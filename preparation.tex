\chapter{Preparation}
\section{Grounding and Shielding}
Add parts from "references\\Grounding and Shielding.pdf"
\section{Laser Current Driver}
\subsection{Design}
\subsubsection{Simulation}
\paragraph{Op Amp Stability}
\subsection{Noise Considerations}
\subsection{Voltage Reference}
\subsection{MOSFET Selection}

\section{LabKraken}
\subsection{Design Goals}
LabKraken is a designed to be a asynchronous, resilient data aquisition suite, that scales to thousands of sensors and accross different networks.
\subsection{Hardware}
\subsection{Software Architecture}
LabKraken needs to scale to thousands of sensors, which need to be served concurrently. This problem is commonly referered to as the C10K problem as dubbed by Dan Kegel back in 1999 \cite{10kProblem} and refers to serving \num{10000} concurrent connections via network sockets. While today millions of concurrent connections can be handled by servers, handling \num{10000} can still be challenging, especially, if the data sources are heterogeneous as is typical for sensor networks of different sensors from different manufacturers.

In order to meet the design goals, an asynchronous architecture was chosen and several different architectures were implemented over time. All in all four complete rewrites of the software were made to arrive at the architecture presented here. The reason for the rewrites is mostly historic and can be explained by the history of the programming language Python, which was used to write the code. The first first version was written for Python 2.6 and exclusively supported sensors made Tinkerforge. In 2015, Python 3.5 was released, which supported a new syntax for asynchronous coroutines. The software was rewritten from scratch to support this new syntax, because it made the code a lot more verbose and easier to follow. With the release of Python 3.7 in 2018 asynchronous generator expressions where mature enough to be used in productions and the programm was again rewritten to use the new syntax. In 2021 a new approach was taken and the programm was once more rewritten with a functional programming style. I will discuss each approach in the next sections to highlight the improvements, that were made over time. Each of these sections discusses the same programm, but written in different styles to show the differences.

\subsubsection{Threaded Design}
The first version of LabKraken used a threaded design approach, because the original libraries of the Tinkerforge sensors are built around threads. The following simplified example shows some code to connect to a temperature sensor over the network and read its data.

\inputpython{source/lab_kraken_threads.py}{1}{26}

\subsubsection{Device Identifiers}
Every sensor network needs device identifiers. Preferably those identifiers should be unique. Typically a device has some kind of internal indetifier. Here are a few examples of the sensors used in our network:

\begin{table}[h]
\centering
\begin{tabularx}{0.95\textwidth}{|l|p{6.5cm}|X|}
    \hline
    Device Type& Identifiers& Example\\
    \hline
    GPIB (SCPI)& \textit{*IDN?} returns \newline \$manufacturer,\$name,\$serial,\$revision& \\
    \hline
    Tinkerforge& Each sensor has a base58 encoded integer device id& QE9 (163684)\\
    \hline
    Labnode& Universal Unique Identifier (UUID) & cc2f2159-e2fb-4ed9-\newline8021-7771890b37ad\\
    \hline
\end{tabularx}
\end{table}

As it can be seen above, these identifiers do not guarantee to uniquely identify a device within a network. The Tinkerforge id is the weakest, as it is a \qty{32}{\bit} integer (4.294.967.295 options), which might easily collide with another id from a different manufacturer. The tinkerforge id is presented as a base58 encoded string. An encoder/decoder example can be found in the TinkerforgeAsync library \cite{TinkerforgeAsync}.

The id string returned by a SCPI device is slightly better, but again does not guarantee uniqueness. As it is shown in the example the same device might return a different id defpending on its settings. This typically done by manufacturers for compatibility reasons.

The only reasonably unique id is the universal unique identifier (UUID) or globally unique identifier (GUID), as dubbed by Microsoft, used in the Labnodes. Their id can be used for networks with participant numbers going into the millions.

Calculating the probability of a collision between two random UUIDs is called the birthday problem \cite{BirthdayProblem} in probability theory. A randomly generated version 4 UUID of variant 1 as defined in RFC 4122 \cite{RFC-UUID} has \qty{122}{\bit} of entropy, that is out of \qty{128}{\bit}, \qty{4}{\bit} are reserved for the UUID version and \qty{2}{\bit} for the variant. This gives the probability of at least one collision in $n$ devices out of $M = 2^{122}$ possibilities:
\begin{align}
    p(n) &= 1 - 1 \cdot \left(1 - \frac{1}{M}\right) \cdot \left(1 - \frac{2}{M}\right) \dots \left(1 - \frac{n-1}{M}\right) \nonumber\\
    &= 1 - \prod_{k=1}^{n-1} \left(1 - \frac{k}{M} \right)
\end{align}
Using the Taylor series $e^x = 1+x \dots$, assuming $n \ll M$ and approximating we can simplify this to:
\begin{align}
    p(n) &\approx 1 - \left(e^\frac{-1}{M} \cdot e^\frac{-2}{M} \dots e^\frac{-(n-1)}{M} \right) \nonumber\\
    &\approx 1 - \left(e^\frac{-n(n-1)/2}{M} \right) \nonumber\\
    &\approx 1 - \left(1 - \frac{n^2}{2 M} \right) = \frac{n^2}{2 M}
\end{align}
For one million devices, this gives a probability of about \num{2e-25}, which is negligible.

In the Kraken implementation, all devices, except for the Labnodes, will be mapped to UUIDs using the underlying configuration database. It is up to the user to ensure the uniqueness of the non-UUID ids reported by the devices to ensure proper mapping.


\subsubsection{Limitations} % FIXME: Different title
There is one inherent limitation to the ethernet bus for instrumentation. The ethernet bus is inherently asynchronous and multiple controllers can talk to the device at the same time. Not only that, but different processes within the same controller can talk to the same device. This makes deterministic statements about the device state challenging.

While it is impossible to rule out the possibility of multiple controllers on a network, care was taken to synchronize the workers within Kraken.
\subsection{Databases}
\subsubsection{Cardinality}
\begin{itemize}
 \item TimescaleDB vs Influx
 \item Example Sensors vs. Experiment
\end{itemize}

\clearpage
\section{Short Introduction to Control Theory}
This section will give a very brief introduction into some basic concepts of control theory. Many systems require control over one or more process variables. For example, temperature control of a room or a device, or creating a current from a voltage. All of this requires control over a process and is established trough feedback, which allows a controller to sense the state of the system.

The focus of this section lies on the principels feedback and control and will be detailed in the following sections.

\subsection{Transfer Functions}

\subsection{Open and Closed Loop Systems}
To understand feedback, one needs to take a look at dynamical systems. There are two types of systems: open and closed loop systems. A system is called open loop, if the output of a system does not influece its input as in figure \ref{fig:open_loop}. On the other hand, if the output is connected to the input of the system it is called closed loop system, an example is shown in figure \ref{fig:closed_loop}. $G(s)$ is called the transfer function of the system, while $R(s)$ is the input, $Y(s)$ is the output and $s$ the Laplace variable.

\begin{figure}[ht]
    \centering
    \begin{subfigure}{0.4\linewidth}
        \import{figures/}{open_loop.tex}
        \caption{Open loop system.}
        \label{fig:open_loop}
    \end{subfigure}
    \begin{subfigure}{0.4\linewidth}
        \import{figures/}{closed_loop.tex}
        \caption{Closed loop system.}
        \label{fig:closed_loop}
    \end{subfigure}
    \caption{Block diagram of closed and open loop systems.}
\end{figure}

It is convenient to express the transfer function as its Laplace transform. The unilateral Laplace transform is definded as:
\begin{equation}
    \mathscr{L}\left( f(t) \right) = F(s) = \int_0^\infty f(t) e^{-st}\,dt.
\end{equation}

with $f: \mathbb{R}^+ \to \mathbb{R}$, that is integrable and grows no faster than $e^{s_0t}$ for $s_0 \in \mathbb{R}$. The latter property is important for deriving the rules of differentiation and integration.

To understand the benefits of using the Laplace representation for transfer function a few useful properties must be discussed. First of all the Laplace transform is linear:
\begin{align}
    \mathscr{L}\left(a \cdot f(t) + b \cdot g(t) \right) &= \int_0^\infty (a \cdot f(t) + b \cdot g(t)) e^{-st}\,dt \nonumber\\
    &= a \int_0^\infty f(t) e^{-st}\,dt + b \int_0^\infty g(t) e^{-st}\,dt \nonumber\\
    &= a \mathscr{L}\left(f(t)\right) + b \mathscr{L}\left(g(t)\right)
\end{align}

Another interesting property is the derivative and integral of a function $f$:

\begin{align}
    \mathscr{L}\left(\frac{df}{dt}\right) &= \int_0^\infty \underbracket{f'(t)}_{v'(t)} \underbracket{\vphantom{f'(t)}e^{-st}}_{u(t)}\,dt \nonumber\\
    &= \left[e^{-st} f(t) \right]_0^\infty - \int_0^\infty (-s)f'(t)\,dt \nonumber\\
    &= -f(0) + s \int_0^\infty f'(t)\,dt \nonumber\\
    &= s F(s) - f(0)
\end{align}

\begin{align}
    \mathscr{L} \left( \int_0^t f(\tau)\,d\tau \right) &= \int_0^\infty \left(\int_0^t f(\tau)\,d\tau e^{-st} \right)\,dt \nonumber\\
    &= \int_0^\infty \underbracket{e^{-st}\vphantom{\int_0^t}}_{v'(t)} \underbracket{\int_0^t f(t)\,d\tau}_{u(t)}\,dt \nonumber\\
    &= \left[\frac{-1}{s} e^{-st} \int_0^t f(t)\,d\tau \right]_0^\infty - \int_0^\infty \frac{-1}{s} e^{-s\tau} f(\tau)\,d\tau \nonumber\\
    &= 0 + \frac{1}{s} \int_0^\infty e^{-s\tau} f(\tau)\,d\tau \nonumber\\
    &= \frac{1}{s} F(s) \label{eqn:lapace_integration}
\end{align}

If the initial state $f(0)$ can be chosen to be $0$, the differentiation becomes a simple multiplication by $s$, while the integration becomes a division by $s$. Finally, the most important aspect is, that a simple relation between the input $r(t)$ and the ouput $y(t)$ of a system can be given. The relation between input and the ouput of a system as shown in figure \ref{fig:open_loop} is given by the convolution, see e.g. \cite{pid_basics}. Assuming the system has an initial state of $0$ for $t<0$, hence $r(t<0) = 0$ and $g(t<0) = 0$, one can calculate:

\begin{equation}
    y(t) = (r \ast g)(t) = \int_0^\infty r(\tau) g(t-\tau)\,d\tau
    \label{eqn:convolution}
\end{equation}

Applying the Laplace transformation, greatly simplifies this:
\begin{align}
    Y(s) &= \int_0^\infty e^{-st} y(t)\,dt \nonumber\\
    \overset{\ref{eqn:convolution}}&{=} \int_0^\infty \underbrace{e^{-st}}_{e^{-s(t-\tau)}e^{-s\tau}} \int_0^\infty r(\tau) g(t-\tau)\,d\tau\,dt \nonumber\\
    &= \int_0^\infty \int_0^t e^{-s(t-\tau)} e^{-s\tau} g(t-\tau) r(\tau)\,d\tau\,dt \nonumber\\
    &= \int_0^\infty e^{-s\tau} r(\tau)\,d\tau \int_0^\infty e^{-st} g(t)\,dt \nonumber\\
    &= R(s) \cdot G(s)
\end{align}

This formula is a lot simpler than the convolution of $r(t)$ and $g(t)$, therefore the use of the Laplace transform has become very popular in control theory.

Another property that is heavily used in control theory is the time delay of functions. To show this property, let $f(t-\theta)$ be
\begin{equation}
    g(t) \coloneqq \begin{cases} f(t-\theta), & t \geq \theta \\ 0, & t < \theta \end{cases} \label{eqn:delayed_f}
\end{equation}

The reason for this definition is, that the system must be causal. This means, it is impossible to get data from the future ($t<\theta$). An example is shown in figure \ref{fig:heaviside}.

\begin{figure}[ht]
    \centering
    \begin{subfigure}{0.4\linewidth}
        \scalebox{0.75}{%
            \import{figures/}{laplace_no_delay.tex}
        } % scalebox
        \caption{Original signal $f(t)$.}
        \label{fig:heaviside}
    \end{subfigure}
    \begin{subfigure}{0.4\linewidth}
        \scalebox{0.75}{%
            \import{figures/}{laplace_time_delay.tex}
        } % scalebox
        \caption{Delayed signal $f(t-2)$.}
        \label{fig:heaviside_delayed}
    \end{subfigure}
\end{figure}

The Laplace transform of a delayed signal can be calculated as follows:

\begin{align}
    \mathscr{L}\left( g(t) \right) &= \int_0^\infty f(t-\theta) e^{-st}\,dt \nonumber\\
    \overset{\ref{eqn:delayed_f}}&{=} \int_\theta^\infty f(t-\theta) e^{-st}\,dt \nonumber\\
    \overset{u \coloneqq t-\theta}&{=} \int_0^\infty f(u) e^{-s(u+\theta)}\,du \nonumber\\
    &= e^{-s\theta} \int_0^\infty f(u) e^{-su} \nonumber\\
    &= e^{-s\theta} F(s) \label{eqn:laplace_delayed}
\end{align}

To satisfy the causaulity requirement, the Heaviside function $H(t)$ can be used:
\begin{align}
    \mathscr{L}\left( f(t-\theta) H(t-\theta) \right) = e^{-s\theta} F(s) \label{eqn:laplace_causality}
\end{align}

Lastly, the Laplace transform of $e^{at}$, which is commonly used in differential equations:
\begin{align}
    \mathscr{L}\left(e^{at} \right) &= \int_0^\infty e^{(a-s)t}\,dt = \frac{1}{a-s} \left[e^{(a-s)t} \right]_0^\infty = \frac{1}{s-a} \label{eqn:laplace_exponential}
\end{align}


Using these tools, it is possible calculate the transfer function of a temperature controller. This is done in the next section.

\subsection{A Model for Temperature Control}
\begin{figure}[ht]
    \centering
    \scalebox{1}{%
        \import{figures/}{first_order_model.tex}
    } % scalebox
    \caption{Simple temperature model of a generic system.}
    \label{fig:first_order_model_room}
\end{figure}

In order to describe a closed-loop system, one has to first create a model for the process and the controller involved. A simple model can be derived from the idea, that the system at temperature $T_{system}$ has a thermal capacitance $C_{system}$, an influx of heat $\dot Q_{load}$ from a thermal load and a controller removing heat from the system through a heat exchanger with a resistance of $R_{force}$. Additionally, there is some leakage through the walls of the system to the ambient environment via $R_{leakage}$. This analogy of thermodynamics with electrondynamics allows to create the model shown in figure \ref{fig:first_order_model_room}. Since this this model is to be used for a room temperature controller, an assumption to simplify it can be made.

Typically, the room temperature is kept constant. Therefore, the controller will keep $T_{system}$ constant and if the outside temperature and the heat load $\dot Q_{load}$ is \textit{reasonably stable}, it is easy to see, that a constant thermal flux must flow through $R$ since it cannot pass through the thermal capacitance $C$. \textit{Reasonably stable} means that those fluxes can be treated as constant with respect to the temperature controller time constants. This will be further discussed in section \ref{} with regards to system stability. If this assumption holds, the thermal flux from the system load will only cause a constant offset of $T_{force}$, since the heat must be removed by the controller through the resistance $R_{force}$, and the model can be simplified further. Here $T_{force}$ and $T_{system}$ was replaced by $T_{in}$ and $T_{out}$ for better readability.:

\begin{figure}[h]
    \centering
    \scalebox{1}{%
        \import{figures/}{first_order_model_kirchhoff.tex}
    } % scalebox
\end{figure}

This is the classic $RC$ circuit. Now, neglecting the constant thermal flux from the system load and exploiting the analogy of thermodynamics and electrondynamics again, using Kirchhoff's second law, one finds:

\begin{align}
    \sum T_i &= 0 \nonumber\\
    T_{in}(t) - \dot{Q}(t) R - \frac 1 C \int \dot{Q}(t)\,dt &= 0 \label{eqn:first_order_model_kirchhoff}
\end{align}

Taking the Laplace transform, applying equation \ref{eqn:lapace_integration} and using $T_{out} = \frac{1}{sC} \dot Q(s)$ to replace $\dot Q$, equation \ref{eqn:first_order_model_kirchhoff} can be written as:
\begin{align*}
    T_{in}(s) - \dot{Q}(s) R - \frac{1}{sC} \dot{Q}(s) &= 0\\
    \dot{Q}(s) = \frac{T_{in}(s)}{R-\frac{1}{sC}} &= \frac{T_{out}}{\frac{1}{sC}}
\end{align*}

This allows to calculate the transfer function of the process $P$:
\begin{align}
    P(s) &= \frac{T_{out}}{T_{in}} = \frac{\frac{1}{sC}}{R-\frac{1}{sC}} \nonumber\\
    &= \frac{1}{sRC + 1} \nonumber\\
    &= \frac{K}{1 + s\tau} \label{eqn:first_order_model}
\end{align}
with the system gain $K$ and the time constant $\tau$. In case of the $RC$ circuit, the gain is $1$, but other systems may have a gain or attenuation factor of $K \neq 1$ in the sensor.

Equation \ref{eqn:first_order_model} is called the transfer function of a first-order model, because its origin is a differential equation of first order. This model describes homogeneous systems, like a room, very well, as can be seen in section \ref{}, but in order to derive the transfer function including the controller and the sensor some more work is required on the sensor transfer function.

Expanding on figure \ref{fig:closed_loop} and equation \ref{eqn:convolution} the closed-loop transfer function becomes:
\begin{equation}
    G(s) = P(s) \cdot S(s)
\end{equation}

and the block diagram becomes

\begin{figure}[ht]
    \centering
    \scalebox{1}{%
        \import{figures/}{open_loop_full.tex}
    }% scalebox
    \caption{Open loop system with sensor.}
\end{figure}

The transfer funciton of the sensor can be modeled as a delay line with delay $\theta$ and $f(t-\theta) = H(t-\theta)$. A gain of $1$ is assumed here, because any system gain is already included in the parameter $K$. Using equation \ref{eqn:laplace_delayed} $S(s)$ can be written as
\begin{equation}
    S(s) = e^{-\theta s} .
\end{equation}

The full process model including the time delay is:
\begin{equation}
    G(s) = \frac{K}{1 + s\tau} e^{-\theta s} \label{eqn:first_order_plus_dead_time_model}
\end{equation}

This is called a first-order plus dead-time model (FOPDT) or first-order plus time-delay model (FOPTD). To fit experimental data to this model it is more convenient to transform the transfer function \ref{eqn:first_order_plus_dead_time_model} into the time domain. To calculate the output response an input $U(s)$ is required. In principal any function can do, but a step function is typically used, for example by \citeauthor{ziegler_nichols} \cite{ziegler_nichols} and many others \cite{tuning_rules,pessen_integral,simc,smic2,pid_controllers_for_time_delay_systems,pi_stabilization_of_fopdt_systems, pid_basics}. It is both simple to calculate and apply to a real system. Using equations \ref{eqn:laplace_delayed} and \ref{eqn:laplace_exponential}, the Heaviside $H(t)$ step function transforms as
\begin{equation}
    \mathscr{L} \left(u(t) \right) = U(s) = \mathscr{L} \left( \Delta u H(t) \right) = \frac{\Delta u}{s}
\end{equation}

with the step size $\Delta u$. The output $Y(s)$ can then be calculated analytically.

\begin{align}
    Y(s) &= \frac{\Delta u}{s} \frac{K}{1 + s\tau} e^{-\theta s} \nonumber\\
    &=  K \Delta u \frac{1}{s (1 + s\tau)} e^{-\theta s} \nonumber\\
    &= K \Delta u \left(\frac{1}{s} - \frac{\tau}{s\tau+1} \right) e^{-\theta s} \nonumber\\
    &= K \Delta u \left(\frac{1}{s} - \frac{1}{s+\frac{1}{\tau}} \right) e^{-\theta s}
\end{align}

To derive $y(t)$, the inverse Laplace transform of $Y(s)$ is required. Unfortunately, this is not as simple as the Laplace transform. Fortunately, using \ref{eqn:laplace_exponential} while making sure causaulity is guaranteed as shown in \ref{eqn:laplace_causality}, the simple first order model can easily be transformed back into the time domain.

\begin{align}
    \mathscr{L}^{-1} \left(Y(s)\right) = y(t) &= K \Delta u \mathscr{L}^{-1} \left(\frac{1}{s} e^{-\theta s} \right)  - K \mathscr{L}^{-1} \left( \frac{1}{s+\frac{1}{\tau}} e^{-\theta s} \right) \nonumber\\
    \overset{\ref{eqn:laplace_exponential}}&{=} K \Delta u \cdot 1 \cdot H(t-\theta) - \left(e^{-\frac{t-\theta}{\tau}} \right) H(t-\theta) \nonumber\\
    &= K \Delta u \left(1- e^{-\frac{t-\theta}{\tau}} \right) H(t-\theta)
\end{align}

The time domain solution of the FOPDT model can now be used extract the parameters $\tau$, $\theta$ and $K$ from a real physical system using a fit to the measurement data. The parameter $\Delta u$ is already known, since it is an input parameter. A simulation of the step response of a first-order model with time delay is shown in figure \ref{fig:fopdt}. Here it can be clearly seen, that the output does not change until the time delay $\theta$ has passed and the Heaviside function changes from $0$ to $1$.

\begin{figure}[ht]
    \centering
    %% Creator: Matplotlib, PGF backend
%%
%% To include the figure in your LaTeX document, write
%%   \input{<filename>.pgf}
%%
%% Make sure the required packages are loaded in your preamble
%%   \usepackage{pgf}
%%
%% Also ensure that all the required font packages are loaded; for instance,
%% the lmodern package is sometimes necessary when using math font.
%%   \usepackage{lmodern}
%%
%% Figures using additional raster images can only be included by \input if
%% they are in the same directory as the main LaTeX file. For loading figures
%% from other directories you can use the `import` package
%%   \usepackage{import}
%%
%% and then include the figures with
%%   \import{<path to file>}{<filename>.pgf}
%%
%% Matplotlib used the following preamble
%%   \usepackage{siunitx}
%%   \sisetup{per-mode = symbol}%
%%   \usepackage{fontspec}
%%   \makeatletter\@ifpackageloaded{underscore}{}{\usepackage[strings]{underscore}}\makeatother
%%
\begingroup%
\makeatletter%
\begin{pgfpicture}%
\pgfpathrectangle{\pgfpointorigin}{\pgfqpoint{5.431103in}{3.356606in}}%
\pgfusepath{use as bounding box, clip}%
\begin{pgfscope}%
\pgfsetbuttcap%
\pgfsetmiterjoin%
\definecolor{currentfill}{rgb}{1.000000,1.000000,1.000000}%
\pgfsetfillcolor{currentfill}%
\pgfsetlinewidth{0.000000pt}%
\definecolor{currentstroke}{rgb}{1.000000,1.000000,1.000000}%
\pgfsetstrokecolor{currentstroke}%
\pgfsetdash{}{0pt}%
\pgfpathmoveto{\pgfqpoint{0.000000in}{0.000000in}}%
\pgfpathlineto{\pgfqpoint{5.431103in}{0.000000in}}%
\pgfpathlineto{\pgfqpoint{5.431103in}{3.356606in}}%
\pgfpathlineto{\pgfqpoint{0.000000in}{3.356606in}}%
\pgfpathlineto{\pgfqpoint{0.000000in}{0.000000in}}%
\pgfpathclose%
\pgfusepath{fill}%
\end{pgfscope}%
\begin{pgfscope}%
\pgfsetbuttcap%
\pgfsetmiterjoin%
\definecolor{currentfill}{rgb}{1.000000,1.000000,1.000000}%
\pgfsetfillcolor{currentfill}%
\pgfsetlinewidth{0.000000pt}%
\definecolor{currentstroke}{rgb}{0.000000,0.000000,0.000000}%
\pgfsetstrokecolor{currentstroke}%
\pgfsetstrokeopacity{0.000000}%
\pgfsetdash{}{0pt}%
\pgfpathmoveto{\pgfqpoint{0.667540in}{0.524170in}}%
\pgfpathlineto{\pgfqpoint{5.222294in}{0.524170in}}%
\pgfpathlineto{\pgfqpoint{5.222294in}{3.168170in}}%
\pgfpathlineto{\pgfqpoint{0.667540in}{3.168170in}}%
\pgfpathlineto{\pgfqpoint{0.667540in}{0.524170in}}%
\pgfpathclose%
\pgfusepath{fill}%
\end{pgfscope}%
\begin{pgfscope}%
\pgfsetbuttcap%
\pgfsetroundjoin%
\definecolor{currentfill}{rgb}{0.000000,0.000000,0.000000}%
\pgfsetfillcolor{currentfill}%
\pgfsetlinewidth{0.803000pt}%
\definecolor{currentstroke}{rgb}{0.000000,0.000000,0.000000}%
\pgfsetstrokecolor{currentstroke}%
\pgfsetdash{}{0pt}%
\pgfsys@defobject{currentmarker}{\pgfqpoint{0.000000in}{-0.048611in}}{\pgfqpoint{0.000000in}{0.000000in}}{%
\pgfpathmoveto{\pgfqpoint{0.000000in}{0.000000in}}%
\pgfpathlineto{\pgfqpoint{0.000000in}{-0.048611in}}%
\pgfusepath{stroke,fill}%
}%
\begin{pgfscope}%
\pgfsys@transformshift{0.667540in}{0.524170in}%
\pgfsys@useobject{currentmarker}{}%
\end{pgfscope}%
\end{pgfscope}%
\begin{pgfscope}%
\definecolor{textcolor}{rgb}{0.000000,0.000000,0.000000}%
\pgfsetstrokecolor{textcolor}%
\pgfsetfillcolor{textcolor}%
\pgftext[x=0.667540in,y=0.426948in,,top]{\color{textcolor}\rmfamily\fontsize{8.000000}{9.600000}\selectfont \(\displaystyle {0}\)}%
\end{pgfscope}%
\begin{pgfscope}%
\pgfsetbuttcap%
\pgfsetroundjoin%
\definecolor{currentfill}{rgb}{0.000000,0.000000,0.000000}%
\pgfsetfillcolor{currentfill}%
\pgfsetlinewidth{0.803000pt}%
\definecolor{currentstroke}{rgb}{0.000000,0.000000,0.000000}%
\pgfsetstrokecolor{currentstroke}%
\pgfsetdash{}{0pt}%
\pgfsys@defobject{currentmarker}{\pgfqpoint{0.000000in}{-0.048611in}}{\pgfqpoint{0.000000in}{0.000000in}}{%
\pgfpathmoveto{\pgfqpoint{0.000000in}{0.000000in}}%
\pgfpathlineto{\pgfqpoint{0.000000in}{-0.048611in}}%
\pgfusepath{stroke,fill}%
}%
\begin{pgfscope}%
\pgfsys@transformshift{1.578491in}{0.524170in}%
\pgfsys@useobject{currentmarker}{}%
\end{pgfscope}%
\end{pgfscope}%
\begin{pgfscope}%
\definecolor{textcolor}{rgb}{0.000000,0.000000,0.000000}%
\pgfsetstrokecolor{textcolor}%
\pgfsetfillcolor{textcolor}%
\pgftext[x=1.578491in,y=0.426948in,,top]{\color{textcolor}\rmfamily\fontsize{8.000000}{9.600000}\selectfont \(\displaystyle {2}\)}%
\end{pgfscope}%
\begin{pgfscope}%
\pgfsetbuttcap%
\pgfsetroundjoin%
\definecolor{currentfill}{rgb}{0.000000,0.000000,0.000000}%
\pgfsetfillcolor{currentfill}%
\pgfsetlinewidth{0.803000pt}%
\definecolor{currentstroke}{rgb}{0.000000,0.000000,0.000000}%
\pgfsetstrokecolor{currentstroke}%
\pgfsetdash{}{0pt}%
\pgfsys@defobject{currentmarker}{\pgfqpoint{0.000000in}{-0.048611in}}{\pgfqpoint{0.000000in}{0.000000in}}{%
\pgfpathmoveto{\pgfqpoint{0.000000in}{0.000000in}}%
\pgfpathlineto{\pgfqpoint{0.000000in}{-0.048611in}}%
\pgfusepath{stroke,fill}%
}%
\begin{pgfscope}%
\pgfsys@transformshift{2.489442in}{0.524170in}%
\pgfsys@useobject{currentmarker}{}%
\end{pgfscope}%
\end{pgfscope}%
\begin{pgfscope}%
\definecolor{textcolor}{rgb}{0.000000,0.000000,0.000000}%
\pgfsetstrokecolor{textcolor}%
\pgfsetfillcolor{textcolor}%
\pgftext[x=2.489442in,y=0.426948in,,top]{\color{textcolor}\rmfamily\fontsize{8.000000}{9.600000}\selectfont \(\displaystyle {4}\)}%
\end{pgfscope}%
\begin{pgfscope}%
\pgfsetbuttcap%
\pgfsetroundjoin%
\definecolor{currentfill}{rgb}{0.000000,0.000000,0.000000}%
\pgfsetfillcolor{currentfill}%
\pgfsetlinewidth{0.803000pt}%
\definecolor{currentstroke}{rgb}{0.000000,0.000000,0.000000}%
\pgfsetstrokecolor{currentstroke}%
\pgfsetdash{}{0pt}%
\pgfsys@defobject{currentmarker}{\pgfqpoint{0.000000in}{-0.048611in}}{\pgfqpoint{0.000000in}{0.000000in}}{%
\pgfpathmoveto{\pgfqpoint{0.000000in}{0.000000in}}%
\pgfpathlineto{\pgfqpoint{0.000000in}{-0.048611in}}%
\pgfusepath{stroke,fill}%
}%
\begin{pgfscope}%
\pgfsys@transformshift{3.400393in}{0.524170in}%
\pgfsys@useobject{currentmarker}{}%
\end{pgfscope}%
\end{pgfscope}%
\begin{pgfscope}%
\definecolor{textcolor}{rgb}{0.000000,0.000000,0.000000}%
\pgfsetstrokecolor{textcolor}%
\pgfsetfillcolor{textcolor}%
\pgftext[x=3.400393in,y=0.426948in,,top]{\color{textcolor}\rmfamily\fontsize{8.000000}{9.600000}\selectfont \(\displaystyle {6}\)}%
\end{pgfscope}%
\begin{pgfscope}%
\pgfsetbuttcap%
\pgfsetroundjoin%
\definecolor{currentfill}{rgb}{0.000000,0.000000,0.000000}%
\pgfsetfillcolor{currentfill}%
\pgfsetlinewidth{0.803000pt}%
\definecolor{currentstroke}{rgb}{0.000000,0.000000,0.000000}%
\pgfsetstrokecolor{currentstroke}%
\pgfsetdash{}{0pt}%
\pgfsys@defobject{currentmarker}{\pgfqpoint{0.000000in}{-0.048611in}}{\pgfqpoint{0.000000in}{0.000000in}}{%
\pgfpathmoveto{\pgfqpoint{0.000000in}{0.000000in}}%
\pgfpathlineto{\pgfqpoint{0.000000in}{-0.048611in}}%
\pgfusepath{stroke,fill}%
}%
\begin{pgfscope}%
\pgfsys@transformshift{4.311344in}{0.524170in}%
\pgfsys@useobject{currentmarker}{}%
\end{pgfscope}%
\end{pgfscope}%
\begin{pgfscope}%
\definecolor{textcolor}{rgb}{0.000000,0.000000,0.000000}%
\pgfsetstrokecolor{textcolor}%
\pgfsetfillcolor{textcolor}%
\pgftext[x=4.311344in,y=0.426948in,,top]{\color{textcolor}\rmfamily\fontsize{8.000000}{9.600000}\selectfont \(\displaystyle {8}\)}%
\end{pgfscope}%
\begin{pgfscope}%
\pgfsetbuttcap%
\pgfsetroundjoin%
\definecolor{currentfill}{rgb}{0.000000,0.000000,0.000000}%
\pgfsetfillcolor{currentfill}%
\pgfsetlinewidth{0.803000pt}%
\definecolor{currentstroke}{rgb}{0.000000,0.000000,0.000000}%
\pgfsetstrokecolor{currentstroke}%
\pgfsetdash{}{0pt}%
\pgfsys@defobject{currentmarker}{\pgfqpoint{0.000000in}{-0.048611in}}{\pgfqpoint{0.000000in}{0.000000in}}{%
\pgfpathmoveto{\pgfqpoint{0.000000in}{0.000000in}}%
\pgfpathlineto{\pgfqpoint{0.000000in}{-0.048611in}}%
\pgfusepath{stroke,fill}%
}%
\begin{pgfscope}%
\pgfsys@transformshift{5.222294in}{0.524170in}%
\pgfsys@useobject{currentmarker}{}%
\end{pgfscope}%
\end{pgfscope}%
\begin{pgfscope}%
\definecolor{textcolor}{rgb}{0.000000,0.000000,0.000000}%
\pgfsetstrokecolor{textcolor}%
\pgfsetfillcolor{textcolor}%
\pgftext[x=5.222294in,y=0.426948in,,top]{\color{textcolor}\rmfamily\fontsize{8.000000}{9.600000}\selectfont \(\displaystyle {10}\)}%
\end{pgfscope}%
\begin{pgfscope}%
\definecolor{textcolor}{rgb}{0.000000,0.000000,0.000000}%
\pgfsetstrokecolor{textcolor}%
\pgfsetfillcolor{textcolor}%
\pgftext[x=2.944917in,y=0.272725in,,top]{\color{textcolor}\rmfamily\fontsize{10.000000}{12.000000}\selectfont Time}%
\end{pgfscope}%
\begin{pgfscope}%
\pgfsetbuttcap%
\pgfsetroundjoin%
\definecolor{currentfill}{rgb}{0.000000,0.000000,0.000000}%
\pgfsetfillcolor{currentfill}%
\pgfsetlinewidth{0.803000pt}%
\definecolor{currentstroke}{rgb}{0.000000,0.000000,0.000000}%
\pgfsetstrokecolor{currentstroke}%
\pgfsetdash{}{0pt}%
\pgfsys@defobject{currentmarker}{\pgfqpoint{-0.048611in}{0.000000in}}{\pgfqpoint{-0.000000in}{0.000000in}}{%
\pgfpathmoveto{\pgfqpoint{-0.000000in}{0.000000in}}%
\pgfpathlineto{\pgfqpoint{-0.048611in}{0.000000in}}%
\pgfusepath{stroke,fill}%
}%
\begin{pgfscope}%
\pgfsys@transformshift{0.667540in}{0.524170in}%
\pgfsys@useobject{currentmarker}{}%
\end{pgfscope}%
\end{pgfscope}%
\begin{pgfscope}%
\definecolor{textcolor}{rgb}{0.000000,0.000000,0.000000}%
\pgfsetstrokecolor{textcolor}%
\pgfsetfillcolor{textcolor}%
\pgftext[x=0.327644in, y=0.485614in, left, base]{\color{textcolor}\rmfamily\fontsize{8.000000}{9.600000}\selectfont \(\displaystyle {\ensuremath{-}1.0}\)}%
\end{pgfscope}%
\begin{pgfscope}%
\pgfsetbuttcap%
\pgfsetroundjoin%
\definecolor{currentfill}{rgb}{0.000000,0.000000,0.000000}%
\pgfsetfillcolor{currentfill}%
\pgfsetlinewidth{0.803000pt}%
\definecolor{currentstroke}{rgb}{0.000000,0.000000,0.000000}%
\pgfsetstrokecolor{currentstroke}%
\pgfsetdash{}{0pt}%
\pgfsys@defobject{currentmarker}{\pgfqpoint{-0.048611in}{0.000000in}}{\pgfqpoint{-0.000000in}{0.000000in}}{%
\pgfpathmoveto{\pgfqpoint{-0.000000in}{0.000000in}}%
\pgfpathlineto{\pgfqpoint{-0.048611in}{0.000000in}}%
\pgfusepath{stroke,fill}%
}%
\begin{pgfscope}%
\pgfsys@transformshift{0.667540in}{1.052970in}%
\pgfsys@useobject{currentmarker}{}%
\end{pgfscope}%
\end{pgfscope}%
\begin{pgfscope}%
\definecolor{textcolor}{rgb}{0.000000,0.000000,0.000000}%
\pgfsetstrokecolor{textcolor}%
\pgfsetfillcolor{textcolor}%
\pgftext[x=0.327644in, y=1.014414in, left, base]{\color{textcolor}\rmfamily\fontsize{8.000000}{9.600000}\selectfont \(\displaystyle {\ensuremath{-}0.5}\)}%
\end{pgfscope}%
\begin{pgfscope}%
\pgfsetbuttcap%
\pgfsetroundjoin%
\definecolor{currentfill}{rgb}{0.000000,0.000000,0.000000}%
\pgfsetfillcolor{currentfill}%
\pgfsetlinewidth{0.803000pt}%
\definecolor{currentstroke}{rgb}{0.000000,0.000000,0.000000}%
\pgfsetstrokecolor{currentstroke}%
\pgfsetdash{}{0pt}%
\pgfsys@defobject{currentmarker}{\pgfqpoint{-0.048611in}{0.000000in}}{\pgfqpoint{-0.000000in}{0.000000in}}{%
\pgfpathmoveto{\pgfqpoint{-0.000000in}{0.000000in}}%
\pgfpathlineto{\pgfqpoint{-0.048611in}{0.000000in}}%
\pgfusepath{stroke,fill}%
}%
\begin{pgfscope}%
\pgfsys@transformshift{0.667540in}{1.581770in}%
\pgfsys@useobject{currentmarker}{}%
\end{pgfscope}%
\end{pgfscope}%
\begin{pgfscope}%
\definecolor{textcolor}{rgb}{0.000000,0.000000,0.000000}%
\pgfsetstrokecolor{textcolor}%
\pgfsetfillcolor{textcolor}%
\pgftext[x=0.419467in, y=1.543214in, left, base]{\color{textcolor}\rmfamily\fontsize{8.000000}{9.600000}\selectfont \(\displaystyle {0.0}\)}%
\end{pgfscope}%
\begin{pgfscope}%
\pgfsetbuttcap%
\pgfsetroundjoin%
\definecolor{currentfill}{rgb}{0.000000,0.000000,0.000000}%
\pgfsetfillcolor{currentfill}%
\pgfsetlinewidth{0.803000pt}%
\definecolor{currentstroke}{rgb}{0.000000,0.000000,0.000000}%
\pgfsetstrokecolor{currentstroke}%
\pgfsetdash{}{0pt}%
\pgfsys@defobject{currentmarker}{\pgfqpoint{-0.048611in}{0.000000in}}{\pgfqpoint{-0.000000in}{0.000000in}}{%
\pgfpathmoveto{\pgfqpoint{-0.000000in}{0.000000in}}%
\pgfpathlineto{\pgfqpoint{-0.048611in}{0.000000in}}%
\pgfusepath{stroke,fill}%
}%
\begin{pgfscope}%
\pgfsys@transformshift{0.667540in}{2.110570in}%
\pgfsys@useobject{currentmarker}{}%
\end{pgfscope}%
\end{pgfscope}%
\begin{pgfscope}%
\definecolor{textcolor}{rgb}{0.000000,0.000000,0.000000}%
\pgfsetstrokecolor{textcolor}%
\pgfsetfillcolor{textcolor}%
\pgftext[x=0.419467in, y=2.072014in, left, base]{\color{textcolor}\rmfamily\fontsize{8.000000}{9.600000}\selectfont \(\displaystyle {0.5}\)}%
\end{pgfscope}%
\begin{pgfscope}%
\pgfsetbuttcap%
\pgfsetroundjoin%
\definecolor{currentfill}{rgb}{0.000000,0.000000,0.000000}%
\pgfsetfillcolor{currentfill}%
\pgfsetlinewidth{0.803000pt}%
\definecolor{currentstroke}{rgb}{0.000000,0.000000,0.000000}%
\pgfsetstrokecolor{currentstroke}%
\pgfsetdash{}{0pt}%
\pgfsys@defobject{currentmarker}{\pgfqpoint{-0.048611in}{0.000000in}}{\pgfqpoint{-0.000000in}{0.000000in}}{%
\pgfpathmoveto{\pgfqpoint{-0.000000in}{0.000000in}}%
\pgfpathlineto{\pgfqpoint{-0.048611in}{0.000000in}}%
\pgfusepath{stroke,fill}%
}%
\begin{pgfscope}%
\pgfsys@transformshift{0.667540in}{2.639370in}%
\pgfsys@useobject{currentmarker}{}%
\end{pgfscope}%
\end{pgfscope}%
\begin{pgfscope}%
\definecolor{textcolor}{rgb}{0.000000,0.000000,0.000000}%
\pgfsetstrokecolor{textcolor}%
\pgfsetfillcolor{textcolor}%
\pgftext[x=0.419467in, y=2.600814in, left, base]{\color{textcolor}\rmfamily\fontsize{8.000000}{9.600000}\selectfont \(\displaystyle {1.0}\)}%
\end{pgfscope}%
\begin{pgfscope}%
\pgfsetbuttcap%
\pgfsetroundjoin%
\definecolor{currentfill}{rgb}{0.000000,0.000000,0.000000}%
\pgfsetfillcolor{currentfill}%
\pgfsetlinewidth{0.803000pt}%
\definecolor{currentstroke}{rgb}{0.000000,0.000000,0.000000}%
\pgfsetstrokecolor{currentstroke}%
\pgfsetdash{}{0pt}%
\pgfsys@defobject{currentmarker}{\pgfqpoint{-0.048611in}{0.000000in}}{\pgfqpoint{-0.000000in}{0.000000in}}{%
\pgfpathmoveto{\pgfqpoint{-0.000000in}{0.000000in}}%
\pgfpathlineto{\pgfqpoint{-0.048611in}{0.000000in}}%
\pgfusepath{stroke,fill}%
}%
\begin{pgfscope}%
\pgfsys@transformshift{0.667540in}{3.168170in}%
\pgfsys@useobject{currentmarker}{}%
\end{pgfscope}%
\end{pgfscope}%
\begin{pgfscope}%
\definecolor{textcolor}{rgb}{0.000000,0.000000,0.000000}%
\pgfsetstrokecolor{textcolor}%
\pgfsetfillcolor{textcolor}%
\pgftext[x=0.419467in, y=3.129614in, left, base]{\color{textcolor}\rmfamily\fontsize{8.000000}{9.600000}\selectfont \(\displaystyle {1.5}\)}%
\end{pgfscope}%
\begin{pgfscope}%
\definecolor{textcolor}{rgb}{0.000000,0.000000,0.000000}%
\pgfsetstrokecolor{textcolor}%
\pgfsetfillcolor{textcolor}%
\pgftext[x=0.272089in,y=1.846170in,,bottom,rotate=90.000000]{\color{textcolor}\rmfamily\fontsize{10.000000}{12.000000}\selectfont Process Output}%
\end{pgfscope}%
\begin{pgfscope}%
\pgfpathrectangle{\pgfqpoint{0.667540in}{0.524170in}}{\pgfqpoint{4.554755in}{2.644000in}}%
\pgfusepath{clip}%
\pgfsetbuttcap%
\pgfsetroundjoin%
\pgfsetlinewidth{1.505625pt}%
\definecolor{currentstroke}{rgb}{0.003922,0.450980,0.698039}%
\pgfsetstrokecolor{currentstroke}%
\pgfsetstrokeopacity{0.700000}%
\pgfsetdash{{5.550000pt}{2.400000pt}}{0.000000pt}%
\pgfpathmoveto{\pgfqpoint{1.853769in}{0.514170in}}%
\pgfpathlineto{\pgfqpoint{1.897324in}{0.613494in}}%
\pgfpathlineto{\pgfqpoint{1.942871in}{0.712297in}}%
\pgfpathlineto{\pgfqpoint{1.988419in}{0.806281in}}%
\pgfpathlineto{\pgfqpoint{2.033966in}{0.895682in}}%
\pgfpathlineto{\pgfqpoint{2.079514in}{0.980723in}}%
\pgfpathlineto{\pgfqpoint{2.125061in}{1.061616in}}%
\pgfpathlineto{\pgfqpoint{2.170609in}{1.138564in}}%
\pgfpathlineto{\pgfqpoint{2.216156in}{1.211759in}}%
\pgfpathlineto{\pgfqpoint{2.261704in}{1.281385in}}%
\pgfpathlineto{\pgfqpoint{2.307252in}{1.347614in}}%
\pgfpathlineto{\pgfqpoint{2.352799in}{1.410614in}}%
\pgfpathlineto{\pgfqpoint{2.398347in}{1.470541in}}%
\pgfpathlineto{\pgfqpoint{2.443894in}{1.527545in}}%
\pgfpathlineto{\pgfqpoint{2.489442in}{1.581770in}}%
\pgfpathlineto{\pgfqpoint{2.534989in}{1.633350in}}%
\pgfpathlineto{\pgfqpoint{2.580537in}{1.682414in}}%
\pgfpathlineto{\pgfqpoint{2.626084in}{1.729085in}}%
\pgfpathlineto{\pgfqpoint{2.671632in}{1.773480in}}%
\pgfpathlineto{\pgfqpoint{2.717179in}{1.815710in}}%
\pgfpathlineto{\pgfqpoint{2.762727in}{1.855880in}}%
\pgfpathlineto{\pgfqpoint{2.808275in}{1.894092in}}%
\pgfpathlineto{\pgfqpoint{2.853822in}{1.930439in}}%
\pgfpathlineto{\pgfqpoint{2.899370in}{1.965014in}}%
\pgfpathlineto{\pgfqpoint{2.944917in}{1.997903in}}%
\pgfpathlineto{\pgfqpoint{2.990465in}{2.029188in}}%
\pgfpathlineto{\pgfqpoint{3.036012in}{2.058947in}}%
\pgfpathlineto{\pgfqpoint{3.081560in}{2.087254in}}%
\pgfpathlineto{\pgfqpoint{3.127107in}{2.114181in}}%
\pgfpathlineto{\pgfqpoint{3.172655in}{2.139795in}}%
\pgfpathlineto{\pgfqpoint{3.218202in}{2.164160in}}%
\pgfpathlineto{\pgfqpoint{3.263750in}{2.187336in}}%
\pgfpathlineto{\pgfqpoint{3.309298in}{2.209382in}}%
\pgfpathlineto{\pgfqpoint{3.354845in}{2.230353in}}%
\pgfpathlineto{\pgfqpoint{3.400393in}{2.250301in}}%
\pgfpathlineto{\pgfqpoint{3.445940in}{2.269276in}}%
\pgfpathlineto{\pgfqpoint{3.491488in}{2.287325in}}%
\pgfpathlineto{\pgfqpoint{3.537035in}{2.304495in}}%
\pgfpathlineto{\pgfqpoint{3.582583in}{2.320827in}}%
\pgfpathlineto{\pgfqpoint{3.628130in}{2.336362in}}%
\pgfpathlineto{\pgfqpoint{3.673678in}{2.351140in}}%
\pgfpathlineto{\pgfqpoint{3.719225in}{2.365197in}}%
\pgfpathlineto{\pgfqpoint{3.764773in}{2.378569in}}%
\pgfpathlineto{\pgfqpoint{3.810321in}{2.391288in}}%
\pgfpathlineto{\pgfqpoint{3.855868in}{2.403387in}}%
\pgfpathlineto{\pgfqpoint{3.901416in}{2.414896in}}%
\pgfpathlineto{\pgfqpoint{3.946963in}{2.425844in}}%
\pgfpathlineto{\pgfqpoint{3.992511in}{2.436258in}}%
\pgfpathlineto{\pgfqpoint{4.038058in}{2.446164in}}%
\pgfpathlineto{\pgfqpoint{4.083606in}{2.455586in}}%
\pgfpathlineto{\pgfqpoint{4.129153in}{2.464550in}}%
\pgfpathlineto{\pgfqpoint{4.174701in}{2.473076in}}%
\pgfpathlineto{\pgfqpoint{4.220248in}{2.481186in}}%
\pgfpathlineto{\pgfqpoint{4.265796in}{2.488901in}}%
\pgfpathlineto{\pgfqpoint{4.311344in}{2.496239in}}%
\pgfpathlineto{\pgfqpoint{4.356891in}{2.503220in}}%
\pgfpathlineto{\pgfqpoint{4.402439in}{2.509860in}}%
\pgfpathlineto{\pgfqpoint{4.447986in}{2.516176in}}%
\pgfpathlineto{\pgfqpoint{4.493534in}{2.522184in}}%
\pgfpathlineto{\pgfqpoint{4.539081in}{2.527900in}}%
\pgfpathlineto{\pgfqpoint{4.584629in}{2.533336in}}%
\pgfpathlineto{\pgfqpoint{4.630176in}{2.538507in}}%
\pgfpathlineto{\pgfqpoint{4.675724in}{2.543427in}}%
\pgfpathlineto{\pgfqpoint{4.721271in}{2.548106in}}%
\pgfpathlineto{\pgfqpoint{4.766819in}{2.552557in}}%
\pgfpathlineto{\pgfqpoint{4.812367in}{2.556791in}}%
\pgfpathlineto{\pgfqpoint{4.857914in}{2.560818in}}%
\pgfpathlineto{\pgfqpoint{4.903462in}{2.564649in}}%
\pgfpathlineto{\pgfqpoint{4.949009in}{2.568293in}}%
\pgfpathlineto{\pgfqpoint{4.994557in}{2.571760in}}%
\pgfpathlineto{\pgfqpoint{5.040104in}{2.575057in}}%
\pgfpathlineto{\pgfqpoint{5.085652in}{2.578194in}}%
\pgfpathlineto{\pgfqpoint{5.131199in}{2.581177in}}%
\pgfpathlineto{\pgfqpoint{5.176747in}{2.584015in}}%
\pgfpathlineto{\pgfqpoint{5.222294in}{2.586715in}}%
\pgfusepath{stroke}%
\end{pgfscope}%
\begin{pgfscope}%
\pgfpathrectangle{\pgfqpoint{0.667540in}{0.524170in}}{\pgfqpoint{4.554755in}{2.644000in}}%
\pgfusepath{clip}%
\pgfsetbuttcap%
\pgfsetroundjoin%
\pgfsetlinewidth{1.505625pt}%
\definecolor{currentstroke}{rgb}{0.007843,0.619608,0.450980}%
\pgfsetstrokecolor{currentstroke}%
\pgfsetstrokeopacity{0.700000}%
\pgfsetdash{{1.500000pt}{2.475000pt}}{0.000000pt}%
\pgfpathmoveto{\pgfqpoint{0.667540in}{1.581770in}}%
\pgfpathlineto{\pgfqpoint{2.489442in}{1.581770in}}%
\pgfpathlineto{\pgfqpoint{2.489487in}{2.639370in}}%
\pgfpathlineto{\pgfqpoint{5.222294in}{2.639370in}}%
\pgfusepath{stroke}%
\end{pgfscope}%
\begin{pgfscope}%
\pgfpathrectangle{\pgfqpoint{0.667540in}{0.524170in}}{\pgfqpoint{4.554755in}{2.644000in}}%
\pgfusepath{clip}%
\pgfsetrectcap%
\pgfsetroundjoin%
\pgfsetlinewidth{1.505625pt}%
\definecolor{currentstroke}{rgb}{0.835294,0.368627,0.000000}%
\pgfsetstrokecolor{currentstroke}%
\pgfsetdash{}{0pt}%
\pgfpathmoveto{\pgfqpoint{0.667540in}{1.581770in}}%
\pgfpathlineto{\pgfqpoint{0.713087in}{1.581770in}}%
\pgfpathlineto{\pgfqpoint{0.758635in}{1.581770in}}%
\pgfpathlineto{\pgfqpoint{0.804183in}{1.581770in}}%
\pgfpathlineto{\pgfqpoint{0.849730in}{1.581770in}}%
\pgfpathlineto{\pgfqpoint{0.895278in}{1.581770in}}%
\pgfpathlineto{\pgfqpoint{0.940825in}{1.581770in}}%
\pgfpathlineto{\pgfqpoint{0.986373in}{1.581770in}}%
\pgfpathlineto{\pgfqpoint{1.031920in}{1.581770in}}%
\pgfpathlineto{\pgfqpoint{1.077468in}{1.581770in}}%
\pgfpathlineto{\pgfqpoint{1.123015in}{1.581770in}}%
\pgfpathlineto{\pgfqpoint{1.168563in}{1.581770in}}%
\pgfpathlineto{\pgfqpoint{1.214110in}{1.581770in}}%
\pgfpathlineto{\pgfqpoint{1.259658in}{1.581770in}}%
\pgfpathlineto{\pgfqpoint{1.305206in}{1.581770in}}%
\pgfpathlineto{\pgfqpoint{1.350753in}{1.581770in}}%
\pgfpathlineto{\pgfqpoint{1.396301in}{1.581770in}}%
\pgfpathlineto{\pgfqpoint{1.441848in}{1.581770in}}%
\pgfpathlineto{\pgfqpoint{1.487396in}{1.581770in}}%
\pgfpathlineto{\pgfqpoint{1.532943in}{1.581770in}}%
\pgfpathlineto{\pgfqpoint{1.578491in}{1.581770in}}%
\pgfpathlineto{\pgfqpoint{1.624038in}{1.581770in}}%
\pgfpathlineto{\pgfqpoint{1.669586in}{1.581770in}}%
\pgfpathlineto{\pgfqpoint{1.715133in}{1.581770in}}%
\pgfpathlineto{\pgfqpoint{1.760681in}{1.581770in}}%
\pgfpathlineto{\pgfqpoint{1.806229in}{1.581770in}}%
\pgfpathlineto{\pgfqpoint{1.851776in}{1.581770in}}%
\pgfpathlineto{\pgfqpoint{1.897324in}{1.581770in}}%
\pgfpathlineto{\pgfqpoint{1.942871in}{1.581770in}}%
\pgfpathlineto{\pgfqpoint{1.988419in}{1.581770in}}%
\pgfpathlineto{\pgfqpoint{2.033966in}{1.581770in}}%
\pgfpathlineto{\pgfqpoint{2.079514in}{1.581770in}}%
\pgfpathlineto{\pgfqpoint{2.125061in}{1.581770in}}%
\pgfpathlineto{\pgfqpoint{2.170609in}{1.581770in}}%
\pgfpathlineto{\pgfqpoint{2.216156in}{1.581770in}}%
\pgfpathlineto{\pgfqpoint{2.261704in}{1.581770in}}%
\pgfpathlineto{\pgfqpoint{2.307252in}{1.581770in}}%
\pgfpathlineto{\pgfqpoint{2.352799in}{1.581770in}}%
\pgfpathlineto{\pgfqpoint{2.398347in}{1.581770in}}%
\pgfpathlineto{\pgfqpoint{2.443894in}{1.581770in}}%
\pgfpathlineto{\pgfqpoint{2.489442in}{1.581770in}}%
\pgfpathlineto{\pgfqpoint{2.534989in}{1.633350in}}%
\pgfpathlineto{\pgfqpoint{2.580537in}{1.682414in}}%
\pgfpathlineto{\pgfqpoint{2.626084in}{1.729085in}}%
\pgfpathlineto{\pgfqpoint{2.671632in}{1.773480in}}%
\pgfpathlineto{\pgfqpoint{2.717179in}{1.815710in}}%
\pgfpathlineto{\pgfqpoint{2.762727in}{1.855880in}}%
\pgfpathlineto{\pgfqpoint{2.808275in}{1.894092in}}%
\pgfpathlineto{\pgfqpoint{2.853822in}{1.930439in}}%
\pgfpathlineto{\pgfqpoint{2.899370in}{1.965014in}}%
\pgfpathlineto{\pgfqpoint{2.944917in}{1.997903in}}%
\pgfpathlineto{\pgfqpoint{2.990465in}{2.029188in}}%
\pgfpathlineto{\pgfqpoint{3.036012in}{2.058946in}}%
\pgfpathlineto{\pgfqpoint{3.081560in}{2.087254in}}%
\pgfpathlineto{\pgfqpoint{3.127107in}{2.114181in}}%
\pgfpathlineto{\pgfqpoint{3.172655in}{2.139795in}}%
\pgfpathlineto{\pgfqpoint{3.218202in}{2.164159in}}%
\pgfpathlineto{\pgfqpoint{3.263750in}{2.187336in}}%
\pgfpathlineto{\pgfqpoint{3.309298in}{2.209382in}}%
\pgfpathlineto{\pgfqpoint{3.354845in}{2.230352in}}%
\pgfpathlineto{\pgfqpoint{3.400393in}{2.250300in}}%
\pgfpathlineto{\pgfqpoint{3.445940in}{2.269275in}}%
\pgfpathlineto{\pgfqpoint{3.491488in}{2.287325in}}%
\pgfpathlineto{\pgfqpoint{3.537035in}{2.304495in}}%
\pgfpathlineto{\pgfqpoint{3.582583in}{2.320827in}}%
\pgfpathlineto{\pgfqpoint{3.628130in}{2.336362in}}%
\pgfpathlineto{\pgfqpoint{3.673678in}{2.351140in}}%
\pgfpathlineto{\pgfqpoint{3.719225in}{2.365197in}}%
\pgfpathlineto{\pgfqpoint{3.764773in}{2.378569in}}%
\pgfpathlineto{\pgfqpoint{3.810321in}{2.391288in}}%
\pgfpathlineto{\pgfqpoint{3.855868in}{2.403387in}}%
\pgfpathlineto{\pgfqpoint{3.901416in}{2.414896in}}%
\pgfpathlineto{\pgfqpoint{3.946963in}{2.425844in}}%
\pgfpathlineto{\pgfqpoint{3.992511in}{2.436258in}}%
\pgfpathlineto{\pgfqpoint{4.038058in}{2.446163in}}%
\pgfpathlineto{\pgfqpoint{4.083606in}{2.455586in}}%
\pgfpathlineto{\pgfqpoint{4.129153in}{2.464549in}}%
\pgfpathlineto{\pgfqpoint{4.174701in}{2.473075in}}%
\pgfpathlineto{\pgfqpoint{4.220248in}{2.481186in}}%
\pgfpathlineto{\pgfqpoint{4.265796in}{2.488900in}}%
\pgfpathlineto{\pgfqpoint{4.311344in}{2.496239in}}%
\pgfpathlineto{\pgfqpoint{4.356891in}{2.503219in}}%
\pgfpathlineto{\pgfqpoint{4.402439in}{2.509860in}}%
\pgfpathlineto{\pgfqpoint{4.447986in}{2.516176in}}%
\pgfpathlineto{\pgfqpoint{4.493534in}{2.522184in}}%
\pgfpathlineto{\pgfqpoint{4.539081in}{2.527899in}}%
\pgfpathlineto{\pgfqpoint{4.584629in}{2.533336in}}%
\pgfpathlineto{\pgfqpoint{4.630176in}{2.538507in}}%
\pgfpathlineto{\pgfqpoint{4.675724in}{2.543426in}}%
\pgfpathlineto{\pgfqpoint{4.721271in}{2.548105in}}%
\pgfpathlineto{\pgfqpoint{4.766819in}{2.552557in}}%
\pgfpathlineto{\pgfqpoint{4.812367in}{2.556790in}}%
\pgfpathlineto{\pgfqpoint{4.857914in}{2.560818in}}%
\pgfpathlineto{\pgfqpoint{4.903462in}{2.564649in}}%
\pgfpathlineto{\pgfqpoint{4.949009in}{2.568293in}}%
\pgfpathlineto{\pgfqpoint{4.994557in}{2.571760in}}%
\pgfpathlineto{\pgfqpoint{5.040104in}{2.575057in}}%
\pgfpathlineto{\pgfqpoint{5.085652in}{2.578194in}}%
\pgfpathlineto{\pgfqpoint{5.131199in}{2.581177in}}%
\pgfpathlineto{\pgfqpoint{5.176747in}{2.584015in}}%
\pgfpathlineto{\pgfqpoint{5.222294in}{2.586715in}}%
\pgfusepath{stroke}%
\end{pgfscope}%
\begin{pgfscope}%
\pgfsetrectcap%
\pgfsetmiterjoin%
\pgfsetlinewidth{0.803000pt}%
\definecolor{currentstroke}{rgb}{0.000000,0.000000,0.000000}%
\pgfsetstrokecolor{currentstroke}%
\pgfsetdash{}{0pt}%
\pgfpathmoveto{\pgfqpoint{0.667540in}{0.524170in}}%
\pgfpathlineto{\pgfqpoint{0.667540in}{3.168170in}}%
\pgfusepath{stroke}%
\end{pgfscope}%
\begin{pgfscope}%
\pgfsetrectcap%
\pgfsetmiterjoin%
\pgfsetlinewidth{0.803000pt}%
\definecolor{currentstroke}{rgb}{0.000000,0.000000,0.000000}%
\pgfsetstrokecolor{currentstroke}%
\pgfsetdash{}{0pt}%
\pgfpathmoveto{\pgfqpoint{5.222294in}{0.524170in}}%
\pgfpathlineto{\pgfqpoint{5.222294in}{3.168170in}}%
\pgfusepath{stroke}%
\end{pgfscope}%
\begin{pgfscope}%
\pgfsetrectcap%
\pgfsetmiterjoin%
\pgfsetlinewidth{0.803000pt}%
\definecolor{currentstroke}{rgb}{0.000000,0.000000,0.000000}%
\pgfsetstrokecolor{currentstroke}%
\pgfsetdash{}{0pt}%
\pgfpathmoveto{\pgfqpoint{0.667540in}{0.524170in}}%
\pgfpathlineto{\pgfqpoint{5.222294in}{0.524170in}}%
\pgfusepath{stroke}%
\end{pgfscope}%
\begin{pgfscope}%
\pgfsetrectcap%
\pgfsetmiterjoin%
\pgfsetlinewidth{0.803000pt}%
\definecolor{currentstroke}{rgb}{0.000000,0.000000,0.000000}%
\pgfsetstrokecolor{currentstroke}%
\pgfsetdash{}{0pt}%
\pgfpathmoveto{\pgfqpoint{0.667540in}{3.168170in}}%
\pgfpathlineto{\pgfqpoint{5.222294in}{3.168170in}}%
\pgfusepath{stroke}%
\end{pgfscope}%
\begin{pgfscope}%
\pgfsetbuttcap%
\pgfsetmiterjoin%
\definecolor{currentfill}{rgb}{1.000000,1.000000,1.000000}%
\pgfsetfillcolor{currentfill}%
\pgfsetfillopacity{0.800000}%
\pgfsetlinewidth{1.003750pt}%
\definecolor{currentstroke}{rgb}{0.800000,0.800000,0.800000}%
\pgfsetstrokecolor{currentstroke}%
\pgfsetstrokeopacity{0.800000}%
\pgfsetdash{}{0pt}%
\pgfpathmoveto{\pgfqpoint{0.745318in}{2.542112in}}%
\pgfpathlineto{\pgfqpoint{1.643290in}{2.542112in}}%
\pgfpathquadraticcurveto{\pgfqpoint{1.665513in}{2.542112in}}{\pgfqpoint{1.665513in}{2.564334in}}%
\pgfpathlineto{\pgfqpoint{1.665513in}{3.090392in}}%
\pgfpathquadraticcurveto{\pgfqpoint{1.665513in}{3.112614in}}{\pgfqpoint{1.643290in}{3.112614in}}%
\pgfpathlineto{\pgfqpoint{0.745318in}{3.112614in}}%
\pgfpathquadraticcurveto{\pgfqpoint{0.723095in}{3.112614in}}{\pgfqpoint{0.723095in}{3.090392in}}%
\pgfpathlineto{\pgfqpoint{0.723095in}{2.564334in}}%
\pgfpathquadraticcurveto{\pgfqpoint{0.723095in}{2.542112in}}{\pgfqpoint{0.745318in}{2.542112in}}%
\pgfpathlineto{\pgfqpoint{0.745318in}{2.542112in}}%
\pgfpathclose%
\pgfusepath{stroke,fill}%
\end{pgfscope}%
\begin{pgfscope}%
\pgfsetbuttcap%
\pgfsetroundjoin%
\pgfsetlinewidth{1.505625pt}%
\definecolor{currentstroke}{rgb}{0.003922,0.450980,0.698039}%
\pgfsetstrokecolor{currentstroke}%
\pgfsetstrokeopacity{0.700000}%
\pgfsetdash{{5.550000pt}{2.400000pt}}{0.000000pt}%
\pgfpathmoveto{\pgfqpoint{0.767540in}{2.980334in}}%
\pgfpathlineto{\pgfqpoint{0.878651in}{2.980334in}}%
\pgfpathlineto{\pgfqpoint{0.989762in}{2.980334in}}%
\pgfusepath{stroke}%
\end{pgfscope}%
\begin{pgfscope}%
\definecolor{textcolor}{rgb}{0.000000,0.000000,0.000000}%
\pgfsetstrokecolor{textcolor}%
\pgfsetfillcolor{textcolor}%
\pgftext[x=1.078651in,y=2.941445in,left,base]{\color{textcolor}\rmfamily\fontsize{8.000000}{9.600000}\selectfont \(\displaystyle 1-e^{-\frac{t-\theta}{\tau} }\)}%
\end{pgfscope}%
\begin{pgfscope}%
\pgfsetbuttcap%
\pgfsetroundjoin%
\pgfsetlinewidth{1.505625pt}%
\definecolor{currentstroke}{rgb}{0.007843,0.619608,0.450980}%
\pgfsetstrokecolor{currentstroke}%
\pgfsetstrokeopacity{0.700000}%
\pgfsetdash{{1.500000pt}{2.475000pt}}{0.000000pt}%
\pgfpathmoveto{\pgfqpoint{0.767540in}{2.819889in}}%
\pgfpathlineto{\pgfqpoint{0.878651in}{2.819889in}}%
\pgfpathlineto{\pgfqpoint{0.989762in}{2.819889in}}%
\pgfusepath{stroke}%
\end{pgfscope}%
\begin{pgfscope}%
\definecolor{textcolor}{rgb}{0.000000,0.000000,0.000000}%
\pgfsetstrokecolor{textcolor}%
\pgfsetfillcolor{textcolor}%
\pgftext[x=1.078651in,y=2.781001in,left,base]{\color{textcolor}\rmfamily\fontsize{8.000000}{9.600000}\selectfont \(\displaystyle H(t- \theta)\)}%
\end{pgfscope}%
\begin{pgfscope}%
\pgfsetrectcap%
\pgfsetroundjoin%
\pgfsetlinewidth{1.505625pt}%
\definecolor{currentstroke}{rgb}{0.835294,0.368627,0.000000}%
\pgfsetstrokecolor{currentstroke}%
\pgfsetdash{}{0pt}%
\pgfpathmoveto{\pgfqpoint{0.767540in}{2.653223in}}%
\pgfpathlineto{\pgfqpoint{0.878651in}{2.653223in}}%
\pgfpathlineto{\pgfqpoint{0.989762in}{2.653223in}}%
\pgfusepath{stroke}%
\end{pgfscope}%
\begin{pgfscope}%
\definecolor{textcolor}{rgb}{0.000000,0.000000,0.000000}%
\pgfsetstrokecolor{textcolor}%
\pgfsetfillcolor{textcolor}%
\pgftext[x=1.078651in,y=2.614334in,left,base]{\color{textcolor}\rmfamily\fontsize{8.000000}{9.600000}\selectfont \(\displaystyle y(t)\)}%
\end{pgfscope}%
\end{pgfpicture}%
\makeatother%
\endgroup%

    \caption{Time domain plot of a first-order plus dead time model, showing induvidual components of the model and the composite function $y(t)$. Model parameters: $K= \Delta u = 1$, $\tau=2$, $\theta=4$.}
    \label{fig:fopdt}
\end{figure}

%\cite{pi_stabilization_of_fopdt_systems}



% https://apmonitor.com/pdc/index.php/Main/FirstOrderPlusDeadTime

\subsection{PID tuning rules}

%\subsubsection{SIMC}
We use $\tau_c = \tau$ as suggested by \cite{simc,smic2} for “\textit{tightest possible subject to maintaining smooth control}“.

\clearpage
\section{Allan Deviation}
The Allan variance \cite{adev} $\sigma_A^2(\tau)$ is a two-sample variance and used as a measure of stability. The Allan deviation $\sigma_A(\tau)$ is the square root of the variance. Originally, the Allan variance was used to quantify the performance of oscillators, namely the frequency stability, but it can be used evaluate any quantity. In order to define the Allan variance, a few terms need to be defined first. A single measurement value of the time series $y(t)$ can be written as
\begin{equation}
    \bar y_k(t) = \frac{1}{\tau} \int_{t_{k}}^{t_{k}+\tau} y(t)\,dt . \label{eqn:allan_variance_measurement}
\end{equation}
This is the $k$-th measurement with a measurement time or integration time $\tau$. The latter term is frequently used for DMMs. $t_k$ is the start of the $k$-th sampling inverval including the dead time $\theta$
\begin{equation}
    t_{k+1} = t_k + T
\end{equation}
with
\begin{equation}
    T \coloneqq \tau + \theta .
\end{equation}

\begin{figure}[hb]
    \centering
    \scalebox{1}{%
        \import{figures/}{allan_variance_definitions.tex}
    }% scalebox
    \caption{Measurement interval according to equation \ref{eqn:allan_variance_measurement}}
    \label{fig:allan_variance_definitions}
\end{figure}

Using this, the deviation over $N$ samples is defined as \cite{adev,psd_to_adev}
\begin{equation}
    \sigma_y^2(N,T,\tau) = \left\langle \frac{1}{N-1} \left(\sum _{k=0}^{N-1}\bar y_k^2(t)-\frac{1}{N}\left(\sum _{k=0}^{N-1} \bar y_k(t)\right)^2\right)\right\rangle
\end{equation}
The $\langle \; \rangle$ denotes the (infinite time) average over all measurands $y_k$ or expected value.

The Allan variance is a special case of this definition with zero dead-time ($\theta=0$) and only 2 samples:
\begin{align}
    \sigma_A^2(\tau) &= \sigma_A^2(N=2,T=\tau,\tau) \label{eqn:allan_coefficients}\\
    &= \left\langle \frac{\left(\bar y_{k+1} - \bar y_k \right)^2}{2} \right\rangle
\end{align}
It can be shown \cite{psd_to_adev}, that \ref{eqn:adev_estimator} is indeed more useful than $\sigma_A^2(N\to\infty,T=\tau,\tau)$, because $\sigma_A^2(N=2,T=\tau,\tau)$ converges for processes, that do not have a convergent $\sigma_A^2(N\to\infty,T=\tau,\tau)$.

In practice, no experiment can take an infinite number of samples, so typically the Allan variance is estimated using a number of samples $m$:
\begin{equation}
    \sigma_A^2(\tau) \approx \frac1 m \sum_{k=1}^m \frac{\left(\bar y_{k+1} - \bar y_{k} \right)^2}{2} \label{eqn:adev_estimator}
\end{equation}
This esitmation can lead to artifacts in the results as discussed later. In order to derive the Allan variance from a set of data points, the different values of $\tau$ are usually obtained by averaging over a number of samples since there is no dead time.

Additionally, the Allan variance is mathematically related to the two-sided power spectral density $S_y(f)$ \cite{psd_to_adev}:
\begin{equation}
    \sigma_A^2(\tau) = 2 \int_0^\infty S_y(f) \frac{\sin^4\left( \pi f \tau \right)}{(\pi f \tau)^2}\,df \label{eqn:psd_to_adev}
\end{equation}

and therefore all processes, that can be observed in the power spectral density can also be seen in the allan deviation. The inverse transform, however, is not always possible as shown by \citeauthor{inverse_adev} \cite{inverse_adev}.

Distinguishing different noise processes using the Allan deviation will be elaborated in the next section.

\subsection{Identifying Noise in Allan Deviation Plots}
It was already mentioned by \citeauthor{adev} in \cite{adev}, that types of noise, whose spectral density follows a power law
\begin{equation}
    S(f) = h_{\alpha} \cdot f^\alpha \label{eqn:power_law}
\end{equation}
can be easily identified in the Allan deviation plot. The constant $h_\alpha$ is called the power (intensity) coefficient. The most common types of noise encountered in experimental data and their representations can be found in table \ref{tab:adev_alpha}. Since those types of noise is present in any measurement or electronic device, it warants a further discussion to understand their root causes and ideas to minimize them. While not a type of noise, linear drift can also be easily identified in the Allan deviation plot. It is therefore included in table \ref{tab:adev_alpha} as well.
% TODO: put in (4) from Generation-Recombination Noise, Allan Variance, and Low-Frequency Gain Instabilities in Microwave Amplifiers

%TODO: $\sigma_A(N=2,T=\tau+\theta,\tau)$
\begin{table}[ht]
    \centering
    \begin{tabular}{lcc}
        \toprule
        Amplitude noise type& Power-law coefficient $\alpha$& Allan deviation $\sigma_A$\\
        \midrule
            White noise & $0$& $\propto \tau^{-1/2}$ \cite{adev_noise_types}\\
            Flicker noise& $-1$& $\propto \tau^0$ \cite{adev_noise_types}\\
            Random walk noise& $-2$& $\propto \tau^{1/2}$ \cite{adev_noise_types}\\
            Burst noise& $0 \textrm{ and } -\!2$& $\propto \tau^{1/2} \textrm{ and } \tau^{-1/2}$\\
            Drift & --& $\propto \tau^1$ \cite{adev_drift}\\
        \bottomrule
    \end{tabular}
    \caption{Power law representations using the Allan variance.}
    \label{tab:adev_alpha}
\end{table}

In order to arrive at a good understanding of the features seen in an Allan deviation plot, this section will provide the reader with examples of each type of noise and the corresponding time domain, power spectral density and Allan deviation plot. Since a complete overview is not available in current literature, all required mathematical descriptions and simulation tools will be discussed here. The simulations were done using Python and the source code is linked to in the discussions.

\clearpage
\minisec{White Noise}
White noise is probably the most common type of noise found in measurement data. Johnson noise found in resistors, caused by the random fluctuation of the charge carriers, is one example of mostly white noise up to bandwidth of \qty{100}{\MHz}, from where on quantum corrections are required \cite{nist_johnson_noise}. Amplifiers also tend to have a white noise spectrum at higher frequencies. For these reasons, it typically makes up for a considerabe amount of noise in a measurement, unless one measures at very low frequencies. White noise is a series of uncorrelated random events and therefore characterised by a uniform power spectral density, which means there is the same power in a given bandwidth at all frequencies. Another one of its important and often used properties is, that the variance $\sigma_{x+y}^2$ of two uncorrelated variables $x$ and $y$ adds:
\begin{equation}
    \sigma_{x+y}^2  = \sigma_x^2 + \sigma_y^2 + \underbrace{2\,\mathrm{Cov}(x,y)}_{\text{uncorrelated}\, =\, 0}\ = \sigma_x^2 + \sigma_y^2 \label{eqn:adding_white_noise}
\end{equation}

This allows simple addition rules of variances from different sources, but it must be stressed here, that this property is only valid for uncorrelated sources like white noise, although it is usually incorrectly applied to all measurements in disreagard of the dominant noise present, which ufortunately obscures rather than clarifies the uncertainties involved.

In order to demonstrate the effect of white noise in Allan deviation plots, it was simulated using the excellent \textit{AllanTools} library \cite{allantools}. The noise generator chosen in the AllanTools library is based on the work of \citeauthor{noise_generation} \cite{noise_generation}. The full Python program code is published online \cite{}. For better comparison, all noise densities are normalized to give an Allan deviation of $\sigma_A(\tau_0)=1$, with $\tau_0$ being the smallest time interval between measurements.

Figure \ref{fig:white_noise_simulated} shows a sample of white noise in three different forms. Figure \ref{fig:white_noise_time} is the time series representation. From this sample, the power spectral density was calculated and is shown in figure \ref{fig:white_noise_psd}. The dashed line shows the expectation value of the power spectral density and the Allan deviation.

\begin{figure}[ht]
    \centering
    \begin{subfigure}{0.32\linewidth}
        \scalebox{0.75}{%
            %% Creator: Matplotlib, PGF backend
%%
%% To include the figure in your LaTeX document, write
%%   \input{<filename>.pgf}
%%
%% Make sure the required packages are loaded in your preamble
%%   \usepackage{pgf}
%%
%% Also ensure that all the required font packages are loaded; for instance,
%% the lmodern package is sometimes necessary when using math font.
%%   \usepackage{lmodern}
%%
%% Figures using additional raster images can only be included by \input if
%% they are in the same directory as the main LaTeX file. For loading figures
%% from other directories you can use the `import` package
%%   \usepackage{import}
%%
%% and then include the figures with
%%   \import{<path to file>}{<filename>.pgf}
%%
%% Matplotlib used the following preamble
%%   \usepackage{siunitx}
%%   \usepackage{fontspec}
%%
\begingroup%
\makeatletter%
\begin{pgfpicture}%
\pgfpathrectangle{\pgfpointorigin}{\pgfqpoint{2.440945in}{1.830709in}}%
\pgfusepath{use as bounding box, clip}%
\begin{pgfscope}%
\pgfsetbuttcap%
\pgfsetmiterjoin%
\definecolor{currentfill}{rgb}{1.000000,1.000000,1.000000}%
\pgfsetfillcolor{currentfill}%
\pgfsetlinewidth{0.000000pt}%
\definecolor{currentstroke}{rgb}{1.000000,1.000000,1.000000}%
\pgfsetstrokecolor{currentstroke}%
\pgfsetdash{}{0pt}%
\pgfpathmoveto{\pgfqpoint{0.000000in}{0.000000in}}%
\pgfpathlineto{\pgfqpoint{2.440945in}{0.000000in}}%
\pgfpathlineto{\pgfqpoint{2.440945in}{1.830709in}}%
\pgfpathlineto{\pgfqpoint{0.000000in}{1.830709in}}%
\pgfpathlineto{\pgfqpoint{0.000000in}{0.000000in}}%
\pgfpathclose%
\pgfusepath{fill}%
\end{pgfscope}%
\begin{pgfscope}%
\pgfsetbuttcap%
\pgfsetmiterjoin%
\definecolor{currentfill}{rgb}{1.000000,1.000000,1.000000}%
\pgfsetfillcolor{currentfill}%
\pgfsetlinewidth{0.000000pt}%
\definecolor{currentstroke}{rgb}{0.000000,0.000000,0.000000}%
\pgfsetstrokecolor{currentstroke}%
\pgfsetstrokeopacity{0.000000}%
\pgfsetdash{}{0pt}%
\pgfpathmoveto{\pgfqpoint{0.563510in}{0.416447in}}%
\pgfpathlineto{\pgfqpoint{2.399275in}{0.416447in}}%
\pgfpathlineto{\pgfqpoint{2.399275in}{1.789039in}}%
\pgfpathlineto{\pgfqpoint{0.563510in}{1.789039in}}%
\pgfpathlineto{\pgfqpoint{0.563510in}{0.416447in}}%
\pgfpathclose%
\pgfusepath{fill}%
\end{pgfscope}%
\begin{pgfscope}%
\pgfpathrectangle{\pgfqpoint{0.563510in}{0.416447in}}{\pgfqpoint{1.835765in}{1.372591in}}%
\pgfusepath{clip}%
\pgfsetrectcap%
\pgfsetroundjoin%
\pgfsetlinewidth{0.803000pt}%
\definecolor{currentstroke}{rgb}{0.450000,0.450000,0.450000}%
\pgfsetstrokecolor{currentstroke}%
\pgfsetdash{}{0pt}%
\pgfpathmoveto{\pgfqpoint{0.646954in}{0.416447in}}%
\pgfpathlineto{\pgfqpoint{0.646954in}{1.789039in}}%
\pgfusepath{stroke}%
\end{pgfscope}%
\begin{pgfscope}%
\pgfsetbuttcap%
\pgfsetroundjoin%
\definecolor{currentfill}{rgb}{0.000000,0.000000,0.000000}%
\pgfsetfillcolor{currentfill}%
\pgfsetlinewidth{0.803000pt}%
\definecolor{currentstroke}{rgb}{0.000000,0.000000,0.000000}%
\pgfsetstrokecolor{currentstroke}%
\pgfsetdash{}{0pt}%
\pgfsys@defobject{currentmarker}{\pgfqpoint{0.000000in}{-0.048611in}}{\pgfqpoint{0.000000in}{0.000000in}}{%
\pgfpathmoveto{\pgfqpoint{0.000000in}{0.000000in}}%
\pgfpathlineto{\pgfqpoint{0.000000in}{-0.048611in}}%
\pgfusepath{stroke,fill}%
}%
\begin{pgfscope}%
\pgfsys@transformshift{0.646954in}{0.416447in}%
\pgfsys@useobject{currentmarker}{}%
\end{pgfscope}%
\end{pgfscope}%
\begin{pgfscope}%
\definecolor{textcolor}{rgb}{0.000000,0.000000,0.000000}%
\pgfsetstrokecolor{textcolor}%
\pgfsetfillcolor{textcolor}%
\pgftext[x=0.646954in,y=0.319225in,,top]{\color{textcolor}\rmfamily\fontsize{8.000000}{9.600000}\selectfont \(\displaystyle {0}\)}%
\end{pgfscope}%
\begin{pgfscope}%
\pgfpathrectangle{\pgfqpoint{0.563510in}{0.416447in}}{\pgfqpoint{1.835765in}{1.372591in}}%
\pgfusepath{clip}%
\pgfsetrectcap%
\pgfsetroundjoin%
\pgfsetlinewidth{0.803000pt}%
\definecolor{currentstroke}{rgb}{0.450000,0.450000,0.450000}%
\pgfsetstrokecolor{currentstroke}%
\pgfsetdash{}{0pt}%
\pgfpathmoveto{\pgfqpoint{1.156317in}{0.416447in}}%
\pgfpathlineto{\pgfqpoint{1.156317in}{1.789039in}}%
\pgfusepath{stroke}%
\end{pgfscope}%
\begin{pgfscope}%
\pgfsetbuttcap%
\pgfsetroundjoin%
\definecolor{currentfill}{rgb}{0.000000,0.000000,0.000000}%
\pgfsetfillcolor{currentfill}%
\pgfsetlinewidth{0.803000pt}%
\definecolor{currentstroke}{rgb}{0.000000,0.000000,0.000000}%
\pgfsetstrokecolor{currentstroke}%
\pgfsetdash{}{0pt}%
\pgfsys@defobject{currentmarker}{\pgfqpoint{0.000000in}{-0.048611in}}{\pgfqpoint{0.000000in}{0.000000in}}{%
\pgfpathmoveto{\pgfqpoint{0.000000in}{0.000000in}}%
\pgfpathlineto{\pgfqpoint{0.000000in}{-0.048611in}}%
\pgfusepath{stroke,fill}%
}%
\begin{pgfscope}%
\pgfsys@transformshift{1.156317in}{0.416447in}%
\pgfsys@useobject{currentmarker}{}%
\end{pgfscope}%
\end{pgfscope}%
\begin{pgfscope}%
\definecolor{textcolor}{rgb}{0.000000,0.000000,0.000000}%
\pgfsetstrokecolor{textcolor}%
\pgfsetfillcolor{textcolor}%
\pgftext[x=1.156317in,y=0.319225in,,top]{\color{textcolor}\rmfamily\fontsize{8.000000}{9.600000}\selectfont \(\displaystyle {5000}\)}%
\end{pgfscope}%
\begin{pgfscope}%
\pgfpathrectangle{\pgfqpoint{0.563510in}{0.416447in}}{\pgfqpoint{1.835765in}{1.372591in}}%
\pgfusepath{clip}%
\pgfsetrectcap%
\pgfsetroundjoin%
\pgfsetlinewidth{0.803000pt}%
\definecolor{currentstroke}{rgb}{0.450000,0.450000,0.450000}%
\pgfsetstrokecolor{currentstroke}%
\pgfsetdash{}{0pt}%
\pgfpathmoveto{\pgfqpoint{1.665680in}{0.416447in}}%
\pgfpathlineto{\pgfqpoint{1.665680in}{1.789039in}}%
\pgfusepath{stroke}%
\end{pgfscope}%
\begin{pgfscope}%
\pgfsetbuttcap%
\pgfsetroundjoin%
\definecolor{currentfill}{rgb}{0.000000,0.000000,0.000000}%
\pgfsetfillcolor{currentfill}%
\pgfsetlinewidth{0.803000pt}%
\definecolor{currentstroke}{rgb}{0.000000,0.000000,0.000000}%
\pgfsetstrokecolor{currentstroke}%
\pgfsetdash{}{0pt}%
\pgfsys@defobject{currentmarker}{\pgfqpoint{0.000000in}{-0.048611in}}{\pgfqpoint{0.000000in}{0.000000in}}{%
\pgfpathmoveto{\pgfqpoint{0.000000in}{0.000000in}}%
\pgfpathlineto{\pgfqpoint{0.000000in}{-0.048611in}}%
\pgfusepath{stroke,fill}%
}%
\begin{pgfscope}%
\pgfsys@transformshift{1.665680in}{0.416447in}%
\pgfsys@useobject{currentmarker}{}%
\end{pgfscope}%
\end{pgfscope}%
\begin{pgfscope}%
\definecolor{textcolor}{rgb}{0.000000,0.000000,0.000000}%
\pgfsetstrokecolor{textcolor}%
\pgfsetfillcolor{textcolor}%
\pgftext[x=1.665680in,y=0.319225in,,top]{\color{textcolor}\rmfamily\fontsize{8.000000}{9.600000}\selectfont \(\displaystyle {10000}\)}%
\end{pgfscope}%
\begin{pgfscope}%
\pgfpathrectangle{\pgfqpoint{0.563510in}{0.416447in}}{\pgfqpoint{1.835765in}{1.372591in}}%
\pgfusepath{clip}%
\pgfsetrectcap%
\pgfsetroundjoin%
\pgfsetlinewidth{0.803000pt}%
\definecolor{currentstroke}{rgb}{0.450000,0.450000,0.450000}%
\pgfsetstrokecolor{currentstroke}%
\pgfsetdash{}{0pt}%
\pgfpathmoveto{\pgfqpoint{2.175043in}{0.416447in}}%
\pgfpathlineto{\pgfqpoint{2.175043in}{1.789039in}}%
\pgfusepath{stroke}%
\end{pgfscope}%
\begin{pgfscope}%
\pgfsetbuttcap%
\pgfsetroundjoin%
\definecolor{currentfill}{rgb}{0.000000,0.000000,0.000000}%
\pgfsetfillcolor{currentfill}%
\pgfsetlinewidth{0.803000pt}%
\definecolor{currentstroke}{rgb}{0.000000,0.000000,0.000000}%
\pgfsetstrokecolor{currentstroke}%
\pgfsetdash{}{0pt}%
\pgfsys@defobject{currentmarker}{\pgfqpoint{0.000000in}{-0.048611in}}{\pgfqpoint{0.000000in}{0.000000in}}{%
\pgfpathmoveto{\pgfqpoint{0.000000in}{0.000000in}}%
\pgfpathlineto{\pgfqpoint{0.000000in}{-0.048611in}}%
\pgfusepath{stroke,fill}%
}%
\begin{pgfscope}%
\pgfsys@transformshift{2.175043in}{0.416447in}%
\pgfsys@useobject{currentmarker}{}%
\end{pgfscope}%
\end{pgfscope}%
\begin{pgfscope}%
\definecolor{textcolor}{rgb}{0.000000,0.000000,0.000000}%
\pgfsetstrokecolor{textcolor}%
\pgfsetfillcolor{textcolor}%
\pgftext[x=2.175043in,y=0.319225in,,top]{\color{textcolor}\rmfamily\fontsize{8.000000}{9.600000}\selectfont \(\displaystyle {15000}\)}%
\end{pgfscope}%
\begin{pgfscope}%
\definecolor{textcolor}{rgb}{0.000000,0.000000,0.000000}%
\pgfsetstrokecolor{textcolor}%
\pgfsetfillcolor{textcolor}%
\pgftext[x=1.481392in,y=0.165003in,,top]{\color{textcolor}\rmfamily\fontsize{10.000000}{12.000000}\selectfont Time in \unit{\second}}%
\end{pgfscope}%
\begin{pgfscope}%
\pgfpathrectangle{\pgfqpoint{0.563510in}{0.416447in}}{\pgfqpoint{1.835765in}{1.372591in}}%
\pgfusepath{clip}%
\pgfsetrectcap%
\pgfsetroundjoin%
\pgfsetlinewidth{0.803000pt}%
\definecolor{currentstroke}{rgb}{0.450000,0.450000,0.450000}%
\pgfsetstrokecolor{currentstroke}%
\pgfsetdash{}{0pt}%
\pgfpathmoveto{\pgfqpoint{0.563510in}{0.574823in}}%
\pgfpathlineto{\pgfqpoint{2.399275in}{0.574823in}}%
\pgfusepath{stroke}%
\end{pgfscope}%
\begin{pgfscope}%
\pgfsetbuttcap%
\pgfsetroundjoin%
\definecolor{currentfill}{rgb}{0.000000,0.000000,0.000000}%
\pgfsetfillcolor{currentfill}%
\pgfsetlinewidth{0.803000pt}%
\definecolor{currentstroke}{rgb}{0.000000,0.000000,0.000000}%
\pgfsetstrokecolor{currentstroke}%
\pgfsetdash{}{0pt}%
\pgfsys@defobject{currentmarker}{\pgfqpoint{-0.048611in}{0.000000in}}{\pgfqpoint{-0.000000in}{0.000000in}}{%
\pgfpathmoveto{\pgfqpoint{-0.000000in}{0.000000in}}%
\pgfpathlineto{\pgfqpoint{-0.048611in}{0.000000in}}%
\pgfusepath{stroke,fill}%
}%
\begin{pgfscope}%
\pgfsys@transformshift{0.563510in}{0.574823in}%
\pgfsys@useobject{currentmarker}{}%
\end{pgfscope}%
\end{pgfscope}%
\begin{pgfscope}%
\definecolor{textcolor}{rgb}{0.000000,0.000000,0.000000}%
\pgfsetstrokecolor{textcolor}%
\pgfsetfillcolor{textcolor}%
\pgftext[x=0.223614in, y=0.536268in, left, base]{\color{textcolor}\rmfamily\fontsize{8.000000}{9.600000}\selectfont \(\displaystyle {\ensuremath{-}5.0}\)}%
\end{pgfscope}%
\begin{pgfscope}%
\pgfpathrectangle{\pgfqpoint{0.563510in}{0.416447in}}{\pgfqpoint{1.835765in}{1.372591in}}%
\pgfusepath{clip}%
\pgfsetrectcap%
\pgfsetroundjoin%
\pgfsetlinewidth{0.803000pt}%
\definecolor{currentstroke}{rgb}{0.450000,0.450000,0.450000}%
\pgfsetstrokecolor{currentstroke}%
\pgfsetdash{}{0pt}%
\pgfpathmoveto{\pgfqpoint{0.563510in}{0.838783in}}%
\pgfpathlineto{\pgfqpoint{2.399275in}{0.838783in}}%
\pgfusepath{stroke}%
\end{pgfscope}%
\begin{pgfscope}%
\pgfsetbuttcap%
\pgfsetroundjoin%
\definecolor{currentfill}{rgb}{0.000000,0.000000,0.000000}%
\pgfsetfillcolor{currentfill}%
\pgfsetlinewidth{0.803000pt}%
\definecolor{currentstroke}{rgb}{0.000000,0.000000,0.000000}%
\pgfsetstrokecolor{currentstroke}%
\pgfsetdash{}{0pt}%
\pgfsys@defobject{currentmarker}{\pgfqpoint{-0.048611in}{0.000000in}}{\pgfqpoint{-0.000000in}{0.000000in}}{%
\pgfpathmoveto{\pgfqpoint{-0.000000in}{0.000000in}}%
\pgfpathlineto{\pgfqpoint{-0.048611in}{0.000000in}}%
\pgfusepath{stroke,fill}%
}%
\begin{pgfscope}%
\pgfsys@transformshift{0.563510in}{0.838783in}%
\pgfsys@useobject{currentmarker}{}%
\end{pgfscope}%
\end{pgfscope}%
\begin{pgfscope}%
\definecolor{textcolor}{rgb}{0.000000,0.000000,0.000000}%
\pgfsetstrokecolor{textcolor}%
\pgfsetfillcolor{textcolor}%
\pgftext[x=0.223614in, y=0.800228in, left, base]{\color{textcolor}\rmfamily\fontsize{8.000000}{9.600000}\selectfont \(\displaystyle {\ensuremath{-}2.5}\)}%
\end{pgfscope}%
\begin{pgfscope}%
\pgfpathrectangle{\pgfqpoint{0.563510in}{0.416447in}}{\pgfqpoint{1.835765in}{1.372591in}}%
\pgfusepath{clip}%
\pgfsetrectcap%
\pgfsetroundjoin%
\pgfsetlinewidth{0.803000pt}%
\definecolor{currentstroke}{rgb}{0.450000,0.450000,0.450000}%
\pgfsetstrokecolor{currentstroke}%
\pgfsetdash{}{0pt}%
\pgfpathmoveto{\pgfqpoint{0.563510in}{1.102743in}}%
\pgfpathlineto{\pgfqpoint{2.399275in}{1.102743in}}%
\pgfusepath{stroke}%
\end{pgfscope}%
\begin{pgfscope}%
\pgfsetbuttcap%
\pgfsetroundjoin%
\definecolor{currentfill}{rgb}{0.000000,0.000000,0.000000}%
\pgfsetfillcolor{currentfill}%
\pgfsetlinewidth{0.803000pt}%
\definecolor{currentstroke}{rgb}{0.000000,0.000000,0.000000}%
\pgfsetstrokecolor{currentstroke}%
\pgfsetdash{}{0pt}%
\pgfsys@defobject{currentmarker}{\pgfqpoint{-0.048611in}{0.000000in}}{\pgfqpoint{-0.000000in}{0.000000in}}{%
\pgfpathmoveto{\pgfqpoint{-0.000000in}{0.000000in}}%
\pgfpathlineto{\pgfqpoint{-0.048611in}{0.000000in}}%
\pgfusepath{stroke,fill}%
}%
\begin{pgfscope}%
\pgfsys@transformshift{0.563510in}{1.102743in}%
\pgfsys@useobject{currentmarker}{}%
\end{pgfscope}%
\end{pgfscope}%
\begin{pgfscope}%
\definecolor{textcolor}{rgb}{0.000000,0.000000,0.000000}%
\pgfsetstrokecolor{textcolor}%
\pgfsetfillcolor{textcolor}%
\pgftext[x=0.315437in, y=1.064188in, left, base]{\color{textcolor}\rmfamily\fontsize{8.000000}{9.600000}\selectfont \(\displaystyle {0.0}\)}%
\end{pgfscope}%
\begin{pgfscope}%
\pgfpathrectangle{\pgfqpoint{0.563510in}{0.416447in}}{\pgfqpoint{1.835765in}{1.372591in}}%
\pgfusepath{clip}%
\pgfsetrectcap%
\pgfsetroundjoin%
\pgfsetlinewidth{0.803000pt}%
\definecolor{currentstroke}{rgb}{0.450000,0.450000,0.450000}%
\pgfsetstrokecolor{currentstroke}%
\pgfsetdash{}{0pt}%
\pgfpathmoveto{\pgfqpoint{0.563510in}{1.366703in}}%
\pgfpathlineto{\pgfqpoint{2.399275in}{1.366703in}}%
\pgfusepath{stroke}%
\end{pgfscope}%
\begin{pgfscope}%
\pgfsetbuttcap%
\pgfsetroundjoin%
\definecolor{currentfill}{rgb}{0.000000,0.000000,0.000000}%
\pgfsetfillcolor{currentfill}%
\pgfsetlinewidth{0.803000pt}%
\definecolor{currentstroke}{rgb}{0.000000,0.000000,0.000000}%
\pgfsetstrokecolor{currentstroke}%
\pgfsetdash{}{0pt}%
\pgfsys@defobject{currentmarker}{\pgfqpoint{-0.048611in}{0.000000in}}{\pgfqpoint{-0.000000in}{0.000000in}}{%
\pgfpathmoveto{\pgfqpoint{-0.000000in}{0.000000in}}%
\pgfpathlineto{\pgfqpoint{-0.048611in}{0.000000in}}%
\pgfusepath{stroke,fill}%
}%
\begin{pgfscope}%
\pgfsys@transformshift{0.563510in}{1.366703in}%
\pgfsys@useobject{currentmarker}{}%
\end{pgfscope}%
\end{pgfscope}%
\begin{pgfscope}%
\definecolor{textcolor}{rgb}{0.000000,0.000000,0.000000}%
\pgfsetstrokecolor{textcolor}%
\pgfsetfillcolor{textcolor}%
\pgftext[x=0.315437in, y=1.328147in, left, base]{\color{textcolor}\rmfamily\fontsize{8.000000}{9.600000}\selectfont \(\displaystyle {2.5}\)}%
\end{pgfscope}%
\begin{pgfscope}%
\pgfpathrectangle{\pgfqpoint{0.563510in}{0.416447in}}{\pgfqpoint{1.835765in}{1.372591in}}%
\pgfusepath{clip}%
\pgfsetrectcap%
\pgfsetroundjoin%
\pgfsetlinewidth{0.803000pt}%
\definecolor{currentstroke}{rgb}{0.450000,0.450000,0.450000}%
\pgfsetstrokecolor{currentstroke}%
\pgfsetdash{}{0pt}%
\pgfpathmoveto{\pgfqpoint{0.563510in}{1.630663in}}%
\pgfpathlineto{\pgfqpoint{2.399275in}{1.630663in}}%
\pgfusepath{stroke}%
\end{pgfscope}%
\begin{pgfscope}%
\pgfsetbuttcap%
\pgfsetroundjoin%
\definecolor{currentfill}{rgb}{0.000000,0.000000,0.000000}%
\pgfsetfillcolor{currentfill}%
\pgfsetlinewidth{0.803000pt}%
\definecolor{currentstroke}{rgb}{0.000000,0.000000,0.000000}%
\pgfsetstrokecolor{currentstroke}%
\pgfsetdash{}{0pt}%
\pgfsys@defobject{currentmarker}{\pgfqpoint{-0.048611in}{0.000000in}}{\pgfqpoint{-0.000000in}{0.000000in}}{%
\pgfpathmoveto{\pgfqpoint{-0.000000in}{0.000000in}}%
\pgfpathlineto{\pgfqpoint{-0.048611in}{0.000000in}}%
\pgfusepath{stroke,fill}%
}%
\begin{pgfscope}%
\pgfsys@transformshift{0.563510in}{1.630663in}%
\pgfsys@useobject{currentmarker}{}%
\end{pgfscope}%
\end{pgfscope}%
\begin{pgfscope}%
\definecolor{textcolor}{rgb}{0.000000,0.000000,0.000000}%
\pgfsetstrokecolor{textcolor}%
\pgfsetfillcolor{textcolor}%
\pgftext[x=0.315437in, y=1.592107in, left, base]{\color{textcolor}\rmfamily\fontsize{8.000000}{9.600000}\selectfont \(\displaystyle {5.0}\)}%
\end{pgfscope}%
\begin{pgfscope}%
\definecolor{textcolor}{rgb}{0.000000,0.000000,0.000000}%
\pgfsetstrokecolor{textcolor}%
\pgfsetfillcolor{textcolor}%
\pgftext[x=0.168059in,y=1.102743in,,bottom,rotate=90.000000]{\color{textcolor}\rmfamily\fontsize{10.000000}{12.000000}\selectfont Ampl. in arb. unit}%
\end{pgfscope}%
\begin{pgfscope}%
\pgfpathrectangle{\pgfqpoint{0.563510in}{0.416447in}}{\pgfqpoint{1.835765in}{1.372591in}}%
\pgfusepath{clip}%
\pgfsetrectcap%
\pgfsetroundjoin%
\pgfsetlinewidth{1.505625pt}%
\definecolor{currentstroke}{rgb}{0.000000,0.447059,0.698039}%
\pgfsetstrokecolor{currentstroke}%
\pgfsetdash{}{0pt}%
\pgfpathmoveto{\pgfqpoint{0.646954in}{1.088145in}}%
\pgfpathlineto{\pgfqpoint{0.647463in}{1.269483in}}%
\pgfpathlineto{\pgfqpoint{0.648787in}{0.953627in}}%
\pgfpathlineto{\pgfqpoint{0.651843in}{0.916594in}}%
\pgfpathlineto{\pgfqpoint{0.652353in}{1.211600in}}%
\pgfpathlineto{\pgfqpoint{0.654390in}{0.826140in}}%
\pgfpathlineto{\pgfqpoint{0.656122in}{1.205016in}}%
\pgfpathlineto{\pgfqpoint{0.658058in}{0.900152in}}%
\pgfpathlineto{\pgfqpoint{0.659586in}{1.334020in}}%
\pgfpathlineto{\pgfqpoint{0.661318in}{0.933019in}}%
\pgfpathlineto{\pgfqpoint{0.662744in}{1.299739in}}%
\pgfpathlineto{\pgfqpoint{0.665087in}{1.389949in}}%
\pgfpathlineto{\pgfqpoint{0.666106in}{0.942800in}}%
\pgfpathlineto{\pgfqpoint{0.668143in}{1.509530in}}%
\pgfpathlineto{\pgfqpoint{0.669569in}{0.932466in}}%
\pgfpathlineto{\pgfqpoint{0.672524in}{1.326809in}}%
\pgfpathlineto{\pgfqpoint{0.673542in}{0.760517in}}%
\pgfpathlineto{\pgfqpoint{0.675784in}{1.327957in}}%
\pgfpathlineto{\pgfqpoint{0.676599in}{1.040516in}}%
\pgfpathlineto{\pgfqpoint{0.679757in}{1.323666in}}%
\pgfpathlineto{\pgfqpoint{0.679960in}{0.974544in}}%
\pgfpathlineto{\pgfqpoint{0.682609in}{1.258517in}}%
\pgfpathlineto{\pgfqpoint{0.683424in}{1.015401in}}%
\pgfpathlineto{\pgfqpoint{0.685360in}{1.333951in}}%
\pgfpathlineto{\pgfqpoint{0.687193in}{0.921864in}}%
\pgfpathlineto{\pgfqpoint{0.689638in}{1.320325in}}%
\pgfpathlineto{\pgfqpoint{0.690759in}{0.887433in}}%
\pgfpathlineto{\pgfqpoint{0.693713in}{1.310187in}}%
\pgfpathlineto{\pgfqpoint{0.694834in}{0.859697in}}%
\pgfpathlineto{\pgfqpoint{0.695547in}{1.427823in}}%
\pgfpathlineto{\pgfqpoint{0.697686in}{0.956742in}}%
\pgfpathlineto{\pgfqpoint{0.699316in}{1.182500in}}%
\pgfpathlineto{\pgfqpoint{0.702270in}{0.841777in}}%
\pgfpathlineto{\pgfqpoint{0.704002in}{1.321872in}}%
\pgfpathlineto{\pgfqpoint{0.704512in}{0.887550in}}%
\pgfpathlineto{\pgfqpoint{0.706243in}{1.342492in}}%
\pgfpathlineto{\pgfqpoint{0.708790in}{0.914939in}}%
\pgfpathlineto{\pgfqpoint{0.709402in}{1.360975in}}%
\pgfpathlineto{\pgfqpoint{0.712661in}{0.817995in}}%
\pgfpathlineto{\pgfqpoint{0.713476in}{1.374449in}}%
\pgfpathlineto{\pgfqpoint{0.714903in}{0.822843in}}%
\pgfpathlineto{\pgfqpoint{0.717246in}{1.225150in}}%
\pgfpathlineto{\pgfqpoint{0.718978in}{0.882925in}}%
\pgfpathlineto{\pgfqpoint{0.720506in}{1.181220in}}%
\pgfpathlineto{\pgfqpoint{0.722237in}{0.908376in}}%
\pgfpathlineto{\pgfqpoint{0.723766in}{1.380680in}}%
\pgfpathlineto{\pgfqpoint{0.725803in}{0.936953in}}%
\pgfpathlineto{\pgfqpoint{0.726822in}{1.216856in}}%
\pgfpathlineto{\pgfqpoint{0.729878in}{0.945892in}}%
\pgfpathlineto{\pgfqpoint{0.730591in}{1.345470in}}%
\pgfpathlineto{\pgfqpoint{0.732119in}{0.997333in}}%
\pgfpathlineto{\pgfqpoint{0.734768in}{1.292359in}}%
\pgfpathlineto{\pgfqpoint{0.735685in}{0.957109in}}%
\pgfpathlineto{\pgfqpoint{0.737926in}{1.248677in}}%
\pgfpathlineto{\pgfqpoint{0.739148in}{0.959164in}}%
\pgfpathlineto{\pgfqpoint{0.741695in}{1.301363in}}%
\pgfpathlineto{\pgfqpoint{0.742612in}{0.967564in}}%
\pgfpathlineto{\pgfqpoint{0.745363in}{1.292893in}}%
\pgfpathlineto{\pgfqpoint{0.746381in}{0.846820in}}%
\pgfpathlineto{\pgfqpoint{0.748317in}{1.292550in}}%
\pgfpathlineto{\pgfqpoint{0.749641in}{1.046231in}}%
\pgfpathlineto{\pgfqpoint{0.751067in}{1.328932in}}%
\pgfpathlineto{\pgfqpoint{0.753512in}{0.936301in}}%
\pgfpathlineto{\pgfqpoint{0.754531in}{1.226564in}}%
\pgfpathlineto{\pgfqpoint{0.756467in}{0.938393in}}%
\pgfpathlineto{\pgfqpoint{0.759013in}{0.796945in}}%
\pgfpathlineto{\pgfqpoint{0.759828in}{1.304834in}}%
\pgfpathlineto{\pgfqpoint{0.761458in}{0.981707in}}%
\pgfpathlineto{\pgfqpoint{0.764718in}{1.340564in}}%
\pgfpathlineto{\pgfqpoint{0.765024in}{0.838846in}}%
\pgfpathlineto{\pgfqpoint{0.767163in}{1.249716in}}%
\pgfpathlineto{\pgfqpoint{0.768488in}{0.934549in}}%
\pgfpathlineto{\pgfqpoint{0.770016in}{1.287470in}}%
\pgfpathlineto{\pgfqpoint{0.772257in}{0.958114in}}%
\pgfpathlineto{\pgfqpoint{0.774804in}{1.279340in}}%
\pgfpathlineto{\pgfqpoint{0.775721in}{0.953478in}}%
\pgfpathlineto{\pgfqpoint{0.778369in}{1.365964in}}%
\pgfpathlineto{\pgfqpoint{0.779286in}{1.032203in}}%
\pgfpathlineto{\pgfqpoint{0.781833in}{0.894166in}}%
\pgfpathlineto{\pgfqpoint{0.782342in}{1.264366in}}%
\pgfpathlineto{\pgfqpoint{0.783972in}{0.833090in}}%
\pgfpathlineto{\pgfqpoint{0.785806in}{1.229082in}}%
\pgfpathlineto{\pgfqpoint{0.788658in}{1.315324in}}%
\pgfpathlineto{\pgfqpoint{0.789066in}{0.960508in}}%
\pgfpathlineto{\pgfqpoint{0.790798in}{1.331333in}}%
\pgfpathlineto{\pgfqpoint{0.793854in}{0.965818in}}%
\pgfpathlineto{\pgfqpoint{0.794873in}{1.377439in}}%
\pgfpathlineto{\pgfqpoint{0.796401in}{0.976477in}}%
\pgfpathlineto{\pgfqpoint{0.798744in}{1.372848in}}%
\pgfpathlineto{\pgfqpoint{0.800985in}{0.905667in}}%
\pgfpathlineto{\pgfqpoint{0.801596in}{1.298506in}}%
\pgfpathlineto{\pgfqpoint{0.803634in}{0.799478in}}%
\pgfpathlineto{\pgfqpoint{0.804652in}{1.232288in}}%
\pgfpathlineto{\pgfqpoint{0.806588in}{0.937763in}}%
\pgfpathlineto{\pgfqpoint{0.809542in}{1.319982in}}%
\pgfpathlineto{\pgfqpoint{0.810969in}{0.937695in}}%
\pgfpathlineto{\pgfqpoint{0.812293in}{1.300533in}}%
\pgfpathlineto{\pgfqpoint{0.813515in}{1.001898in}}%
\pgfpathlineto{\pgfqpoint{0.816164in}{0.878709in}}%
\pgfpathlineto{\pgfqpoint{0.817387in}{1.273301in}}%
\pgfpathlineto{\pgfqpoint{0.819322in}{0.960629in}}%
\pgfpathlineto{\pgfqpoint{0.820748in}{1.303340in}}%
\pgfpathlineto{\pgfqpoint{0.823601in}{0.932087in}}%
\pgfpathlineto{\pgfqpoint{0.824008in}{1.283356in}}%
\pgfpathlineto{\pgfqpoint{0.825434in}{0.909458in}}%
\pgfpathlineto{\pgfqpoint{0.827268in}{1.281713in}}%
\pgfpathlineto{\pgfqpoint{0.829204in}{0.904755in}}%
\pgfpathlineto{\pgfqpoint{0.831445in}{1.272968in}}%
\pgfpathlineto{\pgfqpoint{0.832871in}{0.987614in}}%
\pgfpathlineto{\pgfqpoint{0.835418in}{1.290187in}}%
\pgfpathlineto{\pgfqpoint{0.836640in}{0.918126in}}%
\pgfpathlineto{\pgfqpoint{0.837863in}{1.250105in}}%
\pgfpathlineto{\pgfqpoint{0.840104in}{1.271535in}}%
\pgfpathlineto{\pgfqpoint{0.841021in}{0.936884in}}%
\pgfpathlineto{\pgfqpoint{0.843873in}{1.381943in}}%
\pgfpathlineto{\pgfqpoint{0.845503in}{0.915539in}}%
\pgfpathlineto{\pgfqpoint{0.846216in}{1.434039in}}%
\pgfpathlineto{\pgfqpoint{0.848661in}{0.957078in}}%
\pgfpathlineto{\pgfqpoint{0.849782in}{1.228273in}}%
\pgfpathlineto{\pgfqpoint{0.853042in}{0.783931in}}%
\pgfpathlineto{\pgfqpoint{0.853246in}{1.292848in}}%
\pgfpathlineto{\pgfqpoint{0.856098in}{0.960641in}}%
\pgfpathlineto{\pgfqpoint{0.857524in}{1.252962in}}%
\pgfpathlineto{\pgfqpoint{0.859052in}{0.860303in}}%
\pgfpathlineto{\pgfqpoint{0.860173in}{1.212457in}}%
\pgfpathlineto{\pgfqpoint{0.862618in}{0.832202in}}%
\pgfpathlineto{\pgfqpoint{0.863535in}{1.220398in}}%
\pgfpathlineto{\pgfqpoint{0.865470in}{0.933713in}}%
\pgfpathlineto{\pgfqpoint{0.868425in}{1.323365in}}%
\pgfpathlineto{\pgfqpoint{0.869342in}{0.869206in}}%
\pgfpathlineto{\pgfqpoint{0.870972in}{1.293253in}}%
\pgfpathlineto{\pgfqpoint{0.873824in}{0.905666in}}%
\pgfpathlineto{\pgfqpoint{0.875250in}{1.311423in}}%
\pgfpathlineto{\pgfqpoint{0.876880in}{0.827780in}}%
\pgfpathlineto{\pgfqpoint{0.877593in}{1.228653in}}%
\pgfpathlineto{\pgfqpoint{0.879121in}{0.920526in}}%
\pgfpathlineto{\pgfqpoint{0.881668in}{1.431101in}}%
\pgfpathlineto{\pgfqpoint{0.883094in}{0.986584in}}%
\pgfpathlineto{\pgfqpoint{0.885641in}{0.875385in}}%
\pgfpathlineto{\pgfqpoint{0.886252in}{1.214655in}}%
\pgfpathlineto{\pgfqpoint{0.887984in}{0.922474in}}%
\pgfpathlineto{\pgfqpoint{0.890225in}{1.212049in}}%
\pgfpathlineto{\pgfqpoint{0.891957in}{0.965528in}}%
\pgfpathlineto{\pgfqpoint{0.893485in}{1.226233in}}%
\pgfpathlineto{\pgfqpoint{0.895319in}{0.933114in}}%
\pgfpathlineto{\pgfqpoint{0.896949in}{1.259772in}}%
\pgfpathlineto{\pgfqpoint{0.898783in}{0.956927in}}%
\pgfpathlineto{\pgfqpoint{0.900820in}{1.255167in}}%
\pgfpathlineto{\pgfqpoint{0.903265in}{0.868421in}}%
\pgfpathlineto{\pgfqpoint{0.903673in}{1.417939in}}%
\pgfpathlineto{\pgfqpoint{0.905404in}{0.971215in}}%
\pgfpathlineto{\pgfqpoint{0.906831in}{1.300287in}}%
\pgfpathlineto{\pgfqpoint{0.908766in}{0.949795in}}%
\pgfpathlineto{\pgfqpoint{0.910702in}{1.233397in}}%
\pgfpathlineto{\pgfqpoint{0.912434in}{0.869871in}}%
\pgfpathlineto{\pgfqpoint{0.914064in}{1.251665in}}%
\pgfpathlineto{\pgfqpoint{0.916509in}{0.954338in}}%
\pgfpathlineto{\pgfqpoint{0.917324in}{1.192967in}}%
\pgfpathlineto{\pgfqpoint{0.919361in}{0.895521in}}%
\pgfpathlineto{\pgfqpoint{0.922112in}{1.315207in}}%
\pgfpathlineto{\pgfqpoint{0.923436in}{0.881014in}}%
\pgfpathlineto{\pgfqpoint{0.924353in}{1.309961in}}%
\pgfpathlineto{\pgfqpoint{0.926390in}{0.835662in}}%
\pgfpathlineto{\pgfqpoint{0.928224in}{1.197278in}}%
\pgfpathlineto{\pgfqpoint{0.929345in}{0.939234in}}%
\pgfpathlineto{\pgfqpoint{0.932197in}{1.387084in}}%
\pgfpathlineto{\pgfqpoint{0.934031in}{0.899212in}}%
\pgfpathlineto{\pgfqpoint{0.934642in}{1.282669in}}%
\pgfpathlineto{\pgfqpoint{0.936272in}{0.942610in}}%
\pgfpathlineto{\pgfqpoint{0.938004in}{1.227303in}}%
\pgfpathlineto{\pgfqpoint{0.940652in}{0.812651in}}%
\pgfpathlineto{\pgfqpoint{0.941773in}{1.517291in}}%
\pgfpathlineto{\pgfqpoint{0.943505in}{1.025566in}}%
\pgfpathlineto{\pgfqpoint{0.945135in}{1.269367in}}%
\pgfpathlineto{\pgfqpoint{0.947070in}{0.786927in}}%
\pgfpathlineto{\pgfqpoint{0.949719in}{1.312568in}}%
\pgfpathlineto{\pgfqpoint{0.950840in}{0.928014in}}%
\pgfpathlineto{\pgfqpoint{0.952775in}{1.302053in}}%
\pgfpathlineto{\pgfqpoint{0.954201in}{0.803713in}}%
\pgfpathlineto{\pgfqpoint{0.955730in}{1.150626in}}%
\pgfpathlineto{\pgfqpoint{0.957054in}{0.886302in}}%
\pgfpathlineto{\pgfqpoint{0.959193in}{1.288036in}}%
\pgfpathlineto{\pgfqpoint{0.960823in}{0.895052in}}%
\pgfpathlineto{\pgfqpoint{0.962861in}{1.276902in}}%
\pgfpathlineto{\pgfqpoint{0.965306in}{0.901146in}}%
\pgfpathlineto{\pgfqpoint{0.966019in}{1.260585in}}%
\pgfpathlineto{\pgfqpoint{0.967954in}{0.795812in}}%
\pgfpathlineto{\pgfqpoint{0.970094in}{1.184679in}}%
\pgfpathlineto{\pgfqpoint{0.972335in}{0.904244in}}%
\pgfpathlineto{\pgfqpoint{0.973150in}{1.237771in}}%
\pgfpathlineto{\pgfqpoint{0.974474in}{0.927830in}}%
\pgfpathlineto{\pgfqpoint{0.977021in}{1.405600in}}%
\pgfpathlineto{\pgfqpoint{0.977836in}{0.906579in}}%
\pgfpathlineto{\pgfqpoint{0.980485in}{1.245856in}}%
\pgfpathlineto{\pgfqpoint{0.982115in}{0.840070in}}%
\pgfpathlineto{\pgfqpoint{0.984254in}{1.322996in}}%
\pgfpathlineto{\pgfqpoint{0.985273in}{0.939198in}}%
\pgfpathlineto{\pgfqpoint{0.987004in}{1.292410in}}%
\pgfpathlineto{\pgfqpoint{0.988838in}{0.877983in}}%
\pgfpathlineto{\pgfqpoint{0.990468in}{1.293554in}}%
\pgfpathlineto{\pgfqpoint{0.992200in}{0.936238in}}%
\pgfpathlineto{\pgfqpoint{0.994136in}{1.318859in}}%
\pgfpathlineto{\pgfqpoint{0.995664in}{0.915225in}}%
\pgfpathlineto{\pgfqpoint{0.997599in}{1.270477in}}%
\pgfpathlineto{\pgfqpoint{0.998720in}{0.793440in}}%
\pgfpathlineto{\pgfqpoint{1.001267in}{1.249119in}}%
\pgfpathlineto{\pgfqpoint{1.002285in}{0.934062in}}%
\pgfpathlineto{\pgfqpoint{1.004628in}{1.283507in}}%
\pgfpathlineto{\pgfqpoint{1.005545in}{0.839494in}}%
\pgfpathlineto{\pgfqpoint{1.008805in}{1.367923in}}%
\pgfpathlineto{\pgfqpoint{1.009722in}{0.871523in}}%
\pgfpathlineto{\pgfqpoint{1.011556in}{1.327085in}}%
\pgfpathlineto{\pgfqpoint{1.012880in}{0.946538in}}%
\pgfpathlineto{\pgfqpoint{1.014306in}{1.216983in}}%
\pgfpathlineto{\pgfqpoint{1.016446in}{0.939848in}}%
\pgfpathlineto{\pgfqpoint{1.018483in}{1.185402in}}%
\pgfpathlineto{\pgfqpoint{1.019400in}{0.882445in}}%
\pgfpathlineto{\pgfqpoint{1.022354in}{1.365139in}}%
\pgfpathlineto{\pgfqpoint{1.023373in}{1.042376in}}%
\pgfpathlineto{\pgfqpoint{1.025410in}{1.445162in}}%
\pgfpathlineto{\pgfqpoint{1.026735in}{0.858196in}}%
\pgfpathlineto{\pgfqpoint{1.029587in}{1.318354in}}%
\pgfpathlineto{\pgfqpoint{1.030097in}{1.016902in}}%
\pgfpathlineto{\pgfqpoint{1.032847in}{1.240348in}}%
\pgfpathlineto{\pgfqpoint{1.034681in}{0.894901in}}%
\pgfpathlineto{\pgfqpoint{1.035088in}{1.147109in}}%
\pgfpathlineto{\pgfqpoint{1.037330in}{0.952500in}}%
\pgfpathlineto{\pgfqpoint{1.038450in}{1.272605in}}%
\pgfpathlineto{\pgfqpoint{1.041099in}{0.955958in}}%
\pgfpathlineto{\pgfqpoint{1.043442in}{0.901289in}}%
\pgfpathlineto{\pgfqpoint{1.043748in}{1.249870in}}%
\pgfpathlineto{\pgfqpoint{1.045581in}{0.944471in}}%
\pgfpathlineto{\pgfqpoint{1.047313in}{1.264079in}}%
\pgfpathlineto{\pgfqpoint{1.049147in}{0.967698in}}%
\pgfpathlineto{\pgfqpoint{1.051082in}{1.315524in}}%
\pgfpathlineto{\pgfqpoint{1.052814in}{0.882907in}}%
\pgfpathlineto{\pgfqpoint{1.054139in}{1.276257in}}%
\pgfpathlineto{\pgfqpoint{1.056583in}{0.907218in}}%
\pgfpathlineto{\pgfqpoint{1.059130in}{1.399925in}}%
\pgfpathlineto{\pgfqpoint{1.059436in}{1.024816in}}%
\pgfpathlineto{\pgfqpoint{1.062186in}{0.890788in}}%
\pgfpathlineto{\pgfqpoint{1.062696in}{1.279984in}}%
\pgfpathlineto{\pgfqpoint{1.064530in}{0.944344in}}%
\pgfpathlineto{\pgfqpoint{1.067280in}{1.292660in}}%
\pgfpathlineto{\pgfqpoint{1.069114in}{0.824980in}}%
\pgfpathlineto{\pgfqpoint{1.069725in}{1.221819in}}%
\pgfpathlineto{\pgfqpoint{1.071661in}{0.939848in}}%
\pgfpathlineto{\pgfqpoint{1.073291in}{1.350768in}}%
\pgfpathlineto{\pgfqpoint{1.075430in}{0.801340in}}%
\pgfpathlineto{\pgfqpoint{1.077365in}{1.300060in}}%
\pgfpathlineto{\pgfqpoint{1.078792in}{0.908268in}}%
\pgfpathlineto{\pgfqpoint{1.080524in}{1.253147in}}%
\pgfpathlineto{\pgfqpoint{1.082255in}{0.889274in}}%
\pgfpathlineto{\pgfqpoint{1.083682in}{1.262083in}}%
\pgfpathlineto{\pgfqpoint{1.085413in}{0.934033in}}%
\pgfpathlineto{\pgfqpoint{1.087756in}{1.282176in}}%
\pgfpathlineto{\pgfqpoint{1.088775in}{0.884881in}}%
\pgfpathlineto{\pgfqpoint{1.091526in}{1.324252in}}%
\pgfpathlineto{\pgfqpoint{1.092646in}{0.852334in}}%
\pgfpathlineto{\pgfqpoint{1.094174in}{1.189314in}}%
\pgfpathlineto{\pgfqpoint{1.096416in}{0.838988in}}%
\pgfpathlineto{\pgfqpoint{1.097333in}{1.235294in}}%
\pgfpathlineto{\pgfqpoint{1.099574in}{0.898253in}}%
\pgfpathlineto{\pgfqpoint{1.101306in}{1.298006in}}%
\pgfpathlineto{\pgfqpoint{1.103037in}{0.955507in}}%
\pgfpathlineto{\pgfqpoint{1.104769in}{1.279847in}}%
\pgfpathlineto{\pgfqpoint{1.106297in}{0.977416in}}%
\pgfpathlineto{\pgfqpoint{1.107724in}{1.227346in}}%
\pgfpathlineto{\pgfqpoint{1.109557in}{0.910466in}}%
\pgfpathlineto{\pgfqpoint{1.111493in}{1.284246in}}%
\pgfpathlineto{\pgfqpoint{1.113021in}{0.978475in}}%
\pgfpathlineto{\pgfqpoint{1.115670in}{1.275380in}}%
\pgfpathlineto{\pgfqpoint{1.117198in}{0.803148in}}%
\pgfpathlineto{\pgfqpoint{1.118318in}{1.218470in}}%
\pgfpathlineto{\pgfqpoint{1.120356in}{0.932044in}}%
\pgfpathlineto{\pgfqpoint{1.121578in}{1.259214in}}%
\pgfpathlineto{\pgfqpoint{1.124431in}{0.950322in}}%
\pgfpathlineto{\pgfqpoint{1.125347in}{1.237106in}}%
\pgfpathlineto{\pgfqpoint{1.128200in}{0.767997in}}%
\pgfpathlineto{\pgfqpoint{1.128506in}{1.270903in}}%
\pgfpathlineto{\pgfqpoint{1.131358in}{0.944657in}}%
\pgfpathlineto{\pgfqpoint{1.132580in}{1.211433in}}%
\pgfpathlineto{\pgfqpoint{1.133701in}{0.767334in}}%
\pgfpathlineto{\pgfqpoint{1.136655in}{1.253444in}}%
\pgfpathlineto{\pgfqpoint{1.137165in}{1.017859in}}%
\pgfpathlineto{\pgfqpoint{1.140527in}{0.872759in}}%
\pgfpathlineto{\pgfqpoint{1.141036in}{1.371878in}}%
\pgfpathlineto{\pgfqpoint{1.142462in}{0.924869in}}%
\pgfpathlineto{\pgfqpoint{1.144703in}{1.272683in}}%
\pgfpathlineto{\pgfqpoint{1.146130in}{0.995284in}}%
\pgfpathlineto{\pgfqpoint{1.149186in}{0.871382in}}%
\pgfpathlineto{\pgfqpoint{1.149288in}{1.264760in}}%
\pgfpathlineto{\pgfqpoint{1.151631in}{0.956055in}}%
\pgfpathlineto{\pgfqpoint{1.153159in}{1.264154in}}%
\pgfpathlineto{\pgfqpoint{1.155909in}{1.431416in}}%
\pgfpathlineto{\pgfqpoint{1.156419in}{0.913152in}}%
\pgfpathlineto{\pgfqpoint{1.157947in}{1.318364in}}%
\pgfpathlineto{\pgfqpoint{1.160901in}{0.883711in}}%
\pgfpathlineto{\pgfqpoint{1.163040in}{1.272478in}}%
\pgfpathlineto{\pgfqpoint{1.163652in}{0.938521in}}%
\pgfpathlineto{\pgfqpoint{1.164874in}{1.348191in}}%
\pgfpathlineto{\pgfqpoint{1.167013in}{0.993995in}}%
\pgfpathlineto{\pgfqpoint{1.169866in}{0.935686in}}%
\pgfpathlineto{\pgfqpoint{1.171496in}{1.343034in}}%
\pgfpathlineto{\pgfqpoint{1.172616in}{0.883694in}}%
\pgfpathlineto{\pgfqpoint{1.174043in}{1.292945in}}%
\pgfpathlineto{\pgfqpoint{1.175978in}{0.907775in}}%
\pgfpathlineto{\pgfqpoint{1.176997in}{1.354214in}}%
\pgfpathlineto{\pgfqpoint{1.179136in}{0.989417in}}%
\pgfpathlineto{\pgfqpoint{1.180562in}{1.191205in}}%
\pgfpathlineto{\pgfqpoint{1.183822in}{1.375985in}}%
\pgfpathlineto{\pgfqpoint{1.183924in}{0.976071in}}%
\pgfpathlineto{\pgfqpoint{1.185758in}{1.268904in}}%
\pgfpathlineto{\pgfqpoint{1.188407in}{0.943307in}}%
\pgfpathlineto{\pgfqpoint{1.190138in}{1.228738in}}%
\pgfpathlineto{\pgfqpoint{1.191565in}{0.870327in}}%
\pgfpathlineto{\pgfqpoint{1.193806in}{1.270510in}}%
\pgfpathlineto{\pgfqpoint{1.194621in}{0.865239in}}%
\pgfpathlineto{\pgfqpoint{1.196455in}{1.187543in}}%
\pgfpathlineto{\pgfqpoint{1.198696in}{0.894113in}}%
\pgfpathlineto{\pgfqpoint{1.199613in}{1.204516in}}%
\pgfpathlineto{\pgfqpoint{1.201650in}{0.960229in}}%
\pgfpathlineto{\pgfqpoint{1.204095in}{1.262750in}}%
\pgfpathlineto{\pgfqpoint{1.205521in}{0.850999in}}%
\pgfpathlineto{\pgfqpoint{1.206438in}{1.322639in}}%
\pgfpathlineto{\pgfqpoint{1.208679in}{0.942364in}}%
\pgfpathlineto{\pgfqpoint{1.210717in}{1.366798in}}%
\pgfpathlineto{\pgfqpoint{1.211735in}{0.926294in}}%
\pgfpathlineto{\pgfqpoint{1.213875in}{1.217588in}}%
\pgfpathlineto{\pgfqpoint{1.216014in}{0.813534in}}%
\pgfpathlineto{\pgfqpoint{1.218459in}{1.289258in}}%
\pgfpathlineto{\pgfqpoint{1.218867in}{0.833129in}}%
\pgfpathlineto{\pgfqpoint{1.221006in}{1.322855in}}%
\pgfpathlineto{\pgfqpoint{1.222025in}{1.038594in}}%
\pgfpathlineto{\pgfqpoint{1.224775in}{1.400947in}}%
\pgfpathlineto{\pgfqpoint{1.225998in}{0.944997in}}%
\pgfpathlineto{\pgfqpoint{1.228443in}{1.303219in}}%
\pgfpathlineto{\pgfqpoint{1.229461in}{0.901612in}}%
\pgfpathlineto{\pgfqpoint{1.231295in}{1.251830in}}%
\pgfpathlineto{\pgfqpoint{1.232925in}{0.959574in}}%
\pgfpathlineto{\pgfqpoint{1.234657in}{1.302132in}}%
\pgfpathlineto{\pgfqpoint{1.236796in}{0.838089in}}%
\pgfpathlineto{\pgfqpoint{1.238528in}{1.388139in}}%
\pgfpathlineto{\pgfqpoint{1.239547in}{0.970782in}}%
\pgfpathlineto{\pgfqpoint{1.242399in}{1.411816in}}%
\pgfpathlineto{\pgfqpoint{1.243418in}{0.867969in}}%
\pgfpathlineto{\pgfqpoint{1.245252in}{1.236947in}}%
\pgfpathlineto{\pgfqpoint{1.246678in}{0.863484in}}%
\pgfpathlineto{\pgfqpoint{1.248715in}{1.279211in}}%
\pgfpathlineto{\pgfqpoint{1.251262in}{0.858417in}}%
\pgfpathlineto{\pgfqpoint{1.251873in}{1.338354in}}%
\pgfpathlineto{\pgfqpoint{1.254318in}{0.957511in}}%
\pgfpathlineto{\pgfqpoint{1.256254in}{1.313000in}}%
\pgfpathlineto{\pgfqpoint{1.256661in}{1.018581in}}%
\pgfpathlineto{\pgfqpoint{1.259717in}{1.332764in}}%
\pgfpathlineto{\pgfqpoint{1.260940in}{0.895212in}}%
\pgfpathlineto{\pgfqpoint{1.262366in}{1.335240in}}%
\pgfpathlineto{\pgfqpoint{1.264811in}{1.014508in}}%
\pgfpathlineto{\pgfqpoint{1.265626in}{1.283010in}}%
\pgfpathlineto{\pgfqpoint{1.267867in}{0.982025in}}%
\pgfpathlineto{\pgfqpoint{1.268886in}{1.349580in}}%
\pgfpathlineto{\pgfqpoint{1.271229in}{0.949113in}}%
\pgfpathlineto{\pgfqpoint{1.273368in}{1.302528in}}%
\pgfpathlineto{\pgfqpoint{1.274285in}{0.937544in}}%
\pgfpathlineto{\pgfqpoint{1.276730in}{1.308940in}}%
\pgfpathlineto{\pgfqpoint{1.277851in}{0.816335in}}%
\pgfpathlineto{\pgfqpoint{1.280805in}{1.301228in}}%
\pgfpathlineto{\pgfqpoint{1.281824in}{0.840069in}}%
\pgfpathlineto{\pgfqpoint{1.283759in}{1.381261in}}%
\pgfpathlineto{\pgfqpoint{1.284371in}{0.996796in}}%
\pgfpathlineto{\pgfqpoint{1.287427in}{1.255715in}}%
\pgfpathlineto{\pgfqpoint{1.289057in}{0.897135in}}%
\pgfpathlineto{\pgfqpoint{1.290279in}{1.348194in}}%
\pgfpathlineto{\pgfqpoint{1.291807in}{0.959342in}}%
\pgfpathlineto{\pgfqpoint{1.293743in}{1.396263in}}%
\pgfpathlineto{\pgfqpoint{1.295475in}{0.859745in}}%
\pgfpathlineto{\pgfqpoint{1.296799in}{1.320015in}}%
\pgfpathlineto{\pgfqpoint{1.298735in}{0.951356in}}%
\pgfpathlineto{\pgfqpoint{1.300568in}{1.258572in}}%
\pgfpathlineto{\pgfqpoint{1.301995in}{0.826310in}}%
\pgfpathlineto{\pgfqpoint{1.304032in}{1.205178in}}%
\pgfpathlineto{\pgfqpoint{1.306273in}{0.851375in}}%
\pgfpathlineto{\pgfqpoint{1.307496in}{1.366537in}}%
\pgfpathlineto{\pgfqpoint{1.309329in}{0.958090in}}%
\pgfpathlineto{\pgfqpoint{1.310654in}{1.281428in}}%
\pgfpathlineto{\pgfqpoint{1.313201in}{0.820352in}}%
\pgfpathlineto{\pgfqpoint{1.314219in}{1.224886in}}%
\pgfpathlineto{\pgfqpoint{1.317072in}{0.913423in}}%
\pgfpathlineto{\pgfqpoint{1.317581in}{1.271063in}}%
\pgfpathlineto{\pgfqpoint{1.319007in}{0.916347in}}%
\pgfpathlineto{\pgfqpoint{1.320943in}{1.352492in}}%
\pgfpathlineto{\pgfqpoint{1.322980in}{0.964200in}}%
\pgfpathlineto{\pgfqpoint{1.324814in}{1.294563in}}%
\pgfpathlineto{\pgfqpoint{1.325935in}{0.970393in}}%
\pgfpathlineto{\pgfqpoint{1.329093in}{1.221773in}}%
\pgfpathlineto{\pgfqpoint{1.329602in}{0.868370in}}%
\pgfpathlineto{\pgfqpoint{1.331232in}{1.286943in}}%
\pgfpathlineto{\pgfqpoint{1.333473in}{0.922131in}}%
\pgfpathlineto{\pgfqpoint{1.335307in}{1.320692in}}%
\pgfpathlineto{\pgfqpoint{1.336733in}{0.868601in}}%
\pgfpathlineto{\pgfqpoint{1.339178in}{1.265252in}}%
\pgfpathlineto{\pgfqpoint{1.339789in}{0.854573in}}%
\pgfpathlineto{\pgfqpoint{1.341623in}{1.313070in}}%
\pgfpathlineto{\pgfqpoint{1.343355in}{0.915809in}}%
\pgfpathlineto{\pgfqpoint{1.345902in}{1.306875in}}%
\pgfpathlineto{\pgfqpoint{1.346920in}{0.960213in}}%
\pgfpathlineto{\pgfqpoint{1.348856in}{1.475355in}}%
\pgfpathlineto{\pgfqpoint{1.350792in}{0.879978in}}%
\pgfpathlineto{\pgfqpoint{1.353440in}{1.255477in}}%
\pgfpathlineto{\pgfqpoint{1.354255in}{0.913305in}}%
\pgfpathlineto{\pgfqpoint{1.355682in}{1.384325in}}%
\pgfpathlineto{\pgfqpoint{1.357821in}{0.896685in}}%
\pgfpathlineto{\pgfqpoint{1.359043in}{1.204351in}}%
\pgfpathlineto{\pgfqpoint{1.361998in}{1.319054in}}%
\pgfpathlineto{\pgfqpoint{1.363831in}{0.861402in}}%
\pgfpathlineto{\pgfqpoint{1.364035in}{1.183661in}}%
\pgfpathlineto{\pgfqpoint{1.366989in}{0.883050in}}%
\pgfpathlineto{\pgfqpoint{1.367804in}{1.320230in}}%
\pgfpathlineto{\pgfqpoint{1.368823in}{0.903598in}}%
\pgfpathlineto{\pgfqpoint{1.371777in}{1.287852in}}%
\pgfpathlineto{\pgfqpoint{1.372694in}{0.898535in}}%
\pgfpathlineto{\pgfqpoint{1.374120in}{1.245605in}}%
\pgfpathlineto{\pgfqpoint{1.375750in}{0.884039in}}%
\pgfpathlineto{\pgfqpoint{1.377075in}{1.217325in}}%
\pgfpathlineto{\pgfqpoint{1.378908in}{0.884680in}}%
\pgfpathlineto{\pgfqpoint{1.381353in}{1.352704in}}%
\pgfpathlineto{\pgfqpoint{1.382474in}{0.862580in}}%
\pgfpathlineto{\pgfqpoint{1.383493in}{1.280921in}}%
\pgfpathlineto{\pgfqpoint{1.385225in}{0.900302in}}%
\pgfpathlineto{\pgfqpoint{1.386753in}{1.246263in}}%
\pgfpathlineto{\pgfqpoint{1.389503in}{0.790822in}}%
\pgfpathlineto{\pgfqpoint{1.390420in}{1.221155in}}%
\pgfpathlineto{\pgfqpoint{1.391846in}{0.897221in}}%
\pgfpathlineto{\pgfqpoint{1.393578in}{1.413482in}}%
\pgfpathlineto{\pgfqpoint{1.395310in}{0.966793in}}%
\pgfpathlineto{\pgfqpoint{1.396532in}{1.307949in}}%
\pgfpathlineto{\pgfqpoint{1.398977in}{1.018699in}}%
\pgfpathlineto{\pgfqpoint{1.401219in}{1.278604in}}%
\pgfpathlineto{\pgfqpoint{1.401524in}{0.904772in}}%
\pgfpathlineto{\pgfqpoint{1.404886in}{1.397556in}}%
\pgfpathlineto{\pgfqpoint{1.406720in}{0.960416in}}%
\pgfpathlineto{\pgfqpoint{1.408655in}{1.305317in}}%
\pgfpathlineto{\pgfqpoint{1.410693in}{0.809873in}}%
\pgfpathlineto{\pgfqpoint{1.412017in}{1.356655in}}%
\pgfpathlineto{\pgfqpoint{1.413036in}{0.935303in}}%
\pgfpathlineto{\pgfqpoint{1.415888in}{1.280097in}}%
\pgfpathlineto{\pgfqpoint{1.416194in}{0.941667in}}%
\pgfpathlineto{\pgfqpoint{1.418333in}{1.259403in}}%
\pgfpathlineto{\pgfqpoint{1.419658in}{0.993018in}}%
\pgfpathlineto{\pgfqpoint{1.422204in}{0.713311in}}%
\pgfpathlineto{\pgfqpoint{1.423529in}{1.390133in}}%
\pgfpathlineto{\pgfqpoint{1.425159in}{0.884356in}}%
\pgfpathlineto{\pgfqpoint{1.425872in}{1.318608in}}%
\pgfpathlineto{\pgfqpoint{1.428928in}{0.722526in}}%
\pgfpathlineto{\pgfqpoint{1.429539in}{1.265583in}}%
\pgfpathlineto{\pgfqpoint{1.430965in}{0.953984in}}%
\pgfpathlineto{\pgfqpoint{1.432697in}{1.239912in}}%
\pgfpathlineto{\pgfqpoint{1.434022in}{0.932136in}}%
\pgfpathlineto{\pgfqpoint{1.437078in}{1.347448in}}%
\pgfpathlineto{\pgfqpoint{1.437281in}{0.965684in}}%
\pgfpathlineto{\pgfqpoint{1.439726in}{1.377986in}}%
\pgfpathlineto{\pgfqpoint{1.441356in}{0.909133in}}%
\pgfpathlineto{\pgfqpoint{1.442171in}{1.284672in}}%
\pgfpathlineto{\pgfqpoint{1.444413in}{0.981352in}}%
\pgfpathlineto{\pgfqpoint{1.446552in}{1.314167in}}%
\pgfpathlineto{\pgfqpoint{1.447367in}{0.801826in}}%
\pgfpathlineto{\pgfqpoint{1.448793in}{1.405263in}}%
\pgfpathlineto{\pgfqpoint{1.450321in}{0.927978in}}%
\pgfpathlineto{\pgfqpoint{1.452053in}{1.220219in}}%
\pgfpathlineto{\pgfqpoint{1.454600in}{0.945993in}}%
\pgfpathlineto{\pgfqpoint{1.455415in}{1.358127in}}%
\pgfpathlineto{\pgfqpoint{1.457045in}{0.956005in}}%
\pgfpathlineto{\pgfqpoint{1.458777in}{1.271403in}}%
\pgfpathlineto{\pgfqpoint{1.460305in}{1.013971in}}%
\pgfpathlineto{\pgfqpoint{1.461731in}{1.322123in}}%
\pgfpathlineto{\pgfqpoint{1.464278in}{0.771366in}}%
\pgfpathlineto{\pgfqpoint{1.465398in}{1.312367in}}%
\pgfpathlineto{\pgfqpoint{1.466723in}{0.966210in}}%
\pgfpathlineto{\pgfqpoint{1.468556in}{1.225686in}}%
\pgfpathlineto{\pgfqpoint{1.470594in}{0.893192in}}%
\pgfpathlineto{\pgfqpoint{1.471511in}{1.381133in}}%
\pgfpathlineto{\pgfqpoint{1.473141in}{0.898748in}}%
\pgfpathlineto{\pgfqpoint{1.475382in}{1.296232in}}%
\pgfpathlineto{\pgfqpoint{1.477419in}{0.942267in}}%
\pgfpathlineto{\pgfqpoint{1.478642in}{1.418736in}}%
\pgfpathlineto{\pgfqpoint{1.480170in}{0.760491in}}%
\pgfpathlineto{\pgfqpoint{1.482004in}{1.335711in}}%
\pgfpathlineto{\pgfqpoint{1.482920in}{0.938092in}}%
\pgfpathlineto{\pgfqpoint{1.484958in}{1.333521in}}%
\pgfpathlineto{\pgfqpoint{1.487505in}{0.904215in}}%
\pgfpathlineto{\pgfqpoint{1.487810in}{1.253446in}}%
\pgfpathlineto{\pgfqpoint{1.490663in}{0.791995in}}%
\pgfpathlineto{\pgfqpoint{1.491478in}{1.336156in}}%
\pgfpathlineto{\pgfqpoint{1.493006in}{0.952374in}}%
\pgfpathlineto{\pgfqpoint{1.494636in}{1.293600in}}%
\pgfpathlineto{\pgfqpoint{1.496877in}{0.892637in}}%
\pgfpathlineto{\pgfqpoint{1.497896in}{1.334336in}}%
\pgfpathlineto{\pgfqpoint{1.499831in}{0.845044in}}%
\pgfpathlineto{\pgfqpoint{1.501156in}{1.274859in}}%
\pgfpathlineto{\pgfqpoint{1.503091in}{0.853052in}}%
\pgfpathlineto{\pgfqpoint{1.504721in}{1.202384in}}%
\pgfpathlineto{\pgfqpoint{1.506962in}{0.964265in}}%
\pgfpathlineto{\pgfqpoint{1.507370in}{1.336042in}}%
\pgfpathlineto{\pgfqpoint{1.509000in}{0.986097in}}%
\pgfpathlineto{\pgfqpoint{1.511139in}{1.312190in}}%
\pgfpathlineto{\pgfqpoint{1.512565in}{0.914817in}}%
\pgfpathlineto{\pgfqpoint{1.514501in}{1.238781in}}%
\pgfpathlineto{\pgfqpoint{1.515927in}{0.950416in}}%
\pgfpathlineto{\pgfqpoint{1.517353in}{1.306128in}}%
\pgfpathlineto{\pgfqpoint{1.519289in}{0.854148in}}%
\pgfpathlineto{\pgfqpoint{1.521530in}{1.299557in}}%
\pgfpathlineto{\pgfqpoint{1.522040in}{0.929965in}}%
\pgfpathlineto{\pgfqpoint{1.523669in}{1.287757in}}%
\pgfpathlineto{\pgfqpoint{1.526013in}{0.962549in}}%
\pgfpathlineto{\pgfqpoint{1.527031in}{1.273477in}}%
\pgfpathlineto{\pgfqpoint{1.528559in}{0.978766in}}%
\pgfpathlineto{\pgfqpoint{1.530189in}{1.268471in}}%
\pgfpathlineto{\pgfqpoint{1.532838in}{0.954918in}}%
\pgfpathlineto{\pgfqpoint{1.533551in}{1.286914in}}%
\pgfpathlineto{\pgfqpoint{1.535792in}{1.436153in}}%
\pgfpathlineto{\pgfqpoint{1.537117in}{0.946226in}}%
\pgfpathlineto{\pgfqpoint{1.538747in}{1.236756in}}%
\pgfpathlineto{\pgfqpoint{1.540784in}{0.957323in}}%
\pgfpathlineto{\pgfqpoint{1.543127in}{1.235377in}}%
\pgfpathlineto{\pgfqpoint{1.543840in}{0.890090in}}%
\pgfpathlineto{\pgfqpoint{1.545470in}{1.244513in}}%
\pgfpathlineto{\pgfqpoint{1.547100in}{0.960242in}}%
\pgfpathlineto{\pgfqpoint{1.548425in}{1.300087in}}%
\pgfpathlineto{\pgfqpoint{1.549749in}{0.927775in}}%
\pgfpathlineto{\pgfqpoint{1.551888in}{1.286246in}}%
\pgfpathlineto{\pgfqpoint{1.554231in}{0.940386in}}%
\pgfpathlineto{\pgfqpoint{1.555148in}{1.288885in}}%
\pgfpathlineto{\pgfqpoint{1.556371in}{0.996701in}}%
\pgfpathlineto{\pgfqpoint{1.559121in}{1.213688in}}%
\pgfpathlineto{\pgfqpoint{1.560140in}{0.939610in}}%
\pgfpathlineto{\pgfqpoint{1.562177in}{1.278716in}}%
\pgfpathlineto{\pgfqpoint{1.562992in}{0.911288in}}%
\pgfpathlineto{\pgfqpoint{1.564622in}{1.223963in}}%
\pgfpathlineto{\pgfqpoint{1.567067in}{0.978178in}}%
\pgfpathlineto{\pgfqpoint{1.568392in}{1.237709in}}%
\pgfpathlineto{\pgfqpoint{1.569614in}{0.946243in}}%
\pgfpathlineto{\pgfqpoint{1.571244in}{1.319667in}}%
\pgfpathlineto{\pgfqpoint{1.573078in}{0.980629in}}%
\pgfpathlineto{\pgfqpoint{1.574911in}{1.297874in}}%
\pgfpathlineto{\pgfqpoint{1.575930in}{0.871305in}}%
\pgfpathlineto{\pgfqpoint{1.577662in}{1.213640in}}%
\pgfpathlineto{\pgfqpoint{1.580616in}{0.970338in}}%
\pgfpathlineto{\pgfqpoint{1.581635in}{1.235527in}}%
\pgfpathlineto{\pgfqpoint{1.583876in}{0.840108in}}%
\pgfpathlineto{\pgfqpoint{1.584080in}{1.278455in}}%
\pgfpathlineto{\pgfqpoint{1.586423in}{0.941224in}}%
\pgfpathlineto{\pgfqpoint{1.587544in}{1.227015in}}%
\pgfpathlineto{\pgfqpoint{1.589989in}{1.395039in}}%
\pgfpathlineto{\pgfqpoint{1.590905in}{0.899391in}}%
\pgfpathlineto{\pgfqpoint{1.592230in}{1.223485in}}%
\pgfpathlineto{\pgfqpoint{1.595286in}{1.305835in}}%
\pgfpathlineto{\pgfqpoint{1.595795in}{0.940510in}}%
\pgfpathlineto{\pgfqpoint{1.597323in}{1.251690in}}%
\pgfpathlineto{\pgfqpoint{1.599768in}{0.874544in}}%
\pgfpathlineto{\pgfqpoint{1.600380in}{1.253090in}}%
\pgfpathlineto{\pgfqpoint{1.602010in}{0.884726in}}%
\pgfpathlineto{\pgfqpoint{1.603538in}{1.262627in}}%
\pgfpathlineto{\pgfqpoint{1.605881in}{0.973467in}}%
\pgfpathlineto{\pgfqpoint{1.607409in}{1.316742in}}%
\pgfpathlineto{\pgfqpoint{1.608529in}{0.917104in}}%
\pgfpathlineto{\pgfqpoint{1.611178in}{1.254112in}}%
\pgfpathlineto{\pgfqpoint{1.611993in}{0.878184in}}%
\pgfpathlineto{\pgfqpoint{1.613317in}{1.228935in}}%
\pgfpathlineto{\pgfqpoint{1.615660in}{0.850284in}}%
\pgfpathlineto{\pgfqpoint{1.616985in}{1.332051in}}%
\pgfpathlineto{\pgfqpoint{1.619634in}{0.885928in}}%
\pgfpathlineto{\pgfqpoint{1.620245in}{1.314806in}}%
\pgfpathlineto{\pgfqpoint{1.622078in}{0.812884in}}%
\pgfpathlineto{\pgfqpoint{1.623199in}{1.224111in}}%
\pgfpathlineto{\pgfqpoint{1.625440in}{0.806906in}}%
\pgfpathlineto{\pgfqpoint{1.627580in}{1.356312in}}%
\pgfpathlineto{\pgfqpoint{1.628395in}{0.954216in}}%
\pgfpathlineto{\pgfqpoint{1.630228in}{1.263064in}}%
\pgfpathlineto{\pgfqpoint{1.632266in}{0.933684in}}%
\pgfpathlineto{\pgfqpoint{1.634201in}{1.369910in}}%
\pgfpathlineto{\pgfqpoint{1.634914in}{0.945499in}}%
\pgfpathlineto{\pgfqpoint{1.636137in}{1.174326in}}%
\pgfpathlineto{\pgfqpoint{1.637767in}{0.957556in}}%
\pgfpathlineto{\pgfqpoint{1.640008in}{1.224413in}}%
\pgfpathlineto{\pgfqpoint{1.642147in}{0.688601in}}%
\pgfpathlineto{\pgfqpoint{1.642657in}{1.270347in}}%
\pgfpathlineto{\pgfqpoint{1.645407in}{1.346181in}}%
\pgfpathlineto{\pgfqpoint{1.646019in}{0.883959in}}%
\pgfpathlineto{\pgfqpoint{1.648667in}{1.331387in}}%
\pgfpathlineto{\pgfqpoint{1.649177in}{0.973681in}}%
\pgfpathlineto{\pgfqpoint{1.651214in}{1.319622in}}%
\pgfpathlineto{\pgfqpoint{1.652436in}{0.973104in}}%
\pgfpathlineto{\pgfqpoint{1.655187in}{1.273036in}}%
\pgfpathlineto{\pgfqpoint{1.655900in}{0.969891in}}%
\pgfpathlineto{\pgfqpoint{1.657734in}{1.213285in}}%
\pgfpathlineto{\pgfqpoint{1.660077in}{0.855356in}}%
\pgfpathlineto{\pgfqpoint{1.661911in}{1.270532in}}%
\pgfpathlineto{\pgfqpoint{1.663439in}{0.916920in}}%
\pgfpathlineto{\pgfqpoint{1.664559in}{1.254111in}}%
\pgfpathlineto{\pgfqpoint{1.666801in}{0.963816in}}%
\pgfpathlineto{\pgfqpoint{1.667514in}{1.292216in}}%
\pgfpathlineto{\pgfqpoint{1.669042in}{0.894022in}}%
\pgfpathlineto{\pgfqpoint{1.670774in}{1.169986in}}%
\pgfpathlineto{\pgfqpoint{1.672098in}{0.958655in}}%
\pgfpathlineto{\pgfqpoint{1.673728in}{1.297932in}}%
\pgfpathlineto{\pgfqpoint{1.675969in}{0.880742in}}%
\pgfpathlineto{\pgfqpoint{1.677803in}{1.367038in}}%
\pgfpathlineto{\pgfqpoint{1.678516in}{0.815296in}}%
\pgfpathlineto{\pgfqpoint{1.680350in}{1.220779in}}%
\pgfpathlineto{\pgfqpoint{1.682285in}{0.841234in}}%
\pgfpathlineto{\pgfqpoint{1.684424in}{1.410426in}}%
\pgfpathlineto{\pgfqpoint{1.685036in}{1.074676in}}%
\pgfpathlineto{\pgfqpoint{1.687684in}{0.813398in}}%
\pgfpathlineto{\pgfqpoint{1.688398in}{1.237326in}}%
\pgfpathlineto{\pgfqpoint{1.690639in}{0.962058in}}%
\pgfpathlineto{\pgfqpoint{1.693287in}{1.310612in}}%
\pgfpathlineto{\pgfqpoint{1.695732in}{0.854019in}}%
\pgfpathlineto{\pgfqpoint{1.697057in}{1.178117in}}%
\pgfpathlineto{\pgfqpoint{1.699400in}{1.302878in}}%
\pgfpathlineto{\pgfqpoint{1.700317in}{0.891395in}}%
\pgfpathlineto{\pgfqpoint{1.701743in}{1.374384in}}%
\pgfpathlineto{\pgfqpoint{1.703067in}{0.958150in}}%
\pgfpathlineto{\pgfqpoint{1.705105in}{1.294014in}}%
\pgfpathlineto{\pgfqpoint{1.706225in}{0.905329in}}%
\pgfpathlineto{\pgfqpoint{1.707957in}{1.237091in}}%
\pgfpathlineto{\pgfqpoint{1.709995in}{0.942608in}}%
\pgfpathlineto{\pgfqpoint{1.712134in}{1.380681in}}%
\pgfpathlineto{\pgfqpoint{1.712745in}{1.013452in}}%
\pgfpathlineto{\pgfqpoint{1.714477in}{1.219567in}}%
\pgfpathlineto{\pgfqpoint{1.717227in}{0.903979in}}%
\pgfpathlineto{\pgfqpoint{1.718450in}{1.326299in}}%
\pgfpathlineto{\pgfqpoint{1.719978in}{0.931383in}}%
\pgfpathlineto{\pgfqpoint{1.721302in}{1.198140in}}%
\pgfpathlineto{\pgfqpoint{1.723645in}{0.907612in}}%
\pgfpathlineto{\pgfqpoint{1.724766in}{1.398523in}}%
\pgfpathlineto{\pgfqpoint{1.727211in}{0.907019in}}%
\pgfpathlineto{\pgfqpoint{1.727517in}{1.222543in}}%
\pgfpathlineto{\pgfqpoint{1.730063in}{0.941348in}}%
\pgfpathlineto{\pgfqpoint{1.731388in}{1.310063in}}%
\pgfpathlineto{\pgfqpoint{1.733120in}{0.921657in}}%
\pgfpathlineto{\pgfqpoint{1.734444in}{1.219170in}}%
\pgfpathlineto{\pgfqpoint{1.736787in}{0.882671in}}%
\pgfpathlineto{\pgfqpoint{1.738010in}{1.326160in}}%
\pgfpathlineto{\pgfqpoint{1.739028in}{0.893857in}}%
\pgfpathlineto{\pgfqpoint{1.740556in}{1.261996in}}%
\pgfpathlineto{\pgfqpoint{1.743612in}{0.784032in}}%
\pgfpathlineto{\pgfqpoint{1.743918in}{1.314588in}}%
\pgfpathlineto{\pgfqpoint{1.745854in}{0.918807in}}%
\pgfpathlineto{\pgfqpoint{1.746974in}{1.231852in}}%
\pgfpathlineto{\pgfqpoint{1.749317in}{0.936593in}}%
\pgfpathlineto{\pgfqpoint{1.750540in}{1.276466in}}%
\pgfpathlineto{\pgfqpoint{1.752883in}{0.963854in}}%
\pgfpathlineto{\pgfqpoint{1.753698in}{1.258132in}}%
\pgfpathlineto{\pgfqpoint{1.755124in}{1.009231in}}%
\pgfpathlineto{\pgfqpoint{1.757773in}{0.922701in}}%
\pgfpathlineto{\pgfqpoint{1.759301in}{1.274352in}}%
\pgfpathlineto{\pgfqpoint{1.761135in}{0.839587in}}%
\pgfpathlineto{\pgfqpoint{1.762255in}{1.338524in}}%
\pgfpathlineto{\pgfqpoint{1.763376in}{0.788810in}}%
\pgfpathlineto{\pgfqpoint{1.765209in}{1.217753in}}%
\pgfpathlineto{\pgfqpoint{1.766839in}{0.964165in}}%
\pgfpathlineto{\pgfqpoint{1.769386in}{1.240843in}}%
\pgfpathlineto{\pgfqpoint{1.770711in}{0.926664in}}%
\pgfpathlineto{\pgfqpoint{1.772442in}{1.331986in}}%
\pgfpathlineto{\pgfqpoint{1.773767in}{0.913852in}}%
\pgfpathlineto{\pgfqpoint{1.775702in}{1.285458in}}%
\pgfpathlineto{\pgfqpoint{1.776415in}{0.791597in}}%
\pgfpathlineto{\pgfqpoint{1.778045in}{1.174387in}}%
\pgfpathlineto{\pgfqpoint{1.780694in}{0.936310in}}%
\pgfpathlineto{\pgfqpoint{1.781815in}{1.258021in}}%
\pgfpathlineto{\pgfqpoint{1.783037in}{0.841158in}}%
\pgfpathlineto{\pgfqpoint{1.785788in}{0.746336in}}%
\pgfpathlineto{\pgfqpoint{1.786399in}{1.208682in}}%
\pgfpathlineto{\pgfqpoint{1.787927in}{0.949453in}}%
\pgfpathlineto{\pgfqpoint{1.790474in}{1.389384in}}%
\pgfpathlineto{\pgfqpoint{1.791493in}{0.964056in}}%
\pgfpathlineto{\pgfqpoint{1.793530in}{0.814562in}}%
\pgfpathlineto{\pgfqpoint{1.794956in}{1.300200in}}%
\pgfpathlineto{\pgfqpoint{1.795873in}{0.999230in}}%
\pgfpathlineto{\pgfqpoint{1.798318in}{0.873040in}}%
\pgfpathlineto{\pgfqpoint{1.800356in}{1.272068in}}%
\pgfpathlineto{\pgfqpoint{1.801476in}{0.952289in}}%
\pgfpathlineto{\pgfqpoint{1.803310in}{1.266850in}}%
\pgfpathlineto{\pgfqpoint{1.805245in}{0.913116in}}%
\pgfpathlineto{\pgfqpoint{1.805857in}{1.205776in}}%
\pgfpathlineto{\pgfqpoint{1.807487in}{0.850553in}}%
\pgfpathlineto{\pgfqpoint{1.808913in}{1.263298in}}%
\pgfpathlineto{\pgfqpoint{1.810848in}{0.752087in}}%
\pgfpathlineto{\pgfqpoint{1.813701in}{1.360501in}}%
\pgfpathlineto{\pgfqpoint{1.814618in}{0.943312in}}%
\pgfpathlineto{\pgfqpoint{1.815840in}{1.253084in}}%
\pgfpathlineto{\pgfqpoint{1.817063in}{0.986461in}}%
\pgfpathlineto{\pgfqpoint{1.818795in}{1.299432in}}%
\pgfpathlineto{\pgfqpoint{1.820526in}{0.795635in}}%
\pgfpathlineto{\pgfqpoint{1.822054in}{1.267154in}}%
\pgfpathlineto{\pgfqpoint{1.824194in}{0.890116in}}%
\pgfpathlineto{\pgfqpoint{1.825518in}{1.207666in}}%
\pgfpathlineto{\pgfqpoint{1.828269in}{1.369171in}}%
\pgfpathlineto{\pgfqpoint{1.828574in}{0.906786in}}%
\pgfpathlineto{\pgfqpoint{1.831427in}{1.367580in}}%
\pgfpathlineto{\pgfqpoint{1.832955in}{0.789502in}}%
\pgfpathlineto{\pgfqpoint{1.833668in}{1.260186in}}%
\pgfpathlineto{\pgfqpoint{1.836113in}{0.885079in}}%
\pgfpathlineto{\pgfqpoint{1.836928in}{1.308835in}}%
\pgfpathlineto{\pgfqpoint{1.838965in}{0.927347in}}%
\pgfpathlineto{\pgfqpoint{1.840697in}{1.286892in}}%
\pgfpathlineto{\pgfqpoint{1.843550in}{0.782403in}}%
\pgfpathlineto{\pgfqpoint{1.844976in}{1.189110in}}%
\pgfpathlineto{\pgfqpoint{1.847115in}{0.908890in}}%
\pgfpathlineto{\pgfqpoint{1.848134in}{1.269114in}}%
\pgfpathlineto{\pgfqpoint{1.850069in}{0.970634in}}%
\pgfpathlineto{\pgfqpoint{1.851597in}{1.280205in}}%
\pgfpathlineto{\pgfqpoint{1.853839in}{0.861670in}}%
\pgfpathlineto{\pgfqpoint{1.855367in}{1.275597in}}%
\pgfpathlineto{\pgfqpoint{1.856589in}{0.941708in}}%
\pgfpathlineto{\pgfqpoint{1.858219in}{1.317201in}}%
\pgfpathlineto{\pgfqpoint{1.859849in}{0.911891in}}%
\pgfpathlineto{\pgfqpoint{1.861785in}{1.360355in}}%
\pgfpathlineto{\pgfqpoint{1.864535in}{0.786656in}}%
\pgfpathlineto{\pgfqpoint{1.865962in}{1.265774in}}%
\pgfpathlineto{\pgfqpoint{1.868406in}{0.949530in}}%
\pgfpathlineto{\pgfqpoint{1.869323in}{1.226785in}}%
\pgfpathlineto{\pgfqpoint{1.872074in}{1.361275in}}%
\pgfpathlineto{\pgfqpoint{1.873093in}{0.856092in}}%
\pgfpathlineto{\pgfqpoint{1.874213in}{1.210178in}}%
\pgfpathlineto{\pgfqpoint{1.876760in}{0.944034in}}%
\pgfpathlineto{\pgfqpoint{1.877983in}{1.267274in}}%
\pgfpathlineto{\pgfqpoint{1.879511in}{0.930083in}}%
\pgfpathlineto{\pgfqpoint{1.880631in}{1.183345in}}%
\pgfpathlineto{\pgfqpoint{1.882567in}{1.309654in}}%
\pgfpathlineto{\pgfqpoint{1.884604in}{0.869662in}}%
\pgfpathlineto{\pgfqpoint{1.886642in}{1.369291in}}%
\pgfpathlineto{\pgfqpoint{1.887253in}{0.892915in}}%
\pgfpathlineto{\pgfqpoint{1.889800in}{1.279996in}}%
\pgfpathlineto{\pgfqpoint{1.890920in}{0.874559in}}%
\pgfpathlineto{\pgfqpoint{1.893365in}{1.345763in}}%
\pgfpathlineto{\pgfqpoint{1.893773in}{0.988012in}}%
\pgfpathlineto{\pgfqpoint{1.896829in}{0.889358in}}%
\pgfpathlineto{\pgfqpoint{1.897848in}{1.319886in}}%
\pgfpathlineto{\pgfqpoint{1.898561in}{0.953063in}}%
\pgfpathlineto{\pgfqpoint{1.900802in}{1.303024in}}%
\pgfpathlineto{\pgfqpoint{1.902636in}{0.846161in}}%
\pgfpathlineto{\pgfqpoint{1.903654in}{1.163155in}}%
\pgfpathlineto{\pgfqpoint{1.905182in}{0.852612in}}%
\pgfpathlineto{\pgfqpoint{1.908137in}{1.353449in}}%
\pgfpathlineto{\pgfqpoint{1.908748in}{0.954207in}}%
\pgfpathlineto{\pgfqpoint{1.911091in}{1.270611in}}%
\pgfpathlineto{\pgfqpoint{1.912212in}{0.940064in}}%
\pgfpathlineto{\pgfqpoint{1.913230in}{1.265133in}}%
\pgfpathlineto{\pgfqpoint{1.915981in}{0.913415in}}%
\pgfpathlineto{\pgfqpoint{1.917000in}{1.263887in}}%
\pgfpathlineto{\pgfqpoint{1.918120in}{0.914993in}}%
\pgfpathlineto{\pgfqpoint{1.919852in}{1.199802in}}%
\pgfpathlineto{\pgfqpoint{1.921991in}{1.325202in}}%
\pgfpathlineto{\pgfqpoint{1.923825in}{0.940842in}}%
\pgfpathlineto{\pgfqpoint{1.925251in}{1.225080in}}%
\pgfpathlineto{\pgfqpoint{1.926270in}{0.949518in}}%
\pgfpathlineto{\pgfqpoint{1.929734in}{1.416276in}}%
\pgfpathlineto{\pgfqpoint{1.931262in}{0.886185in}}%
\pgfpathlineto{\pgfqpoint{1.933503in}{1.306443in}}%
\pgfpathlineto{\pgfqpoint{1.934522in}{0.949629in}}%
\pgfpathlineto{\pgfqpoint{1.936050in}{1.273238in}}%
\pgfpathlineto{\pgfqpoint{1.939208in}{0.842961in}}%
\pgfpathlineto{\pgfqpoint{1.939412in}{1.226517in}}%
\pgfpathlineto{\pgfqpoint{1.941449in}{1.449493in}}%
\pgfpathlineto{\pgfqpoint{1.942672in}{0.959146in}}%
\pgfpathlineto{\pgfqpoint{1.944505in}{1.342851in}}%
\pgfpathlineto{\pgfqpoint{1.946135in}{0.918953in}}%
\pgfpathlineto{\pgfqpoint{1.947765in}{1.239785in}}%
\pgfpathlineto{\pgfqpoint{1.949497in}{0.887948in}}%
\pgfpathlineto{\pgfqpoint{1.950720in}{1.272419in}}%
\pgfpathlineto{\pgfqpoint{1.952655in}{0.983607in}}%
\pgfpathlineto{\pgfqpoint{1.955304in}{1.330510in}}%
\pgfpathlineto{\pgfqpoint{1.955711in}{0.936637in}}%
\pgfpathlineto{\pgfqpoint{1.958564in}{0.868952in}}%
\pgfpathlineto{\pgfqpoint{1.958869in}{1.191808in}}%
\pgfpathlineto{\pgfqpoint{1.961518in}{0.915911in}}%
\pgfpathlineto{\pgfqpoint{1.962333in}{1.402875in}}%
\pgfpathlineto{\pgfqpoint{1.963963in}{0.867325in}}%
\pgfpathlineto{\pgfqpoint{1.965899in}{1.312156in}}%
\pgfpathlineto{\pgfqpoint{1.967019in}{0.968516in}}%
\pgfpathlineto{\pgfqpoint{1.968649in}{1.186227in}}%
\pgfpathlineto{\pgfqpoint{1.970483in}{0.859387in}}%
\pgfpathlineto{\pgfqpoint{1.972520in}{1.306134in}}%
\pgfpathlineto{\pgfqpoint{1.974048in}{1.040768in}}%
\pgfpathlineto{\pgfqpoint{1.976595in}{1.331539in}}%
\pgfpathlineto{\pgfqpoint{1.977003in}{0.988170in}}%
\pgfpathlineto{\pgfqpoint{1.979040in}{1.298600in}}%
\pgfpathlineto{\pgfqpoint{1.980670in}{0.982727in}}%
\pgfpathlineto{\pgfqpoint{1.982809in}{1.385669in}}%
\pgfpathlineto{\pgfqpoint{1.983930in}{0.841881in}}%
\pgfpathlineto{\pgfqpoint{1.985560in}{1.217764in}}%
\pgfpathlineto{\pgfqpoint{1.986884in}{0.923414in}}%
\pgfpathlineto{\pgfqpoint{1.988412in}{1.423791in}}%
\pgfpathlineto{\pgfqpoint{1.990857in}{0.914198in}}%
\pgfpathlineto{\pgfqpoint{1.992182in}{1.429693in}}%
\pgfpathlineto{\pgfqpoint{1.993506in}{0.956268in}}%
\pgfpathlineto{\pgfqpoint{1.996257in}{1.379933in}}%
\pgfpathlineto{\pgfqpoint{1.996562in}{0.937071in}}%
\pgfpathlineto{\pgfqpoint{1.998090in}{1.224883in}}%
\pgfpathlineto{\pgfqpoint{2.000739in}{1.283033in}}%
\pgfpathlineto{\pgfqpoint{2.002165in}{0.906055in}}%
\pgfpathlineto{\pgfqpoint{2.003388in}{1.349074in}}%
\pgfpathlineto{\pgfqpoint{2.005833in}{0.886002in}}%
\pgfpathlineto{\pgfqpoint{2.007055in}{1.198749in}}%
\pgfpathlineto{\pgfqpoint{2.008583in}{0.864912in}}%
\pgfpathlineto{\pgfqpoint{2.009704in}{1.248462in}}%
\pgfpathlineto{\pgfqpoint{2.011232in}{0.995170in}}%
\pgfpathlineto{\pgfqpoint{2.012760in}{1.319623in}}%
\pgfpathlineto{\pgfqpoint{2.015612in}{0.804979in}}%
\pgfpathlineto{\pgfqpoint{2.016427in}{1.439194in}}%
\pgfpathlineto{\pgfqpoint{2.017650in}{1.012038in}}%
\pgfpathlineto{\pgfqpoint{2.020095in}{1.325561in}}%
\pgfpathlineto{\pgfqpoint{2.021725in}{0.951829in}}%
\pgfpathlineto{\pgfqpoint{2.023966in}{1.386797in}}%
\pgfpathlineto{\pgfqpoint{2.024272in}{1.010381in}}%
\pgfpathlineto{\pgfqpoint{2.027022in}{1.325088in}}%
\pgfpathlineto{\pgfqpoint{2.027837in}{0.915748in}}%
\pgfpathlineto{\pgfqpoint{2.030486in}{0.789331in}}%
\pgfpathlineto{\pgfqpoint{2.030588in}{1.245234in}}%
\pgfpathlineto{\pgfqpoint{2.033746in}{1.365152in}}%
\pgfpathlineto{\pgfqpoint{2.034153in}{0.908072in}}%
\pgfpathlineto{\pgfqpoint{2.036191in}{1.279909in}}%
\pgfpathlineto{\pgfqpoint{2.037108in}{0.886895in}}%
\pgfpathlineto{\pgfqpoint{2.039654in}{1.363740in}}%
\pgfpathlineto{\pgfqpoint{2.040469in}{0.923776in}}%
\pgfpathlineto{\pgfqpoint{2.042201in}{1.199009in}}%
\pgfpathlineto{\pgfqpoint{2.044035in}{0.868348in}}%
\pgfpathlineto{\pgfqpoint{2.045767in}{1.262310in}}%
\pgfpathlineto{\pgfqpoint{2.047193in}{0.851710in}}%
\pgfpathlineto{\pgfqpoint{2.048619in}{1.303440in}}%
\pgfpathlineto{\pgfqpoint{2.051268in}{0.852416in}}%
\pgfpathlineto{\pgfqpoint{2.052898in}{1.360557in}}%
\pgfpathlineto{\pgfqpoint{2.053407in}{1.005389in}}%
\pgfpathlineto{\pgfqpoint{2.055139in}{1.319350in}}%
\pgfpathlineto{\pgfqpoint{2.057788in}{1.006527in}}%
\pgfpathlineto{\pgfqpoint{2.058908in}{1.265496in}}%
\pgfpathlineto{\pgfqpoint{2.060436in}{0.858437in}}%
\pgfpathlineto{\pgfqpoint{2.061964in}{1.204025in}}%
\pgfpathlineto{\pgfqpoint{2.064511in}{0.795303in}}%
\pgfpathlineto{\pgfqpoint{2.064817in}{1.315938in}}%
\pgfpathlineto{\pgfqpoint{2.067160in}{0.790867in}}%
\pgfpathlineto{\pgfqpoint{2.068179in}{1.290750in}}%
\pgfpathlineto{\pgfqpoint{2.069809in}{0.998943in}}%
\pgfpathlineto{\pgfqpoint{2.071744in}{1.219711in}}%
\pgfpathlineto{\pgfqpoint{2.073782in}{0.962092in}}%
\pgfpathlineto{\pgfqpoint{2.075615in}{1.273510in}}%
\pgfpathlineto{\pgfqpoint{2.076430in}{0.921509in}}%
\pgfpathlineto{\pgfqpoint{2.078977in}{1.329781in}}%
\pgfpathlineto{\pgfqpoint{2.079487in}{1.024109in}}%
\pgfpathlineto{\pgfqpoint{2.081320in}{1.296186in}}%
\pgfpathlineto{\pgfqpoint{2.083256in}{0.874570in}}%
\pgfpathlineto{\pgfqpoint{2.085701in}{1.329056in}}%
\pgfpathlineto{\pgfqpoint{2.086210in}{0.893759in}}%
\pgfpathlineto{\pgfqpoint{2.088044in}{1.355407in}}%
\pgfpathlineto{\pgfqpoint{2.089776in}{0.992643in}}%
\pgfpathlineto{\pgfqpoint{2.090998in}{1.260268in}}%
\pgfpathlineto{\pgfqpoint{2.093443in}{0.954995in}}%
\pgfpathlineto{\pgfqpoint{2.094564in}{1.283267in}}%
\pgfpathlineto{\pgfqpoint{2.095888in}{0.982453in}}%
\pgfpathlineto{\pgfqpoint{2.098537in}{1.350111in}}%
\pgfpathlineto{\pgfqpoint{2.099046in}{1.068687in}}%
\pgfpathlineto{\pgfqpoint{2.102102in}{0.889968in}}%
\pgfpathlineto{\pgfqpoint{2.103529in}{1.357374in}}%
\pgfpathlineto{\pgfqpoint{2.103936in}{0.949989in}}%
\pgfpathlineto{\pgfqpoint{2.105872in}{1.288919in}}%
\pgfpathlineto{\pgfqpoint{2.108418in}{0.909998in}}%
\pgfpathlineto{\pgfqpoint{2.109233in}{1.298902in}}%
\pgfpathlineto{\pgfqpoint{2.110660in}{0.984262in}}%
\pgfpathlineto{\pgfqpoint{2.112188in}{1.255985in}}%
\pgfpathlineto{\pgfqpoint{2.113920in}{0.965810in}}%
\pgfpathlineto{\pgfqpoint{2.116364in}{1.314222in}}%
\pgfpathlineto{\pgfqpoint{2.116976in}{1.022222in}}%
\pgfpathlineto{\pgfqpoint{2.119421in}{0.870834in}}%
\pgfpathlineto{\pgfqpoint{2.120337in}{1.295166in}}%
\pgfpathlineto{\pgfqpoint{2.122884in}{0.896670in}}%
\pgfpathlineto{\pgfqpoint{2.123699in}{1.347436in}}%
\pgfpathlineto{\pgfqpoint{2.125227in}{1.003103in}}%
\pgfpathlineto{\pgfqpoint{2.126857in}{1.244600in}}%
\pgfpathlineto{\pgfqpoint{2.128589in}{0.939673in}}%
\pgfpathlineto{\pgfqpoint{2.130321in}{1.260302in}}%
\pgfpathlineto{\pgfqpoint{2.132257in}{0.751201in}}%
\pgfpathlineto{\pgfqpoint{2.133377in}{1.231809in}}%
\pgfpathlineto{\pgfqpoint{2.136332in}{1.338443in}}%
\pgfpathlineto{\pgfqpoint{2.136637in}{0.905906in}}%
\pgfpathlineto{\pgfqpoint{2.139693in}{0.759560in}}%
\pgfpathlineto{\pgfqpoint{2.139795in}{1.325746in}}%
\pgfpathlineto{\pgfqpoint{2.141629in}{1.014791in}}%
\pgfpathlineto{\pgfqpoint{2.143157in}{1.229783in}}%
\pgfpathlineto{\pgfqpoint{2.145908in}{0.805175in}}%
\pgfpathlineto{\pgfqpoint{2.146926in}{1.346018in}}%
\pgfpathlineto{\pgfqpoint{2.147945in}{0.928898in}}%
\pgfpathlineto{\pgfqpoint{2.150899in}{1.292451in}}%
\pgfpathlineto{\pgfqpoint{2.151307in}{0.856104in}}%
\pgfpathlineto{\pgfqpoint{2.153039in}{1.196391in}}%
\pgfpathlineto{\pgfqpoint{2.154770in}{0.884155in}}%
\pgfpathlineto{\pgfqpoint{2.156095in}{1.204182in}}%
\pgfpathlineto{\pgfqpoint{2.158030in}{0.956671in}}%
\pgfpathlineto{\pgfqpoint{2.159457in}{1.238285in}}%
\pgfpathlineto{\pgfqpoint{2.160985in}{1.078808in}}%
\pgfpathlineto{\pgfqpoint{2.163837in}{0.913093in}}%
\pgfpathlineto{\pgfqpoint{2.164346in}{1.178966in}}%
\pgfpathlineto{\pgfqpoint{2.165976in}{1.434321in}}%
\pgfpathlineto{\pgfqpoint{2.167606in}{0.927484in}}%
\pgfpathlineto{\pgfqpoint{2.169338in}{1.283832in}}%
\pgfpathlineto{\pgfqpoint{2.171579in}{1.387613in}}%
\pgfpathlineto{\pgfqpoint{2.173311in}{0.928112in}}%
\pgfpathlineto{\pgfqpoint{2.174636in}{1.241291in}}%
\pgfpathlineto{\pgfqpoint{2.175756in}{0.969387in}}%
\pgfpathlineto{\pgfqpoint{2.178303in}{1.286883in}}%
\pgfpathlineto{\pgfqpoint{2.179220in}{0.923880in}}%
\pgfpathlineto{\pgfqpoint{2.180850in}{1.304638in}}%
\pgfpathlineto{\pgfqpoint{2.183499in}{0.847720in}}%
\pgfpathlineto{\pgfqpoint{2.184314in}{1.228462in}}%
\pgfpathlineto{\pgfqpoint{2.185434in}{0.942565in}}%
\pgfpathlineto{\pgfqpoint{2.187064in}{1.324347in}}%
\pgfpathlineto{\pgfqpoint{2.191445in}{0.718924in}}%
\pgfpathlineto{\pgfqpoint{2.192667in}{1.335377in}}%
\pgfpathlineto{\pgfqpoint{2.194399in}{0.872446in}}%
\pgfpathlineto{\pgfqpoint{2.196131in}{1.387911in}}%
\pgfpathlineto{\pgfqpoint{2.196946in}{1.009378in}}%
\pgfpathlineto{\pgfqpoint{2.199289in}{1.384724in}}%
\pgfpathlineto{\pgfqpoint{2.200104in}{0.962859in}}%
\pgfpathlineto{\pgfqpoint{2.201937in}{1.296430in}}%
\pgfpathlineto{\pgfqpoint{2.204892in}{0.876375in}}%
\pgfpathlineto{\pgfqpoint{2.205605in}{1.217612in}}%
\pgfpathlineto{\pgfqpoint{2.207744in}{1.382317in}}%
\pgfpathlineto{\pgfqpoint{2.208254in}{0.944173in}}%
\pgfpathlineto{\pgfqpoint{2.209884in}{1.262315in}}%
\pgfpathlineto{\pgfqpoint{2.211615in}{0.932434in}}%
\pgfpathlineto{\pgfqpoint{2.213245in}{1.224948in}}%
\pgfpathlineto{\pgfqpoint{2.215894in}{0.961701in}}%
\pgfpathlineto{\pgfqpoint{2.216607in}{1.210206in}}%
\pgfpathlineto{\pgfqpoint{2.219154in}{0.976510in}}%
\pgfpathlineto{\pgfqpoint{2.219663in}{1.277450in}}%
\pgfpathlineto{\pgfqpoint{2.221803in}{0.939272in}}%
\pgfpathlineto{\pgfqpoint{2.223636in}{1.272664in}}%
\pgfpathlineto{\pgfqpoint{2.225164in}{0.903345in}}%
\pgfpathlineto{\pgfqpoint{2.226183in}{1.240118in}}%
\pgfpathlineto{\pgfqpoint{2.228119in}{0.861836in}}%
\pgfpathlineto{\pgfqpoint{2.229443in}{1.308347in}}%
\pgfpathlineto{\pgfqpoint{2.231786in}{0.753482in}}%
\pgfpathlineto{\pgfqpoint{2.232703in}{1.257464in}}%
\pgfpathlineto{\pgfqpoint{2.235759in}{0.848016in}}%
\pgfpathlineto{\pgfqpoint{2.236269in}{1.328979in}}%
\pgfpathlineto{\pgfqpoint{2.238917in}{0.898957in}}%
\pgfpathlineto{\pgfqpoint{2.239732in}{1.315062in}}%
\pgfpathlineto{\pgfqpoint{2.240955in}{0.846200in}}%
\pgfpathlineto{\pgfqpoint{2.243094in}{1.264002in}}%
\pgfpathlineto{\pgfqpoint{2.245335in}{0.955126in}}%
\pgfpathlineto{\pgfqpoint{2.246048in}{1.226688in}}%
\pgfpathlineto{\pgfqpoint{2.248901in}{1.247495in}}%
\pgfpathlineto{\pgfqpoint{2.249308in}{0.802468in}}%
\pgfpathlineto{\pgfqpoint{2.250938in}{1.259265in}}%
\pgfpathlineto{\pgfqpoint{2.253383in}{0.969887in}}%
\pgfpathlineto{\pgfqpoint{2.253994in}{1.283535in}}%
\pgfpathlineto{\pgfqpoint{2.255624in}{0.858164in}}%
\pgfpathlineto{\pgfqpoint{2.257560in}{1.287204in}}%
\pgfpathlineto{\pgfqpoint{2.258986in}{0.938163in}}%
\pgfpathlineto{\pgfqpoint{2.260820in}{1.575662in}}%
\pgfpathlineto{\pgfqpoint{2.262857in}{0.890602in}}%
\pgfpathlineto{\pgfqpoint{2.265200in}{1.311346in}}%
\pgfpathlineto{\pgfqpoint{2.265913in}{0.966817in}}%
\pgfpathlineto{\pgfqpoint{2.267238in}{1.246900in}}%
\pgfpathlineto{\pgfqpoint{2.269683in}{0.978604in}}%
\pgfpathlineto{\pgfqpoint{2.271618in}{1.299003in}}%
\pgfpathlineto{\pgfqpoint{2.272943in}{0.936623in}}%
\pgfpathlineto{\pgfqpoint{2.274063in}{1.392970in}}%
\pgfpathlineto{\pgfqpoint{2.275490in}{0.973696in}}%
\pgfpathlineto{\pgfqpoint{2.277323in}{1.294744in}}%
\pgfpathlineto{\pgfqpoint{2.278444in}{0.981444in}}%
\pgfpathlineto{\pgfqpoint{2.280176in}{1.235322in}}%
\pgfpathlineto{\pgfqpoint{2.281806in}{0.942472in}}%
\pgfpathlineto{\pgfqpoint{2.283334in}{1.217938in}}%
\pgfpathlineto{\pgfqpoint{2.285881in}{0.911925in}}%
\pgfpathlineto{\pgfqpoint{2.286696in}{1.364267in}}%
\pgfpathlineto{\pgfqpoint{2.289140in}{0.974800in}}%
\pgfpathlineto{\pgfqpoint{2.289854in}{1.254263in}}%
\pgfpathlineto{\pgfqpoint{2.293623in}{0.808827in}}%
\pgfpathlineto{\pgfqpoint{2.294845in}{1.284981in}}%
\pgfpathlineto{\pgfqpoint{2.296272in}{0.943505in}}%
\pgfpathlineto{\pgfqpoint{2.298513in}{1.303539in}}%
\pgfpathlineto{\pgfqpoint{2.300754in}{0.945444in}}%
\pgfpathlineto{\pgfqpoint{2.301263in}{1.314852in}}%
\pgfpathlineto{\pgfqpoint{2.302995in}{0.831377in}}%
\pgfpathlineto{\pgfqpoint{2.305134in}{1.263294in}}%
\pgfpathlineto{\pgfqpoint{2.306764in}{0.949976in}}%
\pgfpathlineto{\pgfqpoint{2.308191in}{1.390829in}}%
\pgfpathlineto{\pgfqpoint{2.309515in}{0.971151in}}%
\pgfpathlineto{\pgfqpoint{2.311552in}{1.249715in}}%
\pgfpathlineto{\pgfqpoint{2.312775in}{0.886978in}}%
\pgfpathlineto{\pgfqpoint{2.315118in}{1.275331in}}%
\pgfpathlineto{\pgfqpoint{2.315831in}{1.131360in}}%
\pgfpathlineto{\pgfqpoint{2.315831in}{1.131360in}}%
\pgfusepath{stroke}%
\end{pgfscope}%
\begin{pgfscope}%
\pgfsetrectcap%
\pgfsetmiterjoin%
\pgfsetlinewidth{0.803000pt}%
\definecolor{currentstroke}{rgb}{0.000000,0.000000,0.000000}%
\pgfsetstrokecolor{currentstroke}%
\pgfsetdash{}{0pt}%
\pgfpathmoveto{\pgfqpoint{0.563510in}{0.416447in}}%
\pgfpathlineto{\pgfqpoint{0.563510in}{1.789039in}}%
\pgfusepath{stroke}%
\end{pgfscope}%
\begin{pgfscope}%
\pgfsetrectcap%
\pgfsetmiterjoin%
\pgfsetlinewidth{0.803000pt}%
\definecolor{currentstroke}{rgb}{0.000000,0.000000,0.000000}%
\pgfsetstrokecolor{currentstroke}%
\pgfsetdash{}{0pt}%
\pgfpathmoveto{\pgfqpoint{2.399275in}{0.416447in}}%
\pgfpathlineto{\pgfqpoint{2.399275in}{1.789039in}}%
\pgfusepath{stroke}%
\end{pgfscope}%
\begin{pgfscope}%
\pgfsetrectcap%
\pgfsetmiterjoin%
\pgfsetlinewidth{0.803000pt}%
\definecolor{currentstroke}{rgb}{0.000000,0.000000,0.000000}%
\pgfsetstrokecolor{currentstroke}%
\pgfsetdash{}{0pt}%
\pgfpathmoveto{\pgfqpoint{0.563510in}{0.416447in}}%
\pgfpathlineto{\pgfqpoint{2.399275in}{0.416447in}}%
\pgfusepath{stroke}%
\end{pgfscope}%
\begin{pgfscope}%
\pgfsetrectcap%
\pgfsetmiterjoin%
\pgfsetlinewidth{0.803000pt}%
\definecolor{currentstroke}{rgb}{0.000000,0.000000,0.000000}%
\pgfsetstrokecolor{currentstroke}%
\pgfsetdash{}{0pt}%
\pgfpathmoveto{\pgfqpoint{0.563510in}{1.789039in}}%
\pgfpathlineto{\pgfqpoint{2.399275in}{1.789039in}}%
\pgfusepath{stroke}%
\end{pgfscope}%
\begin{pgfscope}%
\pgfsetbuttcap%
\pgfsetmiterjoin%
\definecolor{currentfill}{rgb}{1.000000,1.000000,1.000000}%
\pgfsetfillcolor{currentfill}%
\pgfsetfillopacity{0.800000}%
\pgfsetlinewidth{1.003750pt}%
\definecolor{currentstroke}{rgb}{0.800000,0.800000,0.800000}%
\pgfsetstrokecolor{currentstroke}%
\pgfsetstrokeopacity{0.800000}%
\pgfsetdash{}{0pt}%
\pgfpathmoveto{\pgfqpoint{0.641288in}{1.545261in}}%
\pgfpathlineto{\pgfqpoint{1.610399in}{1.545261in}}%
\pgfpathquadraticcurveto{\pgfqpoint{1.632621in}{1.545261in}}{\pgfqpoint{1.632621in}{1.567483in}}%
\pgfpathlineto{\pgfqpoint{1.632621in}{1.711261in}}%
\pgfpathquadraticcurveto{\pgfqpoint{1.632621in}{1.733483in}}{\pgfqpoint{1.610399in}{1.733483in}}%
\pgfpathlineto{\pgfqpoint{0.641288in}{1.733483in}}%
\pgfpathquadraticcurveto{\pgfqpoint{0.619065in}{1.733483in}}{\pgfqpoint{0.619065in}{1.711261in}}%
\pgfpathlineto{\pgfqpoint{0.619065in}{1.567483in}}%
\pgfpathquadraticcurveto{\pgfqpoint{0.619065in}{1.545261in}}{\pgfqpoint{0.641288in}{1.545261in}}%
\pgfpathlineto{\pgfqpoint{0.641288in}{1.545261in}}%
\pgfpathclose%
\pgfusepath{stroke,fill}%
\end{pgfscope}%
\begin{pgfscope}%
\pgfsetrectcap%
\pgfsetroundjoin%
\pgfsetlinewidth{1.505625pt}%
\definecolor{currentstroke}{rgb}{0.000000,0.447059,0.698039}%
\pgfsetstrokecolor{currentstroke}%
\pgfsetdash{}{0pt}%
\pgfpathmoveto{\pgfqpoint{0.663510in}{1.650150in}}%
\pgfpathlineto{\pgfqpoint{0.774621in}{1.650150in}}%
\pgfpathlineto{\pgfqpoint{0.885732in}{1.650150in}}%
\pgfusepath{stroke}%
\end{pgfscope}%
\begin{pgfscope}%
\definecolor{textcolor}{rgb}{0.000000,0.000000,0.000000}%
\pgfsetstrokecolor{textcolor}%
\pgfsetfillcolor{textcolor}%
\pgftext[x=0.974621in,y=1.611261in,left,base]{\color{textcolor}\rmfamily\fontsize{8.000000}{9.600000}\selectfont White noise}%
\end{pgfscope}%
\end{pgfpicture}%
\makeatother%
\endgroup%

        } % scalebox
        \caption{Time domain}
        \label{fig:white_noise_time}
    \end{subfigure}
    \hfill
    \begin{subfigure}{0.32\linewidth}
        \scalebox{0.75}{%
            %% Creator: Matplotlib, PGF backend
%%
%% To include the figure in your LaTeX document, write
%%   \input{<filename>.pgf}
%%
%% Make sure the required packages are loaded in your preamble
%%   \usepackage{pgf}
%%
%% Also ensure that all the required font packages are loaded; for instance,
%% the lmodern package is sometimes necessary when using math font.
%%   \usepackage{lmodern}
%%
%% Figures using additional raster images can only be included by \input if
%% they are in the same directory as the main LaTeX file. For loading figures
%% from other directories you can use the `import` package
%%   \usepackage{import}
%%
%% and then include the figures with
%%   \import{<path to file>}{<filename>.pgf}
%%
%% Matplotlib used the following preamble
%%   \def\mathdefault#1{#1}
%%   \everymath=\expandafter{\the\everymath\displaystyle}
%%   \usepackage{siunitx}
%%   \sisetup{per-mode = symbol}%
%%   \ifdefined\pdftexversion\else  % non-pdftex case.
%%     \usepackage{fontspec}
%%   \fi
%%   \makeatletter\@ifpackageloaded{underscore}{}{\usepackage[strings]{underscore}}\makeatother
%%
\begingroup%
\makeatletter%
\begin{pgfpicture}%
\pgfpathrectangle{\pgfpointorigin}{\pgfqpoint{2.440945in}{1.830709in}}%
\pgfusepath{use as bounding box, clip}%
\begin{pgfscope}%
\pgfsetbuttcap%
\pgfsetmiterjoin%
\definecolor{currentfill}{rgb}{1.000000,1.000000,1.000000}%
\pgfsetfillcolor{currentfill}%
\pgfsetlinewidth{0.000000pt}%
\definecolor{currentstroke}{rgb}{1.000000,1.000000,1.000000}%
\pgfsetstrokecolor{currentstroke}%
\pgfsetdash{}{0pt}%
\pgfpathmoveto{\pgfqpoint{0.000000in}{0.000000in}}%
\pgfpathlineto{\pgfqpoint{2.440945in}{0.000000in}}%
\pgfpathlineto{\pgfqpoint{2.440945in}{1.830709in}}%
\pgfpathlineto{\pgfqpoint{0.000000in}{1.830709in}}%
\pgfpathlineto{\pgfqpoint{0.000000in}{0.000000in}}%
\pgfpathclose%
\pgfusepath{fill}%
\end{pgfscope}%
\begin{pgfscope}%
\pgfsetbuttcap%
\pgfsetmiterjoin%
\definecolor{currentfill}{rgb}{1.000000,1.000000,1.000000}%
\pgfsetfillcolor{currentfill}%
\pgfsetlinewidth{0.000000pt}%
\definecolor{currentstroke}{rgb}{0.000000,0.000000,0.000000}%
\pgfsetstrokecolor{currentstroke}%
\pgfsetstrokeopacity{0.000000}%
\pgfsetdash{}{0pt}%
\pgfpathmoveto{\pgfqpoint{0.514278in}{0.417642in}}%
\pgfpathlineto{\pgfqpoint{2.399275in}{0.417642in}}%
\pgfpathlineto{\pgfqpoint{2.399275in}{1.789039in}}%
\pgfpathlineto{\pgfqpoint{0.514278in}{1.789039in}}%
\pgfpathlineto{\pgfqpoint{0.514278in}{0.417642in}}%
\pgfpathclose%
\pgfusepath{fill}%
\end{pgfscope}%
\begin{pgfscope}%
\pgfpathrectangle{\pgfqpoint{0.514278in}{0.417642in}}{\pgfqpoint{1.884996in}{1.371397in}}%
\pgfusepath{clip}%
\pgfsetrectcap%
\pgfsetroundjoin%
\pgfsetlinewidth{0.803000pt}%
\definecolor{currentstroke}{rgb}{0.450000,0.450000,0.450000}%
\pgfsetstrokecolor{currentstroke}%
\pgfsetdash{}{0pt}%
\pgfpathmoveto{\pgfqpoint{0.916836in}{0.417642in}}%
\pgfpathlineto{\pgfqpoint{0.916836in}{1.789039in}}%
\pgfusepath{stroke}%
\end{pgfscope}%
\begin{pgfscope}%
\pgfsetbuttcap%
\pgfsetroundjoin%
\definecolor{currentfill}{rgb}{0.000000,0.000000,0.000000}%
\pgfsetfillcolor{currentfill}%
\pgfsetlinewidth{0.803000pt}%
\definecolor{currentstroke}{rgb}{0.000000,0.000000,0.000000}%
\pgfsetstrokecolor{currentstroke}%
\pgfsetdash{}{0pt}%
\pgfsys@defobject{currentmarker}{\pgfqpoint{0.000000in}{-0.048611in}}{\pgfqpoint{0.000000in}{0.000000in}}{%
\pgfpathmoveto{\pgfqpoint{0.000000in}{0.000000in}}%
\pgfpathlineto{\pgfqpoint{0.000000in}{-0.048611in}}%
\pgfusepath{stroke,fill}%
}%
\begin{pgfscope}%
\pgfsys@transformshift{0.916836in}{0.417642in}%
\pgfsys@useobject{currentmarker}{}%
\end{pgfscope}%
\end{pgfscope}%
\begin{pgfscope}%
\definecolor{textcolor}{rgb}{0.000000,0.000000,0.000000}%
\pgfsetstrokecolor{textcolor}%
\pgfsetfillcolor{textcolor}%
\pgftext[x=0.916836in,y=0.320420in,,top]{\color{textcolor}{\rmfamily\fontsize{8.000000}{9.600000}\selectfont\catcode`\^=\active\def^{\ifmmode\sp\else\^{}\fi}\catcode`\%=\active\def%{\%}$\mathdefault{10^{-3}}$}}%
\end{pgfscope}%
\begin{pgfscope}%
\pgfpathrectangle{\pgfqpoint{0.514278in}{0.417642in}}{\pgfqpoint{1.884996in}{1.371397in}}%
\pgfusepath{clip}%
\pgfsetrectcap%
\pgfsetroundjoin%
\pgfsetlinewidth{0.803000pt}%
\definecolor{currentstroke}{rgb}{0.450000,0.450000,0.450000}%
\pgfsetstrokecolor{currentstroke}%
\pgfsetdash{}{0pt}%
\pgfpathmoveto{\pgfqpoint{1.434391in}{0.417642in}}%
\pgfpathlineto{\pgfqpoint{1.434391in}{1.789039in}}%
\pgfusepath{stroke}%
\end{pgfscope}%
\begin{pgfscope}%
\pgfsetbuttcap%
\pgfsetroundjoin%
\definecolor{currentfill}{rgb}{0.000000,0.000000,0.000000}%
\pgfsetfillcolor{currentfill}%
\pgfsetlinewidth{0.803000pt}%
\definecolor{currentstroke}{rgb}{0.000000,0.000000,0.000000}%
\pgfsetstrokecolor{currentstroke}%
\pgfsetdash{}{0pt}%
\pgfsys@defobject{currentmarker}{\pgfqpoint{0.000000in}{-0.048611in}}{\pgfqpoint{0.000000in}{0.000000in}}{%
\pgfpathmoveto{\pgfqpoint{0.000000in}{0.000000in}}%
\pgfpathlineto{\pgfqpoint{0.000000in}{-0.048611in}}%
\pgfusepath{stroke,fill}%
}%
\begin{pgfscope}%
\pgfsys@transformshift{1.434391in}{0.417642in}%
\pgfsys@useobject{currentmarker}{}%
\end{pgfscope}%
\end{pgfscope}%
\begin{pgfscope}%
\definecolor{textcolor}{rgb}{0.000000,0.000000,0.000000}%
\pgfsetstrokecolor{textcolor}%
\pgfsetfillcolor{textcolor}%
\pgftext[x=1.434391in,y=0.320420in,,top]{\color{textcolor}{\rmfamily\fontsize{8.000000}{9.600000}\selectfont\catcode`\^=\active\def^{\ifmmode\sp\else\^{}\fi}\catcode`\%=\active\def%{\%}$\mathdefault{10^{-2}}$}}%
\end{pgfscope}%
\begin{pgfscope}%
\pgfpathrectangle{\pgfqpoint{0.514278in}{0.417642in}}{\pgfqpoint{1.884996in}{1.371397in}}%
\pgfusepath{clip}%
\pgfsetrectcap%
\pgfsetroundjoin%
\pgfsetlinewidth{0.803000pt}%
\definecolor{currentstroke}{rgb}{0.450000,0.450000,0.450000}%
\pgfsetstrokecolor{currentstroke}%
\pgfsetdash{}{0pt}%
\pgfpathmoveto{\pgfqpoint{1.951947in}{0.417642in}}%
\pgfpathlineto{\pgfqpoint{1.951947in}{1.789039in}}%
\pgfusepath{stroke}%
\end{pgfscope}%
\begin{pgfscope}%
\pgfsetbuttcap%
\pgfsetroundjoin%
\definecolor{currentfill}{rgb}{0.000000,0.000000,0.000000}%
\pgfsetfillcolor{currentfill}%
\pgfsetlinewidth{0.803000pt}%
\definecolor{currentstroke}{rgb}{0.000000,0.000000,0.000000}%
\pgfsetstrokecolor{currentstroke}%
\pgfsetdash{}{0pt}%
\pgfsys@defobject{currentmarker}{\pgfqpoint{0.000000in}{-0.048611in}}{\pgfqpoint{0.000000in}{0.000000in}}{%
\pgfpathmoveto{\pgfqpoint{0.000000in}{0.000000in}}%
\pgfpathlineto{\pgfqpoint{0.000000in}{-0.048611in}}%
\pgfusepath{stroke,fill}%
}%
\begin{pgfscope}%
\pgfsys@transformshift{1.951947in}{0.417642in}%
\pgfsys@useobject{currentmarker}{}%
\end{pgfscope}%
\end{pgfscope}%
\begin{pgfscope}%
\definecolor{textcolor}{rgb}{0.000000,0.000000,0.000000}%
\pgfsetstrokecolor{textcolor}%
\pgfsetfillcolor{textcolor}%
\pgftext[x=1.951947in,y=0.320420in,,top]{\color{textcolor}{\rmfamily\fontsize{8.000000}{9.600000}\selectfont\catcode`\^=\active\def^{\ifmmode\sp\else\^{}\fi}\catcode`\%=\active\def%{\%}$\mathdefault{10^{-1}}$}}%
\end{pgfscope}%
\begin{pgfscope}%
\pgfpathrectangle{\pgfqpoint{0.514278in}{0.417642in}}{\pgfqpoint{1.884996in}{1.371397in}}%
\pgfusepath{clip}%
\pgfsetrectcap%
\pgfsetroundjoin%
\pgfsetlinewidth{0.803000pt}%
\definecolor{currentstroke}{rgb}{0.850000,0.850000,0.850000}%
\pgfsetstrokecolor{currentstroke}%
\pgfsetdash{}{0pt}%
\pgfpathmoveto{\pgfqpoint{0.555080in}{0.417642in}}%
\pgfpathlineto{\pgfqpoint{0.555080in}{1.789039in}}%
\pgfusepath{stroke}%
\end{pgfscope}%
\begin{pgfscope}%
\pgfsetbuttcap%
\pgfsetroundjoin%
\definecolor{currentfill}{rgb}{0.000000,0.000000,0.000000}%
\pgfsetfillcolor{currentfill}%
\pgfsetlinewidth{0.602250pt}%
\definecolor{currentstroke}{rgb}{0.000000,0.000000,0.000000}%
\pgfsetstrokecolor{currentstroke}%
\pgfsetdash{}{0pt}%
\pgfsys@defobject{currentmarker}{\pgfqpoint{0.000000in}{-0.027778in}}{\pgfqpoint{0.000000in}{0.000000in}}{%
\pgfpathmoveto{\pgfqpoint{0.000000in}{0.000000in}}%
\pgfpathlineto{\pgfqpoint{0.000000in}{-0.027778in}}%
\pgfusepath{stroke,fill}%
}%
\begin{pgfscope}%
\pgfsys@transformshift{0.555080in}{0.417642in}%
\pgfsys@useobject{currentmarker}{}%
\end{pgfscope}%
\end{pgfscope}%
\begin{pgfscope}%
\pgfpathrectangle{\pgfqpoint{0.514278in}{0.417642in}}{\pgfqpoint{1.884996in}{1.371397in}}%
\pgfusepath{clip}%
\pgfsetrectcap%
\pgfsetroundjoin%
\pgfsetlinewidth{0.803000pt}%
\definecolor{currentstroke}{rgb}{0.850000,0.850000,0.850000}%
\pgfsetstrokecolor{currentstroke}%
\pgfsetdash{}{0pt}%
\pgfpathmoveto{\pgfqpoint{0.646217in}{0.417642in}}%
\pgfpathlineto{\pgfqpoint{0.646217in}{1.789039in}}%
\pgfusepath{stroke}%
\end{pgfscope}%
\begin{pgfscope}%
\pgfsetbuttcap%
\pgfsetroundjoin%
\definecolor{currentfill}{rgb}{0.000000,0.000000,0.000000}%
\pgfsetfillcolor{currentfill}%
\pgfsetlinewidth{0.602250pt}%
\definecolor{currentstroke}{rgb}{0.000000,0.000000,0.000000}%
\pgfsetstrokecolor{currentstroke}%
\pgfsetdash{}{0pt}%
\pgfsys@defobject{currentmarker}{\pgfqpoint{0.000000in}{-0.027778in}}{\pgfqpoint{0.000000in}{0.000000in}}{%
\pgfpathmoveto{\pgfqpoint{0.000000in}{0.000000in}}%
\pgfpathlineto{\pgfqpoint{0.000000in}{-0.027778in}}%
\pgfusepath{stroke,fill}%
}%
\begin{pgfscope}%
\pgfsys@transformshift{0.646217in}{0.417642in}%
\pgfsys@useobject{currentmarker}{}%
\end{pgfscope}%
\end{pgfscope}%
\begin{pgfscope}%
\pgfpathrectangle{\pgfqpoint{0.514278in}{0.417642in}}{\pgfqpoint{1.884996in}{1.371397in}}%
\pgfusepath{clip}%
\pgfsetrectcap%
\pgfsetroundjoin%
\pgfsetlinewidth{0.803000pt}%
\definecolor{currentstroke}{rgb}{0.850000,0.850000,0.850000}%
\pgfsetstrokecolor{currentstroke}%
\pgfsetdash{}{0pt}%
\pgfpathmoveto{\pgfqpoint{0.710880in}{0.417642in}}%
\pgfpathlineto{\pgfqpoint{0.710880in}{1.789039in}}%
\pgfusepath{stroke}%
\end{pgfscope}%
\begin{pgfscope}%
\pgfsetbuttcap%
\pgfsetroundjoin%
\definecolor{currentfill}{rgb}{0.000000,0.000000,0.000000}%
\pgfsetfillcolor{currentfill}%
\pgfsetlinewidth{0.602250pt}%
\definecolor{currentstroke}{rgb}{0.000000,0.000000,0.000000}%
\pgfsetstrokecolor{currentstroke}%
\pgfsetdash{}{0pt}%
\pgfsys@defobject{currentmarker}{\pgfqpoint{0.000000in}{-0.027778in}}{\pgfqpoint{0.000000in}{0.000000in}}{%
\pgfpathmoveto{\pgfqpoint{0.000000in}{0.000000in}}%
\pgfpathlineto{\pgfqpoint{0.000000in}{-0.027778in}}%
\pgfusepath{stroke,fill}%
}%
\begin{pgfscope}%
\pgfsys@transformshift{0.710880in}{0.417642in}%
\pgfsys@useobject{currentmarker}{}%
\end{pgfscope}%
\end{pgfscope}%
\begin{pgfscope}%
\pgfpathrectangle{\pgfqpoint{0.514278in}{0.417642in}}{\pgfqpoint{1.884996in}{1.371397in}}%
\pgfusepath{clip}%
\pgfsetrectcap%
\pgfsetroundjoin%
\pgfsetlinewidth{0.803000pt}%
\definecolor{currentstroke}{rgb}{0.850000,0.850000,0.850000}%
\pgfsetstrokecolor{currentstroke}%
\pgfsetdash{}{0pt}%
\pgfpathmoveto{\pgfqpoint{0.761036in}{0.417642in}}%
\pgfpathlineto{\pgfqpoint{0.761036in}{1.789039in}}%
\pgfusepath{stroke}%
\end{pgfscope}%
\begin{pgfscope}%
\pgfsetbuttcap%
\pgfsetroundjoin%
\definecolor{currentfill}{rgb}{0.000000,0.000000,0.000000}%
\pgfsetfillcolor{currentfill}%
\pgfsetlinewidth{0.602250pt}%
\definecolor{currentstroke}{rgb}{0.000000,0.000000,0.000000}%
\pgfsetstrokecolor{currentstroke}%
\pgfsetdash{}{0pt}%
\pgfsys@defobject{currentmarker}{\pgfqpoint{0.000000in}{-0.027778in}}{\pgfqpoint{0.000000in}{0.000000in}}{%
\pgfpathmoveto{\pgfqpoint{0.000000in}{0.000000in}}%
\pgfpathlineto{\pgfqpoint{0.000000in}{-0.027778in}}%
\pgfusepath{stroke,fill}%
}%
\begin{pgfscope}%
\pgfsys@transformshift{0.761036in}{0.417642in}%
\pgfsys@useobject{currentmarker}{}%
\end{pgfscope}%
\end{pgfscope}%
\begin{pgfscope}%
\pgfpathrectangle{\pgfqpoint{0.514278in}{0.417642in}}{\pgfqpoint{1.884996in}{1.371397in}}%
\pgfusepath{clip}%
\pgfsetrectcap%
\pgfsetroundjoin%
\pgfsetlinewidth{0.803000pt}%
\definecolor{currentstroke}{rgb}{0.850000,0.850000,0.850000}%
\pgfsetstrokecolor{currentstroke}%
\pgfsetdash{}{0pt}%
\pgfpathmoveto{\pgfqpoint{0.802017in}{0.417642in}}%
\pgfpathlineto{\pgfqpoint{0.802017in}{1.789039in}}%
\pgfusepath{stroke}%
\end{pgfscope}%
\begin{pgfscope}%
\pgfsetbuttcap%
\pgfsetroundjoin%
\definecolor{currentfill}{rgb}{0.000000,0.000000,0.000000}%
\pgfsetfillcolor{currentfill}%
\pgfsetlinewidth{0.602250pt}%
\definecolor{currentstroke}{rgb}{0.000000,0.000000,0.000000}%
\pgfsetstrokecolor{currentstroke}%
\pgfsetdash{}{0pt}%
\pgfsys@defobject{currentmarker}{\pgfqpoint{0.000000in}{-0.027778in}}{\pgfqpoint{0.000000in}{0.000000in}}{%
\pgfpathmoveto{\pgfqpoint{0.000000in}{0.000000in}}%
\pgfpathlineto{\pgfqpoint{0.000000in}{-0.027778in}}%
\pgfusepath{stroke,fill}%
}%
\begin{pgfscope}%
\pgfsys@transformshift{0.802017in}{0.417642in}%
\pgfsys@useobject{currentmarker}{}%
\end{pgfscope}%
\end{pgfscope}%
\begin{pgfscope}%
\pgfpathrectangle{\pgfqpoint{0.514278in}{0.417642in}}{\pgfqpoint{1.884996in}{1.371397in}}%
\pgfusepath{clip}%
\pgfsetrectcap%
\pgfsetroundjoin%
\pgfsetlinewidth{0.803000pt}%
\definecolor{currentstroke}{rgb}{0.850000,0.850000,0.850000}%
\pgfsetstrokecolor{currentstroke}%
\pgfsetdash{}{0pt}%
\pgfpathmoveto{\pgfqpoint{0.836665in}{0.417642in}}%
\pgfpathlineto{\pgfqpoint{0.836665in}{1.789039in}}%
\pgfusepath{stroke}%
\end{pgfscope}%
\begin{pgfscope}%
\pgfsetbuttcap%
\pgfsetroundjoin%
\definecolor{currentfill}{rgb}{0.000000,0.000000,0.000000}%
\pgfsetfillcolor{currentfill}%
\pgfsetlinewidth{0.602250pt}%
\definecolor{currentstroke}{rgb}{0.000000,0.000000,0.000000}%
\pgfsetstrokecolor{currentstroke}%
\pgfsetdash{}{0pt}%
\pgfsys@defobject{currentmarker}{\pgfqpoint{0.000000in}{-0.027778in}}{\pgfqpoint{0.000000in}{0.000000in}}{%
\pgfpathmoveto{\pgfqpoint{0.000000in}{0.000000in}}%
\pgfpathlineto{\pgfqpoint{0.000000in}{-0.027778in}}%
\pgfusepath{stroke,fill}%
}%
\begin{pgfscope}%
\pgfsys@transformshift{0.836665in}{0.417642in}%
\pgfsys@useobject{currentmarker}{}%
\end{pgfscope}%
\end{pgfscope}%
\begin{pgfscope}%
\pgfpathrectangle{\pgfqpoint{0.514278in}{0.417642in}}{\pgfqpoint{1.884996in}{1.371397in}}%
\pgfusepath{clip}%
\pgfsetrectcap%
\pgfsetroundjoin%
\pgfsetlinewidth{0.803000pt}%
\definecolor{currentstroke}{rgb}{0.850000,0.850000,0.850000}%
\pgfsetstrokecolor{currentstroke}%
\pgfsetdash{}{0pt}%
\pgfpathmoveto{\pgfqpoint{0.866679in}{0.417642in}}%
\pgfpathlineto{\pgfqpoint{0.866679in}{1.789039in}}%
\pgfusepath{stroke}%
\end{pgfscope}%
\begin{pgfscope}%
\pgfsetbuttcap%
\pgfsetroundjoin%
\definecolor{currentfill}{rgb}{0.000000,0.000000,0.000000}%
\pgfsetfillcolor{currentfill}%
\pgfsetlinewidth{0.602250pt}%
\definecolor{currentstroke}{rgb}{0.000000,0.000000,0.000000}%
\pgfsetstrokecolor{currentstroke}%
\pgfsetdash{}{0pt}%
\pgfsys@defobject{currentmarker}{\pgfqpoint{0.000000in}{-0.027778in}}{\pgfqpoint{0.000000in}{0.000000in}}{%
\pgfpathmoveto{\pgfqpoint{0.000000in}{0.000000in}}%
\pgfpathlineto{\pgfqpoint{0.000000in}{-0.027778in}}%
\pgfusepath{stroke,fill}%
}%
\begin{pgfscope}%
\pgfsys@transformshift{0.866679in}{0.417642in}%
\pgfsys@useobject{currentmarker}{}%
\end{pgfscope}%
\end{pgfscope}%
\begin{pgfscope}%
\pgfpathrectangle{\pgfqpoint{0.514278in}{0.417642in}}{\pgfqpoint{1.884996in}{1.371397in}}%
\pgfusepath{clip}%
\pgfsetrectcap%
\pgfsetroundjoin%
\pgfsetlinewidth{0.803000pt}%
\definecolor{currentstroke}{rgb}{0.850000,0.850000,0.850000}%
\pgfsetstrokecolor{currentstroke}%
\pgfsetdash{}{0pt}%
\pgfpathmoveto{\pgfqpoint{0.893154in}{0.417642in}}%
\pgfpathlineto{\pgfqpoint{0.893154in}{1.789039in}}%
\pgfusepath{stroke}%
\end{pgfscope}%
\begin{pgfscope}%
\pgfsetbuttcap%
\pgfsetroundjoin%
\definecolor{currentfill}{rgb}{0.000000,0.000000,0.000000}%
\pgfsetfillcolor{currentfill}%
\pgfsetlinewidth{0.602250pt}%
\definecolor{currentstroke}{rgb}{0.000000,0.000000,0.000000}%
\pgfsetstrokecolor{currentstroke}%
\pgfsetdash{}{0pt}%
\pgfsys@defobject{currentmarker}{\pgfqpoint{0.000000in}{-0.027778in}}{\pgfqpoint{0.000000in}{0.000000in}}{%
\pgfpathmoveto{\pgfqpoint{0.000000in}{0.000000in}}%
\pgfpathlineto{\pgfqpoint{0.000000in}{-0.027778in}}%
\pgfusepath{stroke,fill}%
}%
\begin{pgfscope}%
\pgfsys@transformshift{0.893154in}{0.417642in}%
\pgfsys@useobject{currentmarker}{}%
\end{pgfscope}%
\end{pgfscope}%
\begin{pgfscope}%
\pgfpathrectangle{\pgfqpoint{0.514278in}{0.417642in}}{\pgfqpoint{1.884996in}{1.371397in}}%
\pgfusepath{clip}%
\pgfsetrectcap%
\pgfsetroundjoin%
\pgfsetlinewidth{0.803000pt}%
\definecolor{currentstroke}{rgb}{0.850000,0.850000,0.850000}%
\pgfsetstrokecolor{currentstroke}%
\pgfsetdash{}{0pt}%
\pgfpathmoveto{\pgfqpoint{1.072635in}{0.417642in}}%
\pgfpathlineto{\pgfqpoint{1.072635in}{1.789039in}}%
\pgfusepath{stroke}%
\end{pgfscope}%
\begin{pgfscope}%
\pgfsetbuttcap%
\pgfsetroundjoin%
\definecolor{currentfill}{rgb}{0.000000,0.000000,0.000000}%
\pgfsetfillcolor{currentfill}%
\pgfsetlinewidth{0.602250pt}%
\definecolor{currentstroke}{rgb}{0.000000,0.000000,0.000000}%
\pgfsetstrokecolor{currentstroke}%
\pgfsetdash{}{0pt}%
\pgfsys@defobject{currentmarker}{\pgfqpoint{0.000000in}{-0.027778in}}{\pgfqpoint{0.000000in}{0.000000in}}{%
\pgfpathmoveto{\pgfqpoint{0.000000in}{0.000000in}}%
\pgfpathlineto{\pgfqpoint{0.000000in}{-0.027778in}}%
\pgfusepath{stroke,fill}%
}%
\begin{pgfscope}%
\pgfsys@transformshift{1.072635in}{0.417642in}%
\pgfsys@useobject{currentmarker}{}%
\end{pgfscope}%
\end{pgfscope}%
\begin{pgfscope}%
\pgfpathrectangle{\pgfqpoint{0.514278in}{0.417642in}}{\pgfqpoint{1.884996in}{1.371397in}}%
\pgfusepath{clip}%
\pgfsetrectcap%
\pgfsetroundjoin%
\pgfsetlinewidth{0.803000pt}%
\definecolor{currentstroke}{rgb}{0.850000,0.850000,0.850000}%
\pgfsetstrokecolor{currentstroke}%
\pgfsetdash{}{0pt}%
\pgfpathmoveto{\pgfqpoint{1.163773in}{0.417642in}}%
\pgfpathlineto{\pgfqpoint{1.163773in}{1.789039in}}%
\pgfusepath{stroke}%
\end{pgfscope}%
\begin{pgfscope}%
\pgfsetbuttcap%
\pgfsetroundjoin%
\definecolor{currentfill}{rgb}{0.000000,0.000000,0.000000}%
\pgfsetfillcolor{currentfill}%
\pgfsetlinewidth{0.602250pt}%
\definecolor{currentstroke}{rgb}{0.000000,0.000000,0.000000}%
\pgfsetstrokecolor{currentstroke}%
\pgfsetdash{}{0pt}%
\pgfsys@defobject{currentmarker}{\pgfqpoint{0.000000in}{-0.027778in}}{\pgfqpoint{0.000000in}{0.000000in}}{%
\pgfpathmoveto{\pgfqpoint{0.000000in}{0.000000in}}%
\pgfpathlineto{\pgfqpoint{0.000000in}{-0.027778in}}%
\pgfusepath{stroke,fill}%
}%
\begin{pgfscope}%
\pgfsys@transformshift{1.163773in}{0.417642in}%
\pgfsys@useobject{currentmarker}{}%
\end{pgfscope}%
\end{pgfscope}%
\begin{pgfscope}%
\pgfpathrectangle{\pgfqpoint{0.514278in}{0.417642in}}{\pgfqpoint{1.884996in}{1.371397in}}%
\pgfusepath{clip}%
\pgfsetrectcap%
\pgfsetroundjoin%
\pgfsetlinewidth{0.803000pt}%
\definecolor{currentstroke}{rgb}{0.850000,0.850000,0.850000}%
\pgfsetstrokecolor{currentstroke}%
\pgfsetdash{}{0pt}%
\pgfpathmoveto{\pgfqpoint{1.228435in}{0.417642in}}%
\pgfpathlineto{\pgfqpoint{1.228435in}{1.789039in}}%
\pgfusepath{stroke}%
\end{pgfscope}%
\begin{pgfscope}%
\pgfsetbuttcap%
\pgfsetroundjoin%
\definecolor{currentfill}{rgb}{0.000000,0.000000,0.000000}%
\pgfsetfillcolor{currentfill}%
\pgfsetlinewidth{0.602250pt}%
\definecolor{currentstroke}{rgb}{0.000000,0.000000,0.000000}%
\pgfsetstrokecolor{currentstroke}%
\pgfsetdash{}{0pt}%
\pgfsys@defobject{currentmarker}{\pgfqpoint{0.000000in}{-0.027778in}}{\pgfqpoint{0.000000in}{0.000000in}}{%
\pgfpathmoveto{\pgfqpoint{0.000000in}{0.000000in}}%
\pgfpathlineto{\pgfqpoint{0.000000in}{-0.027778in}}%
\pgfusepath{stroke,fill}%
}%
\begin{pgfscope}%
\pgfsys@transformshift{1.228435in}{0.417642in}%
\pgfsys@useobject{currentmarker}{}%
\end{pgfscope}%
\end{pgfscope}%
\begin{pgfscope}%
\pgfpathrectangle{\pgfqpoint{0.514278in}{0.417642in}}{\pgfqpoint{1.884996in}{1.371397in}}%
\pgfusepath{clip}%
\pgfsetrectcap%
\pgfsetroundjoin%
\pgfsetlinewidth{0.803000pt}%
\definecolor{currentstroke}{rgb}{0.850000,0.850000,0.850000}%
\pgfsetstrokecolor{currentstroke}%
\pgfsetdash{}{0pt}%
\pgfpathmoveto{\pgfqpoint{1.278592in}{0.417642in}}%
\pgfpathlineto{\pgfqpoint{1.278592in}{1.789039in}}%
\pgfusepath{stroke}%
\end{pgfscope}%
\begin{pgfscope}%
\pgfsetbuttcap%
\pgfsetroundjoin%
\definecolor{currentfill}{rgb}{0.000000,0.000000,0.000000}%
\pgfsetfillcolor{currentfill}%
\pgfsetlinewidth{0.602250pt}%
\definecolor{currentstroke}{rgb}{0.000000,0.000000,0.000000}%
\pgfsetstrokecolor{currentstroke}%
\pgfsetdash{}{0pt}%
\pgfsys@defobject{currentmarker}{\pgfqpoint{0.000000in}{-0.027778in}}{\pgfqpoint{0.000000in}{0.000000in}}{%
\pgfpathmoveto{\pgfqpoint{0.000000in}{0.000000in}}%
\pgfpathlineto{\pgfqpoint{0.000000in}{-0.027778in}}%
\pgfusepath{stroke,fill}%
}%
\begin{pgfscope}%
\pgfsys@transformshift{1.278592in}{0.417642in}%
\pgfsys@useobject{currentmarker}{}%
\end{pgfscope}%
\end{pgfscope}%
\begin{pgfscope}%
\pgfpathrectangle{\pgfqpoint{0.514278in}{0.417642in}}{\pgfqpoint{1.884996in}{1.371397in}}%
\pgfusepath{clip}%
\pgfsetrectcap%
\pgfsetroundjoin%
\pgfsetlinewidth{0.803000pt}%
\definecolor{currentstroke}{rgb}{0.850000,0.850000,0.850000}%
\pgfsetstrokecolor{currentstroke}%
\pgfsetdash{}{0pt}%
\pgfpathmoveto{\pgfqpoint{1.319572in}{0.417642in}}%
\pgfpathlineto{\pgfqpoint{1.319572in}{1.789039in}}%
\pgfusepath{stroke}%
\end{pgfscope}%
\begin{pgfscope}%
\pgfsetbuttcap%
\pgfsetroundjoin%
\definecolor{currentfill}{rgb}{0.000000,0.000000,0.000000}%
\pgfsetfillcolor{currentfill}%
\pgfsetlinewidth{0.602250pt}%
\definecolor{currentstroke}{rgb}{0.000000,0.000000,0.000000}%
\pgfsetstrokecolor{currentstroke}%
\pgfsetdash{}{0pt}%
\pgfsys@defobject{currentmarker}{\pgfqpoint{0.000000in}{-0.027778in}}{\pgfqpoint{0.000000in}{0.000000in}}{%
\pgfpathmoveto{\pgfqpoint{0.000000in}{0.000000in}}%
\pgfpathlineto{\pgfqpoint{0.000000in}{-0.027778in}}%
\pgfusepath{stroke,fill}%
}%
\begin{pgfscope}%
\pgfsys@transformshift{1.319572in}{0.417642in}%
\pgfsys@useobject{currentmarker}{}%
\end{pgfscope}%
\end{pgfscope}%
\begin{pgfscope}%
\pgfpathrectangle{\pgfqpoint{0.514278in}{0.417642in}}{\pgfqpoint{1.884996in}{1.371397in}}%
\pgfusepath{clip}%
\pgfsetrectcap%
\pgfsetroundjoin%
\pgfsetlinewidth{0.803000pt}%
\definecolor{currentstroke}{rgb}{0.850000,0.850000,0.850000}%
\pgfsetstrokecolor{currentstroke}%
\pgfsetdash{}{0pt}%
\pgfpathmoveto{\pgfqpoint{1.354221in}{0.417642in}}%
\pgfpathlineto{\pgfqpoint{1.354221in}{1.789039in}}%
\pgfusepath{stroke}%
\end{pgfscope}%
\begin{pgfscope}%
\pgfsetbuttcap%
\pgfsetroundjoin%
\definecolor{currentfill}{rgb}{0.000000,0.000000,0.000000}%
\pgfsetfillcolor{currentfill}%
\pgfsetlinewidth{0.602250pt}%
\definecolor{currentstroke}{rgb}{0.000000,0.000000,0.000000}%
\pgfsetstrokecolor{currentstroke}%
\pgfsetdash{}{0pt}%
\pgfsys@defobject{currentmarker}{\pgfqpoint{0.000000in}{-0.027778in}}{\pgfqpoint{0.000000in}{0.000000in}}{%
\pgfpathmoveto{\pgfqpoint{0.000000in}{0.000000in}}%
\pgfpathlineto{\pgfqpoint{0.000000in}{-0.027778in}}%
\pgfusepath{stroke,fill}%
}%
\begin{pgfscope}%
\pgfsys@transformshift{1.354221in}{0.417642in}%
\pgfsys@useobject{currentmarker}{}%
\end{pgfscope}%
\end{pgfscope}%
\begin{pgfscope}%
\pgfpathrectangle{\pgfqpoint{0.514278in}{0.417642in}}{\pgfqpoint{1.884996in}{1.371397in}}%
\pgfusepath{clip}%
\pgfsetrectcap%
\pgfsetroundjoin%
\pgfsetlinewidth{0.803000pt}%
\definecolor{currentstroke}{rgb}{0.850000,0.850000,0.850000}%
\pgfsetstrokecolor{currentstroke}%
\pgfsetdash{}{0pt}%
\pgfpathmoveto{\pgfqpoint{1.384235in}{0.417642in}}%
\pgfpathlineto{\pgfqpoint{1.384235in}{1.789039in}}%
\pgfusepath{stroke}%
\end{pgfscope}%
\begin{pgfscope}%
\pgfsetbuttcap%
\pgfsetroundjoin%
\definecolor{currentfill}{rgb}{0.000000,0.000000,0.000000}%
\pgfsetfillcolor{currentfill}%
\pgfsetlinewidth{0.602250pt}%
\definecolor{currentstroke}{rgb}{0.000000,0.000000,0.000000}%
\pgfsetstrokecolor{currentstroke}%
\pgfsetdash{}{0pt}%
\pgfsys@defobject{currentmarker}{\pgfqpoint{0.000000in}{-0.027778in}}{\pgfqpoint{0.000000in}{0.000000in}}{%
\pgfpathmoveto{\pgfqpoint{0.000000in}{0.000000in}}%
\pgfpathlineto{\pgfqpoint{0.000000in}{-0.027778in}}%
\pgfusepath{stroke,fill}%
}%
\begin{pgfscope}%
\pgfsys@transformshift{1.384235in}{0.417642in}%
\pgfsys@useobject{currentmarker}{}%
\end{pgfscope}%
\end{pgfscope}%
\begin{pgfscope}%
\pgfpathrectangle{\pgfqpoint{0.514278in}{0.417642in}}{\pgfqpoint{1.884996in}{1.371397in}}%
\pgfusepath{clip}%
\pgfsetrectcap%
\pgfsetroundjoin%
\pgfsetlinewidth{0.803000pt}%
\definecolor{currentstroke}{rgb}{0.850000,0.850000,0.850000}%
\pgfsetstrokecolor{currentstroke}%
\pgfsetdash{}{0pt}%
\pgfpathmoveto{\pgfqpoint{1.410709in}{0.417642in}}%
\pgfpathlineto{\pgfqpoint{1.410709in}{1.789039in}}%
\pgfusepath{stroke}%
\end{pgfscope}%
\begin{pgfscope}%
\pgfsetbuttcap%
\pgfsetroundjoin%
\definecolor{currentfill}{rgb}{0.000000,0.000000,0.000000}%
\pgfsetfillcolor{currentfill}%
\pgfsetlinewidth{0.602250pt}%
\definecolor{currentstroke}{rgb}{0.000000,0.000000,0.000000}%
\pgfsetstrokecolor{currentstroke}%
\pgfsetdash{}{0pt}%
\pgfsys@defobject{currentmarker}{\pgfqpoint{0.000000in}{-0.027778in}}{\pgfqpoint{0.000000in}{0.000000in}}{%
\pgfpathmoveto{\pgfqpoint{0.000000in}{0.000000in}}%
\pgfpathlineto{\pgfqpoint{0.000000in}{-0.027778in}}%
\pgfusepath{stroke,fill}%
}%
\begin{pgfscope}%
\pgfsys@transformshift{1.410709in}{0.417642in}%
\pgfsys@useobject{currentmarker}{}%
\end{pgfscope}%
\end{pgfscope}%
\begin{pgfscope}%
\pgfpathrectangle{\pgfqpoint{0.514278in}{0.417642in}}{\pgfqpoint{1.884996in}{1.371397in}}%
\pgfusepath{clip}%
\pgfsetrectcap%
\pgfsetroundjoin%
\pgfsetlinewidth{0.803000pt}%
\definecolor{currentstroke}{rgb}{0.850000,0.850000,0.850000}%
\pgfsetstrokecolor{currentstroke}%
\pgfsetdash{}{0pt}%
\pgfpathmoveto{\pgfqpoint{1.590191in}{0.417642in}}%
\pgfpathlineto{\pgfqpoint{1.590191in}{1.789039in}}%
\pgfusepath{stroke}%
\end{pgfscope}%
\begin{pgfscope}%
\pgfsetbuttcap%
\pgfsetroundjoin%
\definecolor{currentfill}{rgb}{0.000000,0.000000,0.000000}%
\pgfsetfillcolor{currentfill}%
\pgfsetlinewidth{0.602250pt}%
\definecolor{currentstroke}{rgb}{0.000000,0.000000,0.000000}%
\pgfsetstrokecolor{currentstroke}%
\pgfsetdash{}{0pt}%
\pgfsys@defobject{currentmarker}{\pgfqpoint{0.000000in}{-0.027778in}}{\pgfqpoint{0.000000in}{0.000000in}}{%
\pgfpathmoveto{\pgfqpoint{0.000000in}{0.000000in}}%
\pgfpathlineto{\pgfqpoint{0.000000in}{-0.027778in}}%
\pgfusepath{stroke,fill}%
}%
\begin{pgfscope}%
\pgfsys@transformshift{1.590191in}{0.417642in}%
\pgfsys@useobject{currentmarker}{}%
\end{pgfscope}%
\end{pgfscope}%
\begin{pgfscope}%
\pgfpathrectangle{\pgfqpoint{0.514278in}{0.417642in}}{\pgfqpoint{1.884996in}{1.371397in}}%
\pgfusepath{clip}%
\pgfsetrectcap%
\pgfsetroundjoin%
\pgfsetlinewidth{0.803000pt}%
\definecolor{currentstroke}{rgb}{0.850000,0.850000,0.850000}%
\pgfsetstrokecolor{currentstroke}%
\pgfsetdash{}{0pt}%
\pgfpathmoveto{\pgfqpoint{1.681328in}{0.417642in}}%
\pgfpathlineto{\pgfqpoint{1.681328in}{1.789039in}}%
\pgfusepath{stroke}%
\end{pgfscope}%
\begin{pgfscope}%
\pgfsetbuttcap%
\pgfsetroundjoin%
\definecolor{currentfill}{rgb}{0.000000,0.000000,0.000000}%
\pgfsetfillcolor{currentfill}%
\pgfsetlinewidth{0.602250pt}%
\definecolor{currentstroke}{rgb}{0.000000,0.000000,0.000000}%
\pgfsetstrokecolor{currentstroke}%
\pgfsetdash{}{0pt}%
\pgfsys@defobject{currentmarker}{\pgfqpoint{0.000000in}{-0.027778in}}{\pgfqpoint{0.000000in}{0.000000in}}{%
\pgfpathmoveto{\pgfqpoint{0.000000in}{0.000000in}}%
\pgfpathlineto{\pgfqpoint{0.000000in}{-0.027778in}}%
\pgfusepath{stroke,fill}%
}%
\begin{pgfscope}%
\pgfsys@transformshift{1.681328in}{0.417642in}%
\pgfsys@useobject{currentmarker}{}%
\end{pgfscope}%
\end{pgfscope}%
\begin{pgfscope}%
\pgfpathrectangle{\pgfqpoint{0.514278in}{0.417642in}}{\pgfqpoint{1.884996in}{1.371397in}}%
\pgfusepath{clip}%
\pgfsetrectcap%
\pgfsetroundjoin%
\pgfsetlinewidth{0.803000pt}%
\definecolor{currentstroke}{rgb}{0.850000,0.850000,0.850000}%
\pgfsetstrokecolor{currentstroke}%
\pgfsetdash{}{0pt}%
\pgfpathmoveto{\pgfqpoint{1.745991in}{0.417642in}}%
\pgfpathlineto{\pgfqpoint{1.745991in}{1.789039in}}%
\pgfusepath{stroke}%
\end{pgfscope}%
\begin{pgfscope}%
\pgfsetbuttcap%
\pgfsetroundjoin%
\definecolor{currentfill}{rgb}{0.000000,0.000000,0.000000}%
\pgfsetfillcolor{currentfill}%
\pgfsetlinewidth{0.602250pt}%
\definecolor{currentstroke}{rgb}{0.000000,0.000000,0.000000}%
\pgfsetstrokecolor{currentstroke}%
\pgfsetdash{}{0pt}%
\pgfsys@defobject{currentmarker}{\pgfqpoint{0.000000in}{-0.027778in}}{\pgfqpoint{0.000000in}{0.000000in}}{%
\pgfpathmoveto{\pgfqpoint{0.000000in}{0.000000in}}%
\pgfpathlineto{\pgfqpoint{0.000000in}{-0.027778in}}%
\pgfusepath{stroke,fill}%
}%
\begin{pgfscope}%
\pgfsys@transformshift{1.745991in}{0.417642in}%
\pgfsys@useobject{currentmarker}{}%
\end{pgfscope}%
\end{pgfscope}%
\begin{pgfscope}%
\pgfpathrectangle{\pgfqpoint{0.514278in}{0.417642in}}{\pgfqpoint{1.884996in}{1.371397in}}%
\pgfusepath{clip}%
\pgfsetrectcap%
\pgfsetroundjoin%
\pgfsetlinewidth{0.803000pt}%
\definecolor{currentstroke}{rgb}{0.850000,0.850000,0.850000}%
\pgfsetstrokecolor{currentstroke}%
\pgfsetdash{}{0pt}%
\pgfpathmoveto{\pgfqpoint{1.796147in}{0.417642in}}%
\pgfpathlineto{\pgfqpoint{1.796147in}{1.789039in}}%
\pgfusepath{stroke}%
\end{pgfscope}%
\begin{pgfscope}%
\pgfsetbuttcap%
\pgfsetroundjoin%
\definecolor{currentfill}{rgb}{0.000000,0.000000,0.000000}%
\pgfsetfillcolor{currentfill}%
\pgfsetlinewidth{0.602250pt}%
\definecolor{currentstroke}{rgb}{0.000000,0.000000,0.000000}%
\pgfsetstrokecolor{currentstroke}%
\pgfsetdash{}{0pt}%
\pgfsys@defobject{currentmarker}{\pgfqpoint{0.000000in}{-0.027778in}}{\pgfqpoint{0.000000in}{0.000000in}}{%
\pgfpathmoveto{\pgfqpoint{0.000000in}{0.000000in}}%
\pgfpathlineto{\pgfqpoint{0.000000in}{-0.027778in}}%
\pgfusepath{stroke,fill}%
}%
\begin{pgfscope}%
\pgfsys@transformshift{1.796147in}{0.417642in}%
\pgfsys@useobject{currentmarker}{}%
\end{pgfscope}%
\end{pgfscope}%
\begin{pgfscope}%
\pgfpathrectangle{\pgfqpoint{0.514278in}{0.417642in}}{\pgfqpoint{1.884996in}{1.371397in}}%
\pgfusepath{clip}%
\pgfsetrectcap%
\pgfsetroundjoin%
\pgfsetlinewidth{0.803000pt}%
\definecolor{currentstroke}{rgb}{0.850000,0.850000,0.850000}%
\pgfsetstrokecolor{currentstroke}%
\pgfsetdash{}{0pt}%
\pgfpathmoveto{\pgfqpoint{1.837128in}{0.417642in}}%
\pgfpathlineto{\pgfqpoint{1.837128in}{1.789039in}}%
\pgfusepath{stroke}%
\end{pgfscope}%
\begin{pgfscope}%
\pgfsetbuttcap%
\pgfsetroundjoin%
\definecolor{currentfill}{rgb}{0.000000,0.000000,0.000000}%
\pgfsetfillcolor{currentfill}%
\pgfsetlinewidth{0.602250pt}%
\definecolor{currentstroke}{rgb}{0.000000,0.000000,0.000000}%
\pgfsetstrokecolor{currentstroke}%
\pgfsetdash{}{0pt}%
\pgfsys@defobject{currentmarker}{\pgfqpoint{0.000000in}{-0.027778in}}{\pgfqpoint{0.000000in}{0.000000in}}{%
\pgfpathmoveto{\pgfqpoint{0.000000in}{0.000000in}}%
\pgfpathlineto{\pgfqpoint{0.000000in}{-0.027778in}}%
\pgfusepath{stroke,fill}%
}%
\begin{pgfscope}%
\pgfsys@transformshift{1.837128in}{0.417642in}%
\pgfsys@useobject{currentmarker}{}%
\end{pgfscope}%
\end{pgfscope}%
\begin{pgfscope}%
\pgfpathrectangle{\pgfqpoint{0.514278in}{0.417642in}}{\pgfqpoint{1.884996in}{1.371397in}}%
\pgfusepath{clip}%
\pgfsetrectcap%
\pgfsetroundjoin%
\pgfsetlinewidth{0.803000pt}%
\definecolor{currentstroke}{rgb}{0.850000,0.850000,0.850000}%
\pgfsetstrokecolor{currentstroke}%
\pgfsetdash{}{0pt}%
\pgfpathmoveto{\pgfqpoint{1.871777in}{0.417642in}}%
\pgfpathlineto{\pgfqpoint{1.871777in}{1.789039in}}%
\pgfusepath{stroke}%
\end{pgfscope}%
\begin{pgfscope}%
\pgfsetbuttcap%
\pgfsetroundjoin%
\definecolor{currentfill}{rgb}{0.000000,0.000000,0.000000}%
\pgfsetfillcolor{currentfill}%
\pgfsetlinewidth{0.602250pt}%
\definecolor{currentstroke}{rgb}{0.000000,0.000000,0.000000}%
\pgfsetstrokecolor{currentstroke}%
\pgfsetdash{}{0pt}%
\pgfsys@defobject{currentmarker}{\pgfqpoint{0.000000in}{-0.027778in}}{\pgfqpoint{0.000000in}{0.000000in}}{%
\pgfpathmoveto{\pgfqpoint{0.000000in}{0.000000in}}%
\pgfpathlineto{\pgfqpoint{0.000000in}{-0.027778in}}%
\pgfusepath{stroke,fill}%
}%
\begin{pgfscope}%
\pgfsys@transformshift{1.871777in}{0.417642in}%
\pgfsys@useobject{currentmarker}{}%
\end{pgfscope}%
\end{pgfscope}%
\begin{pgfscope}%
\pgfpathrectangle{\pgfqpoint{0.514278in}{0.417642in}}{\pgfqpoint{1.884996in}{1.371397in}}%
\pgfusepath{clip}%
\pgfsetrectcap%
\pgfsetroundjoin%
\pgfsetlinewidth{0.803000pt}%
\definecolor{currentstroke}{rgb}{0.850000,0.850000,0.850000}%
\pgfsetstrokecolor{currentstroke}%
\pgfsetdash{}{0pt}%
\pgfpathmoveto{\pgfqpoint{1.901791in}{0.417642in}}%
\pgfpathlineto{\pgfqpoint{1.901791in}{1.789039in}}%
\pgfusepath{stroke}%
\end{pgfscope}%
\begin{pgfscope}%
\pgfsetbuttcap%
\pgfsetroundjoin%
\definecolor{currentfill}{rgb}{0.000000,0.000000,0.000000}%
\pgfsetfillcolor{currentfill}%
\pgfsetlinewidth{0.602250pt}%
\definecolor{currentstroke}{rgb}{0.000000,0.000000,0.000000}%
\pgfsetstrokecolor{currentstroke}%
\pgfsetdash{}{0pt}%
\pgfsys@defobject{currentmarker}{\pgfqpoint{0.000000in}{-0.027778in}}{\pgfqpoint{0.000000in}{0.000000in}}{%
\pgfpathmoveto{\pgfqpoint{0.000000in}{0.000000in}}%
\pgfpathlineto{\pgfqpoint{0.000000in}{-0.027778in}}%
\pgfusepath{stroke,fill}%
}%
\begin{pgfscope}%
\pgfsys@transformshift{1.901791in}{0.417642in}%
\pgfsys@useobject{currentmarker}{}%
\end{pgfscope}%
\end{pgfscope}%
\begin{pgfscope}%
\pgfpathrectangle{\pgfqpoint{0.514278in}{0.417642in}}{\pgfqpoint{1.884996in}{1.371397in}}%
\pgfusepath{clip}%
\pgfsetrectcap%
\pgfsetroundjoin%
\pgfsetlinewidth{0.803000pt}%
\definecolor{currentstroke}{rgb}{0.850000,0.850000,0.850000}%
\pgfsetstrokecolor{currentstroke}%
\pgfsetdash{}{0pt}%
\pgfpathmoveto{\pgfqpoint{1.928265in}{0.417642in}}%
\pgfpathlineto{\pgfqpoint{1.928265in}{1.789039in}}%
\pgfusepath{stroke}%
\end{pgfscope}%
\begin{pgfscope}%
\pgfsetbuttcap%
\pgfsetroundjoin%
\definecolor{currentfill}{rgb}{0.000000,0.000000,0.000000}%
\pgfsetfillcolor{currentfill}%
\pgfsetlinewidth{0.602250pt}%
\definecolor{currentstroke}{rgb}{0.000000,0.000000,0.000000}%
\pgfsetstrokecolor{currentstroke}%
\pgfsetdash{}{0pt}%
\pgfsys@defobject{currentmarker}{\pgfqpoint{0.000000in}{-0.027778in}}{\pgfqpoint{0.000000in}{0.000000in}}{%
\pgfpathmoveto{\pgfqpoint{0.000000in}{0.000000in}}%
\pgfpathlineto{\pgfqpoint{0.000000in}{-0.027778in}}%
\pgfusepath{stroke,fill}%
}%
\begin{pgfscope}%
\pgfsys@transformshift{1.928265in}{0.417642in}%
\pgfsys@useobject{currentmarker}{}%
\end{pgfscope}%
\end{pgfscope}%
\begin{pgfscope}%
\pgfpathrectangle{\pgfqpoint{0.514278in}{0.417642in}}{\pgfqpoint{1.884996in}{1.371397in}}%
\pgfusepath{clip}%
\pgfsetrectcap%
\pgfsetroundjoin%
\pgfsetlinewidth{0.803000pt}%
\definecolor{currentstroke}{rgb}{0.850000,0.850000,0.850000}%
\pgfsetstrokecolor{currentstroke}%
\pgfsetdash{}{0pt}%
\pgfpathmoveto{\pgfqpoint{2.107747in}{0.417642in}}%
\pgfpathlineto{\pgfqpoint{2.107747in}{1.789039in}}%
\pgfusepath{stroke}%
\end{pgfscope}%
\begin{pgfscope}%
\pgfsetbuttcap%
\pgfsetroundjoin%
\definecolor{currentfill}{rgb}{0.000000,0.000000,0.000000}%
\pgfsetfillcolor{currentfill}%
\pgfsetlinewidth{0.602250pt}%
\definecolor{currentstroke}{rgb}{0.000000,0.000000,0.000000}%
\pgfsetstrokecolor{currentstroke}%
\pgfsetdash{}{0pt}%
\pgfsys@defobject{currentmarker}{\pgfqpoint{0.000000in}{-0.027778in}}{\pgfqpoint{0.000000in}{0.000000in}}{%
\pgfpathmoveto{\pgfqpoint{0.000000in}{0.000000in}}%
\pgfpathlineto{\pgfqpoint{0.000000in}{-0.027778in}}%
\pgfusepath{stroke,fill}%
}%
\begin{pgfscope}%
\pgfsys@transformshift{2.107747in}{0.417642in}%
\pgfsys@useobject{currentmarker}{}%
\end{pgfscope}%
\end{pgfscope}%
\begin{pgfscope}%
\pgfpathrectangle{\pgfqpoint{0.514278in}{0.417642in}}{\pgfqpoint{1.884996in}{1.371397in}}%
\pgfusepath{clip}%
\pgfsetrectcap%
\pgfsetroundjoin%
\pgfsetlinewidth{0.803000pt}%
\definecolor{currentstroke}{rgb}{0.850000,0.850000,0.850000}%
\pgfsetstrokecolor{currentstroke}%
\pgfsetdash{}{0pt}%
\pgfpathmoveto{\pgfqpoint{2.198884in}{0.417642in}}%
\pgfpathlineto{\pgfqpoint{2.198884in}{1.789039in}}%
\pgfusepath{stroke}%
\end{pgfscope}%
\begin{pgfscope}%
\pgfsetbuttcap%
\pgfsetroundjoin%
\definecolor{currentfill}{rgb}{0.000000,0.000000,0.000000}%
\pgfsetfillcolor{currentfill}%
\pgfsetlinewidth{0.602250pt}%
\definecolor{currentstroke}{rgb}{0.000000,0.000000,0.000000}%
\pgfsetstrokecolor{currentstroke}%
\pgfsetdash{}{0pt}%
\pgfsys@defobject{currentmarker}{\pgfqpoint{0.000000in}{-0.027778in}}{\pgfqpoint{0.000000in}{0.000000in}}{%
\pgfpathmoveto{\pgfqpoint{0.000000in}{0.000000in}}%
\pgfpathlineto{\pgfqpoint{0.000000in}{-0.027778in}}%
\pgfusepath{stroke,fill}%
}%
\begin{pgfscope}%
\pgfsys@transformshift{2.198884in}{0.417642in}%
\pgfsys@useobject{currentmarker}{}%
\end{pgfscope}%
\end{pgfscope}%
\begin{pgfscope}%
\pgfpathrectangle{\pgfqpoint{0.514278in}{0.417642in}}{\pgfqpoint{1.884996in}{1.371397in}}%
\pgfusepath{clip}%
\pgfsetrectcap%
\pgfsetroundjoin%
\pgfsetlinewidth{0.803000pt}%
\definecolor{currentstroke}{rgb}{0.850000,0.850000,0.850000}%
\pgfsetstrokecolor{currentstroke}%
\pgfsetdash{}{0pt}%
\pgfpathmoveto{\pgfqpoint{2.263547in}{0.417642in}}%
\pgfpathlineto{\pgfqpoint{2.263547in}{1.789039in}}%
\pgfusepath{stroke}%
\end{pgfscope}%
\begin{pgfscope}%
\pgfsetbuttcap%
\pgfsetroundjoin%
\definecolor{currentfill}{rgb}{0.000000,0.000000,0.000000}%
\pgfsetfillcolor{currentfill}%
\pgfsetlinewidth{0.602250pt}%
\definecolor{currentstroke}{rgb}{0.000000,0.000000,0.000000}%
\pgfsetstrokecolor{currentstroke}%
\pgfsetdash{}{0pt}%
\pgfsys@defobject{currentmarker}{\pgfqpoint{0.000000in}{-0.027778in}}{\pgfqpoint{0.000000in}{0.000000in}}{%
\pgfpathmoveto{\pgfqpoint{0.000000in}{0.000000in}}%
\pgfpathlineto{\pgfqpoint{0.000000in}{-0.027778in}}%
\pgfusepath{stroke,fill}%
}%
\begin{pgfscope}%
\pgfsys@transformshift{2.263547in}{0.417642in}%
\pgfsys@useobject{currentmarker}{}%
\end{pgfscope}%
\end{pgfscope}%
\begin{pgfscope}%
\pgfpathrectangle{\pgfqpoint{0.514278in}{0.417642in}}{\pgfqpoint{1.884996in}{1.371397in}}%
\pgfusepath{clip}%
\pgfsetrectcap%
\pgfsetroundjoin%
\pgfsetlinewidth{0.803000pt}%
\definecolor{currentstroke}{rgb}{0.850000,0.850000,0.850000}%
\pgfsetstrokecolor{currentstroke}%
\pgfsetdash{}{0pt}%
\pgfpathmoveto{\pgfqpoint{2.313703in}{0.417642in}}%
\pgfpathlineto{\pgfqpoint{2.313703in}{1.789039in}}%
\pgfusepath{stroke}%
\end{pgfscope}%
\begin{pgfscope}%
\pgfsetbuttcap%
\pgfsetroundjoin%
\definecolor{currentfill}{rgb}{0.000000,0.000000,0.000000}%
\pgfsetfillcolor{currentfill}%
\pgfsetlinewidth{0.602250pt}%
\definecolor{currentstroke}{rgb}{0.000000,0.000000,0.000000}%
\pgfsetstrokecolor{currentstroke}%
\pgfsetdash{}{0pt}%
\pgfsys@defobject{currentmarker}{\pgfqpoint{0.000000in}{-0.027778in}}{\pgfqpoint{0.000000in}{0.000000in}}{%
\pgfpathmoveto{\pgfqpoint{0.000000in}{0.000000in}}%
\pgfpathlineto{\pgfqpoint{0.000000in}{-0.027778in}}%
\pgfusepath{stroke,fill}%
}%
\begin{pgfscope}%
\pgfsys@transformshift{2.313703in}{0.417642in}%
\pgfsys@useobject{currentmarker}{}%
\end{pgfscope}%
\end{pgfscope}%
\begin{pgfscope}%
\pgfpathrectangle{\pgfqpoint{0.514278in}{0.417642in}}{\pgfqpoint{1.884996in}{1.371397in}}%
\pgfusepath{clip}%
\pgfsetrectcap%
\pgfsetroundjoin%
\pgfsetlinewidth{0.803000pt}%
\definecolor{currentstroke}{rgb}{0.850000,0.850000,0.850000}%
\pgfsetstrokecolor{currentstroke}%
\pgfsetdash{}{0pt}%
\pgfpathmoveto{\pgfqpoint{2.354684in}{0.417642in}}%
\pgfpathlineto{\pgfqpoint{2.354684in}{1.789039in}}%
\pgfusepath{stroke}%
\end{pgfscope}%
\begin{pgfscope}%
\pgfsetbuttcap%
\pgfsetroundjoin%
\definecolor{currentfill}{rgb}{0.000000,0.000000,0.000000}%
\pgfsetfillcolor{currentfill}%
\pgfsetlinewidth{0.602250pt}%
\definecolor{currentstroke}{rgb}{0.000000,0.000000,0.000000}%
\pgfsetstrokecolor{currentstroke}%
\pgfsetdash{}{0pt}%
\pgfsys@defobject{currentmarker}{\pgfqpoint{0.000000in}{-0.027778in}}{\pgfqpoint{0.000000in}{0.000000in}}{%
\pgfpathmoveto{\pgfqpoint{0.000000in}{0.000000in}}%
\pgfpathlineto{\pgfqpoint{0.000000in}{-0.027778in}}%
\pgfusepath{stroke,fill}%
}%
\begin{pgfscope}%
\pgfsys@transformshift{2.354684in}{0.417642in}%
\pgfsys@useobject{currentmarker}{}%
\end{pgfscope}%
\end{pgfscope}%
\begin{pgfscope}%
\pgfpathrectangle{\pgfqpoint{0.514278in}{0.417642in}}{\pgfqpoint{1.884996in}{1.371397in}}%
\pgfusepath{clip}%
\pgfsetrectcap%
\pgfsetroundjoin%
\pgfsetlinewidth{0.803000pt}%
\definecolor{currentstroke}{rgb}{0.850000,0.850000,0.850000}%
\pgfsetstrokecolor{currentstroke}%
\pgfsetdash{}{0pt}%
\pgfpathmoveto{\pgfqpoint{2.389333in}{0.417642in}}%
\pgfpathlineto{\pgfqpoint{2.389333in}{1.789039in}}%
\pgfusepath{stroke}%
\end{pgfscope}%
\begin{pgfscope}%
\pgfsetbuttcap%
\pgfsetroundjoin%
\definecolor{currentfill}{rgb}{0.000000,0.000000,0.000000}%
\pgfsetfillcolor{currentfill}%
\pgfsetlinewidth{0.602250pt}%
\definecolor{currentstroke}{rgb}{0.000000,0.000000,0.000000}%
\pgfsetstrokecolor{currentstroke}%
\pgfsetdash{}{0pt}%
\pgfsys@defobject{currentmarker}{\pgfqpoint{0.000000in}{-0.027778in}}{\pgfqpoint{0.000000in}{0.000000in}}{%
\pgfpathmoveto{\pgfqpoint{0.000000in}{0.000000in}}%
\pgfpathlineto{\pgfqpoint{0.000000in}{-0.027778in}}%
\pgfusepath{stroke,fill}%
}%
\begin{pgfscope}%
\pgfsys@transformshift{2.389333in}{0.417642in}%
\pgfsys@useobject{currentmarker}{}%
\end{pgfscope}%
\end{pgfscope}%
\begin{pgfscope}%
\definecolor{textcolor}{rgb}{0.000000,0.000000,0.000000}%
\pgfsetstrokecolor{textcolor}%
\pgfsetfillcolor{textcolor}%
\pgftext[x=1.456777in,y=0.165003in,,top]{\color{textcolor}{\rmfamily\fontsize{10.000000}{12.000000}\selectfont\catcode`\^=\active\def^{\ifmmode\sp\else\^{}\fi}\catcode`\%=\active\def%{\%}Frequency in $\unit{\Hz}$}}%
\end{pgfscope}%
\begin{pgfscope}%
\pgfpathrectangle{\pgfqpoint{0.514278in}{0.417642in}}{\pgfqpoint{1.884996in}{1.371397in}}%
\pgfusepath{clip}%
\pgfsetrectcap%
\pgfsetroundjoin%
\pgfsetlinewidth{0.803000pt}%
\definecolor{currentstroke}{rgb}{0.450000,0.450000,0.450000}%
\pgfsetstrokecolor{currentstroke}%
\pgfsetdash{}{0pt}%
\pgfpathmoveto{\pgfqpoint{0.514278in}{0.640670in}}%
\pgfpathlineto{\pgfqpoint{2.399275in}{0.640670in}}%
\pgfusepath{stroke}%
\end{pgfscope}%
\begin{pgfscope}%
\pgfsetbuttcap%
\pgfsetroundjoin%
\definecolor{currentfill}{rgb}{0.000000,0.000000,0.000000}%
\pgfsetfillcolor{currentfill}%
\pgfsetlinewidth{0.803000pt}%
\definecolor{currentstroke}{rgb}{0.000000,0.000000,0.000000}%
\pgfsetstrokecolor{currentstroke}%
\pgfsetdash{}{0pt}%
\pgfsys@defobject{currentmarker}{\pgfqpoint{-0.048611in}{0.000000in}}{\pgfqpoint{-0.000000in}{0.000000in}}{%
\pgfpathmoveto{\pgfqpoint{-0.000000in}{0.000000in}}%
\pgfpathlineto{\pgfqpoint{-0.048611in}{0.000000in}}%
\pgfusepath{stroke,fill}%
}%
\begin{pgfscope}%
\pgfsys@transformshift{0.514278in}{0.640670in}%
\pgfsys@useobject{currentmarker}{}%
\end{pgfscope}%
\end{pgfscope}%
\begin{pgfscope}%
\definecolor{textcolor}{rgb}{0.000000,0.000000,0.000000}%
\pgfsetstrokecolor{textcolor}%
\pgfsetfillcolor{textcolor}%
\pgftext[x=0.241129in, y=0.601518in, left, base]{\color{textcolor}{\rmfamily\fontsize{8.000000}{9.600000}\selectfont\catcode`\^=\active\def^{\ifmmode\sp\else\^{}\fi}\catcode`\%=\active\def%{\%}$\mathdefault{10^{0}}$}}%
\end{pgfscope}%
\begin{pgfscope}%
\pgfpathrectangle{\pgfqpoint{0.514278in}{0.417642in}}{\pgfqpoint{1.884996in}{1.371397in}}%
\pgfusepath{clip}%
\pgfsetrectcap%
\pgfsetroundjoin%
\pgfsetlinewidth{0.803000pt}%
\definecolor{currentstroke}{rgb}{0.450000,0.450000,0.450000}%
\pgfsetstrokecolor{currentstroke}%
\pgfsetdash{}{0pt}%
\pgfpathmoveto{\pgfqpoint{0.514278in}{0.983520in}}%
\pgfpathlineto{\pgfqpoint{2.399275in}{0.983520in}}%
\pgfusepath{stroke}%
\end{pgfscope}%
\begin{pgfscope}%
\pgfsetbuttcap%
\pgfsetroundjoin%
\definecolor{currentfill}{rgb}{0.000000,0.000000,0.000000}%
\pgfsetfillcolor{currentfill}%
\pgfsetlinewidth{0.803000pt}%
\definecolor{currentstroke}{rgb}{0.000000,0.000000,0.000000}%
\pgfsetstrokecolor{currentstroke}%
\pgfsetdash{}{0pt}%
\pgfsys@defobject{currentmarker}{\pgfqpoint{-0.048611in}{0.000000in}}{\pgfqpoint{-0.000000in}{0.000000in}}{%
\pgfpathmoveto{\pgfqpoint{-0.000000in}{0.000000in}}%
\pgfpathlineto{\pgfqpoint{-0.048611in}{0.000000in}}%
\pgfusepath{stroke,fill}%
}%
\begin{pgfscope}%
\pgfsys@transformshift{0.514278in}{0.983520in}%
\pgfsys@useobject{currentmarker}{}%
\end{pgfscope}%
\end{pgfscope}%
\begin{pgfscope}%
\definecolor{textcolor}{rgb}{0.000000,0.000000,0.000000}%
\pgfsetstrokecolor{textcolor}%
\pgfsetfillcolor{textcolor}%
\pgftext[x=0.241129in, y=0.944367in, left, base]{\color{textcolor}{\rmfamily\fontsize{8.000000}{9.600000}\selectfont\catcode`\^=\active\def^{\ifmmode\sp\else\^{}\fi}\catcode`\%=\active\def%{\%}$\mathdefault{10^{2}}$}}%
\end{pgfscope}%
\begin{pgfscope}%
\pgfpathrectangle{\pgfqpoint{0.514278in}{0.417642in}}{\pgfqpoint{1.884996in}{1.371397in}}%
\pgfusepath{clip}%
\pgfsetrectcap%
\pgfsetroundjoin%
\pgfsetlinewidth{0.803000pt}%
\definecolor{currentstroke}{rgb}{0.450000,0.450000,0.450000}%
\pgfsetstrokecolor{currentstroke}%
\pgfsetdash{}{0pt}%
\pgfpathmoveto{\pgfqpoint{0.514278in}{1.326369in}}%
\pgfpathlineto{\pgfqpoint{2.399275in}{1.326369in}}%
\pgfusepath{stroke}%
\end{pgfscope}%
\begin{pgfscope}%
\pgfsetbuttcap%
\pgfsetroundjoin%
\definecolor{currentfill}{rgb}{0.000000,0.000000,0.000000}%
\pgfsetfillcolor{currentfill}%
\pgfsetlinewidth{0.803000pt}%
\definecolor{currentstroke}{rgb}{0.000000,0.000000,0.000000}%
\pgfsetstrokecolor{currentstroke}%
\pgfsetdash{}{0pt}%
\pgfsys@defobject{currentmarker}{\pgfqpoint{-0.048611in}{0.000000in}}{\pgfqpoint{-0.000000in}{0.000000in}}{%
\pgfpathmoveto{\pgfqpoint{-0.000000in}{0.000000in}}%
\pgfpathlineto{\pgfqpoint{-0.048611in}{0.000000in}}%
\pgfusepath{stroke,fill}%
}%
\begin{pgfscope}%
\pgfsys@transformshift{0.514278in}{1.326369in}%
\pgfsys@useobject{currentmarker}{}%
\end{pgfscope}%
\end{pgfscope}%
\begin{pgfscope}%
\definecolor{textcolor}{rgb}{0.000000,0.000000,0.000000}%
\pgfsetstrokecolor{textcolor}%
\pgfsetfillcolor{textcolor}%
\pgftext[x=0.241129in, y=1.287216in, left, base]{\color{textcolor}{\rmfamily\fontsize{8.000000}{9.600000}\selectfont\catcode`\^=\active\def^{\ifmmode\sp\else\^{}\fi}\catcode`\%=\active\def%{\%}$\mathdefault{10^{4}}$}}%
\end{pgfscope}%
\begin{pgfscope}%
\pgfpathrectangle{\pgfqpoint{0.514278in}{0.417642in}}{\pgfqpoint{1.884996in}{1.371397in}}%
\pgfusepath{clip}%
\pgfsetrectcap%
\pgfsetroundjoin%
\pgfsetlinewidth{0.803000pt}%
\definecolor{currentstroke}{rgb}{0.450000,0.450000,0.450000}%
\pgfsetstrokecolor{currentstroke}%
\pgfsetdash{}{0pt}%
\pgfpathmoveto{\pgfqpoint{0.514278in}{1.669218in}}%
\pgfpathlineto{\pgfqpoint{2.399275in}{1.669218in}}%
\pgfusepath{stroke}%
\end{pgfscope}%
\begin{pgfscope}%
\pgfsetbuttcap%
\pgfsetroundjoin%
\definecolor{currentfill}{rgb}{0.000000,0.000000,0.000000}%
\pgfsetfillcolor{currentfill}%
\pgfsetlinewidth{0.803000pt}%
\definecolor{currentstroke}{rgb}{0.000000,0.000000,0.000000}%
\pgfsetstrokecolor{currentstroke}%
\pgfsetdash{}{0pt}%
\pgfsys@defobject{currentmarker}{\pgfqpoint{-0.048611in}{0.000000in}}{\pgfqpoint{-0.000000in}{0.000000in}}{%
\pgfpathmoveto{\pgfqpoint{-0.000000in}{0.000000in}}%
\pgfpathlineto{\pgfqpoint{-0.048611in}{0.000000in}}%
\pgfusepath{stroke,fill}%
}%
\begin{pgfscope}%
\pgfsys@transformshift{0.514278in}{1.669218in}%
\pgfsys@useobject{currentmarker}{}%
\end{pgfscope}%
\end{pgfscope}%
\begin{pgfscope}%
\definecolor{textcolor}{rgb}{0.000000,0.000000,0.000000}%
\pgfsetstrokecolor{textcolor}%
\pgfsetfillcolor{textcolor}%
\pgftext[x=0.241129in, y=1.630065in, left, base]{\color{textcolor}{\rmfamily\fontsize{8.000000}{9.600000}\selectfont\catcode`\^=\active\def^{\ifmmode\sp\else\^{}\fi}\catcode`\%=\active\def%{\%}$\mathdefault{10^{6}}$}}%
\end{pgfscope}%
\begin{pgfscope}%
\definecolor{textcolor}{rgb}{0.000000,0.000000,0.000000}%
\pgfsetstrokecolor{textcolor}%
\pgfsetfillcolor{textcolor}%
\pgftext[x=0.185574in,y=1.103340in,,bottom,rotate=90.000000]{\color{textcolor}{\rmfamily\fontsize{10.000000}{12.000000}\selectfont\catcode`\^=\active\def^{\ifmmode\sp\else\^{}\fi}\catcode`\%=\active\def%{\%}$S_y(f)$ in $\unit{1 \per \Hz}$}}%
\end{pgfscope}%
\begin{pgfscope}%
\pgfpathrectangle{\pgfqpoint{0.514278in}{0.417642in}}{\pgfqpoint{1.884996in}{1.371397in}}%
\pgfusepath{clip}%
\pgfsetbuttcap%
\pgfsetroundjoin%
\pgfsetlinewidth{1.505625pt}%
\definecolor{currentstroke}{rgb}{0.003922,0.450980,0.698039}%
\pgfsetstrokecolor{currentstroke}%
\pgfsetdash{{5.550000pt}{2.400000pt}}{0.000000pt}%
\pgfpathmoveto{\pgfqpoint{0.599960in}{0.692274in}}%
\pgfpathlineto{\pgfqpoint{0.755760in}{0.692274in}}%
\pgfpathlineto{\pgfqpoint{0.846897in}{0.692274in}}%
\pgfpathlineto{\pgfqpoint{0.911560in}{0.692274in}}%
\pgfpathlineto{\pgfqpoint{0.961716in}{0.692274in}}%
\pgfpathlineto{\pgfqpoint{1.002697in}{0.692274in}}%
\pgfpathlineto{\pgfqpoint{1.037345in}{0.692274in}}%
\pgfpathlineto{\pgfqpoint{1.067360in}{0.692274in}}%
\pgfpathlineto{\pgfqpoint{1.093834in}{0.692274in}}%
\pgfpathlineto{\pgfqpoint{1.117516in}{0.692274in}}%
\pgfpathlineto{\pgfqpoint{1.138939in}{0.692274in}}%
\pgfpathlineto{\pgfqpoint{1.158497in}{0.692274in}}%
\pgfpathlineto{\pgfqpoint{1.176488in}{0.692274in}}%
\pgfpathlineto{\pgfqpoint{1.193145in}{0.692274in}}%
\pgfpathlineto{\pgfqpoint{1.208653in}{0.692274in}}%
\pgfpathlineto{\pgfqpoint{1.223159in}{0.692274in}}%
\pgfpathlineto{\pgfqpoint{1.236786in}{0.692274in}}%
\pgfpathlineto{\pgfqpoint{1.249634in}{0.692274in}}%
\pgfpathlineto{\pgfqpoint{1.261786in}{0.692274in}}%
\pgfpathlineto{\pgfqpoint{1.273316in}{0.692274in}}%
\pgfpathlineto{\pgfqpoint{1.284282in}{0.692274in}}%
\pgfpathlineto{\pgfqpoint{1.294739in}{0.692274in}}%
\pgfpathlineto{\pgfqpoint{1.309564in}{0.692274in}}%
\pgfpathlineto{\pgfqpoint{1.323472in}{0.692274in}}%
\pgfpathlineto{\pgfqpoint{1.332288in}{0.692274in}}%
\pgfpathlineto{\pgfqpoint{1.340771in}{0.692274in}}%
\pgfpathlineto{\pgfqpoint{1.348945in}{0.692274in}}%
\pgfpathlineto{\pgfqpoint{1.360675in}{0.692274in}}%
\pgfpathlineto{\pgfqpoint{1.371823in}{0.692274in}}%
\pgfpathlineto{\pgfqpoint{1.382444in}{0.692274in}}%
\pgfpathlineto{\pgfqpoint{1.392586in}{0.692274in}}%
\pgfpathlineto{\pgfqpoint{1.402290in}{0.692274in}}%
\pgfpathlineto{\pgfqpoint{1.414609in}{0.692274in}}%
\pgfpathlineto{\pgfqpoint{1.426288in}{0.692274in}}%
\pgfpathlineto{\pgfqpoint{1.434666in}{0.692274in}}%
\pgfpathlineto{\pgfqpoint{1.442742in}{0.692274in}}%
\pgfpathlineto{\pgfqpoint{1.453078in}{0.692274in}}%
\pgfpathlineto{\pgfqpoint{1.465364in}{0.692274in}}%
\pgfpathlineto{\pgfqpoint{1.477013in}{0.692274in}}%
\pgfpathlineto{\pgfqpoint{1.485916in}{0.692274in}}%
\pgfpathlineto{\pgfqpoint{1.496570in}{0.692274in}}%
\pgfpathlineto{\pgfqpoint{1.506743in}{0.692274in}}%
\pgfpathlineto{\pgfqpoint{1.516475in}{0.692274in}}%
\pgfpathlineto{\pgfqpoint{1.527623in}{0.692274in}}%
\pgfpathlineto{\pgfqpoint{1.538244in}{0.692274in}}%
\pgfpathlineto{\pgfqpoint{1.548386in}{0.692274in}}%
\pgfpathlineto{\pgfqpoint{1.559667in}{0.692274in}}%
\pgfpathlineto{\pgfqpoint{1.570409in}{0.692274in}}%
\pgfpathlineto{\pgfqpoint{1.580661in}{0.692274in}}%
\pgfpathlineto{\pgfqpoint{1.590465in}{0.692274in}}%
\pgfpathlineto{\pgfqpoint{1.599860in}{0.692274in}}%
\pgfpathlineto{\pgfqpoint{1.611390in}{0.692274in}}%
\pgfpathlineto{\pgfqpoint{1.622356in}{0.692274in}}%
\pgfpathlineto{\pgfqpoint{1.631675in}{0.692274in}}%
\pgfpathlineto{\pgfqpoint{1.641716in}{0.692274in}}%
\pgfpathlineto{\pgfqpoint{1.652370in}{0.692274in}}%
\pgfpathlineto{\pgfqpoint{1.663535in}{0.692274in}}%
\pgfpathlineto{\pgfqpoint{1.674171in}{0.692274in}}%
\pgfpathlineto{\pgfqpoint{1.684327in}{0.692274in}}%
\pgfpathlineto{\pgfqpoint{1.694907in}{0.692274in}}%
\pgfpathlineto{\pgfqpoint{1.705010in}{0.692274in}}%
\pgfpathlineto{\pgfqpoint{1.715467in}{0.692274in}}%
\pgfpathlineto{\pgfqpoint{1.726209in}{0.692274in}}%
\pgfpathlineto{\pgfqpoint{1.736461in}{0.692274in}}%
\pgfpathlineto{\pgfqpoint{1.746950in}{0.692274in}}%
\pgfpathlineto{\pgfqpoint{1.756971in}{0.692274in}}%
\pgfpathlineto{\pgfqpoint{1.767189in}{0.692274in}}%
\pgfpathlineto{\pgfqpoint{1.778156in}{0.692274in}}%
\pgfpathlineto{\pgfqpoint{1.788612in}{0.692274in}}%
\pgfpathlineto{\pgfqpoint{1.798604in}{0.692274in}}%
\pgfpathlineto{\pgfqpoint{1.808690in}{0.692274in}}%
\pgfpathlineto{\pgfqpoint{1.819335in}{0.692274in}}%
\pgfpathlineto{\pgfqpoint{1.829971in}{0.692274in}}%
\pgfpathlineto{\pgfqpoint{1.840127in}{0.692274in}}%
\pgfpathlineto{\pgfqpoint{1.850275in}{0.692274in}}%
\pgfpathlineto{\pgfqpoint{1.860810in}{0.692274in}}%
\pgfpathlineto{\pgfqpoint{1.871266in}{0.692274in}}%
\pgfpathlineto{\pgfqpoint{1.881634in}{0.692274in}}%
\pgfpathlineto{\pgfqpoint{1.892260in}{0.692274in}}%
\pgfpathlineto{\pgfqpoint{1.902749in}{0.692274in}}%
\pgfpathlineto{\pgfqpoint{1.913097in}{0.692274in}}%
\pgfpathlineto{\pgfqpoint{1.923613in}{0.692274in}}%
\pgfpathlineto{\pgfqpoint{1.933956in}{0.692274in}}%
\pgfpathlineto{\pgfqpoint{1.944128in}{0.692274in}}%
\pgfpathlineto{\pgfqpoint{1.954404in}{0.692274in}}%
\pgfpathlineto{\pgfqpoint{1.965008in}{0.692274in}}%
\pgfpathlineto{\pgfqpoint{1.975629in}{0.692274in}}%
\pgfpathlineto{\pgfqpoint{1.986007in}{0.692274in}}%
\pgfpathlineto{\pgfqpoint{1.996378in}{0.692274in}}%
\pgfpathlineto{\pgfqpoint{2.006721in}{0.692274in}}%
\pgfpathlineto{\pgfqpoint{2.017021in}{0.692274in}}%
\pgfpathlineto{\pgfqpoint{2.027459in}{0.692274in}}%
\pgfpathlineto{\pgfqpoint{2.037808in}{0.692274in}}%
\pgfpathlineto{\pgfqpoint{2.048239in}{0.692274in}}%
\pgfpathlineto{\pgfqpoint{2.058720in}{0.692274in}}%
\pgfpathlineto{\pgfqpoint{2.069060in}{0.692274in}}%
\pgfpathlineto{\pgfqpoint{2.079412in}{0.692274in}}%
\pgfpathlineto{\pgfqpoint{2.089756in}{0.692274in}}%
\pgfpathlineto{\pgfqpoint{2.100212in}{0.692274in}}%
\pgfpathlineto{\pgfqpoint{2.110610in}{0.692274in}}%
\pgfpathlineto{\pgfqpoint{2.121067in}{0.692274in}}%
\pgfpathlineto{\pgfqpoint{2.131552in}{0.692274in}}%
\pgfpathlineto{\pgfqpoint{2.141925in}{0.692274in}}%
\pgfpathlineto{\pgfqpoint{2.152290in}{0.692274in}}%
\pgfpathlineto{\pgfqpoint{2.162629in}{0.692274in}}%
\pgfpathlineto{\pgfqpoint{2.173026in}{0.692274in}}%
\pgfpathlineto{\pgfqpoint{2.183455in}{0.692274in}}%
\pgfpathlineto{\pgfqpoint{2.193889in}{0.692274in}}%
\pgfpathlineto{\pgfqpoint{2.204307in}{0.692274in}}%
\pgfpathlineto{\pgfqpoint{2.214690in}{0.692274in}}%
\pgfpathlineto{\pgfqpoint{2.225104in}{0.692274in}}%
\pgfpathlineto{\pgfqpoint{2.235523in}{0.692274in}}%
\pgfpathlineto{\pgfqpoint{2.245927in}{0.692274in}}%
\pgfpathlineto{\pgfqpoint{2.256295in}{0.692274in}}%
\pgfpathlineto{\pgfqpoint{2.266681in}{0.692274in}}%
\pgfpathlineto{\pgfqpoint{2.277125in}{0.692274in}}%
\pgfpathlineto{\pgfqpoint{2.287537in}{0.692274in}}%
\pgfpathlineto{\pgfqpoint{2.297901in}{0.692274in}}%
\pgfpathlineto{\pgfqpoint{2.308315in}{0.692274in}}%
\pgfpathlineto{\pgfqpoint{2.313593in}{0.692274in}}%
\pgfusepath{stroke}%
\end{pgfscope}%
\begin{pgfscope}%
\pgfpathrectangle{\pgfqpoint{0.514278in}{0.417642in}}{\pgfqpoint{1.884996in}{1.371397in}}%
\pgfusepath{clip}%
\pgfsetbuttcap%
\pgfsetroundjoin%
\definecolor{currentfill}{rgb}{0.003922,0.450980,0.698039}%
\pgfsetfillcolor{currentfill}%
\pgfsetlinewidth{1.003750pt}%
\definecolor{currentstroke}{rgb}{0.003922,0.450980,0.698039}%
\pgfsetstrokecolor{currentstroke}%
\pgfsetdash{}{0pt}%
\pgfsys@defobject{currentmarker}{\pgfqpoint{-0.006944in}{-0.006944in}}{\pgfqpoint{0.006944in}{0.006944in}}{%
\pgfpathmoveto{\pgfqpoint{0.000000in}{-0.006944in}}%
\pgfpathcurveto{\pgfqpoint{0.001842in}{-0.006944in}}{\pgfqpoint{0.003608in}{-0.006213in}}{\pgfqpoint{0.004910in}{-0.004910in}}%
\pgfpathcurveto{\pgfqpoint{0.006213in}{-0.003608in}}{\pgfqpoint{0.006944in}{-0.001842in}}{\pgfqpoint{0.006944in}{0.000000in}}%
\pgfpathcurveto{\pgfqpoint{0.006944in}{0.001842in}}{\pgfqpoint{0.006213in}{0.003608in}}{\pgfqpoint{0.004910in}{0.004910in}}%
\pgfpathcurveto{\pgfqpoint{0.003608in}{0.006213in}}{\pgfqpoint{0.001842in}{0.006944in}}{\pgfqpoint{0.000000in}{0.006944in}}%
\pgfpathcurveto{\pgfqpoint{-0.001842in}{0.006944in}}{\pgfqpoint{-0.003608in}{0.006213in}}{\pgfqpoint{-0.004910in}{0.004910in}}%
\pgfpathcurveto{\pgfqpoint{-0.006213in}{0.003608in}}{\pgfqpoint{-0.006944in}{0.001842in}}{\pgfqpoint{-0.006944in}{0.000000in}}%
\pgfpathcurveto{\pgfqpoint{-0.006944in}{-0.001842in}}{\pgfqpoint{-0.006213in}{-0.003608in}}{\pgfqpoint{-0.004910in}{-0.004910in}}%
\pgfpathcurveto{\pgfqpoint{-0.003608in}{-0.006213in}}{\pgfqpoint{-0.001842in}{-0.006944in}}{\pgfqpoint{0.000000in}{-0.006944in}}%
\pgfpathlineto{\pgfqpoint{0.000000in}{-0.006944in}}%
\pgfpathclose%
\pgfusepath{stroke,fill}%
}%
\begin{pgfscope}%
\pgfsys@transformshift{0.599960in}{0.663805in}%
\pgfsys@useobject{currentmarker}{}%
\end{pgfscope}%
\begin{pgfscope}%
\pgfsys@transformshift{0.755760in}{0.618622in}%
\pgfsys@useobject{currentmarker}{}%
\end{pgfscope}%
\begin{pgfscope}%
\pgfsys@transformshift{0.846897in}{0.651992in}%
\pgfsys@useobject{currentmarker}{}%
\end{pgfscope}%
\begin{pgfscope}%
\pgfsys@transformshift{0.911560in}{0.646224in}%
\pgfsys@useobject{currentmarker}{}%
\end{pgfscope}%
\begin{pgfscope}%
\pgfsys@transformshift{0.961716in}{0.650738in}%
\pgfsys@useobject{currentmarker}{}%
\end{pgfscope}%
\begin{pgfscope}%
\pgfsys@transformshift{1.002697in}{0.619601in}%
\pgfsys@useobject{currentmarker}{}%
\end{pgfscope}%
\begin{pgfscope}%
\pgfsys@transformshift{1.037345in}{0.703596in}%
\pgfsys@useobject{currentmarker}{}%
\end{pgfscope}%
\begin{pgfscope}%
\pgfsys@transformshift{1.067360in}{0.721542in}%
\pgfsys@useobject{currentmarker}{}%
\end{pgfscope}%
\begin{pgfscope}%
\pgfsys@transformshift{1.093834in}{0.711710in}%
\pgfsys@useobject{currentmarker}{}%
\end{pgfscope}%
\begin{pgfscope}%
\pgfsys@transformshift{1.117516in}{0.701460in}%
\pgfsys@useobject{currentmarker}{}%
\end{pgfscope}%
\begin{pgfscope}%
\pgfsys@transformshift{1.138939in}{0.674348in}%
\pgfsys@useobject{currentmarker}{}%
\end{pgfscope}%
\begin{pgfscope}%
\pgfsys@transformshift{1.158497in}{0.714512in}%
\pgfsys@useobject{currentmarker}{}%
\end{pgfscope}%
\begin{pgfscope}%
\pgfsys@transformshift{1.176488in}{0.691000in}%
\pgfsys@useobject{currentmarker}{}%
\end{pgfscope}%
\begin{pgfscope}%
\pgfsys@transformshift{1.193145in}{0.630301in}%
\pgfsys@useobject{currentmarker}{}%
\end{pgfscope}%
\begin{pgfscope}%
\pgfsys@transformshift{1.208653in}{0.701372in}%
\pgfsys@useobject{currentmarker}{}%
\end{pgfscope}%
\begin{pgfscope}%
\pgfsys@transformshift{1.223159in}{0.729470in}%
\pgfsys@useobject{currentmarker}{}%
\end{pgfscope}%
\begin{pgfscope}%
\pgfsys@transformshift{1.236786in}{0.708209in}%
\pgfsys@useobject{currentmarker}{}%
\end{pgfscope}%
\begin{pgfscope}%
\pgfsys@transformshift{1.249634in}{0.605316in}%
\pgfsys@useobject{currentmarker}{}%
\end{pgfscope}%
\begin{pgfscope}%
\pgfsys@transformshift{1.261786in}{0.661660in}%
\pgfsys@useobject{currentmarker}{}%
\end{pgfscope}%
\begin{pgfscope}%
\pgfsys@transformshift{1.273316in}{0.731416in}%
\pgfsys@useobject{currentmarker}{}%
\end{pgfscope}%
\begin{pgfscope}%
\pgfsys@transformshift{1.284282in}{0.749145in}%
\pgfsys@useobject{currentmarker}{}%
\end{pgfscope}%
\begin{pgfscope}%
\pgfsys@transformshift{1.294739in}{0.714727in}%
\pgfsys@useobject{currentmarker}{}%
\end{pgfscope}%
\begin{pgfscope}%
\pgfsys@transformshift{1.309564in}{0.676465in}%
\pgfsys@useobject{currentmarker}{}%
\end{pgfscope}%
\begin{pgfscope}%
\pgfsys@transformshift{1.323472in}{0.677410in}%
\pgfsys@useobject{currentmarker}{}%
\end{pgfscope}%
\begin{pgfscope}%
\pgfsys@transformshift{1.332288in}{0.659617in}%
\pgfsys@useobject{currentmarker}{}%
\end{pgfscope}%
\begin{pgfscope}%
\pgfsys@transformshift{1.340771in}{0.657152in}%
\pgfsys@useobject{currentmarker}{}%
\end{pgfscope}%
\begin{pgfscope}%
\pgfsys@transformshift{1.348945in}{0.706322in}%
\pgfsys@useobject{currentmarker}{}%
\end{pgfscope}%
\begin{pgfscope}%
\pgfsys@transformshift{1.360675in}{0.696295in}%
\pgfsys@useobject{currentmarker}{}%
\end{pgfscope}%
\begin{pgfscope}%
\pgfsys@transformshift{1.371823in}{0.699681in}%
\pgfsys@useobject{currentmarker}{}%
\end{pgfscope}%
\begin{pgfscope}%
\pgfsys@transformshift{1.382444in}{0.679848in}%
\pgfsys@useobject{currentmarker}{}%
\end{pgfscope}%
\begin{pgfscope}%
\pgfsys@transformshift{1.392586in}{0.711199in}%
\pgfsys@useobject{currentmarker}{}%
\end{pgfscope}%
\begin{pgfscope}%
\pgfsys@transformshift{1.402290in}{0.694159in}%
\pgfsys@useobject{currentmarker}{}%
\end{pgfscope}%
\begin{pgfscope}%
\pgfsys@transformshift{1.414609in}{0.686601in}%
\pgfsys@useobject{currentmarker}{}%
\end{pgfscope}%
\begin{pgfscope}%
\pgfsys@transformshift{1.426288in}{0.659788in}%
\pgfsys@useobject{currentmarker}{}%
\end{pgfscope}%
\begin{pgfscope}%
\pgfsys@transformshift{1.434666in}{0.708572in}%
\pgfsys@useobject{currentmarker}{}%
\end{pgfscope}%
\begin{pgfscope}%
\pgfsys@transformshift{1.442742in}{0.732958in}%
\pgfsys@useobject{currentmarker}{}%
\end{pgfscope}%
\begin{pgfscope}%
\pgfsys@transformshift{1.453078in}{0.663074in}%
\pgfsys@useobject{currentmarker}{}%
\end{pgfscope}%
\begin{pgfscope}%
\pgfsys@transformshift{1.465364in}{0.674540in}%
\pgfsys@useobject{currentmarker}{}%
\end{pgfscope}%
\begin{pgfscope}%
\pgfsys@transformshift{1.477013in}{0.667392in}%
\pgfsys@useobject{currentmarker}{}%
\end{pgfscope}%
\begin{pgfscope}%
\pgfsys@transformshift{1.485916in}{0.608273in}%
\pgfsys@useobject{currentmarker}{}%
\end{pgfscope}%
\begin{pgfscope}%
\pgfsys@transformshift{1.496570in}{0.700949in}%
\pgfsys@useobject{currentmarker}{}%
\end{pgfscope}%
\begin{pgfscope}%
\pgfsys@transformshift{1.506743in}{0.723191in}%
\pgfsys@useobject{currentmarker}{}%
\end{pgfscope}%
\begin{pgfscope}%
\pgfsys@transformshift{1.516475in}{0.704537in}%
\pgfsys@useobject{currentmarker}{}%
\end{pgfscope}%
\begin{pgfscope}%
\pgfsys@transformshift{1.527623in}{0.648969in}%
\pgfsys@useobject{currentmarker}{}%
\end{pgfscope}%
\begin{pgfscope}%
\pgfsys@transformshift{1.538244in}{0.653022in}%
\pgfsys@useobject{currentmarker}{}%
\end{pgfscope}%
\begin{pgfscope}%
\pgfsys@transformshift{1.548386in}{0.696554in}%
\pgfsys@useobject{currentmarker}{}%
\end{pgfscope}%
\begin{pgfscope}%
\pgfsys@transformshift{1.559667in}{0.685699in}%
\pgfsys@useobject{currentmarker}{}%
\end{pgfscope}%
\begin{pgfscope}%
\pgfsys@transformshift{1.570409in}{0.687434in}%
\pgfsys@useobject{currentmarker}{}%
\end{pgfscope}%
\begin{pgfscope}%
\pgfsys@transformshift{1.580661in}{0.667780in}%
\pgfsys@useobject{currentmarker}{}%
\end{pgfscope}%
\begin{pgfscope}%
\pgfsys@transformshift{1.590465in}{0.672550in}%
\pgfsys@useobject{currentmarker}{}%
\end{pgfscope}%
\begin{pgfscope}%
\pgfsys@transformshift{1.599860in}{0.648800in}%
\pgfsys@useobject{currentmarker}{}%
\end{pgfscope}%
\begin{pgfscope}%
\pgfsys@transformshift{1.611390in}{0.661268in}%
\pgfsys@useobject{currentmarker}{}%
\end{pgfscope}%
\begin{pgfscope}%
\pgfsys@transformshift{1.622356in}{0.679397in}%
\pgfsys@useobject{currentmarker}{}%
\end{pgfscope}%
\begin{pgfscope}%
\pgfsys@transformshift{1.631675in}{0.661258in}%
\pgfsys@useobject{currentmarker}{}%
\end{pgfscope}%
\begin{pgfscope}%
\pgfsys@transformshift{1.641716in}{0.674669in}%
\pgfsys@useobject{currentmarker}{}%
\end{pgfscope}%
\begin{pgfscope}%
\pgfsys@transformshift{1.652370in}{0.698131in}%
\pgfsys@useobject{currentmarker}{}%
\end{pgfscope}%
\begin{pgfscope}%
\pgfsys@transformshift{1.663535in}{0.673649in}%
\pgfsys@useobject{currentmarker}{}%
\end{pgfscope}%
\begin{pgfscope}%
\pgfsys@transformshift{1.674171in}{0.680894in}%
\pgfsys@useobject{currentmarker}{}%
\end{pgfscope}%
\begin{pgfscope}%
\pgfsys@transformshift{1.684327in}{0.676831in}%
\pgfsys@useobject{currentmarker}{}%
\end{pgfscope}%
\begin{pgfscope}%
\pgfsys@transformshift{1.694907in}{0.685461in}%
\pgfsys@useobject{currentmarker}{}%
\end{pgfscope}%
\begin{pgfscope}%
\pgfsys@transformshift{1.705010in}{0.716158in}%
\pgfsys@useobject{currentmarker}{}%
\end{pgfscope}%
\begin{pgfscope}%
\pgfsys@transformshift{1.715467in}{0.691709in}%
\pgfsys@useobject{currentmarker}{}%
\end{pgfscope}%
\begin{pgfscope}%
\pgfsys@transformshift{1.726209in}{0.696848in}%
\pgfsys@useobject{currentmarker}{}%
\end{pgfscope}%
\begin{pgfscope}%
\pgfsys@transformshift{1.736461in}{0.702946in}%
\pgfsys@useobject{currentmarker}{}%
\end{pgfscope}%
\begin{pgfscope}%
\pgfsys@transformshift{1.746950in}{0.683494in}%
\pgfsys@useobject{currentmarker}{}%
\end{pgfscope}%
\begin{pgfscope}%
\pgfsys@transformshift{1.756971in}{0.627751in}%
\pgfsys@useobject{currentmarker}{}%
\end{pgfscope}%
\begin{pgfscope}%
\pgfsys@transformshift{1.767189in}{0.696131in}%
\pgfsys@useobject{currentmarker}{}%
\end{pgfscope}%
\begin{pgfscope}%
\pgfsys@transformshift{1.778156in}{0.708459in}%
\pgfsys@useobject{currentmarker}{}%
\end{pgfscope}%
\begin{pgfscope}%
\pgfsys@transformshift{1.788612in}{0.720422in}%
\pgfsys@useobject{currentmarker}{}%
\end{pgfscope}%
\begin{pgfscope}%
\pgfsys@transformshift{1.798604in}{0.684406in}%
\pgfsys@useobject{currentmarker}{}%
\end{pgfscope}%
\begin{pgfscope}%
\pgfsys@transformshift{1.808690in}{0.683552in}%
\pgfsys@useobject{currentmarker}{}%
\end{pgfscope}%
\begin{pgfscope}%
\pgfsys@transformshift{1.819335in}{0.667244in}%
\pgfsys@useobject{currentmarker}{}%
\end{pgfscope}%
\begin{pgfscope}%
\pgfsys@transformshift{1.829971in}{0.694637in}%
\pgfsys@useobject{currentmarker}{}%
\end{pgfscope}%
\begin{pgfscope}%
\pgfsys@transformshift{1.840127in}{0.687990in}%
\pgfsys@useobject{currentmarker}{}%
\end{pgfscope}%
\begin{pgfscope}%
\pgfsys@transformshift{1.850275in}{0.705543in}%
\pgfsys@useobject{currentmarker}{}%
\end{pgfscope}%
\begin{pgfscope}%
\pgfsys@transformshift{1.860810in}{0.669379in}%
\pgfsys@useobject{currentmarker}{}%
\end{pgfscope}%
\begin{pgfscope}%
\pgfsys@transformshift{1.871266in}{0.681188in}%
\pgfsys@useobject{currentmarker}{}%
\end{pgfscope}%
\begin{pgfscope}%
\pgfsys@transformshift{1.881634in}{0.692273in}%
\pgfsys@useobject{currentmarker}{}%
\end{pgfscope}%
\begin{pgfscope}%
\pgfsys@transformshift{1.892260in}{0.696708in}%
\pgfsys@useobject{currentmarker}{}%
\end{pgfscope}%
\begin{pgfscope}%
\pgfsys@transformshift{1.902749in}{0.717949in}%
\pgfsys@useobject{currentmarker}{}%
\end{pgfscope}%
\begin{pgfscope}%
\pgfsys@transformshift{1.913097in}{0.683058in}%
\pgfsys@useobject{currentmarker}{}%
\end{pgfscope}%
\begin{pgfscope}%
\pgfsys@transformshift{1.923613in}{0.676800in}%
\pgfsys@useobject{currentmarker}{}%
\end{pgfscope}%
\begin{pgfscope}%
\pgfsys@transformshift{1.933956in}{0.674384in}%
\pgfsys@useobject{currentmarker}{}%
\end{pgfscope}%
\begin{pgfscope}%
\pgfsys@transformshift{1.944128in}{0.692094in}%
\pgfsys@useobject{currentmarker}{}%
\end{pgfscope}%
\begin{pgfscope}%
\pgfsys@transformshift{1.954404in}{0.704235in}%
\pgfsys@useobject{currentmarker}{}%
\end{pgfscope}%
\begin{pgfscope}%
\pgfsys@transformshift{1.965008in}{0.691487in}%
\pgfsys@useobject{currentmarker}{}%
\end{pgfscope}%
\begin{pgfscope}%
\pgfsys@transformshift{1.975629in}{0.685705in}%
\pgfsys@useobject{currentmarker}{}%
\end{pgfscope}%
\begin{pgfscope}%
\pgfsys@transformshift{1.986007in}{0.692158in}%
\pgfsys@useobject{currentmarker}{}%
\end{pgfscope}%
\begin{pgfscope}%
\pgfsys@transformshift{1.996378in}{0.695510in}%
\pgfsys@useobject{currentmarker}{}%
\end{pgfscope}%
\begin{pgfscope}%
\pgfsys@transformshift{2.006721in}{0.688511in}%
\pgfsys@useobject{currentmarker}{}%
\end{pgfscope}%
\begin{pgfscope}%
\pgfsys@transformshift{2.017021in}{0.695353in}%
\pgfsys@useobject{currentmarker}{}%
\end{pgfscope}%
\begin{pgfscope}%
\pgfsys@transformshift{2.027459in}{0.695166in}%
\pgfsys@useobject{currentmarker}{}%
\end{pgfscope}%
\begin{pgfscope}%
\pgfsys@transformshift{2.037808in}{0.696314in}%
\pgfsys@useobject{currentmarker}{}%
\end{pgfscope}%
\begin{pgfscope}%
\pgfsys@transformshift{2.048239in}{0.700824in}%
\pgfsys@useobject{currentmarker}{}%
\end{pgfscope}%
\begin{pgfscope}%
\pgfsys@transformshift{2.058720in}{0.694145in}%
\pgfsys@useobject{currentmarker}{}%
\end{pgfscope}%
\begin{pgfscope}%
\pgfsys@transformshift{2.069060in}{0.687403in}%
\pgfsys@useobject{currentmarker}{}%
\end{pgfscope}%
\begin{pgfscope}%
\pgfsys@transformshift{2.079412in}{0.696275in}%
\pgfsys@useobject{currentmarker}{}%
\end{pgfscope}%
\begin{pgfscope}%
\pgfsys@transformshift{2.089756in}{0.689918in}%
\pgfsys@useobject{currentmarker}{}%
\end{pgfscope}%
\begin{pgfscope}%
\pgfsys@transformshift{2.100212in}{0.707787in}%
\pgfsys@useobject{currentmarker}{}%
\end{pgfscope}%
\begin{pgfscope}%
\pgfsys@transformshift{2.110610in}{0.698158in}%
\pgfsys@useobject{currentmarker}{}%
\end{pgfscope}%
\begin{pgfscope}%
\pgfsys@transformshift{2.121067in}{0.695132in}%
\pgfsys@useobject{currentmarker}{}%
\end{pgfscope}%
\begin{pgfscope}%
\pgfsys@transformshift{2.131552in}{0.692670in}%
\pgfsys@useobject{currentmarker}{}%
\end{pgfscope}%
\begin{pgfscope}%
\pgfsys@transformshift{2.141925in}{0.694971in}%
\pgfsys@useobject{currentmarker}{}%
\end{pgfscope}%
\begin{pgfscope}%
\pgfsys@transformshift{2.152290in}{0.691283in}%
\pgfsys@useobject{currentmarker}{}%
\end{pgfscope}%
\begin{pgfscope}%
\pgfsys@transformshift{2.162629in}{0.691112in}%
\pgfsys@useobject{currentmarker}{}%
\end{pgfscope}%
\begin{pgfscope}%
\pgfsys@transformshift{2.173026in}{0.693066in}%
\pgfsys@useobject{currentmarker}{}%
\end{pgfscope}%
\begin{pgfscope}%
\pgfsys@transformshift{2.183455in}{0.697970in}%
\pgfsys@useobject{currentmarker}{}%
\end{pgfscope}%
\begin{pgfscope}%
\pgfsys@transformshift{2.193889in}{0.696976in}%
\pgfsys@useobject{currentmarker}{}%
\end{pgfscope}%
\begin{pgfscope}%
\pgfsys@transformshift{2.204307in}{0.684822in}%
\pgfsys@useobject{currentmarker}{}%
\end{pgfscope}%
\begin{pgfscope}%
\pgfsys@transformshift{2.214690in}{0.682311in}%
\pgfsys@useobject{currentmarker}{}%
\end{pgfscope}%
\begin{pgfscope}%
\pgfsys@transformshift{2.225104in}{0.696410in}%
\pgfsys@useobject{currentmarker}{}%
\end{pgfscope}%
\begin{pgfscope}%
\pgfsys@transformshift{2.235523in}{0.699589in}%
\pgfsys@useobject{currentmarker}{}%
\end{pgfscope}%
\begin{pgfscope}%
\pgfsys@transformshift{2.245927in}{0.686990in}%
\pgfsys@useobject{currentmarker}{}%
\end{pgfscope}%
\begin{pgfscope}%
\pgfsys@transformshift{2.256295in}{0.691147in}%
\pgfsys@useobject{currentmarker}{}%
\end{pgfscope}%
\begin{pgfscope}%
\pgfsys@transformshift{2.266681in}{0.697284in}%
\pgfsys@useobject{currentmarker}{}%
\end{pgfscope}%
\begin{pgfscope}%
\pgfsys@transformshift{2.277125in}{0.694587in}%
\pgfsys@useobject{currentmarker}{}%
\end{pgfscope}%
\begin{pgfscope}%
\pgfsys@transformshift{2.287537in}{0.701604in}%
\pgfsys@useobject{currentmarker}{}%
\end{pgfscope}%
\begin{pgfscope}%
\pgfsys@transformshift{2.297901in}{0.687224in}%
\pgfsys@useobject{currentmarker}{}%
\end{pgfscope}%
\begin{pgfscope}%
\pgfsys@transformshift{2.308315in}{0.688592in}%
\pgfsys@useobject{currentmarker}{}%
\end{pgfscope}%
\begin{pgfscope}%
\pgfsys@transformshift{2.313593in}{0.674838in}%
\pgfsys@useobject{currentmarker}{}%
\end{pgfscope}%
\end{pgfscope}%
\begin{pgfscope}%
\pgfsetrectcap%
\pgfsetmiterjoin%
\pgfsetlinewidth{0.803000pt}%
\definecolor{currentstroke}{rgb}{0.000000,0.000000,0.000000}%
\pgfsetstrokecolor{currentstroke}%
\pgfsetdash{}{0pt}%
\pgfpathmoveto{\pgfqpoint{0.514278in}{0.417642in}}%
\pgfpathlineto{\pgfqpoint{0.514278in}{1.789039in}}%
\pgfusepath{stroke}%
\end{pgfscope}%
\begin{pgfscope}%
\pgfsetrectcap%
\pgfsetmiterjoin%
\pgfsetlinewidth{0.803000pt}%
\definecolor{currentstroke}{rgb}{0.000000,0.000000,0.000000}%
\pgfsetstrokecolor{currentstroke}%
\pgfsetdash{}{0pt}%
\pgfpathmoveto{\pgfqpoint{2.399275in}{0.417642in}}%
\pgfpathlineto{\pgfqpoint{2.399275in}{1.789039in}}%
\pgfusepath{stroke}%
\end{pgfscope}%
\begin{pgfscope}%
\pgfsetrectcap%
\pgfsetmiterjoin%
\pgfsetlinewidth{0.803000pt}%
\definecolor{currentstroke}{rgb}{0.000000,0.000000,0.000000}%
\pgfsetstrokecolor{currentstroke}%
\pgfsetdash{}{0pt}%
\pgfpathmoveto{\pgfqpoint{0.514278in}{0.417642in}}%
\pgfpathlineto{\pgfqpoint{2.399275in}{0.417642in}}%
\pgfusepath{stroke}%
\end{pgfscope}%
\begin{pgfscope}%
\pgfsetrectcap%
\pgfsetmiterjoin%
\pgfsetlinewidth{0.803000pt}%
\definecolor{currentstroke}{rgb}{0.000000,0.000000,0.000000}%
\pgfsetstrokecolor{currentstroke}%
\pgfsetdash{}{0pt}%
\pgfpathmoveto{\pgfqpoint{0.514278in}{1.789039in}}%
\pgfpathlineto{\pgfqpoint{2.399275in}{1.789039in}}%
\pgfusepath{stroke}%
\end{pgfscope}%
\begin{pgfscope}%
\pgfsetbuttcap%
\pgfsetmiterjoin%
\definecolor{currentfill}{rgb}{1.000000,1.000000,1.000000}%
\pgfsetfillcolor{currentfill}%
\pgfsetfillopacity{0.800000}%
\pgfsetlinewidth{1.003750pt}%
\definecolor{currentstroke}{rgb}{0.800000,0.800000,0.800000}%
\pgfsetstrokecolor{currentstroke}%
\pgfsetstrokeopacity{0.800000}%
\pgfsetdash{}{0pt}%
\pgfpathmoveto{\pgfqpoint{1.713209in}{1.523128in}}%
\pgfpathlineto{\pgfqpoint{2.321497in}{1.523128in}}%
\pgfpathquadraticcurveto{\pgfqpoint{2.343719in}{1.523128in}}{\pgfqpoint{2.343719in}{1.545351in}}%
\pgfpathlineto{\pgfqpoint{2.343719in}{1.711261in}}%
\pgfpathquadraticcurveto{\pgfqpoint{2.343719in}{1.733483in}}{\pgfqpoint{2.321497in}{1.733483in}}%
\pgfpathlineto{\pgfqpoint{1.713209in}{1.733483in}}%
\pgfpathquadraticcurveto{\pgfqpoint{1.690987in}{1.733483in}}{\pgfqpoint{1.690987in}{1.711261in}}%
\pgfpathlineto{\pgfqpoint{1.690987in}{1.545351in}}%
\pgfpathquadraticcurveto{\pgfqpoint{1.690987in}{1.523128in}}{\pgfqpoint{1.713209in}{1.523128in}}%
\pgfpathlineto{\pgfqpoint{1.713209in}{1.523128in}}%
\pgfpathclose%
\pgfusepath{stroke,fill}%
\end{pgfscope}%
\begin{pgfscope}%
\pgfsetbuttcap%
\pgfsetroundjoin%
\pgfsetlinewidth{1.505625pt}%
\definecolor{currentstroke}{rgb}{0.003922,0.450980,0.698039}%
\pgfsetstrokecolor{currentstroke}%
\pgfsetdash{{5.550000pt}{2.400000pt}}{0.000000pt}%
\pgfpathmoveto{\pgfqpoint{1.735431in}{1.628067in}}%
\pgfpathlineto{\pgfqpoint{1.846542in}{1.628067in}}%
\pgfpathlineto{\pgfqpoint{1.957653in}{1.628067in}}%
\pgfusepath{stroke}%
\end{pgfscope}%
\begin{pgfscope}%
\definecolor{textcolor}{rgb}{0.000000,0.000000,0.000000}%
\pgfsetstrokecolor{textcolor}%
\pgfsetfillcolor{textcolor}%
\pgftext[x=2.046542in,y=1.589178in,left,base]{\color{textcolor}{\rmfamily\fontsize{8.000000}{9.600000}\selectfont\catcode`\^=\active\def^{\ifmmode\sp\else\^{}\fi}\catcode`\%=\active\def%{\%}$\displaystyle h_{0}f^{0}$}}%
\end{pgfscope}%
\end{pgfpicture}%
\makeatother%
\endgroup%

        } % scalebox
        \caption{Power spectral density}
        \label{fig:white_noise_psd}
    \end{subfigure}
    \hfill
    \begin{subfigure}{0.32\linewidth}
        \scalebox{0.75}{%
            %% Creator: Matplotlib, PGF backend
%%
%% To include the figure in your LaTeX document, write
%%   \input{<filename>.pgf}
%%
%% Make sure the required packages are loaded in your preamble
%%   \usepackage{pgf}
%%
%% Also ensure that all the required font packages are loaded; for instance,
%% the lmodern package is sometimes necessary when using math font.
%%   \usepackage{lmodern}
%%
%% Figures using additional raster images can only be included by \input if
%% they are in the same directory as the main LaTeX file. For loading figures
%% from other directories you can use the `import` package
%%   \usepackage{import}
%%
%% and then include the figures with
%%   \import{<path to file>}{<filename>.pgf}
%%
%% Matplotlib used the following preamble
%%   \def\mathdefault#1{#1}
%%   \everymath=\expandafter{\the\everymath\displaystyle}
%%   \usepackage{siunitx}
%%   \sisetup{per-mode = symbol}%
%%   \ifdefined\pdftexversion\else  % non-pdftex case.
%%     \usepackage{fontspec}
%%   \fi
%%   \makeatletter\@ifpackageloaded{underscore}{}{\usepackage[strings]{underscore}}\makeatother
%%
\begingroup%
\makeatletter%
\begin{pgfpicture}%
\pgfpathrectangle{\pgfpointorigin}{\pgfqpoint{2.440945in}{1.830709in}}%
\pgfusepath{use as bounding box, clip}%
\begin{pgfscope}%
\pgfsetbuttcap%
\pgfsetmiterjoin%
\definecolor{currentfill}{rgb}{1.000000,1.000000,1.000000}%
\pgfsetfillcolor{currentfill}%
\pgfsetlinewidth{0.000000pt}%
\definecolor{currentstroke}{rgb}{1.000000,1.000000,1.000000}%
\pgfsetstrokecolor{currentstroke}%
\pgfsetdash{}{0pt}%
\pgfpathmoveto{\pgfqpoint{0.000000in}{0.000000in}}%
\pgfpathlineto{\pgfqpoint{2.440945in}{0.000000in}}%
\pgfpathlineto{\pgfqpoint{2.440945in}{1.830709in}}%
\pgfpathlineto{\pgfqpoint{0.000000in}{1.830709in}}%
\pgfpathlineto{\pgfqpoint{0.000000in}{0.000000in}}%
\pgfpathclose%
\pgfusepath{fill}%
\end{pgfscope}%
\begin{pgfscope}%
\pgfsetbuttcap%
\pgfsetmiterjoin%
\definecolor{currentfill}{rgb}{1.000000,1.000000,1.000000}%
\pgfsetfillcolor{currentfill}%
\pgfsetlinewidth{0.000000pt}%
\definecolor{currentstroke}{rgb}{0.000000,0.000000,0.000000}%
\pgfsetstrokecolor{currentstroke}%
\pgfsetstrokeopacity{0.000000}%
\pgfsetdash{}{0pt}%
\pgfpathmoveto{\pgfqpoint{0.589510in}{0.417642in}}%
\pgfpathlineto{\pgfqpoint{2.399275in}{0.417642in}}%
\pgfpathlineto{\pgfqpoint{2.399275in}{1.789039in}}%
\pgfpathlineto{\pgfqpoint{0.589510in}{1.789039in}}%
\pgfpathlineto{\pgfqpoint{0.589510in}{0.417642in}}%
\pgfpathclose%
\pgfusepath{fill}%
\end{pgfscope}%
\begin{pgfscope}%
\pgfpathrectangle{\pgfqpoint{0.589510in}{0.417642in}}{\pgfqpoint{1.809765in}{1.371397in}}%
\pgfusepath{clip}%
\pgfsetrectcap%
\pgfsetroundjoin%
\pgfsetlinewidth{0.803000pt}%
\definecolor{currentstroke}{rgb}{0.450000,0.450000,0.450000}%
\pgfsetstrokecolor{currentstroke}%
\pgfsetdash{}{0pt}%
\pgfpathmoveto{\pgfqpoint{0.671772in}{0.417642in}}%
\pgfpathlineto{\pgfqpoint{0.671772in}{1.789039in}}%
\pgfusepath{stroke}%
\end{pgfscope}%
\begin{pgfscope}%
\pgfsetbuttcap%
\pgfsetroundjoin%
\definecolor{currentfill}{rgb}{0.000000,0.000000,0.000000}%
\pgfsetfillcolor{currentfill}%
\pgfsetlinewidth{0.803000pt}%
\definecolor{currentstroke}{rgb}{0.000000,0.000000,0.000000}%
\pgfsetstrokecolor{currentstroke}%
\pgfsetdash{}{0pt}%
\pgfsys@defobject{currentmarker}{\pgfqpoint{0.000000in}{-0.048611in}}{\pgfqpoint{0.000000in}{0.000000in}}{%
\pgfpathmoveto{\pgfqpoint{0.000000in}{0.000000in}}%
\pgfpathlineto{\pgfqpoint{0.000000in}{-0.048611in}}%
\pgfusepath{stroke,fill}%
}%
\begin{pgfscope}%
\pgfsys@transformshift{0.671772in}{0.417642in}%
\pgfsys@useobject{currentmarker}{}%
\end{pgfscope}%
\end{pgfscope}%
\begin{pgfscope}%
\definecolor{textcolor}{rgb}{0.000000,0.000000,0.000000}%
\pgfsetstrokecolor{textcolor}%
\pgfsetfillcolor{textcolor}%
\pgftext[x=0.671772in,y=0.320420in,,top]{\color{textcolor}{\rmfamily\fontsize{8.000000}{9.600000}\selectfont\catcode`\^=\active\def^{\ifmmode\sp\else\^{}\fi}\catcode`\%=\active\def%{\%}$\mathdefault{10^{0}}$}}%
\end{pgfscope}%
\begin{pgfscope}%
\pgfpathrectangle{\pgfqpoint{0.589510in}{0.417642in}}{\pgfqpoint{1.809765in}{1.371397in}}%
\pgfusepath{clip}%
\pgfsetrectcap%
\pgfsetroundjoin%
\pgfsetlinewidth{0.803000pt}%
\definecolor{currentstroke}{rgb}{0.450000,0.450000,0.450000}%
\pgfsetstrokecolor{currentstroke}%
\pgfsetdash{}{0pt}%
\pgfpathmoveto{\pgfqpoint{1.128522in}{0.417642in}}%
\pgfpathlineto{\pgfqpoint{1.128522in}{1.789039in}}%
\pgfusepath{stroke}%
\end{pgfscope}%
\begin{pgfscope}%
\pgfsetbuttcap%
\pgfsetroundjoin%
\definecolor{currentfill}{rgb}{0.000000,0.000000,0.000000}%
\pgfsetfillcolor{currentfill}%
\pgfsetlinewidth{0.803000pt}%
\definecolor{currentstroke}{rgb}{0.000000,0.000000,0.000000}%
\pgfsetstrokecolor{currentstroke}%
\pgfsetdash{}{0pt}%
\pgfsys@defobject{currentmarker}{\pgfqpoint{0.000000in}{-0.048611in}}{\pgfqpoint{0.000000in}{0.000000in}}{%
\pgfpathmoveto{\pgfqpoint{0.000000in}{0.000000in}}%
\pgfpathlineto{\pgfqpoint{0.000000in}{-0.048611in}}%
\pgfusepath{stroke,fill}%
}%
\begin{pgfscope}%
\pgfsys@transformshift{1.128522in}{0.417642in}%
\pgfsys@useobject{currentmarker}{}%
\end{pgfscope}%
\end{pgfscope}%
\begin{pgfscope}%
\definecolor{textcolor}{rgb}{0.000000,0.000000,0.000000}%
\pgfsetstrokecolor{textcolor}%
\pgfsetfillcolor{textcolor}%
\pgftext[x=1.128522in,y=0.320420in,,top]{\color{textcolor}{\rmfamily\fontsize{8.000000}{9.600000}\selectfont\catcode`\^=\active\def^{\ifmmode\sp\else\^{}\fi}\catcode`\%=\active\def%{\%}$\mathdefault{10^{1}}$}}%
\end{pgfscope}%
\begin{pgfscope}%
\pgfpathrectangle{\pgfqpoint{0.589510in}{0.417642in}}{\pgfqpoint{1.809765in}{1.371397in}}%
\pgfusepath{clip}%
\pgfsetrectcap%
\pgfsetroundjoin%
\pgfsetlinewidth{0.803000pt}%
\definecolor{currentstroke}{rgb}{0.450000,0.450000,0.450000}%
\pgfsetstrokecolor{currentstroke}%
\pgfsetdash{}{0pt}%
\pgfpathmoveto{\pgfqpoint{1.585272in}{0.417642in}}%
\pgfpathlineto{\pgfqpoint{1.585272in}{1.789039in}}%
\pgfusepath{stroke}%
\end{pgfscope}%
\begin{pgfscope}%
\pgfsetbuttcap%
\pgfsetroundjoin%
\definecolor{currentfill}{rgb}{0.000000,0.000000,0.000000}%
\pgfsetfillcolor{currentfill}%
\pgfsetlinewidth{0.803000pt}%
\definecolor{currentstroke}{rgb}{0.000000,0.000000,0.000000}%
\pgfsetstrokecolor{currentstroke}%
\pgfsetdash{}{0pt}%
\pgfsys@defobject{currentmarker}{\pgfqpoint{0.000000in}{-0.048611in}}{\pgfqpoint{0.000000in}{0.000000in}}{%
\pgfpathmoveto{\pgfqpoint{0.000000in}{0.000000in}}%
\pgfpathlineto{\pgfqpoint{0.000000in}{-0.048611in}}%
\pgfusepath{stroke,fill}%
}%
\begin{pgfscope}%
\pgfsys@transformshift{1.585272in}{0.417642in}%
\pgfsys@useobject{currentmarker}{}%
\end{pgfscope}%
\end{pgfscope}%
\begin{pgfscope}%
\definecolor{textcolor}{rgb}{0.000000,0.000000,0.000000}%
\pgfsetstrokecolor{textcolor}%
\pgfsetfillcolor{textcolor}%
\pgftext[x=1.585272in,y=0.320420in,,top]{\color{textcolor}{\rmfamily\fontsize{8.000000}{9.600000}\selectfont\catcode`\^=\active\def^{\ifmmode\sp\else\^{}\fi}\catcode`\%=\active\def%{\%}$\mathdefault{10^{2}}$}}%
\end{pgfscope}%
\begin{pgfscope}%
\pgfpathrectangle{\pgfqpoint{0.589510in}{0.417642in}}{\pgfqpoint{1.809765in}{1.371397in}}%
\pgfusepath{clip}%
\pgfsetrectcap%
\pgfsetroundjoin%
\pgfsetlinewidth{0.803000pt}%
\definecolor{currentstroke}{rgb}{0.450000,0.450000,0.450000}%
\pgfsetstrokecolor{currentstroke}%
\pgfsetdash{}{0pt}%
\pgfpathmoveto{\pgfqpoint{2.042022in}{0.417642in}}%
\pgfpathlineto{\pgfqpoint{2.042022in}{1.789039in}}%
\pgfusepath{stroke}%
\end{pgfscope}%
\begin{pgfscope}%
\pgfsetbuttcap%
\pgfsetroundjoin%
\definecolor{currentfill}{rgb}{0.000000,0.000000,0.000000}%
\pgfsetfillcolor{currentfill}%
\pgfsetlinewidth{0.803000pt}%
\definecolor{currentstroke}{rgb}{0.000000,0.000000,0.000000}%
\pgfsetstrokecolor{currentstroke}%
\pgfsetdash{}{0pt}%
\pgfsys@defobject{currentmarker}{\pgfqpoint{0.000000in}{-0.048611in}}{\pgfqpoint{0.000000in}{0.000000in}}{%
\pgfpathmoveto{\pgfqpoint{0.000000in}{0.000000in}}%
\pgfpathlineto{\pgfqpoint{0.000000in}{-0.048611in}}%
\pgfusepath{stroke,fill}%
}%
\begin{pgfscope}%
\pgfsys@transformshift{2.042022in}{0.417642in}%
\pgfsys@useobject{currentmarker}{}%
\end{pgfscope}%
\end{pgfscope}%
\begin{pgfscope}%
\definecolor{textcolor}{rgb}{0.000000,0.000000,0.000000}%
\pgfsetstrokecolor{textcolor}%
\pgfsetfillcolor{textcolor}%
\pgftext[x=2.042022in,y=0.320420in,,top]{\color{textcolor}{\rmfamily\fontsize{8.000000}{9.600000}\selectfont\catcode`\^=\active\def^{\ifmmode\sp\else\^{}\fi}\catcode`\%=\active\def%{\%}$\mathdefault{10^{3}}$}}%
\end{pgfscope}%
\begin{pgfscope}%
\pgfpathrectangle{\pgfqpoint{0.589510in}{0.417642in}}{\pgfqpoint{1.809765in}{1.371397in}}%
\pgfusepath{clip}%
\pgfsetrectcap%
\pgfsetroundjoin%
\pgfsetlinewidth{0.803000pt}%
\definecolor{currentstroke}{rgb}{0.850000,0.850000,0.850000}%
\pgfsetstrokecolor{currentstroke}%
\pgfsetdash{}{0pt}%
\pgfpathmoveto{\pgfqpoint{0.601020in}{0.417642in}}%
\pgfpathlineto{\pgfqpoint{0.601020in}{1.789039in}}%
\pgfusepath{stroke}%
\end{pgfscope}%
\begin{pgfscope}%
\pgfsetbuttcap%
\pgfsetroundjoin%
\definecolor{currentfill}{rgb}{0.000000,0.000000,0.000000}%
\pgfsetfillcolor{currentfill}%
\pgfsetlinewidth{0.602250pt}%
\definecolor{currentstroke}{rgb}{0.000000,0.000000,0.000000}%
\pgfsetstrokecolor{currentstroke}%
\pgfsetdash{}{0pt}%
\pgfsys@defobject{currentmarker}{\pgfqpoint{0.000000in}{-0.027778in}}{\pgfqpoint{0.000000in}{0.000000in}}{%
\pgfpathmoveto{\pgfqpoint{0.000000in}{0.000000in}}%
\pgfpathlineto{\pgfqpoint{0.000000in}{-0.027778in}}%
\pgfusepath{stroke,fill}%
}%
\begin{pgfscope}%
\pgfsys@transformshift{0.601020in}{0.417642in}%
\pgfsys@useobject{currentmarker}{}%
\end{pgfscope}%
\end{pgfscope}%
\begin{pgfscope}%
\pgfpathrectangle{\pgfqpoint{0.589510in}{0.417642in}}{\pgfqpoint{1.809765in}{1.371397in}}%
\pgfusepath{clip}%
\pgfsetrectcap%
\pgfsetroundjoin%
\pgfsetlinewidth{0.803000pt}%
\definecolor{currentstroke}{rgb}{0.850000,0.850000,0.850000}%
\pgfsetstrokecolor{currentstroke}%
\pgfsetdash{}{0pt}%
\pgfpathmoveto{\pgfqpoint{0.627508in}{0.417642in}}%
\pgfpathlineto{\pgfqpoint{0.627508in}{1.789039in}}%
\pgfusepath{stroke}%
\end{pgfscope}%
\begin{pgfscope}%
\pgfsetbuttcap%
\pgfsetroundjoin%
\definecolor{currentfill}{rgb}{0.000000,0.000000,0.000000}%
\pgfsetfillcolor{currentfill}%
\pgfsetlinewidth{0.602250pt}%
\definecolor{currentstroke}{rgb}{0.000000,0.000000,0.000000}%
\pgfsetstrokecolor{currentstroke}%
\pgfsetdash{}{0pt}%
\pgfsys@defobject{currentmarker}{\pgfqpoint{0.000000in}{-0.027778in}}{\pgfqpoint{0.000000in}{0.000000in}}{%
\pgfpathmoveto{\pgfqpoint{0.000000in}{0.000000in}}%
\pgfpathlineto{\pgfqpoint{0.000000in}{-0.027778in}}%
\pgfusepath{stroke,fill}%
}%
\begin{pgfscope}%
\pgfsys@transformshift{0.627508in}{0.417642in}%
\pgfsys@useobject{currentmarker}{}%
\end{pgfscope}%
\end{pgfscope}%
\begin{pgfscope}%
\pgfpathrectangle{\pgfqpoint{0.589510in}{0.417642in}}{\pgfqpoint{1.809765in}{1.371397in}}%
\pgfusepath{clip}%
\pgfsetrectcap%
\pgfsetroundjoin%
\pgfsetlinewidth{0.803000pt}%
\definecolor{currentstroke}{rgb}{0.850000,0.850000,0.850000}%
\pgfsetstrokecolor{currentstroke}%
\pgfsetdash{}{0pt}%
\pgfpathmoveto{\pgfqpoint{0.650872in}{0.417642in}}%
\pgfpathlineto{\pgfqpoint{0.650872in}{1.789039in}}%
\pgfusepath{stroke}%
\end{pgfscope}%
\begin{pgfscope}%
\pgfsetbuttcap%
\pgfsetroundjoin%
\definecolor{currentfill}{rgb}{0.000000,0.000000,0.000000}%
\pgfsetfillcolor{currentfill}%
\pgfsetlinewidth{0.602250pt}%
\definecolor{currentstroke}{rgb}{0.000000,0.000000,0.000000}%
\pgfsetstrokecolor{currentstroke}%
\pgfsetdash{}{0pt}%
\pgfsys@defobject{currentmarker}{\pgfqpoint{0.000000in}{-0.027778in}}{\pgfqpoint{0.000000in}{0.000000in}}{%
\pgfpathmoveto{\pgfqpoint{0.000000in}{0.000000in}}%
\pgfpathlineto{\pgfqpoint{0.000000in}{-0.027778in}}%
\pgfusepath{stroke,fill}%
}%
\begin{pgfscope}%
\pgfsys@transformshift{0.650872in}{0.417642in}%
\pgfsys@useobject{currentmarker}{}%
\end{pgfscope}%
\end{pgfscope}%
\begin{pgfscope}%
\pgfpathrectangle{\pgfqpoint{0.589510in}{0.417642in}}{\pgfqpoint{1.809765in}{1.371397in}}%
\pgfusepath{clip}%
\pgfsetrectcap%
\pgfsetroundjoin%
\pgfsetlinewidth{0.803000pt}%
\definecolor{currentstroke}{rgb}{0.850000,0.850000,0.850000}%
\pgfsetstrokecolor{currentstroke}%
\pgfsetdash{}{0pt}%
\pgfpathmoveto{\pgfqpoint{0.809267in}{0.417642in}}%
\pgfpathlineto{\pgfqpoint{0.809267in}{1.789039in}}%
\pgfusepath{stroke}%
\end{pgfscope}%
\begin{pgfscope}%
\pgfsetbuttcap%
\pgfsetroundjoin%
\definecolor{currentfill}{rgb}{0.000000,0.000000,0.000000}%
\pgfsetfillcolor{currentfill}%
\pgfsetlinewidth{0.602250pt}%
\definecolor{currentstroke}{rgb}{0.000000,0.000000,0.000000}%
\pgfsetstrokecolor{currentstroke}%
\pgfsetdash{}{0pt}%
\pgfsys@defobject{currentmarker}{\pgfqpoint{0.000000in}{-0.027778in}}{\pgfqpoint{0.000000in}{0.000000in}}{%
\pgfpathmoveto{\pgfqpoint{0.000000in}{0.000000in}}%
\pgfpathlineto{\pgfqpoint{0.000000in}{-0.027778in}}%
\pgfusepath{stroke,fill}%
}%
\begin{pgfscope}%
\pgfsys@transformshift{0.809267in}{0.417642in}%
\pgfsys@useobject{currentmarker}{}%
\end{pgfscope}%
\end{pgfscope}%
\begin{pgfscope}%
\pgfpathrectangle{\pgfqpoint{0.589510in}{0.417642in}}{\pgfqpoint{1.809765in}{1.371397in}}%
\pgfusepath{clip}%
\pgfsetrectcap%
\pgfsetroundjoin%
\pgfsetlinewidth{0.803000pt}%
\definecolor{currentstroke}{rgb}{0.850000,0.850000,0.850000}%
\pgfsetstrokecolor{currentstroke}%
\pgfsetdash{}{0pt}%
\pgfpathmoveto{\pgfqpoint{0.889697in}{0.417642in}}%
\pgfpathlineto{\pgfqpoint{0.889697in}{1.789039in}}%
\pgfusepath{stroke}%
\end{pgfscope}%
\begin{pgfscope}%
\pgfsetbuttcap%
\pgfsetroundjoin%
\definecolor{currentfill}{rgb}{0.000000,0.000000,0.000000}%
\pgfsetfillcolor{currentfill}%
\pgfsetlinewidth{0.602250pt}%
\definecolor{currentstroke}{rgb}{0.000000,0.000000,0.000000}%
\pgfsetstrokecolor{currentstroke}%
\pgfsetdash{}{0pt}%
\pgfsys@defobject{currentmarker}{\pgfqpoint{0.000000in}{-0.027778in}}{\pgfqpoint{0.000000in}{0.000000in}}{%
\pgfpathmoveto{\pgfqpoint{0.000000in}{0.000000in}}%
\pgfpathlineto{\pgfqpoint{0.000000in}{-0.027778in}}%
\pgfusepath{stroke,fill}%
}%
\begin{pgfscope}%
\pgfsys@transformshift{0.889697in}{0.417642in}%
\pgfsys@useobject{currentmarker}{}%
\end{pgfscope}%
\end{pgfscope}%
\begin{pgfscope}%
\pgfpathrectangle{\pgfqpoint{0.589510in}{0.417642in}}{\pgfqpoint{1.809765in}{1.371397in}}%
\pgfusepath{clip}%
\pgfsetrectcap%
\pgfsetroundjoin%
\pgfsetlinewidth{0.803000pt}%
\definecolor{currentstroke}{rgb}{0.850000,0.850000,0.850000}%
\pgfsetstrokecolor{currentstroke}%
\pgfsetdash{}{0pt}%
\pgfpathmoveto{\pgfqpoint{0.946763in}{0.417642in}}%
\pgfpathlineto{\pgfqpoint{0.946763in}{1.789039in}}%
\pgfusepath{stroke}%
\end{pgfscope}%
\begin{pgfscope}%
\pgfsetbuttcap%
\pgfsetroundjoin%
\definecolor{currentfill}{rgb}{0.000000,0.000000,0.000000}%
\pgfsetfillcolor{currentfill}%
\pgfsetlinewidth{0.602250pt}%
\definecolor{currentstroke}{rgb}{0.000000,0.000000,0.000000}%
\pgfsetstrokecolor{currentstroke}%
\pgfsetdash{}{0pt}%
\pgfsys@defobject{currentmarker}{\pgfqpoint{0.000000in}{-0.027778in}}{\pgfqpoint{0.000000in}{0.000000in}}{%
\pgfpathmoveto{\pgfqpoint{0.000000in}{0.000000in}}%
\pgfpathlineto{\pgfqpoint{0.000000in}{-0.027778in}}%
\pgfusepath{stroke,fill}%
}%
\begin{pgfscope}%
\pgfsys@transformshift{0.946763in}{0.417642in}%
\pgfsys@useobject{currentmarker}{}%
\end{pgfscope}%
\end{pgfscope}%
\begin{pgfscope}%
\pgfpathrectangle{\pgfqpoint{0.589510in}{0.417642in}}{\pgfqpoint{1.809765in}{1.371397in}}%
\pgfusepath{clip}%
\pgfsetrectcap%
\pgfsetroundjoin%
\pgfsetlinewidth{0.803000pt}%
\definecolor{currentstroke}{rgb}{0.850000,0.850000,0.850000}%
\pgfsetstrokecolor{currentstroke}%
\pgfsetdash{}{0pt}%
\pgfpathmoveto{\pgfqpoint{0.991026in}{0.417642in}}%
\pgfpathlineto{\pgfqpoint{0.991026in}{1.789039in}}%
\pgfusepath{stroke}%
\end{pgfscope}%
\begin{pgfscope}%
\pgfsetbuttcap%
\pgfsetroundjoin%
\definecolor{currentfill}{rgb}{0.000000,0.000000,0.000000}%
\pgfsetfillcolor{currentfill}%
\pgfsetlinewidth{0.602250pt}%
\definecolor{currentstroke}{rgb}{0.000000,0.000000,0.000000}%
\pgfsetstrokecolor{currentstroke}%
\pgfsetdash{}{0pt}%
\pgfsys@defobject{currentmarker}{\pgfqpoint{0.000000in}{-0.027778in}}{\pgfqpoint{0.000000in}{0.000000in}}{%
\pgfpathmoveto{\pgfqpoint{0.000000in}{0.000000in}}%
\pgfpathlineto{\pgfqpoint{0.000000in}{-0.027778in}}%
\pgfusepath{stroke,fill}%
}%
\begin{pgfscope}%
\pgfsys@transformshift{0.991026in}{0.417642in}%
\pgfsys@useobject{currentmarker}{}%
\end{pgfscope}%
\end{pgfscope}%
\begin{pgfscope}%
\pgfpathrectangle{\pgfqpoint{0.589510in}{0.417642in}}{\pgfqpoint{1.809765in}{1.371397in}}%
\pgfusepath{clip}%
\pgfsetrectcap%
\pgfsetroundjoin%
\pgfsetlinewidth{0.803000pt}%
\definecolor{currentstroke}{rgb}{0.850000,0.850000,0.850000}%
\pgfsetstrokecolor{currentstroke}%
\pgfsetdash{}{0pt}%
\pgfpathmoveto{\pgfqpoint{1.027192in}{0.417642in}}%
\pgfpathlineto{\pgfqpoint{1.027192in}{1.789039in}}%
\pgfusepath{stroke}%
\end{pgfscope}%
\begin{pgfscope}%
\pgfsetbuttcap%
\pgfsetroundjoin%
\definecolor{currentfill}{rgb}{0.000000,0.000000,0.000000}%
\pgfsetfillcolor{currentfill}%
\pgfsetlinewidth{0.602250pt}%
\definecolor{currentstroke}{rgb}{0.000000,0.000000,0.000000}%
\pgfsetstrokecolor{currentstroke}%
\pgfsetdash{}{0pt}%
\pgfsys@defobject{currentmarker}{\pgfqpoint{0.000000in}{-0.027778in}}{\pgfqpoint{0.000000in}{0.000000in}}{%
\pgfpathmoveto{\pgfqpoint{0.000000in}{0.000000in}}%
\pgfpathlineto{\pgfqpoint{0.000000in}{-0.027778in}}%
\pgfusepath{stroke,fill}%
}%
\begin{pgfscope}%
\pgfsys@transformshift{1.027192in}{0.417642in}%
\pgfsys@useobject{currentmarker}{}%
\end{pgfscope}%
\end{pgfscope}%
\begin{pgfscope}%
\pgfpathrectangle{\pgfqpoint{0.589510in}{0.417642in}}{\pgfqpoint{1.809765in}{1.371397in}}%
\pgfusepath{clip}%
\pgfsetrectcap%
\pgfsetroundjoin%
\pgfsetlinewidth{0.803000pt}%
\definecolor{currentstroke}{rgb}{0.850000,0.850000,0.850000}%
\pgfsetstrokecolor{currentstroke}%
\pgfsetdash{}{0pt}%
\pgfpathmoveto{\pgfqpoint{1.057770in}{0.417642in}}%
\pgfpathlineto{\pgfqpoint{1.057770in}{1.789039in}}%
\pgfusepath{stroke}%
\end{pgfscope}%
\begin{pgfscope}%
\pgfsetbuttcap%
\pgfsetroundjoin%
\definecolor{currentfill}{rgb}{0.000000,0.000000,0.000000}%
\pgfsetfillcolor{currentfill}%
\pgfsetlinewidth{0.602250pt}%
\definecolor{currentstroke}{rgb}{0.000000,0.000000,0.000000}%
\pgfsetstrokecolor{currentstroke}%
\pgfsetdash{}{0pt}%
\pgfsys@defobject{currentmarker}{\pgfqpoint{0.000000in}{-0.027778in}}{\pgfqpoint{0.000000in}{0.000000in}}{%
\pgfpathmoveto{\pgfqpoint{0.000000in}{0.000000in}}%
\pgfpathlineto{\pgfqpoint{0.000000in}{-0.027778in}}%
\pgfusepath{stroke,fill}%
}%
\begin{pgfscope}%
\pgfsys@transformshift{1.057770in}{0.417642in}%
\pgfsys@useobject{currentmarker}{}%
\end{pgfscope}%
\end{pgfscope}%
\begin{pgfscope}%
\pgfpathrectangle{\pgfqpoint{0.589510in}{0.417642in}}{\pgfqpoint{1.809765in}{1.371397in}}%
\pgfusepath{clip}%
\pgfsetrectcap%
\pgfsetroundjoin%
\pgfsetlinewidth{0.803000pt}%
\definecolor{currentstroke}{rgb}{0.850000,0.850000,0.850000}%
\pgfsetstrokecolor{currentstroke}%
\pgfsetdash{}{0pt}%
\pgfpathmoveto{\pgfqpoint{1.084258in}{0.417642in}}%
\pgfpathlineto{\pgfqpoint{1.084258in}{1.789039in}}%
\pgfusepath{stroke}%
\end{pgfscope}%
\begin{pgfscope}%
\pgfsetbuttcap%
\pgfsetroundjoin%
\definecolor{currentfill}{rgb}{0.000000,0.000000,0.000000}%
\pgfsetfillcolor{currentfill}%
\pgfsetlinewidth{0.602250pt}%
\definecolor{currentstroke}{rgb}{0.000000,0.000000,0.000000}%
\pgfsetstrokecolor{currentstroke}%
\pgfsetdash{}{0pt}%
\pgfsys@defobject{currentmarker}{\pgfqpoint{0.000000in}{-0.027778in}}{\pgfqpoint{0.000000in}{0.000000in}}{%
\pgfpathmoveto{\pgfqpoint{0.000000in}{0.000000in}}%
\pgfpathlineto{\pgfqpoint{0.000000in}{-0.027778in}}%
\pgfusepath{stroke,fill}%
}%
\begin{pgfscope}%
\pgfsys@transformshift{1.084258in}{0.417642in}%
\pgfsys@useobject{currentmarker}{}%
\end{pgfscope}%
\end{pgfscope}%
\begin{pgfscope}%
\pgfpathrectangle{\pgfqpoint{0.589510in}{0.417642in}}{\pgfqpoint{1.809765in}{1.371397in}}%
\pgfusepath{clip}%
\pgfsetrectcap%
\pgfsetroundjoin%
\pgfsetlinewidth{0.803000pt}%
\definecolor{currentstroke}{rgb}{0.850000,0.850000,0.850000}%
\pgfsetstrokecolor{currentstroke}%
\pgfsetdash{}{0pt}%
\pgfpathmoveto{\pgfqpoint{1.107622in}{0.417642in}}%
\pgfpathlineto{\pgfqpoint{1.107622in}{1.789039in}}%
\pgfusepath{stroke}%
\end{pgfscope}%
\begin{pgfscope}%
\pgfsetbuttcap%
\pgfsetroundjoin%
\definecolor{currentfill}{rgb}{0.000000,0.000000,0.000000}%
\pgfsetfillcolor{currentfill}%
\pgfsetlinewidth{0.602250pt}%
\definecolor{currentstroke}{rgb}{0.000000,0.000000,0.000000}%
\pgfsetstrokecolor{currentstroke}%
\pgfsetdash{}{0pt}%
\pgfsys@defobject{currentmarker}{\pgfqpoint{0.000000in}{-0.027778in}}{\pgfqpoint{0.000000in}{0.000000in}}{%
\pgfpathmoveto{\pgfqpoint{0.000000in}{0.000000in}}%
\pgfpathlineto{\pgfqpoint{0.000000in}{-0.027778in}}%
\pgfusepath{stroke,fill}%
}%
\begin{pgfscope}%
\pgfsys@transformshift{1.107622in}{0.417642in}%
\pgfsys@useobject{currentmarker}{}%
\end{pgfscope}%
\end{pgfscope}%
\begin{pgfscope}%
\pgfpathrectangle{\pgfqpoint{0.589510in}{0.417642in}}{\pgfqpoint{1.809765in}{1.371397in}}%
\pgfusepath{clip}%
\pgfsetrectcap%
\pgfsetroundjoin%
\pgfsetlinewidth{0.803000pt}%
\definecolor{currentstroke}{rgb}{0.850000,0.850000,0.850000}%
\pgfsetstrokecolor{currentstroke}%
\pgfsetdash{}{0pt}%
\pgfpathmoveto{\pgfqpoint{1.266017in}{0.417642in}}%
\pgfpathlineto{\pgfqpoint{1.266017in}{1.789039in}}%
\pgfusepath{stroke}%
\end{pgfscope}%
\begin{pgfscope}%
\pgfsetbuttcap%
\pgfsetroundjoin%
\definecolor{currentfill}{rgb}{0.000000,0.000000,0.000000}%
\pgfsetfillcolor{currentfill}%
\pgfsetlinewidth{0.602250pt}%
\definecolor{currentstroke}{rgb}{0.000000,0.000000,0.000000}%
\pgfsetstrokecolor{currentstroke}%
\pgfsetdash{}{0pt}%
\pgfsys@defobject{currentmarker}{\pgfqpoint{0.000000in}{-0.027778in}}{\pgfqpoint{0.000000in}{0.000000in}}{%
\pgfpathmoveto{\pgfqpoint{0.000000in}{0.000000in}}%
\pgfpathlineto{\pgfqpoint{0.000000in}{-0.027778in}}%
\pgfusepath{stroke,fill}%
}%
\begin{pgfscope}%
\pgfsys@transformshift{1.266017in}{0.417642in}%
\pgfsys@useobject{currentmarker}{}%
\end{pgfscope}%
\end{pgfscope}%
\begin{pgfscope}%
\pgfpathrectangle{\pgfqpoint{0.589510in}{0.417642in}}{\pgfqpoint{1.809765in}{1.371397in}}%
\pgfusepath{clip}%
\pgfsetrectcap%
\pgfsetroundjoin%
\pgfsetlinewidth{0.803000pt}%
\definecolor{currentstroke}{rgb}{0.850000,0.850000,0.850000}%
\pgfsetstrokecolor{currentstroke}%
\pgfsetdash{}{0pt}%
\pgfpathmoveto{\pgfqpoint{1.346447in}{0.417642in}}%
\pgfpathlineto{\pgfqpoint{1.346447in}{1.789039in}}%
\pgfusepath{stroke}%
\end{pgfscope}%
\begin{pgfscope}%
\pgfsetbuttcap%
\pgfsetroundjoin%
\definecolor{currentfill}{rgb}{0.000000,0.000000,0.000000}%
\pgfsetfillcolor{currentfill}%
\pgfsetlinewidth{0.602250pt}%
\definecolor{currentstroke}{rgb}{0.000000,0.000000,0.000000}%
\pgfsetstrokecolor{currentstroke}%
\pgfsetdash{}{0pt}%
\pgfsys@defobject{currentmarker}{\pgfqpoint{0.000000in}{-0.027778in}}{\pgfqpoint{0.000000in}{0.000000in}}{%
\pgfpathmoveto{\pgfqpoint{0.000000in}{0.000000in}}%
\pgfpathlineto{\pgfqpoint{0.000000in}{-0.027778in}}%
\pgfusepath{stroke,fill}%
}%
\begin{pgfscope}%
\pgfsys@transformshift{1.346447in}{0.417642in}%
\pgfsys@useobject{currentmarker}{}%
\end{pgfscope}%
\end{pgfscope}%
\begin{pgfscope}%
\pgfpathrectangle{\pgfqpoint{0.589510in}{0.417642in}}{\pgfqpoint{1.809765in}{1.371397in}}%
\pgfusepath{clip}%
\pgfsetrectcap%
\pgfsetroundjoin%
\pgfsetlinewidth{0.803000pt}%
\definecolor{currentstroke}{rgb}{0.850000,0.850000,0.850000}%
\pgfsetstrokecolor{currentstroke}%
\pgfsetdash{}{0pt}%
\pgfpathmoveto{\pgfqpoint{1.403513in}{0.417642in}}%
\pgfpathlineto{\pgfqpoint{1.403513in}{1.789039in}}%
\pgfusepath{stroke}%
\end{pgfscope}%
\begin{pgfscope}%
\pgfsetbuttcap%
\pgfsetroundjoin%
\definecolor{currentfill}{rgb}{0.000000,0.000000,0.000000}%
\pgfsetfillcolor{currentfill}%
\pgfsetlinewidth{0.602250pt}%
\definecolor{currentstroke}{rgb}{0.000000,0.000000,0.000000}%
\pgfsetstrokecolor{currentstroke}%
\pgfsetdash{}{0pt}%
\pgfsys@defobject{currentmarker}{\pgfqpoint{0.000000in}{-0.027778in}}{\pgfqpoint{0.000000in}{0.000000in}}{%
\pgfpathmoveto{\pgfqpoint{0.000000in}{0.000000in}}%
\pgfpathlineto{\pgfqpoint{0.000000in}{-0.027778in}}%
\pgfusepath{stroke,fill}%
}%
\begin{pgfscope}%
\pgfsys@transformshift{1.403513in}{0.417642in}%
\pgfsys@useobject{currentmarker}{}%
\end{pgfscope}%
\end{pgfscope}%
\begin{pgfscope}%
\pgfpathrectangle{\pgfqpoint{0.589510in}{0.417642in}}{\pgfqpoint{1.809765in}{1.371397in}}%
\pgfusepath{clip}%
\pgfsetrectcap%
\pgfsetroundjoin%
\pgfsetlinewidth{0.803000pt}%
\definecolor{currentstroke}{rgb}{0.850000,0.850000,0.850000}%
\pgfsetstrokecolor{currentstroke}%
\pgfsetdash{}{0pt}%
\pgfpathmoveto{\pgfqpoint{1.447776in}{0.417642in}}%
\pgfpathlineto{\pgfqpoint{1.447776in}{1.789039in}}%
\pgfusepath{stroke}%
\end{pgfscope}%
\begin{pgfscope}%
\pgfsetbuttcap%
\pgfsetroundjoin%
\definecolor{currentfill}{rgb}{0.000000,0.000000,0.000000}%
\pgfsetfillcolor{currentfill}%
\pgfsetlinewidth{0.602250pt}%
\definecolor{currentstroke}{rgb}{0.000000,0.000000,0.000000}%
\pgfsetstrokecolor{currentstroke}%
\pgfsetdash{}{0pt}%
\pgfsys@defobject{currentmarker}{\pgfqpoint{0.000000in}{-0.027778in}}{\pgfqpoint{0.000000in}{0.000000in}}{%
\pgfpathmoveto{\pgfqpoint{0.000000in}{0.000000in}}%
\pgfpathlineto{\pgfqpoint{0.000000in}{-0.027778in}}%
\pgfusepath{stroke,fill}%
}%
\begin{pgfscope}%
\pgfsys@transformshift{1.447776in}{0.417642in}%
\pgfsys@useobject{currentmarker}{}%
\end{pgfscope}%
\end{pgfscope}%
\begin{pgfscope}%
\pgfpathrectangle{\pgfqpoint{0.589510in}{0.417642in}}{\pgfqpoint{1.809765in}{1.371397in}}%
\pgfusepath{clip}%
\pgfsetrectcap%
\pgfsetroundjoin%
\pgfsetlinewidth{0.803000pt}%
\definecolor{currentstroke}{rgb}{0.850000,0.850000,0.850000}%
\pgfsetstrokecolor{currentstroke}%
\pgfsetdash{}{0pt}%
\pgfpathmoveto{\pgfqpoint{1.483942in}{0.417642in}}%
\pgfpathlineto{\pgfqpoint{1.483942in}{1.789039in}}%
\pgfusepath{stroke}%
\end{pgfscope}%
\begin{pgfscope}%
\pgfsetbuttcap%
\pgfsetroundjoin%
\definecolor{currentfill}{rgb}{0.000000,0.000000,0.000000}%
\pgfsetfillcolor{currentfill}%
\pgfsetlinewidth{0.602250pt}%
\definecolor{currentstroke}{rgb}{0.000000,0.000000,0.000000}%
\pgfsetstrokecolor{currentstroke}%
\pgfsetdash{}{0pt}%
\pgfsys@defobject{currentmarker}{\pgfqpoint{0.000000in}{-0.027778in}}{\pgfqpoint{0.000000in}{0.000000in}}{%
\pgfpathmoveto{\pgfqpoint{0.000000in}{0.000000in}}%
\pgfpathlineto{\pgfqpoint{0.000000in}{-0.027778in}}%
\pgfusepath{stroke,fill}%
}%
\begin{pgfscope}%
\pgfsys@transformshift{1.483942in}{0.417642in}%
\pgfsys@useobject{currentmarker}{}%
\end{pgfscope}%
\end{pgfscope}%
\begin{pgfscope}%
\pgfpathrectangle{\pgfqpoint{0.589510in}{0.417642in}}{\pgfqpoint{1.809765in}{1.371397in}}%
\pgfusepath{clip}%
\pgfsetrectcap%
\pgfsetroundjoin%
\pgfsetlinewidth{0.803000pt}%
\definecolor{currentstroke}{rgb}{0.850000,0.850000,0.850000}%
\pgfsetstrokecolor{currentstroke}%
\pgfsetdash{}{0pt}%
\pgfpathmoveto{\pgfqpoint{1.514520in}{0.417642in}}%
\pgfpathlineto{\pgfqpoint{1.514520in}{1.789039in}}%
\pgfusepath{stroke}%
\end{pgfscope}%
\begin{pgfscope}%
\pgfsetbuttcap%
\pgfsetroundjoin%
\definecolor{currentfill}{rgb}{0.000000,0.000000,0.000000}%
\pgfsetfillcolor{currentfill}%
\pgfsetlinewidth{0.602250pt}%
\definecolor{currentstroke}{rgb}{0.000000,0.000000,0.000000}%
\pgfsetstrokecolor{currentstroke}%
\pgfsetdash{}{0pt}%
\pgfsys@defobject{currentmarker}{\pgfqpoint{0.000000in}{-0.027778in}}{\pgfqpoint{0.000000in}{0.000000in}}{%
\pgfpathmoveto{\pgfqpoint{0.000000in}{0.000000in}}%
\pgfpathlineto{\pgfqpoint{0.000000in}{-0.027778in}}%
\pgfusepath{stroke,fill}%
}%
\begin{pgfscope}%
\pgfsys@transformshift{1.514520in}{0.417642in}%
\pgfsys@useobject{currentmarker}{}%
\end{pgfscope}%
\end{pgfscope}%
\begin{pgfscope}%
\pgfpathrectangle{\pgfqpoint{0.589510in}{0.417642in}}{\pgfqpoint{1.809765in}{1.371397in}}%
\pgfusepath{clip}%
\pgfsetrectcap%
\pgfsetroundjoin%
\pgfsetlinewidth{0.803000pt}%
\definecolor{currentstroke}{rgb}{0.850000,0.850000,0.850000}%
\pgfsetstrokecolor{currentstroke}%
\pgfsetdash{}{0pt}%
\pgfpathmoveto{\pgfqpoint{1.541008in}{0.417642in}}%
\pgfpathlineto{\pgfqpoint{1.541008in}{1.789039in}}%
\pgfusepath{stroke}%
\end{pgfscope}%
\begin{pgfscope}%
\pgfsetbuttcap%
\pgfsetroundjoin%
\definecolor{currentfill}{rgb}{0.000000,0.000000,0.000000}%
\pgfsetfillcolor{currentfill}%
\pgfsetlinewidth{0.602250pt}%
\definecolor{currentstroke}{rgb}{0.000000,0.000000,0.000000}%
\pgfsetstrokecolor{currentstroke}%
\pgfsetdash{}{0pt}%
\pgfsys@defobject{currentmarker}{\pgfqpoint{0.000000in}{-0.027778in}}{\pgfqpoint{0.000000in}{0.000000in}}{%
\pgfpathmoveto{\pgfqpoint{0.000000in}{0.000000in}}%
\pgfpathlineto{\pgfqpoint{0.000000in}{-0.027778in}}%
\pgfusepath{stroke,fill}%
}%
\begin{pgfscope}%
\pgfsys@transformshift{1.541008in}{0.417642in}%
\pgfsys@useobject{currentmarker}{}%
\end{pgfscope}%
\end{pgfscope}%
\begin{pgfscope}%
\pgfpathrectangle{\pgfqpoint{0.589510in}{0.417642in}}{\pgfqpoint{1.809765in}{1.371397in}}%
\pgfusepath{clip}%
\pgfsetrectcap%
\pgfsetroundjoin%
\pgfsetlinewidth{0.803000pt}%
\definecolor{currentstroke}{rgb}{0.850000,0.850000,0.850000}%
\pgfsetstrokecolor{currentstroke}%
\pgfsetdash{}{0pt}%
\pgfpathmoveto{\pgfqpoint{1.564372in}{0.417642in}}%
\pgfpathlineto{\pgfqpoint{1.564372in}{1.789039in}}%
\pgfusepath{stroke}%
\end{pgfscope}%
\begin{pgfscope}%
\pgfsetbuttcap%
\pgfsetroundjoin%
\definecolor{currentfill}{rgb}{0.000000,0.000000,0.000000}%
\pgfsetfillcolor{currentfill}%
\pgfsetlinewidth{0.602250pt}%
\definecolor{currentstroke}{rgb}{0.000000,0.000000,0.000000}%
\pgfsetstrokecolor{currentstroke}%
\pgfsetdash{}{0pt}%
\pgfsys@defobject{currentmarker}{\pgfqpoint{0.000000in}{-0.027778in}}{\pgfqpoint{0.000000in}{0.000000in}}{%
\pgfpathmoveto{\pgfqpoint{0.000000in}{0.000000in}}%
\pgfpathlineto{\pgfqpoint{0.000000in}{-0.027778in}}%
\pgfusepath{stroke,fill}%
}%
\begin{pgfscope}%
\pgfsys@transformshift{1.564372in}{0.417642in}%
\pgfsys@useobject{currentmarker}{}%
\end{pgfscope}%
\end{pgfscope}%
\begin{pgfscope}%
\pgfpathrectangle{\pgfqpoint{0.589510in}{0.417642in}}{\pgfqpoint{1.809765in}{1.371397in}}%
\pgfusepath{clip}%
\pgfsetrectcap%
\pgfsetroundjoin%
\pgfsetlinewidth{0.803000pt}%
\definecolor{currentstroke}{rgb}{0.850000,0.850000,0.850000}%
\pgfsetstrokecolor{currentstroke}%
\pgfsetdash{}{0pt}%
\pgfpathmoveto{\pgfqpoint{1.722767in}{0.417642in}}%
\pgfpathlineto{\pgfqpoint{1.722767in}{1.789039in}}%
\pgfusepath{stroke}%
\end{pgfscope}%
\begin{pgfscope}%
\pgfsetbuttcap%
\pgfsetroundjoin%
\definecolor{currentfill}{rgb}{0.000000,0.000000,0.000000}%
\pgfsetfillcolor{currentfill}%
\pgfsetlinewidth{0.602250pt}%
\definecolor{currentstroke}{rgb}{0.000000,0.000000,0.000000}%
\pgfsetstrokecolor{currentstroke}%
\pgfsetdash{}{0pt}%
\pgfsys@defobject{currentmarker}{\pgfqpoint{0.000000in}{-0.027778in}}{\pgfqpoint{0.000000in}{0.000000in}}{%
\pgfpathmoveto{\pgfqpoint{0.000000in}{0.000000in}}%
\pgfpathlineto{\pgfqpoint{0.000000in}{-0.027778in}}%
\pgfusepath{stroke,fill}%
}%
\begin{pgfscope}%
\pgfsys@transformshift{1.722767in}{0.417642in}%
\pgfsys@useobject{currentmarker}{}%
\end{pgfscope}%
\end{pgfscope}%
\begin{pgfscope}%
\pgfpathrectangle{\pgfqpoint{0.589510in}{0.417642in}}{\pgfqpoint{1.809765in}{1.371397in}}%
\pgfusepath{clip}%
\pgfsetrectcap%
\pgfsetroundjoin%
\pgfsetlinewidth{0.803000pt}%
\definecolor{currentstroke}{rgb}{0.850000,0.850000,0.850000}%
\pgfsetstrokecolor{currentstroke}%
\pgfsetdash{}{0pt}%
\pgfpathmoveto{\pgfqpoint{1.803197in}{0.417642in}}%
\pgfpathlineto{\pgfqpoint{1.803197in}{1.789039in}}%
\pgfusepath{stroke}%
\end{pgfscope}%
\begin{pgfscope}%
\pgfsetbuttcap%
\pgfsetroundjoin%
\definecolor{currentfill}{rgb}{0.000000,0.000000,0.000000}%
\pgfsetfillcolor{currentfill}%
\pgfsetlinewidth{0.602250pt}%
\definecolor{currentstroke}{rgb}{0.000000,0.000000,0.000000}%
\pgfsetstrokecolor{currentstroke}%
\pgfsetdash{}{0pt}%
\pgfsys@defobject{currentmarker}{\pgfqpoint{0.000000in}{-0.027778in}}{\pgfqpoint{0.000000in}{0.000000in}}{%
\pgfpathmoveto{\pgfqpoint{0.000000in}{0.000000in}}%
\pgfpathlineto{\pgfqpoint{0.000000in}{-0.027778in}}%
\pgfusepath{stroke,fill}%
}%
\begin{pgfscope}%
\pgfsys@transformshift{1.803197in}{0.417642in}%
\pgfsys@useobject{currentmarker}{}%
\end{pgfscope}%
\end{pgfscope}%
\begin{pgfscope}%
\pgfpathrectangle{\pgfqpoint{0.589510in}{0.417642in}}{\pgfqpoint{1.809765in}{1.371397in}}%
\pgfusepath{clip}%
\pgfsetrectcap%
\pgfsetroundjoin%
\pgfsetlinewidth{0.803000pt}%
\definecolor{currentstroke}{rgb}{0.850000,0.850000,0.850000}%
\pgfsetstrokecolor{currentstroke}%
\pgfsetdash{}{0pt}%
\pgfpathmoveto{\pgfqpoint{1.860263in}{0.417642in}}%
\pgfpathlineto{\pgfqpoint{1.860263in}{1.789039in}}%
\pgfusepath{stroke}%
\end{pgfscope}%
\begin{pgfscope}%
\pgfsetbuttcap%
\pgfsetroundjoin%
\definecolor{currentfill}{rgb}{0.000000,0.000000,0.000000}%
\pgfsetfillcolor{currentfill}%
\pgfsetlinewidth{0.602250pt}%
\definecolor{currentstroke}{rgb}{0.000000,0.000000,0.000000}%
\pgfsetstrokecolor{currentstroke}%
\pgfsetdash{}{0pt}%
\pgfsys@defobject{currentmarker}{\pgfqpoint{0.000000in}{-0.027778in}}{\pgfqpoint{0.000000in}{0.000000in}}{%
\pgfpathmoveto{\pgfqpoint{0.000000in}{0.000000in}}%
\pgfpathlineto{\pgfqpoint{0.000000in}{-0.027778in}}%
\pgfusepath{stroke,fill}%
}%
\begin{pgfscope}%
\pgfsys@transformshift{1.860263in}{0.417642in}%
\pgfsys@useobject{currentmarker}{}%
\end{pgfscope}%
\end{pgfscope}%
\begin{pgfscope}%
\pgfpathrectangle{\pgfqpoint{0.589510in}{0.417642in}}{\pgfqpoint{1.809765in}{1.371397in}}%
\pgfusepath{clip}%
\pgfsetrectcap%
\pgfsetroundjoin%
\pgfsetlinewidth{0.803000pt}%
\definecolor{currentstroke}{rgb}{0.850000,0.850000,0.850000}%
\pgfsetstrokecolor{currentstroke}%
\pgfsetdash{}{0pt}%
\pgfpathmoveto{\pgfqpoint{1.904526in}{0.417642in}}%
\pgfpathlineto{\pgfqpoint{1.904526in}{1.789039in}}%
\pgfusepath{stroke}%
\end{pgfscope}%
\begin{pgfscope}%
\pgfsetbuttcap%
\pgfsetroundjoin%
\definecolor{currentfill}{rgb}{0.000000,0.000000,0.000000}%
\pgfsetfillcolor{currentfill}%
\pgfsetlinewidth{0.602250pt}%
\definecolor{currentstroke}{rgb}{0.000000,0.000000,0.000000}%
\pgfsetstrokecolor{currentstroke}%
\pgfsetdash{}{0pt}%
\pgfsys@defobject{currentmarker}{\pgfqpoint{0.000000in}{-0.027778in}}{\pgfqpoint{0.000000in}{0.000000in}}{%
\pgfpathmoveto{\pgfqpoint{0.000000in}{0.000000in}}%
\pgfpathlineto{\pgfqpoint{0.000000in}{-0.027778in}}%
\pgfusepath{stroke,fill}%
}%
\begin{pgfscope}%
\pgfsys@transformshift{1.904526in}{0.417642in}%
\pgfsys@useobject{currentmarker}{}%
\end{pgfscope}%
\end{pgfscope}%
\begin{pgfscope}%
\pgfpathrectangle{\pgfqpoint{0.589510in}{0.417642in}}{\pgfqpoint{1.809765in}{1.371397in}}%
\pgfusepath{clip}%
\pgfsetrectcap%
\pgfsetroundjoin%
\pgfsetlinewidth{0.803000pt}%
\definecolor{currentstroke}{rgb}{0.850000,0.850000,0.850000}%
\pgfsetstrokecolor{currentstroke}%
\pgfsetdash{}{0pt}%
\pgfpathmoveto{\pgfqpoint{1.940693in}{0.417642in}}%
\pgfpathlineto{\pgfqpoint{1.940693in}{1.789039in}}%
\pgfusepath{stroke}%
\end{pgfscope}%
\begin{pgfscope}%
\pgfsetbuttcap%
\pgfsetroundjoin%
\definecolor{currentfill}{rgb}{0.000000,0.000000,0.000000}%
\pgfsetfillcolor{currentfill}%
\pgfsetlinewidth{0.602250pt}%
\definecolor{currentstroke}{rgb}{0.000000,0.000000,0.000000}%
\pgfsetstrokecolor{currentstroke}%
\pgfsetdash{}{0pt}%
\pgfsys@defobject{currentmarker}{\pgfqpoint{0.000000in}{-0.027778in}}{\pgfqpoint{0.000000in}{0.000000in}}{%
\pgfpathmoveto{\pgfqpoint{0.000000in}{0.000000in}}%
\pgfpathlineto{\pgfqpoint{0.000000in}{-0.027778in}}%
\pgfusepath{stroke,fill}%
}%
\begin{pgfscope}%
\pgfsys@transformshift{1.940693in}{0.417642in}%
\pgfsys@useobject{currentmarker}{}%
\end{pgfscope}%
\end{pgfscope}%
\begin{pgfscope}%
\pgfpathrectangle{\pgfqpoint{0.589510in}{0.417642in}}{\pgfqpoint{1.809765in}{1.371397in}}%
\pgfusepath{clip}%
\pgfsetrectcap%
\pgfsetroundjoin%
\pgfsetlinewidth{0.803000pt}%
\definecolor{currentstroke}{rgb}{0.850000,0.850000,0.850000}%
\pgfsetstrokecolor{currentstroke}%
\pgfsetdash{}{0pt}%
\pgfpathmoveto{\pgfqpoint{1.971270in}{0.417642in}}%
\pgfpathlineto{\pgfqpoint{1.971270in}{1.789039in}}%
\pgfusepath{stroke}%
\end{pgfscope}%
\begin{pgfscope}%
\pgfsetbuttcap%
\pgfsetroundjoin%
\definecolor{currentfill}{rgb}{0.000000,0.000000,0.000000}%
\pgfsetfillcolor{currentfill}%
\pgfsetlinewidth{0.602250pt}%
\definecolor{currentstroke}{rgb}{0.000000,0.000000,0.000000}%
\pgfsetstrokecolor{currentstroke}%
\pgfsetdash{}{0pt}%
\pgfsys@defobject{currentmarker}{\pgfqpoint{0.000000in}{-0.027778in}}{\pgfqpoint{0.000000in}{0.000000in}}{%
\pgfpathmoveto{\pgfqpoint{0.000000in}{0.000000in}}%
\pgfpathlineto{\pgfqpoint{0.000000in}{-0.027778in}}%
\pgfusepath{stroke,fill}%
}%
\begin{pgfscope}%
\pgfsys@transformshift{1.971270in}{0.417642in}%
\pgfsys@useobject{currentmarker}{}%
\end{pgfscope}%
\end{pgfscope}%
\begin{pgfscope}%
\pgfpathrectangle{\pgfqpoint{0.589510in}{0.417642in}}{\pgfqpoint{1.809765in}{1.371397in}}%
\pgfusepath{clip}%
\pgfsetrectcap%
\pgfsetroundjoin%
\pgfsetlinewidth{0.803000pt}%
\definecolor{currentstroke}{rgb}{0.850000,0.850000,0.850000}%
\pgfsetstrokecolor{currentstroke}%
\pgfsetdash{}{0pt}%
\pgfpathmoveto{\pgfqpoint{1.997758in}{0.417642in}}%
\pgfpathlineto{\pgfqpoint{1.997758in}{1.789039in}}%
\pgfusepath{stroke}%
\end{pgfscope}%
\begin{pgfscope}%
\pgfsetbuttcap%
\pgfsetroundjoin%
\definecolor{currentfill}{rgb}{0.000000,0.000000,0.000000}%
\pgfsetfillcolor{currentfill}%
\pgfsetlinewidth{0.602250pt}%
\definecolor{currentstroke}{rgb}{0.000000,0.000000,0.000000}%
\pgfsetstrokecolor{currentstroke}%
\pgfsetdash{}{0pt}%
\pgfsys@defobject{currentmarker}{\pgfqpoint{0.000000in}{-0.027778in}}{\pgfqpoint{0.000000in}{0.000000in}}{%
\pgfpathmoveto{\pgfqpoint{0.000000in}{0.000000in}}%
\pgfpathlineto{\pgfqpoint{0.000000in}{-0.027778in}}%
\pgfusepath{stroke,fill}%
}%
\begin{pgfscope}%
\pgfsys@transformshift{1.997758in}{0.417642in}%
\pgfsys@useobject{currentmarker}{}%
\end{pgfscope}%
\end{pgfscope}%
\begin{pgfscope}%
\pgfpathrectangle{\pgfqpoint{0.589510in}{0.417642in}}{\pgfqpoint{1.809765in}{1.371397in}}%
\pgfusepath{clip}%
\pgfsetrectcap%
\pgfsetroundjoin%
\pgfsetlinewidth{0.803000pt}%
\definecolor{currentstroke}{rgb}{0.850000,0.850000,0.850000}%
\pgfsetstrokecolor{currentstroke}%
\pgfsetdash{}{0pt}%
\pgfpathmoveto{\pgfqpoint{2.021122in}{0.417642in}}%
\pgfpathlineto{\pgfqpoint{2.021122in}{1.789039in}}%
\pgfusepath{stroke}%
\end{pgfscope}%
\begin{pgfscope}%
\pgfsetbuttcap%
\pgfsetroundjoin%
\definecolor{currentfill}{rgb}{0.000000,0.000000,0.000000}%
\pgfsetfillcolor{currentfill}%
\pgfsetlinewidth{0.602250pt}%
\definecolor{currentstroke}{rgb}{0.000000,0.000000,0.000000}%
\pgfsetstrokecolor{currentstroke}%
\pgfsetdash{}{0pt}%
\pgfsys@defobject{currentmarker}{\pgfqpoint{0.000000in}{-0.027778in}}{\pgfqpoint{0.000000in}{0.000000in}}{%
\pgfpathmoveto{\pgfqpoint{0.000000in}{0.000000in}}%
\pgfpathlineto{\pgfqpoint{0.000000in}{-0.027778in}}%
\pgfusepath{stroke,fill}%
}%
\begin{pgfscope}%
\pgfsys@transformshift{2.021122in}{0.417642in}%
\pgfsys@useobject{currentmarker}{}%
\end{pgfscope}%
\end{pgfscope}%
\begin{pgfscope}%
\pgfpathrectangle{\pgfqpoint{0.589510in}{0.417642in}}{\pgfqpoint{1.809765in}{1.371397in}}%
\pgfusepath{clip}%
\pgfsetrectcap%
\pgfsetroundjoin%
\pgfsetlinewidth{0.803000pt}%
\definecolor{currentstroke}{rgb}{0.850000,0.850000,0.850000}%
\pgfsetstrokecolor{currentstroke}%
\pgfsetdash{}{0pt}%
\pgfpathmoveto{\pgfqpoint{2.179517in}{0.417642in}}%
\pgfpathlineto{\pgfqpoint{2.179517in}{1.789039in}}%
\pgfusepath{stroke}%
\end{pgfscope}%
\begin{pgfscope}%
\pgfsetbuttcap%
\pgfsetroundjoin%
\definecolor{currentfill}{rgb}{0.000000,0.000000,0.000000}%
\pgfsetfillcolor{currentfill}%
\pgfsetlinewidth{0.602250pt}%
\definecolor{currentstroke}{rgb}{0.000000,0.000000,0.000000}%
\pgfsetstrokecolor{currentstroke}%
\pgfsetdash{}{0pt}%
\pgfsys@defobject{currentmarker}{\pgfqpoint{0.000000in}{-0.027778in}}{\pgfqpoint{0.000000in}{0.000000in}}{%
\pgfpathmoveto{\pgfqpoint{0.000000in}{0.000000in}}%
\pgfpathlineto{\pgfqpoint{0.000000in}{-0.027778in}}%
\pgfusepath{stroke,fill}%
}%
\begin{pgfscope}%
\pgfsys@transformshift{2.179517in}{0.417642in}%
\pgfsys@useobject{currentmarker}{}%
\end{pgfscope}%
\end{pgfscope}%
\begin{pgfscope}%
\pgfpathrectangle{\pgfqpoint{0.589510in}{0.417642in}}{\pgfqpoint{1.809765in}{1.371397in}}%
\pgfusepath{clip}%
\pgfsetrectcap%
\pgfsetroundjoin%
\pgfsetlinewidth{0.803000pt}%
\definecolor{currentstroke}{rgb}{0.850000,0.850000,0.850000}%
\pgfsetstrokecolor{currentstroke}%
\pgfsetdash{}{0pt}%
\pgfpathmoveto{\pgfqpoint{2.259947in}{0.417642in}}%
\pgfpathlineto{\pgfqpoint{2.259947in}{1.789039in}}%
\pgfusepath{stroke}%
\end{pgfscope}%
\begin{pgfscope}%
\pgfsetbuttcap%
\pgfsetroundjoin%
\definecolor{currentfill}{rgb}{0.000000,0.000000,0.000000}%
\pgfsetfillcolor{currentfill}%
\pgfsetlinewidth{0.602250pt}%
\definecolor{currentstroke}{rgb}{0.000000,0.000000,0.000000}%
\pgfsetstrokecolor{currentstroke}%
\pgfsetdash{}{0pt}%
\pgfsys@defobject{currentmarker}{\pgfqpoint{0.000000in}{-0.027778in}}{\pgfqpoint{0.000000in}{0.000000in}}{%
\pgfpathmoveto{\pgfqpoint{0.000000in}{0.000000in}}%
\pgfpathlineto{\pgfqpoint{0.000000in}{-0.027778in}}%
\pgfusepath{stroke,fill}%
}%
\begin{pgfscope}%
\pgfsys@transformshift{2.259947in}{0.417642in}%
\pgfsys@useobject{currentmarker}{}%
\end{pgfscope}%
\end{pgfscope}%
\begin{pgfscope}%
\pgfpathrectangle{\pgfqpoint{0.589510in}{0.417642in}}{\pgfqpoint{1.809765in}{1.371397in}}%
\pgfusepath{clip}%
\pgfsetrectcap%
\pgfsetroundjoin%
\pgfsetlinewidth{0.803000pt}%
\definecolor{currentstroke}{rgb}{0.850000,0.850000,0.850000}%
\pgfsetstrokecolor{currentstroke}%
\pgfsetdash{}{0pt}%
\pgfpathmoveto{\pgfqpoint{2.317013in}{0.417642in}}%
\pgfpathlineto{\pgfqpoint{2.317013in}{1.789039in}}%
\pgfusepath{stroke}%
\end{pgfscope}%
\begin{pgfscope}%
\pgfsetbuttcap%
\pgfsetroundjoin%
\definecolor{currentfill}{rgb}{0.000000,0.000000,0.000000}%
\pgfsetfillcolor{currentfill}%
\pgfsetlinewidth{0.602250pt}%
\definecolor{currentstroke}{rgb}{0.000000,0.000000,0.000000}%
\pgfsetstrokecolor{currentstroke}%
\pgfsetdash{}{0pt}%
\pgfsys@defobject{currentmarker}{\pgfqpoint{0.000000in}{-0.027778in}}{\pgfqpoint{0.000000in}{0.000000in}}{%
\pgfpathmoveto{\pgfqpoint{0.000000in}{0.000000in}}%
\pgfpathlineto{\pgfqpoint{0.000000in}{-0.027778in}}%
\pgfusepath{stroke,fill}%
}%
\begin{pgfscope}%
\pgfsys@transformshift{2.317013in}{0.417642in}%
\pgfsys@useobject{currentmarker}{}%
\end{pgfscope}%
\end{pgfscope}%
\begin{pgfscope}%
\pgfpathrectangle{\pgfqpoint{0.589510in}{0.417642in}}{\pgfqpoint{1.809765in}{1.371397in}}%
\pgfusepath{clip}%
\pgfsetrectcap%
\pgfsetroundjoin%
\pgfsetlinewidth{0.803000pt}%
\definecolor{currentstroke}{rgb}{0.850000,0.850000,0.850000}%
\pgfsetstrokecolor{currentstroke}%
\pgfsetdash{}{0pt}%
\pgfpathmoveto{\pgfqpoint{2.361277in}{0.417642in}}%
\pgfpathlineto{\pgfqpoint{2.361277in}{1.789039in}}%
\pgfusepath{stroke}%
\end{pgfscope}%
\begin{pgfscope}%
\pgfsetbuttcap%
\pgfsetroundjoin%
\definecolor{currentfill}{rgb}{0.000000,0.000000,0.000000}%
\pgfsetfillcolor{currentfill}%
\pgfsetlinewidth{0.602250pt}%
\definecolor{currentstroke}{rgb}{0.000000,0.000000,0.000000}%
\pgfsetstrokecolor{currentstroke}%
\pgfsetdash{}{0pt}%
\pgfsys@defobject{currentmarker}{\pgfqpoint{0.000000in}{-0.027778in}}{\pgfqpoint{0.000000in}{0.000000in}}{%
\pgfpathmoveto{\pgfqpoint{0.000000in}{0.000000in}}%
\pgfpathlineto{\pgfqpoint{0.000000in}{-0.027778in}}%
\pgfusepath{stroke,fill}%
}%
\begin{pgfscope}%
\pgfsys@transformshift{2.361277in}{0.417642in}%
\pgfsys@useobject{currentmarker}{}%
\end{pgfscope}%
\end{pgfscope}%
\begin{pgfscope}%
\pgfpathrectangle{\pgfqpoint{0.589510in}{0.417642in}}{\pgfqpoint{1.809765in}{1.371397in}}%
\pgfusepath{clip}%
\pgfsetrectcap%
\pgfsetroundjoin%
\pgfsetlinewidth{0.803000pt}%
\definecolor{currentstroke}{rgb}{0.850000,0.850000,0.850000}%
\pgfsetstrokecolor{currentstroke}%
\pgfsetdash{}{0pt}%
\pgfpathmoveto{\pgfqpoint{2.397443in}{0.417642in}}%
\pgfpathlineto{\pgfqpoint{2.397443in}{1.789039in}}%
\pgfusepath{stroke}%
\end{pgfscope}%
\begin{pgfscope}%
\pgfsetbuttcap%
\pgfsetroundjoin%
\definecolor{currentfill}{rgb}{0.000000,0.000000,0.000000}%
\pgfsetfillcolor{currentfill}%
\pgfsetlinewidth{0.602250pt}%
\definecolor{currentstroke}{rgb}{0.000000,0.000000,0.000000}%
\pgfsetstrokecolor{currentstroke}%
\pgfsetdash{}{0pt}%
\pgfsys@defobject{currentmarker}{\pgfqpoint{0.000000in}{-0.027778in}}{\pgfqpoint{0.000000in}{0.000000in}}{%
\pgfpathmoveto{\pgfqpoint{0.000000in}{0.000000in}}%
\pgfpathlineto{\pgfqpoint{0.000000in}{-0.027778in}}%
\pgfusepath{stroke,fill}%
}%
\begin{pgfscope}%
\pgfsys@transformshift{2.397443in}{0.417642in}%
\pgfsys@useobject{currentmarker}{}%
\end{pgfscope}%
\end{pgfscope}%
\begin{pgfscope}%
\definecolor{textcolor}{rgb}{0.000000,0.000000,0.000000}%
\pgfsetstrokecolor{textcolor}%
\pgfsetfillcolor{textcolor}%
\pgftext[x=1.494392in,y=0.165003in,,top]{\color{textcolor}{\rmfamily\fontsize{10.000000}{12.000000}\selectfont\catcode`\^=\active\def^{\ifmmode\sp\else\^{}\fi}\catcode`\%=\active\def%{\%}$\tau$ in \unit{\second}}}%
\end{pgfscope}%
\begin{pgfscope}%
\pgfpathrectangle{\pgfqpoint{0.589510in}{0.417642in}}{\pgfqpoint{1.809765in}{1.371397in}}%
\pgfusepath{clip}%
\pgfsetrectcap%
\pgfsetroundjoin%
\pgfsetlinewidth{0.803000pt}%
\definecolor{currentstroke}{rgb}{0.450000,0.450000,0.450000}%
\pgfsetstrokecolor{currentstroke}%
\pgfsetdash{}{0pt}%
\pgfpathmoveto{\pgfqpoint{0.589510in}{0.417642in}}%
\pgfpathlineto{\pgfqpoint{2.399275in}{0.417642in}}%
\pgfusepath{stroke}%
\end{pgfscope}%
\begin{pgfscope}%
\pgfsetbuttcap%
\pgfsetroundjoin%
\definecolor{currentfill}{rgb}{0.000000,0.000000,0.000000}%
\pgfsetfillcolor{currentfill}%
\pgfsetlinewidth{0.803000pt}%
\definecolor{currentstroke}{rgb}{0.000000,0.000000,0.000000}%
\pgfsetstrokecolor{currentstroke}%
\pgfsetdash{}{0pt}%
\pgfsys@defobject{currentmarker}{\pgfqpoint{-0.048611in}{0.000000in}}{\pgfqpoint{-0.000000in}{0.000000in}}{%
\pgfpathmoveto{\pgfqpoint{-0.000000in}{0.000000in}}%
\pgfpathlineto{\pgfqpoint{-0.048611in}{0.000000in}}%
\pgfusepath{stroke,fill}%
}%
\begin{pgfscope}%
\pgfsys@transformshift{0.589510in}{0.417642in}%
\pgfsys@useobject{currentmarker}{}%
\end{pgfscope}%
\end{pgfscope}%
\begin{pgfscope}%
\definecolor{textcolor}{rgb}{0.000000,0.000000,0.000000}%
\pgfsetstrokecolor{textcolor}%
\pgfsetfillcolor{textcolor}%
\pgftext[x=0.236114in, y=0.378489in, left, base]{\color{textcolor}{\rmfamily\fontsize{8.000000}{9.600000}\selectfont\catcode`\^=\active\def^{\ifmmode\sp\else\^{}\fi}\catcode`\%=\active\def%{\%}$\mathdefault{10^{-2}}$}}%
\end{pgfscope}%
\begin{pgfscope}%
\pgfpathrectangle{\pgfqpoint{0.589510in}{0.417642in}}{\pgfqpoint{1.809765in}{1.371397in}}%
\pgfusepath{clip}%
\pgfsetrectcap%
\pgfsetroundjoin%
\pgfsetlinewidth{0.803000pt}%
\definecolor{currentstroke}{rgb}{0.450000,0.450000,0.450000}%
\pgfsetstrokecolor{currentstroke}%
\pgfsetdash{}{0pt}%
\pgfpathmoveto{\pgfqpoint{0.589510in}{0.622360in}}%
\pgfpathlineto{\pgfqpoint{2.399275in}{0.622360in}}%
\pgfusepath{stroke}%
\end{pgfscope}%
\begin{pgfscope}%
\pgfsetbuttcap%
\pgfsetroundjoin%
\definecolor{currentfill}{rgb}{0.000000,0.000000,0.000000}%
\pgfsetfillcolor{currentfill}%
\pgfsetlinewidth{0.803000pt}%
\definecolor{currentstroke}{rgb}{0.000000,0.000000,0.000000}%
\pgfsetstrokecolor{currentstroke}%
\pgfsetdash{}{0pt}%
\pgfsys@defobject{currentmarker}{\pgfqpoint{-0.048611in}{0.000000in}}{\pgfqpoint{-0.000000in}{0.000000in}}{%
\pgfpathmoveto{\pgfqpoint{-0.000000in}{0.000000in}}%
\pgfpathlineto{\pgfqpoint{-0.048611in}{0.000000in}}%
\pgfusepath{stroke,fill}%
}%
\begin{pgfscope}%
\pgfsys@transformshift{0.589510in}{0.622360in}%
\pgfsys@useobject{currentmarker}{}%
\end{pgfscope}%
\end{pgfscope}%
\begin{pgfscope}%
\definecolor{textcolor}{rgb}{0.000000,0.000000,0.000000}%
\pgfsetstrokecolor{textcolor}%
\pgfsetfillcolor{textcolor}%
\pgftext[x=0.236114in, y=0.583207in, left, base]{\color{textcolor}{\rmfamily\fontsize{8.000000}{9.600000}\selectfont\catcode`\^=\active\def^{\ifmmode\sp\else\^{}\fi}\catcode`\%=\active\def%{\%}$\mathdefault{10^{-1}}$}}%
\end{pgfscope}%
\begin{pgfscope}%
\pgfpathrectangle{\pgfqpoint{0.589510in}{0.417642in}}{\pgfqpoint{1.809765in}{1.371397in}}%
\pgfusepath{clip}%
\pgfsetrectcap%
\pgfsetroundjoin%
\pgfsetlinewidth{0.803000pt}%
\definecolor{currentstroke}{rgb}{0.450000,0.450000,0.450000}%
\pgfsetstrokecolor{currentstroke}%
\pgfsetdash{}{0pt}%
\pgfpathmoveto{\pgfqpoint{0.589510in}{0.827077in}}%
\pgfpathlineto{\pgfqpoint{2.399275in}{0.827077in}}%
\pgfusepath{stroke}%
\end{pgfscope}%
\begin{pgfscope}%
\pgfsetbuttcap%
\pgfsetroundjoin%
\definecolor{currentfill}{rgb}{0.000000,0.000000,0.000000}%
\pgfsetfillcolor{currentfill}%
\pgfsetlinewidth{0.803000pt}%
\definecolor{currentstroke}{rgb}{0.000000,0.000000,0.000000}%
\pgfsetstrokecolor{currentstroke}%
\pgfsetdash{}{0pt}%
\pgfsys@defobject{currentmarker}{\pgfqpoint{-0.048611in}{0.000000in}}{\pgfqpoint{-0.000000in}{0.000000in}}{%
\pgfpathmoveto{\pgfqpoint{-0.000000in}{0.000000in}}%
\pgfpathlineto{\pgfqpoint{-0.048611in}{0.000000in}}%
\pgfusepath{stroke,fill}%
}%
\begin{pgfscope}%
\pgfsys@transformshift{0.589510in}{0.827077in}%
\pgfsys@useobject{currentmarker}{}%
\end{pgfscope}%
\end{pgfscope}%
\begin{pgfscope}%
\definecolor{textcolor}{rgb}{0.000000,0.000000,0.000000}%
\pgfsetstrokecolor{textcolor}%
\pgfsetfillcolor{textcolor}%
\pgftext[x=0.316361in, y=0.787924in, left, base]{\color{textcolor}{\rmfamily\fontsize{8.000000}{9.600000}\selectfont\catcode`\^=\active\def^{\ifmmode\sp\else\^{}\fi}\catcode`\%=\active\def%{\%}$\mathdefault{10^{0}}$}}%
\end{pgfscope}%
\begin{pgfscope}%
\pgfpathrectangle{\pgfqpoint{0.589510in}{0.417642in}}{\pgfqpoint{1.809765in}{1.371397in}}%
\pgfusepath{clip}%
\pgfsetrectcap%
\pgfsetroundjoin%
\pgfsetlinewidth{0.803000pt}%
\definecolor{currentstroke}{rgb}{0.450000,0.450000,0.450000}%
\pgfsetstrokecolor{currentstroke}%
\pgfsetdash{}{0pt}%
\pgfpathmoveto{\pgfqpoint{0.589510in}{1.031795in}}%
\pgfpathlineto{\pgfqpoint{2.399275in}{1.031795in}}%
\pgfusepath{stroke}%
\end{pgfscope}%
\begin{pgfscope}%
\pgfsetbuttcap%
\pgfsetroundjoin%
\definecolor{currentfill}{rgb}{0.000000,0.000000,0.000000}%
\pgfsetfillcolor{currentfill}%
\pgfsetlinewidth{0.803000pt}%
\definecolor{currentstroke}{rgb}{0.000000,0.000000,0.000000}%
\pgfsetstrokecolor{currentstroke}%
\pgfsetdash{}{0pt}%
\pgfsys@defobject{currentmarker}{\pgfqpoint{-0.048611in}{0.000000in}}{\pgfqpoint{-0.000000in}{0.000000in}}{%
\pgfpathmoveto{\pgfqpoint{-0.000000in}{0.000000in}}%
\pgfpathlineto{\pgfqpoint{-0.048611in}{0.000000in}}%
\pgfusepath{stroke,fill}%
}%
\begin{pgfscope}%
\pgfsys@transformshift{0.589510in}{1.031795in}%
\pgfsys@useobject{currentmarker}{}%
\end{pgfscope}%
\end{pgfscope}%
\begin{pgfscope}%
\definecolor{textcolor}{rgb}{0.000000,0.000000,0.000000}%
\pgfsetstrokecolor{textcolor}%
\pgfsetfillcolor{textcolor}%
\pgftext[x=0.316361in, y=0.992642in, left, base]{\color{textcolor}{\rmfamily\fontsize{8.000000}{9.600000}\selectfont\catcode`\^=\active\def^{\ifmmode\sp\else\^{}\fi}\catcode`\%=\active\def%{\%}$\mathdefault{10^{1}}$}}%
\end{pgfscope}%
\begin{pgfscope}%
\pgfpathrectangle{\pgfqpoint{0.589510in}{0.417642in}}{\pgfqpoint{1.809765in}{1.371397in}}%
\pgfusepath{clip}%
\pgfsetrectcap%
\pgfsetroundjoin%
\pgfsetlinewidth{0.803000pt}%
\definecolor{currentstroke}{rgb}{0.450000,0.450000,0.450000}%
\pgfsetstrokecolor{currentstroke}%
\pgfsetdash{}{0pt}%
\pgfpathmoveto{\pgfqpoint{0.589510in}{1.236512in}}%
\pgfpathlineto{\pgfqpoint{2.399275in}{1.236512in}}%
\pgfusepath{stroke}%
\end{pgfscope}%
\begin{pgfscope}%
\pgfsetbuttcap%
\pgfsetroundjoin%
\definecolor{currentfill}{rgb}{0.000000,0.000000,0.000000}%
\pgfsetfillcolor{currentfill}%
\pgfsetlinewidth{0.803000pt}%
\definecolor{currentstroke}{rgb}{0.000000,0.000000,0.000000}%
\pgfsetstrokecolor{currentstroke}%
\pgfsetdash{}{0pt}%
\pgfsys@defobject{currentmarker}{\pgfqpoint{-0.048611in}{0.000000in}}{\pgfqpoint{-0.000000in}{0.000000in}}{%
\pgfpathmoveto{\pgfqpoint{-0.000000in}{0.000000in}}%
\pgfpathlineto{\pgfqpoint{-0.048611in}{0.000000in}}%
\pgfusepath{stroke,fill}%
}%
\begin{pgfscope}%
\pgfsys@transformshift{0.589510in}{1.236512in}%
\pgfsys@useobject{currentmarker}{}%
\end{pgfscope}%
\end{pgfscope}%
\begin{pgfscope}%
\definecolor{textcolor}{rgb}{0.000000,0.000000,0.000000}%
\pgfsetstrokecolor{textcolor}%
\pgfsetfillcolor{textcolor}%
\pgftext[x=0.316361in, y=1.197359in, left, base]{\color{textcolor}{\rmfamily\fontsize{8.000000}{9.600000}\selectfont\catcode`\^=\active\def^{\ifmmode\sp\else\^{}\fi}\catcode`\%=\active\def%{\%}$\mathdefault{10^{2}}$}}%
\end{pgfscope}%
\begin{pgfscope}%
\pgfpathrectangle{\pgfqpoint{0.589510in}{0.417642in}}{\pgfqpoint{1.809765in}{1.371397in}}%
\pgfusepath{clip}%
\pgfsetrectcap%
\pgfsetroundjoin%
\pgfsetlinewidth{0.803000pt}%
\definecolor{currentstroke}{rgb}{0.450000,0.450000,0.450000}%
\pgfsetstrokecolor{currentstroke}%
\pgfsetdash{}{0pt}%
\pgfpathmoveto{\pgfqpoint{0.589510in}{1.441230in}}%
\pgfpathlineto{\pgfqpoint{2.399275in}{1.441230in}}%
\pgfusepath{stroke}%
\end{pgfscope}%
\begin{pgfscope}%
\pgfsetbuttcap%
\pgfsetroundjoin%
\definecolor{currentfill}{rgb}{0.000000,0.000000,0.000000}%
\pgfsetfillcolor{currentfill}%
\pgfsetlinewidth{0.803000pt}%
\definecolor{currentstroke}{rgb}{0.000000,0.000000,0.000000}%
\pgfsetstrokecolor{currentstroke}%
\pgfsetdash{}{0pt}%
\pgfsys@defobject{currentmarker}{\pgfqpoint{-0.048611in}{0.000000in}}{\pgfqpoint{-0.000000in}{0.000000in}}{%
\pgfpathmoveto{\pgfqpoint{-0.000000in}{0.000000in}}%
\pgfpathlineto{\pgfqpoint{-0.048611in}{0.000000in}}%
\pgfusepath{stroke,fill}%
}%
\begin{pgfscope}%
\pgfsys@transformshift{0.589510in}{1.441230in}%
\pgfsys@useobject{currentmarker}{}%
\end{pgfscope}%
\end{pgfscope}%
\begin{pgfscope}%
\definecolor{textcolor}{rgb}{0.000000,0.000000,0.000000}%
\pgfsetstrokecolor{textcolor}%
\pgfsetfillcolor{textcolor}%
\pgftext[x=0.316361in, y=1.402077in, left, base]{\color{textcolor}{\rmfamily\fontsize{8.000000}{9.600000}\selectfont\catcode`\^=\active\def^{\ifmmode\sp\else\^{}\fi}\catcode`\%=\active\def%{\%}$\mathdefault{10^{3}}$}}%
\end{pgfscope}%
\begin{pgfscope}%
\pgfpathrectangle{\pgfqpoint{0.589510in}{0.417642in}}{\pgfqpoint{1.809765in}{1.371397in}}%
\pgfusepath{clip}%
\pgfsetrectcap%
\pgfsetroundjoin%
\pgfsetlinewidth{0.803000pt}%
\definecolor{currentstroke}{rgb}{0.450000,0.450000,0.450000}%
\pgfsetstrokecolor{currentstroke}%
\pgfsetdash{}{0pt}%
\pgfpathmoveto{\pgfqpoint{0.589510in}{1.645947in}}%
\pgfpathlineto{\pgfqpoint{2.399275in}{1.645947in}}%
\pgfusepath{stroke}%
\end{pgfscope}%
\begin{pgfscope}%
\pgfsetbuttcap%
\pgfsetroundjoin%
\definecolor{currentfill}{rgb}{0.000000,0.000000,0.000000}%
\pgfsetfillcolor{currentfill}%
\pgfsetlinewidth{0.803000pt}%
\definecolor{currentstroke}{rgb}{0.000000,0.000000,0.000000}%
\pgfsetstrokecolor{currentstroke}%
\pgfsetdash{}{0pt}%
\pgfsys@defobject{currentmarker}{\pgfqpoint{-0.048611in}{0.000000in}}{\pgfqpoint{-0.000000in}{0.000000in}}{%
\pgfpathmoveto{\pgfqpoint{-0.000000in}{0.000000in}}%
\pgfpathlineto{\pgfqpoint{-0.048611in}{0.000000in}}%
\pgfusepath{stroke,fill}%
}%
\begin{pgfscope}%
\pgfsys@transformshift{0.589510in}{1.645947in}%
\pgfsys@useobject{currentmarker}{}%
\end{pgfscope}%
\end{pgfscope}%
\begin{pgfscope}%
\definecolor{textcolor}{rgb}{0.000000,0.000000,0.000000}%
\pgfsetstrokecolor{textcolor}%
\pgfsetfillcolor{textcolor}%
\pgftext[x=0.316361in, y=1.606795in, left, base]{\color{textcolor}{\rmfamily\fontsize{8.000000}{9.600000}\selectfont\catcode`\^=\active\def^{\ifmmode\sp\else\^{}\fi}\catcode`\%=\active\def%{\%}$\mathdefault{10^{4}}$}}%
\end{pgfscope}%
\begin{pgfscope}%
\pgfsetbuttcap%
\pgfsetroundjoin%
\definecolor{currentfill}{rgb}{0.000000,0.000000,0.000000}%
\pgfsetfillcolor{currentfill}%
\pgfsetlinewidth{0.602250pt}%
\definecolor{currentstroke}{rgb}{0.000000,0.000000,0.000000}%
\pgfsetstrokecolor{currentstroke}%
\pgfsetdash{}{0pt}%
\pgfsys@defobject{currentmarker}{\pgfqpoint{-0.027778in}{0.000000in}}{\pgfqpoint{-0.000000in}{0.000000in}}{%
\pgfpathmoveto{\pgfqpoint{-0.000000in}{0.000000in}}%
\pgfpathlineto{\pgfqpoint{-0.027778in}{0.000000in}}%
\pgfusepath{stroke,fill}%
}%
\begin{pgfscope}%
\pgfsys@transformshift{0.589510in}{0.479268in}%
\pgfsys@useobject{currentmarker}{}%
\end{pgfscope}%
\end{pgfscope}%
\begin{pgfscope}%
\pgfsetbuttcap%
\pgfsetroundjoin%
\definecolor{currentfill}{rgb}{0.000000,0.000000,0.000000}%
\pgfsetfillcolor{currentfill}%
\pgfsetlinewidth{0.602250pt}%
\definecolor{currentstroke}{rgb}{0.000000,0.000000,0.000000}%
\pgfsetstrokecolor{currentstroke}%
\pgfsetdash{}{0pt}%
\pgfsys@defobject{currentmarker}{\pgfqpoint{-0.027778in}{0.000000in}}{\pgfqpoint{-0.000000in}{0.000000in}}{%
\pgfpathmoveto{\pgfqpoint{-0.000000in}{0.000000in}}%
\pgfpathlineto{\pgfqpoint{-0.027778in}{0.000000in}}%
\pgfusepath{stroke,fill}%
}%
\begin{pgfscope}%
\pgfsys@transformshift{0.589510in}{0.515317in}%
\pgfsys@useobject{currentmarker}{}%
\end{pgfscope}%
\end{pgfscope}%
\begin{pgfscope}%
\pgfsetbuttcap%
\pgfsetroundjoin%
\definecolor{currentfill}{rgb}{0.000000,0.000000,0.000000}%
\pgfsetfillcolor{currentfill}%
\pgfsetlinewidth{0.602250pt}%
\definecolor{currentstroke}{rgb}{0.000000,0.000000,0.000000}%
\pgfsetstrokecolor{currentstroke}%
\pgfsetdash{}{0pt}%
\pgfsys@defobject{currentmarker}{\pgfqpoint{-0.027778in}{0.000000in}}{\pgfqpoint{-0.000000in}{0.000000in}}{%
\pgfpathmoveto{\pgfqpoint{-0.000000in}{0.000000in}}%
\pgfpathlineto{\pgfqpoint{-0.027778in}{0.000000in}}%
\pgfusepath{stroke,fill}%
}%
\begin{pgfscope}%
\pgfsys@transformshift{0.589510in}{0.540894in}%
\pgfsys@useobject{currentmarker}{}%
\end{pgfscope}%
\end{pgfscope}%
\begin{pgfscope}%
\pgfsetbuttcap%
\pgfsetroundjoin%
\definecolor{currentfill}{rgb}{0.000000,0.000000,0.000000}%
\pgfsetfillcolor{currentfill}%
\pgfsetlinewidth{0.602250pt}%
\definecolor{currentstroke}{rgb}{0.000000,0.000000,0.000000}%
\pgfsetstrokecolor{currentstroke}%
\pgfsetdash{}{0pt}%
\pgfsys@defobject{currentmarker}{\pgfqpoint{-0.027778in}{0.000000in}}{\pgfqpoint{-0.000000in}{0.000000in}}{%
\pgfpathmoveto{\pgfqpoint{-0.000000in}{0.000000in}}%
\pgfpathlineto{\pgfqpoint{-0.027778in}{0.000000in}}%
\pgfusepath{stroke,fill}%
}%
\begin{pgfscope}%
\pgfsys@transformshift{0.589510in}{0.560733in}%
\pgfsys@useobject{currentmarker}{}%
\end{pgfscope}%
\end{pgfscope}%
\begin{pgfscope}%
\pgfsetbuttcap%
\pgfsetroundjoin%
\definecolor{currentfill}{rgb}{0.000000,0.000000,0.000000}%
\pgfsetfillcolor{currentfill}%
\pgfsetlinewidth{0.602250pt}%
\definecolor{currentstroke}{rgb}{0.000000,0.000000,0.000000}%
\pgfsetstrokecolor{currentstroke}%
\pgfsetdash{}{0pt}%
\pgfsys@defobject{currentmarker}{\pgfqpoint{-0.027778in}{0.000000in}}{\pgfqpoint{-0.000000in}{0.000000in}}{%
\pgfpathmoveto{\pgfqpoint{-0.000000in}{0.000000in}}%
\pgfpathlineto{\pgfqpoint{-0.027778in}{0.000000in}}%
\pgfusepath{stroke,fill}%
}%
\begin{pgfscope}%
\pgfsys@transformshift{0.589510in}{0.576943in}%
\pgfsys@useobject{currentmarker}{}%
\end{pgfscope}%
\end{pgfscope}%
\begin{pgfscope}%
\pgfsetbuttcap%
\pgfsetroundjoin%
\definecolor{currentfill}{rgb}{0.000000,0.000000,0.000000}%
\pgfsetfillcolor{currentfill}%
\pgfsetlinewidth{0.602250pt}%
\definecolor{currentstroke}{rgb}{0.000000,0.000000,0.000000}%
\pgfsetstrokecolor{currentstroke}%
\pgfsetdash{}{0pt}%
\pgfsys@defobject{currentmarker}{\pgfqpoint{-0.027778in}{0.000000in}}{\pgfqpoint{-0.000000in}{0.000000in}}{%
\pgfpathmoveto{\pgfqpoint{-0.000000in}{0.000000in}}%
\pgfpathlineto{\pgfqpoint{-0.027778in}{0.000000in}}%
\pgfusepath{stroke,fill}%
}%
\begin{pgfscope}%
\pgfsys@transformshift{0.589510in}{0.590648in}%
\pgfsys@useobject{currentmarker}{}%
\end{pgfscope}%
\end{pgfscope}%
\begin{pgfscope}%
\pgfsetbuttcap%
\pgfsetroundjoin%
\definecolor{currentfill}{rgb}{0.000000,0.000000,0.000000}%
\pgfsetfillcolor{currentfill}%
\pgfsetlinewidth{0.602250pt}%
\definecolor{currentstroke}{rgb}{0.000000,0.000000,0.000000}%
\pgfsetstrokecolor{currentstroke}%
\pgfsetdash{}{0pt}%
\pgfsys@defobject{currentmarker}{\pgfqpoint{-0.027778in}{0.000000in}}{\pgfqpoint{-0.000000in}{0.000000in}}{%
\pgfpathmoveto{\pgfqpoint{-0.000000in}{0.000000in}}%
\pgfpathlineto{\pgfqpoint{-0.027778in}{0.000000in}}%
\pgfusepath{stroke,fill}%
}%
\begin{pgfscope}%
\pgfsys@transformshift{0.589510in}{0.602520in}%
\pgfsys@useobject{currentmarker}{}%
\end{pgfscope}%
\end{pgfscope}%
\begin{pgfscope}%
\pgfsetbuttcap%
\pgfsetroundjoin%
\definecolor{currentfill}{rgb}{0.000000,0.000000,0.000000}%
\pgfsetfillcolor{currentfill}%
\pgfsetlinewidth{0.602250pt}%
\definecolor{currentstroke}{rgb}{0.000000,0.000000,0.000000}%
\pgfsetstrokecolor{currentstroke}%
\pgfsetdash{}{0pt}%
\pgfsys@defobject{currentmarker}{\pgfqpoint{-0.027778in}{0.000000in}}{\pgfqpoint{-0.000000in}{0.000000in}}{%
\pgfpathmoveto{\pgfqpoint{-0.000000in}{0.000000in}}%
\pgfpathlineto{\pgfqpoint{-0.027778in}{0.000000in}}%
\pgfusepath{stroke,fill}%
}%
\begin{pgfscope}%
\pgfsys@transformshift{0.589510in}{0.612992in}%
\pgfsys@useobject{currentmarker}{}%
\end{pgfscope}%
\end{pgfscope}%
\begin{pgfscope}%
\pgfsetbuttcap%
\pgfsetroundjoin%
\definecolor{currentfill}{rgb}{0.000000,0.000000,0.000000}%
\pgfsetfillcolor{currentfill}%
\pgfsetlinewidth{0.602250pt}%
\definecolor{currentstroke}{rgb}{0.000000,0.000000,0.000000}%
\pgfsetstrokecolor{currentstroke}%
\pgfsetdash{}{0pt}%
\pgfsys@defobject{currentmarker}{\pgfqpoint{-0.027778in}{0.000000in}}{\pgfqpoint{-0.000000in}{0.000000in}}{%
\pgfpathmoveto{\pgfqpoint{-0.000000in}{0.000000in}}%
\pgfpathlineto{\pgfqpoint{-0.027778in}{0.000000in}}%
\pgfusepath{stroke,fill}%
}%
\begin{pgfscope}%
\pgfsys@transformshift{0.589510in}{0.683986in}%
\pgfsys@useobject{currentmarker}{}%
\end{pgfscope}%
\end{pgfscope}%
\begin{pgfscope}%
\pgfsetbuttcap%
\pgfsetroundjoin%
\definecolor{currentfill}{rgb}{0.000000,0.000000,0.000000}%
\pgfsetfillcolor{currentfill}%
\pgfsetlinewidth{0.602250pt}%
\definecolor{currentstroke}{rgb}{0.000000,0.000000,0.000000}%
\pgfsetstrokecolor{currentstroke}%
\pgfsetdash{}{0pt}%
\pgfsys@defobject{currentmarker}{\pgfqpoint{-0.027778in}{0.000000in}}{\pgfqpoint{-0.000000in}{0.000000in}}{%
\pgfpathmoveto{\pgfqpoint{-0.000000in}{0.000000in}}%
\pgfpathlineto{\pgfqpoint{-0.027778in}{0.000000in}}%
\pgfusepath{stroke,fill}%
}%
\begin{pgfscope}%
\pgfsys@transformshift{0.589510in}{0.720035in}%
\pgfsys@useobject{currentmarker}{}%
\end{pgfscope}%
\end{pgfscope}%
\begin{pgfscope}%
\pgfsetbuttcap%
\pgfsetroundjoin%
\definecolor{currentfill}{rgb}{0.000000,0.000000,0.000000}%
\pgfsetfillcolor{currentfill}%
\pgfsetlinewidth{0.602250pt}%
\definecolor{currentstroke}{rgb}{0.000000,0.000000,0.000000}%
\pgfsetstrokecolor{currentstroke}%
\pgfsetdash{}{0pt}%
\pgfsys@defobject{currentmarker}{\pgfqpoint{-0.027778in}{0.000000in}}{\pgfqpoint{-0.000000in}{0.000000in}}{%
\pgfpathmoveto{\pgfqpoint{-0.000000in}{0.000000in}}%
\pgfpathlineto{\pgfqpoint{-0.027778in}{0.000000in}}%
\pgfusepath{stroke,fill}%
}%
\begin{pgfscope}%
\pgfsys@transformshift{0.589510in}{0.745612in}%
\pgfsys@useobject{currentmarker}{}%
\end{pgfscope}%
\end{pgfscope}%
\begin{pgfscope}%
\pgfsetbuttcap%
\pgfsetroundjoin%
\definecolor{currentfill}{rgb}{0.000000,0.000000,0.000000}%
\pgfsetfillcolor{currentfill}%
\pgfsetlinewidth{0.602250pt}%
\definecolor{currentstroke}{rgb}{0.000000,0.000000,0.000000}%
\pgfsetstrokecolor{currentstroke}%
\pgfsetdash{}{0pt}%
\pgfsys@defobject{currentmarker}{\pgfqpoint{-0.027778in}{0.000000in}}{\pgfqpoint{-0.000000in}{0.000000in}}{%
\pgfpathmoveto{\pgfqpoint{-0.000000in}{0.000000in}}%
\pgfpathlineto{\pgfqpoint{-0.027778in}{0.000000in}}%
\pgfusepath{stroke,fill}%
}%
\begin{pgfscope}%
\pgfsys@transformshift{0.589510in}{0.765451in}%
\pgfsys@useobject{currentmarker}{}%
\end{pgfscope}%
\end{pgfscope}%
\begin{pgfscope}%
\pgfsetbuttcap%
\pgfsetroundjoin%
\definecolor{currentfill}{rgb}{0.000000,0.000000,0.000000}%
\pgfsetfillcolor{currentfill}%
\pgfsetlinewidth{0.602250pt}%
\definecolor{currentstroke}{rgb}{0.000000,0.000000,0.000000}%
\pgfsetstrokecolor{currentstroke}%
\pgfsetdash{}{0pt}%
\pgfsys@defobject{currentmarker}{\pgfqpoint{-0.027778in}{0.000000in}}{\pgfqpoint{-0.000000in}{0.000000in}}{%
\pgfpathmoveto{\pgfqpoint{-0.000000in}{0.000000in}}%
\pgfpathlineto{\pgfqpoint{-0.027778in}{0.000000in}}%
\pgfusepath{stroke,fill}%
}%
\begin{pgfscope}%
\pgfsys@transformshift{0.589510in}{0.781661in}%
\pgfsys@useobject{currentmarker}{}%
\end{pgfscope}%
\end{pgfscope}%
\begin{pgfscope}%
\pgfsetbuttcap%
\pgfsetroundjoin%
\definecolor{currentfill}{rgb}{0.000000,0.000000,0.000000}%
\pgfsetfillcolor{currentfill}%
\pgfsetlinewidth{0.602250pt}%
\definecolor{currentstroke}{rgb}{0.000000,0.000000,0.000000}%
\pgfsetstrokecolor{currentstroke}%
\pgfsetdash{}{0pt}%
\pgfsys@defobject{currentmarker}{\pgfqpoint{-0.027778in}{0.000000in}}{\pgfqpoint{-0.000000in}{0.000000in}}{%
\pgfpathmoveto{\pgfqpoint{-0.000000in}{0.000000in}}%
\pgfpathlineto{\pgfqpoint{-0.027778in}{0.000000in}}%
\pgfusepath{stroke,fill}%
}%
\begin{pgfscope}%
\pgfsys@transformshift{0.589510in}{0.795366in}%
\pgfsys@useobject{currentmarker}{}%
\end{pgfscope}%
\end{pgfscope}%
\begin{pgfscope}%
\pgfsetbuttcap%
\pgfsetroundjoin%
\definecolor{currentfill}{rgb}{0.000000,0.000000,0.000000}%
\pgfsetfillcolor{currentfill}%
\pgfsetlinewidth{0.602250pt}%
\definecolor{currentstroke}{rgb}{0.000000,0.000000,0.000000}%
\pgfsetstrokecolor{currentstroke}%
\pgfsetdash{}{0pt}%
\pgfsys@defobject{currentmarker}{\pgfqpoint{-0.027778in}{0.000000in}}{\pgfqpoint{-0.000000in}{0.000000in}}{%
\pgfpathmoveto{\pgfqpoint{-0.000000in}{0.000000in}}%
\pgfpathlineto{\pgfqpoint{-0.027778in}{0.000000in}}%
\pgfusepath{stroke,fill}%
}%
\begin{pgfscope}%
\pgfsys@transformshift{0.589510in}{0.807238in}%
\pgfsys@useobject{currentmarker}{}%
\end{pgfscope}%
\end{pgfscope}%
\begin{pgfscope}%
\pgfsetbuttcap%
\pgfsetroundjoin%
\definecolor{currentfill}{rgb}{0.000000,0.000000,0.000000}%
\pgfsetfillcolor{currentfill}%
\pgfsetlinewidth{0.602250pt}%
\definecolor{currentstroke}{rgb}{0.000000,0.000000,0.000000}%
\pgfsetstrokecolor{currentstroke}%
\pgfsetdash{}{0pt}%
\pgfsys@defobject{currentmarker}{\pgfqpoint{-0.027778in}{0.000000in}}{\pgfqpoint{-0.000000in}{0.000000in}}{%
\pgfpathmoveto{\pgfqpoint{-0.000000in}{0.000000in}}%
\pgfpathlineto{\pgfqpoint{-0.027778in}{0.000000in}}%
\pgfusepath{stroke,fill}%
}%
\begin{pgfscope}%
\pgfsys@transformshift{0.589510in}{0.817710in}%
\pgfsys@useobject{currentmarker}{}%
\end{pgfscope}%
\end{pgfscope}%
\begin{pgfscope}%
\pgfsetbuttcap%
\pgfsetroundjoin%
\definecolor{currentfill}{rgb}{0.000000,0.000000,0.000000}%
\pgfsetfillcolor{currentfill}%
\pgfsetlinewidth{0.602250pt}%
\definecolor{currentstroke}{rgb}{0.000000,0.000000,0.000000}%
\pgfsetstrokecolor{currentstroke}%
\pgfsetdash{}{0pt}%
\pgfsys@defobject{currentmarker}{\pgfqpoint{-0.027778in}{0.000000in}}{\pgfqpoint{-0.000000in}{0.000000in}}{%
\pgfpathmoveto{\pgfqpoint{-0.000000in}{0.000000in}}%
\pgfpathlineto{\pgfqpoint{-0.027778in}{0.000000in}}%
\pgfusepath{stroke,fill}%
}%
\begin{pgfscope}%
\pgfsys@transformshift{0.589510in}{0.888703in}%
\pgfsys@useobject{currentmarker}{}%
\end{pgfscope}%
\end{pgfscope}%
\begin{pgfscope}%
\pgfsetbuttcap%
\pgfsetroundjoin%
\definecolor{currentfill}{rgb}{0.000000,0.000000,0.000000}%
\pgfsetfillcolor{currentfill}%
\pgfsetlinewidth{0.602250pt}%
\definecolor{currentstroke}{rgb}{0.000000,0.000000,0.000000}%
\pgfsetstrokecolor{currentstroke}%
\pgfsetdash{}{0pt}%
\pgfsys@defobject{currentmarker}{\pgfqpoint{-0.027778in}{0.000000in}}{\pgfqpoint{-0.000000in}{0.000000in}}{%
\pgfpathmoveto{\pgfqpoint{-0.000000in}{0.000000in}}%
\pgfpathlineto{\pgfqpoint{-0.027778in}{0.000000in}}%
\pgfusepath{stroke,fill}%
}%
\begin{pgfscope}%
\pgfsys@transformshift{0.589510in}{0.924752in}%
\pgfsys@useobject{currentmarker}{}%
\end{pgfscope}%
\end{pgfscope}%
\begin{pgfscope}%
\pgfsetbuttcap%
\pgfsetroundjoin%
\definecolor{currentfill}{rgb}{0.000000,0.000000,0.000000}%
\pgfsetfillcolor{currentfill}%
\pgfsetlinewidth{0.602250pt}%
\definecolor{currentstroke}{rgb}{0.000000,0.000000,0.000000}%
\pgfsetstrokecolor{currentstroke}%
\pgfsetdash{}{0pt}%
\pgfsys@defobject{currentmarker}{\pgfqpoint{-0.027778in}{0.000000in}}{\pgfqpoint{-0.000000in}{0.000000in}}{%
\pgfpathmoveto{\pgfqpoint{-0.000000in}{0.000000in}}%
\pgfpathlineto{\pgfqpoint{-0.027778in}{0.000000in}}%
\pgfusepath{stroke,fill}%
}%
\begin{pgfscope}%
\pgfsys@transformshift{0.589510in}{0.950329in}%
\pgfsys@useobject{currentmarker}{}%
\end{pgfscope}%
\end{pgfscope}%
\begin{pgfscope}%
\pgfsetbuttcap%
\pgfsetroundjoin%
\definecolor{currentfill}{rgb}{0.000000,0.000000,0.000000}%
\pgfsetfillcolor{currentfill}%
\pgfsetlinewidth{0.602250pt}%
\definecolor{currentstroke}{rgb}{0.000000,0.000000,0.000000}%
\pgfsetstrokecolor{currentstroke}%
\pgfsetdash{}{0pt}%
\pgfsys@defobject{currentmarker}{\pgfqpoint{-0.027778in}{0.000000in}}{\pgfqpoint{-0.000000in}{0.000000in}}{%
\pgfpathmoveto{\pgfqpoint{-0.000000in}{0.000000in}}%
\pgfpathlineto{\pgfqpoint{-0.027778in}{0.000000in}}%
\pgfusepath{stroke,fill}%
}%
\begin{pgfscope}%
\pgfsys@transformshift{0.589510in}{0.970168in}%
\pgfsys@useobject{currentmarker}{}%
\end{pgfscope}%
\end{pgfscope}%
\begin{pgfscope}%
\pgfsetbuttcap%
\pgfsetroundjoin%
\definecolor{currentfill}{rgb}{0.000000,0.000000,0.000000}%
\pgfsetfillcolor{currentfill}%
\pgfsetlinewidth{0.602250pt}%
\definecolor{currentstroke}{rgb}{0.000000,0.000000,0.000000}%
\pgfsetstrokecolor{currentstroke}%
\pgfsetdash{}{0pt}%
\pgfsys@defobject{currentmarker}{\pgfqpoint{-0.027778in}{0.000000in}}{\pgfqpoint{-0.000000in}{0.000000in}}{%
\pgfpathmoveto{\pgfqpoint{-0.000000in}{0.000000in}}%
\pgfpathlineto{\pgfqpoint{-0.027778in}{0.000000in}}%
\pgfusepath{stroke,fill}%
}%
\begin{pgfscope}%
\pgfsys@transformshift{0.589510in}{0.986378in}%
\pgfsys@useobject{currentmarker}{}%
\end{pgfscope}%
\end{pgfscope}%
\begin{pgfscope}%
\pgfsetbuttcap%
\pgfsetroundjoin%
\definecolor{currentfill}{rgb}{0.000000,0.000000,0.000000}%
\pgfsetfillcolor{currentfill}%
\pgfsetlinewidth{0.602250pt}%
\definecolor{currentstroke}{rgb}{0.000000,0.000000,0.000000}%
\pgfsetstrokecolor{currentstroke}%
\pgfsetdash{}{0pt}%
\pgfsys@defobject{currentmarker}{\pgfqpoint{-0.027778in}{0.000000in}}{\pgfqpoint{-0.000000in}{0.000000in}}{%
\pgfpathmoveto{\pgfqpoint{-0.000000in}{0.000000in}}%
\pgfpathlineto{\pgfqpoint{-0.027778in}{0.000000in}}%
\pgfusepath{stroke,fill}%
}%
\begin{pgfscope}%
\pgfsys@transformshift{0.589510in}{1.000083in}%
\pgfsys@useobject{currentmarker}{}%
\end{pgfscope}%
\end{pgfscope}%
\begin{pgfscope}%
\pgfsetbuttcap%
\pgfsetroundjoin%
\definecolor{currentfill}{rgb}{0.000000,0.000000,0.000000}%
\pgfsetfillcolor{currentfill}%
\pgfsetlinewidth{0.602250pt}%
\definecolor{currentstroke}{rgb}{0.000000,0.000000,0.000000}%
\pgfsetstrokecolor{currentstroke}%
\pgfsetdash{}{0pt}%
\pgfsys@defobject{currentmarker}{\pgfqpoint{-0.027778in}{0.000000in}}{\pgfqpoint{-0.000000in}{0.000000in}}{%
\pgfpathmoveto{\pgfqpoint{-0.000000in}{0.000000in}}%
\pgfpathlineto{\pgfqpoint{-0.027778in}{0.000000in}}%
\pgfusepath{stroke,fill}%
}%
\begin{pgfscope}%
\pgfsys@transformshift{0.589510in}{1.011955in}%
\pgfsys@useobject{currentmarker}{}%
\end{pgfscope}%
\end{pgfscope}%
\begin{pgfscope}%
\pgfsetbuttcap%
\pgfsetroundjoin%
\definecolor{currentfill}{rgb}{0.000000,0.000000,0.000000}%
\pgfsetfillcolor{currentfill}%
\pgfsetlinewidth{0.602250pt}%
\definecolor{currentstroke}{rgb}{0.000000,0.000000,0.000000}%
\pgfsetstrokecolor{currentstroke}%
\pgfsetdash{}{0pt}%
\pgfsys@defobject{currentmarker}{\pgfqpoint{-0.027778in}{0.000000in}}{\pgfqpoint{-0.000000in}{0.000000in}}{%
\pgfpathmoveto{\pgfqpoint{-0.000000in}{0.000000in}}%
\pgfpathlineto{\pgfqpoint{-0.027778in}{0.000000in}}%
\pgfusepath{stroke,fill}%
}%
\begin{pgfscope}%
\pgfsys@transformshift{0.589510in}{1.022427in}%
\pgfsys@useobject{currentmarker}{}%
\end{pgfscope}%
\end{pgfscope}%
\begin{pgfscope}%
\pgfsetbuttcap%
\pgfsetroundjoin%
\definecolor{currentfill}{rgb}{0.000000,0.000000,0.000000}%
\pgfsetfillcolor{currentfill}%
\pgfsetlinewidth{0.602250pt}%
\definecolor{currentstroke}{rgb}{0.000000,0.000000,0.000000}%
\pgfsetstrokecolor{currentstroke}%
\pgfsetdash{}{0pt}%
\pgfsys@defobject{currentmarker}{\pgfqpoint{-0.027778in}{0.000000in}}{\pgfqpoint{-0.000000in}{0.000000in}}{%
\pgfpathmoveto{\pgfqpoint{-0.000000in}{0.000000in}}%
\pgfpathlineto{\pgfqpoint{-0.027778in}{0.000000in}}%
\pgfusepath{stroke,fill}%
}%
\begin{pgfscope}%
\pgfsys@transformshift{0.589510in}{1.093421in}%
\pgfsys@useobject{currentmarker}{}%
\end{pgfscope}%
\end{pgfscope}%
\begin{pgfscope}%
\pgfsetbuttcap%
\pgfsetroundjoin%
\definecolor{currentfill}{rgb}{0.000000,0.000000,0.000000}%
\pgfsetfillcolor{currentfill}%
\pgfsetlinewidth{0.602250pt}%
\definecolor{currentstroke}{rgb}{0.000000,0.000000,0.000000}%
\pgfsetstrokecolor{currentstroke}%
\pgfsetdash{}{0pt}%
\pgfsys@defobject{currentmarker}{\pgfqpoint{-0.027778in}{0.000000in}}{\pgfqpoint{-0.000000in}{0.000000in}}{%
\pgfpathmoveto{\pgfqpoint{-0.000000in}{0.000000in}}%
\pgfpathlineto{\pgfqpoint{-0.027778in}{0.000000in}}%
\pgfusepath{stroke,fill}%
}%
\begin{pgfscope}%
\pgfsys@transformshift{0.589510in}{1.129470in}%
\pgfsys@useobject{currentmarker}{}%
\end{pgfscope}%
\end{pgfscope}%
\begin{pgfscope}%
\pgfsetbuttcap%
\pgfsetroundjoin%
\definecolor{currentfill}{rgb}{0.000000,0.000000,0.000000}%
\pgfsetfillcolor{currentfill}%
\pgfsetlinewidth{0.602250pt}%
\definecolor{currentstroke}{rgb}{0.000000,0.000000,0.000000}%
\pgfsetstrokecolor{currentstroke}%
\pgfsetdash{}{0pt}%
\pgfsys@defobject{currentmarker}{\pgfqpoint{-0.027778in}{0.000000in}}{\pgfqpoint{-0.000000in}{0.000000in}}{%
\pgfpathmoveto{\pgfqpoint{-0.000000in}{0.000000in}}%
\pgfpathlineto{\pgfqpoint{-0.027778in}{0.000000in}}%
\pgfusepath{stroke,fill}%
}%
\begin{pgfscope}%
\pgfsys@transformshift{0.589510in}{1.155047in}%
\pgfsys@useobject{currentmarker}{}%
\end{pgfscope}%
\end{pgfscope}%
\begin{pgfscope}%
\pgfsetbuttcap%
\pgfsetroundjoin%
\definecolor{currentfill}{rgb}{0.000000,0.000000,0.000000}%
\pgfsetfillcolor{currentfill}%
\pgfsetlinewidth{0.602250pt}%
\definecolor{currentstroke}{rgb}{0.000000,0.000000,0.000000}%
\pgfsetstrokecolor{currentstroke}%
\pgfsetdash{}{0pt}%
\pgfsys@defobject{currentmarker}{\pgfqpoint{-0.027778in}{0.000000in}}{\pgfqpoint{-0.000000in}{0.000000in}}{%
\pgfpathmoveto{\pgfqpoint{-0.000000in}{0.000000in}}%
\pgfpathlineto{\pgfqpoint{-0.027778in}{0.000000in}}%
\pgfusepath{stroke,fill}%
}%
\begin{pgfscope}%
\pgfsys@transformshift{0.589510in}{1.174886in}%
\pgfsys@useobject{currentmarker}{}%
\end{pgfscope}%
\end{pgfscope}%
\begin{pgfscope}%
\pgfsetbuttcap%
\pgfsetroundjoin%
\definecolor{currentfill}{rgb}{0.000000,0.000000,0.000000}%
\pgfsetfillcolor{currentfill}%
\pgfsetlinewidth{0.602250pt}%
\definecolor{currentstroke}{rgb}{0.000000,0.000000,0.000000}%
\pgfsetstrokecolor{currentstroke}%
\pgfsetdash{}{0pt}%
\pgfsys@defobject{currentmarker}{\pgfqpoint{-0.027778in}{0.000000in}}{\pgfqpoint{-0.000000in}{0.000000in}}{%
\pgfpathmoveto{\pgfqpoint{-0.000000in}{0.000000in}}%
\pgfpathlineto{\pgfqpoint{-0.027778in}{0.000000in}}%
\pgfusepath{stroke,fill}%
}%
\begin{pgfscope}%
\pgfsys@transformshift{0.589510in}{1.191096in}%
\pgfsys@useobject{currentmarker}{}%
\end{pgfscope}%
\end{pgfscope}%
\begin{pgfscope}%
\pgfsetbuttcap%
\pgfsetroundjoin%
\definecolor{currentfill}{rgb}{0.000000,0.000000,0.000000}%
\pgfsetfillcolor{currentfill}%
\pgfsetlinewidth{0.602250pt}%
\definecolor{currentstroke}{rgb}{0.000000,0.000000,0.000000}%
\pgfsetstrokecolor{currentstroke}%
\pgfsetdash{}{0pt}%
\pgfsys@defobject{currentmarker}{\pgfqpoint{-0.027778in}{0.000000in}}{\pgfqpoint{-0.000000in}{0.000000in}}{%
\pgfpathmoveto{\pgfqpoint{-0.000000in}{0.000000in}}%
\pgfpathlineto{\pgfqpoint{-0.027778in}{0.000000in}}%
\pgfusepath{stroke,fill}%
}%
\begin{pgfscope}%
\pgfsys@transformshift{0.589510in}{1.204801in}%
\pgfsys@useobject{currentmarker}{}%
\end{pgfscope}%
\end{pgfscope}%
\begin{pgfscope}%
\pgfsetbuttcap%
\pgfsetroundjoin%
\definecolor{currentfill}{rgb}{0.000000,0.000000,0.000000}%
\pgfsetfillcolor{currentfill}%
\pgfsetlinewidth{0.602250pt}%
\definecolor{currentstroke}{rgb}{0.000000,0.000000,0.000000}%
\pgfsetstrokecolor{currentstroke}%
\pgfsetdash{}{0pt}%
\pgfsys@defobject{currentmarker}{\pgfqpoint{-0.027778in}{0.000000in}}{\pgfqpoint{-0.000000in}{0.000000in}}{%
\pgfpathmoveto{\pgfqpoint{-0.000000in}{0.000000in}}%
\pgfpathlineto{\pgfqpoint{-0.027778in}{0.000000in}}%
\pgfusepath{stroke,fill}%
}%
\begin{pgfscope}%
\pgfsys@transformshift{0.589510in}{1.216673in}%
\pgfsys@useobject{currentmarker}{}%
\end{pgfscope}%
\end{pgfscope}%
\begin{pgfscope}%
\pgfsetbuttcap%
\pgfsetroundjoin%
\definecolor{currentfill}{rgb}{0.000000,0.000000,0.000000}%
\pgfsetfillcolor{currentfill}%
\pgfsetlinewidth{0.602250pt}%
\definecolor{currentstroke}{rgb}{0.000000,0.000000,0.000000}%
\pgfsetstrokecolor{currentstroke}%
\pgfsetdash{}{0pt}%
\pgfsys@defobject{currentmarker}{\pgfqpoint{-0.027778in}{0.000000in}}{\pgfqpoint{-0.000000in}{0.000000in}}{%
\pgfpathmoveto{\pgfqpoint{-0.000000in}{0.000000in}}%
\pgfpathlineto{\pgfqpoint{-0.027778in}{0.000000in}}%
\pgfusepath{stroke,fill}%
}%
\begin{pgfscope}%
\pgfsys@transformshift{0.589510in}{1.227145in}%
\pgfsys@useobject{currentmarker}{}%
\end{pgfscope}%
\end{pgfscope}%
\begin{pgfscope}%
\pgfsetbuttcap%
\pgfsetroundjoin%
\definecolor{currentfill}{rgb}{0.000000,0.000000,0.000000}%
\pgfsetfillcolor{currentfill}%
\pgfsetlinewidth{0.602250pt}%
\definecolor{currentstroke}{rgb}{0.000000,0.000000,0.000000}%
\pgfsetstrokecolor{currentstroke}%
\pgfsetdash{}{0pt}%
\pgfsys@defobject{currentmarker}{\pgfqpoint{-0.027778in}{0.000000in}}{\pgfqpoint{-0.000000in}{0.000000in}}{%
\pgfpathmoveto{\pgfqpoint{-0.000000in}{0.000000in}}%
\pgfpathlineto{\pgfqpoint{-0.027778in}{0.000000in}}%
\pgfusepath{stroke,fill}%
}%
\begin{pgfscope}%
\pgfsys@transformshift{0.589510in}{1.298138in}%
\pgfsys@useobject{currentmarker}{}%
\end{pgfscope}%
\end{pgfscope}%
\begin{pgfscope}%
\pgfsetbuttcap%
\pgfsetroundjoin%
\definecolor{currentfill}{rgb}{0.000000,0.000000,0.000000}%
\pgfsetfillcolor{currentfill}%
\pgfsetlinewidth{0.602250pt}%
\definecolor{currentstroke}{rgb}{0.000000,0.000000,0.000000}%
\pgfsetstrokecolor{currentstroke}%
\pgfsetdash{}{0pt}%
\pgfsys@defobject{currentmarker}{\pgfqpoint{-0.027778in}{0.000000in}}{\pgfqpoint{-0.000000in}{0.000000in}}{%
\pgfpathmoveto{\pgfqpoint{-0.000000in}{0.000000in}}%
\pgfpathlineto{\pgfqpoint{-0.027778in}{0.000000in}}%
\pgfusepath{stroke,fill}%
}%
\begin{pgfscope}%
\pgfsys@transformshift{0.589510in}{1.334187in}%
\pgfsys@useobject{currentmarker}{}%
\end{pgfscope}%
\end{pgfscope}%
\begin{pgfscope}%
\pgfsetbuttcap%
\pgfsetroundjoin%
\definecolor{currentfill}{rgb}{0.000000,0.000000,0.000000}%
\pgfsetfillcolor{currentfill}%
\pgfsetlinewidth{0.602250pt}%
\definecolor{currentstroke}{rgb}{0.000000,0.000000,0.000000}%
\pgfsetstrokecolor{currentstroke}%
\pgfsetdash{}{0pt}%
\pgfsys@defobject{currentmarker}{\pgfqpoint{-0.027778in}{0.000000in}}{\pgfqpoint{-0.000000in}{0.000000in}}{%
\pgfpathmoveto{\pgfqpoint{-0.000000in}{0.000000in}}%
\pgfpathlineto{\pgfqpoint{-0.027778in}{0.000000in}}%
\pgfusepath{stroke,fill}%
}%
\begin{pgfscope}%
\pgfsys@transformshift{0.589510in}{1.359764in}%
\pgfsys@useobject{currentmarker}{}%
\end{pgfscope}%
\end{pgfscope}%
\begin{pgfscope}%
\pgfsetbuttcap%
\pgfsetroundjoin%
\definecolor{currentfill}{rgb}{0.000000,0.000000,0.000000}%
\pgfsetfillcolor{currentfill}%
\pgfsetlinewidth{0.602250pt}%
\definecolor{currentstroke}{rgb}{0.000000,0.000000,0.000000}%
\pgfsetstrokecolor{currentstroke}%
\pgfsetdash{}{0pt}%
\pgfsys@defobject{currentmarker}{\pgfqpoint{-0.027778in}{0.000000in}}{\pgfqpoint{-0.000000in}{0.000000in}}{%
\pgfpathmoveto{\pgfqpoint{-0.000000in}{0.000000in}}%
\pgfpathlineto{\pgfqpoint{-0.027778in}{0.000000in}}%
\pgfusepath{stroke,fill}%
}%
\begin{pgfscope}%
\pgfsys@transformshift{0.589510in}{1.379604in}%
\pgfsys@useobject{currentmarker}{}%
\end{pgfscope}%
\end{pgfscope}%
\begin{pgfscope}%
\pgfsetbuttcap%
\pgfsetroundjoin%
\definecolor{currentfill}{rgb}{0.000000,0.000000,0.000000}%
\pgfsetfillcolor{currentfill}%
\pgfsetlinewidth{0.602250pt}%
\definecolor{currentstroke}{rgb}{0.000000,0.000000,0.000000}%
\pgfsetstrokecolor{currentstroke}%
\pgfsetdash{}{0pt}%
\pgfsys@defobject{currentmarker}{\pgfqpoint{-0.027778in}{0.000000in}}{\pgfqpoint{-0.000000in}{0.000000in}}{%
\pgfpathmoveto{\pgfqpoint{-0.000000in}{0.000000in}}%
\pgfpathlineto{\pgfqpoint{-0.027778in}{0.000000in}}%
\pgfusepath{stroke,fill}%
}%
\begin{pgfscope}%
\pgfsys@transformshift{0.589510in}{1.395813in}%
\pgfsys@useobject{currentmarker}{}%
\end{pgfscope}%
\end{pgfscope}%
\begin{pgfscope}%
\pgfsetbuttcap%
\pgfsetroundjoin%
\definecolor{currentfill}{rgb}{0.000000,0.000000,0.000000}%
\pgfsetfillcolor{currentfill}%
\pgfsetlinewidth{0.602250pt}%
\definecolor{currentstroke}{rgb}{0.000000,0.000000,0.000000}%
\pgfsetstrokecolor{currentstroke}%
\pgfsetdash{}{0pt}%
\pgfsys@defobject{currentmarker}{\pgfqpoint{-0.027778in}{0.000000in}}{\pgfqpoint{-0.000000in}{0.000000in}}{%
\pgfpathmoveto{\pgfqpoint{-0.000000in}{0.000000in}}%
\pgfpathlineto{\pgfqpoint{-0.027778in}{0.000000in}}%
\pgfusepath{stroke,fill}%
}%
\begin{pgfscope}%
\pgfsys@transformshift{0.589510in}{1.409519in}%
\pgfsys@useobject{currentmarker}{}%
\end{pgfscope}%
\end{pgfscope}%
\begin{pgfscope}%
\pgfsetbuttcap%
\pgfsetroundjoin%
\definecolor{currentfill}{rgb}{0.000000,0.000000,0.000000}%
\pgfsetfillcolor{currentfill}%
\pgfsetlinewidth{0.602250pt}%
\definecolor{currentstroke}{rgb}{0.000000,0.000000,0.000000}%
\pgfsetstrokecolor{currentstroke}%
\pgfsetdash{}{0pt}%
\pgfsys@defobject{currentmarker}{\pgfqpoint{-0.027778in}{0.000000in}}{\pgfqpoint{-0.000000in}{0.000000in}}{%
\pgfpathmoveto{\pgfqpoint{-0.000000in}{0.000000in}}%
\pgfpathlineto{\pgfqpoint{-0.027778in}{0.000000in}}%
\pgfusepath{stroke,fill}%
}%
\begin{pgfscope}%
\pgfsys@transformshift{0.589510in}{1.421391in}%
\pgfsys@useobject{currentmarker}{}%
\end{pgfscope}%
\end{pgfscope}%
\begin{pgfscope}%
\pgfsetbuttcap%
\pgfsetroundjoin%
\definecolor{currentfill}{rgb}{0.000000,0.000000,0.000000}%
\pgfsetfillcolor{currentfill}%
\pgfsetlinewidth{0.602250pt}%
\definecolor{currentstroke}{rgb}{0.000000,0.000000,0.000000}%
\pgfsetstrokecolor{currentstroke}%
\pgfsetdash{}{0pt}%
\pgfsys@defobject{currentmarker}{\pgfqpoint{-0.027778in}{0.000000in}}{\pgfqpoint{-0.000000in}{0.000000in}}{%
\pgfpathmoveto{\pgfqpoint{-0.000000in}{0.000000in}}%
\pgfpathlineto{\pgfqpoint{-0.027778in}{0.000000in}}%
\pgfusepath{stroke,fill}%
}%
\begin{pgfscope}%
\pgfsys@transformshift{0.589510in}{1.431862in}%
\pgfsys@useobject{currentmarker}{}%
\end{pgfscope}%
\end{pgfscope}%
\begin{pgfscope}%
\pgfsetbuttcap%
\pgfsetroundjoin%
\definecolor{currentfill}{rgb}{0.000000,0.000000,0.000000}%
\pgfsetfillcolor{currentfill}%
\pgfsetlinewidth{0.602250pt}%
\definecolor{currentstroke}{rgb}{0.000000,0.000000,0.000000}%
\pgfsetstrokecolor{currentstroke}%
\pgfsetdash{}{0pt}%
\pgfsys@defobject{currentmarker}{\pgfqpoint{-0.027778in}{0.000000in}}{\pgfqpoint{-0.000000in}{0.000000in}}{%
\pgfpathmoveto{\pgfqpoint{-0.000000in}{0.000000in}}%
\pgfpathlineto{\pgfqpoint{-0.027778in}{0.000000in}}%
\pgfusepath{stroke,fill}%
}%
\begin{pgfscope}%
\pgfsys@transformshift{0.589510in}{1.502856in}%
\pgfsys@useobject{currentmarker}{}%
\end{pgfscope}%
\end{pgfscope}%
\begin{pgfscope}%
\pgfsetbuttcap%
\pgfsetroundjoin%
\definecolor{currentfill}{rgb}{0.000000,0.000000,0.000000}%
\pgfsetfillcolor{currentfill}%
\pgfsetlinewidth{0.602250pt}%
\definecolor{currentstroke}{rgb}{0.000000,0.000000,0.000000}%
\pgfsetstrokecolor{currentstroke}%
\pgfsetdash{}{0pt}%
\pgfsys@defobject{currentmarker}{\pgfqpoint{-0.027778in}{0.000000in}}{\pgfqpoint{-0.000000in}{0.000000in}}{%
\pgfpathmoveto{\pgfqpoint{-0.000000in}{0.000000in}}%
\pgfpathlineto{\pgfqpoint{-0.027778in}{0.000000in}}%
\pgfusepath{stroke,fill}%
}%
\begin{pgfscope}%
\pgfsys@transformshift{0.589510in}{1.538905in}%
\pgfsys@useobject{currentmarker}{}%
\end{pgfscope}%
\end{pgfscope}%
\begin{pgfscope}%
\pgfsetbuttcap%
\pgfsetroundjoin%
\definecolor{currentfill}{rgb}{0.000000,0.000000,0.000000}%
\pgfsetfillcolor{currentfill}%
\pgfsetlinewidth{0.602250pt}%
\definecolor{currentstroke}{rgb}{0.000000,0.000000,0.000000}%
\pgfsetstrokecolor{currentstroke}%
\pgfsetdash{}{0pt}%
\pgfsys@defobject{currentmarker}{\pgfqpoint{-0.027778in}{0.000000in}}{\pgfqpoint{-0.000000in}{0.000000in}}{%
\pgfpathmoveto{\pgfqpoint{-0.000000in}{0.000000in}}%
\pgfpathlineto{\pgfqpoint{-0.027778in}{0.000000in}}%
\pgfusepath{stroke,fill}%
}%
\begin{pgfscope}%
\pgfsys@transformshift{0.589510in}{1.564482in}%
\pgfsys@useobject{currentmarker}{}%
\end{pgfscope}%
\end{pgfscope}%
\begin{pgfscope}%
\pgfsetbuttcap%
\pgfsetroundjoin%
\definecolor{currentfill}{rgb}{0.000000,0.000000,0.000000}%
\pgfsetfillcolor{currentfill}%
\pgfsetlinewidth{0.602250pt}%
\definecolor{currentstroke}{rgb}{0.000000,0.000000,0.000000}%
\pgfsetstrokecolor{currentstroke}%
\pgfsetdash{}{0pt}%
\pgfsys@defobject{currentmarker}{\pgfqpoint{-0.027778in}{0.000000in}}{\pgfqpoint{-0.000000in}{0.000000in}}{%
\pgfpathmoveto{\pgfqpoint{-0.000000in}{0.000000in}}%
\pgfpathlineto{\pgfqpoint{-0.027778in}{0.000000in}}%
\pgfusepath{stroke,fill}%
}%
\begin{pgfscope}%
\pgfsys@transformshift{0.589510in}{1.584321in}%
\pgfsys@useobject{currentmarker}{}%
\end{pgfscope}%
\end{pgfscope}%
\begin{pgfscope}%
\pgfsetbuttcap%
\pgfsetroundjoin%
\definecolor{currentfill}{rgb}{0.000000,0.000000,0.000000}%
\pgfsetfillcolor{currentfill}%
\pgfsetlinewidth{0.602250pt}%
\definecolor{currentstroke}{rgb}{0.000000,0.000000,0.000000}%
\pgfsetstrokecolor{currentstroke}%
\pgfsetdash{}{0pt}%
\pgfsys@defobject{currentmarker}{\pgfqpoint{-0.027778in}{0.000000in}}{\pgfqpoint{-0.000000in}{0.000000in}}{%
\pgfpathmoveto{\pgfqpoint{-0.000000in}{0.000000in}}%
\pgfpathlineto{\pgfqpoint{-0.027778in}{0.000000in}}%
\pgfusepath{stroke,fill}%
}%
\begin{pgfscope}%
\pgfsys@transformshift{0.589510in}{1.600531in}%
\pgfsys@useobject{currentmarker}{}%
\end{pgfscope}%
\end{pgfscope}%
\begin{pgfscope}%
\pgfsetbuttcap%
\pgfsetroundjoin%
\definecolor{currentfill}{rgb}{0.000000,0.000000,0.000000}%
\pgfsetfillcolor{currentfill}%
\pgfsetlinewidth{0.602250pt}%
\definecolor{currentstroke}{rgb}{0.000000,0.000000,0.000000}%
\pgfsetstrokecolor{currentstroke}%
\pgfsetdash{}{0pt}%
\pgfsys@defobject{currentmarker}{\pgfqpoint{-0.027778in}{0.000000in}}{\pgfqpoint{-0.000000in}{0.000000in}}{%
\pgfpathmoveto{\pgfqpoint{-0.000000in}{0.000000in}}%
\pgfpathlineto{\pgfqpoint{-0.027778in}{0.000000in}}%
\pgfusepath{stroke,fill}%
}%
\begin{pgfscope}%
\pgfsys@transformshift{0.589510in}{1.614236in}%
\pgfsys@useobject{currentmarker}{}%
\end{pgfscope}%
\end{pgfscope}%
\begin{pgfscope}%
\pgfsetbuttcap%
\pgfsetroundjoin%
\definecolor{currentfill}{rgb}{0.000000,0.000000,0.000000}%
\pgfsetfillcolor{currentfill}%
\pgfsetlinewidth{0.602250pt}%
\definecolor{currentstroke}{rgb}{0.000000,0.000000,0.000000}%
\pgfsetstrokecolor{currentstroke}%
\pgfsetdash{}{0pt}%
\pgfsys@defobject{currentmarker}{\pgfqpoint{-0.027778in}{0.000000in}}{\pgfqpoint{-0.000000in}{0.000000in}}{%
\pgfpathmoveto{\pgfqpoint{-0.000000in}{0.000000in}}%
\pgfpathlineto{\pgfqpoint{-0.027778in}{0.000000in}}%
\pgfusepath{stroke,fill}%
}%
\begin{pgfscope}%
\pgfsys@transformshift{0.589510in}{1.626108in}%
\pgfsys@useobject{currentmarker}{}%
\end{pgfscope}%
\end{pgfscope}%
\begin{pgfscope}%
\pgfsetbuttcap%
\pgfsetroundjoin%
\definecolor{currentfill}{rgb}{0.000000,0.000000,0.000000}%
\pgfsetfillcolor{currentfill}%
\pgfsetlinewidth{0.602250pt}%
\definecolor{currentstroke}{rgb}{0.000000,0.000000,0.000000}%
\pgfsetstrokecolor{currentstroke}%
\pgfsetdash{}{0pt}%
\pgfsys@defobject{currentmarker}{\pgfqpoint{-0.027778in}{0.000000in}}{\pgfqpoint{-0.000000in}{0.000000in}}{%
\pgfpathmoveto{\pgfqpoint{-0.000000in}{0.000000in}}%
\pgfpathlineto{\pgfqpoint{-0.027778in}{0.000000in}}%
\pgfusepath{stroke,fill}%
}%
\begin{pgfscope}%
\pgfsys@transformshift{0.589510in}{1.636580in}%
\pgfsys@useobject{currentmarker}{}%
\end{pgfscope}%
\end{pgfscope}%
\begin{pgfscope}%
\pgfsetbuttcap%
\pgfsetroundjoin%
\definecolor{currentfill}{rgb}{0.000000,0.000000,0.000000}%
\pgfsetfillcolor{currentfill}%
\pgfsetlinewidth{0.602250pt}%
\definecolor{currentstroke}{rgb}{0.000000,0.000000,0.000000}%
\pgfsetstrokecolor{currentstroke}%
\pgfsetdash{}{0pt}%
\pgfsys@defobject{currentmarker}{\pgfqpoint{-0.027778in}{0.000000in}}{\pgfqpoint{-0.000000in}{0.000000in}}{%
\pgfpathmoveto{\pgfqpoint{-0.000000in}{0.000000in}}%
\pgfpathlineto{\pgfqpoint{-0.027778in}{0.000000in}}%
\pgfusepath{stroke,fill}%
}%
\begin{pgfscope}%
\pgfsys@transformshift{0.589510in}{1.707573in}%
\pgfsys@useobject{currentmarker}{}%
\end{pgfscope}%
\end{pgfscope}%
\begin{pgfscope}%
\pgfsetbuttcap%
\pgfsetroundjoin%
\definecolor{currentfill}{rgb}{0.000000,0.000000,0.000000}%
\pgfsetfillcolor{currentfill}%
\pgfsetlinewidth{0.602250pt}%
\definecolor{currentstroke}{rgb}{0.000000,0.000000,0.000000}%
\pgfsetstrokecolor{currentstroke}%
\pgfsetdash{}{0pt}%
\pgfsys@defobject{currentmarker}{\pgfqpoint{-0.027778in}{0.000000in}}{\pgfqpoint{-0.000000in}{0.000000in}}{%
\pgfpathmoveto{\pgfqpoint{-0.000000in}{0.000000in}}%
\pgfpathlineto{\pgfqpoint{-0.027778in}{0.000000in}}%
\pgfusepath{stroke,fill}%
}%
\begin{pgfscope}%
\pgfsys@transformshift{0.589510in}{1.743622in}%
\pgfsys@useobject{currentmarker}{}%
\end{pgfscope}%
\end{pgfscope}%
\begin{pgfscope}%
\pgfsetbuttcap%
\pgfsetroundjoin%
\definecolor{currentfill}{rgb}{0.000000,0.000000,0.000000}%
\pgfsetfillcolor{currentfill}%
\pgfsetlinewidth{0.602250pt}%
\definecolor{currentstroke}{rgb}{0.000000,0.000000,0.000000}%
\pgfsetstrokecolor{currentstroke}%
\pgfsetdash{}{0pt}%
\pgfsys@defobject{currentmarker}{\pgfqpoint{-0.027778in}{0.000000in}}{\pgfqpoint{-0.000000in}{0.000000in}}{%
\pgfpathmoveto{\pgfqpoint{-0.000000in}{0.000000in}}%
\pgfpathlineto{\pgfqpoint{-0.027778in}{0.000000in}}%
\pgfusepath{stroke,fill}%
}%
\begin{pgfscope}%
\pgfsys@transformshift{0.589510in}{1.769200in}%
\pgfsys@useobject{currentmarker}{}%
\end{pgfscope}%
\end{pgfscope}%
\begin{pgfscope}%
\pgfsetbuttcap%
\pgfsetroundjoin%
\definecolor{currentfill}{rgb}{0.000000,0.000000,0.000000}%
\pgfsetfillcolor{currentfill}%
\pgfsetlinewidth{0.602250pt}%
\definecolor{currentstroke}{rgb}{0.000000,0.000000,0.000000}%
\pgfsetstrokecolor{currentstroke}%
\pgfsetdash{}{0pt}%
\pgfsys@defobject{currentmarker}{\pgfqpoint{-0.027778in}{0.000000in}}{\pgfqpoint{-0.000000in}{0.000000in}}{%
\pgfpathmoveto{\pgfqpoint{-0.000000in}{0.000000in}}%
\pgfpathlineto{\pgfqpoint{-0.027778in}{0.000000in}}%
\pgfusepath{stroke,fill}%
}%
\begin{pgfscope}%
\pgfsys@transformshift{0.589510in}{1.789039in}%
\pgfsys@useobject{currentmarker}{}%
\end{pgfscope}%
\end{pgfscope}%
\begin{pgfscope}%
\definecolor{textcolor}{rgb}{0.000000,0.000000,0.000000}%
\pgfsetstrokecolor{textcolor}%
\pgfsetfillcolor{textcolor}%
\pgftext[x=0.180559in,y=1.103340in,,bottom,rotate=90.000000]{\color{textcolor}{\rmfamily\fontsize{10.000000}{12.000000}\selectfont\catcode`\^=\active\def^{\ifmmode\sp\else\^{}\fi}\catcode`\%=\active\def%{\%}ADEV $\sigma_A(\tau)$}}%
\end{pgfscope}%
\begin{pgfscope}%
\pgfpathrectangle{\pgfqpoint{0.589510in}{0.417642in}}{\pgfqpoint{1.809765in}{1.371397in}}%
\pgfusepath{clip}%
\pgfsetbuttcap%
\pgfsetroundjoin%
\pgfsetlinewidth{1.505625pt}%
\definecolor{currentstroke}{rgb}{0.003922,0.450980,0.698039}%
\pgfsetstrokecolor{currentstroke}%
\pgfsetdash{{5.550000pt}{2.400000pt}}{0.000000pt}%
\pgfpathmoveto{\pgfqpoint{0.671772in}{0.827077in}}%
\pgfpathlineto{\pgfqpoint{0.809267in}{0.796264in}}%
\pgfpathlineto{\pgfqpoint{0.946763in}{0.765451in}}%
\pgfpathlineto{\pgfqpoint{1.128522in}{0.724718in}}%
\pgfpathlineto{\pgfqpoint{1.266017in}{0.693905in}}%
\pgfpathlineto{\pgfqpoint{1.403513in}{0.663092in}}%
\pgfpathlineto{\pgfqpoint{1.585272in}{0.622360in}}%
\pgfpathlineto{\pgfqpoint{1.722767in}{0.591546in}}%
\pgfpathlineto{\pgfqpoint{1.860263in}{0.560733in}}%
\pgfpathlineto{\pgfqpoint{2.042022in}{0.520001in}}%
\pgfpathlineto{\pgfqpoint{2.179517in}{0.489188in}}%
\pgfpathlineto{\pgfqpoint{2.317013in}{0.458375in}}%
\pgfusepath{stroke}%
\end{pgfscope}%
\begin{pgfscope}%
\pgfpathrectangle{\pgfqpoint{0.589510in}{0.417642in}}{\pgfqpoint{1.809765in}{1.371397in}}%
\pgfusepath{clip}%
\pgfsetbuttcap%
\pgfsetroundjoin%
\definecolor{currentfill}{rgb}{0.003922,0.450980,0.698039}%
\pgfsetfillcolor{currentfill}%
\pgfsetlinewidth{1.003750pt}%
\definecolor{currentstroke}{rgb}{0.003922,0.450980,0.698039}%
\pgfsetstrokecolor{currentstroke}%
\pgfsetdash{}{0pt}%
\pgfsys@defobject{currentmarker}{\pgfqpoint{-0.020833in}{-0.020833in}}{\pgfqpoint{0.020833in}{0.020833in}}{%
\pgfpathmoveto{\pgfqpoint{0.000000in}{-0.020833in}}%
\pgfpathcurveto{\pgfqpoint{0.005525in}{-0.020833in}}{\pgfqpoint{0.010825in}{-0.018638in}}{\pgfqpoint{0.014731in}{-0.014731in}}%
\pgfpathcurveto{\pgfqpoint{0.018638in}{-0.010825in}}{\pgfqpoint{0.020833in}{-0.005525in}}{\pgfqpoint{0.020833in}{0.000000in}}%
\pgfpathcurveto{\pgfqpoint{0.020833in}{0.005525in}}{\pgfqpoint{0.018638in}{0.010825in}}{\pgfqpoint{0.014731in}{0.014731in}}%
\pgfpathcurveto{\pgfqpoint{0.010825in}{0.018638in}}{\pgfqpoint{0.005525in}{0.020833in}}{\pgfqpoint{0.000000in}{0.020833in}}%
\pgfpathcurveto{\pgfqpoint{-0.005525in}{0.020833in}}{\pgfqpoint{-0.010825in}{0.018638in}}{\pgfqpoint{-0.014731in}{0.014731in}}%
\pgfpathcurveto{\pgfqpoint{-0.018638in}{0.010825in}}{\pgfqpoint{-0.020833in}{0.005525in}}{\pgfqpoint{-0.020833in}{0.000000in}}%
\pgfpathcurveto{\pgfqpoint{-0.020833in}{-0.005525in}}{\pgfqpoint{-0.018638in}{-0.010825in}}{\pgfqpoint{-0.014731in}{-0.014731in}}%
\pgfpathcurveto{\pgfqpoint{-0.010825in}{-0.018638in}}{\pgfqpoint{-0.005525in}{-0.020833in}}{\pgfqpoint{0.000000in}{-0.020833in}}%
\pgfpathlineto{\pgfqpoint{0.000000in}{-0.020833in}}%
\pgfpathclose%
\pgfusepath{stroke,fill}%
}%
\begin{pgfscope}%
\pgfsys@transformshift{0.671772in}{0.827704in}%
\pgfsys@useobject{currentmarker}{}%
\end{pgfscope}%
\begin{pgfscope}%
\pgfsys@transformshift{0.809267in}{0.796917in}%
\pgfsys@useobject{currentmarker}{}%
\end{pgfscope}%
\begin{pgfscope}%
\pgfsys@transformshift{0.946763in}{0.765573in}%
\pgfsys@useobject{currentmarker}{}%
\end{pgfscope}%
\begin{pgfscope}%
\pgfsys@transformshift{1.128522in}{0.722995in}%
\pgfsys@useobject{currentmarker}{}%
\end{pgfscope}%
\begin{pgfscope}%
\pgfsys@transformshift{1.266017in}{0.689209in}%
\pgfsys@useobject{currentmarker}{}%
\end{pgfscope}%
\begin{pgfscope}%
\pgfsys@transformshift{1.403513in}{0.662309in}%
\pgfsys@useobject{currentmarker}{}%
\end{pgfscope}%
\begin{pgfscope}%
\pgfsys@transformshift{1.585272in}{0.624538in}%
\pgfsys@useobject{currentmarker}{}%
\end{pgfscope}%
\begin{pgfscope}%
\pgfsys@transformshift{1.722767in}{0.589221in}%
\pgfsys@useobject{currentmarker}{}%
\end{pgfscope}%
\begin{pgfscope}%
\pgfsys@transformshift{1.860263in}{0.544348in}%
\pgfsys@useobject{currentmarker}{}%
\end{pgfscope}%
\begin{pgfscope}%
\pgfsys@transformshift{2.042022in}{0.497050in}%
\pgfsys@useobject{currentmarker}{}%
\end{pgfscope}%
\begin{pgfscope}%
\pgfsys@transformshift{2.179517in}{0.507227in}%
\pgfsys@useobject{currentmarker}{}%
\end{pgfscope}%
\begin{pgfscope}%
\pgfsys@transformshift{2.317013in}{0.455319in}%
\pgfsys@useobject{currentmarker}{}%
\end{pgfscope}%
\end{pgfscope}%
\begin{pgfscope}%
\pgfsetrectcap%
\pgfsetmiterjoin%
\pgfsetlinewidth{0.803000pt}%
\definecolor{currentstroke}{rgb}{0.000000,0.000000,0.000000}%
\pgfsetstrokecolor{currentstroke}%
\pgfsetdash{}{0pt}%
\pgfpathmoveto{\pgfqpoint{0.589510in}{0.417642in}}%
\pgfpathlineto{\pgfqpoint{0.589510in}{1.789039in}}%
\pgfusepath{stroke}%
\end{pgfscope}%
\begin{pgfscope}%
\pgfsetrectcap%
\pgfsetmiterjoin%
\pgfsetlinewidth{0.803000pt}%
\definecolor{currentstroke}{rgb}{0.000000,0.000000,0.000000}%
\pgfsetstrokecolor{currentstroke}%
\pgfsetdash{}{0pt}%
\pgfpathmoveto{\pgfqpoint{2.399275in}{0.417642in}}%
\pgfpathlineto{\pgfqpoint{2.399275in}{1.789039in}}%
\pgfusepath{stroke}%
\end{pgfscope}%
\begin{pgfscope}%
\pgfsetrectcap%
\pgfsetmiterjoin%
\pgfsetlinewidth{0.803000pt}%
\definecolor{currentstroke}{rgb}{0.000000,0.000000,0.000000}%
\pgfsetstrokecolor{currentstroke}%
\pgfsetdash{}{0pt}%
\pgfpathmoveto{\pgfqpoint{0.589510in}{0.417642in}}%
\pgfpathlineto{\pgfqpoint{2.399275in}{0.417642in}}%
\pgfusepath{stroke}%
\end{pgfscope}%
\begin{pgfscope}%
\pgfsetrectcap%
\pgfsetmiterjoin%
\pgfsetlinewidth{0.803000pt}%
\definecolor{currentstroke}{rgb}{0.000000,0.000000,0.000000}%
\pgfsetstrokecolor{currentstroke}%
\pgfsetdash{}{0pt}%
\pgfpathmoveto{\pgfqpoint{0.589510in}{1.789039in}}%
\pgfpathlineto{\pgfqpoint{2.399275in}{1.789039in}}%
\pgfusepath{stroke}%
\end{pgfscope}%
\begin{pgfscope}%
\pgfsetbuttcap%
\pgfsetmiterjoin%
\definecolor{currentfill}{rgb}{1.000000,1.000000,1.000000}%
\pgfsetfillcolor{currentfill}%
\pgfsetfillopacity{0.800000}%
\pgfsetlinewidth{1.003750pt}%
\definecolor{currentstroke}{rgb}{0.800000,0.800000,0.800000}%
\pgfsetstrokecolor{currentstroke}%
\pgfsetstrokeopacity{0.800000}%
\pgfsetdash{}{0pt}%
\pgfpathmoveto{\pgfqpoint{1.290639in}{1.472371in}}%
\pgfpathlineto{\pgfqpoint{2.321497in}{1.472371in}}%
\pgfpathquadraticcurveto{\pgfqpoint{2.343719in}{1.472371in}}{\pgfqpoint{2.343719in}{1.494593in}}%
\pgfpathlineto{\pgfqpoint{2.343719in}{1.711261in}}%
\pgfpathquadraticcurveto{\pgfqpoint{2.343719in}{1.733483in}}{\pgfqpoint{2.321497in}{1.733483in}}%
\pgfpathlineto{\pgfqpoint{1.290639in}{1.733483in}}%
\pgfpathquadraticcurveto{\pgfqpoint{1.268417in}{1.733483in}}{\pgfqpoint{1.268417in}{1.711261in}}%
\pgfpathlineto{\pgfqpoint{1.268417in}{1.494593in}}%
\pgfpathquadraticcurveto{\pgfqpoint{1.268417in}{1.472371in}}{\pgfqpoint{1.290639in}{1.472371in}}%
\pgfpathlineto{\pgfqpoint{1.290639in}{1.472371in}}%
\pgfpathclose%
\pgfusepath{stroke,fill}%
\end{pgfscope}%
\begin{pgfscope}%
\pgfsetbuttcap%
\pgfsetroundjoin%
\pgfsetlinewidth{1.505625pt}%
\definecolor{currentstroke}{rgb}{0.003922,0.450980,0.698039}%
\pgfsetstrokecolor{currentstroke}%
\pgfsetdash{{5.550000pt}{2.400000pt}}{0.000000pt}%
\pgfpathmoveto{\pgfqpoint{1.312861in}{1.596639in}}%
\pgfpathlineto{\pgfqpoint{1.423972in}{1.596639in}}%
\pgfpathlineto{\pgfqpoint{1.535084in}{1.596639in}}%
\pgfusepath{stroke}%
\end{pgfscope}%
\begin{pgfscope}%
\definecolor{textcolor}{rgb}{0.000000,0.000000,0.000000}%
\pgfsetstrokecolor{textcolor}%
\pgfsetfillcolor{textcolor}%
\pgftext[x=1.623972in,y=1.557750in,left,base]{\color{textcolor}{\rmfamily\fontsize{8.000000}{9.600000}\selectfont\catcode`\^=\active\def^{\ifmmode\sp\else\^{}\fi}\catcode`\%=\active\def%{\%}$\displaystyle \propto\sqrt{h_{0}}\tau^{-0.5}$}}%
\end{pgfscope}%
\end{pgfpicture}%
\makeatother%
\endgroup%

        } % scalebox
        \caption{Allan deviation}
        \label{fig:white_noise_adev}
    \end{subfigure}
    \caption{Different representations of white noise.}
    \label{fig:white_noise_simulated}
\end{figure}

From this simulation, several features can be observed. First of all, the power spectral density is flat and constant with $h_0 = 2$, which is in accordance with table \ref{tab:adev_alpha} and the normalization mentioned earlier. Figure \ref{fig:white_noise_adev} shows the typical $\tau^{-\frac1 2}$ dependence of white noise in the Allan deviation plot. This immediately explains, why filtering white noise scales with $\frac{1}{\sqrt{n}}$ with $n$ being the number of samples averaged.

\clearpage
\minisec{Burst Noise}
Burst noise, popcorn noise, or sometimes referred to as random telegraph signal is a random bi-stable change in a signal and is caused by a generation recombination processes. This, for example, happens in semiconductors if there is a site, that can trap an electrons for a prologned period of time and then randomly release it. Imporities causing lattice defects are discussed in this context \cite{kay2012operational,burst_noise_psd,popcorn_noise_orgin,technote_ti_popcorn_noise}. Such latttice defects can also be introduced by ion implantation during doping. Fortunately, this type of noise has become less prevalent in modern manufacturing processes, because the quality of the semiconductors has improved. But if a trap site is located very close to an important structure, for example a high precision Zener diode, its effect might be so strong, that it can be clearly seen.

The discussion is split into two parts. First the power spectral density is calculated and then the Allan variance is caclulated using that result.

The spectral density of burst noise caused by a single trap site was derived in \cite{burst_noise_wiener_khinchin} by \citeauthor{burst_noise_wiener_khinchin}. The author used the autocorrelation function of the burst noise signal and applied the Wiener-Khinchin (Wiener-Хи́нчин) theorem, which connects the autocorrelation function with the power spectral density. A more detailed derivation can be found in \cite{fundamentals_of_noise_processes}, in this paper the preconditions, like stationarity of the process, are also discussed. The burst noise signal consists of two energie levels, called $0$ and $1$, split by $\Delta y$. Multiple burst noise signals can be superimposed in a real device. This would then result in mutiple levels, but they can be treated separately. The measurement interval over an even number of transitions, so that one ends in the same state as the measurement has started, is the time $T$. The mean lifetime of the levels is called $\bar \tau_0$ and $\bar \tau_1$:
\begin{equation}
    \bar \tau_{0} \approx \frac 1 N \sum_{i}^N \tau_{0,i} \qquad \bar \tau_{1} \approx \frac 1 N \sum_{i}^N \tau_{1,i}
\end{equation}

Figure \ref{fig:burst_noise} shows a burst noise signal along with the definitions above.

\begin{figure}[hb]
    \centering
    \scalebox{1}{%
        \import{figures/}{burst_noise.tex}
    } % scalebox
    \caption{A random burst noise signal.}
    \label{fig:burst_noise}
\end{figure}

Using these definitions, one can then derive \cite{burst_noise_wiener_khinchin}:
\begin{align}
    R_{xx} (T) &= \Delta y^2 \cdot \frac{\bar \tau_1 \bar \tau_0 e^{-\left(\frac{1}{\bar \tau_1}+\frac{1}{\bar \tau_0}\right)T}}{\left(\bar \tau_1 + \bar \tau_0\right)^2} \quad \text{and} \label{eqn:burst_noise_correlation}\\
    S(\omega) &= 4 R_{xx}(0) \frac{\frac{1}{\bar \tau_1} + \frac{1}{\bar \tau_0}}{\left(\frac{1}{\bar \tau_1} + \frac{1}{\bar \tau_0}\right)^2 + \omega^2} \qquad \omega > 0 . \label{eqn:burst_noise_psd}
\end{align}
Note, that the power spectral density is the one-sided version, hence an additional factor of $2$ is included. The d.c. term was ommitted here and can usually be neglected, because it is not relevant for calculating the power spectral density as it only contributes a single peak at $\omega=0$. Using the following definitions of the average time constant and the duty cycle

\begin{align}
    \frac{1}{\bar \tau} &= \frac{1}{\bar \tau_1} + \frac{1}{\bar \tau_0} \quad \mathrm{and} \label{eqn:definition_bar_tau}\\
    D_i &= \frac{\bar \tau_i}{\bar \tau_1 + \bar \tau_0} \quad i \in \{0 ; 1\}
\end{align}

equations \ref{eqn:burst_noise_correlation} and \ref{eqn:burst_noise_psd} can be rewritten to give a more intuitive form:

\begin{align}
    R_{xx} (T) &= \Delta y^2 D_1 D_0 \, e^{-\left(\frac{1}{\bar \tau_1}+\frac{1}{\bar \tau_0}\right)T}\\
    S(\omega) &= 4 R_{xx}(0) \frac{\bar \tau}{1 + \omega^2 \bar \tau^2} \label{eqn:burst_noise_lorentzian}
\end{align}

The special case $\bar \tau_0 = \bar \tau_1$ with $D_i=\frac 1 2$ is the previously mentioned case of random telegraph noise.

$R_{xx} (0)$ can be identified as the mean squared value of $y$:
\begin{equation}
    y_{RMS} = \sqrt{R_{xx}(0)} \,.
\end{equation}

Equation \ref{eqn:burst_noise_lorentzian} is a Lorentzian function and from this it can be easily seen, that a single trap site has a power spectral density, which is proportional to $\frac{1}{f^2}$ at high frequencies and is flat at low frequencies.

With the spectral density in hand, it is now possible to calculate the Allan variance as it was done by \citeauthor{allen_dev_flicker} in \cite{allen_dev_flicker} for the classic example of random telegraph noise where $\bar \tau_1 = \bar \tau_0$. Do note, that table I given by \citeauthor{allen_dev_flicker} shows the total number of events instead of the instantationous number of events typically given. Hence, their notation must be multiplied by $\frac{1}{\tau^2}$ (or $\frac{1}{T^2}$ in their notation). For the generic case with $\bar \tau_1$, $\bar \tau_0$ and the definition of $\bar \tau$ given in equation \ref{eqn:definition_bar_tau} one finds for the Allan variance of burst noise:
\begin{equation}
    \sigma^2_A(\tau) = R_{xx}(0) \frac{\bar \tau^2}{\tau^2} \left(4 e^{-\frac{\tau}{\bar \tau}} - e^{-\frac{2 \tau}{\bar \tau}} + 2 \frac{\tau}{\bar \tau} - 3 \right) \label{eqn:burst_noise_avar}
\end{equation}

Having arrived at equations \ref{eqn:burst_noise_lorentzian} and \ref{eqn:burst_noise_avar} of the power spectral density and Allan variance, it it now possible to model it. For this purpose, parts of the Python library \textit{qtt} \cite{qtt} was used. The algorithm written by \citeauthor{qtt} implements continous-time Markov chains to simulate the burst noise signal. The result can be see in figure \ref{fig:burst_noise_simulated}. For these simulations one time constants, namely the lifetime of the lower state $\bar \tau_0$ was held constant, while the lifetime of the upper state was varied to show the effect of different $\bar \tau$. By looking at the time domain in figure \ref{fig:burst_noise_time} it can be seen, that the maximum average number of state changes can be observed, when $\bar \tau_1 = \bar \tau_0$. If $\bar \tau_1 > \bar \tau_0$ the system will favour the upper, while if $\bar \tau_1 < \bar \tau_0$ it will favour the lower state instead. This explaines why the noise is strongest for random telegraph noise when $\bar \tau_1 = \bar \tau_0$, which can also be seen in power spectral density in figure \ref{fig:burst_noise_psd}. Looking at the Allan deviation in figure \ref{fig:burst_noise_adev} confirms this, but also shows another interesting implication as it shows an obvious maximum. If the application allows a choice over the sampling interval $\tau$, the effect of the burst noise can mitigated by staying well clear of the maximum.

The small deviation from the analytical solution in figure \ref{fig:burst_noise_adev}  at large $\tau$ is a typical so called end-of-data error. As it was discussed above, the Allan deviation can only be estimated given a limited number of samples using equation \ref{eqn:adev_estimator} and going to longer $\tau$ means there are fewer samples to average over.

\begin{figure}[ht]
    \centering
    \begin{subfigure}{0.8\linewidth}
        \centering
        \scalebox{1}{%
            %% Creator: Matplotlib, PGF backend
%%
%% To include the figure in your LaTeX document, write
%%   \input{<filename>.pgf}
%%
%% Make sure the required packages are loaded in your preamble
%%   \usepackage{pgf}
%%
%% Also ensure that all the required font packages are loaded; for instance,
%% the lmodern package is sometimes necessary when using math font.
%%   \usepackage{lmodern}
%%
%% Figures using additional raster images can only be included by \input if
%% they are in the same directory as the main LaTeX file. For loading figures
%% from other directories you can use the `import` package
%%   \usepackage{import}
%%
%% and then include the figures with
%%   \import{<path to file>}{<filename>.pgf}
%%
%% Matplotlib used the following preamble
%%   \usepackage{siunitx}
%%   \usepackage{fontspec}
%%
\begingroup%
\makeatletter%
\begin{pgfpicture}%
\pgfpathrectangle{\pgfpointorigin}{\pgfqpoint{4.068242in}{2.514312in}}%
\pgfusepath{use as bounding box, clip}%
\begin{pgfscope}%
\pgfsetbuttcap%
\pgfsetmiterjoin%
\definecolor{currentfill}{rgb}{1.000000,1.000000,1.000000}%
\pgfsetfillcolor{currentfill}%
\pgfsetlinewidth{0.000000pt}%
\definecolor{currentstroke}{rgb}{1.000000,1.000000,1.000000}%
\pgfsetstrokecolor{currentstroke}%
\pgfsetdash{}{0pt}%
\pgfpathmoveto{\pgfqpoint{0.000000in}{0.000000in}}%
\pgfpathlineto{\pgfqpoint{4.068242in}{0.000000in}}%
\pgfpathlineto{\pgfqpoint{4.068242in}{2.514312in}}%
\pgfpathlineto{\pgfqpoint{0.000000in}{2.514312in}}%
\pgfpathlineto{\pgfqpoint{0.000000in}{0.000000in}}%
\pgfpathclose%
\pgfusepath{fill}%
\end{pgfscope}%
\begin{pgfscope}%
\pgfsetbuttcap%
\pgfsetmiterjoin%
\definecolor{currentfill}{rgb}{1.000000,1.000000,1.000000}%
\pgfsetfillcolor{currentfill}%
\pgfsetlinewidth{0.000000pt}%
\definecolor{currentstroke}{rgb}{0.000000,0.000000,0.000000}%
\pgfsetstrokecolor{currentstroke}%
\pgfsetstrokeopacity{0.000000}%
\pgfsetdash{}{0pt}%
\pgfpathmoveto{\pgfqpoint{0.471687in}{0.416447in}}%
\pgfpathlineto{\pgfqpoint{4.026572in}{0.416447in}}%
\pgfpathlineto{\pgfqpoint{4.026572in}{2.472642in}}%
\pgfpathlineto{\pgfqpoint{0.471687in}{2.472642in}}%
\pgfpathlineto{\pgfqpoint{0.471687in}{0.416447in}}%
\pgfpathclose%
\pgfusepath{fill}%
\end{pgfscope}%
\begin{pgfscope}%
\pgfpathrectangle{\pgfqpoint{0.471687in}{0.416447in}}{\pgfqpoint{3.554884in}{2.056194in}}%
\pgfusepath{clip}%
\pgfsetrectcap%
\pgfsetroundjoin%
\pgfsetlinewidth{0.803000pt}%
\definecolor{currentstroke}{rgb}{0.450000,0.450000,0.450000}%
\pgfsetstrokecolor{currentstroke}%
\pgfsetdash{}{0pt}%
\pgfpathmoveto{\pgfqpoint{0.633273in}{0.416447in}}%
\pgfpathlineto{\pgfqpoint{0.633273in}{2.472642in}}%
\pgfusepath{stroke}%
\end{pgfscope}%
\begin{pgfscope}%
\pgfsetbuttcap%
\pgfsetroundjoin%
\definecolor{currentfill}{rgb}{0.000000,0.000000,0.000000}%
\pgfsetfillcolor{currentfill}%
\pgfsetlinewidth{0.803000pt}%
\definecolor{currentstroke}{rgb}{0.000000,0.000000,0.000000}%
\pgfsetstrokecolor{currentstroke}%
\pgfsetdash{}{0pt}%
\pgfsys@defobject{currentmarker}{\pgfqpoint{0.000000in}{-0.048611in}}{\pgfqpoint{0.000000in}{0.000000in}}{%
\pgfpathmoveto{\pgfqpoint{0.000000in}{0.000000in}}%
\pgfpathlineto{\pgfqpoint{0.000000in}{-0.048611in}}%
\pgfusepath{stroke,fill}%
}%
\begin{pgfscope}%
\pgfsys@transformshift{0.633273in}{0.416447in}%
\pgfsys@useobject{currentmarker}{}%
\end{pgfscope}%
\end{pgfscope}%
\begin{pgfscope}%
\definecolor{textcolor}{rgb}{0.000000,0.000000,0.000000}%
\pgfsetstrokecolor{textcolor}%
\pgfsetfillcolor{textcolor}%
\pgftext[x=0.633273in,y=0.319225in,,top]{\color{textcolor}\rmfamily\fontsize{8.000000}{9.600000}\selectfont \(\displaystyle {0}\)}%
\end{pgfscope}%
\begin{pgfscope}%
\pgfpathrectangle{\pgfqpoint{0.471687in}{0.416447in}}{\pgfqpoint{3.554884in}{2.056194in}}%
\pgfusepath{clip}%
\pgfsetrectcap%
\pgfsetroundjoin%
\pgfsetlinewidth{0.803000pt}%
\definecolor{currentstroke}{rgb}{0.450000,0.450000,0.450000}%
\pgfsetstrokecolor{currentstroke}%
\pgfsetdash{}{0pt}%
\pgfpathmoveto{\pgfqpoint{1.279939in}{0.416447in}}%
\pgfpathlineto{\pgfqpoint{1.279939in}{2.472642in}}%
\pgfusepath{stroke}%
\end{pgfscope}%
\begin{pgfscope}%
\pgfsetbuttcap%
\pgfsetroundjoin%
\definecolor{currentfill}{rgb}{0.000000,0.000000,0.000000}%
\pgfsetfillcolor{currentfill}%
\pgfsetlinewidth{0.803000pt}%
\definecolor{currentstroke}{rgb}{0.000000,0.000000,0.000000}%
\pgfsetstrokecolor{currentstroke}%
\pgfsetdash{}{0pt}%
\pgfsys@defobject{currentmarker}{\pgfqpoint{0.000000in}{-0.048611in}}{\pgfqpoint{0.000000in}{0.000000in}}{%
\pgfpathmoveto{\pgfqpoint{0.000000in}{0.000000in}}%
\pgfpathlineto{\pgfqpoint{0.000000in}{-0.048611in}}%
\pgfusepath{stroke,fill}%
}%
\begin{pgfscope}%
\pgfsys@transformshift{1.279939in}{0.416447in}%
\pgfsys@useobject{currentmarker}{}%
\end{pgfscope}%
\end{pgfscope}%
\begin{pgfscope}%
\definecolor{textcolor}{rgb}{0.000000,0.000000,0.000000}%
\pgfsetstrokecolor{textcolor}%
\pgfsetfillcolor{textcolor}%
\pgftext[x=1.279939in,y=0.319225in,,top]{\color{textcolor}\rmfamily\fontsize{8.000000}{9.600000}\selectfont \(\displaystyle {2}\)}%
\end{pgfscope}%
\begin{pgfscope}%
\pgfpathrectangle{\pgfqpoint{0.471687in}{0.416447in}}{\pgfqpoint{3.554884in}{2.056194in}}%
\pgfusepath{clip}%
\pgfsetrectcap%
\pgfsetroundjoin%
\pgfsetlinewidth{0.803000pt}%
\definecolor{currentstroke}{rgb}{0.450000,0.450000,0.450000}%
\pgfsetstrokecolor{currentstroke}%
\pgfsetdash{}{0pt}%
\pgfpathmoveto{\pgfqpoint{1.926605in}{0.416447in}}%
\pgfpathlineto{\pgfqpoint{1.926605in}{2.472642in}}%
\pgfusepath{stroke}%
\end{pgfscope}%
\begin{pgfscope}%
\pgfsetbuttcap%
\pgfsetroundjoin%
\definecolor{currentfill}{rgb}{0.000000,0.000000,0.000000}%
\pgfsetfillcolor{currentfill}%
\pgfsetlinewidth{0.803000pt}%
\definecolor{currentstroke}{rgb}{0.000000,0.000000,0.000000}%
\pgfsetstrokecolor{currentstroke}%
\pgfsetdash{}{0pt}%
\pgfsys@defobject{currentmarker}{\pgfqpoint{0.000000in}{-0.048611in}}{\pgfqpoint{0.000000in}{0.000000in}}{%
\pgfpathmoveto{\pgfqpoint{0.000000in}{0.000000in}}%
\pgfpathlineto{\pgfqpoint{0.000000in}{-0.048611in}}%
\pgfusepath{stroke,fill}%
}%
\begin{pgfscope}%
\pgfsys@transformshift{1.926605in}{0.416447in}%
\pgfsys@useobject{currentmarker}{}%
\end{pgfscope}%
\end{pgfscope}%
\begin{pgfscope}%
\definecolor{textcolor}{rgb}{0.000000,0.000000,0.000000}%
\pgfsetstrokecolor{textcolor}%
\pgfsetfillcolor{textcolor}%
\pgftext[x=1.926605in,y=0.319225in,,top]{\color{textcolor}\rmfamily\fontsize{8.000000}{9.600000}\selectfont \(\displaystyle {4}\)}%
\end{pgfscope}%
\begin{pgfscope}%
\pgfpathrectangle{\pgfqpoint{0.471687in}{0.416447in}}{\pgfqpoint{3.554884in}{2.056194in}}%
\pgfusepath{clip}%
\pgfsetrectcap%
\pgfsetroundjoin%
\pgfsetlinewidth{0.803000pt}%
\definecolor{currentstroke}{rgb}{0.450000,0.450000,0.450000}%
\pgfsetstrokecolor{currentstroke}%
\pgfsetdash{}{0pt}%
\pgfpathmoveto{\pgfqpoint{2.573271in}{0.416447in}}%
\pgfpathlineto{\pgfqpoint{2.573271in}{2.472642in}}%
\pgfusepath{stroke}%
\end{pgfscope}%
\begin{pgfscope}%
\pgfsetbuttcap%
\pgfsetroundjoin%
\definecolor{currentfill}{rgb}{0.000000,0.000000,0.000000}%
\pgfsetfillcolor{currentfill}%
\pgfsetlinewidth{0.803000pt}%
\definecolor{currentstroke}{rgb}{0.000000,0.000000,0.000000}%
\pgfsetstrokecolor{currentstroke}%
\pgfsetdash{}{0pt}%
\pgfsys@defobject{currentmarker}{\pgfqpoint{0.000000in}{-0.048611in}}{\pgfqpoint{0.000000in}{0.000000in}}{%
\pgfpathmoveto{\pgfqpoint{0.000000in}{0.000000in}}%
\pgfpathlineto{\pgfqpoint{0.000000in}{-0.048611in}}%
\pgfusepath{stroke,fill}%
}%
\begin{pgfscope}%
\pgfsys@transformshift{2.573271in}{0.416447in}%
\pgfsys@useobject{currentmarker}{}%
\end{pgfscope}%
\end{pgfscope}%
\begin{pgfscope}%
\definecolor{textcolor}{rgb}{0.000000,0.000000,0.000000}%
\pgfsetstrokecolor{textcolor}%
\pgfsetfillcolor{textcolor}%
\pgftext[x=2.573271in,y=0.319225in,,top]{\color{textcolor}\rmfamily\fontsize{8.000000}{9.600000}\selectfont \(\displaystyle {6}\)}%
\end{pgfscope}%
\begin{pgfscope}%
\pgfpathrectangle{\pgfqpoint{0.471687in}{0.416447in}}{\pgfqpoint{3.554884in}{2.056194in}}%
\pgfusepath{clip}%
\pgfsetrectcap%
\pgfsetroundjoin%
\pgfsetlinewidth{0.803000pt}%
\definecolor{currentstroke}{rgb}{0.450000,0.450000,0.450000}%
\pgfsetstrokecolor{currentstroke}%
\pgfsetdash{}{0pt}%
\pgfpathmoveto{\pgfqpoint{3.219937in}{0.416447in}}%
\pgfpathlineto{\pgfqpoint{3.219937in}{2.472642in}}%
\pgfusepath{stroke}%
\end{pgfscope}%
\begin{pgfscope}%
\pgfsetbuttcap%
\pgfsetroundjoin%
\definecolor{currentfill}{rgb}{0.000000,0.000000,0.000000}%
\pgfsetfillcolor{currentfill}%
\pgfsetlinewidth{0.803000pt}%
\definecolor{currentstroke}{rgb}{0.000000,0.000000,0.000000}%
\pgfsetstrokecolor{currentstroke}%
\pgfsetdash{}{0pt}%
\pgfsys@defobject{currentmarker}{\pgfqpoint{0.000000in}{-0.048611in}}{\pgfqpoint{0.000000in}{0.000000in}}{%
\pgfpathmoveto{\pgfqpoint{0.000000in}{0.000000in}}%
\pgfpathlineto{\pgfqpoint{0.000000in}{-0.048611in}}%
\pgfusepath{stroke,fill}%
}%
\begin{pgfscope}%
\pgfsys@transformshift{3.219937in}{0.416447in}%
\pgfsys@useobject{currentmarker}{}%
\end{pgfscope}%
\end{pgfscope}%
\begin{pgfscope}%
\definecolor{textcolor}{rgb}{0.000000,0.000000,0.000000}%
\pgfsetstrokecolor{textcolor}%
\pgfsetfillcolor{textcolor}%
\pgftext[x=3.219937in,y=0.319225in,,top]{\color{textcolor}\rmfamily\fontsize{8.000000}{9.600000}\selectfont \(\displaystyle {8}\)}%
\end{pgfscope}%
\begin{pgfscope}%
\pgfpathrectangle{\pgfqpoint{0.471687in}{0.416447in}}{\pgfqpoint{3.554884in}{2.056194in}}%
\pgfusepath{clip}%
\pgfsetrectcap%
\pgfsetroundjoin%
\pgfsetlinewidth{0.803000pt}%
\definecolor{currentstroke}{rgb}{0.450000,0.450000,0.450000}%
\pgfsetstrokecolor{currentstroke}%
\pgfsetdash{}{0pt}%
\pgfpathmoveto{\pgfqpoint{3.866603in}{0.416447in}}%
\pgfpathlineto{\pgfqpoint{3.866603in}{2.472642in}}%
\pgfusepath{stroke}%
\end{pgfscope}%
\begin{pgfscope}%
\pgfsetbuttcap%
\pgfsetroundjoin%
\definecolor{currentfill}{rgb}{0.000000,0.000000,0.000000}%
\pgfsetfillcolor{currentfill}%
\pgfsetlinewidth{0.803000pt}%
\definecolor{currentstroke}{rgb}{0.000000,0.000000,0.000000}%
\pgfsetstrokecolor{currentstroke}%
\pgfsetdash{}{0pt}%
\pgfsys@defobject{currentmarker}{\pgfqpoint{0.000000in}{-0.048611in}}{\pgfqpoint{0.000000in}{0.000000in}}{%
\pgfpathmoveto{\pgfqpoint{0.000000in}{0.000000in}}%
\pgfpathlineto{\pgfqpoint{0.000000in}{-0.048611in}}%
\pgfusepath{stroke,fill}%
}%
\begin{pgfscope}%
\pgfsys@transformshift{3.866603in}{0.416447in}%
\pgfsys@useobject{currentmarker}{}%
\end{pgfscope}%
\end{pgfscope}%
\begin{pgfscope}%
\definecolor{textcolor}{rgb}{0.000000,0.000000,0.000000}%
\pgfsetstrokecolor{textcolor}%
\pgfsetfillcolor{textcolor}%
\pgftext[x=3.866603in,y=0.319225in,,top]{\color{textcolor}\rmfamily\fontsize{8.000000}{9.600000}\selectfont \(\displaystyle {10}\)}%
\end{pgfscope}%
\begin{pgfscope}%
\definecolor{textcolor}{rgb}{0.000000,0.000000,0.000000}%
\pgfsetstrokecolor{textcolor}%
\pgfsetfillcolor{textcolor}%
\pgftext[x=2.249130in,y=0.165003in,,top]{\color{textcolor}\rmfamily\fontsize{10.000000}{12.000000}\selectfont Time in \unit{\second}}%
\end{pgfscope}%
\begin{pgfscope}%
\pgfpathrectangle{\pgfqpoint{0.471687in}{0.416447in}}{\pgfqpoint{3.554884in}{2.056194in}}%
\pgfusepath{clip}%
\pgfsetrectcap%
\pgfsetroundjoin%
\pgfsetlinewidth{0.803000pt}%
\definecolor{currentstroke}{rgb}{0.450000,0.450000,0.450000}%
\pgfsetstrokecolor{currentstroke}%
\pgfsetdash{}{0pt}%
\pgfpathmoveto{\pgfqpoint{0.471687in}{0.509911in}}%
\pgfpathlineto{\pgfqpoint{4.026572in}{0.509911in}}%
\pgfusepath{stroke}%
\end{pgfscope}%
\begin{pgfscope}%
\pgfsetbuttcap%
\pgfsetroundjoin%
\definecolor{currentfill}{rgb}{0.000000,0.000000,0.000000}%
\pgfsetfillcolor{currentfill}%
\pgfsetlinewidth{0.803000pt}%
\definecolor{currentstroke}{rgb}{0.000000,0.000000,0.000000}%
\pgfsetstrokecolor{currentstroke}%
\pgfsetdash{}{0pt}%
\pgfsys@defobject{currentmarker}{\pgfqpoint{-0.048611in}{0.000000in}}{\pgfqpoint{-0.000000in}{0.000000in}}{%
\pgfpathmoveto{\pgfqpoint{-0.000000in}{0.000000in}}%
\pgfpathlineto{\pgfqpoint{-0.048611in}{0.000000in}}%
\pgfusepath{stroke,fill}%
}%
\begin{pgfscope}%
\pgfsys@transformshift{0.471687in}{0.509911in}%
\pgfsys@useobject{currentmarker}{}%
\end{pgfscope}%
\end{pgfscope}%
\begin{pgfscope}%
\definecolor{textcolor}{rgb}{0.000000,0.000000,0.000000}%
\pgfsetstrokecolor{textcolor}%
\pgfsetfillcolor{textcolor}%
\pgftext[x=0.223614in, y=0.471355in, left, base]{\color{textcolor}\rmfamily\fontsize{8.000000}{9.600000}\selectfont \(\displaystyle {0.0}\)}%
\end{pgfscope}%
\begin{pgfscope}%
\pgfpathrectangle{\pgfqpoint{0.471687in}{0.416447in}}{\pgfqpoint{3.554884in}{2.056194in}}%
\pgfusepath{clip}%
\pgfsetrectcap%
\pgfsetroundjoin%
\pgfsetlinewidth{0.803000pt}%
\definecolor{currentstroke}{rgb}{0.450000,0.450000,0.450000}%
\pgfsetstrokecolor{currentstroke}%
\pgfsetdash{}{0pt}%
\pgfpathmoveto{\pgfqpoint{0.471687in}{0.821455in}}%
\pgfpathlineto{\pgfqpoint{4.026572in}{0.821455in}}%
\pgfusepath{stroke}%
\end{pgfscope}%
\begin{pgfscope}%
\pgfsetbuttcap%
\pgfsetroundjoin%
\definecolor{currentfill}{rgb}{0.000000,0.000000,0.000000}%
\pgfsetfillcolor{currentfill}%
\pgfsetlinewidth{0.803000pt}%
\definecolor{currentstroke}{rgb}{0.000000,0.000000,0.000000}%
\pgfsetstrokecolor{currentstroke}%
\pgfsetdash{}{0pt}%
\pgfsys@defobject{currentmarker}{\pgfqpoint{-0.048611in}{0.000000in}}{\pgfqpoint{-0.000000in}{0.000000in}}{%
\pgfpathmoveto{\pgfqpoint{-0.000000in}{0.000000in}}%
\pgfpathlineto{\pgfqpoint{-0.048611in}{0.000000in}}%
\pgfusepath{stroke,fill}%
}%
\begin{pgfscope}%
\pgfsys@transformshift{0.471687in}{0.821455in}%
\pgfsys@useobject{currentmarker}{}%
\end{pgfscope}%
\end{pgfscope}%
\begin{pgfscope}%
\definecolor{textcolor}{rgb}{0.000000,0.000000,0.000000}%
\pgfsetstrokecolor{textcolor}%
\pgfsetfillcolor{textcolor}%
\pgftext[x=0.223614in, y=0.782900in, left, base]{\color{textcolor}\rmfamily\fontsize{8.000000}{9.600000}\selectfont \(\displaystyle {0.5}\)}%
\end{pgfscope}%
\begin{pgfscope}%
\pgfpathrectangle{\pgfqpoint{0.471687in}{0.416447in}}{\pgfqpoint{3.554884in}{2.056194in}}%
\pgfusepath{clip}%
\pgfsetrectcap%
\pgfsetroundjoin%
\pgfsetlinewidth{0.803000pt}%
\definecolor{currentstroke}{rgb}{0.450000,0.450000,0.450000}%
\pgfsetstrokecolor{currentstroke}%
\pgfsetdash{}{0pt}%
\pgfpathmoveto{\pgfqpoint{0.471687in}{1.133000in}}%
\pgfpathlineto{\pgfqpoint{4.026572in}{1.133000in}}%
\pgfusepath{stroke}%
\end{pgfscope}%
\begin{pgfscope}%
\pgfsetbuttcap%
\pgfsetroundjoin%
\definecolor{currentfill}{rgb}{0.000000,0.000000,0.000000}%
\pgfsetfillcolor{currentfill}%
\pgfsetlinewidth{0.803000pt}%
\definecolor{currentstroke}{rgb}{0.000000,0.000000,0.000000}%
\pgfsetstrokecolor{currentstroke}%
\pgfsetdash{}{0pt}%
\pgfsys@defobject{currentmarker}{\pgfqpoint{-0.048611in}{0.000000in}}{\pgfqpoint{-0.000000in}{0.000000in}}{%
\pgfpathmoveto{\pgfqpoint{-0.000000in}{0.000000in}}%
\pgfpathlineto{\pgfqpoint{-0.048611in}{0.000000in}}%
\pgfusepath{stroke,fill}%
}%
\begin{pgfscope}%
\pgfsys@transformshift{0.471687in}{1.133000in}%
\pgfsys@useobject{currentmarker}{}%
\end{pgfscope}%
\end{pgfscope}%
\begin{pgfscope}%
\definecolor{textcolor}{rgb}{0.000000,0.000000,0.000000}%
\pgfsetstrokecolor{textcolor}%
\pgfsetfillcolor{textcolor}%
\pgftext[x=0.223614in, y=1.094444in, left, base]{\color{textcolor}\rmfamily\fontsize{8.000000}{9.600000}\selectfont \(\displaystyle {1.0}\)}%
\end{pgfscope}%
\begin{pgfscope}%
\pgfpathrectangle{\pgfqpoint{0.471687in}{0.416447in}}{\pgfqpoint{3.554884in}{2.056194in}}%
\pgfusepath{clip}%
\pgfsetrectcap%
\pgfsetroundjoin%
\pgfsetlinewidth{0.803000pt}%
\definecolor{currentstroke}{rgb}{0.450000,0.450000,0.450000}%
\pgfsetstrokecolor{currentstroke}%
\pgfsetdash{}{0pt}%
\pgfpathmoveto{\pgfqpoint{0.471687in}{1.444545in}}%
\pgfpathlineto{\pgfqpoint{4.026572in}{1.444545in}}%
\pgfusepath{stroke}%
\end{pgfscope}%
\begin{pgfscope}%
\pgfsetbuttcap%
\pgfsetroundjoin%
\definecolor{currentfill}{rgb}{0.000000,0.000000,0.000000}%
\pgfsetfillcolor{currentfill}%
\pgfsetlinewidth{0.803000pt}%
\definecolor{currentstroke}{rgb}{0.000000,0.000000,0.000000}%
\pgfsetstrokecolor{currentstroke}%
\pgfsetdash{}{0pt}%
\pgfsys@defobject{currentmarker}{\pgfqpoint{-0.048611in}{0.000000in}}{\pgfqpoint{-0.000000in}{0.000000in}}{%
\pgfpathmoveto{\pgfqpoint{-0.000000in}{0.000000in}}%
\pgfpathlineto{\pgfqpoint{-0.048611in}{0.000000in}}%
\pgfusepath{stroke,fill}%
}%
\begin{pgfscope}%
\pgfsys@transformshift{0.471687in}{1.444545in}%
\pgfsys@useobject{currentmarker}{}%
\end{pgfscope}%
\end{pgfscope}%
\begin{pgfscope}%
\definecolor{textcolor}{rgb}{0.000000,0.000000,0.000000}%
\pgfsetstrokecolor{textcolor}%
\pgfsetfillcolor{textcolor}%
\pgftext[x=0.223614in, y=1.405989in, left, base]{\color{textcolor}\rmfamily\fontsize{8.000000}{9.600000}\selectfont \(\displaystyle {1.5}\)}%
\end{pgfscope}%
\begin{pgfscope}%
\pgfpathrectangle{\pgfqpoint{0.471687in}{0.416447in}}{\pgfqpoint{3.554884in}{2.056194in}}%
\pgfusepath{clip}%
\pgfsetrectcap%
\pgfsetroundjoin%
\pgfsetlinewidth{0.803000pt}%
\definecolor{currentstroke}{rgb}{0.450000,0.450000,0.450000}%
\pgfsetstrokecolor{currentstroke}%
\pgfsetdash{}{0pt}%
\pgfpathmoveto{\pgfqpoint{0.471687in}{1.756089in}}%
\pgfpathlineto{\pgfqpoint{4.026572in}{1.756089in}}%
\pgfusepath{stroke}%
\end{pgfscope}%
\begin{pgfscope}%
\pgfsetbuttcap%
\pgfsetroundjoin%
\definecolor{currentfill}{rgb}{0.000000,0.000000,0.000000}%
\pgfsetfillcolor{currentfill}%
\pgfsetlinewidth{0.803000pt}%
\definecolor{currentstroke}{rgb}{0.000000,0.000000,0.000000}%
\pgfsetstrokecolor{currentstroke}%
\pgfsetdash{}{0pt}%
\pgfsys@defobject{currentmarker}{\pgfqpoint{-0.048611in}{0.000000in}}{\pgfqpoint{-0.000000in}{0.000000in}}{%
\pgfpathmoveto{\pgfqpoint{-0.000000in}{0.000000in}}%
\pgfpathlineto{\pgfqpoint{-0.048611in}{0.000000in}}%
\pgfusepath{stroke,fill}%
}%
\begin{pgfscope}%
\pgfsys@transformshift{0.471687in}{1.756089in}%
\pgfsys@useobject{currentmarker}{}%
\end{pgfscope}%
\end{pgfscope}%
\begin{pgfscope}%
\definecolor{textcolor}{rgb}{0.000000,0.000000,0.000000}%
\pgfsetstrokecolor{textcolor}%
\pgfsetfillcolor{textcolor}%
\pgftext[x=0.223614in, y=1.717534in, left, base]{\color{textcolor}\rmfamily\fontsize{8.000000}{9.600000}\selectfont \(\displaystyle {2.0}\)}%
\end{pgfscope}%
\begin{pgfscope}%
\pgfpathrectangle{\pgfqpoint{0.471687in}{0.416447in}}{\pgfqpoint{3.554884in}{2.056194in}}%
\pgfusepath{clip}%
\pgfsetrectcap%
\pgfsetroundjoin%
\pgfsetlinewidth{0.803000pt}%
\definecolor{currentstroke}{rgb}{0.450000,0.450000,0.450000}%
\pgfsetstrokecolor{currentstroke}%
\pgfsetdash{}{0pt}%
\pgfpathmoveto{\pgfqpoint{0.471687in}{2.067634in}}%
\pgfpathlineto{\pgfqpoint{4.026572in}{2.067634in}}%
\pgfusepath{stroke}%
\end{pgfscope}%
\begin{pgfscope}%
\pgfsetbuttcap%
\pgfsetroundjoin%
\definecolor{currentfill}{rgb}{0.000000,0.000000,0.000000}%
\pgfsetfillcolor{currentfill}%
\pgfsetlinewidth{0.803000pt}%
\definecolor{currentstroke}{rgb}{0.000000,0.000000,0.000000}%
\pgfsetstrokecolor{currentstroke}%
\pgfsetdash{}{0pt}%
\pgfsys@defobject{currentmarker}{\pgfqpoint{-0.048611in}{0.000000in}}{\pgfqpoint{-0.000000in}{0.000000in}}{%
\pgfpathmoveto{\pgfqpoint{-0.000000in}{0.000000in}}%
\pgfpathlineto{\pgfqpoint{-0.048611in}{0.000000in}}%
\pgfusepath{stroke,fill}%
}%
\begin{pgfscope}%
\pgfsys@transformshift{0.471687in}{2.067634in}%
\pgfsys@useobject{currentmarker}{}%
\end{pgfscope}%
\end{pgfscope}%
\begin{pgfscope}%
\definecolor{textcolor}{rgb}{0.000000,0.000000,0.000000}%
\pgfsetstrokecolor{textcolor}%
\pgfsetfillcolor{textcolor}%
\pgftext[x=0.223614in, y=2.029078in, left, base]{\color{textcolor}\rmfamily\fontsize{8.000000}{9.600000}\selectfont \(\displaystyle {2.5}\)}%
\end{pgfscope}%
\begin{pgfscope}%
\pgfpathrectangle{\pgfqpoint{0.471687in}{0.416447in}}{\pgfqpoint{3.554884in}{2.056194in}}%
\pgfusepath{clip}%
\pgfsetrectcap%
\pgfsetroundjoin%
\pgfsetlinewidth{0.803000pt}%
\definecolor{currentstroke}{rgb}{0.450000,0.450000,0.450000}%
\pgfsetstrokecolor{currentstroke}%
\pgfsetdash{}{0pt}%
\pgfpathmoveto{\pgfqpoint{0.471687in}{2.379178in}}%
\pgfpathlineto{\pgfqpoint{4.026572in}{2.379178in}}%
\pgfusepath{stroke}%
\end{pgfscope}%
\begin{pgfscope}%
\pgfsetbuttcap%
\pgfsetroundjoin%
\definecolor{currentfill}{rgb}{0.000000,0.000000,0.000000}%
\pgfsetfillcolor{currentfill}%
\pgfsetlinewidth{0.803000pt}%
\definecolor{currentstroke}{rgb}{0.000000,0.000000,0.000000}%
\pgfsetstrokecolor{currentstroke}%
\pgfsetdash{}{0pt}%
\pgfsys@defobject{currentmarker}{\pgfqpoint{-0.048611in}{0.000000in}}{\pgfqpoint{-0.000000in}{0.000000in}}{%
\pgfpathmoveto{\pgfqpoint{-0.000000in}{0.000000in}}%
\pgfpathlineto{\pgfqpoint{-0.048611in}{0.000000in}}%
\pgfusepath{stroke,fill}%
}%
\begin{pgfscope}%
\pgfsys@transformshift{0.471687in}{2.379178in}%
\pgfsys@useobject{currentmarker}{}%
\end{pgfscope}%
\end{pgfscope}%
\begin{pgfscope}%
\definecolor{textcolor}{rgb}{0.000000,0.000000,0.000000}%
\pgfsetstrokecolor{textcolor}%
\pgfsetfillcolor{textcolor}%
\pgftext[x=0.223614in, y=2.340623in, left, base]{\color{textcolor}\rmfamily\fontsize{8.000000}{9.600000}\selectfont \(\displaystyle {3.0}\)}%
\end{pgfscope}%
\begin{pgfscope}%
\definecolor{textcolor}{rgb}{0.000000,0.000000,0.000000}%
\pgfsetstrokecolor{textcolor}%
\pgfsetfillcolor{textcolor}%
\pgftext[x=0.168059in,y=1.444545in,,bottom,rotate=90.000000]{\color{textcolor}\rmfamily\fontsize{10.000000}{12.000000}\selectfont Amplitude in arb. unit}%
\end{pgfscope}%
\begin{pgfscope}%
\pgfpathrectangle{\pgfqpoint{0.471687in}{0.416447in}}{\pgfqpoint{3.554884in}{2.056194in}}%
\pgfusepath{clip}%
\pgfsetrectcap%
\pgfsetroundjoin%
\pgfsetlinewidth{1.505625pt}%
\definecolor{currentstroke}{rgb}{0.000000,0.447059,0.698039}%
\pgfsetstrokecolor{currentstroke}%
\pgfsetdash{}{0pt}%
\pgfpathmoveto{\pgfqpoint{0.633273in}{0.509911in}}%
\pgfpathlineto{\pgfqpoint{0.736740in}{0.509911in}}%
\pgfpathlineto{\pgfqpoint{0.736740in}{1.133000in}}%
\pgfpathlineto{\pgfqpoint{0.748056in}{1.133000in}}%
\pgfpathlineto{\pgfqpoint{0.748056in}{0.509911in}}%
\pgfpathlineto{\pgfqpoint{0.974389in}{0.509911in}}%
\pgfpathlineto{\pgfqpoint{0.974389in}{1.133000in}}%
\pgfpathlineto{\pgfqpoint{0.987323in}{1.133000in}}%
\pgfpathlineto{\pgfqpoint{0.987323in}{0.509911in}}%
\pgfpathlineto{\pgfqpoint{1.073006in}{0.509911in}}%
\pgfpathlineto{\pgfqpoint{1.073006in}{1.133000in}}%
\pgfpathlineto{\pgfqpoint{1.106956in}{1.133000in}}%
\pgfpathlineto{\pgfqpoint{1.106956in}{0.509911in}}%
\pgfpathlineto{\pgfqpoint{1.905588in}{0.509911in}}%
\pgfpathlineto{\pgfqpoint{1.905588in}{1.133000in}}%
\pgfpathlineto{\pgfqpoint{1.989655in}{1.133000in}}%
\pgfpathlineto{\pgfqpoint{1.989655in}{0.509911in}}%
\pgfpathlineto{\pgfqpoint{2.249938in}{0.509911in}}%
\pgfpathlineto{\pgfqpoint{2.249938in}{1.133000in}}%
\pgfpathlineto{\pgfqpoint{2.264488in}{1.133000in}}%
\pgfpathlineto{\pgfqpoint{2.264488in}{0.509911in}}%
\pgfpathlineto{\pgfqpoint{2.321071in}{0.509911in}}%
\pgfpathlineto{\pgfqpoint{2.321071in}{1.133000in}}%
\pgfpathlineto{\pgfqpoint{2.342088in}{1.133000in}}%
\pgfpathlineto{\pgfqpoint{2.342088in}{0.509911in}}%
\pgfpathlineto{\pgfqpoint{2.670271in}{0.509911in}}%
\pgfpathlineto{\pgfqpoint{2.670271in}{1.133000in}}%
\pgfpathlineto{\pgfqpoint{2.673504in}{1.133000in}}%
\pgfpathlineto{\pgfqpoint{2.673504in}{0.509911in}}%
\pgfpathlineto{\pgfqpoint{2.983904in}{0.509911in}}%
\pgfpathlineto{\pgfqpoint{2.983904in}{1.133000in}}%
\pgfpathlineto{\pgfqpoint{2.998454in}{1.133000in}}%
\pgfpathlineto{\pgfqpoint{2.998454in}{0.509911in}}%
\pgfpathlineto{\pgfqpoint{3.541653in}{0.509911in}}%
\pgfpathlineto{\pgfqpoint{3.541653in}{1.133000in}}%
\pgfpathlineto{\pgfqpoint{3.585303in}{1.133000in}}%
\pgfpathlineto{\pgfqpoint{3.585303in}{0.509911in}}%
\pgfpathlineto{\pgfqpoint{3.864986in}{0.509911in}}%
\pgfpathlineto{\pgfqpoint{3.864986in}{0.509911in}}%
\pgfusepath{stroke}%
\end{pgfscope}%
\begin{pgfscope}%
\pgfpathrectangle{\pgfqpoint{0.471687in}{0.416447in}}{\pgfqpoint{3.554884in}{2.056194in}}%
\pgfusepath{clip}%
\pgfsetrectcap%
\pgfsetroundjoin%
\pgfsetlinewidth{1.505625pt}%
\definecolor{currentstroke}{rgb}{0.000000,0.619608,0.450980}%
\pgfsetstrokecolor{currentstroke}%
\pgfsetdash{}{0pt}%
\pgfpathmoveto{\pgfqpoint{0.633273in}{1.133000in}}%
\pgfpathlineto{\pgfqpoint{0.736740in}{1.133000in}}%
\pgfpathlineto{\pgfqpoint{0.736740in}{1.756089in}}%
\pgfpathlineto{\pgfqpoint{0.917806in}{1.756089in}}%
\pgfpathlineto{\pgfqpoint{0.917806in}{1.133000in}}%
\pgfpathlineto{\pgfqpoint{1.144139in}{1.133000in}}%
\pgfpathlineto{\pgfqpoint{1.144139in}{1.756089in}}%
\pgfpathlineto{\pgfqpoint{1.619439in}{1.756089in}}%
\pgfpathlineto{\pgfqpoint{1.619439in}{1.133000in}}%
\pgfpathlineto{\pgfqpoint{1.705122in}{1.133000in}}%
\pgfpathlineto{\pgfqpoint{1.705122in}{1.756089in}}%
\pgfpathlineto{\pgfqpoint{2.097971in}{1.756089in}}%
\pgfpathlineto{\pgfqpoint{2.097971in}{1.133000in}}%
\pgfpathlineto{\pgfqpoint{2.896604in}{1.133000in}}%
\pgfpathlineto{\pgfqpoint{2.896604in}{1.756089in}}%
\pgfpathlineto{\pgfqpoint{3.864986in}{1.756089in}}%
\pgfpathlineto{\pgfqpoint{3.864986in}{1.756089in}}%
\pgfusepath{stroke}%
\end{pgfscope}%
\begin{pgfscope}%
\pgfpathrectangle{\pgfqpoint{0.471687in}{0.416447in}}{\pgfqpoint{3.554884in}{2.056194in}}%
\pgfusepath{clip}%
\pgfsetrectcap%
\pgfsetroundjoin%
\pgfsetlinewidth{1.505625pt}%
\definecolor{currentstroke}{rgb}{0.835294,0.368627,0.000000}%
\pgfsetstrokecolor{currentstroke}%
\pgfsetdash{}{0pt}%
\pgfpathmoveto{\pgfqpoint{0.633273in}{2.379178in}}%
\pgfpathlineto{\pgfqpoint{0.812723in}{2.379178in}}%
\pgfpathlineto{\pgfqpoint{0.812723in}{1.756089in}}%
\pgfpathlineto{\pgfqpoint{0.917806in}{1.756089in}}%
\pgfpathlineto{\pgfqpoint{0.917806in}{2.379178in}}%
\pgfpathlineto{\pgfqpoint{3.864986in}{2.379178in}}%
\pgfpathlineto{\pgfqpoint{3.864986in}{2.379178in}}%
\pgfusepath{stroke}%
\end{pgfscope}%
\begin{pgfscope}%
\pgfsetrectcap%
\pgfsetmiterjoin%
\pgfsetlinewidth{0.803000pt}%
\definecolor{currentstroke}{rgb}{0.000000,0.000000,0.000000}%
\pgfsetstrokecolor{currentstroke}%
\pgfsetdash{}{0pt}%
\pgfpathmoveto{\pgfqpoint{0.471687in}{0.416447in}}%
\pgfpathlineto{\pgfqpoint{0.471687in}{2.472642in}}%
\pgfusepath{stroke}%
\end{pgfscope}%
\begin{pgfscope}%
\pgfsetrectcap%
\pgfsetmiterjoin%
\pgfsetlinewidth{0.803000pt}%
\definecolor{currentstroke}{rgb}{0.000000,0.000000,0.000000}%
\pgfsetstrokecolor{currentstroke}%
\pgfsetdash{}{0pt}%
\pgfpathmoveto{\pgfqpoint{4.026572in}{0.416447in}}%
\pgfpathlineto{\pgfqpoint{4.026572in}{2.472642in}}%
\pgfusepath{stroke}%
\end{pgfscope}%
\begin{pgfscope}%
\pgfsetrectcap%
\pgfsetmiterjoin%
\pgfsetlinewidth{0.803000pt}%
\definecolor{currentstroke}{rgb}{0.000000,0.000000,0.000000}%
\pgfsetstrokecolor{currentstroke}%
\pgfsetdash{}{0pt}%
\pgfpathmoveto{\pgfqpoint{0.471687in}{0.416447in}}%
\pgfpathlineto{\pgfqpoint{4.026572in}{0.416447in}}%
\pgfusepath{stroke}%
\end{pgfscope}%
\begin{pgfscope}%
\pgfsetrectcap%
\pgfsetmiterjoin%
\pgfsetlinewidth{0.803000pt}%
\definecolor{currentstroke}{rgb}{0.000000,0.000000,0.000000}%
\pgfsetstrokecolor{currentstroke}%
\pgfsetdash{}{0pt}%
\pgfpathmoveto{\pgfqpoint{0.471687in}{2.472642in}}%
\pgfpathlineto{\pgfqpoint{4.026572in}{2.472642in}}%
\pgfusepath{stroke}%
\end{pgfscope}%
\begin{pgfscope}%
\pgfsetbuttcap%
\pgfsetmiterjoin%
\definecolor{currentfill}{rgb}{1.000000,1.000000,1.000000}%
\pgfsetfillcolor{currentfill}%
\pgfsetfillopacity{0.800000}%
\pgfsetlinewidth{1.003750pt}%
\definecolor{currentstroke}{rgb}{0.800000,0.800000,0.800000}%
\pgfsetstrokecolor{currentstroke}%
\pgfsetstrokeopacity{0.800000}%
\pgfsetdash{}{0pt}%
\pgfpathmoveto{\pgfqpoint{3.108484in}{1.919086in}}%
\pgfpathlineto{\pgfqpoint{3.948794in}{1.919086in}}%
\pgfpathquadraticcurveto{\pgfqpoint{3.971016in}{1.919086in}}{\pgfqpoint{3.971016in}{1.941309in}}%
\pgfpathlineto{\pgfqpoint{3.971016in}{2.394864in}}%
\pgfpathquadraticcurveto{\pgfqpoint{3.971016in}{2.417086in}}{\pgfqpoint{3.948794in}{2.417086in}}%
\pgfpathlineto{\pgfqpoint{3.108484in}{2.417086in}}%
\pgfpathquadraticcurveto{\pgfqpoint{3.086261in}{2.417086in}}{\pgfqpoint{3.086261in}{2.394864in}}%
\pgfpathlineto{\pgfqpoint{3.086261in}{1.941309in}}%
\pgfpathquadraticcurveto{\pgfqpoint{3.086261in}{1.919086in}}{\pgfqpoint{3.108484in}{1.919086in}}%
\pgfpathlineto{\pgfqpoint{3.108484in}{1.919086in}}%
\pgfpathclose%
\pgfusepath{stroke,fill}%
\end{pgfscope}%
\begin{pgfscope}%
\pgfsetrectcap%
\pgfsetroundjoin%
\pgfsetlinewidth{1.505625pt}%
\definecolor{currentstroke}{rgb}{0.000000,0.447059,0.698039}%
\pgfsetstrokecolor{currentstroke}%
\pgfsetdash{}{0pt}%
\pgfpathmoveto{\pgfqpoint{3.130706in}{2.333753in}}%
\pgfpathlineto{\pgfqpoint{3.130706in}{2.333753in}}%
\pgfpathlineto{\pgfqpoint{3.241817in}{2.333753in}}%
\pgfpathlineto{\pgfqpoint{3.241817in}{2.333753in}}%
\pgfpathlineto{\pgfqpoint{3.352928in}{2.333753in}}%
\pgfusepath{stroke}%
\end{pgfscope}%
\begin{pgfscope}%
\definecolor{textcolor}{rgb}{0.000000,0.000000,0.000000}%
\pgfsetstrokecolor{textcolor}%
\pgfsetfillcolor{textcolor}%
\pgftext[x=3.441817in,y=2.294864in,left,base]{\color{textcolor}\rmfamily\fontsize{8.000000}{9.600000}\selectfont \(\displaystyle \bar\tau_1=\qty{0.1}{\s}\)}%
\end{pgfscope}%
\begin{pgfscope}%
\pgfsetrectcap%
\pgfsetroundjoin%
\pgfsetlinewidth{1.505625pt}%
\definecolor{currentstroke}{rgb}{0.000000,0.619608,0.450980}%
\pgfsetstrokecolor{currentstroke}%
\pgfsetdash{}{0pt}%
\pgfpathmoveto{\pgfqpoint{3.130706in}{2.178864in}}%
\pgfpathlineto{\pgfqpoint{3.130706in}{2.178864in}}%
\pgfpathlineto{\pgfqpoint{3.241817in}{2.178864in}}%
\pgfpathlineto{\pgfqpoint{3.241817in}{2.178864in}}%
\pgfpathlineto{\pgfqpoint{3.352928in}{2.178864in}}%
\pgfusepath{stroke}%
\end{pgfscope}%
\begin{pgfscope}%
\definecolor{textcolor}{rgb}{0.000000,0.000000,0.000000}%
\pgfsetstrokecolor{textcolor}%
\pgfsetfillcolor{textcolor}%
\pgftext[x=3.441817in,y=2.139975in,left,base]{\color{textcolor}\rmfamily\fontsize{8.000000}{9.600000}\selectfont \(\displaystyle \bar\tau_1=\qty{1}{\s}\)}%
\end{pgfscope}%
\begin{pgfscope}%
\pgfsetrectcap%
\pgfsetroundjoin%
\pgfsetlinewidth{1.505625pt}%
\definecolor{currentstroke}{rgb}{0.835294,0.368627,0.000000}%
\pgfsetstrokecolor{currentstroke}%
\pgfsetdash{}{0pt}%
\pgfpathmoveto{\pgfqpoint{3.130706in}{2.023975in}}%
\pgfpathlineto{\pgfqpoint{3.130706in}{2.023975in}}%
\pgfpathlineto{\pgfqpoint{3.241817in}{2.023975in}}%
\pgfpathlineto{\pgfqpoint{3.241817in}{2.023975in}}%
\pgfpathlineto{\pgfqpoint{3.352928in}{2.023975in}}%
\pgfusepath{stroke}%
\end{pgfscope}%
\begin{pgfscope}%
\definecolor{textcolor}{rgb}{0.000000,0.000000,0.000000}%
\pgfsetstrokecolor{textcolor}%
\pgfsetfillcolor{textcolor}%
\pgftext[x=3.441817in,y=1.985086in,left,base]{\color{textcolor}\rmfamily\fontsize{8.000000}{9.600000}\selectfont \(\displaystyle \bar\tau_1=\qty{10}{\s}\)}%
\end{pgfscope}%
\end{pgfpicture}%
\makeatother%
\endgroup%

        } % scalebox
        \caption{Time domain}
        \label{fig:burst_noise_time}
    \end{subfigure}
    \begin{subfigure}{0.8\linewidth}
        \centering
        \scalebox{1}{%
            %% Creator: Matplotlib, PGF backend
%%
%% To include the figure in your LaTeX document, write
%%   \input{<filename>.pgf}
%%
%% Make sure the required packages are loaded in your preamble
%%   \usepackage{pgf}
%%
%% Also ensure that all the required font packages are loaded; for instance,
%% the lmodern package is sometimes necessary when using math font.
%%   \usepackage{lmodern}
%%
%% Figures using additional raster images can only be included by \input if
%% they are in the same directory as the main LaTeX file. For loading figures
%% from other directories you can use the `import` package
%%   \usepackage{import}
%%
%% and then include the figures with
%%   \import{<path to file>}{<filename>.pgf}
%%
%% Matplotlib used the following preamble
%%   \usepackage{siunitx}
%%   \sisetup{per-mode = symbol}%
%%   \usepackage{fontspec}
%%   \makeatletter\@ifpackageloaded{underscore}{}{\usepackage[strings]{underscore}}\makeatother
%%
\begingroup%
\makeatletter%
\begin{pgfpicture}%
\pgfpathrectangle{\pgfpointorigin}{\pgfqpoint{4.068242in}{2.514312in}}%
\pgfusepath{use as bounding box, clip}%
\begin{pgfscope}%
\pgfsetbuttcap%
\pgfsetmiterjoin%
\definecolor{currentfill}{rgb}{1.000000,1.000000,1.000000}%
\pgfsetfillcolor{currentfill}%
\pgfsetlinewidth{0.000000pt}%
\definecolor{currentstroke}{rgb}{1.000000,1.000000,1.000000}%
\pgfsetstrokecolor{currentstroke}%
\pgfsetdash{}{0pt}%
\pgfpathmoveto{\pgfqpoint{0.000000in}{0.000000in}}%
\pgfpathlineto{\pgfqpoint{4.068242in}{0.000000in}}%
\pgfpathlineto{\pgfqpoint{4.068242in}{2.514312in}}%
\pgfpathlineto{\pgfqpoint{0.000000in}{2.514312in}}%
\pgfpathlineto{\pgfqpoint{0.000000in}{0.000000in}}%
\pgfpathclose%
\pgfusepath{fill}%
\end{pgfscope}%
\begin{pgfscope}%
\pgfsetbuttcap%
\pgfsetmiterjoin%
\definecolor{currentfill}{rgb}{1.000000,1.000000,1.000000}%
\pgfsetfillcolor{currentfill}%
\pgfsetlinewidth{0.000000pt}%
\definecolor{currentstroke}{rgb}{0.000000,0.000000,0.000000}%
\pgfsetstrokecolor{currentstroke}%
\pgfsetstrokeopacity{0.000000}%
\pgfsetdash{}{0pt}%
\pgfpathmoveto{\pgfqpoint{0.594525in}{0.417642in}}%
\pgfpathlineto{\pgfqpoint{4.026572in}{0.417642in}}%
\pgfpathlineto{\pgfqpoint{4.026572in}{2.433919in}}%
\pgfpathlineto{\pgfqpoint{0.594525in}{2.433919in}}%
\pgfpathlineto{\pgfqpoint{0.594525in}{0.417642in}}%
\pgfpathclose%
\pgfusepath{fill}%
\end{pgfscope}%
\begin{pgfscope}%
\pgfpathrectangle{\pgfqpoint{0.594525in}{0.417642in}}{\pgfqpoint{3.432047in}{2.016277in}}%
\pgfusepath{clip}%
\pgfsetrectcap%
\pgfsetroundjoin%
\pgfsetlinewidth{0.803000pt}%
\definecolor{currentstroke}{rgb}{0.450000,0.450000,0.450000}%
\pgfsetstrokecolor{currentstroke}%
\pgfsetdash{}{0pt}%
\pgfpathmoveto{\pgfqpoint{0.750527in}{0.417642in}}%
\pgfpathlineto{\pgfqpoint{0.750527in}{2.433919in}}%
\pgfusepath{stroke}%
\end{pgfscope}%
\begin{pgfscope}%
\pgfsetbuttcap%
\pgfsetroundjoin%
\definecolor{currentfill}{rgb}{0.000000,0.000000,0.000000}%
\pgfsetfillcolor{currentfill}%
\pgfsetlinewidth{0.803000pt}%
\definecolor{currentstroke}{rgb}{0.000000,0.000000,0.000000}%
\pgfsetstrokecolor{currentstroke}%
\pgfsetdash{}{0pt}%
\pgfsys@defobject{currentmarker}{\pgfqpoint{0.000000in}{-0.048611in}}{\pgfqpoint{0.000000in}{0.000000in}}{%
\pgfpathmoveto{\pgfqpoint{0.000000in}{0.000000in}}%
\pgfpathlineto{\pgfqpoint{0.000000in}{-0.048611in}}%
\pgfusepath{stroke,fill}%
}%
\begin{pgfscope}%
\pgfsys@transformshift{0.750527in}{0.417642in}%
\pgfsys@useobject{currentmarker}{}%
\end{pgfscope}%
\end{pgfscope}%
\begin{pgfscope}%
\definecolor{textcolor}{rgb}{0.000000,0.000000,0.000000}%
\pgfsetstrokecolor{textcolor}%
\pgfsetfillcolor{textcolor}%
\pgftext[x=0.750527in,y=0.320420in,,top]{\color{textcolor}\rmfamily\fontsize{8.000000}{9.600000}\selectfont \(\displaystyle {10^{-2}}\)}%
\end{pgfscope}%
\begin{pgfscope}%
\pgfpathrectangle{\pgfqpoint{0.594525in}{0.417642in}}{\pgfqpoint{3.432047in}{2.016277in}}%
\pgfusepath{clip}%
\pgfsetrectcap%
\pgfsetroundjoin%
\pgfsetlinewidth{0.803000pt}%
\definecolor{currentstroke}{rgb}{0.450000,0.450000,0.450000}%
\pgfsetstrokecolor{currentstroke}%
\pgfsetdash{}{0pt}%
\pgfpathmoveto{\pgfqpoint{1.531514in}{0.417642in}}%
\pgfpathlineto{\pgfqpoint{1.531514in}{2.433919in}}%
\pgfusepath{stroke}%
\end{pgfscope}%
\begin{pgfscope}%
\pgfsetbuttcap%
\pgfsetroundjoin%
\definecolor{currentfill}{rgb}{0.000000,0.000000,0.000000}%
\pgfsetfillcolor{currentfill}%
\pgfsetlinewidth{0.803000pt}%
\definecolor{currentstroke}{rgb}{0.000000,0.000000,0.000000}%
\pgfsetstrokecolor{currentstroke}%
\pgfsetdash{}{0pt}%
\pgfsys@defobject{currentmarker}{\pgfqpoint{0.000000in}{-0.048611in}}{\pgfqpoint{0.000000in}{0.000000in}}{%
\pgfpathmoveto{\pgfqpoint{0.000000in}{0.000000in}}%
\pgfpathlineto{\pgfqpoint{0.000000in}{-0.048611in}}%
\pgfusepath{stroke,fill}%
}%
\begin{pgfscope}%
\pgfsys@transformshift{1.531514in}{0.417642in}%
\pgfsys@useobject{currentmarker}{}%
\end{pgfscope}%
\end{pgfscope}%
\begin{pgfscope}%
\definecolor{textcolor}{rgb}{0.000000,0.000000,0.000000}%
\pgfsetstrokecolor{textcolor}%
\pgfsetfillcolor{textcolor}%
\pgftext[x=1.531514in,y=0.320420in,,top]{\color{textcolor}\rmfamily\fontsize{8.000000}{9.600000}\selectfont \(\displaystyle {10^{-1}}\)}%
\end{pgfscope}%
\begin{pgfscope}%
\pgfpathrectangle{\pgfqpoint{0.594525in}{0.417642in}}{\pgfqpoint{3.432047in}{2.016277in}}%
\pgfusepath{clip}%
\pgfsetrectcap%
\pgfsetroundjoin%
\pgfsetlinewidth{0.803000pt}%
\definecolor{currentstroke}{rgb}{0.450000,0.450000,0.450000}%
\pgfsetstrokecolor{currentstroke}%
\pgfsetdash{}{0pt}%
\pgfpathmoveto{\pgfqpoint{2.312501in}{0.417642in}}%
\pgfpathlineto{\pgfqpoint{2.312501in}{2.433919in}}%
\pgfusepath{stroke}%
\end{pgfscope}%
\begin{pgfscope}%
\pgfsetbuttcap%
\pgfsetroundjoin%
\definecolor{currentfill}{rgb}{0.000000,0.000000,0.000000}%
\pgfsetfillcolor{currentfill}%
\pgfsetlinewidth{0.803000pt}%
\definecolor{currentstroke}{rgb}{0.000000,0.000000,0.000000}%
\pgfsetstrokecolor{currentstroke}%
\pgfsetdash{}{0pt}%
\pgfsys@defobject{currentmarker}{\pgfqpoint{0.000000in}{-0.048611in}}{\pgfqpoint{0.000000in}{0.000000in}}{%
\pgfpathmoveto{\pgfqpoint{0.000000in}{0.000000in}}%
\pgfpathlineto{\pgfqpoint{0.000000in}{-0.048611in}}%
\pgfusepath{stroke,fill}%
}%
\begin{pgfscope}%
\pgfsys@transformshift{2.312501in}{0.417642in}%
\pgfsys@useobject{currentmarker}{}%
\end{pgfscope}%
\end{pgfscope}%
\begin{pgfscope}%
\definecolor{textcolor}{rgb}{0.000000,0.000000,0.000000}%
\pgfsetstrokecolor{textcolor}%
\pgfsetfillcolor{textcolor}%
\pgftext[x=2.312501in,y=0.320420in,,top]{\color{textcolor}\rmfamily\fontsize{8.000000}{9.600000}\selectfont \(\displaystyle {10^{0}}\)}%
\end{pgfscope}%
\begin{pgfscope}%
\pgfpathrectangle{\pgfqpoint{0.594525in}{0.417642in}}{\pgfqpoint{3.432047in}{2.016277in}}%
\pgfusepath{clip}%
\pgfsetrectcap%
\pgfsetroundjoin%
\pgfsetlinewidth{0.803000pt}%
\definecolor{currentstroke}{rgb}{0.450000,0.450000,0.450000}%
\pgfsetstrokecolor{currentstroke}%
\pgfsetdash{}{0pt}%
\pgfpathmoveto{\pgfqpoint{3.093488in}{0.417642in}}%
\pgfpathlineto{\pgfqpoint{3.093488in}{2.433919in}}%
\pgfusepath{stroke}%
\end{pgfscope}%
\begin{pgfscope}%
\pgfsetbuttcap%
\pgfsetroundjoin%
\definecolor{currentfill}{rgb}{0.000000,0.000000,0.000000}%
\pgfsetfillcolor{currentfill}%
\pgfsetlinewidth{0.803000pt}%
\definecolor{currentstroke}{rgb}{0.000000,0.000000,0.000000}%
\pgfsetstrokecolor{currentstroke}%
\pgfsetdash{}{0pt}%
\pgfsys@defobject{currentmarker}{\pgfqpoint{0.000000in}{-0.048611in}}{\pgfqpoint{0.000000in}{0.000000in}}{%
\pgfpathmoveto{\pgfqpoint{0.000000in}{0.000000in}}%
\pgfpathlineto{\pgfqpoint{0.000000in}{-0.048611in}}%
\pgfusepath{stroke,fill}%
}%
\begin{pgfscope}%
\pgfsys@transformshift{3.093488in}{0.417642in}%
\pgfsys@useobject{currentmarker}{}%
\end{pgfscope}%
\end{pgfscope}%
\begin{pgfscope}%
\definecolor{textcolor}{rgb}{0.000000,0.000000,0.000000}%
\pgfsetstrokecolor{textcolor}%
\pgfsetfillcolor{textcolor}%
\pgftext[x=3.093488in,y=0.320420in,,top]{\color{textcolor}\rmfamily\fontsize{8.000000}{9.600000}\selectfont \(\displaystyle {10^{1}}\)}%
\end{pgfscope}%
\begin{pgfscope}%
\pgfpathrectangle{\pgfqpoint{0.594525in}{0.417642in}}{\pgfqpoint{3.432047in}{2.016277in}}%
\pgfusepath{clip}%
\pgfsetrectcap%
\pgfsetroundjoin%
\pgfsetlinewidth{0.803000pt}%
\definecolor{currentstroke}{rgb}{0.450000,0.450000,0.450000}%
\pgfsetstrokecolor{currentstroke}%
\pgfsetdash{}{0pt}%
\pgfpathmoveto{\pgfqpoint{3.874475in}{0.417642in}}%
\pgfpathlineto{\pgfqpoint{3.874475in}{2.433919in}}%
\pgfusepath{stroke}%
\end{pgfscope}%
\begin{pgfscope}%
\pgfsetbuttcap%
\pgfsetroundjoin%
\definecolor{currentfill}{rgb}{0.000000,0.000000,0.000000}%
\pgfsetfillcolor{currentfill}%
\pgfsetlinewidth{0.803000pt}%
\definecolor{currentstroke}{rgb}{0.000000,0.000000,0.000000}%
\pgfsetstrokecolor{currentstroke}%
\pgfsetdash{}{0pt}%
\pgfsys@defobject{currentmarker}{\pgfqpoint{0.000000in}{-0.048611in}}{\pgfqpoint{0.000000in}{0.000000in}}{%
\pgfpathmoveto{\pgfqpoint{0.000000in}{0.000000in}}%
\pgfpathlineto{\pgfqpoint{0.000000in}{-0.048611in}}%
\pgfusepath{stroke,fill}%
}%
\begin{pgfscope}%
\pgfsys@transformshift{3.874475in}{0.417642in}%
\pgfsys@useobject{currentmarker}{}%
\end{pgfscope}%
\end{pgfscope}%
\begin{pgfscope}%
\definecolor{textcolor}{rgb}{0.000000,0.000000,0.000000}%
\pgfsetstrokecolor{textcolor}%
\pgfsetfillcolor{textcolor}%
\pgftext[x=3.874475in,y=0.320420in,,top]{\color{textcolor}\rmfamily\fontsize{8.000000}{9.600000}\selectfont \(\displaystyle {10^{2}}\)}%
\end{pgfscope}%
\begin{pgfscope}%
\pgfpathrectangle{\pgfqpoint{0.594525in}{0.417642in}}{\pgfqpoint{3.432047in}{2.016277in}}%
\pgfusepath{clip}%
\pgfsetrectcap%
\pgfsetroundjoin%
\pgfsetlinewidth{0.803000pt}%
\definecolor{currentstroke}{rgb}{0.850000,0.850000,0.850000}%
\pgfsetstrokecolor{currentstroke}%
\pgfsetdash{}{0pt}%
\pgfpathmoveto{\pgfqpoint{0.629550in}{0.417642in}}%
\pgfpathlineto{\pgfqpoint{0.629550in}{2.433919in}}%
\pgfusepath{stroke}%
\end{pgfscope}%
\begin{pgfscope}%
\pgfsetbuttcap%
\pgfsetroundjoin%
\definecolor{currentfill}{rgb}{0.000000,0.000000,0.000000}%
\pgfsetfillcolor{currentfill}%
\pgfsetlinewidth{0.602250pt}%
\definecolor{currentstroke}{rgb}{0.000000,0.000000,0.000000}%
\pgfsetstrokecolor{currentstroke}%
\pgfsetdash{}{0pt}%
\pgfsys@defobject{currentmarker}{\pgfqpoint{0.000000in}{-0.027778in}}{\pgfqpoint{0.000000in}{0.000000in}}{%
\pgfpathmoveto{\pgfqpoint{0.000000in}{0.000000in}}%
\pgfpathlineto{\pgfqpoint{0.000000in}{-0.027778in}}%
\pgfusepath{stroke,fill}%
}%
\begin{pgfscope}%
\pgfsys@transformshift{0.629550in}{0.417642in}%
\pgfsys@useobject{currentmarker}{}%
\end{pgfscope}%
\end{pgfscope}%
\begin{pgfscope}%
\pgfpathrectangle{\pgfqpoint{0.594525in}{0.417642in}}{\pgfqpoint{3.432047in}{2.016277in}}%
\pgfusepath{clip}%
\pgfsetrectcap%
\pgfsetroundjoin%
\pgfsetlinewidth{0.803000pt}%
\definecolor{currentstroke}{rgb}{0.850000,0.850000,0.850000}%
\pgfsetstrokecolor{currentstroke}%
\pgfsetdash{}{0pt}%
\pgfpathmoveto{\pgfqpoint{0.674841in}{0.417642in}}%
\pgfpathlineto{\pgfqpoint{0.674841in}{2.433919in}}%
\pgfusepath{stroke}%
\end{pgfscope}%
\begin{pgfscope}%
\pgfsetbuttcap%
\pgfsetroundjoin%
\definecolor{currentfill}{rgb}{0.000000,0.000000,0.000000}%
\pgfsetfillcolor{currentfill}%
\pgfsetlinewidth{0.602250pt}%
\definecolor{currentstroke}{rgb}{0.000000,0.000000,0.000000}%
\pgfsetstrokecolor{currentstroke}%
\pgfsetdash{}{0pt}%
\pgfsys@defobject{currentmarker}{\pgfqpoint{0.000000in}{-0.027778in}}{\pgfqpoint{0.000000in}{0.000000in}}{%
\pgfpathmoveto{\pgfqpoint{0.000000in}{0.000000in}}%
\pgfpathlineto{\pgfqpoint{0.000000in}{-0.027778in}}%
\pgfusepath{stroke,fill}%
}%
\begin{pgfscope}%
\pgfsys@transformshift{0.674841in}{0.417642in}%
\pgfsys@useobject{currentmarker}{}%
\end{pgfscope}%
\end{pgfscope}%
\begin{pgfscope}%
\pgfpathrectangle{\pgfqpoint{0.594525in}{0.417642in}}{\pgfqpoint{3.432047in}{2.016277in}}%
\pgfusepath{clip}%
\pgfsetrectcap%
\pgfsetroundjoin%
\pgfsetlinewidth{0.803000pt}%
\definecolor{currentstroke}{rgb}{0.850000,0.850000,0.850000}%
\pgfsetstrokecolor{currentstroke}%
\pgfsetdash{}{0pt}%
\pgfpathmoveto{\pgfqpoint{0.714791in}{0.417642in}}%
\pgfpathlineto{\pgfqpoint{0.714791in}{2.433919in}}%
\pgfusepath{stroke}%
\end{pgfscope}%
\begin{pgfscope}%
\pgfsetbuttcap%
\pgfsetroundjoin%
\definecolor{currentfill}{rgb}{0.000000,0.000000,0.000000}%
\pgfsetfillcolor{currentfill}%
\pgfsetlinewidth{0.602250pt}%
\definecolor{currentstroke}{rgb}{0.000000,0.000000,0.000000}%
\pgfsetstrokecolor{currentstroke}%
\pgfsetdash{}{0pt}%
\pgfsys@defobject{currentmarker}{\pgfqpoint{0.000000in}{-0.027778in}}{\pgfqpoint{0.000000in}{0.000000in}}{%
\pgfpathmoveto{\pgfqpoint{0.000000in}{0.000000in}}%
\pgfpathlineto{\pgfqpoint{0.000000in}{-0.027778in}}%
\pgfusepath{stroke,fill}%
}%
\begin{pgfscope}%
\pgfsys@transformshift{0.714791in}{0.417642in}%
\pgfsys@useobject{currentmarker}{}%
\end{pgfscope}%
\end{pgfscope}%
\begin{pgfscope}%
\pgfpathrectangle{\pgfqpoint{0.594525in}{0.417642in}}{\pgfqpoint{3.432047in}{2.016277in}}%
\pgfusepath{clip}%
\pgfsetrectcap%
\pgfsetroundjoin%
\pgfsetlinewidth{0.803000pt}%
\definecolor{currentstroke}{rgb}{0.850000,0.850000,0.850000}%
\pgfsetstrokecolor{currentstroke}%
\pgfsetdash{}{0pt}%
\pgfpathmoveto{\pgfqpoint{0.985627in}{0.417642in}}%
\pgfpathlineto{\pgfqpoint{0.985627in}{2.433919in}}%
\pgfusepath{stroke}%
\end{pgfscope}%
\begin{pgfscope}%
\pgfsetbuttcap%
\pgfsetroundjoin%
\definecolor{currentfill}{rgb}{0.000000,0.000000,0.000000}%
\pgfsetfillcolor{currentfill}%
\pgfsetlinewidth{0.602250pt}%
\definecolor{currentstroke}{rgb}{0.000000,0.000000,0.000000}%
\pgfsetstrokecolor{currentstroke}%
\pgfsetdash{}{0pt}%
\pgfsys@defobject{currentmarker}{\pgfqpoint{0.000000in}{-0.027778in}}{\pgfqpoint{0.000000in}{0.000000in}}{%
\pgfpathmoveto{\pgfqpoint{0.000000in}{0.000000in}}%
\pgfpathlineto{\pgfqpoint{0.000000in}{-0.027778in}}%
\pgfusepath{stroke,fill}%
}%
\begin{pgfscope}%
\pgfsys@transformshift{0.985627in}{0.417642in}%
\pgfsys@useobject{currentmarker}{}%
\end{pgfscope}%
\end{pgfscope}%
\begin{pgfscope}%
\pgfpathrectangle{\pgfqpoint{0.594525in}{0.417642in}}{\pgfqpoint{3.432047in}{2.016277in}}%
\pgfusepath{clip}%
\pgfsetrectcap%
\pgfsetroundjoin%
\pgfsetlinewidth{0.803000pt}%
\definecolor{currentstroke}{rgb}{0.850000,0.850000,0.850000}%
\pgfsetstrokecolor{currentstroke}%
\pgfsetdash{}{0pt}%
\pgfpathmoveto{\pgfqpoint{1.123152in}{0.417642in}}%
\pgfpathlineto{\pgfqpoint{1.123152in}{2.433919in}}%
\pgfusepath{stroke}%
\end{pgfscope}%
\begin{pgfscope}%
\pgfsetbuttcap%
\pgfsetroundjoin%
\definecolor{currentfill}{rgb}{0.000000,0.000000,0.000000}%
\pgfsetfillcolor{currentfill}%
\pgfsetlinewidth{0.602250pt}%
\definecolor{currentstroke}{rgb}{0.000000,0.000000,0.000000}%
\pgfsetstrokecolor{currentstroke}%
\pgfsetdash{}{0pt}%
\pgfsys@defobject{currentmarker}{\pgfqpoint{0.000000in}{-0.027778in}}{\pgfqpoint{0.000000in}{0.000000in}}{%
\pgfpathmoveto{\pgfqpoint{0.000000in}{0.000000in}}%
\pgfpathlineto{\pgfqpoint{0.000000in}{-0.027778in}}%
\pgfusepath{stroke,fill}%
}%
\begin{pgfscope}%
\pgfsys@transformshift{1.123152in}{0.417642in}%
\pgfsys@useobject{currentmarker}{}%
\end{pgfscope}%
\end{pgfscope}%
\begin{pgfscope}%
\pgfpathrectangle{\pgfqpoint{0.594525in}{0.417642in}}{\pgfqpoint{3.432047in}{2.016277in}}%
\pgfusepath{clip}%
\pgfsetrectcap%
\pgfsetroundjoin%
\pgfsetlinewidth{0.803000pt}%
\definecolor{currentstroke}{rgb}{0.850000,0.850000,0.850000}%
\pgfsetstrokecolor{currentstroke}%
\pgfsetdash{}{0pt}%
\pgfpathmoveto{\pgfqpoint{1.220728in}{0.417642in}}%
\pgfpathlineto{\pgfqpoint{1.220728in}{2.433919in}}%
\pgfusepath{stroke}%
\end{pgfscope}%
\begin{pgfscope}%
\pgfsetbuttcap%
\pgfsetroundjoin%
\definecolor{currentfill}{rgb}{0.000000,0.000000,0.000000}%
\pgfsetfillcolor{currentfill}%
\pgfsetlinewidth{0.602250pt}%
\definecolor{currentstroke}{rgb}{0.000000,0.000000,0.000000}%
\pgfsetstrokecolor{currentstroke}%
\pgfsetdash{}{0pt}%
\pgfsys@defobject{currentmarker}{\pgfqpoint{0.000000in}{-0.027778in}}{\pgfqpoint{0.000000in}{0.000000in}}{%
\pgfpathmoveto{\pgfqpoint{0.000000in}{0.000000in}}%
\pgfpathlineto{\pgfqpoint{0.000000in}{-0.027778in}}%
\pgfusepath{stroke,fill}%
}%
\begin{pgfscope}%
\pgfsys@transformshift{1.220728in}{0.417642in}%
\pgfsys@useobject{currentmarker}{}%
\end{pgfscope}%
\end{pgfscope}%
\begin{pgfscope}%
\pgfpathrectangle{\pgfqpoint{0.594525in}{0.417642in}}{\pgfqpoint{3.432047in}{2.016277in}}%
\pgfusepath{clip}%
\pgfsetrectcap%
\pgfsetroundjoin%
\pgfsetlinewidth{0.803000pt}%
\definecolor{currentstroke}{rgb}{0.850000,0.850000,0.850000}%
\pgfsetstrokecolor{currentstroke}%
\pgfsetdash{}{0pt}%
\pgfpathmoveto{\pgfqpoint{1.296413in}{0.417642in}}%
\pgfpathlineto{\pgfqpoint{1.296413in}{2.433919in}}%
\pgfusepath{stroke}%
\end{pgfscope}%
\begin{pgfscope}%
\pgfsetbuttcap%
\pgfsetroundjoin%
\definecolor{currentfill}{rgb}{0.000000,0.000000,0.000000}%
\pgfsetfillcolor{currentfill}%
\pgfsetlinewidth{0.602250pt}%
\definecolor{currentstroke}{rgb}{0.000000,0.000000,0.000000}%
\pgfsetstrokecolor{currentstroke}%
\pgfsetdash{}{0pt}%
\pgfsys@defobject{currentmarker}{\pgfqpoint{0.000000in}{-0.027778in}}{\pgfqpoint{0.000000in}{0.000000in}}{%
\pgfpathmoveto{\pgfqpoint{0.000000in}{0.000000in}}%
\pgfpathlineto{\pgfqpoint{0.000000in}{-0.027778in}}%
\pgfusepath{stroke,fill}%
}%
\begin{pgfscope}%
\pgfsys@transformshift{1.296413in}{0.417642in}%
\pgfsys@useobject{currentmarker}{}%
\end{pgfscope}%
\end{pgfscope}%
\begin{pgfscope}%
\pgfpathrectangle{\pgfqpoint{0.594525in}{0.417642in}}{\pgfqpoint{3.432047in}{2.016277in}}%
\pgfusepath{clip}%
\pgfsetrectcap%
\pgfsetroundjoin%
\pgfsetlinewidth{0.803000pt}%
\definecolor{currentstroke}{rgb}{0.850000,0.850000,0.850000}%
\pgfsetstrokecolor{currentstroke}%
\pgfsetdash{}{0pt}%
\pgfpathmoveto{\pgfqpoint{1.358253in}{0.417642in}}%
\pgfpathlineto{\pgfqpoint{1.358253in}{2.433919in}}%
\pgfusepath{stroke}%
\end{pgfscope}%
\begin{pgfscope}%
\pgfsetbuttcap%
\pgfsetroundjoin%
\definecolor{currentfill}{rgb}{0.000000,0.000000,0.000000}%
\pgfsetfillcolor{currentfill}%
\pgfsetlinewidth{0.602250pt}%
\definecolor{currentstroke}{rgb}{0.000000,0.000000,0.000000}%
\pgfsetstrokecolor{currentstroke}%
\pgfsetdash{}{0pt}%
\pgfsys@defobject{currentmarker}{\pgfqpoint{0.000000in}{-0.027778in}}{\pgfqpoint{0.000000in}{0.000000in}}{%
\pgfpathmoveto{\pgfqpoint{0.000000in}{0.000000in}}%
\pgfpathlineto{\pgfqpoint{0.000000in}{-0.027778in}}%
\pgfusepath{stroke,fill}%
}%
\begin{pgfscope}%
\pgfsys@transformshift{1.358253in}{0.417642in}%
\pgfsys@useobject{currentmarker}{}%
\end{pgfscope}%
\end{pgfscope}%
\begin{pgfscope}%
\pgfpathrectangle{\pgfqpoint{0.594525in}{0.417642in}}{\pgfqpoint{3.432047in}{2.016277in}}%
\pgfusepath{clip}%
\pgfsetrectcap%
\pgfsetroundjoin%
\pgfsetlinewidth{0.803000pt}%
\definecolor{currentstroke}{rgb}{0.850000,0.850000,0.850000}%
\pgfsetstrokecolor{currentstroke}%
\pgfsetdash{}{0pt}%
\pgfpathmoveto{\pgfqpoint{1.410538in}{0.417642in}}%
\pgfpathlineto{\pgfqpoint{1.410538in}{2.433919in}}%
\pgfusepath{stroke}%
\end{pgfscope}%
\begin{pgfscope}%
\pgfsetbuttcap%
\pgfsetroundjoin%
\definecolor{currentfill}{rgb}{0.000000,0.000000,0.000000}%
\pgfsetfillcolor{currentfill}%
\pgfsetlinewidth{0.602250pt}%
\definecolor{currentstroke}{rgb}{0.000000,0.000000,0.000000}%
\pgfsetstrokecolor{currentstroke}%
\pgfsetdash{}{0pt}%
\pgfsys@defobject{currentmarker}{\pgfqpoint{0.000000in}{-0.027778in}}{\pgfqpoint{0.000000in}{0.000000in}}{%
\pgfpathmoveto{\pgfqpoint{0.000000in}{0.000000in}}%
\pgfpathlineto{\pgfqpoint{0.000000in}{-0.027778in}}%
\pgfusepath{stroke,fill}%
}%
\begin{pgfscope}%
\pgfsys@transformshift{1.410538in}{0.417642in}%
\pgfsys@useobject{currentmarker}{}%
\end{pgfscope}%
\end{pgfscope}%
\begin{pgfscope}%
\pgfpathrectangle{\pgfqpoint{0.594525in}{0.417642in}}{\pgfqpoint{3.432047in}{2.016277in}}%
\pgfusepath{clip}%
\pgfsetrectcap%
\pgfsetroundjoin%
\pgfsetlinewidth{0.803000pt}%
\definecolor{currentstroke}{rgb}{0.850000,0.850000,0.850000}%
\pgfsetstrokecolor{currentstroke}%
\pgfsetdash{}{0pt}%
\pgfpathmoveto{\pgfqpoint{1.455829in}{0.417642in}}%
\pgfpathlineto{\pgfqpoint{1.455829in}{2.433919in}}%
\pgfusepath{stroke}%
\end{pgfscope}%
\begin{pgfscope}%
\pgfsetbuttcap%
\pgfsetroundjoin%
\definecolor{currentfill}{rgb}{0.000000,0.000000,0.000000}%
\pgfsetfillcolor{currentfill}%
\pgfsetlinewidth{0.602250pt}%
\definecolor{currentstroke}{rgb}{0.000000,0.000000,0.000000}%
\pgfsetstrokecolor{currentstroke}%
\pgfsetdash{}{0pt}%
\pgfsys@defobject{currentmarker}{\pgfqpoint{0.000000in}{-0.027778in}}{\pgfqpoint{0.000000in}{0.000000in}}{%
\pgfpathmoveto{\pgfqpoint{0.000000in}{0.000000in}}%
\pgfpathlineto{\pgfqpoint{0.000000in}{-0.027778in}}%
\pgfusepath{stroke,fill}%
}%
\begin{pgfscope}%
\pgfsys@transformshift{1.455829in}{0.417642in}%
\pgfsys@useobject{currentmarker}{}%
\end{pgfscope}%
\end{pgfscope}%
\begin{pgfscope}%
\pgfpathrectangle{\pgfqpoint{0.594525in}{0.417642in}}{\pgfqpoint{3.432047in}{2.016277in}}%
\pgfusepath{clip}%
\pgfsetrectcap%
\pgfsetroundjoin%
\pgfsetlinewidth{0.803000pt}%
\definecolor{currentstroke}{rgb}{0.850000,0.850000,0.850000}%
\pgfsetstrokecolor{currentstroke}%
\pgfsetdash{}{0pt}%
\pgfpathmoveto{\pgfqpoint{1.495778in}{0.417642in}}%
\pgfpathlineto{\pgfqpoint{1.495778in}{2.433919in}}%
\pgfusepath{stroke}%
\end{pgfscope}%
\begin{pgfscope}%
\pgfsetbuttcap%
\pgfsetroundjoin%
\definecolor{currentfill}{rgb}{0.000000,0.000000,0.000000}%
\pgfsetfillcolor{currentfill}%
\pgfsetlinewidth{0.602250pt}%
\definecolor{currentstroke}{rgb}{0.000000,0.000000,0.000000}%
\pgfsetstrokecolor{currentstroke}%
\pgfsetdash{}{0pt}%
\pgfsys@defobject{currentmarker}{\pgfqpoint{0.000000in}{-0.027778in}}{\pgfqpoint{0.000000in}{0.000000in}}{%
\pgfpathmoveto{\pgfqpoint{0.000000in}{0.000000in}}%
\pgfpathlineto{\pgfqpoint{0.000000in}{-0.027778in}}%
\pgfusepath{stroke,fill}%
}%
\begin{pgfscope}%
\pgfsys@transformshift{1.495778in}{0.417642in}%
\pgfsys@useobject{currentmarker}{}%
\end{pgfscope}%
\end{pgfscope}%
\begin{pgfscope}%
\pgfpathrectangle{\pgfqpoint{0.594525in}{0.417642in}}{\pgfqpoint{3.432047in}{2.016277in}}%
\pgfusepath{clip}%
\pgfsetrectcap%
\pgfsetroundjoin%
\pgfsetlinewidth{0.803000pt}%
\definecolor{currentstroke}{rgb}{0.850000,0.850000,0.850000}%
\pgfsetstrokecolor{currentstroke}%
\pgfsetdash{}{0pt}%
\pgfpathmoveto{\pgfqpoint{1.766615in}{0.417642in}}%
\pgfpathlineto{\pgfqpoint{1.766615in}{2.433919in}}%
\pgfusepath{stroke}%
\end{pgfscope}%
\begin{pgfscope}%
\pgfsetbuttcap%
\pgfsetroundjoin%
\definecolor{currentfill}{rgb}{0.000000,0.000000,0.000000}%
\pgfsetfillcolor{currentfill}%
\pgfsetlinewidth{0.602250pt}%
\definecolor{currentstroke}{rgb}{0.000000,0.000000,0.000000}%
\pgfsetstrokecolor{currentstroke}%
\pgfsetdash{}{0pt}%
\pgfsys@defobject{currentmarker}{\pgfqpoint{0.000000in}{-0.027778in}}{\pgfqpoint{0.000000in}{0.000000in}}{%
\pgfpathmoveto{\pgfqpoint{0.000000in}{0.000000in}}%
\pgfpathlineto{\pgfqpoint{0.000000in}{-0.027778in}}%
\pgfusepath{stroke,fill}%
}%
\begin{pgfscope}%
\pgfsys@transformshift{1.766615in}{0.417642in}%
\pgfsys@useobject{currentmarker}{}%
\end{pgfscope}%
\end{pgfscope}%
\begin{pgfscope}%
\pgfpathrectangle{\pgfqpoint{0.594525in}{0.417642in}}{\pgfqpoint{3.432047in}{2.016277in}}%
\pgfusepath{clip}%
\pgfsetrectcap%
\pgfsetroundjoin%
\pgfsetlinewidth{0.803000pt}%
\definecolor{currentstroke}{rgb}{0.850000,0.850000,0.850000}%
\pgfsetstrokecolor{currentstroke}%
\pgfsetdash{}{0pt}%
\pgfpathmoveto{\pgfqpoint{1.904140in}{0.417642in}}%
\pgfpathlineto{\pgfqpoint{1.904140in}{2.433919in}}%
\pgfusepath{stroke}%
\end{pgfscope}%
\begin{pgfscope}%
\pgfsetbuttcap%
\pgfsetroundjoin%
\definecolor{currentfill}{rgb}{0.000000,0.000000,0.000000}%
\pgfsetfillcolor{currentfill}%
\pgfsetlinewidth{0.602250pt}%
\definecolor{currentstroke}{rgb}{0.000000,0.000000,0.000000}%
\pgfsetstrokecolor{currentstroke}%
\pgfsetdash{}{0pt}%
\pgfsys@defobject{currentmarker}{\pgfqpoint{0.000000in}{-0.027778in}}{\pgfqpoint{0.000000in}{0.000000in}}{%
\pgfpathmoveto{\pgfqpoint{0.000000in}{0.000000in}}%
\pgfpathlineto{\pgfqpoint{0.000000in}{-0.027778in}}%
\pgfusepath{stroke,fill}%
}%
\begin{pgfscope}%
\pgfsys@transformshift{1.904140in}{0.417642in}%
\pgfsys@useobject{currentmarker}{}%
\end{pgfscope}%
\end{pgfscope}%
\begin{pgfscope}%
\pgfpathrectangle{\pgfqpoint{0.594525in}{0.417642in}}{\pgfqpoint{3.432047in}{2.016277in}}%
\pgfusepath{clip}%
\pgfsetrectcap%
\pgfsetroundjoin%
\pgfsetlinewidth{0.803000pt}%
\definecolor{currentstroke}{rgb}{0.850000,0.850000,0.850000}%
\pgfsetstrokecolor{currentstroke}%
\pgfsetdash{}{0pt}%
\pgfpathmoveto{\pgfqpoint{2.001715in}{0.417642in}}%
\pgfpathlineto{\pgfqpoint{2.001715in}{2.433919in}}%
\pgfusepath{stroke}%
\end{pgfscope}%
\begin{pgfscope}%
\pgfsetbuttcap%
\pgfsetroundjoin%
\definecolor{currentfill}{rgb}{0.000000,0.000000,0.000000}%
\pgfsetfillcolor{currentfill}%
\pgfsetlinewidth{0.602250pt}%
\definecolor{currentstroke}{rgb}{0.000000,0.000000,0.000000}%
\pgfsetstrokecolor{currentstroke}%
\pgfsetdash{}{0pt}%
\pgfsys@defobject{currentmarker}{\pgfqpoint{0.000000in}{-0.027778in}}{\pgfqpoint{0.000000in}{0.000000in}}{%
\pgfpathmoveto{\pgfqpoint{0.000000in}{0.000000in}}%
\pgfpathlineto{\pgfqpoint{0.000000in}{-0.027778in}}%
\pgfusepath{stroke,fill}%
}%
\begin{pgfscope}%
\pgfsys@transformshift{2.001715in}{0.417642in}%
\pgfsys@useobject{currentmarker}{}%
\end{pgfscope}%
\end{pgfscope}%
\begin{pgfscope}%
\pgfpathrectangle{\pgfqpoint{0.594525in}{0.417642in}}{\pgfqpoint{3.432047in}{2.016277in}}%
\pgfusepath{clip}%
\pgfsetrectcap%
\pgfsetroundjoin%
\pgfsetlinewidth{0.803000pt}%
\definecolor{currentstroke}{rgb}{0.850000,0.850000,0.850000}%
\pgfsetstrokecolor{currentstroke}%
\pgfsetdash{}{0pt}%
\pgfpathmoveto{\pgfqpoint{2.077401in}{0.417642in}}%
\pgfpathlineto{\pgfqpoint{2.077401in}{2.433919in}}%
\pgfusepath{stroke}%
\end{pgfscope}%
\begin{pgfscope}%
\pgfsetbuttcap%
\pgfsetroundjoin%
\definecolor{currentfill}{rgb}{0.000000,0.000000,0.000000}%
\pgfsetfillcolor{currentfill}%
\pgfsetlinewidth{0.602250pt}%
\definecolor{currentstroke}{rgb}{0.000000,0.000000,0.000000}%
\pgfsetstrokecolor{currentstroke}%
\pgfsetdash{}{0pt}%
\pgfsys@defobject{currentmarker}{\pgfqpoint{0.000000in}{-0.027778in}}{\pgfqpoint{0.000000in}{0.000000in}}{%
\pgfpathmoveto{\pgfqpoint{0.000000in}{0.000000in}}%
\pgfpathlineto{\pgfqpoint{0.000000in}{-0.027778in}}%
\pgfusepath{stroke,fill}%
}%
\begin{pgfscope}%
\pgfsys@transformshift{2.077401in}{0.417642in}%
\pgfsys@useobject{currentmarker}{}%
\end{pgfscope}%
\end{pgfscope}%
\begin{pgfscope}%
\pgfpathrectangle{\pgfqpoint{0.594525in}{0.417642in}}{\pgfqpoint{3.432047in}{2.016277in}}%
\pgfusepath{clip}%
\pgfsetrectcap%
\pgfsetroundjoin%
\pgfsetlinewidth{0.803000pt}%
\definecolor{currentstroke}{rgb}{0.850000,0.850000,0.850000}%
\pgfsetstrokecolor{currentstroke}%
\pgfsetdash{}{0pt}%
\pgfpathmoveto{\pgfqpoint{2.139240in}{0.417642in}}%
\pgfpathlineto{\pgfqpoint{2.139240in}{2.433919in}}%
\pgfusepath{stroke}%
\end{pgfscope}%
\begin{pgfscope}%
\pgfsetbuttcap%
\pgfsetroundjoin%
\definecolor{currentfill}{rgb}{0.000000,0.000000,0.000000}%
\pgfsetfillcolor{currentfill}%
\pgfsetlinewidth{0.602250pt}%
\definecolor{currentstroke}{rgb}{0.000000,0.000000,0.000000}%
\pgfsetstrokecolor{currentstroke}%
\pgfsetdash{}{0pt}%
\pgfsys@defobject{currentmarker}{\pgfqpoint{0.000000in}{-0.027778in}}{\pgfqpoint{0.000000in}{0.000000in}}{%
\pgfpathmoveto{\pgfqpoint{0.000000in}{0.000000in}}%
\pgfpathlineto{\pgfqpoint{0.000000in}{-0.027778in}}%
\pgfusepath{stroke,fill}%
}%
\begin{pgfscope}%
\pgfsys@transformshift{2.139240in}{0.417642in}%
\pgfsys@useobject{currentmarker}{}%
\end{pgfscope}%
\end{pgfscope}%
\begin{pgfscope}%
\pgfpathrectangle{\pgfqpoint{0.594525in}{0.417642in}}{\pgfqpoint{3.432047in}{2.016277in}}%
\pgfusepath{clip}%
\pgfsetrectcap%
\pgfsetroundjoin%
\pgfsetlinewidth{0.803000pt}%
\definecolor{currentstroke}{rgb}{0.850000,0.850000,0.850000}%
\pgfsetstrokecolor{currentstroke}%
\pgfsetdash{}{0pt}%
\pgfpathmoveto{\pgfqpoint{2.191525in}{0.417642in}}%
\pgfpathlineto{\pgfqpoint{2.191525in}{2.433919in}}%
\pgfusepath{stroke}%
\end{pgfscope}%
\begin{pgfscope}%
\pgfsetbuttcap%
\pgfsetroundjoin%
\definecolor{currentfill}{rgb}{0.000000,0.000000,0.000000}%
\pgfsetfillcolor{currentfill}%
\pgfsetlinewidth{0.602250pt}%
\definecolor{currentstroke}{rgb}{0.000000,0.000000,0.000000}%
\pgfsetstrokecolor{currentstroke}%
\pgfsetdash{}{0pt}%
\pgfsys@defobject{currentmarker}{\pgfqpoint{0.000000in}{-0.027778in}}{\pgfqpoint{0.000000in}{0.000000in}}{%
\pgfpathmoveto{\pgfqpoint{0.000000in}{0.000000in}}%
\pgfpathlineto{\pgfqpoint{0.000000in}{-0.027778in}}%
\pgfusepath{stroke,fill}%
}%
\begin{pgfscope}%
\pgfsys@transformshift{2.191525in}{0.417642in}%
\pgfsys@useobject{currentmarker}{}%
\end{pgfscope}%
\end{pgfscope}%
\begin{pgfscope}%
\pgfpathrectangle{\pgfqpoint{0.594525in}{0.417642in}}{\pgfqpoint{3.432047in}{2.016277in}}%
\pgfusepath{clip}%
\pgfsetrectcap%
\pgfsetroundjoin%
\pgfsetlinewidth{0.803000pt}%
\definecolor{currentstroke}{rgb}{0.850000,0.850000,0.850000}%
\pgfsetstrokecolor{currentstroke}%
\pgfsetdash{}{0pt}%
\pgfpathmoveto{\pgfqpoint{2.236816in}{0.417642in}}%
\pgfpathlineto{\pgfqpoint{2.236816in}{2.433919in}}%
\pgfusepath{stroke}%
\end{pgfscope}%
\begin{pgfscope}%
\pgfsetbuttcap%
\pgfsetroundjoin%
\definecolor{currentfill}{rgb}{0.000000,0.000000,0.000000}%
\pgfsetfillcolor{currentfill}%
\pgfsetlinewidth{0.602250pt}%
\definecolor{currentstroke}{rgb}{0.000000,0.000000,0.000000}%
\pgfsetstrokecolor{currentstroke}%
\pgfsetdash{}{0pt}%
\pgfsys@defobject{currentmarker}{\pgfqpoint{0.000000in}{-0.027778in}}{\pgfqpoint{0.000000in}{0.000000in}}{%
\pgfpathmoveto{\pgfqpoint{0.000000in}{0.000000in}}%
\pgfpathlineto{\pgfqpoint{0.000000in}{-0.027778in}}%
\pgfusepath{stroke,fill}%
}%
\begin{pgfscope}%
\pgfsys@transformshift{2.236816in}{0.417642in}%
\pgfsys@useobject{currentmarker}{}%
\end{pgfscope}%
\end{pgfscope}%
\begin{pgfscope}%
\pgfpathrectangle{\pgfqpoint{0.594525in}{0.417642in}}{\pgfqpoint{3.432047in}{2.016277in}}%
\pgfusepath{clip}%
\pgfsetrectcap%
\pgfsetroundjoin%
\pgfsetlinewidth{0.803000pt}%
\definecolor{currentstroke}{rgb}{0.850000,0.850000,0.850000}%
\pgfsetstrokecolor{currentstroke}%
\pgfsetdash{}{0pt}%
\pgfpathmoveto{\pgfqpoint{2.276765in}{0.417642in}}%
\pgfpathlineto{\pgfqpoint{2.276765in}{2.433919in}}%
\pgfusepath{stroke}%
\end{pgfscope}%
\begin{pgfscope}%
\pgfsetbuttcap%
\pgfsetroundjoin%
\definecolor{currentfill}{rgb}{0.000000,0.000000,0.000000}%
\pgfsetfillcolor{currentfill}%
\pgfsetlinewidth{0.602250pt}%
\definecolor{currentstroke}{rgb}{0.000000,0.000000,0.000000}%
\pgfsetstrokecolor{currentstroke}%
\pgfsetdash{}{0pt}%
\pgfsys@defobject{currentmarker}{\pgfqpoint{0.000000in}{-0.027778in}}{\pgfqpoint{0.000000in}{0.000000in}}{%
\pgfpathmoveto{\pgfqpoint{0.000000in}{0.000000in}}%
\pgfpathlineto{\pgfqpoint{0.000000in}{-0.027778in}}%
\pgfusepath{stroke,fill}%
}%
\begin{pgfscope}%
\pgfsys@transformshift{2.276765in}{0.417642in}%
\pgfsys@useobject{currentmarker}{}%
\end{pgfscope}%
\end{pgfscope}%
\begin{pgfscope}%
\pgfpathrectangle{\pgfqpoint{0.594525in}{0.417642in}}{\pgfqpoint{3.432047in}{2.016277in}}%
\pgfusepath{clip}%
\pgfsetrectcap%
\pgfsetroundjoin%
\pgfsetlinewidth{0.803000pt}%
\definecolor{currentstroke}{rgb}{0.850000,0.850000,0.850000}%
\pgfsetstrokecolor{currentstroke}%
\pgfsetdash{}{0pt}%
\pgfpathmoveto{\pgfqpoint{2.547602in}{0.417642in}}%
\pgfpathlineto{\pgfqpoint{2.547602in}{2.433919in}}%
\pgfusepath{stroke}%
\end{pgfscope}%
\begin{pgfscope}%
\pgfsetbuttcap%
\pgfsetroundjoin%
\definecolor{currentfill}{rgb}{0.000000,0.000000,0.000000}%
\pgfsetfillcolor{currentfill}%
\pgfsetlinewidth{0.602250pt}%
\definecolor{currentstroke}{rgb}{0.000000,0.000000,0.000000}%
\pgfsetstrokecolor{currentstroke}%
\pgfsetdash{}{0pt}%
\pgfsys@defobject{currentmarker}{\pgfqpoint{0.000000in}{-0.027778in}}{\pgfqpoint{0.000000in}{0.000000in}}{%
\pgfpathmoveto{\pgfqpoint{0.000000in}{0.000000in}}%
\pgfpathlineto{\pgfqpoint{0.000000in}{-0.027778in}}%
\pgfusepath{stroke,fill}%
}%
\begin{pgfscope}%
\pgfsys@transformshift{2.547602in}{0.417642in}%
\pgfsys@useobject{currentmarker}{}%
\end{pgfscope}%
\end{pgfscope}%
\begin{pgfscope}%
\pgfpathrectangle{\pgfqpoint{0.594525in}{0.417642in}}{\pgfqpoint{3.432047in}{2.016277in}}%
\pgfusepath{clip}%
\pgfsetrectcap%
\pgfsetroundjoin%
\pgfsetlinewidth{0.803000pt}%
\definecolor{currentstroke}{rgb}{0.850000,0.850000,0.850000}%
\pgfsetstrokecolor{currentstroke}%
\pgfsetdash{}{0pt}%
\pgfpathmoveto{\pgfqpoint{2.685127in}{0.417642in}}%
\pgfpathlineto{\pgfqpoint{2.685127in}{2.433919in}}%
\pgfusepath{stroke}%
\end{pgfscope}%
\begin{pgfscope}%
\pgfsetbuttcap%
\pgfsetroundjoin%
\definecolor{currentfill}{rgb}{0.000000,0.000000,0.000000}%
\pgfsetfillcolor{currentfill}%
\pgfsetlinewidth{0.602250pt}%
\definecolor{currentstroke}{rgb}{0.000000,0.000000,0.000000}%
\pgfsetstrokecolor{currentstroke}%
\pgfsetdash{}{0pt}%
\pgfsys@defobject{currentmarker}{\pgfqpoint{0.000000in}{-0.027778in}}{\pgfqpoint{0.000000in}{0.000000in}}{%
\pgfpathmoveto{\pgfqpoint{0.000000in}{0.000000in}}%
\pgfpathlineto{\pgfqpoint{0.000000in}{-0.027778in}}%
\pgfusepath{stroke,fill}%
}%
\begin{pgfscope}%
\pgfsys@transformshift{2.685127in}{0.417642in}%
\pgfsys@useobject{currentmarker}{}%
\end{pgfscope}%
\end{pgfscope}%
\begin{pgfscope}%
\pgfpathrectangle{\pgfqpoint{0.594525in}{0.417642in}}{\pgfqpoint{3.432047in}{2.016277in}}%
\pgfusepath{clip}%
\pgfsetrectcap%
\pgfsetroundjoin%
\pgfsetlinewidth{0.803000pt}%
\definecolor{currentstroke}{rgb}{0.850000,0.850000,0.850000}%
\pgfsetstrokecolor{currentstroke}%
\pgfsetdash{}{0pt}%
\pgfpathmoveto{\pgfqpoint{2.782702in}{0.417642in}}%
\pgfpathlineto{\pgfqpoint{2.782702in}{2.433919in}}%
\pgfusepath{stroke}%
\end{pgfscope}%
\begin{pgfscope}%
\pgfsetbuttcap%
\pgfsetroundjoin%
\definecolor{currentfill}{rgb}{0.000000,0.000000,0.000000}%
\pgfsetfillcolor{currentfill}%
\pgfsetlinewidth{0.602250pt}%
\definecolor{currentstroke}{rgb}{0.000000,0.000000,0.000000}%
\pgfsetstrokecolor{currentstroke}%
\pgfsetdash{}{0pt}%
\pgfsys@defobject{currentmarker}{\pgfqpoint{0.000000in}{-0.027778in}}{\pgfqpoint{0.000000in}{0.000000in}}{%
\pgfpathmoveto{\pgfqpoint{0.000000in}{0.000000in}}%
\pgfpathlineto{\pgfqpoint{0.000000in}{-0.027778in}}%
\pgfusepath{stroke,fill}%
}%
\begin{pgfscope}%
\pgfsys@transformshift{2.782702in}{0.417642in}%
\pgfsys@useobject{currentmarker}{}%
\end{pgfscope}%
\end{pgfscope}%
\begin{pgfscope}%
\pgfpathrectangle{\pgfqpoint{0.594525in}{0.417642in}}{\pgfqpoint{3.432047in}{2.016277in}}%
\pgfusepath{clip}%
\pgfsetrectcap%
\pgfsetroundjoin%
\pgfsetlinewidth{0.803000pt}%
\definecolor{currentstroke}{rgb}{0.850000,0.850000,0.850000}%
\pgfsetstrokecolor{currentstroke}%
\pgfsetdash{}{0pt}%
\pgfpathmoveto{\pgfqpoint{2.858388in}{0.417642in}}%
\pgfpathlineto{\pgfqpoint{2.858388in}{2.433919in}}%
\pgfusepath{stroke}%
\end{pgfscope}%
\begin{pgfscope}%
\pgfsetbuttcap%
\pgfsetroundjoin%
\definecolor{currentfill}{rgb}{0.000000,0.000000,0.000000}%
\pgfsetfillcolor{currentfill}%
\pgfsetlinewidth{0.602250pt}%
\definecolor{currentstroke}{rgb}{0.000000,0.000000,0.000000}%
\pgfsetstrokecolor{currentstroke}%
\pgfsetdash{}{0pt}%
\pgfsys@defobject{currentmarker}{\pgfqpoint{0.000000in}{-0.027778in}}{\pgfqpoint{0.000000in}{0.000000in}}{%
\pgfpathmoveto{\pgfqpoint{0.000000in}{0.000000in}}%
\pgfpathlineto{\pgfqpoint{0.000000in}{-0.027778in}}%
\pgfusepath{stroke,fill}%
}%
\begin{pgfscope}%
\pgfsys@transformshift{2.858388in}{0.417642in}%
\pgfsys@useobject{currentmarker}{}%
\end{pgfscope}%
\end{pgfscope}%
\begin{pgfscope}%
\pgfpathrectangle{\pgfqpoint{0.594525in}{0.417642in}}{\pgfqpoint{3.432047in}{2.016277in}}%
\pgfusepath{clip}%
\pgfsetrectcap%
\pgfsetroundjoin%
\pgfsetlinewidth{0.803000pt}%
\definecolor{currentstroke}{rgb}{0.850000,0.850000,0.850000}%
\pgfsetstrokecolor{currentstroke}%
\pgfsetdash{}{0pt}%
\pgfpathmoveto{\pgfqpoint{2.920227in}{0.417642in}}%
\pgfpathlineto{\pgfqpoint{2.920227in}{2.433919in}}%
\pgfusepath{stroke}%
\end{pgfscope}%
\begin{pgfscope}%
\pgfsetbuttcap%
\pgfsetroundjoin%
\definecolor{currentfill}{rgb}{0.000000,0.000000,0.000000}%
\pgfsetfillcolor{currentfill}%
\pgfsetlinewidth{0.602250pt}%
\definecolor{currentstroke}{rgb}{0.000000,0.000000,0.000000}%
\pgfsetstrokecolor{currentstroke}%
\pgfsetdash{}{0pt}%
\pgfsys@defobject{currentmarker}{\pgfqpoint{0.000000in}{-0.027778in}}{\pgfqpoint{0.000000in}{0.000000in}}{%
\pgfpathmoveto{\pgfqpoint{0.000000in}{0.000000in}}%
\pgfpathlineto{\pgfqpoint{0.000000in}{-0.027778in}}%
\pgfusepath{stroke,fill}%
}%
\begin{pgfscope}%
\pgfsys@transformshift{2.920227in}{0.417642in}%
\pgfsys@useobject{currentmarker}{}%
\end{pgfscope}%
\end{pgfscope}%
\begin{pgfscope}%
\pgfpathrectangle{\pgfqpoint{0.594525in}{0.417642in}}{\pgfqpoint{3.432047in}{2.016277in}}%
\pgfusepath{clip}%
\pgfsetrectcap%
\pgfsetroundjoin%
\pgfsetlinewidth{0.803000pt}%
\definecolor{currentstroke}{rgb}{0.850000,0.850000,0.850000}%
\pgfsetstrokecolor{currentstroke}%
\pgfsetdash{}{0pt}%
\pgfpathmoveto{\pgfqpoint{2.972512in}{0.417642in}}%
\pgfpathlineto{\pgfqpoint{2.972512in}{2.433919in}}%
\pgfusepath{stroke}%
\end{pgfscope}%
\begin{pgfscope}%
\pgfsetbuttcap%
\pgfsetroundjoin%
\definecolor{currentfill}{rgb}{0.000000,0.000000,0.000000}%
\pgfsetfillcolor{currentfill}%
\pgfsetlinewidth{0.602250pt}%
\definecolor{currentstroke}{rgb}{0.000000,0.000000,0.000000}%
\pgfsetstrokecolor{currentstroke}%
\pgfsetdash{}{0pt}%
\pgfsys@defobject{currentmarker}{\pgfqpoint{0.000000in}{-0.027778in}}{\pgfqpoint{0.000000in}{0.000000in}}{%
\pgfpathmoveto{\pgfqpoint{0.000000in}{0.000000in}}%
\pgfpathlineto{\pgfqpoint{0.000000in}{-0.027778in}}%
\pgfusepath{stroke,fill}%
}%
\begin{pgfscope}%
\pgfsys@transformshift{2.972512in}{0.417642in}%
\pgfsys@useobject{currentmarker}{}%
\end{pgfscope}%
\end{pgfscope}%
\begin{pgfscope}%
\pgfpathrectangle{\pgfqpoint{0.594525in}{0.417642in}}{\pgfqpoint{3.432047in}{2.016277in}}%
\pgfusepath{clip}%
\pgfsetrectcap%
\pgfsetroundjoin%
\pgfsetlinewidth{0.803000pt}%
\definecolor{currentstroke}{rgb}{0.850000,0.850000,0.850000}%
\pgfsetstrokecolor{currentstroke}%
\pgfsetdash{}{0pt}%
\pgfpathmoveto{\pgfqpoint{3.017803in}{0.417642in}}%
\pgfpathlineto{\pgfqpoint{3.017803in}{2.433919in}}%
\pgfusepath{stroke}%
\end{pgfscope}%
\begin{pgfscope}%
\pgfsetbuttcap%
\pgfsetroundjoin%
\definecolor{currentfill}{rgb}{0.000000,0.000000,0.000000}%
\pgfsetfillcolor{currentfill}%
\pgfsetlinewidth{0.602250pt}%
\definecolor{currentstroke}{rgb}{0.000000,0.000000,0.000000}%
\pgfsetstrokecolor{currentstroke}%
\pgfsetdash{}{0pt}%
\pgfsys@defobject{currentmarker}{\pgfqpoint{0.000000in}{-0.027778in}}{\pgfqpoint{0.000000in}{0.000000in}}{%
\pgfpathmoveto{\pgfqpoint{0.000000in}{0.000000in}}%
\pgfpathlineto{\pgfqpoint{0.000000in}{-0.027778in}}%
\pgfusepath{stroke,fill}%
}%
\begin{pgfscope}%
\pgfsys@transformshift{3.017803in}{0.417642in}%
\pgfsys@useobject{currentmarker}{}%
\end{pgfscope}%
\end{pgfscope}%
\begin{pgfscope}%
\pgfpathrectangle{\pgfqpoint{0.594525in}{0.417642in}}{\pgfqpoint{3.432047in}{2.016277in}}%
\pgfusepath{clip}%
\pgfsetrectcap%
\pgfsetroundjoin%
\pgfsetlinewidth{0.803000pt}%
\definecolor{currentstroke}{rgb}{0.850000,0.850000,0.850000}%
\pgfsetstrokecolor{currentstroke}%
\pgfsetdash{}{0pt}%
\pgfpathmoveto{\pgfqpoint{3.057752in}{0.417642in}}%
\pgfpathlineto{\pgfqpoint{3.057752in}{2.433919in}}%
\pgfusepath{stroke}%
\end{pgfscope}%
\begin{pgfscope}%
\pgfsetbuttcap%
\pgfsetroundjoin%
\definecolor{currentfill}{rgb}{0.000000,0.000000,0.000000}%
\pgfsetfillcolor{currentfill}%
\pgfsetlinewidth{0.602250pt}%
\definecolor{currentstroke}{rgb}{0.000000,0.000000,0.000000}%
\pgfsetstrokecolor{currentstroke}%
\pgfsetdash{}{0pt}%
\pgfsys@defobject{currentmarker}{\pgfqpoint{0.000000in}{-0.027778in}}{\pgfqpoint{0.000000in}{0.000000in}}{%
\pgfpathmoveto{\pgfqpoint{0.000000in}{0.000000in}}%
\pgfpathlineto{\pgfqpoint{0.000000in}{-0.027778in}}%
\pgfusepath{stroke,fill}%
}%
\begin{pgfscope}%
\pgfsys@transformshift{3.057752in}{0.417642in}%
\pgfsys@useobject{currentmarker}{}%
\end{pgfscope}%
\end{pgfscope}%
\begin{pgfscope}%
\pgfpathrectangle{\pgfqpoint{0.594525in}{0.417642in}}{\pgfqpoint{3.432047in}{2.016277in}}%
\pgfusepath{clip}%
\pgfsetrectcap%
\pgfsetroundjoin%
\pgfsetlinewidth{0.803000pt}%
\definecolor{currentstroke}{rgb}{0.850000,0.850000,0.850000}%
\pgfsetstrokecolor{currentstroke}%
\pgfsetdash{}{0pt}%
\pgfpathmoveto{\pgfqpoint{3.328589in}{0.417642in}}%
\pgfpathlineto{\pgfqpoint{3.328589in}{2.433919in}}%
\pgfusepath{stroke}%
\end{pgfscope}%
\begin{pgfscope}%
\pgfsetbuttcap%
\pgfsetroundjoin%
\definecolor{currentfill}{rgb}{0.000000,0.000000,0.000000}%
\pgfsetfillcolor{currentfill}%
\pgfsetlinewidth{0.602250pt}%
\definecolor{currentstroke}{rgb}{0.000000,0.000000,0.000000}%
\pgfsetstrokecolor{currentstroke}%
\pgfsetdash{}{0pt}%
\pgfsys@defobject{currentmarker}{\pgfqpoint{0.000000in}{-0.027778in}}{\pgfqpoint{0.000000in}{0.000000in}}{%
\pgfpathmoveto{\pgfqpoint{0.000000in}{0.000000in}}%
\pgfpathlineto{\pgfqpoint{0.000000in}{-0.027778in}}%
\pgfusepath{stroke,fill}%
}%
\begin{pgfscope}%
\pgfsys@transformshift{3.328589in}{0.417642in}%
\pgfsys@useobject{currentmarker}{}%
\end{pgfscope}%
\end{pgfscope}%
\begin{pgfscope}%
\pgfpathrectangle{\pgfqpoint{0.594525in}{0.417642in}}{\pgfqpoint{3.432047in}{2.016277in}}%
\pgfusepath{clip}%
\pgfsetrectcap%
\pgfsetroundjoin%
\pgfsetlinewidth{0.803000pt}%
\definecolor{currentstroke}{rgb}{0.850000,0.850000,0.850000}%
\pgfsetstrokecolor{currentstroke}%
\pgfsetdash{}{0pt}%
\pgfpathmoveto{\pgfqpoint{3.466114in}{0.417642in}}%
\pgfpathlineto{\pgfqpoint{3.466114in}{2.433919in}}%
\pgfusepath{stroke}%
\end{pgfscope}%
\begin{pgfscope}%
\pgfsetbuttcap%
\pgfsetroundjoin%
\definecolor{currentfill}{rgb}{0.000000,0.000000,0.000000}%
\pgfsetfillcolor{currentfill}%
\pgfsetlinewidth{0.602250pt}%
\definecolor{currentstroke}{rgb}{0.000000,0.000000,0.000000}%
\pgfsetstrokecolor{currentstroke}%
\pgfsetdash{}{0pt}%
\pgfsys@defobject{currentmarker}{\pgfqpoint{0.000000in}{-0.027778in}}{\pgfqpoint{0.000000in}{0.000000in}}{%
\pgfpathmoveto{\pgfqpoint{0.000000in}{0.000000in}}%
\pgfpathlineto{\pgfqpoint{0.000000in}{-0.027778in}}%
\pgfusepath{stroke,fill}%
}%
\begin{pgfscope}%
\pgfsys@transformshift{3.466114in}{0.417642in}%
\pgfsys@useobject{currentmarker}{}%
\end{pgfscope}%
\end{pgfscope}%
\begin{pgfscope}%
\pgfpathrectangle{\pgfqpoint{0.594525in}{0.417642in}}{\pgfqpoint{3.432047in}{2.016277in}}%
\pgfusepath{clip}%
\pgfsetrectcap%
\pgfsetroundjoin%
\pgfsetlinewidth{0.803000pt}%
\definecolor{currentstroke}{rgb}{0.850000,0.850000,0.850000}%
\pgfsetstrokecolor{currentstroke}%
\pgfsetdash{}{0pt}%
\pgfpathmoveto{\pgfqpoint{3.563689in}{0.417642in}}%
\pgfpathlineto{\pgfqpoint{3.563689in}{2.433919in}}%
\pgfusepath{stroke}%
\end{pgfscope}%
\begin{pgfscope}%
\pgfsetbuttcap%
\pgfsetroundjoin%
\definecolor{currentfill}{rgb}{0.000000,0.000000,0.000000}%
\pgfsetfillcolor{currentfill}%
\pgfsetlinewidth{0.602250pt}%
\definecolor{currentstroke}{rgb}{0.000000,0.000000,0.000000}%
\pgfsetstrokecolor{currentstroke}%
\pgfsetdash{}{0pt}%
\pgfsys@defobject{currentmarker}{\pgfqpoint{0.000000in}{-0.027778in}}{\pgfqpoint{0.000000in}{0.000000in}}{%
\pgfpathmoveto{\pgfqpoint{0.000000in}{0.000000in}}%
\pgfpathlineto{\pgfqpoint{0.000000in}{-0.027778in}}%
\pgfusepath{stroke,fill}%
}%
\begin{pgfscope}%
\pgfsys@transformshift{3.563689in}{0.417642in}%
\pgfsys@useobject{currentmarker}{}%
\end{pgfscope}%
\end{pgfscope}%
\begin{pgfscope}%
\pgfpathrectangle{\pgfqpoint{0.594525in}{0.417642in}}{\pgfqpoint{3.432047in}{2.016277in}}%
\pgfusepath{clip}%
\pgfsetrectcap%
\pgfsetroundjoin%
\pgfsetlinewidth{0.803000pt}%
\definecolor{currentstroke}{rgb}{0.850000,0.850000,0.850000}%
\pgfsetstrokecolor{currentstroke}%
\pgfsetdash{}{0pt}%
\pgfpathmoveto{\pgfqpoint{3.639375in}{0.417642in}}%
\pgfpathlineto{\pgfqpoint{3.639375in}{2.433919in}}%
\pgfusepath{stroke}%
\end{pgfscope}%
\begin{pgfscope}%
\pgfsetbuttcap%
\pgfsetroundjoin%
\definecolor{currentfill}{rgb}{0.000000,0.000000,0.000000}%
\pgfsetfillcolor{currentfill}%
\pgfsetlinewidth{0.602250pt}%
\definecolor{currentstroke}{rgb}{0.000000,0.000000,0.000000}%
\pgfsetstrokecolor{currentstroke}%
\pgfsetdash{}{0pt}%
\pgfsys@defobject{currentmarker}{\pgfqpoint{0.000000in}{-0.027778in}}{\pgfqpoint{0.000000in}{0.000000in}}{%
\pgfpathmoveto{\pgfqpoint{0.000000in}{0.000000in}}%
\pgfpathlineto{\pgfqpoint{0.000000in}{-0.027778in}}%
\pgfusepath{stroke,fill}%
}%
\begin{pgfscope}%
\pgfsys@transformshift{3.639375in}{0.417642in}%
\pgfsys@useobject{currentmarker}{}%
\end{pgfscope}%
\end{pgfscope}%
\begin{pgfscope}%
\pgfpathrectangle{\pgfqpoint{0.594525in}{0.417642in}}{\pgfqpoint{3.432047in}{2.016277in}}%
\pgfusepath{clip}%
\pgfsetrectcap%
\pgfsetroundjoin%
\pgfsetlinewidth{0.803000pt}%
\definecolor{currentstroke}{rgb}{0.850000,0.850000,0.850000}%
\pgfsetstrokecolor{currentstroke}%
\pgfsetdash{}{0pt}%
\pgfpathmoveto{\pgfqpoint{3.701214in}{0.417642in}}%
\pgfpathlineto{\pgfqpoint{3.701214in}{2.433919in}}%
\pgfusepath{stroke}%
\end{pgfscope}%
\begin{pgfscope}%
\pgfsetbuttcap%
\pgfsetroundjoin%
\definecolor{currentfill}{rgb}{0.000000,0.000000,0.000000}%
\pgfsetfillcolor{currentfill}%
\pgfsetlinewidth{0.602250pt}%
\definecolor{currentstroke}{rgb}{0.000000,0.000000,0.000000}%
\pgfsetstrokecolor{currentstroke}%
\pgfsetdash{}{0pt}%
\pgfsys@defobject{currentmarker}{\pgfqpoint{0.000000in}{-0.027778in}}{\pgfqpoint{0.000000in}{0.000000in}}{%
\pgfpathmoveto{\pgfqpoint{0.000000in}{0.000000in}}%
\pgfpathlineto{\pgfqpoint{0.000000in}{-0.027778in}}%
\pgfusepath{stroke,fill}%
}%
\begin{pgfscope}%
\pgfsys@transformshift{3.701214in}{0.417642in}%
\pgfsys@useobject{currentmarker}{}%
\end{pgfscope}%
\end{pgfscope}%
\begin{pgfscope}%
\pgfpathrectangle{\pgfqpoint{0.594525in}{0.417642in}}{\pgfqpoint{3.432047in}{2.016277in}}%
\pgfusepath{clip}%
\pgfsetrectcap%
\pgfsetroundjoin%
\pgfsetlinewidth{0.803000pt}%
\definecolor{currentstroke}{rgb}{0.850000,0.850000,0.850000}%
\pgfsetstrokecolor{currentstroke}%
\pgfsetdash{}{0pt}%
\pgfpathmoveto{\pgfqpoint{3.753499in}{0.417642in}}%
\pgfpathlineto{\pgfqpoint{3.753499in}{2.433919in}}%
\pgfusepath{stroke}%
\end{pgfscope}%
\begin{pgfscope}%
\pgfsetbuttcap%
\pgfsetroundjoin%
\definecolor{currentfill}{rgb}{0.000000,0.000000,0.000000}%
\pgfsetfillcolor{currentfill}%
\pgfsetlinewidth{0.602250pt}%
\definecolor{currentstroke}{rgb}{0.000000,0.000000,0.000000}%
\pgfsetstrokecolor{currentstroke}%
\pgfsetdash{}{0pt}%
\pgfsys@defobject{currentmarker}{\pgfqpoint{0.000000in}{-0.027778in}}{\pgfqpoint{0.000000in}{0.000000in}}{%
\pgfpathmoveto{\pgfqpoint{0.000000in}{0.000000in}}%
\pgfpathlineto{\pgfqpoint{0.000000in}{-0.027778in}}%
\pgfusepath{stroke,fill}%
}%
\begin{pgfscope}%
\pgfsys@transformshift{3.753499in}{0.417642in}%
\pgfsys@useobject{currentmarker}{}%
\end{pgfscope}%
\end{pgfscope}%
\begin{pgfscope}%
\pgfpathrectangle{\pgfqpoint{0.594525in}{0.417642in}}{\pgfqpoint{3.432047in}{2.016277in}}%
\pgfusepath{clip}%
\pgfsetrectcap%
\pgfsetroundjoin%
\pgfsetlinewidth{0.803000pt}%
\definecolor{currentstroke}{rgb}{0.850000,0.850000,0.850000}%
\pgfsetstrokecolor{currentstroke}%
\pgfsetdash{}{0pt}%
\pgfpathmoveto{\pgfqpoint{3.798790in}{0.417642in}}%
\pgfpathlineto{\pgfqpoint{3.798790in}{2.433919in}}%
\pgfusepath{stroke}%
\end{pgfscope}%
\begin{pgfscope}%
\pgfsetbuttcap%
\pgfsetroundjoin%
\definecolor{currentfill}{rgb}{0.000000,0.000000,0.000000}%
\pgfsetfillcolor{currentfill}%
\pgfsetlinewidth{0.602250pt}%
\definecolor{currentstroke}{rgb}{0.000000,0.000000,0.000000}%
\pgfsetstrokecolor{currentstroke}%
\pgfsetdash{}{0pt}%
\pgfsys@defobject{currentmarker}{\pgfqpoint{0.000000in}{-0.027778in}}{\pgfqpoint{0.000000in}{0.000000in}}{%
\pgfpathmoveto{\pgfqpoint{0.000000in}{0.000000in}}%
\pgfpathlineto{\pgfqpoint{0.000000in}{-0.027778in}}%
\pgfusepath{stroke,fill}%
}%
\begin{pgfscope}%
\pgfsys@transformshift{3.798790in}{0.417642in}%
\pgfsys@useobject{currentmarker}{}%
\end{pgfscope}%
\end{pgfscope}%
\begin{pgfscope}%
\pgfpathrectangle{\pgfqpoint{0.594525in}{0.417642in}}{\pgfqpoint{3.432047in}{2.016277in}}%
\pgfusepath{clip}%
\pgfsetrectcap%
\pgfsetroundjoin%
\pgfsetlinewidth{0.803000pt}%
\definecolor{currentstroke}{rgb}{0.850000,0.850000,0.850000}%
\pgfsetstrokecolor{currentstroke}%
\pgfsetdash{}{0pt}%
\pgfpathmoveto{\pgfqpoint{3.838739in}{0.417642in}}%
\pgfpathlineto{\pgfqpoint{3.838739in}{2.433919in}}%
\pgfusepath{stroke}%
\end{pgfscope}%
\begin{pgfscope}%
\pgfsetbuttcap%
\pgfsetroundjoin%
\definecolor{currentfill}{rgb}{0.000000,0.000000,0.000000}%
\pgfsetfillcolor{currentfill}%
\pgfsetlinewidth{0.602250pt}%
\definecolor{currentstroke}{rgb}{0.000000,0.000000,0.000000}%
\pgfsetstrokecolor{currentstroke}%
\pgfsetdash{}{0pt}%
\pgfsys@defobject{currentmarker}{\pgfqpoint{0.000000in}{-0.027778in}}{\pgfqpoint{0.000000in}{0.000000in}}{%
\pgfpathmoveto{\pgfqpoint{0.000000in}{0.000000in}}%
\pgfpathlineto{\pgfqpoint{0.000000in}{-0.027778in}}%
\pgfusepath{stroke,fill}%
}%
\begin{pgfscope}%
\pgfsys@transformshift{3.838739in}{0.417642in}%
\pgfsys@useobject{currentmarker}{}%
\end{pgfscope}%
\end{pgfscope}%
\begin{pgfscope}%
\definecolor{textcolor}{rgb}{0.000000,0.000000,0.000000}%
\pgfsetstrokecolor{textcolor}%
\pgfsetfillcolor{textcolor}%
\pgftext[x=2.310548in,y=0.165003in,,top]{\color{textcolor}\rmfamily\fontsize{10.000000}{12.000000}\selectfont Frequency in \(\displaystyle \unit{\Hz}\)}%
\end{pgfscope}%
\begin{pgfscope}%
\pgfpathrectangle{\pgfqpoint{0.594525in}{0.417642in}}{\pgfqpoint{3.432047in}{2.016277in}}%
\pgfusepath{clip}%
\pgfsetrectcap%
\pgfsetroundjoin%
\pgfsetlinewidth{0.803000pt}%
\definecolor{currentstroke}{rgb}{0.450000,0.450000,0.450000}%
\pgfsetstrokecolor{currentstroke}%
\pgfsetdash{}{0pt}%
\pgfpathmoveto{\pgfqpoint{0.594525in}{0.517495in}}%
\pgfpathlineto{\pgfqpoint{4.026572in}{0.517495in}}%
\pgfusepath{stroke}%
\end{pgfscope}%
\begin{pgfscope}%
\pgfsetbuttcap%
\pgfsetroundjoin%
\definecolor{currentfill}{rgb}{0.000000,0.000000,0.000000}%
\pgfsetfillcolor{currentfill}%
\pgfsetlinewidth{0.803000pt}%
\definecolor{currentstroke}{rgb}{0.000000,0.000000,0.000000}%
\pgfsetstrokecolor{currentstroke}%
\pgfsetdash{}{0pt}%
\pgfsys@defobject{currentmarker}{\pgfqpoint{-0.048611in}{0.000000in}}{\pgfqpoint{-0.000000in}{0.000000in}}{%
\pgfpathmoveto{\pgfqpoint{-0.000000in}{0.000000in}}%
\pgfpathlineto{\pgfqpoint{-0.048611in}{0.000000in}}%
\pgfusepath{stroke,fill}%
}%
\begin{pgfscope}%
\pgfsys@transformshift{0.594525in}{0.517495in}%
\pgfsys@useobject{currentmarker}{}%
\end{pgfscope}%
\end{pgfscope}%
\begin{pgfscope}%
\definecolor{textcolor}{rgb}{0.000000,0.000000,0.000000}%
\pgfsetstrokecolor{textcolor}%
\pgfsetfillcolor{textcolor}%
\pgftext[x=0.241129in, y=0.478342in, left, base]{\color{textcolor}\rmfamily\fontsize{8.000000}{9.600000}\selectfont \(\displaystyle {10^{-6}}\)}%
\end{pgfscope}%
\begin{pgfscope}%
\pgfpathrectangle{\pgfqpoint{0.594525in}{0.417642in}}{\pgfqpoint{3.432047in}{2.016277in}}%
\pgfusepath{clip}%
\pgfsetrectcap%
\pgfsetroundjoin%
\pgfsetlinewidth{0.803000pt}%
\definecolor{currentstroke}{rgb}{0.450000,0.450000,0.450000}%
\pgfsetstrokecolor{currentstroke}%
\pgfsetdash{}{0pt}%
\pgfpathmoveto{\pgfqpoint{0.594525in}{0.836827in}}%
\pgfpathlineto{\pgfqpoint{4.026572in}{0.836827in}}%
\pgfusepath{stroke}%
\end{pgfscope}%
\begin{pgfscope}%
\pgfsetbuttcap%
\pgfsetroundjoin%
\definecolor{currentfill}{rgb}{0.000000,0.000000,0.000000}%
\pgfsetfillcolor{currentfill}%
\pgfsetlinewidth{0.803000pt}%
\definecolor{currentstroke}{rgb}{0.000000,0.000000,0.000000}%
\pgfsetstrokecolor{currentstroke}%
\pgfsetdash{}{0pt}%
\pgfsys@defobject{currentmarker}{\pgfqpoint{-0.048611in}{0.000000in}}{\pgfqpoint{-0.000000in}{0.000000in}}{%
\pgfpathmoveto{\pgfqpoint{-0.000000in}{0.000000in}}%
\pgfpathlineto{\pgfqpoint{-0.048611in}{0.000000in}}%
\pgfusepath{stroke,fill}%
}%
\begin{pgfscope}%
\pgfsys@transformshift{0.594525in}{0.836827in}%
\pgfsys@useobject{currentmarker}{}%
\end{pgfscope}%
\end{pgfscope}%
\begin{pgfscope}%
\definecolor{textcolor}{rgb}{0.000000,0.000000,0.000000}%
\pgfsetstrokecolor{textcolor}%
\pgfsetfillcolor{textcolor}%
\pgftext[x=0.241129in, y=0.797674in, left, base]{\color{textcolor}\rmfamily\fontsize{8.000000}{9.600000}\selectfont \(\displaystyle {10^{-5}}\)}%
\end{pgfscope}%
\begin{pgfscope}%
\pgfpathrectangle{\pgfqpoint{0.594525in}{0.417642in}}{\pgfqpoint{3.432047in}{2.016277in}}%
\pgfusepath{clip}%
\pgfsetrectcap%
\pgfsetroundjoin%
\pgfsetlinewidth{0.803000pt}%
\definecolor{currentstroke}{rgb}{0.450000,0.450000,0.450000}%
\pgfsetstrokecolor{currentstroke}%
\pgfsetdash{}{0pt}%
\pgfpathmoveto{\pgfqpoint{0.594525in}{1.156160in}}%
\pgfpathlineto{\pgfqpoint{4.026572in}{1.156160in}}%
\pgfusepath{stroke}%
\end{pgfscope}%
\begin{pgfscope}%
\pgfsetbuttcap%
\pgfsetroundjoin%
\definecolor{currentfill}{rgb}{0.000000,0.000000,0.000000}%
\pgfsetfillcolor{currentfill}%
\pgfsetlinewidth{0.803000pt}%
\definecolor{currentstroke}{rgb}{0.000000,0.000000,0.000000}%
\pgfsetstrokecolor{currentstroke}%
\pgfsetdash{}{0pt}%
\pgfsys@defobject{currentmarker}{\pgfqpoint{-0.048611in}{0.000000in}}{\pgfqpoint{-0.000000in}{0.000000in}}{%
\pgfpathmoveto{\pgfqpoint{-0.000000in}{0.000000in}}%
\pgfpathlineto{\pgfqpoint{-0.048611in}{0.000000in}}%
\pgfusepath{stroke,fill}%
}%
\begin{pgfscope}%
\pgfsys@transformshift{0.594525in}{1.156160in}%
\pgfsys@useobject{currentmarker}{}%
\end{pgfscope}%
\end{pgfscope}%
\begin{pgfscope}%
\definecolor{textcolor}{rgb}{0.000000,0.000000,0.000000}%
\pgfsetstrokecolor{textcolor}%
\pgfsetfillcolor{textcolor}%
\pgftext[x=0.241129in, y=1.117007in, left, base]{\color{textcolor}\rmfamily\fontsize{8.000000}{9.600000}\selectfont \(\displaystyle {10^{-4}}\)}%
\end{pgfscope}%
\begin{pgfscope}%
\pgfpathrectangle{\pgfqpoint{0.594525in}{0.417642in}}{\pgfqpoint{3.432047in}{2.016277in}}%
\pgfusepath{clip}%
\pgfsetrectcap%
\pgfsetroundjoin%
\pgfsetlinewidth{0.803000pt}%
\definecolor{currentstroke}{rgb}{0.450000,0.450000,0.450000}%
\pgfsetstrokecolor{currentstroke}%
\pgfsetdash{}{0pt}%
\pgfpathmoveto{\pgfqpoint{0.594525in}{1.475492in}}%
\pgfpathlineto{\pgfqpoint{4.026572in}{1.475492in}}%
\pgfusepath{stroke}%
\end{pgfscope}%
\begin{pgfscope}%
\pgfsetbuttcap%
\pgfsetroundjoin%
\definecolor{currentfill}{rgb}{0.000000,0.000000,0.000000}%
\pgfsetfillcolor{currentfill}%
\pgfsetlinewidth{0.803000pt}%
\definecolor{currentstroke}{rgb}{0.000000,0.000000,0.000000}%
\pgfsetstrokecolor{currentstroke}%
\pgfsetdash{}{0pt}%
\pgfsys@defobject{currentmarker}{\pgfqpoint{-0.048611in}{0.000000in}}{\pgfqpoint{-0.000000in}{0.000000in}}{%
\pgfpathmoveto{\pgfqpoint{-0.000000in}{0.000000in}}%
\pgfpathlineto{\pgfqpoint{-0.048611in}{0.000000in}}%
\pgfusepath{stroke,fill}%
}%
\begin{pgfscope}%
\pgfsys@transformshift{0.594525in}{1.475492in}%
\pgfsys@useobject{currentmarker}{}%
\end{pgfscope}%
\end{pgfscope}%
\begin{pgfscope}%
\definecolor{textcolor}{rgb}{0.000000,0.000000,0.000000}%
\pgfsetstrokecolor{textcolor}%
\pgfsetfillcolor{textcolor}%
\pgftext[x=0.241129in, y=1.436339in, left, base]{\color{textcolor}\rmfamily\fontsize{8.000000}{9.600000}\selectfont \(\displaystyle {10^{-3}}\)}%
\end{pgfscope}%
\begin{pgfscope}%
\pgfpathrectangle{\pgfqpoint{0.594525in}{0.417642in}}{\pgfqpoint{3.432047in}{2.016277in}}%
\pgfusepath{clip}%
\pgfsetrectcap%
\pgfsetroundjoin%
\pgfsetlinewidth{0.803000pt}%
\definecolor{currentstroke}{rgb}{0.450000,0.450000,0.450000}%
\pgfsetstrokecolor{currentstroke}%
\pgfsetdash{}{0pt}%
\pgfpathmoveto{\pgfqpoint{0.594525in}{1.794824in}}%
\pgfpathlineto{\pgfqpoint{4.026572in}{1.794824in}}%
\pgfusepath{stroke}%
\end{pgfscope}%
\begin{pgfscope}%
\pgfsetbuttcap%
\pgfsetroundjoin%
\definecolor{currentfill}{rgb}{0.000000,0.000000,0.000000}%
\pgfsetfillcolor{currentfill}%
\pgfsetlinewidth{0.803000pt}%
\definecolor{currentstroke}{rgb}{0.000000,0.000000,0.000000}%
\pgfsetstrokecolor{currentstroke}%
\pgfsetdash{}{0pt}%
\pgfsys@defobject{currentmarker}{\pgfqpoint{-0.048611in}{0.000000in}}{\pgfqpoint{-0.000000in}{0.000000in}}{%
\pgfpathmoveto{\pgfqpoint{-0.000000in}{0.000000in}}%
\pgfpathlineto{\pgfqpoint{-0.048611in}{0.000000in}}%
\pgfusepath{stroke,fill}%
}%
\begin{pgfscope}%
\pgfsys@transformshift{0.594525in}{1.794824in}%
\pgfsys@useobject{currentmarker}{}%
\end{pgfscope}%
\end{pgfscope}%
\begin{pgfscope}%
\definecolor{textcolor}{rgb}{0.000000,0.000000,0.000000}%
\pgfsetstrokecolor{textcolor}%
\pgfsetfillcolor{textcolor}%
\pgftext[x=0.241129in, y=1.755671in, left, base]{\color{textcolor}\rmfamily\fontsize{8.000000}{9.600000}\selectfont \(\displaystyle {10^{-2}}\)}%
\end{pgfscope}%
\begin{pgfscope}%
\pgfpathrectangle{\pgfqpoint{0.594525in}{0.417642in}}{\pgfqpoint{3.432047in}{2.016277in}}%
\pgfusepath{clip}%
\pgfsetrectcap%
\pgfsetroundjoin%
\pgfsetlinewidth{0.803000pt}%
\definecolor{currentstroke}{rgb}{0.450000,0.450000,0.450000}%
\pgfsetstrokecolor{currentstroke}%
\pgfsetdash{}{0pt}%
\pgfpathmoveto{\pgfqpoint{0.594525in}{2.114156in}}%
\pgfpathlineto{\pgfqpoint{4.026572in}{2.114156in}}%
\pgfusepath{stroke}%
\end{pgfscope}%
\begin{pgfscope}%
\pgfsetbuttcap%
\pgfsetroundjoin%
\definecolor{currentfill}{rgb}{0.000000,0.000000,0.000000}%
\pgfsetfillcolor{currentfill}%
\pgfsetlinewidth{0.803000pt}%
\definecolor{currentstroke}{rgb}{0.000000,0.000000,0.000000}%
\pgfsetstrokecolor{currentstroke}%
\pgfsetdash{}{0pt}%
\pgfsys@defobject{currentmarker}{\pgfqpoint{-0.048611in}{0.000000in}}{\pgfqpoint{-0.000000in}{0.000000in}}{%
\pgfpathmoveto{\pgfqpoint{-0.000000in}{0.000000in}}%
\pgfpathlineto{\pgfqpoint{-0.048611in}{0.000000in}}%
\pgfusepath{stroke,fill}%
}%
\begin{pgfscope}%
\pgfsys@transformshift{0.594525in}{2.114156in}%
\pgfsys@useobject{currentmarker}{}%
\end{pgfscope}%
\end{pgfscope}%
\begin{pgfscope}%
\definecolor{textcolor}{rgb}{0.000000,0.000000,0.000000}%
\pgfsetstrokecolor{textcolor}%
\pgfsetfillcolor{textcolor}%
\pgftext[x=0.241129in, y=2.075004in, left, base]{\color{textcolor}\rmfamily\fontsize{8.000000}{9.600000}\selectfont \(\displaystyle {10^{-1}}\)}%
\end{pgfscope}%
\begin{pgfscope}%
\pgfpathrectangle{\pgfqpoint{0.594525in}{0.417642in}}{\pgfqpoint{3.432047in}{2.016277in}}%
\pgfusepath{clip}%
\pgfsetrectcap%
\pgfsetroundjoin%
\pgfsetlinewidth{0.803000pt}%
\definecolor{currentstroke}{rgb}{0.450000,0.450000,0.450000}%
\pgfsetstrokecolor{currentstroke}%
\pgfsetdash{}{0pt}%
\pgfpathmoveto{\pgfqpoint{0.594525in}{2.433489in}}%
\pgfpathlineto{\pgfqpoint{4.026572in}{2.433489in}}%
\pgfusepath{stroke}%
\end{pgfscope}%
\begin{pgfscope}%
\pgfsetbuttcap%
\pgfsetroundjoin%
\definecolor{currentfill}{rgb}{0.000000,0.000000,0.000000}%
\pgfsetfillcolor{currentfill}%
\pgfsetlinewidth{0.803000pt}%
\definecolor{currentstroke}{rgb}{0.000000,0.000000,0.000000}%
\pgfsetstrokecolor{currentstroke}%
\pgfsetdash{}{0pt}%
\pgfsys@defobject{currentmarker}{\pgfqpoint{-0.048611in}{0.000000in}}{\pgfqpoint{-0.000000in}{0.000000in}}{%
\pgfpathmoveto{\pgfqpoint{-0.000000in}{0.000000in}}%
\pgfpathlineto{\pgfqpoint{-0.048611in}{0.000000in}}%
\pgfusepath{stroke,fill}%
}%
\begin{pgfscope}%
\pgfsys@transformshift{0.594525in}{2.433489in}%
\pgfsys@useobject{currentmarker}{}%
\end{pgfscope}%
\end{pgfscope}%
\begin{pgfscope}%
\definecolor{textcolor}{rgb}{0.000000,0.000000,0.000000}%
\pgfsetstrokecolor{textcolor}%
\pgfsetfillcolor{textcolor}%
\pgftext[x=0.321376in, y=2.394336in, left, base]{\color{textcolor}\rmfamily\fontsize{8.000000}{9.600000}\selectfont \(\displaystyle {10^{0}}\)}%
\end{pgfscope}%
\begin{pgfscope}%
\pgfpathrectangle{\pgfqpoint{0.594525in}{0.417642in}}{\pgfqpoint{3.432047in}{2.016277in}}%
\pgfusepath{clip}%
\pgfsetrectcap%
\pgfsetroundjoin%
\pgfsetlinewidth{0.803000pt}%
\definecolor{currentstroke}{rgb}{0.850000,0.850000,0.850000}%
\pgfsetstrokecolor{currentstroke}%
\pgfsetdash{}{0pt}%
\pgfpathmoveto{\pgfqpoint{0.594525in}{0.421366in}}%
\pgfpathlineto{\pgfqpoint{4.026572in}{0.421366in}}%
\pgfusepath{stroke}%
\end{pgfscope}%
\begin{pgfscope}%
\pgfsetbuttcap%
\pgfsetroundjoin%
\definecolor{currentfill}{rgb}{0.000000,0.000000,0.000000}%
\pgfsetfillcolor{currentfill}%
\pgfsetlinewidth{0.602250pt}%
\definecolor{currentstroke}{rgb}{0.000000,0.000000,0.000000}%
\pgfsetstrokecolor{currentstroke}%
\pgfsetdash{}{0pt}%
\pgfsys@defobject{currentmarker}{\pgfqpoint{-0.027778in}{0.000000in}}{\pgfqpoint{-0.000000in}{0.000000in}}{%
\pgfpathmoveto{\pgfqpoint{-0.000000in}{0.000000in}}%
\pgfpathlineto{\pgfqpoint{-0.027778in}{0.000000in}}%
\pgfusepath{stroke,fill}%
}%
\begin{pgfscope}%
\pgfsys@transformshift{0.594525in}{0.421366in}%
\pgfsys@useobject{currentmarker}{}%
\end{pgfscope}%
\end{pgfscope}%
\begin{pgfscope}%
\pgfpathrectangle{\pgfqpoint{0.594525in}{0.417642in}}{\pgfqpoint{3.432047in}{2.016277in}}%
\pgfusepath{clip}%
\pgfsetrectcap%
\pgfsetroundjoin%
\pgfsetlinewidth{0.803000pt}%
\definecolor{currentstroke}{rgb}{0.850000,0.850000,0.850000}%
\pgfsetstrokecolor{currentstroke}%
\pgfsetdash{}{0pt}%
\pgfpathmoveto{\pgfqpoint{0.594525in}{0.446651in}}%
\pgfpathlineto{\pgfqpoint{4.026572in}{0.446651in}}%
\pgfusepath{stroke}%
\end{pgfscope}%
\begin{pgfscope}%
\pgfsetbuttcap%
\pgfsetroundjoin%
\definecolor{currentfill}{rgb}{0.000000,0.000000,0.000000}%
\pgfsetfillcolor{currentfill}%
\pgfsetlinewidth{0.602250pt}%
\definecolor{currentstroke}{rgb}{0.000000,0.000000,0.000000}%
\pgfsetstrokecolor{currentstroke}%
\pgfsetdash{}{0pt}%
\pgfsys@defobject{currentmarker}{\pgfqpoint{-0.027778in}{0.000000in}}{\pgfqpoint{-0.000000in}{0.000000in}}{%
\pgfpathmoveto{\pgfqpoint{-0.000000in}{0.000000in}}%
\pgfpathlineto{\pgfqpoint{-0.027778in}{0.000000in}}%
\pgfusepath{stroke,fill}%
}%
\begin{pgfscope}%
\pgfsys@transformshift{0.594525in}{0.446651in}%
\pgfsys@useobject{currentmarker}{}%
\end{pgfscope}%
\end{pgfscope}%
\begin{pgfscope}%
\pgfpathrectangle{\pgfqpoint{0.594525in}{0.417642in}}{\pgfqpoint{3.432047in}{2.016277in}}%
\pgfusepath{clip}%
\pgfsetrectcap%
\pgfsetroundjoin%
\pgfsetlinewidth{0.803000pt}%
\definecolor{currentstroke}{rgb}{0.850000,0.850000,0.850000}%
\pgfsetstrokecolor{currentstroke}%
\pgfsetdash{}{0pt}%
\pgfpathmoveto{\pgfqpoint{0.594525in}{0.468030in}}%
\pgfpathlineto{\pgfqpoint{4.026572in}{0.468030in}}%
\pgfusepath{stroke}%
\end{pgfscope}%
\begin{pgfscope}%
\pgfsetbuttcap%
\pgfsetroundjoin%
\definecolor{currentfill}{rgb}{0.000000,0.000000,0.000000}%
\pgfsetfillcolor{currentfill}%
\pgfsetlinewidth{0.602250pt}%
\definecolor{currentstroke}{rgb}{0.000000,0.000000,0.000000}%
\pgfsetstrokecolor{currentstroke}%
\pgfsetdash{}{0pt}%
\pgfsys@defobject{currentmarker}{\pgfqpoint{-0.027778in}{0.000000in}}{\pgfqpoint{-0.000000in}{0.000000in}}{%
\pgfpathmoveto{\pgfqpoint{-0.000000in}{0.000000in}}%
\pgfpathlineto{\pgfqpoint{-0.027778in}{0.000000in}}%
\pgfusepath{stroke,fill}%
}%
\begin{pgfscope}%
\pgfsys@transformshift{0.594525in}{0.468030in}%
\pgfsys@useobject{currentmarker}{}%
\end{pgfscope}%
\end{pgfscope}%
\begin{pgfscope}%
\pgfpathrectangle{\pgfqpoint{0.594525in}{0.417642in}}{\pgfqpoint{3.432047in}{2.016277in}}%
\pgfusepath{clip}%
\pgfsetrectcap%
\pgfsetroundjoin%
\pgfsetlinewidth{0.803000pt}%
\definecolor{currentstroke}{rgb}{0.850000,0.850000,0.850000}%
\pgfsetstrokecolor{currentstroke}%
\pgfsetdash{}{0pt}%
\pgfpathmoveto{\pgfqpoint{0.594525in}{0.486548in}}%
\pgfpathlineto{\pgfqpoint{4.026572in}{0.486548in}}%
\pgfusepath{stroke}%
\end{pgfscope}%
\begin{pgfscope}%
\pgfsetbuttcap%
\pgfsetroundjoin%
\definecolor{currentfill}{rgb}{0.000000,0.000000,0.000000}%
\pgfsetfillcolor{currentfill}%
\pgfsetlinewidth{0.602250pt}%
\definecolor{currentstroke}{rgb}{0.000000,0.000000,0.000000}%
\pgfsetstrokecolor{currentstroke}%
\pgfsetdash{}{0pt}%
\pgfsys@defobject{currentmarker}{\pgfqpoint{-0.027778in}{0.000000in}}{\pgfqpoint{-0.000000in}{0.000000in}}{%
\pgfpathmoveto{\pgfqpoint{-0.000000in}{0.000000in}}%
\pgfpathlineto{\pgfqpoint{-0.027778in}{0.000000in}}%
\pgfusepath{stroke,fill}%
}%
\begin{pgfscope}%
\pgfsys@transformshift{0.594525in}{0.486548in}%
\pgfsys@useobject{currentmarker}{}%
\end{pgfscope}%
\end{pgfscope}%
\begin{pgfscope}%
\pgfpathrectangle{\pgfqpoint{0.594525in}{0.417642in}}{\pgfqpoint{3.432047in}{2.016277in}}%
\pgfusepath{clip}%
\pgfsetrectcap%
\pgfsetroundjoin%
\pgfsetlinewidth{0.803000pt}%
\definecolor{currentstroke}{rgb}{0.850000,0.850000,0.850000}%
\pgfsetstrokecolor{currentstroke}%
\pgfsetdash{}{0pt}%
\pgfpathmoveto{\pgfqpoint{0.594525in}{0.502883in}}%
\pgfpathlineto{\pgfqpoint{4.026572in}{0.502883in}}%
\pgfusepath{stroke}%
\end{pgfscope}%
\begin{pgfscope}%
\pgfsetbuttcap%
\pgfsetroundjoin%
\definecolor{currentfill}{rgb}{0.000000,0.000000,0.000000}%
\pgfsetfillcolor{currentfill}%
\pgfsetlinewidth{0.602250pt}%
\definecolor{currentstroke}{rgb}{0.000000,0.000000,0.000000}%
\pgfsetstrokecolor{currentstroke}%
\pgfsetdash{}{0pt}%
\pgfsys@defobject{currentmarker}{\pgfqpoint{-0.027778in}{0.000000in}}{\pgfqpoint{-0.000000in}{0.000000in}}{%
\pgfpathmoveto{\pgfqpoint{-0.000000in}{0.000000in}}%
\pgfpathlineto{\pgfqpoint{-0.027778in}{0.000000in}}%
\pgfusepath{stroke,fill}%
}%
\begin{pgfscope}%
\pgfsys@transformshift{0.594525in}{0.502883in}%
\pgfsys@useobject{currentmarker}{}%
\end{pgfscope}%
\end{pgfscope}%
\begin{pgfscope}%
\pgfpathrectangle{\pgfqpoint{0.594525in}{0.417642in}}{\pgfqpoint{3.432047in}{2.016277in}}%
\pgfusepath{clip}%
\pgfsetrectcap%
\pgfsetroundjoin%
\pgfsetlinewidth{0.803000pt}%
\definecolor{currentstroke}{rgb}{0.850000,0.850000,0.850000}%
\pgfsetstrokecolor{currentstroke}%
\pgfsetdash{}{0pt}%
\pgfpathmoveto{\pgfqpoint{0.594525in}{0.613623in}}%
\pgfpathlineto{\pgfqpoint{4.026572in}{0.613623in}}%
\pgfusepath{stroke}%
\end{pgfscope}%
\begin{pgfscope}%
\pgfsetbuttcap%
\pgfsetroundjoin%
\definecolor{currentfill}{rgb}{0.000000,0.000000,0.000000}%
\pgfsetfillcolor{currentfill}%
\pgfsetlinewidth{0.602250pt}%
\definecolor{currentstroke}{rgb}{0.000000,0.000000,0.000000}%
\pgfsetstrokecolor{currentstroke}%
\pgfsetdash{}{0pt}%
\pgfsys@defobject{currentmarker}{\pgfqpoint{-0.027778in}{0.000000in}}{\pgfqpoint{-0.000000in}{0.000000in}}{%
\pgfpathmoveto{\pgfqpoint{-0.000000in}{0.000000in}}%
\pgfpathlineto{\pgfqpoint{-0.027778in}{0.000000in}}%
\pgfusepath{stroke,fill}%
}%
\begin{pgfscope}%
\pgfsys@transformshift{0.594525in}{0.613623in}%
\pgfsys@useobject{currentmarker}{}%
\end{pgfscope}%
\end{pgfscope}%
\begin{pgfscope}%
\pgfpathrectangle{\pgfqpoint{0.594525in}{0.417642in}}{\pgfqpoint{3.432047in}{2.016277in}}%
\pgfusepath{clip}%
\pgfsetrectcap%
\pgfsetroundjoin%
\pgfsetlinewidth{0.803000pt}%
\definecolor{currentstroke}{rgb}{0.850000,0.850000,0.850000}%
\pgfsetstrokecolor{currentstroke}%
\pgfsetdash{}{0pt}%
\pgfpathmoveto{\pgfqpoint{0.594525in}{0.669855in}}%
\pgfpathlineto{\pgfqpoint{4.026572in}{0.669855in}}%
\pgfusepath{stroke}%
\end{pgfscope}%
\begin{pgfscope}%
\pgfsetbuttcap%
\pgfsetroundjoin%
\definecolor{currentfill}{rgb}{0.000000,0.000000,0.000000}%
\pgfsetfillcolor{currentfill}%
\pgfsetlinewidth{0.602250pt}%
\definecolor{currentstroke}{rgb}{0.000000,0.000000,0.000000}%
\pgfsetstrokecolor{currentstroke}%
\pgfsetdash{}{0pt}%
\pgfsys@defobject{currentmarker}{\pgfqpoint{-0.027778in}{0.000000in}}{\pgfqpoint{-0.000000in}{0.000000in}}{%
\pgfpathmoveto{\pgfqpoint{-0.000000in}{0.000000in}}%
\pgfpathlineto{\pgfqpoint{-0.027778in}{0.000000in}}%
\pgfusepath{stroke,fill}%
}%
\begin{pgfscope}%
\pgfsys@transformshift{0.594525in}{0.669855in}%
\pgfsys@useobject{currentmarker}{}%
\end{pgfscope}%
\end{pgfscope}%
\begin{pgfscope}%
\pgfpathrectangle{\pgfqpoint{0.594525in}{0.417642in}}{\pgfqpoint{3.432047in}{2.016277in}}%
\pgfusepath{clip}%
\pgfsetrectcap%
\pgfsetroundjoin%
\pgfsetlinewidth{0.803000pt}%
\definecolor{currentstroke}{rgb}{0.850000,0.850000,0.850000}%
\pgfsetstrokecolor{currentstroke}%
\pgfsetdash{}{0pt}%
\pgfpathmoveto{\pgfqpoint{0.594525in}{0.709752in}}%
\pgfpathlineto{\pgfqpoint{4.026572in}{0.709752in}}%
\pgfusepath{stroke}%
\end{pgfscope}%
\begin{pgfscope}%
\pgfsetbuttcap%
\pgfsetroundjoin%
\definecolor{currentfill}{rgb}{0.000000,0.000000,0.000000}%
\pgfsetfillcolor{currentfill}%
\pgfsetlinewidth{0.602250pt}%
\definecolor{currentstroke}{rgb}{0.000000,0.000000,0.000000}%
\pgfsetstrokecolor{currentstroke}%
\pgfsetdash{}{0pt}%
\pgfsys@defobject{currentmarker}{\pgfqpoint{-0.027778in}{0.000000in}}{\pgfqpoint{-0.000000in}{0.000000in}}{%
\pgfpathmoveto{\pgfqpoint{-0.000000in}{0.000000in}}%
\pgfpathlineto{\pgfqpoint{-0.027778in}{0.000000in}}%
\pgfusepath{stroke,fill}%
}%
\begin{pgfscope}%
\pgfsys@transformshift{0.594525in}{0.709752in}%
\pgfsys@useobject{currentmarker}{}%
\end{pgfscope}%
\end{pgfscope}%
\begin{pgfscope}%
\pgfpathrectangle{\pgfqpoint{0.594525in}{0.417642in}}{\pgfqpoint{3.432047in}{2.016277in}}%
\pgfusepath{clip}%
\pgfsetrectcap%
\pgfsetroundjoin%
\pgfsetlinewidth{0.803000pt}%
\definecolor{currentstroke}{rgb}{0.850000,0.850000,0.850000}%
\pgfsetstrokecolor{currentstroke}%
\pgfsetdash{}{0pt}%
\pgfpathmoveto{\pgfqpoint{0.594525in}{0.740699in}}%
\pgfpathlineto{\pgfqpoint{4.026572in}{0.740699in}}%
\pgfusepath{stroke}%
\end{pgfscope}%
\begin{pgfscope}%
\pgfsetbuttcap%
\pgfsetroundjoin%
\definecolor{currentfill}{rgb}{0.000000,0.000000,0.000000}%
\pgfsetfillcolor{currentfill}%
\pgfsetlinewidth{0.602250pt}%
\definecolor{currentstroke}{rgb}{0.000000,0.000000,0.000000}%
\pgfsetstrokecolor{currentstroke}%
\pgfsetdash{}{0pt}%
\pgfsys@defobject{currentmarker}{\pgfqpoint{-0.027778in}{0.000000in}}{\pgfqpoint{-0.000000in}{0.000000in}}{%
\pgfpathmoveto{\pgfqpoint{-0.000000in}{0.000000in}}%
\pgfpathlineto{\pgfqpoint{-0.027778in}{0.000000in}}%
\pgfusepath{stroke,fill}%
}%
\begin{pgfscope}%
\pgfsys@transformshift{0.594525in}{0.740699in}%
\pgfsys@useobject{currentmarker}{}%
\end{pgfscope}%
\end{pgfscope}%
\begin{pgfscope}%
\pgfpathrectangle{\pgfqpoint{0.594525in}{0.417642in}}{\pgfqpoint{3.432047in}{2.016277in}}%
\pgfusepath{clip}%
\pgfsetrectcap%
\pgfsetroundjoin%
\pgfsetlinewidth{0.803000pt}%
\definecolor{currentstroke}{rgb}{0.850000,0.850000,0.850000}%
\pgfsetstrokecolor{currentstroke}%
\pgfsetdash{}{0pt}%
\pgfpathmoveto{\pgfqpoint{0.594525in}{0.765984in}}%
\pgfpathlineto{\pgfqpoint{4.026572in}{0.765984in}}%
\pgfusepath{stroke}%
\end{pgfscope}%
\begin{pgfscope}%
\pgfsetbuttcap%
\pgfsetroundjoin%
\definecolor{currentfill}{rgb}{0.000000,0.000000,0.000000}%
\pgfsetfillcolor{currentfill}%
\pgfsetlinewidth{0.602250pt}%
\definecolor{currentstroke}{rgb}{0.000000,0.000000,0.000000}%
\pgfsetstrokecolor{currentstroke}%
\pgfsetdash{}{0pt}%
\pgfsys@defobject{currentmarker}{\pgfqpoint{-0.027778in}{0.000000in}}{\pgfqpoint{-0.000000in}{0.000000in}}{%
\pgfpathmoveto{\pgfqpoint{-0.000000in}{0.000000in}}%
\pgfpathlineto{\pgfqpoint{-0.027778in}{0.000000in}}%
\pgfusepath{stroke,fill}%
}%
\begin{pgfscope}%
\pgfsys@transformshift{0.594525in}{0.765984in}%
\pgfsys@useobject{currentmarker}{}%
\end{pgfscope}%
\end{pgfscope}%
\begin{pgfscope}%
\pgfpathrectangle{\pgfqpoint{0.594525in}{0.417642in}}{\pgfqpoint{3.432047in}{2.016277in}}%
\pgfusepath{clip}%
\pgfsetrectcap%
\pgfsetroundjoin%
\pgfsetlinewidth{0.803000pt}%
\definecolor{currentstroke}{rgb}{0.850000,0.850000,0.850000}%
\pgfsetstrokecolor{currentstroke}%
\pgfsetdash{}{0pt}%
\pgfpathmoveto{\pgfqpoint{0.594525in}{0.787362in}}%
\pgfpathlineto{\pgfqpoint{4.026572in}{0.787362in}}%
\pgfusepath{stroke}%
\end{pgfscope}%
\begin{pgfscope}%
\pgfsetbuttcap%
\pgfsetroundjoin%
\definecolor{currentfill}{rgb}{0.000000,0.000000,0.000000}%
\pgfsetfillcolor{currentfill}%
\pgfsetlinewidth{0.602250pt}%
\definecolor{currentstroke}{rgb}{0.000000,0.000000,0.000000}%
\pgfsetstrokecolor{currentstroke}%
\pgfsetdash{}{0pt}%
\pgfsys@defobject{currentmarker}{\pgfqpoint{-0.027778in}{0.000000in}}{\pgfqpoint{-0.000000in}{0.000000in}}{%
\pgfpathmoveto{\pgfqpoint{-0.000000in}{0.000000in}}%
\pgfpathlineto{\pgfqpoint{-0.027778in}{0.000000in}}%
\pgfusepath{stroke,fill}%
}%
\begin{pgfscope}%
\pgfsys@transformshift{0.594525in}{0.787362in}%
\pgfsys@useobject{currentmarker}{}%
\end{pgfscope}%
\end{pgfscope}%
\begin{pgfscope}%
\pgfpathrectangle{\pgfqpoint{0.594525in}{0.417642in}}{\pgfqpoint{3.432047in}{2.016277in}}%
\pgfusepath{clip}%
\pgfsetrectcap%
\pgfsetroundjoin%
\pgfsetlinewidth{0.803000pt}%
\definecolor{currentstroke}{rgb}{0.850000,0.850000,0.850000}%
\pgfsetstrokecolor{currentstroke}%
\pgfsetdash{}{0pt}%
\pgfpathmoveto{\pgfqpoint{0.594525in}{0.805881in}}%
\pgfpathlineto{\pgfqpoint{4.026572in}{0.805881in}}%
\pgfusepath{stroke}%
\end{pgfscope}%
\begin{pgfscope}%
\pgfsetbuttcap%
\pgfsetroundjoin%
\definecolor{currentfill}{rgb}{0.000000,0.000000,0.000000}%
\pgfsetfillcolor{currentfill}%
\pgfsetlinewidth{0.602250pt}%
\definecolor{currentstroke}{rgb}{0.000000,0.000000,0.000000}%
\pgfsetstrokecolor{currentstroke}%
\pgfsetdash{}{0pt}%
\pgfsys@defobject{currentmarker}{\pgfqpoint{-0.027778in}{0.000000in}}{\pgfqpoint{-0.000000in}{0.000000in}}{%
\pgfpathmoveto{\pgfqpoint{-0.000000in}{0.000000in}}%
\pgfpathlineto{\pgfqpoint{-0.027778in}{0.000000in}}%
\pgfusepath{stroke,fill}%
}%
\begin{pgfscope}%
\pgfsys@transformshift{0.594525in}{0.805881in}%
\pgfsys@useobject{currentmarker}{}%
\end{pgfscope}%
\end{pgfscope}%
\begin{pgfscope}%
\pgfpathrectangle{\pgfqpoint{0.594525in}{0.417642in}}{\pgfqpoint{3.432047in}{2.016277in}}%
\pgfusepath{clip}%
\pgfsetrectcap%
\pgfsetroundjoin%
\pgfsetlinewidth{0.803000pt}%
\definecolor{currentstroke}{rgb}{0.850000,0.850000,0.850000}%
\pgfsetstrokecolor{currentstroke}%
\pgfsetdash{}{0pt}%
\pgfpathmoveto{\pgfqpoint{0.594525in}{0.822215in}}%
\pgfpathlineto{\pgfqpoint{4.026572in}{0.822215in}}%
\pgfusepath{stroke}%
\end{pgfscope}%
\begin{pgfscope}%
\pgfsetbuttcap%
\pgfsetroundjoin%
\definecolor{currentfill}{rgb}{0.000000,0.000000,0.000000}%
\pgfsetfillcolor{currentfill}%
\pgfsetlinewidth{0.602250pt}%
\definecolor{currentstroke}{rgb}{0.000000,0.000000,0.000000}%
\pgfsetstrokecolor{currentstroke}%
\pgfsetdash{}{0pt}%
\pgfsys@defobject{currentmarker}{\pgfqpoint{-0.027778in}{0.000000in}}{\pgfqpoint{-0.000000in}{0.000000in}}{%
\pgfpathmoveto{\pgfqpoint{-0.000000in}{0.000000in}}%
\pgfpathlineto{\pgfqpoint{-0.027778in}{0.000000in}}%
\pgfusepath{stroke,fill}%
}%
\begin{pgfscope}%
\pgfsys@transformshift{0.594525in}{0.822215in}%
\pgfsys@useobject{currentmarker}{}%
\end{pgfscope}%
\end{pgfscope}%
\begin{pgfscope}%
\pgfpathrectangle{\pgfqpoint{0.594525in}{0.417642in}}{\pgfqpoint{3.432047in}{2.016277in}}%
\pgfusepath{clip}%
\pgfsetrectcap%
\pgfsetroundjoin%
\pgfsetlinewidth{0.803000pt}%
\definecolor{currentstroke}{rgb}{0.850000,0.850000,0.850000}%
\pgfsetstrokecolor{currentstroke}%
\pgfsetdash{}{0pt}%
\pgfpathmoveto{\pgfqpoint{0.594525in}{0.932956in}}%
\pgfpathlineto{\pgfqpoint{4.026572in}{0.932956in}}%
\pgfusepath{stroke}%
\end{pgfscope}%
\begin{pgfscope}%
\pgfsetbuttcap%
\pgfsetroundjoin%
\definecolor{currentfill}{rgb}{0.000000,0.000000,0.000000}%
\pgfsetfillcolor{currentfill}%
\pgfsetlinewidth{0.602250pt}%
\definecolor{currentstroke}{rgb}{0.000000,0.000000,0.000000}%
\pgfsetstrokecolor{currentstroke}%
\pgfsetdash{}{0pt}%
\pgfsys@defobject{currentmarker}{\pgfqpoint{-0.027778in}{0.000000in}}{\pgfqpoint{-0.000000in}{0.000000in}}{%
\pgfpathmoveto{\pgfqpoint{-0.000000in}{0.000000in}}%
\pgfpathlineto{\pgfqpoint{-0.027778in}{0.000000in}}%
\pgfusepath{stroke,fill}%
}%
\begin{pgfscope}%
\pgfsys@transformshift{0.594525in}{0.932956in}%
\pgfsys@useobject{currentmarker}{}%
\end{pgfscope}%
\end{pgfscope}%
\begin{pgfscope}%
\pgfpathrectangle{\pgfqpoint{0.594525in}{0.417642in}}{\pgfqpoint{3.432047in}{2.016277in}}%
\pgfusepath{clip}%
\pgfsetrectcap%
\pgfsetroundjoin%
\pgfsetlinewidth{0.803000pt}%
\definecolor{currentstroke}{rgb}{0.850000,0.850000,0.850000}%
\pgfsetstrokecolor{currentstroke}%
\pgfsetdash{}{0pt}%
\pgfpathmoveto{\pgfqpoint{0.594525in}{0.989187in}}%
\pgfpathlineto{\pgfqpoint{4.026572in}{0.989187in}}%
\pgfusepath{stroke}%
\end{pgfscope}%
\begin{pgfscope}%
\pgfsetbuttcap%
\pgfsetroundjoin%
\definecolor{currentfill}{rgb}{0.000000,0.000000,0.000000}%
\pgfsetfillcolor{currentfill}%
\pgfsetlinewidth{0.602250pt}%
\definecolor{currentstroke}{rgb}{0.000000,0.000000,0.000000}%
\pgfsetstrokecolor{currentstroke}%
\pgfsetdash{}{0pt}%
\pgfsys@defobject{currentmarker}{\pgfqpoint{-0.027778in}{0.000000in}}{\pgfqpoint{-0.000000in}{0.000000in}}{%
\pgfpathmoveto{\pgfqpoint{-0.000000in}{0.000000in}}%
\pgfpathlineto{\pgfqpoint{-0.027778in}{0.000000in}}%
\pgfusepath{stroke,fill}%
}%
\begin{pgfscope}%
\pgfsys@transformshift{0.594525in}{0.989187in}%
\pgfsys@useobject{currentmarker}{}%
\end{pgfscope}%
\end{pgfscope}%
\begin{pgfscope}%
\pgfpathrectangle{\pgfqpoint{0.594525in}{0.417642in}}{\pgfqpoint{3.432047in}{2.016277in}}%
\pgfusepath{clip}%
\pgfsetrectcap%
\pgfsetroundjoin%
\pgfsetlinewidth{0.803000pt}%
\definecolor{currentstroke}{rgb}{0.850000,0.850000,0.850000}%
\pgfsetstrokecolor{currentstroke}%
\pgfsetdash{}{0pt}%
\pgfpathmoveto{\pgfqpoint{0.594525in}{1.029084in}}%
\pgfpathlineto{\pgfqpoint{4.026572in}{1.029084in}}%
\pgfusepath{stroke}%
\end{pgfscope}%
\begin{pgfscope}%
\pgfsetbuttcap%
\pgfsetroundjoin%
\definecolor{currentfill}{rgb}{0.000000,0.000000,0.000000}%
\pgfsetfillcolor{currentfill}%
\pgfsetlinewidth{0.602250pt}%
\definecolor{currentstroke}{rgb}{0.000000,0.000000,0.000000}%
\pgfsetstrokecolor{currentstroke}%
\pgfsetdash{}{0pt}%
\pgfsys@defobject{currentmarker}{\pgfqpoint{-0.027778in}{0.000000in}}{\pgfqpoint{-0.000000in}{0.000000in}}{%
\pgfpathmoveto{\pgfqpoint{-0.000000in}{0.000000in}}%
\pgfpathlineto{\pgfqpoint{-0.027778in}{0.000000in}}%
\pgfusepath{stroke,fill}%
}%
\begin{pgfscope}%
\pgfsys@transformshift{0.594525in}{1.029084in}%
\pgfsys@useobject{currentmarker}{}%
\end{pgfscope}%
\end{pgfscope}%
\begin{pgfscope}%
\pgfpathrectangle{\pgfqpoint{0.594525in}{0.417642in}}{\pgfqpoint{3.432047in}{2.016277in}}%
\pgfusepath{clip}%
\pgfsetrectcap%
\pgfsetroundjoin%
\pgfsetlinewidth{0.803000pt}%
\definecolor{currentstroke}{rgb}{0.850000,0.850000,0.850000}%
\pgfsetstrokecolor{currentstroke}%
\pgfsetdash{}{0pt}%
\pgfpathmoveto{\pgfqpoint{0.594525in}{1.060031in}}%
\pgfpathlineto{\pgfqpoint{4.026572in}{1.060031in}}%
\pgfusepath{stroke}%
\end{pgfscope}%
\begin{pgfscope}%
\pgfsetbuttcap%
\pgfsetroundjoin%
\definecolor{currentfill}{rgb}{0.000000,0.000000,0.000000}%
\pgfsetfillcolor{currentfill}%
\pgfsetlinewidth{0.602250pt}%
\definecolor{currentstroke}{rgb}{0.000000,0.000000,0.000000}%
\pgfsetstrokecolor{currentstroke}%
\pgfsetdash{}{0pt}%
\pgfsys@defobject{currentmarker}{\pgfqpoint{-0.027778in}{0.000000in}}{\pgfqpoint{-0.000000in}{0.000000in}}{%
\pgfpathmoveto{\pgfqpoint{-0.000000in}{0.000000in}}%
\pgfpathlineto{\pgfqpoint{-0.027778in}{0.000000in}}%
\pgfusepath{stroke,fill}%
}%
\begin{pgfscope}%
\pgfsys@transformshift{0.594525in}{1.060031in}%
\pgfsys@useobject{currentmarker}{}%
\end{pgfscope}%
\end{pgfscope}%
\begin{pgfscope}%
\pgfpathrectangle{\pgfqpoint{0.594525in}{0.417642in}}{\pgfqpoint{3.432047in}{2.016277in}}%
\pgfusepath{clip}%
\pgfsetrectcap%
\pgfsetroundjoin%
\pgfsetlinewidth{0.803000pt}%
\definecolor{currentstroke}{rgb}{0.850000,0.850000,0.850000}%
\pgfsetstrokecolor{currentstroke}%
\pgfsetdash{}{0pt}%
\pgfpathmoveto{\pgfqpoint{0.594525in}{1.085316in}}%
\pgfpathlineto{\pgfqpoint{4.026572in}{1.085316in}}%
\pgfusepath{stroke}%
\end{pgfscope}%
\begin{pgfscope}%
\pgfsetbuttcap%
\pgfsetroundjoin%
\definecolor{currentfill}{rgb}{0.000000,0.000000,0.000000}%
\pgfsetfillcolor{currentfill}%
\pgfsetlinewidth{0.602250pt}%
\definecolor{currentstroke}{rgb}{0.000000,0.000000,0.000000}%
\pgfsetstrokecolor{currentstroke}%
\pgfsetdash{}{0pt}%
\pgfsys@defobject{currentmarker}{\pgfqpoint{-0.027778in}{0.000000in}}{\pgfqpoint{-0.000000in}{0.000000in}}{%
\pgfpathmoveto{\pgfqpoint{-0.000000in}{0.000000in}}%
\pgfpathlineto{\pgfqpoint{-0.027778in}{0.000000in}}%
\pgfusepath{stroke,fill}%
}%
\begin{pgfscope}%
\pgfsys@transformshift{0.594525in}{1.085316in}%
\pgfsys@useobject{currentmarker}{}%
\end{pgfscope}%
\end{pgfscope}%
\begin{pgfscope}%
\pgfpathrectangle{\pgfqpoint{0.594525in}{0.417642in}}{\pgfqpoint{3.432047in}{2.016277in}}%
\pgfusepath{clip}%
\pgfsetrectcap%
\pgfsetroundjoin%
\pgfsetlinewidth{0.803000pt}%
\definecolor{currentstroke}{rgb}{0.850000,0.850000,0.850000}%
\pgfsetstrokecolor{currentstroke}%
\pgfsetdash{}{0pt}%
\pgfpathmoveto{\pgfqpoint{0.594525in}{1.106694in}}%
\pgfpathlineto{\pgfqpoint{4.026572in}{1.106694in}}%
\pgfusepath{stroke}%
\end{pgfscope}%
\begin{pgfscope}%
\pgfsetbuttcap%
\pgfsetroundjoin%
\definecolor{currentfill}{rgb}{0.000000,0.000000,0.000000}%
\pgfsetfillcolor{currentfill}%
\pgfsetlinewidth{0.602250pt}%
\definecolor{currentstroke}{rgb}{0.000000,0.000000,0.000000}%
\pgfsetstrokecolor{currentstroke}%
\pgfsetdash{}{0pt}%
\pgfsys@defobject{currentmarker}{\pgfqpoint{-0.027778in}{0.000000in}}{\pgfqpoint{-0.000000in}{0.000000in}}{%
\pgfpathmoveto{\pgfqpoint{-0.000000in}{0.000000in}}%
\pgfpathlineto{\pgfqpoint{-0.027778in}{0.000000in}}%
\pgfusepath{stroke,fill}%
}%
\begin{pgfscope}%
\pgfsys@transformshift{0.594525in}{1.106694in}%
\pgfsys@useobject{currentmarker}{}%
\end{pgfscope}%
\end{pgfscope}%
\begin{pgfscope}%
\pgfpathrectangle{\pgfqpoint{0.594525in}{0.417642in}}{\pgfqpoint{3.432047in}{2.016277in}}%
\pgfusepath{clip}%
\pgfsetrectcap%
\pgfsetroundjoin%
\pgfsetlinewidth{0.803000pt}%
\definecolor{currentstroke}{rgb}{0.850000,0.850000,0.850000}%
\pgfsetstrokecolor{currentstroke}%
\pgfsetdash{}{0pt}%
\pgfpathmoveto{\pgfqpoint{0.594525in}{1.125213in}}%
\pgfpathlineto{\pgfqpoint{4.026572in}{1.125213in}}%
\pgfusepath{stroke}%
\end{pgfscope}%
\begin{pgfscope}%
\pgfsetbuttcap%
\pgfsetroundjoin%
\definecolor{currentfill}{rgb}{0.000000,0.000000,0.000000}%
\pgfsetfillcolor{currentfill}%
\pgfsetlinewidth{0.602250pt}%
\definecolor{currentstroke}{rgb}{0.000000,0.000000,0.000000}%
\pgfsetstrokecolor{currentstroke}%
\pgfsetdash{}{0pt}%
\pgfsys@defobject{currentmarker}{\pgfqpoint{-0.027778in}{0.000000in}}{\pgfqpoint{-0.000000in}{0.000000in}}{%
\pgfpathmoveto{\pgfqpoint{-0.000000in}{0.000000in}}%
\pgfpathlineto{\pgfqpoint{-0.027778in}{0.000000in}}%
\pgfusepath{stroke,fill}%
}%
\begin{pgfscope}%
\pgfsys@transformshift{0.594525in}{1.125213in}%
\pgfsys@useobject{currentmarker}{}%
\end{pgfscope}%
\end{pgfscope}%
\begin{pgfscope}%
\pgfpathrectangle{\pgfqpoint{0.594525in}{0.417642in}}{\pgfqpoint{3.432047in}{2.016277in}}%
\pgfusepath{clip}%
\pgfsetrectcap%
\pgfsetroundjoin%
\pgfsetlinewidth{0.803000pt}%
\definecolor{currentstroke}{rgb}{0.850000,0.850000,0.850000}%
\pgfsetstrokecolor{currentstroke}%
\pgfsetdash{}{0pt}%
\pgfpathmoveto{\pgfqpoint{0.594525in}{1.141548in}}%
\pgfpathlineto{\pgfqpoint{4.026572in}{1.141548in}}%
\pgfusepath{stroke}%
\end{pgfscope}%
\begin{pgfscope}%
\pgfsetbuttcap%
\pgfsetroundjoin%
\definecolor{currentfill}{rgb}{0.000000,0.000000,0.000000}%
\pgfsetfillcolor{currentfill}%
\pgfsetlinewidth{0.602250pt}%
\definecolor{currentstroke}{rgb}{0.000000,0.000000,0.000000}%
\pgfsetstrokecolor{currentstroke}%
\pgfsetdash{}{0pt}%
\pgfsys@defobject{currentmarker}{\pgfqpoint{-0.027778in}{0.000000in}}{\pgfqpoint{-0.000000in}{0.000000in}}{%
\pgfpathmoveto{\pgfqpoint{-0.000000in}{0.000000in}}%
\pgfpathlineto{\pgfqpoint{-0.027778in}{0.000000in}}%
\pgfusepath{stroke,fill}%
}%
\begin{pgfscope}%
\pgfsys@transformshift{0.594525in}{1.141548in}%
\pgfsys@useobject{currentmarker}{}%
\end{pgfscope}%
\end{pgfscope}%
\begin{pgfscope}%
\pgfpathrectangle{\pgfqpoint{0.594525in}{0.417642in}}{\pgfqpoint{3.432047in}{2.016277in}}%
\pgfusepath{clip}%
\pgfsetrectcap%
\pgfsetroundjoin%
\pgfsetlinewidth{0.803000pt}%
\definecolor{currentstroke}{rgb}{0.850000,0.850000,0.850000}%
\pgfsetstrokecolor{currentstroke}%
\pgfsetdash{}{0pt}%
\pgfpathmoveto{\pgfqpoint{0.594525in}{1.252288in}}%
\pgfpathlineto{\pgfqpoint{4.026572in}{1.252288in}}%
\pgfusepath{stroke}%
\end{pgfscope}%
\begin{pgfscope}%
\pgfsetbuttcap%
\pgfsetroundjoin%
\definecolor{currentfill}{rgb}{0.000000,0.000000,0.000000}%
\pgfsetfillcolor{currentfill}%
\pgfsetlinewidth{0.602250pt}%
\definecolor{currentstroke}{rgb}{0.000000,0.000000,0.000000}%
\pgfsetstrokecolor{currentstroke}%
\pgfsetdash{}{0pt}%
\pgfsys@defobject{currentmarker}{\pgfqpoint{-0.027778in}{0.000000in}}{\pgfqpoint{-0.000000in}{0.000000in}}{%
\pgfpathmoveto{\pgfqpoint{-0.000000in}{0.000000in}}%
\pgfpathlineto{\pgfqpoint{-0.027778in}{0.000000in}}%
\pgfusepath{stroke,fill}%
}%
\begin{pgfscope}%
\pgfsys@transformshift{0.594525in}{1.252288in}%
\pgfsys@useobject{currentmarker}{}%
\end{pgfscope}%
\end{pgfscope}%
\begin{pgfscope}%
\pgfpathrectangle{\pgfqpoint{0.594525in}{0.417642in}}{\pgfqpoint{3.432047in}{2.016277in}}%
\pgfusepath{clip}%
\pgfsetrectcap%
\pgfsetroundjoin%
\pgfsetlinewidth{0.803000pt}%
\definecolor{currentstroke}{rgb}{0.850000,0.850000,0.850000}%
\pgfsetstrokecolor{currentstroke}%
\pgfsetdash{}{0pt}%
\pgfpathmoveto{\pgfqpoint{0.594525in}{1.308520in}}%
\pgfpathlineto{\pgfqpoint{4.026572in}{1.308520in}}%
\pgfusepath{stroke}%
\end{pgfscope}%
\begin{pgfscope}%
\pgfsetbuttcap%
\pgfsetroundjoin%
\definecolor{currentfill}{rgb}{0.000000,0.000000,0.000000}%
\pgfsetfillcolor{currentfill}%
\pgfsetlinewidth{0.602250pt}%
\definecolor{currentstroke}{rgb}{0.000000,0.000000,0.000000}%
\pgfsetstrokecolor{currentstroke}%
\pgfsetdash{}{0pt}%
\pgfsys@defobject{currentmarker}{\pgfqpoint{-0.027778in}{0.000000in}}{\pgfqpoint{-0.000000in}{0.000000in}}{%
\pgfpathmoveto{\pgfqpoint{-0.000000in}{0.000000in}}%
\pgfpathlineto{\pgfqpoint{-0.027778in}{0.000000in}}%
\pgfusepath{stroke,fill}%
}%
\begin{pgfscope}%
\pgfsys@transformshift{0.594525in}{1.308520in}%
\pgfsys@useobject{currentmarker}{}%
\end{pgfscope}%
\end{pgfscope}%
\begin{pgfscope}%
\pgfpathrectangle{\pgfqpoint{0.594525in}{0.417642in}}{\pgfqpoint{3.432047in}{2.016277in}}%
\pgfusepath{clip}%
\pgfsetrectcap%
\pgfsetroundjoin%
\pgfsetlinewidth{0.803000pt}%
\definecolor{currentstroke}{rgb}{0.850000,0.850000,0.850000}%
\pgfsetstrokecolor{currentstroke}%
\pgfsetdash{}{0pt}%
\pgfpathmoveto{\pgfqpoint{0.594525in}{1.348417in}}%
\pgfpathlineto{\pgfqpoint{4.026572in}{1.348417in}}%
\pgfusepath{stroke}%
\end{pgfscope}%
\begin{pgfscope}%
\pgfsetbuttcap%
\pgfsetroundjoin%
\definecolor{currentfill}{rgb}{0.000000,0.000000,0.000000}%
\pgfsetfillcolor{currentfill}%
\pgfsetlinewidth{0.602250pt}%
\definecolor{currentstroke}{rgb}{0.000000,0.000000,0.000000}%
\pgfsetstrokecolor{currentstroke}%
\pgfsetdash{}{0pt}%
\pgfsys@defobject{currentmarker}{\pgfqpoint{-0.027778in}{0.000000in}}{\pgfqpoint{-0.000000in}{0.000000in}}{%
\pgfpathmoveto{\pgfqpoint{-0.000000in}{0.000000in}}%
\pgfpathlineto{\pgfqpoint{-0.027778in}{0.000000in}}%
\pgfusepath{stroke,fill}%
}%
\begin{pgfscope}%
\pgfsys@transformshift{0.594525in}{1.348417in}%
\pgfsys@useobject{currentmarker}{}%
\end{pgfscope}%
\end{pgfscope}%
\begin{pgfscope}%
\pgfpathrectangle{\pgfqpoint{0.594525in}{0.417642in}}{\pgfqpoint{3.432047in}{2.016277in}}%
\pgfusepath{clip}%
\pgfsetrectcap%
\pgfsetroundjoin%
\pgfsetlinewidth{0.803000pt}%
\definecolor{currentstroke}{rgb}{0.850000,0.850000,0.850000}%
\pgfsetstrokecolor{currentstroke}%
\pgfsetdash{}{0pt}%
\pgfpathmoveto{\pgfqpoint{0.594525in}{1.379363in}}%
\pgfpathlineto{\pgfqpoint{4.026572in}{1.379363in}}%
\pgfusepath{stroke}%
\end{pgfscope}%
\begin{pgfscope}%
\pgfsetbuttcap%
\pgfsetroundjoin%
\definecolor{currentfill}{rgb}{0.000000,0.000000,0.000000}%
\pgfsetfillcolor{currentfill}%
\pgfsetlinewidth{0.602250pt}%
\definecolor{currentstroke}{rgb}{0.000000,0.000000,0.000000}%
\pgfsetstrokecolor{currentstroke}%
\pgfsetdash{}{0pt}%
\pgfsys@defobject{currentmarker}{\pgfqpoint{-0.027778in}{0.000000in}}{\pgfqpoint{-0.000000in}{0.000000in}}{%
\pgfpathmoveto{\pgfqpoint{-0.000000in}{0.000000in}}%
\pgfpathlineto{\pgfqpoint{-0.027778in}{0.000000in}}%
\pgfusepath{stroke,fill}%
}%
\begin{pgfscope}%
\pgfsys@transformshift{0.594525in}{1.379363in}%
\pgfsys@useobject{currentmarker}{}%
\end{pgfscope}%
\end{pgfscope}%
\begin{pgfscope}%
\pgfpathrectangle{\pgfqpoint{0.594525in}{0.417642in}}{\pgfqpoint{3.432047in}{2.016277in}}%
\pgfusepath{clip}%
\pgfsetrectcap%
\pgfsetroundjoin%
\pgfsetlinewidth{0.803000pt}%
\definecolor{currentstroke}{rgb}{0.850000,0.850000,0.850000}%
\pgfsetstrokecolor{currentstroke}%
\pgfsetdash{}{0pt}%
\pgfpathmoveto{\pgfqpoint{0.594525in}{1.404648in}}%
\pgfpathlineto{\pgfqpoint{4.026572in}{1.404648in}}%
\pgfusepath{stroke}%
\end{pgfscope}%
\begin{pgfscope}%
\pgfsetbuttcap%
\pgfsetroundjoin%
\definecolor{currentfill}{rgb}{0.000000,0.000000,0.000000}%
\pgfsetfillcolor{currentfill}%
\pgfsetlinewidth{0.602250pt}%
\definecolor{currentstroke}{rgb}{0.000000,0.000000,0.000000}%
\pgfsetstrokecolor{currentstroke}%
\pgfsetdash{}{0pt}%
\pgfsys@defobject{currentmarker}{\pgfqpoint{-0.027778in}{0.000000in}}{\pgfqpoint{-0.000000in}{0.000000in}}{%
\pgfpathmoveto{\pgfqpoint{-0.000000in}{0.000000in}}%
\pgfpathlineto{\pgfqpoint{-0.027778in}{0.000000in}}%
\pgfusepath{stroke,fill}%
}%
\begin{pgfscope}%
\pgfsys@transformshift{0.594525in}{1.404648in}%
\pgfsys@useobject{currentmarker}{}%
\end{pgfscope}%
\end{pgfscope}%
\begin{pgfscope}%
\pgfpathrectangle{\pgfqpoint{0.594525in}{0.417642in}}{\pgfqpoint{3.432047in}{2.016277in}}%
\pgfusepath{clip}%
\pgfsetrectcap%
\pgfsetroundjoin%
\pgfsetlinewidth{0.803000pt}%
\definecolor{currentstroke}{rgb}{0.850000,0.850000,0.850000}%
\pgfsetstrokecolor{currentstroke}%
\pgfsetdash{}{0pt}%
\pgfpathmoveto{\pgfqpoint{0.594525in}{1.426027in}}%
\pgfpathlineto{\pgfqpoint{4.026572in}{1.426027in}}%
\pgfusepath{stroke}%
\end{pgfscope}%
\begin{pgfscope}%
\pgfsetbuttcap%
\pgfsetroundjoin%
\definecolor{currentfill}{rgb}{0.000000,0.000000,0.000000}%
\pgfsetfillcolor{currentfill}%
\pgfsetlinewidth{0.602250pt}%
\definecolor{currentstroke}{rgb}{0.000000,0.000000,0.000000}%
\pgfsetstrokecolor{currentstroke}%
\pgfsetdash{}{0pt}%
\pgfsys@defobject{currentmarker}{\pgfqpoint{-0.027778in}{0.000000in}}{\pgfqpoint{-0.000000in}{0.000000in}}{%
\pgfpathmoveto{\pgfqpoint{-0.000000in}{0.000000in}}%
\pgfpathlineto{\pgfqpoint{-0.027778in}{0.000000in}}%
\pgfusepath{stroke,fill}%
}%
\begin{pgfscope}%
\pgfsys@transformshift{0.594525in}{1.426027in}%
\pgfsys@useobject{currentmarker}{}%
\end{pgfscope}%
\end{pgfscope}%
\begin{pgfscope}%
\pgfpathrectangle{\pgfqpoint{0.594525in}{0.417642in}}{\pgfqpoint{3.432047in}{2.016277in}}%
\pgfusepath{clip}%
\pgfsetrectcap%
\pgfsetroundjoin%
\pgfsetlinewidth{0.803000pt}%
\definecolor{currentstroke}{rgb}{0.850000,0.850000,0.850000}%
\pgfsetstrokecolor{currentstroke}%
\pgfsetdash{}{0pt}%
\pgfpathmoveto{\pgfqpoint{0.594525in}{1.444545in}}%
\pgfpathlineto{\pgfqpoint{4.026572in}{1.444545in}}%
\pgfusepath{stroke}%
\end{pgfscope}%
\begin{pgfscope}%
\pgfsetbuttcap%
\pgfsetroundjoin%
\definecolor{currentfill}{rgb}{0.000000,0.000000,0.000000}%
\pgfsetfillcolor{currentfill}%
\pgfsetlinewidth{0.602250pt}%
\definecolor{currentstroke}{rgb}{0.000000,0.000000,0.000000}%
\pgfsetstrokecolor{currentstroke}%
\pgfsetdash{}{0pt}%
\pgfsys@defobject{currentmarker}{\pgfqpoint{-0.027778in}{0.000000in}}{\pgfqpoint{-0.000000in}{0.000000in}}{%
\pgfpathmoveto{\pgfqpoint{-0.000000in}{0.000000in}}%
\pgfpathlineto{\pgfqpoint{-0.027778in}{0.000000in}}%
\pgfusepath{stroke,fill}%
}%
\begin{pgfscope}%
\pgfsys@transformshift{0.594525in}{1.444545in}%
\pgfsys@useobject{currentmarker}{}%
\end{pgfscope}%
\end{pgfscope}%
\begin{pgfscope}%
\pgfpathrectangle{\pgfqpoint{0.594525in}{0.417642in}}{\pgfqpoint{3.432047in}{2.016277in}}%
\pgfusepath{clip}%
\pgfsetrectcap%
\pgfsetroundjoin%
\pgfsetlinewidth{0.803000pt}%
\definecolor{currentstroke}{rgb}{0.850000,0.850000,0.850000}%
\pgfsetstrokecolor{currentstroke}%
\pgfsetdash{}{0pt}%
\pgfpathmoveto{\pgfqpoint{0.594525in}{1.460880in}}%
\pgfpathlineto{\pgfqpoint{4.026572in}{1.460880in}}%
\pgfusepath{stroke}%
\end{pgfscope}%
\begin{pgfscope}%
\pgfsetbuttcap%
\pgfsetroundjoin%
\definecolor{currentfill}{rgb}{0.000000,0.000000,0.000000}%
\pgfsetfillcolor{currentfill}%
\pgfsetlinewidth{0.602250pt}%
\definecolor{currentstroke}{rgb}{0.000000,0.000000,0.000000}%
\pgfsetstrokecolor{currentstroke}%
\pgfsetdash{}{0pt}%
\pgfsys@defobject{currentmarker}{\pgfqpoint{-0.027778in}{0.000000in}}{\pgfqpoint{-0.000000in}{0.000000in}}{%
\pgfpathmoveto{\pgfqpoint{-0.000000in}{0.000000in}}%
\pgfpathlineto{\pgfqpoint{-0.027778in}{0.000000in}}%
\pgfusepath{stroke,fill}%
}%
\begin{pgfscope}%
\pgfsys@transformshift{0.594525in}{1.460880in}%
\pgfsys@useobject{currentmarker}{}%
\end{pgfscope}%
\end{pgfscope}%
\begin{pgfscope}%
\pgfpathrectangle{\pgfqpoint{0.594525in}{0.417642in}}{\pgfqpoint{3.432047in}{2.016277in}}%
\pgfusepath{clip}%
\pgfsetrectcap%
\pgfsetroundjoin%
\pgfsetlinewidth{0.803000pt}%
\definecolor{currentstroke}{rgb}{0.850000,0.850000,0.850000}%
\pgfsetstrokecolor{currentstroke}%
\pgfsetdash{}{0pt}%
\pgfpathmoveto{\pgfqpoint{0.594525in}{1.571620in}}%
\pgfpathlineto{\pgfqpoint{4.026572in}{1.571620in}}%
\pgfusepath{stroke}%
\end{pgfscope}%
\begin{pgfscope}%
\pgfsetbuttcap%
\pgfsetroundjoin%
\definecolor{currentfill}{rgb}{0.000000,0.000000,0.000000}%
\pgfsetfillcolor{currentfill}%
\pgfsetlinewidth{0.602250pt}%
\definecolor{currentstroke}{rgb}{0.000000,0.000000,0.000000}%
\pgfsetstrokecolor{currentstroke}%
\pgfsetdash{}{0pt}%
\pgfsys@defobject{currentmarker}{\pgfqpoint{-0.027778in}{0.000000in}}{\pgfqpoint{-0.000000in}{0.000000in}}{%
\pgfpathmoveto{\pgfqpoint{-0.000000in}{0.000000in}}%
\pgfpathlineto{\pgfqpoint{-0.027778in}{0.000000in}}%
\pgfusepath{stroke,fill}%
}%
\begin{pgfscope}%
\pgfsys@transformshift{0.594525in}{1.571620in}%
\pgfsys@useobject{currentmarker}{}%
\end{pgfscope}%
\end{pgfscope}%
\begin{pgfscope}%
\pgfpathrectangle{\pgfqpoint{0.594525in}{0.417642in}}{\pgfqpoint{3.432047in}{2.016277in}}%
\pgfusepath{clip}%
\pgfsetrectcap%
\pgfsetroundjoin%
\pgfsetlinewidth{0.803000pt}%
\definecolor{currentstroke}{rgb}{0.850000,0.850000,0.850000}%
\pgfsetstrokecolor{currentstroke}%
\pgfsetdash{}{0pt}%
\pgfpathmoveto{\pgfqpoint{0.594525in}{1.627852in}}%
\pgfpathlineto{\pgfqpoint{4.026572in}{1.627852in}}%
\pgfusepath{stroke}%
\end{pgfscope}%
\begin{pgfscope}%
\pgfsetbuttcap%
\pgfsetroundjoin%
\definecolor{currentfill}{rgb}{0.000000,0.000000,0.000000}%
\pgfsetfillcolor{currentfill}%
\pgfsetlinewidth{0.602250pt}%
\definecolor{currentstroke}{rgb}{0.000000,0.000000,0.000000}%
\pgfsetstrokecolor{currentstroke}%
\pgfsetdash{}{0pt}%
\pgfsys@defobject{currentmarker}{\pgfqpoint{-0.027778in}{0.000000in}}{\pgfqpoint{-0.000000in}{0.000000in}}{%
\pgfpathmoveto{\pgfqpoint{-0.000000in}{0.000000in}}%
\pgfpathlineto{\pgfqpoint{-0.027778in}{0.000000in}}%
\pgfusepath{stroke,fill}%
}%
\begin{pgfscope}%
\pgfsys@transformshift{0.594525in}{1.627852in}%
\pgfsys@useobject{currentmarker}{}%
\end{pgfscope}%
\end{pgfscope}%
\begin{pgfscope}%
\pgfpathrectangle{\pgfqpoint{0.594525in}{0.417642in}}{\pgfqpoint{3.432047in}{2.016277in}}%
\pgfusepath{clip}%
\pgfsetrectcap%
\pgfsetroundjoin%
\pgfsetlinewidth{0.803000pt}%
\definecolor{currentstroke}{rgb}{0.850000,0.850000,0.850000}%
\pgfsetstrokecolor{currentstroke}%
\pgfsetdash{}{0pt}%
\pgfpathmoveto{\pgfqpoint{0.594525in}{1.667749in}}%
\pgfpathlineto{\pgfqpoint{4.026572in}{1.667749in}}%
\pgfusepath{stroke}%
\end{pgfscope}%
\begin{pgfscope}%
\pgfsetbuttcap%
\pgfsetroundjoin%
\definecolor{currentfill}{rgb}{0.000000,0.000000,0.000000}%
\pgfsetfillcolor{currentfill}%
\pgfsetlinewidth{0.602250pt}%
\definecolor{currentstroke}{rgb}{0.000000,0.000000,0.000000}%
\pgfsetstrokecolor{currentstroke}%
\pgfsetdash{}{0pt}%
\pgfsys@defobject{currentmarker}{\pgfqpoint{-0.027778in}{0.000000in}}{\pgfqpoint{-0.000000in}{0.000000in}}{%
\pgfpathmoveto{\pgfqpoint{-0.000000in}{0.000000in}}%
\pgfpathlineto{\pgfqpoint{-0.027778in}{0.000000in}}%
\pgfusepath{stroke,fill}%
}%
\begin{pgfscope}%
\pgfsys@transformshift{0.594525in}{1.667749in}%
\pgfsys@useobject{currentmarker}{}%
\end{pgfscope}%
\end{pgfscope}%
\begin{pgfscope}%
\pgfpathrectangle{\pgfqpoint{0.594525in}{0.417642in}}{\pgfqpoint{3.432047in}{2.016277in}}%
\pgfusepath{clip}%
\pgfsetrectcap%
\pgfsetroundjoin%
\pgfsetlinewidth{0.803000pt}%
\definecolor{currentstroke}{rgb}{0.850000,0.850000,0.850000}%
\pgfsetstrokecolor{currentstroke}%
\pgfsetdash{}{0pt}%
\pgfpathmoveto{\pgfqpoint{0.594525in}{1.698696in}}%
\pgfpathlineto{\pgfqpoint{4.026572in}{1.698696in}}%
\pgfusepath{stroke}%
\end{pgfscope}%
\begin{pgfscope}%
\pgfsetbuttcap%
\pgfsetroundjoin%
\definecolor{currentfill}{rgb}{0.000000,0.000000,0.000000}%
\pgfsetfillcolor{currentfill}%
\pgfsetlinewidth{0.602250pt}%
\definecolor{currentstroke}{rgb}{0.000000,0.000000,0.000000}%
\pgfsetstrokecolor{currentstroke}%
\pgfsetdash{}{0pt}%
\pgfsys@defobject{currentmarker}{\pgfqpoint{-0.027778in}{0.000000in}}{\pgfqpoint{-0.000000in}{0.000000in}}{%
\pgfpathmoveto{\pgfqpoint{-0.000000in}{0.000000in}}%
\pgfpathlineto{\pgfqpoint{-0.027778in}{0.000000in}}%
\pgfusepath{stroke,fill}%
}%
\begin{pgfscope}%
\pgfsys@transformshift{0.594525in}{1.698696in}%
\pgfsys@useobject{currentmarker}{}%
\end{pgfscope}%
\end{pgfscope}%
\begin{pgfscope}%
\pgfpathrectangle{\pgfqpoint{0.594525in}{0.417642in}}{\pgfqpoint{3.432047in}{2.016277in}}%
\pgfusepath{clip}%
\pgfsetrectcap%
\pgfsetroundjoin%
\pgfsetlinewidth{0.803000pt}%
\definecolor{currentstroke}{rgb}{0.850000,0.850000,0.850000}%
\pgfsetstrokecolor{currentstroke}%
\pgfsetdash{}{0pt}%
\pgfpathmoveto{\pgfqpoint{0.594525in}{1.723981in}}%
\pgfpathlineto{\pgfqpoint{4.026572in}{1.723981in}}%
\pgfusepath{stroke}%
\end{pgfscope}%
\begin{pgfscope}%
\pgfsetbuttcap%
\pgfsetroundjoin%
\definecolor{currentfill}{rgb}{0.000000,0.000000,0.000000}%
\pgfsetfillcolor{currentfill}%
\pgfsetlinewidth{0.602250pt}%
\definecolor{currentstroke}{rgb}{0.000000,0.000000,0.000000}%
\pgfsetstrokecolor{currentstroke}%
\pgfsetdash{}{0pt}%
\pgfsys@defobject{currentmarker}{\pgfqpoint{-0.027778in}{0.000000in}}{\pgfqpoint{-0.000000in}{0.000000in}}{%
\pgfpathmoveto{\pgfqpoint{-0.000000in}{0.000000in}}%
\pgfpathlineto{\pgfqpoint{-0.027778in}{0.000000in}}%
\pgfusepath{stroke,fill}%
}%
\begin{pgfscope}%
\pgfsys@transformshift{0.594525in}{1.723981in}%
\pgfsys@useobject{currentmarker}{}%
\end{pgfscope}%
\end{pgfscope}%
\begin{pgfscope}%
\pgfpathrectangle{\pgfqpoint{0.594525in}{0.417642in}}{\pgfqpoint{3.432047in}{2.016277in}}%
\pgfusepath{clip}%
\pgfsetrectcap%
\pgfsetroundjoin%
\pgfsetlinewidth{0.803000pt}%
\definecolor{currentstroke}{rgb}{0.850000,0.850000,0.850000}%
\pgfsetstrokecolor{currentstroke}%
\pgfsetdash{}{0pt}%
\pgfpathmoveto{\pgfqpoint{0.594525in}{1.745359in}}%
\pgfpathlineto{\pgfqpoint{4.026572in}{1.745359in}}%
\pgfusepath{stroke}%
\end{pgfscope}%
\begin{pgfscope}%
\pgfsetbuttcap%
\pgfsetroundjoin%
\definecolor{currentfill}{rgb}{0.000000,0.000000,0.000000}%
\pgfsetfillcolor{currentfill}%
\pgfsetlinewidth{0.602250pt}%
\definecolor{currentstroke}{rgb}{0.000000,0.000000,0.000000}%
\pgfsetstrokecolor{currentstroke}%
\pgfsetdash{}{0pt}%
\pgfsys@defobject{currentmarker}{\pgfqpoint{-0.027778in}{0.000000in}}{\pgfqpoint{-0.000000in}{0.000000in}}{%
\pgfpathmoveto{\pgfqpoint{-0.000000in}{0.000000in}}%
\pgfpathlineto{\pgfqpoint{-0.027778in}{0.000000in}}%
\pgfusepath{stroke,fill}%
}%
\begin{pgfscope}%
\pgfsys@transformshift{0.594525in}{1.745359in}%
\pgfsys@useobject{currentmarker}{}%
\end{pgfscope}%
\end{pgfscope}%
\begin{pgfscope}%
\pgfpathrectangle{\pgfqpoint{0.594525in}{0.417642in}}{\pgfqpoint{3.432047in}{2.016277in}}%
\pgfusepath{clip}%
\pgfsetrectcap%
\pgfsetroundjoin%
\pgfsetlinewidth{0.803000pt}%
\definecolor{currentstroke}{rgb}{0.850000,0.850000,0.850000}%
\pgfsetstrokecolor{currentstroke}%
\pgfsetdash{}{0pt}%
\pgfpathmoveto{\pgfqpoint{0.594525in}{1.763878in}}%
\pgfpathlineto{\pgfqpoint{4.026572in}{1.763878in}}%
\pgfusepath{stroke}%
\end{pgfscope}%
\begin{pgfscope}%
\pgfsetbuttcap%
\pgfsetroundjoin%
\definecolor{currentfill}{rgb}{0.000000,0.000000,0.000000}%
\pgfsetfillcolor{currentfill}%
\pgfsetlinewidth{0.602250pt}%
\definecolor{currentstroke}{rgb}{0.000000,0.000000,0.000000}%
\pgfsetstrokecolor{currentstroke}%
\pgfsetdash{}{0pt}%
\pgfsys@defobject{currentmarker}{\pgfqpoint{-0.027778in}{0.000000in}}{\pgfqpoint{-0.000000in}{0.000000in}}{%
\pgfpathmoveto{\pgfqpoint{-0.000000in}{0.000000in}}%
\pgfpathlineto{\pgfqpoint{-0.027778in}{0.000000in}}%
\pgfusepath{stroke,fill}%
}%
\begin{pgfscope}%
\pgfsys@transformshift{0.594525in}{1.763878in}%
\pgfsys@useobject{currentmarker}{}%
\end{pgfscope}%
\end{pgfscope}%
\begin{pgfscope}%
\pgfpathrectangle{\pgfqpoint{0.594525in}{0.417642in}}{\pgfqpoint{3.432047in}{2.016277in}}%
\pgfusepath{clip}%
\pgfsetrectcap%
\pgfsetroundjoin%
\pgfsetlinewidth{0.803000pt}%
\definecolor{currentstroke}{rgb}{0.850000,0.850000,0.850000}%
\pgfsetstrokecolor{currentstroke}%
\pgfsetdash{}{0pt}%
\pgfpathmoveto{\pgfqpoint{0.594525in}{1.780212in}}%
\pgfpathlineto{\pgfqpoint{4.026572in}{1.780212in}}%
\pgfusepath{stroke}%
\end{pgfscope}%
\begin{pgfscope}%
\pgfsetbuttcap%
\pgfsetroundjoin%
\definecolor{currentfill}{rgb}{0.000000,0.000000,0.000000}%
\pgfsetfillcolor{currentfill}%
\pgfsetlinewidth{0.602250pt}%
\definecolor{currentstroke}{rgb}{0.000000,0.000000,0.000000}%
\pgfsetstrokecolor{currentstroke}%
\pgfsetdash{}{0pt}%
\pgfsys@defobject{currentmarker}{\pgfqpoint{-0.027778in}{0.000000in}}{\pgfqpoint{-0.000000in}{0.000000in}}{%
\pgfpathmoveto{\pgfqpoint{-0.000000in}{0.000000in}}%
\pgfpathlineto{\pgfqpoint{-0.027778in}{0.000000in}}%
\pgfusepath{stroke,fill}%
}%
\begin{pgfscope}%
\pgfsys@transformshift{0.594525in}{1.780212in}%
\pgfsys@useobject{currentmarker}{}%
\end{pgfscope}%
\end{pgfscope}%
\begin{pgfscope}%
\pgfpathrectangle{\pgfqpoint{0.594525in}{0.417642in}}{\pgfqpoint{3.432047in}{2.016277in}}%
\pgfusepath{clip}%
\pgfsetrectcap%
\pgfsetroundjoin%
\pgfsetlinewidth{0.803000pt}%
\definecolor{currentstroke}{rgb}{0.850000,0.850000,0.850000}%
\pgfsetstrokecolor{currentstroke}%
\pgfsetdash{}{0pt}%
\pgfpathmoveto{\pgfqpoint{0.594525in}{1.890953in}}%
\pgfpathlineto{\pgfqpoint{4.026572in}{1.890953in}}%
\pgfusepath{stroke}%
\end{pgfscope}%
\begin{pgfscope}%
\pgfsetbuttcap%
\pgfsetroundjoin%
\definecolor{currentfill}{rgb}{0.000000,0.000000,0.000000}%
\pgfsetfillcolor{currentfill}%
\pgfsetlinewidth{0.602250pt}%
\definecolor{currentstroke}{rgb}{0.000000,0.000000,0.000000}%
\pgfsetstrokecolor{currentstroke}%
\pgfsetdash{}{0pt}%
\pgfsys@defobject{currentmarker}{\pgfqpoint{-0.027778in}{0.000000in}}{\pgfqpoint{-0.000000in}{0.000000in}}{%
\pgfpathmoveto{\pgfqpoint{-0.000000in}{0.000000in}}%
\pgfpathlineto{\pgfqpoint{-0.027778in}{0.000000in}}%
\pgfusepath{stroke,fill}%
}%
\begin{pgfscope}%
\pgfsys@transformshift{0.594525in}{1.890953in}%
\pgfsys@useobject{currentmarker}{}%
\end{pgfscope}%
\end{pgfscope}%
\begin{pgfscope}%
\pgfpathrectangle{\pgfqpoint{0.594525in}{0.417642in}}{\pgfqpoint{3.432047in}{2.016277in}}%
\pgfusepath{clip}%
\pgfsetrectcap%
\pgfsetroundjoin%
\pgfsetlinewidth{0.803000pt}%
\definecolor{currentstroke}{rgb}{0.850000,0.850000,0.850000}%
\pgfsetstrokecolor{currentstroke}%
\pgfsetdash{}{0pt}%
\pgfpathmoveto{\pgfqpoint{0.594525in}{1.947184in}}%
\pgfpathlineto{\pgfqpoint{4.026572in}{1.947184in}}%
\pgfusepath{stroke}%
\end{pgfscope}%
\begin{pgfscope}%
\pgfsetbuttcap%
\pgfsetroundjoin%
\definecolor{currentfill}{rgb}{0.000000,0.000000,0.000000}%
\pgfsetfillcolor{currentfill}%
\pgfsetlinewidth{0.602250pt}%
\definecolor{currentstroke}{rgb}{0.000000,0.000000,0.000000}%
\pgfsetstrokecolor{currentstroke}%
\pgfsetdash{}{0pt}%
\pgfsys@defobject{currentmarker}{\pgfqpoint{-0.027778in}{0.000000in}}{\pgfqpoint{-0.000000in}{0.000000in}}{%
\pgfpathmoveto{\pgfqpoint{-0.000000in}{0.000000in}}%
\pgfpathlineto{\pgfqpoint{-0.027778in}{0.000000in}}%
\pgfusepath{stroke,fill}%
}%
\begin{pgfscope}%
\pgfsys@transformshift{0.594525in}{1.947184in}%
\pgfsys@useobject{currentmarker}{}%
\end{pgfscope}%
\end{pgfscope}%
\begin{pgfscope}%
\pgfpathrectangle{\pgfqpoint{0.594525in}{0.417642in}}{\pgfqpoint{3.432047in}{2.016277in}}%
\pgfusepath{clip}%
\pgfsetrectcap%
\pgfsetroundjoin%
\pgfsetlinewidth{0.803000pt}%
\definecolor{currentstroke}{rgb}{0.850000,0.850000,0.850000}%
\pgfsetstrokecolor{currentstroke}%
\pgfsetdash{}{0pt}%
\pgfpathmoveto{\pgfqpoint{0.594525in}{1.987081in}}%
\pgfpathlineto{\pgfqpoint{4.026572in}{1.987081in}}%
\pgfusepath{stroke}%
\end{pgfscope}%
\begin{pgfscope}%
\pgfsetbuttcap%
\pgfsetroundjoin%
\definecolor{currentfill}{rgb}{0.000000,0.000000,0.000000}%
\pgfsetfillcolor{currentfill}%
\pgfsetlinewidth{0.602250pt}%
\definecolor{currentstroke}{rgb}{0.000000,0.000000,0.000000}%
\pgfsetstrokecolor{currentstroke}%
\pgfsetdash{}{0pt}%
\pgfsys@defobject{currentmarker}{\pgfqpoint{-0.027778in}{0.000000in}}{\pgfqpoint{-0.000000in}{0.000000in}}{%
\pgfpathmoveto{\pgfqpoint{-0.000000in}{0.000000in}}%
\pgfpathlineto{\pgfqpoint{-0.027778in}{0.000000in}}%
\pgfusepath{stroke,fill}%
}%
\begin{pgfscope}%
\pgfsys@transformshift{0.594525in}{1.987081in}%
\pgfsys@useobject{currentmarker}{}%
\end{pgfscope}%
\end{pgfscope}%
\begin{pgfscope}%
\pgfpathrectangle{\pgfqpoint{0.594525in}{0.417642in}}{\pgfqpoint{3.432047in}{2.016277in}}%
\pgfusepath{clip}%
\pgfsetrectcap%
\pgfsetroundjoin%
\pgfsetlinewidth{0.803000pt}%
\definecolor{currentstroke}{rgb}{0.850000,0.850000,0.850000}%
\pgfsetstrokecolor{currentstroke}%
\pgfsetdash{}{0pt}%
\pgfpathmoveto{\pgfqpoint{0.594525in}{2.018028in}}%
\pgfpathlineto{\pgfqpoint{4.026572in}{2.018028in}}%
\pgfusepath{stroke}%
\end{pgfscope}%
\begin{pgfscope}%
\pgfsetbuttcap%
\pgfsetroundjoin%
\definecolor{currentfill}{rgb}{0.000000,0.000000,0.000000}%
\pgfsetfillcolor{currentfill}%
\pgfsetlinewidth{0.602250pt}%
\definecolor{currentstroke}{rgb}{0.000000,0.000000,0.000000}%
\pgfsetstrokecolor{currentstroke}%
\pgfsetdash{}{0pt}%
\pgfsys@defobject{currentmarker}{\pgfqpoint{-0.027778in}{0.000000in}}{\pgfqpoint{-0.000000in}{0.000000in}}{%
\pgfpathmoveto{\pgfqpoint{-0.000000in}{0.000000in}}%
\pgfpathlineto{\pgfqpoint{-0.027778in}{0.000000in}}%
\pgfusepath{stroke,fill}%
}%
\begin{pgfscope}%
\pgfsys@transformshift{0.594525in}{2.018028in}%
\pgfsys@useobject{currentmarker}{}%
\end{pgfscope}%
\end{pgfscope}%
\begin{pgfscope}%
\pgfpathrectangle{\pgfqpoint{0.594525in}{0.417642in}}{\pgfqpoint{3.432047in}{2.016277in}}%
\pgfusepath{clip}%
\pgfsetrectcap%
\pgfsetroundjoin%
\pgfsetlinewidth{0.803000pt}%
\definecolor{currentstroke}{rgb}{0.850000,0.850000,0.850000}%
\pgfsetstrokecolor{currentstroke}%
\pgfsetdash{}{0pt}%
\pgfpathmoveto{\pgfqpoint{0.594525in}{2.043313in}}%
\pgfpathlineto{\pgfqpoint{4.026572in}{2.043313in}}%
\pgfusepath{stroke}%
\end{pgfscope}%
\begin{pgfscope}%
\pgfsetbuttcap%
\pgfsetroundjoin%
\definecolor{currentfill}{rgb}{0.000000,0.000000,0.000000}%
\pgfsetfillcolor{currentfill}%
\pgfsetlinewidth{0.602250pt}%
\definecolor{currentstroke}{rgb}{0.000000,0.000000,0.000000}%
\pgfsetstrokecolor{currentstroke}%
\pgfsetdash{}{0pt}%
\pgfsys@defobject{currentmarker}{\pgfqpoint{-0.027778in}{0.000000in}}{\pgfqpoint{-0.000000in}{0.000000in}}{%
\pgfpathmoveto{\pgfqpoint{-0.000000in}{0.000000in}}%
\pgfpathlineto{\pgfqpoint{-0.027778in}{0.000000in}}%
\pgfusepath{stroke,fill}%
}%
\begin{pgfscope}%
\pgfsys@transformshift{0.594525in}{2.043313in}%
\pgfsys@useobject{currentmarker}{}%
\end{pgfscope}%
\end{pgfscope}%
\begin{pgfscope}%
\pgfpathrectangle{\pgfqpoint{0.594525in}{0.417642in}}{\pgfqpoint{3.432047in}{2.016277in}}%
\pgfusepath{clip}%
\pgfsetrectcap%
\pgfsetroundjoin%
\pgfsetlinewidth{0.803000pt}%
\definecolor{currentstroke}{rgb}{0.850000,0.850000,0.850000}%
\pgfsetstrokecolor{currentstroke}%
\pgfsetdash{}{0pt}%
\pgfpathmoveto{\pgfqpoint{0.594525in}{2.064691in}}%
\pgfpathlineto{\pgfqpoint{4.026572in}{2.064691in}}%
\pgfusepath{stroke}%
\end{pgfscope}%
\begin{pgfscope}%
\pgfsetbuttcap%
\pgfsetroundjoin%
\definecolor{currentfill}{rgb}{0.000000,0.000000,0.000000}%
\pgfsetfillcolor{currentfill}%
\pgfsetlinewidth{0.602250pt}%
\definecolor{currentstroke}{rgb}{0.000000,0.000000,0.000000}%
\pgfsetstrokecolor{currentstroke}%
\pgfsetdash{}{0pt}%
\pgfsys@defobject{currentmarker}{\pgfqpoint{-0.027778in}{0.000000in}}{\pgfqpoint{-0.000000in}{0.000000in}}{%
\pgfpathmoveto{\pgfqpoint{-0.000000in}{0.000000in}}%
\pgfpathlineto{\pgfqpoint{-0.027778in}{0.000000in}}%
\pgfusepath{stroke,fill}%
}%
\begin{pgfscope}%
\pgfsys@transformshift{0.594525in}{2.064691in}%
\pgfsys@useobject{currentmarker}{}%
\end{pgfscope}%
\end{pgfscope}%
\begin{pgfscope}%
\pgfpathrectangle{\pgfqpoint{0.594525in}{0.417642in}}{\pgfqpoint{3.432047in}{2.016277in}}%
\pgfusepath{clip}%
\pgfsetrectcap%
\pgfsetroundjoin%
\pgfsetlinewidth{0.803000pt}%
\definecolor{currentstroke}{rgb}{0.850000,0.850000,0.850000}%
\pgfsetstrokecolor{currentstroke}%
\pgfsetdash{}{0pt}%
\pgfpathmoveto{\pgfqpoint{0.594525in}{2.083210in}}%
\pgfpathlineto{\pgfqpoint{4.026572in}{2.083210in}}%
\pgfusepath{stroke}%
\end{pgfscope}%
\begin{pgfscope}%
\pgfsetbuttcap%
\pgfsetroundjoin%
\definecolor{currentfill}{rgb}{0.000000,0.000000,0.000000}%
\pgfsetfillcolor{currentfill}%
\pgfsetlinewidth{0.602250pt}%
\definecolor{currentstroke}{rgb}{0.000000,0.000000,0.000000}%
\pgfsetstrokecolor{currentstroke}%
\pgfsetdash{}{0pt}%
\pgfsys@defobject{currentmarker}{\pgfqpoint{-0.027778in}{0.000000in}}{\pgfqpoint{-0.000000in}{0.000000in}}{%
\pgfpathmoveto{\pgfqpoint{-0.000000in}{0.000000in}}%
\pgfpathlineto{\pgfqpoint{-0.027778in}{0.000000in}}%
\pgfusepath{stroke,fill}%
}%
\begin{pgfscope}%
\pgfsys@transformshift{0.594525in}{2.083210in}%
\pgfsys@useobject{currentmarker}{}%
\end{pgfscope}%
\end{pgfscope}%
\begin{pgfscope}%
\pgfpathrectangle{\pgfqpoint{0.594525in}{0.417642in}}{\pgfqpoint{3.432047in}{2.016277in}}%
\pgfusepath{clip}%
\pgfsetrectcap%
\pgfsetroundjoin%
\pgfsetlinewidth{0.803000pt}%
\definecolor{currentstroke}{rgb}{0.850000,0.850000,0.850000}%
\pgfsetstrokecolor{currentstroke}%
\pgfsetdash{}{0pt}%
\pgfpathmoveto{\pgfqpoint{0.594525in}{2.099545in}}%
\pgfpathlineto{\pgfqpoint{4.026572in}{2.099545in}}%
\pgfusepath{stroke}%
\end{pgfscope}%
\begin{pgfscope}%
\pgfsetbuttcap%
\pgfsetroundjoin%
\definecolor{currentfill}{rgb}{0.000000,0.000000,0.000000}%
\pgfsetfillcolor{currentfill}%
\pgfsetlinewidth{0.602250pt}%
\definecolor{currentstroke}{rgb}{0.000000,0.000000,0.000000}%
\pgfsetstrokecolor{currentstroke}%
\pgfsetdash{}{0pt}%
\pgfsys@defobject{currentmarker}{\pgfqpoint{-0.027778in}{0.000000in}}{\pgfqpoint{-0.000000in}{0.000000in}}{%
\pgfpathmoveto{\pgfqpoint{-0.000000in}{0.000000in}}%
\pgfpathlineto{\pgfqpoint{-0.027778in}{0.000000in}}%
\pgfusepath{stroke,fill}%
}%
\begin{pgfscope}%
\pgfsys@transformshift{0.594525in}{2.099545in}%
\pgfsys@useobject{currentmarker}{}%
\end{pgfscope}%
\end{pgfscope}%
\begin{pgfscope}%
\pgfpathrectangle{\pgfqpoint{0.594525in}{0.417642in}}{\pgfqpoint{3.432047in}{2.016277in}}%
\pgfusepath{clip}%
\pgfsetrectcap%
\pgfsetroundjoin%
\pgfsetlinewidth{0.803000pt}%
\definecolor{currentstroke}{rgb}{0.850000,0.850000,0.850000}%
\pgfsetstrokecolor{currentstroke}%
\pgfsetdash{}{0pt}%
\pgfpathmoveto{\pgfqpoint{0.594525in}{2.210285in}}%
\pgfpathlineto{\pgfqpoint{4.026572in}{2.210285in}}%
\pgfusepath{stroke}%
\end{pgfscope}%
\begin{pgfscope}%
\pgfsetbuttcap%
\pgfsetroundjoin%
\definecolor{currentfill}{rgb}{0.000000,0.000000,0.000000}%
\pgfsetfillcolor{currentfill}%
\pgfsetlinewidth{0.602250pt}%
\definecolor{currentstroke}{rgb}{0.000000,0.000000,0.000000}%
\pgfsetstrokecolor{currentstroke}%
\pgfsetdash{}{0pt}%
\pgfsys@defobject{currentmarker}{\pgfqpoint{-0.027778in}{0.000000in}}{\pgfqpoint{-0.000000in}{0.000000in}}{%
\pgfpathmoveto{\pgfqpoint{-0.000000in}{0.000000in}}%
\pgfpathlineto{\pgfqpoint{-0.027778in}{0.000000in}}%
\pgfusepath{stroke,fill}%
}%
\begin{pgfscope}%
\pgfsys@transformshift{0.594525in}{2.210285in}%
\pgfsys@useobject{currentmarker}{}%
\end{pgfscope}%
\end{pgfscope}%
\begin{pgfscope}%
\pgfpathrectangle{\pgfqpoint{0.594525in}{0.417642in}}{\pgfqpoint{3.432047in}{2.016277in}}%
\pgfusepath{clip}%
\pgfsetrectcap%
\pgfsetroundjoin%
\pgfsetlinewidth{0.803000pt}%
\definecolor{currentstroke}{rgb}{0.850000,0.850000,0.850000}%
\pgfsetstrokecolor{currentstroke}%
\pgfsetdash{}{0pt}%
\pgfpathmoveto{\pgfqpoint{0.594525in}{2.266517in}}%
\pgfpathlineto{\pgfqpoint{4.026572in}{2.266517in}}%
\pgfusepath{stroke}%
\end{pgfscope}%
\begin{pgfscope}%
\pgfsetbuttcap%
\pgfsetroundjoin%
\definecolor{currentfill}{rgb}{0.000000,0.000000,0.000000}%
\pgfsetfillcolor{currentfill}%
\pgfsetlinewidth{0.602250pt}%
\definecolor{currentstroke}{rgb}{0.000000,0.000000,0.000000}%
\pgfsetstrokecolor{currentstroke}%
\pgfsetdash{}{0pt}%
\pgfsys@defobject{currentmarker}{\pgfqpoint{-0.027778in}{0.000000in}}{\pgfqpoint{-0.000000in}{0.000000in}}{%
\pgfpathmoveto{\pgfqpoint{-0.000000in}{0.000000in}}%
\pgfpathlineto{\pgfqpoint{-0.027778in}{0.000000in}}%
\pgfusepath{stroke,fill}%
}%
\begin{pgfscope}%
\pgfsys@transformshift{0.594525in}{2.266517in}%
\pgfsys@useobject{currentmarker}{}%
\end{pgfscope}%
\end{pgfscope}%
\begin{pgfscope}%
\pgfpathrectangle{\pgfqpoint{0.594525in}{0.417642in}}{\pgfqpoint{3.432047in}{2.016277in}}%
\pgfusepath{clip}%
\pgfsetrectcap%
\pgfsetroundjoin%
\pgfsetlinewidth{0.803000pt}%
\definecolor{currentstroke}{rgb}{0.850000,0.850000,0.850000}%
\pgfsetstrokecolor{currentstroke}%
\pgfsetdash{}{0pt}%
\pgfpathmoveto{\pgfqpoint{0.594525in}{2.306414in}}%
\pgfpathlineto{\pgfqpoint{4.026572in}{2.306414in}}%
\pgfusepath{stroke}%
\end{pgfscope}%
\begin{pgfscope}%
\pgfsetbuttcap%
\pgfsetroundjoin%
\definecolor{currentfill}{rgb}{0.000000,0.000000,0.000000}%
\pgfsetfillcolor{currentfill}%
\pgfsetlinewidth{0.602250pt}%
\definecolor{currentstroke}{rgb}{0.000000,0.000000,0.000000}%
\pgfsetstrokecolor{currentstroke}%
\pgfsetdash{}{0pt}%
\pgfsys@defobject{currentmarker}{\pgfqpoint{-0.027778in}{0.000000in}}{\pgfqpoint{-0.000000in}{0.000000in}}{%
\pgfpathmoveto{\pgfqpoint{-0.000000in}{0.000000in}}%
\pgfpathlineto{\pgfqpoint{-0.027778in}{0.000000in}}%
\pgfusepath{stroke,fill}%
}%
\begin{pgfscope}%
\pgfsys@transformshift{0.594525in}{2.306414in}%
\pgfsys@useobject{currentmarker}{}%
\end{pgfscope}%
\end{pgfscope}%
\begin{pgfscope}%
\pgfpathrectangle{\pgfqpoint{0.594525in}{0.417642in}}{\pgfqpoint{3.432047in}{2.016277in}}%
\pgfusepath{clip}%
\pgfsetrectcap%
\pgfsetroundjoin%
\pgfsetlinewidth{0.803000pt}%
\definecolor{currentstroke}{rgb}{0.850000,0.850000,0.850000}%
\pgfsetstrokecolor{currentstroke}%
\pgfsetdash{}{0pt}%
\pgfpathmoveto{\pgfqpoint{0.594525in}{2.337360in}}%
\pgfpathlineto{\pgfqpoint{4.026572in}{2.337360in}}%
\pgfusepath{stroke}%
\end{pgfscope}%
\begin{pgfscope}%
\pgfsetbuttcap%
\pgfsetroundjoin%
\definecolor{currentfill}{rgb}{0.000000,0.000000,0.000000}%
\pgfsetfillcolor{currentfill}%
\pgfsetlinewidth{0.602250pt}%
\definecolor{currentstroke}{rgb}{0.000000,0.000000,0.000000}%
\pgfsetstrokecolor{currentstroke}%
\pgfsetdash{}{0pt}%
\pgfsys@defobject{currentmarker}{\pgfqpoint{-0.027778in}{0.000000in}}{\pgfqpoint{-0.000000in}{0.000000in}}{%
\pgfpathmoveto{\pgfqpoint{-0.000000in}{0.000000in}}%
\pgfpathlineto{\pgfqpoint{-0.027778in}{0.000000in}}%
\pgfusepath{stroke,fill}%
}%
\begin{pgfscope}%
\pgfsys@transformshift{0.594525in}{2.337360in}%
\pgfsys@useobject{currentmarker}{}%
\end{pgfscope}%
\end{pgfscope}%
\begin{pgfscope}%
\pgfpathrectangle{\pgfqpoint{0.594525in}{0.417642in}}{\pgfqpoint{3.432047in}{2.016277in}}%
\pgfusepath{clip}%
\pgfsetrectcap%
\pgfsetroundjoin%
\pgfsetlinewidth{0.803000pt}%
\definecolor{currentstroke}{rgb}{0.850000,0.850000,0.850000}%
\pgfsetstrokecolor{currentstroke}%
\pgfsetdash{}{0pt}%
\pgfpathmoveto{\pgfqpoint{0.594525in}{2.362645in}}%
\pgfpathlineto{\pgfqpoint{4.026572in}{2.362645in}}%
\pgfusepath{stroke}%
\end{pgfscope}%
\begin{pgfscope}%
\pgfsetbuttcap%
\pgfsetroundjoin%
\definecolor{currentfill}{rgb}{0.000000,0.000000,0.000000}%
\pgfsetfillcolor{currentfill}%
\pgfsetlinewidth{0.602250pt}%
\definecolor{currentstroke}{rgb}{0.000000,0.000000,0.000000}%
\pgfsetstrokecolor{currentstroke}%
\pgfsetdash{}{0pt}%
\pgfsys@defobject{currentmarker}{\pgfqpoint{-0.027778in}{0.000000in}}{\pgfqpoint{-0.000000in}{0.000000in}}{%
\pgfpathmoveto{\pgfqpoint{-0.000000in}{0.000000in}}%
\pgfpathlineto{\pgfqpoint{-0.027778in}{0.000000in}}%
\pgfusepath{stroke,fill}%
}%
\begin{pgfscope}%
\pgfsys@transformshift{0.594525in}{2.362645in}%
\pgfsys@useobject{currentmarker}{}%
\end{pgfscope}%
\end{pgfscope}%
\begin{pgfscope}%
\pgfpathrectangle{\pgfqpoint{0.594525in}{0.417642in}}{\pgfqpoint{3.432047in}{2.016277in}}%
\pgfusepath{clip}%
\pgfsetrectcap%
\pgfsetroundjoin%
\pgfsetlinewidth{0.803000pt}%
\definecolor{currentstroke}{rgb}{0.850000,0.850000,0.850000}%
\pgfsetstrokecolor{currentstroke}%
\pgfsetdash{}{0pt}%
\pgfpathmoveto{\pgfqpoint{0.594525in}{2.384024in}}%
\pgfpathlineto{\pgfqpoint{4.026572in}{2.384024in}}%
\pgfusepath{stroke}%
\end{pgfscope}%
\begin{pgfscope}%
\pgfsetbuttcap%
\pgfsetroundjoin%
\definecolor{currentfill}{rgb}{0.000000,0.000000,0.000000}%
\pgfsetfillcolor{currentfill}%
\pgfsetlinewidth{0.602250pt}%
\definecolor{currentstroke}{rgb}{0.000000,0.000000,0.000000}%
\pgfsetstrokecolor{currentstroke}%
\pgfsetdash{}{0pt}%
\pgfsys@defobject{currentmarker}{\pgfqpoint{-0.027778in}{0.000000in}}{\pgfqpoint{-0.000000in}{0.000000in}}{%
\pgfpathmoveto{\pgfqpoint{-0.000000in}{0.000000in}}%
\pgfpathlineto{\pgfqpoint{-0.027778in}{0.000000in}}%
\pgfusepath{stroke,fill}%
}%
\begin{pgfscope}%
\pgfsys@transformshift{0.594525in}{2.384024in}%
\pgfsys@useobject{currentmarker}{}%
\end{pgfscope}%
\end{pgfscope}%
\begin{pgfscope}%
\pgfpathrectangle{\pgfqpoint{0.594525in}{0.417642in}}{\pgfqpoint{3.432047in}{2.016277in}}%
\pgfusepath{clip}%
\pgfsetrectcap%
\pgfsetroundjoin%
\pgfsetlinewidth{0.803000pt}%
\definecolor{currentstroke}{rgb}{0.850000,0.850000,0.850000}%
\pgfsetstrokecolor{currentstroke}%
\pgfsetdash{}{0pt}%
\pgfpathmoveto{\pgfqpoint{0.594525in}{2.402542in}}%
\pgfpathlineto{\pgfqpoint{4.026572in}{2.402542in}}%
\pgfusepath{stroke}%
\end{pgfscope}%
\begin{pgfscope}%
\pgfsetbuttcap%
\pgfsetroundjoin%
\definecolor{currentfill}{rgb}{0.000000,0.000000,0.000000}%
\pgfsetfillcolor{currentfill}%
\pgfsetlinewidth{0.602250pt}%
\definecolor{currentstroke}{rgb}{0.000000,0.000000,0.000000}%
\pgfsetstrokecolor{currentstroke}%
\pgfsetdash{}{0pt}%
\pgfsys@defobject{currentmarker}{\pgfqpoint{-0.027778in}{0.000000in}}{\pgfqpoint{-0.000000in}{0.000000in}}{%
\pgfpathmoveto{\pgfqpoint{-0.000000in}{0.000000in}}%
\pgfpathlineto{\pgfqpoint{-0.027778in}{0.000000in}}%
\pgfusepath{stroke,fill}%
}%
\begin{pgfscope}%
\pgfsys@transformshift{0.594525in}{2.402542in}%
\pgfsys@useobject{currentmarker}{}%
\end{pgfscope}%
\end{pgfscope}%
\begin{pgfscope}%
\pgfpathrectangle{\pgfqpoint{0.594525in}{0.417642in}}{\pgfqpoint{3.432047in}{2.016277in}}%
\pgfusepath{clip}%
\pgfsetrectcap%
\pgfsetroundjoin%
\pgfsetlinewidth{0.803000pt}%
\definecolor{currentstroke}{rgb}{0.850000,0.850000,0.850000}%
\pgfsetstrokecolor{currentstroke}%
\pgfsetdash{}{0pt}%
\pgfpathmoveto{\pgfqpoint{0.594525in}{2.418877in}}%
\pgfpathlineto{\pgfqpoint{4.026572in}{2.418877in}}%
\pgfusepath{stroke}%
\end{pgfscope}%
\begin{pgfscope}%
\pgfsetbuttcap%
\pgfsetroundjoin%
\definecolor{currentfill}{rgb}{0.000000,0.000000,0.000000}%
\pgfsetfillcolor{currentfill}%
\pgfsetlinewidth{0.602250pt}%
\definecolor{currentstroke}{rgb}{0.000000,0.000000,0.000000}%
\pgfsetstrokecolor{currentstroke}%
\pgfsetdash{}{0pt}%
\pgfsys@defobject{currentmarker}{\pgfqpoint{-0.027778in}{0.000000in}}{\pgfqpoint{-0.000000in}{0.000000in}}{%
\pgfpathmoveto{\pgfqpoint{-0.000000in}{0.000000in}}%
\pgfpathlineto{\pgfqpoint{-0.027778in}{0.000000in}}%
\pgfusepath{stroke,fill}%
}%
\begin{pgfscope}%
\pgfsys@transformshift{0.594525in}{2.418877in}%
\pgfsys@useobject{currentmarker}{}%
\end{pgfscope}%
\end{pgfscope}%
\begin{pgfscope}%
\definecolor{textcolor}{rgb}{0.000000,0.000000,0.000000}%
\pgfsetstrokecolor{textcolor}%
\pgfsetfillcolor{textcolor}%
\pgftext[x=0.185574in,y=1.425780in,,bottom,rotate=90.000000]{\color{textcolor}\rmfamily\fontsize{10.000000}{12.000000}\selectfont \(\displaystyle S_y(f)\) in \(\displaystyle \unit{1 \per \Hz}\)}%
\end{pgfscope}%
\begin{pgfscope}%
\pgfpathrectangle{\pgfqpoint{0.594525in}{0.417642in}}{\pgfqpoint{3.432047in}{2.016277in}}%
\pgfusepath{clip}%
\pgfsetbuttcap%
\pgfsetroundjoin%
\pgfsetlinewidth{1.505625pt}%
\definecolor{currentstroke}{rgb}{0.003922,0.450980,0.698039}%
\pgfsetstrokecolor{currentstroke}%
\pgfsetdash{{5.550000pt}{2.400000pt}}{0.000000pt}%
\pgfpathmoveto{\pgfqpoint{0.750527in}{1.947423in}}%
\pgfpathlineto{\pgfqpoint{1.690929in}{1.946274in}}%
\pgfpathlineto{\pgfqpoint{1.880739in}{1.943924in}}%
\pgfpathlineto{\pgfqpoint{2.001715in}{1.940370in}}%
\pgfpathlineto{\pgfqpoint{2.093949in}{1.935485in}}%
\pgfpathlineto{\pgfqpoint{2.171567in}{1.928998in}}%
\pgfpathlineto{\pgfqpoint{2.234689in}{1.921426in}}%
\pgfpathlineto{\pgfqpoint{2.289705in}{1.912626in}}%
\pgfpathlineto{\pgfqpoint{2.343283in}{1.901638in}}%
\pgfpathlineto{\pgfqpoint{2.390889in}{1.889543in}}%
\pgfpathlineto{\pgfqpoint{2.438527in}{1.874986in}}%
\pgfpathlineto{\pgfqpoint{2.485423in}{1.858085in}}%
\pgfpathlineto{\pgfqpoint{2.531985in}{1.838696in}}%
\pgfpathlineto{\pgfqpoint{2.578384in}{1.816803in}}%
\pgfpathlineto{\pgfqpoint{2.625316in}{1.792152in}}%
\pgfpathlineto{\pgfqpoint{2.672456in}{1.765047in}}%
\pgfpathlineto{\pgfqpoint{2.727080in}{1.731029in}}%
\pgfpathlineto{\pgfqpoint{2.790249in}{1.688740in}}%
\pgfpathlineto{\pgfqpoint{2.860754in}{1.638588in}}%
\pgfpathlineto{\pgfqpoint{2.946854in}{1.574265in}}%
\pgfpathlineto{\pgfqpoint{3.056241in}{1.489357in}}%
\pgfpathlineto{\pgfqpoint{3.205059in}{1.370671in}}%
\pgfpathlineto{\pgfqpoint{3.447823in}{1.173807in}}%
\pgfpathlineto{\pgfqpoint{3.870569in}{0.828580in}}%
\pgfpathlineto{\pgfqpoint{3.870569in}{0.828580in}}%
\pgfusepath{stroke}%
\end{pgfscope}%
\begin{pgfscope}%
\pgfpathrectangle{\pgfqpoint{0.594525in}{0.417642in}}{\pgfqpoint{3.432047in}{2.016277in}}%
\pgfusepath{clip}%
\pgfsetbuttcap%
\pgfsetroundjoin%
\definecolor{currentfill}{rgb}{0.003922,0.450980,0.698039}%
\pgfsetfillcolor{currentfill}%
\pgfsetlinewidth{1.003750pt}%
\definecolor{currentstroke}{rgb}{0.003922,0.450980,0.698039}%
\pgfsetstrokecolor{currentstroke}%
\pgfsetdash{}{0pt}%
\pgfsys@defobject{currentmarker}{\pgfqpoint{-0.006944in}{-0.006944in}}{\pgfqpoint{0.006944in}{0.006944in}}{%
\pgfpathmoveto{\pgfqpoint{0.000000in}{-0.006944in}}%
\pgfpathcurveto{\pgfqpoint{0.001842in}{-0.006944in}}{\pgfqpoint{0.003608in}{-0.006213in}}{\pgfqpoint{0.004910in}{-0.004910in}}%
\pgfpathcurveto{\pgfqpoint{0.006213in}{-0.003608in}}{\pgfqpoint{0.006944in}{-0.001842in}}{\pgfqpoint{0.006944in}{0.000000in}}%
\pgfpathcurveto{\pgfqpoint{0.006944in}{0.001842in}}{\pgfqpoint{0.006213in}{0.003608in}}{\pgfqpoint{0.004910in}{0.004910in}}%
\pgfpathcurveto{\pgfqpoint{0.003608in}{0.006213in}}{\pgfqpoint{0.001842in}{0.006944in}}{\pgfqpoint{0.000000in}{0.006944in}}%
\pgfpathcurveto{\pgfqpoint{-0.001842in}{0.006944in}}{\pgfqpoint{-0.003608in}{0.006213in}}{\pgfqpoint{-0.004910in}{0.004910in}}%
\pgfpathcurveto{\pgfqpoint{-0.006213in}{0.003608in}}{\pgfqpoint{-0.006944in}{0.001842in}}{\pgfqpoint{-0.006944in}{0.000000in}}%
\pgfpathcurveto{\pgfqpoint{-0.006944in}{-0.001842in}}{\pgfqpoint{-0.006213in}{-0.003608in}}{\pgfqpoint{-0.004910in}{-0.004910in}}%
\pgfpathcurveto{\pgfqpoint{-0.003608in}{-0.006213in}}{\pgfqpoint{-0.001842in}{-0.006944in}}{\pgfqpoint{0.000000in}{-0.006944in}}%
\pgfpathlineto{\pgfqpoint{0.000000in}{-0.006944in}}%
\pgfpathclose%
\pgfusepath{stroke,fill}%
}%
\begin{pgfscope}%
\pgfsys@transformshift{0.750527in}{1.928874in}%
\pgfsys@useobject{currentmarker}{}%
\end{pgfscope}%
\begin{pgfscope}%
\pgfsys@transformshift{0.985627in}{1.949594in}%
\pgfsys@useobject{currentmarker}{}%
\end{pgfscope}%
\begin{pgfscope}%
\pgfsys@transformshift{1.123152in}{1.957338in}%
\pgfsys@useobject{currentmarker}{}%
\end{pgfscope}%
\begin{pgfscope}%
\pgfsys@transformshift{1.220728in}{1.944865in}%
\pgfsys@useobject{currentmarker}{}%
\end{pgfscope}%
\begin{pgfscope}%
\pgfsys@transformshift{1.296413in}{1.941415in}%
\pgfsys@useobject{currentmarker}{}%
\end{pgfscope}%
\begin{pgfscope}%
\pgfsys@transformshift{1.358253in}{1.936367in}%
\pgfsys@useobject{currentmarker}{}%
\end{pgfscope}%
\begin{pgfscope}%
\pgfsys@transformshift{1.410538in}{1.941183in}%
\pgfsys@useobject{currentmarker}{}%
\end{pgfscope}%
\begin{pgfscope}%
\pgfsys@transformshift{1.455829in}{1.946675in}%
\pgfsys@useobject{currentmarker}{}%
\end{pgfscope}%
\begin{pgfscope}%
\pgfsys@transformshift{1.495778in}{1.950385in}%
\pgfsys@useobject{currentmarker}{}%
\end{pgfscope}%
\begin{pgfscope}%
\pgfsys@transformshift{1.531514in}{1.938377in}%
\pgfsys@useobject{currentmarker}{}%
\end{pgfscope}%
\begin{pgfscope}%
\pgfsys@transformshift{1.563841in}{1.929005in}%
\pgfsys@useobject{currentmarker}{}%
\end{pgfscope}%
\begin{pgfscope}%
\pgfsys@transformshift{1.593354in}{1.936773in}%
\pgfsys@useobject{currentmarker}{}%
\end{pgfscope}%
\begin{pgfscope}%
\pgfsys@transformshift{1.620502in}{1.939118in}%
\pgfsys@useobject{currentmarker}{}%
\end{pgfscope}%
\begin{pgfscope}%
\pgfsys@transformshift{1.645638in}{1.944714in}%
\pgfsys@useobject{currentmarker}{}%
\end{pgfscope}%
\begin{pgfscope}%
\pgfsys@transformshift{1.669039in}{1.952663in}%
\pgfsys@useobject{currentmarker}{}%
\end{pgfscope}%
\begin{pgfscope}%
\pgfsys@transformshift{1.690929in}{1.947396in}%
\pgfsys@useobject{currentmarker}{}%
\end{pgfscope}%
\begin{pgfscope}%
\pgfsys@transformshift{1.711492in}{1.926045in}%
\pgfsys@useobject{currentmarker}{}%
\end{pgfscope}%
\begin{pgfscope}%
\pgfsys@transformshift{1.730879in}{1.947583in}%
\pgfsys@useobject{currentmarker}{}%
\end{pgfscope}%
\begin{pgfscope}%
\pgfsys@transformshift{1.749217in}{1.940807in}%
\pgfsys@useobject{currentmarker}{}%
\end{pgfscope}%
\begin{pgfscope}%
\pgfsys@transformshift{1.766615in}{1.942597in}%
\pgfsys@useobject{currentmarker}{}%
\end{pgfscope}%
\begin{pgfscope}%
\pgfsys@transformshift{1.783163in}{1.941872in}%
\pgfsys@useobject{currentmarker}{}%
\end{pgfscope}%
\begin{pgfscope}%
\pgfsys@transformshift{1.798942in}{1.945008in}%
\pgfsys@useobject{currentmarker}{}%
\end{pgfscope}%
\begin{pgfscope}%
\pgfsys@transformshift{1.814019in}{1.951307in}%
\pgfsys@useobject{currentmarker}{}%
\end{pgfscope}%
\begin{pgfscope}%
\pgfsys@transformshift{1.828454in}{1.939010in}%
\pgfsys@useobject{currentmarker}{}%
\end{pgfscope}%
\begin{pgfscope}%
\pgfsys@transformshift{1.842300in}{1.921499in}%
\pgfsys@useobject{currentmarker}{}%
\end{pgfscope}%
\begin{pgfscope}%
\pgfsys@transformshift{1.855603in}{1.934294in}%
\pgfsys@useobject{currentmarker}{}%
\end{pgfscope}%
\begin{pgfscope}%
\pgfsys@transformshift{1.868404in}{1.940051in}%
\pgfsys@useobject{currentmarker}{}%
\end{pgfscope}%
\begin{pgfscope}%
\pgfsys@transformshift{1.880739in}{1.936995in}%
\pgfsys@useobject{currentmarker}{}%
\end{pgfscope}%
\begin{pgfscope}%
\pgfsys@transformshift{1.892641in}{1.925546in}%
\pgfsys@useobject{currentmarker}{}%
\end{pgfscope}%
\begin{pgfscope}%
\pgfsys@transformshift{1.904140in}{1.937562in}%
\pgfsys@useobject{currentmarker}{}%
\end{pgfscope}%
\begin{pgfscope}%
\pgfsys@transformshift{1.915261in}{1.926059in}%
\pgfsys@useobject{currentmarker}{}%
\end{pgfscope}%
\begin{pgfscope}%
\pgfsys@transformshift{1.926030in}{1.929280in}%
\pgfsys@useobject{currentmarker}{}%
\end{pgfscope}%
\begin{pgfscope}%
\pgfsys@transformshift{1.936467in}{1.931825in}%
\pgfsys@useobject{currentmarker}{}%
\end{pgfscope}%
\begin{pgfscope}%
\pgfsys@transformshift{1.946592in}{1.930487in}%
\pgfsys@useobject{currentmarker}{}%
\end{pgfscope}%
\begin{pgfscope}%
\pgfsys@transformshift{1.956424in}{1.943180in}%
\pgfsys@useobject{currentmarker}{}%
\end{pgfscope}%
\begin{pgfscope}%
\pgfsys@transformshift{1.965979in}{1.958029in}%
\pgfsys@useobject{currentmarker}{}%
\end{pgfscope}%
\begin{pgfscope}%
\pgfsys@transformshift{1.975272in}{1.950325in}%
\pgfsys@useobject{currentmarker}{}%
\end{pgfscope}%
\begin{pgfscope}%
\pgfsys@transformshift{1.984318in}{1.922211in}%
\pgfsys@useobject{currentmarker}{}%
\end{pgfscope}%
\begin{pgfscope}%
\pgfsys@transformshift{1.993128in}{1.924279in}%
\pgfsys@useobject{currentmarker}{}%
\end{pgfscope}%
\begin{pgfscope}%
\pgfsys@transformshift{2.001715in}{1.925673in}%
\pgfsys@useobject{currentmarker}{}%
\end{pgfscope}%
\begin{pgfscope}%
\pgfsys@transformshift{2.010090in}{1.922900in}%
\pgfsys@useobject{currentmarker}{}%
\end{pgfscope}%
\begin{pgfscope}%
\pgfsys@transformshift{2.018264in}{1.922806in}%
\pgfsys@useobject{currentmarker}{}%
\end{pgfscope}%
\begin{pgfscope}%
\pgfsys@transformshift{2.026245in}{1.926506in}%
\pgfsys@useobject{currentmarker}{}%
\end{pgfscope}%
\begin{pgfscope}%
\pgfsys@transformshift{2.034042in}{1.931899in}%
\pgfsys@useobject{currentmarker}{}%
\end{pgfscope}%
\begin{pgfscope}%
\pgfsys@transformshift{2.041665in}{1.938074in}%
\pgfsys@useobject{currentmarker}{}%
\end{pgfscope}%
\begin{pgfscope}%
\pgfsys@transformshift{2.049119in}{1.929735in}%
\pgfsys@useobject{currentmarker}{}%
\end{pgfscope}%
\begin{pgfscope}%
\pgfsys@transformshift{2.056414in}{1.944196in}%
\pgfsys@useobject{currentmarker}{}%
\end{pgfscope}%
\begin{pgfscope}%
\pgfsys@transformshift{2.063555in}{1.943092in}%
\pgfsys@useobject{currentmarker}{}%
\end{pgfscope}%
\begin{pgfscope}%
\pgfsys@transformshift{2.070548in}{1.924422in}%
\pgfsys@useobject{currentmarker}{}%
\end{pgfscope}%
\begin{pgfscope}%
\pgfsys@transformshift{2.077401in}{1.925785in}%
\pgfsys@useobject{currentmarker}{}%
\end{pgfscope}%
\begin{pgfscope}%
\pgfsys@transformshift{2.084117in}{1.925073in}%
\pgfsys@useobject{currentmarker}{}%
\end{pgfscope}%
\begin{pgfscope}%
\pgfsys@transformshift{2.093949in}{1.935161in}%
\pgfsys@useobject{currentmarker}{}%
\end{pgfscope}%
\begin{pgfscope}%
\pgfsys@transformshift{2.103504in}{1.943385in}%
\pgfsys@useobject{currentmarker}{}%
\end{pgfscope}%
\begin{pgfscope}%
\pgfsys@transformshift{2.109728in}{1.936490in}%
\pgfsys@useobject{currentmarker}{}%
\end{pgfscope}%
\begin{pgfscope}%
\pgfsys@transformshift{2.115839in}{1.930926in}%
\pgfsys@useobject{currentmarker}{}%
\end{pgfscope}%
\begin{pgfscope}%
\pgfsys@transformshift{2.124805in}{1.942939in}%
\pgfsys@useobject{currentmarker}{}%
\end{pgfscope}%
\begin{pgfscope}%
\pgfsys@transformshift{2.133540in}{1.936087in}%
\pgfsys@useobject{currentmarker}{}%
\end{pgfscope}%
\begin{pgfscope}%
\pgfsys@transformshift{2.139240in}{1.936039in}%
\pgfsys@useobject{currentmarker}{}%
\end{pgfscope}%
\begin{pgfscope}%
\pgfsys@transformshift{2.147615in}{1.931423in}%
\pgfsys@useobject{currentmarker}{}%
\end{pgfscope}%
\begin{pgfscope}%
\pgfsys@transformshift{2.155789in}{1.926676in}%
\pgfsys@useobject{currentmarker}{}%
\end{pgfscope}%
\begin{pgfscope}%
\pgfsys@transformshift{2.163770in}{1.935270in}%
\pgfsys@useobject{currentmarker}{}%
\end{pgfscope}%
\begin{pgfscope}%
\pgfsys@transformshift{2.171567in}{1.928472in}%
\pgfsys@useobject{currentmarker}{}%
\end{pgfscope}%
\begin{pgfscope}%
\pgfsys@transformshift{2.179190in}{1.912586in}%
\pgfsys@useobject{currentmarker}{}%
\end{pgfscope}%
\begin{pgfscope}%
\pgfsys@transformshift{2.186644in}{1.912375in}%
\pgfsys@useobject{currentmarker}{}%
\end{pgfscope}%
\begin{pgfscope}%
\pgfsys@transformshift{2.193939in}{1.924428in}%
\pgfsys@useobject{currentmarker}{}%
\end{pgfscope}%
\begin{pgfscope}%
\pgfsys@transformshift{2.203427in}{1.922481in}%
\pgfsys@useobject{currentmarker}{}%
\end{pgfscope}%
\begin{pgfscope}%
\pgfsys@transformshift{2.210373in}{1.942305in}%
\pgfsys@useobject{currentmarker}{}%
\end{pgfscope}%
\begin{pgfscope}%
\pgfsys@transformshift{2.217179in}{1.920947in}%
\pgfsys@useobject{currentmarker}{}%
\end{pgfscope}%
\begin{pgfscope}%
\pgfsys@transformshift{2.226047in}{1.912864in}%
\pgfsys@useobject{currentmarker}{}%
\end{pgfscope}%
\begin{pgfscope}%
\pgfsys@transformshift{2.234689in}{1.913822in}%
\pgfsys@useobject{currentmarker}{}%
\end{pgfscope}%
\begin{pgfscope}%
\pgfsys@transformshift{2.243116in}{1.902053in}%
\pgfsys@useobject{currentmarker}{}%
\end{pgfscope}%
\begin{pgfscope}%
\pgfsys@transformshift{2.251339in}{1.915023in}%
\pgfsys@useobject{currentmarker}{}%
\end{pgfscope}%
\begin{pgfscope}%
\pgfsys@transformshift{2.259368in}{1.927677in}%
\pgfsys@useobject{currentmarker}{}%
\end{pgfscope}%
\begin{pgfscope}%
\pgfsys@transformshift{2.267210in}{1.915267in}%
\pgfsys@useobject{currentmarker}{}%
\end{pgfscope}%
\begin{pgfscope}%
\pgfsys@transformshift{2.274876in}{1.911942in}%
\pgfsys@useobject{currentmarker}{}%
\end{pgfscope}%
\begin{pgfscope}%
\pgfsys@transformshift{2.282372in}{1.920130in}%
\pgfsys@useobject{currentmarker}{}%
\end{pgfscope}%
\begin{pgfscope}%
\pgfsys@transformshift{2.289705in}{1.912516in}%
\pgfsys@useobject{currentmarker}{}%
\end{pgfscope}%
\begin{pgfscope}%
\pgfsys@transformshift{2.296884in}{1.911121in}%
\pgfsys@useobject{currentmarker}{}%
\end{pgfscope}%
\begin{pgfscope}%
\pgfsys@transformshift{2.303914in}{1.911113in}%
\pgfsys@useobject{currentmarker}{}%
\end{pgfscope}%
\begin{pgfscope}%
\pgfsys@transformshift{2.312501in}{1.908461in}%
\pgfsys@useobject{currentmarker}{}%
\end{pgfscope}%
\begin{pgfscope}%
\pgfsys@transformshift{2.320876in}{1.906886in}%
\pgfsys@useobject{currentmarker}{}%
\end{pgfscope}%
\begin{pgfscope}%
\pgfsys@transformshift{2.327431in}{1.899330in}%
\pgfsys@useobject{currentmarker}{}%
\end{pgfscope}%
\begin{pgfscope}%
\pgfsys@transformshift{2.335450in}{1.903407in}%
\pgfsys@useobject{currentmarker}{}%
\end{pgfscope}%
\begin{pgfscope}%
\pgfsys@transformshift{2.343283in}{1.900802in}%
\pgfsys@useobject{currentmarker}{}%
\end{pgfscope}%
\begin{pgfscope}%
\pgfsys@transformshift{2.350940in}{1.906052in}%
\pgfsys@useobject{currentmarker}{}%
\end{pgfscope}%
\begin{pgfscope}%
\pgfsys@transformshift{2.359905in}{1.893294in}%
\pgfsys@useobject{currentmarker}{}%
\end{pgfscope}%
\begin{pgfscope}%
\pgfsys@transformshift{2.367200in}{1.879564in}%
\pgfsys@useobject{currentmarker}{}%
\end{pgfscope}%
\begin{pgfscope}%
\pgfsys@transformshift{2.374341in}{1.893276in}%
\pgfsys@useobject{currentmarker}{}%
\end{pgfscope}%
\begin{pgfscope}%
\pgfsys@transformshift{2.382716in}{1.880078in}%
\pgfsys@useobject{currentmarker}{}%
\end{pgfscope}%
\begin{pgfscope}%
\pgfsys@transformshift{2.390889in}{1.889316in}%
\pgfsys@useobject{currentmarker}{}%
\end{pgfscope}%
\begin{pgfscope}%
\pgfsys@transformshift{2.398870in}{1.887966in}%
\pgfsys@useobject{currentmarker}{}%
\end{pgfscope}%
\begin{pgfscope}%
\pgfsys@transformshift{2.406668in}{1.876536in}%
\pgfsys@useobject{currentmarker}{}%
\end{pgfscope}%
\begin{pgfscope}%
\pgfsys@transformshift{2.414290in}{1.869427in}%
\pgfsys@useobject{currentmarker}{}%
\end{pgfscope}%
\begin{pgfscope}%
\pgfsys@transformshift{2.421745in}{1.872920in}%
\pgfsys@useobject{currentmarker}{}%
\end{pgfscope}%
\begin{pgfscope}%
\pgfsys@transformshift{2.430240in}{1.865832in}%
\pgfsys@useobject{currentmarker}{}%
\end{pgfscope}%
\begin{pgfscope}%
\pgfsys@transformshift{2.438527in}{1.881926in}%
\pgfsys@useobject{currentmarker}{}%
\end{pgfscope}%
\begin{pgfscope}%
\pgfsys@transformshift{2.445473in}{1.873196in}%
\pgfsys@useobject{currentmarker}{}%
\end{pgfscope}%
\begin{pgfscope}%
\pgfsys@transformshift{2.453401in}{1.866123in}%
\pgfsys@useobject{currentmarker}{}%
\end{pgfscope}%
\begin{pgfscope}%
\pgfsys@transformshift{2.461148in}{1.859446in}%
\pgfsys@useobject{currentmarker}{}%
\end{pgfscope}%
\begin{pgfscope}%
\pgfsys@transformshift{2.468721in}{1.855960in}%
\pgfsys@useobject{currentmarker}{}%
\end{pgfscope}%
\begin{pgfscope}%
\pgfsys@transformshift{2.477175in}{1.859998in}%
\pgfsys@useobject{currentmarker}{}%
\end{pgfscope}%
\begin{pgfscope}%
\pgfsys@transformshift{2.485423in}{1.852361in}%
\pgfsys@useobject{currentmarker}{}%
\end{pgfscope}%
\begin{pgfscope}%
\pgfsys@transformshift{2.493475in}{1.859905in}%
\pgfsys@useobject{currentmarker}{}%
\end{pgfscope}%
\begin{pgfscope}%
\pgfsys@transformshift{2.501340in}{1.851489in}%
\pgfsys@useobject{currentmarker}{}%
\end{pgfscope}%
\begin{pgfscope}%
\pgfsys@transformshift{2.509027in}{1.854478in}%
\pgfsys@useobject{currentmarker}{}%
\end{pgfscope}%
\begin{pgfscope}%
\pgfsys@transformshift{2.516544in}{1.841466in}%
\pgfsys@useobject{currentmarker}{}%
\end{pgfscope}%
\begin{pgfscope}%
\pgfsys@transformshift{2.523898in}{1.845763in}%
\pgfsys@useobject{currentmarker}{}%
\end{pgfscope}%
\begin{pgfscope}%
\pgfsys@transformshift{2.531985in}{1.836507in}%
\pgfsys@useobject{currentmarker}{}%
\end{pgfscope}%
\begin{pgfscope}%
\pgfsys@transformshift{2.539883in}{1.822836in}%
\pgfsys@useobject{currentmarker}{}%
\end{pgfscope}%
\begin{pgfscope}%
\pgfsys@transformshift{2.547602in}{1.829430in}%
\pgfsys@useobject{currentmarker}{}%
\end{pgfscope}%
\begin{pgfscope}%
\pgfsys@transformshift{2.555149in}{1.831423in}%
\pgfsys@useobject{currentmarker}{}%
\end{pgfscope}%
\begin{pgfscope}%
\pgfsys@transformshift{2.562531in}{1.829970in}%
\pgfsys@useobject{currentmarker}{}%
\end{pgfscope}%
\begin{pgfscope}%
\pgfsys@transformshift{2.570550in}{1.807216in}%
\pgfsys@useobject{currentmarker}{}%
\end{pgfscope}%
\begin{pgfscope}%
\pgfsys@transformshift{2.578384in}{1.818835in}%
\pgfsys@useobject{currentmarker}{}%
\end{pgfscope}%
\begin{pgfscope}%
\pgfsys@transformshift{2.586040in}{1.809862in}%
\pgfsys@useobject{currentmarker}{}%
\end{pgfscope}%
\begin{pgfscope}%
\pgfsys@transformshift{2.594268in}{1.799094in}%
\pgfsys@useobject{currentmarker}{}%
\end{pgfscope}%
\begin{pgfscope}%
\pgfsys@transformshift{2.602300in}{1.799365in}%
\pgfsys@useobject{currentmarker}{}%
\end{pgfscope}%
\begin{pgfscope}%
\pgfsys@transformshift{2.610147in}{1.805535in}%
\pgfsys@useobject{currentmarker}{}%
\end{pgfscope}%
\begin{pgfscope}%
\pgfsys@transformshift{2.617816in}{1.793230in}%
\pgfsys@useobject{currentmarker}{}%
\end{pgfscope}%
\begin{pgfscope}%
\pgfsys@transformshift{2.625316in}{1.785349in}%
\pgfsys@useobject{currentmarker}{}%
\end{pgfscope}%
\begin{pgfscope}%
\pgfsys@transformshift{2.633313in}{1.786791in}%
\pgfsys@useobject{currentmarker}{}%
\end{pgfscope}%
\begin{pgfscope}%
\pgfsys@transformshift{2.641125in}{1.785999in}%
\pgfsys@useobject{currentmarker}{}%
\end{pgfscope}%
\begin{pgfscope}%
\pgfsys@transformshift{2.648762in}{1.784933in}%
\pgfsys@useobject{currentmarker}{}%
\end{pgfscope}%
\begin{pgfscope}%
\pgfsys@transformshift{2.656845in}{1.771944in}%
\pgfsys@useobject{currentmarker}{}%
\end{pgfscope}%
\begin{pgfscope}%
\pgfsys@transformshift{2.664741in}{1.775236in}%
\pgfsys@useobject{currentmarker}{}%
\end{pgfscope}%
\begin{pgfscope}%
\pgfsys@transformshift{2.672456in}{1.762444in}%
\pgfsys@useobject{currentmarker}{}%
\end{pgfscope}%
\begin{pgfscope}%
\pgfsys@transformshift{2.680574in}{1.769577in}%
\pgfsys@useobject{currentmarker}{}%
\end{pgfscope}%
\begin{pgfscope}%
\pgfsys@transformshift{2.688502in}{1.758800in}%
\pgfsys@useobject{currentmarker}{}%
\end{pgfscope}%
\begin{pgfscope}%
\pgfsys@transformshift{2.696248in}{1.757204in}%
\pgfsys@useobject{currentmarker}{}%
\end{pgfscope}%
\begin{pgfscope}%
\pgfsys@transformshift{2.703822in}{1.752826in}%
\pgfsys@useobject{currentmarker}{}%
\end{pgfscope}%
\begin{pgfscope}%
\pgfsys@transformshift{2.711753in}{1.735788in}%
\pgfsys@useobject{currentmarker}{}%
\end{pgfscope}%
\begin{pgfscope}%
\pgfsys@transformshift{2.719503in}{1.740777in}%
\pgfsys@useobject{currentmarker}{}%
\end{pgfscope}%
\begin{pgfscope}%
\pgfsys@transformshift{2.727080in}{1.730526in}%
\pgfsys@useobject{currentmarker}{}%
\end{pgfscope}%
\begin{pgfscope}%
\pgfsys@transformshift{2.734980in}{1.725072in}%
\pgfsys@useobject{currentmarker}{}%
\end{pgfscope}%
\begin{pgfscope}%
\pgfsys@transformshift{2.743177in}{1.718298in}%
\pgfsys@useobject{currentmarker}{}%
\end{pgfscope}%
\begin{pgfscope}%
\pgfsys@transformshift{2.751180in}{1.715113in}%
\pgfsys@useobject{currentmarker}{}%
\end{pgfscope}%
\begin{pgfscope}%
\pgfsys@transformshift{2.758998in}{1.702223in}%
\pgfsys@useobject{currentmarker}{}%
\end{pgfscope}%
\begin{pgfscope}%
\pgfsys@transformshift{2.766641in}{1.695459in}%
\pgfsys@useobject{currentmarker}{}%
\end{pgfscope}%
\begin{pgfscope}%
\pgfsys@transformshift{2.774115in}{1.701715in}%
\pgfsys@useobject{currentmarker}{}%
\end{pgfscope}%
\begin{pgfscope}%
\pgfsys@transformshift{2.782278in}{1.696444in}%
\pgfsys@useobject{currentmarker}{}%
\end{pgfscope}%
\begin{pgfscope}%
\pgfsys@transformshift{2.790249in}{1.692675in}%
\pgfsys@useobject{currentmarker}{}%
\end{pgfscope}%
\begin{pgfscope}%
\pgfsys@transformshift{2.798037in}{1.680659in}%
\pgfsys@useobject{currentmarker}{}%
\end{pgfscope}%
\begin{pgfscope}%
\pgfsys@transformshift{2.806047in}{1.675688in}%
\pgfsys@useobject{currentmarker}{}%
\end{pgfscope}%
\begin{pgfscope}%
\pgfsys@transformshift{2.813871in}{1.666706in}%
\pgfsys@useobject{currentmarker}{}%
\end{pgfscope}%
\begin{pgfscope}%
\pgfsys@transformshift{2.821519in}{1.670644in}%
\pgfsys@useobject{currentmarker}{}%
\end{pgfscope}%
\begin{pgfscope}%
\pgfsys@transformshift{2.829368in}{1.661094in}%
\pgfsys@useobject{currentmarker}{}%
\end{pgfscope}%
\begin{pgfscope}%
\pgfsys@transformshift{2.837040in}{1.652782in}%
\pgfsys@useobject{currentmarker}{}%
\end{pgfscope}%
\begin{pgfscope}%
\pgfsys@transformshift{2.844895in}{1.653153in}%
\pgfsys@useobject{currentmarker}{}%
\end{pgfscope}%
\begin{pgfscope}%
\pgfsys@transformshift{2.852917in}{1.649206in}%
\pgfsys@useobject{currentmarker}{}%
\end{pgfscope}%
\begin{pgfscope}%
\pgfsys@transformshift{2.860754in}{1.637217in}%
\pgfsys@useobject{currentmarker}{}%
\end{pgfscope}%
\begin{pgfscope}%
\pgfsys@transformshift{2.868413in}{1.628682in}%
\pgfsys@useobject{currentmarker}{}%
\end{pgfscope}%
\begin{pgfscope}%
\pgfsys@transformshift{2.876226in}{1.631635in}%
\pgfsys@useobject{currentmarker}{}%
\end{pgfscope}%
\begin{pgfscope}%
\pgfsys@transformshift{2.884177in}{1.622290in}%
\pgfsys@useobject{currentmarker}{}%
\end{pgfscope}%
\begin{pgfscope}%
\pgfsys@transformshift{2.891946in}{1.611425in}%
\pgfsys@useobject{currentmarker}{}%
\end{pgfscope}%
\begin{pgfscope}%
\pgfsys@transformshift{2.899841in}{1.602435in}%
\pgfsys@useobject{currentmarker}{}%
\end{pgfscope}%
\begin{pgfscope}%
\pgfsys@transformshift{2.907557in}{1.595553in}%
\pgfsys@useobject{currentmarker}{}%
\end{pgfscope}%
\begin{pgfscope}%
\pgfsys@transformshift{2.915388in}{1.600072in}%
\pgfsys@useobject{currentmarker}{}%
\end{pgfscope}%
\begin{pgfscope}%
\pgfsys@transformshift{2.923322in}{1.595865in}%
\pgfsys@useobject{currentmarker}{}%
\end{pgfscope}%
\begin{pgfscope}%
\pgfsys@transformshift{2.931075in}{1.587107in}%
\pgfsys@useobject{currentmarker}{}%
\end{pgfscope}%
\begin{pgfscope}%
\pgfsys@transformshift{2.938923in}{1.583499in}%
\pgfsys@useobject{currentmarker}{}%
\end{pgfscope}%
\begin{pgfscope}%
\pgfsys@transformshift{2.946854in}{1.577353in}%
\pgfsys@useobject{currentmarker}{}%
\end{pgfscope}%
\begin{pgfscope}%
\pgfsys@transformshift{2.954604in}{1.571811in}%
\pgfsys@useobject{currentmarker}{}%
\end{pgfscope}%
\begin{pgfscope}%
\pgfsys@transformshift{2.962430in}{1.570342in}%
\pgfsys@useobject{currentmarker}{}%
\end{pgfscope}%
\begin{pgfscope}%
\pgfsys@transformshift{2.970324in}{1.562496in}%
\pgfsys@useobject{currentmarker}{}%
\end{pgfscope}%
\begin{pgfscope}%
\pgfsys@transformshift{2.978039in}{1.556038in}%
\pgfsys@useobject{currentmarker}{}%
\end{pgfscope}%
\begin{pgfscope}%
\pgfsys@transformshift{2.985815in}{1.546345in}%
\pgfsys@useobject{currentmarker}{}%
\end{pgfscope}%
\begin{pgfscope}%
\pgfsys@transformshift{2.993644in}{1.535846in}%
\pgfsys@useobject{currentmarker}{}%
\end{pgfscope}%
\begin{pgfscope}%
\pgfsys@transformshift{3.001519in}{1.533106in}%
\pgfsys@useobject{currentmarker}{}%
\end{pgfscope}%
\begin{pgfscope}%
\pgfsys@transformshift{3.009433in}{1.528406in}%
\pgfsys@useobject{currentmarker}{}%
\end{pgfscope}%
\begin{pgfscope}%
\pgfsys@transformshift{3.017166in}{1.520273in}%
\pgfsys@useobject{currentmarker}{}%
\end{pgfscope}%
\begin{pgfscope}%
\pgfsys@transformshift{3.024935in}{1.515058in}%
\pgfsys@useobject{currentmarker}{}%
\end{pgfscope}%
\begin{pgfscope}%
\pgfsys@transformshift{3.032732in}{1.503170in}%
\pgfsys@useobject{currentmarker}{}%
\end{pgfscope}%
\begin{pgfscope}%
\pgfsys@transformshift{3.040553in}{1.502677in}%
\pgfsys@useobject{currentmarker}{}%
\end{pgfscope}%
\begin{pgfscope}%
\pgfsys@transformshift{3.048391in}{1.495652in}%
\pgfsys@useobject{currentmarker}{}%
\end{pgfscope}%
\begin{pgfscope}%
\pgfsys@transformshift{3.056241in}{1.492841in}%
\pgfsys@useobject{currentmarker}{}%
\end{pgfscope}%
\begin{pgfscope}%
\pgfsys@transformshift{3.064099in}{1.483616in}%
\pgfsys@useobject{currentmarker}{}%
\end{pgfscope}%
\begin{pgfscope}%
\pgfsys@transformshift{3.071960in}{1.478927in}%
\pgfsys@useobject{currentmarker}{}%
\end{pgfscope}%
\begin{pgfscope}%
\pgfsys@transformshift{3.079819in}{1.472494in}%
\pgfsys@useobject{currentmarker}{}%
\end{pgfscope}%
\begin{pgfscope}%
\pgfsys@transformshift{3.087673in}{1.470666in}%
\pgfsys@useobject{currentmarker}{}%
\end{pgfscope}%
\begin{pgfscope}%
\pgfsys@transformshift{3.095517in}{1.459565in}%
\pgfsys@useobject{currentmarker}{}%
\end{pgfscope}%
\begin{pgfscope}%
\pgfsys@transformshift{3.103349in}{1.456283in}%
\pgfsys@useobject{currentmarker}{}%
\end{pgfscope}%
\begin{pgfscope}%
\pgfsys@transformshift{3.111166in}{1.454025in}%
\pgfsys@useobject{currentmarker}{}%
\end{pgfscope}%
\begin{pgfscope}%
\pgfsys@transformshift{3.118963in}{1.440052in}%
\pgfsys@useobject{currentmarker}{}%
\end{pgfscope}%
\begin{pgfscope}%
\pgfsys@transformshift{3.126739in}{1.441429in}%
\pgfsys@useobject{currentmarker}{}%
\end{pgfscope}%
\begin{pgfscope}%
\pgfsys@transformshift{3.134491in}{1.428415in}%
\pgfsys@useobject{currentmarker}{}%
\end{pgfscope}%
\begin{pgfscope}%
\pgfsys@transformshift{3.142364in}{1.423594in}%
\pgfsys@useobject{currentmarker}{}%
\end{pgfscope}%
\begin{pgfscope}%
\pgfsys@transformshift{3.150202in}{1.413662in}%
\pgfsys@useobject{currentmarker}{}%
\end{pgfscope}%
\begin{pgfscope}%
\pgfsys@transformshift{3.158002in}{1.408565in}%
\pgfsys@useobject{currentmarker}{}%
\end{pgfscope}%
\begin{pgfscope}%
\pgfsys@transformshift{3.165902in}{1.401889in}%
\pgfsys@useobject{currentmarker}{}%
\end{pgfscope}%
\begin{pgfscope}%
\pgfsys@transformshift{3.173756in}{1.393385in}%
\pgfsys@useobject{currentmarker}{}%
\end{pgfscope}%
\begin{pgfscope}%
\pgfsys@transformshift{3.181562in}{1.391538in}%
\pgfsys@useobject{currentmarker}{}%
\end{pgfscope}%
\begin{pgfscope}%
\pgfsys@transformshift{3.189449in}{1.388507in}%
\pgfsys@useobject{currentmarker}{}%
\end{pgfscope}%
\begin{pgfscope}%
\pgfsys@transformshift{3.197281in}{1.377593in}%
\pgfsys@useobject{currentmarker}{}%
\end{pgfscope}%
\begin{pgfscope}%
\pgfsys@transformshift{3.205059in}{1.373192in}%
\pgfsys@useobject{currentmarker}{}%
\end{pgfscope}%
\begin{pgfscope}%
\pgfsys@transformshift{3.212901in}{1.367013in}%
\pgfsys@useobject{currentmarker}{}%
\end{pgfscope}%
\begin{pgfscope}%
\pgfsys@transformshift{3.220799in}{1.358061in}%
\pgfsys@useobject{currentmarker}{}%
\end{pgfscope}%
\begin{pgfscope}%
\pgfsys@transformshift{3.228631in}{1.354471in}%
\pgfsys@useobject{currentmarker}{}%
\end{pgfscope}%
\begin{pgfscope}%
\pgfsys@transformshift{3.236397in}{1.350627in}%
\pgfsys@useobject{currentmarker}{}%
\end{pgfscope}%
\begin{pgfscope}%
\pgfsys@transformshift{3.244207in}{1.344372in}%
\pgfsys@useobject{currentmarker}{}%
\end{pgfscope}%
\begin{pgfscope}%
\pgfsys@transformshift{3.252054in}{1.334880in}%
\pgfsys@useobject{currentmarker}{}%
\end{pgfscope}%
\begin{pgfscope}%
\pgfsys@transformshift{3.259932in}{1.330755in}%
\pgfsys@useobject{currentmarker}{}%
\end{pgfscope}%
\begin{pgfscope}%
\pgfsys@transformshift{3.267732in}{1.327173in}%
\pgfsys@useobject{currentmarker}{}%
\end{pgfscope}%
\begin{pgfscope}%
\pgfsys@transformshift{3.275554in}{1.315546in}%
\pgfsys@useobject{currentmarker}{}%
\end{pgfscope}%
\begin{pgfscope}%
\pgfsys@transformshift{3.283395in}{1.309213in}%
\pgfsys@useobject{currentmarker}{}%
\end{pgfscope}%
\begin{pgfscope}%
\pgfsys@transformshift{3.291153in}{1.307280in}%
\pgfsys@useobject{currentmarker}{}%
\end{pgfscope}%
\begin{pgfscope}%
\pgfsys@transformshift{3.299015in}{1.299628in}%
\pgfsys@useobject{currentmarker}{}%
\end{pgfscope}%
\begin{pgfscope}%
\pgfsys@transformshift{3.306880in}{1.291580in}%
\pgfsys@useobject{currentmarker}{}%
\end{pgfscope}%
\begin{pgfscope}%
\pgfsys@transformshift{3.314655in}{1.287993in}%
\pgfsys@useobject{currentmarker}{}%
\end{pgfscope}%
\begin{pgfscope}%
\pgfsys@transformshift{3.322514in}{1.280271in}%
\pgfsys@useobject{currentmarker}{}%
\end{pgfscope}%
\begin{pgfscope}%
\pgfsys@transformshift{3.330365in}{1.272103in}%
\pgfsys@useobject{currentmarker}{}%
\end{pgfscope}%
\begin{pgfscope}%
\pgfsys@transformshift{3.338203in}{1.271662in}%
\pgfsys@useobject{currentmarker}{}%
\end{pgfscope}%
\begin{pgfscope}%
\pgfsys@transformshift{3.346025in}{1.258757in}%
\pgfsys@useobject{currentmarker}{}%
\end{pgfscope}%
\begin{pgfscope}%
\pgfsys@transformshift{3.353828in}{1.259672in}%
\pgfsys@useobject{currentmarker}{}%
\end{pgfscope}%
\begin{pgfscope}%
\pgfsys@transformshift{3.361686in}{1.248067in}%
\pgfsys@useobject{currentmarker}{}%
\end{pgfscope}%
\begin{pgfscope}%
\pgfsys@transformshift{3.369517in}{1.248281in}%
\pgfsys@useobject{currentmarker}{}%
\end{pgfscope}%
\begin{pgfscope}%
\pgfsys@transformshift{3.377318in}{1.238893in}%
\pgfsys@useobject{currentmarker}{}%
\end{pgfscope}%
\begin{pgfscope}%
\pgfsys@transformshift{3.385159in}{1.228444in}%
\pgfsys@useobject{currentmarker}{}%
\end{pgfscope}%
\begin{pgfscope}%
\pgfsys@transformshift{3.393033in}{1.224075in}%
\pgfsys@useobject{currentmarker}{}%
\end{pgfscope}%
\begin{pgfscope}%
\pgfsys@transformshift{3.400865in}{1.218029in}%
\pgfsys@useobject{currentmarker}{}%
\end{pgfscope}%
\begin{pgfscope}%
\pgfsys@transformshift{3.408655in}{1.212633in}%
\pgfsys@useobject{currentmarker}{}%
\end{pgfscope}%
\begin{pgfscope}%
\pgfsys@transformshift{3.416466in}{1.210181in}%
\pgfsys@useobject{currentmarker}{}%
\end{pgfscope}%
\begin{pgfscope}%
\pgfsys@transformshift{3.424294in}{1.204504in}%
\pgfsys@useobject{currentmarker}{}%
\end{pgfscope}%
\begin{pgfscope}%
\pgfsys@transformshift{3.432132in}{1.198012in}%
\pgfsys@useobject{currentmarker}{}%
\end{pgfscope}%
\begin{pgfscope}%
\pgfsys@transformshift{3.439976in}{1.192516in}%
\pgfsys@useobject{currentmarker}{}%
\end{pgfscope}%
\begin{pgfscope}%
\pgfsys@transformshift{3.447823in}{1.181366in}%
\pgfsys@useobject{currentmarker}{}%
\end{pgfscope}%
\begin{pgfscope}%
\pgfsys@transformshift{3.455666in}{1.176764in}%
\pgfsys@useobject{currentmarker}{}%
\end{pgfscope}%
\begin{pgfscope}%
\pgfsys@transformshift{3.463447in}{1.172225in}%
\pgfsys@useobject{currentmarker}{}%
\end{pgfscope}%
\begin{pgfscope}%
\pgfsys@transformshift{3.471275in}{1.167378in}%
\pgfsys@useobject{currentmarker}{}%
\end{pgfscope}%
\begin{pgfscope}%
\pgfsys@transformshift{3.479145in}{1.163834in}%
\pgfsys@useobject{currentmarker}{}%
\end{pgfscope}%
\begin{pgfscope}%
\pgfsys@transformshift{3.486942in}{1.158801in}%
\pgfsys@useobject{currentmarker}{}%
\end{pgfscope}%
\begin{pgfscope}%
\pgfsys@transformshift{3.494773in}{1.148010in}%
\pgfsys@useobject{currentmarker}{}%
\end{pgfscope}%
\begin{pgfscope}%
\pgfsys@transformshift{3.502629in}{1.142291in}%
\pgfsys@useobject{currentmarker}{}%
\end{pgfscope}%
\begin{pgfscope}%
\pgfsys@transformshift{3.510457in}{1.138598in}%
\pgfsys@useobject{currentmarker}{}%
\end{pgfscope}%
\begin{pgfscope}%
\pgfsys@transformshift{3.518253in}{1.128641in}%
\pgfsys@useobject{currentmarker}{}%
\end{pgfscope}%
\begin{pgfscope}%
\pgfsys@transformshift{3.526064in}{1.122716in}%
\pgfsys@useobject{currentmarker}{}%
\end{pgfscope}%
\begin{pgfscope}%
\pgfsys@transformshift{3.533930in}{1.119541in}%
\pgfsys@useobject{currentmarker}{}%
\end{pgfscope}%
\begin{pgfscope}%
\pgfsys@transformshift{3.541754in}{1.115778in}%
\pgfsys@useobject{currentmarker}{}%
\end{pgfscope}%
\begin{pgfscope}%
\pgfsys@transformshift{3.549578in}{1.109153in}%
\pgfsys@useobject{currentmarker}{}%
\end{pgfscope}%
\begin{pgfscope}%
\pgfsys@transformshift{3.557399in}{1.103011in}%
\pgfsys@useobject{currentmarker}{}%
\end{pgfscope}%
\begin{pgfscope}%
\pgfsys@transformshift{3.565212in}{1.097825in}%
\pgfsys@useobject{currentmarker}{}%
\end{pgfscope}%
\begin{pgfscope}%
\pgfsys@transformshift{3.573056in}{1.091347in}%
\pgfsys@useobject{currentmarker}{}%
\end{pgfscope}%
\begin{pgfscope}%
\pgfsys@transformshift{3.580883in}{1.085471in}%
\pgfsys@useobject{currentmarker}{}%
\end{pgfscope}%
\begin{pgfscope}%
\pgfsys@transformshift{3.588731in}{1.078181in}%
\pgfsys@useobject{currentmarker}{}%
\end{pgfscope}%
\begin{pgfscope}%
\pgfsys@transformshift{3.596556in}{1.073649in}%
\pgfsys@useobject{currentmarker}{}%
\end{pgfscope}%
\begin{pgfscope}%
\pgfsys@transformshift{3.604354in}{1.066865in}%
\pgfsys@useobject{currentmarker}{}%
\end{pgfscope}%
\begin{pgfscope}%
\pgfsys@transformshift{3.612198in}{1.067836in}%
\pgfsys@useobject{currentmarker}{}%
\end{pgfscope}%
\begin{pgfscope}%
\pgfsys@transformshift{3.620044in}{1.057106in}%
\pgfsys@useobject{currentmarker}{}%
\end{pgfscope}%
\begin{pgfscope}%
\pgfsys@transformshift{3.627888in}{1.056021in}%
\pgfsys@useobject{currentmarker}{}%
\end{pgfscope}%
\begin{pgfscope}%
\pgfsys@transformshift{3.635726in}{1.052030in}%
\pgfsys@useobject{currentmarker}{}%
\end{pgfscope}%
\begin{pgfscope}%
\pgfsys@transformshift{3.643555in}{1.045165in}%
\pgfsys@useobject{currentmarker}{}%
\end{pgfscope}%
\begin{pgfscope}%
\pgfsys@transformshift{3.651371in}{1.038030in}%
\pgfsys@useobject{currentmarker}{}%
\end{pgfscope}%
\begin{pgfscope}%
\pgfsys@transformshift{3.659170in}{1.036278in}%
\pgfsys@useobject{currentmarker}{}%
\end{pgfscope}%
\begin{pgfscope}%
\pgfsys@transformshift{3.667014in}{1.032882in}%
\pgfsys@useobject{currentmarker}{}%
\end{pgfscope}%
\begin{pgfscope}%
\pgfsys@transformshift{3.674863in}{1.027090in}%
\pgfsys@useobject{currentmarker}{}%
\end{pgfscope}%
\begin{pgfscope}%
\pgfsys@transformshift{3.682684in}{1.021139in}%
\pgfsys@useobject{currentmarker}{}%
\end{pgfscope}%
\begin{pgfscope}%
\pgfsys@transformshift{3.690504in}{1.015861in}%
\pgfsys@useobject{currentmarker}{}%
\end{pgfscope}%
\begin{pgfscope}%
\pgfsys@transformshift{3.698319in}{1.009210in}%
\pgfsys@useobject{currentmarker}{}%
\end{pgfscope}%
\begin{pgfscope}%
\pgfsys@transformshift{3.706153in}{1.006736in}%
\pgfsys@useobject{currentmarker}{}%
\end{pgfscope}%
\begin{pgfscope}%
\pgfsys@transformshift{3.714000in}{1.003398in}%
\pgfsys@useobject{currentmarker}{}%
\end{pgfscope}%
\begin{pgfscope}%
\pgfsys@transformshift{3.721830in}{0.996407in}%
\pgfsys@useobject{currentmarker}{}%
\end{pgfscope}%
\begin{pgfscope}%
\pgfsys@transformshift{3.729665in}{0.992891in}%
\pgfsys@useobject{currentmarker}{}%
\end{pgfscope}%
\begin{pgfscope}%
\pgfsys@transformshift{3.737501in}{0.991977in}%
\pgfsys@useobject{currentmarker}{}%
\end{pgfscope}%
\begin{pgfscope}%
\pgfsys@transformshift{3.745309in}{0.986370in}%
\pgfsys@useobject{currentmarker}{}%
\end{pgfscope}%
\begin{pgfscope}%
\pgfsys@transformshift{3.753135in}{0.984132in}%
\pgfsys@useobject{currentmarker}{}%
\end{pgfscope}%
\begin{pgfscope}%
\pgfsys@transformshift{3.760975in}{0.975872in}%
\pgfsys@useobject{currentmarker}{}%
\end{pgfscope}%
\begin{pgfscope}%
\pgfsys@transformshift{3.768799in}{0.977165in}%
\pgfsys@useobject{currentmarker}{}%
\end{pgfscope}%
\begin{pgfscope}%
\pgfsys@transformshift{3.776628in}{0.973750in}%
\pgfsys@useobject{currentmarker}{}%
\end{pgfscope}%
\begin{pgfscope}%
\pgfsys@transformshift{3.784458in}{0.970796in}%
\pgfsys@useobject{currentmarker}{}%
\end{pgfscope}%
\begin{pgfscope}%
\pgfsys@transformshift{3.792284in}{0.966927in}%
\pgfsys@useobject{currentmarker}{}%
\end{pgfscope}%
\begin{pgfscope}%
\pgfsys@transformshift{3.800123in}{0.966400in}%
\pgfsys@useobject{currentmarker}{}%
\end{pgfscope}%
\begin{pgfscope}%
\pgfsys@transformshift{3.807950in}{0.960010in}%
\pgfsys@useobject{currentmarker}{}%
\end{pgfscope}%
\begin{pgfscope}%
\pgfsys@transformshift{3.815762in}{0.960040in}%
\pgfsys@useobject{currentmarker}{}%
\end{pgfscope}%
\begin{pgfscope}%
\pgfsys@transformshift{3.823596in}{0.957662in}%
\pgfsys@useobject{currentmarker}{}%
\end{pgfscope}%
\begin{pgfscope}%
\pgfsys@transformshift{3.831425in}{0.955426in}%
\pgfsys@useobject{currentmarker}{}%
\end{pgfscope}%
\begin{pgfscope}%
\pgfsys@transformshift{3.839267in}{0.953319in}%
\pgfsys@useobject{currentmarker}{}%
\end{pgfscope}%
\begin{pgfscope}%
\pgfsys@transformshift{3.847096in}{0.952455in}%
\pgfsys@useobject{currentmarker}{}%
\end{pgfscope}%
\begin{pgfscope}%
\pgfsys@transformshift{3.854911in}{0.952616in}%
\pgfsys@useobject{currentmarker}{}%
\end{pgfscope}%
\begin{pgfscope}%
\pgfsys@transformshift{3.862743in}{0.953465in}%
\pgfsys@useobject{currentmarker}{}%
\end{pgfscope}%
\begin{pgfscope}%
\pgfsys@transformshift{3.870569in}{0.949998in}%
\pgfsys@useobject{currentmarker}{}%
\end{pgfscope}%
\end{pgfscope}%
\begin{pgfscope}%
\pgfpathrectangle{\pgfqpoint{0.594525in}{0.417642in}}{\pgfqpoint{3.432047in}{2.016277in}}%
\pgfusepath{clip}%
\pgfsetbuttcap%
\pgfsetroundjoin%
\pgfsetlinewidth{1.505625pt}%
\definecolor{currentstroke}{rgb}{0.007843,0.619608,0.450980}%
\pgfsetstrokecolor{currentstroke}%
\pgfsetdash{{5.550000pt}{2.400000pt}}{0.000000pt}%
\pgfpathmoveto{\pgfqpoint{0.750527in}{2.337223in}}%
\pgfpathlineto{\pgfqpoint{1.123152in}{2.336134in}}%
\pgfpathlineto{\pgfqpoint{1.296413in}{2.333980in}}%
\pgfpathlineto{\pgfqpoint{1.410538in}{2.330810in}}%
\pgfpathlineto{\pgfqpoint{1.495778in}{2.326694in}}%
\pgfpathlineto{\pgfqpoint{1.563841in}{2.321715in}}%
\pgfpathlineto{\pgfqpoint{1.620502in}{2.315967in}}%
\pgfpathlineto{\pgfqpoint{1.690929in}{2.306119in}}%
\pgfpathlineto{\pgfqpoint{1.749217in}{2.295095in}}%
\pgfpathlineto{\pgfqpoint{1.798942in}{2.283207in}}%
\pgfpathlineto{\pgfqpoint{1.842300in}{2.270725in}}%
\pgfpathlineto{\pgfqpoint{1.892641in}{2.253549in}}%
\pgfpathlineto{\pgfqpoint{1.936467in}{2.236139in}}%
\pgfpathlineto{\pgfqpoint{1.984318in}{2.214500in}}%
\pgfpathlineto{\pgfqpoint{2.034042in}{2.189188in}}%
\pgfpathlineto{\pgfqpoint{2.084117in}{2.160989in}}%
\pgfpathlineto{\pgfqpoint{2.139240in}{2.127143in}}%
\pgfpathlineto{\pgfqpoint{2.193939in}{2.091074in}}%
\pgfpathlineto{\pgfqpoint{2.259368in}{2.045298in}}%
\pgfpathlineto{\pgfqpoint{2.335450in}{1.989322in}}%
\pgfpathlineto{\pgfqpoint{2.430240in}{1.916719in}}%
\pgfpathlineto{\pgfqpoint{2.555149in}{1.818099in}}%
\pgfpathlineto{\pgfqpoint{2.743177in}{1.666552in}}%
\pgfpathlineto{\pgfqpoint{3.118963in}{1.360230in}}%
\pgfpathlineto{\pgfqpoint{3.870569in}{0.745712in}}%
\pgfpathlineto{\pgfqpoint{3.870569in}{0.745712in}}%
\pgfusepath{stroke}%
\end{pgfscope}%
\begin{pgfscope}%
\pgfpathrectangle{\pgfqpoint{0.594525in}{0.417642in}}{\pgfqpoint{3.432047in}{2.016277in}}%
\pgfusepath{clip}%
\pgfsetbuttcap%
\pgfsetroundjoin%
\definecolor{currentfill}{rgb}{0.007843,0.619608,0.450980}%
\pgfsetfillcolor{currentfill}%
\pgfsetlinewidth{1.003750pt}%
\definecolor{currentstroke}{rgb}{0.007843,0.619608,0.450980}%
\pgfsetstrokecolor{currentstroke}%
\pgfsetdash{}{0pt}%
\pgfsys@defobject{currentmarker}{\pgfqpoint{-0.006944in}{-0.006944in}}{\pgfqpoint{0.006944in}{0.006944in}}{%
\pgfpathmoveto{\pgfqpoint{0.000000in}{-0.006944in}}%
\pgfpathcurveto{\pgfqpoint{0.001842in}{-0.006944in}}{\pgfqpoint{0.003608in}{-0.006213in}}{\pgfqpoint{0.004910in}{-0.004910in}}%
\pgfpathcurveto{\pgfqpoint{0.006213in}{-0.003608in}}{\pgfqpoint{0.006944in}{-0.001842in}}{\pgfqpoint{0.006944in}{0.000000in}}%
\pgfpathcurveto{\pgfqpoint{0.006944in}{0.001842in}}{\pgfqpoint{0.006213in}{0.003608in}}{\pgfqpoint{0.004910in}{0.004910in}}%
\pgfpathcurveto{\pgfqpoint{0.003608in}{0.006213in}}{\pgfqpoint{0.001842in}{0.006944in}}{\pgfqpoint{0.000000in}{0.006944in}}%
\pgfpathcurveto{\pgfqpoint{-0.001842in}{0.006944in}}{\pgfqpoint{-0.003608in}{0.006213in}}{\pgfqpoint{-0.004910in}{0.004910in}}%
\pgfpathcurveto{\pgfqpoint{-0.006213in}{0.003608in}}{\pgfqpoint{-0.006944in}{0.001842in}}{\pgfqpoint{-0.006944in}{0.000000in}}%
\pgfpathcurveto{\pgfqpoint{-0.006944in}{-0.001842in}}{\pgfqpoint{-0.006213in}{-0.003608in}}{\pgfqpoint{-0.004910in}{-0.004910in}}%
\pgfpathcurveto{\pgfqpoint{-0.003608in}{-0.006213in}}{\pgfqpoint{-0.001842in}{-0.006944in}}{\pgfqpoint{0.000000in}{-0.006944in}}%
\pgfpathlineto{\pgfqpoint{0.000000in}{-0.006944in}}%
\pgfpathclose%
\pgfusepath{stroke,fill}%
}%
\begin{pgfscope}%
\pgfsys@transformshift{0.750527in}{2.312872in}%
\pgfsys@useobject{currentmarker}{}%
\end{pgfscope}%
\begin{pgfscope}%
\pgfsys@transformshift{0.985627in}{2.341687in}%
\pgfsys@useobject{currentmarker}{}%
\end{pgfscope}%
\begin{pgfscope}%
\pgfsys@transformshift{1.123152in}{2.327223in}%
\pgfsys@useobject{currentmarker}{}%
\end{pgfscope}%
\begin{pgfscope}%
\pgfsys@transformshift{1.220728in}{2.322175in}%
\pgfsys@useobject{currentmarker}{}%
\end{pgfscope}%
\begin{pgfscope}%
\pgfsys@transformshift{1.296413in}{2.331902in}%
\pgfsys@useobject{currentmarker}{}%
\end{pgfscope}%
\begin{pgfscope}%
\pgfsys@transformshift{1.358253in}{2.334081in}%
\pgfsys@useobject{currentmarker}{}%
\end{pgfscope}%
\begin{pgfscope}%
\pgfsys@transformshift{1.410538in}{2.342270in}%
\pgfsys@useobject{currentmarker}{}%
\end{pgfscope}%
\begin{pgfscope}%
\pgfsys@transformshift{1.455829in}{2.329571in}%
\pgfsys@useobject{currentmarker}{}%
\end{pgfscope}%
\begin{pgfscope}%
\pgfsys@transformshift{1.495778in}{2.333138in}%
\pgfsys@useobject{currentmarker}{}%
\end{pgfscope}%
\begin{pgfscope}%
\pgfsys@transformshift{1.531514in}{2.324519in}%
\pgfsys@useobject{currentmarker}{}%
\end{pgfscope}%
\begin{pgfscope}%
\pgfsys@transformshift{1.563841in}{2.321269in}%
\pgfsys@useobject{currentmarker}{}%
\end{pgfscope}%
\begin{pgfscope}%
\pgfsys@transformshift{1.593354in}{2.319662in}%
\pgfsys@useobject{currentmarker}{}%
\end{pgfscope}%
\begin{pgfscope}%
\pgfsys@transformshift{1.620502in}{2.318968in}%
\pgfsys@useobject{currentmarker}{}%
\end{pgfscope}%
\begin{pgfscope}%
\pgfsys@transformshift{1.645638in}{2.303840in}%
\pgfsys@useobject{currentmarker}{}%
\end{pgfscope}%
\begin{pgfscope}%
\pgfsys@transformshift{1.669039in}{2.293263in}%
\pgfsys@useobject{currentmarker}{}%
\end{pgfscope}%
\begin{pgfscope}%
\pgfsys@transformshift{1.690929in}{2.299007in}%
\pgfsys@useobject{currentmarker}{}%
\end{pgfscope}%
\begin{pgfscope}%
\pgfsys@transformshift{1.711492in}{2.302164in}%
\pgfsys@useobject{currentmarker}{}%
\end{pgfscope}%
\begin{pgfscope}%
\pgfsys@transformshift{1.730879in}{2.294600in}%
\pgfsys@useobject{currentmarker}{}%
\end{pgfscope}%
\begin{pgfscope}%
\pgfsys@transformshift{1.749217in}{2.299021in}%
\pgfsys@useobject{currentmarker}{}%
\end{pgfscope}%
\begin{pgfscope}%
\pgfsys@transformshift{1.766615in}{2.298245in}%
\pgfsys@useobject{currentmarker}{}%
\end{pgfscope}%
\begin{pgfscope}%
\pgfsys@transformshift{1.783163in}{2.270928in}%
\pgfsys@useobject{currentmarker}{}%
\end{pgfscope}%
\begin{pgfscope}%
\pgfsys@transformshift{1.798942in}{2.276321in}%
\pgfsys@useobject{currentmarker}{}%
\end{pgfscope}%
\begin{pgfscope}%
\pgfsys@transformshift{1.814019in}{2.276563in}%
\pgfsys@useobject{currentmarker}{}%
\end{pgfscope}%
\begin{pgfscope}%
\pgfsys@transformshift{1.828454in}{2.283492in}%
\pgfsys@useobject{currentmarker}{}%
\end{pgfscope}%
\begin{pgfscope}%
\pgfsys@transformshift{1.842300in}{2.290871in}%
\pgfsys@useobject{currentmarker}{}%
\end{pgfscope}%
\begin{pgfscope}%
\pgfsys@transformshift{1.855603in}{2.256449in}%
\pgfsys@useobject{currentmarker}{}%
\end{pgfscope}%
\begin{pgfscope}%
\pgfsys@transformshift{1.868404in}{2.252423in}%
\pgfsys@useobject{currentmarker}{}%
\end{pgfscope}%
\begin{pgfscope}%
\pgfsys@transformshift{1.880739in}{2.270300in}%
\pgfsys@useobject{currentmarker}{}%
\end{pgfscope}%
\begin{pgfscope}%
\pgfsys@transformshift{1.892641in}{2.262432in}%
\pgfsys@useobject{currentmarker}{}%
\end{pgfscope}%
\begin{pgfscope}%
\pgfsys@transformshift{1.904140in}{2.248129in}%
\pgfsys@useobject{currentmarker}{}%
\end{pgfscope}%
\begin{pgfscope}%
\pgfsys@transformshift{1.915261in}{2.253258in}%
\pgfsys@useobject{currentmarker}{}%
\end{pgfscope}%
\begin{pgfscope}%
\pgfsys@transformshift{1.926030in}{2.243927in}%
\pgfsys@useobject{currentmarker}{}%
\end{pgfscope}%
\begin{pgfscope}%
\pgfsys@transformshift{1.936467in}{2.225273in}%
\pgfsys@useobject{currentmarker}{}%
\end{pgfscope}%
\begin{pgfscope}%
\pgfsys@transformshift{1.946592in}{2.226196in}%
\pgfsys@useobject{currentmarker}{}%
\end{pgfscope}%
\begin{pgfscope}%
\pgfsys@transformshift{1.956424in}{2.246161in}%
\pgfsys@useobject{currentmarker}{}%
\end{pgfscope}%
\begin{pgfscope}%
\pgfsys@transformshift{1.965979in}{2.246572in}%
\pgfsys@useobject{currentmarker}{}%
\end{pgfscope}%
\begin{pgfscope}%
\pgfsys@transformshift{1.975272in}{2.221757in}%
\pgfsys@useobject{currentmarker}{}%
\end{pgfscope}%
\begin{pgfscope}%
\pgfsys@transformshift{1.984318in}{2.219103in}%
\pgfsys@useobject{currentmarker}{}%
\end{pgfscope}%
\begin{pgfscope}%
\pgfsys@transformshift{1.993128in}{2.212058in}%
\pgfsys@useobject{currentmarker}{}%
\end{pgfscope}%
\begin{pgfscope}%
\pgfsys@transformshift{2.001715in}{2.210222in}%
\pgfsys@useobject{currentmarker}{}%
\end{pgfscope}%
\begin{pgfscope}%
\pgfsys@transformshift{2.010090in}{2.210599in}%
\pgfsys@useobject{currentmarker}{}%
\end{pgfscope}%
\begin{pgfscope}%
\pgfsys@transformshift{2.018264in}{2.185799in}%
\pgfsys@useobject{currentmarker}{}%
\end{pgfscope}%
\begin{pgfscope}%
\pgfsys@transformshift{2.026245in}{2.188264in}%
\pgfsys@useobject{currentmarker}{}%
\end{pgfscope}%
\begin{pgfscope}%
\pgfsys@transformshift{2.034042in}{2.192759in}%
\pgfsys@useobject{currentmarker}{}%
\end{pgfscope}%
\begin{pgfscope}%
\pgfsys@transformshift{2.041665in}{2.183839in}%
\pgfsys@useobject{currentmarker}{}%
\end{pgfscope}%
\begin{pgfscope}%
\pgfsys@transformshift{2.049119in}{2.173392in}%
\pgfsys@useobject{currentmarker}{}%
\end{pgfscope}%
\begin{pgfscope}%
\pgfsys@transformshift{2.056414in}{2.163033in}%
\pgfsys@useobject{currentmarker}{}%
\end{pgfscope}%
\begin{pgfscope}%
\pgfsys@transformshift{2.063555in}{2.180108in}%
\pgfsys@useobject{currentmarker}{}%
\end{pgfscope}%
\begin{pgfscope}%
\pgfsys@transformshift{2.070548in}{2.176949in}%
\pgfsys@useobject{currentmarker}{}%
\end{pgfscope}%
\begin{pgfscope}%
\pgfsys@transformshift{2.077401in}{2.173534in}%
\pgfsys@useobject{currentmarker}{}%
\end{pgfscope}%
\begin{pgfscope}%
\pgfsys@transformshift{2.084117in}{2.158367in}%
\pgfsys@useobject{currentmarker}{}%
\end{pgfscope}%
\begin{pgfscope}%
\pgfsys@transformshift{2.093949in}{2.155150in}%
\pgfsys@useobject{currentmarker}{}%
\end{pgfscope}%
\begin{pgfscope}%
\pgfsys@transformshift{2.103504in}{2.153829in}%
\pgfsys@useobject{currentmarker}{}%
\end{pgfscope}%
\begin{pgfscope}%
\pgfsys@transformshift{2.109728in}{2.141879in}%
\pgfsys@useobject{currentmarker}{}%
\end{pgfscope}%
\begin{pgfscope}%
\pgfsys@transformshift{2.115839in}{2.132821in}%
\pgfsys@useobject{currentmarker}{}%
\end{pgfscope}%
\begin{pgfscope}%
\pgfsys@transformshift{2.124805in}{2.142289in}%
\pgfsys@useobject{currentmarker}{}%
\end{pgfscope}%
\begin{pgfscope}%
\pgfsys@transformshift{2.133540in}{2.113769in}%
\pgfsys@useobject{currentmarker}{}%
\end{pgfscope}%
\begin{pgfscope}%
\pgfsys@transformshift{2.139240in}{2.124439in}%
\pgfsys@useobject{currentmarker}{}%
\end{pgfscope}%
\begin{pgfscope}%
\pgfsys@transformshift{2.147615in}{2.116199in}%
\pgfsys@useobject{currentmarker}{}%
\end{pgfscope}%
\begin{pgfscope}%
\pgfsys@transformshift{2.155789in}{2.116386in}%
\pgfsys@useobject{currentmarker}{}%
\end{pgfscope}%
\begin{pgfscope}%
\pgfsys@transformshift{2.163770in}{2.091014in}%
\pgfsys@useobject{currentmarker}{}%
\end{pgfscope}%
\begin{pgfscope}%
\pgfsys@transformshift{2.171567in}{2.095018in}%
\pgfsys@useobject{currentmarker}{}%
\end{pgfscope}%
\begin{pgfscope}%
\pgfsys@transformshift{2.179190in}{2.115256in}%
\pgfsys@useobject{currentmarker}{}%
\end{pgfscope}%
\begin{pgfscope}%
\pgfsys@transformshift{2.186644in}{2.107445in}%
\pgfsys@useobject{currentmarker}{}%
\end{pgfscope}%
\begin{pgfscope}%
\pgfsys@transformshift{2.193939in}{2.078225in}%
\pgfsys@useobject{currentmarker}{}%
\end{pgfscope}%
\begin{pgfscope}%
\pgfsys@transformshift{2.203427in}{2.082310in}%
\pgfsys@useobject{currentmarker}{}%
\end{pgfscope}%
\begin{pgfscope}%
\pgfsys@transformshift{2.210373in}{2.084450in}%
\pgfsys@useobject{currentmarker}{}%
\end{pgfscope}%
\begin{pgfscope}%
\pgfsys@transformshift{2.217179in}{2.067392in}%
\pgfsys@useobject{currentmarker}{}%
\end{pgfscope}%
\begin{pgfscope}%
\pgfsys@transformshift{2.226047in}{2.073761in}%
\pgfsys@useobject{currentmarker}{}%
\end{pgfscope}%
\begin{pgfscope}%
\pgfsys@transformshift{2.234689in}{2.084131in}%
\pgfsys@useobject{currentmarker}{}%
\end{pgfscope}%
\begin{pgfscope}%
\pgfsys@transformshift{2.243116in}{2.067875in}%
\pgfsys@useobject{currentmarker}{}%
\end{pgfscope}%
\begin{pgfscope}%
\pgfsys@transformshift{2.251339in}{2.064152in}%
\pgfsys@useobject{currentmarker}{}%
\end{pgfscope}%
\begin{pgfscope}%
\pgfsys@transformshift{2.259368in}{2.048809in}%
\pgfsys@useobject{currentmarker}{}%
\end{pgfscope}%
\begin{pgfscope}%
\pgfsys@transformshift{2.267210in}{2.035917in}%
\pgfsys@useobject{currentmarker}{}%
\end{pgfscope}%
\begin{pgfscope}%
\pgfsys@transformshift{2.274876in}{2.037395in}%
\pgfsys@useobject{currentmarker}{}%
\end{pgfscope}%
\begin{pgfscope}%
\pgfsys@transformshift{2.282372in}{2.033088in}%
\pgfsys@useobject{currentmarker}{}%
\end{pgfscope}%
\begin{pgfscope}%
\pgfsys@transformshift{2.289705in}{2.014323in}%
\pgfsys@useobject{currentmarker}{}%
\end{pgfscope}%
\begin{pgfscope}%
\pgfsys@transformshift{2.296884in}{2.005601in}%
\pgfsys@useobject{currentmarker}{}%
\end{pgfscope}%
\begin{pgfscope}%
\pgfsys@transformshift{2.303914in}{2.010100in}%
\pgfsys@useobject{currentmarker}{}%
\end{pgfscope}%
\begin{pgfscope}%
\pgfsys@transformshift{2.312501in}{1.998748in}%
\pgfsys@useobject{currentmarker}{}%
\end{pgfscope}%
\begin{pgfscope}%
\pgfsys@transformshift{2.320876in}{2.003698in}%
\pgfsys@useobject{currentmarker}{}%
\end{pgfscope}%
\begin{pgfscope}%
\pgfsys@transformshift{2.327431in}{1.996319in}%
\pgfsys@useobject{currentmarker}{}%
\end{pgfscope}%
\begin{pgfscope}%
\pgfsys@transformshift{2.335450in}{1.993753in}%
\pgfsys@useobject{currentmarker}{}%
\end{pgfscope}%
\begin{pgfscope}%
\pgfsys@transformshift{2.343283in}{1.978344in}%
\pgfsys@useobject{currentmarker}{}%
\end{pgfscope}%
\begin{pgfscope}%
\pgfsys@transformshift{2.350940in}{1.976436in}%
\pgfsys@useobject{currentmarker}{}%
\end{pgfscope}%
\begin{pgfscope}%
\pgfsys@transformshift{2.359905in}{1.975299in}%
\pgfsys@useobject{currentmarker}{}%
\end{pgfscope}%
\begin{pgfscope}%
\pgfsys@transformshift{2.367200in}{1.969444in}%
\pgfsys@useobject{currentmarker}{}%
\end{pgfscope}%
\begin{pgfscope}%
\pgfsys@transformshift{2.374341in}{1.966938in}%
\pgfsys@useobject{currentmarker}{}%
\end{pgfscope}%
\begin{pgfscope}%
\pgfsys@transformshift{2.382716in}{1.955844in}%
\pgfsys@useobject{currentmarker}{}%
\end{pgfscope}%
\begin{pgfscope}%
\pgfsys@transformshift{2.390889in}{1.934756in}%
\pgfsys@useobject{currentmarker}{}%
\end{pgfscope}%
\begin{pgfscope}%
\pgfsys@transformshift{2.398870in}{1.935840in}%
\pgfsys@useobject{currentmarker}{}%
\end{pgfscope}%
\begin{pgfscope}%
\pgfsys@transformshift{2.406668in}{1.929706in}%
\pgfsys@useobject{currentmarker}{}%
\end{pgfscope}%
\begin{pgfscope}%
\pgfsys@transformshift{2.414290in}{1.922728in}%
\pgfsys@useobject{currentmarker}{}%
\end{pgfscope}%
\begin{pgfscope}%
\pgfsys@transformshift{2.421745in}{1.934868in}%
\pgfsys@useobject{currentmarker}{}%
\end{pgfscope}%
\begin{pgfscope}%
\pgfsys@transformshift{2.430240in}{1.921215in}%
\pgfsys@useobject{currentmarker}{}%
\end{pgfscope}%
\begin{pgfscope}%
\pgfsys@transformshift{2.438527in}{1.904432in}%
\pgfsys@useobject{currentmarker}{}%
\end{pgfscope}%
\begin{pgfscope}%
\pgfsys@transformshift{2.445473in}{1.906030in}%
\pgfsys@useobject{currentmarker}{}%
\end{pgfscope}%
\begin{pgfscope}%
\pgfsys@transformshift{2.453401in}{1.901020in}%
\pgfsys@useobject{currentmarker}{}%
\end{pgfscope}%
\begin{pgfscope}%
\pgfsys@transformshift{2.461148in}{1.892082in}%
\pgfsys@useobject{currentmarker}{}%
\end{pgfscope}%
\begin{pgfscope}%
\pgfsys@transformshift{2.468721in}{1.886508in}%
\pgfsys@useobject{currentmarker}{}%
\end{pgfscope}%
\begin{pgfscope}%
\pgfsys@transformshift{2.477175in}{1.879606in}%
\pgfsys@useobject{currentmarker}{}%
\end{pgfscope}%
\begin{pgfscope}%
\pgfsys@transformshift{2.485423in}{1.871759in}%
\pgfsys@useobject{currentmarker}{}%
\end{pgfscope}%
\begin{pgfscope}%
\pgfsys@transformshift{2.493475in}{1.877308in}%
\pgfsys@useobject{currentmarker}{}%
\end{pgfscope}%
\begin{pgfscope}%
\pgfsys@transformshift{2.501340in}{1.854179in}%
\pgfsys@useobject{currentmarker}{}%
\end{pgfscope}%
\begin{pgfscope}%
\pgfsys@transformshift{2.509027in}{1.858872in}%
\pgfsys@useobject{currentmarker}{}%
\end{pgfscope}%
\begin{pgfscope}%
\pgfsys@transformshift{2.516544in}{1.860354in}%
\pgfsys@useobject{currentmarker}{}%
\end{pgfscope}%
\begin{pgfscope}%
\pgfsys@transformshift{2.523898in}{1.847614in}%
\pgfsys@useobject{currentmarker}{}%
\end{pgfscope}%
\begin{pgfscope}%
\pgfsys@transformshift{2.531985in}{1.832204in}%
\pgfsys@useobject{currentmarker}{}%
\end{pgfscope}%
\begin{pgfscope}%
\pgfsys@transformshift{2.539883in}{1.830552in}%
\pgfsys@useobject{currentmarker}{}%
\end{pgfscope}%
\begin{pgfscope}%
\pgfsys@transformshift{2.547602in}{1.824957in}%
\pgfsys@useobject{currentmarker}{}%
\end{pgfscope}%
\begin{pgfscope}%
\pgfsys@transformshift{2.555149in}{1.813347in}%
\pgfsys@useobject{currentmarker}{}%
\end{pgfscope}%
\begin{pgfscope}%
\pgfsys@transformshift{2.562531in}{1.809799in}%
\pgfsys@useobject{currentmarker}{}%
\end{pgfscope}%
\begin{pgfscope}%
\pgfsys@transformshift{2.570550in}{1.801004in}%
\pgfsys@useobject{currentmarker}{}%
\end{pgfscope}%
\begin{pgfscope}%
\pgfsys@transformshift{2.578384in}{1.803959in}%
\pgfsys@useobject{currentmarker}{}%
\end{pgfscope}%
\begin{pgfscope}%
\pgfsys@transformshift{2.586040in}{1.795614in}%
\pgfsys@useobject{currentmarker}{}%
\end{pgfscope}%
\begin{pgfscope}%
\pgfsys@transformshift{2.594268in}{1.790854in}%
\pgfsys@useobject{currentmarker}{}%
\end{pgfscope}%
\begin{pgfscope}%
\pgfsys@transformshift{2.602300in}{1.784354in}%
\pgfsys@useobject{currentmarker}{}%
\end{pgfscope}%
\begin{pgfscope}%
\pgfsys@transformshift{2.610147in}{1.770069in}%
\pgfsys@useobject{currentmarker}{}%
\end{pgfscope}%
\begin{pgfscope}%
\pgfsys@transformshift{2.617816in}{1.763158in}%
\pgfsys@useobject{currentmarker}{}%
\end{pgfscope}%
\begin{pgfscope}%
\pgfsys@transformshift{2.625316in}{1.762591in}%
\pgfsys@useobject{currentmarker}{}%
\end{pgfscope}%
\begin{pgfscope}%
\pgfsys@transformshift{2.633313in}{1.751288in}%
\pgfsys@useobject{currentmarker}{}%
\end{pgfscope}%
\begin{pgfscope}%
\pgfsys@transformshift{2.641125in}{1.757589in}%
\pgfsys@useobject{currentmarker}{}%
\end{pgfscope}%
\begin{pgfscope}%
\pgfsys@transformshift{2.648762in}{1.742875in}%
\pgfsys@useobject{currentmarker}{}%
\end{pgfscope}%
\begin{pgfscope}%
\pgfsys@transformshift{2.656845in}{1.734751in}%
\pgfsys@useobject{currentmarker}{}%
\end{pgfscope}%
\begin{pgfscope}%
\pgfsys@transformshift{2.664741in}{1.735067in}%
\pgfsys@useobject{currentmarker}{}%
\end{pgfscope}%
\begin{pgfscope}%
\pgfsys@transformshift{2.672456in}{1.731989in}%
\pgfsys@useobject{currentmarker}{}%
\end{pgfscope}%
\begin{pgfscope}%
\pgfsys@transformshift{2.680574in}{1.716662in}%
\pgfsys@useobject{currentmarker}{}%
\end{pgfscope}%
\begin{pgfscope}%
\pgfsys@transformshift{2.688502in}{1.708710in}%
\pgfsys@useobject{currentmarker}{}%
\end{pgfscope}%
\begin{pgfscope}%
\pgfsys@transformshift{2.696248in}{1.700337in}%
\pgfsys@useobject{currentmarker}{}%
\end{pgfscope}%
\begin{pgfscope}%
\pgfsys@transformshift{2.703822in}{1.697143in}%
\pgfsys@useobject{currentmarker}{}%
\end{pgfscope}%
\begin{pgfscope}%
\pgfsys@transformshift{2.711753in}{1.688899in}%
\pgfsys@useobject{currentmarker}{}%
\end{pgfscope}%
\begin{pgfscope}%
\pgfsys@transformshift{2.719503in}{1.692055in}%
\pgfsys@useobject{currentmarker}{}%
\end{pgfscope}%
\begin{pgfscope}%
\pgfsys@transformshift{2.727080in}{1.679544in}%
\pgfsys@useobject{currentmarker}{}%
\end{pgfscope}%
\begin{pgfscope}%
\pgfsys@transformshift{2.734980in}{1.672842in}%
\pgfsys@useobject{currentmarker}{}%
\end{pgfscope}%
\begin{pgfscope}%
\pgfsys@transformshift{2.743177in}{1.664884in}%
\pgfsys@useobject{currentmarker}{}%
\end{pgfscope}%
\begin{pgfscope}%
\pgfsys@transformshift{2.751180in}{1.668670in}%
\pgfsys@useobject{currentmarker}{}%
\end{pgfscope}%
\begin{pgfscope}%
\pgfsys@transformshift{2.758998in}{1.647537in}%
\pgfsys@useobject{currentmarker}{}%
\end{pgfscope}%
\begin{pgfscope}%
\pgfsys@transformshift{2.766641in}{1.641199in}%
\pgfsys@useobject{currentmarker}{}%
\end{pgfscope}%
\begin{pgfscope}%
\pgfsys@transformshift{2.774115in}{1.635412in}%
\pgfsys@useobject{currentmarker}{}%
\end{pgfscope}%
\begin{pgfscope}%
\pgfsys@transformshift{2.782278in}{1.638310in}%
\pgfsys@useobject{currentmarker}{}%
\end{pgfscope}%
\begin{pgfscope}%
\pgfsys@transformshift{2.790249in}{1.629929in}%
\pgfsys@useobject{currentmarker}{}%
\end{pgfscope}%
\begin{pgfscope}%
\pgfsys@transformshift{2.798037in}{1.614088in}%
\pgfsys@useobject{currentmarker}{}%
\end{pgfscope}%
\begin{pgfscope}%
\pgfsys@transformshift{2.806047in}{1.614473in}%
\pgfsys@useobject{currentmarker}{}%
\end{pgfscope}%
\begin{pgfscope}%
\pgfsys@transformshift{2.813871in}{1.600726in}%
\pgfsys@useobject{currentmarker}{}%
\end{pgfscope}%
\begin{pgfscope}%
\pgfsys@transformshift{2.821519in}{1.599082in}%
\pgfsys@useobject{currentmarker}{}%
\end{pgfscope}%
\begin{pgfscope}%
\pgfsys@transformshift{2.829368in}{1.599589in}%
\pgfsys@useobject{currentmarker}{}%
\end{pgfscope}%
\begin{pgfscope}%
\pgfsys@transformshift{2.837040in}{1.589361in}%
\pgfsys@useobject{currentmarker}{}%
\end{pgfscope}%
\begin{pgfscope}%
\pgfsys@transformshift{2.844895in}{1.591960in}%
\pgfsys@useobject{currentmarker}{}%
\end{pgfscope}%
\begin{pgfscope}%
\pgfsys@transformshift{2.852917in}{1.575274in}%
\pgfsys@useobject{currentmarker}{}%
\end{pgfscope}%
\begin{pgfscope}%
\pgfsys@transformshift{2.860754in}{1.570242in}%
\pgfsys@useobject{currentmarker}{}%
\end{pgfscope}%
\begin{pgfscope}%
\pgfsys@transformshift{2.868413in}{1.564665in}%
\pgfsys@useobject{currentmarker}{}%
\end{pgfscope}%
\begin{pgfscope}%
\pgfsys@transformshift{2.876226in}{1.549913in}%
\pgfsys@useobject{currentmarker}{}%
\end{pgfscope}%
\begin{pgfscope}%
\pgfsys@transformshift{2.884177in}{1.558970in}%
\pgfsys@useobject{currentmarker}{}%
\end{pgfscope}%
\begin{pgfscope}%
\pgfsys@transformshift{2.891946in}{1.542427in}%
\pgfsys@useobject{currentmarker}{}%
\end{pgfscope}%
\begin{pgfscope}%
\pgfsys@transformshift{2.899841in}{1.546029in}%
\pgfsys@useobject{currentmarker}{}%
\end{pgfscope}%
\begin{pgfscope}%
\pgfsys@transformshift{2.907557in}{1.530025in}%
\pgfsys@useobject{currentmarker}{}%
\end{pgfscope}%
\begin{pgfscope}%
\pgfsys@transformshift{2.915388in}{1.529997in}%
\pgfsys@useobject{currentmarker}{}%
\end{pgfscope}%
\begin{pgfscope}%
\pgfsys@transformshift{2.923322in}{1.519984in}%
\pgfsys@useobject{currentmarker}{}%
\end{pgfscope}%
\begin{pgfscope}%
\pgfsys@transformshift{2.931075in}{1.517926in}%
\pgfsys@useobject{currentmarker}{}%
\end{pgfscope}%
\begin{pgfscope}%
\pgfsys@transformshift{2.938923in}{1.512908in}%
\pgfsys@useobject{currentmarker}{}%
\end{pgfscope}%
\begin{pgfscope}%
\pgfsys@transformshift{2.946854in}{1.499336in}%
\pgfsys@useobject{currentmarker}{}%
\end{pgfscope}%
\begin{pgfscope}%
\pgfsys@transformshift{2.954604in}{1.492570in}%
\pgfsys@useobject{currentmarker}{}%
\end{pgfscope}%
\begin{pgfscope}%
\pgfsys@transformshift{2.962430in}{1.488204in}%
\pgfsys@useobject{currentmarker}{}%
\end{pgfscope}%
\begin{pgfscope}%
\pgfsys@transformshift{2.970324in}{1.485822in}%
\pgfsys@useobject{currentmarker}{}%
\end{pgfscope}%
\begin{pgfscope}%
\pgfsys@transformshift{2.978039in}{1.477367in}%
\pgfsys@useobject{currentmarker}{}%
\end{pgfscope}%
\begin{pgfscope}%
\pgfsys@transformshift{2.985815in}{1.467965in}%
\pgfsys@useobject{currentmarker}{}%
\end{pgfscope}%
\begin{pgfscope}%
\pgfsys@transformshift{2.993644in}{1.466007in}%
\pgfsys@useobject{currentmarker}{}%
\end{pgfscope}%
\begin{pgfscope}%
\pgfsys@transformshift{3.001519in}{1.452237in}%
\pgfsys@useobject{currentmarker}{}%
\end{pgfscope}%
\begin{pgfscope}%
\pgfsys@transformshift{3.009433in}{1.455087in}%
\pgfsys@useobject{currentmarker}{}%
\end{pgfscope}%
\begin{pgfscope}%
\pgfsys@transformshift{3.017166in}{1.444148in}%
\pgfsys@useobject{currentmarker}{}%
\end{pgfscope}%
\begin{pgfscope}%
\pgfsys@transformshift{3.024935in}{1.436237in}%
\pgfsys@useobject{currentmarker}{}%
\end{pgfscope}%
\begin{pgfscope}%
\pgfsys@transformshift{3.032732in}{1.428828in}%
\pgfsys@useobject{currentmarker}{}%
\end{pgfscope}%
\begin{pgfscope}%
\pgfsys@transformshift{3.040553in}{1.432246in}%
\pgfsys@useobject{currentmarker}{}%
\end{pgfscope}%
\begin{pgfscope}%
\pgfsys@transformshift{3.048391in}{1.418699in}%
\pgfsys@useobject{currentmarker}{}%
\end{pgfscope}%
\begin{pgfscope}%
\pgfsys@transformshift{3.056241in}{1.411626in}%
\pgfsys@useobject{currentmarker}{}%
\end{pgfscope}%
\begin{pgfscope}%
\pgfsys@transformshift{3.064099in}{1.400894in}%
\pgfsys@useobject{currentmarker}{}%
\end{pgfscope}%
\begin{pgfscope}%
\pgfsys@transformshift{3.071960in}{1.398123in}%
\pgfsys@useobject{currentmarker}{}%
\end{pgfscope}%
\begin{pgfscope}%
\pgfsys@transformshift{3.079819in}{1.394982in}%
\pgfsys@useobject{currentmarker}{}%
\end{pgfscope}%
\begin{pgfscope}%
\pgfsys@transformshift{3.087673in}{1.386823in}%
\pgfsys@useobject{currentmarker}{}%
\end{pgfscope}%
\begin{pgfscope}%
\pgfsys@transformshift{3.095517in}{1.380144in}%
\pgfsys@useobject{currentmarker}{}%
\end{pgfscope}%
\begin{pgfscope}%
\pgfsys@transformshift{3.103349in}{1.376718in}%
\pgfsys@useobject{currentmarker}{}%
\end{pgfscope}%
\begin{pgfscope}%
\pgfsys@transformshift{3.111166in}{1.368263in}%
\pgfsys@useobject{currentmarker}{}%
\end{pgfscope}%
\begin{pgfscope}%
\pgfsys@transformshift{3.118963in}{1.368114in}%
\pgfsys@useobject{currentmarker}{}%
\end{pgfscope}%
\begin{pgfscope}%
\pgfsys@transformshift{3.126739in}{1.353066in}%
\pgfsys@useobject{currentmarker}{}%
\end{pgfscope}%
\begin{pgfscope}%
\pgfsys@transformshift{3.134491in}{1.347703in}%
\pgfsys@useobject{currentmarker}{}%
\end{pgfscope}%
\begin{pgfscope}%
\pgfsys@transformshift{3.142364in}{1.343550in}%
\pgfsys@useobject{currentmarker}{}%
\end{pgfscope}%
\begin{pgfscope}%
\pgfsys@transformshift{3.150202in}{1.334570in}%
\pgfsys@useobject{currentmarker}{}%
\end{pgfscope}%
\begin{pgfscope}%
\pgfsys@transformshift{3.158002in}{1.325484in}%
\pgfsys@useobject{currentmarker}{}%
\end{pgfscope}%
\begin{pgfscope}%
\pgfsys@transformshift{3.165902in}{1.322108in}%
\pgfsys@useobject{currentmarker}{}%
\end{pgfscope}%
\begin{pgfscope}%
\pgfsys@transformshift{3.173756in}{1.317947in}%
\pgfsys@useobject{currentmarker}{}%
\end{pgfscope}%
\begin{pgfscope}%
\pgfsys@transformshift{3.181562in}{1.305417in}%
\pgfsys@useobject{currentmarker}{}%
\end{pgfscope}%
\begin{pgfscope}%
\pgfsys@transformshift{3.189449in}{1.305961in}%
\pgfsys@useobject{currentmarker}{}%
\end{pgfscope}%
\begin{pgfscope}%
\pgfsys@transformshift{3.197281in}{1.300219in}%
\pgfsys@useobject{currentmarker}{}%
\end{pgfscope}%
\begin{pgfscope}%
\pgfsys@transformshift{3.205059in}{1.289502in}%
\pgfsys@useobject{currentmarker}{}%
\end{pgfscope}%
\begin{pgfscope}%
\pgfsys@transformshift{3.212901in}{1.288984in}%
\pgfsys@useobject{currentmarker}{}%
\end{pgfscope}%
\begin{pgfscope}%
\pgfsys@transformshift{3.220799in}{1.281645in}%
\pgfsys@useobject{currentmarker}{}%
\end{pgfscope}%
\begin{pgfscope}%
\pgfsys@transformshift{3.228631in}{1.275754in}%
\pgfsys@useobject{currentmarker}{}%
\end{pgfscope}%
\begin{pgfscope}%
\pgfsys@transformshift{3.236397in}{1.267001in}%
\pgfsys@useobject{currentmarker}{}%
\end{pgfscope}%
\begin{pgfscope}%
\pgfsys@transformshift{3.244207in}{1.264374in}%
\pgfsys@useobject{currentmarker}{}%
\end{pgfscope}%
\begin{pgfscope}%
\pgfsys@transformshift{3.252054in}{1.255234in}%
\pgfsys@useobject{currentmarker}{}%
\end{pgfscope}%
\begin{pgfscope}%
\pgfsys@transformshift{3.259932in}{1.250595in}%
\pgfsys@useobject{currentmarker}{}%
\end{pgfscope}%
\begin{pgfscope}%
\pgfsys@transformshift{3.267732in}{1.242916in}%
\pgfsys@useobject{currentmarker}{}%
\end{pgfscope}%
\begin{pgfscope}%
\pgfsys@transformshift{3.275554in}{1.236193in}%
\pgfsys@useobject{currentmarker}{}%
\end{pgfscope}%
\begin{pgfscope}%
\pgfsys@transformshift{3.283395in}{1.228386in}%
\pgfsys@useobject{currentmarker}{}%
\end{pgfscope}%
\begin{pgfscope}%
\pgfsys@transformshift{3.291153in}{1.225093in}%
\pgfsys@useobject{currentmarker}{}%
\end{pgfscope}%
\begin{pgfscope}%
\pgfsys@transformshift{3.299015in}{1.217284in}%
\pgfsys@useobject{currentmarker}{}%
\end{pgfscope}%
\begin{pgfscope}%
\pgfsys@transformshift{3.306880in}{1.211545in}%
\pgfsys@useobject{currentmarker}{}%
\end{pgfscope}%
\begin{pgfscope}%
\pgfsys@transformshift{3.314655in}{1.203353in}%
\pgfsys@useobject{currentmarker}{}%
\end{pgfscope}%
\begin{pgfscope}%
\pgfsys@transformshift{3.322514in}{1.193682in}%
\pgfsys@useobject{currentmarker}{}%
\end{pgfscope}%
\begin{pgfscope}%
\pgfsys@transformshift{3.330365in}{1.189060in}%
\pgfsys@useobject{currentmarker}{}%
\end{pgfscope}%
\begin{pgfscope}%
\pgfsys@transformshift{3.338203in}{1.184324in}%
\pgfsys@useobject{currentmarker}{}%
\end{pgfscope}%
\begin{pgfscope}%
\pgfsys@transformshift{3.346025in}{1.178167in}%
\pgfsys@useobject{currentmarker}{}%
\end{pgfscope}%
\begin{pgfscope}%
\pgfsys@transformshift{3.353828in}{1.173559in}%
\pgfsys@useobject{currentmarker}{}%
\end{pgfscope}%
\begin{pgfscope}%
\pgfsys@transformshift{3.361686in}{1.167920in}%
\pgfsys@useobject{currentmarker}{}%
\end{pgfscope}%
\begin{pgfscope}%
\pgfsys@transformshift{3.369517in}{1.160531in}%
\pgfsys@useobject{currentmarker}{}%
\end{pgfscope}%
\begin{pgfscope}%
\pgfsys@transformshift{3.377318in}{1.157265in}%
\pgfsys@useobject{currentmarker}{}%
\end{pgfscope}%
\begin{pgfscope}%
\pgfsys@transformshift{3.385159in}{1.146412in}%
\pgfsys@useobject{currentmarker}{}%
\end{pgfscope}%
\begin{pgfscope}%
\pgfsys@transformshift{3.393033in}{1.143415in}%
\pgfsys@useobject{currentmarker}{}%
\end{pgfscope}%
\begin{pgfscope}%
\pgfsys@transformshift{3.400865in}{1.138258in}%
\pgfsys@useobject{currentmarker}{}%
\end{pgfscope}%
\begin{pgfscope}%
\pgfsys@transformshift{3.408655in}{1.127803in}%
\pgfsys@useobject{currentmarker}{}%
\end{pgfscope}%
\begin{pgfscope}%
\pgfsys@transformshift{3.416466in}{1.123552in}%
\pgfsys@useobject{currentmarker}{}%
\end{pgfscope}%
\begin{pgfscope}%
\pgfsys@transformshift{3.424294in}{1.119173in}%
\pgfsys@useobject{currentmarker}{}%
\end{pgfscope}%
\begin{pgfscope}%
\pgfsys@transformshift{3.432132in}{1.113106in}%
\pgfsys@useobject{currentmarker}{}%
\end{pgfscope}%
\begin{pgfscope}%
\pgfsys@transformshift{3.439976in}{1.109386in}%
\pgfsys@useobject{currentmarker}{}%
\end{pgfscope}%
\begin{pgfscope}%
\pgfsys@transformshift{3.447823in}{1.100879in}%
\pgfsys@useobject{currentmarker}{}%
\end{pgfscope}%
\begin{pgfscope}%
\pgfsys@transformshift{3.455666in}{1.094077in}%
\pgfsys@useobject{currentmarker}{}%
\end{pgfscope}%
\begin{pgfscope}%
\pgfsys@transformshift{3.463447in}{1.089718in}%
\pgfsys@useobject{currentmarker}{}%
\end{pgfscope}%
\begin{pgfscope}%
\pgfsys@transformshift{3.471275in}{1.080117in}%
\pgfsys@useobject{currentmarker}{}%
\end{pgfscope}%
\begin{pgfscope}%
\pgfsys@transformshift{3.479145in}{1.077536in}%
\pgfsys@useobject{currentmarker}{}%
\end{pgfscope}%
\begin{pgfscope}%
\pgfsys@transformshift{3.486942in}{1.072077in}%
\pgfsys@useobject{currentmarker}{}%
\end{pgfscope}%
\begin{pgfscope}%
\pgfsys@transformshift{3.494773in}{1.068861in}%
\pgfsys@useobject{currentmarker}{}%
\end{pgfscope}%
\begin{pgfscope}%
\pgfsys@transformshift{3.502629in}{1.057550in}%
\pgfsys@useobject{currentmarker}{}%
\end{pgfscope}%
\begin{pgfscope}%
\pgfsys@transformshift{3.510457in}{1.053673in}%
\pgfsys@useobject{currentmarker}{}%
\end{pgfscope}%
\begin{pgfscope}%
\pgfsys@transformshift{3.518253in}{1.050134in}%
\pgfsys@useobject{currentmarker}{}%
\end{pgfscope}%
\begin{pgfscope}%
\pgfsys@transformshift{3.526064in}{1.043666in}%
\pgfsys@useobject{currentmarker}{}%
\end{pgfscope}%
\begin{pgfscope}%
\pgfsys@transformshift{3.533930in}{1.037843in}%
\pgfsys@useobject{currentmarker}{}%
\end{pgfscope}%
\begin{pgfscope}%
\pgfsys@transformshift{3.541754in}{1.033860in}%
\pgfsys@useobject{currentmarker}{}%
\end{pgfscope}%
\begin{pgfscope}%
\pgfsys@transformshift{3.549578in}{1.026047in}%
\pgfsys@useobject{currentmarker}{}%
\end{pgfscope}%
\begin{pgfscope}%
\pgfsys@transformshift{3.557399in}{1.017522in}%
\pgfsys@useobject{currentmarker}{}%
\end{pgfscope}%
\begin{pgfscope}%
\pgfsys@transformshift{3.565212in}{1.015202in}%
\pgfsys@useobject{currentmarker}{}%
\end{pgfscope}%
\begin{pgfscope}%
\pgfsys@transformshift{3.573056in}{1.007339in}%
\pgfsys@useobject{currentmarker}{}%
\end{pgfscope}%
\begin{pgfscope}%
\pgfsys@transformshift{3.580883in}{1.004465in}%
\pgfsys@useobject{currentmarker}{}%
\end{pgfscope}%
\begin{pgfscope}%
\pgfsys@transformshift{3.588731in}{0.993937in}%
\pgfsys@useobject{currentmarker}{}%
\end{pgfscope}%
\begin{pgfscope}%
\pgfsys@transformshift{3.596556in}{0.992775in}%
\pgfsys@useobject{currentmarker}{}%
\end{pgfscope}%
\begin{pgfscope}%
\pgfsys@transformshift{3.604354in}{0.986978in}%
\pgfsys@useobject{currentmarker}{}%
\end{pgfscope}%
\begin{pgfscope}%
\pgfsys@transformshift{3.612198in}{0.983277in}%
\pgfsys@useobject{currentmarker}{}%
\end{pgfscope}%
\begin{pgfscope}%
\pgfsys@transformshift{3.620044in}{0.976143in}%
\pgfsys@useobject{currentmarker}{}%
\end{pgfscope}%
\begin{pgfscope}%
\pgfsys@transformshift{3.627888in}{0.973304in}%
\pgfsys@useobject{currentmarker}{}%
\end{pgfscope}%
\begin{pgfscope}%
\pgfsys@transformshift{3.635726in}{0.966732in}%
\pgfsys@useobject{currentmarker}{}%
\end{pgfscope}%
\begin{pgfscope}%
\pgfsys@transformshift{3.643555in}{0.960707in}%
\pgfsys@useobject{currentmarker}{}%
\end{pgfscope}%
\begin{pgfscope}%
\pgfsys@transformshift{3.651371in}{0.955502in}%
\pgfsys@useobject{currentmarker}{}%
\end{pgfscope}%
\begin{pgfscope}%
\pgfsys@transformshift{3.659170in}{0.953343in}%
\pgfsys@useobject{currentmarker}{}%
\end{pgfscope}%
\begin{pgfscope}%
\pgfsys@transformshift{3.667014in}{0.943937in}%
\pgfsys@useobject{currentmarker}{}%
\end{pgfscope}%
\begin{pgfscope}%
\pgfsys@transformshift{3.674863in}{0.942055in}%
\pgfsys@useobject{currentmarker}{}%
\end{pgfscope}%
\begin{pgfscope}%
\pgfsys@transformshift{3.682684in}{0.938829in}%
\pgfsys@useobject{currentmarker}{}%
\end{pgfscope}%
\begin{pgfscope}%
\pgfsys@transformshift{3.690504in}{0.932359in}%
\pgfsys@useobject{currentmarker}{}%
\end{pgfscope}%
\begin{pgfscope}%
\pgfsys@transformshift{3.698319in}{0.928450in}%
\pgfsys@useobject{currentmarker}{}%
\end{pgfscope}%
\begin{pgfscope}%
\pgfsys@transformshift{3.706153in}{0.923952in}%
\pgfsys@useobject{currentmarker}{}%
\end{pgfscope}%
\begin{pgfscope}%
\pgfsys@transformshift{3.714000in}{0.921470in}%
\pgfsys@useobject{currentmarker}{}%
\end{pgfscope}%
\begin{pgfscope}%
\pgfsys@transformshift{3.721830in}{0.915759in}%
\pgfsys@useobject{currentmarker}{}%
\end{pgfscope}%
\begin{pgfscope}%
\pgfsys@transformshift{3.729665in}{0.910090in}%
\pgfsys@useobject{currentmarker}{}%
\end{pgfscope}%
\begin{pgfscope}%
\pgfsys@transformshift{3.737501in}{0.907638in}%
\pgfsys@useobject{currentmarker}{}%
\end{pgfscope}%
\begin{pgfscope}%
\pgfsys@transformshift{3.745309in}{0.902834in}%
\pgfsys@useobject{currentmarker}{}%
\end{pgfscope}%
\begin{pgfscope}%
\pgfsys@transformshift{3.753135in}{0.900327in}%
\pgfsys@useobject{currentmarker}{}%
\end{pgfscope}%
\begin{pgfscope}%
\pgfsys@transformshift{3.760975in}{0.895926in}%
\pgfsys@useobject{currentmarker}{}%
\end{pgfscope}%
\begin{pgfscope}%
\pgfsys@transformshift{3.768799in}{0.895633in}%
\pgfsys@useobject{currentmarker}{}%
\end{pgfscope}%
\begin{pgfscope}%
\pgfsys@transformshift{3.776628in}{0.890559in}%
\pgfsys@useobject{currentmarker}{}%
\end{pgfscope}%
\begin{pgfscope}%
\pgfsys@transformshift{3.784458in}{0.886817in}%
\pgfsys@useobject{currentmarker}{}%
\end{pgfscope}%
\begin{pgfscope}%
\pgfsys@transformshift{3.792284in}{0.883504in}%
\pgfsys@useobject{currentmarker}{}%
\end{pgfscope}%
\begin{pgfscope}%
\pgfsys@transformshift{3.800123in}{0.880033in}%
\pgfsys@useobject{currentmarker}{}%
\end{pgfscope}%
\begin{pgfscope}%
\pgfsys@transformshift{3.807950in}{0.876751in}%
\pgfsys@useobject{currentmarker}{}%
\end{pgfscope}%
\begin{pgfscope}%
\pgfsys@transformshift{3.815762in}{0.876603in}%
\pgfsys@useobject{currentmarker}{}%
\end{pgfscope}%
\begin{pgfscope}%
\pgfsys@transformshift{3.823596in}{0.874755in}%
\pgfsys@useobject{currentmarker}{}%
\end{pgfscope}%
\begin{pgfscope}%
\pgfsys@transformshift{3.831425in}{0.872852in}%
\pgfsys@useobject{currentmarker}{}%
\end{pgfscope}%
\begin{pgfscope}%
\pgfsys@transformshift{3.839267in}{0.870759in}%
\pgfsys@useobject{currentmarker}{}%
\end{pgfscope}%
\begin{pgfscope}%
\pgfsys@transformshift{3.847096in}{0.868349in}%
\pgfsys@useobject{currentmarker}{}%
\end{pgfscope}%
\begin{pgfscope}%
\pgfsys@transformshift{3.854911in}{0.868101in}%
\pgfsys@useobject{currentmarker}{}%
\end{pgfscope}%
\begin{pgfscope}%
\pgfsys@transformshift{3.862743in}{0.868406in}%
\pgfsys@useobject{currentmarker}{}%
\end{pgfscope}%
\begin{pgfscope}%
\pgfsys@transformshift{3.870569in}{0.866777in}%
\pgfsys@useobject{currentmarker}{}%
\end{pgfscope}%
\end{pgfscope}%
\begin{pgfscope}%
\pgfpathrectangle{\pgfqpoint{0.594525in}{0.417642in}}{\pgfqpoint{3.432047in}{2.016277in}}%
\pgfusepath{clip}%
\pgfsetbuttcap%
\pgfsetroundjoin%
\pgfsetlinewidth{1.505625pt}%
\definecolor{currentstroke}{rgb}{0.835294,0.368627,0.000000}%
\pgfsetstrokecolor{currentstroke}%
\pgfsetdash{{5.550000pt}{2.400000pt}}{0.000000pt}%
\pgfpathmoveto{\pgfqpoint{0.750527in}{2.266308in}}%
\pgfpathlineto{\pgfqpoint{0.985627in}{2.264961in}}%
\pgfpathlineto{\pgfqpoint{1.123152in}{2.262746in}}%
\pgfpathlineto{\pgfqpoint{1.220728in}{2.259703in}}%
\pgfpathlineto{\pgfqpoint{1.296413in}{2.255885in}}%
\pgfpathlineto{\pgfqpoint{1.358253in}{2.251358in}}%
\pgfpathlineto{\pgfqpoint{1.410538in}{2.246191in}}%
\pgfpathlineto{\pgfqpoint{1.455829in}{2.240460in}}%
\pgfpathlineto{\pgfqpoint{1.531514in}{2.227600in}}%
\pgfpathlineto{\pgfqpoint{1.593354in}{2.213346in}}%
\pgfpathlineto{\pgfqpoint{1.645638in}{2.198197in}}%
\pgfpathlineto{\pgfqpoint{1.690929in}{2.182554in}}%
\pgfpathlineto{\pgfqpoint{1.730879in}{2.166727in}}%
\pgfpathlineto{\pgfqpoint{1.783163in}{2.143119in}}%
\pgfpathlineto{\pgfqpoint{1.828454in}{2.120094in}}%
\pgfpathlineto{\pgfqpoint{1.880739in}{2.090744in}}%
\pgfpathlineto{\pgfqpoint{1.936467in}{2.056542in}}%
\pgfpathlineto{\pgfqpoint{1.993128in}{2.019139in}}%
\pgfpathlineto{\pgfqpoint{2.063555in}{1.969686in}}%
\pgfpathlineto{\pgfqpoint{2.139240in}{1.913782in}}%
\pgfpathlineto{\pgfqpoint{2.234689in}{1.840491in}}%
\pgfpathlineto{\pgfqpoint{2.367200in}{1.735651in}}%
\pgfpathlineto{\pgfqpoint{2.570550in}{1.571478in}}%
\pgfpathlineto{\pgfqpoint{2.993644in}{1.226334in}}%
\pgfpathlineto{\pgfqpoint{3.870569in}{0.509291in}}%
\pgfpathlineto{\pgfqpoint{3.870569in}{0.509291in}}%
\pgfusepath{stroke}%
\end{pgfscope}%
\begin{pgfscope}%
\pgfpathrectangle{\pgfqpoint{0.594525in}{0.417642in}}{\pgfqpoint{3.432047in}{2.016277in}}%
\pgfusepath{clip}%
\pgfsetbuttcap%
\pgfsetroundjoin%
\definecolor{currentfill}{rgb}{0.835294,0.368627,0.000000}%
\pgfsetfillcolor{currentfill}%
\pgfsetlinewidth{1.003750pt}%
\definecolor{currentstroke}{rgb}{0.835294,0.368627,0.000000}%
\pgfsetstrokecolor{currentstroke}%
\pgfsetdash{}{0pt}%
\pgfsys@defobject{currentmarker}{\pgfqpoint{-0.006944in}{-0.006944in}}{\pgfqpoint{0.006944in}{0.006944in}}{%
\pgfpathmoveto{\pgfqpoint{0.000000in}{-0.006944in}}%
\pgfpathcurveto{\pgfqpoint{0.001842in}{-0.006944in}}{\pgfqpoint{0.003608in}{-0.006213in}}{\pgfqpoint{0.004910in}{-0.004910in}}%
\pgfpathcurveto{\pgfqpoint{0.006213in}{-0.003608in}}{\pgfqpoint{0.006944in}{-0.001842in}}{\pgfqpoint{0.006944in}{0.000000in}}%
\pgfpathcurveto{\pgfqpoint{0.006944in}{0.001842in}}{\pgfqpoint{0.006213in}{0.003608in}}{\pgfqpoint{0.004910in}{0.004910in}}%
\pgfpathcurveto{\pgfqpoint{0.003608in}{0.006213in}}{\pgfqpoint{0.001842in}{0.006944in}}{\pgfqpoint{0.000000in}{0.006944in}}%
\pgfpathcurveto{\pgfqpoint{-0.001842in}{0.006944in}}{\pgfqpoint{-0.003608in}{0.006213in}}{\pgfqpoint{-0.004910in}{0.004910in}}%
\pgfpathcurveto{\pgfqpoint{-0.006213in}{0.003608in}}{\pgfqpoint{-0.006944in}{0.001842in}}{\pgfqpoint{-0.006944in}{0.000000in}}%
\pgfpathcurveto{\pgfqpoint{-0.006944in}{-0.001842in}}{\pgfqpoint{-0.006213in}{-0.003608in}}{\pgfqpoint{-0.004910in}{-0.004910in}}%
\pgfpathcurveto{\pgfqpoint{-0.003608in}{-0.006213in}}{\pgfqpoint{-0.001842in}{-0.006944in}}{\pgfqpoint{0.000000in}{-0.006944in}}%
\pgfpathlineto{\pgfqpoint{0.000000in}{-0.006944in}}%
\pgfpathclose%
\pgfusepath{stroke,fill}%
}%
\begin{pgfscope}%
\pgfsys@transformshift{0.750527in}{2.242922in}%
\pgfsys@useobject{currentmarker}{}%
\end{pgfscope}%
\begin{pgfscope}%
\pgfsys@transformshift{0.985627in}{2.267711in}%
\pgfsys@useobject{currentmarker}{}%
\end{pgfscope}%
\begin{pgfscope}%
\pgfsys@transformshift{1.123152in}{2.258194in}%
\pgfsys@useobject{currentmarker}{}%
\end{pgfscope}%
\begin{pgfscope}%
\pgfsys@transformshift{1.220728in}{2.255756in}%
\pgfsys@useobject{currentmarker}{}%
\end{pgfscope}%
\begin{pgfscope}%
\pgfsys@transformshift{1.296413in}{2.258359in}%
\pgfsys@useobject{currentmarker}{}%
\end{pgfscope}%
\begin{pgfscope}%
\pgfsys@transformshift{1.358253in}{2.248696in}%
\pgfsys@useobject{currentmarker}{}%
\end{pgfscope}%
\begin{pgfscope}%
\pgfsys@transformshift{1.410538in}{2.230399in}%
\pgfsys@useobject{currentmarker}{}%
\end{pgfscope}%
\begin{pgfscope}%
\pgfsys@transformshift{1.455829in}{2.232375in}%
\pgfsys@useobject{currentmarker}{}%
\end{pgfscope}%
\begin{pgfscope}%
\pgfsys@transformshift{1.495778in}{2.224078in}%
\pgfsys@useobject{currentmarker}{}%
\end{pgfscope}%
\begin{pgfscope}%
\pgfsys@transformshift{1.531514in}{2.216072in}%
\pgfsys@useobject{currentmarker}{}%
\end{pgfscope}%
\begin{pgfscope}%
\pgfsys@transformshift{1.563841in}{2.204274in}%
\pgfsys@useobject{currentmarker}{}%
\end{pgfscope}%
\begin{pgfscope}%
\pgfsys@transformshift{1.593354in}{2.184573in}%
\pgfsys@useobject{currentmarker}{}%
\end{pgfscope}%
\begin{pgfscope}%
\pgfsys@transformshift{1.620502in}{2.178099in}%
\pgfsys@useobject{currentmarker}{}%
\end{pgfscope}%
\begin{pgfscope}%
\pgfsys@transformshift{1.645638in}{2.195228in}%
\pgfsys@useobject{currentmarker}{}%
\end{pgfscope}%
\begin{pgfscope}%
\pgfsys@transformshift{1.669039in}{2.193206in}%
\pgfsys@useobject{currentmarker}{}%
\end{pgfscope}%
\begin{pgfscope}%
\pgfsys@transformshift{1.690929in}{2.173130in}%
\pgfsys@useobject{currentmarker}{}%
\end{pgfscope}%
\begin{pgfscope}%
\pgfsys@transformshift{1.711492in}{2.155337in}%
\pgfsys@useobject{currentmarker}{}%
\end{pgfscope}%
\begin{pgfscope}%
\pgfsys@transformshift{1.730879in}{2.155788in}%
\pgfsys@useobject{currentmarker}{}%
\end{pgfscope}%
\begin{pgfscope}%
\pgfsys@transformshift{1.749217in}{2.156894in}%
\pgfsys@useobject{currentmarker}{}%
\end{pgfscope}%
\begin{pgfscope}%
\pgfsys@transformshift{1.766615in}{2.144229in}%
\pgfsys@useobject{currentmarker}{}%
\end{pgfscope}%
\begin{pgfscope}%
\pgfsys@transformshift{1.783163in}{2.137224in}%
\pgfsys@useobject{currentmarker}{}%
\end{pgfscope}%
\begin{pgfscope}%
\pgfsys@transformshift{1.798942in}{2.144611in}%
\pgfsys@useobject{currentmarker}{}%
\end{pgfscope}%
\begin{pgfscope}%
\pgfsys@transformshift{1.814019in}{2.130034in}%
\pgfsys@useobject{currentmarker}{}%
\end{pgfscope}%
\begin{pgfscope}%
\pgfsys@transformshift{1.828454in}{2.124470in}%
\pgfsys@useobject{currentmarker}{}%
\end{pgfscope}%
\begin{pgfscope}%
\pgfsys@transformshift{1.842300in}{2.108289in}%
\pgfsys@useobject{currentmarker}{}%
\end{pgfscope}%
\begin{pgfscope}%
\pgfsys@transformshift{1.855603in}{2.102036in}%
\pgfsys@useobject{currentmarker}{}%
\end{pgfscope}%
\begin{pgfscope}%
\pgfsys@transformshift{1.868404in}{2.104159in}%
\pgfsys@useobject{currentmarker}{}%
\end{pgfscope}%
\begin{pgfscope}%
\pgfsys@transformshift{1.880739in}{2.082110in}%
\pgfsys@useobject{currentmarker}{}%
\end{pgfscope}%
\begin{pgfscope}%
\pgfsys@transformshift{1.892641in}{2.077393in}%
\pgfsys@useobject{currentmarker}{}%
\end{pgfscope}%
\begin{pgfscope}%
\pgfsys@transformshift{1.904140in}{2.080732in}%
\pgfsys@useobject{currentmarker}{}%
\end{pgfscope}%
\begin{pgfscope}%
\pgfsys@transformshift{1.915261in}{2.066059in}%
\pgfsys@useobject{currentmarker}{}%
\end{pgfscope}%
\begin{pgfscope}%
\pgfsys@transformshift{1.926030in}{2.043936in}%
\pgfsys@useobject{currentmarker}{}%
\end{pgfscope}%
\begin{pgfscope}%
\pgfsys@transformshift{1.936467in}{2.034250in}%
\pgfsys@useobject{currentmarker}{}%
\end{pgfscope}%
\begin{pgfscope}%
\pgfsys@transformshift{1.946592in}{2.039148in}%
\pgfsys@useobject{currentmarker}{}%
\end{pgfscope}%
\begin{pgfscope}%
\pgfsys@transformshift{1.956424in}{2.038467in}%
\pgfsys@useobject{currentmarker}{}%
\end{pgfscope}%
\begin{pgfscope}%
\pgfsys@transformshift{1.965979in}{2.047961in}%
\pgfsys@useobject{currentmarker}{}%
\end{pgfscope}%
\begin{pgfscope}%
\pgfsys@transformshift{1.975272in}{2.029116in}%
\pgfsys@useobject{currentmarker}{}%
\end{pgfscope}%
\begin{pgfscope}%
\pgfsys@transformshift{1.984318in}{2.025141in}%
\pgfsys@useobject{currentmarker}{}%
\end{pgfscope}%
\begin{pgfscope}%
\pgfsys@transformshift{1.993128in}{2.023049in}%
\pgfsys@useobject{currentmarker}{}%
\end{pgfscope}%
\begin{pgfscope}%
\pgfsys@transformshift{2.001715in}{2.014879in}%
\pgfsys@useobject{currentmarker}{}%
\end{pgfscope}%
\begin{pgfscope}%
\pgfsys@transformshift{2.010090in}{2.009596in}%
\pgfsys@useobject{currentmarker}{}%
\end{pgfscope}%
\begin{pgfscope}%
\pgfsys@transformshift{2.018264in}{1.992857in}%
\pgfsys@useobject{currentmarker}{}%
\end{pgfscope}%
\begin{pgfscope}%
\pgfsys@transformshift{2.026245in}{1.981543in}%
\pgfsys@useobject{currentmarker}{}%
\end{pgfscope}%
\begin{pgfscope}%
\pgfsys@transformshift{2.034042in}{1.980208in}%
\pgfsys@useobject{currentmarker}{}%
\end{pgfscope}%
\begin{pgfscope}%
\pgfsys@transformshift{2.041665in}{1.982414in}%
\pgfsys@useobject{currentmarker}{}%
\end{pgfscope}%
\begin{pgfscope}%
\pgfsys@transformshift{2.049119in}{1.991603in}%
\pgfsys@useobject{currentmarker}{}%
\end{pgfscope}%
\begin{pgfscope}%
\pgfsys@transformshift{2.056414in}{1.955989in}%
\pgfsys@useobject{currentmarker}{}%
\end{pgfscope}%
\begin{pgfscope}%
\pgfsys@transformshift{2.063555in}{1.949986in}%
\pgfsys@useobject{currentmarker}{}%
\end{pgfscope}%
\begin{pgfscope}%
\pgfsys@transformshift{2.070548in}{1.954281in}%
\pgfsys@useobject{currentmarker}{}%
\end{pgfscope}%
\begin{pgfscope}%
\pgfsys@transformshift{2.077401in}{1.951415in}%
\pgfsys@useobject{currentmarker}{}%
\end{pgfscope}%
\begin{pgfscope}%
\pgfsys@transformshift{2.084117in}{1.939417in}%
\pgfsys@useobject{currentmarker}{}%
\end{pgfscope}%
\begin{pgfscope}%
\pgfsys@transformshift{2.093949in}{1.950091in}%
\pgfsys@useobject{currentmarker}{}%
\end{pgfscope}%
\begin{pgfscope}%
\pgfsys@transformshift{2.103504in}{1.936985in}%
\pgfsys@useobject{currentmarker}{}%
\end{pgfscope}%
\begin{pgfscope}%
\pgfsys@transformshift{2.109728in}{1.928830in}%
\pgfsys@useobject{currentmarker}{}%
\end{pgfscope}%
\begin{pgfscope}%
\pgfsys@transformshift{2.115839in}{1.912373in}%
\pgfsys@useobject{currentmarker}{}%
\end{pgfscope}%
\begin{pgfscope}%
\pgfsys@transformshift{2.124805in}{1.910191in}%
\pgfsys@useobject{currentmarker}{}%
\end{pgfscope}%
\begin{pgfscope}%
\pgfsys@transformshift{2.133540in}{1.919928in}%
\pgfsys@useobject{currentmarker}{}%
\end{pgfscope}%
\begin{pgfscope}%
\pgfsys@transformshift{2.139240in}{1.906125in}%
\pgfsys@useobject{currentmarker}{}%
\end{pgfscope}%
\begin{pgfscope}%
\pgfsys@transformshift{2.147615in}{1.915295in}%
\pgfsys@useobject{currentmarker}{}%
\end{pgfscope}%
\begin{pgfscope}%
\pgfsys@transformshift{2.155789in}{1.912846in}%
\pgfsys@useobject{currentmarker}{}%
\end{pgfscope}%
\begin{pgfscope}%
\pgfsys@transformshift{2.163770in}{1.915975in}%
\pgfsys@useobject{currentmarker}{}%
\end{pgfscope}%
\begin{pgfscope}%
\pgfsys@transformshift{2.171567in}{1.907866in}%
\pgfsys@useobject{currentmarker}{}%
\end{pgfscope}%
\begin{pgfscope}%
\pgfsys@transformshift{2.179190in}{1.890533in}%
\pgfsys@useobject{currentmarker}{}%
\end{pgfscope}%
\begin{pgfscope}%
\pgfsys@transformshift{2.186644in}{1.877368in}%
\pgfsys@useobject{currentmarker}{}%
\end{pgfscope}%
\begin{pgfscope}%
\pgfsys@transformshift{2.193939in}{1.892414in}%
\pgfsys@useobject{currentmarker}{}%
\end{pgfscope}%
\begin{pgfscope}%
\pgfsys@transformshift{2.203427in}{1.896018in}%
\pgfsys@useobject{currentmarker}{}%
\end{pgfscope}%
\begin{pgfscope}%
\pgfsys@transformshift{2.210373in}{1.880621in}%
\pgfsys@useobject{currentmarker}{}%
\end{pgfscope}%
\begin{pgfscope}%
\pgfsys@transformshift{2.217179in}{1.862126in}%
\pgfsys@useobject{currentmarker}{}%
\end{pgfscope}%
\begin{pgfscope}%
\pgfsys@transformshift{2.226047in}{1.862298in}%
\pgfsys@useobject{currentmarker}{}%
\end{pgfscope}%
\begin{pgfscope}%
\pgfsys@transformshift{2.234689in}{1.850041in}%
\pgfsys@useobject{currentmarker}{}%
\end{pgfscope}%
\begin{pgfscope}%
\pgfsys@transformshift{2.243116in}{1.836644in}%
\pgfsys@useobject{currentmarker}{}%
\end{pgfscope}%
\begin{pgfscope}%
\pgfsys@transformshift{2.251339in}{1.839732in}%
\pgfsys@useobject{currentmarker}{}%
\end{pgfscope}%
\begin{pgfscope}%
\pgfsys@transformshift{2.259368in}{1.820793in}%
\pgfsys@useobject{currentmarker}{}%
\end{pgfscope}%
\begin{pgfscope}%
\pgfsys@transformshift{2.267210in}{1.826608in}%
\pgfsys@useobject{currentmarker}{}%
\end{pgfscope}%
\begin{pgfscope}%
\pgfsys@transformshift{2.274876in}{1.817028in}%
\pgfsys@useobject{currentmarker}{}%
\end{pgfscope}%
\begin{pgfscope}%
\pgfsys@transformshift{2.282372in}{1.816160in}%
\pgfsys@useobject{currentmarker}{}%
\end{pgfscope}%
\begin{pgfscope}%
\pgfsys@transformshift{2.289705in}{1.795115in}%
\pgfsys@useobject{currentmarker}{}%
\end{pgfscope}%
\begin{pgfscope}%
\pgfsys@transformshift{2.296884in}{1.784473in}%
\pgfsys@useobject{currentmarker}{}%
\end{pgfscope}%
\begin{pgfscope}%
\pgfsys@transformshift{2.303914in}{1.799479in}%
\pgfsys@useobject{currentmarker}{}%
\end{pgfscope}%
\begin{pgfscope}%
\pgfsys@transformshift{2.312501in}{1.783885in}%
\pgfsys@useobject{currentmarker}{}%
\end{pgfscope}%
\begin{pgfscope}%
\pgfsys@transformshift{2.320876in}{1.782203in}%
\pgfsys@useobject{currentmarker}{}%
\end{pgfscope}%
\begin{pgfscope}%
\pgfsys@transformshift{2.327431in}{1.766910in}%
\pgfsys@useobject{currentmarker}{}%
\end{pgfscope}%
\begin{pgfscope}%
\pgfsys@transformshift{2.335450in}{1.763729in}%
\pgfsys@useobject{currentmarker}{}%
\end{pgfscope}%
\begin{pgfscope}%
\pgfsys@transformshift{2.343283in}{1.759899in}%
\pgfsys@useobject{currentmarker}{}%
\end{pgfscope}%
\begin{pgfscope}%
\pgfsys@transformshift{2.350940in}{1.750471in}%
\pgfsys@useobject{currentmarker}{}%
\end{pgfscope}%
\begin{pgfscope}%
\pgfsys@transformshift{2.359905in}{1.747630in}%
\pgfsys@useobject{currentmarker}{}%
\end{pgfscope}%
\begin{pgfscope}%
\pgfsys@transformshift{2.367200in}{1.759412in}%
\pgfsys@useobject{currentmarker}{}%
\end{pgfscope}%
\begin{pgfscope}%
\pgfsys@transformshift{2.374341in}{1.739285in}%
\pgfsys@useobject{currentmarker}{}%
\end{pgfscope}%
\begin{pgfscope}%
\pgfsys@transformshift{2.382716in}{1.729515in}%
\pgfsys@useobject{currentmarker}{}%
\end{pgfscope}%
\begin{pgfscope}%
\pgfsys@transformshift{2.390889in}{1.718967in}%
\pgfsys@useobject{currentmarker}{}%
\end{pgfscope}%
\begin{pgfscope}%
\pgfsys@transformshift{2.398870in}{1.718020in}%
\pgfsys@useobject{currentmarker}{}%
\end{pgfscope}%
\begin{pgfscope}%
\pgfsys@transformshift{2.406668in}{1.709188in}%
\pgfsys@useobject{currentmarker}{}%
\end{pgfscope}%
\begin{pgfscope}%
\pgfsys@transformshift{2.414290in}{1.720707in}%
\pgfsys@useobject{currentmarker}{}%
\end{pgfscope}%
\begin{pgfscope}%
\pgfsys@transformshift{2.421745in}{1.695925in}%
\pgfsys@useobject{currentmarker}{}%
\end{pgfscope}%
\begin{pgfscope}%
\pgfsys@transformshift{2.430240in}{1.677145in}%
\pgfsys@useobject{currentmarker}{}%
\end{pgfscope}%
\begin{pgfscope}%
\pgfsys@transformshift{2.438527in}{1.674665in}%
\pgfsys@useobject{currentmarker}{}%
\end{pgfscope}%
\begin{pgfscope}%
\pgfsys@transformshift{2.445473in}{1.671102in}%
\pgfsys@useobject{currentmarker}{}%
\end{pgfscope}%
\begin{pgfscope}%
\pgfsys@transformshift{2.453401in}{1.685078in}%
\pgfsys@useobject{currentmarker}{}%
\end{pgfscope}%
\begin{pgfscope}%
\pgfsys@transformshift{2.461148in}{1.652248in}%
\pgfsys@useobject{currentmarker}{}%
\end{pgfscope}%
\begin{pgfscope}%
\pgfsys@transformshift{2.468721in}{1.661166in}%
\pgfsys@useobject{currentmarker}{}%
\end{pgfscope}%
\begin{pgfscope}%
\pgfsys@transformshift{2.477175in}{1.654755in}%
\pgfsys@useobject{currentmarker}{}%
\end{pgfscope}%
\begin{pgfscope}%
\pgfsys@transformshift{2.485423in}{1.647235in}%
\pgfsys@useobject{currentmarker}{}%
\end{pgfscope}%
\begin{pgfscope}%
\pgfsys@transformshift{2.493475in}{1.627739in}%
\pgfsys@useobject{currentmarker}{}%
\end{pgfscope}%
\begin{pgfscope}%
\pgfsys@transformshift{2.501340in}{1.629400in}%
\pgfsys@useobject{currentmarker}{}%
\end{pgfscope}%
\begin{pgfscope}%
\pgfsys@transformshift{2.509027in}{1.615493in}%
\pgfsys@useobject{currentmarker}{}%
\end{pgfscope}%
\begin{pgfscope}%
\pgfsys@transformshift{2.516544in}{1.602405in}%
\pgfsys@useobject{currentmarker}{}%
\end{pgfscope}%
\begin{pgfscope}%
\pgfsys@transformshift{2.523898in}{1.596418in}%
\pgfsys@useobject{currentmarker}{}%
\end{pgfscope}%
\begin{pgfscope}%
\pgfsys@transformshift{2.531985in}{1.602197in}%
\pgfsys@useobject{currentmarker}{}%
\end{pgfscope}%
\begin{pgfscope}%
\pgfsys@transformshift{2.539883in}{1.600115in}%
\pgfsys@useobject{currentmarker}{}%
\end{pgfscope}%
\begin{pgfscope}%
\pgfsys@transformshift{2.547602in}{1.595470in}%
\pgfsys@useobject{currentmarker}{}%
\end{pgfscope}%
\begin{pgfscope}%
\pgfsys@transformshift{2.555149in}{1.575269in}%
\pgfsys@useobject{currentmarker}{}%
\end{pgfscope}%
\begin{pgfscope}%
\pgfsys@transformshift{2.562531in}{1.577498in}%
\pgfsys@useobject{currentmarker}{}%
\end{pgfscope}%
\begin{pgfscope}%
\pgfsys@transformshift{2.570550in}{1.572853in}%
\pgfsys@useobject{currentmarker}{}%
\end{pgfscope}%
\begin{pgfscope}%
\pgfsys@transformshift{2.578384in}{1.564076in}%
\pgfsys@useobject{currentmarker}{}%
\end{pgfscope}%
\begin{pgfscope}%
\pgfsys@transformshift{2.586040in}{1.562440in}%
\pgfsys@useobject{currentmarker}{}%
\end{pgfscope}%
\begin{pgfscope}%
\pgfsys@transformshift{2.594268in}{1.559649in}%
\pgfsys@useobject{currentmarker}{}%
\end{pgfscope}%
\begin{pgfscope}%
\pgfsys@transformshift{2.602300in}{1.557994in}%
\pgfsys@useobject{currentmarker}{}%
\end{pgfscope}%
\begin{pgfscope}%
\pgfsys@transformshift{2.610147in}{1.543039in}%
\pgfsys@useobject{currentmarker}{}%
\end{pgfscope}%
\begin{pgfscope}%
\pgfsys@transformshift{2.617816in}{1.537611in}%
\pgfsys@useobject{currentmarker}{}%
\end{pgfscope}%
\begin{pgfscope}%
\pgfsys@transformshift{2.625316in}{1.526578in}%
\pgfsys@useobject{currentmarker}{}%
\end{pgfscope}%
\begin{pgfscope}%
\pgfsys@transformshift{2.633313in}{1.526751in}%
\pgfsys@useobject{currentmarker}{}%
\end{pgfscope}%
\begin{pgfscope}%
\pgfsys@transformshift{2.641125in}{1.517120in}%
\pgfsys@useobject{currentmarker}{}%
\end{pgfscope}%
\begin{pgfscope}%
\pgfsys@transformshift{2.648762in}{1.520533in}%
\pgfsys@useobject{currentmarker}{}%
\end{pgfscope}%
\begin{pgfscope}%
\pgfsys@transformshift{2.656845in}{1.495295in}%
\pgfsys@useobject{currentmarker}{}%
\end{pgfscope}%
\begin{pgfscope}%
\pgfsys@transformshift{2.664741in}{1.494986in}%
\pgfsys@useobject{currentmarker}{}%
\end{pgfscope}%
\begin{pgfscope}%
\pgfsys@transformshift{2.672456in}{1.493313in}%
\pgfsys@useobject{currentmarker}{}%
\end{pgfscope}%
\begin{pgfscope}%
\pgfsys@transformshift{2.680574in}{1.488856in}%
\pgfsys@useobject{currentmarker}{}%
\end{pgfscope}%
\begin{pgfscope}%
\pgfsys@transformshift{2.688502in}{1.470535in}%
\pgfsys@useobject{currentmarker}{}%
\end{pgfscope}%
\begin{pgfscope}%
\pgfsys@transformshift{2.696248in}{1.469258in}%
\pgfsys@useobject{currentmarker}{}%
\end{pgfscope}%
\begin{pgfscope}%
\pgfsys@transformshift{2.703822in}{1.462565in}%
\pgfsys@useobject{currentmarker}{}%
\end{pgfscope}%
\begin{pgfscope}%
\pgfsys@transformshift{2.711753in}{1.466159in}%
\pgfsys@useobject{currentmarker}{}%
\end{pgfscope}%
\begin{pgfscope}%
\pgfsys@transformshift{2.719503in}{1.454392in}%
\pgfsys@useobject{currentmarker}{}%
\end{pgfscope}%
\begin{pgfscope}%
\pgfsys@transformshift{2.727080in}{1.453268in}%
\pgfsys@useobject{currentmarker}{}%
\end{pgfscope}%
\begin{pgfscope}%
\pgfsys@transformshift{2.734980in}{1.441298in}%
\pgfsys@useobject{currentmarker}{}%
\end{pgfscope}%
\begin{pgfscope}%
\pgfsys@transformshift{2.743177in}{1.443303in}%
\pgfsys@useobject{currentmarker}{}%
\end{pgfscope}%
\begin{pgfscope}%
\pgfsys@transformshift{2.751180in}{1.426114in}%
\pgfsys@useobject{currentmarker}{}%
\end{pgfscope}%
\begin{pgfscope}%
\pgfsys@transformshift{2.758998in}{1.415950in}%
\pgfsys@useobject{currentmarker}{}%
\end{pgfscope}%
\begin{pgfscope}%
\pgfsys@transformshift{2.766641in}{1.413140in}%
\pgfsys@useobject{currentmarker}{}%
\end{pgfscope}%
\begin{pgfscope}%
\pgfsys@transformshift{2.774115in}{1.417712in}%
\pgfsys@useobject{currentmarker}{}%
\end{pgfscope}%
\begin{pgfscope}%
\pgfsys@transformshift{2.782278in}{1.395287in}%
\pgfsys@useobject{currentmarker}{}%
\end{pgfscope}%
\begin{pgfscope}%
\pgfsys@transformshift{2.790249in}{1.398763in}%
\pgfsys@useobject{currentmarker}{}%
\end{pgfscope}%
\begin{pgfscope}%
\pgfsys@transformshift{2.798037in}{1.382873in}%
\pgfsys@useobject{currentmarker}{}%
\end{pgfscope}%
\begin{pgfscope}%
\pgfsys@transformshift{2.806047in}{1.382456in}%
\pgfsys@useobject{currentmarker}{}%
\end{pgfscope}%
\begin{pgfscope}%
\pgfsys@transformshift{2.813871in}{1.375556in}%
\pgfsys@useobject{currentmarker}{}%
\end{pgfscope}%
\begin{pgfscope}%
\pgfsys@transformshift{2.821519in}{1.376608in}%
\pgfsys@useobject{currentmarker}{}%
\end{pgfscope}%
\begin{pgfscope}%
\pgfsys@transformshift{2.829368in}{1.364229in}%
\pgfsys@useobject{currentmarker}{}%
\end{pgfscope}%
\begin{pgfscope}%
\pgfsys@transformshift{2.837040in}{1.351091in}%
\pgfsys@useobject{currentmarker}{}%
\end{pgfscope}%
\begin{pgfscope}%
\pgfsys@transformshift{2.844895in}{1.351069in}%
\pgfsys@useobject{currentmarker}{}%
\end{pgfscope}%
\begin{pgfscope}%
\pgfsys@transformshift{2.852917in}{1.346645in}%
\pgfsys@useobject{currentmarker}{}%
\end{pgfscope}%
\begin{pgfscope}%
\pgfsys@transformshift{2.860754in}{1.335580in}%
\pgfsys@useobject{currentmarker}{}%
\end{pgfscope}%
\begin{pgfscope}%
\pgfsys@transformshift{2.868413in}{1.324545in}%
\pgfsys@useobject{currentmarker}{}%
\end{pgfscope}%
\begin{pgfscope}%
\pgfsys@transformshift{2.876226in}{1.322872in}%
\pgfsys@useobject{currentmarker}{}%
\end{pgfscope}%
\begin{pgfscope}%
\pgfsys@transformshift{2.884177in}{1.315381in}%
\pgfsys@useobject{currentmarker}{}%
\end{pgfscope}%
\begin{pgfscope}%
\pgfsys@transformshift{2.891946in}{1.310877in}%
\pgfsys@useobject{currentmarker}{}%
\end{pgfscope}%
\begin{pgfscope}%
\pgfsys@transformshift{2.899841in}{1.311630in}%
\pgfsys@useobject{currentmarker}{}%
\end{pgfscope}%
\begin{pgfscope}%
\pgfsys@transformshift{2.907557in}{1.298928in}%
\pgfsys@useobject{currentmarker}{}%
\end{pgfscope}%
\begin{pgfscope}%
\pgfsys@transformshift{2.915388in}{1.293082in}%
\pgfsys@useobject{currentmarker}{}%
\end{pgfscope}%
\begin{pgfscope}%
\pgfsys@transformshift{2.923322in}{1.290849in}%
\pgfsys@useobject{currentmarker}{}%
\end{pgfscope}%
\begin{pgfscope}%
\pgfsys@transformshift{2.931075in}{1.281180in}%
\pgfsys@useobject{currentmarker}{}%
\end{pgfscope}%
\begin{pgfscope}%
\pgfsys@transformshift{2.938923in}{1.275929in}%
\pgfsys@useobject{currentmarker}{}%
\end{pgfscope}%
\begin{pgfscope}%
\pgfsys@transformshift{2.946854in}{1.266086in}%
\pgfsys@useobject{currentmarker}{}%
\end{pgfscope}%
\begin{pgfscope}%
\pgfsys@transformshift{2.954604in}{1.257792in}%
\pgfsys@useobject{currentmarker}{}%
\end{pgfscope}%
\begin{pgfscope}%
\pgfsys@transformshift{2.962430in}{1.255149in}%
\pgfsys@useobject{currentmarker}{}%
\end{pgfscope}%
\begin{pgfscope}%
\pgfsys@transformshift{2.970324in}{1.250399in}%
\pgfsys@useobject{currentmarker}{}%
\end{pgfscope}%
\begin{pgfscope}%
\pgfsys@transformshift{2.978039in}{1.246859in}%
\pgfsys@useobject{currentmarker}{}%
\end{pgfscope}%
\begin{pgfscope}%
\pgfsys@transformshift{2.985815in}{1.236412in}%
\pgfsys@useobject{currentmarker}{}%
\end{pgfscope}%
\begin{pgfscope}%
\pgfsys@transformshift{2.993644in}{1.236908in}%
\pgfsys@useobject{currentmarker}{}%
\end{pgfscope}%
\begin{pgfscope}%
\pgfsys@transformshift{3.001519in}{1.215458in}%
\pgfsys@useobject{currentmarker}{}%
\end{pgfscope}%
\begin{pgfscope}%
\pgfsys@transformshift{3.009433in}{1.216849in}%
\pgfsys@useobject{currentmarker}{}%
\end{pgfscope}%
\begin{pgfscope}%
\pgfsys@transformshift{3.017166in}{1.204607in}%
\pgfsys@useobject{currentmarker}{}%
\end{pgfscope}%
\begin{pgfscope}%
\pgfsys@transformshift{3.024935in}{1.204762in}%
\pgfsys@useobject{currentmarker}{}%
\end{pgfscope}%
\begin{pgfscope}%
\pgfsys@transformshift{3.032732in}{1.199263in}%
\pgfsys@useobject{currentmarker}{}%
\end{pgfscope}%
\begin{pgfscope}%
\pgfsys@transformshift{3.040553in}{1.195386in}%
\pgfsys@useobject{currentmarker}{}%
\end{pgfscope}%
\begin{pgfscope}%
\pgfsys@transformshift{3.048391in}{1.192342in}%
\pgfsys@useobject{currentmarker}{}%
\end{pgfscope}%
\begin{pgfscope}%
\pgfsys@transformshift{3.056241in}{1.181419in}%
\pgfsys@useobject{currentmarker}{}%
\end{pgfscope}%
\begin{pgfscope}%
\pgfsys@transformshift{3.064099in}{1.173918in}%
\pgfsys@useobject{currentmarker}{}%
\end{pgfscope}%
\begin{pgfscope}%
\pgfsys@transformshift{3.071960in}{1.169274in}%
\pgfsys@useobject{currentmarker}{}%
\end{pgfscope}%
\begin{pgfscope}%
\pgfsys@transformshift{3.079819in}{1.156078in}%
\pgfsys@useobject{currentmarker}{}%
\end{pgfscope}%
\begin{pgfscope}%
\pgfsys@transformshift{3.087673in}{1.150215in}%
\pgfsys@useobject{currentmarker}{}%
\end{pgfscope}%
\begin{pgfscope}%
\pgfsys@transformshift{3.095517in}{1.138747in}%
\pgfsys@useobject{currentmarker}{}%
\end{pgfscope}%
\begin{pgfscope}%
\pgfsys@transformshift{3.103349in}{1.140384in}%
\pgfsys@useobject{currentmarker}{}%
\end{pgfscope}%
\begin{pgfscope}%
\pgfsys@transformshift{3.111166in}{1.129719in}%
\pgfsys@useobject{currentmarker}{}%
\end{pgfscope}%
\begin{pgfscope}%
\pgfsys@transformshift{3.118963in}{1.130123in}%
\pgfsys@useobject{currentmarker}{}%
\end{pgfscope}%
\begin{pgfscope}%
\pgfsys@transformshift{3.126739in}{1.124494in}%
\pgfsys@useobject{currentmarker}{}%
\end{pgfscope}%
\begin{pgfscope}%
\pgfsys@transformshift{3.134491in}{1.115415in}%
\pgfsys@useobject{currentmarker}{}%
\end{pgfscope}%
\begin{pgfscope}%
\pgfsys@transformshift{3.142364in}{1.103524in}%
\pgfsys@useobject{currentmarker}{}%
\end{pgfscope}%
\begin{pgfscope}%
\pgfsys@transformshift{3.150202in}{1.096645in}%
\pgfsys@useobject{currentmarker}{}%
\end{pgfscope}%
\begin{pgfscope}%
\pgfsys@transformshift{3.158002in}{1.089515in}%
\pgfsys@useobject{currentmarker}{}%
\end{pgfscope}%
\begin{pgfscope}%
\pgfsys@transformshift{3.165902in}{1.085629in}%
\pgfsys@useobject{currentmarker}{}%
\end{pgfscope}%
\begin{pgfscope}%
\pgfsys@transformshift{3.173756in}{1.079553in}%
\pgfsys@useobject{currentmarker}{}%
\end{pgfscope}%
\begin{pgfscope}%
\pgfsys@transformshift{3.181562in}{1.073667in}%
\pgfsys@useobject{currentmarker}{}%
\end{pgfscope}%
\begin{pgfscope}%
\pgfsys@transformshift{3.189449in}{1.068513in}%
\pgfsys@useobject{currentmarker}{}%
\end{pgfscope}%
\begin{pgfscope}%
\pgfsys@transformshift{3.197281in}{1.064873in}%
\pgfsys@useobject{currentmarker}{}%
\end{pgfscope}%
\begin{pgfscope}%
\pgfsys@transformshift{3.205059in}{1.058285in}%
\pgfsys@useobject{currentmarker}{}%
\end{pgfscope}%
\begin{pgfscope}%
\pgfsys@transformshift{3.212901in}{1.055008in}%
\pgfsys@useobject{currentmarker}{}%
\end{pgfscope}%
\begin{pgfscope}%
\pgfsys@transformshift{3.220799in}{1.047326in}%
\pgfsys@useobject{currentmarker}{}%
\end{pgfscope}%
\begin{pgfscope}%
\pgfsys@transformshift{3.228631in}{1.043061in}%
\pgfsys@useobject{currentmarker}{}%
\end{pgfscope}%
\begin{pgfscope}%
\pgfsys@transformshift{3.236397in}{1.033133in}%
\pgfsys@useobject{currentmarker}{}%
\end{pgfscope}%
\begin{pgfscope}%
\pgfsys@transformshift{3.244207in}{1.022663in}%
\pgfsys@useobject{currentmarker}{}%
\end{pgfscope}%
\begin{pgfscope}%
\pgfsys@transformshift{3.252054in}{1.017629in}%
\pgfsys@useobject{currentmarker}{}%
\end{pgfscope}%
\begin{pgfscope}%
\pgfsys@transformshift{3.259932in}{1.015853in}%
\pgfsys@useobject{currentmarker}{}%
\end{pgfscope}%
\begin{pgfscope}%
\pgfsys@transformshift{3.267732in}{1.003607in}%
\pgfsys@useobject{currentmarker}{}%
\end{pgfscope}%
\begin{pgfscope}%
\pgfsys@transformshift{3.275554in}{0.996605in}%
\pgfsys@useobject{currentmarker}{}%
\end{pgfscope}%
\begin{pgfscope}%
\pgfsys@transformshift{3.283395in}{0.999848in}%
\pgfsys@useobject{currentmarker}{}%
\end{pgfscope}%
\begin{pgfscope}%
\pgfsys@transformshift{3.291153in}{0.987825in}%
\pgfsys@useobject{currentmarker}{}%
\end{pgfscope}%
\begin{pgfscope}%
\pgfsys@transformshift{3.299015in}{0.980948in}%
\pgfsys@useobject{currentmarker}{}%
\end{pgfscope}%
\begin{pgfscope}%
\pgfsys@transformshift{3.306880in}{0.972393in}%
\pgfsys@useobject{currentmarker}{}%
\end{pgfscope}%
\begin{pgfscope}%
\pgfsys@transformshift{3.314655in}{0.971079in}%
\pgfsys@useobject{currentmarker}{}%
\end{pgfscope}%
\begin{pgfscope}%
\pgfsys@transformshift{3.322514in}{0.967335in}%
\pgfsys@useobject{currentmarker}{}%
\end{pgfscope}%
\begin{pgfscope}%
\pgfsys@transformshift{3.330365in}{0.959972in}%
\pgfsys@useobject{currentmarker}{}%
\end{pgfscope}%
\begin{pgfscope}%
\pgfsys@transformshift{3.338203in}{0.953885in}%
\pgfsys@useobject{currentmarker}{}%
\end{pgfscope}%
\begin{pgfscope}%
\pgfsys@transformshift{3.346025in}{0.942309in}%
\pgfsys@useobject{currentmarker}{}%
\end{pgfscope}%
\begin{pgfscope}%
\pgfsys@transformshift{3.353828in}{0.935709in}%
\pgfsys@useobject{currentmarker}{}%
\end{pgfscope}%
\begin{pgfscope}%
\pgfsys@transformshift{3.361686in}{0.926802in}%
\pgfsys@useobject{currentmarker}{}%
\end{pgfscope}%
\begin{pgfscope}%
\pgfsys@transformshift{3.369517in}{0.930654in}%
\pgfsys@useobject{currentmarker}{}%
\end{pgfscope}%
\begin{pgfscope}%
\pgfsys@transformshift{3.377318in}{0.924878in}%
\pgfsys@useobject{currentmarker}{}%
\end{pgfscope}%
\begin{pgfscope}%
\pgfsys@transformshift{3.385159in}{0.914307in}%
\pgfsys@useobject{currentmarker}{}%
\end{pgfscope}%
\begin{pgfscope}%
\pgfsys@transformshift{3.393033in}{0.905580in}%
\pgfsys@useobject{currentmarker}{}%
\end{pgfscope}%
\begin{pgfscope}%
\pgfsys@transformshift{3.400865in}{0.899063in}%
\pgfsys@useobject{currentmarker}{}%
\end{pgfscope}%
\begin{pgfscope}%
\pgfsys@transformshift{3.408655in}{0.899178in}%
\pgfsys@useobject{currentmarker}{}%
\end{pgfscope}%
\begin{pgfscope}%
\pgfsys@transformshift{3.416466in}{0.890228in}%
\pgfsys@useobject{currentmarker}{}%
\end{pgfscope}%
\begin{pgfscope}%
\pgfsys@transformshift{3.424294in}{0.877738in}%
\pgfsys@useobject{currentmarker}{}%
\end{pgfscope}%
\begin{pgfscope}%
\pgfsys@transformshift{3.432132in}{0.878056in}%
\pgfsys@useobject{currentmarker}{}%
\end{pgfscope}%
\begin{pgfscope}%
\pgfsys@transformshift{3.439976in}{0.874204in}%
\pgfsys@useobject{currentmarker}{}%
\end{pgfscope}%
\begin{pgfscope}%
\pgfsys@transformshift{3.447823in}{0.863637in}%
\pgfsys@useobject{currentmarker}{}%
\end{pgfscope}%
\begin{pgfscope}%
\pgfsys@transformshift{3.455666in}{0.855131in}%
\pgfsys@useobject{currentmarker}{}%
\end{pgfscope}%
\begin{pgfscope}%
\pgfsys@transformshift{3.463447in}{0.857028in}%
\pgfsys@useobject{currentmarker}{}%
\end{pgfscope}%
\begin{pgfscope}%
\pgfsys@transformshift{3.471275in}{0.849301in}%
\pgfsys@useobject{currentmarker}{}%
\end{pgfscope}%
\begin{pgfscope}%
\pgfsys@transformshift{3.479145in}{0.842214in}%
\pgfsys@useobject{currentmarker}{}%
\end{pgfscope}%
\begin{pgfscope}%
\pgfsys@transformshift{3.486942in}{0.839563in}%
\pgfsys@useobject{currentmarker}{}%
\end{pgfscope}%
\begin{pgfscope}%
\pgfsys@transformshift{3.494773in}{0.827045in}%
\pgfsys@useobject{currentmarker}{}%
\end{pgfscope}%
\begin{pgfscope}%
\pgfsys@transformshift{3.502629in}{0.825559in}%
\pgfsys@useobject{currentmarker}{}%
\end{pgfscope}%
\begin{pgfscope}%
\pgfsys@transformshift{3.510457in}{0.820198in}%
\pgfsys@useobject{currentmarker}{}%
\end{pgfscope}%
\begin{pgfscope}%
\pgfsys@transformshift{3.518253in}{0.815070in}%
\pgfsys@useobject{currentmarker}{}%
\end{pgfscope}%
\begin{pgfscope}%
\pgfsys@transformshift{3.526064in}{0.806851in}%
\pgfsys@useobject{currentmarker}{}%
\end{pgfscope}%
\begin{pgfscope}%
\pgfsys@transformshift{3.533930in}{0.802045in}%
\pgfsys@useobject{currentmarker}{}%
\end{pgfscope}%
\begin{pgfscope}%
\pgfsys@transformshift{3.541754in}{0.796165in}%
\pgfsys@useobject{currentmarker}{}%
\end{pgfscope}%
\begin{pgfscope}%
\pgfsys@transformshift{3.549578in}{0.795657in}%
\pgfsys@useobject{currentmarker}{}%
\end{pgfscope}%
\begin{pgfscope}%
\pgfsys@transformshift{3.557399in}{0.782476in}%
\pgfsys@useobject{currentmarker}{}%
\end{pgfscope}%
\begin{pgfscope}%
\pgfsys@transformshift{3.565212in}{0.778372in}%
\pgfsys@useobject{currentmarker}{}%
\end{pgfscope}%
\begin{pgfscope}%
\pgfsys@transformshift{3.573056in}{0.773704in}%
\pgfsys@useobject{currentmarker}{}%
\end{pgfscope}%
\begin{pgfscope}%
\pgfsys@transformshift{3.580883in}{0.769087in}%
\pgfsys@useobject{currentmarker}{}%
\end{pgfscope}%
\begin{pgfscope}%
\pgfsys@transformshift{3.588731in}{0.759799in}%
\pgfsys@useobject{currentmarker}{}%
\end{pgfscope}%
\begin{pgfscope}%
\pgfsys@transformshift{3.596556in}{0.761066in}%
\pgfsys@useobject{currentmarker}{}%
\end{pgfscope}%
\begin{pgfscope}%
\pgfsys@transformshift{3.604354in}{0.749908in}%
\pgfsys@useobject{currentmarker}{}%
\end{pgfscope}%
\begin{pgfscope}%
\pgfsys@transformshift{3.612198in}{0.747835in}%
\pgfsys@useobject{currentmarker}{}%
\end{pgfscope}%
\begin{pgfscope}%
\pgfsys@transformshift{3.620044in}{0.738252in}%
\pgfsys@useobject{currentmarker}{}%
\end{pgfscope}%
\begin{pgfscope}%
\pgfsys@transformshift{3.627888in}{0.736554in}%
\pgfsys@useobject{currentmarker}{}%
\end{pgfscope}%
\begin{pgfscope}%
\pgfsys@transformshift{3.635726in}{0.734400in}%
\pgfsys@useobject{currentmarker}{}%
\end{pgfscope}%
\begin{pgfscope}%
\pgfsys@transformshift{3.643555in}{0.726742in}%
\pgfsys@useobject{currentmarker}{}%
\end{pgfscope}%
\begin{pgfscope}%
\pgfsys@transformshift{3.651371in}{0.721791in}%
\pgfsys@useobject{currentmarker}{}%
\end{pgfscope}%
\begin{pgfscope}%
\pgfsys@transformshift{3.659170in}{0.716195in}%
\pgfsys@useobject{currentmarker}{}%
\end{pgfscope}%
\begin{pgfscope}%
\pgfsys@transformshift{3.667014in}{0.709154in}%
\pgfsys@useobject{currentmarker}{}%
\end{pgfscope}%
\begin{pgfscope}%
\pgfsys@transformshift{3.674863in}{0.706190in}%
\pgfsys@useobject{currentmarker}{}%
\end{pgfscope}%
\begin{pgfscope}%
\pgfsys@transformshift{3.682684in}{0.704662in}%
\pgfsys@useobject{currentmarker}{}%
\end{pgfscope}%
\begin{pgfscope}%
\pgfsys@transformshift{3.690504in}{0.699259in}%
\pgfsys@useobject{currentmarker}{}%
\end{pgfscope}%
\begin{pgfscope}%
\pgfsys@transformshift{3.698319in}{0.697705in}%
\pgfsys@useobject{currentmarker}{}%
\end{pgfscope}%
\begin{pgfscope}%
\pgfsys@transformshift{3.706153in}{0.689456in}%
\pgfsys@useobject{currentmarker}{}%
\end{pgfscope}%
\begin{pgfscope}%
\pgfsys@transformshift{3.714000in}{0.685822in}%
\pgfsys@useobject{currentmarker}{}%
\end{pgfscope}%
\begin{pgfscope}%
\pgfsys@transformshift{3.721830in}{0.680301in}%
\pgfsys@useobject{currentmarker}{}%
\end{pgfscope}%
\begin{pgfscope}%
\pgfsys@transformshift{3.729665in}{0.680140in}%
\pgfsys@useobject{currentmarker}{}%
\end{pgfscope}%
\begin{pgfscope}%
\pgfsys@transformshift{3.737501in}{0.674910in}%
\pgfsys@useobject{currentmarker}{}%
\end{pgfscope}%
\begin{pgfscope}%
\pgfsys@transformshift{3.745309in}{0.668317in}%
\pgfsys@useobject{currentmarker}{}%
\end{pgfscope}%
\begin{pgfscope}%
\pgfsys@transformshift{3.753135in}{0.667080in}%
\pgfsys@useobject{currentmarker}{}%
\end{pgfscope}%
\begin{pgfscope}%
\pgfsys@transformshift{3.760975in}{0.663041in}%
\pgfsys@useobject{currentmarker}{}%
\end{pgfscope}%
\begin{pgfscope}%
\pgfsys@transformshift{3.768799in}{0.655285in}%
\pgfsys@useobject{currentmarker}{}%
\end{pgfscope}%
\begin{pgfscope}%
\pgfsys@transformshift{3.776628in}{0.657072in}%
\pgfsys@useobject{currentmarker}{}%
\end{pgfscope}%
\begin{pgfscope}%
\pgfsys@transformshift{3.784458in}{0.651905in}%
\pgfsys@useobject{currentmarker}{}%
\end{pgfscope}%
\begin{pgfscope}%
\pgfsys@transformshift{3.792284in}{0.649929in}%
\pgfsys@useobject{currentmarker}{}%
\end{pgfscope}%
\begin{pgfscope}%
\pgfsys@transformshift{3.800123in}{0.644844in}%
\pgfsys@useobject{currentmarker}{}%
\end{pgfscope}%
\begin{pgfscope}%
\pgfsys@transformshift{3.807950in}{0.646798in}%
\pgfsys@useobject{currentmarker}{}%
\end{pgfscope}%
\begin{pgfscope}%
\pgfsys@transformshift{3.815762in}{0.646343in}%
\pgfsys@useobject{currentmarker}{}%
\end{pgfscope}%
\begin{pgfscope}%
\pgfsys@transformshift{3.823596in}{0.639796in}%
\pgfsys@useobject{currentmarker}{}%
\end{pgfscope}%
\begin{pgfscope}%
\pgfsys@transformshift{3.831425in}{0.641296in}%
\pgfsys@useobject{currentmarker}{}%
\end{pgfscope}%
\begin{pgfscope}%
\pgfsys@transformshift{3.839267in}{0.635313in}%
\pgfsys@useobject{currentmarker}{}%
\end{pgfscope}%
\begin{pgfscope}%
\pgfsys@transformshift{3.847096in}{0.636731in}%
\pgfsys@useobject{currentmarker}{}%
\end{pgfscope}%
\begin{pgfscope}%
\pgfsys@transformshift{3.854911in}{0.635184in}%
\pgfsys@useobject{currentmarker}{}%
\end{pgfscope}%
\begin{pgfscope}%
\pgfsys@transformshift{3.862743in}{0.633221in}%
\pgfsys@useobject{currentmarker}{}%
\end{pgfscope}%
\begin{pgfscope}%
\pgfsys@transformshift{3.870569in}{0.632988in}%
\pgfsys@useobject{currentmarker}{}%
\end{pgfscope}%
\end{pgfscope}%
\begin{pgfscope}%
\pgfsetrectcap%
\pgfsetmiterjoin%
\pgfsetlinewidth{0.803000pt}%
\definecolor{currentstroke}{rgb}{0.000000,0.000000,0.000000}%
\pgfsetstrokecolor{currentstroke}%
\pgfsetdash{}{0pt}%
\pgfpathmoveto{\pgfqpoint{0.594525in}{0.417642in}}%
\pgfpathlineto{\pgfqpoint{0.594525in}{2.433919in}}%
\pgfusepath{stroke}%
\end{pgfscope}%
\begin{pgfscope}%
\pgfsetrectcap%
\pgfsetmiterjoin%
\pgfsetlinewidth{0.803000pt}%
\definecolor{currentstroke}{rgb}{0.000000,0.000000,0.000000}%
\pgfsetstrokecolor{currentstroke}%
\pgfsetdash{}{0pt}%
\pgfpathmoveto{\pgfqpoint{4.026572in}{0.417642in}}%
\pgfpathlineto{\pgfqpoint{4.026572in}{2.433919in}}%
\pgfusepath{stroke}%
\end{pgfscope}%
\begin{pgfscope}%
\pgfsetrectcap%
\pgfsetmiterjoin%
\pgfsetlinewidth{0.803000pt}%
\definecolor{currentstroke}{rgb}{0.000000,0.000000,0.000000}%
\pgfsetstrokecolor{currentstroke}%
\pgfsetdash{}{0pt}%
\pgfpathmoveto{\pgfqpoint{0.594525in}{0.417642in}}%
\pgfpathlineto{\pgfqpoint{4.026572in}{0.417642in}}%
\pgfusepath{stroke}%
\end{pgfscope}%
\begin{pgfscope}%
\pgfsetrectcap%
\pgfsetmiterjoin%
\pgfsetlinewidth{0.803000pt}%
\definecolor{currentstroke}{rgb}{0.000000,0.000000,0.000000}%
\pgfsetstrokecolor{currentstroke}%
\pgfsetdash{}{0pt}%
\pgfpathmoveto{\pgfqpoint{0.594525in}{2.433919in}}%
\pgfpathlineto{\pgfqpoint{4.026572in}{2.433919in}}%
\pgfusepath{stroke}%
\end{pgfscope}%
\begin{pgfscope}%
\pgfsetbuttcap%
\pgfsetmiterjoin%
\definecolor{currentfill}{rgb}{1.000000,1.000000,1.000000}%
\pgfsetfillcolor{currentfill}%
\pgfsetfillopacity{0.800000}%
\pgfsetlinewidth{1.003750pt}%
\definecolor{currentstroke}{rgb}{0.800000,0.800000,0.800000}%
\pgfsetstrokecolor{currentstroke}%
\pgfsetstrokeopacity{0.800000}%
\pgfsetdash{}{0pt}%
\pgfpathmoveto{\pgfqpoint{0.672303in}{0.473198in}}%
\pgfpathlineto{\pgfqpoint{1.512613in}{0.473198in}}%
\pgfpathquadraticcurveto{\pgfqpoint{1.534835in}{0.473198in}}{\pgfqpoint{1.534835in}{0.495420in}}%
\pgfpathlineto{\pgfqpoint{1.534835in}{0.948975in}}%
\pgfpathquadraticcurveto{\pgfqpoint{1.534835in}{0.971197in}}{\pgfqpoint{1.512613in}{0.971197in}}%
\pgfpathlineto{\pgfqpoint{0.672303in}{0.971197in}}%
\pgfpathquadraticcurveto{\pgfqpoint{0.650080in}{0.971197in}}{\pgfqpoint{0.650080in}{0.948975in}}%
\pgfpathlineto{\pgfqpoint{0.650080in}{0.495420in}}%
\pgfpathquadraticcurveto{\pgfqpoint{0.650080in}{0.473198in}}{\pgfqpoint{0.672303in}{0.473198in}}%
\pgfpathlineto{\pgfqpoint{0.672303in}{0.473198in}}%
\pgfpathclose%
\pgfusepath{stroke,fill}%
\end{pgfscope}%
\begin{pgfscope}%
\pgfsetbuttcap%
\pgfsetroundjoin%
\pgfsetlinewidth{1.505625pt}%
\definecolor{currentstroke}{rgb}{0.003922,0.450980,0.698039}%
\pgfsetstrokecolor{currentstroke}%
\pgfsetdash{{5.550000pt}{2.400000pt}}{0.000000pt}%
\pgfpathmoveto{\pgfqpoint{0.694525in}{0.887864in}}%
\pgfpathlineto{\pgfqpoint{0.805636in}{0.887864in}}%
\pgfpathlineto{\pgfqpoint{0.916747in}{0.887864in}}%
\pgfusepath{stroke}%
\end{pgfscope}%
\begin{pgfscope}%
\definecolor{textcolor}{rgb}{0.000000,0.000000,0.000000}%
\pgfsetstrokecolor{textcolor}%
\pgfsetfillcolor{textcolor}%
\pgftext[x=1.005636in,y=0.848975in,left,base]{\color{textcolor}\rmfamily\fontsize{8.000000}{9.600000}\selectfont \(\displaystyle \bar\tau_1=\qty{0.1}{\s}\)}%
\end{pgfscope}%
\begin{pgfscope}%
\pgfsetbuttcap%
\pgfsetroundjoin%
\pgfsetlinewidth{1.505625pt}%
\definecolor{currentstroke}{rgb}{0.007843,0.619608,0.450980}%
\pgfsetstrokecolor{currentstroke}%
\pgfsetdash{{5.550000pt}{2.400000pt}}{0.000000pt}%
\pgfpathmoveto{\pgfqpoint{0.694525in}{0.732975in}}%
\pgfpathlineto{\pgfqpoint{0.805636in}{0.732975in}}%
\pgfpathlineto{\pgfqpoint{0.916747in}{0.732975in}}%
\pgfusepath{stroke}%
\end{pgfscope}%
\begin{pgfscope}%
\definecolor{textcolor}{rgb}{0.000000,0.000000,0.000000}%
\pgfsetstrokecolor{textcolor}%
\pgfsetfillcolor{textcolor}%
\pgftext[x=1.005636in,y=0.694086in,left,base]{\color{textcolor}\rmfamily\fontsize{8.000000}{9.600000}\selectfont \(\displaystyle \bar\tau_1=\qty{1}{\s}\)}%
\end{pgfscope}%
\begin{pgfscope}%
\pgfsetbuttcap%
\pgfsetroundjoin%
\pgfsetlinewidth{1.505625pt}%
\definecolor{currentstroke}{rgb}{0.835294,0.368627,0.000000}%
\pgfsetstrokecolor{currentstroke}%
\pgfsetdash{{5.550000pt}{2.400000pt}}{0.000000pt}%
\pgfpathmoveto{\pgfqpoint{0.694525in}{0.578086in}}%
\pgfpathlineto{\pgfqpoint{0.805636in}{0.578086in}}%
\pgfpathlineto{\pgfqpoint{0.916747in}{0.578086in}}%
\pgfusepath{stroke}%
\end{pgfscope}%
\begin{pgfscope}%
\definecolor{textcolor}{rgb}{0.000000,0.000000,0.000000}%
\pgfsetstrokecolor{textcolor}%
\pgfsetfillcolor{textcolor}%
\pgftext[x=1.005636in,y=0.539197in,left,base]{\color{textcolor}\rmfamily\fontsize{8.000000}{9.600000}\selectfont \(\displaystyle \bar\tau_1=\qty{10}{\s}\)}%
\end{pgfscope}%
\end{pgfpicture}%
\makeatother%
\endgroup%

        } % scalebox
        \caption{Power spectral density}
        \label{fig:burst_noise_psd}
    \end{subfigure}
    \begin{subfigure}{0.8\linewidth}
        \centering
        \scalebox{1}{%
            %% Creator: Matplotlib, PGF backend
%%
%% To include the figure in your LaTeX document, write
%%   \input{<filename>.pgf}
%%
%% Make sure the required packages are loaded in your preamble
%%   \usepackage{pgf}
%%
%% Also ensure that all the required font packages are loaded; for instance,
%% the lmodern package is sometimes necessary when using math font.
%%   \usepackage{lmodern}
%%
%% Figures using additional raster images can only be included by \input if
%% they are in the same directory as the main LaTeX file. For loading figures
%% from other directories you can use the `import` package
%%   \usepackage{import}
%%
%% and then include the figures with
%%   \import{<path to file>}{<filename>.pgf}
%%
%% Matplotlib used the following preamble
%%   \usepackage{siunitx}
%%   \sisetup{per-mode = symbol}%
%%   \usepackage{fontspec}
%%   \makeatletter\@ifpackageloaded{underscore}{}{\usepackage[strings]{underscore}}\makeatother
%%
\begingroup%
\makeatletter%
\begin{pgfpicture}%
\pgfpathrectangle{\pgfpointorigin}{\pgfqpoint{4.068242in}{2.514312in}}%
\pgfusepath{use as bounding box, clip}%
\begin{pgfscope}%
\pgfsetbuttcap%
\pgfsetmiterjoin%
\definecolor{currentfill}{rgb}{1.000000,1.000000,1.000000}%
\pgfsetfillcolor{currentfill}%
\pgfsetlinewidth{0.000000pt}%
\definecolor{currentstroke}{rgb}{1.000000,1.000000,1.000000}%
\pgfsetstrokecolor{currentstroke}%
\pgfsetdash{}{0pt}%
\pgfpathmoveto{\pgfqpoint{0.000000in}{0.000000in}}%
\pgfpathlineto{\pgfqpoint{4.068242in}{0.000000in}}%
\pgfpathlineto{\pgfqpoint{4.068242in}{2.514312in}}%
\pgfpathlineto{\pgfqpoint{0.000000in}{2.514312in}}%
\pgfpathlineto{\pgfqpoint{0.000000in}{0.000000in}}%
\pgfpathclose%
\pgfusepath{fill}%
\end{pgfscope}%
\begin{pgfscope}%
\pgfsetbuttcap%
\pgfsetmiterjoin%
\definecolor{currentfill}{rgb}{1.000000,1.000000,1.000000}%
\pgfsetfillcolor{currentfill}%
\pgfsetlinewidth{0.000000pt}%
\definecolor{currentstroke}{rgb}{0.000000,0.000000,0.000000}%
\pgfsetstrokecolor{currentstroke}%
\pgfsetstrokeopacity{0.000000}%
\pgfsetdash{}{0pt}%
\pgfpathmoveto{\pgfqpoint{0.589510in}{0.417642in}}%
\pgfpathlineto{\pgfqpoint{4.026572in}{0.417642in}}%
\pgfpathlineto{\pgfqpoint{4.026572in}{2.472642in}}%
\pgfpathlineto{\pgfqpoint{0.589510in}{2.472642in}}%
\pgfpathlineto{\pgfqpoint{0.589510in}{0.417642in}}%
\pgfpathclose%
\pgfusepath{fill}%
\end{pgfscope}%
\begin{pgfscope}%
\pgfpathrectangle{\pgfqpoint{0.589510in}{0.417642in}}{\pgfqpoint{3.437062in}{2.055000in}}%
\pgfusepath{clip}%
\pgfsetrectcap%
\pgfsetroundjoin%
\pgfsetlinewidth{0.803000pt}%
\definecolor{currentstroke}{rgb}{0.450000,0.450000,0.450000}%
\pgfsetstrokecolor{currentstroke}%
\pgfsetdash{}{0pt}%
\pgfpathmoveto{\pgfqpoint{0.745740in}{0.417642in}}%
\pgfpathlineto{\pgfqpoint{0.745740in}{2.472642in}}%
\pgfusepath{stroke}%
\end{pgfscope}%
\begin{pgfscope}%
\pgfsetbuttcap%
\pgfsetroundjoin%
\definecolor{currentfill}{rgb}{0.000000,0.000000,0.000000}%
\pgfsetfillcolor{currentfill}%
\pgfsetlinewidth{0.803000pt}%
\definecolor{currentstroke}{rgb}{0.000000,0.000000,0.000000}%
\pgfsetstrokecolor{currentstroke}%
\pgfsetdash{}{0pt}%
\pgfsys@defobject{currentmarker}{\pgfqpoint{0.000000in}{-0.048611in}}{\pgfqpoint{0.000000in}{0.000000in}}{%
\pgfpathmoveto{\pgfqpoint{0.000000in}{0.000000in}}%
\pgfpathlineto{\pgfqpoint{0.000000in}{-0.048611in}}%
\pgfusepath{stroke,fill}%
}%
\begin{pgfscope}%
\pgfsys@transformshift{0.745740in}{0.417642in}%
\pgfsys@useobject{currentmarker}{}%
\end{pgfscope}%
\end{pgfscope}%
\begin{pgfscope}%
\definecolor{textcolor}{rgb}{0.000000,0.000000,0.000000}%
\pgfsetstrokecolor{textcolor}%
\pgfsetfillcolor{textcolor}%
\pgftext[x=0.745740in,y=0.320420in,,top]{\color{textcolor}\rmfamily\fontsize{8.000000}{9.600000}\selectfont \(\displaystyle {10^{-2}}\)}%
\end{pgfscope}%
\begin{pgfscope}%
\pgfpathrectangle{\pgfqpoint{0.589510in}{0.417642in}}{\pgfqpoint{3.437062in}{2.055000in}}%
\pgfusepath{clip}%
\pgfsetrectcap%
\pgfsetroundjoin%
\pgfsetlinewidth{0.803000pt}%
\definecolor{currentstroke}{rgb}{0.450000,0.450000,0.450000}%
\pgfsetstrokecolor{currentstroke}%
\pgfsetdash{}{0pt}%
\pgfpathmoveto{\pgfqpoint{1.526890in}{0.417642in}}%
\pgfpathlineto{\pgfqpoint{1.526890in}{2.472642in}}%
\pgfusepath{stroke}%
\end{pgfscope}%
\begin{pgfscope}%
\pgfsetbuttcap%
\pgfsetroundjoin%
\definecolor{currentfill}{rgb}{0.000000,0.000000,0.000000}%
\pgfsetfillcolor{currentfill}%
\pgfsetlinewidth{0.803000pt}%
\definecolor{currentstroke}{rgb}{0.000000,0.000000,0.000000}%
\pgfsetstrokecolor{currentstroke}%
\pgfsetdash{}{0pt}%
\pgfsys@defobject{currentmarker}{\pgfqpoint{0.000000in}{-0.048611in}}{\pgfqpoint{0.000000in}{0.000000in}}{%
\pgfpathmoveto{\pgfqpoint{0.000000in}{0.000000in}}%
\pgfpathlineto{\pgfqpoint{0.000000in}{-0.048611in}}%
\pgfusepath{stroke,fill}%
}%
\begin{pgfscope}%
\pgfsys@transformshift{1.526890in}{0.417642in}%
\pgfsys@useobject{currentmarker}{}%
\end{pgfscope}%
\end{pgfscope}%
\begin{pgfscope}%
\definecolor{textcolor}{rgb}{0.000000,0.000000,0.000000}%
\pgfsetstrokecolor{textcolor}%
\pgfsetfillcolor{textcolor}%
\pgftext[x=1.526890in,y=0.320420in,,top]{\color{textcolor}\rmfamily\fontsize{8.000000}{9.600000}\selectfont \(\displaystyle {10^{-1}}\)}%
\end{pgfscope}%
\begin{pgfscope}%
\pgfpathrectangle{\pgfqpoint{0.589510in}{0.417642in}}{\pgfqpoint{3.437062in}{2.055000in}}%
\pgfusepath{clip}%
\pgfsetrectcap%
\pgfsetroundjoin%
\pgfsetlinewidth{0.803000pt}%
\definecolor{currentstroke}{rgb}{0.450000,0.450000,0.450000}%
\pgfsetstrokecolor{currentstroke}%
\pgfsetdash{}{0pt}%
\pgfpathmoveto{\pgfqpoint{2.308041in}{0.417642in}}%
\pgfpathlineto{\pgfqpoint{2.308041in}{2.472642in}}%
\pgfusepath{stroke}%
\end{pgfscope}%
\begin{pgfscope}%
\pgfsetbuttcap%
\pgfsetroundjoin%
\definecolor{currentfill}{rgb}{0.000000,0.000000,0.000000}%
\pgfsetfillcolor{currentfill}%
\pgfsetlinewidth{0.803000pt}%
\definecolor{currentstroke}{rgb}{0.000000,0.000000,0.000000}%
\pgfsetstrokecolor{currentstroke}%
\pgfsetdash{}{0pt}%
\pgfsys@defobject{currentmarker}{\pgfqpoint{0.000000in}{-0.048611in}}{\pgfqpoint{0.000000in}{0.000000in}}{%
\pgfpathmoveto{\pgfqpoint{0.000000in}{0.000000in}}%
\pgfpathlineto{\pgfqpoint{0.000000in}{-0.048611in}}%
\pgfusepath{stroke,fill}%
}%
\begin{pgfscope}%
\pgfsys@transformshift{2.308041in}{0.417642in}%
\pgfsys@useobject{currentmarker}{}%
\end{pgfscope}%
\end{pgfscope}%
\begin{pgfscope}%
\definecolor{textcolor}{rgb}{0.000000,0.000000,0.000000}%
\pgfsetstrokecolor{textcolor}%
\pgfsetfillcolor{textcolor}%
\pgftext[x=2.308041in,y=0.320420in,,top]{\color{textcolor}\rmfamily\fontsize{8.000000}{9.600000}\selectfont \(\displaystyle {10^{0}}\)}%
\end{pgfscope}%
\begin{pgfscope}%
\pgfpathrectangle{\pgfqpoint{0.589510in}{0.417642in}}{\pgfqpoint{3.437062in}{2.055000in}}%
\pgfusepath{clip}%
\pgfsetrectcap%
\pgfsetroundjoin%
\pgfsetlinewidth{0.803000pt}%
\definecolor{currentstroke}{rgb}{0.450000,0.450000,0.450000}%
\pgfsetstrokecolor{currentstroke}%
\pgfsetdash{}{0pt}%
\pgfpathmoveto{\pgfqpoint{3.089191in}{0.417642in}}%
\pgfpathlineto{\pgfqpoint{3.089191in}{2.472642in}}%
\pgfusepath{stroke}%
\end{pgfscope}%
\begin{pgfscope}%
\pgfsetbuttcap%
\pgfsetroundjoin%
\definecolor{currentfill}{rgb}{0.000000,0.000000,0.000000}%
\pgfsetfillcolor{currentfill}%
\pgfsetlinewidth{0.803000pt}%
\definecolor{currentstroke}{rgb}{0.000000,0.000000,0.000000}%
\pgfsetstrokecolor{currentstroke}%
\pgfsetdash{}{0pt}%
\pgfsys@defobject{currentmarker}{\pgfqpoint{0.000000in}{-0.048611in}}{\pgfqpoint{0.000000in}{0.000000in}}{%
\pgfpathmoveto{\pgfqpoint{0.000000in}{0.000000in}}%
\pgfpathlineto{\pgfqpoint{0.000000in}{-0.048611in}}%
\pgfusepath{stroke,fill}%
}%
\begin{pgfscope}%
\pgfsys@transformshift{3.089191in}{0.417642in}%
\pgfsys@useobject{currentmarker}{}%
\end{pgfscope}%
\end{pgfscope}%
\begin{pgfscope}%
\definecolor{textcolor}{rgb}{0.000000,0.000000,0.000000}%
\pgfsetstrokecolor{textcolor}%
\pgfsetfillcolor{textcolor}%
\pgftext[x=3.089191in,y=0.320420in,,top]{\color{textcolor}\rmfamily\fontsize{8.000000}{9.600000}\selectfont \(\displaystyle {10^{1}}\)}%
\end{pgfscope}%
\begin{pgfscope}%
\pgfpathrectangle{\pgfqpoint{0.589510in}{0.417642in}}{\pgfqpoint{3.437062in}{2.055000in}}%
\pgfusepath{clip}%
\pgfsetrectcap%
\pgfsetroundjoin%
\pgfsetlinewidth{0.803000pt}%
\definecolor{currentstroke}{rgb}{0.450000,0.450000,0.450000}%
\pgfsetstrokecolor{currentstroke}%
\pgfsetdash{}{0pt}%
\pgfpathmoveto{\pgfqpoint{3.870342in}{0.417642in}}%
\pgfpathlineto{\pgfqpoint{3.870342in}{2.472642in}}%
\pgfusepath{stroke}%
\end{pgfscope}%
\begin{pgfscope}%
\pgfsetbuttcap%
\pgfsetroundjoin%
\definecolor{currentfill}{rgb}{0.000000,0.000000,0.000000}%
\pgfsetfillcolor{currentfill}%
\pgfsetlinewidth{0.803000pt}%
\definecolor{currentstroke}{rgb}{0.000000,0.000000,0.000000}%
\pgfsetstrokecolor{currentstroke}%
\pgfsetdash{}{0pt}%
\pgfsys@defobject{currentmarker}{\pgfqpoint{0.000000in}{-0.048611in}}{\pgfqpoint{0.000000in}{0.000000in}}{%
\pgfpathmoveto{\pgfqpoint{0.000000in}{0.000000in}}%
\pgfpathlineto{\pgfqpoint{0.000000in}{-0.048611in}}%
\pgfusepath{stroke,fill}%
}%
\begin{pgfscope}%
\pgfsys@transformshift{3.870342in}{0.417642in}%
\pgfsys@useobject{currentmarker}{}%
\end{pgfscope}%
\end{pgfscope}%
\begin{pgfscope}%
\definecolor{textcolor}{rgb}{0.000000,0.000000,0.000000}%
\pgfsetstrokecolor{textcolor}%
\pgfsetfillcolor{textcolor}%
\pgftext[x=3.870342in,y=0.320420in,,top]{\color{textcolor}\rmfamily\fontsize{8.000000}{9.600000}\selectfont \(\displaystyle {10^{2}}\)}%
\end{pgfscope}%
\begin{pgfscope}%
\pgfpathrectangle{\pgfqpoint{0.589510in}{0.417642in}}{\pgfqpoint{3.437062in}{2.055000in}}%
\pgfusepath{clip}%
\pgfsetrectcap%
\pgfsetroundjoin%
\pgfsetlinewidth{0.803000pt}%
\definecolor{currentstroke}{rgb}{0.850000,0.850000,0.850000}%
\pgfsetstrokecolor{currentstroke}%
\pgfsetdash{}{0pt}%
\pgfpathmoveto{\pgfqpoint{0.624738in}{0.417642in}}%
\pgfpathlineto{\pgfqpoint{0.624738in}{2.472642in}}%
\pgfusepath{stroke}%
\end{pgfscope}%
\begin{pgfscope}%
\pgfsetbuttcap%
\pgfsetroundjoin%
\definecolor{currentfill}{rgb}{0.000000,0.000000,0.000000}%
\pgfsetfillcolor{currentfill}%
\pgfsetlinewidth{0.602250pt}%
\definecolor{currentstroke}{rgb}{0.000000,0.000000,0.000000}%
\pgfsetstrokecolor{currentstroke}%
\pgfsetdash{}{0pt}%
\pgfsys@defobject{currentmarker}{\pgfqpoint{0.000000in}{-0.027778in}}{\pgfqpoint{0.000000in}{0.000000in}}{%
\pgfpathmoveto{\pgfqpoint{0.000000in}{0.000000in}}%
\pgfpathlineto{\pgfqpoint{0.000000in}{-0.027778in}}%
\pgfusepath{stroke,fill}%
}%
\begin{pgfscope}%
\pgfsys@transformshift{0.624738in}{0.417642in}%
\pgfsys@useobject{currentmarker}{}%
\end{pgfscope}%
\end{pgfscope}%
\begin{pgfscope}%
\pgfpathrectangle{\pgfqpoint{0.589510in}{0.417642in}}{\pgfqpoint{3.437062in}{2.055000in}}%
\pgfusepath{clip}%
\pgfsetrectcap%
\pgfsetroundjoin%
\pgfsetlinewidth{0.803000pt}%
\definecolor{currentstroke}{rgb}{0.850000,0.850000,0.850000}%
\pgfsetstrokecolor{currentstroke}%
\pgfsetdash{}{0pt}%
\pgfpathmoveto{\pgfqpoint{0.670039in}{0.417642in}}%
\pgfpathlineto{\pgfqpoint{0.670039in}{2.472642in}}%
\pgfusepath{stroke}%
\end{pgfscope}%
\begin{pgfscope}%
\pgfsetbuttcap%
\pgfsetroundjoin%
\definecolor{currentfill}{rgb}{0.000000,0.000000,0.000000}%
\pgfsetfillcolor{currentfill}%
\pgfsetlinewidth{0.602250pt}%
\definecolor{currentstroke}{rgb}{0.000000,0.000000,0.000000}%
\pgfsetstrokecolor{currentstroke}%
\pgfsetdash{}{0pt}%
\pgfsys@defobject{currentmarker}{\pgfqpoint{0.000000in}{-0.027778in}}{\pgfqpoint{0.000000in}{0.000000in}}{%
\pgfpathmoveto{\pgfqpoint{0.000000in}{0.000000in}}%
\pgfpathlineto{\pgfqpoint{0.000000in}{-0.027778in}}%
\pgfusepath{stroke,fill}%
}%
\begin{pgfscope}%
\pgfsys@transformshift{0.670039in}{0.417642in}%
\pgfsys@useobject{currentmarker}{}%
\end{pgfscope}%
\end{pgfscope}%
\begin{pgfscope}%
\pgfpathrectangle{\pgfqpoint{0.589510in}{0.417642in}}{\pgfqpoint{3.437062in}{2.055000in}}%
\pgfusepath{clip}%
\pgfsetrectcap%
\pgfsetroundjoin%
\pgfsetlinewidth{0.803000pt}%
\definecolor{currentstroke}{rgb}{0.850000,0.850000,0.850000}%
\pgfsetstrokecolor{currentstroke}%
\pgfsetdash{}{0pt}%
\pgfpathmoveto{\pgfqpoint{0.709996in}{0.417642in}}%
\pgfpathlineto{\pgfqpoint{0.709996in}{2.472642in}}%
\pgfusepath{stroke}%
\end{pgfscope}%
\begin{pgfscope}%
\pgfsetbuttcap%
\pgfsetroundjoin%
\definecolor{currentfill}{rgb}{0.000000,0.000000,0.000000}%
\pgfsetfillcolor{currentfill}%
\pgfsetlinewidth{0.602250pt}%
\definecolor{currentstroke}{rgb}{0.000000,0.000000,0.000000}%
\pgfsetstrokecolor{currentstroke}%
\pgfsetdash{}{0pt}%
\pgfsys@defobject{currentmarker}{\pgfqpoint{0.000000in}{-0.027778in}}{\pgfqpoint{0.000000in}{0.000000in}}{%
\pgfpathmoveto{\pgfqpoint{0.000000in}{0.000000in}}%
\pgfpathlineto{\pgfqpoint{0.000000in}{-0.027778in}}%
\pgfusepath{stroke,fill}%
}%
\begin{pgfscope}%
\pgfsys@transformshift{0.709996in}{0.417642in}%
\pgfsys@useobject{currentmarker}{}%
\end{pgfscope}%
\end{pgfscope}%
\begin{pgfscope}%
\pgfpathrectangle{\pgfqpoint{0.589510in}{0.417642in}}{\pgfqpoint{3.437062in}{2.055000in}}%
\pgfusepath{clip}%
\pgfsetrectcap%
\pgfsetroundjoin%
\pgfsetlinewidth{0.803000pt}%
\definecolor{currentstroke}{rgb}{0.850000,0.850000,0.850000}%
\pgfsetstrokecolor{currentstroke}%
\pgfsetdash{}{0pt}%
\pgfpathmoveto{\pgfqpoint{0.980890in}{0.417642in}}%
\pgfpathlineto{\pgfqpoint{0.980890in}{2.472642in}}%
\pgfusepath{stroke}%
\end{pgfscope}%
\begin{pgfscope}%
\pgfsetbuttcap%
\pgfsetroundjoin%
\definecolor{currentfill}{rgb}{0.000000,0.000000,0.000000}%
\pgfsetfillcolor{currentfill}%
\pgfsetlinewidth{0.602250pt}%
\definecolor{currentstroke}{rgb}{0.000000,0.000000,0.000000}%
\pgfsetstrokecolor{currentstroke}%
\pgfsetdash{}{0pt}%
\pgfsys@defobject{currentmarker}{\pgfqpoint{0.000000in}{-0.027778in}}{\pgfqpoint{0.000000in}{0.000000in}}{%
\pgfpathmoveto{\pgfqpoint{0.000000in}{0.000000in}}%
\pgfpathlineto{\pgfqpoint{0.000000in}{-0.027778in}}%
\pgfusepath{stroke,fill}%
}%
\begin{pgfscope}%
\pgfsys@transformshift{0.980890in}{0.417642in}%
\pgfsys@useobject{currentmarker}{}%
\end{pgfscope}%
\end{pgfscope}%
\begin{pgfscope}%
\pgfpathrectangle{\pgfqpoint{0.589510in}{0.417642in}}{\pgfqpoint{3.437062in}{2.055000in}}%
\pgfusepath{clip}%
\pgfsetrectcap%
\pgfsetroundjoin%
\pgfsetlinewidth{0.803000pt}%
\definecolor{currentstroke}{rgb}{0.850000,0.850000,0.850000}%
\pgfsetstrokecolor{currentstroke}%
\pgfsetdash{}{0pt}%
\pgfpathmoveto{\pgfqpoint{1.118443in}{0.417642in}}%
\pgfpathlineto{\pgfqpoint{1.118443in}{2.472642in}}%
\pgfusepath{stroke}%
\end{pgfscope}%
\begin{pgfscope}%
\pgfsetbuttcap%
\pgfsetroundjoin%
\definecolor{currentfill}{rgb}{0.000000,0.000000,0.000000}%
\pgfsetfillcolor{currentfill}%
\pgfsetlinewidth{0.602250pt}%
\definecolor{currentstroke}{rgb}{0.000000,0.000000,0.000000}%
\pgfsetstrokecolor{currentstroke}%
\pgfsetdash{}{0pt}%
\pgfsys@defobject{currentmarker}{\pgfqpoint{0.000000in}{-0.027778in}}{\pgfqpoint{0.000000in}{0.000000in}}{%
\pgfpathmoveto{\pgfqpoint{0.000000in}{0.000000in}}%
\pgfpathlineto{\pgfqpoint{0.000000in}{-0.027778in}}%
\pgfusepath{stroke,fill}%
}%
\begin{pgfscope}%
\pgfsys@transformshift{1.118443in}{0.417642in}%
\pgfsys@useobject{currentmarker}{}%
\end{pgfscope}%
\end{pgfscope}%
\begin{pgfscope}%
\pgfpathrectangle{\pgfqpoint{0.589510in}{0.417642in}}{\pgfqpoint{3.437062in}{2.055000in}}%
\pgfusepath{clip}%
\pgfsetrectcap%
\pgfsetroundjoin%
\pgfsetlinewidth{0.803000pt}%
\definecolor{currentstroke}{rgb}{0.850000,0.850000,0.850000}%
\pgfsetstrokecolor{currentstroke}%
\pgfsetdash{}{0pt}%
\pgfpathmoveto{\pgfqpoint{1.216039in}{0.417642in}}%
\pgfpathlineto{\pgfqpoint{1.216039in}{2.472642in}}%
\pgfusepath{stroke}%
\end{pgfscope}%
\begin{pgfscope}%
\pgfsetbuttcap%
\pgfsetroundjoin%
\definecolor{currentfill}{rgb}{0.000000,0.000000,0.000000}%
\pgfsetfillcolor{currentfill}%
\pgfsetlinewidth{0.602250pt}%
\definecolor{currentstroke}{rgb}{0.000000,0.000000,0.000000}%
\pgfsetstrokecolor{currentstroke}%
\pgfsetdash{}{0pt}%
\pgfsys@defobject{currentmarker}{\pgfqpoint{0.000000in}{-0.027778in}}{\pgfqpoint{0.000000in}{0.000000in}}{%
\pgfpathmoveto{\pgfqpoint{0.000000in}{0.000000in}}%
\pgfpathlineto{\pgfqpoint{0.000000in}{-0.027778in}}%
\pgfusepath{stroke,fill}%
}%
\begin{pgfscope}%
\pgfsys@transformshift{1.216039in}{0.417642in}%
\pgfsys@useobject{currentmarker}{}%
\end{pgfscope}%
\end{pgfscope}%
\begin{pgfscope}%
\pgfpathrectangle{\pgfqpoint{0.589510in}{0.417642in}}{\pgfqpoint{3.437062in}{2.055000in}}%
\pgfusepath{clip}%
\pgfsetrectcap%
\pgfsetroundjoin%
\pgfsetlinewidth{0.803000pt}%
\definecolor{currentstroke}{rgb}{0.850000,0.850000,0.850000}%
\pgfsetstrokecolor{currentstroke}%
\pgfsetdash{}{0pt}%
\pgfpathmoveto{\pgfqpoint{1.291741in}{0.417642in}}%
\pgfpathlineto{\pgfqpoint{1.291741in}{2.472642in}}%
\pgfusepath{stroke}%
\end{pgfscope}%
\begin{pgfscope}%
\pgfsetbuttcap%
\pgfsetroundjoin%
\definecolor{currentfill}{rgb}{0.000000,0.000000,0.000000}%
\pgfsetfillcolor{currentfill}%
\pgfsetlinewidth{0.602250pt}%
\definecolor{currentstroke}{rgb}{0.000000,0.000000,0.000000}%
\pgfsetstrokecolor{currentstroke}%
\pgfsetdash{}{0pt}%
\pgfsys@defobject{currentmarker}{\pgfqpoint{0.000000in}{-0.027778in}}{\pgfqpoint{0.000000in}{0.000000in}}{%
\pgfpathmoveto{\pgfqpoint{0.000000in}{0.000000in}}%
\pgfpathlineto{\pgfqpoint{0.000000in}{-0.027778in}}%
\pgfusepath{stroke,fill}%
}%
\begin{pgfscope}%
\pgfsys@transformshift{1.291741in}{0.417642in}%
\pgfsys@useobject{currentmarker}{}%
\end{pgfscope}%
\end{pgfscope}%
\begin{pgfscope}%
\pgfpathrectangle{\pgfqpoint{0.589510in}{0.417642in}}{\pgfqpoint{3.437062in}{2.055000in}}%
\pgfusepath{clip}%
\pgfsetrectcap%
\pgfsetroundjoin%
\pgfsetlinewidth{0.803000pt}%
\definecolor{currentstroke}{rgb}{0.850000,0.850000,0.850000}%
\pgfsetstrokecolor{currentstroke}%
\pgfsetdash{}{0pt}%
\pgfpathmoveto{\pgfqpoint{1.353593in}{0.417642in}}%
\pgfpathlineto{\pgfqpoint{1.353593in}{2.472642in}}%
\pgfusepath{stroke}%
\end{pgfscope}%
\begin{pgfscope}%
\pgfsetbuttcap%
\pgfsetroundjoin%
\definecolor{currentfill}{rgb}{0.000000,0.000000,0.000000}%
\pgfsetfillcolor{currentfill}%
\pgfsetlinewidth{0.602250pt}%
\definecolor{currentstroke}{rgb}{0.000000,0.000000,0.000000}%
\pgfsetstrokecolor{currentstroke}%
\pgfsetdash{}{0pt}%
\pgfsys@defobject{currentmarker}{\pgfqpoint{0.000000in}{-0.027778in}}{\pgfqpoint{0.000000in}{0.000000in}}{%
\pgfpathmoveto{\pgfqpoint{0.000000in}{0.000000in}}%
\pgfpathlineto{\pgfqpoint{0.000000in}{-0.027778in}}%
\pgfusepath{stroke,fill}%
}%
\begin{pgfscope}%
\pgfsys@transformshift{1.353593in}{0.417642in}%
\pgfsys@useobject{currentmarker}{}%
\end{pgfscope}%
\end{pgfscope}%
\begin{pgfscope}%
\pgfpathrectangle{\pgfqpoint{0.589510in}{0.417642in}}{\pgfqpoint{3.437062in}{2.055000in}}%
\pgfusepath{clip}%
\pgfsetrectcap%
\pgfsetroundjoin%
\pgfsetlinewidth{0.803000pt}%
\definecolor{currentstroke}{rgb}{0.850000,0.850000,0.850000}%
\pgfsetstrokecolor{currentstroke}%
\pgfsetdash{}{0pt}%
\pgfpathmoveto{\pgfqpoint{1.405889in}{0.417642in}}%
\pgfpathlineto{\pgfqpoint{1.405889in}{2.472642in}}%
\pgfusepath{stroke}%
\end{pgfscope}%
\begin{pgfscope}%
\pgfsetbuttcap%
\pgfsetroundjoin%
\definecolor{currentfill}{rgb}{0.000000,0.000000,0.000000}%
\pgfsetfillcolor{currentfill}%
\pgfsetlinewidth{0.602250pt}%
\definecolor{currentstroke}{rgb}{0.000000,0.000000,0.000000}%
\pgfsetstrokecolor{currentstroke}%
\pgfsetdash{}{0pt}%
\pgfsys@defobject{currentmarker}{\pgfqpoint{0.000000in}{-0.027778in}}{\pgfqpoint{0.000000in}{0.000000in}}{%
\pgfpathmoveto{\pgfqpoint{0.000000in}{0.000000in}}%
\pgfpathlineto{\pgfqpoint{0.000000in}{-0.027778in}}%
\pgfusepath{stroke,fill}%
}%
\begin{pgfscope}%
\pgfsys@transformshift{1.405889in}{0.417642in}%
\pgfsys@useobject{currentmarker}{}%
\end{pgfscope}%
\end{pgfscope}%
\begin{pgfscope}%
\pgfpathrectangle{\pgfqpoint{0.589510in}{0.417642in}}{\pgfqpoint{3.437062in}{2.055000in}}%
\pgfusepath{clip}%
\pgfsetrectcap%
\pgfsetroundjoin%
\pgfsetlinewidth{0.803000pt}%
\definecolor{currentstroke}{rgb}{0.850000,0.850000,0.850000}%
\pgfsetstrokecolor{currentstroke}%
\pgfsetdash{}{0pt}%
\pgfpathmoveto{\pgfqpoint{1.451189in}{0.417642in}}%
\pgfpathlineto{\pgfqpoint{1.451189in}{2.472642in}}%
\pgfusepath{stroke}%
\end{pgfscope}%
\begin{pgfscope}%
\pgfsetbuttcap%
\pgfsetroundjoin%
\definecolor{currentfill}{rgb}{0.000000,0.000000,0.000000}%
\pgfsetfillcolor{currentfill}%
\pgfsetlinewidth{0.602250pt}%
\definecolor{currentstroke}{rgb}{0.000000,0.000000,0.000000}%
\pgfsetstrokecolor{currentstroke}%
\pgfsetdash{}{0pt}%
\pgfsys@defobject{currentmarker}{\pgfqpoint{0.000000in}{-0.027778in}}{\pgfqpoint{0.000000in}{0.000000in}}{%
\pgfpathmoveto{\pgfqpoint{0.000000in}{0.000000in}}%
\pgfpathlineto{\pgfqpoint{0.000000in}{-0.027778in}}%
\pgfusepath{stroke,fill}%
}%
\begin{pgfscope}%
\pgfsys@transformshift{1.451189in}{0.417642in}%
\pgfsys@useobject{currentmarker}{}%
\end{pgfscope}%
\end{pgfscope}%
\begin{pgfscope}%
\pgfpathrectangle{\pgfqpoint{0.589510in}{0.417642in}}{\pgfqpoint{3.437062in}{2.055000in}}%
\pgfusepath{clip}%
\pgfsetrectcap%
\pgfsetroundjoin%
\pgfsetlinewidth{0.803000pt}%
\definecolor{currentstroke}{rgb}{0.850000,0.850000,0.850000}%
\pgfsetstrokecolor{currentstroke}%
\pgfsetdash{}{0pt}%
\pgfpathmoveto{\pgfqpoint{1.491147in}{0.417642in}}%
\pgfpathlineto{\pgfqpoint{1.491147in}{2.472642in}}%
\pgfusepath{stroke}%
\end{pgfscope}%
\begin{pgfscope}%
\pgfsetbuttcap%
\pgfsetroundjoin%
\definecolor{currentfill}{rgb}{0.000000,0.000000,0.000000}%
\pgfsetfillcolor{currentfill}%
\pgfsetlinewidth{0.602250pt}%
\definecolor{currentstroke}{rgb}{0.000000,0.000000,0.000000}%
\pgfsetstrokecolor{currentstroke}%
\pgfsetdash{}{0pt}%
\pgfsys@defobject{currentmarker}{\pgfqpoint{0.000000in}{-0.027778in}}{\pgfqpoint{0.000000in}{0.000000in}}{%
\pgfpathmoveto{\pgfqpoint{0.000000in}{0.000000in}}%
\pgfpathlineto{\pgfqpoint{0.000000in}{-0.027778in}}%
\pgfusepath{stroke,fill}%
}%
\begin{pgfscope}%
\pgfsys@transformshift{1.491147in}{0.417642in}%
\pgfsys@useobject{currentmarker}{}%
\end{pgfscope}%
\end{pgfscope}%
\begin{pgfscope}%
\pgfpathrectangle{\pgfqpoint{0.589510in}{0.417642in}}{\pgfqpoint{3.437062in}{2.055000in}}%
\pgfusepath{clip}%
\pgfsetrectcap%
\pgfsetroundjoin%
\pgfsetlinewidth{0.803000pt}%
\definecolor{currentstroke}{rgb}{0.850000,0.850000,0.850000}%
\pgfsetstrokecolor{currentstroke}%
\pgfsetdash{}{0pt}%
\pgfpathmoveto{\pgfqpoint{1.762040in}{0.417642in}}%
\pgfpathlineto{\pgfqpoint{1.762040in}{2.472642in}}%
\pgfusepath{stroke}%
\end{pgfscope}%
\begin{pgfscope}%
\pgfsetbuttcap%
\pgfsetroundjoin%
\definecolor{currentfill}{rgb}{0.000000,0.000000,0.000000}%
\pgfsetfillcolor{currentfill}%
\pgfsetlinewidth{0.602250pt}%
\definecolor{currentstroke}{rgb}{0.000000,0.000000,0.000000}%
\pgfsetstrokecolor{currentstroke}%
\pgfsetdash{}{0pt}%
\pgfsys@defobject{currentmarker}{\pgfqpoint{0.000000in}{-0.027778in}}{\pgfqpoint{0.000000in}{0.000000in}}{%
\pgfpathmoveto{\pgfqpoint{0.000000in}{0.000000in}}%
\pgfpathlineto{\pgfqpoint{0.000000in}{-0.027778in}}%
\pgfusepath{stroke,fill}%
}%
\begin{pgfscope}%
\pgfsys@transformshift{1.762040in}{0.417642in}%
\pgfsys@useobject{currentmarker}{}%
\end{pgfscope}%
\end{pgfscope}%
\begin{pgfscope}%
\pgfpathrectangle{\pgfqpoint{0.589510in}{0.417642in}}{\pgfqpoint{3.437062in}{2.055000in}}%
\pgfusepath{clip}%
\pgfsetrectcap%
\pgfsetroundjoin%
\pgfsetlinewidth{0.803000pt}%
\definecolor{currentstroke}{rgb}{0.850000,0.850000,0.850000}%
\pgfsetstrokecolor{currentstroke}%
\pgfsetdash{}{0pt}%
\pgfpathmoveto{\pgfqpoint{1.899594in}{0.417642in}}%
\pgfpathlineto{\pgfqpoint{1.899594in}{2.472642in}}%
\pgfusepath{stroke}%
\end{pgfscope}%
\begin{pgfscope}%
\pgfsetbuttcap%
\pgfsetroundjoin%
\definecolor{currentfill}{rgb}{0.000000,0.000000,0.000000}%
\pgfsetfillcolor{currentfill}%
\pgfsetlinewidth{0.602250pt}%
\definecolor{currentstroke}{rgb}{0.000000,0.000000,0.000000}%
\pgfsetstrokecolor{currentstroke}%
\pgfsetdash{}{0pt}%
\pgfsys@defobject{currentmarker}{\pgfqpoint{0.000000in}{-0.027778in}}{\pgfqpoint{0.000000in}{0.000000in}}{%
\pgfpathmoveto{\pgfqpoint{0.000000in}{0.000000in}}%
\pgfpathlineto{\pgfqpoint{0.000000in}{-0.027778in}}%
\pgfusepath{stroke,fill}%
}%
\begin{pgfscope}%
\pgfsys@transformshift{1.899594in}{0.417642in}%
\pgfsys@useobject{currentmarker}{}%
\end{pgfscope}%
\end{pgfscope}%
\begin{pgfscope}%
\pgfpathrectangle{\pgfqpoint{0.589510in}{0.417642in}}{\pgfqpoint{3.437062in}{2.055000in}}%
\pgfusepath{clip}%
\pgfsetrectcap%
\pgfsetroundjoin%
\pgfsetlinewidth{0.803000pt}%
\definecolor{currentstroke}{rgb}{0.850000,0.850000,0.850000}%
\pgfsetstrokecolor{currentstroke}%
\pgfsetdash{}{0pt}%
\pgfpathmoveto{\pgfqpoint{1.997190in}{0.417642in}}%
\pgfpathlineto{\pgfqpoint{1.997190in}{2.472642in}}%
\pgfusepath{stroke}%
\end{pgfscope}%
\begin{pgfscope}%
\pgfsetbuttcap%
\pgfsetroundjoin%
\definecolor{currentfill}{rgb}{0.000000,0.000000,0.000000}%
\pgfsetfillcolor{currentfill}%
\pgfsetlinewidth{0.602250pt}%
\definecolor{currentstroke}{rgb}{0.000000,0.000000,0.000000}%
\pgfsetstrokecolor{currentstroke}%
\pgfsetdash{}{0pt}%
\pgfsys@defobject{currentmarker}{\pgfqpoint{0.000000in}{-0.027778in}}{\pgfqpoint{0.000000in}{0.000000in}}{%
\pgfpathmoveto{\pgfqpoint{0.000000in}{0.000000in}}%
\pgfpathlineto{\pgfqpoint{0.000000in}{-0.027778in}}%
\pgfusepath{stroke,fill}%
}%
\begin{pgfscope}%
\pgfsys@transformshift{1.997190in}{0.417642in}%
\pgfsys@useobject{currentmarker}{}%
\end{pgfscope}%
\end{pgfscope}%
\begin{pgfscope}%
\pgfpathrectangle{\pgfqpoint{0.589510in}{0.417642in}}{\pgfqpoint{3.437062in}{2.055000in}}%
\pgfusepath{clip}%
\pgfsetrectcap%
\pgfsetroundjoin%
\pgfsetlinewidth{0.803000pt}%
\definecolor{currentstroke}{rgb}{0.850000,0.850000,0.850000}%
\pgfsetstrokecolor{currentstroke}%
\pgfsetdash{}{0pt}%
\pgfpathmoveto{\pgfqpoint{2.072891in}{0.417642in}}%
\pgfpathlineto{\pgfqpoint{2.072891in}{2.472642in}}%
\pgfusepath{stroke}%
\end{pgfscope}%
\begin{pgfscope}%
\pgfsetbuttcap%
\pgfsetroundjoin%
\definecolor{currentfill}{rgb}{0.000000,0.000000,0.000000}%
\pgfsetfillcolor{currentfill}%
\pgfsetlinewidth{0.602250pt}%
\definecolor{currentstroke}{rgb}{0.000000,0.000000,0.000000}%
\pgfsetstrokecolor{currentstroke}%
\pgfsetdash{}{0pt}%
\pgfsys@defobject{currentmarker}{\pgfqpoint{0.000000in}{-0.027778in}}{\pgfqpoint{0.000000in}{0.000000in}}{%
\pgfpathmoveto{\pgfqpoint{0.000000in}{0.000000in}}%
\pgfpathlineto{\pgfqpoint{0.000000in}{-0.027778in}}%
\pgfusepath{stroke,fill}%
}%
\begin{pgfscope}%
\pgfsys@transformshift{2.072891in}{0.417642in}%
\pgfsys@useobject{currentmarker}{}%
\end{pgfscope}%
\end{pgfscope}%
\begin{pgfscope}%
\pgfpathrectangle{\pgfqpoint{0.589510in}{0.417642in}}{\pgfqpoint{3.437062in}{2.055000in}}%
\pgfusepath{clip}%
\pgfsetrectcap%
\pgfsetroundjoin%
\pgfsetlinewidth{0.803000pt}%
\definecolor{currentstroke}{rgb}{0.850000,0.850000,0.850000}%
\pgfsetstrokecolor{currentstroke}%
\pgfsetdash{}{0pt}%
\pgfpathmoveto{\pgfqpoint{2.134743in}{0.417642in}}%
\pgfpathlineto{\pgfqpoint{2.134743in}{2.472642in}}%
\pgfusepath{stroke}%
\end{pgfscope}%
\begin{pgfscope}%
\pgfsetbuttcap%
\pgfsetroundjoin%
\definecolor{currentfill}{rgb}{0.000000,0.000000,0.000000}%
\pgfsetfillcolor{currentfill}%
\pgfsetlinewidth{0.602250pt}%
\definecolor{currentstroke}{rgb}{0.000000,0.000000,0.000000}%
\pgfsetstrokecolor{currentstroke}%
\pgfsetdash{}{0pt}%
\pgfsys@defobject{currentmarker}{\pgfqpoint{0.000000in}{-0.027778in}}{\pgfqpoint{0.000000in}{0.000000in}}{%
\pgfpathmoveto{\pgfqpoint{0.000000in}{0.000000in}}%
\pgfpathlineto{\pgfqpoint{0.000000in}{-0.027778in}}%
\pgfusepath{stroke,fill}%
}%
\begin{pgfscope}%
\pgfsys@transformshift{2.134743in}{0.417642in}%
\pgfsys@useobject{currentmarker}{}%
\end{pgfscope}%
\end{pgfscope}%
\begin{pgfscope}%
\pgfpathrectangle{\pgfqpoint{0.589510in}{0.417642in}}{\pgfqpoint{3.437062in}{2.055000in}}%
\pgfusepath{clip}%
\pgfsetrectcap%
\pgfsetroundjoin%
\pgfsetlinewidth{0.803000pt}%
\definecolor{currentstroke}{rgb}{0.850000,0.850000,0.850000}%
\pgfsetstrokecolor{currentstroke}%
\pgfsetdash{}{0pt}%
\pgfpathmoveto{\pgfqpoint{2.187039in}{0.417642in}}%
\pgfpathlineto{\pgfqpoint{2.187039in}{2.472642in}}%
\pgfusepath{stroke}%
\end{pgfscope}%
\begin{pgfscope}%
\pgfsetbuttcap%
\pgfsetroundjoin%
\definecolor{currentfill}{rgb}{0.000000,0.000000,0.000000}%
\pgfsetfillcolor{currentfill}%
\pgfsetlinewidth{0.602250pt}%
\definecolor{currentstroke}{rgb}{0.000000,0.000000,0.000000}%
\pgfsetstrokecolor{currentstroke}%
\pgfsetdash{}{0pt}%
\pgfsys@defobject{currentmarker}{\pgfqpoint{0.000000in}{-0.027778in}}{\pgfqpoint{0.000000in}{0.000000in}}{%
\pgfpathmoveto{\pgfqpoint{0.000000in}{0.000000in}}%
\pgfpathlineto{\pgfqpoint{0.000000in}{-0.027778in}}%
\pgfusepath{stroke,fill}%
}%
\begin{pgfscope}%
\pgfsys@transformshift{2.187039in}{0.417642in}%
\pgfsys@useobject{currentmarker}{}%
\end{pgfscope}%
\end{pgfscope}%
\begin{pgfscope}%
\pgfpathrectangle{\pgfqpoint{0.589510in}{0.417642in}}{\pgfqpoint{3.437062in}{2.055000in}}%
\pgfusepath{clip}%
\pgfsetrectcap%
\pgfsetroundjoin%
\pgfsetlinewidth{0.803000pt}%
\definecolor{currentstroke}{rgb}{0.850000,0.850000,0.850000}%
\pgfsetstrokecolor{currentstroke}%
\pgfsetdash{}{0pt}%
\pgfpathmoveto{\pgfqpoint{2.232339in}{0.417642in}}%
\pgfpathlineto{\pgfqpoint{2.232339in}{2.472642in}}%
\pgfusepath{stroke}%
\end{pgfscope}%
\begin{pgfscope}%
\pgfsetbuttcap%
\pgfsetroundjoin%
\definecolor{currentfill}{rgb}{0.000000,0.000000,0.000000}%
\pgfsetfillcolor{currentfill}%
\pgfsetlinewidth{0.602250pt}%
\definecolor{currentstroke}{rgb}{0.000000,0.000000,0.000000}%
\pgfsetstrokecolor{currentstroke}%
\pgfsetdash{}{0pt}%
\pgfsys@defobject{currentmarker}{\pgfqpoint{0.000000in}{-0.027778in}}{\pgfqpoint{0.000000in}{0.000000in}}{%
\pgfpathmoveto{\pgfqpoint{0.000000in}{0.000000in}}%
\pgfpathlineto{\pgfqpoint{0.000000in}{-0.027778in}}%
\pgfusepath{stroke,fill}%
}%
\begin{pgfscope}%
\pgfsys@transformshift{2.232339in}{0.417642in}%
\pgfsys@useobject{currentmarker}{}%
\end{pgfscope}%
\end{pgfscope}%
\begin{pgfscope}%
\pgfpathrectangle{\pgfqpoint{0.589510in}{0.417642in}}{\pgfqpoint{3.437062in}{2.055000in}}%
\pgfusepath{clip}%
\pgfsetrectcap%
\pgfsetroundjoin%
\pgfsetlinewidth{0.803000pt}%
\definecolor{currentstroke}{rgb}{0.850000,0.850000,0.850000}%
\pgfsetstrokecolor{currentstroke}%
\pgfsetdash{}{0pt}%
\pgfpathmoveto{\pgfqpoint{2.272297in}{0.417642in}}%
\pgfpathlineto{\pgfqpoint{2.272297in}{2.472642in}}%
\pgfusepath{stroke}%
\end{pgfscope}%
\begin{pgfscope}%
\pgfsetbuttcap%
\pgfsetroundjoin%
\definecolor{currentfill}{rgb}{0.000000,0.000000,0.000000}%
\pgfsetfillcolor{currentfill}%
\pgfsetlinewidth{0.602250pt}%
\definecolor{currentstroke}{rgb}{0.000000,0.000000,0.000000}%
\pgfsetstrokecolor{currentstroke}%
\pgfsetdash{}{0pt}%
\pgfsys@defobject{currentmarker}{\pgfqpoint{0.000000in}{-0.027778in}}{\pgfqpoint{0.000000in}{0.000000in}}{%
\pgfpathmoveto{\pgfqpoint{0.000000in}{0.000000in}}%
\pgfpathlineto{\pgfqpoint{0.000000in}{-0.027778in}}%
\pgfusepath{stroke,fill}%
}%
\begin{pgfscope}%
\pgfsys@transformshift{2.272297in}{0.417642in}%
\pgfsys@useobject{currentmarker}{}%
\end{pgfscope}%
\end{pgfscope}%
\begin{pgfscope}%
\pgfpathrectangle{\pgfqpoint{0.589510in}{0.417642in}}{\pgfqpoint{3.437062in}{2.055000in}}%
\pgfusepath{clip}%
\pgfsetrectcap%
\pgfsetroundjoin%
\pgfsetlinewidth{0.803000pt}%
\definecolor{currentstroke}{rgb}{0.850000,0.850000,0.850000}%
\pgfsetstrokecolor{currentstroke}%
\pgfsetdash{}{0pt}%
\pgfpathmoveto{\pgfqpoint{2.543190in}{0.417642in}}%
\pgfpathlineto{\pgfqpoint{2.543190in}{2.472642in}}%
\pgfusepath{stroke}%
\end{pgfscope}%
\begin{pgfscope}%
\pgfsetbuttcap%
\pgfsetroundjoin%
\definecolor{currentfill}{rgb}{0.000000,0.000000,0.000000}%
\pgfsetfillcolor{currentfill}%
\pgfsetlinewidth{0.602250pt}%
\definecolor{currentstroke}{rgb}{0.000000,0.000000,0.000000}%
\pgfsetstrokecolor{currentstroke}%
\pgfsetdash{}{0pt}%
\pgfsys@defobject{currentmarker}{\pgfqpoint{0.000000in}{-0.027778in}}{\pgfqpoint{0.000000in}{0.000000in}}{%
\pgfpathmoveto{\pgfqpoint{0.000000in}{0.000000in}}%
\pgfpathlineto{\pgfqpoint{0.000000in}{-0.027778in}}%
\pgfusepath{stroke,fill}%
}%
\begin{pgfscope}%
\pgfsys@transformshift{2.543190in}{0.417642in}%
\pgfsys@useobject{currentmarker}{}%
\end{pgfscope}%
\end{pgfscope}%
\begin{pgfscope}%
\pgfpathrectangle{\pgfqpoint{0.589510in}{0.417642in}}{\pgfqpoint{3.437062in}{2.055000in}}%
\pgfusepath{clip}%
\pgfsetrectcap%
\pgfsetroundjoin%
\pgfsetlinewidth{0.803000pt}%
\definecolor{currentstroke}{rgb}{0.850000,0.850000,0.850000}%
\pgfsetstrokecolor{currentstroke}%
\pgfsetdash{}{0pt}%
\pgfpathmoveto{\pgfqpoint{2.680744in}{0.417642in}}%
\pgfpathlineto{\pgfqpoint{2.680744in}{2.472642in}}%
\pgfusepath{stroke}%
\end{pgfscope}%
\begin{pgfscope}%
\pgfsetbuttcap%
\pgfsetroundjoin%
\definecolor{currentfill}{rgb}{0.000000,0.000000,0.000000}%
\pgfsetfillcolor{currentfill}%
\pgfsetlinewidth{0.602250pt}%
\definecolor{currentstroke}{rgb}{0.000000,0.000000,0.000000}%
\pgfsetstrokecolor{currentstroke}%
\pgfsetdash{}{0pt}%
\pgfsys@defobject{currentmarker}{\pgfqpoint{0.000000in}{-0.027778in}}{\pgfqpoint{0.000000in}{0.000000in}}{%
\pgfpathmoveto{\pgfqpoint{0.000000in}{0.000000in}}%
\pgfpathlineto{\pgfqpoint{0.000000in}{-0.027778in}}%
\pgfusepath{stroke,fill}%
}%
\begin{pgfscope}%
\pgfsys@transformshift{2.680744in}{0.417642in}%
\pgfsys@useobject{currentmarker}{}%
\end{pgfscope}%
\end{pgfscope}%
\begin{pgfscope}%
\pgfpathrectangle{\pgfqpoint{0.589510in}{0.417642in}}{\pgfqpoint{3.437062in}{2.055000in}}%
\pgfusepath{clip}%
\pgfsetrectcap%
\pgfsetroundjoin%
\pgfsetlinewidth{0.803000pt}%
\definecolor{currentstroke}{rgb}{0.850000,0.850000,0.850000}%
\pgfsetstrokecolor{currentstroke}%
\pgfsetdash{}{0pt}%
\pgfpathmoveto{\pgfqpoint{2.778340in}{0.417642in}}%
\pgfpathlineto{\pgfqpoint{2.778340in}{2.472642in}}%
\pgfusepath{stroke}%
\end{pgfscope}%
\begin{pgfscope}%
\pgfsetbuttcap%
\pgfsetroundjoin%
\definecolor{currentfill}{rgb}{0.000000,0.000000,0.000000}%
\pgfsetfillcolor{currentfill}%
\pgfsetlinewidth{0.602250pt}%
\definecolor{currentstroke}{rgb}{0.000000,0.000000,0.000000}%
\pgfsetstrokecolor{currentstroke}%
\pgfsetdash{}{0pt}%
\pgfsys@defobject{currentmarker}{\pgfqpoint{0.000000in}{-0.027778in}}{\pgfqpoint{0.000000in}{0.000000in}}{%
\pgfpathmoveto{\pgfqpoint{0.000000in}{0.000000in}}%
\pgfpathlineto{\pgfqpoint{0.000000in}{-0.027778in}}%
\pgfusepath{stroke,fill}%
}%
\begin{pgfscope}%
\pgfsys@transformshift{2.778340in}{0.417642in}%
\pgfsys@useobject{currentmarker}{}%
\end{pgfscope}%
\end{pgfscope}%
\begin{pgfscope}%
\pgfpathrectangle{\pgfqpoint{0.589510in}{0.417642in}}{\pgfqpoint{3.437062in}{2.055000in}}%
\pgfusepath{clip}%
\pgfsetrectcap%
\pgfsetroundjoin%
\pgfsetlinewidth{0.803000pt}%
\definecolor{currentstroke}{rgb}{0.850000,0.850000,0.850000}%
\pgfsetstrokecolor{currentstroke}%
\pgfsetdash{}{0pt}%
\pgfpathmoveto{\pgfqpoint{2.854041in}{0.417642in}}%
\pgfpathlineto{\pgfqpoint{2.854041in}{2.472642in}}%
\pgfusepath{stroke}%
\end{pgfscope}%
\begin{pgfscope}%
\pgfsetbuttcap%
\pgfsetroundjoin%
\definecolor{currentfill}{rgb}{0.000000,0.000000,0.000000}%
\pgfsetfillcolor{currentfill}%
\pgfsetlinewidth{0.602250pt}%
\definecolor{currentstroke}{rgb}{0.000000,0.000000,0.000000}%
\pgfsetstrokecolor{currentstroke}%
\pgfsetdash{}{0pt}%
\pgfsys@defobject{currentmarker}{\pgfqpoint{0.000000in}{-0.027778in}}{\pgfqpoint{0.000000in}{0.000000in}}{%
\pgfpathmoveto{\pgfqpoint{0.000000in}{0.000000in}}%
\pgfpathlineto{\pgfqpoint{0.000000in}{-0.027778in}}%
\pgfusepath{stroke,fill}%
}%
\begin{pgfscope}%
\pgfsys@transformshift{2.854041in}{0.417642in}%
\pgfsys@useobject{currentmarker}{}%
\end{pgfscope}%
\end{pgfscope}%
\begin{pgfscope}%
\pgfpathrectangle{\pgfqpoint{0.589510in}{0.417642in}}{\pgfqpoint{3.437062in}{2.055000in}}%
\pgfusepath{clip}%
\pgfsetrectcap%
\pgfsetroundjoin%
\pgfsetlinewidth{0.803000pt}%
\definecolor{currentstroke}{rgb}{0.850000,0.850000,0.850000}%
\pgfsetstrokecolor{currentstroke}%
\pgfsetdash{}{0pt}%
\pgfpathmoveto{\pgfqpoint{2.915894in}{0.417642in}}%
\pgfpathlineto{\pgfqpoint{2.915894in}{2.472642in}}%
\pgfusepath{stroke}%
\end{pgfscope}%
\begin{pgfscope}%
\pgfsetbuttcap%
\pgfsetroundjoin%
\definecolor{currentfill}{rgb}{0.000000,0.000000,0.000000}%
\pgfsetfillcolor{currentfill}%
\pgfsetlinewidth{0.602250pt}%
\definecolor{currentstroke}{rgb}{0.000000,0.000000,0.000000}%
\pgfsetstrokecolor{currentstroke}%
\pgfsetdash{}{0pt}%
\pgfsys@defobject{currentmarker}{\pgfqpoint{0.000000in}{-0.027778in}}{\pgfqpoint{0.000000in}{0.000000in}}{%
\pgfpathmoveto{\pgfqpoint{0.000000in}{0.000000in}}%
\pgfpathlineto{\pgfqpoint{0.000000in}{-0.027778in}}%
\pgfusepath{stroke,fill}%
}%
\begin{pgfscope}%
\pgfsys@transformshift{2.915894in}{0.417642in}%
\pgfsys@useobject{currentmarker}{}%
\end{pgfscope}%
\end{pgfscope}%
\begin{pgfscope}%
\pgfpathrectangle{\pgfqpoint{0.589510in}{0.417642in}}{\pgfqpoint{3.437062in}{2.055000in}}%
\pgfusepath{clip}%
\pgfsetrectcap%
\pgfsetroundjoin%
\pgfsetlinewidth{0.803000pt}%
\definecolor{currentstroke}{rgb}{0.850000,0.850000,0.850000}%
\pgfsetstrokecolor{currentstroke}%
\pgfsetdash{}{0pt}%
\pgfpathmoveto{\pgfqpoint{2.968189in}{0.417642in}}%
\pgfpathlineto{\pgfqpoint{2.968189in}{2.472642in}}%
\pgfusepath{stroke}%
\end{pgfscope}%
\begin{pgfscope}%
\pgfsetbuttcap%
\pgfsetroundjoin%
\definecolor{currentfill}{rgb}{0.000000,0.000000,0.000000}%
\pgfsetfillcolor{currentfill}%
\pgfsetlinewidth{0.602250pt}%
\definecolor{currentstroke}{rgb}{0.000000,0.000000,0.000000}%
\pgfsetstrokecolor{currentstroke}%
\pgfsetdash{}{0pt}%
\pgfsys@defobject{currentmarker}{\pgfqpoint{0.000000in}{-0.027778in}}{\pgfqpoint{0.000000in}{0.000000in}}{%
\pgfpathmoveto{\pgfqpoint{0.000000in}{0.000000in}}%
\pgfpathlineto{\pgfqpoint{0.000000in}{-0.027778in}}%
\pgfusepath{stroke,fill}%
}%
\begin{pgfscope}%
\pgfsys@transformshift{2.968189in}{0.417642in}%
\pgfsys@useobject{currentmarker}{}%
\end{pgfscope}%
\end{pgfscope}%
\begin{pgfscope}%
\pgfpathrectangle{\pgfqpoint{0.589510in}{0.417642in}}{\pgfqpoint{3.437062in}{2.055000in}}%
\pgfusepath{clip}%
\pgfsetrectcap%
\pgfsetroundjoin%
\pgfsetlinewidth{0.803000pt}%
\definecolor{currentstroke}{rgb}{0.850000,0.850000,0.850000}%
\pgfsetstrokecolor{currentstroke}%
\pgfsetdash{}{0pt}%
\pgfpathmoveto{\pgfqpoint{3.013490in}{0.417642in}}%
\pgfpathlineto{\pgfqpoint{3.013490in}{2.472642in}}%
\pgfusepath{stroke}%
\end{pgfscope}%
\begin{pgfscope}%
\pgfsetbuttcap%
\pgfsetroundjoin%
\definecolor{currentfill}{rgb}{0.000000,0.000000,0.000000}%
\pgfsetfillcolor{currentfill}%
\pgfsetlinewidth{0.602250pt}%
\definecolor{currentstroke}{rgb}{0.000000,0.000000,0.000000}%
\pgfsetstrokecolor{currentstroke}%
\pgfsetdash{}{0pt}%
\pgfsys@defobject{currentmarker}{\pgfqpoint{0.000000in}{-0.027778in}}{\pgfqpoint{0.000000in}{0.000000in}}{%
\pgfpathmoveto{\pgfqpoint{0.000000in}{0.000000in}}%
\pgfpathlineto{\pgfqpoint{0.000000in}{-0.027778in}}%
\pgfusepath{stroke,fill}%
}%
\begin{pgfscope}%
\pgfsys@transformshift{3.013490in}{0.417642in}%
\pgfsys@useobject{currentmarker}{}%
\end{pgfscope}%
\end{pgfscope}%
\begin{pgfscope}%
\pgfpathrectangle{\pgfqpoint{0.589510in}{0.417642in}}{\pgfqpoint{3.437062in}{2.055000in}}%
\pgfusepath{clip}%
\pgfsetrectcap%
\pgfsetroundjoin%
\pgfsetlinewidth{0.803000pt}%
\definecolor{currentstroke}{rgb}{0.850000,0.850000,0.850000}%
\pgfsetstrokecolor{currentstroke}%
\pgfsetdash{}{0pt}%
\pgfpathmoveto{\pgfqpoint{3.053448in}{0.417642in}}%
\pgfpathlineto{\pgfqpoint{3.053448in}{2.472642in}}%
\pgfusepath{stroke}%
\end{pgfscope}%
\begin{pgfscope}%
\pgfsetbuttcap%
\pgfsetroundjoin%
\definecolor{currentfill}{rgb}{0.000000,0.000000,0.000000}%
\pgfsetfillcolor{currentfill}%
\pgfsetlinewidth{0.602250pt}%
\definecolor{currentstroke}{rgb}{0.000000,0.000000,0.000000}%
\pgfsetstrokecolor{currentstroke}%
\pgfsetdash{}{0pt}%
\pgfsys@defobject{currentmarker}{\pgfqpoint{0.000000in}{-0.027778in}}{\pgfqpoint{0.000000in}{0.000000in}}{%
\pgfpathmoveto{\pgfqpoint{0.000000in}{0.000000in}}%
\pgfpathlineto{\pgfqpoint{0.000000in}{-0.027778in}}%
\pgfusepath{stroke,fill}%
}%
\begin{pgfscope}%
\pgfsys@transformshift{3.053448in}{0.417642in}%
\pgfsys@useobject{currentmarker}{}%
\end{pgfscope}%
\end{pgfscope}%
\begin{pgfscope}%
\pgfpathrectangle{\pgfqpoint{0.589510in}{0.417642in}}{\pgfqpoint{3.437062in}{2.055000in}}%
\pgfusepath{clip}%
\pgfsetrectcap%
\pgfsetroundjoin%
\pgfsetlinewidth{0.803000pt}%
\definecolor{currentstroke}{rgb}{0.850000,0.850000,0.850000}%
\pgfsetstrokecolor{currentstroke}%
\pgfsetdash{}{0pt}%
\pgfpathmoveto{\pgfqpoint{3.324341in}{0.417642in}}%
\pgfpathlineto{\pgfqpoint{3.324341in}{2.472642in}}%
\pgfusepath{stroke}%
\end{pgfscope}%
\begin{pgfscope}%
\pgfsetbuttcap%
\pgfsetroundjoin%
\definecolor{currentfill}{rgb}{0.000000,0.000000,0.000000}%
\pgfsetfillcolor{currentfill}%
\pgfsetlinewidth{0.602250pt}%
\definecolor{currentstroke}{rgb}{0.000000,0.000000,0.000000}%
\pgfsetstrokecolor{currentstroke}%
\pgfsetdash{}{0pt}%
\pgfsys@defobject{currentmarker}{\pgfqpoint{0.000000in}{-0.027778in}}{\pgfqpoint{0.000000in}{0.000000in}}{%
\pgfpathmoveto{\pgfqpoint{0.000000in}{0.000000in}}%
\pgfpathlineto{\pgfqpoint{0.000000in}{-0.027778in}}%
\pgfusepath{stroke,fill}%
}%
\begin{pgfscope}%
\pgfsys@transformshift{3.324341in}{0.417642in}%
\pgfsys@useobject{currentmarker}{}%
\end{pgfscope}%
\end{pgfscope}%
\begin{pgfscope}%
\pgfpathrectangle{\pgfqpoint{0.589510in}{0.417642in}}{\pgfqpoint{3.437062in}{2.055000in}}%
\pgfusepath{clip}%
\pgfsetrectcap%
\pgfsetroundjoin%
\pgfsetlinewidth{0.803000pt}%
\definecolor{currentstroke}{rgb}{0.850000,0.850000,0.850000}%
\pgfsetstrokecolor{currentstroke}%
\pgfsetdash{}{0pt}%
\pgfpathmoveto{\pgfqpoint{3.461895in}{0.417642in}}%
\pgfpathlineto{\pgfqpoint{3.461895in}{2.472642in}}%
\pgfusepath{stroke}%
\end{pgfscope}%
\begin{pgfscope}%
\pgfsetbuttcap%
\pgfsetroundjoin%
\definecolor{currentfill}{rgb}{0.000000,0.000000,0.000000}%
\pgfsetfillcolor{currentfill}%
\pgfsetlinewidth{0.602250pt}%
\definecolor{currentstroke}{rgb}{0.000000,0.000000,0.000000}%
\pgfsetstrokecolor{currentstroke}%
\pgfsetdash{}{0pt}%
\pgfsys@defobject{currentmarker}{\pgfqpoint{0.000000in}{-0.027778in}}{\pgfqpoint{0.000000in}{0.000000in}}{%
\pgfpathmoveto{\pgfqpoint{0.000000in}{0.000000in}}%
\pgfpathlineto{\pgfqpoint{0.000000in}{-0.027778in}}%
\pgfusepath{stroke,fill}%
}%
\begin{pgfscope}%
\pgfsys@transformshift{3.461895in}{0.417642in}%
\pgfsys@useobject{currentmarker}{}%
\end{pgfscope}%
\end{pgfscope}%
\begin{pgfscope}%
\pgfpathrectangle{\pgfqpoint{0.589510in}{0.417642in}}{\pgfqpoint{3.437062in}{2.055000in}}%
\pgfusepath{clip}%
\pgfsetrectcap%
\pgfsetroundjoin%
\pgfsetlinewidth{0.803000pt}%
\definecolor{currentstroke}{rgb}{0.850000,0.850000,0.850000}%
\pgfsetstrokecolor{currentstroke}%
\pgfsetdash{}{0pt}%
\pgfpathmoveto{\pgfqpoint{3.559491in}{0.417642in}}%
\pgfpathlineto{\pgfqpoint{3.559491in}{2.472642in}}%
\pgfusepath{stroke}%
\end{pgfscope}%
\begin{pgfscope}%
\pgfsetbuttcap%
\pgfsetroundjoin%
\definecolor{currentfill}{rgb}{0.000000,0.000000,0.000000}%
\pgfsetfillcolor{currentfill}%
\pgfsetlinewidth{0.602250pt}%
\definecolor{currentstroke}{rgb}{0.000000,0.000000,0.000000}%
\pgfsetstrokecolor{currentstroke}%
\pgfsetdash{}{0pt}%
\pgfsys@defobject{currentmarker}{\pgfqpoint{0.000000in}{-0.027778in}}{\pgfqpoint{0.000000in}{0.000000in}}{%
\pgfpathmoveto{\pgfqpoint{0.000000in}{0.000000in}}%
\pgfpathlineto{\pgfqpoint{0.000000in}{-0.027778in}}%
\pgfusepath{stroke,fill}%
}%
\begin{pgfscope}%
\pgfsys@transformshift{3.559491in}{0.417642in}%
\pgfsys@useobject{currentmarker}{}%
\end{pgfscope}%
\end{pgfscope}%
\begin{pgfscope}%
\pgfpathrectangle{\pgfqpoint{0.589510in}{0.417642in}}{\pgfqpoint{3.437062in}{2.055000in}}%
\pgfusepath{clip}%
\pgfsetrectcap%
\pgfsetroundjoin%
\pgfsetlinewidth{0.803000pt}%
\definecolor{currentstroke}{rgb}{0.850000,0.850000,0.850000}%
\pgfsetstrokecolor{currentstroke}%
\pgfsetdash{}{0pt}%
\pgfpathmoveto{\pgfqpoint{3.635192in}{0.417642in}}%
\pgfpathlineto{\pgfqpoint{3.635192in}{2.472642in}}%
\pgfusepath{stroke}%
\end{pgfscope}%
\begin{pgfscope}%
\pgfsetbuttcap%
\pgfsetroundjoin%
\definecolor{currentfill}{rgb}{0.000000,0.000000,0.000000}%
\pgfsetfillcolor{currentfill}%
\pgfsetlinewidth{0.602250pt}%
\definecolor{currentstroke}{rgb}{0.000000,0.000000,0.000000}%
\pgfsetstrokecolor{currentstroke}%
\pgfsetdash{}{0pt}%
\pgfsys@defobject{currentmarker}{\pgfqpoint{0.000000in}{-0.027778in}}{\pgfqpoint{0.000000in}{0.000000in}}{%
\pgfpathmoveto{\pgfqpoint{0.000000in}{0.000000in}}%
\pgfpathlineto{\pgfqpoint{0.000000in}{-0.027778in}}%
\pgfusepath{stroke,fill}%
}%
\begin{pgfscope}%
\pgfsys@transformshift{3.635192in}{0.417642in}%
\pgfsys@useobject{currentmarker}{}%
\end{pgfscope}%
\end{pgfscope}%
\begin{pgfscope}%
\pgfpathrectangle{\pgfqpoint{0.589510in}{0.417642in}}{\pgfqpoint{3.437062in}{2.055000in}}%
\pgfusepath{clip}%
\pgfsetrectcap%
\pgfsetroundjoin%
\pgfsetlinewidth{0.803000pt}%
\definecolor{currentstroke}{rgb}{0.850000,0.850000,0.850000}%
\pgfsetstrokecolor{currentstroke}%
\pgfsetdash{}{0pt}%
\pgfpathmoveto{\pgfqpoint{3.697044in}{0.417642in}}%
\pgfpathlineto{\pgfqpoint{3.697044in}{2.472642in}}%
\pgfusepath{stroke}%
\end{pgfscope}%
\begin{pgfscope}%
\pgfsetbuttcap%
\pgfsetroundjoin%
\definecolor{currentfill}{rgb}{0.000000,0.000000,0.000000}%
\pgfsetfillcolor{currentfill}%
\pgfsetlinewidth{0.602250pt}%
\definecolor{currentstroke}{rgb}{0.000000,0.000000,0.000000}%
\pgfsetstrokecolor{currentstroke}%
\pgfsetdash{}{0pt}%
\pgfsys@defobject{currentmarker}{\pgfqpoint{0.000000in}{-0.027778in}}{\pgfqpoint{0.000000in}{0.000000in}}{%
\pgfpathmoveto{\pgfqpoint{0.000000in}{0.000000in}}%
\pgfpathlineto{\pgfqpoint{0.000000in}{-0.027778in}}%
\pgfusepath{stroke,fill}%
}%
\begin{pgfscope}%
\pgfsys@transformshift{3.697044in}{0.417642in}%
\pgfsys@useobject{currentmarker}{}%
\end{pgfscope}%
\end{pgfscope}%
\begin{pgfscope}%
\pgfpathrectangle{\pgfqpoint{0.589510in}{0.417642in}}{\pgfqpoint{3.437062in}{2.055000in}}%
\pgfusepath{clip}%
\pgfsetrectcap%
\pgfsetroundjoin%
\pgfsetlinewidth{0.803000pt}%
\definecolor{currentstroke}{rgb}{0.850000,0.850000,0.850000}%
\pgfsetstrokecolor{currentstroke}%
\pgfsetdash{}{0pt}%
\pgfpathmoveto{\pgfqpoint{3.749340in}{0.417642in}}%
\pgfpathlineto{\pgfqpoint{3.749340in}{2.472642in}}%
\pgfusepath{stroke}%
\end{pgfscope}%
\begin{pgfscope}%
\pgfsetbuttcap%
\pgfsetroundjoin%
\definecolor{currentfill}{rgb}{0.000000,0.000000,0.000000}%
\pgfsetfillcolor{currentfill}%
\pgfsetlinewidth{0.602250pt}%
\definecolor{currentstroke}{rgb}{0.000000,0.000000,0.000000}%
\pgfsetstrokecolor{currentstroke}%
\pgfsetdash{}{0pt}%
\pgfsys@defobject{currentmarker}{\pgfqpoint{0.000000in}{-0.027778in}}{\pgfqpoint{0.000000in}{0.000000in}}{%
\pgfpathmoveto{\pgfqpoint{0.000000in}{0.000000in}}%
\pgfpathlineto{\pgfqpoint{0.000000in}{-0.027778in}}%
\pgfusepath{stroke,fill}%
}%
\begin{pgfscope}%
\pgfsys@transformshift{3.749340in}{0.417642in}%
\pgfsys@useobject{currentmarker}{}%
\end{pgfscope}%
\end{pgfscope}%
\begin{pgfscope}%
\pgfpathrectangle{\pgfqpoint{0.589510in}{0.417642in}}{\pgfqpoint{3.437062in}{2.055000in}}%
\pgfusepath{clip}%
\pgfsetrectcap%
\pgfsetroundjoin%
\pgfsetlinewidth{0.803000pt}%
\definecolor{currentstroke}{rgb}{0.850000,0.850000,0.850000}%
\pgfsetstrokecolor{currentstroke}%
\pgfsetdash{}{0pt}%
\pgfpathmoveto{\pgfqpoint{3.794640in}{0.417642in}}%
\pgfpathlineto{\pgfqpoint{3.794640in}{2.472642in}}%
\pgfusepath{stroke}%
\end{pgfscope}%
\begin{pgfscope}%
\pgfsetbuttcap%
\pgfsetroundjoin%
\definecolor{currentfill}{rgb}{0.000000,0.000000,0.000000}%
\pgfsetfillcolor{currentfill}%
\pgfsetlinewidth{0.602250pt}%
\definecolor{currentstroke}{rgb}{0.000000,0.000000,0.000000}%
\pgfsetstrokecolor{currentstroke}%
\pgfsetdash{}{0pt}%
\pgfsys@defobject{currentmarker}{\pgfqpoint{0.000000in}{-0.027778in}}{\pgfqpoint{0.000000in}{0.000000in}}{%
\pgfpathmoveto{\pgfqpoint{0.000000in}{0.000000in}}%
\pgfpathlineto{\pgfqpoint{0.000000in}{-0.027778in}}%
\pgfusepath{stroke,fill}%
}%
\begin{pgfscope}%
\pgfsys@transformshift{3.794640in}{0.417642in}%
\pgfsys@useobject{currentmarker}{}%
\end{pgfscope}%
\end{pgfscope}%
\begin{pgfscope}%
\pgfpathrectangle{\pgfqpoint{0.589510in}{0.417642in}}{\pgfqpoint{3.437062in}{2.055000in}}%
\pgfusepath{clip}%
\pgfsetrectcap%
\pgfsetroundjoin%
\pgfsetlinewidth{0.803000pt}%
\definecolor{currentstroke}{rgb}{0.850000,0.850000,0.850000}%
\pgfsetstrokecolor{currentstroke}%
\pgfsetdash{}{0pt}%
\pgfpathmoveto{\pgfqpoint{3.834598in}{0.417642in}}%
\pgfpathlineto{\pgfqpoint{3.834598in}{2.472642in}}%
\pgfusepath{stroke}%
\end{pgfscope}%
\begin{pgfscope}%
\pgfsetbuttcap%
\pgfsetroundjoin%
\definecolor{currentfill}{rgb}{0.000000,0.000000,0.000000}%
\pgfsetfillcolor{currentfill}%
\pgfsetlinewidth{0.602250pt}%
\definecolor{currentstroke}{rgb}{0.000000,0.000000,0.000000}%
\pgfsetstrokecolor{currentstroke}%
\pgfsetdash{}{0pt}%
\pgfsys@defobject{currentmarker}{\pgfqpoint{0.000000in}{-0.027778in}}{\pgfqpoint{0.000000in}{0.000000in}}{%
\pgfpathmoveto{\pgfqpoint{0.000000in}{0.000000in}}%
\pgfpathlineto{\pgfqpoint{0.000000in}{-0.027778in}}%
\pgfusepath{stroke,fill}%
}%
\begin{pgfscope}%
\pgfsys@transformshift{3.834598in}{0.417642in}%
\pgfsys@useobject{currentmarker}{}%
\end{pgfscope}%
\end{pgfscope}%
\begin{pgfscope}%
\definecolor{textcolor}{rgb}{0.000000,0.000000,0.000000}%
\pgfsetstrokecolor{textcolor}%
\pgfsetfillcolor{textcolor}%
\pgftext[x=2.308041in,y=0.165003in,,top]{\color{textcolor}\rmfamily\fontsize{10.000000}{12.000000}\selectfont \(\displaystyle \tau\) in \unit{\second}}%
\end{pgfscope}%
\begin{pgfscope}%
\pgfpathrectangle{\pgfqpoint{0.589510in}{0.417642in}}{\pgfqpoint{3.437062in}{2.055000in}}%
\pgfusepath{clip}%
\pgfsetrectcap%
\pgfsetroundjoin%
\pgfsetlinewidth{0.803000pt}%
\definecolor{currentstroke}{rgb}{0.450000,0.450000,0.450000}%
\pgfsetstrokecolor{currentstroke}%
\pgfsetdash{}{0pt}%
\pgfpathmoveto{\pgfqpoint{0.589510in}{1.728810in}}%
\pgfpathlineto{\pgfqpoint{4.026572in}{1.728810in}}%
\pgfusepath{stroke}%
\end{pgfscope}%
\begin{pgfscope}%
\pgfsetbuttcap%
\pgfsetroundjoin%
\definecolor{currentfill}{rgb}{0.000000,0.000000,0.000000}%
\pgfsetfillcolor{currentfill}%
\pgfsetlinewidth{0.803000pt}%
\definecolor{currentstroke}{rgb}{0.000000,0.000000,0.000000}%
\pgfsetstrokecolor{currentstroke}%
\pgfsetdash{}{0pt}%
\pgfsys@defobject{currentmarker}{\pgfqpoint{-0.048611in}{0.000000in}}{\pgfqpoint{-0.000000in}{0.000000in}}{%
\pgfpathmoveto{\pgfqpoint{-0.000000in}{0.000000in}}%
\pgfpathlineto{\pgfqpoint{-0.048611in}{0.000000in}}%
\pgfusepath{stroke,fill}%
}%
\begin{pgfscope}%
\pgfsys@transformshift{0.589510in}{1.728810in}%
\pgfsys@useobject{currentmarker}{}%
\end{pgfscope}%
\end{pgfscope}%
\begin{pgfscope}%
\definecolor{textcolor}{rgb}{0.000000,0.000000,0.000000}%
\pgfsetstrokecolor{textcolor}%
\pgfsetfillcolor{textcolor}%
\pgftext[x=0.236114in, y=1.689657in, left, base]{\color{textcolor}\rmfamily\fontsize{8.000000}{9.600000}\selectfont \(\displaystyle {10^{-1}}\)}%
\end{pgfscope}%
\begin{pgfscope}%
\pgfpathrectangle{\pgfqpoint{0.589510in}{0.417642in}}{\pgfqpoint{3.437062in}{2.055000in}}%
\pgfusepath{clip}%
\pgfsetrectcap%
\pgfsetroundjoin%
\pgfsetlinewidth{0.803000pt}%
\definecolor{currentstroke}{rgb}{0.850000,0.850000,0.850000}%
\pgfsetstrokecolor{currentstroke}%
\pgfsetdash{}{0pt}%
\pgfpathmoveto{\pgfqpoint{0.589510in}{0.795369in}}%
\pgfpathlineto{\pgfqpoint{4.026572in}{0.795369in}}%
\pgfusepath{stroke}%
\end{pgfscope}%
\begin{pgfscope}%
\pgfsetbuttcap%
\pgfsetroundjoin%
\definecolor{currentfill}{rgb}{0.000000,0.000000,0.000000}%
\pgfsetfillcolor{currentfill}%
\pgfsetlinewidth{0.602250pt}%
\definecolor{currentstroke}{rgb}{0.000000,0.000000,0.000000}%
\pgfsetstrokecolor{currentstroke}%
\pgfsetdash{}{0pt}%
\pgfsys@defobject{currentmarker}{\pgfqpoint{-0.027778in}{0.000000in}}{\pgfqpoint{-0.000000in}{0.000000in}}{%
\pgfpathmoveto{\pgfqpoint{-0.000000in}{0.000000in}}%
\pgfpathlineto{\pgfqpoint{-0.027778in}{0.000000in}}%
\pgfusepath{stroke,fill}%
}%
\begin{pgfscope}%
\pgfsys@transformshift{0.589510in}{0.795369in}%
\pgfsys@useobject{currentmarker}{}%
\end{pgfscope}%
\end{pgfscope}%
\begin{pgfscope}%
\pgfpathrectangle{\pgfqpoint{0.589510in}{0.417642in}}{\pgfqpoint{3.437062in}{2.055000in}}%
\pgfusepath{clip}%
\pgfsetrectcap%
\pgfsetroundjoin%
\pgfsetlinewidth{0.803000pt}%
\definecolor{currentstroke}{rgb}{0.850000,0.850000,0.850000}%
\pgfsetstrokecolor{currentstroke}%
\pgfsetdash{}{0pt}%
\pgfpathmoveto{\pgfqpoint{0.589510in}{1.030531in}}%
\pgfpathlineto{\pgfqpoint{4.026572in}{1.030531in}}%
\pgfusepath{stroke}%
\end{pgfscope}%
\begin{pgfscope}%
\pgfsetbuttcap%
\pgfsetroundjoin%
\definecolor{currentfill}{rgb}{0.000000,0.000000,0.000000}%
\pgfsetfillcolor{currentfill}%
\pgfsetlinewidth{0.602250pt}%
\definecolor{currentstroke}{rgb}{0.000000,0.000000,0.000000}%
\pgfsetstrokecolor{currentstroke}%
\pgfsetdash{}{0pt}%
\pgfsys@defobject{currentmarker}{\pgfqpoint{-0.027778in}{0.000000in}}{\pgfqpoint{-0.000000in}{0.000000in}}{%
\pgfpathmoveto{\pgfqpoint{-0.000000in}{0.000000in}}%
\pgfpathlineto{\pgfqpoint{-0.027778in}{0.000000in}}%
\pgfusepath{stroke,fill}%
}%
\begin{pgfscope}%
\pgfsys@transformshift{0.589510in}{1.030531in}%
\pgfsys@useobject{currentmarker}{}%
\end{pgfscope}%
\end{pgfscope}%
\begin{pgfscope}%
\pgfpathrectangle{\pgfqpoint{0.589510in}{0.417642in}}{\pgfqpoint{3.437062in}{2.055000in}}%
\pgfusepath{clip}%
\pgfsetrectcap%
\pgfsetroundjoin%
\pgfsetlinewidth{0.803000pt}%
\definecolor{currentstroke}{rgb}{0.850000,0.850000,0.850000}%
\pgfsetstrokecolor{currentstroke}%
\pgfsetdash{}{0pt}%
\pgfpathmoveto{\pgfqpoint{0.589510in}{1.197380in}}%
\pgfpathlineto{\pgfqpoint{4.026572in}{1.197380in}}%
\pgfusepath{stroke}%
\end{pgfscope}%
\begin{pgfscope}%
\pgfsetbuttcap%
\pgfsetroundjoin%
\definecolor{currentfill}{rgb}{0.000000,0.000000,0.000000}%
\pgfsetfillcolor{currentfill}%
\pgfsetlinewidth{0.602250pt}%
\definecolor{currentstroke}{rgb}{0.000000,0.000000,0.000000}%
\pgfsetstrokecolor{currentstroke}%
\pgfsetdash{}{0pt}%
\pgfsys@defobject{currentmarker}{\pgfqpoint{-0.027778in}{0.000000in}}{\pgfqpoint{-0.000000in}{0.000000in}}{%
\pgfpathmoveto{\pgfqpoint{-0.000000in}{0.000000in}}%
\pgfpathlineto{\pgfqpoint{-0.027778in}{0.000000in}}%
\pgfusepath{stroke,fill}%
}%
\begin{pgfscope}%
\pgfsys@transformshift{0.589510in}{1.197380in}%
\pgfsys@useobject{currentmarker}{}%
\end{pgfscope}%
\end{pgfscope}%
\begin{pgfscope}%
\pgfpathrectangle{\pgfqpoint{0.589510in}{0.417642in}}{\pgfqpoint{3.437062in}{2.055000in}}%
\pgfusepath{clip}%
\pgfsetrectcap%
\pgfsetroundjoin%
\pgfsetlinewidth{0.803000pt}%
\definecolor{currentstroke}{rgb}{0.850000,0.850000,0.850000}%
\pgfsetstrokecolor{currentstroke}%
\pgfsetdash{}{0pt}%
\pgfpathmoveto{\pgfqpoint{0.589510in}{1.326799in}}%
\pgfpathlineto{\pgfqpoint{4.026572in}{1.326799in}}%
\pgfusepath{stroke}%
\end{pgfscope}%
\begin{pgfscope}%
\pgfsetbuttcap%
\pgfsetroundjoin%
\definecolor{currentfill}{rgb}{0.000000,0.000000,0.000000}%
\pgfsetfillcolor{currentfill}%
\pgfsetlinewidth{0.602250pt}%
\definecolor{currentstroke}{rgb}{0.000000,0.000000,0.000000}%
\pgfsetstrokecolor{currentstroke}%
\pgfsetdash{}{0pt}%
\pgfsys@defobject{currentmarker}{\pgfqpoint{-0.027778in}{0.000000in}}{\pgfqpoint{-0.000000in}{0.000000in}}{%
\pgfpathmoveto{\pgfqpoint{-0.000000in}{0.000000in}}%
\pgfpathlineto{\pgfqpoint{-0.027778in}{0.000000in}}%
\pgfusepath{stroke,fill}%
}%
\begin{pgfscope}%
\pgfsys@transformshift{0.589510in}{1.326799in}%
\pgfsys@useobject{currentmarker}{}%
\end{pgfscope}%
\end{pgfscope}%
\begin{pgfscope}%
\pgfpathrectangle{\pgfqpoint{0.589510in}{0.417642in}}{\pgfqpoint{3.437062in}{2.055000in}}%
\pgfusepath{clip}%
\pgfsetrectcap%
\pgfsetroundjoin%
\pgfsetlinewidth{0.803000pt}%
\definecolor{currentstroke}{rgb}{0.850000,0.850000,0.850000}%
\pgfsetstrokecolor{currentstroke}%
\pgfsetdash{}{0pt}%
\pgfpathmoveto{\pgfqpoint{0.589510in}{1.432542in}}%
\pgfpathlineto{\pgfqpoint{4.026572in}{1.432542in}}%
\pgfusepath{stroke}%
\end{pgfscope}%
\begin{pgfscope}%
\pgfsetbuttcap%
\pgfsetroundjoin%
\definecolor{currentfill}{rgb}{0.000000,0.000000,0.000000}%
\pgfsetfillcolor{currentfill}%
\pgfsetlinewidth{0.602250pt}%
\definecolor{currentstroke}{rgb}{0.000000,0.000000,0.000000}%
\pgfsetstrokecolor{currentstroke}%
\pgfsetdash{}{0pt}%
\pgfsys@defobject{currentmarker}{\pgfqpoint{-0.027778in}{0.000000in}}{\pgfqpoint{-0.000000in}{0.000000in}}{%
\pgfpathmoveto{\pgfqpoint{-0.000000in}{0.000000in}}%
\pgfpathlineto{\pgfqpoint{-0.027778in}{0.000000in}}%
\pgfusepath{stroke,fill}%
}%
\begin{pgfscope}%
\pgfsys@transformshift{0.589510in}{1.432542in}%
\pgfsys@useobject{currentmarker}{}%
\end{pgfscope}%
\end{pgfscope}%
\begin{pgfscope}%
\pgfpathrectangle{\pgfqpoint{0.589510in}{0.417642in}}{\pgfqpoint{3.437062in}{2.055000in}}%
\pgfusepath{clip}%
\pgfsetrectcap%
\pgfsetroundjoin%
\pgfsetlinewidth{0.803000pt}%
\definecolor{currentstroke}{rgb}{0.850000,0.850000,0.850000}%
\pgfsetstrokecolor{currentstroke}%
\pgfsetdash{}{0pt}%
\pgfpathmoveto{\pgfqpoint{0.589510in}{1.521946in}}%
\pgfpathlineto{\pgfqpoint{4.026572in}{1.521946in}}%
\pgfusepath{stroke}%
\end{pgfscope}%
\begin{pgfscope}%
\pgfsetbuttcap%
\pgfsetroundjoin%
\definecolor{currentfill}{rgb}{0.000000,0.000000,0.000000}%
\pgfsetfillcolor{currentfill}%
\pgfsetlinewidth{0.602250pt}%
\definecolor{currentstroke}{rgb}{0.000000,0.000000,0.000000}%
\pgfsetstrokecolor{currentstroke}%
\pgfsetdash{}{0pt}%
\pgfsys@defobject{currentmarker}{\pgfqpoint{-0.027778in}{0.000000in}}{\pgfqpoint{-0.000000in}{0.000000in}}{%
\pgfpathmoveto{\pgfqpoint{-0.000000in}{0.000000in}}%
\pgfpathlineto{\pgfqpoint{-0.027778in}{0.000000in}}%
\pgfusepath{stroke,fill}%
}%
\begin{pgfscope}%
\pgfsys@transformshift{0.589510in}{1.521946in}%
\pgfsys@useobject{currentmarker}{}%
\end{pgfscope}%
\end{pgfscope}%
\begin{pgfscope}%
\pgfpathrectangle{\pgfqpoint{0.589510in}{0.417642in}}{\pgfqpoint{3.437062in}{2.055000in}}%
\pgfusepath{clip}%
\pgfsetrectcap%
\pgfsetroundjoin%
\pgfsetlinewidth{0.803000pt}%
\definecolor{currentstroke}{rgb}{0.850000,0.850000,0.850000}%
\pgfsetstrokecolor{currentstroke}%
\pgfsetdash{}{0pt}%
\pgfpathmoveto{\pgfqpoint{0.589510in}{1.599392in}}%
\pgfpathlineto{\pgfqpoint{4.026572in}{1.599392in}}%
\pgfusepath{stroke}%
\end{pgfscope}%
\begin{pgfscope}%
\pgfsetbuttcap%
\pgfsetroundjoin%
\definecolor{currentfill}{rgb}{0.000000,0.000000,0.000000}%
\pgfsetfillcolor{currentfill}%
\pgfsetlinewidth{0.602250pt}%
\definecolor{currentstroke}{rgb}{0.000000,0.000000,0.000000}%
\pgfsetstrokecolor{currentstroke}%
\pgfsetdash{}{0pt}%
\pgfsys@defobject{currentmarker}{\pgfqpoint{-0.027778in}{0.000000in}}{\pgfqpoint{-0.000000in}{0.000000in}}{%
\pgfpathmoveto{\pgfqpoint{-0.000000in}{0.000000in}}%
\pgfpathlineto{\pgfqpoint{-0.027778in}{0.000000in}}%
\pgfusepath{stroke,fill}%
}%
\begin{pgfscope}%
\pgfsys@transformshift{0.589510in}{1.599392in}%
\pgfsys@useobject{currentmarker}{}%
\end{pgfscope}%
\end{pgfscope}%
\begin{pgfscope}%
\pgfpathrectangle{\pgfqpoint{0.589510in}{0.417642in}}{\pgfqpoint{3.437062in}{2.055000in}}%
\pgfusepath{clip}%
\pgfsetrectcap%
\pgfsetroundjoin%
\pgfsetlinewidth{0.803000pt}%
\definecolor{currentstroke}{rgb}{0.850000,0.850000,0.850000}%
\pgfsetstrokecolor{currentstroke}%
\pgfsetdash{}{0pt}%
\pgfpathmoveto{\pgfqpoint{0.589510in}{1.667703in}}%
\pgfpathlineto{\pgfqpoint{4.026572in}{1.667703in}}%
\pgfusepath{stroke}%
\end{pgfscope}%
\begin{pgfscope}%
\pgfsetbuttcap%
\pgfsetroundjoin%
\definecolor{currentfill}{rgb}{0.000000,0.000000,0.000000}%
\pgfsetfillcolor{currentfill}%
\pgfsetlinewidth{0.602250pt}%
\definecolor{currentstroke}{rgb}{0.000000,0.000000,0.000000}%
\pgfsetstrokecolor{currentstroke}%
\pgfsetdash{}{0pt}%
\pgfsys@defobject{currentmarker}{\pgfqpoint{-0.027778in}{0.000000in}}{\pgfqpoint{-0.000000in}{0.000000in}}{%
\pgfpathmoveto{\pgfqpoint{-0.000000in}{0.000000in}}%
\pgfpathlineto{\pgfqpoint{-0.027778in}{0.000000in}}%
\pgfusepath{stroke,fill}%
}%
\begin{pgfscope}%
\pgfsys@transformshift{0.589510in}{1.667703in}%
\pgfsys@useobject{currentmarker}{}%
\end{pgfscope}%
\end{pgfscope}%
\begin{pgfscope}%
\pgfpathrectangle{\pgfqpoint{0.589510in}{0.417642in}}{\pgfqpoint{3.437062in}{2.055000in}}%
\pgfusepath{clip}%
\pgfsetrectcap%
\pgfsetroundjoin%
\pgfsetlinewidth{0.803000pt}%
\definecolor{currentstroke}{rgb}{0.850000,0.850000,0.850000}%
\pgfsetstrokecolor{currentstroke}%
\pgfsetdash{}{0pt}%
\pgfpathmoveto{\pgfqpoint{0.589510in}{2.130821in}}%
\pgfpathlineto{\pgfqpoint{4.026572in}{2.130821in}}%
\pgfusepath{stroke}%
\end{pgfscope}%
\begin{pgfscope}%
\pgfsetbuttcap%
\pgfsetroundjoin%
\definecolor{currentfill}{rgb}{0.000000,0.000000,0.000000}%
\pgfsetfillcolor{currentfill}%
\pgfsetlinewidth{0.602250pt}%
\definecolor{currentstroke}{rgb}{0.000000,0.000000,0.000000}%
\pgfsetstrokecolor{currentstroke}%
\pgfsetdash{}{0pt}%
\pgfsys@defobject{currentmarker}{\pgfqpoint{-0.027778in}{0.000000in}}{\pgfqpoint{-0.000000in}{0.000000in}}{%
\pgfpathmoveto{\pgfqpoint{-0.000000in}{0.000000in}}%
\pgfpathlineto{\pgfqpoint{-0.027778in}{0.000000in}}%
\pgfusepath{stroke,fill}%
}%
\begin{pgfscope}%
\pgfsys@transformshift{0.589510in}{2.130821in}%
\pgfsys@useobject{currentmarker}{}%
\end{pgfscope}%
\end{pgfscope}%
\begin{pgfscope}%
\pgfpathrectangle{\pgfqpoint{0.589510in}{0.417642in}}{\pgfqpoint{3.437062in}{2.055000in}}%
\pgfusepath{clip}%
\pgfsetrectcap%
\pgfsetroundjoin%
\pgfsetlinewidth{0.803000pt}%
\definecolor{currentstroke}{rgb}{0.850000,0.850000,0.850000}%
\pgfsetstrokecolor{currentstroke}%
\pgfsetdash{}{0pt}%
\pgfpathmoveto{\pgfqpoint{0.589510in}{2.365983in}}%
\pgfpathlineto{\pgfqpoint{4.026572in}{2.365983in}}%
\pgfusepath{stroke}%
\end{pgfscope}%
\begin{pgfscope}%
\pgfsetbuttcap%
\pgfsetroundjoin%
\definecolor{currentfill}{rgb}{0.000000,0.000000,0.000000}%
\pgfsetfillcolor{currentfill}%
\pgfsetlinewidth{0.602250pt}%
\definecolor{currentstroke}{rgb}{0.000000,0.000000,0.000000}%
\pgfsetstrokecolor{currentstroke}%
\pgfsetdash{}{0pt}%
\pgfsys@defobject{currentmarker}{\pgfqpoint{-0.027778in}{0.000000in}}{\pgfqpoint{-0.000000in}{0.000000in}}{%
\pgfpathmoveto{\pgfqpoint{-0.000000in}{0.000000in}}%
\pgfpathlineto{\pgfqpoint{-0.027778in}{0.000000in}}%
\pgfusepath{stroke,fill}%
}%
\begin{pgfscope}%
\pgfsys@transformshift{0.589510in}{2.365983in}%
\pgfsys@useobject{currentmarker}{}%
\end{pgfscope}%
\end{pgfscope}%
\begin{pgfscope}%
\definecolor{textcolor}{rgb}{0.000000,0.000000,0.000000}%
\pgfsetstrokecolor{textcolor}%
\pgfsetfillcolor{textcolor}%
\pgftext[x=0.180559in,y=1.445142in,,bottom,rotate=90.000000]{\color{textcolor}\rmfamily\fontsize{10.000000}{12.000000}\selectfont ADEV \(\displaystyle \sigma_A(\tau)\)}%
\end{pgfscope}%
\begin{pgfscope}%
\pgfpathrectangle{\pgfqpoint{0.589510in}{0.417642in}}{\pgfqpoint{3.437062in}{2.055000in}}%
\pgfusepath{clip}%
\pgfsetbuttcap%
\pgfsetroundjoin%
\pgfsetlinewidth{1.505625pt}%
\definecolor{currentstroke}{rgb}{0.003922,0.450980,0.698039}%
\pgfsetstrokecolor{currentstroke}%
\pgfsetdash{{5.550000pt}{2.400000pt}}{0.000000pt}%
\pgfpathmoveto{\pgfqpoint{0.745740in}{1.559906in}}%
\pgfpathlineto{\pgfqpoint{0.883294in}{1.665824in}}%
\pgfpathlineto{\pgfqpoint{1.056591in}{1.790983in}}%
\pgfpathlineto{\pgfqpoint{1.216039in}{1.893684in}}%
\pgfpathlineto{\pgfqpoint{1.380747in}{1.980727in}}%
\pgfpathlineto{\pgfqpoint{1.559224in}{2.043219in}}%
\pgfpathlineto{\pgfqpoint{1.726296in}{2.061342in}}%
\pgfpathlineto{\pgfqpoint{1.893892in}{2.033564in}}%
\pgfpathlineto{\pgfqpoint{2.059042in}{1.963607in}}%
\pgfpathlineto{\pgfqpoint{2.223750in}{1.863241in}}%
\pgfpathlineto{\pgfqpoint{2.389127in}{1.744648in}}%
\pgfpathlineto{\pgfqpoint{2.554041in}{1.616807in}}%
\pgfpathlineto{\pgfqpoint{2.718686in}{1.483847in}}%
\pgfpathlineto{\pgfqpoint{2.883588in}{1.347581in}}%
\pgfpathlineto{\pgfqpoint{3.047937in}{1.209936in}}%
\pgfpathlineto{\pgfqpoint{3.212425in}{1.071070in}}%
\pgfpathlineto{\pgfqpoint{3.376951in}{0.931498in}}%
\pgfpathlineto{\pgfqpoint{3.541419in}{0.791565in}}%
\pgfpathlineto{\pgfqpoint{3.705862in}{0.651400in}}%
\pgfpathlineto{\pgfqpoint{3.870342in}{0.511051in}}%
\pgfusepath{stroke}%
\end{pgfscope}%
\begin{pgfscope}%
\pgfpathrectangle{\pgfqpoint{0.589510in}{0.417642in}}{\pgfqpoint{3.437062in}{2.055000in}}%
\pgfusepath{clip}%
\pgfsetbuttcap%
\pgfsetroundjoin%
\definecolor{currentfill}{rgb}{0.003922,0.450980,0.698039}%
\pgfsetfillcolor{currentfill}%
\pgfsetlinewidth{1.003750pt}%
\definecolor{currentstroke}{rgb}{0.003922,0.450980,0.698039}%
\pgfsetstrokecolor{currentstroke}%
\pgfsetdash{}{0pt}%
\pgfsys@defobject{currentmarker}{\pgfqpoint{-0.020833in}{-0.020833in}}{\pgfqpoint{0.020833in}{0.020833in}}{%
\pgfpathmoveto{\pgfqpoint{0.000000in}{-0.020833in}}%
\pgfpathcurveto{\pgfqpoint{0.005525in}{-0.020833in}}{\pgfqpoint{0.010825in}{-0.018638in}}{\pgfqpoint{0.014731in}{-0.014731in}}%
\pgfpathcurveto{\pgfqpoint{0.018638in}{-0.010825in}}{\pgfqpoint{0.020833in}{-0.005525in}}{\pgfqpoint{0.020833in}{0.000000in}}%
\pgfpathcurveto{\pgfqpoint{0.020833in}{0.005525in}}{\pgfqpoint{0.018638in}{0.010825in}}{\pgfqpoint{0.014731in}{0.014731in}}%
\pgfpathcurveto{\pgfqpoint{0.010825in}{0.018638in}}{\pgfqpoint{0.005525in}{0.020833in}}{\pgfqpoint{0.000000in}{0.020833in}}%
\pgfpathcurveto{\pgfqpoint{-0.005525in}{0.020833in}}{\pgfqpoint{-0.010825in}{0.018638in}}{\pgfqpoint{-0.014731in}{0.014731in}}%
\pgfpathcurveto{\pgfqpoint{-0.018638in}{0.010825in}}{\pgfqpoint{-0.020833in}{0.005525in}}{\pgfqpoint{-0.020833in}{0.000000in}}%
\pgfpathcurveto{\pgfqpoint{-0.020833in}{-0.005525in}}{\pgfqpoint{-0.018638in}{-0.010825in}}{\pgfqpoint{-0.014731in}{-0.014731in}}%
\pgfpathcurveto{\pgfqpoint{-0.010825in}{-0.018638in}}{\pgfqpoint{-0.005525in}{-0.020833in}}{\pgfqpoint{0.000000in}{-0.020833in}}%
\pgfpathlineto{\pgfqpoint{0.000000in}{-0.020833in}}%
\pgfpathclose%
\pgfusepath{stroke,fill}%
}%
\begin{pgfscope}%
\pgfsys@transformshift{0.745740in}{1.595988in}%
\pgfsys@useobject{currentmarker}{}%
\end{pgfscope}%
\begin{pgfscope}%
\pgfsys@transformshift{0.883294in}{1.682684in}%
\pgfsys@useobject{currentmarker}{}%
\end{pgfscope}%
\begin{pgfscope}%
\pgfsys@transformshift{1.056591in}{1.797122in}%
\pgfsys@useobject{currentmarker}{}%
\end{pgfscope}%
\begin{pgfscope}%
\pgfsys@transformshift{1.216039in}{1.895780in}%
\pgfsys@useobject{currentmarker}{}%
\end{pgfscope}%
\begin{pgfscope}%
\pgfsys@transformshift{1.380747in}{1.980802in}%
\pgfsys@useobject{currentmarker}{}%
\end{pgfscope}%
\begin{pgfscope}%
\pgfsys@transformshift{1.559224in}{2.042134in}%
\pgfsys@useobject{currentmarker}{}%
\end{pgfscope}%
\begin{pgfscope}%
\pgfsys@transformshift{1.726296in}{2.059848in}%
\pgfsys@useobject{currentmarker}{}%
\end{pgfscope}%
\begin{pgfscope}%
\pgfsys@transformshift{1.893892in}{2.032510in}%
\pgfsys@useobject{currentmarker}{}%
\end{pgfscope}%
\begin{pgfscope}%
\pgfsys@transformshift{2.059042in}{1.963610in}%
\pgfsys@useobject{currentmarker}{}%
\end{pgfscope}%
\begin{pgfscope}%
\pgfsys@transformshift{2.223750in}{1.862324in}%
\pgfsys@useobject{currentmarker}{}%
\end{pgfscope}%
\begin{pgfscope}%
\pgfsys@transformshift{2.389127in}{1.742498in}%
\pgfsys@useobject{currentmarker}{}%
\end{pgfscope}%
\begin{pgfscope}%
\pgfsys@transformshift{2.554041in}{1.614059in}%
\pgfsys@useobject{currentmarker}{}%
\end{pgfscope}%
\begin{pgfscope}%
\pgfsys@transformshift{2.718686in}{1.480139in}%
\pgfsys@useobject{currentmarker}{}%
\end{pgfscope}%
\begin{pgfscope}%
\pgfsys@transformshift{2.883588in}{1.344195in}%
\pgfsys@useobject{currentmarker}{}%
\end{pgfscope}%
\begin{pgfscope}%
\pgfsys@transformshift{3.047937in}{1.211691in}%
\pgfsys@useobject{currentmarker}{}%
\end{pgfscope}%
\begin{pgfscope}%
\pgfsys@transformshift{3.212425in}{1.071840in}%
\pgfsys@useobject{currentmarker}{}%
\end{pgfscope}%
\begin{pgfscope}%
\pgfsys@transformshift{3.376951in}{0.932165in}%
\pgfsys@useobject{currentmarker}{}%
\end{pgfscope}%
\begin{pgfscope}%
\pgfsys@transformshift{3.541419in}{0.793422in}%
\pgfsys@useobject{currentmarker}{}%
\end{pgfscope}%
\begin{pgfscope}%
\pgfsys@transformshift{3.705862in}{0.660521in}%
\pgfsys@useobject{currentmarker}{}%
\end{pgfscope}%
\begin{pgfscope}%
\pgfsys@transformshift{3.870342in}{0.518723in}%
\pgfsys@useobject{currentmarker}{}%
\end{pgfscope}%
\end{pgfscope}%
\begin{pgfscope}%
\pgfpathrectangle{\pgfqpoint{0.589510in}{0.417642in}}{\pgfqpoint{3.437062in}{2.055000in}}%
\pgfusepath{clip}%
\pgfsetbuttcap%
\pgfsetroundjoin%
\pgfsetlinewidth{1.505625pt}%
\definecolor{currentstroke}{rgb}{0.007843,0.619608,0.450980}%
\pgfsetstrokecolor{currentstroke}%
\pgfsetdash{{5.550000pt}{2.400000pt}}{0.000000pt}%
\pgfpathmoveto{\pgfqpoint{0.745740in}{1.405882in}}%
\pgfpathlineto{\pgfqpoint{0.883294in}{1.521298in}}%
\pgfpathlineto{\pgfqpoint{1.056591in}{1.665114in}}%
\pgfpathlineto{\pgfqpoint{1.216039in}{1.794963in}}%
\pgfpathlineto{\pgfqpoint{1.380747in}{1.925088in}}%
\pgfpathlineto{\pgfqpoint{1.559224in}{2.058696in}}%
\pgfpathlineto{\pgfqpoint{1.726296in}{2.172645in}}%
\pgfpathlineto{\pgfqpoint{1.893892in}{2.270086in}}%
\pgfpathlineto{\pgfqpoint{2.059042in}{2.341547in}}%
\pgfpathlineto{\pgfqpoint{2.223750in}{2.379022in}}%
\pgfpathlineto{\pgfqpoint{2.389127in}{2.374190in}}%
\pgfpathlineto{\pgfqpoint{2.554041in}{2.324611in}}%
\pgfpathlineto{\pgfqpoint{2.718686in}{2.237995in}}%
\pgfpathlineto{\pgfqpoint{2.883588in}{2.127384in}}%
\pgfpathlineto{\pgfqpoint{3.047937in}{2.004124in}}%
\pgfpathlineto{\pgfqpoint{3.212425in}{1.873652in}}%
\pgfpathlineto{\pgfqpoint{3.376951in}{1.739083in}}%
\pgfpathlineto{\pgfqpoint{3.541419in}{1.602169in}}%
\pgfpathlineto{\pgfqpoint{3.705862in}{1.463842in}}%
\pgfpathlineto{\pgfqpoint{3.870342in}{1.324616in}}%
\pgfusepath{stroke}%
\end{pgfscope}%
\begin{pgfscope}%
\pgfpathrectangle{\pgfqpoint{0.589510in}{0.417642in}}{\pgfqpoint{3.437062in}{2.055000in}}%
\pgfusepath{clip}%
\pgfsetbuttcap%
\pgfsetroundjoin%
\definecolor{currentfill}{rgb}{0.007843,0.619608,0.450980}%
\pgfsetfillcolor{currentfill}%
\pgfsetlinewidth{1.003750pt}%
\definecolor{currentstroke}{rgb}{0.007843,0.619608,0.450980}%
\pgfsetstrokecolor{currentstroke}%
\pgfsetdash{}{0pt}%
\pgfsys@defobject{currentmarker}{\pgfqpoint{-0.020833in}{-0.020833in}}{\pgfqpoint{0.020833in}{0.020833in}}{%
\pgfpathmoveto{\pgfqpoint{0.000000in}{-0.020833in}}%
\pgfpathcurveto{\pgfqpoint{0.005525in}{-0.020833in}}{\pgfqpoint{0.010825in}{-0.018638in}}{\pgfqpoint{0.014731in}{-0.014731in}}%
\pgfpathcurveto{\pgfqpoint{0.018638in}{-0.010825in}}{\pgfqpoint{0.020833in}{-0.005525in}}{\pgfqpoint{0.020833in}{0.000000in}}%
\pgfpathcurveto{\pgfqpoint{0.020833in}{0.005525in}}{\pgfqpoint{0.018638in}{0.010825in}}{\pgfqpoint{0.014731in}{0.014731in}}%
\pgfpathcurveto{\pgfqpoint{0.010825in}{0.018638in}}{\pgfqpoint{0.005525in}{0.020833in}}{\pgfqpoint{0.000000in}{0.020833in}}%
\pgfpathcurveto{\pgfqpoint{-0.005525in}{0.020833in}}{\pgfqpoint{-0.010825in}{0.018638in}}{\pgfqpoint{-0.014731in}{0.014731in}}%
\pgfpathcurveto{\pgfqpoint{-0.018638in}{0.010825in}}{\pgfqpoint{-0.020833in}{0.005525in}}{\pgfqpoint{-0.020833in}{0.000000in}}%
\pgfpathcurveto{\pgfqpoint{-0.020833in}{-0.005525in}}{\pgfqpoint{-0.018638in}{-0.010825in}}{\pgfqpoint{-0.014731in}{-0.014731in}}%
\pgfpathcurveto{\pgfqpoint{-0.010825in}{-0.018638in}}{\pgfqpoint{-0.005525in}{-0.020833in}}{\pgfqpoint{0.000000in}{-0.020833in}}%
\pgfpathlineto{\pgfqpoint{0.000000in}{-0.020833in}}%
\pgfpathclose%
\pgfusepath{stroke,fill}%
}%
\begin{pgfscope}%
\pgfsys@transformshift{0.745740in}{1.439250in}%
\pgfsys@useobject{currentmarker}{}%
\end{pgfscope}%
\begin{pgfscope}%
\pgfsys@transformshift{0.883294in}{1.536085in}%
\pgfsys@useobject{currentmarker}{}%
\end{pgfscope}%
\begin{pgfscope}%
\pgfsys@transformshift{1.056591in}{1.669922in}%
\pgfsys@useobject{currentmarker}{}%
\end{pgfscope}%
\begin{pgfscope}%
\pgfsys@transformshift{1.216039in}{1.796284in}%
\pgfsys@useobject{currentmarker}{}%
\end{pgfscope}%
\begin{pgfscope}%
\pgfsys@transformshift{1.380747in}{1.925105in}%
\pgfsys@useobject{currentmarker}{}%
\end{pgfscope}%
\begin{pgfscope}%
\pgfsys@transformshift{1.559224in}{2.058323in}%
\pgfsys@useobject{currentmarker}{}%
\end{pgfscope}%
\begin{pgfscope}%
\pgfsys@transformshift{1.726296in}{2.172197in}%
\pgfsys@useobject{currentmarker}{}%
\end{pgfscope}%
\begin{pgfscope}%
\pgfsys@transformshift{1.893892in}{2.269642in}%
\pgfsys@useobject{currentmarker}{}%
\end{pgfscope}%
\begin{pgfscope}%
\pgfsys@transformshift{2.059042in}{2.341414in}%
\pgfsys@useobject{currentmarker}{}%
\end{pgfscope}%
\begin{pgfscope}%
\pgfsys@transformshift{2.223750in}{2.379233in}%
\pgfsys@useobject{currentmarker}{}%
\end{pgfscope}%
\begin{pgfscope}%
\pgfsys@transformshift{2.389127in}{2.374711in}%
\pgfsys@useobject{currentmarker}{}%
\end{pgfscope}%
\begin{pgfscope}%
\pgfsys@transformshift{2.554041in}{2.326537in}%
\pgfsys@useobject{currentmarker}{}%
\end{pgfscope}%
\begin{pgfscope}%
\pgfsys@transformshift{2.718686in}{2.239534in}%
\pgfsys@useobject{currentmarker}{}%
\end{pgfscope}%
\begin{pgfscope}%
\pgfsys@transformshift{2.883588in}{2.125773in}%
\pgfsys@useobject{currentmarker}{}%
\end{pgfscope}%
\begin{pgfscope}%
\pgfsys@transformshift{3.047937in}{2.002365in}%
\pgfsys@useobject{currentmarker}{}%
\end{pgfscope}%
\begin{pgfscope}%
\pgfsys@transformshift{3.212425in}{1.870829in}%
\pgfsys@useobject{currentmarker}{}%
\end{pgfscope}%
\begin{pgfscope}%
\pgfsys@transformshift{3.376951in}{1.732480in}%
\pgfsys@useobject{currentmarker}{}%
\end{pgfscope}%
\begin{pgfscope}%
\pgfsys@transformshift{3.541419in}{1.599369in}%
\pgfsys@useobject{currentmarker}{}%
\end{pgfscope}%
\begin{pgfscope}%
\pgfsys@transformshift{3.705862in}{1.470071in}%
\pgfsys@useobject{currentmarker}{}%
\end{pgfscope}%
\begin{pgfscope}%
\pgfsys@transformshift{3.870342in}{1.322559in}%
\pgfsys@useobject{currentmarker}{}%
\end{pgfscope}%
\end{pgfscope}%
\begin{pgfscope}%
\pgfpathrectangle{\pgfqpoint{0.589510in}{0.417642in}}{\pgfqpoint{3.437062in}{2.055000in}}%
\pgfusepath{clip}%
\pgfsetbuttcap%
\pgfsetroundjoin%
\pgfsetlinewidth{1.505625pt}%
\definecolor{currentstroke}{rgb}{0.835294,0.368627,0.000000}%
\pgfsetstrokecolor{currentstroke}%
\pgfsetdash{{5.550000pt}{2.400000pt}}{0.000000pt}%
\pgfpathmoveto{\pgfqpoint{0.745740in}{0.913474in}}%
\pgfpathlineto{\pgfqpoint{0.883294in}{1.029862in}}%
\pgfpathlineto{\pgfqpoint{1.056591in}{1.175613in}}%
\pgfpathlineto{\pgfqpoint{1.216039in}{1.308344in}}%
\pgfpathlineto{\pgfqpoint{1.380747in}{1.443219in}}%
\pgfpathlineto{\pgfqpoint{1.559224in}{1.585203in}}%
\pgfpathlineto{\pgfqpoint{1.726296in}{1.711754in}}%
\pgfpathlineto{\pgfqpoint{1.893892in}{1.828794in}}%
\pgfpathlineto{\pgfqpoint{2.059042in}{1.929055in}}%
\pgfpathlineto{\pgfqpoint{2.223750in}{2.006719in}}%
\pgfpathlineto{\pgfqpoint{2.389127in}{2.052968in}}%
\pgfpathlineto{\pgfqpoint{2.554041in}{2.058337in}}%
\pgfpathlineto{\pgfqpoint{2.718686in}{2.018832in}}%
\pgfpathlineto{\pgfqpoint{2.883588in}{1.939550in}}%
\pgfpathlineto{\pgfqpoint{3.047937in}{1.833671in}}%
\pgfpathlineto{\pgfqpoint{3.212425in}{1.712667in}}%
\pgfpathlineto{\pgfqpoint{3.376951in}{1.583476in}}%
\pgfpathlineto{\pgfqpoint{3.541419in}{1.449716in}}%
\pgfpathlineto{\pgfqpoint{3.705862in}{1.313276in}}%
\pgfpathlineto{\pgfqpoint{3.870342in}{1.175191in}}%
\pgfusepath{stroke}%
\end{pgfscope}%
\begin{pgfscope}%
\pgfpathrectangle{\pgfqpoint{0.589510in}{0.417642in}}{\pgfqpoint{3.437062in}{2.055000in}}%
\pgfusepath{clip}%
\pgfsetbuttcap%
\pgfsetroundjoin%
\definecolor{currentfill}{rgb}{0.835294,0.368627,0.000000}%
\pgfsetfillcolor{currentfill}%
\pgfsetlinewidth{1.003750pt}%
\definecolor{currentstroke}{rgb}{0.835294,0.368627,0.000000}%
\pgfsetstrokecolor{currentstroke}%
\pgfsetdash{}{0pt}%
\pgfsys@defobject{currentmarker}{\pgfqpoint{-0.020833in}{-0.020833in}}{\pgfqpoint{0.020833in}{0.020833in}}{%
\pgfpathmoveto{\pgfqpoint{0.000000in}{-0.020833in}}%
\pgfpathcurveto{\pgfqpoint{0.005525in}{-0.020833in}}{\pgfqpoint{0.010825in}{-0.018638in}}{\pgfqpoint{0.014731in}{-0.014731in}}%
\pgfpathcurveto{\pgfqpoint{0.018638in}{-0.010825in}}{\pgfqpoint{0.020833in}{-0.005525in}}{\pgfqpoint{0.020833in}{0.000000in}}%
\pgfpathcurveto{\pgfqpoint{0.020833in}{0.005525in}}{\pgfqpoint{0.018638in}{0.010825in}}{\pgfqpoint{0.014731in}{0.014731in}}%
\pgfpathcurveto{\pgfqpoint{0.010825in}{0.018638in}}{\pgfqpoint{0.005525in}{0.020833in}}{\pgfqpoint{0.000000in}{0.020833in}}%
\pgfpathcurveto{\pgfqpoint{-0.005525in}{0.020833in}}{\pgfqpoint{-0.010825in}{0.018638in}}{\pgfqpoint{-0.014731in}{0.014731in}}%
\pgfpathcurveto{\pgfqpoint{-0.018638in}{0.010825in}}{\pgfqpoint{-0.020833in}{0.005525in}}{\pgfqpoint{-0.020833in}{0.000000in}}%
\pgfpathcurveto{\pgfqpoint{-0.020833in}{-0.005525in}}{\pgfqpoint{-0.018638in}{-0.010825in}}{\pgfqpoint{-0.014731in}{-0.014731in}}%
\pgfpathcurveto{\pgfqpoint{-0.010825in}{-0.018638in}}{\pgfqpoint{-0.005525in}{-0.020833in}}{\pgfqpoint{0.000000in}{-0.020833in}}%
\pgfpathlineto{\pgfqpoint{0.000000in}{-0.020833in}}%
\pgfpathclose%
\pgfusepath{stroke,fill}%
}%
\begin{pgfscope}%
\pgfsys@transformshift{0.745740in}{0.947471in}%
\pgfsys@useobject{currentmarker}{}%
\end{pgfscope}%
\begin{pgfscope}%
\pgfsys@transformshift{0.883294in}{1.045335in}%
\pgfsys@useobject{currentmarker}{}%
\end{pgfscope}%
\begin{pgfscope}%
\pgfsys@transformshift{1.056591in}{1.181110in}%
\pgfsys@useobject{currentmarker}{}%
\end{pgfscope}%
\begin{pgfscope}%
\pgfsys@transformshift{1.216039in}{1.310404in}%
\pgfsys@useobject{currentmarker}{}%
\end{pgfscope}%
\begin{pgfscope}%
\pgfsys@transformshift{1.380747in}{1.444139in}%
\pgfsys@useobject{currentmarker}{}%
\end{pgfscope}%
\begin{pgfscope}%
\pgfsys@transformshift{1.559224in}{1.585886in}%
\pgfsys@useobject{currentmarker}{}%
\end{pgfscope}%
\begin{pgfscope}%
\pgfsys@transformshift{1.726296in}{1.712219in}%
\pgfsys@useobject{currentmarker}{}%
\end{pgfscope}%
\begin{pgfscope}%
\pgfsys@transformshift{1.893892in}{1.828510in}%
\pgfsys@useobject{currentmarker}{}%
\end{pgfscope}%
\begin{pgfscope}%
\pgfsys@transformshift{2.059042in}{1.928187in}%
\pgfsys@useobject{currentmarker}{}%
\end{pgfscope}%
\begin{pgfscope}%
\pgfsys@transformshift{2.223750in}{2.005395in}%
\pgfsys@useobject{currentmarker}{}%
\end{pgfscope}%
\begin{pgfscope}%
\pgfsys@transformshift{2.389127in}{2.051102in}%
\pgfsys@useobject{currentmarker}{}%
\end{pgfscope}%
\begin{pgfscope}%
\pgfsys@transformshift{2.554041in}{2.055275in}%
\pgfsys@useobject{currentmarker}{}%
\end{pgfscope}%
\begin{pgfscope}%
\pgfsys@transformshift{2.718686in}{2.013974in}%
\pgfsys@useobject{currentmarker}{}%
\end{pgfscope}%
\begin{pgfscope}%
\pgfsys@transformshift{2.883588in}{1.934223in}%
\pgfsys@useobject{currentmarker}{}%
\end{pgfscope}%
\begin{pgfscope}%
\pgfsys@transformshift{3.047937in}{1.829737in}%
\pgfsys@useobject{currentmarker}{}%
\end{pgfscope}%
\begin{pgfscope}%
\pgfsys@transformshift{3.212425in}{1.710686in}%
\pgfsys@useobject{currentmarker}{}%
\end{pgfscope}%
\begin{pgfscope}%
\pgfsys@transformshift{3.376951in}{1.583962in}%
\pgfsys@useobject{currentmarker}{}%
\end{pgfscope}%
\begin{pgfscope}%
\pgfsys@transformshift{3.541419in}{1.447015in}%
\pgfsys@useobject{currentmarker}{}%
\end{pgfscope}%
\begin{pgfscope}%
\pgfsys@transformshift{3.705862in}{1.301664in}%
\pgfsys@useobject{currentmarker}{}%
\end{pgfscope}%
\begin{pgfscope}%
\pgfsys@transformshift{3.870342in}{1.156630in}%
\pgfsys@useobject{currentmarker}{}%
\end{pgfscope}%
\end{pgfscope}%
\begin{pgfscope}%
\pgfsetrectcap%
\pgfsetmiterjoin%
\pgfsetlinewidth{0.803000pt}%
\definecolor{currentstroke}{rgb}{0.000000,0.000000,0.000000}%
\pgfsetstrokecolor{currentstroke}%
\pgfsetdash{}{0pt}%
\pgfpathmoveto{\pgfqpoint{0.589510in}{0.417642in}}%
\pgfpathlineto{\pgfqpoint{0.589510in}{2.472642in}}%
\pgfusepath{stroke}%
\end{pgfscope}%
\begin{pgfscope}%
\pgfsetrectcap%
\pgfsetmiterjoin%
\pgfsetlinewidth{0.803000pt}%
\definecolor{currentstroke}{rgb}{0.000000,0.000000,0.000000}%
\pgfsetstrokecolor{currentstroke}%
\pgfsetdash{}{0pt}%
\pgfpathmoveto{\pgfqpoint{4.026572in}{0.417642in}}%
\pgfpathlineto{\pgfqpoint{4.026572in}{2.472642in}}%
\pgfusepath{stroke}%
\end{pgfscope}%
\begin{pgfscope}%
\pgfsetrectcap%
\pgfsetmiterjoin%
\pgfsetlinewidth{0.803000pt}%
\definecolor{currentstroke}{rgb}{0.000000,0.000000,0.000000}%
\pgfsetstrokecolor{currentstroke}%
\pgfsetdash{}{0pt}%
\pgfpathmoveto{\pgfqpoint{0.589510in}{0.417642in}}%
\pgfpathlineto{\pgfqpoint{4.026572in}{0.417642in}}%
\pgfusepath{stroke}%
\end{pgfscope}%
\begin{pgfscope}%
\pgfsetrectcap%
\pgfsetmiterjoin%
\pgfsetlinewidth{0.803000pt}%
\definecolor{currentstroke}{rgb}{0.000000,0.000000,0.000000}%
\pgfsetstrokecolor{currentstroke}%
\pgfsetdash{}{0pt}%
\pgfpathmoveto{\pgfqpoint{0.589510in}{2.472642in}}%
\pgfpathlineto{\pgfqpoint{4.026572in}{2.472642in}}%
\pgfusepath{stroke}%
\end{pgfscope}%
\begin{pgfscope}%
\pgfsetbuttcap%
\pgfsetmiterjoin%
\definecolor{currentfill}{rgb}{1.000000,1.000000,1.000000}%
\pgfsetfillcolor{currentfill}%
\pgfsetfillopacity{0.800000}%
\pgfsetlinewidth{1.003750pt}%
\definecolor{currentstroke}{rgb}{0.800000,0.800000,0.800000}%
\pgfsetstrokecolor{currentstroke}%
\pgfsetstrokeopacity{0.800000}%
\pgfsetdash{}{0pt}%
\pgfpathmoveto{\pgfqpoint{3.108484in}{1.919086in}}%
\pgfpathlineto{\pgfqpoint{3.948794in}{1.919086in}}%
\pgfpathquadraticcurveto{\pgfqpoint{3.971016in}{1.919086in}}{\pgfqpoint{3.971016in}{1.941309in}}%
\pgfpathlineto{\pgfqpoint{3.971016in}{2.394864in}}%
\pgfpathquadraticcurveto{\pgfqpoint{3.971016in}{2.417086in}}{\pgfqpoint{3.948794in}{2.417086in}}%
\pgfpathlineto{\pgfqpoint{3.108484in}{2.417086in}}%
\pgfpathquadraticcurveto{\pgfqpoint{3.086261in}{2.417086in}}{\pgfqpoint{3.086261in}{2.394864in}}%
\pgfpathlineto{\pgfqpoint{3.086261in}{1.941309in}}%
\pgfpathquadraticcurveto{\pgfqpoint{3.086261in}{1.919086in}}{\pgfqpoint{3.108484in}{1.919086in}}%
\pgfpathlineto{\pgfqpoint{3.108484in}{1.919086in}}%
\pgfpathclose%
\pgfusepath{stroke,fill}%
\end{pgfscope}%
\begin{pgfscope}%
\pgfsetbuttcap%
\pgfsetroundjoin%
\definecolor{currentfill}{rgb}{0.003922,0.450980,0.698039}%
\pgfsetfillcolor{currentfill}%
\pgfsetlinewidth{1.003750pt}%
\definecolor{currentstroke}{rgb}{0.003922,0.450980,0.698039}%
\pgfsetstrokecolor{currentstroke}%
\pgfsetdash{}{0pt}%
\pgfsys@defobject{currentmarker}{\pgfqpoint{-0.020833in}{-0.020833in}}{\pgfqpoint{0.020833in}{0.020833in}}{%
\pgfpathmoveto{\pgfqpoint{0.000000in}{-0.020833in}}%
\pgfpathcurveto{\pgfqpoint{0.005525in}{-0.020833in}}{\pgfqpoint{0.010825in}{-0.018638in}}{\pgfqpoint{0.014731in}{-0.014731in}}%
\pgfpathcurveto{\pgfqpoint{0.018638in}{-0.010825in}}{\pgfqpoint{0.020833in}{-0.005525in}}{\pgfqpoint{0.020833in}{0.000000in}}%
\pgfpathcurveto{\pgfqpoint{0.020833in}{0.005525in}}{\pgfqpoint{0.018638in}{0.010825in}}{\pgfqpoint{0.014731in}{0.014731in}}%
\pgfpathcurveto{\pgfqpoint{0.010825in}{0.018638in}}{\pgfqpoint{0.005525in}{0.020833in}}{\pgfqpoint{0.000000in}{0.020833in}}%
\pgfpathcurveto{\pgfqpoint{-0.005525in}{0.020833in}}{\pgfqpoint{-0.010825in}{0.018638in}}{\pgfqpoint{-0.014731in}{0.014731in}}%
\pgfpathcurveto{\pgfqpoint{-0.018638in}{0.010825in}}{\pgfqpoint{-0.020833in}{0.005525in}}{\pgfqpoint{-0.020833in}{0.000000in}}%
\pgfpathcurveto{\pgfqpoint{-0.020833in}{-0.005525in}}{\pgfqpoint{-0.018638in}{-0.010825in}}{\pgfqpoint{-0.014731in}{-0.014731in}}%
\pgfpathcurveto{\pgfqpoint{-0.010825in}{-0.018638in}}{\pgfqpoint{-0.005525in}{-0.020833in}}{\pgfqpoint{0.000000in}{-0.020833in}}%
\pgfpathlineto{\pgfqpoint{0.000000in}{-0.020833in}}%
\pgfpathclose%
\pgfusepath{stroke,fill}%
}%
\begin{pgfscope}%
\pgfsys@transformshift{3.241817in}{2.333753in}%
\pgfsys@useobject{currentmarker}{}%
\end{pgfscope}%
\end{pgfscope}%
\begin{pgfscope}%
\definecolor{textcolor}{rgb}{0.000000,0.000000,0.000000}%
\pgfsetstrokecolor{textcolor}%
\pgfsetfillcolor{textcolor}%
\pgftext[x=3.441817in,y=2.294864in,left,base]{\color{textcolor}\rmfamily\fontsize{8.000000}{9.600000}\selectfont \(\displaystyle \bar\tau_1=\qty{0.1}{\s}\)}%
\end{pgfscope}%
\begin{pgfscope}%
\pgfsetbuttcap%
\pgfsetroundjoin%
\definecolor{currentfill}{rgb}{0.007843,0.619608,0.450980}%
\pgfsetfillcolor{currentfill}%
\pgfsetlinewidth{1.003750pt}%
\definecolor{currentstroke}{rgb}{0.007843,0.619608,0.450980}%
\pgfsetstrokecolor{currentstroke}%
\pgfsetdash{}{0pt}%
\pgfsys@defobject{currentmarker}{\pgfqpoint{-0.020833in}{-0.020833in}}{\pgfqpoint{0.020833in}{0.020833in}}{%
\pgfpathmoveto{\pgfqpoint{0.000000in}{-0.020833in}}%
\pgfpathcurveto{\pgfqpoint{0.005525in}{-0.020833in}}{\pgfqpoint{0.010825in}{-0.018638in}}{\pgfqpoint{0.014731in}{-0.014731in}}%
\pgfpathcurveto{\pgfqpoint{0.018638in}{-0.010825in}}{\pgfqpoint{0.020833in}{-0.005525in}}{\pgfqpoint{0.020833in}{0.000000in}}%
\pgfpathcurveto{\pgfqpoint{0.020833in}{0.005525in}}{\pgfqpoint{0.018638in}{0.010825in}}{\pgfqpoint{0.014731in}{0.014731in}}%
\pgfpathcurveto{\pgfqpoint{0.010825in}{0.018638in}}{\pgfqpoint{0.005525in}{0.020833in}}{\pgfqpoint{0.000000in}{0.020833in}}%
\pgfpathcurveto{\pgfqpoint{-0.005525in}{0.020833in}}{\pgfqpoint{-0.010825in}{0.018638in}}{\pgfqpoint{-0.014731in}{0.014731in}}%
\pgfpathcurveto{\pgfqpoint{-0.018638in}{0.010825in}}{\pgfqpoint{-0.020833in}{0.005525in}}{\pgfqpoint{-0.020833in}{0.000000in}}%
\pgfpathcurveto{\pgfqpoint{-0.020833in}{-0.005525in}}{\pgfqpoint{-0.018638in}{-0.010825in}}{\pgfqpoint{-0.014731in}{-0.014731in}}%
\pgfpathcurveto{\pgfqpoint{-0.010825in}{-0.018638in}}{\pgfqpoint{-0.005525in}{-0.020833in}}{\pgfqpoint{0.000000in}{-0.020833in}}%
\pgfpathlineto{\pgfqpoint{0.000000in}{-0.020833in}}%
\pgfpathclose%
\pgfusepath{stroke,fill}%
}%
\begin{pgfscope}%
\pgfsys@transformshift{3.241817in}{2.178864in}%
\pgfsys@useobject{currentmarker}{}%
\end{pgfscope}%
\end{pgfscope}%
\begin{pgfscope}%
\definecolor{textcolor}{rgb}{0.000000,0.000000,0.000000}%
\pgfsetstrokecolor{textcolor}%
\pgfsetfillcolor{textcolor}%
\pgftext[x=3.441817in,y=2.139975in,left,base]{\color{textcolor}\rmfamily\fontsize{8.000000}{9.600000}\selectfont \(\displaystyle \bar\tau_1=\qty{1}{\s}\)}%
\end{pgfscope}%
\begin{pgfscope}%
\pgfsetbuttcap%
\pgfsetroundjoin%
\definecolor{currentfill}{rgb}{0.835294,0.368627,0.000000}%
\pgfsetfillcolor{currentfill}%
\pgfsetlinewidth{1.003750pt}%
\definecolor{currentstroke}{rgb}{0.835294,0.368627,0.000000}%
\pgfsetstrokecolor{currentstroke}%
\pgfsetdash{}{0pt}%
\pgfsys@defobject{currentmarker}{\pgfqpoint{-0.020833in}{-0.020833in}}{\pgfqpoint{0.020833in}{0.020833in}}{%
\pgfpathmoveto{\pgfqpoint{0.000000in}{-0.020833in}}%
\pgfpathcurveto{\pgfqpoint{0.005525in}{-0.020833in}}{\pgfqpoint{0.010825in}{-0.018638in}}{\pgfqpoint{0.014731in}{-0.014731in}}%
\pgfpathcurveto{\pgfqpoint{0.018638in}{-0.010825in}}{\pgfqpoint{0.020833in}{-0.005525in}}{\pgfqpoint{0.020833in}{0.000000in}}%
\pgfpathcurveto{\pgfqpoint{0.020833in}{0.005525in}}{\pgfqpoint{0.018638in}{0.010825in}}{\pgfqpoint{0.014731in}{0.014731in}}%
\pgfpathcurveto{\pgfqpoint{0.010825in}{0.018638in}}{\pgfqpoint{0.005525in}{0.020833in}}{\pgfqpoint{0.000000in}{0.020833in}}%
\pgfpathcurveto{\pgfqpoint{-0.005525in}{0.020833in}}{\pgfqpoint{-0.010825in}{0.018638in}}{\pgfqpoint{-0.014731in}{0.014731in}}%
\pgfpathcurveto{\pgfqpoint{-0.018638in}{0.010825in}}{\pgfqpoint{-0.020833in}{0.005525in}}{\pgfqpoint{-0.020833in}{0.000000in}}%
\pgfpathcurveto{\pgfqpoint{-0.020833in}{-0.005525in}}{\pgfqpoint{-0.018638in}{-0.010825in}}{\pgfqpoint{-0.014731in}{-0.014731in}}%
\pgfpathcurveto{\pgfqpoint{-0.010825in}{-0.018638in}}{\pgfqpoint{-0.005525in}{-0.020833in}}{\pgfqpoint{0.000000in}{-0.020833in}}%
\pgfpathlineto{\pgfqpoint{0.000000in}{-0.020833in}}%
\pgfpathclose%
\pgfusepath{stroke,fill}%
}%
\begin{pgfscope}%
\pgfsys@transformshift{3.241817in}{2.023975in}%
\pgfsys@useobject{currentmarker}{}%
\end{pgfscope}%
\end{pgfscope}%
\begin{pgfscope}%
\definecolor{textcolor}{rgb}{0.000000,0.000000,0.000000}%
\pgfsetstrokecolor{textcolor}%
\pgfsetfillcolor{textcolor}%
\pgftext[x=3.441817in,y=1.985086in,left,base]{\color{textcolor}\rmfamily\fontsize{8.000000}{9.600000}\selectfont \(\displaystyle \bar\tau_1=\qty{10}{\s}\)}%
\end{pgfscope}%
\end{pgfpicture}%
\makeatother%
\endgroup%

        } % scalebox
        \caption{Allan deviation}
        \label{fig:burst_noise_adev}
    \end{subfigure}
    \caption{Different representations of burst noise for different $\bar \tau_1$ and fixed $\bar \tau_0 = \qty{1}{\s}$.}
    \label{fig:burst_noise_simulated}
\end{figure}

The burst noise equations can used to gain further insight into other types of noise. The first one is Shot noise, which is commonly found in photodetectors and lasers. Here, electrons or photons are created at discrete intervals resulting in an instantationous signal. This means, that the lifetime of the upper level is very short in comparison to the lower level ($\tau_1 \ll \tau_0$) equation \ref{eqn:burst_noise_psd} becomes:
\begin{align}
    S_{Shot}(\omega) = S_{\tau_1 \ll \tau_0}(\omega) &= 4 \Delta y^2 \frac{\tau_1}{\tau_0} \frac{\frac{1}{\bar \tau_1}}{\left(\frac{1}{\bar \tau_1}\right)^2 + \omega^2}\nonumber\\
    &= 4 \Delta y^2 \frac{1}{\tau_0} \frac{1}{\frac{1}{\tau_1^2}+\omega^2}\\
    \overset{\omega \ll 1/\tau_0}&{\approx} 4 \Delta y^2 \frac{\tau_1^2}{\tau_0} = \text{const.}
\end{align}

For the typical case, a very large number of such events happen. When not counting single events, but rather a stream, the relation $\omega \ll 1/\tau_0$ is valid and hence the results is a white spectrum as $S_{Shot}(\omega)$ is constant with respect to $\omega$ --- just as observed in photodetectors and lasers.

The other interesting case is a case, where many trap sites with different time constants are contributing to the noise. This can change the shape of the spectrum from $f^{-2}$ to $f^{-1}$ and is discussed in the next section.

\clearpage
\minisec{Flicker Noise}
Flicker noise is also called $\frac 1 f$-noise and it can be observed in many naturally occuring phenomenen. Its origin is not clear, although there have been many explanations. An overview can be found in \cite{flicker_noise_overview, flicker_noise_overview2, origins_1_f_noise}. This work concentrates on flicker noise in electronic devices. In thick-film resistors, for example, it was shown to extend over at least 6 decades without any visible flattening \cite{1_f_noise_thick_film}. In transistors, flicker noise is caused by the existance of generation-recombination noise or burst noise discussed in the previous section \cite{origins_1_f_noise}. If there are many uncorrelated trap sites, that contribute to the total noise, the envelope of the noise spectral density changes from $\frac{1}{f^2}$ to $\frac{1}{f^1}$ as shown in figure \ref{fig:flicker_noise_evelope}

\begin{figure}[hb]
    \centering
    %% Creator: Matplotlib, PGF backend
%%
%% To include the figure in your LaTeX document, write
%%   \input{<filename>.pgf}
%%
%% Make sure the required packages are loaded in your preamble
%%   \usepackage{pgf}
%%
%% Also ensure that all the required font packages are loaded; for instance,
%% the lmodern package is sometimes necessary when using math font.
%%   \usepackage{lmodern}
%%
%% Figures using additional raster images can only be included by \input if
%% they are in the same directory as the main LaTeX file. For loading figures
%% from other directories you can use the `import` package
%%   \usepackage{import}
%%
%% and then include the figures with
%%   \import{<path to file>}{<filename>.pgf}
%%
%% Matplotlib used the following preamble
%%   \usepackage{siunitx}
%%   \usepackage{fontspec}
%%   \makeatletter\@ifpackageloaded{underscore}{}{\usepackage[strings]{underscore}}\makeatother
%%
\begingroup%
\makeatletter%
\begin{pgfpicture}%
\pgfpathrectangle{\pgfpointorigin}{\pgfqpoint{4.060000in}{2.510000in}}%
\pgfusepath{use as bounding box, clip}%
\begin{pgfscope}%
\pgfsetbuttcap%
\pgfsetmiterjoin%
\definecolor{currentfill}{rgb}{1.000000,1.000000,1.000000}%
\pgfsetfillcolor{currentfill}%
\pgfsetlinewidth{0.000000pt}%
\definecolor{currentstroke}{rgb}{1.000000,1.000000,1.000000}%
\pgfsetstrokecolor{currentstroke}%
\pgfsetdash{}{0pt}%
\pgfpathmoveto{\pgfqpoint{0.000000in}{0.000000in}}%
\pgfpathlineto{\pgfqpoint{4.060000in}{0.000000in}}%
\pgfpathlineto{\pgfqpoint{4.060000in}{2.510000in}}%
\pgfpathlineto{\pgfqpoint{0.000000in}{2.510000in}}%
\pgfpathlineto{\pgfqpoint{0.000000in}{0.000000in}}%
\pgfpathclose%
\pgfusepath{fill}%
\end{pgfscope}%
\begin{pgfscope}%
\pgfsetbuttcap%
\pgfsetmiterjoin%
\definecolor{currentfill}{rgb}{1.000000,1.000000,1.000000}%
\pgfsetfillcolor{currentfill}%
\pgfsetlinewidth{0.000000pt}%
\definecolor{currentstroke}{rgb}{0.000000,0.000000,0.000000}%
\pgfsetstrokecolor{currentstroke}%
\pgfsetstrokeopacity{0.000000}%
\pgfsetdash{}{0pt}%
\pgfpathmoveto{\pgfqpoint{0.594525in}{0.417642in}}%
\pgfpathlineto{\pgfqpoint{4.018330in}{0.417642in}}%
\pgfpathlineto{\pgfqpoint{4.018330in}{2.429177in}}%
\pgfpathlineto{\pgfqpoint{0.594525in}{2.429177in}}%
\pgfpathlineto{\pgfqpoint{0.594525in}{0.417642in}}%
\pgfpathclose%
\pgfusepath{fill}%
\end{pgfscope}%
\begin{pgfscope}%
\pgfpathrectangle{\pgfqpoint{0.594525in}{0.417642in}}{\pgfqpoint{3.423805in}{2.011535in}}%
\pgfusepath{clip}%
\pgfsetrectcap%
\pgfsetroundjoin%
\pgfsetlinewidth{0.803000pt}%
\definecolor{currentstroke}{rgb}{0.450000,0.450000,0.450000}%
\pgfsetstrokecolor{currentstroke}%
\pgfsetdash{}{0pt}%
\pgfpathmoveto{\pgfqpoint{0.750152in}{0.417642in}}%
\pgfpathlineto{\pgfqpoint{0.750152in}{2.429177in}}%
\pgfusepath{stroke}%
\end{pgfscope}%
\begin{pgfscope}%
\pgfsetbuttcap%
\pgfsetroundjoin%
\definecolor{currentfill}{rgb}{0.000000,0.000000,0.000000}%
\pgfsetfillcolor{currentfill}%
\pgfsetlinewidth{0.803000pt}%
\definecolor{currentstroke}{rgb}{0.000000,0.000000,0.000000}%
\pgfsetstrokecolor{currentstroke}%
\pgfsetdash{}{0pt}%
\pgfsys@defobject{currentmarker}{\pgfqpoint{0.000000in}{-0.048611in}}{\pgfqpoint{0.000000in}{0.000000in}}{%
\pgfpathmoveto{\pgfqpoint{0.000000in}{0.000000in}}%
\pgfpathlineto{\pgfqpoint{0.000000in}{-0.048611in}}%
\pgfusepath{stroke,fill}%
}%
\begin{pgfscope}%
\pgfsys@transformshift{0.750152in}{0.417642in}%
\pgfsys@useobject{currentmarker}{}%
\end{pgfscope}%
\end{pgfscope}%
\begin{pgfscope}%
\definecolor{textcolor}{rgb}{0.000000,0.000000,0.000000}%
\pgfsetstrokecolor{textcolor}%
\pgfsetfillcolor{textcolor}%
\pgftext[x=0.750152in,y=0.320420in,,top]{\color{textcolor}\rmfamily\fontsize{8.000000}{9.600000}\selectfont \(\displaystyle {10^{-2}}\)}%
\end{pgfscope}%
\begin{pgfscope}%
\pgfpathrectangle{\pgfqpoint{0.594525in}{0.417642in}}{\pgfqpoint{3.423805in}{2.011535in}}%
\pgfusepath{clip}%
\pgfsetrectcap%
\pgfsetroundjoin%
\pgfsetlinewidth{0.803000pt}%
\definecolor{currentstroke}{rgb}{0.450000,0.450000,0.450000}%
\pgfsetstrokecolor{currentstroke}%
\pgfsetdash{}{0pt}%
\pgfpathmoveto{\pgfqpoint{1.528290in}{0.417642in}}%
\pgfpathlineto{\pgfqpoint{1.528290in}{2.429177in}}%
\pgfusepath{stroke}%
\end{pgfscope}%
\begin{pgfscope}%
\pgfsetbuttcap%
\pgfsetroundjoin%
\definecolor{currentfill}{rgb}{0.000000,0.000000,0.000000}%
\pgfsetfillcolor{currentfill}%
\pgfsetlinewidth{0.803000pt}%
\definecolor{currentstroke}{rgb}{0.000000,0.000000,0.000000}%
\pgfsetstrokecolor{currentstroke}%
\pgfsetdash{}{0pt}%
\pgfsys@defobject{currentmarker}{\pgfqpoint{0.000000in}{-0.048611in}}{\pgfqpoint{0.000000in}{0.000000in}}{%
\pgfpathmoveto{\pgfqpoint{0.000000in}{0.000000in}}%
\pgfpathlineto{\pgfqpoint{0.000000in}{-0.048611in}}%
\pgfusepath{stroke,fill}%
}%
\begin{pgfscope}%
\pgfsys@transformshift{1.528290in}{0.417642in}%
\pgfsys@useobject{currentmarker}{}%
\end{pgfscope}%
\end{pgfscope}%
\begin{pgfscope}%
\definecolor{textcolor}{rgb}{0.000000,0.000000,0.000000}%
\pgfsetstrokecolor{textcolor}%
\pgfsetfillcolor{textcolor}%
\pgftext[x=1.528290in,y=0.320420in,,top]{\color{textcolor}\rmfamily\fontsize{8.000000}{9.600000}\selectfont \(\displaystyle {10^{-1}}\)}%
\end{pgfscope}%
\begin{pgfscope}%
\pgfpathrectangle{\pgfqpoint{0.594525in}{0.417642in}}{\pgfqpoint{3.423805in}{2.011535in}}%
\pgfusepath{clip}%
\pgfsetrectcap%
\pgfsetroundjoin%
\pgfsetlinewidth{0.803000pt}%
\definecolor{currentstroke}{rgb}{0.450000,0.450000,0.450000}%
\pgfsetstrokecolor{currentstroke}%
\pgfsetdash{}{0pt}%
\pgfpathmoveto{\pgfqpoint{2.306427in}{0.417642in}}%
\pgfpathlineto{\pgfqpoint{2.306427in}{2.429177in}}%
\pgfusepath{stroke}%
\end{pgfscope}%
\begin{pgfscope}%
\pgfsetbuttcap%
\pgfsetroundjoin%
\definecolor{currentfill}{rgb}{0.000000,0.000000,0.000000}%
\pgfsetfillcolor{currentfill}%
\pgfsetlinewidth{0.803000pt}%
\definecolor{currentstroke}{rgb}{0.000000,0.000000,0.000000}%
\pgfsetstrokecolor{currentstroke}%
\pgfsetdash{}{0pt}%
\pgfsys@defobject{currentmarker}{\pgfqpoint{0.000000in}{-0.048611in}}{\pgfqpoint{0.000000in}{0.000000in}}{%
\pgfpathmoveto{\pgfqpoint{0.000000in}{0.000000in}}%
\pgfpathlineto{\pgfqpoint{0.000000in}{-0.048611in}}%
\pgfusepath{stroke,fill}%
}%
\begin{pgfscope}%
\pgfsys@transformshift{2.306427in}{0.417642in}%
\pgfsys@useobject{currentmarker}{}%
\end{pgfscope}%
\end{pgfscope}%
\begin{pgfscope}%
\definecolor{textcolor}{rgb}{0.000000,0.000000,0.000000}%
\pgfsetstrokecolor{textcolor}%
\pgfsetfillcolor{textcolor}%
\pgftext[x=2.306427in,y=0.320420in,,top]{\color{textcolor}\rmfamily\fontsize{8.000000}{9.600000}\selectfont \(\displaystyle {10^{0}}\)}%
\end{pgfscope}%
\begin{pgfscope}%
\pgfpathrectangle{\pgfqpoint{0.594525in}{0.417642in}}{\pgfqpoint{3.423805in}{2.011535in}}%
\pgfusepath{clip}%
\pgfsetrectcap%
\pgfsetroundjoin%
\pgfsetlinewidth{0.803000pt}%
\definecolor{currentstroke}{rgb}{0.450000,0.450000,0.450000}%
\pgfsetstrokecolor{currentstroke}%
\pgfsetdash{}{0pt}%
\pgfpathmoveto{\pgfqpoint{3.084565in}{0.417642in}}%
\pgfpathlineto{\pgfqpoint{3.084565in}{2.429177in}}%
\pgfusepath{stroke}%
\end{pgfscope}%
\begin{pgfscope}%
\pgfsetbuttcap%
\pgfsetroundjoin%
\definecolor{currentfill}{rgb}{0.000000,0.000000,0.000000}%
\pgfsetfillcolor{currentfill}%
\pgfsetlinewidth{0.803000pt}%
\definecolor{currentstroke}{rgb}{0.000000,0.000000,0.000000}%
\pgfsetstrokecolor{currentstroke}%
\pgfsetdash{}{0pt}%
\pgfsys@defobject{currentmarker}{\pgfqpoint{0.000000in}{-0.048611in}}{\pgfqpoint{0.000000in}{0.000000in}}{%
\pgfpathmoveto{\pgfqpoint{0.000000in}{0.000000in}}%
\pgfpathlineto{\pgfqpoint{0.000000in}{-0.048611in}}%
\pgfusepath{stroke,fill}%
}%
\begin{pgfscope}%
\pgfsys@transformshift{3.084565in}{0.417642in}%
\pgfsys@useobject{currentmarker}{}%
\end{pgfscope}%
\end{pgfscope}%
\begin{pgfscope}%
\definecolor{textcolor}{rgb}{0.000000,0.000000,0.000000}%
\pgfsetstrokecolor{textcolor}%
\pgfsetfillcolor{textcolor}%
\pgftext[x=3.084565in,y=0.320420in,,top]{\color{textcolor}\rmfamily\fontsize{8.000000}{9.600000}\selectfont \(\displaystyle {10^{1}}\)}%
\end{pgfscope}%
\begin{pgfscope}%
\pgfpathrectangle{\pgfqpoint{0.594525in}{0.417642in}}{\pgfqpoint{3.423805in}{2.011535in}}%
\pgfusepath{clip}%
\pgfsetrectcap%
\pgfsetroundjoin%
\pgfsetlinewidth{0.803000pt}%
\definecolor{currentstroke}{rgb}{0.450000,0.450000,0.450000}%
\pgfsetstrokecolor{currentstroke}%
\pgfsetdash{}{0pt}%
\pgfpathmoveto{\pgfqpoint{3.862702in}{0.417642in}}%
\pgfpathlineto{\pgfqpoint{3.862702in}{2.429177in}}%
\pgfusepath{stroke}%
\end{pgfscope}%
\begin{pgfscope}%
\pgfsetbuttcap%
\pgfsetroundjoin%
\definecolor{currentfill}{rgb}{0.000000,0.000000,0.000000}%
\pgfsetfillcolor{currentfill}%
\pgfsetlinewidth{0.803000pt}%
\definecolor{currentstroke}{rgb}{0.000000,0.000000,0.000000}%
\pgfsetstrokecolor{currentstroke}%
\pgfsetdash{}{0pt}%
\pgfsys@defobject{currentmarker}{\pgfqpoint{0.000000in}{-0.048611in}}{\pgfqpoint{0.000000in}{0.000000in}}{%
\pgfpathmoveto{\pgfqpoint{0.000000in}{0.000000in}}%
\pgfpathlineto{\pgfqpoint{0.000000in}{-0.048611in}}%
\pgfusepath{stroke,fill}%
}%
\begin{pgfscope}%
\pgfsys@transformshift{3.862702in}{0.417642in}%
\pgfsys@useobject{currentmarker}{}%
\end{pgfscope}%
\end{pgfscope}%
\begin{pgfscope}%
\definecolor{textcolor}{rgb}{0.000000,0.000000,0.000000}%
\pgfsetstrokecolor{textcolor}%
\pgfsetfillcolor{textcolor}%
\pgftext[x=3.862702in,y=0.320420in,,top]{\color{textcolor}\rmfamily\fontsize{8.000000}{9.600000}\selectfont \(\displaystyle {10^{2}}\)}%
\end{pgfscope}%
\begin{pgfscope}%
\pgfpathrectangle{\pgfqpoint{0.594525in}{0.417642in}}{\pgfqpoint{3.423805in}{2.011535in}}%
\pgfusepath{clip}%
\pgfsetrectcap%
\pgfsetroundjoin%
\pgfsetlinewidth{0.803000pt}%
\definecolor{currentstroke}{rgb}{0.850000,0.850000,0.850000}%
\pgfsetstrokecolor{currentstroke}%
\pgfsetdash{}{0pt}%
\pgfpathmoveto{\pgfqpoint{0.629617in}{0.417642in}}%
\pgfpathlineto{\pgfqpoint{0.629617in}{2.429177in}}%
\pgfusepath{stroke}%
\end{pgfscope}%
\begin{pgfscope}%
\pgfsetbuttcap%
\pgfsetroundjoin%
\definecolor{currentfill}{rgb}{0.000000,0.000000,0.000000}%
\pgfsetfillcolor{currentfill}%
\pgfsetlinewidth{0.602250pt}%
\definecolor{currentstroke}{rgb}{0.000000,0.000000,0.000000}%
\pgfsetstrokecolor{currentstroke}%
\pgfsetdash{}{0pt}%
\pgfsys@defobject{currentmarker}{\pgfqpoint{0.000000in}{-0.027778in}}{\pgfqpoint{0.000000in}{0.000000in}}{%
\pgfpathmoveto{\pgfqpoint{0.000000in}{0.000000in}}%
\pgfpathlineto{\pgfqpoint{0.000000in}{-0.027778in}}%
\pgfusepath{stroke,fill}%
}%
\begin{pgfscope}%
\pgfsys@transformshift{0.629617in}{0.417642in}%
\pgfsys@useobject{currentmarker}{}%
\end{pgfscope}%
\end{pgfscope}%
\begin{pgfscope}%
\pgfpathrectangle{\pgfqpoint{0.594525in}{0.417642in}}{\pgfqpoint{3.423805in}{2.011535in}}%
\pgfusepath{clip}%
\pgfsetrectcap%
\pgfsetroundjoin%
\pgfsetlinewidth{0.803000pt}%
\definecolor{currentstroke}{rgb}{0.850000,0.850000,0.850000}%
\pgfsetstrokecolor{currentstroke}%
\pgfsetdash{}{0pt}%
\pgfpathmoveto{\pgfqpoint{0.674743in}{0.417642in}}%
\pgfpathlineto{\pgfqpoint{0.674743in}{2.429177in}}%
\pgfusepath{stroke}%
\end{pgfscope}%
\begin{pgfscope}%
\pgfsetbuttcap%
\pgfsetroundjoin%
\definecolor{currentfill}{rgb}{0.000000,0.000000,0.000000}%
\pgfsetfillcolor{currentfill}%
\pgfsetlinewidth{0.602250pt}%
\definecolor{currentstroke}{rgb}{0.000000,0.000000,0.000000}%
\pgfsetstrokecolor{currentstroke}%
\pgfsetdash{}{0pt}%
\pgfsys@defobject{currentmarker}{\pgfqpoint{0.000000in}{-0.027778in}}{\pgfqpoint{0.000000in}{0.000000in}}{%
\pgfpathmoveto{\pgfqpoint{0.000000in}{0.000000in}}%
\pgfpathlineto{\pgfqpoint{0.000000in}{-0.027778in}}%
\pgfusepath{stroke,fill}%
}%
\begin{pgfscope}%
\pgfsys@transformshift{0.674743in}{0.417642in}%
\pgfsys@useobject{currentmarker}{}%
\end{pgfscope}%
\end{pgfscope}%
\begin{pgfscope}%
\pgfpathrectangle{\pgfqpoint{0.594525in}{0.417642in}}{\pgfqpoint{3.423805in}{2.011535in}}%
\pgfusepath{clip}%
\pgfsetrectcap%
\pgfsetroundjoin%
\pgfsetlinewidth{0.803000pt}%
\definecolor{currentstroke}{rgb}{0.850000,0.850000,0.850000}%
\pgfsetstrokecolor{currentstroke}%
\pgfsetdash{}{0pt}%
\pgfpathmoveto{\pgfqpoint{0.714547in}{0.417642in}}%
\pgfpathlineto{\pgfqpoint{0.714547in}{2.429177in}}%
\pgfusepath{stroke}%
\end{pgfscope}%
\begin{pgfscope}%
\pgfsetbuttcap%
\pgfsetroundjoin%
\definecolor{currentfill}{rgb}{0.000000,0.000000,0.000000}%
\pgfsetfillcolor{currentfill}%
\pgfsetlinewidth{0.602250pt}%
\definecolor{currentstroke}{rgb}{0.000000,0.000000,0.000000}%
\pgfsetstrokecolor{currentstroke}%
\pgfsetdash{}{0pt}%
\pgfsys@defobject{currentmarker}{\pgfqpoint{0.000000in}{-0.027778in}}{\pgfqpoint{0.000000in}{0.000000in}}{%
\pgfpathmoveto{\pgfqpoint{0.000000in}{0.000000in}}%
\pgfpathlineto{\pgfqpoint{0.000000in}{-0.027778in}}%
\pgfusepath{stroke,fill}%
}%
\begin{pgfscope}%
\pgfsys@transformshift{0.714547in}{0.417642in}%
\pgfsys@useobject{currentmarker}{}%
\end{pgfscope}%
\end{pgfscope}%
\begin{pgfscope}%
\pgfpathrectangle{\pgfqpoint{0.594525in}{0.417642in}}{\pgfqpoint{3.423805in}{2.011535in}}%
\pgfusepath{clip}%
\pgfsetrectcap%
\pgfsetroundjoin%
\pgfsetlinewidth{0.803000pt}%
\definecolor{currentstroke}{rgb}{0.850000,0.850000,0.850000}%
\pgfsetstrokecolor{currentstroke}%
\pgfsetdash{}{0pt}%
\pgfpathmoveto{\pgfqpoint{0.984395in}{0.417642in}}%
\pgfpathlineto{\pgfqpoint{0.984395in}{2.429177in}}%
\pgfusepath{stroke}%
\end{pgfscope}%
\begin{pgfscope}%
\pgfsetbuttcap%
\pgfsetroundjoin%
\definecolor{currentfill}{rgb}{0.000000,0.000000,0.000000}%
\pgfsetfillcolor{currentfill}%
\pgfsetlinewidth{0.602250pt}%
\definecolor{currentstroke}{rgb}{0.000000,0.000000,0.000000}%
\pgfsetstrokecolor{currentstroke}%
\pgfsetdash{}{0pt}%
\pgfsys@defobject{currentmarker}{\pgfqpoint{0.000000in}{-0.027778in}}{\pgfqpoint{0.000000in}{0.000000in}}{%
\pgfpathmoveto{\pgfqpoint{0.000000in}{0.000000in}}%
\pgfpathlineto{\pgfqpoint{0.000000in}{-0.027778in}}%
\pgfusepath{stroke,fill}%
}%
\begin{pgfscope}%
\pgfsys@transformshift{0.984395in}{0.417642in}%
\pgfsys@useobject{currentmarker}{}%
\end{pgfscope}%
\end{pgfscope}%
\begin{pgfscope}%
\pgfpathrectangle{\pgfqpoint{0.594525in}{0.417642in}}{\pgfqpoint{3.423805in}{2.011535in}}%
\pgfusepath{clip}%
\pgfsetrectcap%
\pgfsetroundjoin%
\pgfsetlinewidth{0.803000pt}%
\definecolor{currentstroke}{rgb}{0.850000,0.850000,0.850000}%
\pgfsetstrokecolor{currentstroke}%
\pgfsetdash{}{0pt}%
\pgfpathmoveto{\pgfqpoint{1.121418in}{0.417642in}}%
\pgfpathlineto{\pgfqpoint{1.121418in}{2.429177in}}%
\pgfusepath{stroke}%
\end{pgfscope}%
\begin{pgfscope}%
\pgfsetbuttcap%
\pgfsetroundjoin%
\definecolor{currentfill}{rgb}{0.000000,0.000000,0.000000}%
\pgfsetfillcolor{currentfill}%
\pgfsetlinewidth{0.602250pt}%
\definecolor{currentstroke}{rgb}{0.000000,0.000000,0.000000}%
\pgfsetstrokecolor{currentstroke}%
\pgfsetdash{}{0pt}%
\pgfsys@defobject{currentmarker}{\pgfqpoint{0.000000in}{-0.027778in}}{\pgfqpoint{0.000000in}{0.000000in}}{%
\pgfpathmoveto{\pgfqpoint{0.000000in}{0.000000in}}%
\pgfpathlineto{\pgfqpoint{0.000000in}{-0.027778in}}%
\pgfusepath{stroke,fill}%
}%
\begin{pgfscope}%
\pgfsys@transformshift{1.121418in}{0.417642in}%
\pgfsys@useobject{currentmarker}{}%
\end{pgfscope}%
\end{pgfscope}%
\begin{pgfscope}%
\pgfpathrectangle{\pgfqpoint{0.594525in}{0.417642in}}{\pgfqpoint{3.423805in}{2.011535in}}%
\pgfusepath{clip}%
\pgfsetrectcap%
\pgfsetroundjoin%
\pgfsetlinewidth{0.803000pt}%
\definecolor{currentstroke}{rgb}{0.850000,0.850000,0.850000}%
\pgfsetstrokecolor{currentstroke}%
\pgfsetdash{}{0pt}%
\pgfpathmoveto{\pgfqpoint{1.218638in}{0.417642in}}%
\pgfpathlineto{\pgfqpoint{1.218638in}{2.429177in}}%
\pgfusepath{stroke}%
\end{pgfscope}%
\begin{pgfscope}%
\pgfsetbuttcap%
\pgfsetroundjoin%
\definecolor{currentfill}{rgb}{0.000000,0.000000,0.000000}%
\pgfsetfillcolor{currentfill}%
\pgfsetlinewidth{0.602250pt}%
\definecolor{currentstroke}{rgb}{0.000000,0.000000,0.000000}%
\pgfsetstrokecolor{currentstroke}%
\pgfsetdash{}{0pt}%
\pgfsys@defobject{currentmarker}{\pgfqpoint{0.000000in}{-0.027778in}}{\pgfqpoint{0.000000in}{0.000000in}}{%
\pgfpathmoveto{\pgfqpoint{0.000000in}{0.000000in}}%
\pgfpathlineto{\pgfqpoint{0.000000in}{-0.027778in}}%
\pgfusepath{stroke,fill}%
}%
\begin{pgfscope}%
\pgfsys@transformshift{1.218638in}{0.417642in}%
\pgfsys@useobject{currentmarker}{}%
\end{pgfscope}%
\end{pgfscope}%
\begin{pgfscope}%
\pgfpathrectangle{\pgfqpoint{0.594525in}{0.417642in}}{\pgfqpoint{3.423805in}{2.011535in}}%
\pgfusepath{clip}%
\pgfsetrectcap%
\pgfsetroundjoin%
\pgfsetlinewidth{0.803000pt}%
\definecolor{currentstroke}{rgb}{0.850000,0.850000,0.850000}%
\pgfsetstrokecolor{currentstroke}%
\pgfsetdash{}{0pt}%
\pgfpathmoveto{\pgfqpoint{1.294047in}{0.417642in}}%
\pgfpathlineto{\pgfqpoint{1.294047in}{2.429177in}}%
\pgfusepath{stroke}%
\end{pgfscope}%
\begin{pgfscope}%
\pgfsetbuttcap%
\pgfsetroundjoin%
\definecolor{currentfill}{rgb}{0.000000,0.000000,0.000000}%
\pgfsetfillcolor{currentfill}%
\pgfsetlinewidth{0.602250pt}%
\definecolor{currentstroke}{rgb}{0.000000,0.000000,0.000000}%
\pgfsetstrokecolor{currentstroke}%
\pgfsetdash{}{0pt}%
\pgfsys@defobject{currentmarker}{\pgfqpoint{0.000000in}{-0.027778in}}{\pgfqpoint{0.000000in}{0.000000in}}{%
\pgfpathmoveto{\pgfqpoint{0.000000in}{0.000000in}}%
\pgfpathlineto{\pgfqpoint{0.000000in}{-0.027778in}}%
\pgfusepath{stroke,fill}%
}%
\begin{pgfscope}%
\pgfsys@transformshift{1.294047in}{0.417642in}%
\pgfsys@useobject{currentmarker}{}%
\end{pgfscope}%
\end{pgfscope}%
\begin{pgfscope}%
\pgfpathrectangle{\pgfqpoint{0.594525in}{0.417642in}}{\pgfqpoint{3.423805in}{2.011535in}}%
\pgfusepath{clip}%
\pgfsetrectcap%
\pgfsetroundjoin%
\pgfsetlinewidth{0.803000pt}%
\definecolor{currentstroke}{rgb}{0.850000,0.850000,0.850000}%
\pgfsetstrokecolor{currentstroke}%
\pgfsetdash{}{0pt}%
\pgfpathmoveto{\pgfqpoint{1.355661in}{0.417642in}}%
\pgfpathlineto{\pgfqpoint{1.355661in}{2.429177in}}%
\pgfusepath{stroke}%
\end{pgfscope}%
\begin{pgfscope}%
\pgfsetbuttcap%
\pgfsetroundjoin%
\definecolor{currentfill}{rgb}{0.000000,0.000000,0.000000}%
\pgfsetfillcolor{currentfill}%
\pgfsetlinewidth{0.602250pt}%
\definecolor{currentstroke}{rgb}{0.000000,0.000000,0.000000}%
\pgfsetstrokecolor{currentstroke}%
\pgfsetdash{}{0pt}%
\pgfsys@defobject{currentmarker}{\pgfqpoint{0.000000in}{-0.027778in}}{\pgfqpoint{0.000000in}{0.000000in}}{%
\pgfpathmoveto{\pgfqpoint{0.000000in}{0.000000in}}%
\pgfpathlineto{\pgfqpoint{0.000000in}{-0.027778in}}%
\pgfusepath{stroke,fill}%
}%
\begin{pgfscope}%
\pgfsys@transformshift{1.355661in}{0.417642in}%
\pgfsys@useobject{currentmarker}{}%
\end{pgfscope}%
\end{pgfscope}%
\begin{pgfscope}%
\pgfpathrectangle{\pgfqpoint{0.594525in}{0.417642in}}{\pgfqpoint{3.423805in}{2.011535in}}%
\pgfusepath{clip}%
\pgfsetrectcap%
\pgfsetroundjoin%
\pgfsetlinewidth{0.803000pt}%
\definecolor{currentstroke}{rgb}{0.850000,0.850000,0.850000}%
\pgfsetstrokecolor{currentstroke}%
\pgfsetdash{}{0pt}%
\pgfpathmoveto{\pgfqpoint{1.407755in}{0.417642in}}%
\pgfpathlineto{\pgfqpoint{1.407755in}{2.429177in}}%
\pgfusepath{stroke}%
\end{pgfscope}%
\begin{pgfscope}%
\pgfsetbuttcap%
\pgfsetroundjoin%
\definecolor{currentfill}{rgb}{0.000000,0.000000,0.000000}%
\pgfsetfillcolor{currentfill}%
\pgfsetlinewidth{0.602250pt}%
\definecolor{currentstroke}{rgb}{0.000000,0.000000,0.000000}%
\pgfsetstrokecolor{currentstroke}%
\pgfsetdash{}{0pt}%
\pgfsys@defobject{currentmarker}{\pgfqpoint{0.000000in}{-0.027778in}}{\pgfqpoint{0.000000in}{0.000000in}}{%
\pgfpathmoveto{\pgfqpoint{0.000000in}{0.000000in}}%
\pgfpathlineto{\pgfqpoint{0.000000in}{-0.027778in}}%
\pgfusepath{stroke,fill}%
}%
\begin{pgfscope}%
\pgfsys@transformshift{1.407755in}{0.417642in}%
\pgfsys@useobject{currentmarker}{}%
\end{pgfscope}%
\end{pgfscope}%
\begin{pgfscope}%
\pgfpathrectangle{\pgfqpoint{0.594525in}{0.417642in}}{\pgfqpoint{3.423805in}{2.011535in}}%
\pgfusepath{clip}%
\pgfsetrectcap%
\pgfsetroundjoin%
\pgfsetlinewidth{0.803000pt}%
\definecolor{currentstroke}{rgb}{0.850000,0.850000,0.850000}%
\pgfsetstrokecolor{currentstroke}%
\pgfsetdash{}{0pt}%
\pgfpathmoveto{\pgfqpoint{1.452880in}{0.417642in}}%
\pgfpathlineto{\pgfqpoint{1.452880in}{2.429177in}}%
\pgfusepath{stroke}%
\end{pgfscope}%
\begin{pgfscope}%
\pgfsetbuttcap%
\pgfsetroundjoin%
\definecolor{currentfill}{rgb}{0.000000,0.000000,0.000000}%
\pgfsetfillcolor{currentfill}%
\pgfsetlinewidth{0.602250pt}%
\definecolor{currentstroke}{rgb}{0.000000,0.000000,0.000000}%
\pgfsetstrokecolor{currentstroke}%
\pgfsetdash{}{0pt}%
\pgfsys@defobject{currentmarker}{\pgfqpoint{0.000000in}{-0.027778in}}{\pgfqpoint{0.000000in}{0.000000in}}{%
\pgfpathmoveto{\pgfqpoint{0.000000in}{0.000000in}}%
\pgfpathlineto{\pgfqpoint{0.000000in}{-0.027778in}}%
\pgfusepath{stroke,fill}%
}%
\begin{pgfscope}%
\pgfsys@transformshift{1.452880in}{0.417642in}%
\pgfsys@useobject{currentmarker}{}%
\end{pgfscope}%
\end{pgfscope}%
\begin{pgfscope}%
\pgfpathrectangle{\pgfqpoint{0.594525in}{0.417642in}}{\pgfqpoint{3.423805in}{2.011535in}}%
\pgfusepath{clip}%
\pgfsetrectcap%
\pgfsetroundjoin%
\pgfsetlinewidth{0.803000pt}%
\definecolor{currentstroke}{rgb}{0.850000,0.850000,0.850000}%
\pgfsetstrokecolor{currentstroke}%
\pgfsetdash{}{0pt}%
\pgfpathmoveto{\pgfqpoint{1.492684in}{0.417642in}}%
\pgfpathlineto{\pgfqpoint{1.492684in}{2.429177in}}%
\pgfusepath{stroke}%
\end{pgfscope}%
\begin{pgfscope}%
\pgfsetbuttcap%
\pgfsetroundjoin%
\definecolor{currentfill}{rgb}{0.000000,0.000000,0.000000}%
\pgfsetfillcolor{currentfill}%
\pgfsetlinewidth{0.602250pt}%
\definecolor{currentstroke}{rgb}{0.000000,0.000000,0.000000}%
\pgfsetstrokecolor{currentstroke}%
\pgfsetdash{}{0pt}%
\pgfsys@defobject{currentmarker}{\pgfqpoint{0.000000in}{-0.027778in}}{\pgfqpoint{0.000000in}{0.000000in}}{%
\pgfpathmoveto{\pgfqpoint{0.000000in}{0.000000in}}%
\pgfpathlineto{\pgfqpoint{0.000000in}{-0.027778in}}%
\pgfusepath{stroke,fill}%
}%
\begin{pgfscope}%
\pgfsys@transformshift{1.492684in}{0.417642in}%
\pgfsys@useobject{currentmarker}{}%
\end{pgfscope}%
\end{pgfscope}%
\begin{pgfscope}%
\pgfpathrectangle{\pgfqpoint{0.594525in}{0.417642in}}{\pgfqpoint{3.423805in}{2.011535in}}%
\pgfusepath{clip}%
\pgfsetrectcap%
\pgfsetroundjoin%
\pgfsetlinewidth{0.803000pt}%
\definecolor{currentstroke}{rgb}{0.850000,0.850000,0.850000}%
\pgfsetstrokecolor{currentstroke}%
\pgfsetdash{}{0pt}%
\pgfpathmoveto{\pgfqpoint{1.762533in}{0.417642in}}%
\pgfpathlineto{\pgfqpoint{1.762533in}{2.429177in}}%
\pgfusepath{stroke}%
\end{pgfscope}%
\begin{pgfscope}%
\pgfsetbuttcap%
\pgfsetroundjoin%
\definecolor{currentfill}{rgb}{0.000000,0.000000,0.000000}%
\pgfsetfillcolor{currentfill}%
\pgfsetlinewidth{0.602250pt}%
\definecolor{currentstroke}{rgb}{0.000000,0.000000,0.000000}%
\pgfsetstrokecolor{currentstroke}%
\pgfsetdash{}{0pt}%
\pgfsys@defobject{currentmarker}{\pgfqpoint{0.000000in}{-0.027778in}}{\pgfqpoint{0.000000in}{0.000000in}}{%
\pgfpathmoveto{\pgfqpoint{0.000000in}{0.000000in}}%
\pgfpathlineto{\pgfqpoint{0.000000in}{-0.027778in}}%
\pgfusepath{stroke,fill}%
}%
\begin{pgfscope}%
\pgfsys@transformshift{1.762533in}{0.417642in}%
\pgfsys@useobject{currentmarker}{}%
\end{pgfscope}%
\end{pgfscope}%
\begin{pgfscope}%
\pgfpathrectangle{\pgfqpoint{0.594525in}{0.417642in}}{\pgfqpoint{3.423805in}{2.011535in}}%
\pgfusepath{clip}%
\pgfsetrectcap%
\pgfsetroundjoin%
\pgfsetlinewidth{0.803000pt}%
\definecolor{currentstroke}{rgb}{0.850000,0.850000,0.850000}%
\pgfsetstrokecolor{currentstroke}%
\pgfsetdash{}{0pt}%
\pgfpathmoveto{\pgfqpoint{1.899556in}{0.417642in}}%
\pgfpathlineto{\pgfqpoint{1.899556in}{2.429177in}}%
\pgfusepath{stroke}%
\end{pgfscope}%
\begin{pgfscope}%
\pgfsetbuttcap%
\pgfsetroundjoin%
\definecolor{currentfill}{rgb}{0.000000,0.000000,0.000000}%
\pgfsetfillcolor{currentfill}%
\pgfsetlinewidth{0.602250pt}%
\definecolor{currentstroke}{rgb}{0.000000,0.000000,0.000000}%
\pgfsetstrokecolor{currentstroke}%
\pgfsetdash{}{0pt}%
\pgfsys@defobject{currentmarker}{\pgfqpoint{0.000000in}{-0.027778in}}{\pgfqpoint{0.000000in}{0.000000in}}{%
\pgfpathmoveto{\pgfqpoint{0.000000in}{0.000000in}}%
\pgfpathlineto{\pgfqpoint{0.000000in}{-0.027778in}}%
\pgfusepath{stroke,fill}%
}%
\begin{pgfscope}%
\pgfsys@transformshift{1.899556in}{0.417642in}%
\pgfsys@useobject{currentmarker}{}%
\end{pgfscope}%
\end{pgfscope}%
\begin{pgfscope}%
\pgfpathrectangle{\pgfqpoint{0.594525in}{0.417642in}}{\pgfqpoint{3.423805in}{2.011535in}}%
\pgfusepath{clip}%
\pgfsetrectcap%
\pgfsetroundjoin%
\pgfsetlinewidth{0.803000pt}%
\definecolor{currentstroke}{rgb}{0.850000,0.850000,0.850000}%
\pgfsetstrokecolor{currentstroke}%
\pgfsetdash{}{0pt}%
\pgfpathmoveto{\pgfqpoint{1.996775in}{0.417642in}}%
\pgfpathlineto{\pgfqpoint{1.996775in}{2.429177in}}%
\pgfusepath{stroke}%
\end{pgfscope}%
\begin{pgfscope}%
\pgfsetbuttcap%
\pgfsetroundjoin%
\definecolor{currentfill}{rgb}{0.000000,0.000000,0.000000}%
\pgfsetfillcolor{currentfill}%
\pgfsetlinewidth{0.602250pt}%
\definecolor{currentstroke}{rgb}{0.000000,0.000000,0.000000}%
\pgfsetstrokecolor{currentstroke}%
\pgfsetdash{}{0pt}%
\pgfsys@defobject{currentmarker}{\pgfqpoint{0.000000in}{-0.027778in}}{\pgfqpoint{0.000000in}{0.000000in}}{%
\pgfpathmoveto{\pgfqpoint{0.000000in}{0.000000in}}%
\pgfpathlineto{\pgfqpoint{0.000000in}{-0.027778in}}%
\pgfusepath{stroke,fill}%
}%
\begin{pgfscope}%
\pgfsys@transformshift{1.996775in}{0.417642in}%
\pgfsys@useobject{currentmarker}{}%
\end{pgfscope}%
\end{pgfscope}%
\begin{pgfscope}%
\pgfpathrectangle{\pgfqpoint{0.594525in}{0.417642in}}{\pgfqpoint{3.423805in}{2.011535in}}%
\pgfusepath{clip}%
\pgfsetrectcap%
\pgfsetroundjoin%
\pgfsetlinewidth{0.803000pt}%
\definecolor{currentstroke}{rgb}{0.850000,0.850000,0.850000}%
\pgfsetstrokecolor{currentstroke}%
\pgfsetdash{}{0pt}%
\pgfpathmoveto{\pgfqpoint{2.072185in}{0.417642in}}%
\pgfpathlineto{\pgfqpoint{2.072185in}{2.429177in}}%
\pgfusepath{stroke}%
\end{pgfscope}%
\begin{pgfscope}%
\pgfsetbuttcap%
\pgfsetroundjoin%
\definecolor{currentfill}{rgb}{0.000000,0.000000,0.000000}%
\pgfsetfillcolor{currentfill}%
\pgfsetlinewidth{0.602250pt}%
\definecolor{currentstroke}{rgb}{0.000000,0.000000,0.000000}%
\pgfsetstrokecolor{currentstroke}%
\pgfsetdash{}{0pt}%
\pgfsys@defobject{currentmarker}{\pgfqpoint{0.000000in}{-0.027778in}}{\pgfqpoint{0.000000in}{0.000000in}}{%
\pgfpathmoveto{\pgfqpoint{0.000000in}{0.000000in}}%
\pgfpathlineto{\pgfqpoint{0.000000in}{-0.027778in}}%
\pgfusepath{stroke,fill}%
}%
\begin{pgfscope}%
\pgfsys@transformshift{2.072185in}{0.417642in}%
\pgfsys@useobject{currentmarker}{}%
\end{pgfscope}%
\end{pgfscope}%
\begin{pgfscope}%
\pgfpathrectangle{\pgfqpoint{0.594525in}{0.417642in}}{\pgfqpoint{3.423805in}{2.011535in}}%
\pgfusepath{clip}%
\pgfsetrectcap%
\pgfsetroundjoin%
\pgfsetlinewidth{0.803000pt}%
\definecolor{currentstroke}{rgb}{0.850000,0.850000,0.850000}%
\pgfsetstrokecolor{currentstroke}%
\pgfsetdash{}{0pt}%
\pgfpathmoveto{\pgfqpoint{2.133799in}{0.417642in}}%
\pgfpathlineto{\pgfqpoint{2.133799in}{2.429177in}}%
\pgfusepath{stroke}%
\end{pgfscope}%
\begin{pgfscope}%
\pgfsetbuttcap%
\pgfsetroundjoin%
\definecolor{currentfill}{rgb}{0.000000,0.000000,0.000000}%
\pgfsetfillcolor{currentfill}%
\pgfsetlinewidth{0.602250pt}%
\definecolor{currentstroke}{rgb}{0.000000,0.000000,0.000000}%
\pgfsetstrokecolor{currentstroke}%
\pgfsetdash{}{0pt}%
\pgfsys@defobject{currentmarker}{\pgfqpoint{0.000000in}{-0.027778in}}{\pgfqpoint{0.000000in}{0.000000in}}{%
\pgfpathmoveto{\pgfqpoint{0.000000in}{0.000000in}}%
\pgfpathlineto{\pgfqpoint{0.000000in}{-0.027778in}}%
\pgfusepath{stroke,fill}%
}%
\begin{pgfscope}%
\pgfsys@transformshift{2.133799in}{0.417642in}%
\pgfsys@useobject{currentmarker}{}%
\end{pgfscope}%
\end{pgfscope}%
\begin{pgfscope}%
\pgfpathrectangle{\pgfqpoint{0.594525in}{0.417642in}}{\pgfqpoint{3.423805in}{2.011535in}}%
\pgfusepath{clip}%
\pgfsetrectcap%
\pgfsetroundjoin%
\pgfsetlinewidth{0.803000pt}%
\definecolor{currentstroke}{rgb}{0.850000,0.850000,0.850000}%
\pgfsetstrokecolor{currentstroke}%
\pgfsetdash{}{0pt}%
\pgfpathmoveto{\pgfqpoint{2.185892in}{0.417642in}}%
\pgfpathlineto{\pgfqpoint{2.185892in}{2.429177in}}%
\pgfusepath{stroke}%
\end{pgfscope}%
\begin{pgfscope}%
\pgfsetbuttcap%
\pgfsetroundjoin%
\definecolor{currentfill}{rgb}{0.000000,0.000000,0.000000}%
\pgfsetfillcolor{currentfill}%
\pgfsetlinewidth{0.602250pt}%
\definecolor{currentstroke}{rgb}{0.000000,0.000000,0.000000}%
\pgfsetstrokecolor{currentstroke}%
\pgfsetdash{}{0pt}%
\pgfsys@defobject{currentmarker}{\pgfqpoint{0.000000in}{-0.027778in}}{\pgfqpoint{0.000000in}{0.000000in}}{%
\pgfpathmoveto{\pgfqpoint{0.000000in}{0.000000in}}%
\pgfpathlineto{\pgfqpoint{0.000000in}{-0.027778in}}%
\pgfusepath{stroke,fill}%
}%
\begin{pgfscope}%
\pgfsys@transformshift{2.185892in}{0.417642in}%
\pgfsys@useobject{currentmarker}{}%
\end{pgfscope}%
\end{pgfscope}%
\begin{pgfscope}%
\pgfpathrectangle{\pgfqpoint{0.594525in}{0.417642in}}{\pgfqpoint{3.423805in}{2.011535in}}%
\pgfusepath{clip}%
\pgfsetrectcap%
\pgfsetroundjoin%
\pgfsetlinewidth{0.803000pt}%
\definecolor{currentstroke}{rgb}{0.850000,0.850000,0.850000}%
\pgfsetstrokecolor{currentstroke}%
\pgfsetdash{}{0pt}%
\pgfpathmoveto{\pgfqpoint{2.231018in}{0.417642in}}%
\pgfpathlineto{\pgfqpoint{2.231018in}{2.429177in}}%
\pgfusepath{stroke}%
\end{pgfscope}%
\begin{pgfscope}%
\pgfsetbuttcap%
\pgfsetroundjoin%
\definecolor{currentfill}{rgb}{0.000000,0.000000,0.000000}%
\pgfsetfillcolor{currentfill}%
\pgfsetlinewidth{0.602250pt}%
\definecolor{currentstroke}{rgb}{0.000000,0.000000,0.000000}%
\pgfsetstrokecolor{currentstroke}%
\pgfsetdash{}{0pt}%
\pgfsys@defobject{currentmarker}{\pgfqpoint{0.000000in}{-0.027778in}}{\pgfqpoint{0.000000in}{0.000000in}}{%
\pgfpathmoveto{\pgfqpoint{0.000000in}{0.000000in}}%
\pgfpathlineto{\pgfqpoint{0.000000in}{-0.027778in}}%
\pgfusepath{stroke,fill}%
}%
\begin{pgfscope}%
\pgfsys@transformshift{2.231018in}{0.417642in}%
\pgfsys@useobject{currentmarker}{}%
\end{pgfscope}%
\end{pgfscope}%
\begin{pgfscope}%
\pgfpathrectangle{\pgfqpoint{0.594525in}{0.417642in}}{\pgfqpoint{3.423805in}{2.011535in}}%
\pgfusepath{clip}%
\pgfsetrectcap%
\pgfsetroundjoin%
\pgfsetlinewidth{0.803000pt}%
\definecolor{currentstroke}{rgb}{0.850000,0.850000,0.850000}%
\pgfsetstrokecolor{currentstroke}%
\pgfsetdash{}{0pt}%
\pgfpathmoveto{\pgfqpoint{2.270822in}{0.417642in}}%
\pgfpathlineto{\pgfqpoint{2.270822in}{2.429177in}}%
\pgfusepath{stroke}%
\end{pgfscope}%
\begin{pgfscope}%
\pgfsetbuttcap%
\pgfsetroundjoin%
\definecolor{currentfill}{rgb}{0.000000,0.000000,0.000000}%
\pgfsetfillcolor{currentfill}%
\pgfsetlinewidth{0.602250pt}%
\definecolor{currentstroke}{rgb}{0.000000,0.000000,0.000000}%
\pgfsetstrokecolor{currentstroke}%
\pgfsetdash{}{0pt}%
\pgfsys@defobject{currentmarker}{\pgfqpoint{0.000000in}{-0.027778in}}{\pgfqpoint{0.000000in}{0.000000in}}{%
\pgfpathmoveto{\pgfqpoint{0.000000in}{0.000000in}}%
\pgfpathlineto{\pgfqpoint{0.000000in}{-0.027778in}}%
\pgfusepath{stroke,fill}%
}%
\begin{pgfscope}%
\pgfsys@transformshift{2.270822in}{0.417642in}%
\pgfsys@useobject{currentmarker}{}%
\end{pgfscope}%
\end{pgfscope}%
\begin{pgfscope}%
\pgfpathrectangle{\pgfqpoint{0.594525in}{0.417642in}}{\pgfqpoint{3.423805in}{2.011535in}}%
\pgfusepath{clip}%
\pgfsetrectcap%
\pgfsetroundjoin%
\pgfsetlinewidth{0.803000pt}%
\definecolor{currentstroke}{rgb}{0.850000,0.850000,0.850000}%
\pgfsetstrokecolor{currentstroke}%
\pgfsetdash{}{0pt}%
\pgfpathmoveto{\pgfqpoint{2.540670in}{0.417642in}}%
\pgfpathlineto{\pgfqpoint{2.540670in}{2.429177in}}%
\pgfusepath{stroke}%
\end{pgfscope}%
\begin{pgfscope}%
\pgfsetbuttcap%
\pgfsetroundjoin%
\definecolor{currentfill}{rgb}{0.000000,0.000000,0.000000}%
\pgfsetfillcolor{currentfill}%
\pgfsetlinewidth{0.602250pt}%
\definecolor{currentstroke}{rgb}{0.000000,0.000000,0.000000}%
\pgfsetstrokecolor{currentstroke}%
\pgfsetdash{}{0pt}%
\pgfsys@defobject{currentmarker}{\pgfqpoint{0.000000in}{-0.027778in}}{\pgfqpoint{0.000000in}{0.000000in}}{%
\pgfpathmoveto{\pgfqpoint{0.000000in}{0.000000in}}%
\pgfpathlineto{\pgfqpoint{0.000000in}{-0.027778in}}%
\pgfusepath{stroke,fill}%
}%
\begin{pgfscope}%
\pgfsys@transformshift{2.540670in}{0.417642in}%
\pgfsys@useobject{currentmarker}{}%
\end{pgfscope}%
\end{pgfscope}%
\begin{pgfscope}%
\pgfpathrectangle{\pgfqpoint{0.594525in}{0.417642in}}{\pgfqpoint{3.423805in}{2.011535in}}%
\pgfusepath{clip}%
\pgfsetrectcap%
\pgfsetroundjoin%
\pgfsetlinewidth{0.803000pt}%
\definecolor{currentstroke}{rgb}{0.850000,0.850000,0.850000}%
\pgfsetstrokecolor{currentstroke}%
\pgfsetdash{}{0pt}%
\pgfpathmoveto{\pgfqpoint{2.677693in}{0.417642in}}%
\pgfpathlineto{\pgfqpoint{2.677693in}{2.429177in}}%
\pgfusepath{stroke}%
\end{pgfscope}%
\begin{pgfscope}%
\pgfsetbuttcap%
\pgfsetroundjoin%
\definecolor{currentfill}{rgb}{0.000000,0.000000,0.000000}%
\pgfsetfillcolor{currentfill}%
\pgfsetlinewidth{0.602250pt}%
\definecolor{currentstroke}{rgb}{0.000000,0.000000,0.000000}%
\pgfsetstrokecolor{currentstroke}%
\pgfsetdash{}{0pt}%
\pgfsys@defobject{currentmarker}{\pgfqpoint{0.000000in}{-0.027778in}}{\pgfqpoint{0.000000in}{0.000000in}}{%
\pgfpathmoveto{\pgfqpoint{0.000000in}{0.000000in}}%
\pgfpathlineto{\pgfqpoint{0.000000in}{-0.027778in}}%
\pgfusepath{stroke,fill}%
}%
\begin{pgfscope}%
\pgfsys@transformshift{2.677693in}{0.417642in}%
\pgfsys@useobject{currentmarker}{}%
\end{pgfscope}%
\end{pgfscope}%
\begin{pgfscope}%
\pgfpathrectangle{\pgfqpoint{0.594525in}{0.417642in}}{\pgfqpoint{3.423805in}{2.011535in}}%
\pgfusepath{clip}%
\pgfsetrectcap%
\pgfsetroundjoin%
\pgfsetlinewidth{0.803000pt}%
\definecolor{currentstroke}{rgb}{0.850000,0.850000,0.850000}%
\pgfsetstrokecolor{currentstroke}%
\pgfsetdash{}{0pt}%
\pgfpathmoveto{\pgfqpoint{2.774913in}{0.417642in}}%
\pgfpathlineto{\pgfqpoint{2.774913in}{2.429177in}}%
\pgfusepath{stroke}%
\end{pgfscope}%
\begin{pgfscope}%
\pgfsetbuttcap%
\pgfsetroundjoin%
\definecolor{currentfill}{rgb}{0.000000,0.000000,0.000000}%
\pgfsetfillcolor{currentfill}%
\pgfsetlinewidth{0.602250pt}%
\definecolor{currentstroke}{rgb}{0.000000,0.000000,0.000000}%
\pgfsetstrokecolor{currentstroke}%
\pgfsetdash{}{0pt}%
\pgfsys@defobject{currentmarker}{\pgfqpoint{0.000000in}{-0.027778in}}{\pgfqpoint{0.000000in}{0.000000in}}{%
\pgfpathmoveto{\pgfqpoint{0.000000in}{0.000000in}}%
\pgfpathlineto{\pgfqpoint{0.000000in}{-0.027778in}}%
\pgfusepath{stroke,fill}%
}%
\begin{pgfscope}%
\pgfsys@transformshift{2.774913in}{0.417642in}%
\pgfsys@useobject{currentmarker}{}%
\end{pgfscope}%
\end{pgfscope}%
\begin{pgfscope}%
\pgfpathrectangle{\pgfqpoint{0.594525in}{0.417642in}}{\pgfqpoint{3.423805in}{2.011535in}}%
\pgfusepath{clip}%
\pgfsetrectcap%
\pgfsetroundjoin%
\pgfsetlinewidth{0.803000pt}%
\definecolor{currentstroke}{rgb}{0.850000,0.850000,0.850000}%
\pgfsetstrokecolor{currentstroke}%
\pgfsetdash{}{0pt}%
\pgfpathmoveto{\pgfqpoint{2.850322in}{0.417642in}}%
\pgfpathlineto{\pgfqpoint{2.850322in}{2.429177in}}%
\pgfusepath{stroke}%
\end{pgfscope}%
\begin{pgfscope}%
\pgfsetbuttcap%
\pgfsetroundjoin%
\definecolor{currentfill}{rgb}{0.000000,0.000000,0.000000}%
\pgfsetfillcolor{currentfill}%
\pgfsetlinewidth{0.602250pt}%
\definecolor{currentstroke}{rgb}{0.000000,0.000000,0.000000}%
\pgfsetstrokecolor{currentstroke}%
\pgfsetdash{}{0pt}%
\pgfsys@defobject{currentmarker}{\pgfqpoint{0.000000in}{-0.027778in}}{\pgfqpoint{0.000000in}{0.000000in}}{%
\pgfpathmoveto{\pgfqpoint{0.000000in}{0.000000in}}%
\pgfpathlineto{\pgfqpoint{0.000000in}{-0.027778in}}%
\pgfusepath{stroke,fill}%
}%
\begin{pgfscope}%
\pgfsys@transformshift{2.850322in}{0.417642in}%
\pgfsys@useobject{currentmarker}{}%
\end{pgfscope}%
\end{pgfscope}%
\begin{pgfscope}%
\pgfpathrectangle{\pgfqpoint{0.594525in}{0.417642in}}{\pgfqpoint{3.423805in}{2.011535in}}%
\pgfusepath{clip}%
\pgfsetrectcap%
\pgfsetroundjoin%
\pgfsetlinewidth{0.803000pt}%
\definecolor{currentstroke}{rgb}{0.850000,0.850000,0.850000}%
\pgfsetstrokecolor{currentstroke}%
\pgfsetdash{}{0pt}%
\pgfpathmoveto{\pgfqpoint{2.911936in}{0.417642in}}%
\pgfpathlineto{\pgfqpoint{2.911936in}{2.429177in}}%
\pgfusepath{stroke}%
\end{pgfscope}%
\begin{pgfscope}%
\pgfsetbuttcap%
\pgfsetroundjoin%
\definecolor{currentfill}{rgb}{0.000000,0.000000,0.000000}%
\pgfsetfillcolor{currentfill}%
\pgfsetlinewidth{0.602250pt}%
\definecolor{currentstroke}{rgb}{0.000000,0.000000,0.000000}%
\pgfsetstrokecolor{currentstroke}%
\pgfsetdash{}{0pt}%
\pgfsys@defobject{currentmarker}{\pgfqpoint{0.000000in}{-0.027778in}}{\pgfqpoint{0.000000in}{0.000000in}}{%
\pgfpathmoveto{\pgfqpoint{0.000000in}{0.000000in}}%
\pgfpathlineto{\pgfqpoint{0.000000in}{-0.027778in}}%
\pgfusepath{stroke,fill}%
}%
\begin{pgfscope}%
\pgfsys@transformshift{2.911936in}{0.417642in}%
\pgfsys@useobject{currentmarker}{}%
\end{pgfscope}%
\end{pgfscope}%
\begin{pgfscope}%
\pgfpathrectangle{\pgfqpoint{0.594525in}{0.417642in}}{\pgfqpoint{3.423805in}{2.011535in}}%
\pgfusepath{clip}%
\pgfsetrectcap%
\pgfsetroundjoin%
\pgfsetlinewidth{0.803000pt}%
\definecolor{currentstroke}{rgb}{0.850000,0.850000,0.850000}%
\pgfsetstrokecolor{currentstroke}%
\pgfsetdash{}{0pt}%
\pgfpathmoveto{\pgfqpoint{2.964030in}{0.417642in}}%
\pgfpathlineto{\pgfqpoint{2.964030in}{2.429177in}}%
\pgfusepath{stroke}%
\end{pgfscope}%
\begin{pgfscope}%
\pgfsetbuttcap%
\pgfsetroundjoin%
\definecolor{currentfill}{rgb}{0.000000,0.000000,0.000000}%
\pgfsetfillcolor{currentfill}%
\pgfsetlinewidth{0.602250pt}%
\definecolor{currentstroke}{rgb}{0.000000,0.000000,0.000000}%
\pgfsetstrokecolor{currentstroke}%
\pgfsetdash{}{0pt}%
\pgfsys@defobject{currentmarker}{\pgfqpoint{0.000000in}{-0.027778in}}{\pgfqpoint{0.000000in}{0.000000in}}{%
\pgfpathmoveto{\pgfqpoint{0.000000in}{0.000000in}}%
\pgfpathlineto{\pgfqpoint{0.000000in}{-0.027778in}}%
\pgfusepath{stroke,fill}%
}%
\begin{pgfscope}%
\pgfsys@transformshift{2.964030in}{0.417642in}%
\pgfsys@useobject{currentmarker}{}%
\end{pgfscope}%
\end{pgfscope}%
\begin{pgfscope}%
\pgfpathrectangle{\pgfqpoint{0.594525in}{0.417642in}}{\pgfqpoint{3.423805in}{2.011535in}}%
\pgfusepath{clip}%
\pgfsetrectcap%
\pgfsetroundjoin%
\pgfsetlinewidth{0.803000pt}%
\definecolor{currentstroke}{rgb}{0.850000,0.850000,0.850000}%
\pgfsetstrokecolor{currentstroke}%
\pgfsetdash{}{0pt}%
\pgfpathmoveto{\pgfqpoint{3.009156in}{0.417642in}}%
\pgfpathlineto{\pgfqpoint{3.009156in}{2.429177in}}%
\pgfusepath{stroke}%
\end{pgfscope}%
\begin{pgfscope}%
\pgfsetbuttcap%
\pgfsetroundjoin%
\definecolor{currentfill}{rgb}{0.000000,0.000000,0.000000}%
\pgfsetfillcolor{currentfill}%
\pgfsetlinewidth{0.602250pt}%
\definecolor{currentstroke}{rgb}{0.000000,0.000000,0.000000}%
\pgfsetstrokecolor{currentstroke}%
\pgfsetdash{}{0pt}%
\pgfsys@defobject{currentmarker}{\pgfqpoint{0.000000in}{-0.027778in}}{\pgfqpoint{0.000000in}{0.000000in}}{%
\pgfpathmoveto{\pgfqpoint{0.000000in}{0.000000in}}%
\pgfpathlineto{\pgfqpoint{0.000000in}{-0.027778in}}%
\pgfusepath{stroke,fill}%
}%
\begin{pgfscope}%
\pgfsys@transformshift{3.009156in}{0.417642in}%
\pgfsys@useobject{currentmarker}{}%
\end{pgfscope}%
\end{pgfscope}%
\begin{pgfscope}%
\pgfpathrectangle{\pgfqpoint{0.594525in}{0.417642in}}{\pgfqpoint{3.423805in}{2.011535in}}%
\pgfusepath{clip}%
\pgfsetrectcap%
\pgfsetroundjoin%
\pgfsetlinewidth{0.803000pt}%
\definecolor{currentstroke}{rgb}{0.850000,0.850000,0.850000}%
\pgfsetstrokecolor{currentstroke}%
\pgfsetdash{}{0pt}%
\pgfpathmoveto{\pgfqpoint{3.048959in}{0.417642in}}%
\pgfpathlineto{\pgfqpoint{3.048959in}{2.429177in}}%
\pgfusepath{stroke}%
\end{pgfscope}%
\begin{pgfscope}%
\pgfsetbuttcap%
\pgfsetroundjoin%
\definecolor{currentfill}{rgb}{0.000000,0.000000,0.000000}%
\pgfsetfillcolor{currentfill}%
\pgfsetlinewidth{0.602250pt}%
\definecolor{currentstroke}{rgb}{0.000000,0.000000,0.000000}%
\pgfsetstrokecolor{currentstroke}%
\pgfsetdash{}{0pt}%
\pgfsys@defobject{currentmarker}{\pgfqpoint{0.000000in}{-0.027778in}}{\pgfqpoint{0.000000in}{0.000000in}}{%
\pgfpathmoveto{\pgfqpoint{0.000000in}{0.000000in}}%
\pgfpathlineto{\pgfqpoint{0.000000in}{-0.027778in}}%
\pgfusepath{stroke,fill}%
}%
\begin{pgfscope}%
\pgfsys@transformshift{3.048959in}{0.417642in}%
\pgfsys@useobject{currentmarker}{}%
\end{pgfscope}%
\end{pgfscope}%
\begin{pgfscope}%
\pgfpathrectangle{\pgfqpoint{0.594525in}{0.417642in}}{\pgfqpoint{3.423805in}{2.011535in}}%
\pgfusepath{clip}%
\pgfsetrectcap%
\pgfsetroundjoin%
\pgfsetlinewidth{0.803000pt}%
\definecolor{currentstroke}{rgb}{0.850000,0.850000,0.850000}%
\pgfsetstrokecolor{currentstroke}%
\pgfsetdash{}{0pt}%
\pgfpathmoveto{\pgfqpoint{3.318808in}{0.417642in}}%
\pgfpathlineto{\pgfqpoint{3.318808in}{2.429177in}}%
\pgfusepath{stroke}%
\end{pgfscope}%
\begin{pgfscope}%
\pgfsetbuttcap%
\pgfsetroundjoin%
\definecolor{currentfill}{rgb}{0.000000,0.000000,0.000000}%
\pgfsetfillcolor{currentfill}%
\pgfsetlinewidth{0.602250pt}%
\definecolor{currentstroke}{rgb}{0.000000,0.000000,0.000000}%
\pgfsetstrokecolor{currentstroke}%
\pgfsetdash{}{0pt}%
\pgfsys@defobject{currentmarker}{\pgfqpoint{0.000000in}{-0.027778in}}{\pgfqpoint{0.000000in}{0.000000in}}{%
\pgfpathmoveto{\pgfqpoint{0.000000in}{0.000000in}}%
\pgfpathlineto{\pgfqpoint{0.000000in}{-0.027778in}}%
\pgfusepath{stroke,fill}%
}%
\begin{pgfscope}%
\pgfsys@transformshift{3.318808in}{0.417642in}%
\pgfsys@useobject{currentmarker}{}%
\end{pgfscope}%
\end{pgfscope}%
\begin{pgfscope}%
\pgfpathrectangle{\pgfqpoint{0.594525in}{0.417642in}}{\pgfqpoint{3.423805in}{2.011535in}}%
\pgfusepath{clip}%
\pgfsetrectcap%
\pgfsetroundjoin%
\pgfsetlinewidth{0.803000pt}%
\definecolor{currentstroke}{rgb}{0.850000,0.850000,0.850000}%
\pgfsetstrokecolor{currentstroke}%
\pgfsetdash{}{0pt}%
\pgfpathmoveto{\pgfqpoint{3.455831in}{0.417642in}}%
\pgfpathlineto{\pgfqpoint{3.455831in}{2.429177in}}%
\pgfusepath{stroke}%
\end{pgfscope}%
\begin{pgfscope}%
\pgfsetbuttcap%
\pgfsetroundjoin%
\definecolor{currentfill}{rgb}{0.000000,0.000000,0.000000}%
\pgfsetfillcolor{currentfill}%
\pgfsetlinewidth{0.602250pt}%
\definecolor{currentstroke}{rgb}{0.000000,0.000000,0.000000}%
\pgfsetstrokecolor{currentstroke}%
\pgfsetdash{}{0pt}%
\pgfsys@defobject{currentmarker}{\pgfqpoint{0.000000in}{-0.027778in}}{\pgfqpoint{0.000000in}{0.000000in}}{%
\pgfpathmoveto{\pgfqpoint{0.000000in}{0.000000in}}%
\pgfpathlineto{\pgfqpoint{0.000000in}{-0.027778in}}%
\pgfusepath{stroke,fill}%
}%
\begin{pgfscope}%
\pgfsys@transformshift{3.455831in}{0.417642in}%
\pgfsys@useobject{currentmarker}{}%
\end{pgfscope}%
\end{pgfscope}%
\begin{pgfscope}%
\pgfpathrectangle{\pgfqpoint{0.594525in}{0.417642in}}{\pgfqpoint{3.423805in}{2.011535in}}%
\pgfusepath{clip}%
\pgfsetrectcap%
\pgfsetroundjoin%
\pgfsetlinewidth{0.803000pt}%
\definecolor{currentstroke}{rgb}{0.850000,0.850000,0.850000}%
\pgfsetstrokecolor{currentstroke}%
\pgfsetdash{}{0pt}%
\pgfpathmoveto{\pgfqpoint{3.553050in}{0.417642in}}%
\pgfpathlineto{\pgfqpoint{3.553050in}{2.429177in}}%
\pgfusepath{stroke}%
\end{pgfscope}%
\begin{pgfscope}%
\pgfsetbuttcap%
\pgfsetroundjoin%
\definecolor{currentfill}{rgb}{0.000000,0.000000,0.000000}%
\pgfsetfillcolor{currentfill}%
\pgfsetlinewidth{0.602250pt}%
\definecolor{currentstroke}{rgb}{0.000000,0.000000,0.000000}%
\pgfsetstrokecolor{currentstroke}%
\pgfsetdash{}{0pt}%
\pgfsys@defobject{currentmarker}{\pgfqpoint{0.000000in}{-0.027778in}}{\pgfqpoint{0.000000in}{0.000000in}}{%
\pgfpathmoveto{\pgfqpoint{0.000000in}{0.000000in}}%
\pgfpathlineto{\pgfqpoint{0.000000in}{-0.027778in}}%
\pgfusepath{stroke,fill}%
}%
\begin{pgfscope}%
\pgfsys@transformshift{3.553050in}{0.417642in}%
\pgfsys@useobject{currentmarker}{}%
\end{pgfscope}%
\end{pgfscope}%
\begin{pgfscope}%
\pgfpathrectangle{\pgfqpoint{0.594525in}{0.417642in}}{\pgfqpoint{3.423805in}{2.011535in}}%
\pgfusepath{clip}%
\pgfsetrectcap%
\pgfsetroundjoin%
\pgfsetlinewidth{0.803000pt}%
\definecolor{currentstroke}{rgb}{0.850000,0.850000,0.850000}%
\pgfsetstrokecolor{currentstroke}%
\pgfsetdash{}{0pt}%
\pgfpathmoveto{\pgfqpoint{3.628460in}{0.417642in}}%
\pgfpathlineto{\pgfqpoint{3.628460in}{2.429177in}}%
\pgfusepath{stroke}%
\end{pgfscope}%
\begin{pgfscope}%
\pgfsetbuttcap%
\pgfsetroundjoin%
\definecolor{currentfill}{rgb}{0.000000,0.000000,0.000000}%
\pgfsetfillcolor{currentfill}%
\pgfsetlinewidth{0.602250pt}%
\definecolor{currentstroke}{rgb}{0.000000,0.000000,0.000000}%
\pgfsetstrokecolor{currentstroke}%
\pgfsetdash{}{0pt}%
\pgfsys@defobject{currentmarker}{\pgfqpoint{0.000000in}{-0.027778in}}{\pgfqpoint{0.000000in}{0.000000in}}{%
\pgfpathmoveto{\pgfqpoint{0.000000in}{0.000000in}}%
\pgfpathlineto{\pgfqpoint{0.000000in}{-0.027778in}}%
\pgfusepath{stroke,fill}%
}%
\begin{pgfscope}%
\pgfsys@transformshift{3.628460in}{0.417642in}%
\pgfsys@useobject{currentmarker}{}%
\end{pgfscope}%
\end{pgfscope}%
\begin{pgfscope}%
\pgfpathrectangle{\pgfqpoint{0.594525in}{0.417642in}}{\pgfqpoint{3.423805in}{2.011535in}}%
\pgfusepath{clip}%
\pgfsetrectcap%
\pgfsetroundjoin%
\pgfsetlinewidth{0.803000pt}%
\definecolor{currentstroke}{rgb}{0.850000,0.850000,0.850000}%
\pgfsetstrokecolor{currentstroke}%
\pgfsetdash{}{0pt}%
\pgfpathmoveto{\pgfqpoint{3.690074in}{0.417642in}}%
\pgfpathlineto{\pgfqpoint{3.690074in}{2.429177in}}%
\pgfusepath{stroke}%
\end{pgfscope}%
\begin{pgfscope}%
\pgfsetbuttcap%
\pgfsetroundjoin%
\definecolor{currentfill}{rgb}{0.000000,0.000000,0.000000}%
\pgfsetfillcolor{currentfill}%
\pgfsetlinewidth{0.602250pt}%
\definecolor{currentstroke}{rgb}{0.000000,0.000000,0.000000}%
\pgfsetstrokecolor{currentstroke}%
\pgfsetdash{}{0pt}%
\pgfsys@defobject{currentmarker}{\pgfqpoint{0.000000in}{-0.027778in}}{\pgfqpoint{0.000000in}{0.000000in}}{%
\pgfpathmoveto{\pgfqpoint{0.000000in}{0.000000in}}%
\pgfpathlineto{\pgfqpoint{0.000000in}{-0.027778in}}%
\pgfusepath{stroke,fill}%
}%
\begin{pgfscope}%
\pgfsys@transformshift{3.690074in}{0.417642in}%
\pgfsys@useobject{currentmarker}{}%
\end{pgfscope}%
\end{pgfscope}%
\begin{pgfscope}%
\pgfpathrectangle{\pgfqpoint{0.594525in}{0.417642in}}{\pgfqpoint{3.423805in}{2.011535in}}%
\pgfusepath{clip}%
\pgfsetrectcap%
\pgfsetroundjoin%
\pgfsetlinewidth{0.803000pt}%
\definecolor{currentstroke}{rgb}{0.850000,0.850000,0.850000}%
\pgfsetstrokecolor{currentstroke}%
\pgfsetdash{}{0pt}%
\pgfpathmoveto{\pgfqpoint{3.742167in}{0.417642in}}%
\pgfpathlineto{\pgfqpoint{3.742167in}{2.429177in}}%
\pgfusepath{stroke}%
\end{pgfscope}%
\begin{pgfscope}%
\pgfsetbuttcap%
\pgfsetroundjoin%
\definecolor{currentfill}{rgb}{0.000000,0.000000,0.000000}%
\pgfsetfillcolor{currentfill}%
\pgfsetlinewidth{0.602250pt}%
\definecolor{currentstroke}{rgb}{0.000000,0.000000,0.000000}%
\pgfsetstrokecolor{currentstroke}%
\pgfsetdash{}{0pt}%
\pgfsys@defobject{currentmarker}{\pgfqpoint{0.000000in}{-0.027778in}}{\pgfqpoint{0.000000in}{0.000000in}}{%
\pgfpathmoveto{\pgfqpoint{0.000000in}{0.000000in}}%
\pgfpathlineto{\pgfqpoint{0.000000in}{-0.027778in}}%
\pgfusepath{stroke,fill}%
}%
\begin{pgfscope}%
\pgfsys@transformshift{3.742167in}{0.417642in}%
\pgfsys@useobject{currentmarker}{}%
\end{pgfscope}%
\end{pgfscope}%
\begin{pgfscope}%
\pgfpathrectangle{\pgfqpoint{0.594525in}{0.417642in}}{\pgfqpoint{3.423805in}{2.011535in}}%
\pgfusepath{clip}%
\pgfsetrectcap%
\pgfsetroundjoin%
\pgfsetlinewidth{0.803000pt}%
\definecolor{currentstroke}{rgb}{0.850000,0.850000,0.850000}%
\pgfsetstrokecolor{currentstroke}%
\pgfsetdash{}{0pt}%
\pgfpathmoveto{\pgfqpoint{3.787293in}{0.417642in}}%
\pgfpathlineto{\pgfqpoint{3.787293in}{2.429177in}}%
\pgfusepath{stroke}%
\end{pgfscope}%
\begin{pgfscope}%
\pgfsetbuttcap%
\pgfsetroundjoin%
\definecolor{currentfill}{rgb}{0.000000,0.000000,0.000000}%
\pgfsetfillcolor{currentfill}%
\pgfsetlinewidth{0.602250pt}%
\definecolor{currentstroke}{rgb}{0.000000,0.000000,0.000000}%
\pgfsetstrokecolor{currentstroke}%
\pgfsetdash{}{0pt}%
\pgfsys@defobject{currentmarker}{\pgfqpoint{0.000000in}{-0.027778in}}{\pgfqpoint{0.000000in}{0.000000in}}{%
\pgfpathmoveto{\pgfqpoint{0.000000in}{0.000000in}}%
\pgfpathlineto{\pgfqpoint{0.000000in}{-0.027778in}}%
\pgfusepath{stroke,fill}%
}%
\begin{pgfscope}%
\pgfsys@transformshift{3.787293in}{0.417642in}%
\pgfsys@useobject{currentmarker}{}%
\end{pgfscope}%
\end{pgfscope}%
\begin{pgfscope}%
\pgfpathrectangle{\pgfqpoint{0.594525in}{0.417642in}}{\pgfqpoint{3.423805in}{2.011535in}}%
\pgfusepath{clip}%
\pgfsetrectcap%
\pgfsetroundjoin%
\pgfsetlinewidth{0.803000pt}%
\definecolor{currentstroke}{rgb}{0.850000,0.850000,0.850000}%
\pgfsetstrokecolor{currentstroke}%
\pgfsetdash{}{0pt}%
\pgfpathmoveto{\pgfqpoint{3.827097in}{0.417642in}}%
\pgfpathlineto{\pgfqpoint{3.827097in}{2.429177in}}%
\pgfusepath{stroke}%
\end{pgfscope}%
\begin{pgfscope}%
\pgfsetbuttcap%
\pgfsetroundjoin%
\definecolor{currentfill}{rgb}{0.000000,0.000000,0.000000}%
\pgfsetfillcolor{currentfill}%
\pgfsetlinewidth{0.602250pt}%
\definecolor{currentstroke}{rgb}{0.000000,0.000000,0.000000}%
\pgfsetstrokecolor{currentstroke}%
\pgfsetdash{}{0pt}%
\pgfsys@defobject{currentmarker}{\pgfqpoint{0.000000in}{-0.027778in}}{\pgfqpoint{0.000000in}{0.000000in}}{%
\pgfpathmoveto{\pgfqpoint{0.000000in}{0.000000in}}%
\pgfpathlineto{\pgfqpoint{0.000000in}{-0.027778in}}%
\pgfusepath{stroke,fill}%
}%
\begin{pgfscope}%
\pgfsys@transformshift{3.827097in}{0.417642in}%
\pgfsys@useobject{currentmarker}{}%
\end{pgfscope}%
\end{pgfscope}%
\begin{pgfscope}%
\definecolor{textcolor}{rgb}{0.000000,0.000000,0.000000}%
\pgfsetstrokecolor{textcolor}%
\pgfsetfillcolor{textcolor}%
\pgftext[x=2.306427in,y=0.165003in,,top]{\color{textcolor}\rmfamily\fontsize{10.000000}{12.000000}\selectfont Frequency in \(\displaystyle \unit{\Hz}\)}%
\end{pgfscope}%
\begin{pgfscope}%
\pgfpathrectangle{\pgfqpoint{0.594525in}{0.417642in}}{\pgfqpoint{3.423805in}{2.011535in}}%
\pgfusepath{clip}%
\pgfsetrectcap%
\pgfsetroundjoin%
\pgfsetlinewidth{0.803000pt}%
\definecolor{currentstroke}{rgb}{0.450000,0.450000,0.450000}%
\pgfsetstrokecolor{currentstroke}%
\pgfsetdash{}{0pt}%
\pgfpathmoveto{\pgfqpoint{0.594525in}{0.417642in}}%
\pgfpathlineto{\pgfqpoint{4.018330in}{0.417642in}}%
\pgfusepath{stroke}%
\end{pgfscope}%
\begin{pgfscope}%
\pgfsetbuttcap%
\pgfsetroundjoin%
\definecolor{currentfill}{rgb}{0.000000,0.000000,0.000000}%
\pgfsetfillcolor{currentfill}%
\pgfsetlinewidth{0.803000pt}%
\definecolor{currentstroke}{rgb}{0.000000,0.000000,0.000000}%
\pgfsetstrokecolor{currentstroke}%
\pgfsetdash{}{0pt}%
\pgfsys@defobject{currentmarker}{\pgfqpoint{-0.048611in}{0.000000in}}{\pgfqpoint{-0.000000in}{0.000000in}}{%
\pgfpathmoveto{\pgfqpoint{-0.000000in}{0.000000in}}%
\pgfpathlineto{\pgfqpoint{-0.048611in}{0.000000in}}%
\pgfusepath{stroke,fill}%
}%
\begin{pgfscope}%
\pgfsys@transformshift{0.594525in}{0.417642in}%
\pgfsys@useobject{currentmarker}{}%
\end{pgfscope}%
\end{pgfscope}%
\begin{pgfscope}%
\definecolor{textcolor}{rgb}{0.000000,0.000000,0.000000}%
\pgfsetstrokecolor{textcolor}%
\pgfsetfillcolor{textcolor}%
\pgftext[x=0.241129in, y=0.378489in, left, base]{\color{textcolor}\rmfamily\fontsize{8.000000}{9.600000}\selectfont \(\displaystyle {10^{-4}}\)}%
\end{pgfscope}%
\begin{pgfscope}%
\pgfpathrectangle{\pgfqpoint{0.594525in}{0.417642in}}{\pgfqpoint{3.423805in}{2.011535in}}%
\pgfusepath{clip}%
\pgfsetrectcap%
\pgfsetroundjoin%
\pgfsetlinewidth{0.803000pt}%
\definecolor{currentstroke}{rgb}{0.450000,0.450000,0.450000}%
\pgfsetstrokecolor{currentstroke}%
\pgfsetdash{}{0pt}%
\pgfpathmoveto{\pgfqpoint{0.594525in}{0.819949in}}%
\pgfpathlineto{\pgfqpoint{4.018330in}{0.819949in}}%
\pgfusepath{stroke}%
\end{pgfscope}%
\begin{pgfscope}%
\pgfsetbuttcap%
\pgfsetroundjoin%
\definecolor{currentfill}{rgb}{0.000000,0.000000,0.000000}%
\pgfsetfillcolor{currentfill}%
\pgfsetlinewidth{0.803000pt}%
\definecolor{currentstroke}{rgb}{0.000000,0.000000,0.000000}%
\pgfsetstrokecolor{currentstroke}%
\pgfsetdash{}{0pt}%
\pgfsys@defobject{currentmarker}{\pgfqpoint{-0.048611in}{0.000000in}}{\pgfqpoint{-0.000000in}{0.000000in}}{%
\pgfpathmoveto{\pgfqpoint{-0.000000in}{0.000000in}}%
\pgfpathlineto{\pgfqpoint{-0.048611in}{0.000000in}}%
\pgfusepath{stroke,fill}%
}%
\begin{pgfscope}%
\pgfsys@transformshift{0.594525in}{0.819949in}%
\pgfsys@useobject{currentmarker}{}%
\end{pgfscope}%
\end{pgfscope}%
\begin{pgfscope}%
\definecolor{textcolor}{rgb}{0.000000,0.000000,0.000000}%
\pgfsetstrokecolor{textcolor}%
\pgfsetfillcolor{textcolor}%
\pgftext[x=0.241129in, y=0.780796in, left, base]{\color{textcolor}\rmfamily\fontsize{8.000000}{9.600000}\selectfont \(\displaystyle {10^{-3}}\)}%
\end{pgfscope}%
\begin{pgfscope}%
\pgfpathrectangle{\pgfqpoint{0.594525in}{0.417642in}}{\pgfqpoint{3.423805in}{2.011535in}}%
\pgfusepath{clip}%
\pgfsetrectcap%
\pgfsetroundjoin%
\pgfsetlinewidth{0.803000pt}%
\definecolor{currentstroke}{rgb}{0.450000,0.450000,0.450000}%
\pgfsetstrokecolor{currentstroke}%
\pgfsetdash{}{0pt}%
\pgfpathmoveto{\pgfqpoint{0.594525in}{1.222256in}}%
\pgfpathlineto{\pgfqpoint{4.018330in}{1.222256in}}%
\pgfusepath{stroke}%
\end{pgfscope}%
\begin{pgfscope}%
\pgfsetbuttcap%
\pgfsetroundjoin%
\definecolor{currentfill}{rgb}{0.000000,0.000000,0.000000}%
\pgfsetfillcolor{currentfill}%
\pgfsetlinewidth{0.803000pt}%
\definecolor{currentstroke}{rgb}{0.000000,0.000000,0.000000}%
\pgfsetstrokecolor{currentstroke}%
\pgfsetdash{}{0pt}%
\pgfsys@defobject{currentmarker}{\pgfqpoint{-0.048611in}{0.000000in}}{\pgfqpoint{-0.000000in}{0.000000in}}{%
\pgfpathmoveto{\pgfqpoint{-0.000000in}{0.000000in}}%
\pgfpathlineto{\pgfqpoint{-0.048611in}{0.000000in}}%
\pgfusepath{stroke,fill}%
}%
\begin{pgfscope}%
\pgfsys@transformshift{0.594525in}{1.222256in}%
\pgfsys@useobject{currentmarker}{}%
\end{pgfscope}%
\end{pgfscope}%
\begin{pgfscope}%
\definecolor{textcolor}{rgb}{0.000000,0.000000,0.000000}%
\pgfsetstrokecolor{textcolor}%
\pgfsetfillcolor{textcolor}%
\pgftext[x=0.241129in, y=1.183103in, left, base]{\color{textcolor}\rmfamily\fontsize{8.000000}{9.600000}\selectfont \(\displaystyle {10^{-2}}\)}%
\end{pgfscope}%
\begin{pgfscope}%
\pgfpathrectangle{\pgfqpoint{0.594525in}{0.417642in}}{\pgfqpoint{3.423805in}{2.011535in}}%
\pgfusepath{clip}%
\pgfsetrectcap%
\pgfsetroundjoin%
\pgfsetlinewidth{0.803000pt}%
\definecolor{currentstroke}{rgb}{0.450000,0.450000,0.450000}%
\pgfsetstrokecolor{currentstroke}%
\pgfsetdash{}{0pt}%
\pgfpathmoveto{\pgfqpoint{0.594525in}{1.624563in}}%
\pgfpathlineto{\pgfqpoint{4.018330in}{1.624563in}}%
\pgfusepath{stroke}%
\end{pgfscope}%
\begin{pgfscope}%
\pgfsetbuttcap%
\pgfsetroundjoin%
\definecolor{currentfill}{rgb}{0.000000,0.000000,0.000000}%
\pgfsetfillcolor{currentfill}%
\pgfsetlinewidth{0.803000pt}%
\definecolor{currentstroke}{rgb}{0.000000,0.000000,0.000000}%
\pgfsetstrokecolor{currentstroke}%
\pgfsetdash{}{0pt}%
\pgfsys@defobject{currentmarker}{\pgfqpoint{-0.048611in}{0.000000in}}{\pgfqpoint{-0.000000in}{0.000000in}}{%
\pgfpathmoveto{\pgfqpoint{-0.000000in}{0.000000in}}%
\pgfpathlineto{\pgfqpoint{-0.048611in}{0.000000in}}%
\pgfusepath{stroke,fill}%
}%
\begin{pgfscope}%
\pgfsys@transformshift{0.594525in}{1.624563in}%
\pgfsys@useobject{currentmarker}{}%
\end{pgfscope}%
\end{pgfscope}%
\begin{pgfscope}%
\definecolor{textcolor}{rgb}{0.000000,0.000000,0.000000}%
\pgfsetstrokecolor{textcolor}%
\pgfsetfillcolor{textcolor}%
\pgftext[x=0.241129in, y=1.585410in, left, base]{\color{textcolor}\rmfamily\fontsize{8.000000}{9.600000}\selectfont \(\displaystyle {10^{-1}}\)}%
\end{pgfscope}%
\begin{pgfscope}%
\pgfpathrectangle{\pgfqpoint{0.594525in}{0.417642in}}{\pgfqpoint{3.423805in}{2.011535in}}%
\pgfusepath{clip}%
\pgfsetrectcap%
\pgfsetroundjoin%
\pgfsetlinewidth{0.803000pt}%
\definecolor{currentstroke}{rgb}{0.450000,0.450000,0.450000}%
\pgfsetstrokecolor{currentstroke}%
\pgfsetdash{}{0pt}%
\pgfpathmoveto{\pgfqpoint{0.594525in}{2.026870in}}%
\pgfpathlineto{\pgfqpoint{4.018330in}{2.026870in}}%
\pgfusepath{stroke}%
\end{pgfscope}%
\begin{pgfscope}%
\pgfsetbuttcap%
\pgfsetroundjoin%
\definecolor{currentfill}{rgb}{0.000000,0.000000,0.000000}%
\pgfsetfillcolor{currentfill}%
\pgfsetlinewidth{0.803000pt}%
\definecolor{currentstroke}{rgb}{0.000000,0.000000,0.000000}%
\pgfsetstrokecolor{currentstroke}%
\pgfsetdash{}{0pt}%
\pgfsys@defobject{currentmarker}{\pgfqpoint{-0.048611in}{0.000000in}}{\pgfqpoint{-0.000000in}{0.000000in}}{%
\pgfpathmoveto{\pgfqpoint{-0.000000in}{0.000000in}}%
\pgfpathlineto{\pgfqpoint{-0.048611in}{0.000000in}}%
\pgfusepath{stroke,fill}%
}%
\begin{pgfscope}%
\pgfsys@transformshift{0.594525in}{2.026870in}%
\pgfsys@useobject{currentmarker}{}%
\end{pgfscope}%
\end{pgfscope}%
\begin{pgfscope}%
\definecolor{textcolor}{rgb}{0.000000,0.000000,0.000000}%
\pgfsetstrokecolor{textcolor}%
\pgfsetfillcolor{textcolor}%
\pgftext[x=0.321376in, y=1.987717in, left, base]{\color{textcolor}\rmfamily\fontsize{8.000000}{9.600000}\selectfont \(\displaystyle {10^{0}}\)}%
\end{pgfscope}%
\begin{pgfscope}%
\pgfpathrectangle{\pgfqpoint{0.594525in}{0.417642in}}{\pgfqpoint{3.423805in}{2.011535in}}%
\pgfusepath{clip}%
\pgfsetrectcap%
\pgfsetroundjoin%
\pgfsetlinewidth{0.803000pt}%
\definecolor{currentstroke}{rgb}{0.450000,0.450000,0.450000}%
\pgfsetstrokecolor{currentstroke}%
\pgfsetdash{}{0pt}%
\pgfpathmoveto{\pgfqpoint{0.594525in}{2.429177in}}%
\pgfpathlineto{\pgfqpoint{4.018330in}{2.429177in}}%
\pgfusepath{stroke}%
\end{pgfscope}%
\begin{pgfscope}%
\pgfsetbuttcap%
\pgfsetroundjoin%
\definecolor{currentfill}{rgb}{0.000000,0.000000,0.000000}%
\pgfsetfillcolor{currentfill}%
\pgfsetlinewidth{0.803000pt}%
\definecolor{currentstroke}{rgb}{0.000000,0.000000,0.000000}%
\pgfsetstrokecolor{currentstroke}%
\pgfsetdash{}{0pt}%
\pgfsys@defobject{currentmarker}{\pgfqpoint{-0.048611in}{0.000000in}}{\pgfqpoint{-0.000000in}{0.000000in}}{%
\pgfpathmoveto{\pgfqpoint{-0.000000in}{0.000000in}}%
\pgfpathlineto{\pgfqpoint{-0.048611in}{0.000000in}}%
\pgfusepath{stroke,fill}%
}%
\begin{pgfscope}%
\pgfsys@transformshift{0.594525in}{2.429177in}%
\pgfsys@useobject{currentmarker}{}%
\end{pgfscope}%
\end{pgfscope}%
\begin{pgfscope}%
\definecolor{textcolor}{rgb}{0.000000,0.000000,0.000000}%
\pgfsetstrokecolor{textcolor}%
\pgfsetfillcolor{textcolor}%
\pgftext[x=0.321376in, y=2.390024in, left, base]{\color{textcolor}\rmfamily\fontsize{8.000000}{9.600000}\selectfont \(\displaystyle {10^{1}}\)}%
\end{pgfscope}%
\begin{pgfscope}%
\pgfpathrectangle{\pgfqpoint{0.594525in}{0.417642in}}{\pgfqpoint{3.423805in}{2.011535in}}%
\pgfusepath{clip}%
\pgfsetrectcap%
\pgfsetroundjoin%
\pgfsetlinewidth{0.803000pt}%
\definecolor{currentstroke}{rgb}{0.850000,0.850000,0.850000}%
\pgfsetstrokecolor{currentstroke}%
\pgfsetdash{}{0pt}%
\pgfpathmoveto{\pgfqpoint{0.594525in}{0.538748in}}%
\pgfpathlineto{\pgfqpoint{4.018330in}{0.538748in}}%
\pgfusepath{stroke}%
\end{pgfscope}%
\begin{pgfscope}%
\pgfsetbuttcap%
\pgfsetroundjoin%
\definecolor{currentfill}{rgb}{0.000000,0.000000,0.000000}%
\pgfsetfillcolor{currentfill}%
\pgfsetlinewidth{0.602250pt}%
\definecolor{currentstroke}{rgb}{0.000000,0.000000,0.000000}%
\pgfsetstrokecolor{currentstroke}%
\pgfsetdash{}{0pt}%
\pgfsys@defobject{currentmarker}{\pgfqpoint{-0.027778in}{0.000000in}}{\pgfqpoint{-0.000000in}{0.000000in}}{%
\pgfpathmoveto{\pgfqpoint{-0.000000in}{0.000000in}}%
\pgfpathlineto{\pgfqpoint{-0.027778in}{0.000000in}}%
\pgfusepath{stroke,fill}%
}%
\begin{pgfscope}%
\pgfsys@transformshift{0.594525in}{0.538748in}%
\pgfsys@useobject{currentmarker}{}%
\end{pgfscope}%
\end{pgfscope}%
\begin{pgfscope}%
\pgfpathrectangle{\pgfqpoint{0.594525in}{0.417642in}}{\pgfqpoint{3.423805in}{2.011535in}}%
\pgfusepath{clip}%
\pgfsetrectcap%
\pgfsetroundjoin%
\pgfsetlinewidth{0.803000pt}%
\definecolor{currentstroke}{rgb}{0.850000,0.850000,0.850000}%
\pgfsetstrokecolor{currentstroke}%
\pgfsetdash{}{0pt}%
\pgfpathmoveto{\pgfqpoint{0.594525in}{0.609591in}}%
\pgfpathlineto{\pgfqpoint{4.018330in}{0.609591in}}%
\pgfusepath{stroke}%
\end{pgfscope}%
\begin{pgfscope}%
\pgfsetbuttcap%
\pgfsetroundjoin%
\definecolor{currentfill}{rgb}{0.000000,0.000000,0.000000}%
\pgfsetfillcolor{currentfill}%
\pgfsetlinewidth{0.602250pt}%
\definecolor{currentstroke}{rgb}{0.000000,0.000000,0.000000}%
\pgfsetstrokecolor{currentstroke}%
\pgfsetdash{}{0pt}%
\pgfsys@defobject{currentmarker}{\pgfqpoint{-0.027778in}{0.000000in}}{\pgfqpoint{-0.000000in}{0.000000in}}{%
\pgfpathmoveto{\pgfqpoint{-0.000000in}{0.000000in}}%
\pgfpathlineto{\pgfqpoint{-0.027778in}{0.000000in}}%
\pgfusepath{stroke,fill}%
}%
\begin{pgfscope}%
\pgfsys@transformshift{0.594525in}{0.609591in}%
\pgfsys@useobject{currentmarker}{}%
\end{pgfscope}%
\end{pgfscope}%
\begin{pgfscope}%
\pgfpathrectangle{\pgfqpoint{0.594525in}{0.417642in}}{\pgfqpoint{3.423805in}{2.011535in}}%
\pgfusepath{clip}%
\pgfsetrectcap%
\pgfsetroundjoin%
\pgfsetlinewidth{0.803000pt}%
\definecolor{currentstroke}{rgb}{0.850000,0.850000,0.850000}%
\pgfsetstrokecolor{currentstroke}%
\pgfsetdash{}{0pt}%
\pgfpathmoveto{\pgfqpoint{0.594525in}{0.659855in}}%
\pgfpathlineto{\pgfqpoint{4.018330in}{0.659855in}}%
\pgfusepath{stroke}%
\end{pgfscope}%
\begin{pgfscope}%
\pgfsetbuttcap%
\pgfsetroundjoin%
\definecolor{currentfill}{rgb}{0.000000,0.000000,0.000000}%
\pgfsetfillcolor{currentfill}%
\pgfsetlinewidth{0.602250pt}%
\definecolor{currentstroke}{rgb}{0.000000,0.000000,0.000000}%
\pgfsetstrokecolor{currentstroke}%
\pgfsetdash{}{0pt}%
\pgfsys@defobject{currentmarker}{\pgfqpoint{-0.027778in}{0.000000in}}{\pgfqpoint{-0.000000in}{0.000000in}}{%
\pgfpathmoveto{\pgfqpoint{-0.000000in}{0.000000in}}%
\pgfpathlineto{\pgfqpoint{-0.027778in}{0.000000in}}%
\pgfusepath{stroke,fill}%
}%
\begin{pgfscope}%
\pgfsys@transformshift{0.594525in}{0.659855in}%
\pgfsys@useobject{currentmarker}{}%
\end{pgfscope}%
\end{pgfscope}%
\begin{pgfscope}%
\pgfpathrectangle{\pgfqpoint{0.594525in}{0.417642in}}{\pgfqpoint{3.423805in}{2.011535in}}%
\pgfusepath{clip}%
\pgfsetrectcap%
\pgfsetroundjoin%
\pgfsetlinewidth{0.803000pt}%
\definecolor{currentstroke}{rgb}{0.850000,0.850000,0.850000}%
\pgfsetstrokecolor{currentstroke}%
\pgfsetdash{}{0pt}%
\pgfpathmoveto{\pgfqpoint{0.594525in}{0.698843in}}%
\pgfpathlineto{\pgfqpoint{4.018330in}{0.698843in}}%
\pgfusepath{stroke}%
\end{pgfscope}%
\begin{pgfscope}%
\pgfsetbuttcap%
\pgfsetroundjoin%
\definecolor{currentfill}{rgb}{0.000000,0.000000,0.000000}%
\pgfsetfillcolor{currentfill}%
\pgfsetlinewidth{0.602250pt}%
\definecolor{currentstroke}{rgb}{0.000000,0.000000,0.000000}%
\pgfsetstrokecolor{currentstroke}%
\pgfsetdash{}{0pt}%
\pgfsys@defobject{currentmarker}{\pgfqpoint{-0.027778in}{0.000000in}}{\pgfqpoint{-0.000000in}{0.000000in}}{%
\pgfpathmoveto{\pgfqpoint{-0.000000in}{0.000000in}}%
\pgfpathlineto{\pgfqpoint{-0.027778in}{0.000000in}}%
\pgfusepath{stroke,fill}%
}%
\begin{pgfscope}%
\pgfsys@transformshift{0.594525in}{0.698843in}%
\pgfsys@useobject{currentmarker}{}%
\end{pgfscope}%
\end{pgfscope}%
\begin{pgfscope}%
\pgfpathrectangle{\pgfqpoint{0.594525in}{0.417642in}}{\pgfqpoint{3.423805in}{2.011535in}}%
\pgfusepath{clip}%
\pgfsetrectcap%
\pgfsetroundjoin%
\pgfsetlinewidth{0.803000pt}%
\definecolor{currentstroke}{rgb}{0.850000,0.850000,0.850000}%
\pgfsetstrokecolor{currentstroke}%
\pgfsetdash{}{0pt}%
\pgfpathmoveto{\pgfqpoint{0.594525in}{0.730698in}}%
\pgfpathlineto{\pgfqpoint{4.018330in}{0.730698in}}%
\pgfusepath{stroke}%
\end{pgfscope}%
\begin{pgfscope}%
\pgfsetbuttcap%
\pgfsetroundjoin%
\definecolor{currentfill}{rgb}{0.000000,0.000000,0.000000}%
\pgfsetfillcolor{currentfill}%
\pgfsetlinewidth{0.602250pt}%
\definecolor{currentstroke}{rgb}{0.000000,0.000000,0.000000}%
\pgfsetstrokecolor{currentstroke}%
\pgfsetdash{}{0pt}%
\pgfsys@defobject{currentmarker}{\pgfqpoint{-0.027778in}{0.000000in}}{\pgfqpoint{-0.000000in}{0.000000in}}{%
\pgfpathmoveto{\pgfqpoint{-0.000000in}{0.000000in}}%
\pgfpathlineto{\pgfqpoint{-0.027778in}{0.000000in}}%
\pgfusepath{stroke,fill}%
}%
\begin{pgfscope}%
\pgfsys@transformshift{0.594525in}{0.730698in}%
\pgfsys@useobject{currentmarker}{}%
\end{pgfscope}%
\end{pgfscope}%
\begin{pgfscope}%
\pgfpathrectangle{\pgfqpoint{0.594525in}{0.417642in}}{\pgfqpoint{3.423805in}{2.011535in}}%
\pgfusepath{clip}%
\pgfsetrectcap%
\pgfsetroundjoin%
\pgfsetlinewidth{0.803000pt}%
\definecolor{currentstroke}{rgb}{0.850000,0.850000,0.850000}%
\pgfsetstrokecolor{currentstroke}%
\pgfsetdash{}{0pt}%
\pgfpathmoveto{\pgfqpoint{0.594525in}{0.757631in}}%
\pgfpathlineto{\pgfqpoint{4.018330in}{0.757631in}}%
\pgfusepath{stroke}%
\end{pgfscope}%
\begin{pgfscope}%
\pgfsetbuttcap%
\pgfsetroundjoin%
\definecolor{currentfill}{rgb}{0.000000,0.000000,0.000000}%
\pgfsetfillcolor{currentfill}%
\pgfsetlinewidth{0.602250pt}%
\definecolor{currentstroke}{rgb}{0.000000,0.000000,0.000000}%
\pgfsetstrokecolor{currentstroke}%
\pgfsetdash{}{0pt}%
\pgfsys@defobject{currentmarker}{\pgfqpoint{-0.027778in}{0.000000in}}{\pgfqpoint{-0.000000in}{0.000000in}}{%
\pgfpathmoveto{\pgfqpoint{-0.000000in}{0.000000in}}%
\pgfpathlineto{\pgfqpoint{-0.027778in}{0.000000in}}%
\pgfusepath{stroke,fill}%
}%
\begin{pgfscope}%
\pgfsys@transformshift{0.594525in}{0.757631in}%
\pgfsys@useobject{currentmarker}{}%
\end{pgfscope}%
\end{pgfscope}%
\begin{pgfscope}%
\pgfpathrectangle{\pgfqpoint{0.594525in}{0.417642in}}{\pgfqpoint{3.423805in}{2.011535in}}%
\pgfusepath{clip}%
\pgfsetrectcap%
\pgfsetroundjoin%
\pgfsetlinewidth{0.803000pt}%
\definecolor{currentstroke}{rgb}{0.850000,0.850000,0.850000}%
\pgfsetstrokecolor{currentstroke}%
\pgfsetdash{}{0pt}%
\pgfpathmoveto{\pgfqpoint{0.594525in}{0.780961in}}%
\pgfpathlineto{\pgfqpoint{4.018330in}{0.780961in}}%
\pgfusepath{stroke}%
\end{pgfscope}%
\begin{pgfscope}%
\pgfsetbuttcap%
\pgfsetroundjoin%
\definecolor{currentfill}{rgb}{0.000000,0.000000,0.000000}%
\pgfsetfillcolor{currentfill}%
\pgfsetlinewidth{0.602250pt}%
\definecolor{currentstroke}{rgb}{0.000000,0.000000,0.000000}%
\pgfsetstrokecolor{currentstroke}%
\pgfsetdash{}{0pt}%
\pgfsys@defobject{currentmarker}{\pgfqpoint{-0.027778in}{0.000000in}}{\pgfqpoint{-0.000000in}{0.000000in}}{%
\pgfpathmoveto{\pgfqpoint{-0.000000in}{0.000000in}}%
\pgfpathlineto{\pgfqpoint{-0.027778in}{0.000000in}}%
\pgfusepath{stroke,fill}%
}%
\begin{pgfscope}%
\pgfsys@transformshift{0.594525in}{0.780961in}%
\pgfsys@useobject{currentmarker}{}%
\end{pgfscope}%
\end{pgfscope}%
\begin{pgfscope}%
\pgfpathrectangle{\pgfqpoint{0.594525in}{0.417642in}}{\pgfqpoint{3.423805in}{2.011535in}}%
\pgfusepath{clip}%
\pgfsetrectcap%
\pgfsetroundjoin%
\pgfsetlinewidth{0.803000pt}%
\definecolor{currentstroke}{rgb}{0.850000,0.850000,0.850000}%
\pgfsetstrokecolor{currentstroke}%
\pgfsetdash{}{0pt}%
\pgfpathmoveto{\pgfqpoint{0.594525in}{0.801540in}}%
\pgfpathlineto{\pgfqpoint{4.018330in}{0.801540in}}%
\pgfusepath{stroke}%
\end{pgfscope}%
\begin{pgfscope}%
\pgfsetbuttcap%
\pgfsetroundjoin%
\definecolor{currentfill}{rgb}{0.000000,0.000000,0.000000}%
\pgfsetfillcolor{currentfill}%
\pgfsetlinewidth{0.602250pt}%
\definecolor{currentstroke}{rgb}{0.000000,0.000000,0.000000}%
\pgfsetstrokecolor{currentstroke}%
\pgfsetdash{}{0pt}%
\pgfsys@defobject{currentmarker}{\pgfqpoint{-0.027778in}{0.000000in}}{\pgfqpoint{-0.000000in}{0.000000in}}{%
\pgfpathmoveto{\pgfqpoint{-0.000000in}{0.000000in}}%
\pgfpathlineto{\pgfqpoint{-0.027778in}{0.000000in}}%
\pgfusepath{stroke,fill}%
}%
\begin{pgfscope}%
\pgfsys@transformshift{0.594525in}{0.801540in}%
\pgfsys@useobject{currentmarker}{}%
\end{pgfscope}%
\end{pgfscope}%
\begin{pgfscope}%
\pgfpathrectangle{\pgfqpoint{0.594525in}{0.417642in}}{\pgfqpoint{3.423805in}{2.011535in}}%
\pgfusepath{clip}%
\pgfsetrectcap%
\pgfsetroundjoin%
\pgfsetlinewidth{0.803000pt}%
\definecolor{currentstroke}{rgb}{0.850000,0.850000,0.850000}%
\pgfsetstrokecolor{currentstroke}%
\pgfsetdash{}{0pt}%
\pgfpathmoveto{\pgfqpoint{0.594525in}{0.941055in}}%
\pgfpathlineto{\pgfqpoint{4.018330in}{0.941055in}}%
\pgfusepath{stroke}%
\end{pgfscope}%
\begin{pgfscope}%
\pgfsetbuttcap%
\pgfsetroundjoin%
\definecolor{currentfill}{rgb}{0.000000,0.000000,0.000000}%
\pgfsetfillcolor{currentfill}%
\pgfsetlinewidth{0.602250pt}%
\definecolor{currentstroke}{rgb}{0.000000,0.000000,0.000000}%
\pgfsetstrokecolor{currentstroke}%
\pgfsetdash{}{0pt}%
\pgfsys@defobject{currentmarker}{\pgfqpoint{-0.027778in}{0.000000in}}{\pgfqpoint{-0.000000in}{0.000000in}}{%
\pgfpathmoveto{\pgfqpoint{-0.000000in}{0.000000in}}%
\pgfpathlineto{\pgfqpoint{-0.027778in}{0.000000in}}%
\pgfusepath{stroke,fill}%
}%
\begin{pgfscope}%
\pgfsys@transformshift{0.594525in}{0.941055in}%
\pgfsys@useobject{currentmarker}{}%
\end{pgfscope}%
\end{pgfscope}%
\begin{pgfscope}%
\pgfpathrectangle{\pgfqpoint{0.594525in}{0.417642in}}{\pgfqpoint{3.423805in}{2.011535in}}%
\pgfusepath{clip}%
\pgfsetrectcap%
\pgfsetroundjoin%
\pgfsetlinewidth{0.803000pt}%
\definecolor{currentstroke}{rgb}{0.850000,0.850000,0.850000}%
\pgfsetstrokecolor{currentstroke}%
\pgfsetdash{}{0pt}%
\pgfpathmoveto{\pgfqpoint{0.594525in}{1.011898in}}%
\pgfpathlineto{\pgfqpoint{4.018330in}{1.011898in}}%
\pgfusepath{stroke}%
\end{pgfscope}%
\begin{pgfscope}%
\pgfsetbuttcap%
\pgfsetroundjoin%
\definecolor{currentfill}{rgb}{0.000000,0.000000,0.000000}%
\pgfsetfillcolor{currentfill}%
\pgfsetlinewidth{0.602250pt}%
\definecolor{currentstroke}{rgb}{0.000000,0.000000,0.000000}%
\pgfsetstrokecolor{currentstroke}%
\pgfsetdash{}{0pt}%
\pgfsys@defobject{currentmarker}{\pgfqpoint{-0.027778in}{0.000000in}}{\pgfqpoint{-0.000000in}{0.000000in}}{%
\pgfpathmoveto{\pgfqpoint{-0.000000in}{0.000000in}}%
\pgfpathlineto{\pgfqpoint{-0.027778in}{0.000000in}}%
\pgfusepath{stroke,fill}%
}%
\begin{pgfscope}%
\pgfsys@transformshift{0.594525in}{1.011898in}%
\pgfsys@useobject{currentmarker}{}%
\end{pgfscope}%
\end{pgfscope}%
\begin{pgfscope}%
\pgfpathrectangle{\pgfqpoint{0.594525in}{0.417642in}}{\pgfqpoint{3.423805in}{2.011535in}}%
\pgfusepath{clip}%
\pgfsetrectcap%
\pgfsetroundjoin%
\pgfsetlinewidth{0.803000pt}%
\definecolor{currentstroke}{rgb}{0.850000,0.850000,0.850000}%
\pgfsetstrokecolor{currentstroke}%
\pgfsetdash{}{0pt}%
\pgfpathmoveto{\pgfqpoint{0.594525in}{1.062162in}}%
\pgfpathlineto{\pgfqpoint{4.018330in}{1.062162in}}%
\pgfusepath{stroke}%
\end{pgfscope}%
\begin{pgfscope}%
\pgfsetbuttcap%
\pgfsetroundjoin%
\definecolor{currentfill}{rgb}{0.000000,0.000000,0.000000}%
\pgfsetfillcolor{currentfill}%
\pgfsetlinewidth{0.602250pt}%
\definecolor{currentstroke}{rgb}{0.000000,0.000000,0.000000}%
\pgfsetstrokecolor{currentstroke}%
\pgfsetdash{}{0pt}%
\pgfsys@defobject{currentmarker}{\pgfqpoint{-0.027778in}{0.000000in}}{\pgfqpoint{-0.000000in}{0.000000in}}{%
\pgfpathmoveto{\pgfqpoint{-0.000000in}{0.000000in}}%
\pgfpathlineto{\pgfqpoint{-0.027778in}{0.000000in}}%
\pgfusepath{stroke,fill}%
}%
\begin{pgfscope}%
\pgfsys@transformshift{0.594525in}{1.062162in}%
\pgfsys@useobject{currentmarker}{}%
\end{pgfscope}%
\end{pgfscope}%
\begin{pgfscope}%
\pgfpathrectangle{\pgfqpoint{0.594525in}{0.417642in}}{\pgfqpoint{3.423805in}{2.011535in}}%
\pgfusepath{clip}%
\pgfsetrectcap%
\pgfsetroundjoin%
\pgfsetlinewidth{0.803000pt}%
\definecolor{currentstroke}{rgb}{0.850000,0.850000,0.850000}%
\pgfsetstrokecolor{currentstroke}%
\pgfsetdash{}{0pt}%
\pgfpathmoveto{\pgfqpoint{0.594525in}{1.101150in}}%
\pgfpathlineto{\pgfqpoint{4.018330in}{1.101150in}}%
\pgfusepath{stroke}%
\end{pgfscope}%
\begin{pgfscope}%
\pgfsetbuttcap%
\pgfsetroundjoin%
\definecolor{currentfill}{rgb}{0.000000,0.000000,0.000000}%
\pgfsetfillcolor{currentfill}%
\pgfsetlinewidth{0.602250pt}%
\definecolor{currentstroke}{rgb}{0.000000,0.000000,0.000000}%
\pgfsetstrokecolor{currentstroke}%
\pgfsetdash{}{0pt}%
\pgfsys@defobject{currentmarker}{\pgfqpoint{-0.027778in}{0.000000in}}{\pgfqpoint{-0.000000in}{0.000000in}}{%
\pgfpathmoveto{\pgfqpoint{-0.000000in}{0.000000in}}%
\pgfpathlineto{\pgfqpoint{-0.027778in}{0.000000in}}%
\pgfusepath{stroke,fill}%
}%
\begin{pgfscope}%
\pgfsys@transformshift{0.594525in}{1.101150in}%
\pgfsys@useobject{currentmarker}{}%
\end{pgfscope}%
\end{pgfscope}%
\begin{pgfscope}%
\pgfpathrectangle{\pgfqpoint{0.594525in}{0.417642in}}{\pgfqpoint{3.423805in}{2.011535in}}%
\pgfusepath{clip}%
\pgfsetrectcap%
\pgfsetroundjoin%
\pgfsetlinewidth{0.803000pt}%
\definecolor{currentstroke}{rgb}{0.850000,0.850000,0.850000}%
\pgfsetstrokecolor{currentstroke}%
\pgfsetdash{}{0pt}%
\pgfpathmoveto{\pgfqpoint{0.594525in}{1.133005in}}%
\pgfpathlineto{\pgfqpoint{4.018330in}{1.133005in}}%
\pgfusepath{stroke}%
\end{pgfscope}%
\begin{pgfscope}%
\pgfsetbuttcap%
\pgfsetroundjoin%
\definecolor{currentfill}{rgb}{0.000000,0.000000,0.000000}%
\pgfsetfillcolor{currentfill}%
\pgfsetlinewidth{0.602250pt}%
\definecolor{currentstroke}{rgb}{0.000000,0.000000,0.000000}%
\pgfsetstrokecolor{currentstroke}%
\pgfsetdash{}{0pt}%
\pgfsys@defobject{currentmarker}{\pgfqpoint{-0.027778in}{0.000000in}}{\pgfqpoint{-0.000000in}{0.000000in}}{%
\pgfpathmoveto{\pgfqpoint{-0.000000in}{0.000000in}}%
\pgfpathlineto{\pgfqpoint{-0.027778in}{0.000000in}}%
\pgfusepath{stroke,fill}%
}%
\begin{pgfscope}%
\pgfsys@transformshift{0.594525in}{1.133005in}%
\pgfsys@useobject{currentmarker}{}%
\end{pgfscope}%
\end{pgfscope}%
\begin{pgfscope}%
\pgfpathrectangle{\pgfqpoint{0.594525in}{0.417642in}}{\pgfqpoint{3.423805in}{2.011535in}}%
\pgfusepath{clip}%
\pgfsetrectcap%
\pgfsetroundjoin%
\pgfsetlinewidth{0.803000pt}%
\definecolor{currentstroke}{rgb}{0.850000,0.850000,0.850000}%
\pgfsetstrokecolor{currentstroke}%
\pgfsetdash{}{0pt}%
\pgfpathmoveto{\pgfqpoint{0.594525in}{1.159938in}}%
\pgfpathlineto{\pgfqpoint{4.018330in}{1.159938in}}%
\pgfusepath{stroke}%
\end{pgfscope}%
\begin{pgfscope}%
\pgfsetbuttcap%
\pgfsetroundjoin%
\definecolor{currentfill}{rgb}{0.000000,0.000000,0.000000}%
\pgfsetfillcolor{currentfill}%
\pgfsetlinewidth{0.602250pt}%
\definecolor{currentstroke}{rgb}{0.000000,0.000000,0.000000}%
\pgfsetstrokecolor{currentstroke}%
\pgfsetdash{}{0pt}%
\pgfsys@defobject{currentmarker}{\pgfqpoint{-0.027778in}{0.000000in}}{\pgfqpoint{-0.000000in}{0.000000in}}{%
\pgfpathmoveto{\pgfqpoint{-0.000000in}{0.000000in}}%
\pgfpathlineto{\pgfqpoint{-0.027778in}{0.000000in}}%
\pgfusepath{stroke,fill}%
}%
\begin{pgfscope}%
\pgfsys@transformshift{0.594525in}{1.159938in}%
\pgfsys@useobject{currentmarker}{}%
\end{pgfscope}%
\end{pgfscope}%
\begin{pgfscope}%
\pgfpathrectangle{\pgfqpoint{0.594525in}{0.417642in}}{\pgfqpoint{3.423805in}{2.011535in}}%
\pgfusepath{clip}%
\pgfsetrectcap%
\pgfsetroundjoin%
\pgfsetlinewidth{0.803000pt}%
\definecolor{currentstroke}{rgb}{0.850000,0.850000,0.850000}%
\pgfsetstrokecolor{currentstroke}%
\pgfsetdash{}{0pt}%
\pgfpathmoveto{\pgfqpoint{0.594525in}{1.183268in}}%
\pgfpathlineto{\pgfqpoint{4.018330in}{1.183268in}}%
\pgfusepath{stroke}%
\end{pgfscope}%
\begin{pgfscope}%
\pgfsetbuttcap%
\pgfsetroundjoin%
\definecolor{currentfill}{rgb}{0.000000,0.000000,0.000000}%
\pgfsetfillcolor{currentfill}%
\pgfsetlinewidth{0.602250pt}%
\definecolor{currentstroke}{rgb}{0.000000,0.000000,0.000000}%
\pgfsetstrokecolor{currentstroke}%
\pgfsetdash{}{0pt}%
\pgfsys@defobject{currentmarker}{\pgfqpoint{-0.027778in}{0.000000in}}{\pgfqpoint{-0.000000in}{0.000000in}}{%
\pgfpathmoveto{\pgfqpoint{-0.000000in}{0.000000in}}%
\pgfpathlineto{\pgfqpoint{-0.027778in}{0.000000in}}%
\pgfusepath{stroke,fill}%
}%
\begin{pgfscope}%
\pgfsys@transformshift{0.594525in}{1.183268in}%
\pgfsys@useobject{currentmarker}{}%
\end{pgfscope}%
\end{pgfscope}%
\begin{pgfscope}%
\pgfpathrectangle{\pgfqpoint{0.594525in}{0.417642in}}{\pgfqpoint{3.423805in}{2.011535in}}%
\pgfusepath{clip}%
\pgfsetrectcap%
\pgfsetroundjoin%
\pgfsetlinewidth{0.803000pt}%
\definecolor{currentstroke}{rgb}{0.850000,0.850000,0.850000}%
\pgfsetstrokecolor{currentstroke}%
\pgfsetdash{}{0pt}%
\pgfpathmoveto{\pgfqpoint{0.594525in}{1.203847in}}%
\pgfpathlineto{\pgfqpoint{4.018330in}{1.203847in}}%
\pgfusepath{stroke}%
\end{pgfscope}%
\begin{pgfscope}%
\pgfsetbuttcap%
\pgfsetroundjoin%
\definecolor{currentfill}{rgb}{0.000000,0.000000,0.000000}%
\pgfsetfillcolor{currentfill}%
\pgfsetlinewidth{0.602250pt}%
\definecolor{currentstroke}{rgb}{0.000000,0.000000,0.000000}%
\pgfsetstrokecolor{currentstroke}%
\pgfsetdash{}{0pt}%
\pgfsys@defobject{currentmarker}{\pgfqpoint{-0.027778in}{0.000000in}}{\pgfqpoint{-0.000000in}{0.000000in}}{%
\pgfpathmoveto{\pgfqpoint{-0.000000in}{0.000000in}}%
\pgfpathlineto{\pgfqpoint{-0.027778in}{0.000000in}}%
\pgfusepath{stroke,fill}%
}%
\begin{pgfscope}%
\pgfsys@transformshift{0.594525in}{1.203847in}%
\pgfsys@useobject{currentmarker}{}%
\end{pgfscope}%
\end{pgfscope}%
\begin{pgfscope}%
\pgfpathrectangle{\pgfqpoint{0.594525in}{0.417642in}}{\pgfqpoint{3.423805in}{2.011535in}}%
\pgfusepath{clip}%
\pgfsetrectcap%
\pgfsetroundjoin%
\pgfsetlinewidth{0.803000pt}%
\definecolor{currentstroke}{rgb}{0.850000,0.850000,0.850000}%
\pgfsetstrokecolor{currentstroke}%
\pgfsetdash{}{0pt}%
\pgfpathmoveto{\pgfqpoint{0.594525in}{1.343363in}}%
\pgfpathlineto{\pgfqpoint{4.018330in}{1.343363in}}%
\pgfusepath{stroke}%
\end{pgfscope}%
\begin{pgfscope}%
\pgfsetbuttcap%
\pgfsetroundjoin%
\definecolor{currentfill}{rgb}{0.000000,0.000000,0.000000}%
\pgfsetfillcolor{currentfill}%
\pgfsetlinewidth{0.602250pt}%
\definecolor{currentstroke}{rgb}{0.000000,0.000000,0.000000}%
\pgfsetstrokecolor{currentstroke}%
\pgfsetdash{}{0pt}%
\pgfsys@defobject{currentmarker}{\pgfqpoint{-0.027778in}{0.000000in}}{\pgfqpoint{-0.000000in}{0.000000in}}{%
\pgfpathmoveto{\pgfqpoint{-0.000000in}{0.000000in}}%
\pgfpathlineto{\pgfqpoint{-0.027778in}{0.000000in}}%
\pgfusepath{stroke,fill}%
}%
\begin{pgfscope}%
\pgfsys@transformshift{0.594525in}{1.343363in}%
\pgfsys@useobject{currentmarker}{}%
\end{pgfscope}%
\end{pgfscope}%
\begin{pgfscope}%
\pgfpathrectangle{\pgfqpoint{0.594525in}{0.417642in}}{\pgfqpoint{3.423805in}{2.011535in}}%
\pgfusepath{clip}%
\pgfsetrectcap%
\pgfsetroundjoin%
\pgfsetlinewidth{0.803000pt}%
\definecolor{currentstroke}{rgb}{0.850000,0.850000,0.850000}%
\pgfsetstrokecolor{currentstroke}%
\pgfsetdash{}{0pt}%
\pgfpathmoveto{\pgfqpoint{0.594525in}{1.414205in}}%
\pgfpathlineto{\pgfqpoint{4.018330in}{1.414205in}}%
\pgfusepath{stroke}%
\end{pgfscope}%
\begin{pgfscope}%
\pgfsetbuttcap%
\pgfsetroundjoin%
\definecolor{currentfill}{rgb}{0.000000,0.000000,0.000000}%
\pgfsetfillcolor{currentfill}%
\pgfsetlinewidth{0.602250pt}%
\definecolor{currentstroke}{rgb}{0.000000,0.000000,0.000000}%
\pgfsetstrokecolor{currentstroke}%
\pgfsetdash{}{0pt}%
\pgfsys@defobject{currentmarker}{\pgfqpoint{-0.027778in}{0.000000in}}{\pgfqpoint{-0.000000in}{0.000000in}}{%
\pgfpathmoveto{\pgfqpoint{-0.000000in}{0.000000in}}%
\pgfpathlineto{\pgfqpoint{-0.027778in}{0.000000in}}%
\pgfusepath{stroke,fill}%
}%
\begin{pgfscope}%
\pgfsys@transformshift{0.594525in}{1.414205in}%
\pgfsys@useobject{currentmarker}{}%
\end{pgfscope}%
\end{pgfscope}%
\begin{pgfscope}%
\pgfpathrectangle{\pgfqpoint{0.594525in}{0.417642in}}{\pgfqpoint{3.423805in}{2.011535in}}%
\pgfusepath{clip}%
\pgfsetrectcap%
\pgfsetroundjoin%
\pgfsetlinewidth{0.803000pt}%
\definecolor{currentstroke}{rgb}{0.850000,0.850000,0.850000}%
\pgfsetstrokecolor{currentstroke}%
\pgfsetdash{}{0pt}%
\pgfpathmoveto{\pgfqpoint{0.594525in}{1.464469in}}%
\pgfpathlineto{\pgfqpoint{4.018330in}{1.464469in}}%
\pgfusepath{stroke}%
\end{pgfscope}%
\begin{pgfscope}%
\pgfsetbuttcap%
\pgfsetroundjoin%
\definecolor{currentfill}{rgb}{0.000000,0.000000,0.000000}%
\pgfsetfillcolor{currentfill}%
\pgfsetlinewidth{0.602250pt}%
\definecolor{currentstroke}{rgb}{0.000000,0.000000,0.000000}%
\pgfsetstrokecolor{currentstroke}%
\pgfsetdash{}{0pt}%
\pgfsys@defobject{currentmarker}{\pgfqpoint{-0.027778in}{0.000000in}}{\pgfqpoint{-0.000000in}{0.000000in}}{%
\pgfpathmoveto{\pgfqpoint{-0.000000in}{0.000000in}}%
\pgfpathlineto{\pgfqpoint{-0.027778in}{0.000000in}}%
\pgfusepath{stroke,fill}%
}%
\begin{pgfscope}%
\pgfsys@transformshift{0.594525in}{1.464469in}%
\pgfsys@useobject{currentmarker}{}%
\end{pgfscope}%
\end{pgfscope}%
\begin{pgfscope}%
\pgfpathrectangle{\pgfqpoint{0.594525in}{0.417642in}}{\pgfqpoint{3.423805in}{2.011535in}}%
\pgfusepath{clip}%
\pgfsetrectcap%
\pgfsetroundjoin%
\pgfsetlinewidth{0.803000pt}%
\definecolor{currentstroke}{rgb}{0.850000,0.850000,0.850000}%
\pgfsetstrokecolor{currentstroke}%
\pgfsetdash{}{0pt}%
\pgfpathmoveto{\pgfqpoint{0.594525in}{1.503457in}}%
\pgfpathlineto{\pgfqpoint{4.018330in}{1.503457in}}%
\pgfusepath{stroke}%
\end{pgfscope}%
\begin{pgfscope}%
\pgfsetbuttcap%
\pgfsetroundjoin%
\definecolor{currentfill}{rgb}{0.000000,0.000000,0.000000}%
\pgfsetfillcolor{currentfill}%
\pgfsetlinewidth{0.602250pt}%
\definecolor{currentstroke}{rgb}{0.000000,0.000000,0.000000}%
\pgfsetstrokecolor{currentstroke}%
\pgfsetdash{}{0pt}%
\pgfsys@defobject{currentmarker}{\pgfqpoint{-0.027778in}{0.000000in}}{\pgfqpoint{-0.000000in}{0.000000in}}{%
\pgfpathmoveto{\pgfqpoint{-0.000000in}{0.000000in}}%
\pgfpathlineto{\pgfqpoint{-0.027778in}{0.000000in}}%
\pgfusepath{stroke,fill}%
}%
\begin{pgfscope}%
\pgfsys@transformshift{0.594525in}{1.503457in}%
\pgfsys@useobject{currentmarker}{}%
\end{pgfscope}%
\end{pgfscope}%
\begin{pgfscope}%
\pgfpathrectangle{\pgfqpoint{0.594525in}{0.417642in}}{\pgfqpoint{3.423805in}{2.011535in}}%
\pgfusepath{clip}%
\pgfsetrectcap%
\pgfsetroundjoin%
\pgfsetlinewidth{0.803000pt}%
\definecolor{currentstroke}{rgb}{0.850000,0.850000,0.850000}%
\pgfsetstrokecolor{currentstroke}%
\pgfsetdash{}{0pt}%
\pgfpathmoveto{\pgfqpoint{0.594525in}{1.535312in}}%
\pgfpathlineto{\pgfqpoint{4.018330in}{1.535312in}}%
\pgfusepath{stroke}%
\end{pgfscope}%
\begin{pgfscope}%
\pgfsetbuttcap%
\pgfsetroundjoin%
\definecolor{currentfill}{rgb}{0.000000,0.000000,0.000000}%
\pgfsetfillcolor{currentfill}%
\pgfsetlinewidth{0.602250pt}%
\definecolor{currentstroke}{rgb}{0.000000,0.000000,0.000000}%
\pgfsetstrokecolor{currentstroke}%
\pgfsetdash{}{0pt}%
\pgfsys@defobject{currentmarker}{\pgfqpoint{-0.027778in}{0.000000in}}{\pgfqpoint{-0.000000in}{0.000000in}}{%
\pgfpathmoveto{\pgfqpoint{-0.000000in}{0.000000in}}%
\pgfpathlineto{\pgfqpoint{-0.027778in}{0.000000in}}%
\pgfusepath{stroke,fill}%
}%
\begin{pgfscope}%
\pgfsys@transformshift{0.594525in}{1.535312in}%
\pgfsys@useobject{currentmarker}{}%
\end{pgfscope}%
\end{pgfscope}%
\begin{pgfscope}%
\pgfpathrectangle{\pgfqpoint{0.594525in}{0.417642in}}{\pgfqpoint{3.423805in}{2.011535in}}%
\pgfusepath{clip}%
\pgfsetrectcap%
\pgfsetroundjoin%
\pgfsetlinewidth{0.803000pt}%
\definecolor{currentstroke}{rgb}{0.850000,0.850000,0.850000}%
\pgfsetstrokecolor{currentstroke}%
\pgfsetdash{}{0pt}%
\pgfpathmoveto{\pgfqpoint{0.594525in}{1.562245in}}%
\pgfpathlineto{\pgfqpoint{4.018330in}{1.562245in}}%
\pgfusepath{stroke}%
\end{pgfscope}%
\begin{pgfscope}%
\pgfsetbuttcap%
\pgfsetroundjoin%
\definecolor{currentfill}{rgb}{0.000000,0.000000,0.000000}%
\pgfsetfillcolor{currentfill}%
\pgfsetlinewidth{0.602250pt}%
\definecolor{currentstroke}{rgb}{0.000000,0.000000,0.000000}%
\pgfsetstrokecolor{currentstroke}%
\pgfsetdash{}{0pt}%
\pgfsys@defobject{currentmarker}{\pgfqpoint{-0.027778in}{0.000000in}}{\pgfqpoint{-0.000000in}{0.000000in}}{%
\pgfpathmoveto{\pgfqpoint{-0.000000in}{0.000000in}}%
\pgfpathlineto{\pgfqpoint{-0.027778in}{0.000000in}}%
\pgfusepath{stroke,fill}%
}%
\begin{pgfscope}%
\pgfsys@transformshift{0.594525in}{1.562245in}%
\pgfsys@useobject{currentmarker}{}%
\end{pgfscope}%
\end{pgfscope}%
\begin{pgfscope}%
\pgfpathrectangle{\pgfqpoint{0.594525in}{0.417642in}}{\pgfqpoint{3.423805in}{2.011535in}}%
\pgfusepath{clip}%
\pgfsetrectcap%
\pgfsetroundjoin%
\pgfsetlinewidth{0.803000pt}%
\definecolor{currentstroke}{rgb}{0.850000,0.850000,0.850000}%
\pgfsetstrokecolor{currentstroke}%
\pgfsetdash{}{0pt}%
\pgfpathmoveto{\pgfqpoint{0.594525in}{1.585576in}}%
\pgfpathlineto{\pgfqpoint{4.018330in}{1.585576in}}%
\pgfusepath{stroke}%
\end{pgfscope}%
\begin{pgfscope}%
\pgfsetbuttcap%
\pgfsetroundjoin%
\definecolor{currentfill}{rgb}{0.000000,0.000000,0.000000}%
\pgfsetfillcolor{currentfill}%
\pgfsetlinewidth{0.602250pt}%
\definecolor{currentstroke}{rgb}{0.000000,0.000000,0.000000}%
\pgfsetstrokecolor{currentstroke}%
\pgfsetdash{}{0pt}%
\pgfsys@defobject{currentmarker}{\pgfqpoint{-0.027778in}{0.000000in}}{\pgfqpoint{-0.000000in}{0.000000in}}{%
\pgfpathmoveto{\pgfqpoint{-0.000000in}{0.000000in}}%
\pgfpathlineto{\pgfqpoint{-0.027778in}{0.000000in}}%
\pgfusepath{stroke,fill}%
}%
\begin{pgfscope}%
\pgfsys@transformshift{0.594525in}{1.585576in}%
\pgfsys@useobject{currentmarker}{}%
\end{pgfscope}%
\end{pgfscope}%
\begin{pgfscope}%
\pgfpathrectangle{\pgfqpoint{0.594525in}{0.417642in}}{\pgfqpoint{3.423805in}{2.011535in}}%
\pgfusepath{clip}%
\pgfsetrectcap%
\pgfsetroundjoin%
\pgfsetlinewidth{0.803000pt}%
\definecolor{currentstroke}{rgb}{0.850000,0.850000,0.850000}%
\pgfsetstrokecolor{currentstroke}%
\pgfsetdash{}{0pt}%
\pgfpathmoveto{\pgfqpoint{0.594525in}{1.606155in}}%
\pgfpathlineto{\pgfqpoint{4.018330in}{1.606155in}}%
\pgfusepath{stroke}%
\end{pgfscope}%
\begin{pgfscope}%
\pgfsetbuttcap%
\pgfsetroundjoin%
\definecolor{currentfill}{rgb}{0.000000,0.000000,0.000000}%
\pgfsetfillcolor{currentfill}%
\pgfsetlinewidth{0.602250pt}%
\definecolor{currentstroke}{rgb}{0.000000,0.000000,0.000000}%
\pgfsetstrokecolor{currentstroke}%
\pgfsetdash{}{0pt}%
\pgfsys@defobject{currentmarker}{\pgfqpoint{-0.027778in}{0.000000in}}{\pgfqpoint{-0.000000in}{0.000000in}}{%
\pgfpathmoveto{\pgfqpoint{-0.000000in}{0.000000in}}%
\pgfpathlineto{\pgfqpoint{-0.027778in}{0.000000in}}%
\pgfusepath{stroke,fill}%
}%
\begin{pgfscope}%
\pgfsys@transformshift{0.594525in}{1.606155in}%
\pgfsys@useobject{currentmarker}{}%
\end{pgfscope}%
\end{pgfscope}%
\begin{pgfscope}%
\pgfpathrectangle{\pgfqpoint{0.594525in}{0.417642in}}{\pgfqpoint{3.423805in}{2.011535in}}%
\pgfusepath{clip}%
\pgfsetrectcap%
\pgfsetroundjoin%
\pgfsetlinewidth{0.803000pt}%
\definecolor{currentstroke}{rgb}{0.850000,0.850000,0.850000}%
\pgfsetstrokecolor{currentstroke}%
\pgfsetdash{}{0pt}%
\pgfpathmoveto{\pgfqpoint{0.594525in}{1.745670in}}%
\pgfpathlineto{\pgfqpoint{4.018330in}{1.745670in}}%
\pgfusepath{stroke}%
\end{pgfscope}%
\begin{pgfscope}%
\pgfsetbuttcap%
\pgfsetroundjoin%
\definecolor{currentfill}{rgb}{0.000000,0.000000,0.000000}%
\pgfsetfillcolor{currentfill}%
\pgfsetlinewidth{0.602250pt}%
\definecolor{currentstroke}{rgb}{0.000000,0.000000,0.000000}%
\pgfsetstrokecolor{currentstroke}%
\pgfsetdash{}{0pt}%
\pgfsys@defobject{currentmarker}{\pgfqpoint{-0.027778in}{0.000000in}}{\pgfqpoint{-0.000000in}{0.000000in}}{%
\pgfpathmoveto{\pgfqpoint{-0.000000in}{0.000000in}}%
\pgfpathlineto{\pgfqpoint{-0.027778in}{0.000000in}}%
\pgfusepath{stroke,fill}%
}%
\begin{pgfscope}%
\pgfsys@transformshift{0.594525in}{1.745670in}%
\pgfsys@useobject{currentmarker}{}%
\end{pgfscope}%
\end{pgfscope}%
\begin{pgfscope}%
\pgfpathrectangle{\pgfqpoint{0.594525in}{0.417642in}}{\pgfqpoint{3.423805in}{2.011535in}}%
\pgfusepath{clip}%
\pgfsetrectcap%
\pgfsetroundjoin%
\pgfsetlinewidth{0.803000pt}%
\definecolor{currentstroke}{rgb}{0.850000,0.850000,0.850000}%
\pgfsetstrokecolor{currentstroke}%
\pgfsetdash{}{0pt}%
\pgfpathmoveto{\pgfqpoint{0.594525in}{1.816512in}}%
\pgfpathlineto{\pgfqpoint{4.018330in}{1.816512in}}%
\pgfusepath{stroke}%
\end{pgfscope}%
\begin{pgfscope}%
\pgfsetbuttcap%
\pgfsetroundjoin%
\definecolor{currentfill}{rgb}{0.000000,0.000000,0.000000}%
\pgfsetfillcolor{currentfill}%
\pgfsetlinewidth{0.602250pt}%
\definecolor{currentstroke}{rgb}{0.000000,0.000000,0.000000}%
\pgfsetstrokecolor{currentstroke}%
\pgfsetdash{}{0pt}%
\pgfsys@defobject{currentmarker}{\pgfqpoint{-0.027778in}{0.000000in}}{\pgfqpoint{-0.000000in}{0.000000in}}{%
\pgfpathmoveto{\pgfqpoint{-0.000000in}{0.000000in}}%
\pgfpathlineto{\pgfqpoint{-0.027778in}{0.000000in}}%
\pgfusepath{stroke,fill}%
}%
\begin{pgfscope}%
\pgfsys@transformshift{0.594525in}{1.816512in}%
\pgfsys@useobject{currentmarker}{}%
\end{pgfscope}%
\end{pgfscope}%
\begin{pgfscope}%
\pgfpathrectangle{\pgfqpoint{0.594525in}{0.417642in}}{\pgfqpoint{3.423805in}{2.011535in}}%
\pgfusepath{clip}%
\pgfsetrectcap%
\pgfsetroundjoin%
\pgfsetlinewidth{0.803000pt}%
\definecolor{currentstroke}{rgb}{0.850000,0.850000,0.850000}%
\pgfsetstrokecolor{currentstroke}%
\pgfsetdash{}{0pt}%
\pgfpathmoveto{\pgfqpoint{0.594525in}{1.866776in}}%
\pgfpathlineto{\pgfqpoint{4.018330in}{1.866776in}}%
\pgfusepath{stroke}%
\end{pgfscope}%
\begin{pgfscope}%
\pgfsetbuttcap%
\pgfsetroundjoin%
\definecolor{currentfill}{rgb}{0.000000,0.000000,0.000000}%
\pgfsetfillcolor{currentfill}%
\pgfsetlinewidth{0.602250pt}%
\definecolor{currentstroke}{rgb}{0.000000,0.000000,0.000000}%
\pgfsetstrokecolor{currentstroke}%
\pgfsetdash{}{0pt}%
\pgfsys@defobject{currentmarker}{\pgfqpoint{-0.027778in}{0.000000in}}{\pgfqpoint{-0.000000in}{0.000000in}}{%
\pgfpathmoveto{\pgfqpoint{-0.000000in}{0.000000in}}%
\pgfpathlineto{\pgfqpoint{-0.027778in}{0.000000in}}%
\pgfusepath{stroke,fill}%
}%
\begin{pgfscope}%
\pgfsys@transformshift{0.594525in}{1.866776in}%
\pgfsys@useobject{currentmarker}{}%
\end{pgfscope}%
\end{pgfscope}%
\begin{pgfscope}%
\pgfpathrectangle{\pgfqpoint{0.594525in}{0.417642in}}{\pgfqpoint{3.423805in}{2.011535in}}%
\pgfusepath{clip}%
\pgfsetrectcap%
\pgfsetroundjoin%
\pgfsetlinewidth{0.803000pt}%
\definecolor{currentstroke}{rgb}{0.850000,0.850000,0.850000}%
\pgfsetstrokecolor{currentstroke}%
\pgfsetdash{}{0pt}%
\pgfpathmoveto{\pgfqpoint{0.594525in}{1.905764in}}%
\pgfpathlineto{\pgfqpoint{4.018330in}{1.905764in}}%
\pgfusepath{stroke}%
\end{pgfscope}%
\begin{pgfscope}%
\pgfsetbuttcap%
\pgfsetroundjoin%
\definecolor{currentfill}{rgb}{0.000000,0.000000,0.000000}%
\pgfsetfillcolor{currentfill}%
\pgfsetlinewidth{0.602250pt}%
\definecolor{currentstroke}{rgb}{0.000000,0.000000,0.000000}%
\pgfsetstrokecolor{currentstroke}%
\pgfsetdash{}{0pt}%
\pgfsys@defobject{currentmarker}{\pgfqpoint{-0.027778in}{0.000000in}}{\pgfqpoint{-0.000000in}{0.000000in}}{%
\pgfpathmoveto{\pgfqpoint{-0.000000in}{0.000000in}}%
\pgfpathlineto{\pgfqpoint{-0.027778in}{0.000000in}}%
\pgfusepath{stroke,fill}%
}%
\begin{pgfscope}%
\pgfsys@transformshift{0.594525in}{1.905764in}%
\pgfsys@useobject{currentmarker}{}%
\end{pgfscope}%
\end{pgfscope}%
\begin{pgfscope}%
\pgfpathrectangle{\pgfqpoint{0.594525in}{0.417642in}}{\pgfqpoint{3.423805in}{2.011535in}}%
\pgfusepath{clip}%
\pgfsetrectcap%
\pgfsetroundjoin%
\pgfsetlinewidth{0.803000pt}%
\definecolor{currentstroke}{rgb}{0.850000,0.850000,0.850000}%
\pgfsetstrokecolor{currentstroke}%
\pgfsetdash{}{0pt}%
\pgfpathmoveto{\pgfqpoint{0.594525in}{1.937619in}}%
\pgfpathlineto{\pgfqpoint{4.018330in}{1.937619in}}%
\pgfusepath{stroke}%
\end{pgfscope}%
\begin{pgfscope}%
\pgfsetbuttcap%
\pgfsetroundjoin%
\definecolor{currentfill}{rgb}{0.000000,0.000000,0.000000}%
\pgfsetfillcolor{currentfill}%
\pgfsetlinewidth{0.602250pt}%
\definecolor{currentstroke}{rgb}{0.000000,0.000000,0.000000}%
\pgfsetstrokecolor{currentstroke}%
\pgfsetdash{}{0pt}%
\pgfsys@defobject{currentmarker}{\pgfqpoint{-0.027778in}{0.000000in}}{\pgfqpoint{-0.000000in}{0.000000in}}{%
\pgfpathmoveto{\pgfqpoint{-0.000000in}{0.000000in}}%
\pgfpathlineto{\pgfqpoint{-0.027778in}{0.000000in}}%
\pgfusepath{stroke,fill}%
}%
\begin{pgfscope}%
\pgfsys@transformshift{0.594525in}{1.937619in}%
\pgfsys@useobject{currentmarker}{}%
\end{pgfscope}%
\end{pgfscope}%
\begin{pgfscope}%
\pgfpathrectangle{\pgfqpoint{0.594525in}{0.417642in}}{\pgfqpoint{3.423805in}{2.011535in}}%
\pgfusepath{clip}%
\pgfsetrectcap%
\pgfsetroundjoin%
\pgfsetlinewidth{0.803000pt}%
\definecolor{currentstroke}{rgb}{0.850000,0.850000,0.850000}%
\pgfsetstrokecolor{currentstroke}%
\pgfsetdash{}{0pt}%
\pgfpathmoveto{\pgfqpoint{0.594525in}{1.964552in}}%
\pgfpathlineto{\pgfqpoint{4.018330in}{1.964552in}}%
\pgfusepath{stroke}%
\end{pgfscope}%
\begin{pgfscope}%
\pgfsetbuttcap%
\pgfsetroundjoin%
\definecolor{currentfill}{rgb}{0.000000,0.000000,0.000000}%
\pgfsetfillcolor{currentfill}%
\pgfsetlinewidth{0.602250pt}%
\definecolor{currentstroke}{rgb}{0.000000,0.000000,0.000000}%
\pgfsetstrokecolor{currentstroke}%
\pgfsetdash{}{0pt}%
\pgfsys@defobject{currentmarker}{\pgfqpoint{-0.027778in}{0.000000in}}{\pgfqpoint{-0.000000in}{0.000000in}}{%
\pgfpathmoveto{\pgfqpoint{-0.000000in}{0.000000in}}%
\pgfpathlineto{\pgfqpoint{-0.027778in}{0.000000in}}%
\pgfusepath{stroke,fill}%
}%
\begin{pgfscope}%
\pgfsys@transformshift{0.594525in}{1.964552in}%
\pgfsys@useobject{currentmarker}{}%
\end{pgfscope}%
\end{pgfscope}%
\begin{pgfscope}%
\pgfpathrectangle{\pgfqpoint{0.594525in}{0.417642in}}{\pgfqpoint{3.423805in}{2.011535in}}%
\pgfusepath{clip}%
\pgfsetrectcap%
\pgfsetroundjoin%
\pgfsetlinewidth{0.803000pt}%
\definecolor{currentstroke}{rgb}{0.850000,0.850000,0.850000}%
\pgfsetstrokecolor{currentstroke}%
\pgfsetdash{}{0pt}%
\pgfpathmoveto{\pgfqpoint{0.594525in}{1.987883in}}%
\pgfpathlineto{\pgfqpoint{4.018330in}{1.987883in}}%
\pgfusepath{stroke}%
\end{pgfscope}%
\begin{pgfscope}%
\pgfsetbuttcap%
\pgfsetroundjoin%
\definecolor{currentfill}{rgb}{0.000000,0.000000,0.000000}%
\pgfsetfillcolor{currentfill}%
\pgfsetlinewidth{0.602250pt}%
\definecolor{currentstroke}{rgb}{0.000000,0.000000,0.000000}%
\pgfsetstrokecolor{currentstroke}%
\pgfsetdash{}{0pt}%
\pgfsys@defobject{currentmarker}{\pgfqpoint{-0.027778in}{0.000000in}}{\pgfqpoint{-0.000000in}{0.000000in}}{%
\pgfpathmoveto{\pgfqpoint{-0.000000in}{0.000000in}}%
\pgfpathlineto{\pgfqpoint{-0.027778in}{0.000000in}}%
\pgfusepath{stroke,fill}%
}%
\begin{pgfscope}%
\pgfsys@transformshift{0.594525in}{1.987883in}%
\pgfsys@useobject{currentmarker}{}%
\end{pgfscope}%
\end{pgfscope}%
\begin{pgfscope}%
\pgfpathrectangle{\pgfqpoint{0.594525in}{0.417642in}}{\pgfqpoint{3.423805in}{2.011535in}}%
\pgfusepath{clip}%
\pgfsetrectcap%
\pgfsetroundjoin%
\pgfsetlinewidth{0.803000pt}%
\definecolor{currentstroke}{rgb}{0.850000,0.850000,0.850000}%
\pgfsetstrokecolor{currentstroke}%
\pgfsetdash{}{0pt}%
\pgfpathmoveto{\pgfqpoint{0.594525in}{2.008462in}}%
\pgfpathlineto{\pgfqpoint{4.018330in}{2.008462in}}%
\pgfusepath{stroke}%
\end{pgfscope}%
\begin{pgfscope}%
\pgfsetbuttcap%
\pgfsetroundjoin%
\definecolor{currentfill}{rgb}{0.000000,0.000000,0.000000}%
\pgfsetfillcolor{currentfill}%
\pgfsetlinewidth{0.602250pt}%
\definecolor{currentstroke}{rgb}{0.000000,0.000000,0.000000}%
\pgfsetstrokecolor{currentstroke}%
\pgfsetdash{}{0pt}%
\pgfsys@defobject{currentmarker}{\pgfqpoint{-0.027778in}{0.000000in}}{\pgfqpoint{-0.000000in}{0.000000in}}{%
\pgfpathmoveto{\pgfqpoint{-0.000000in}{0.000000in}}%
\pgfpathlineto{\pgfqpoint{-0.027778in}{0.000000in}}%
\pgfusepath{stroke,fill}%
}%
\begin{pgfscope}%
\pgfsys@transformshift{0.594525in}{2.008462in}%
\pgfsys@useobject{currentmarker}{}%
\end{pgfscope}%
\end{pgfscope}%
\begin{pgfscope}%
\pgfpathrectangle{\pgfqpoint{0.594525in}{0.417642in}}{\pgfqpoint{3.423805in}{2.011535in}}%
\pgfusepath{clip}%
\pgfsetrectcap%
\pgfsetroundjoin%
\pgfsetlinewidth{0.803000pt}%
\definecolor{currentstroke}{rgb}{0.850000,0.850000,0.850000}%
\pgfsetstrokecolor{currentstroke}%
\pgfsetdash{}{0pt}%
\pgfpathmoveto{\pgfqpoint{0.594525in}{2.147977in}}%
\pgfpathlineto{\pgfqpoint{4.018330in}{2.147977in}}%
\pgfusepath{stroke}%
\end{pgfscope}%
\begin{pgfscope}%
\pgfsetbuttcap%
\pgfsetroundjoin%
\definecolor{currentfill}{rgb}{0.000000,0.000000,0.000000}%
\pgfsetfillcolor{currentfill}%
\pgfsetlinewidth{0.602250pt}%
\definecolor{currentstroke}{rgb}{0.000000,0.000000,0.000000}%
\pgfsetstrokecolor{currentstroke}%
\pgfsetdash{}{0pt}%
\pgfsys@defobject{currentmarker}{\pgfqpoint{-0.027778in}{0.000000in}}{\pgfqpoint{-0.000000in}{0.000000in}}{%
\pgfpathmoveto{\pgfqpoint{-0.000000in}{0.000000in}}%
\pgfpathlineto{\pgfqpoint{-0.027778in}{0.000000in}}%
\pgfusepath{stroke,fill}%
}%
\begin{pgfscope}%
\pgfsys@transformshift{0.594525in}{2.147977in}%
\pgfsys@useobject{currentmarker}{}%
\end{pgfscope}%
\end{pgfscope}%
\begin{pgfscope}%
\pgfpathrectangle{\pgfqpoint{0.594525in}{0.417642in}}{\pgfqpoint{3.423805in}{2.011535in}}%
\pgfusepath{clip}%
\pgfsetrectcap%
\pgfsetroundjoin%
\pgfsetlinewidth{0.803000pt}%
\definecolor{currentstroke}{rgb}{0.850000,0.850000,0.850000}%
\pgfsetstrokecolor{currentstroke}%
\pgfsetdash{}{0pt}%
\pgfpathmoveto{\pgfqpoint{0.594525in}{2.218819in}}%
\pgfpathlineto{\pgfqpoint{4.018330in}{2.218819in}}%
\pgfusepath{stroke}%
\end{pgfscope}%
\begin{pgfscope}%
\pgfsetbuttcap%
\pgfsetroundjoin%
\definecolor{currentfill}{rgb}{0.000000,0.000000,0.000000}%
\pgfsetfillcolor{currentfill}%
\pgfsetlinewidth{0.602250pt}%
\definecolor{currentstroke}{rgb}{0.000000,0.000000,0.000000}%
\pgfsetstrokecolor{currentstroke}%
\pgfsetdash{}{0pt}%
\pgfsys@defobject{currentmarker}{\pgfqpoint{-0.027778in}{0.000000in}}{\pgfqpoint{-0.000000in}{0.000000in}}{%
\pgfpathmoveto{\pgfqpoint{-0.000000in}{0.000000in}}%
\pgfpathlineto{\pgfqpoint{-0.027778in}{0.000000in}}%
\pgfusepath{stroke,fill}%
}%
\begin{pgfscope}%
\pgfsys@transformshift{0.594525in}{2.218819in}%
\pgfsys@useobject{currentmarker}{}%
\end{pgfscope}%
\end{pgfscope}%
\begin{pgfscope}%
\pgfpathrectangle{\pgfqpoint{0.594525in}{0.417642in}}{\pgfqpoint{3.423805in}{2.011535in}}%
\pgfusepath{clip}%
\pgfsetrectcap%
\pgfsetroundjoin%
\pgfsetlinewidth{0.803000pt}%
\definecolor{currentstroke}{rgb}{0.850000,0.850000,0.850000}%
\pgfsetstrokecolor{currentstroke}%
\pgfsetdash{}{0pt}%
\pgfpathmoveto{\pgfqpoint{0.594525in}{2.269083in}}%
\pgfpathlineto{\pgfqpoint{4.018330in}{2.269083in}}%
\pgfusepath{stroke}%
\end{pgfscope}%
\begin{pgfscope}%
\pgfsetbuttcap%
\pgfsetroundjoin%
\definecolor{currentfill}{rgb}{0.000000,0.000000,0.000000}%
\pgfsetfillcolor{currentfill}%
\pgfsetlinewidth{0.602250pt}%
\definecolor{currentstroke}{rgb}{0.000000,0.000000,0.000000}%
\pgfsetstrokecolor{currentstroke}%
\pgfsetdash{}{0pt}%
\pgfsys@defobject{currentmarker}{\pgfqpoint{-0.027778in}{0.000000in}}{\pgfqpoint{-0.000000in}{0.000000in}}{%
\pgfpathmoveto{\pgfqpoint{-0.000000in}{0.000000in}}%
\pgfpathlineto{\pgfqpoint{-0.027778in}{0.000000in}}%
\pgfusepath{stroke,fill}%
}%
\begin{pgfscope}%
\pgfsys@transformshift{0.594525in}{2.269083in}%
\pgfsys@useobject{currentmarker}{}%
\end{pgfscope}%
\end{pgfscope}%
\begin{pgfscope}%
\pgfpathrectangle{\pgfqpoint{0.594525in}{0.417642in}}{\pgfqpoint{3.423805in}{2.011535in}}%
\pgfusepath{clip}%
\pgfsetrectcap%
\pgfsetroundjoin%
\pgfsetlinewidth{0.803000pt}%
\definecolor{currentstroke}{rgb}{0.850000,0.850000,0.850000}%
\pgfsetstrokecolor{currentstroke}%
\pgfsetdash{}{0pt}%
\pgfpathmoveto{\pgfqpoint{0.594525in}{2.308071in}}%
\pgfpathlineto{\pgfqpoint{4.018330in}{2.308071in}}%
\pgfusepath{stroke}%
\end{pgfscope}%
\begin{pgfscope}%
\pgfsetbuttcap%
\pgfsetroundjoin%
\definecolor{currentfill}{rgb}{0.000000,0.000000,0.000000}%
\pgfsetfillcolor{currentfill}%
\pgfsetlinewidth{0.602250pt}%
\definecolor{currentstroke}{rgb}{0.000000,0.000000,0.000000}%
\pgfsetstrokecolor{currentstroke}%
\pgfsetdash{}{0pt}%
\pgfsys@defobject{currentmarker}{\pgfqpoint{-0.027778in}{0.000000in}}{\pgfqpoint{-0.000000in}{0.000000in}}{%
\pgfpathmoveto{\pgfqpoint{-0.000000in}{0.000000in}}%
\pgfpathlineto{\pgfqpoint{-0.027778in}{0.000000in}}%
\pgfusepath{stroke,fill}%
}%
\begin{pgfscope}%
\pgfsys@transformshift{0.594525in}{2.308071in}%
\pgfsys@useobject{currentmarker}{}%
\end{pgfscope}%
\end{pgfscope}%
\begin{pgfscope}%
\pgfpathrectangle{\pgfqpoint{0.594525in}{0.417642in}}{\pgfqpoint{3.423805in}{2.011535in}}%
\pgfusepath{clip}%
\pgfsetrectcap%
\pgfsetroundjoin%
\pgfsetlinewidth{0.803000pt}%
\definecolor{currentstroke}{rgb}{0.850000,0.850000,0.850000}%
\pgfsetstrokecolor{currentstroke}%
\pgfsetdash{}{0pt}%
\pgfpathmoveto{\pgfqpoint{0.594525in}{2.339926in}}%
\pgfpathlineto{\pgfqpoint{4.018330in}{2.339926in}}%
\pgfusepath{stroke}%
\end{pgfscope}%
\begin{pgfscope}%
\pgfsetbuttcap%
\pgfsetroundjoin%
\definecolor{currentfill}{rgb}{0.000000,0.000000,0.000000}%
\pgfsetfillcolor{currentfill}%
\pgfsetlinewidth{0.602250pt}%
\definecolor{currentstroke}{rgb}{0.000000,0.000000,0.000000}%
\pgfsetstrokecolor{currentstroke}%
\pgfsetdash{}{0pt}%
\pgfsys@defobject{currentmarker}{\pgfqpoint{-0.027778in}{0.000000in}}{\pgfqpoint{-0.000000in}{0.000000in}}{%
\pgfpathmoveto{\pgfqpoint{-0.000000in}{0.000000in}}%
\pgfpathlineto{\pgfqpoint{-0.027778in}{0.000000in}}%
\pgfusepath{stroke,fill}%
}%
\begin{pgfscope}%
\pgfsys@transformshift{0.594525in}{2.339926in}%
\pgfsys@useobject{currentmarker}{}%
\end{pgfscope}%
\end{pgfscope}%
\begin{pgfscope}%
\pgfpathrectangle{\pgfqpoint{0.594525in}{0.417642in}}{\pgfqpoint{3.423805in}{2.011535in}}%
\pgfusepath{clip}%
\pgfsetrectcap%
\pgfsetroundjoin%
\pgfsetlinewidth{0.803000pt}%
\definecolor{currentstroke}{rgb}{0.850000,0.850000,0.850000}%
\pgfsetstrokecolor{currentstroke}%
\pgfsetdash{}{0pt}%
\pgfpathmoveto{\pgfqpoint{0.594525in}{2.366859in}}%
\pgfpathlineto{\pgfqpoint{4.018330in}{2.366859in}}%
\pgfusepath{stroke}%
\end{pgfscope}%
\begin{pgfscope}%
\pgfsetbuttcap%
\pgfsetroundjoin%
\definecolor{currentfill}{rgb}{0.000000,0.000000,0.000000}%
\pgfsetfillcolor{currentfill}%
\pgfsetlinewidth{0.602250pt}%
\definecolor{currentstroke}{rgb}{0.000000,0.000000,0.000000}%
\pgfsetstrokecolor{currentstroke}%
\pgfsetdash{}{0pt}%
\pgfsys@defobject{currentmarker}{\pgfqpoint{-0.027778in}{0.000000in}}{\pgfqpoint{-0.000000in}{0.000000in}}{%
\pgfpathmoveto{\pgfqpoint{-0.000000in}{0.000000in}}%
\pgfpathlineto{\pgfqpoint{-0.027778in}{0.000000in}}%
\pgfusepath{stroke,fill}%
}%
\begin{pgfscope}%
\pgfsys@transformshift{0.594525in}{2.366859in}%
\pgfsys@useobject{currentmarker}{}%
\end{pgfscope}%
\end{pgfscope}%
\begin{pgfscope}%
\pgfpathrectangle{\pgfqpoint{0.594525in}{0.417642in}}{\pgfqpoint{3.423805in}{2.011535in}}%
\pgfusepath{clip}%
\pgfsetrectcap%
\pgfsetroundjoin%
\pgfsetlinewidth{0.803000pt}%
\definecolor{currentstroke}{rgb}{0.850000,0.850000,0.850000}%
\pgfsetstrokecolor{currentstroke}%
\pgfsetdash{}{0pt}%
\pgfpathmoveto{\pgfqpoint{0.594525in}{2.390190in}}%
\pgfpathlineto{\pgfqpoint{4.018330in}{2.390190in}}%
\pgfusepath{stroke}%
\end{pgfscope}%
\begin{pgfscope}%
\pgfsetbuttcap%
\pgfsetroundjoin%
\definecolor{currentfill}{rgb}{0.000000,0.000000,0.000000}%
\pgfsetfillcolor{currentfill}%
\pgfsetlinewidth{0.602250pt}%
\definecolor{currentstroke}{rgb}{0.000000,0.000000,0.000000}%
\pgfsetstrokecolor{currentstroke}%
\pgfsetdash{}{0pt}%
\pgfsys@defobject{currentmarker}{\pgfqpoint{-0.027778in}{0.000000in}}{\pgfqpoint{-0.000000in}{0.000000in}}{%
\pgfpathmoveto{\pgfqpoint{-0.000000in}{0.000000in}}%
\pgfpathlineto{\pgfqpoint{-0.027778in}{0.000000in}}%
\pgfusepath{stroke,fill}%
}%
\begin{pgfscope}%
\pgfsys@transformshift{0.594525in}{2.390190in}%
\pgfsys@useobject{currentmarker}{}%
\end{pgfscope}%
\end{pgfscope}%
\begin{pgfscope}%
\pgfpathrectangle{\pgfqpoint{0.594525in}{0.417642in}}{\pgfqpoint{3.423805in}{2.011535in}}%
\pgfusepath{clip}%
\pgfsetrectcap%
\pgfsetroundjoin%
\pgfsetlinewidth{0.803000pt}%
\definecolor{currentstroke}{rgb}{0.850000,0.850000,0.850000}%
\pgfsetstrokecolor{currentstroke}%
\pgfsetdash{}{0pt}%
\pgfpathmoveto{\pgfqpoint{0.594525in}{2.410769in}}%
\pgfpathlineto{\pgfqpoint{4.018330in}{2.410769in}}%
\pgfusepath{stroke}%
\end{pgfscope}%
\begin{pgfscope}%
\pgfsetbuttcap%
\pgfsetroundjoin%
\definecolor{currentfill}{rgb}{0.000000,0.000000,0.000000}%
\pgfsetfillcolor{currentfill}%
\pgfsetlinewidth{0.602250pt}%
\definecolor{currentstroke}{rgb}{0.000000,0.000000,0.000000}%
\pgfsetstrokecolor{currentstroke}%
\pgfsetdash{}{0pt}%
\pgfsys@defobject{currentmarker}{\pgfqpoint{-0.027778in}{0.000000in}}{\pgfqpoint{-0.000000in}{0.000000in}}{%
\pgfpathmoveto{\pgfqpoint{-0.000000in}{0.000000in}}%
\pgfpathlineto{\pgfqpoint{-0.027778in}{0.000000in}}%
\pgfusepath{stroke,fill}%
}%
\begin{pgfscope}%
\pgfsys@transformshift{0.594525in}{2.410769in}%
\pgfsys@useobject{currentmarker}{}%
\end{pgfscope}%
\end{pgfscope}%
\begin{pgfscope}%
\definecolor{textcolor}{rgb}{0.000000,0.000000,0.000000}%
\pgfsetstrokecolor{textcolor}%
\pgfsetfillcolor{textcolor}%
\pgftext[x=0.185574in,y=1.423410in,,bottom,rotate=90.000000]{\color{textcolor}\rmfamily\fontsize{10.000000}{12.000000}\selectfont  \(\displaystyle S_y(f)\) in \(\displaystyle \unit{1 \per \Hz}\)}%
\end{pgfscope}%
\begin{pgfscope}%
\pgfpathrectangle{\pgfqpoint{0.594525in}{0.417642in}}{\pgfqpoint{3.423805in}{2.011535in}}%
\pgfusepath{clip}%
\pgfsetbuttcap%
\pgfsetroundjoin%
\pgfsetlinewidth{1.505625pt}%
\definecolor{currentstroke}{rgb}{0.000000,0.447059,0.698039}%
\pgfsetstrokecolor{currentstroke}%
\pgfsetdash{{5.550000pt}{2.400000pt}}{0.000000pt}%
\pgfpathmoveto{\pgfqpoint{0.750152in}{1.101150in}}%
\pgfpathlineto{\pgfqpoint{0.913971in}{1.101150in}}%
\pgfpathlineto{\pgfqpoint{1.077789in}{1.101149in}}%
\pgfpathlineto{\pgfqpoint{1.241608in}{1.101149in}}%
\pgfpathlineto{\pgfqpoint{1.405426in}{1.101149in}}%
\pgfpathlineto{\pgfqpoint{1.569244in}{1.101147in}}%
\pgfpathlineto{\pgfqpoint{1.733063in}{1.101144in}}%
\pgfpathlineto{\pgfqpoint{1.896881in}{1.101134in}}%
\pgfpathlineto{\pgfqpoint{2.060700in}{1.101109in}}%
\pgfpathlineto{\pgfqpoint{2.224518in}{1.101043in}}%
\pgfpathlineto{\pgfqpoint{2.388337in}{1.100870in}}%
\pgfpathlineto{\pgfqpoint{2.552155in}{1.100413in}}%
\pgfpathlineto{\pgfqpoint{2.715973in}{1.099214in}}%
\pgfpathlineto{\pgfqpoint{2.879792in}{1.096091in}}%
\pgfpathlineto{\pgfqpoint{3.043610in}{1.088116in}}%
\pgfpathlineto{\pgfqpoint{3.207429in}{1.068682in}}%
\pgfpathlineto{\pgfqpoint{3.371247in}{1.025885in}}%
\pgfpathlineto{\pgfqpoint{3.535066in}{0.946760in}}%
\pgfpathlineto{\pgfqpoint{3.698884in}{0.829160in}}%
\pgfpathlineto{\pgfqpoint{3.862702in}{0.684273in}}%
\pgfusepath{stroke}%
\end{pgfscope}%
\begin{pgfscope}%
\pgfpathrectangle{\pgfqpoint{0.594525in}{0.417642in}}{\pgfqpoint{3.423805in}{2.011535in}}%
\pgfusepath{clip}%
\pgfsetbuttcap%
\pgfsetroundjoin%
\pgfsetlinewidth{1.505625pt}%
\definecolor{currentstroke}{rgb}{0.000000,0.619608,0.450980}%
\pgfsetstrokecolor{currentstroke}%
\pgfsetdash{{5.550000pt}{2.400000pt}}{0.000000pt}%
\pgfpathmoveto{\pgfqpoint{0.750152in}{1.503455in}}%
\pgfpathlineto{\pgfqpoint{0.913971in}{1.503452in}}%
\pgfpathlineto{\pgfqpoint{1.077789in}{1.503445in}}%
\pgfpathlineto{\pgfqpoint{1.241608in}{1.503425in}}%
\pgfpathlineto{\pgfqpoint{1.405426in}{1.503373in}}%
\pgfpathlineto{\pgfqpoint{1.569244in}{1.503237in}}%
\pgfpathlineto{\pgfqpoint{1.733063in}{1.502878in}}%
\pgfpathlineto{\pgfqpoint{1.896881in}{1.501936in}}%
\pgfpathlineto{\pgfqpoint{2.060700in}{1.499475in}}%
\pgfpathlineto{\pgfqpoint{2.224518in}{1.493147in}}%
\pgfpathlineto{\pgfqpoint{2.388337in}{1.477486in}}%
\pgfpathlineto{\pgfqpoint{2.552155in}{1.441876in}}%
\pgfpathlineto{\pgfqpoint{2.715973in}{1.372655in}}%
\pgfpathlineto{\pgfqpoint{2.879792in}{1.263993in}}%
\pgfpathlineto{\pgfqpoint{3.043610in}{1.124575in}}%
\pgfpathlineto{\pgfqpoint{3.207429in}{0.968046in}}%
\pgfpathlineto{\pgfqpoint{3.371247in}{0.803791in}}%
\pgfpathlineto{\pgfqpoint{3.535066in}{0.636387in}}%
\pgfpathlineto{\pgfqpoint{3.698884in}{0.467755in}}%
\pgfpathlineto{\pgfqpoint{3.757118in}{0.407642in}}%
\pgfusepath{stroke}%
\end{pgfscope}%
\begin{pgfscope}%
\pgfpathrectangle{\pgfqpoint{0.594525in}{0.417642in}}{\pgfqpoint{3.423805in}{2.011535in}}%
\pgfusepath{clip}%
\pgfsetbuttcap%
\pgfsetroundjoin%
\pgfsetlinewidth{1.505625pt}%
\definecolor{currentstroke}{rgb}{0.835294,0.368627,0.000000}%
\pgfsetstrokecolor{currentstroke}%
\pgfsetdash{{5.550000pt}{2.400000pt}}{0.000000pt}%
\pgfpathmoveto{\pgfqpoint{0.750152in}{1.905591in}}%
\pgfpathlineto{\pgfqpoint{0.913971in}{1.905310in}}%
\pgfpathlineto{\pgfqpoint{1.077789in}{1.904569in}}%
\pgfpathlineto{\pgfqpoint{1.241608in}{1.902631in}}%
\pgfpathlineto{\pgfqpoint{1.405426in}{1.897622in}}%
\pgfpathlineto{\pgfqpoint{1.569244in}{1.885066in}}%
\pgfpathlineto{\pgfqpoint{1.733063in}{1.855727in}}%
\pgfpathlineto{\pgfqpoint{1.896881in}{1.795997in}}%
\pgfpathlineto{\pgfqpoint{2.060700in}{1.696883in}}%
\pgfpathlineto{\pgfqpoint{2.224518in}{1.563834in}}%
\pgfpathlineto{\pgfqpoint{2.388337in}{1.410478in}}%
\pgfpathlineto{\pgfqpoint{2.552155in}{1.247574in}}%
\pgfpathlineto{\pgfqpoint{2.715973in}{1.080708in}}%
\pgfpathlineto{\pgfqpoint{2.879792in}{0.912283in}}%
\pgfpathlineto{\pgfqpoint{3.043610in}{0.743258in}}%
\pgfpathlineto{\pgfqpoint{3.207429in}{0.574006in}}%
\pgfpathlineto{\pgfqpoint{3.368369in}{0.407642in}}%
\pgfusepath{stroke}%
\end{pgfscope}%
\begin{pgfscope}%
\pgfpathrectangle{\pgfqpoint{0.594525in}{0.417642in}}{\pgfqpoint{3.423805in}{2.011535in}}%
\pgfusepath{clip}%
\pgfsetbuttcap%
\pgfsetroundjoin%
\pgfsetlinewidth{1.505625pt}%
\definecolor{currentstroke}{rgb}{0.800000,0.474510,0.654902}%
\pgfsetstrokecolor{currentstroke}%
\pgfsetdash{{5.550000pt}{2.400000pt}}{0.000000pt}%
\pgfpathmoveto{\pgfqpoint{0.750152in}{2.291625in}}%
\pgfpathlineto{\pgfqpoint{0.913971in}{2.267659in}}%
\pgfpathlineto{\pgfqpoint{1.077789in}{2.216791in}}%
\pgfpathlineto{\pgfqpoint{1.241608in}{2.127610in}}%
\pgfpathlineto{\pgfqpoint{1.405426in}{2.001847in}}%
\pgfpathlineto{\pgfqpoint{1.569244in}{1.852341in}}%
\pgfpathlineto{\pgfqpoint{1.733063in}{1.691125in}}%
\pgfpathlineto{\pgfqpoint{1.896881in}{1.524937in}}%
\pgfpathlineto{\pgfqpoint{2.060700in}{1.356775in}}%
\pgfpathlineto{\pgfqpoint{2.224518in}{1.187852in}}%
\pgfpathlineto{\pgfqpoint{2.388337in}{1.018638in}}%
\pgfpathlineto{\pgfqpoint{2.552155in}{0.849313in}}%
\pgfpathlineto{\pgfqpoint{2.715973in}{0.679946in}}%
\pgfpathlineto{\pgfqpoint{2.879792in}{0.510563in}}%
\pgfpathlineto{\pgfqpoint{2.979329in}{0.407642in}}%
\pgfusepath{stroke}%
\end{pgfscope}%
\begin{pgfscope}%
\pgfpathrectangle{\pgfqpoint{0.594525in}{0.417642in}}{\pgfqpoint{3.423805in}{2.011535in}}%
\pgfusepath{clip}%
\pgfsetrectcap%
\pgfsetroundjoin%
\pgfsetlinewidth{1.505625pt}%
\definecolor{currentstroke}{rgb}{0.000000,0.000000,0.000000}%
\pgfsetstrokecolor{currentstroke}%
\pgfsetdash{}{0pt}%
\pgfpathmoveto{\pgfqpoint{0.750152in}{2.311714in}}%
\pgfpathlineto{\pgfqpoint{0.913971in}{2.290485in}}%
\pgfpathlineto{\pgfqpoint{1.077789in}{2.246597in}}%
\pgfpathlineto{\pgfqpoint{1.241608in}{2.174364in}}%
\pgfpathlineto{\pgfqpoint{1.405426in}{2.085506in}}%
\pgfpathlineto{\pgfqpoint{1.569244in}{2.002007in}}%
\pgfpathlineto{\pgfqpoint{1.733063in}{1.930696in}}%
\pgfpathlineto{\pgfqpoint{1.896881in}{1.856833in}}%
\pgfpathlineto{\pgfqpoint{2.060700in}{1.767596in}}%
\pgfpathlineto{\pgfqpoint{2.224518in}{1.671721in}}%
\pgfpathlineto{\pgfqpoint{2.388337in}{1.586817in}}%
\pgfpathlineto{\pgfqpoint{2.552155in}{1.513206in}}%
\pgfpathlineto{\pgfqpoint{2.715973in}{1.433444in}}%
\pgfpathlineto{\pgfqpoint{2.879792in}{1.338240in}}%
\pgfpathlineto{\pgfqpoint{3.043610in}{1.239973in}}%
\pgfpathlineto{\pgfqpoint{3.207429in}{1.153723in}}%
\pgfpathlineto{\pgfqpoint{3.371247in}{1.073322in}}%
\pgfpathlineto{\pgfqpoint{3.535066in}{0.976858in}}%
\pgfpathlineto{\pgfqpoint{3.698884in}{0.852102in}}%
\pgfpathlineto{\pgfqpoint{3.862702in}{0.704407in}}%
\pgfusepath{stroke}%
\end{pgfscope}%
\begin{pgfscope}%
\pgfsetrectcap%
\pgfsetmiterjoin%
\pgfsetlinewidth{0.803000pt}%
\definecolor{currentstroke}{rgb}{0.000000,0.000000,0.000000}%
\pgfsetstrokecolor{currentstroke}%
\pgfsetdash{}{0pt}%
\pgfpathmoveto{\pgfqpoint{0.594525in}{0.417642in}}%
\pgfpathlineto{\pgfqpoint{0.594525in}{2.429177in}}%
\pgfusepath{stroke}%
\end{pgfscope}%
\begin{pgfscope}%
\pgfsetrectcap%
\pgfsetmiterjoin%
\pgfsetlinewidth{0.803000pt}%
\definecolor{currentstroke}{rgb}{0.000000,0.000000,0.000000}%
\pgfsetstrokecolor{currentstroke}%
\pgfsetdash{}{0pt}%
\pgfpathmoveto{\pgfqpoint{4.018330in}{0.417642in}}%
\pgfpathlineto{\pgfqpoint{4.018330in}{2.429177in}}%
\pgfusepath{stroke}%
\end{pgfscope}%
\begin{pgfscope}%
\pgfsetrectcap%
\pgfsetmiterjoin%
\pgfsetlinewidth{0.803000pt}%
\definecolor{currentstroke}{rgb}{0.000000,0.000000,0.000000}%
\pgfsetstrokecolor{currentstroke}%
\pgfsetdash{}{0pt}%
\pgfpathmoveto{\pgfqpoint{0.594525in}{0.417642in}}%
\pgfpathlineto{\pgfqpoint{4.018330in}{0.417642in}}%
\pgfusepath{stroke}%
\end{pgfscope}%
\begin{pgfscope}%
\pgfsetrectcap%
\pgfsetmiterjoin%
\pgfsetlinewidth{0.803000pt}%
\definecolor{currentstroke}{rgb}{0.000000,0.000000,0.000000}%
\pgfsetstrokecolor{currentstroke}%
\pgfsetdash{}{0pt}%
\pgfpathmoveto{\pgfqpoint{0.594525in}{2.429177in}}%
\pgfpathlineto{\pgfqpoint{4.018330in}{2.429177in}}%
\pgfusepath{stroke}%
\end{pgfscope}%
\begin{pgfscope}%
\pgfsetbuttcap%
\pgfsetmiterjoin%
\definecolor{currentfill}{rgb}{1.000000,1.000000,1.000000}%
\pgfsetfillcolor{currentfill}%
\pgfsetfillopacity{0.800000}%
\pgfsetlinewidth{1.003750pt}%
\definecolor{currentstroke}{rgb}{0.800000,0.800000,0.800000}%
\pgfsetstrokecolor{currentstroke}%
\pgfsetstrokeopacity{0.800000}%
\pgfsetdash{}{0pt}%
\pgfpathmoveto{\pgfqpoint{0.672303in}{0.473197in}}%
\pgfpathlineto{\pgfqpoint{1.839313in}{0.473197in}}%
\pgfpathquadraticcurveto{\pgfqpoint{1.861536in}{0.473197in}}{\pgfqpoint{1.861536in}{0.495420in}}%
\pgfpathlineto{\pgfqpoint{1.861536in}{1.258752in}}%
\pgfpathquadraticcurveto{\pgfqpoint{1.861536in}{1.280975in}}{\pgfqpoint{1.839313in}{1.280975in}}%
\pgfpathlineto{\pgfqpoint{0.672303in}{1.280975in}}%
\pgfpathquadraticcurveto{\pgfqpoint{0.650080in}{1.280975in}}{\pgfqpoint{0.650080in}{1.258752in}}%
\pgfpathlineto{\pgfqpoint{0.650080in}{0.495420in}}%
\pgfpathquadraticcurveto{\pgfqpoint{0.650080in}{0.473197in}}{\pgfqpoint{0.672303in}{0.473197in}}%
\pgfpathlineto{\pgfqpoint{0.672303in}{0.473197in}}%
\pgfpathclose%
\pgfusepath{stroke,fill}%
\end{pgfscope}%
\begin{pgfscope}%
\pgfsetbuttcap%
\pgfsetroundjoin%
\pgfsetlinewidth{1.505625pt}%
\definecolor{currentstroke}{rgb}{0.000000,0.447059,0.698039}%
\pgfsetstrokecolor{currentstroke}%
\pgfsetdash{{5.550000pt}{2.400000pt}}{0.000000pt}%
\pgfpathmoveto{\pgfqpoint{0.694525in}{1.197641in}}%
\pgfpathlineto{\pgfqpoint{0.805636in}{1.197641in}}%
\pgfpathlineto{\pgfqpoint{0.916747in}{1.197641in}}%
\pgfusepath{stroke}%
\end{pgfscope}%
\begin{pgfscope}%
\definecolor{textcolor}{rgb}{0.000000,0.000000,0.000000}%
\pgfsetstrokecolor{textcolor}%
\pgfsetfillcolor{textcolor}%
\pgftext[x=1.005636in,y=1.158752in,left,base]{\color{textcolor}\rmfamily\fontsize{8.000000}{9.600000}\selectfont \(\displaystyle \bar\tau_1=\tau_0=\qty{0.01}{\s}\)}%
\end{pgfscope}%
\begin{pgfscope}%
\pgfsetbuttcap%
\pgfsetroundjoin%
\pgfsetlinewidth{1.505625pt}%
\definecolor{currentstroke}{rgb}{0.000000,0.619608,0.450980}%
\pgfsetstrokecolor{currentstroke}%
\pgfsetdash{{5.550000pt}{2.400000pt}}{0.000000pt}%
\pgfpathmoveto{\pgfqpoint{0.694525in}{1.042752in}}%
\pgfpathlineto{\pgfqpoint{0.805636in}{1.042752in}}%
\pgfpathlineto{\pgfqpoint{0.916747in}{1.042752in}}%
\pgfusepath{stroke}%
\end{pgfscope}%
\begin{pgfscope}%
\definecolor{textcolor}{rgb}{0.000000,0.000000,0.000000}%
\pgfsetstrokecolor{textcolor}%
\pgfsetfillcolor{textcolor}%
\pgftext[x=1.005636in,y=1.003864in,left,base]{\color{textcolor}\rmfamily\fontsize{8.000000}{9.600000}\selectfont \(\displaystyle \bar\tau_1=\tau_0=\qty{0.1}{\s}\)}%
\end{pgfscope}%
\begin{pgfscope}%
\pgfsetbuttcap%
\pgfsetroundjoin%
\pgfsetlinewidth{1.505625pt}%
\definecolor{currentstroke}{rgb}{0.835294,0.368627,0.000000}%
\pgfsetstrokecolor{currentstroke}%
\pgfsetdash{{5.550000pt}{2.400000pt}}{0.000000pt}%
\pgfpathmoveto{\pgfqpoint{0.694525in}{0.887864in}}%
\pgfpathlineto{\pgfqpoint{0.805636in}{0.887864in}}%
\pgfpathlineto{\pgfqpoint{0.916747in}{0.887864in}}%
\pgfusepath{stroke}%
\end{pgfscope}%
\begin{pgfscope}%
\definecolor{textcolor}{rgb}{0.000000,0.000000,0.000000}%
\pgfsetstrokecolor{textcolor}%
\pgfsetfillcolor{textcolor}%
\pgftext[x=1.005636in,y=0.848975in,left,base]{\color{textcolor}\rmfamily\fontsize{8.000000}{9.600000}\selectfont \(\displaystyle \bar\tau_1=\tau_0=\qty{1}{\s}\)}%
\end{pgfscope}%
\begin{pgfscope}%
\pgfsetbuttcap%
\pgfsetroundjoin%
\pgfsetlinewidth{1.505625pt}%
\definecolor{currentstroke}{rgb}{0.800000,0.474510,0.654902}%
\pgfsetstrokecolor{currentstroke}%
\pgfsetdash{{5.550000pt}{2.400000pt}}{0.000000pt}%
\pgfpathmoveto{\pgfqpoint{0.694525in}{0.732975in}}%
\pgfpathlineto{\pgfqpoint{0.805636in}{0.732975in}}%
\pgfpathlineto{\pgfqpoint{0.916747in}{0.732975in}}%
\pgfusepath{stroke}%
\end{pgfscope}%
\begin{pgfscope}%
\definecolor{textcolor}{rgb}{0.000000,0.000000,0.000000}%
\pgfsetstrokecolor{textcolor}%
\pgfsetfillcolor{textcolor}%
\pgftext[x=1.005636in,y=0.694086in,left,base]{\color{textcolor}\rmfamily\fontsize{8.000000}{9.600000}\selectfont \(\displaystyle \bar\tau_1=\tau_0=\qty{10}{\s}\)}%
\end{pgfscope}%
\begin{pgfscope}%
\pgfsetrectcap%
\pgfsetroundjoin%
\pgfsetlinewidth{1.505625pt}%
\definecolor{currentstroke}{rgb}{0.000000,0.000000,0.000000}%
\pgfsetstrokecolor{currentstroke}%
\pgfsetdash{}{0pt}%
\pgfpathmoveto{\pgfqpoint{0.694525in}{0.578086in}}%
\pgfpathlineto{\pgfqpoint{0.805636in}{0.578086in}}%
\pgfpathlineto{\pgfqpoint{0.916747in}{0.578086in}}%
\pgfusepath{stroke}%
\end{pgfscope}%
\begin{pgfscope}%
\definecolor{textcolor}{rgb}{0.000000,0.000000,0.000000}%
\pgfsetstrokecolor{textcolor}%
\pgfsetfillcolor{textcolor}%
\pgftext[x=1.005636in,y=0.539197in,left,base]{\color{textcolor}\rmfamily\fontsize{8.000000}{9.600000}\selectfont Envelope}%
\end{pgfscope}%
\end{pgfpicture}%
\makeatother%
\endgroup%

    \caption{Multiple overlapping Lorentzian noise sources forming a $\frac 1 f$-like shape.}
    \label{fig:flicker_noise_evelope}
\end{figure}

Given that no trap site can store an electron indefinetely, the number of trap sites $N$ with a certain time constant $\frac 1 2 \bar \tau = \bar \tau_0 = \bar \tau_1$ must decline for longer time scales. Assuming $N$ is inversely proportional to the time constant $\bar \tau$
\begin{equation}
    N(\tau) \propto \frac{1}{\bar \tau}\,, \label{eqn:flicker_noise_weight_function}
\end{equation}

 which can be motivated if the trapping process is thermally activated \cite{1_f_noise_motivation} and using equation \ref{eqn:burst_noise_lorentzian} from the previous section, multiplying the weight function \ref{eqn:flicker_noise_weight_function} and integrating over all possible storage times gives:
\begin{align}
    S(\omega) &= \lim_{t \to \infty} \int_0^t N(\bar \tau) \, 4 R_{xx}(0) \frac{\bar \tau}{1 + \omega^2 \bar \tau^2} \, d\bar\tau \nonumber\\
    \overset{\bar \tau_0 = \bar \tau_1}&{=} 4 R_{xx}(0)\, C_N \lim_{t \to \infty} \int_0^t \frac{1}{1 + \omega^2 \bar\tau^2} \, d\bar\tau \nonumber\\
    &= \frac{4 R_{xx}(0)\, C_N}{\omega} \lim_{t \to \infty}  \arctan{\bar\tau \omega} \Big|_{\bar\tau=0}^t \nonumber\\
    &= \frac{4 R_{xx}(0)\, C_N}{\omega} \cdot \frac{\pi}{2} \nonumber\\
    &= \frac{2 \pi R_{xx}(0)\, C_N}{\omega}\\
    S(f) &= h_{-1} f^{-1}
\end{align}

$C_N$ is the proportionality constant of \ref{eqn:flicker_noise_weight_function} and $h_{-1}$ is the power coefficient introduced in \ref{eqn:power_law}. This shows, that for a large number of distributed trap sites, a noise spectrum of $f^{-1}$ is found.

Using equation \ref{eqn:psd_to_adev}, the Allan variance can be calculated from the power spectral density:
\begin{align}
    \sigma_A^2(\tau) &= 2 h_{-1} \int_0^\infty \frac{1}{f} \frac{\sin^4\left( \pi f \tau \right)}{(\pi f \tau)^2}\,df \nonumber\\
    &=2 \ln 2 \, h_{-1}
\end{align}


Again, using the \textit{AllanTools} library \cite{allantools}, flicker noise was simulated to give an impresion of its properties.

\begin{figure}[ht]
    \centering
    \begin{subfigure}{0.32\linewidth}
        \centering
        \scalebox{0.75}{%
            %% Creator: Matplotlib, PGF backend
%%
%% To include the figure in your LaTeX document, write
%%   \input{<filename>.pgf}
%%
%% Make sure the required packages are loaded in your preamble
%%   \usepackage{pgf}
%%
%% Also ensure that all the required font packages are loaded; for instance,
%% the lmodern package is sometimes necessary when using math font.
%%   \usepackage{lmodern}
%%
%% Figures using additional raster images can only be included by \input if
%% they are in the same directory as the main LaTeX file. For loading figures
%% from other directories you can use the `import` package
%%   \usepackage{import}
%%
%% and then include the figures with
%%   \import{<path to file>}{<filename>.pgf}
%%
%% Matplotlib used the following preamble
%%   \usepackage{siunitx}
%%   \usepackage{fontspec}
%%   \makeatletter\@ifpackageloaded{underscore}{}{\usepackage[strings]{underscore}}\makeatother
%%
\begingroup%
\makeatletter%
\begin{pgfpicture}%
\pgfpathrectangle{\pgfpointorigin}{\pgfqpoint{2.440000in}{1.830000in}}%
\pgfusepath{use as bounding box, clip}%
\begin{pgfscope}%
\pgfsetbuttcap%
\pgfsetmiterjoin%
\definecolor{currentfill}{rgb}{1.000000,1.000000,1.000000}%
\pgfsetfillcolor{currentfill}%
\pgfsetlinewidth{0.000000pt}%
\definecolor{currentstroke}{rgb}{1.000000,1.000000,1.000000}%
\pgfsetstrokecolor{currentstroke}%
\pgfsetdash{}{0pt}%
\pgfpathmoveto{\pgfqpoint{0.000000in}{0.000000in}}%
\pgfpathlineto{\pgfqpoint{2.440000in}{0.000000in}}%
\pgfpathlineto{\pgfqpoint{2.440000in}{1.830000in}}%
\pgfpathlineto{\pgfqpoint{0.000000in}{1.830000in}}%
\pgfpathlineto{\pgfqpoint{0.000000in}{0.000000in}}%
\pgfpathclose%
\pgfusepath{fill}%
\end{pgfscope}%
\begin{pgfscope}%
\pgfsetbuttcap%
\pgfsetmiterjoin%
\definecolor{currentfill}{rgb}{1.000000,1.000000,1.000000}%
\pgfsetfillcolor{currentfill}%
\pgfsetlinewidth{0.000000pt}%
\definecolor{currentstroke}{rgb}{0.000000,0.000000,0.000000}%
\pgfsetstrokecolor{currentstroke}%
\pgfsetstrokeopacity{0.000000}%
\pgfsetdash{}{0pt}%
\pgfpathmoveto{\pgfqpoint{0.530716in}{0.416447in}}%
\pgfpathlineto{\pgfqpoint{2.398330in}{0.416447in}}%
\pgfpathlineto{\pgfqpoint{2.398330in}{1.788330in}}%
\pgfpathlineto{\pgfqpoint{0.530716in}{1.788330in}}%
\pgfpathlineto{\pgfqpoint{0.530716in}{0.416447in}}%
\pgfpathclose%
\pgfusepath{fill}%
\end{pgfscope}%
\begin{pgfscope}%
\pgfpathrectangle{\pgfqpoint{0.530716in}{0.416447in}}{\pgfqpoint{1.867614in}{1.371882in}}%
\pgfusepath{clip}%
\pgfsetrectcap%
\pgfsetroundjoin%
\pgfsetlinewidth{0.803000pt}%
\definecolor{currentstroke}{rgb}{0.450000,0.450000,0.450000}%
\pgfsetstrokecolor{currentstroke}%
\pgfsetdash{}{0pt}%
\pgfpathmoveto{\pgfqpoint{0.615608in}{0.416447in}}%
\pgfpathlineto{\pgfqpoint{0.615608in}{1.788330in}}%
\pgfusepath{stroke}%
\end{pgfscope}%
\begin{pgfscope}%
\pgfsetbuttcap%
\pgfsetroundjoin%
\definecolor{currentfill}{rgb}{0.000000,0.000000,0.000000}%
\pgfsetfillcolor{currentfill}%
\pgfsetlinewidth{0.803000pt}%
\definecolor{currentstroke}{rgb}{0.000000,0.000000,0.000000}%
\pgfsetstrokecolor{currentstroke}%
\pgfsetdash{}{0pt}%
\pgfsys@defobject{currentmarker}{\pgfqpoint{0.000000in}{-0.048611in}}{\pgfqpoint{0.000000in}{0.000000in}}{%
\pgfpathmoveto{\pgfqpoint{0.000000in}{0.000000in}}%
\pgfpathlineto{\pgfqpoint{0.000000in}{-0.048611in}}%
\pgfusepath{stroke,fill}%
}%
\begin{pgfscope}%
\pgfsys@transformshift{0.615608in}{0.416447in}%
\pgfsys@useobject{currentmarker}{}%
\end{pgfscope}%
\end{pgfscope}%
\begin{pgfscope}%
\definecolor{textcolor}{rgb}{0.000000,0.000000,0.000000}%
\pgfsetstrokecolor{textcolor}%
\pgfsetfillcolor{textcolor}%
\pgftext[x=0.615608in,y=0.319225in,,top]{\color{textcolor}\rmfamily\fontsize{8.000000}{9.600000}\selectfont \(\displaystyle {0}\)}%
\end{pgfscope}%
\begin{pgfscope}%
\pgfpathrectangle{\pgfqpoint{0.530716in}{0.416447in}}{\pgfqpoint{1.867614in}{1.371882in}}%
\pgfusepath{clip}%
\pgfsetrectcap%
\pgfsetroundjoin%
\pgfsetlinewidth{0.803000pt}%
\definecolor{currentstroke}{rgb}{0.450000,0.450000,0.450000}%
\pgfsetstrokecolor{currentstroke}%
\pgfsetdash{}{0pt}%
\pgfpathmoveto{\pgfqpoint{1.133808in}{0.416447in}}%
\pgfpathlineto{\pgfqpoint{1.133808in}{1.788330in}}%
\pgfusepath{stroke}%
\end{pgfscope}%
\begin{pgfscope}%
\pgfsetbuttcap%
\pgfsetroundjoin%
\definecolor{currentfill}{rgb}{0.000000,0.000000,0.000000}%
\pgfsetfillcolor{currentfill}%
\pgfsetlinewidth{0.803000pt}%
\definecolor{currentstroke}{rgb}{0.000000,0.000000,0.000000}%
\pgfsetstrokecolor{currentstroke}%
\pgfsetdash{}{0pt}%
\pgfsys@defobject{currentmarker}{\pgfqpoint{0.000000in}{-0.048611in}}{\pgfqpoint{0.000000in}{0.000000in}}{%
\pgfpathmoveto{\pgfqpoint{0.000000in}{0.000000in}}%
\pgfpathlineto{\pgfqpoint{0.000000in}{-0.048611in}}%
\pgfusepath{stroke,fill}%
}%
\begin{pgfscope}%
\pgfsys@transformshift{1.133808in}{0.416447in}%
\pgfsys@useobject{currentmarker}{}%
\end{pgfscope}%
\end{pgfscope}%
\begin{pgfscope}%
\definecolor{textcolor}{rgb}{0.000000,0.000000,0.000000}%
\pgfsetstrokecolor{textcolor}%
\pgfsetfillcolor{textcolor}%
\pgftext[x=1.133808in,y=0.319225in,,top]{\color{textcolor}\rmfamily\fontsize{8.000000}{9.600000}\selectfont \(\displaystyle {5000}\)}%
\end{pgfscope}%
\begin{pgfscope}%
\pgfpathrectangle{\pgfqpoint{0.530716in}{0.416447in}}{\pgfqpoint{1.867614in}{1.371882in}}%
\pgfusepath{clip}%
\pgfsetrectcap%
\pgfsetroundjoin%
\pgfsetlinewidth{0.803000pt}%
\definecolor{currentstroke}{rgb}{0.450000,0.450000,0.450000}%
\pgfsetstrokecolor{currentstroke}%
\pgfsetdash{}{0pt}%
\pgfpathmoveto{\pgfqpoint{1.652008in}{0.416447in}}%
\pgfpathlineto{\pgfqpoint{1.652008in}{1.788330in}}%
\pgfusepath{stroke}%
\end{pgfscope}%
\begin{pgfscope}%
\pgfsetbuttcap%
\pgfsetroundjoin%
\definecolor{currentfill}{rgb}{0.000000,0.000000,0.000000}%
\pgfsetfillcolor{currentfill}%
\pgfsetlinewidth{0.803000pt}%
\definecolor{currentstroke}{rgb}{0.000000,0.000000,0.000000}%
\pgfsetstrokecolor{currentstroke}%
\pgfsetdash{}{0pt}%
\pgfsys@defobject{currentmarker}{\pgfqpoint{0.000000in}{-0.048611in}}{\pgfqpoint{0.000000in}{0.000000in}}{%
\pgfpathmoveto{\pgfqpoint{0.000000in}{0.000000in}}%
\pgfpathlineto{\pgfqpoint{0.000000in}{-0.048611in}}%
\pgfusepath{stroke,fill}%
}%
\begin{pgfscope}%
\pgfsys@transformshift{1.652008in}{0.416447in}%
\pgfsys@useobject{currentmarker}{}%
\end{pgfscope}%
\end{pgfscope}%
\begin{pgfscope}%
\definecolor{textcolor}{rgb}{0.000000,0.000000,0.000000}%
\pgfsetstrokecolor{textcolor}%
\pgfsetfillcolor{textcolor}%
\pgftext[x=1.652008in,y=0.319225in,,top]{\color{textcolor}\rmfamily\fontsize{8.000000}{9.600000}\selectfont \(\displaystyle {10000}\)}%
\end{pgfscope}%
\begin{pgfscope}%
\pgfpathrectangle{\pgfqpoint{0.530716in}{0.416447in}}{\pgfqpoint{1.867614in}{1.371882in}}%
\pgfusepath{clip}%
\pgfsetrectcap%
\pgfsetroundjoin%
\pgfsetlinewidth{0.803000pt}%
\definecolor{currentstroke}{rgb}{0.450000,0.450000,0.450000}%
\pgfsetstrokecolor{currentstroke}%
\pgfsetdash{}{0pt}%
\pgfpathmoveto{\pgfqpoint{2.170208in}{0.416447in}}%
\pgfpathlineto{\pgfqpoint{2.170208in}{1.788330in}}%
\pgfusepath{stroke}%
\end{pgfscope}%
\begin{pgfscope}%
\pgfsetbuttcap%
\pgfsetroundjoin%
\definecolor{currentfill}{rgb}{0.000000,0.000000,0.000000}%
\pgfsetfillcolor{currentfill}%
\pgfsetlinewidth{0.803000pt}%
\definecolor{currentstroke}{rgb}{0.000000,0.000000,0.000000}%
\pgfsetstrokecolor{currentstroke}%
\pgfsetdash{}{0pt}%
\pgfsys@defobject{currentmarker}{\pgfqpoint{0.000000in}{-0.048611in}}{\pgfqpoint{0.000000in}{0.000000in}}{%
\pgfpathmoveto{\pgfqpoint{0.000000in}{0.000000in}}%
\pgfpathlineto{\pgfqpoint{0.000000in}{-0.048611in}}%
\pgfusepath{stroke,fill}%
}%
\begin{pgfscope}%
\pgfsys@transformshift{2.170208in}{0.416447in}%
\pgfsys@useobject{currentmarker}{}%
\end{pgfscope}%
\end{pgfscope}%
\begin{pgfscope}%
\definecolor{textcolor}{rgb}{0.000000,0.000000,0.000000}%
\pgfsetstrokecolor{textcolor}%
\pgfsetfillcolor{textcolor}%
\pgftext[x=2.170208in,y=0.319225in,,top]{\color{textcolor}\rmfamily\fontsize{8.000000}{9.600000}\selectfont \(\displaystyle {15000}\)}%
\end{pgfscope}%
\begin{pgfscope}%
\definecolor{textcolor}{rgb}{0.000000,0.000000,0.000000}%
\pgfsetstrokecolor{textcolor}%
\pgfsetfillcolor{textcolor}%
\pgftext[x=1.464523in,y=0.165003in,,top]{\color{textcolor}\rmfamily\fontsize{10.000000}{12.000000}\selectfont Time in \(\displaystyle \unit{\second}\)}%
\end{pgfscope}%
\begin{pgfscope}%
\pgfpathrectangle{\pgfqpoint{0.530716in}{0.416447in}}{\pgfqpoint{1.867614in}{1.371882in}}%
\pgfusepath{clip}%
\pgfsetrectcap%
\pgfsetroundjoin%
\pgfsetlinewidth{0.803000pt}%
\definecolor{currentstroke}{rgb}{0.450000,0.450000,0.450000}%
\pgfsetstrokecolor{currentstroke}%
\pgfsetdash{}{0pt}%
\pgfpathmoveto{\pgfqpoint{0.530716in}{0.416447in}}%
\pgfpathlineto{\pgfqpoint{2.398330in}{0.416447in}}%
\pgfusepath{stroke}%
\end{pgfscope}%
\begin{pgfscope}%
\pgfsetbuttcap%
\pgfsetroundjoin%
\definecolor{currentfill}{rgb}{0.000000,0.000000,0.000000}%
\pgfsetfillcolor{currentfill}%
\pgfsetlinewidth{0.803000pt}%
\definecolor{currentstroke}{rgb}{0.000000,0.000000,0.000000}%
\pgfsetstrokecolor{currentstroke}%
\pgfsetdash{}{0pt}%
\pgfsys@defobject{currentmarker}{\pgfqpoint{-0.048611in}{0.000000in}}{\pgfqpoint{-0.000000in}{0.000000in}}{%
\pgfpathmoveto{\pgfqpoint{-0.000000in}{0.000000in}}%
\pgfpathlineto{\pgfqpoint{-0.048611in}{0.000000in}}%
\pgfusepath{stroke,fill}%
}%
\begin{pgfscope}%
\pgfsys@transformshift{0.530716in}{0.416447in}%
\pgfsys@useobject{currentmarker}{}%
\end{pgfscope}%
\end{pgfscope}%
\begin{pgfscope}%
\definecolor{textcolor}{rgb}{0.000000,0.000000,0.000000}%
\pgfsetstrokecolor{textcolor}%
\pgfsetfillcolor{textcolor}%
\pgftext[x=0.223614in, y=0.377892in, left, base]{\color{textcolor}\rmfamily\fontsize{8.000000}{9.600000}\selectfont \(\displaystyle {\ensuremath{-}15}\)}%
\end{pgfscope}%
\begin{pgfscope}%
\pgfpathrectangle{\pgfqpoint{0.530716in}{0.416447in}}{\pgfqpoint{1.867614in}{1.371882in}}%
\pgfusepath{clip}%
\pgfsetrectcap%
\pgfsetroundjoin%
\pgfsetlinewidth{0.803000pt}%
\definecolor{currentstroke}{rgb}{0.450000,0.450000,0.450000}%
\pgfsetstrokecolor{currentstroke}%
\pgfsetdash{}{0pt}%
\pgfpathmoveto{\pgfqpoint{0.530716in}{0.645095in}}%
\pgfpathlineto{\pgfqpoint{2.398330in}{0.645095in}}%
\pgfusepath{stroke}%
\end{pgfscope}%
\begin{pgfscope}%
\pgfsetbuttcap%
\pgfsetroundjoin%
\definecolor{currentfill}{rgb}{0.000000,0.000000,0.000000}%
\pgfsetfillcolor{currentfill}%
\pgfsetlinewidth{0.803000pt}%
\definecolor{currentstroke}{rgb}{0.000000,0.000000,0.000000}%
\pgfsetstrokecolor{currentstroke}%
\pgfsetdash{}{0pt}%
\pgfsys@defobject{currentmarker}{\pgfqpoint{-0.048611in}{0.000000in}}{\pgfqpoint{-0.000000in}{0.000000in}}{%
\pgfpathmoveto{\pgfqpoint{-0.000000in}{0.000000in}}%
\pgfpathlineto{\pgfqpoint{-0.048611in}{0.000000in}}%
\pgfusepath{stroke,fill}%
}%
\begin{pgfscope}%
\pgfsys@transformshift{0.530716in}{0.645095in}%
\pgfsys@useobject{currentmarker}{}%
\end{pgfscope}%
\end{pgfscope}%
\begin{pgfscope}%
\definecolor{textcolor}{rgb}{0.000000,0.000000,0.000000}%
\pgfsetstrokecolor{textcolor}%
\pgfsetfillcolor{textcolor}%
\pgftext[x=0.223614in, y=0.606539in, left, base]{\color{textcolor}\rmfamily\fontsize{8.000000}{9.600000}\selectfont \(\displaystyle {\ensuremath{-}10}\)}%
\end{pgfscope}%
\begin{pgfscope}%
\pgfpathrectangle{\pgfqpoint{0.530716in}{0.416447in}}{\pgfqpoint{1.867614in}{1.371882in}}%
\pgfusepath{clip}%
\pgfsetrectcap%
\pgfsetroundjoin%
\pgfsetlinewidth{0.803000pt}%
\definecolor{currentstroke}{rgb}{0.450000,0.450000,0.450000}%
\pgfsetstrokecolor{currentstroke}%
\pgfsetdash{}{0pt}%
\pgfpathmoveto{\pgfqpoint{0.530716in}{0.873742in}}%
\pgfpathlineto{\pgfqpoint{2.398330in}{0.873742in}}%
\pgfusepath{stroke}%
\end{pgfscope}%
\begin{pgfscope}%
\pgfsetbuttcap%
\pgfsetroundjoin%
\definecolor{currentfill}{rgb}{0.000000,0.000000,0.000000}%
\pgfsetfillcolor{currentfill}%
\pgfsetlinewidth{0.803000pt}%
\definecolor{currentstroke}{rgb}{0.000000,0.000000,0.000000}%
\pgfsetstrokecolor{currentstroke}%
\pgfsetdash{}{0pt}%
\pgfsys@defobject{currentmarker}{\pgfqpoint{-0.048611in}{0.000000in}}{\pgfqpoint{-0.000000in}{0.000000in}}{%
\pgfpathmoveto{\pgfqpoint{-0.000000in}{0.000000in}}%
\pgfpathlineto{\pgfqpoint{-0.048611in}{0.000000in}}%
\pgfusepath{stroke,fill}%
}%
\begin{pgfscope}%
\pgfsys@transformshift{0.530716in}{0.873742in}%
\pgfsys@useobject{currentmarker}{}%
\end{pgfscope}%
\end{pgfscope}%
\begin{pgfscope}%
\definecolor{textcolor}{rgb}{0.000000,0.000000,0.000000}%
\pgfsetstrokecolor{textcolor}%
\pgfsetfillcolor{textcolor}%
\pgftext[x=0.282643in, y=0.835186in, left, base]{\color{textcolor}\rmfamily\fontsize{8.000000}{9.600000}\selectfont \(\displaystyle {\ensuremath{-}5}\)}%
\end{pgfscope}%
\begin{pgfscope}%
\pgfpathrectangle{\pgfqpoint{0.530716in}{0.416447in}}{\pgfqpoint{1.867614in}{1.371882in}}%
\pgfusepath{clip}%
\pgfsetrectcap%
\pgfsetroundjoin%
\pgfsetlinewidth{0.803000pt}%
\definecolor{currentstroke}{rgb}{0.450000,0.450000,0.450000}%
\pgfsetstrokecolor{currentstroke}%
\pgfsetdash{}{0pt}%
\pgfpathmoveto{\pgfqpoint{0.530716in}{1.102389in}}%
\pgfpathlineto{\pgfqpoint{2.398330in}{1.102389in}}%
\pgfusepath{stroke}%
\end{pgfscope}%
\begin{pgfscope}%
\pgfsetbuttcap%
\pgfsetroundjoin%
\definecolor{currentfill}{rgb}{0.000000,0.000000,0.000000}%
\pgfsetfillcolor{currentfill}%
\pgfsetlinewidth{0.803000pt}%
\definecolor{currentstroke}{rgb}{0.000000,0.000000,0.000000}%
\pgfsetstrokecolor{currentstroke}%
\pgfsetdash{}{0pt}%
\pgfsys@defobject{currentmarker}{\pgfqpoint{-0.048611in}{0.000000in}}{\pgfqpoint{-0.000000in}{0.000000in}}{%
\pgfpathmoveto{\pgfqpoint{-0.000000in}{0.000000in}}%
\pgfpathlineto{\pgfqpoint{-0.048611in}{0.000000in}}%
\pgfusepath{stroke,fill}%
}%
\begin{pgfscope}%
\pgfsys@transformshift{0.530716in}{1.102389in}%
\pgfsys@useobject{currentmarker}{}%
\end{pgfscope}%
\end{pgfscope}%
\begin{pgfscope}%
\definecolor{textcolor}{rgb}{0.000000,0.000000,0.000000}%
\pgfsetstrokecolor{textcolor}%
\pgfsetfillcolor{textcolor}%
\pgftext[x=0.374465in, y=1.063833in, left, base]{\color{textcolor}\rmfamily\fontsize{8.000000}{9.600000}\selectfont \(\displaystyle {0}\)}%
\end{pgfscope}%
\begin{pgfscope}%
\pgfpathrectangle{\pgfqpoint{0.530716in}{0.416447in}}{\pgfqpoint{1.867614in}{1.371882in}}%
\pgfusepath{clip}%
\pgfsetrectcap%
\pgfsetroundjoin%
\pgfsetlinewidth{0.803000pt}%
\definecolor{currentstroke}{rgb}{0.450000,0.450000,0.450000}%
\pgfsetstrokecolor{currentstroke}%
\pgfsetdash{}{0pt}%
\pgfpathmoveto{\pgfqpoint{0.530716in}{1.331036in}}%
\pgfpathlineto{\pgfqpoint{2.398330in}{1.331036in}}%
\pgfusepath{stroke}%
\end{pgfscope}%
\begin{pgfscope}%
\pgfsetbuttcap%
\pgfsetroundjoin%
\definecolor{currentfill}{rgb}{0.000000,0.000000,0.000000}%
\pgfsetfillcolor{currentfill}%
\pgfsetlinewidth{0.803000pt}%
\definecolor{currentstroke}{rgb}{0.000000,0.000000,0.000000}%
\pgfsetstrokecolor{currentstroke}%
\pgfsetdash{}{0pt}%
\pgfsys@defobject{currentmarker}{\pgfqpoint{-0.048611in}{0.000000in}}{\pgfqpoint{-0.000000in}{0.000000in}}{%
\pgfpathmoveto{\pgfqpoint{-0.000000in}{0.000000in}}%
\pgfpathlineto{\pgfqpoint{-0.048611in}{0.000000in}}%
\pgfusepath{stroke,fill}%
}%
\begin{pgfscope}%
\pgfsys@transformshift{0.530716in}{1.331036in}%
\pgfsys@useobject{currentmarker}{}%
\end{pgfscope}%
\end{pgfscope}%
\begin{pgfscope}%
\definecolor{textcolor}{rgb}{0.000000,0.000000,0.000000}%
\pgfsetstrokecolor{textcolor}%
\pgfsetfillcolor{textcolor}%
\pgftext[x=0.374465in, y=1.292480in, left, base]{\color{textcolor}\rmfamily\fontsize{8.000000}{9.600000}\selectfont \(\displaystyle {5}\)}%
\end{pgfscope}%
\begin{pgfscope}%
\pgfpathrectangle{\pgfqpoint{0.530716in}{0.416447in}}{\pgfqpoint{1.867614in}{1.371882in}}%
\pgfusepath{clip}%
\pgfsetrectcap%
\pgfsetroundjoin%
\pgfsetlinewidth{0.803000pt}%
\definecolor{currentstroke}{rgb}{0.450000,0.450000,0.450000}%
\pgfsetstrokecolor{currentstroke}%
\pgfsetdash{}{0pt}%
\pgfpathmoveto{\pgfqpoint{0.530716in}{1.559683in}}%
\pgfpathlineto{\pgfqpoint{2.398330in}{1.559683in}}%
\pgfusepath{stroke}%
\end{pgfscope}%
\begin{pgfscope}%
\pgfsetbuttcap%
\pgfsetroundjoin%
\definecolor{currentfill}{rgb}{0.000000,0.000000,0.000000}%
\pgfsetfillcolor{currentfill}%
\pgfsetlinewidth{0.803000pt}%
\definecolor{currentstroke}{rgb}{0.000000,0.000000,0.000000}%
\pgfsetstrokecolor{currentstroke}%
\pgfsetdash{}{0pt}%
\pgfsys@defobject{currentmarker}{\pgfqpoint{-0.048611in}{0.000000in}}{\pgfqpoint{-0.000000in}{0.000000in}}{%
\pgfpathmoveto{\pgfqpoint{-0.000000in}{0.000000in}}%
\pgfpathlineto{\pgfqpoint{-0.048611in}{0.000000in}}%
\pgfusepath{stroke,fill}%
}%
\begin{pgfscope}%
\pgfsys@transformshift{0.530716in}{1.559683in}%
\pgfsys@useobject{currentmarker}{}%
\end{pgfscope}%
\end{pgfscope}%
\begin{pgfscope}%
\definecolor{textcolor}{rgb}{0.000000,0.000000,0.000000}%
\pgfsetstrokecolor{textcolor}%
\pgfsetfillcolor{textcolor}%
\pgftext[x=0.315437in, y=1.521127in, left, base]{\color{textcolor}\rmfamily\fontsize{8.000000}{9.600000}\selectfont \(\displaystyle {10}\)}%
\end{pgfscope}%
\begin{pgfscope}%
\pgfpathrectangle{\pgfqpoint{0.530716in}{0.416447in}}{\pgfqpoint{1.867614in}{1.371882in}}%
\pgfusepath{clip}%
\pgfsetrectcap%
\pgfsetroundjoin%
\pgfsetlinewidth{0.803000pt}%
\definecolor{currentstroke}{rgb}{0.450000,0.450000,0.450000}%
\pgfsetstrokecolor{currentstroke}%
\pgfsetdash{}{0pt}%
\pgfpathmoveto{\pgfqpoint{0.530716in}{1.788330in}}%
\pgfpathlineto{\pgfqpoint{2.398330in}{1.788330in}}%
\pgfusepath{stroke}%
\end{pgfscope}%
\begin{pgfscope}%
\pgfsetbuttcap%
\pgfsetroundjoin%
\definecolor{currentfill}{rgb}{0.000000,0.000000,0.000000}%
\pgfsetfillcolor{currentfill}%
\pgfsetlinewidth{0.803000pt}%
\definecolor{currentstroke}{rgb}{0.000000,0.000000,0.000000}%
\pgfsetstrokecolor{currentstroke}%
\pgfsetdash{}{0pt}%
\pgfsys@defobject{currentmarker}{\pgfqpoint{-0.048611in}{0.000000in}}{\pgfqpoint{-0.000000in}{0.000000in}}{%
\pgfpathmoveto{\pgfqpoint{-0.000000in}{0.000000in}}%
\pgfpathlineto{\pgfqpoint{-0.048611in}{0.000000in}}%
\pgfusepath{stroke,fill}%
}%
\begin{pgfscope}%
\pgfsys@transformshift{0.530716in}{1.788330in}%
\pgfsys@useobject{currentmarker}{}%
\end{pgfscope}%
\end{pgfscope}%
\begin{pgfscope}%
\definecolor{textcolor}{rgb}{0.000000,0.000000,0.000000}%
\pgfsetstrokecolor{textcolor}%
\pgfsetfillcolor{textcolor}%
\pgftext[x=0.315437in, y=1.749774in, left, base]{\color{textcolor}\rmfamily\fontsize{8.000000}{9.600000}\selectfont \(\displaystyle {15}\)}%
\end{pgfscope}%
\begin{pgfscope}%
\definecolor{textcolor}{rgb}{0.000000,0.000000,0.000000}%
\pgfsetstrokecolor{textcolor}%
\pgfsetfillcolor{textcolor}%
\pgftext[x=0.168059in,y=1.102389in,,bottom,rotate=90.000000]{\color{textcolor}\rmfamily\fontsize{10.000000}{12.000000}\selectfont Ampl. in arb. unit}%
\end{pgfscope}%
\begin{pgfscope}%
\pgfpathrectangle{\pgfqpoint{0.530716in}{0.416447in}}{\pgfqpoint{1.867614in}{1.371882in}}%
\pgfusepath{clip}%
\pgfsetrectcap%
\pgfsetroundjoin%
\pgfsetlinewidth{1.505625pt}%
\definecolor{currentstroke}{rgb}{0.000000,0.619608,0.450980}%
\pgfsetstrokecolor{currentstroke}%
\pgfsetdash{}{0pt}%
\pgfpathmoveto{\pgfqpoint{0.615608in}{1.109968in}}%
\pgfpathlineto{\pgfqpoint{0.616126in}{1.247326in}}%
\pgfpathlineto{\pgfqpoint{0.616748in}{1.156083in}}%
\pgfpathlineto{\pgfqpoint{0.617473in}{0.927633in}}%
\pgfpathlineto{\pgfqpoint{0.617991in}{0.975135in}}%
\pgfpathlineto{\pgfqpoint{0.618717in}{1.122777in}}%
\pgfpathlineto{\pgfqpoint{0.618199in}{0.946292in}}%
\pgfpathlineto{\pgfqpoint{0.619028in}{1.079568in}}%
\pgfpathlineto{\pgfqpoint{0.619442in}{0.878404in}}%
\pgfpathlineto{\pgfqpoint{0.620271in}{0.923986in}}%
\pgfpathlineto{\pgfqpoint{0.621204in}{1.096724in}}%
\pgfpathlineto{\pgfqpoint{0.620582in}{0.896272in}}%
\pgfpathlineto{\pgfqpoint{0.621411in}{0.999151in}}%
\pgfpathlineto{\pgfqpoint{0.622448in}{1.116002in}}%
\pgfpathlineto{\pgfqpoint{0.622033in}{0.913785in}}%
\pgfpathlineto{\pgfqpoint{0.622552in}{1.103577in}}%
\pgfpathlineto{\pgfqpoint{0.623692in}{0.934054in}}%
\pgfpathlineto{\pgfqpoint{0.623070in}{1.206526in}}%
\pgfpathlineto{\pgfqpoint{0.623795in}{0.986921in}}%
\pgfpathlineto{\pgfqpoint{0.624002in}{1.135508in}}%
\pgfpathlineto{\pgfqpoint{0.625039in}{1.051314in}}%
\pgfpathlineto{\pgfqpoint{0.625246in}{1.029245in}}%
\pgfpathlineto{\pgfqpoint{0.626179in}{0.934340in}}%
\pgfpathlineto{\pgfqpoint{0.625557in}{1.043243in}}%
\pgfpathlineto{\pgfqpoint{0.626283in}{0.960896in}}%
\pgfpathlineto{\pgfqpoint{0.627215in}{1.184323in}}%
\pgfpathlineto{\pgfqpoint{0.626904in}{0.926771in}}%
\pgfpathlineto{\pgfqpoint{0.627526in}{1.080835in}}%
\pgfpathlineto{\pgfqpoint{0.627630in}{0.995270in}}%
\pgfpathlineto{\pgfqpoint{0.628148in}{1.170213in}}%
\pgfpathlineto{\pgfqpoint{0.628355in}{1.102870in}}%
\pgfpathlineto{\pgfqpoint{0.628459in}{1.238820in}}%
\pgfpathlineto{\pgfqpoint{0.628977in}{0.964016in}}%
\pgfpathlineto{\pgfqpoint{0.629288in}{1.040594in}}%
\pgfpathlineto{\pgfqpoint{0.630221in}{0.974169in}}%
\pgfpathlineto{\pgfqpoint{0.630117in}{1.126885in}}%
\pgfpathlineto{\pgfqpoint{0.630325in}{1.033228in}}%
\pgfpathlineto{\pgfqpoint{0.630532in}{1.102945in}}%
\pgfpathlineto{\pgfqpoint{0.630739in}{0.936584in}}%
\pgfpathlineto{\pgfqpoint{0.631465in}{1.064450in}}%
\pgfpathlineto{\pgfqpoint{0.631568in}{1.005820in}}%
\pgfpathlineto{\pgfqpoint{0.631672in}{1.157160in}}%
\pgfpathlineto{\pgfqpoint{0.632397in}{1.053697in}}%
\pgfpathlineto{\pgfqpoint{0.632812in}{1.261184in}}%
\pgfpathlineto{\pgfqpoint{0.633226in}{1.028221in}}%
\pgfpathlineto{\pgfqpoint{0.633537in}{1.102756in}}%
\pgfpathlineto{\pgfqpoint{0.634056in}{1.318893in}}%
\pgfpathlineto{\pgfqpoint{0.634366in}{1.089169in}}%
\pgfpathlineto{\pgfqpoint{0.634781in}{1.185422in}}%
\pgfpathlineto{\pgfqpoint{0.636128in}{1.004114in}}%
\pgfpathlineto{\pgfqpoint{0.636232in}{1.069302in}}%
\pgfpathlineto{\pgfqpoint{0.637165in}{1.384778in}}%
\pgfpathlineto{\pgfqpoint{0.636750in}{1.024661in}}%
\pgfpathlineto{\pgfqpoint{0.637787in}{1.273526in}}%
\pgfpathlineto{\pgfqpoint{0.638616in}{1.075928in}}%
\pgfpathlineto{\pgfqpoint{0.638305in}{1.327134in}}%
\pgfpathlineto{\pgfqpoint{0.639030in}{1.085869in}}%
\pgfpathlineto{\pgfqpoint{0.639134in}{1.070207in}}%
\pgfpathlineto{\pgfqpoint{0.639238in}{1.148529in}}%
\pgfpathlineto{\pgfqpoint{0.639652in}{1.078496in}}%
\pgfpathlineto{\pgfqpoint{0.639756in}{1.248867in}}%
\pgfpathlineto{\pgfqpoint{0.639963in}{1.047850in}}%
\pgfpathlineto{\pgfqpoint{0.640689in}{1.186677in}}%
\pgfpathlineto{\pgfqpoint{0.641103in}{1.043034in}}%
\pgfpathlineto{\pgfqpoint{0.641621in}{1.279634in}}%
\pgfpathlineto{\pgfqpoint{0.641725in}{1.272612in}}%
\pgfpathlineto{\pgfqpoint{0.642658in}{0.915205in}}%
\pgfpathlineto{\pgfqpoint{0.642969in}{0.947263in}}%
\pgfpathlineto{\pgfqpoint{0.643487in}{1.143344in}}%
\pgfpathlineto{\pgfqpoint{0.644109in}{1.091762in}}%
\pgfpathlineto{\pgfqpoint{0.644212in}{1.044015in}}%
\pgfpathlineto{\pgfqpoint{0.644730in}{1.204014in}}%
\pgfpathlineto{\pgfqpoint{0.644834in}{1.060292in}}%
\pgfpathlineto{\pgfqpoint{0.644938in}{1.234721in}}%
\pgfpathlineto{\pgfqpoint{0.645041in}{1.030227in}}%
\pgfpathlineto{\pgfqpoint{0.645974in}{1.118830in}}%
\pgfpathlineto{\pgfqpoint{0.646078in}{1.056137in}}%
\pgfpathlineto{\pgfqpoint{0.646492in}{1.182899in}}%
\pgfpathlineto{\pgfqpoint{0.647011in}{1.138926in}}%
\pgfpathlineto{\pgfqpoint{0.647114in}{1.140008in}}%
\pgfpathlineto{\pgfqpoint{0.648047in}{1.290493in}}%
\pgfpathlineto{\pgfqpoint{0.647632in}{1.125555in}}%
\pgfpathlineto{\pgfqpoint{0.648254in}{1.258669in}}%
\pgfpathlineto{\pgfqpoint{0.649187in}{1.102484in}}%
\pgfpathlineto{\pgfqpoint{0.648980in}{1.310698in}}%
\pgfpathlineto{\pgfqpoint{0.649291in}{1.222125in}}%
\pgfpathlineto{\pgfqpoint{0.649602in}{1.263061in}}%
\pgfpathlineto{\pgfqpoint{0.649809in}{1.154456in}}%
\pgfpathlineto{\pgfqpoint{0.649912in}{1.182549in}}%
\pgfpathlineto{\pgfqpoint{0.650742in}{1.085100in}}%
\pgfpathlineto{\pgfqpoint{0.650120in}{1.224078in}}%
\pgfpathlineto{\pgfqpoint{0.651156in}{1.094216in}}%
\pgfpathlineto{\pgfqpoint{0.651985in}{1.186370in}}%
\pgfpathlineto{\pgfqpoint{0.651467in}{0.983281in}}%
\pgfpathlineto{\pgfqpoint{0.652193in}{1.131481in}}%
\pgfpathlineto{\pgfqpoint{0.652711in}{1.051937in}}%
\pgfpathlineto{\pgfqpoint{0.652918in}{1.211500in}}%
\pgfpathlineto{\pgfqpoint{0.653125in}{1.169683in}}%
\pgfpathlineto{\pgfqpoint{0.654265in}{1.355290in}}%
\pgfpathlineto{\pgfqpoint{0.653747in}{1.094167in}}%
\pgfpathlineto{\pgfqpoint{0.654473in}{1.271570in}}%
\pgfpathlineto{\pgfqpoint{0.655094in}{1.030819in}}%
\pgfpathlineto{\pgfqpoint{0.654680in}{1.375345in}}%
\pgfpathlineto{\pgfqpoint{0.655613in}{1.258499in}}%
\pgfpathlineto{\pgfqpoint{0.655716in}{1.260332in}}%
\pgfpathlineto{\pgfqpoint{0.656545in}{1.068641in}}%
\pgfpathlineto{\pgfqpoint{0.656338in}{1.270436in}}%
\pgfpathlineto{\pgfqpoint{0.656753in}{1.172868in}}%
\pgfpathlineto{\pgfqpoint{0.656856in}{1.254799in}}%
\pgfpathlineto{\pgfqpoint{0.657582in}{1.079117in}}%
\pgfpathlineto{\pgfqpoint{0.657789in}{1.124243in}}%
\pgfpathlineto{\pgfqpoint{0.659136in}{1.370961in}}%
\pgfpathlineto{\pgfqpoint{0.658204in}{1.014680in}}%
\pgfpathlineto{\pgfqpoint{0.659344in}{1.318152in}}%
\pgfpathlineto{\pgfqpoint{0.659447in}{1.315279in}}%
\pgfpathlineto{\pgfqpoint{0.659551in}{1.369055in}}%
\pgfpathlineto{\pgfqpoint{0.660069in}{1.130721in}}%
\pgfpathlineto{\pgfqpoint{0.660173in}{1.035011in}}%
\pgfpathlineto{\pgfqpoint{0.660691in}{1.317579in}}%
\pgfpathlineto{\pgfqpoint{0.661106in}{1.171398in}}%
\pgfpathlineto{\pgfqpoint{0.661209in}{1.162462in}}%
\pgfpathlineto{\pgfqpoint{0.662038in}{1.016417in}}%
\pgfpathlineto{\pgfqpoint{0.662349in}{1.049586in}}%
\pgfpathlineto{\pgfqpoint{0.663178in}{1.149404in}}%
\pgfpathlineto{\pgfqpoint{0.663075in}{0.960070in}}%
\pgfpathlineto{\pgfqpoint{0.663489in}{1.085031in}}%
\pgfpathlineto{\pgfqpoint{0.663697in}{1.067189in}}%
\pgfpathlineto{\pgfqpoint{0.663800in}{1.120083in}}%
\pgfpathlineto{\pgfqpoint{0.663904in}{1.155310in}}%
\pgfpathlineto{\pgfqpoint{0.664318in}{0.921190in}}%
\pgfpathlineto{\pgfqpoint{0.664422in}{0.901534in}}%
\pgfpathlineto{\pgfqpoint{0.664526in}{1.070855in}}%
\pgfpathlineto{\pgfqpoint{0.665044in}{1.325961in}}%
\pgfpathlineto{\pgfqpoint{0.665459in}{1.040316in}}%
\pgfpathlineto{\pgfqpoint{0.665562in}{1.109233in}}%
\pgfpathlineto{\pgfqpoint{0.665977in}{0.969365in}}%
\pgfpathlineto{\pgfqpoint{0.666391in}{1.216163in}}%
\pgfpathlineto{\pgfqpoint{0.666599in}{1.091381in}}%
\pgfpathlineto{\pgfqpoint{0.667428in}{1.234182in}}%
\pgfpathlineto{\pgfqpoint{0.667220in}{0.999829in}}%
\pgfpathlineto{\pgfqpoint{0.667635in}{1.143840in}}%
\pgfpathlineto{\pgfqpoint{0.668775in}{1.000218in}}%
\pgfpathlineto{\pgfqpoint{0.669293in}{1.113969in}}%
\pgfpathlineto{\pgfqpoint{0.669397in}{0.976961in}}%
\pgfpathlineto{\pgfqpoint{0.669811in}{1.017805in}}%
\pgfpathlineto{\pgfqpoint{0.670019in}{0.877657in}}%
\pgfpathlineto{\pgfqpoint{0.670433in}{1.023914in}}%
\pgfpathlineto{\pgfqpoint{0.670951in}{0.947396in}}%
\pgfpathlineto{\pgfqpoint{0.671055in}{0.903614in}}%
\pgfpathlineto{\pgfqpoint{0.671470in}{1.032991in}}%
\pgfpathlineto{\pgfqpoint{0.671677in}{1.008405in}}%
\pgfpathlineto{\pgfqpoint{0.671781in}{1.100311in}}%
\pgfpathlineto{\pgfqpoint{0.671884in}{0.882713in}}%
\pgfpathlineto{\pgfqpoint{0.672817in}{1.054638in}}%
\pgfpathlineto{\pgfqpoint{0.672921in}{1.062854in}}%
\pgfpathlineto{\pgfqpoint{0.673128in}{0.995090in}}%
\pgfpathlineto{\pgfqpoint{0.673232in}{0.983791in}}%
\pgfpathlineto{\pgfqpoint{0.673335in}{1.031025in}}%
\pgfpathlineto{\pgfqpoint{0.673542in}{1.030278in}}%
\pgfpathlineto{\pgfqpoint{0.673646in}{1.168069in}}%
\pgfpathlineto{\pgfqpoint{0.674372in}{0.868565in}}%
\pgfpathlineto{\pgfqpoint{0.674682in}{1.091968in}}%
\pgfpathlineto{\pgfqpoint{0.675097in}{0.912814in}}%
\pgfpathlineto{\pgfqpoint{0.674993in}{1.093222in}}%
\pgfpathlineto{\pgfqpoint{0.675823in}{1.072513in}}%
\pgfpathlineto{\pgfqpoint{0.675926in}{1.209647in}}%
\pgfpathlineto{\pgfqpoint{0.676341in}{1.026099in}}%
\pgfpathlineto{\pgfqpoint{0.676859in}{1.077727in}}%
\pgfpathlineto{\pgfqpoint{0.677066in}{1.147373in}}%
\pgfpathlineto{\pgfqpoint{0.677273in}{1.061705in}}%
\pgfpathlineto{\pgfqpoint{0.677377in}{1.002634in}}%
\pgfpathlineto{\pgfqpoint{0.677999in}{1.171161in}}%
\pgfpathlineto{\pgfqpoint{0.678103in}{1.148351in}}%
\pgfpathlineto{\pgfqpoint{0.678206in}{1.249771in}}%
\pgfpathlineto{\pgfqpoint{0.678517in}{0.955343in}}%
\pgfpathlineto{\pgfqpoint{0.679139in}{1.231168in}}%
\pgfpathlineto{\pgfqpoint{0.679761in}{1.099783in}}%
\pgfpathlineto{\pgfqpoint{0.680072in}{1.287756in}}%
\pgfpathlineto{\pgfqpoint{0.680175in}{1.331526in}}%
\pgfpathlineto{\pgfqpoint{0.680383in}{1.141880in}}%
\pgfpathlineto{\pgfqpoint{0.680694in}{1.235282in}}%
\pgfpathlineto{\pgfqpoint{0.681730in}{0.979615in}}%
\pgfpathlineto{\pgfqpoint{0.681834in}{1.012818in}}%
\pgfpathlineto{\pgfqpoint{0.682455in}{0.862686in}}%
\pgfpathlineto{\pgfqpoint{0.682974in}{1.177631in}}%
\pgfpathlineto{\pgfqpoint{0.683285in}{1.237057in}}%
\pgfpathlineto{\pgfqpoint{0.684321in}{1.018330in}}%
\pgfpathlineto{\pgfqpoint{0.684632in}{1.161639in}}%
\pgfpathlineto{\pgfqpoint{0.684736in}{0.932602in}}%
\pgfpathlineto{\pgfqpoint{0.685254in}{1.008606in}}%
\pgfpathlineto{\pgfqpoint{0.685565in}{0.905783in}}%
\pgfpathlineto{\pgfqpoint{0.685876in}{1.157345in}}%
\pgfpathlineto{\pgfqpoint{0.686187in}{1.030815in}}%
\pgfpathlineto{\pgfqpoint{0.687119in}{1.149627in}}%
\pgfpathlineto{\pgfqpoint{0.686394in}{0.990265in}}%
\pgfpathlineto{\pgfqpoint{0.687430in}{1.130410in}}%
\pgfpathlineto{\pgfqpoint{0.688467in}{0.921278in}}%
\pgfpathlineto{\pgfqpoint{0.688156in}{1.136896in}}%
\pgfpathlineto{\pgfqpoint{0.688674in}{0.940937in}}%
\pgfpathlineto{\pgfqpoint{0.689607in}{1.133108in}}%
\pgfpathlineto{\pgfqpoint{0.688881in}{0.923344in}}%
\pgfpathlineto{\pgfqpoint{0.689814in}{1.101194in}}%
\pgfpathlineto{\pgfqpoint{0.690332in}{0.930051in}}%
\pgfpathlineto{\pgfqpoint{0.691058in}{0.983864in}}%
\pgfpathlineto{\pgfqpoint{0.692094in}{1.198619in}}%
\pgfpathlineto{\pgfqpoint{0.691472in}{0.909891in}}%
\pgfpathlineto{\pgfqpoint{0.692198in}{1.018679in}}%
\pgfpathlineto{\pgfqpoint{0.692612in}{1.064261in}}%
\pgfpathlineto{\pgfqpoint{0.693441in}{0.856163in}}%
\pgfpathlineto{\pgfqpoint{0.694478in}{1.202750in}}%
\pgfpathlineto{\pgfqpoint{0.694685in}{1.170551in}}%
\pgfpathlineto{\pgfqpoint{0.695203in}{1.223114in}}%
\pgfpathlineto{\pgfqpoint{0.695825in}{1.064803in}}%
\pgfpathlineto{\pgfqpoint{0.696447in}{1.017195in}}%
\pgfpathlineto{\pgfqpoint{0.696240in}{1.126195in}}%
\pgfpathlineto{\pgfqpoint{0.696551in}{1.085595in}}%
\pgfpathlineto{\pgfqpoint{0.696965in}{1.205680in}}%
\pgfpathlineto{\pgfqpoint{0.697276in}{1.036144in}}%
\pgfpathlineto{\pgfqpoint{0.697587in}{1.135212in}}%
\pgfpathlineto{\pgfqpoint{0.697794in}{1.001011in}}%
\pgfpathlineto{\pgfqpoint{0.698416in}{1.172742in}}%
\pgfpathlineto{\pgfqpoint{0.698727in}{1.081647in}}%
\pgfpathlineto{\pgfqpoint{0.699245in}{1.159623in}}%
\pgfpathlineto{\pgfqpoint{0.699556in}{1.049042in}}%
\pgfpathlineto{\pgfqpoint{0.700696in}{1.332203in}}%
\pgfpathlineto{\pgfqpoint{0.699971in}{1.046985in}}%
\pgfpathlineto{\pgfqpoint{0.700800in}{1.213566in}}%
\pgfpathlineto{\pgfqpoint{0.701836in}{1.025260in}}%
\pgfpathlineto{\pgfqpoint{0.701111in}{1.324386in}}%
\pgfpathlineto{\pgfqpoint{0.702043in}{1.093651in}}%
\pgfpathlineto{\pgfqpoint{0.702769in}{1.068868in}}%
\pgfpathlineto{\pgfqpoint{0.703287in}{1.313606in}}%
\pgfpathlineto{\pgfqpoint{0.703805in}{1.069494in}}%
\pgfpathlineto{\pgfqpoint{0.704427in}{1.103830in}}%
\pgfpathlineto{\pgfqpoint{0.705567in}{1.321166in}}%
\pgfpathlineto{\pgfqpoint{0.704738in}{1.043167in}}%
\pgfpathlineto{\pgfqpoint{0.705671in}{1.230464in}}%
\pgfpathlineto{\pgfqpoint{0.705982in}{1.076347in}}%
\pgfpathlineto{\pgfqpoint{0.706189in}{1.243016in}}%
\pgfpathlineto{\pgfqpoint{0.706604in}{1.238745in}}%
\pgfpathlineto{\pgfqpoint{0.706707in}{1.369455in}}%
\pgfpathlineto{\pgfqpoint{0.706915in}{1.190005in}}%
\pgfpathlineto{\pgfqpoint{0.707744in}{1.344204in}}%
\pgfpathlineto{\pgfqpoint{0.707847in}{1.332380in}}%
\pgfpathlineto{\pgfqpoint{0.708055in}{1.396677in}}%
\pgfpathlineto{\pgfqpoint{0.708158in}{1.431033in}}%
\pgfpathlineto{\pgfqpoint{0.708676in}{1.291253in}}%
\pgfpathlineto{\pgfqpoint{0.708780in}{1.324712in}}%
\pgfpathlineto{\pgfqpoint{0.709402in}{1.205474in}}%
\pgfpathlineto{\pgfqpoint{0.709091in}{1.367225in}}%
\pgfpathlineto{\pgfqpoint{0.709609in}{1.318540in}}%
\pgfpathlineto{\pgfqpoint{0.709713in}{1.438912in}}%
\pgfpathlineto{\pgfqpoint{0.710646in}{1.300159in}}%
\pgfpathlineto{\pgfqpoint{0.710853in}{1.380412in}}%
\pgfpathlineto{\pgfqpoint{0.711164in}{1.242706in}}%
\pgfpathlineto{\pgfqpoint{0.711267in}{1.345182in}}%
\pgfpathlineto{\pgfqpoint{0.712200in}{1.105091in}}%
\pgfpathlineto{\pgfqpoint{0.711993in}{1.389306in}}%
\pgfpathlineto{\pgfqpoint{0.712511in}{1.132968in}}%
\pgfpathlineto{\pgfqpoint{0.713340in}{1.341996in}}%
\pgfpathlineto{\pgfqpoint{0.712926in}{1.110635in}}%
\pgfpathlineto{\pgfqpoint{0.713547in}{1.118039in}}%
\pgfpathlineto{\pgfqpoint{0.714066in}{1.201082in}}%
\pgfpathlineto{\pgfqpoint{0.714169in}{1.106315in}}%
\pgfpathlineto{\pgfqpoint{0.715102in}{1.362534in}}%
\pgfpathlineto{\pgfqpoint{0.715413in}{1.298016in}}%
\pgfpathlineto{\pgfqpoint{0.715724in}{1.408682in}}%
\pgfpathlineto{\pgfqpoint{0.716138in}{1.218169in}}%
\pgfpathlineto{\pgfqpoint{0.716346in}{1.227198in}}%
\pgfpathlineto{\pgfqpoint{0.716553in}{1.098795in}}%
\pgfpathlineto{\pgfqpoint{0.716657in}{1.136028in}}%
\pgfpathlineto{\pgfqpoint{0.716864in}{0.988168in}}%
\pgfpathlineto{\pgfqpoint{0.717693in}{1.220958in}}%
\pgfpathlineto{\pgfqpoint{0.717797in}{1.242328in}}%
\pgfpathlineto{\pgfqpoint{0.717900in}{1.191113in}}%
\pgfpathlineto{\pgfqpoint{0.718108in}{1.202484in}}%
\pgfpathlineto{\pgfqpoint{0.718522in}{1.045396in}}%
\pgfpathlineto{\pgfqpoint{0.719144in}{1.274342in}}%
\pgfpathlineto{\pgfqpoint{0.719248in}{1.285523in}}%
\pgfpathlineto{\pgfqpoint{0.719351in}{1.243777in}}%
\pgfpathlineto{\pgfqpoint{0.719455in}{1.182102in}}%
\pgfpathlineto{\pgfqpoint{0.720180in}{1.336145in}}%
\pgfpathlineto{\pgfqpoint{0.720284in}{1.298841in}}%
\pgfpathlineto{\pgfqpoint{0.720388in}{1.418685in}}%
\pgfpathlineto{\pgfqpoint{0.721320in}{1.238205in}}%
\pgfpathlineto{\pgfqpoint{0.721424in}{1.247921in}}%
\pgfpathlineto{\pgfqpoint{0.721631in}{1.439234in}}%
\pgfpathlineto{\pgfqpoint{0.722357in}{1.160599in}}%
\pgfpathlineto{\pgfqpoint{0.722461in}{1.245068in}}%
\pgfpathlineto{\pgfqpoint{0.723290in}{1.373145in}}%
\pgfpathlineto{\pgfqpoint{0.722979in}{1.143850in}}%
\pgfpathlineto{\pgfqpoint{0.723601in}{1.303081in}}%
\pgfpathlineto{\pgfqpoint{0.724015in}{1.205716in}}%
\pgfpathlineto{\pgfqpoint{0.724222in}{1.398695in}}%
\pgfpathlineto{\pgfqpoint{0.724533in}{1.341115in}}%
\pgfpathlineto{\pgfqpoint{0.724844in}{1.240083in}}%
\pgfpathlineto{\pgfqpoint{0.725052in}{1.344355in}}%
\pgfpathlineto{\pgfqpoint{0.725362in}{1.341679in}}%
\pgfpathlineto{\pgfqpoint{0.725984in}{1.081774in}}%
\pgfpathlineto{\pgfqpoint{0.726502in}{1.234568in}}%
\pgfpathlineto{\pgfqpoint{0.727021in}{1.090505in}}%
\pgfpathlineto{\pgfqpoint{0.727228in}{1.250662in}}%
\pgfpathlineto{\pgfqpoint{0.728057in}{1.336278in}}%
\pgfpathlineto{\pgfqpoint{0.727643in}{1.176681in}}%
\pgfpathlineto{\pgfqpoint{0.728264in}{1.265814in}}%
\pgfpathlineto{\pgfqpoint{0.729197in}{1.127633in}}%
\pgfpathlineto{\pgfqpoint{0.728886in}{1.311202in}}%
\pgfpathlineto{\pgfqpoint{0.729404in}{1.260618in}}%
\pgfpathlineto{\pgfqpoint{0.729612in}{1.077335in}}%
\pgfpathlineto{\pgfqpoint{0.730441in}{1.328026in}}%
\pgfpathlineto{\pgfqpoint{0.730544in}{1.256550in}}%
\pgfpathlineto{\pgfqpoint{0.731166in}{1.449202in}}%
\pgfpathlineto{\pgfqpoint{0.731374in}{1.343516in}}%
\pgfpathlineto{\pgfqpoint{0.731684in}{1.449794in}}%
\pgfpathlineto{\pgfqpoint{0.732306in}{1.278835in}}%
\pgfpathlineto{\pgfqpoint{0.733239in}{1.449483in}}%
\pgfpathlineto{\pgfqpoint{0.733965in}{1.371395in}}%
\pgfpathlineto{\pgfqpoint{0.735001in}{1.097645in}}%
\pgfpathlineto{\pgfqpoint{0.735312in}{1.222663in}}%
\pgfpathlineto{\pgfqpoint{0.736348in}{1.425951in}}%
\pgfpathlineto{\pgfqpoint{0.735726in}{1.133940in}}%
\pgfpathlineto{\pgfqpoint{0.736452in}{1.396287in}}%
\pgfpathlineto{\pgfqpoint{0.736556in}{1.395618in}}%
\pgfpathlineto{\pgfqpoint{0.736763in}{1.423626in}}%
\pgfpathlineto{\pgfqpoint{0.737177in}{1.296976in}}%
\pgfpathlineto{\pgfqpoint{0.737281in}{1.296220in}}%
\pgfpathlineto{\pgfqpoint{0.737696in}{1.154854in}}%
\pgfpathlineto{\pgfqpoint{0.738007in}{1.343568in}}%
\pgfpathlineto{\pgfqpoint{0.738421in}{1.189255in}}%
\pgfpathlineto{\pgfqpoint{0.738628in}{1.258773in}}%
\pgfpathlineto{\pgfqpoint{0.738939in}{1.131070in}}%
\pgfpathlineto{\pgfqpoint{0.739147in}{1.195327in}}%
\pgfpathlineto{\pgfqpoint{0.739250in}{1.087135in}}%
\pgfpathlineto{\pgfqpoint{0.739768in}{1.308428in}}%
\pgfpathlineto{\pgfqpoint{0.740183in}{1.272624in}}%
\pgfpathlineto{\pgfqpoint{0.740701in}{1.103929in}}%
\pgfpathlineto{\pgfqpoint{0.740805in}{1.279946in}}%
\pgfpathlineto{\pgfqpoint{0.741323in}{1.178076in}}%
\pgfpathlineto{\pgfqpoint{0.741841in}{1.331514in}}%
\pgfpathlineto{\pgfqpoint{0.741738in}{1.156497in}}%
\pgfpathlineto{\pgfqpoint{0.742463in}{1.227608in}}%
\pgfpathlineto{\pgfqpoint{0.742670in}{1.220213in}}%
\pgfpathlineto{\pgfqpoint{0.742774in}{1.290268in}}%
\pgfpathlineto{\pgfqpoint{0.743085in}{1.187699in}}%
\pgfpathlineto{\pgfqpoint{0.743914in}{1.442133in}}%
\pgfpathlineto{\pgfqpoint{0.745261in}{1.095221in}}%
\pgfpathlineto{\pgfqpoint{0.745676in}{1.303923in}}%
\pgfpathlineto{\pgfqpoint{0.746505in}{1.232119in}}%
\pgfpathlineto{\pgfqpoint{0.746920in}{1.114817in}}%
\pgfpathlineto{\pgfqpoint{0.747230in}{1.283516in}}%
\pgfpathlineto{\pgfqpoint{0.747541in}{1.282703in}}%
\pgfpathlineto{\pgfqpoint{0.747645in}{1.254953in}}%
\pgfpathlineto{\pgfqpoint{0.747749in}{1.401384in}}%
\pgfpathlineto{\pgfqpoint{0.748371in}{1.314264in}}%
\pgfpathlineto{\pgfqpoint{0.748474in}{1.352980in}}%
\pgfpathlineto{\pgfqpoint{0.748785in}{1.097009in}}%
\pgfpathlineto{\pgfqpoint{0.748889in}{1.204242in}}%
\pgfpathlineto{\pgfqpoint{0.749096in}{1.024667in}}%
\pgfpathlineto{\pgfqpoint{0.749303in}{1.321989in}}%
\pgfpathlineto{\pgfqpoint{0.749925in}{1.248186in}}%
\pgfpathlineto{\pgfqpoint{0.750340in}{1.300076in}}%
\pgfpathlineto{\pgfqpoint{0.750547in}{1.201464in}}%
\pgfpathlineto{\pgfqpoint{0.750651in}{1.261258in}}%
\pgfpathlineto{\pgfqpoint{0.750754in}{1.184976in}}%
\pgfpathlineto{\pgfqpoint{0.751480in}{1.377112in}}%
\pgfpathlineto{\pgfqpoint{0.751583in}{1.357403in}}%
\pgfpathlineto{\pgfqpoint{0.753138in}{1.082597in}}%
\pgfpathlineto{\pgfqpoint{0.753656in}{1.341465in}}%
\pgfpathlineto{\pgfqpoint{0.754382in}{1.230683in}}%
\pgfpathlineto{\pgfqpoint{0.754485in}{1.225815in}}%
\pgfpathlineto{\pgfqpoint{0.754589in}{1.254673in}}%
\pgfpathlineto{\pgfqpoint{0.754796in}{1.341118in}}%
\pgfpathlineto{\pgfqpoint{0.755003in}{1.054117in}}%
\pgfpathlineto{\pgfqpoint{0.755418in}{1.082418in}}%
\pgfpathlineto{\pgfqpoint{0.755936in}{0.966725in}}%
\pgfpathlineto{\pgfqpoint{0.755729in}{1.181405in}}%
\pgfpathlineto{\pgfqpoint{0.756454in}{1.087211in}}%
\pgfpathlineto{\pgfqpoint{0.757387in}{1.045208in}}%
\pgfpathlineto{\pgfqpoint{0.757594in}{1.210298in}}%
\pgfpathlineto{\pgfqpoint{0.757802in}{0.994439in}}%
\pgfpathlineto{\pgfqpoint{0.758424in}{1.245027in}}%
\pgfpathlineto{\pgfqpoint{0.758631in}{1.163722in}}%
\pgfpathlineto{\pgfqpoint{0.759253in}{1.365120in}}%
\pgfpathlineto{\pgfqpoint{0.759771in}{1.354662in}}%
\pgfpathlineto{\pgfqpoint{0.760600in}{1.124259in}}%
\pgfpathlineto{\pgfqpoint{0.760911in}{1.282703in}}%
\pgfpathlineto{\pgfqpoint{0.761222in}{1.133328in}}%
\pgfpathlineto{\pgfqpoint{0.761947in}{1.418904in}}%
\pgfpathlineto{\pgfqpoint{0.763191in}{1.119258in}}%
\pgfpathlineto{\pgfqpoint{0.763295in}{1.334814in}}%
\pgfpathlineto{\pgfqpoint{0.764331in}{1.243833in}}%
\pgfpathlineto{\pgfqpoint{0.764849in}{1.409041in}}%
\pgfpathlineto{\pgfqpoint{0.765367in}{1.235881in}}%
\pgfpathlineto{\pgfqpoint{0.765471in}{1.296183in}}%
\pgfpathlineto{\pgfqpoint{0.766093in}{1.527159in}}%
\pgfpathlineto{\pgfqpoint{0.766300in}{1.290792in}}%
\pgfpathlineto{\pgfqpoint{0.766715in}{1.438197in}}%
\pgfpathlineto{\pgfqpoint{0.767233in}{1.248492in}}%
\pgfpathlineto{\pgfqpoint{0.768062in}{1.311692in}}%
\pgfpathlineto{\pgfqpoint{0.768166in}{1.314648in}}%
\pgfpathlineto{\pgfqpoint{0.768788in}{1.247716in}}%
\pgfpathlineto{\pgfqpoint{0.769099in}{1.398559in}}%
\pgfpathlineto{\pgfqpoint{0.769202in}{1.312527in}}%
\pgfpathlineto{\pgfqpoint{0.769409in}{1.378801in}}%
\pgfpathlineto{\pgfqpoint{0.769513in}{1.243577in}}%
\pgfpathlineto{\pgfqpoint{0.769720in}{1.120023in}}%
\pgfpathlineto{\pgfqpoint{0.770031in}{1.409615in}}%
\pgfpathlineto{\pgfqpoint{0.770342in}{1.398026in}}%
\pgfpathlineto{\pgfqpoint{0.770549in}{1.535165in}}%
\pgfpathlineto{\pgfqpoint{0.771171in}{1.354033in}}%
\pgfpathlineto{\pgfqpoint{0.771379in}{1.358290in}}%
\pgfpathlineto{\pgfqpoint{0.772311in}{1.200569in}}%
\pgfpathlineto{\pgfqpoint{0.771897in}{1.482621in}}%
\pgfpathlineto{\pgfqpoint{0.772519in}{1.273165in}}%
\pgfpathlineto{\pgfqpoint{0.773762in}{1.526496in}}%
\pgfpathlineto{\pgfqpoint{0.775006in}{1.041784in}}%
\pgfpathlineto{\pgfqpoint{0.775110in}{1.152414in}}%
\pgfpathlineto{\pgfqpoint{0.775317in}{1.377804in}}%
\pgfpathlineto{\pgfqpoint{0.776250in}{1.297696in}}%
\pgfpathlineto{\pgfqpoint{0.776975in}{1.363585in}}%
\pgfpathlineto{\pgfqpoint{0.777390in}{1.095914in}}%
\pgfpathlineto{\pgfqpoint{0.777908in}{1.292308in}}%
\pgfpathlineto{\pgfqpoint{0.778426in}{1.223707in}}%
\pgfpathlineto{\pgfqpoint{0.778841in}{1.103244in}}%
\pgfpathlineto{\pgfqpoint{0.779359in}{1.269975in}}%
\pgfpathlineto{\pgfqpoint{0.779463in}{1.165576in}}%
\pgfpathlineto{\pgfqpoint{0.779773in}{1.269323in}}%
\pgfpathlineto{\pgfqpoint{0.780292in}{1.136907in}}%
\pgfpathlineto{\pgfqpoint{0.780395in}{0.975741in}}%
\pgfpathlineto{\pgfqpoint{0.781017in}{1.270501in}}%
\pgfpathlineto{\pgfqpoint{0.781328in}{1.156036in}}%
\pgfpathlineto{\pgfqpoint{0.781432in}{1.262207in}}%
\pgfpathlineto{\pgfqpoint{0.781846in}{1.089053in}}%
\pgfpathlineto{\pgfqpoint{0.782364in}{1.177707in}}%
\pgfpathlineto{\pgfqpoint{0.782468in}{1.079604in}}%
\pgfpathlineto{\pgfqpoint{0.782883in}{1.393544in}}%
\pgfpathlineto{\pgfqpoint{0.783297in}{1.293719in}}%
\pgfpathlineto{\pgfqpoint{0.783504in}{1.150406in}}%
\pgfpathlineto{\pgfqpoint{0.784126in}{1.352389in}}%
\pgfpathlineto{\pgfqpoint{0.784230in}{1.307490in}}%
\pgfpathlineto{\pgfqpoint{0.784334in}{1.348532in}}%
\pgfpathlineto{\pgfqpoint{0.784645in}{1.226549in}}%
\pgfpathlineto{\pgfqpoint{0.784955in}{1.238532in}}%
\pgfpathlineto{\pgfqpoint{0.785059in}{1.184567in}}%
\pgfpathlineto{\pgfqpoint{0.785266in}{1.383250in}}%
\pgfpathlineto{\pgfqpoint{0.785888in}{1.224645in}}%
\pgfpathlineto{\pgfqpoint{0.786406in}{1.504051in}}%
\pgfpathlineto{\pgfqpoint{0.787236in}{1.419919in}}%
\pgfpathlineto{\pgfqpoint{0.787754in}{1.220462in}}%
\pgfpathlineto{\pgfqpoint{0.787546in}{1.447459in}}%
\pgfpathlineto{\pgfqpoint{0.788376in}{1.324154in}}%
\pgfpathlineto{\pgfqpoint{0.788479in}{1.338902in}}%
\pgfpathlineto{\pgfqpoint{0.788790in}{1.258324in}}%
\pgfpathlineto{\pgfqpoint{0.788894in}{1.209483in}}%
\pgfpathlineto{\pgfqpoint{0.789412in}{1.437326in}}%
\pgfpathlineto{\pgfqpoint{0.789619in}{1.376621in}}%
\pgfpathlineto{\pgfqpoint{0.790034in}{1.238588in}}%
\pgfpathlineto{\pgfqpoint{0.790759in}{1.287793in}}%
\pgfpathlineto{\pgfqpoint{0.791174in}{1.417919in}}%
\pgfpathlineto{\pgfqpoint{0.790967in}{1.235377in}}%
\pgfpathlineto{\pgfqpoint{0.792003in}{1.414696in}}%
\pgfpathlineto{\pgfqpoint{0.792936in}{1.213456in}}%
\pgfpathlineto{\pgfqpoint{0.792418in}{1.491742in}}%
\pgfpathlineto{\pgfqpoint{0.793454in}{1.329175in}}%
\pgfpathlineto{\pgfqpoint{0.794179in}{1.291671in}}%
\pgfpathlineto{\pgfqpoint{0.794698in}{1.481443in}}%
\pgfpathlineto{\pgfqpoint{0.795527in}{1.199202in}}%
\pgfpathlineto{\pgfqpoint{0.795941in}{1.323603in}}%
\pgfpathlineto{\pgfqpoint{0.796045in}{1.328099in}}%
\pgfpathlineto{\pgfqpoint{0.796770in}{1.427844in}}%
\pgfpathlineto{\pgfqpoint{0.796874in}{1.199182in}}%
\pgfpathlineto{\pgfqpoint{0.796978in}{1.324726in}}%
\pgfpathlineto{\pgfqpoint{0.797185in}{1.167167in}}%
\pgfpathlineto{\pgfqpoint{0.797703in}{1.449935in}}%
\pgfpathlineto{\pgfqpoint{0.797910in}{1.523406in}}%
\pgfpathlineto{\pgfqpoint{0.798221in}{1.313897in}}%
\pgfpathlineto{\pgfqpoint{0.798325in}{1.337234in}}%
\pgfpathlineto{\pgfqpoint{0.799465in}{1.265942in}}%
\pgfpathlineto{\pgfqpoint{0.799050in}{1.495366in}}%
\pgfpathlineto{\pgfqpoint{0.799569in}{1.270126in}}%
\pgfpathlineto{\pgfqpoint{0.799672in}{1.298604in}}%
\pgfpathlineto{\pgfqpoint{0.800191in}{1.151392in}}%
\pgfpathlineto{\pgfqpoint{0.800398in}{1.263376in}}%
\pgfpathlineto{\pgfqpoint{0.801020in}{1.153462in}}%
\pgfpathlineto{\pgfqpoint{0.800916in}{1.323327in}}%
\pgfpathlineto{\pgfqpoint{0.801538in}{1.211759in}}%
\pgfpathlineto{\pgfqpoint{0.801641in}{1.213909in}}%
\pgfpathlineto{\pgfqpoint{0.802574in}{1.060827in}}%
\pgfpathlineto{\pgfqpoint{0.802056in}{1.385532in}}%
\pgfpathlineto{\pgfqpoint{0.802678in}{1.155497in}}%
\pgfpathlineto{\pgfqpoint{0.803403in}{1.318404in}}%
\pgfpathlineto{\pgfqpoint{0.802885in}{1.091980in}}%
\pgfpathlineto{\pgfqpoint{0.803611in}{1.194635in}}%
\pgfpathlineto{\pgfqpoint{0.803714in}{1.073754in}}%
\pgfpathlineto{\pgfqpoint{0.804751in}{1.109272in}}%
\pgfpathlineto{\pgfqpoint{0.805373in}{1.291865in}}%
\pgfpathlineto{\pgfqpoint{0.805891in}{1.204092in}}%
\pgfpathlineto{\pgfqpoint{0.805994in}{1.206039in}}%
\pgfpathlineto{\pgfqpoint{0.806098in}{1.195597in}}%
\pgfpathlineto{\pgfqpoint{0.806202in}{1.195921in}}%
\pgfpathlineto{\pgfqpoint{0.806305in}{1.073943in}}%
\pgfpathlineto{\pgfqpoint{0.807134in}{1.233030in}}%
\pgfpathlineto{\pgfqpoint{0.807238in}{1.169874in}}%
\pgfpathlineto{\pgfqpoint{0.807860in}{1.322866in}}%
\pgfpathlineto{\pgfqpoint{0.808378in}{1.306507in}}%
\pgfpathlineto{\pgfqpoint{0.808793in}{1.160129in}}%
\pgfpathlineto{\pgfqpoint{0.809000in}{1.331539in}}%
\pgfpathlineto{\pgfqpoint{0.810140in}{1.141125in}}%
\pgfpathlineto{\pgfqpoint{0.809829in}{1.365732in}}%
\pgfpathlineto{\pgfqpoint{0.810244in}{1.141727in}}%
\pgfpathlineto{\pgfqpoint{0.810451in}{1.338943in}}%
\pgfpathlineto{\pgfqpoint{0.811384in}{1.245539in}}%
\pgfpathlineto{\pgfqpoint{0.812005in}{1.181255in}}%
\pgfpathlineto{\pgfqpoint{0.812627in}{1.325767in}}%
\pgfpathlineto{\pgfqpoint{0.813042in}{1.173764in}}%
\pgfpathlineto{\pgfqpoint{0.813249in}{1.108029in}}%
\pgfpathlineto{\pgfqpoint{0.813456in}{1.302804in}}%
\pgfpathlineto{\pgfqpoint{0.813560in}{1.404391in}}%
\pgfpathlineto{\pgfqpoint{0.814182in}{1.164577in}}%
\pgfpathlineto{\pgfqpoint{0.814389in}{1.166594in}}%
\pgfpathlineto{\pgfqpoint{0.815115in}{1.025580in}}%
\pgfpathlineto{\pgfqpoint{0.814907in}{1.309962in}}%
\pgfpathlineto{\pgfqpoint{0.815322in}{1.204344in}}%
\pgfpathlineto{\pgfqpoint{0.815944in}{1.374002in}}%
\pgfpathlineto{\pgfqpoint{0.815529in}{1.162979in}}%
\pgfpathlineto{\pgfqpoint{0.816462in}{1.277308in}}%
\pgfpathlineto{\pgfqpoint{0.816566in}{1.282925in}}%
\pgfpathlineto{\pgfqpoint{0.816669in}{1.237941in}}%
\pgfpathlineto{\pgfqpoint{0.816773in}{1.262578in}}%
\pgfpathlineto{\pgfqpoint{0.817602in}{1.148076in}}%
\pgfpathlineto{\pgfqpoint{0.817395in}{1.324326in}}%
\pgfpathlineto{\pgfqpoint{0.818017in}{1.161849in}}%
\pgfpathlineto{\pgfqpoint{0.818328in}{1.381415in}}%
\pgfpathlineto{\pgfqpoint{0.818224in}{1.098965in}}%
\pgfpathlineto{\pgfqpoint{0.819157in}{1.322156in}}%
\pgfpathlineto{\pgfqpoint{0.819778in}{1.118201in}}%
\pgfpathlineto{\pgfqpoint{0.819675in}{1.352333in}}%
\pgfpathlineto{\pgfqpoint{0.820193in}{1.228953in}}%
\pgfpathlineto{\pgfqpoint{0.820400in}{1.348904in}}%
\pgfpathlineto{\pgfqpoint{0.820815in}{1.149184in}}%
\pgfpathlineto{\pgfqpoint{0.821333in}{1.327251in}}%
\pgfpathlineto{\pgfqpoint{0.821437in}{1.329859in}}%
\pgfpathlineto{\pgfqpoint{0.821540in}{1.257378in}}%
\pgfpathlineto{\pgfqpoint{0.822266in}{1.425246in}}%
\pgfpathlineto{\pgfqpoint{0.822577in}{1.303225in}}%
\pgfpathlineto{\pgfqpoint{0.823199in}{1.108260in}}%
\pgfpathlineto{\pgfqpoint{0.823613in}{1.238779in}}%
\pgfpathlineto{\pgfqpoint{0.823717in}{1.291529in}}%
\pgfpathlineto{\pgfqpoint{0.824131in}{1.103933in}}%
\pgfpathlineto{\pgfqpoint{0.824442in}{1.171672in}}%
\pgfpathlineto{\pgfqpoint{0.825271in}{0.955039in}}%
\pgfpathlineto{\pgfqpoint{0.825064in}{1.258697in}}%
\pgfpathlineto{\pgfqpoint{0.825479in}{1.228689in}}%
\pgfpathlineto{\pgfqpoint{0.826411in}{1.440773in}}%
\pgfpathlineto{\pgfqpoint{0.825997in}{1.153957in}}%
\pgfpathlineto{\pgfqpoint{0.826722in}{1.280429in}}%
\pgfpathlineto{\pgfqpoint{0.827344in}{1.138283in}}%
\pgfpathlineto{\pgfqpoint{0.827862in}{1.212175in}}%
\pgfpathlineto{\pgfqpoint{0.827966in}{1.266467in}}%
\pgfpathlineto{\pgfqpoint{0.828484in}{1.084091in}}%
\pgfpathlineto{\pgfqpoint{0.829002in}{1.244901in}}%
\pgfpathlineto{\pgfqpoint{0.829624in}{1.024350in}}%
\pgfpathlineto{\pgfqpoint{0.829935in}{1.254898in}}%
\pgfpathlineto{\pgfqpoint{0.830142in}{1.328258in}}%
\pgfpathlineto{\pgfqpoint{0.830557in}{1.206231in}}%
\pgfpathlineto{\pgfqpoint{0.830868in}{1.266801in}}%
\pgfpathlineto{\pgfqpoint{0.831386in}{1.041728in}}%
\pgfpathlineto{\pgfqpoint{0.831179in}{1.277884in}}%
\pgfpathlineto{\pgfqpoint{0.832112in}{1.201596in}}%
\pgfpathlineto{\pgfqpoint{0.832319in}{1.235625in}}%
\pgfpathlineto{\pgfqpoint{0.832941in}{1.070480in}}%
\pgfpathlineto{\pgfqpoint{0.832733in}{1.245722in}}%
\pgfpathlineto{\pgfqpoint{0.833355in}{1.095658in}}%
\pgfpathlineto{\pgfqpoint{0.833666in}{1.252742in}}%
\pgfpathlineto{\pgfqpoint{0.834392in}{1.177872in}}%
\pgfpathlineto{\pgfqpoint{0.835014in}{0.952925in}}%
\pgfpathlineto{\pgfqpoint{0.835221in}{1.240551in}}%
\pgfpathlineto{\pgfqpoint{0.835428in}{1.190412in}}%
\pgfpathlineto{\pgfqpoint{0.835843in}{1.108322in}}%
\pgfpathlineto{\pgfqpoint{0.836154in}{1.259698in}}%
\pgfpathlineto{\pgfqpoint{0.836361in}{1.232773in}}%
\pgfpathlineto{\pgfqpoint{0.836983in}{1.317258in}}%
\pgfpathlineto{\pgfqpoint{0.837086in}{1.177823in}}%
\pgfpathlineto{\pgfqpoint{0.837812in}{1.293867in}}%
\pgfpathlineto{\pgfqpoint{0.838226in}{1.049257in}}%
\pgfpathlineto{\pgfqpoint{0.839781in}{1.328888in}}%
\pgfpathlineto{\pgfqpoint{0.840092in}{1.093890in}}%
\pgfpathlineto{\pgfqpoint{0.840817in}{1.131177in}}%
\pgfpathlineto{\pgfqpoint{0.841232in}{1.320548in}}%
\pgfpathlineto{\pgfqpoint{0.841854in}{1.095170in}}%
\pgfpathlineto{\pgfqpoint{0.841957in}{1.265092in}}%
\pgfpathlineto{\pgfqpoint{0.842476in}{1.076763in}}%
\pgfpathlineto{\pgfqpoint{0.842683in}{1.299864in}}%
\pgfpathlineto{\pgfqpoint{0.843097in}{1.190393in}}%
\pgfpathlineto{\pgfqpoint{0.843305in}{1.137869in}}%
\pgfpathlineto{\pgfqpoint{0.843408in}{1.193616in}}%
\pgfpathlineto{\pgfqpoint{0.843512in}{1.306836in}}%
\pgfpathlineto{\pgfqpoint{0.843927in}{1.119952in}}%
\pgfpathlineto{\pgfqpoint{0.844445in}{1.216823in}}%
\pgfpathlineto{\pgfqpoint{0.844548in}{1.128384in}}%
\pgfpathlineto{\pgfqpoint{0.845378in}{1.330923in}}%
\pgfpathlineto{\pgfqpoint{0.845481in}{1.243765in}}%
\pgfpathlineto{\pgfqpoint{0.846414in}{1.112433in}}%
\pgfpathlineto{\pgfqpoint{0.846207in}{1.336755in}}%
\pgfpathlineto{\pgfqpoint{0.846621in}{1.172993in}}%
\pgfpathlineto{\pgfqpoint{0.846829in}{1.089139in}}%
\pgfpathlineto{\pgfqpoint{0.847865in}{1.342354in}}%
\pgfpathlineto{\pgfqpoint{0.849109in}{1.107504in}}%
\pgfpathlineto{\pgfqpoint{0.849523in}{1.011156in}}%
\pgfpathlineto{\pgfqpoint{0.850249in}{1.278596in}}%
\pgfpathlineto{\pgfqpoint{0.851285in}{1.047801in}}%
\pgfpathlineto{\pgfqpoint{0.851492in}{1.057898in}}%
\pgfpathlineto{\pgfqpoint{0.851803in}{1.036380in}}%
\pgfpathlineto{\pgfqpoint{0.852840in}{1.297332in}}%
\pgfpathlineto{\pgfqpoint{0.852943in}{1.296336in}}%
\pgfpathlineto{\pgfqpoint{0.853254in}{1.134244in}}%
\pgfpathlineto{\pgfqpoint{0.854187in}{1.192456in}}%
\pgfpathlineto{\pgfqpoint{0.854394in}{1.470915in}}%
\pgfpathlineto{\pgfqpoint{0.855431in}{1.372372in}}%
\pgfpathlineto{\pgfqpoint{0.856260in}{1.422505in}}%
\pgfpathlineto{\pgfqpoint{0.855845in}{1.217055in}}%
\pgfpathlineto{\pgfqpoint{0.856467in}{1.380983in}}%
\pgfpathlineto{\pgfqpoint{0.857193in}{1.194674in}}%
\pgfpathlineto{\pgfqpoint{0.857607in}{1.327591in}}%
\pgfpathlineto{\pgfqpoint{0.858436in}{1.133366in}}%
\pgfpathlineto{\pgfqpoint{0.858229in}{1.332637in}}%
\pgfpathlineto{\pgfqpoint{0.858954in}{1.221875in}}%
\pgfpathlineto{\pgfqpoint{0.859265in}{1.333899in}}%
\pgfpathlineto{\pgfqpoint{0.859887in}{1.151980in}}%
\pgfpathlineto{\pgfqpoint{0.859991in}{1.238276in}}%
\pgfpathlineto{\pgfqpoint{0.860302in}{1.062870in}}%
\pgfpathlineto{\pgfqpoint{0.860613in}{1.279131in}}%
\pgfpathlineto{\pgfqpoint{0.861131in}{1.179491in}}%
\pgfpathlineto{\pgfqpoint{0.861649in}{1.337885in}}%
\pgfpathlineto{\pgfqpoint{0.862271in}{1.108571in}}%
\pgfpathlineto{\pgfqpoint{0.863100in}{1.249270in}}%
\pgfpathlineto{\pgfqpoint{0.863307in}{1.077886in}}%
\pgfpathlineto{\pgfqpoint{0.863929in}{0.980896in}}%
\pgfpathlineto{\pgfqpoint{0.863618in}{1.078665in}}%
\pgfpathlineto{\pgfqpoint{0.864447in}{1.044174in}}%
\pgfpathlineto{\pgfqpoint{0.864551in}{1.051697in}}%
\pgfpathlineto{\pgfqpoint{0.865380in}{1.270453in}}%
\pgfpathlineto{\pgfqpoint{0.865691in}{1.200808in}}%
\pgfpathlineto{\pgfqpoint{0.866209in}{1.114449in}}%
\pgfpathlineto{\pgfqpoint{0.866106in}{1.225647in}}%
\pgfpathlineto{\pgfqpoint{0.866313in}{1.225217in}}%
\pgfpathlineto{\pgfqpoint{0.866416in}{1.277058in}}%
\pgfpathlineto{\pgfqpoint{0.866727in}{1.098230in}}%
\pgfpathlineto{\pgfqpoint{0.867349in}{1.243844in}}%
\pgfpathlineto{\pgfqpoint{0.868282in}{1.073541in}}%
\pgfpathlineto{\pgfqpoint{0.868386in}{1.250374in}}%
\pgfpathlineto{\pgfqpoint{0.868489in}{1.299364in}}%
\pgfpathlineto{\pgfqpoint{0.869215in}{1.142016in}}%
\pgfpathlineto{\pgfqpoint{0.869526in}{1.071462in}}%
\pgfpathlineto{\pgfqpoint{0.869837in}{1.155340in}}%
\pgfpathlineto{\pgfqpoint{0.870769in}{1.142154in}}%
\pgfpathlineto{\pgfqpoint{0.870977in}{1.272206in}}%
\pgfpathlineto{\pgfqpoint{0.871806in}{1.166436in}}%
\pgfpathlineto{\pgfqpoint{0.871599in}{1.407235in}}%
\pgfpathlineto{\pgfqpoint{0.872117in}{1.230420in}}%
\pgfpathlineto{\pgfqpoint{0.872739in}{1.416057in}}%
\pgfpathlineto{\pgfqpoint{0.872324in}{1.186161in}}%
\pgfpathlineto{\pgfqpoint{0.873153in}{1.256460in}}%
\pgfpathlineto{\pgfqpoint{0.873360in}{1.257162in}}%
\pgfpathlineto{\pgfqpoint{0.873464in}{1.121537in}}%
\pgfpathlineto{\pgfqpoint{0.873879in}{1.272667in}}%
\pgfpathlineto{\pgfqpoint{0.874397in}{1.250694in}}%
\pgfpathlineto{\pgfqpoint{0.874500in}{1.249892in}}%
\pgfpathlineto{\pgfqpoint{0.874708in}{1.313847in}}%
\pgfpathlineto{\pgfqpoint{0.875019in}{1.148428in}}%
\pgfpathlineto{\pgfqpoint{0.875122in}{1.130828in}}%
\pgfpathlineto{\pgfqpoint{0.875226in}{1.371541in}}%
\pgfpathlineto{\pgfqpoint{0.876262in}{1.247114in}}%
\pgfpathlineto{\pgfqpoint{0.876366in}{1.087606in}}%
\pgfpathlineto{\pgfqpoint{0.876781in}{1.374907in}}%
\pgfpathlineto{\pgfqpoint{0.877299in}{1.306267in}}%
\pgfpathlineto{\pgfqpoint{0.877506in}{1.433267in}}%
\pgfpathlineto{\pgfqpoint{0.877817in}{1.234817in}}%
\pgfpathlineto{\pgfqpoint{0.877921in}{1.296044in}}%
\pgfpathlineto{\pgfqpoint{0.878853in}{1.129552in}}%
\pgfpathlineto{\pgfqpoint{0.878439in}{1.326121in}}%
\pgfpathlineto{\pgfqpoint{0.879061in}{1.163542in}}%
\pgfpathlineto{\pgfqpoint{0.879682in}{1.386181in}}%
\pgfpathlineto{\pgfqpoint{0.879890in}{1.104197in}}%
\pgfpathlineto{\pgfqpoint{0.880097in}{1.279445in}}%
\pgfpathlineto{\pgfqpoint{0.880408in}{1.060990in}}%
\pgfpathlineto{\pgfqpoint{0.881444in}{1.084034in}}%
\pgfpathlineto{\pgfqpoint{0.882792in}{1.377654in}}%
\pgfpathlineto{\pgfqpoint{0.883724in}{1.187211in}}%
\pgfpathlineto{\pgfqpoint{0.883310in}{1.390941in}}%
\pgfpathlineto{\pgfqpoint{0.884139in}{1.240680in}}%
\pgfpathlineto{\pgfqpoint{0.884346in}{1.301176in}}%
\pgfpathlineto{\pgfqpoint{0.884968in}{1.089505in}}%
\pgfpathlineto{\pgfqpoint{0.885486in}{1.121500in}}%
\pgfpathlineto{\pgfqpoint{0.885694in}{0.978563in}}%
\pgfpathlineto{\pgfqpoint{0.886004in}{1.219183in}}%
\pgfpathlineto{\pgfqpoint{0.886315in}{1.147077in}}%
\pgfpathlineto{\pgfqpoint{0.887352in}{1.349280in}}%
\pgfpathlineto{\pgfqpoint{0.887766in}{1.339789in}}%
\pgfpathlineto{\pgfqpoint{0.887974in}{1.238289in}}%
\pgfpathlineto{\pgfqpoint{0.888595in}{1.400103in}}%
\pgfpathlineto{\pgfqpoint{0.888699in}{1.356971in}}%
\pgfpathlineto{\pgfqpoint{0.889114in}{1.431771in}}%
\pgfpathlineto{\pgfqpoint{0.889425in}{1.249388in}}%
\pgfpathlineto{\pgfqpoint{0.889839in}{1.120320in}}%
\pgfpathlineto{\pgfqpoint{0.890254in}{1.304091in}}%
\pgfpathlineto{\pgfqpoint{0.890357in}{1.294446in}}%
\pgfpathlineto{\pgfqpoint{0.890461in}{1.338741in}}%
\pgfpathlineto{\pgfqpoint{0.891186in}{1.206041in}}%
\pgfpathlineto{\pgfqpoint{0.891705in}{1.138251in}}%
\pgfpathlineto{\pgfqpoint{0.892119in}{1.255459in}}%
\pgfpathlineto{\pgfqpoint{0.892223in}{1.186796in}}%
\pgfpathlineto{\pgfqpoint{0.892430in}{1.248890in}}%
\pgfpathlineto{\pgfqpoint{0.892741in}{1.048205in}}%
\pgfpathlineto{\pgfqpoint{0.893259in}{1.184931in}}%
\pgfpathlineto{\pgfqpoint{0.893985in}{1.110140in}}%
\pgfpathlineto{\pgfqpoint{0.894192in}{1.236239in}}%
\pgfpathlineto{\pgfqpoint{0.894399in}{1.156601in}}%
\pgfpathlineto{\pgfqpoint{0.895436in}{1.342359in}}%
\pgfpathlineto{\pgfqpoint{0.895539in}{1.432406in}}%
\pgfpathlineto{\pgfqpoint{0.896368in}{1.300833in}}%
\pgfpathlineto{\pgfqpoint{0.896576in}{1.393687in}}%
\pgfpathlineto{\pgfqpoint{0.896887in}{1.165521in}}%
\pgfpathlineto{\pgfqpoint{0.897405in}{1.410658in}}%
\pgfpathlineto{\pgfqpoint{0.897612in}{1.233042in}}%
\pgfpathlineto{\pgfqpoint{0.898856in}{1.496813in}}%
\pgfpathlineto{\pgfqpoint{0.899892in}{1.240674in}}%
\pgfpathlineto{\pgfqpoint{0.900100in}{1.333411in}}%
\pgfpathlineto{\pgfqpoint{0.900203in}{1.251913in}}%
\pgfpathlineto{\pgfqpoint{0.900618in}{1.477453in}}%
\pgfpathlineto{\pgfqpoint{0.901240in}{1.286252in}}%
\pgfpathlineto{\pgfqpoint{0.901343in}{1.231634in}}%
\pgfpathlineto{\pgfqpoint{0.902069in}{1.368034in}}%
\pgfpathlineto{\pgfqpoint{0.902172in}{1.335511in}}%
\pgfpathlineto{\pgfqpoint{0.902276in}{1.359493in}}%
\pgfpathlineto{\pgfqpoint{0.902691in}{1.249659in}}%
\pgfpathlineto{\pgfqpoint{0.902794in}{1.264444in}}%
\pgfpathlineto{\pgfqpoint{0.903623in}{1.315794in}}%
\pgfpathlineto{\pgfqpoint{0.903934in}{1.145205in}}%
\pgfpathlineto{\pgfqpoint{0.904660in}{1.209812in}}%
\pgfpathlineto{\pgfqpoint{0.904349in}{1.085739in}}%
\pgfpathlineto{\pgfqpoint{0.904867in}{1.151246in}}%
\pgfpathlineto{\pgfqpoint{0.905385in}{1.051201in}}%
\pgfpathlineto{\pgfqpoint{0.905282in}{1.180535in}}%
\pgfpathlineto{\pgfqpoint{0.905696in}{1.159665in}}%
\pgfpathlineto{\pgfqpoint{0.905800in}{1.345307in}}%
\pgfpathlineto{\pgfqpoint{0.906214in}{1.080711in}}%
\pgfpathlineto{\pgfqpoint{0.906732in}{1.154986in}}%
\pgfpathlineto{\pgfqpoint{0.906836in}{1.121539in}}%
\pgfpathlineto{\pgfqpoint{0.907147in}{1.323286in}}%
\pgfpathlineto{\pgfqpoint{0.907354in}{1.305011in}}%
\pgfpathlineto{\pgfqpoint{0.907458in}{1.299199in}}%
\pgfpathlineto{\pgfqpoint{0.907562in}{1.301486in}}%
\pgfpathlineto{\pgfqpoint{0.908080in}{1.134839in}}%
\pgfpathlineto{\pgfqpoint{0.908391in}{1.341281in}}%
\pgfpathlineto{\pgfqpoint{0.908598in}{1.334747in}}%
\pgfpathlineto{\pgfqpoint{0.908909in}{1.141678in}}%
\pgfpathlineto{\pgfqpoint{0.909531in}{1.396737in}}%
\pgfpathlineto{\pgfqpoint{0.909634in}{1.478674in}}%
\pgfpathlineto{\pgfqpoint{0.910360in}{1.209225in}}%
\pgfpathlineto{\pgfqpoint{0.910878in}{1.154277in}}%
\pgfpathlineto{\pgfqpoint{0.911293in}{1.285996in}}%
\pgfpathlineto{\pgfqpoint{0.911085in}{1.147068in}}%
\pgfpathlineto{\pgfqpoint{0.912018in}{1.254711in}}%
\pgfpathlineto{\pgfqpoint{0.913158in}{1.092273in}}%
\pgfpathlineto{\pgfqpoint{0.912847in}{1.290417in}}%
\pgfpathlineto{\pgfqpoint{0.913262in}{1.120464in}}%
\pgfpathlineto{\pgfqpoint{0.913365in}{1.154750in}}%
\pgfpathlineto{\pgfqpoint{0.913676in}{1.079988in}}%
\pgfpathlineto{\pgfqpoint{0.913884in}{1.130969in}}%
\pgfpathlineto{\pgfqpoint{0.914402in}{0.887371in}}%
\pgfpathlineto{\pgfqpoint{0.914920in}{1.098045in}}%
\pgfpathlineto{\pgfqpoint{0.915024in}{1.102740in}}%
\pgfpathlineto{\pgfqpoint{0.915127in}{0.954582in}}%
\pgfpathlineto{\pgfqpoint{0.915542in}{1.383787in}}%
\pgfpathlineto{\pgfqpoint{0.916060in}{1.184805in}}%
\pgfpathlineto{\pgfqpoint{0.916786in}{1.058188in}}%
\pgfpathlineto{\pgfqpoint{0.916993in}{1.185904in}}%
\pgfpathlineto{\pgfqpoint{0.917304in}{1.097496in}}%
\pgfpathlineto{\pgfqpoint{0.918029in}{1.306605in}}%
\pgfpathlineto{\pgfqpoint{0.918547in}{1.221755in}}%
\pgfpathlineto{\pgfqpoint{0.919687in}{1.387617in}}%
\pgfpathlineto{\pgfqpoint{0.919377in}{1.205962in}}%
\pgfpathlineto{\pgfqpoint{0.919895in}{1.266975in}}%
\pgfpathlineto{\pgfqpoint{0.920931in}{1.006840in}}%
\pgfpathlineto{\pgfqpoint{0.920206in}{1.273600in}}%
\pgfpathlineto{\pgfqpoint{0.921138in}{1.020191in}}%
\pgfpathlineto{\pgfqpoint{0.922071in}{1.332589in}}%
\pgfpathlineto{\pgfqpoint{0.922278in}{1.177209in}}%
\pgfpathlineto{\pgfqpoint{0.922693in}{1.236485in}}%
\pgfpathlineto{\pgfqpoint{0.923522in}{1.021447in}}%
\pgfpathlineto{\pgfqpoint{0.924559in}{1.304790in}}%
\pgfpathlineto{\pgfqpoint{0.924662in}{1.285793in}}%
\pgfpathlineto{\pgfqpoint{0.925180in}{1.107395in}}%
\pgfpathlineto{\pgfqpoint{0.925388in}{1.347692in}}%
\pgfpathlineto{\pgfqpoint{0.925802in}{1.186532in}}%
\pgfpathlineto{\pgfqpoint{0.926735in}{1.257883in}}%
\pgfpathlineto{\pgfqpoint{0.926424in}{1.070285in}}%
\pgfpathlineto{\pgfqpoint{0.926839in}{1.243996in}}%
\pgfpathlineto{\pgfqpoint{0.927564in}{1.259399in}}%
\pgfpathlineto{\pgfqpoint{0.928186in}{0.918341in}}%
\pgfpathlineto{\pgfqpoint{0.928808in}{1.188177in}}%
\pgfpathlineto{\pgfqpoint{0.929430in}{1.112763in}}%
\pgfpathlineto{\pgfqpoint{0.929533in}{0.946480in}}%
\pgfpathlineto{\pgfqpoint{0.930466in}{1.047928in}}%
\pgfpathlineto{\pgfqpoint{0.931088in}{0.984202in}}%
\pgfpathlineto{\pgfqpoint{0.931502in}{1.196495in}}%
\pgfpathlineto{\pgfqpoint{0.932435in}{1.015075in}}%
\pgfpathlineto{\pgfqpoint{0.932642in}{1.080727in}}%
\pgfpathlineto{\pgfqpoint{0.933057in}{0.959856in}}%
\pgfpathlineto{\pgfqpoint{0.933264in}{1.171181in}}%
\pgfpathlineto{\pgfqpoint{0.933472in}{1.126028in}}%
\pgfpathlineto{\pgfqpoint{0.934715in}{1.284188in}}%
\pgfpathlineto{\pgfqpoint{0.933679in}{1.049387in}}%
\pgfpathlineto{\pgfqpoint{0.934819in}{1.247370in}}%
\pgfpathlineto{\pgfqpoint{0.935752in}{1.006590in}}%
\pgfpathlineto{\pgfqpoint{0.935959in}{1.111843in}}%
\pgfpathlineto{\pgfqpoint{0.937099in}{1.317894in}}%
\pgfpathlineto{\pgfqpoint{0.936166in}{1.077224in}}%
\pgfpathlineto{\pgfqpoint{0.937306in}{1.243373in}}%
\pgfpathlineto{\pgfqpoint{0.938135in}{1.036082in}}%
\pgfpathlineto{\pgfqpoint{0.938550in}{1.073282in}}%
\pgfpathlineto{\pgfqpoint{0.938757in}{1.262709in}}%
\pgfpathlineto{\pgfqpoint{0.939483in}{1.044853in}}%
\pgfpathlineto{\pgfqpoint{0.939690in}{1.169636in}}%
\pgfpathlineto{\pgfqpoint{0.940105in}{1.027465in}}%
\pgfpathlineto{\pgfqpoint{0.940519in}{1.232130in}}%
\pgfpathlineto{\pgfqpoint{0.940830in}{1.151472in}}%
\pgfpathlineto{\pgfqpoint{0.940934in}{1.216133in}}%
\pgfpathlineto{\pgfqpoint{0.941037in}{1.041402in}}%
\pgfpathlineto{\pgfqpoint{0.941763in}{1.096098in}}%
\pgfpathlineto{\pgfqpoint{0.942592in}{0.893348in}}%
\pgfpathlineto{\pgfqpoint{0.941970in}{1.142698in}}%
\pgfpathlineto{\pgfqpoint{0.943006in}{1.023786in}}%
\pgfpathlineto{\pgfqpoint{0.943110in}{1.076697in}}%
\pgfpathlineto{\pgfqpoint{0.943525in}{0.887218in}}%
\pgfpathlineto{\pgfqpoint{0.943939in}{0.978405in}}%
\pgfpathlineto{\pgfqpoint{0.944665in}{0.875535in}}%
\pgfpathlineto{\pgfqpoint{0.944354in}{1.030358in}}%
\pgfpathlineto{\pgfqpoint{0.944872in}{1.013260in}}%
\pgfpathlineto{\pgfqpoint{0.946323in}{1.280077in}}%
\pgfpathlineto{\pgfqpoint{0.945494in}{0.964683in}}%
\pgfpathlineto{\pgfqpoint{0.946427in}{1.149572in}}%
\pgfpathlineto{\pgfqpoint{0.946841in}{0.941993in}}%
\pgfpathlineto{\pgfqpoint{0.947463in}{1.166731in}}%
\pgfpathlineto{\pgfqpoint{0.947567in}{1.055770in}}%
\pgfpathlineto{\pgfqpoint{0.947774in}{1.004276in}}%
\pgfpathlineto{\pgfqpoint{0.947981in}{1.139151in}}%
\pgfpathlineto{\pgfqpoint{0.948085in}{1.138049in}}%
\pgfpathlineto{\pgfqpoint{0.948707in}{1.322428in}}%
\pgfpathlineto{\pgfqpoint{0.948292in}{1.068125in}}%
\pgfpathlineto{\pgfqpoint{0.949121in}{1.168784in}}%
\pgfpathlineto{\pgfqpoint{0.949225in}{1.077340in}}%
\pgfpathlineto{\pgfqpoint{0.949432in}{1.236350in}}%
\pgfpathlineto{\pgfqpoint{0.950158in}{1.187332in}}%
\pgfpathlineto{\pgfqpoint{0.950469in}{1.013912in}}%
\pgfpathlineto{\pgfqpoint{0.950572in}{1.198587in}}%
\pgfpathlineto{\pgfqpoint{0.951298in}{1.149853in}}%
\pgfpathlineto{\pgfqpoint{0.951401in}{1.340405in}}%
\pgfpathlineto{\pgfqpoint{0.952230in}{1.048532in}}%
\pgfpathlineto{\pgfqpoint{0.952334in}{1.150797in}}%
\pgfpathlineto{\pgfqpoint{0.953267in}{1.286898in}}%
\pgfpathlineto{\pgfqpoint{0.953060in}{1.113670in}}%
\pgfpathlineto{\pgfqpoint{0.953474in}{1.208568in}}%
\pgfpathlineto{\pgfqpoint{0.953578in}{1.138011in}}%
\pgfpathlineto{\pgfqpoint{0.953889in}{1.355646in}}%
\pgfpathlineto{\pgfqpoint{0.954511in}{1.153986in}}%
\pgfpathlineto{\pgfqpoint{0.955029in}{1.320674in}}%
\pgfpathlineto{\pgfqpoint{0.955547in}{1.221100in}}%
\pgfpathlineto{\pgfqpoint{0.956480in}{1.251882in}}%
\pgfpathlineto{\pgfqpoint{0.956583in}{1.037284in}}%
\pgfpathlineto{\pgfqpoint{0.957102in}{1.225965in}}%
\pgfpathlineto{\pgfqpoint{0.957723in}{1.148093in}}%
\pgfpathlineto{\pgfqpoint{0.957827in}{1.076822in}}%
\pgfpathlineto{\pgfqpoint{0.958345in}{1.326586in}}%
\pgfpathlineto{\pgfqpoint{0.958656in}{1.217004in}}%
\pgfpathlineto{\pgfqpoint{0.959485in}{1.377535in}}%
\pgfpathlineto{\pgfqpoint{0.959278in}{1.141956in}}%
\pgfpathlineto{\pgfqpoint{0.959693in}{1.210996in}}%
\pgfpathlineto{\pgfqpoint{0.959796in}{1.121160in}}%
\pgfpathlineto{\pgfqpoint{0.960625in}{1.269945in}}%
\pgfpathlineto{\pgfqpoint{0.960729in}{1.128787in}}%
\pgfpathlineto{\pgfqpoint{0.961040in}{1.341162in}}%
\pgfpathlineto{\pgfqpoint{0.961247in}{1.124612in}}%
\pgfpathlineto{\pgfqpoint{0.961869in}{1.176580in}}%
\pgfpathlineto{\pgfqpoint{0.962387in}{1.353759in}}%
\pgfpathlineto{\pgfqpoint{0.963009in}{1.259328in}}%
\pgfpathlineto{\pgfqpoint{0.964045in}{0.954990in}}%
\pgfpathlineto{\pgfqpoint{0.964356in}{1.087723in}}%
\pgfpathlineto{\pgfqpoint{0.965082in}{1.283766in}}%
\pgfpathlineto{\pgfqpoint{0.964875in}{1.054416in}}%
\pgfpathlineto{\pgfqpoint{0.965393in}{1.092690in}}%
\pgfpathlineto{\pgfqpoint{0.965496in}{1.047306in}}%
\pgfpathlineto{\pgfqpoint{0.965911in}{1.263874in}}%
\pgfpathlineto{\pgfqpoint{0.966015in}{1.224710in}}%
\pgfpathlineto{\pgfqpoint{0.966222in}{1.304083in}}%
\pgfpathlineto{\pgfqpoint{0.966844in}{1.103992in}}%
\pgfpathlineto{\pgfqpoint{0.966947in}{1.097213in}}%
\pgfpathlineto{\pgfqpoint{0.967155in}{1.137856in}}%
\pgfpathlineto{\pgfqpoint{0.967673in}{1.385204in}}%
\pgfpathlineto{\pgfqpoint{0.967362in}{1.123465in}}%
\pgfpathlineto{\pgfqpoint{0.968295in}{1.277742in}}%
\pgfpathlineto{\pgfqpoint{0.968398in}{1.082488in}}%
\pgfpathlineto{\pgfqpoint{0.969124in}{1.348393in}}%
\pgfpathlineto{\pgfqpoint{0.969435in}{1.202590in}}%
\pgfpathlineto{\pgfqpoint{0.969538in}{1.267494in}}%
\pgfpathlineto{\pgfqpoint{0.969953in}{1.136402in}}%
\pgfpathlineto{\pgfqpoint{0.970264in}{1.219304in}}%
\pgfpathlineto{\pgfqpoint{0.970367in}{1.082099in}}%
\pgfpathlineto{\pgfqpoint{0.971093in}{1.364555in}}%
\pgfpathlineto{\pgfqpoint{0.971300in}{1.297011in}}%
\pgfpathlineto{\pgfqpoint{0.971404in}{1.440822in}}%
\pgfpathlineto{\pgfqpoint{0.971818in}{1.201993in}}%
\pgfpathlineto{\pgfqpoint{0.972440in}{1.350144in}}%
\pgfpathlineto{\pgfqpoint{0.973477in}{0.961442in}}%
\pgfpathlineto{\pgfqpoint{0.973580in}{1.213648in}}%
\pgfpathlineto{\pgfqpoint{0.974202in}{1.351366in}}%
\pgfpathlineto{\pgfqpoint{0.973995in}{1.110914in}}%
\pgfpathlineto{\pgfqpoint{0.974513in}{1.298348in}}%
\pgfpathlineto{\pgfqpoint{0.975653in}{1.063995in}}%
\pgfpathlineto{\pgfqpoint{0.976068in}{1.253993in}}%
\pgfpathlineto{\pgfqpoint{0.976275in}{1.008153in}}%
\pgfpathlineto{\pgfqpoint{0.976793in}{1.195130in}}%
\pgfpathlineto{\pgfqpoint{0.976897in}{1.201733in}}%
\pgfpathlineto{\pgfqpoint{0.977104in}{1.026542in}}%
\pgfpathlineto{\pgfqpoint{0.977726in}{1.266578in}}%
\pgfpathlineto{\pgfqpoint{0.977933in}{1.210124in}}%
\pgfpathlineto{\pgfqpoint{0.978762in}{1.070148in}}%
\pgfpathlineto{\pgfqpoint{0.978244in}{1.251657in}}%
\pgfpathlineto{\pgfqpoint{0.979177in}{1.130997in}}%
\pgfpathlineto{\pgfqpoint{0.979488in}{1.273211in}}%
\pgfpathlineto{\pgfqpoint{0.980213in}{1.063760in}}%
\pgfpathlineto{\pgfqpoint{0.980317in}{1.163273in}}%
\pgfpathlineto{\pgfqpoint{0.980421in}{0.987761in}}%
\pgfpathlineto{\pgfqpoint{0.981042in}{1.233876in}}%
\pgfpathlineto{\pgfqpoint{0.981353in}{1.176816in}}%
\pgfpathlineto{\pgfqpoint{0.981871in}{1.319899in}}%
\pgfpathlineto{\pgfqpoint{0.981664in}{1.094966in}}%
\pgfpathlineto{\pgfqpoint{0.982597in}{1.310361in}}%
\pgfpathlineto{\pgfqpoint{0.983530in}{1.114040in}}%
\pgfpathlineto{\pgfqpoint{0.983115in}{1.369839in}}%
\pgfpathlineto{\pgfqpoint{0.983633in}{1.193079in}}%
\pgfpathlineto{\pgfqpoint{0.983737in}{1.366788in}}%
\pgfpathlineto{\pgfqpoint{0.984359in}{1.129265in}}%
\pgfpathlineto{\pgfqpoint{0.984566in}{1.137910in}}%
\pgfpathlineto{\pgfqpoint{0.984670in}{1.015439in}}%
\pgfpathlineto{\pgfqpoint{0.985395in}{1.291372in}}%
\pgfpathlineto{\pgfqpoint{0.985499in}{1.232250in}}%
\pgfpathlineto{\pgfqpoint{0.986432in}{1.084710in}}%
\pgfpathlineto{\pgfqpoint{0.986743in}{1.387764in}}%
\pgfpathlineto{\pgfqpoint{0.987261in}{1.112010in}}%
\pgfpathlineto{\pgfqpoint{0.987779in}{1.399369in}}%
\pgfpathlineto{\pgfqpoint{0.987883in}{1.209085in}}%
\pgfpathlineto{\pgfqpoint{0.988090in}{1.356860in}}%
\pgfpathlineto{\pgfqpoint{0.988712in}{1.185057in}}%
\pgfpathlineto{\pgfqpoint{0.989023in}{1.328099in}}%
\pgfpathlineto{\pgfqpoint{0.989437in}{1.175661in}}%
\pgfpathlineto{\pgfqpoint{0.990266in}{1.201366in}}%
\pgfpathlineto{\pgfqpoint{0.990370in}{1.205471in}}%
\pgfpathlineto{\pgfqpoint{0.990577in}{1.323337in}}%
\pgfpathlineto{\pgfqpoint{0.990888in}{1.103817in}}%
\pgfpathlineto{\pgfqpoint{0.991406in}{1.240476in}}%
\pgfpathlineto{\pgfqpoint{0.991510in}{1.115827in}}%
\pgfpathlineto{\pgfqpoint{0.992546in}{1.205185in}}%
\pgfpathlineto{\pgfqpoint{0.992650in}{1.237998in}}%
\pgfpathlineto{\pgfqpoint{0.993168in}{1.116601in}}%
\pgfpathlineto{\pgfqpoint{0.993272in}{1.072945in}}%
\pgfpathlineto{\pgfqpoint{0.993997in}{1.234262in}}%
\pgfpathlineto{\pgfqpoint{0.994101in}{1.156872in}}%
\pgfpathlineto{\pgfqpoint{0.994723in}{1.299748in}}%
\pgfpathlineto{\pgfqpoint{0.994516in}{1.054487in}}%
\pgfpathlineto{\pgfqpoint{0.995241in}{1.283949in}}%
\pgfpathlineto{\pgfqpoint{0.995345in}{1.153056in}}%
\pgfpathlineto{\pgfqpoint{0.995967in}{1.316315in}}%
\pgfpathlineto{\pgfqpoint{0.996381in}{1.245433in}}%
\pgfpathlineto{\pgfqpoint{0.996588in}{1.227122in}}%
\pgfpathlineto{\pgfqpoint{0.997521in}{1.397281in}}%
\pgfpathlineto{\pgfqpoint{0.996899in}{1.176815in}}%
\pgfpathlineto{\pgfqpoint{0.997728in}{1.387960in}}%
\pgfpathlineto{\pgfqpoint{0.998558in}{1.309242in}}%
\pgfpathlineto{\pgfqpoint{0.998350in}{1.432232in}}%
\pgfpathlineto{\pgfqpoint{0.998868in}{1.334758in}}%
\pgfpathlineto{\pgfqpoint{0.999076in}{1.458046in}}%
\pgfpathlineto{\pgfqpoint{0.999594in}{1.275678in}}%
\pgfpathlineto{\pgfqpoint{0.999905in}{1.341792in}}%
\pgfpathlineto{\pgfqpoint{1.000319in}{1.110240in}}%
\pgfpathlineto{\pgfqpoint{1.000734in}{1.511188in}}%
\pgfpathlineto{\pgfqpoint{1.000941in}{1.359568in}}%
\pgfpathlineto{\pgfqpoint{1.001149in}{1.488308in}}%
\pgfpathlineto{\pgfqpoint{1.001356in}{1.208982in}}%
\pgfpathlineto{\pgfqpoint{1.001770in}{1.378511in}}%
\pgfpathlineto{\pgfqpoint{1.002289in}{1.090667in}}%
\pgfpathlineto{\pgfqpoint{1.003014in}{1.114532in}}%
\pgfpathlineto{\pgfqpoint{1.003118in}{1.117897in}}%
\pgfpathlineto{\pgfqpoint{1.003740in}{1.031388in}}%
\pgfpathlineto{\pgfqpoint{1.004361in}{1.248550in}}%
\pgfpathlineto{\pgfqpoint{1.004672in}{1.029891in}}%
\pgfpathlineto{\pgfqpoint{1.004880in}{1.270069in}}%
\pgfpathlineto{\pgfqpoint{1.005398in}{1.151296in}}%
\pgfpathlineto{\pgfqpoint{1.006227in}{1.284848in}}%
\pgfpathlineto{\pgfqpoint{1.005812in}{1.145317in}}%
\pgfpathlineto{\pgfqpoint{1.006434in}{1.169918in}}%
\pgfpathlineto{\pgfqpoint{1.007056in}{0.937839in}}%
\pgfpathlineto{\pgfqpoint{1.007989in}{1.020808in}}%
\pgfpathlineto{\pgfqpoint{1.008300in}{1.169527in}}%
\pgfpathlineto{\pgfqpoint{1.009129in}{1.061511in}}%
\pgfpathlineto{\pgfqpoint{1.010062in}{0.912295in}}%
\pgfpathlineto{\pgfqpoint{1.010165in}{1.031876in}}%
\pgfpathlineto{\pgfqpoint{1.010269in}{1.087994in}}%
\pgfpathlineto{\pgfqpoint{1.010891in}{0.887642in}}%
\pgfpathlineto{\pgfqpoint{1.011098in}{0.969040in}}%
\pgfpathlineto{\pgfqpoint{1.011305in}{0.909687in}}%
\pgfpathlineto{\pgfqpoint{1.011616in}{0.991874in}}%
\pgfpathlineto{\pgfqpoint{1.012031in}{0.968634in}}%
\pgfpathlineto{\pgfqpoint{1.012653in}{1.127119in}}%
\pgfpathlineto{\pgfqpoint{1.013793in}{0.965358in}}%
\pgfpathlineto{\pgfqpoint{1.014104in}{1.177954in}}%
\pgfpathlineto{\pgfqpoint{1.014933in}{1.042284in}}%
\pgfpathlineto{\pgfqpoint{1.015036in}{1.050438in}}%
\pgfpathlineto{\pgfqpoint{1.015140in}{0.965809in}}%
\pgfpathlineto{\pgfqpoint{1.015244in}{1.059627in}}%
\pgfpathlineto{\pgfqpoint{1.016176in}{1.024294in}}%
\pgfpathlineto{\pgfqpoint{1.016487in}{1.151337in}}%
\pgfpathlineto{\pgfqpoint{1.016902in}{0.929764in}}%
\pgfpathlineto{\pgfqpoint{1.017316in}{1.092565in}}%
\pgfpathlineto{\pgfqpoint{1.018249in}{0.872521in}}%
\pgfpathlineto{\pgfqpoint{1.018456in}{0.978010in}}%
\pgfpathlineto{\pgfqpoint{1.019389in}{1.121962in}}%
\pgfpathlineto{\pgfqpoint{1.018975in}{0.899807in}}%
\pgfpathlineto{\pgfqpoint{1.019596in}{1.119071in}}%
\pgfpathlineto{\pgfqpoint{1.020322in}{1.203611in}}%
\pgfpathlineto{\pgfqpoint{1.020736in}{0.988479in}}%
\pgfpathlineto{\pgfqpoint{1.021773in}{1.216846in}}%
\pgfpathlineto{\pgfqpoint{1.021151in}{0.964484in}}%
\pgfpathlineto{\pgfqpoint{1.021980in}{1.117752in}}%
\pgfpathlineto{\pgfqpoint{1.022084in}{1.008478in}}%
\pgfpathlineto{\pgfqpoint{1.022913in}{1.209241in}}%
\pgfpathlineto{\pgfqpoint{1.023224in}{1.183678in}}%
\pgfpathlineto{\pgfqpoint{1.023431in}{1.256976in}}%
\pgfpathlineto{\pgfqpoint{1.024157in}{0.972229in}}%
\pgfpathlineto{\pgfqpoint{1.024571in}{1.150962in}}%
\pgfpathlineto{\pgfqpoint{1.024778in}{1.029738in}}%
\pgfpathlineto{\pgfqpoint{1.025089in}{1.224853in}}%
\pgfpathlineto{\pgfqpoint{1.025608in}{1.094191in}}%
\pgfpathlineto{\pgfqpoint{1.026644in}{1.041392in}}%
\pgfpathlineto{\pgfqpoint{1.026851in}{1.270833in}}%
\pgfpathlineto{\pgfqpoint{1.027059in}{1.107736in}}%
\pgfpathlineto{\pgfqpoint{1.027680in}{1.282827in}}%
\pgfpathlineto{\pgfqpoint{1.027888in}{1.156279in}}%
\pgfpathlineto{\pgfqpoint{1.028199in}{1.485452in}}%
\pgfpathlineto{\pgfqpoint{1.028509in}{1.129165in}}%
\pgfpathlineto{\pgfqpoint{1.028924in}{1.209211in}}%
\pgfpathlineto{\pgfqpoint{1.029650in}{0.952712in}}%
\pgfpathlineto{\pgfqpoint{1.030064in}{1.071809in}}%
\pgfpathlineto{\pgfqpoint{1.030375in}{1.142863in}}%
\pgfpathlineto{\pgfqpoint{1.030479in}{1.035545in}}%
\pgfpathlineto{\pgfqpoint{1.030582in}{1.121731in}}%
\pgfpathlineto{\pgfqpoint{1.030686in}{0.926874in}}%
\pgfpathlineto{\pgfqpoint{1.031722in}{0.993258in}}%
\pgfpathlineto{\pgfqpoint{1.032137in}{1.193736in}}%
\pgfpathlineto{\pgfqpoint{1.032344in}{0.934276in}}%
\pgfpathlineto{\pgfqpoint{1.032759in}{0.976249in}}%
\pgfpathlineto{\pgfqpoint{1.033173in}{1.161218in}}%
\pgfpathlineto{\pgfqpoint{1.033381in}{0.984338in}}%
\pgfpathlineto{\pgfqpoint{1.033484in}{0.835528in}}%
\pgfpathlineto{\pgfqpoint{1.034313in}{1.079699in}}%
\pgfpathlineto{\pgfqpoint{1.034417in}{0.946541in}}%
\pgfpathlineto{\pgfqpoint{1.034935in}{1.263960in}}%
\pgfpathlineto{\pgfqpoint{1.035661in}{1.052137in}}%
\pgfpathlineto{\pgfqpoint{1.036075in}{1.215526in}}%
\pgfpathlineto{\pgfqpoint{1.036697in}{1.072666in}}%
\pgfpathlineto{\pgfqpoint{1.037008in}{1.123659in}}%
\pgfpathlineto{\pgfqpoint{1.037112in}{1.033918in}}%
\pgfpathlineto{\pgfqpoint{1.037319in}{1.159516in}}%
\pgfpathlineto{\pgfqpoint{1.038044in}{0.924810in}}%
\pgfpathlineto{\pgfqpoint{1.038148in}{0.973716in}}%
\pgfpathlineto{\pgfqpoint{1.039806in}{1.200143in}}%
\pgfpathlineto{\pgfqpoint{1.038666in}{0.961454in}}%
\pgfpathlineto{\pgfqpoint{1.040014in}{1.146831in}}%
\pgfpathlineto{\pgfqpoint{1.040532in}{1.000406in}}%
\pgfpathlineto{\pgfqpoint{1.040739in}{1.161005in}}%
\pgfpathlineto{\pgfqpoint{1.041050in}{1.100582in}}%
\pgfpathlineto{\pgfqpoint{1.041672in}{1.241134in}}%
\pgfpathlineto{\pgfqpoint{1.041775in}{1.081701in}}%
\pgfpathlineto{\pgfqpoint{1.042086in}{1.138425in}}%
\pgfpathlineto{\pgfqpoint{1.042397in}{1.004876in}}%
\pgfpathlineto{\pgfqpoint{1.043123in}{1.009269in}}%
\pgfpathlineto{\pgfqpoint{1.043848in}{1.245543in}}%
\pgfpathlineto{\pgfqpoint{1.044263in}{1.168160in}}%
\pgfpathlineto{\pgfqpoint{1.044677in}{1.298012in}}%
\pgfpathlineto{\pgfqpoint{1.044988in}{1.116932in}}%
\pgfpathlineto{\pgfqpoint{1.045092in}{0.976157in}}%
\pgfpathlineto{\pgfqpoint{1.045714in}{1.193364in}}%
\pgfpathlineto{\pgfqpoint{1.046025in}{1.189741in}}%
\pgfpathlineto{\pgfqpoint{1.046957in}{1.006669in}}%
\pgfpathlineto{\pgfqpoint{1.047372in}{1.083128in}}%
\pgfpathlineto{\pgfqpoint{1.047994in}{0.990785in}}%
\pgfpathlineto{\pgfqpoint{1.048512in}{1.210644in}}%
\pgfpathlineto{\pgfqpoint{1.049134in}{1.033687in}}%
\pgfpathlineto{\pgfqpoint{1.049341in}{1.242533in}}%
\pgfpathlineto{\pgfqpoint{1.049652in}{1.100885in}}%
\pgfpathlineto{\pgfqpoint{1.049963in}{1.214771in}}%
\pgfpathlineto{\pgfqpoint{1.050067in}{1.000063in}}%
\pgfpathlineto{\pgfqpoint{1.050585in}{0.809075in}}%
\pgfpathlineto{\pgfqpoint{1.050378in}{1.088117in}}%
\pgfpathlineto{\pgfqpoint{1.051207in}{0.953797in}}%
\pgfpathlineto{\pgfqpoint{1.051518in}{0.796038in}}%
\pgfpathlineto{\pgfqpoint{1.051414in}{0.995000in}}%
\pgfpathlineto{\pgfqpoint{1.052243in}{0.893953in}}%
\pgfpathlineto{\pgfqpoint{1.053487in}{1.100352in}}%
\pgfpathlineto{\pgfqpoint{1.052865in}{0.775949in}}%
\pgfpathlineto{\pgfqpoint{1.053590in}{1.068153in}}%
\pgfpathlineto{\pgfqpoint{1.053798in}{1.043183in}}%
\pgfpathlineto{\pgfqpoint{1.054419in}{0.895922in}}%
\pgfpathlineto{\pgfqpoint{1.054730in}{1.101463in}}%
\pgfpathlineto{\pgfqpoint{1.054938in}{0.915796in}}%
\pgfpathlineto{\pgfqpoint{1.055663in}{1.146391in}}%
\pgfpathlineto{\pgfqpoint{1.056181in}{1.108450in}}%
\pgfpathlineto{\pgfqpoint{1.056285in}{1.017867in}}%
\pgfpathlineto{\pgfqpoint{1.056700in}{1.157510in}}%
\pgfpathlineto{\pgfqpoint{1.057218in}{1.134640in}}%
\pgfpathlineto{\pgfqpoint{1.058461in}{0.835198in}}%
\pgfpathlineto{\pgfqpoint{1.058772in}{0.876412in}}%
\pgfpathlineto{\pgfqpoint{1.059912in}{1.131906in}}%
\pgfpathlineto{\pgfqpoint{1.060223in}{1.067222in}}%
\pgfpathlineto{\pgfqpoint{1.060534in}{1.106035in}}%
\pgfpathlineto{\pgfqpoint{1.061363in}{0.862919in}}%
\pgfpathlineto{\pgfqpoint{1.061467in}{0.964838in}}%
\pgfpathlineto{\pgfqpoint{1.061674in}{0.812589in}}%
\pgfpathlineto{\pgfqpoint{1.062503in}{0.950356in}}%
\pgfpathlineto{\pgfqpoint{1.062814in}{0.899963in}}%
\pgfpathlineto{\pgfqpoint{1.062918in}{1.013579in}}%
\pgfpathlineto{\pgfqpoint{1.063333in}{0.969255in}}%
\pgfpathlineto{\pgfqpoint{1.064058in}{1.130671in}}%
\pgfpathlineto{\pgfqpoint{1.063643in}{0.923298in}}%
\pgfpathlineto{\pgfqpoint{1.064473in}{1.060651in}}%
\pgfpathlineto{\pgfqpoint{1.064680in}{1.075044in}}%
\pgfpathlineto{\pgfqpoint{1.064783in}{1.010560in}}%
\pgfpathlineto{\pgfqpoint{1.065094in}{0.862147in}}%
\pgfpathlineto{\pgfqpoint{1.065613in}{1.137044in}}%
\pgfpathlineto{\pgfqpoint{1.065820in}{1.050171in}}%
\pgfpathlineto{\pgfqpoint{1.065924in}{1.043514in}}%
\pgfpathlineto{\pgfqpoint{1.066649in}{0.821784in}}%
\pgfpathlineto{\pgfqpoint{1.067064in}{0.948918in}}%
\pgfpathlineto{\pgfqpoint{1.067893in}{1.153094in}}%
\pgfpathlineto{\pgfqpoint{1.068307in}{1.032474in}}%
\pgfpathlineto{\pgfqpoint{1.069033in}{0.858922in}}%
\pgfpathlineto{\pgfqpoint{1.069136in}{1.075921in}}%
\pgfpathlineto{\pgfqpoint{1.069344in}{0.973165in}}%
\pgfpathlineto{\pgfqpoint{1.069551in}{1.017246in}}%
\pgfpathlineto{\pgfqpoint{1.070069in}{0.928404in}}%
\pgfpathlineto{\pgfqpoint{1.070173in}{0.860909in}}%
\pgfpathlineto{\pgfqpoint{1.070898in}{1.058830in}}%
\pgfpathlineto{\pgfqpoint{1.071209in}{0.891101in}}%
\pgfpathlineto{\pgfqpoint{1.072453in}{1.106810in}}%
\pgfpathlineto{\pgfqpoint{1.073489in}{0.788994in}}%
\pgfpathlineto{\pgfqpoint{1.073697in}{0.878096in}}%
\pgfpathlineto{\pgfqpoint{1.073904in}{1.041696in}}%
\pgfpathlineto{\pgfqpoint{1.074629in}{0.818428in}}%
\pgfpathlineto{\pgfqpoint{1.074733in}{0.734994in}}%
\pgfpathlineto{\pgfqpoint{1.075251in}{0.987179in}}%
\pgfpathlineto{\pgfqpoint{1.075458in}{0.914538in}}%
\pgfpathlineto{\pgfqpoint{1.076288in}{0.831234in}}%
\pgfpathlineto{\pgfqpoint{1.076598in}{1.025764in}}%
\pgfpathlineto{\pgfqpoint{1.077531in}{0.824584in}}%
\pgfpathlineto{\pgfqpoint{1.077117in}{1.071626in}}%
\pgfpathlineto{\pgfqpoint{1.077635in}{0.937088in}}%
\pgfpathlineto{\pgfqpoint{1.078982in}{1.175182in}}%
\pgfpathlineto{\pgfqpoint{1.080330in}{0.946631in}}%
\pgfpathlineto{\pgfqpoint{1.080537in}{0.988400in}}%
\pgfpathlineto{\pgfqpoint{1.081366in}{1.151137in}}%
\pgfpathlineto{\pgfqpoint{1.081159in}{0.974195in}}%
\pgfpathlineto{\pgfqpoint{1.081573in}{1.122308in}}%
\pgfpathlineto{\pgfqpoint{1.082091in}{0.816975in}}%
\pgfpathlineto{\pgfqpoint{1.082713in}{1.041781in}}%
\pgfpathlineto{\pgfqpoint{1.083439in}{0.903775in}}%
\pgfpathlineto{\pgfqpoint{1.083957in}{0.916521in}}%
\pgfpathlineto{\pgfqpoint{1.084475in}{1.171407in}}%
\pgfpathlineto{\pgfqpoint{1.085201in}{1.110719in}}%
\pgfpathlineto{\pgfqpoint{1.086237in}{0.970147in}}%
\pgfpathlineto{\pgfqpoint{1.086133in}{1.156270in}}%
\pgfpathlineto{\pgfqpoint{1.086341in}{0.977679in}}%
\pgfpathlineto{\pgfqpoint{1.087273in}{1.168213in}}%
\pgfpathlineto{\pgfqpoint{1.087584in}{1.052916in}}%
\pgfpathlineto{\pgfqpoint{1.088103in}{0.986162in}}%
\pgfpathlineto{\pgfqpoint{1.088206in}{1.138414in}}%
\pgfpathlineto{\pgfqpoint{1.088621in}{1.043088in}}%
\pgfpathlineto{\pgfqpoint{1.088932in}{1.136385in}}%
\pgfpathlineto{\pgfqpoint{1.088828in}{1.001353in}}%
\pgfpathlineto{\pgfqpoint{1.089657in}{1.016611in}}%
\pgfpathlineto{\pgfqpoint{1.089761in}{0.950364in}}%
\pgfpathlineto{\pgfqpoint{1.089864in}{1.121594in}}%
\pgfpathlineto{\pgfqpoint{1.090590in}{1.058837in}}%
\pgfpathlineto{\pgfqpoint{1.090901in}{1.187935in}}%
\pgfpathlineto{\pgfqpoint{1.091523in}{0.935610in}}%
\pgfpathlineto{\pgfqpoint{1.091626in}{1.005686in}}%
\pgfpathlineto{\pgfqpoint{1.092559in}{1.262556in}}%
\pgfpathlineto{\pgfqpoint{1.093077in}{1.172413in}}%
\pgfpathlineto{\pgfqpoint{1.094010in}{0.921751in}}%
\pgfpathlineto{\pgfqpoint{1.094321in}{1.005337in}}%
\pgfpathlineto{\pgfqpoint{1.094839in}{1.179577in}}%
\pgfpathlineto{\pgfqpoint{1.095357in}{1.010917in}}%
\pgfpathlineto{\pgfqpoint{1.095668in}{1.126171in}}%
\pgfpathlineto{\pgfqpoint{1.095772in}{1.051498in}}%
\pgfpathlineto{\pgfqpoint{1.096290in}{0.734354in}}%
\pgfpathlineto{\pgfqpoint{1.096808in}{1.051623in}}%
\pgfpathlineto{\pgfqpoint{1.097223in}{0.906096in}}%
\pgfpathlineto{\pgfqpoint{1.097948in}{1.198810in}}%
\pgfpathlineto{\pgfqpoint{1.098363in}{1.001028in}}%
\pgfpathlineto{\pgfqpoint{1.099088in}{1.006387in}}%
\pgfpathlineto{\pgfqpoint{1.099192in}{1.143103in}}%
\pgfpathlineto{\pgfqpoint{1.099607in}{0.963841in}}%
\pgfpathlineto{\pgfqpoint{1.100228in}{1.119094in}}%
\pgfpathlineto{\pgfqpoint{1.100747in}{1.158973in}}%
\pgfpathlineto{\pgfqpoint{1.100643in}{1.040813in}}%
\pgfpathlineto{\pgfqpoint{1.100850in}{1.062810in}}%
\pgfpathlineto{\pgfqpoint{1.101368in}{0.914148in}}%
\pgfpathlineto{\pgfqpoint{1.101058in}{1.084874in}}%
\pgfpathlineto{\pgfqpoint{1.101990in}{1.043020in}}%
\pgfpathlineto{\pgfqpoint{1.102094in}{1.030417in}}%
\pgfpathlineto{\pgfqpoint{1.102198in}{1.071777in}}%
\pgfpathlineto{\pgfqpoint{1.102716in}{0.930809in}}%
\pgfpathlineto{\pgfqpoint{1.103234in}{1.153745in}}%
\pgfpathlineto{\pgfqpoint{1.103649in}{1.051629in}}%
\pgfpathlineto{\pgfqpoint{1.104167in}{1.097787in}}%
\pgfpathlineto{\pgfqpoint{1.104270in}{1.258664in}}%
\pgfpathlineto{\pgfqpoint{1.105099in}{1.158908in}}%
\pgfpathlineto{\pgfqpoint{1.105203in}{0.938127in}}%
\pgfpathlineto{\pgfqpoint{1.105514in}{1.174583in}}%
\pgfpathlineto{\pgfqpoint{1.106136in}{1.095979in}}%
\pgfpathlineto{\pgfqpoint{1.107172in}{1.224512in}}%
\pgfpathlineto{\pgfqpoint{1.106550in}{0.993855in}}%
\pgfpathlineto{\pgfqpoint{1.107276in}{1.144204in}}%
\pgfpathlineto{\pgfqpoint{1.107380in}{1.146643in}}%
\pgfpathlineto{\pgfqpoint{1.108416in}{0.980590in}}%
\pgfpathlineto{\pgfqpoint{1.107794in}{1.155808in}}%
\pgfpathlineto{\pgfqpoint{1.108520in}{1.120951in}}%
\pgfpathlineto{\pgfqpoint{1.109452in}{1.000409in}}%
\pgfpathlineto{\pgfqpoint{1.109141in}{1.126700in}}%
\pgfpathlineto{\pgfqpoint{1.109556in}{1.023918in}}%
\pgfpathlineto{\pgfqpoint{1.109763in}{1.140788in}}%
\pgfpathlineto{\pgfqpoint{1.110281in}{0.929789in}}%
\pgfpathlineto{\pgfqpoint{1.110592in}{1.138953in}}%
\pgfpathlineto{\pgfqpoint{1.110800in}{0.835606in}}%
\pgfpathlineto{\pgfqpoint{1.111836in}{0.955872in}}%
\pgfpathlineto{\pgfqpoint{1.112251in}{1.004959in}}%
\pgfpathlineto{\pgfqpoint{1.112354in}{0.993272in}}%
\pgfpathlineto{\pgfqpoint{1.112769in}{0.790004in}}%
\pgfpathlineto{\pgfqpoint{1.112562in}{0.996285in}}%
\pgfpathlineto{\pgfqpoint{1.113598in}{0.829730in}}%
\pgfpathlineto{\pgfqpoint{1.114634in}{1.040012in}}%
\pgfpathlineto{\pgfqpoint{1.114842in}{1.006769in}}%
\pgfpathlineto{\pgfqpoint{1.115256in}{0.914949in}}%
\pgfpathlineto{\pgfqpoint{1.115774in}{1.034416in}}%
\pgfpathlineto{\pgfqpoint{1.115878in}{1.071853in}}%
\pgfpathlineto{\pgfqpoint{1.116189in}{0.941760in}}%
\pgfpathlineto{\pgfqpoint{1.116396in}{1.048010in}}%
\pgfpathlineto{\pgfqpoint{1.116500in}{0.894007in}}%
\pgfpathlineto{\pgfqpoint{1.117433in}{0.984242in}}%
\pgfpathlineto{\pgfqpoint{1.118158in}{0.791072in}}%
\pgfpathlineto{\pgfqpoint{1.118676in}{1.205373in}}%
\pgfpathlineto{\pgfqpoint{1.119713in}{0.879132in}}%
\pgfpathlineto{\pgfqpoint{1.120127in}{0.954328in}}%
\pgfpathlineto{\pgfqpoint{1.120231in}{1.187513in}}%
\pgfpathlineto{\pgfqpoint{1.120853in}{0.908134in}}%
\pgfpathlineto{\pgfqpoint{1.121267in}{1.087350in}}%
\pgfpathlineto{\pgfqpoint{1.121371in}{1.145115in}}%
\pgfpathlineto{\pgfqpoint{1.121889in}{0.998542in}}%
\pgfpathlineto{\pgfqpoint{1.122200in}{1.125415in}}%
\pgfpathlineto{\pgfqpoint{1.123029in}{0.866156in}}%
\pgfpathlineto{\pgfqpoint{1.123547in}{0.880715in}}%
\pgfpathlineto{\pgfqpoint{1.125724in}{1.214124in}}%
\pgfpathlineto{\pgfqpoint{1.125827in}{1.181686in}}%
\pgfpathlineto{\pgfqpoint{1.126553in}{0.939925in}}%
\pgfpathlineto{\pgfqpoint{1.126968in}{1.116201in}}%
\pgfpathlineto{\pgfqpoint{1.127278in}{1.193878in}}%
\pgfpathlineto{\pgfqpoint{1.127797in}{1.043320in}}%
\pgfpathlineto{\pgfqpoint{1.128833in}{1.208121in}}%
\pgfpathlineto{\pgfqpoint{1.128211in}{1.018691in}}%
\pgfpathlineto{\pgfqpoint{1.128937in}{1.184044in}}%
\pgfpathlineto{\pgfqpoint{1.130180in}{1.002881in}}%
\pgfpathlineto{\pgfqpoint{1.130595in}{1.230156in}}%
\pgfpathlineto{\pgfqpoint{1.131113in}{0.902379in}}%
\pgfpathlineto{\pgfqpoint{1.131320in}{1.048624in}}%
\pgfpathlineto{\pgfqpoint{1.131631in}{0.900211in}}%
\pgfpathlineto{\pgfqpoint{1.132564in}{1.243229in}}%
\pgfpathlineto{\pgfqpoint{1.132668in}{1.012678in}}%
\pgfpathlineto{\pgfqpoint{1.133393in}{1.312966in}}%
\pgfpathlineto{\pgfqpoint{1.133704in}{1.116956in}}%
\pgfpathlineto{\pgfqpoint{1.133911in}{0.984308in}}%
\pgfpathlineto{\pgfqpoint{1.134430in}{1.120492in}}%
\pgfpathlineto{\pgfqpoint{1.134637in}{1.059847in}}%
\pgfpathlineto{\pgfqpoint{1.134741in}{1.178478in}}%
\pgfpathlineto{\pgfqpoint{1.135362in}{0.946714in}}%
\pgfpathlineto{\pgfqpoint{1.135777in}{1.111572in}}%
\pgfpathlineto{\pgfqpoint{1.136088in}{0.966963in}}%
\pgfpathlineto{\pgfqpoint{1.136191in}{1.181918in}}%
\pgfpathlineto{\pgfqpoint{1.136813in}{0.994582in}}%
\pgfpathlineto{\pgfqpoint{1.137435in}{1.355347in}}%
\pgfpathlineto{\pgfqpoint{1.137953in}{1.111443in}}%
\pgfpathlineto{\pgfqpoint{1.138368in}{1.230835in}}%
\pgfpathlineto{\pgfqpoint{1.138472in}{1.045508in}}%
\pgfpathlineto{\pgfqpoint{1.139197in}{1.153117in}}%
\pgfpathlineto{\pgfqpoint{1.140233in}{1.022364in}}%
\pgfpathlineto{\pgfqpoint{1.140026in}{1.164074in}}%
\pgfpathlineto{\pgfqpoint{1.140337in}{1.119644in}}%
\pgfpathlineto{\pgfqpoint{1.141270in}{1.015281in}}%
\pgfpathlineto{\pgfqpoint{1.140648in}{1.210727in}}%
\pgfpathlineto{\pgfqpoint{1.141477in}{1.091067in}}%
\pgfpathlineto{\pgfqpoint{1.141581in}{1.094328in}}%
\pgfpathlineto{\pgfqpoint{1.142514in}{1.292936in}}%
\pgfpathlineto{\pgfqpoint{1.142306in}{1.058909in}}%
\pgfpathlineto{\pgfqpoint{1.142824in}{1.195956in}}%
\pgfpathlineto{\pgfqpoint{1.143239in}{0.979895in}}%
\pgfpathlineto{\pgfqpoint{1.143964in}{1.070155in}}%
\pgfpathlineto{\pgfqpoint{1.144483in}{1.155899in}}%
\pgfpathlineto{\pgfqpoint{1.144897in}{1.032138in}}%
\pgfpathlineto{\pgfqpoint{1.145001in}{1.076371in}}%
\pgfpathlineto{\pgfqpoint{1.145415in}{1.049523in}}%
\pgfpathlineto{\pgfqpoint{1.145208in}{1.246114in}}%
\pgfpathlineto{\pgfqpoint{1.145623in}{1.128345in}}%
\pgfpathlineto{\pgfqpoint{1.145726in}{1.229634in}}%
\pgfpathlineto{\pgfqpoint{1.146141in}{1.020813in}}%
\pgfpathlineto{\pgfqpoint{1.146555in}{1.114958in}}%
\pgfpathlineto{\pgfqpoint{1.147592in}{1.029648in}}%
\pgfpathlineto{\pgfqpoint{1.147488in}{1.161756in}}%
\pgfpathlineto{\pgfqpoint{1.147696in}{1.079438in}}%
\pgfpathlineto{\pgfqpoint{1.148628in}{1.165660in}}%
\pgfpathlineto{\pgfqpoint{1.148214in}{1.037894in}}%
\pgfpathlineto{\pgfqpoint{1.148836in}{1.121298in}}%
\pgfpathlineto{\pgfqpoint{1.148939in}{1.105606in}}%
\pgfpathlineto{\pgfqpoint{1.149043in}{1.162026in}}%
\pgfpathlineto{\pgfqpoint{1.149146in}{1.158776in}}%
\pgfpathlineto{\pgfqpoint{1.149976in}{1.325981in}}%
\pgfpathlineto{\pgfqpoint{1.149665in}{1.085052in}}%
\pgfpathlineto{\pgfqpoint{1.150183in}{1.115026in}}%
\pgfpathlineto{\pgfqpoint{1.150805in}{1.197936in}}%
\pgfpathlineto{\pgfqpoint{1.150390in}{1.010973in}}%
\pgfpathlineto{\pgfqpoint{1.151116in}{1.073575in}}%
\pgfpathlineto{\pgfqpoint{1.152048in}{0.942181in}}%
\pgfpathlineto{\pgfqpoint{1.151841in}{1.184269in}}%
\pgfpathlineto{\pgfqpoint{1.152256in}{1.028396in}}%
\pgfpathlineto{\pgfqpoint{1.152774in}{1.131850in}}%
\pgfpathlineto{\pgfqpoint{1.152878in}{0.962541in}}%
\pgfpathlineto{\pgfqpoint{1.153085in}{1.038602in}}%
\pgfpathlineto{\pgfqpoint{1.153810in}{0.927711in}}%
\pgfpathlineto{\pgfqpoint{1.153292in}{1.133585in}}%
\pgfpathlineto{\pgfqpoint{1.154225in}{0.977050in}}%
\pgfpathlineto{\pgfqpoint{1.155054in}{1.231055in}}%
\pgfpathlineto{\pgfqpoint{1.155469in}{1.106961in}}%
\pgfpathlineto{\pgfqpoint{1.156401in}{0.993273in}}%
\pgfpathlineto{\pgfqpoint{1.156505in}{1.161871in}}%
\pgfpathlineto{\pgfqpoint{1.156816in}{1.028617in}}%
\pgfpathlineto{\pgfqpoint{1.156919in}{1.262504in}}%
\pgfpathlineto{\pgfqpoint{1.157438in}{1.194222in}}%
\pgfpathlineto{\pgfqpoint{1.158060in}{1.288532in}}%
\pgfpathlineto{\pgfqpoint{1.158163in}{1.116790in}}%
\pgfpathlineto{\pgfqpoint{1.158474in}{1.237774in}}%
\pgfpathlineto{\pgfqpoint{1.158992in}{1.257177in}}%
\pgfpathlineto{\pgfqpoint{1.160132in}{0.969874in}}%
\pgfpathlineto{\pgfqpoint{1.161065in}{1.181267in}}%
\pgfpathlineto{\pgfqpoint{1.160443in}{0.905646in}}%
\pgfpathlineto{\pgfqpoint{1.161376in}{1.098632in}}%
\pgfpathlineto{\pgfqpoint{1.161480in}{1.016306in}}%
\pgfpathlineto{\pgfqpoint{1.161791in}{1.319626in}}%
\pgfpathlineto{\pgfqpoint{1.162516in}{1.020371in}}%
\pgfpathlineto{\pgfqpoint{1.163760in}{1.282190in}}%
\pgfpathlineto{\pgfqpoint{1.163863in}{1.249560in}}%
\pgfpathlineto{\pgfqpoint{1.164071in}{1.084950in}}%
\pgfpathlineto{\pgfqpoint{1.164900in}{1.156544in}}%
\pgfpathlineto{\pgfqpoint{1.165314in}{1.257104in}}%
\pgfpathlineto{\pgfqpoint{1.165729in}{1.071473in}}%
\pgfpathlineto{\pgfqpoint{1.165936in}{1.197667in}}%
\pgfpathlineto{\pgfqpoint{1.166454in}{1.009003in}}%
\pgfpathlineto{\pgfqpoint{1.167076in}{1.144896in}}%
\pgfpathlineto{\pgfqpoint{1.167802in}{1.011153in}}%
\pgfpathlineto{\pgfqpoint{1.167698in}{1.147853in}}%
\pgfpathlineto{\pgfqpoint{1.168113in}{1.081904in}}%
\pgfpathlineto{\pgfqpoint{1.168216in}{1.163700in}}%
\pgfpathlineto{\pgfqpoint{1.168942in}{1.009903in}}%
\pgfpathlineto{\pgfqpoint{1.169045in}{1.075022in}}%
\pgfpathlineto{\pgfqpoint{1.169771in}{0.854693in}}%
\pgfpathlineto{\pgfqpoint{1.169460in}{1.129566in}}%
\pgfpathlineto{\pgfqpoint{1.170185in}{1.039056in}}%
\pgfpathlineto{\pgfqpoint{1.170289in}{1.115331in}}%
\pgfpathlineto{\pgfqpoint{1.170807in}{0.876762in}}%
\pgfpathlineto{\pgfqpoint{1.171222in}{0.999256in}}%
\pgfpathlineto{\pgfqpoint{1.171533in}{0.879540in}}%
\pgfpathlineto{\pgfqpoint{1.171947in}{1.093861in}}%
\pgfpathlineto{\pgfqpoint{1.172362in}{0.938312in}}%
\pgfpathlineto{\pgfqpoint{1.172673in}{1.191325in}}%
\pgfpathlineto{\pgfqpoint{1.173295in}{0.921427in}}%
\pgfpathlineto{\pgfqpoint{1.173709in}{1.142900in}}%
\pgfpathlineto{\pgfqpoint{1.174020in}{0.837911in}}%
\pgfpathlineto{\pgfqpoint{1.174953in}{0.888455in}}%
\pgfpathlineto{\pgfqpoint{1.175782in}{0.857811in}}%
\pgfpathlineto{\pgfqpoint{1.176093in}{1.045231in}}%
\pgfpathlineto{\pgfqpoint{1.177337in}{0.832992in}}%
\pgfpathlineto{\pgfqpoint{1.177855in}{1.031891in}}%
\pgfpathlineto{\pgfqpoint{1.178477in}{0.904682in}}%
\pgfpathlineto{\pgfqpoint{1.178580in}{0.892596in}}%
\pgfpathlineto{\pgfqpoint{1.178684in}{0.960824in}}%
\pgfpathlineto{\pgfqpoint{1.178891in}{0.923112in}}%
\pgfpathlineto{\pgfqpoint{1.178995in}{1.014502in}}%
\pgfpathlineto{\pgfqpoint{1.179513in}{0.883264in}}%
\pgfpathlineto{\pgfqpoint{1.179824in}{0.937915in}}%
\pgfpathlineto{\pgfqpoint{1.179928in}{0.853824in}}%
\pgfpathlineto{\pgfqpoint{1.180549in}{1.059253in}}%
\pgfpathlineto{\pgfqpoint{1.180757in}{1.024149in}}%
\pgfpathlineto{\pgfqpoint{1.180860in}{1.092440in}}%
\pgfpathlineto{\pgfqpoint{1.181171in}{0.921207in}}%
\pgfpathlineto{\pgfqpoint{1.181689in}{0.976424in}}%
\pgfpathlineto{\pgfqpoint{1.181793in}{0.856958in}}%
\pgfpathlineto{\pgfqpoint{1.182415in}{1.077676in}}%
\pgfpathlineto{\pgfqpoint{1.182726in}{0.973983in}}%
\pgfpathlineto{\pgfqpoint{1.182829in}{0.978560in}}%
\pgfpathlineto{\pgfqpoint{1.183451in}{1.180260in}}%
\pgfpathlineto{\pgfqpoint{1.183866in}{0.914352in}}%
\pgfpathlineto{\pgfqpoint{1.183970in}{1.040931in}}%
\pgfpathlineto{\pgfqpoint{1.184073in}{1.040155in}}%
\pgfpathlineto{\pgfqpoint{1.184799in}{1.168423in}}%
\pgfpathlineto{\pgfqpoint{1.184488in}{0.983274in}}%
\pgfpathlineto{\pgfqpoint{1.185110in}{1.030055in}}%
\pgfpathlineto{\pgfqpoint{1.186250in}{0.821334in}}%
\pgfpathlineto{\pgfqpoint{1.185628in}{1.083106in}}%
\pgfpathlineto{\pgfqpoint{1.186457in}{0.925422in}}%
\pgfpathlineto{\pgfqpoint{1.186975in}{1.100972in}}%
\pgfpathlineto{\pgfqpoint{1.187390in}{0.952733in}}%
\pgfpathlineto{\pgfqpoint{1.187493in}{0.886477in}}%
\pgfpathlineto{\pgfqpoint{1.188219in}{1.000600in}}%
\pgfpathlineto{\pgfqpoint{1.188426in}{0.962275in}}%
\pgfpathlineto{\pgfqpoint{1.188530in}{0.962154in}}%
\pgfpathlineto{\pgfqpoint{1.189462in}{0.944036in}}%
\pgfpathlineto{\pgfqpoint{1.189877in}{1.210259in}}%
\pgfpathlineto{\pgfqpoint{1.190188in}{0.977269in}}%
\pgfpathlineto{\pgfqpoint{1.191224in}{0.992429in}}%
\pgfpathlineto{\pgfqpoint{1.192468in}{1.170397in}}%
\pgfpathlineto{\pgfqpoint{1.192572in}{1.152145in}}%
\pgfpathlineto{\pgfqpoint{1.193297in}{0.887481in}}%
\pgfpathlineto{\pgfqpoint{1.193712in}{0.953180in}}%
\pgfpathlineto{\pgfqpoint{1.194230in}{1.192474in}}%
\pgfpathlineto{\pgfqpoint{1.194541in}{0.881831in}}%
\pgfpathlineto{\pgfqpoint{1.194852in}{1.057946in}}%
\pgfpathlineto{\pgfqpoint{1.195059in}{1.003919in}}%
\pgfpathlineto{\pgfqpoint{1.195163in}{1.057577in}}%
\pgfpathlineto{\pgfqpoint{1.195266in}{1.128758in}}%
\pgfpathlineto{\pgfqpoint{1.195370in}{0.970164in}}%
\pgfpathlineto{\pgfqpoint{1.196303in}{1.107101in}}%
\pgfpathlineto{\pgfqpoint{1.196406in}{1.100858in}}%
\pgfpathlineto{\pgfqpoint{1.197443in}{0.936826in}}%
\pgfpathlineto{\pgfqpoint{1.197132in}{1.211929in}}%
\pgfpathlineto{\pgfqpoint{1.197546in}{1.009792in}}%
\pgfpathlineto{\pgfqpoint{1.197650in}{1.011663in}}%
\pgfpathlineto{\pgfqpoint{1.198686in}{1.129378in}}%
\pgfpathlineto{\pgfqpoint{1.198065in}{0.918445in}}%
\pgfpathlineto{\pgfqpoint{1.198894in}{1.079916in}}%
\pgfpathlineto{\pgfqpoint{1.198997in}{1.018191in}}%
\pgfpathlineto{\pgfqpoint{1.199619in}{1.222330in}}%
\pgfpathlineto{\pgfqpoint{1.199930in}{1.093730in}}%
\pgfpathlineto{\pgfqpoint{1.200966in}{1.180627in}}%
\pgfpathlineto{\pgfqpoint{1.200241in}{0.888153in}}%
\pgfpathlineto{\pgfqpoint{1.201070in}{1.121298in}}%
\pgfpathlineto{\pgfqpoint{1.201277in}{1.098871in}}%
\pgfpathlineto{\pgfqpoint{1.201899in}{1.192690in}}%
\pgfpathlineto{\pgfqpoint{1.201588in}{1.036359in}}%
\pgfpathlineto{\pgfqpoint{1.202314in}{1.068814in}}%
\pgfpathlineto{\pgfqpoint{1.202625in}{0.924571in}}%
\pgfpathlineto{\pgfqpoint{1.203143in}{1.227590in}}%
\pgfpathlineto{\pgfqpoint{1.203350in}{1.073692in}}%
\pgfpathlineto{\pgfqpoint{1.203454in}{1.279386in}}%
\pgfpathlineto{\pgfqpoint{1.204387in}{1.011645in}}%
\pgfpathlineto{\pgfqpoint{1.204801in}{1.090543in}}%
\pgfpathlineto{\pgfqpoint{1.204698in}{0.945111in}}%
\pgfpathlineto{\pgfqpoint{1.205216in}{1.019558in}}%
\pgfpathlineto{\pgfqpoint{1.205319in}{0.906034in}}%
\pgfpathlineto{\pgfqpoint{1.205630in}{1.091074in}}%
\pgfpathlineto{\pgfqpoint{1.206252in}{1.043847in}}%
\pgfpathlineto{\pgfqpoint{1.206770in}{0.870930in}}%
\pgfpathlineto{\pgfqpoint{1.207081in}{0.965062in}}%
\pgfpathlineto{\pgfqpoint{1.207185in}{1.123455in}}%
\pgfpathlineto{\pgfqpoint{1.208118in}{1.092621in}}%
\pgfpathlineto{\pgfqpoint{1.208947in}{0.823841in}}%
\pgfpathlineto{\pgfqpoint{1.208429in}{1.120679in}}%
\pgfpathlineto{\pgfqpoint{1.209258in}{0.952223in}}%
\pgfpathlineto{\pgfqpoint{1.209465in}{0.969838in}}%
\pgfpathlineto{\pgfqpoint{1.210087in}{1.106012in}}%
\pgfpathlineto{\pgfqpoint{1.209776in}{0.965495in}}%
\pgfpathlineto{\pgfqpoint{1.210605in}{1.002229in}}%
\pgfpathlineto{\pgfqpoint{1.210812in}{0.839856in}}%
\pgfpathlineto{\pgfqpoint{1.211641in}{1.013445in}}%
\pgfpathlineto{\pgfqpoint{1.211745in}{0.888542in}}%
\pgfpathlineto{\pgfqpoint{1.212574in}{1.005007in}}%
\pgfpathlineto{\pgfqpoint{1.211952in}{0.867299in}}%
\pgfpathlineto{\pgfqpoint{1.212678in}{0.920816in}}%
\pgfpathlineto{\pgfqpoint{1.212781in}{0.855202in}}%
\pgfpathlineto{\pgfqpoint{1.213507in}{1.043149in}}%
\pgfpathlineto{\pgfqpoint{1.213611in}{0.982367in}}%
\pgfpathlineto{\pgfqpoint{1.213714in}{1.028474in}}%
\pgfpathlineto{\pgfqpoint{1.214025in}{0.895205in}}%
\pgfpathlineto{\pgfqpoint{1.214543in}{0.906872in}}%
\pgfpathlineto{\pgfqpoint{1.214647in}{0.898907in}}%
\pgfpathlineto{\pgfqpoint{1.214751in}{0.963860in}}%
\pgfpathlineto{\pgfqpoint{1.215269in}{1.151692in}}%
\pgfpathlineto{\pgfqpoint{1.215683in}{0.783578in}}%
\pgfpathlineto{\pgfqpoint{1.215787in}{0.904434in}}%
\pgfpathlineto{\pgfqpoint{1.216202in}{1.155412in}}%
\pgfpathlineto{\pgfqpoint{1.217134in}{1.094160in}}%
\pgfpathlineto{\pgfqpoint{1.218067in}{0.903660in}}%
\pgfpathlineto{\pgfqpoint{1.217756in}{1.189647in}}%
\pgfpathlineto{\pgfqpoint{1.218274in}{0.971953in}}%
\pgfpathlineto{\pgfqpoint{1.219000in}{1.049905in}}%
\pgfpathlineto{\pgfqpoint{1.218896in}{0.927771in}}%
\pgfpathlineto{\pgfqpoint{1.219103in}{0.978826in}}%
\pgfpathlineto{\pgfqpoint{1.219207in}{0.880062in}}%
\pgfpathlineto{\pgfqpoint{1.219622in}{1.071936in}}%
\pgfpathlineto{\pgfqpoint{1.220140in}{1.019417in}}%
\pgfpathlineto{\pgfqpoint{1.221073in}{0.747979in}}%
\pgfpathlineto{\pgfqpoint{1.220451in}{1.032390in}}%
\pgfpathlineto{\pgfqpoint{1.221280in}{0.859421in}}%
\pgfpathlineto{\pgfqpoint{1.221384in}{1.090963in}}%
\pgfpathlineto{\pgfqpoint{1.222316in}{0.758987in}}%
\pgfpathlineto{\pgfqpoint{1.222420in}{0.685871in}}%
\pgfpathlineto{\pgfqpoint{1.222938in}{0.962500in}}%
\pgfpathlineto{\pgfqpoint{1.223353in}{0.803636in}}%
\pgfpathlineto{\pgfqpoint{1.223871in}{0.778225in}}%
\pgfpathlineto{\pgfqpoint{1.223975in}{0.829138in}}%
\pgfpathlineto{\pgfqpoint{1.224285in}{0.920095in}}%
\pgfpathlineto{\pgfqpoint{1.224493in}{0.741683in}}%
\pgfpathlineto{\pgfqpoint{1.224804in}{0.843494in}}%
\pgfpathlineto{\pgfqpoint{1.225736in}{0.660787in}}%
\pgfpathlineto{\pgfqpoint{1.225426in}{0.898778in}}%
\pgfpathlineto{\pgfqpoint{1.225944in}{0.752924in}}%
\pgfpathlineto{\pgfqpoint{1.226876in}{0.907129in}}%
\pgfpathlineto{\pgfqpoint{1.226255in}{0.672284in}}%
\pgfpathlineto{\pgfqpoint{1.226980in}{0.867949in}}%
\pgfpathlineto{\pgfqpoint{1.227706in}{0.615829in}}%
\pgfpathlineto{\pgfqpoint{1.228120in}{0.809781in}}%
\pgfpathlineto{\pgfqpoint{1.228742in}{0.739182in}}%
\pgfpathlineto{\pgfqpoint{1.229260in}{0.910620in}}%
\pgfpathlineto{\pgfqpoint{1.229882in}{0.937010in}}%
\pgfpathlineto{\pgfqpoint{1.230400in}{0.698090in}}%
\pgfpathlineto{\pgfqpoint{1.231022in}{1.058702in}}%
\pgfpathlineto{\pgfqpoint{1.231644in}{0.958779in}}%
\pgfpathlineto{\pgfqpoint{1.232058in}{0.778140in}}%
\pgfpathlineto{\pgfqpoint{1.231851in}{0.984637in}}%
\pgfpathlineto{\pgfqpoint{1.232784in}{0.925876in}}%
\pgfpathlineto{\pgfqpoint{1.233924in}{1.091033in}}%
\pgfpathlineto{\pgfqpoint{1.233509in}{0.901803in}}%
\pgfpathlineto{\pgfqpoint{1.234028in}{1.060522in}}%
\pgfpathlineto{\pgfqpoint{1.234546in}{0.885512in}}%
\pgfpathlineto{\pgfqpoint{1.234857in}{1.068029in}}%
\pgfpathlineto{\pgfqpoint{1.235064in}{1.042826in}}%
\pgfpathlineto{\pgfqpoint{1.235686in}{1.199218in}}%
\pgfpathlineto{\pgfqpoint{1.235271in}{0.962997in}}%
\pgfpathlineto{\pgfqpoint{1.236100in}{1.076351in}}%
\pgfpathlineto{\pgfqpoint{1.236930in}{1.139774in}}%
\pgfpathlineto{\pgfqpoint{1.237551in}{0.902536in}}%
\pgfpathlineto{\pgfqpoint{1.237759in}{0.926396in}}%
\pgfpathlineto{\pgfqpoint{1.238070in}{0.862694in}}%
\pgfpathlineto{\pgfqpoint{1.239002in}{1.134875in}}%
\pgfpathlineto{\pgfqpoint{1.239624in}{1.236599in}}%
\pgfpathlineto{\pgfqpoint{1.240453in}{0.864169in}}%
\pgfpathlineto{\pgfqpoint{1.241697in}{1.198014in}}%
\pgfpathlineto{\pgfqpoint{1.240868in}{0.846702in}}%
\pgfpathlineto{\pgfqpoint{1.241904in}{1.076733in}}%
\pgfpathlineto{\pgfqpoint{1.242422in}{0.966632in}}%
\pgfpathlineto{\pgfqpoint{1.242215in}{1.142882in}}%
\pgfpathlineto{\pgfqpoint{1.242941in}{1.103598in}}%
\pgfpathlineto{\pgfqpoint{1.243873in}{1.314206in}}%
\pgfpathlineto{\pgfqpoint{1.244081in}{1.182522in}}%
\pgfpathlineto{\pgfqpoint{1.245013in}{1.249973in}}%
\pgfpathlineto{\pgfqpoint{1.245428in}{0.976304in}}%
\pgfpathlineto{\pgfqpoint{1.245843in}{1.222906in}}%
\pgfpathlineto{\pgfqpoint{1.246568in}{1.077864in}}%
\pgfpathlineto{\pgfqpoint{1.246672in}{1.133304in}}%
\pgfpathlineto{\pgfqpoint{1.247294in}{1.023828in}}%
\pgfpathlineto{\pgfqpoint{1.247501in}{1.083871in}}%
\pgfpathlineto{\pgfqpoint{1.247915in}{0.976052in}}%
\pgfpathlineto{\pgfqpoint{1.248330in}{1.220515in}}%
\pgfpathlineto{\pgfqpoint{1.248641in}{1.051653in}}%
\pgfpathlineto{\pgfqpoint{1.248745in}{1.141268in}}%
\pgfpathlineto{\pgfqpoint{1.249263in}{0.974253in}}%
\pgfpathlineto{\pgfqpoint{1.249677in}{1.027099in}}%
\pgfpathlineto{\pgfqpoint{1.249885in}{1.047761in}}%
\pgfpathlineto{\pgfqpoint{1.250714in}{0.988405in}}%
\pgfpathlineto{\pgfqpoint{1.250506in}{1.195862in}}%
\pgfpathlineto{\pgfqpoint{1.250921in}{1.018476in}}%
\pgfpathlineto{\pgfqpoint{1.251025in}{1.167348in}}%
\pgfpathlineto{\pgfqpoint{1.251336in}{0.983036in}}%
\pgfpathlineto{\pgfqpoint{1.252061in}{1.061173in}}%
\pgfpathlineto{\pgfqpoint{1.252890in}{1.170226in}}%
\pgfpathlineto{\pgfqpoint{1.252683in}{0.966233in}}%
\pgfpathlineto{\pgfqpoint{1.252994in}{1.048131in}}%
\pgfpathlineto{\pgfqpoint{1.253408in}{1.082218in}}%
\pgfpathlineto{\pgfqpoint{1.254237in}{0.891587in}}%
\pgfpathlineto{\pgfqpoint{1.254548in}{1.081967in}}%
\pgfpathlineto{\pgfqpoint{1.255377in}{0.995682in}}%
\pgfpathlineto{\pgfqpoint{1.255688in}{0.911244in}}%
\pgfpathlineto{\pgfqpoint{1.255792in}{1.067793in}}%
\pgfpathlineto{\pgfqpoint{1.255999in}{1.118025in}}%
\pgfpathlineto{\pgfqpoint{1.256103in}{1.054978in}}%
\pgfpathlineto{\pgfqpoint{1.256207in}{0.934316in}}%
\pgfpathlineto{\pgfqpoint{1.257036in}{1.184606in}}%
\pgfpathlineto{\pgfqpoint{1.257139in}{1.058826in}}%
\pgfpathlineto{\pgfqpoint{1.257347in}{1.120795in}}%
\pgfpathlineto{\pgfqpoint{1.258176in}{0.885519in}}%
\pgfpathlineto{\pgfqpoint{1.258383in}{1.159039in}}%
\pgfpathlineto{\pgfqpoint{1.259316in}{1.028856in}}%
\pgfpathlineto{\pgfqpoint{1.259627in}{0.865413in}}%
\pgfpathlineto{\pgfqpoint{1.260145in}{1.079385in}}%
\pgfpathlineto{\pgfqpoint{1.260352in}{1.018251in}}%
\pgfpathlineto{\pgfqpoint{1.260456in}{1.147210in}}%
\pgfpathlineto{\pgfqpoint{1.260767in}{0.961376in}}%
\pgfpathlineto{\pgfqpoint{1.261389in}{1.014516in}}%
\pgfpathlineto{\pgfqpoint{1.261492in}{0.851773in}}%
\pgfpathlineto{\pgfqpoint{1.262425in}{1.047916in}}%
\pgfpathlineto{\pgfqpoint{1.262943in}{1.087668in}}%
\pgfpathlineto{\pgfqpoint{1.263047in}{0.942602in}}%
\pgfpathlineto{\pgfqpoint{1.263254in}{0.997533in}}%
\pgfpathlineto{\pgfqpoint{1.263358in}{0.988176in}}%
\pgfpathlineto{\pgfqpoint{1.263461in}{1.185173in}}%
\pgfpathlineto{\pgfqpoint{1.263772in}{0.911652in}}%
\pgfpathlineto{\pgfqpoint{1.264394in}{1.085180in}}%
\pgfpathlineto{\pgfqpoint{1.264705in}{1.005247in}}%
\pgfpathlineto{\pgfqpoint{1.265120in}{1.131145in}}%
\pgfpathlineto{\pgfqpoint{1.265327in}{1.068655in}}%
\pgfpathlineto{\pgfqpoint{1.265431in}{1.121102in}}%
\pgfpathlineto{\pgfqpoint{1.266156in}{0.974418in}}%
\pgfpathlineto{\pgfqpoint{1.266363in}{1.033043in}}%
\pgfpathlineto{\pgfqpoint{1.266674in}{1.141506in}}%
\pgfpathlineto{\pgfqpoint{1.266882in}{1.022833in}}%
\pgfpathlineto{\pgfqpoint{1.266985in}{0.970742in}}%
\pgfpathlineto{\pgfqpoint{1.267503in}{1.158514in}}%
\pgfpathlineto{\pgfqpoint{1.267814in}{1.033196in}}%
\pgfpathlineto{\pgfqpoint{1.267918in}{1.110347in}}%
\pgfpathlineto{\pgfqpoint{1.268540in}{0.909133in}}%
\pgfpathlineto{\pgfqpoint{1.268747in}{1.093677in}}%
\pgfpathlineto{\pgfqpoint{1.268851in}{0.921924in}}%
\pgfpathlineto{\pgfqpoint{1.269576in}{1.108680in}}%
\pgfpathlineto{\pgfqpoint{1.269783in}{1.058647in}}%
\pgfpathlineto{\pgfqpoint{1.270094in}{1.253741in}}%
\pgfpathlineto{\pgfqpoint{1.270923in}{1.123266in}}%
\pgfpathlineto{\pgfqpoint{1.271027in}{1.027868in}}%
\pgfpathlineto{\pgfqpoint{1.271545in}{1.216498in}}%
\pgfpathlineto{\pgfqpoint{1.272064in}{1.084544in}}%
\pgfpathlineto{\pgfqpoint{1.272167in}{1.131147in}}%
\pgfpathlineto{\pgfqpoint{1.272582in}{1.024144in}}%
\pgfpathlineto{\pgfqpoint{1.273100in}{1.065062in}}%
\pgfpathlineto{\pgfqpoint{1.274033in}{0.925526in}}%
\pgfpathlineto{\pgfqpoint{1.273618in}{1.221265in}}%
\pgfpathlineto{\pgfqpoint{1.274136in}{0.979384in}}%
\pgfpathlineto{\pgfqpoint{1.275173in}{1.207578in}}%
\pgfpathlineto{\pgfqpoint{1.274551in}{0.861698in}}%
\pgfpathlineto{\pgfqpoint{1.275276in}{1.116464in}}%
\pgfpathlineto{\pgfqpoint{1.275380in}{0.945234in}}%
\pgfpathlineto{\pgfqpoint{1.276209in}{1.143834in}}%
\pgfpathlineto{\pgfqpoint{1.276313in}{1.126220in}}%
\pgfpathlineto{\pgfqpoint{1.276935in}{1.065362in}}%
\pgfpathlineto{\pgfqpoint{1.276727in}{1.251534in}}%
\pgfpathlineto{\pgfqpoint{1.277142in}{1.110903in}}%
\pgfpathlineto{\pgfqpoint{1.277246in}{1.183985in}}%
\pgfpathlineto{\pgfqpoint{1.277971in}{0.949266in}}%
\pgfpathlineto{\pgfqpoint{1.279318in}{1.123871in}}%
\pgfpathlineto{\pgfqpoint{1.278696in}{0.936648in}}%
\pgfpathlineto{\pgfqpoint{1.279837in}{1.084630in}}%
\pgfpathlineto{\pgfqpoint{1.280458in}{0.926885in}}%
\pgfpathlineto{\pgfqpoint{1.280977in}{0.945887in}}%
\pgfpathlineto{\pgfqpoint{1.281702in}{1.162318in}}%
\pgfpathlineto{\pgfqpoint{1.282013in}{0.868293in}}%
\pgfpathlineto{\pgfqpoint{1.282220in}{0.840500in}}%
\pgfpathlineto{\pgfqpoint{1.282324in}{0.919909in}}%
\pgfpathlineto{\pgfqpoint{1.282635in}{0.888677in}}%
\pgfpathlineto{\pgfqpoint{1.283360in}{0.993241in}}%
\pgfpathlineto{\pgfqpoint{1.283153in}{0.828638in}}%
\pgfpathlineto{\pgfqpoint{1.283671in}{0.888691in}}%
\pgfpathlineto{\pgfqpoint{1.284604in}{0.758277in}}%
\pgfpathlineto{\pgfqpoint{1.284086in}{0.959839in}}%
\pgfpathlineto{\pgfqpoint{1.284811in}{0.830801in}}%
\pgfpathlineto{\pgfqpoint{1.285537in}{0.904566in}}%
\pgfpathlineto{\pgfqpoint{1.285122in}{0.786255in}}%
\pgfpathlineto{\pgfqpoint{1.285951in}{0.844558in}}%
\pgfpathlineto{\pgfqpoint{1.286470in}{0.686805in}}%
\pgfpathlineto{\pgfqpoint{1.287091in}{0.914145in}}%
\pgfpathlineto{\pgfqpoint{1.287195in}{0.755529in}}%
\pgfpathlineto{\pgfqpoint{1.287610in}{1.030120in}}%
\pgfpathlineto{\pgfqpoint{1.288128in}{0.950372in}}%
\pgfpathlineto{\pgfqpoint{1.288231in}{0.875333in}}%
\pgfpathlineto{\pgfqpoint{1.288853in}{1.050048in}}%
\pgfpathlineto{\pgfqpoint{1.289061in}{0.960133in}}%
\pgfpathlineto{\pgfqpoint{1.290201in}{1.118737in}}%
\pgfpathlineto{\pgfqpoint{1.289475in}{0.879002in}}%
\pgfpathlineto{\pgfqpoint{1.290304in}{1.115812in}}%
\pgfpathlineto{\pgfqpoint{1.291030in}{0.853713in}}%
\pgfpathlineto{\pgfqpoint{1.291444in}{0.939727in}}%
\pgfpathlineto{\pgfqpoint{1.292584in}{1.147791in}}%
\pgfpathlineto{\pgfqpoint{1.292688in}{1.041728in}}%
\pgfpathlineto{\pgfqpoint{1.292792in}{1.102886in}}%
\pgfpathlineto{\pgfqpoint{1.293413in}{0.825616in}}%
\pgfpathlineto{\pgfqpoint{1.293517in}{0.798135in}}%
\pgfpathlineto{\pgfqpoint{1.293724in}{0.908385in}}%
\pgfpathlineto{\pgfqpoint{1.294139in}{0.894341in}}%
\pgfpathlineto{\pgfqpoint{1.294864in}{0.847183in}}%
\pgfpathlineto{\pgfqpoint{1.295486in}{1.089499in}}%
\pgfpathlineto{\pgfqpoint{1.296004in}{1.162481in}}%
\pgfpathlineto{\pgfqpoint{1.296523in}{0.993710in}}%
\pgfpathlineto{\pgfqpoint{1.296626in}{1.177596in}}%
\pgfpathlineto{\pgfqpoint{1.297352in}{0.925084in}}%
\pgfpathlineto{\pgfqpoint{1.297559in}{1.007228in}}%
\pgfpathlineto{\pgfqpoint{1.297766in}{0.895166in}}%
\pgfpathlineto{\pgfqpoint{1.297870in}{1.062389in}}%
\pgfpathlineto{\pgfqpoint{1.297974in}{1.030498in}}%
\pgfpathlineto{\pgfqpoint{1.298077in}{1.120101in}}%
\pgfpathlineto{\pgfqpoint{1.298906in}{0.884412in}}%
\pgfpathlineto{\pgfqpoint{1.299010in}{1.041539in}}%
\pgfpathlineto{\pgfqpoint{1.299321in}{0.853950in}}%
\pgfpathlineto{\pgfqpoint{1.299425in}{1.042160in}}%
\pgfpathlineto{\pgfqpoint{1.300254in}{0.945360in}}%
\pgfpathlineto{\pgfqpoint{1.300357in}{0.898783in}}%
\pgfpathlineto{\pgfqpoint{1.300875in}{1.111977in}}%
\pgfpathlineto{\pgfqpoint{1.301083in}{1.054871in}}%
\pgfpathlineto{\pgfqpoint{1.301290in}{1.234580in}}%
\pgfpathlineto{\pgfqpoint{1.301601in}{0.887105in}}%
\pgfpathlineto{\pgfqpoint{1.302016in}{0.964557in}}%
\pgfpathlineto{\pgfqpoint{1.302119in}{0.847018in}}%
\pgfpathlineto{\pgfqpoint{1.302326in}{1.057917in}}%
\pgfpathlineto{\pgfqpoint{1.303156in}{0.904424in}}%
\pgfpathlineto{\pgfqpoint{1.304192in}{1.019840in}}%
\pgfpathlineto{\pgfqpoint{1.303674in}{0.848865in}}%
\pgfpathlineto{\pgfqpoint{1.304296in}{1.010534in}}%
\pgfpathlineto{\pgfqpoint{1.304399in}{0.920286in}}%
\pgfpathlineto{\pgfqpoint{1.305228in}{1.079956in}}%
\pgfpathlineto{\pgfqpoint{1.305332in}{0.935652in}}%
\pgfpathlineto{\pgfqpoint{1.305747in}{1.074076in}}%
\pgfpathlineto{\pgfqpoint{1.306057in}{0.851782in}}%
\pgfpathlineto{\pgfqpoint{1.306265in}{0.866978in}}%
\pgfpathlineto{\pgfqpoint{1.306368in}{0.827290in}}%
\pgfpathlineto{\pgfqpoint{1.306679in}{0.974494in}}%
\pgfpathlineto{\pgfqpoint{1.307094in}{0.966826in}}%
\pgfpathlineto{\pgfqpoint{1.307301in}{1.049208in}}%
\pgfpathlineto{\pgfqpoint{1.307716in}{0.915177in}}%
\pgfpathlineto{\pgfqpoint{1.308130in}{0.980059in}}%
\pgfpathlineto{\pgfqpoint{1.308234in}{0.979043in}}%
\pgfpathlineto{\pgfqpoint{1.308338in}{0.987656in}}%
\pgfpathlineto{\pgfqpoint{1.308648in}{0.815221in}}%
\pgfpathlineto{\pgfqpoint{1.309374in}{0.873291in}}%
\pgfpathlineto{\pgfqpoint{1.309581in}{1.030859in}}%
\pgfpathlineto{\pgfqpoint{1.310203in}{0.786354in}}%
\pgfpathlineto{\pgfqpoint{1.310307in}{0.836882in}}%
\pgfpathlineto{\pgfqpoint{1.310410in}{0.787611in}}%
\pgfpathlineto{\pgfqpoint{1.310929in}{0.984576in}}%
\pgfpathlineto{\pgfqpoint{1.312069in}{1.134629in}}%
\pgfpathlineto{\pgfqpoint{1.311343in}{0.883203in}}%
\pgfpathlineto{\pgfqpoint{1.312172in}{1.105847in}}%
\pgfpathlineto{\pgfqpoint{1.312276in}{0.904640in}}%
\pgfpathlineto{\pgfqpoint{1.313001in}{1.138923in}}%
\pgfpathlineto{\pgfqpoint{1.313312in}{0.978050in}}%
\pgfpathlineto{\pgfqpoint{1.313830in}{1.259317in}}%
\pgfpathlineto{\pgfqpoint{1.314245in}{0.903129in}}%
\pgfpathlineto{\pgfqpoint{1.314660in}{0.834751in}}%
\pgfpathlineto{\pgfqpoint{1.315074in}{0.980379in}}%
\pgfpathlineto{\pgfqpoint{1.315281in}{0.912278in}}%
\pgfpathlineto{\pgfqpoint{1.315592in}{0.719358in}}%
\pgfpathlineto{\pgfqpoint{1.316421in}{1.094948in}}%
\pgfpathlineto{\pgfqpoint{1.316940in}{1.148471in}}%
\pgfpathlineto{\pgfqpoint{1.317562in}{0.848027in}}%
\pgfpathlineto{\pgfqpoint{1.318183in}{1.083542in}}%
\pgfpathlineto{\pgfqpoint{1.318805in}{1.078327in}}%
\pgfpathlineto{\pgfqpoint{1.319012in}{0.920544in}}%
\pgfpathlineto{\pgfqpoint{1.319842in}{1.141640in}}%
\pgfpathlineto{\pgfqpoint{1.319945in}{1.186559in}}%
\pgfpathlineto{\pgfqpoint{1.320360in}{1.000380in}}%
\pgfpathlineto{\pgfqpoint{1.320567in}{1.067579in}}%
\pgfpathlineto{\pgfqpoint{1.321603in}{0.773509in}}%
\pgfpathlineto{\pgfqpoint{1.322018in}{0.719108in}}%
\pgfpathlineto{\pgfqpoint{1.321811in}{0.852126in}}%
\pgfpathlineto{\pgfqpoint{1.322225in}{0.789756in}}%
\pgfpathlineto{\pgfqpoint{1.322433in}{1.039372in}}%
\pgfpathlineto{\pgfqpoint{1.323365in}{0.931753in}}%
\pgfpathlineto{\pgfqpoint{1.324091in}{0.796117in}}%
\pgfpathlineto{\pgfqpoint{1.323987in}{0.936734in}}%
\pgfpathlineto{\pgfqpoint{1.324505in}{0.810783in}}%
\pgfpathlineto{\pgfqpoint{1.325231in}{1.151196in}}%
\pgfpathlineto{\pgfqpoint{1.325645in}{0.935045in}}%
\pgfpathlineto{\pgfqpoint{1.325749in}{0.823359in}}%
\pgfpathlineto{\pgfqpoint{1.326578in}{1.063092in}}%
\pgfpathlineto{\pgfqpoint{1.326682in}{1.137946in}}%
\pgfpathlineto{\pgfqpoint{1.327407in}{0.921121in}}%
\pgfpathlineto{\pgfqpoint{1.327615in}{1.037359in}}%
\pgfpathlineto{\pgfqpoint{1.328444in}{0.805088in}}%
\pgfpathlineto{\pgfqpoint{1.328858in}{0.873724in}}%
\pgfpathlineto{\pgfqpoint{1.329687in}{1.186228in}}%
\pgfpathlineto{\pgfqpoint{1.329169in}{0.782636in}}%
\pgfpathlineto{\pgfqpoint{1.330102in}{0.927798in}}%
\pgfpathlineto{\pgfqpoint{1.330827in}{0.951156in}}%
\pgfpathlineto{\pgfqpoint{1.331657in}{0.708225in}}%
\pgfpathlineto{\pgfqpoint{1.332486in}{1.078881in}}%
\pgfpathlineto{\pgfqpoint{1.332900in}{0.982974in}}%
\pgfpathlineto{\pgfqpoint{1.333729in}{0.777750in}}%
\pgfpathlineto{\pgfqpoint{1.334040in}{0.880223in}}%
\pgfpathlineto{\pgfqpoint{1.334662in}{0.836159in}}%
\pgfpathlineto{\pgfqpoint{1.335077in}{1.031681in}}%
\pgfpathlineto{\pgfqpoint{1.336320in}{0.737202in}}%
\pgfpathlineto{\pgfqpoint{1.336631in}{1.098895in}}%
\pgfpathlineto{\pgfqpoint{1.337771in}{1.023954in}}%
\pgfpathlineto{\pgfqpoint{1.337875in}{0.922041in}}%
\pgfpathlineto{\pgfqpoint{1.338704in}{1.185581in}}%
\pgfpathlineto{\pgfqpoint{1.338911in}{0.934633in}}%
\pgfpathlineto{\pgfqpoint{1.339015in}{1.073009in}}%
\pgfpathlineto{\pgfqpoint{1.339533in}{0.852651in}}%
\pgfpathlineto{\pgfqpoint{1.340051in}{1.072867in}}%
\pgfpathlineto{\pgfqpoint{1.340259in}{1.129753in}}%
\pgfpathlineto{\pgfqpoint{1.340673in}{1.013982in}}%
\pgfpathlineto{\pgfqpoint{1.340984in}{1.088913in}}%
\pgfpathlineto{\pgfqpoint{1.341088in}{0.934545in}}%
\pgfpathlineto{\pgfqpoint{1.341917in}{1.250694in}}%
\pgfpathlineto{\pgfqpoint{1.342021in}{1.196434in}}%
\pgfpathlineto{\pgfqpoint{1.342746in}{1.070798in}}%
\pgfpathlineto{\pgfqpoint{1.342331in}{1.208436in}}%
\pgfpathlineto{\pgfqpoint{1.342953in}{1.141459in}}%
\pgfpathlineto{\pgfqpoint{1.343057in}{1.267743in}}%
\pgfpathlineto{\pgfqpoint{1.343886in}{1.005907in}}%
\pgfpathlineto{\pgfqpoint{1.343990in}{1.071128in}}%
\pgfpathlineto{\pgfqpoint{1.344922in}{0.858573in}}%
\pgfpathlineto{\pgfqpoint{1.344612in}{1.161336in}}%
\pgfpathlineto{\pgfqpoint{1.345130in}{0.997389in}}%
\pgfpathlineto{\pgfqpoint{1.346063in}{0.867212in}}%
\pgfpathlineto{\pgfqpoint{1.345337in}{1.012390in}}%
\pgfpathlineto{\pgfqpoint{1.346166in}{0.987389in}}%
\pgfpathlineto{\pgfqpoint{1.346373in}{1.059756in}}%
\pgfpathlineto{\pgfqpoint{1.346581in}{1.036108in}}%
\pgfpathlineto{\pgfqpoint{1.346684in}{0.906384in}}%
\pgfpathlineto{\pgfqpoint{1.347203in}{1.161150in}}%
\pgfpathlineto{\pgfqpoint{1.347721in}{0.997981in}}%
\pgfpathlineto{\pgfqpoint{1.347928in}{1.002060in}}%
\pgfpathlineto{\pgfqpoint{1.348964in}{1.233690in}}%
\pgfpathlineto{\pgfqpoint{1.348446in}{0.910905in}}%
\pgfpathlineto{\pgfqpoint{1.349068in}{1.187239in}}%
\pgfpathlineto{\pgfqpoint{1.350001in}{0.891680in}}%
\pgfpathlineto{\pgfqpoint{1.349379in}{1.204900in}}%
\pgfpathlineto{\pgfqpoint{1.350312in}{0.992322in}}%
\pgfpathlineto{\pgfqpoint{1.350934in}{1.191669in}}%
\pgfpathlineto{\pgfqpoint{1.351245in}{0.951398in}}%
\pgfpathlineto{\pgfqpoint{1.351348in}{0.986765in}}%
\pgfpathlineto{\pgfqpoint{1.351763in}{1.053619in}}%
\pgfpathlineto{\pgfqpoint{1.351866in}{1.009360in}}%
\pgfpathlineto{\pgfqpoint{1.352177in}{0.805306in}}%
\pgfpathlineto{\pgfqpoint{1.352592in}{1.027016in}}%
\pgfpathlineto{\pgfqpoint{1.352903in}{0.968143in}}%
\pgfpathlineto{\pgfqpoint{1.353006in}{1.099518in}}%
\pgfpathlineto{\pgfqpoint{1.353939in}{0.887228in}}%
\pgfpathlineto{\pgfqpoint{1.354354in}{1.084582in}}%
\pgfpathlineto{\pgfqpoint{1.355079in}{0.986804in}}%
\pgfpathlineto{\pgfqpoint{1.355390in}{1.068557in}}%
\pgfpathlineto{\pgfqpoint{1.356219in}{0.800982in}}%
\pgfpathlineto{\pgfqpoint{1.356634in}{1.017261in}}%
\pgfpathlineto{\pgfqpoint{1.357359in}{0.874753in}}%
\pgfpathlineto{\pgfqpoint{1.359225in}{1.110553in}}%
\pgfpathlineto{\pgfqpoint{1.357981in}{0.859650in}}%
\pgfpathlineto{\pgfqpoint{1.359432in}{0.982787in}}%
\pgfpathlineto{\pgfqpoint{1.359847in}{1.029405in}}%
\pgfpathlineto{\pgfqpoint{1.360261in}{0.862759in}}%
\pgfpathlineto{\pgfqpoint{1.360572in}{0.859092in}}%
\pgfpathlineto{\pgfqpoint{1.361505in}{1.137949in}}%
\pgfpathlineto{\pgfqpoint{1.362127in}{0.883974in}}%
\pgfpathlineto{\pgfqpoint{1.362645in}{1.061153in}}%
\pgfpathlineto{\pgfqpoint{1.362749in}{1.190148in}}%
\pgfpathlineto{\pgfqpoint{1.362956in}{0.975455in}}%
\pgfpathlineto{\pgfqpoint{1.363578in}{0.983399in}}%
\pgfpathlineto{\pgfqpoint{1.363889in}{0.850586in}}%
\pgfpathlineto{\pgfqpoint{1.364096in}{1.154176in}}%
\pgfpathlineto{\pgfqpoint{1.364614in}{1.006665in}}%
\pgfpathlineto{\pgfqpoint{1.365340in}{1.118697in}}%
\pgfpathlineto{\pgfqpoint{1.364821in}{0.953822in}}%
\pgfpathlineto{\pgfqpoint{1.365650in}{0.993152in}}%
\pgfpathlineto{\pgfqpoint{1.366791in}{0.821841in}}%
\pgfpathlineto{\pgfqpoint{1.365858in}{1.036597in}}%
\pgfpathlineto{\pgfqpoint{1.366998in}{0.874452in}}%
\pgfpathlineto{\pgfqpoint{1.367931in}{1.244766in}}%
\pgfpathlineto{\pgfqpoint{1.368345in}{1.143736in}}%
\pgfpathlineto{\pgfqpoint{1.368967in}{0.954466in}}%
\pgfpathlineto{\pgfqpoint{1.369382in}{1.157260in}}%
\pgfpathlineto{\pgfqpoint{1.369589in}{0.996518in}}%
\pgfpathlineto{\pgfqpoint{1.370211in}{1.220143in}}%
\pgfpathlineto{\pgfqpoint{1.370936in}{1.166417in}}%
\pgfpathlineto{\pgfqpoint{1.371869in}{0.916350in}}%
\pgfpathlineto{\pgfqpoint{1.372076in}{0.965194in}}%
\pgfpathlineto{\pgfqpoint{1.372594in}{0.800861in}}%
\pgfpathlineto{\pgfqpoint{1.372491in}{1.023040in}}%
\pgfpathlineto{\pgfqpoint{1.372698in}{0.974987in}}%
\pgfpathlineto{\pgfqpoint{1.372802in}{1.065432in}}%
\pgfpathlineto{\pgfqpoint{1.373527in}{0.815734in}}%
\pgfpathlineto{\pgfqpoint{1.373631in}{0.981859in}}%
\pgfpathlineto{\pgfqpoint{1.374045in}{0.856753in}}%
\pgfpathlineto{\pgfqpoint{1.374667in}{1.023269in}}%
\pgfpathlineto{\pgfqpoint{1.375393in}{1.241488in}}%
\pgfpathlineto{\pgfqpoint{1.374978in}{0.939042in}}%
\pgfpathlineto{\pgfqpoint{1.375807in}{1.130807in}}%
\pgfpathlineto{\pgfqpoint{1.376118in}{1.138214in}}%
\pgfpathlineto{\pgfqpoint{1.377051in}{0.901320in}}%
\pgfpathlineto{\pgfqpoint{1.378191in}{1.151180in}}%
\pgfpathlineto{\pgfqpoint{1.379124in}{1.036568in}}%
\pgfpathlineto{\pgfqpoint{1.378916in}{1.236974in}}%
\pgfpathlineto{\pgfqpoint{1.379331in}{1.072813in}}%
\pgfpathlineto{\pgfqpoint{1.379435in}{1.074023in}}%
\pgfpathlineto{\pgfqpoint{1.379642in}{0.925071in}}%
\pgfpathlineto{\pgfqpoint{1.379849in}{1.161875in}}%
\pgfpathlineto{\pgfqpoint{1.380264in}{1.146129in}}%
\pgfpathlineto{\pgfqpoint{1.380678in}{1.063434in}}%
\pgfpathlineto{\pgfqpoint{1.381404in}{1.212626in}}%
\pgfpathlineto{\pgfqpoint{1.381922in}{0.958148in}}%
\pgfpathlineto{\pgfqpoint{1.382855in}{1.043528in}}%
\pgfpathlineto{\pgfqpoint{1.383269in}{0.997137in}}%
\pgfpathlineto{\pgfqpoint{1.383891in}{1.278969in}}%
\pgfpathlineto{\pgfqpoint{1.384928in}{0.968235in}}%
\pgfpathlineto{\pgfqpoint{1.385031in}{1.063310in}}%
\pgfpathlineto{\pgfqpoint{1.386378in}{0.875743in}}%
\pgfpathlineto{\pgfqpoint{1.386689in}{1.233765in}}%
\pgfpathlineto{\pgfqpoint{1.387726in}{1.112579in}}%
\pgfpathlineto{\pgfqpoint{1.387933in}{1.124314in}}%
\pgfpathlineto{\pgfqpoint{1.388969in}{0.875403in}}%
\pgfpathlineto{\pgfqpoint{1.390213in}{1.138826in}}%
\pgfpathlineto{\pgfqpoint{1.390317in}{0.923979in}}%
\pgfpathlineto{\pgfqpoint{1.391353in}{1.077601in}}%
\pgfpathlineto{\pgfqpoint{1.391560in}{1.070548in}}%
\pgfpathlineto{\pgfqpoint{1.391664in}{1.111375in}}%
\pgfpathlineto{\pgfqpoint{1.392182in}{0.947254in}}%
\pgfpathlineto{\pgfqpoint{1.392286in}{0.952584in}}%
\pgfpathlineto{\pgfqpoint{1.392390in}{0.945386in}}%
\pgfpathlineto{\pgfqpoint{1.392493in}{1.119621in}}%
\pgfpathlineto{\pgfqpoint{1.393322in}{0.865170in}}%
\pgfpathlineto{\pgfqpoint{1.393426in}{0.959593in}}%
\pgfpathlineto{\pgfqpoint{1.393633in}{0.899793in}}%
\pgfpathlineto{\pgfqpoint{1.393944in}{1.175285in}}%
\pgfpathlineto{\pgfqpoint{1.394255in}{0.961796in}}%
\pgfpathlineto{\pgfqpoint{1.394462in}{1.146071in}}%
\pgfpathlineto{\pgfqpoint{1.394670in}{0.896236in}}%
\pgfpathlineto{\pgfqpoint{1.395292in}{1.017542in}}%
\pgfpathlineto{\pgfqpoint{1.396639in}{0.795299in}}%
\pgfpathlineto{\pgfqpoint{1.395499in}{1.061722in}}%
\pgfpathlineto{\pgfqpoint{1.396742in}{0.829056in}}%
\pgfpathlineto{\pgfqpoint{1.398090in}{1.197798in}}%
\pgfpathlineto{\pgfqpoint{1.398712in}{1.087080in}}%
\pgfpathlineto{\pgfqpoint{1.399748in}{0.892437in}}%
\pgfpathlineto{\pgfqpoint{1.400059in}{0.903298in}}%
\pgfpathlineto{\pgfqpoint{1.400370in}{1.047661in}}%
\pgfpathlineto{\pgfqpoint{1.400681in}{0.783726in}}%
\pgfpathlineto{\pgfqpoint{1.400888in}{0.859699in}}%
\pgfpathlineto{\pgfqpoint{1.400992in}{0.795777in}}%
\pgfpathlineto{\pgfqpoint{1.401510in}{1.041478in}}%
\pgfpathlineto{\pgfqpoint{1.401717in}{0.899971in}}%
\pgfpathlineto{\pgfqpoint{1.402546in}{1.061313in}}%
\pgfpathlineto{\pgfqpoint{1.402754in}{0.875078in}}%
\pgfpathlineto{\pgfqpoint{1.402857in}{0.858352in}}%
\pgfpathlineto{\pgfqpoint{1.402961in}{0.977260in}}%
\pgfpathlineto{\pgfqpoint{1.403272in}{0.929252in}}%
\pgfpathlineto{\pgfqpoint{1.403375in}{0.982243in}}%
\pgfpathlineto{\pgfqpoint{1.403894in}{0.870782in}}%
\pgfpathlineto{\pgfqpoint{1.404205in}{0.931026in}}%
\pgfpathlineto{\pgfqpoint{1.404308in}{0.676778in}}%
\pgfpathlineto{\pgfqpoint{1.404930in}{0.982924in}}%
\pgfpathlineto{\pgfqpoint{1.405345in}{0.850278in}}%
\pgfpathlineto{\pgfqpoint{1.405552in}{0.810008in}}%
\pgfpathlineto{\pgfqpoint{1.405656in}{1.036411in}}%
\pgfpathlineto{\pgfqpoint{1.405759in}{1.015890in}}%
\pgfpathlineto{\pgfqpoint{1.406692in}{1.134333in}}%
\pgfpathlineto{\pgfqpoint{1.406070in}{0.890858in}}%
\pgfpathlineto{\pgfqpoint{1.406796in}{1.032303in}}%
\pgfpathlineto{\pgfqpoint{1.407314in}{0.809275in}}%
\pgfpathlineto{\pgfqpoint{1.407106in}{1.070847in}}%
\pgfpathlineto{\pgfqpoint{1.407936in}{0.810327in}}%
\pgfpathlineto{\pgfqpoint{1.408972in}{1.073943in}}%
\pgfpathlineto{\pgfqpoint{1.409076in}{0.873330in}}%
\pgfpathlineto{\pgfqpoint{1.409905in}{1.060014in}}%
\pgfpathlineto{\pgfqpoint{1.410216in}{0.968359in}}%
\pgfpathlineto{\pgfqpoint{1.410319in}{0.970406in}}%
\pgfpathlineto{\pgfqpoint{1.410423in}{1.025584in}}%
\pgfpathlineto{\pgfqpoint{1.410527in}{0.862632in}}%
\pgfpathlineto{\pgfqpoint{1.411045in}{0.894167in}}%
\pgfpathlineto{\pgfqpoint{1.411148in}{0.666173in}}%
\pgfpathlineto{\pgfqpoint{1.411978in}{0.999948in}}%
\pgfpathlineto{\pgfqpoint{1.412081in}{0.926617in}}%
\pgfpathlineto{\pgfqpoint{1.413118in}{1.098860in}}%
\pgfpathlineto{\pgfqpoint{1.412599in}{0.908319in}}%
\pgfpathlineto{\pgfqpoint{1.413532in}{1.085344in}}%
\pgfpathlineto{\pgfqpoint{1.414672in}{0.886070in}}%
\pgfpathlineto{\pgfqpoint{1.414776in}{0.915979in}}%
\pgfpathlineto{\pgfqpoint{1.416020in}{1.121012in}}%
\pgfpathlineto{\pgfqpoint{1.416330in}{0.844060in}}%
\pgfpathlineto{\pgfqpoint{1.417160in}{0.878806in}}%
\pgfpathlineto{\pgfqpoint{1.417263in}{1.121643in}}%
\pgfpathlineto{\pgfqpoint{1.418300in}{1.112078in}}%
\pgfpathlineto{\pgfqpoint{1.418611in}{1.176026in}}%
\pgfpathlineto{\pgfqpoint{1.419336in}{1.009538in}}%
\pgfpathlineto{\pgfqpoint{1.419647in}{0.965723in}}%
\pgfpathlineto{\pgfqpoint{1.420372in}{1.273305in}}%
\pgfpathlineto{\pgfqpoint{1.420476in}{1.056965in}}%
\pgfpathlineto{\pgfqpoint{1.421512in}{1.105089in}}%
\pgfpathlineto{\pgfqpoint{1.421616in}{1.106869in}}%
\pgfpathlineto{\pgfqpoint{1.421720in}{1.047787in}}%
\pgfpathlineto{\pgfqpoint{1.422134in}{1.286260in}}%
\pgfpathlineto{\pgfqpoint{1.422652in}{1.117294in}}%
\pgfpathlineto{\pgfqpoint{1.422756in}{1.195902in}}%
\pgfpathlineto{\pgfqpoint{1.423482in}{1.031118in}}%
\pgfpathlineto{\pgfqpoint{1.423689in}{1.073354in}}%
\pgfpathlineto{\pgfqpoint{1.424103in}{0.936362in}}%
\pgfpathlineto{\pgfqpoint{1.424622in}{1.132647in}}%
\pgfpathlineto{\pgfqpoint{1.424725in}{1.188546in}}%
\pgfpathlineto{\pgfqpoint{1.424829in}{0.987464in}}%
\pgfpathlineto{\pgfqpoint{1.425451in}{1.059278in}}%
\pgfpathlineto{\pgfqpoint{1.425658in}{0.971254in}}%
\pgfpathlineto{\pgfqpoint{1.426384in}{1.119789in}}%
\pgfpathlineto{\pgfqpoint{1.426487in}{1.164390in}}%
\pgfpathlineto{\pgfqpoint{1.426902in}{1.006498in}}%
\pgfpathlineto{\pgfqpoint{1.427316in}{1.070593in}}%
\pgfpathlineto{\pgfqpoint{1.427627in}{1.094353in}}%
\pgfpathlineto{\pgfqpoint{1.427524in}{1.021062in}}%
\pgfpathlineto{\pgfqpoint{1.428042in}{1.041361in}}%
\pgfpathlineto{\pgfqpoint{1.428145in}{1.016020in}}%
\pgfpathlineto{\pgfqpoint{1.428456in}{1.153882in}}%
\pgfpathlineto{\pgfqpoint{1.428560in}{1.202031in}}%
\pgfpathlineto{\pgfqpoint{1.428975in}{1.040728in}}%
\pgfpathlineto{\pgfqpoint{1.429389in}{1.144712in}}%
\pgfpathlineto{\pgfqpoint{1.429907in}{0.801748in}}%
\pgfpathlineto{\pgfqpoint{1.430633in}{0.906895in}}%
\pgfpathlineto{\pgfqpoint{1.431566in}{1.324607in}}%
\pgfpathlineto{\pgfqpoint{1.431876in}{1.156922in}}%
\pgfpathlineto{\pgfqpoint{1.432187in}{1.032188in}}%
\pgfpathlineto{\pgfqpoint{1.432809in}{1.199611in}}%
\pgfpathlineto{\pgfqpoint{1.433016in}{1.120637in}}%
\pgfpathlineto{\pgfqpoint{1.434364in}{0.882973in}}%
\pgfpathlineto{\pgfqpoint{1.433327in}{1.171722in}}%
\pgfpathlineto{\pgfqpoint{1.434571in}{0.894150in}}%
\pgfpathlineto{\pgfqpoint{1.435400in}{1.093541in}}%
\pgfpathlineto{\pgfqpoint{1.435711in}{1.045789in}}%
\pgfpathlineto{\pgfqpoint{1.435815in}{1.051204in}}%
\pgfpathlineto{\pgfqpoint{1.436022in}{1.008707in}}%
\pgfpathlineto{\pgfqpoint{1.436126in}{0.978630in}}%
\pgfpathlineto{\pgfqpoint{1.436437in}{1.157181in}}%
\pgfpathlineto{\pgfqpoint{1.436748in}{1.073533in}}%
\pgfpathlineto{\pgfqpoint{1.437680in}{1.170985in}}%
\pgfpathlineto{\pgfqpoint{1.437266in}{0.968979in}}%
\pgfpathlineto{\pgfqpoint{1.437888in}{1.157670in}}%
\pgfpathlineto{\pgfqpoint{1.437991in}{0.969522in}}%
\pgfpathlineto{\pgfqpoint{1.438198in}{1.232593in}}%
\pgfpathlineto{\pgfqpoint{1.438924in}{1.007548in}}%
\pgfpathlineto{\pgfqpoint{1.439753in}{0.989145in}}%
\pgfpathlineto{\pgfqpoint{1.440168in}{1.284770in}}%
\pgfpathlineto{\pgfqpoint{1.440997in}{1.095811in}}%
\pgfpathlineto{\pgfqpoint{1.441308in}{1.097080in}}%
\pgfpathlineto{\pgfqpoint{1.441515in}{1.258970in}}%
\pgfpathlineto{\pgfqpoint{1.442137in}{1.078292in}}%
\pgfpathlineto{\pgfqpoint{1.442344in}{1.142652in}}%
\pgfpathlineto{\pgfqpoint{1.442551in}{1.007951in}}%
\pgfpathlineto{\pgfqpoint{1.442862in}{1.186260in}}%
\pgfpathlineto{\pgfqpoint{1.443380in}{1.108020in}}%
\pgfpathlineto{\pgfqpoint{1.444521in}{1.366990in}}%
\pgfpathlineto{\pgfqpoint{1.444728in}{1.211922in}}%
\pgfpathlineto{\pgfqpoint{1.444831in}{1.200639in}}%
\pgfpathlineto{\pgfqpoint{1.444935in}{1.248306in}}%
\pgfpathlineto{\pgfqpoint{1.445350in}{1.341865in}}%
\pgfpathlineto{\pgfqpoint{1.445971in}{1.230180in}}%
\pgfpathlineto{\pgfqpoint{1.447112in}{0.875924in}}%
\pgfpathlineto{\pgfqpoint{1.446490in}{1.328883in}}%
\pgfpathlineto{\pgfqpoint{1.447733in}{0.990480in}}%
\pgfpathlineto{\pgfqpoint{1.448252in}{1.255169in}}%
\pgfpathlineto{\pgfqpoint{1.448873in}{1.124925in}}%
\pgfpathlineto{\pgfqpoint{1.449910in}{0.968618in}}%
\pgfpathlineto{\pgfqpoint{1.449288in}{1.201420in}}%
\pgfpathlineto{\pgfqpoint{1.450117in}{0.990378in}}%
\pgfpathlineto{\pgfqpoint{1.450221in}{0.983302in}}%
\pgfpathlineto{\pgfqpoint{1.450324in}{1.046255in}}%
\pgfpathlineto{\pgfqpoint{1.450428in}{1.048071in}}%
\pgfpathlineto{\pgfqpoint{1.451050in}{0.943115in}}%
\pgfpathlineto{\pgfqpoint{1.451153in}{1.065955in}}%
\pgfpathlineto{\pgfqpoint{1.451361in}{1.022847in}}%
\pgfpathlineto{\pgfqpoint{1.451983in}{1.152220in}}%
\pgfpathlineto{\pgfqpoint{1.452294in}{0.919251in}}%
\pgfpathlineto{\pgfqpoint{1.452397in}{1.007111in}}%
\pgfpathlineto{\pgfqpoint{1.452604in}{1.106540in}}%
\pgfpathlineto{\pgfqpoint{1.452708in}{1.098655in}}%
\pgfpathlineto{\pgfqpoint{1.452812in}{1.172827in}}%
\pgfpathlineto{\pgfqpoint{1.453537in}{0.940645in}}%
\pgfpathlineto{\pgfqpoint{1.453744in}{1.048887in}}%
\pgfpathlineto{\pgfqpoint{1.453952in}{0.957491in}}%
\pgfpathlineto{\pgfqpoint{1.454055in}{1.059014in}}%
\pgfpathlineto{\pgfqpoint{1.454366in}{1.028692in}}%
\pgfpathlineto{\pgfqpoint{1.454677in}{1.227572in}}%
\pgfpathlineto{\pgfqpoint{1.455403in}{0.908930in}}%
\pgfpathlineto{\pgfqpoint{1.456335in}{0.824460in}}%
\pgfpathlineto{\pgfqpoint{1.455817in}{1.071952in}}%
\pgfpathlineto{\pgfqpoint{1.456543in}{0.879113in}}%
\pgfpathlineto{\pgfqpoint{1.456646in}{0.838316in}}%
\pgfpathlineto{\pgfqpoint{1.457061in}{1.013507in}}%
\pgfpathlineto{\pgfqpoint{1.457476in}{0.943267in}}%
\pgfpathlineto{\pgfqpoint{1.458408in}{1.142705in}}%
\pgfpathlineto{\pgfqpoint{1.458616in}{0.998832in}}%
\pgfpathlineto{\pgfqpoint{1.459341in}{0.847891in}}%
\pgfpathlineto{\pgfqpoint{1.458823in}{1.021224in}}%
\pgfpathlineto{\pgfqpoint{1.459652in}{0.964433in}}%
\pgfpathlineto{\pgfqpoint{1.459859in}{1.105004in}}%
\pgfpathlineto{\pgfqpoint{1.460481in}{0.886136in}}%
\pgfpathlineto{\pgfqpoint{1.460792in}{0.991572in}}%
\pgfpathlineto{\pgfqpoint{1.461414in}{1.053321in}}%
\pgfpathlineto{\pgfqpoint{1.461621in}{0.893449in}}%
\pgfpathlineto{\pgfqpoint{1.461828in}{1.188682in}}%
\pgfpathlineto{\pgfqpoint{1.462761in}{1.108451in}}%
\pgfpathlineto{\pgfqpoint{1.462968in}{1.225770in}}%
\pgfpathlineto{\pgfqpoint{1.463279in}{0.907123in}}%
\pgfpathlineto{\pgfqpoint{1.463694in}{1.056588in}}%
\pgfpathlineto{\pgfqpoint{1.463798in}{0.963708in}}%
\pgfpathlineto{\pgfqpoint{1.464316in}{1.147193in}}%
\pgfpathlineto{\pgfqpoint{1.464730in}{1.090390in}}%
\pgfpathlineto{\pgfqpoint{1.465145in}{1.337853in}}%
\pgfpathlineto{\pgfqpoint{1.465352in}{0.942822in}}%
\pgfpathlineto{\pgfqpoint{1.465663in}{1.040006in}}%
\pgfpathlineto{\pgfqpoint{1.465767in}{1.042103in}}%
\pgfpathlineto{\pgfqpoint{1.465870in}{1.031905in}}%
\pgfpathlineto{\pgfqpoint{1.466181in}{0.951553in}}%
\pgfpathlineto{\pgfqpoint{1.467010in}{1.128666in}}%
\pgfpathlineto{\pgfqpoint{1.467425in}{0.855758in}}%
\pgfpathlineto{\pgfqpoint{1.468047in}{1.001053in}}%
\pgfpathlineto{\pgfqpoint{1.468150in}{1.167261in}}%
\pgfpathlineto{\pgfqpoint{1.468669in}{0.871485in}}%
\pgfpathlineto{\pgfqpoint{1.469083in}{1.101206in}}%
\pgfpathlineto{\pgfqpoint{1.470016in}{0.958322in}}%
\pgfpathlineto{\pgfqpoint{1.469290in}{1.133406in}}%
\pgfpathlineto{\pgfqpoint{1.470223in}{0.989940in}}%
\pgfpathlineto{\pgfqpoint{1.470327in}{1.241447in}}%
\pgfpathlineto{\pgfqpoint{1.470845in}{0.924384in}}%
\pgfpathlineto{\pgfqpoint{1.471363in}{1.060448in}}%
\pgfpathlineto{\pgfqpoint{1.471674in}{1.168193in}}%
\pgfpathlineto{\pgfqpoint{1.471571in}{1.035831in}}%
\pgfpathlineto{\pgfqpoint{1.471881in}{1.084545in}}%
\pgfpathlineto{\pgfqpoint{1.473022in}{0.926186in}}%
\pgfpathlineto{\pgfqpoint{1.473851in}{1.169165in}}%
\pgfpathlineto{\pgfqpoint{1.473954in}{0.924578in}}%
\pgfpathlineto{\pgfqpoint{1.474265in}{1.051770in}}%
\pgfpathlineto{\pgfqpoint{1.474576in}{0.866389in}}%
\pgfpathlineto{\pgfqpoint{1.474991in}{1.217794in}}%
\pgfpathlineto{\pgfqpoint{1.475094in}{1.192712in}}%
\pgfpathlineto{\pgfqpoint{1.475302in}{1.290314in}}%
\pgfpathlineto{\pgfqpoint{1.475509in}{1.173435in}}%
\pgfpathlineto{\pgfqpoint{1.475820in}{1.183592in}}%
\pgfpathlineto{\pgfqpoint{1.476442in}{1.251914in}}%
\pgfpathlineto{\pgfqpoint{1.476753in}{1.074764in}}%
\pgfpathlineto{\pgfqpoint{1.476856in}{1.289702in}}%
\pgfpathlineto{\pgfqpoint{1.477789in}{0.995287in}}%
\pgfpathlineto{\pgfqpoint{1.477893in}{0.975116in}}%
\pgfpathlineto{\pgfqpoint{1.477996in}{1.161913in}}%
\pgfpathlineto{\pgfqpoint{1.478307in}{1.269836in}}%
\pgfpathlineto{\pgfqpoint{1.478825in}{1.133995in}}%
\pgfpathlineto{\pgfqpoint{1.479240in}{1.258682in}}%
\pgfpathlineto{\pgfqpoint{1.480276in}{1.013406in}}%
\pgfpathlineto{\pgfqpoint{1.480069in}{1.308257in}}%
\pgfpathlineto{\pgfqpoint{1.480691in}{1.155003in}}%
\pgfpathlineto{\pgfqpoint{1.481313in}{1.293561in}}%
\pgfpathlineto{\pgfqpoint{1.480898in}{1.070439in}}%
\pgfpathlineto{\pgfqpoint{1.481727in}{1.118787in}}%
\pgfpathlineto{\pgfqpoint{1.482142in}{1.061800in}}%
\pgfpathlineto{\pgfqpoint{1.482038in}{1.165619in}}%
\pgfpathlineto{\pgfqpoint{1.482453in}{1.158006in}}%
\pgfpathlineto{\pgfqpoint{1.482764in}{1.274297in}}%
\pgfpathlineto{\pgfqpoint{1.483075in}{1.111506in}}%
\pgfpathlineto{\pgfqpoint{1.483178in}{1.155529in}}%
\pgfpathlineto{\pgfqpoint{1.483696in}{0.978065in}}%
\pgfpathlineto{\pgfqpoint{1.484215in}{1.104123in}}%
\pgfpathlineto{\pgfqpoint{1.485251in}{1.202983in}}%
\pgfpathlineto{\pgfqpoint{1.484940in}{1.030196in}}%
\pgfpathlineto{\pgfqpoint{1.485355in}{1.126899in}}%
\pgfpathlineto{\pgfqpoint{1.485458in}{1.068707in}}%
\pgfpathlineto{\pgfqpoint{1.486184in}{1.237459in}}%
\pgfpathlineto{\pgfqpoint{1.486287in}{1.234132in}}%
\pgfpathlineto{\pgfqpoint{1.486598in}{1.055515in}}%
\pgfpathlineto{\pgfqpoint{1.486495in}{1.271886in}}%
\pgfpathlineto{\pgfqpoint{1.487635in}{1.124927in}}%
\pgfpathlineto{\pgfqpoint{1.488464in}{1.050122in}}%
\pgfpathlineto{\pgfqpoint{1.488257in}{1.169394in}}%
\pgfpathlineto{\pgfqpoint{1.488671in}{1.108590in}}%
\pgfpathlineto{\pgfqpoint{1.488775in}{1.165489in}}%
\pgfpathlineto{\pgfqpoint{1.488982in}{0.976124in}}%
\pgfpathlineto{\pgfqpoint{1.489604in}{1.042814in}}%
\pgfpathlineto{\pgfqpoint{1.490329in}{1.159491in}}%
\pgfpathlineto{\pgfqpoint{1.490537in}{1.008796in}}%
\pgfpathlineto{\pgfqpoint{1.490640in}{1.064662in}}%
\pgfpathlineto{\pgfqpoint{1.490951in}{1.226236in}}%
\pgfpathlineto{\pgfqpoint{1.491677in}{0.959215in}}%
\pgfpathlineto{\pgfqpoint{1.492195in}{1.157149in}}%
\pgfpathlineto{\pgfqpoint{1.492402in}{0.924182in}}%
\pgfpathlineto{\pgfqpoint{1.492817in}{1.024853in}}%
\pgfpathlineto{\pgfqpoint{1.493024in}{1.159069in}}%
\pgfpathlineto{\pgfqpoint{1.493231in}{1.015871in}}%
\pgfpathlineto{\pgfqpoint{1.493957in}{1.060117in}}%
\pgfpathlineto{\pgfqpoint{1.494060in}{0.952748in}}%
\pgfpathlineto{\pgfqpoint{1.494786in}{1.187874in}}%
\pgfpathlineto{\pgfqpoint{1.494890in}{1.176892in}}%
\pgfpathlineto{\pgfqpoint{1.495822in}{0.998852in}}%
\pgfpathlineto{\pgfqpoint{1.496030in}{1.111431in}}%
\pgfpathlineto{\pgfqpoint{1.496444in}{1.274306in}}%
\pgfpathlineto{\pgfqpoint{1.496237in}{1.007102in}}%
\pgfpathlineto{\pgfqpoint{1.497066in}{1.177148in}}%
\pgfpathlineto{\pgfqpoint{1.498102in}{1.064054in}}%
\pgfpathlineto{\pgfqpoint{1.498413in}{1.189424in}}%
\pgfpathlineto{\pgfqpoint{1.499035in}{0.957372in}}%
\pgfpathlineto{\pgfqpoint{1.499139in}{0.990190in}}%
\pgfpathlineto{\pgfqpoint{1.500279in}{1.257293in}}%
\pgfpathlineto{\pgfqpoint{1.499657in}{0.936003in}}%
\pgfpathlineto{\pgfqpoint{1.500590in}{1.162032in}}%
\pgfpathlineto{\pgfqpoint{1.501315in}{1.234913in}}%
\pgfpathlineto{\pgfqpoint{1.501730in}{1.040220in}}%
\pgfpathlineto{\pgfqpoint{1.502352in}{1.265395in}}%
\pgfpathlineto{\pgfqpoint{1.502974in}{1.222935in}}%
\pgfpathlineto{\pgfqpoint{1.503077in}{1.038629in}}%
\pgfpathlineto{\pgfqpoint{1.504114in}{1.166177in}}%
\pgfpathlineto{\pgfqpoint{1.504217in}{1.187072in}}%
\pgfpathlineto{\pgfqpoint{1.504424in}{1.093523in}}%
\pgfpathlineto{\pgfqpoint{1.504528in}{0.967568in}}%
\pgfpathlineto{\pgfqpoint{1.505357in}{1.219788in}}%
\pgfpathlineto{\pgfqpoint{1.505565in}{1.017421in}}%
\pgfpathlineto{\pgfqpoint{1.505979in}{1.183535in}}%
\pgfpathlineto{\pgfqpoint{1.505875in}{1.001664in}}%
\pgfpathlineto{\pgfqpoint{1.506808in}{1.075085in}}%
\pgfpathlineto{\pgfqpoint{1.506912in}{1.035252in}}%
\pgfpathlineto{\pgfqpoint{1.507326in}{1.245596in}}%
\pgfpathlineto{\pgfqpoint{1.507430in}{1.096469in}}%
\pgfpathlineto{\pgfqpoint{1.507534in}{1.249035in}}%
\pgfpathlineto{\pgfqpoint{1.507637in}{1.018377in}}%
\pgfpathlineto{\pgfqpoint{1.508363in}{1.034603in}}%
\pgfpathlineto{\pgfqpoint{1.508466in}{0.965902in}}%
\pgfpathlineto{\pgfqpoint{1.508777in}{1.204600in}}%
\pgfpathlineto{\pgfqpoint{1.509192in}{1.194793in}}%
\pgfpathlineto{\pgfqpoint{1.509296in}{1.223034in}}%
\pgfpathlineto{\pgfqpoint{1.509814in}{1.140157in}}%
\pgfpathlineto{\pgfqpoint{1.510436in}{1.056109in}}%
\pgfpathlineto{\pgfqpoint{1.510228in}{1.239662in}}%
\pgfpathlineto{\pgfqpoint{1.510850in}{1.085825in}}%
\pgfpathlineto{\pgfqpoint{1.510954in}{1.216466in}}%
\pgfpathlineto{\pgfqpoint{1.511576in}{0.937981in}}%
\pgfpathlineto{\pgfqpoint{1.511783in}{1.059919in}}%
\pgfpathlineto{\pgfqpoint{1.511990in}{0.958107in}}%
\pgfpathlineto{\pgfqpoint{1.512716in}{1.161137in}}%
\pgfpathlineto{\pgfqpoint{1.513130in}{1.084599in}}%
\pgfpathlineto{\pgfqpoint{1.513027in}{1.263569in}}%
\pgfpathlineto{\pgfqpoint{1.513856in}{1.138246in}}%
\pgfpathlineto{\pgfqpoint{1.514167in}{1.293530in}}%
\pgfpathlineto{\pgfqpoint{1.514892in}{1.249971in}}%
\pgfpathlineto{\pgfqpoint{1.515307in}{1.022220in}}%
\pgfpathlineto{\pgfqpoint{1.515929in}{1.312923in}}%
\pgfpathlineto{\pgfqpoint{1.516758in}{1.387017in}}%
\pgfpathlineto{\pgfqpoint{1.517069in}{1.130063in}}%
\pgfpathlineto{\pgfqpoint{1.518001in}{1.334611in}}%
\pgfpathlineto{\pgfqpoint{1.517379in}{1.096083in}}%
\pgfpathlineto{\pgfqpoint{1.518209in}{1.218213in}}%
\pgfpathlineto{\pgfqpoint{1.518416in}{1.118555in}}%
\pgfpathlineto{\pgfqpoint{1.518830in}{1.257656in}}%
\pgfpathlineto{\pgfqpoint{1.519349in}{1.197968in}}%
\pgfpathlineto{\pgfqpoint{1.519867in}{1.398942in}}%
\pgfpathlineto{\pgfqpoint{1.519763in}{1.171012in}}%
\pgfpathlineto{\pgfqpoint{1.520489in}{1.249663in}}%
\pgfpathlineto{\pgfqpoint{1.520696in}{1.058236in}}%
\pgfpathlineto{\pgfqpoint{1.521525in}{1.190454in}}%
\pgfpathlineto{\pgfqpoint{1.521732in}{1.346005in}}%
\pgfpathlineto{\pgfqpoint{1.522147in}{1.060838in}}%
\pgfpathlineto{\pgfqpoint{1.522561in}{1.229243in}}%
\pgfpathlineto{\pgfqpoint{1.523598in}{1.112190in}}%
\pgfpathlineto{\pgfqpoint{1.523391in}{1.303698in}}%
\pgfpathlineto{\pgfqpoint{1.523702in}{1.194317in}}%
\pgfpathlineto{\pgfqpoint{1.523909in}{1.148312in}}%
\pgfpathlineto{\pgfqpoint{1.524323in}{1.250617in}}%
\pgfpathlineto{\pgfqpoint{1.524427in}{1.194026in}}%
\pgfpathlineto{\pgfqpoint{1.524842in}{1.325467in}}%
\pgfpathlineto{\pgfqpoint{1.524945in}{1.173012in}}%
\pgfpathlineto{\pgfqpoint{1.525567in}{1.274885in}}%
\pgfpathlineto{\pgfqpoint{1.525671in}{1.324111in}}%
\pgfpathlineto{\pgfqpoint{1.525878in}{1.134833in}}%
\pgfpathlineto{\pgfqpoint{1.526396in}{1.174562in}}%
\pgfpathlineto{\pgfqpoint{1.527018in}{0.977831in}}%
\pgfpathlineto{\pgfqpoint{1.527433in}{1.236035in}}%
\pgfpathlineto{\pgfqpoint{1.527536in}{1.131833in}}%
\pgfpathlineto{\pgfqpoint{1.528054in}{1.029683in}}%
\pgfpathlineto{\pgfqpoint{1.527951in}{1.186511in}}%
\pgfpathlineto{\pgfqpoint{1.528573in}{1.154730in}}%
\pgfpathlineto{\pgfqpoint{1.529713in}{1.279666in}}%
\pgfpathlineto{\pgfqpoint{1.529194in}{1.153728in}}%
\pgfpathlineto{\pgfqpoint{1.529816in}{1.245111in}}%
\pgfpathlineto{\pgfqpoint{1.530853in}{1.073370in}}%
\pgfpathlineto{\pgfqpoint{1.530024in}{1.247985in}}%
\pgfpathlineto{\pgfqpoint{1.530956in}{1.164470in}}%
\pgfpathlineto{\pgfqpoint{1.531578in}{1.194133in}}%
\pgfpathlineto{\pgfqpoint{1.531371in}{1.041813in}}%
\pgfpathlineto{\pgfqpoint{1.531889in}{1.159297in}}%
\pgfpathlineto{\pgfqpoint{1.532511in}{1.098917in}}%
\pgfpathlineto{\pgfqpoint{1.532096in}{1.174500in}}%
\pgfpathlineto{\pgfqpoint{1.532615in}{1.163636in}}%
\pgfpathlineto{\pgfqpoint{1.532718in}{1.274596in}}%
\pgfpathlineto{\pgfqpoint{1.533029in}{0.967163in}}%
\pgfpathlineto{\pgfqpoint{1.533547in}{0.977819in}}%
\pgfpathlineto{\pgfqpoint{1.533651in}{0.911945in}}%
\pgfpathlineto{\pgfqpoint{1.534273in}{1.113637in}}%
\pgfpathlineto{\pgfqpoint{1.534480in}{1.025072in}}%
\pgfpathlineto{\pgfqpoint{1.534791in}{1.090261in}}%
\pgfpathlineto{\pgfqpoint{1.534998in}{0.949128in}}%
\pgfpathlineto{\pgfqpoint{1.535413in}{1.017546in}}%
\pgfpathlineto{\pgfqpoint{1.535827in}{0.866519in}}%
\pgfpathlineto{\pgfqpoint{1.536242in}{1.122557in}}%
\pgfpathlineto{\pgfqpoint{1.536346in}{1.099715in}}%
\pgfpathlineto{\pgfqpoint{1.536449in}{1.098931in}}%
\pgfpathlineto{\pgfqpoint{1.537278in}{0.931529in}}%
\pgfpathlineto{\pgfqpoint{1.537382in}{1.124420in}}%
\pgfpathlineto{\pgfqpoint{1.537486in}{1.107679in}}%
\pgfpathlineto{\pgfqpoint{1.537589in}{1.102733in}}%
\pgfpathlineto{\pgfqpoint{1.538626in}{0.950082in}}%
\pgfpathlineto{\pgfqpoint{1.538729in}{1.041430in}}%
\pgfpathlineto{\pgfqpoint{1.539558in}{1.168136in}}%
\pgfpathlineto{\pgfqpoint{1.539040in}{0.972866in}}%
\pgfpathlineto{\pgfqpoint{1.539662in}{1.071967in}}%
\pgfpathlineto{\pgfqpoint{1.539973in}{1.171522in}}%
\pgfpathlineto{\pgfqpoint{1.540802in}{0.948287in}}%
\pgfpathlineto{\pgfqpoint{1.542149in}{1.244532in}}%
\pgfpathlineto{\pgfqpoint{1.542253in}{1.157044in}}%
\pgfpathlineto{\pgfqpoint{1.542357in}{1.162882in}}%
\pgfpathlineto{\pgfqpoint{1.543289in}{1.007819in}}%
\pgfpathlineto{\pgfqpoint{1.543497in}{1.019221in}}%
\pgfpathlineto{\pgfqpoint{1.543704in}{1.002988in}}%
\pgfpathlineto{\pgfqpoint{1.544533in}{1.117058in}}%
\pgfpathlineto{\pgfqpoint{1.544637in}{0.988842in}}%
\pgfpathlineto{\pgfqpoint{1.545570in}{1.205960in}}%
\pgfpathlineto{\pgfqpoint{1.545673in}{1.101253in}}%
\pgfpathlineto{\pgfqpoint{1.546606in}{1.150182in}}%
\pgfpathlineto{\pgfqpoint{1.547331in}{1.269254in}}%
\pgfpathlineto{\pgfqpoint{1.547539in}{1.073538in}}%
\pgfpathlineto{\pgfqpoint{1.547642in}{1.125254in}}%
\pgfpathlineto{\pgfqpoint{1.548368in}{1.292327in}}%
\pgfpathlineto{\pgfqpoint{1.548575in}{1.107117in}}%
\pgfpathlineto{\pgfqpoint{1.548679in}{1.148039in}}%
\pgfpathlineto{\pgfqpoint{1.549922in}{0.951799in}}%
\pgfpathlineto{\pgfqpoint{1.549197in}{1.248899in}}%
\pgfpathlineto{\pgfqpoint{1.550026in}{0.972616in}}%
\pgfpathlineto{\pgfqpoint{1.551062in}{1.189109in}}%
\pgfpathlineto{\pgfqpoint{1.551581in}{1.147715in}}%
\pgfpathlineto{\pgfqpoint{1.552203in}{1.212119in}}%
\pgfpathlineto{\pgfqpoint{1.552721in}{1.046461in}}%
\pgfpathlineto{\pgfqpoint{1.553446in}{1.167918in}}%
\pgfpathlineto{\pgfqpoint{1.552928in}{0.999440in}}%
\pgfpathlineto{\pgfqpoint{1.553757in}{1.081749in}}%
\pgfpathlineto{\pgfqpoint{1.554275in}{0.947340in}}%
\pgfpathlineto{\pgfqpoint{1.554690in}{1.092693in}}%
\pgfpathlineto{\pgfqpoint{1.554794in}{1.027043in}}%
\pgfpathlineto{\pgfqpoint{1.556037in}{1.294812in}}%
\pgfpathlineto{\pgfqpoint{1.555104in}{1.001350in}}%
\pgfpathlineto{\pgfqpoint{1.556141in}{1.197257in}}%
\pgfpathlineto{\pgfqpoint{1.556452in}{0.983598in}}%
\pgfpathlineto{\pgfqpoint{1.556970in}{1.218706in}}%
\pgfpathlineto{\pgfqpoint{1.557074in}{1.139830in}}%
\pgfpathlineto{\pgfqpoint{1.557281in}{1.343321in}}%
\pgfpathlineto{\pgfqpoint{1.557799in}{1.103274in}}%
\pgfpathlineto{\pgfqpoint{1.558110in}{1.231658in}}%
\pgfpathlineto{\pgfqpoint{1.558317in}{1.071659in}}%
\pgfpathlineto{\pgfqpoint{1.558939in}{1.267784in}}%
\pgfpathlineto{\pgfqpoint{1.559250in}{1.200600in}}%
\pgfpathlineto{\pgfqpoint{1.559354in}{1.206468in}}%
\pgfpathlineto{\pgfqpoint{1.560597in}{1.380791in}}%
\pgfpathlineto{\pgfqpoint{1.560805in}{1.125792in}}%
\pgfpathlineto{\pgfqpoint{1.561634in}{1.294510in}}%
\pgfpathlineto{\pgfqpoint{1.561841in}{1.422732in}}%
\pgfpathlineto{\pgfqpoint{1.562152in}{1.202253in}}%
\pgfpathlineto{\pgfqpoint{1.562463in}{1.292970in}}%
\pgfpathlineto{\pgfqpoint{1.563292in}{1.089668in}}%
\pgfpathlineto{\pgfqpoint{1.563707in}{1.169381in}}%
\pgfpathlineto{\pgfqpoint{1.565261in}{1.348556in}}%
\pgfpathlineto{\pgfqpoint{1.566194in}{1.138200in}}%
\pgfpathlineto{\pgfqpoint{1.566401in}{1.192998in}}%
\pgfpathlineto{\pgfqpoint{1.566505in}{1.286266in}}%
\pgfpathlineto{\pgfqpoint{1.566712in}{1.142750in}}%
\pgfpathlineto{\pgfqpoint{1.567438in}{1.165763in}}%
\pgfpathlineto{\pgfqpoint{1.567852in}{1.254038in}}%
\pgfpathlineto{\pgfqpoint{1.568474in}{1.012940in}}%
\pgfpathlineto{\pgfqpoint{1.568681in}{1.243220in}}%
\pgfpathlineto{\pgfqpoint{1.569510in}{0.967829in}}%
\pgfpathlineto{\pgfqpoint{1.569614in}{1.111831in}}%
\pgfpathlineto{\pgfqpoint{1.569718in}{0.970504in}}%
\pgfpathlineto{\pgfqpoint{1.570547in}{1.168781in}}%
\pgfpathlineto{\pgfqpoint{1.570650in}{1.215460in}}%
\pgfpathlineto{\pgfqpoint{1.571376in}{1.047956in}}%
\pgfpathlineto{\pgfqpoint{1.571894in}{0.932473in}}%
\pgfpathlineto{\pgfqpoint{1.572205in}{1.083117in}}%
\pgfpathlineto{\pgfqpoint{1.572412in}{1.063832in}}%
\pgfpathlineto{\pgfqpoint{1.572516in}{1.151207in}}%
\pgfpathlineto{\pgfqpoint{1.573034in}{0.933797in}}%
\pgfpathlineto{\pgfqpoint{1.573345in}{1.039216in}}%
\pgfpathlineto{\pgfqpoint{1.574174in}{0.939542in}}%
\pgfpathlineto{\pgfqpoint{1.574278in}{1.097614in}}%
\pgfpathlineto{\pgfqpoint{1.574485in}{0.945033in}}%
\pgfpathlineto{\pgfqpoint{1.575003in}{1.151813in}}%
\pgfpathlineto{\pgfqpoint{1.574900in}{0.910267in}}%
\pgfpathlineto{\pgfqpoint{1.575729in}{1.055831in}}%
\pgfpathlineto{\pgfqpoint{1.575936in}{0.902470in}}%
\pgfpathlineto{\pgfqpoint{1.576351in}{1.175613in}}%
\pgfpathlineto{\pgfqpoint{1.576454in}{1.052211in}}%
\pgfpathlineto{\pgfqpoint{1.576558in}{1.142425in}}%
\pgfpathlineto{\pgfqpoint{1.576765in}{0.947743in}}%
\pgfpathlineto{\pgfqpoint{1.577594in}{1.084742in}}%
\pgfpathlineto{\pgfqpoint{1.577698in}{1.080656in}}%
\pgfpathlineto{\pgfqpoint{1.577802in}{0.897712in}}%
\pgfpathlineto{\pgfqpoint{1.578734in}{1.166760in}}%
\pgfpathlineto{\pgfqpoint{1.579667in}{0.949849in}}%
\pgfpathlineto{\pgfqpoint{1.579874in}{1.116963in}}%
\pgfpathlineto{\pgfqpoint{1.580393in}{1.265746in}}%
\pgfpathlineto{\pgfqpoint{1.580807in}{1.136673in}}%
\pgfpathlineto{\pgfqpoint{1.581844in}{0.878674in}}%
\pgfpathlineto{\pgfqpoint{1.582051in}{0.927518in}}%
\pgfpathlineto{\pgfqpoint{1.583087in}{1.172352in}}%
\pgfpathlineto{\pgfqpoint{1.583295in}{1.104045in}}%
\pgfpathlineto{\pgfqpoint{1.583398in}{1.099304in}}%
\pgfpathlineto{\pgfqpoint{1.583502in}{1.030523in}}%
\pgfpathlineto{\pgfqpoint{1.584331in}{1.175862in}}%
\pgfpathlineto{\pgfqpoint{1.584435in}{1.122610in}}%
\pgfpathlineto{\pgfqpoint{1.584642in}{1.164162in}}%
\pgfpathlineto{\pgfqpoint{1.584745in}{1.072543in}}%
\pgfpathlineto{\pgfqpoint{1.584849in}{1.110841in}}%
\pgfpathlineto{\pgfqpoint{1.584953in}{0.959994in}}%
\pgfpathlineto{\pgfqpoint{1.585678in}{1.251033in}}%
\pgfpathlineto{\pgfqpoint{1.585886in}{1.177443in}}%
\pgfpathlineto{\pgfqpoint{1.585989in}{1.238556in}}%
\pgfpathlineto{\pgfqpoint{1.586611in}{1.083500in}}%
\pgfpathlineto{\pgfqpoint{1.586818in}{1.086979in}}%
\pgfpathlineto{\pgfqpoint{1.586922in}{1.131541in}}%
\pgfpathlineto{\pgfqpoint{1.587233in}{0.925465in}}%
\pgfpathlineto{\pgfqpoint{1.587647in}{1.073824in}}%
\pgfpathlineto{\pgfqpoint{1.587958in}{0.940756in}}%
\pgfpathlineto{\pgfqpoint{1.588580in}{1.152518in}}%
\pgfpathlineto{\pgfqpoint{1.588684in}{1.106207in}}%
\pgfpathlineto{\pgfqpoint{1.588787in}{1.203324in}}%
\pgfpathlineto{\pgfqpoint{1.589513in}{1.087034in}}%
\pgfpathlineto{\pgfqpoint{1.589720in}{1.176790in}}%
\pgfpathlineto{\pgfqpoint{1.589927in}{1.015075in}}%
\pgfpathlineto{\pgfqpoint{1.590238in}{1.225285in}}%
\pgfpathlineto{\pgfqpoint{1.590860in}{1.061538in}}%
\pgfpathlineto{\pgfqpoint{1.591378in}{1.156742in}}%
\pgfpathlineto{\pgfqpoint{1.591275in}{0.995230in}}%
\pgfpathlineto{\pgfqpoint{1.592000in}{1.138504in}}%
\pgfpathlineto{\pgfqpoint{1.592208in}{1.183481in}}%
\pgfpathlineto{\pgfqpoint{1.592518in}{1.059243in}}%
\pgfpathlineto{\pgfqpoint{1.592622in}{0.948729in}}%
\pgfpathlineto{\pgfqpoint{1.593244in}{1.177739in}}%
\pgfpathlineto{\pgfqpoint{1.593555in}{1.077437in}}%
\pgfpathlineto{\pgfqpoint{1.593659in}{1.140678in}}%
\pgfpathlineto{\pgfqpoint{1.594488in}{0.975526in}}%
\pgfpathlineto{\pgfqpoint{1.594695in}{0.983726in}}%
\pgfpathlineto{\pgfqpoint{1.595731in}{1.149849in}}%
\pgfpathlineto{\pgfqpoint{1.595939in}{1.121356in}}%
\pgfpathlineto{\pgfqpoint{1.596664in}{0.994894in}}%
\pgfpathlineto{\pgfqpoint{1.596560in}{1.165047in}}%
\pgfpathlineto{\pgfqpoint{1.596975in}{1.073760in}}%
\pgfpathlineto{\pgfqpoint{1.597079in}{1.174872in}}%
\pgfpathlineto{\pgfqpoint{1.597390in}{0.891572in}}%
\pgfpathlineto{\pgfqpoint{1.598011in}{1.147884in}}%
\pgfpathlineto{\pgfqpoint{1.598426in}{0.959024in}}%
\pgfpathlineto{\pgfqpoint{1.599255in}{1.057786in}}%
\pgfpathlineto{\pgfqpoint{1.599566in}{0.970231in}}%
\pgfpathlineto{\pgfqpoint{1.600602in}{1.239222in}}%
\pgfpathlineto{\pgfqpoint{1.601121in}{1.015923in}}%
\pgfpathlineto{\pgfqpoint{1.601742in}{1.191146in}}%
\pgfpathlineto{\pgfqpoint{1.602468in}{1.379203in}}%
\pgfpathlineto{\pgfqpoint{1.602053in}{1.119403in}}%
\pgfpathlineto{\pgfqpoint{1.602675in}{1.164346in}}%
\pgfpathlineto{\pgfqpoint{1.603297in}{1.268546in}}%
\pgfpathlineto{\pgfqpoint{1.603815in}{1.090083in}}%
\pgfpathlineto{\pgfqpoint{1.604023in}{1.071485in}}%
\pgfpathlineto{\pgfqpoint{1.604955in}{1.221300in}}%
\pgfpathlineto{\pgfqpoint{1.605163in}{1.012773in}}%
\pgfpathlineto{\pgfqpoint{1.605784in}{1.250071in}}%
\pgfpathlineto{\pgfqpoint{1.606199in}{1.100922in}}%
\pgfpathlineto{\pgfqpoint{1.606406in}{1.052152in}}%
\pgfpathlineto{\pgfqpoint{1.606510in}{1.123454in}}%
\pgfpathlineto{\pgfqpoint{1.607132in}{1.290074in}}%
\pgfpathlineto{\pgfqpoint{1.606924in}{1.091575in}}%
\pgfpathlineto{\pgfqpoint{1.607546in}{1.104132in}}%
\pgfpathlineto{\pgfqpoint{1.608583in}{0.932187in}}%
\pgfpathlineto{\pgfqpoint{1.608375in}{1.130154in}}%
\pgfpathlineto{\pgfqpoint{1.608686in}{1.014889in}}%
\pgfpathlineto{\pgfqpoint{1.609723in}{1.183131in}}%
\pgfpathlineto{\pgfqpoint{1.609205in}{0.926291in}}%
\pgfpathlineto{\pgfqpoint{1.609826in}{1.141869in}}%
\pgfpathlineto{\pgfqpoint{1.610448in}{0.930915in}}%
\pgfpathlineto{\pgfqpoint{1.610863in}{1.122788in}}%
\pgfpathlineto{\pgfqpoint{1.610966in}{1.201853in}}%
\pgfpathlineto{\pgfqpoint{1.611070in}{0.951340in}}%
\pgfpathlineto{\pgfqpoint{1.611692in}{1.048518in}}%
\pgfpathlineto{\pgfqpoint{1.611796in}{0.933253in}}%
\pgfpathlineto{\pgfqpoint{1.612106in}{1.100861in}}%
\pgfpathlineto{\pgfqpoint{1.612728in}{1.055762in}}%
\pgfpathlineto{\pgfqpoint{1.613557in}{1.285936in}}%
\pgfpathlineto{\pgfqpoint{1.613039in}{0.991230in}}%
\pgfpathlineto{\pgfqpoint{1.613868in}{1.097289in}}%
\pgfpathlineto{\pgfqpoint{1.614283in}{1.175676in}}%
\pgfpathlineto{\pgfqpoint{1.614076in}{1.016353in}}%
\pgfpathlineto{\pgfqpoint{1.614490in}{1.082153in}}%
\pgfpathlineto{\pgfqpoint{1.614594in}{0.973971in}}%
\pgfpathlineto{\pgfqpoint{1.615527in}{1.108614in}}%
\pgfpathlineto{\pgfqpoint{1.616148in}{1.271379in}}%
\pgfpathlineto{\pgfqpoint{1.616356in}{1.086780in}}%
\pgfpathlineto{\pgfqpoint{1.616667in}{1.172939in}}%
\pgfpathlineto{\pgfqpoint{1.617185in}{1.222422in}}%
\pgfpathlineto{\pgfqpoint{1.617081in}{1.118439in}}%
\pgfpathlineto{\pgfqpoint{1.617392in}{1.139357in}}%
\pgfpathlineto{\pgfqpoint{1.618014in}{0.999227in}}%
\pgfpathlineto{\pgfqpoint{1.618221in}{1.200577in}}%
\pgfpathlineto{\pgfqpoint{1.618325in}{1.081547in}}%
\pgfpathlineto{\pgfqpoint{1.618428in}{1.215750in}}%
\pgfpathlineto{\pgfqpoint{1.619361in}{0.994587in}}%
\pgfpathlineto{\pgfqpoint{1.619465in}{0.950689in}}%
\pgfpathlineto{\pgfqpoint{1.619776in}{1.092759in}}%
\pgfpathlineto{\pgfqpoint{1.619879in}{1.060187in}}%
\pgfpathlineto{\pgfqpoint{1.619983in}{1.244651in}}%
\pgfpathlineto{\pgfqpoint{1.620709in}{1.055945in}}%
\pgfpathlineto{\pgfqpoint{1.621019in}{1.152565in}}%
\pgfpathlineto{\pgfqpoint{1.622056in}{1.011326in}}%
\pgfpathlineto{\pgfqpoint{1.621538in}{1.231595in}}%
\pgfpathlineto{\pgfqpoint{1.622367in}{1.077660in}}%
\pgfpathlineto{\pgfqpoint{1.622470in}{1.099454in}}%
\pgfpathlineto{\pgfqpoint{1.622574in}{1.019648in}}%
\pgfpathlineto{\pgfqpoint{1.623196in}{1.081615in}}%
\pgfpathlineto{\pgfqpoint{1.623921in}{0.953643in}}%
\pgfpathlineto{\pgfqpoint{1.624129in}{1.143695in}}%
\pgfpathlineto{\pgfqpoint{1.624232in}{1.012904in}}%
\pgfpathlineto{\pgfqpoint{1.624751in}{1.140197in}}%
\pgfpathlineto{\pgfqpoint{1.625165in}{0.944645in}}%
\pgfpathlineto{\pgfqpoint{1.625372in}{1.069873in}}%
\pgfpathlineto{\pgfqpoint{1.625476in}{1.103037in}}%
\pgfpathlineto{\pgfqpoint{1.626201in}{0.985188in}}%
\pgfpathlineto{\pgfqpoint{1.626305in}{1.046536in}}%
\pgfpathlineto{\pgfqpoint{1.627134in}{0.947343in}}%
\pgfpathlineto{\pgfqpoint{1.626616in}{1.128292in}}%
\pgfpathlineto{\pgfqpoint{1.627342in}{1.036602in}}%
\pgfpathlineto{\pgfqpoint{1.627549in}{1.019550in}}%
\pgfpathlineto{\pgfqpoint{1.627860in}{1.113462in}}%
\pgfpathlineto{\pgfqpoint{1.628067in}{0.749224in}}%
\pgfpathlineto{\pgfqpoint{1.628896in}{1.077696in}}%
\pgfpathlineto{\pgfqpoint{1.629103in}{1.022261in}}%
\pgfpathlineto{\pgfqpoint{1.629207in}{1.106102in}}%
\pgfpathlineto{\pgfqpoint{1.629518in}{0.934311in}}%
\pgfpathlineto{\pgfqpoint{1.629829in}{1.135654in}}%
\pgfpathlineto{\pgfqpoint{1.630347in}{1.060904in}}%
\pgfpathlineto{\pgfqpoint{1.631073in}{0.970038in}}%
\pgfpathlineto{\pgfqpoint{1.630658in}{1.089082in}}%
\pgfpathlineto{\pgfqpoint{1.631280in}{1.086746in}}%
\pgfpathlineto{\pgfqpoint{1.631383in}{1.225099in}}%
\pgfpathlineto{\pgfqpoint{1.632005in}{0.957283in}}%
\pgfpathlineto{\pgfqpoint{1.632213in}{0.988019in}}%
\pgfpathlineto{\pgfqpoint{1.632316in}{0.967016in}}%
\pgfpathlineto{\pgfqpoint{1.632524in}{1.102363in}}%
\pgfpathlineto{\pgfqpoint{1.633042in}{0.986544in}}%
\pgfpathlineto{\pgfqpoint{1.633353in}{1.102398in}}%
\pgfpathlineto{\pgfqpoint{1.633871in}{0.873946in}}%
\pgfpathlineto{\pgfqpoint{1.634078in}{0.986044in}}%
\pgfpathlineto{\pgfqpoint{1.634389in}{0.890179in}}%
\pgfpathlineto{\pgfqpoint{1.634804in}{1.098044in}}%
\pgfpathlineto{\pgfqpoint{1.635011in}{1.156111in}}%
\pgfpathlineto{\pgfqpoint{1.635218in}{0.966983in}}%
\pgfpathlineto{\pgfqpoint{1.635633in}{0.930686in}}%
\pgfpathlineto{\pgfqpoint{1.635425in}{1.071936in}}%
\pgfpathlineto{\pgfqpoint{1.636151in}{0.951384in}}%
\pgfpathlineto{\pgfqpoint{1.637291in}{1.144869in}}%
\pgfpathlineto{\pgfqpoint{1.637395in}{1.116288in}}%
\pgfpathlineto{\pgfqpoint{1.637498in}{1.111789in}}%
\pgfpathlineto{\pgfqpoint{1.637602in}{1.119614in}}%
\pgfpathlineto{\pgfqpoint{1.638535in}{0.972199in}}%
\pgfpathlineto{\pgfqpoint{1.638638in}{0.991447in}}%
\pgfpathlineto{\pgfqpoint{1.639053in}{1.194810in}}%
\pgfpathlineto{\pgfqpoint{1.639260in}{0.985390in}}%
\pgfpathlineto{\pgfqpoint{1.639778in}{1.060094in}}%
\pgfpathlineto{\pgfqpoint{1.640607in}{1.120527in}}%
\pgfpathlineto{\pgfqpoint{1.640400in}{0.964026in}}%
\pgfpathlineto{\pgfqpoint{1.640815in}{1.047206in}}%
\pgfpathlineto{\pgfqpoint{1.640918in}{1.043902in}}%
\pgfpathlineto{\pgfqpoint{1.641126in}{0.993470in}}%
\pgfpathlineto{\pgfqpoint{1.641540in}{1.163687in}}%
\pgfpathlineto{\pgfqpoint{1.641851in}{1.065495in}}%
\pgfpathlineto{\pgfqpoint{1.642058in}{1.017177in}}%
\pgfpathlineto{\pgfqpoint{1.643198in}{1.275941in}}%
\pgfpathlineto{\pgfqpoint{1.643820in}{1.019183in}}%
\pgfpathlineto{\pgfqpoint{1.644442in}{1.126768in}}%
\pgfpathlineto{\pgfqpoint{1.644753in}{1.036907in}}%
\pgfpathlineto{\pgfqpoint{1.645375in}{1.158617in}}%
\pgfpathlineto{\pgfqpoint{1.646308in}{0.932238in}}%
\pgfpathlineto{\pgfqpoint{1.645997in}{1.280922in}}%
\pgfpathlineto{\pgfqpoint{1.646515in}{1.007701in}}%
\pgfpathlineto{\pgfqpoint{1.646722in}{1.145383in}}%
\pgfpathlineto{\pgfqpoint{1.647655in}{1.118671in}}%
\pgfpathlineto{\pgfqpoint{1.647862in}{1.006021in}}%
\pgfpathlineto{\pgfqpoint{1.648173in}{1.174270in}}%
\pgfpathlineto{\pgfqpoint{1.648795in}{1.010020in}}%
\pgfpathlineto{\pgfqpoint{1.649417in}{1.245045in}}%
\pgfpathlineto{\pgfqpoint{1.649728in}{0.974863in}}%
\pgfpathlineto{\pgfqpoint{1.649831in}{1.010447in}}%
\pgfpathlineto{\pgfqpoint{1.649935in}{1.007927in}}%
\pgfpathlineto{\pgfqpoint{1.650142in}{1.082884in}}%
\pgfpathlineto{\pgfqpoint{1.650557in}{0.906788in}}%
\pgfpathlineto{\pgfqpoint{1.651075in}{1.076624in}}%
\pgfpathlineto{\pgfqpoint{1.651179in}{0.951252in}}%
\pgfpathlineto{\pgfqpoint{1.651386in}{1.141638in}}%
\pgfpathlineto{\pgfqpoint{1.652215in}{1.014769in}}%
\pgfpathlineto{\pgfqpoint{1.653044in}{1.137696in}}%
\pgfpathlineto{\pgfqpoint{1.652733in}{0.978496in}}%
\pgfpathlineto{\pgfqpoint{1.653355in}{1.090461in}}%
\pgfpathlineto{\pgfqpoint{1.653562in}{0.992001in}}%
\pgfpathlineto{\pgfqpoint{1.653873in}{1.150405in}}%
\pgfpathlineto{\pgfqpoint{1.654495in}{1.269093in}}%
\pgfpathlineto{\pgfqpoint{1.654702in}{1.172392in}}%
\pgfpathlineto{\pgfqpoint{1.655428in}{1.001424in}}%
\pgfpathlineto{\pgfqpoint{1.655117in}{1.201002in}}%
\pgfpathlineto{\pgfqpoint{1.655843in}{1.103663in}}%
\pgfpathlineto{\pgfqpoint{1.655946in}{1.226108in}}%
\pgfpathlineto{\pgfqpoint{1.656464in}{1.060666in}}%
\pgfpathlineto{\pgfqpoint{1.656983in}{1.158827in}}%
\pgfpathlineto{\pgfqpoint{1.658019in}{0.931814in}}%
\pgfpathlineto{\pgfqpoint{1.658123in}{0.996986in}}%
\pgfpathlineto{\pgfqpoint{1.658434in}{1.183461in}}%
\pgfpathlineto{\pgfqpoint{1.659159in}{1.030399in}}%
\pgfpathlineto{\pgfqpoint{1.659263in}{0.970892in}}%
\pgfpathlineto{\pgfqpoint{1.659574in}{1.137727in}}%
\pgfpathlineto{\pgfqpoint{1.660092in}{0.977063in}}%
\pgfpathlineto{\pgfqpoint{1.661128in}{1.170128in}}%
\pgfpathlineto{\pgfqpoint{1.660921in}{0.975538in}}%
\pgfpathlineto{\pgfqpoint{1.661232in}{1.087753in}}%
\pgfpathlineto{\pgfqpoint{1.661439in}{0.879817in}}%
\pgfpathlineto{\pgfqpoint{1.662268in}{1.130885in}}%
\pgfpathlineto{\pgfqpoint{1.662683in}{0.940889in}}%
\pgfpathlineto{\pgfqpoint{1.663305in}{1.014981in}}%
\pgfpathlineto{\pgfqpoint{1.664341in}{1.188545in}}%
\pgfpathlineto{\pgfqpoint{1.664237in}{0.974084in}}%
\pgfpathlineto{\pgfqpoint{1.664445in}{1.080090in}}%
\pgfpathlineto{\pgfqpoint{1.664548in}{1.128912in}}%
\pgfpathlineto{\pgfqpoint{1.665066in}{0.873248in}}%
\pgfpathlineto{\pgfqpoint{1.665170in}{0.977032in}}%
\pgfpathlineto{\pgfqpoint{1.666103in}{0.781182in}}%
\pgfpathlineto{\pgfqpoint{1.665377in}{0.987742in}}%
\pgfpathlineto{\pgfqpoint{1.666414in}{0.879225in}}%
\pgfpathlineto{\pgfqpoint{1.666932in}{1.006443in}}%
\pgfpathlineto{\pgfqpoint{1.667243in}{0.817508in}}%
\pgfpathlineto{\pgfqpoint{1.667450in}{0.894333in}}%
\pgfpathlineto{\pgfqpoint{1.667761in}{0.791392in}}%
\pgfpathlineto{\pgfqpoint{1.667968in}{0.950537in}}%
\pgfpathlineto{\pgfqpoint{1.668279in}{0.915213in}}%
\pgfpathlineto{\pgfqpoint{1.668383in}{0.994927in}}%
\pgfpathlineto{\pgfqpoint{1.668901in}{0.755325in}}%
\pgfpathlineto{\pgfqpoint{1.669316in}{0.877111in}}%
\pgfpathlineto{\pgfqpoint{1.669419in}{0.875862in}}%
\pgfpathlineto{\pgfqpoint{1.670456in}{1.119484in}}%
\pgfpathlineto{\pgfqpoint{1.670559in}{0.968907in}}%
\pgfpathlineto{\pgfqpoint{1.671078in}{1.219150in}}%
\pgfpathlineto{\pgfqpoint{1.671803in}{1.112462in}}%
\pgfpathlineto{\pgfqpoint{1.672010in}{1.041436in}}%
\pgfpathlineto{\pgfqpoint{1.672425in}{1.187723in}}%
\pgfpathlineto{\pgfqpoint{1.672736in}{1.134547in}}%
\pgfpathlineto{\pgfqpoint{1.673254in}{1.247539in}}%
\pgfpathlineto{\pgfqpoint{1.673358in}{1.076915in}}%
\pgfpathlineto{\pgfqpoint{1.673461in}{1.107717in}}%
\pgfpathlineto{\pgfqpoint{1.674394in}{0.867458in}}%
\pgfpathlineto{\pgfqpoint{1.673876in}{1.136723in}}%
\pgfpathlineto{\pgfqpoint{1.674601in}{1.062546in}}%
\pgfpathlineto{\pgfqpoint{1.675223in}{1.139544in}}%
\pgfpathlineto{\pgfqpoint{1.674912in}{0.930417in}}%
\pgfpathlineto{\pgfqpoint{1.675534in}{1.031174in}}%
\pgfpathlineto{\pgfqpoint{1.676156in}{0.868709in}}%
\pgfpathlineto{\pgfqpoint{1.675741in}{1.088876in}}%
\pgfpathlineto{\pgfqpoint{1.676674in}{1.014207in}}%
\pgfpathlineto{\pgfqpoint{1.677192in}{1.143281in}}%
\pgfpathlineto{\pgfqpoint{1.677400in}{0.961409in}}%
\pgfpathlineto{\pgfqpoint{1.677711in}{1.062995in}}%
\pgfpathlineto{\pgfqpoint{1.678332in}{0.957163in}}%
\pgfpathlineto{\pgfqpoint{1.678021in}{1.106705in}}%
\pgfpathlineto{\pgfqpoint{1.678851in}{1.023852in}}%
\pgfpathlineto{\pgfqpoint{1.679265in}{1.111900in}}%
\pgfpathlineto{\pgfqpoint{1.679887in}{1.047814in}}%
\pgfpathlineto{\pgfqpoint{1.679991in}{1.029183in}}%
\pgfpathlineto{\pgfqpoint{1.680198in}{1.179167in}}%
\pgfpathlineto{\pgfqpoint{1.680509in}{1.114866in}}%
\pgfpathlineto{\pgfqpoint{1.681131in}{1.199387in}}%
\pgfpathlineto{\pgfqpoint{1.681649in}{1.136276in}}%
\pgfpathlineto{\pgfqpoint{1.682582in}{0.862473in}}%
\pgfpathlineto{\pgfqpoint{1.682789in}{1.103942in}}%
\pgfpathlineto{\pgfqpoint{1.682893in}{1.097094in}}%
\pgfpathlineto{\pgfqpoint{1.683203in}{1.210957in}}%
\pgfpathlineto{\pgfqpoint{1.683514in}{1.012773in}}%
\pgfpathlineto{\pgfqpoint{1.683929in}{1.119581in}}%
\pgfpathlineto{\pgfqpoint{1.684758in}{1.016108in}}%
\pgfpathlineto{\pgfqpoint{1.684654in}{1.179377in}}%
\pgfpathlineto{\pgfqpoint{1.684862in}{1.137948in}}%
\pgfpathlineto{\pgfqpoint{1.684965in}{1.247971in}}%
\pgfpathlineto{\pgfqpoint{1.685587in}{1.066764in}}%
\pgfpathlineto{\pgfqpoint{1.686002in}{1.217168in}}%
\pgfpathlineto{\pgfqpoint{1.687245in}{0.955277in}}%
\pgfpathlineto{\pgfqpoint{1.686313in}{1.242876in}}%
\pgfpathlineto{\pgfqpoint{1.687349in}{0.985558in}}%
\pgfpathlineto{\pgfqpoint{1.687453in}{0.961307in}}%
\pgfpathlineto{\pgfqpoint{1.687764in}{1.091209in}}%
\pgfpathlineto{\pgfqpoint{1.688696in}{1.250527in}}%
\pgfpathlineto{\pgfqpoint{1.688489in}{1.003602in}}%
\pgfpathlineto{\pgfqpoint{1.688800in}{1.196115in}}%
\pgfpathlineto{\pgfqpoint{1.690044in}{0.952831in}}%
\pgfpathlineto{\pgfqpoint{1.691287in}{1.250826in}}%
\pgfpathlineto{\pgfqpoint{1.692324in}{0.965759in}}%
\pgfpathlineto{\pgfqpoint{1.692946in}{1.061510in}}%
\pgfpathlineto{\pgfqpoint{1.693049in}{1.183571in}}%
\pgfpathlineto{\pgfqpoint{1.693257in}{0.987423in}}%
\pgfpathlineto{\pgfqpoint{1.693982in}{1.130857in}}%
\pgfpathlineto{\pgfqpoint{1.694708in}{1.046052in}}%
\pgfpathlineto{\pgfqpoint{1.694189in}{1.166590in}}%
\pgfpathlineto{\pgfqpoint{1.695018in}{1.161780in}}%
\pgfpathlineto{\pgfqpoint{1.695848in}{1.080293in}}%
\pgfpathlineto{\pgfqpoint{1.695433in}{1.205833in}}%
\pgfpathlineto{\pgfqpoint{1.696159in}{1.090809in}}%
\pgfpathlineto{\pgfqpoint{1.697091in}{1.028864in}}%
\pgfpathlineto{\pgfqpoint{1.697299in}{1.186376in}}%
\pgfpathlineto{\pgfqpoint{1.697817in}{1.042430in}}%
\pgfpathlineto{\pgfqpoint{1.698439in}{1.057082in}}%
\pgfpathlineto{\pgfqpoint{1.699268in}{1.244500in}}%
\pgfpathlineto{\pgfqpoint{1.698646in}{1.036040in}}%
\pgfpathlineto{\pgfqpoint{1.699579in}{1.092081in}}%
\pgfpathlineto{\pgfqpoint{1.700200in}{0.978329in}}%
\pgfpathlineto{\pgfqpoint{1.700408in}{1.155386in}}%
\pgfpathlineto{\pgfqpoint{1.700511in}{1.176591in}}%
\pgfpathlineto{\pgfqpoint{1.700926in}{1.041997in}}%
\pgfpathlineto{\pgfqpoint{1.701030in}{1.099584in}}%
\pgfpathlineto{\pgfqpoint{1.701237in}{0.985333in}}%
\pgfpathlineto{\pgfqpoint{1.701651in}{1.218988in}}%
\pgfpathlineto{\pgfqpoint{1.702066in}{1.159180in}}%
\pgfpathlineto{\pgfqpoint{1.703310in}{0.918965in}}%
\pgfpathlineto{\pgfqpoint{1.702377in}{1.185208in}}%
\pgfpathlineto{\pgfqpoint{1.703621in}{1.065905in}}%
\pgfpathlineto{\pgfqpoint{1.703828in}{1.201887in}}%
\pgfpathlineto{\pgfqpoint{1.704450in}{0.997017in}}%
\pgfpathlineto{\pgfqpoint{1.704657in}{1.010889in}}%
\pgfpathlineto{\pgfqpoint{1.705072in}{0.949332in}}%
\pgfpathlineto{\pgfqpoint{1.705279in}{1.091796in}}%
\pgfpathlineto{\pgfqpoint{1.705382in}{0.988371in}}%
\pgfpathlineto{\pgfqpoint{1.705693in}{1.164072in}}%
\pgfpathlineto{\pgfqpoint{1.706004in}{0.950109in}}%
\pgfpathlineto{\pgfqpoint{1.706419in}{1.003596in}}%
\pgfpathlineto{\pgfqpoint{1.706833in}{1.157157in}}%
\pgfpathlineto{\pgfqpoint{1.707352in}{0.942645in}}%
\pgfpathlineto{\pgfqpoint{1.707766in}{1.085080in}}%
\pgfpathlineto{\pgfqpoint{1.708181in}{0.823341in}}%
\pgfpathlineto{\pgfqpoint{1.708492in}{0.998469in}}%
\pgfpathlineto{\pgfqpoint{1.708803in}{1.058872in}}%
\pgfpathlineto{\pgfqpoint{1.709217in}{0.835371in}}%
\pgfpathlineto{\pgfqpoint{1.709321in}{0.974642in}}%
\pgfpathlineto{\pgfqpoint{1.709424in}{0.859451in}}%
\pgfpathlineto{\pgfqpoint{1.709839in}{1.044396in}}%
\pgfpathlineto{\pgfqpoint{1.710254in}{0.996096in}}%
\pgfpathlineto{\pgfqpoint{1.710875in}{1.159703in}}%
\pgfpathlineto{\pgfqpoint{1.711186in}{0.929692in}}%
\pgfpathlineto{\pgfqpoint{1.711290in}{1.006925in}}%
\pgfpathlineto{\pgfqpoint{1.712119in}{1.157342in}}%
\pgfpathlineto{\pgfqpoint{1.712534in}{0.861109in}}%
\pgfpathlineto{\pgfqpoint{1.713777in}{1.105349in}}%
\pgfpathlineto{\pgfqpoint{1.714606in}{0.888626in}}%
\pgfpathlineto{\pgfqpoint{1.714088in}{1.153025in}}%
\pgfpathlineto{\pgfqpoint{1.715125in}{0.994208in}}%
\pgfpathlineto{\pgfqpoint{1.715954in}{1.143042in}}%
\pgfpathlineto{\pgfqpoint{1.716057in}{0.967706in}}%
\pgfpathlineto{\pgfqpoint{1.716265in}{1.046665in}}%
\pgfpathlineto{\pgfqpoint{1.717197in}{1.259583in}}%
\pgfpathlineto{\pgfqpoint{1.717405in}{1.153057in}}%
\pgfpathlineto{\pgfqpoint{1.717923in}{0.966545in}}%
\pgfpathlineto{\pgfqpoint{1.718648in}{1.003262in}}%
\pgfpathlineto{\pgfqpoint{1.718856in}{1.180504in}}%
\pgfpathlineto{\pgfqpoint{1.719167in}{0.982856in}}%
\pgfpathlineto{\pgfqpoint{1.719478in}{1.024134in}}%
\pgfpathlineto{\pgfqpoint{1.719581in}{0.924435in}}%
\pgfpathlineto{\pgfqpoint{1.719892in}{1.139143in}}%
\pgfpathlineto{\pgfqpoint{1.720410in}{1.097236in}}%
\pgfpathlineto{\pgfqpoint{1.721032in}{1.157737in}}%
\pgfpathlineto{\pgfqpoint{1.720618in}{0.989360in}}%
\pgfpathlineto{\pgfqpoint{1.721447in}{1.103874in}}%
\pgfpathlineto{\pgfqpoint{1.721758in}{0.936064in}}%
\pgfpathlineto{\pgfqpoint{1.722587in}{1.029149in}}%
\pgfpathlineto{\pgfqpoint{1.723623in}{1.097557in}}%
\pgfpathlineto{\pgfqpoint{1.723001in}{0.877932in}}%
\pgfpathlineto{\pgfqpoint{1.723727in}{1.052589in}}%
\pgfpathlineto{\pgfqpoint{1.724141in}{1.223968in}}%
\pgfpathlineto{\pgfqpoint{1.724349in}{0.992245in}}%
\pgfpathlineto{\pgfqpoint{1.724867in}{1.135368in}}%
\pgfpathlineto{\pgfqpoint{1.726007in}{0.910379in}}%
\pgfpathlineto{\pgfqpoint{1.725592in}{1.220471in}}%
\pgfpathlineto{\pgfqpoint{1.726110in}{0.975502in}}%
\pgfpathlineto{\pgfqpoint{1.726318in}{1.130898in}}%
\pgfpathlineto{\pgfqpoint{1.726629in}{0.925684in}}%
\pgfpathlineto{\pgfqpoint{1.727147in}{0.987931in}}%
\pgfpathlineto{\pgfqpoint{1.728080in}{0.658662in}}%
\pgfpathlineto{\pgfqpoint{1.728494in}{0.761234in}}%
\pgfpathlineto{\pgfqpoint{1.729012in}{0.989575in}}%
\pgfpathlineto{\pgfqpoint{1.729738in}{0.966475in}}%
\pgfpathlineto{\pgfqpoint{1.729945in}{0.778001in}}%
\pgfpathlineto{\pgfqpoint{1.730878in}{0.901383in}}%
\pgfpathlineto{\pgfqpoint{1.732122in}{1.278780in}}%
\pgfpathlineto{\pgfqpoint{1.731292in}{0.819069in}}%
\pgfpathlineto{\pgfqpoint{1.732225in}{1.199737in}}%
\pgfpathlineto{\pgfqpoint{1.732847in}{0.921180in}}%
\pgfpathlineto{\pgfqpoint{1.733262in}{1.133885in}}%
\pgfpathlineto{\pgfqpoint{1.733365in}{1.162660in}}%
\pgfpathlineto{\pgfqpoint{1.733573in}{0.973266in}}%
\pgfpathlineto{\pgfqpoint{1.733780in}{0.999966in}}%
\pgfpathlineto{\pgfqpoint{1.734402in}{0.910777in}}%
\pgfpathlineto{\pgfqpoint{1.734194in}{1.090170in}}%
\pgfpathlineto{\pgfqpoint{1.734609in}{0.976598in}}%
\pgfpathlineto{\pgfqpoint{1.735231in}{1.098373in}}%
\pgfpathlineto{\pgfqpoint{1.735645in}{0.973822in}}%
\pgfpathlineto{\pgfqpoint{1.735749in}{1.033112in}}%
\pgfpathlineto{\pgfqpoint{1.735853in}{0.973350in}}%
\pgfpathlineto{\pgfqpoint{1.736682in}{1.012688in}}%
\pgfpathlineto{\pgfqpoint{1.736785in}{1.130256in}}%
\pgfpathlineto{\pgfqpoint{1.737096in}{0.915292in}}%
\pgfpathlineto{\pgfqpoint{1.737718in}{0.942118in}}%
\pgfpathlineto{\pgfqpoint{1.737822in}{0.923536in}}%
\pgfpathlineto{\pgfqpoint{1.737925in}{1.050651in}}%
\pgfpathlineto{\pgfqpoint{1.738236in}{0.974255in}}%
\pgfpathlineto{\pgfqpoint{1.738340in}{1.099441in}}%
\pgfpathlineto{\pgfqpoint{1.739169in}{0.845170in}}%
\pgfpathlineto{\pgfqpoint{1.739273in}{0.933842in}}%
\pgfpathlineto{\pgfqpoint{1.739687in}{1.068537in}}%
\pgfpathlineto{\pgfqpoint{1.740516in}{0.886854in}}%
\pgfpathlineto{\pgfqpoint{1.741553in}{1.021560in}}%
\pgfpathlineto{\pgfqpoint{1.741138in}{0.847662in}}%
\pgfpathlineto{\pgfqpoint{1.741656in}{0.990001in}}%
\pgfpathlineto{\pgfqpoint{1.741864in}{0.896469in}}%
\pgfpathlineto{\pgfqpoint{1.742486in}{1.128425in}}%
\pgfpathlineto{\pgfqpoint{1.742693in}{1.030814in}}%
\pgfpathlineto{\pgfqpoint{1.742797in}{1.082789in}}%
\pgfpathlineto{\pgfqpoint{1.743211in}{0.894699in}}%
\pgfpathlineto{\pgfqpoint{1.743626in}{1.057670in}}%
\pgfpathlineto{\pgfqpoint{1.743833in}{0.920365in}}%
\pgfpathlineto{\pgfqpoint{1.744869in}{0.922335in}}%
\pgfpathlineto{\pgfqpoint{1.745698in}{0.805821in}}%
\pgfpathlineto{\pgfqpoint{1.746217in}{0.866466in}}%
\pgfpathlineto{\pgfqpoint{1.746424in}{0.806183in}}%
\pgfpathlineto{\pgfqpoint{1.747253in}{0.972384in}}%
\pgfpathlineto{\pgfqpoint{1.748082in}{0.800381in}}%
\pgfpathlineto{\pgfqpoint{1.747460in}{0.992562in}}%
\pgfpathlineto{\pgfqpoint{1.748289in}{0.829717in}}%
\pgfpathlineto{\pgfqpoint{1.748393in}{0.983135in}}%
\pgfpathlineto{\pgfqpoint{1.749119in}{0.723281in}}%
\pgfpathlineto{\pgfqpoint{1.749326in}{0.785692in}}%
\pgfpathlineto{\pgfqpoint{1.749429in}{0.760258in}}%
\pgfpathlineto{\pgfqpoint{1.749740in}{0.912867in}}%
\pgfpathlineto{\pgfqpoint{1.750259in}{1.052088in}}%
\pgfpathlineto{\pgfqpoint{1.750155in}{0.877302in}}%
\pgfpathlineto{\pgfqpoint{1.750880in}{0.949071in}}%
\pgfpathlineto{\pgfqpoint{1.751399in}{0.728220in}}%
\pgfpathlineto{\pgfqpoint{1.751917in}{0.853806in}}%
\pgfpathlineto{\pgfqpoint{1.752642in}{1.036843in}}%
\pgfpathlineto{\pgfqpoint{1.752124in}{0.823673in}}%
\pgfpathlineto{\pgfqpoint{1.753264in}{1.016276in}}%
\pgfpathlineto{\pgfqpoint{1.753368in}{1.006601in}}%
\pgfpathlineto{\pgfqpoint{1.753471in}{0.823267in}}%
\pgfpathlineto{\pgfqpoint{1.754093in}{1.083848in}}%
\pgfpathlineto{\pgfqpoint{1.754404in}{1.068336in}}%
\pgfpathlineto{\pgfqpoint{1.754922in}{0.896167in}}%
\pgfpathlineto{\pgfqpoint{1.755337in}{1.070432in}}%
\pgfpathlineto{\pgfqpoint{1.755648in}{0.996231in}}%
\pgfpathlineto{\pgfqpoint{1.755959in}{1.116141in}}%
\pgfpathlineto{\pgfqpoint{1.756270in}{0.969819in}}%
\pgfpathlineto{\pgfqpoint{1.756684in}{0.994842in}}%
\pgfpathlineto{\pgfqpoint{1.756892in}{0.940133in}}%
\pgfpathlineto{\pgfqpoint{1.756995in}{1.032752in}}%
\pgfpathlineto{\pgfqpoint{1.757306in}{0.945331in}}%
\pgfpathlineto{\pgfqpoint{1.757410in}{0.879425in}}%
\pgfpathlineto{\pgfqpoint{1.758135in}{1.119977in}}%
\pgfpathlineto{\pgfqpoint{1.759793in}{0.865443in}}%
\pgfpathlineto{\pgfqpoint{1.760726in}{1.167255in}}%
\pgfpathlineto{\pgfqpoint{1.761244in}{1.003444in}}%
\pgfpathlineto{\pgfqpoint{1.761970in}{0.823509in}}%
\pgfpathlineto{\pgfqpoint{1.762177in}{1.039169in}}%
\pgfpathlineto{\pgfqpoint{1.762384in}{0.959283in}}%
\pgfpathlineto{\pgfqpoint{1.762592in}{1.090048in}}%
\pgfpathlineto{\pgfqpoint{1.763006in}{0.872267in}}%
\pgfpathlineto{\pgfqpoint{1.763317in}{0.941232in}}%
\pgfpathlineto{\pgfqpoint{1.763628in}{0.822553in}}%
\pgfpathlineto{\pgfqpoint{1.763939in}{1.097462in}}%
\pgfpathlineto{\pgfqpoint{1.764665in}{0.822182in}}%
\pgfpathlineto{\pgfqpoint{1.765286in}{0.972378in}}%
\pgfpathlineto{\pgfqpoint{1.765390in}{1.032479in}}%
\pgfpathlineto{\pgfqpoint{1.766116in}{0.903465in}}%
\pgfpathlineto{\pgfqpoint{1.766323in}{1.010551in}}%
\pgfpathlineto{\pgfqpoint{1.766945in}{0.845430in}}%
\pgfpathlineto{\pgfqpoint{1.767463in}{0.967710in}}%
\pgfpathlineto{\pgfqpoint{1.768499in}{1.128289in}}%
\pgfpathlineto{\pgfqpoint{1.767670in}{0.965351in}}%
\pgfpathlineto{\pgfqpoint{1.768603in}{1.012940in}}%
\pgfpathlineto{\pgfqpoint{1.769743in}{0.875323in}}%
\pgfpathlineto{\pgfqpoint{1.769328in}{1.080529in}}%
\pgfpathlineto{\pgfqpoint{1.769847in}{0.890389in}}%
\pgfpathlineto{\pgfqpoint{1.770572in}{1.075873in}}%
\pgfpathlineto{\pgfqpoint{1.770054in}{0.851723in}}%
\pgfpathlineto{\pgfqpoint{1.771194in}{0.966224in}}%
\pgfpathlineto{\pgfqpoint{1.771401in}{0.769861in}}%
\pgfpathlineto{\pgfqpoint{1.772023in}{1.068440in}}%
\pgfpathlineto{\pgfqpoint{1.772541in}{1.172737in}}%
\pgfpathlineto{\pgfqpoint{1.772334in}{0.853146in}}%
\pgfpathlineto{\pgfqpoint{1.772956in}{1.033120in}}%
\pgfpathlineto{\pgfqpoint{1.773370in}{0.968854in}}%
\pgfpathlineto{\pgfqpoint{1.773163in}{1.081489in}}%
\pgfpathlineto{\pgfqpoint{1.773785in}{1.053840in}}%
\pgfpathlineto{\pgfqpoint{1.773889in}{1.102690in}}%
\pgfpathlineto{\pgfqpoint{1.774199in}{0.811130in}}%
\pgfpathlineto{\pgfqpoint{1.774614in}{0.932715in}}%
\pgfpathlineto{\pgfqpoint{1.774718in}{0.883106in}}%
\pgfpathlineto{\pgfqpoint{1.775029in}{1.050144in}}%
\pgfpathlineto{\pgfqpoint{1.775650in}{0.911639in}}%
\pgfpathlineto{\pgfqpoint{1.776272in}{1.047303in}}%
\pgfpathlineto{\pgfqpoint{1.775858in}{0.839287in}}%
\pgfpathlineto{\pgfqpoint{1.776894in}{1.021427in}}%
\pgfpathlineto{\pgfqpoint{1.776998in}{0.962807in}}%
\pgfpathlineto{\pgfqpoint{1.777309in}{1.120569in}}%
\pgfpathlineto{\pgfqpoint{1.777723in}{1.109016in}}%
\pgfpathlineto{\pgfqpoint{1.777930in}{1.182024in}}%
\pgfpathlineto{\pgfqpoint{1.778345in}{1.023986in}}%
\pgfpathlineto{\pgfqpoint{1.778552in}{1.038661in}}%
\pgfpathlineto{\pgfqpoint{1.778863in}{0.839935in}}%
\pgfpathlineto{\pgfqpoint{1.779071in}{1.154802in}}%
\pgfpathlineto{\pgfqpoint{1.779589in}{1.052176in}}%
\pgfpathlineto{\pgfqpoint{1.780003in}{0.952449in}}%
\pgfpathlineto{\pgfqpoint{1.780418in}{1.007865in}}%
\pgfpathlineto{\pgfqpoint{1.781454in}{1.165531in}}%
\pgfpathlineto{\pgfqpoint{1.781558in}{1.109062in}}%
\pgfpathlineto{\pgfqpoint{1.781972in}{1.111411in}}%
\pgfpathlineto{\pgfqpoint{1.782802in}{0.876161in}}%
\pgfpathlineto{\pgfqpoint{1.784149in}{1.254635in}}%
\pgfpathlineto{\pgfqpoint{1.784253in}{1.176400in}}%
\pgfpathlineto{\pgfqpoint{1.784356in}{1.180479in}}%
\pgfpathlineto{\pgfqpoint{1.785393in}{0.956252in}}%
\pgfpathlineto{\pgfqpoint{1.784563in}{1.251074in}}%
\pgfpathlineto{\pgfqpoint{1.785496in}{1.132361in}}%
\pgfpathlineto{\pgfqpoint{1.785600in}{1.130145in}}%
\pgfpathlineto{\pgfqpoint{1.785911in}{1.271938in}}%
\pgfpathlineto{\pgfqpoint{1.786222in}{1.008001in}}%
\pgfpathlineto{\pgfqpoint{1.786740in}{1.044345in}}%
\pgfpathlineto{\pgfqpoint{1.786844in}{0.995733in}}%
\pgfpathlineto{\pgfqpoint{1.787984in}{0.795557in}}%
\pgfpathlineto{\pgfqpoint{1.789227in}{1.083171in}}%
\pgfpathlineto{\pgfqpoint{1.789435in}{1.080322in}}%
\pgfpathlineto{\pgfqpoint{1.790264in}{0.937810in}}%
\pgfpathlineto{\pgfqpoint{1.790056in}{1.111084in}}%
\pgfpathlineto{\pgfqpoint{1.790471in}{0.944700in}}%
\pgfpathlineto{\pgfqpoint{1.790575in}{1.204993in}}%
\pgfpathlineto{\pgfqpoint{1.791611in}{1.156033in}}%
\pgfpathlineto{\pgfqpoint{1.791818in}{0.998198in}}%
\pgfpathlineto{\pgfqpoint{1.792336in}{1.240539in}}%
\pgfpathlineto{\pgfqpoint{1.792751in}{1.140711in}}%
\pgfpathlineto{\pgfqpoint{1.792855in}{1.217723in}}%
\pgfpathlineto{\pgfqpoint{1.793684in}{1.059464in}}%
\pgfpathlineto{\pgfqpoint{1.793787in}{1.174216in}}%
\pgfpathlineto{\pgfqpoint{1.794409in}{0.902102in}}%
\pgfpathlineto{\pgfqpoint{1.794927in}{1.086261in}}%
\pgfpathlineto{\pgfqpoint{1.796275in}{0.879289in}}%
\pgfpathlineto{\pgfqpoint{1.795757in}{1.091974in}}%
\pgfpathlineto{\pgfqpoint{1.796689in}{0.906920in}}%
\pgfpathlineto{\pgfqpoint{1.797933in}{1.086797in}}%
\pgfpathlineto{\pgfqpoint{1.797311in}{0.856302in}}%
\pgfpathlineto{\pgfqpoint{1.798037in}{0.990604in}}%
\pgfpathlineto{\pgfqpoint{1.798140in}{0.861002in}}%
\pgfpathlineto{\pgfqpoint{1.798969in}{1.006705in}}%
\pgfpathlineto{\pgfqpoint{1.799073in}{0.997841in}}%
\pgfpathlineto{\pgfqpoint{1.799177in}{1.031112in}}%
\pgfpathlineto{\pgfqpoint{1.799488in}{0.903852in}}%
\pgfpathlineto{\pgfqpoint{1.799591in}{0.921001in}}%
\pgfpathlineto{\pgfqpoint{1.799695in}{0.716690in}}%
\pgfpathlineto{\pgfqpoint{1.800213in}{0.930375in}}%
\pgfpathlineto{\pgfqpoint{1.800731in}{0.799739in}}%
\pgfpathlineto{\pgfqpoint{1.801975in}{0.987531in}}%
\pgfpathlineto{\pgfqpoint{1.802390in}{0.736832in}}%
\pgfpathlineto{\pgfqpoint{1.802804in}{1.056072in}}%
\pgfpathlineto{\pgfqpoint{1.803115in}{0.929082in}}%
\pgfpathlineto{\pgfqpoint{1.803633in}{1.125001in}}%
\pgfpathlineto{\pgfqpoint{1.803530in}{0.844572in}}%
\pgfpathlineto{\pgfqpoint{1.804359in}{1.009925in}}%
\pgfpathlineto{\pgfqpoint{1.805188in}{0.886728in}}%
\pgfpathlineto{\pgfqpoint{1.804877in}{1.082699in}}%
\pgfpathlineto{\pgfqpoint{1.805499in}{0.939991in}}%
\pgfpathlineto{\pgfqpoint{1.805913in}{0.772182in}}%
\pgfpathlineto{\pgfqpoint{1.806535in}{0.913768in}}%
\pgfpathlineto{\pgfqpoint{1.807261in}{0.902244in}}%
\pgfpathlineto{\pgfqpoint{1.807779in}{1.144817in}}%
\pgfpathlineto{\pgfqpoint{1.808712in}{0.964162in}}%
\pgfpathlineto{\pgfqpoint{1.808401in}{1.166350in}}%
\pgfpathlineto{\pgfqpoint{1.809022in}{1.049415in}}%
\pgfpathlineto{\pgfqpoint{1.809437in}{1.157164in}}%
\pgfpathlineto{\pgfqpoint{1.809748in}{0.901109in}}%
\pgfpathlineto{\pgfqpoint{1.809852in}{0.892699in}}%
\pgfpathlineto{\pgfqpoint{1.809955in}{0.960000in}}%
\pgfpathlineto{\pgfqpoint{1.810266in}{1.036855in}}%
\pgfpathlineto{\pgfqpoint{1.810473in}{0.893540in}}%
\pgfpathlineto{\pgfqpoint{1.810681in}{0.925287in}}%
\pgfpathlineto{\pgfqpoint{1.810784in}{0.842635in}}%
\pgfpathlineto{\pgfqpoint{1.811406in}{1.146948in}}%
\pgfpathlineto{\pgfqpoint{1.811510in}{1.044541in}}%
\pgfpathlineto{\pgfqpoint{1.812028in}{1.179893in}}%
\pgfpathlineto{\pgfqpoint{1.812235in}{1.012067in}}%
\pgfpathlineto{\pgfqpoint{1.812754in}{1.168842in}}%
\pgfpathlineto{\pgfqpoint{1.812857in}{1.200204in}}%
\pgfpathlineto{\pgfqpoint{1.813272in}{1.007039in}}%
\pgfpathlineto{\pgfqpoint{1.813375in}{1.071413in}}%
\pgfpathlineto{\pgfqpoint{1.813790in}{0.934221in}}%
\pgfpathlineto{\pgfqpoint{1.813997in}{1.129947in}}%
\pgfpathlineto{\pgfqpoint{1.814515in}{1.035243in}}%
\pgfpathlineto{\pgfqpoint{1.815448in}{1.136969in}}%
\pgfpathlineto{\pgfqpoint{1.815241in}{1.015347in}}%
\pgfpathlineto{\pgfqpoint{1.815552in}{1.043470in}}%
\pgfpathlineto{\pgfqpoint{1.815863in}{1.038882in}}%
\pgfpathlineto{\pgfqpoint{1.816070in}{1.105180in}}%
\pgfpathlineto{\pgfqpoint{1.817003in}{1.180488in}}%
\pgfpathlineto{\pgfqpoint{1.816795in}{0.959963in}}%
\pgfpathlineto{\pgfqpoint{1.817210in}{1.163812in}}%
\pgfpathlineto{\pgfqpoint{1.817417in}{1.228206in}}%
\pgfpathlineto{\pgfqpoint{1.818350in}{0.935274in}}%
\pgfpathlineto{\pgfqpoint{1.818868in}{1.167030in}}%
\pgfpathlineto{\pgfqpoint{1.818661in}{0.908451in}}%
\pgfpathlineto{\pgfqpoint{1.819697in}{1.116165in}}%
\pgfpathlineto{\pgfqpoint{1.820527in}{0.980228in}}%
\pgfpathlineto{\pgfqpoint{1.820216in}{1.193538in}}%
\pgfpathlineto{\pgfqpoint{1.820630in}{1.182686in}}%
\pgfpathlineto{\pgfqpoint{1.821045in}{1.308982in}}%
\pgfpathlineto{\pgfqpoint{1.820941in}{1.156003in}}%
\pgfpathlineto{\pgfqpoint{1.821563in}{1.160559in}}%
\pgfpathlineto{\pgfqpoint{1.821667in}{1.144167in}}%
\pgfpathlineto{\pgfqpoint{1.821874in}{1.187496in}}%
\pgfpathlineto{\pgfqpoint{1.821977in}{1.163312in}}%
\pgfpathlineto{\pgfqpoint{1.822185in}{0.996007in}}%
\pgfpathlineto{\pgfqpoint{1.823221in}{1.301549in}}%
\pgfpathlineto{\pgfqpoint{1.823325in}{1.087672in}}%
\pgfpathlineto{\pgfqpoint{1.824361in}{1.154085in}}%
\pgfpathlineto{\pgfqpoint{1.824568in}{1.178398in}}%
\pgfpathlineto{\pgfqpoint{1.825812in}{0.946904in}}%
\pgfpathlineto{\pgfqpoint{1.826019in}{0.907992in}}%
\pgfpathlineto{\pgfqpoint{1.826123in}{0.939602in}}%
\pgfpathlineto{\pgfqpoint{1.826745in}{0.927400in}}%
\pgfpathlineto{\pgfqpoint{1.827263in}{1.211088in}}%
\pgfpathlineto{\pgfqpoint{1.827989in}{0.933742in}}%
\pgfpathlineto{\pgfqpoint{1.828403in}{1.096893in}}%
\pgfpathlineto{\pgfqpoint{1.828507in}{1.093074in}}%
\pgfpathlineto{\pgfqpoint{1.828610in}{1.121232in}}%
\pgfpathlineto{\pgfqpoint{1.828714in}{1.037141in}}%
\pgfpathlineto{\pgfqpoint{1.829129in}{1.233528in}}%
\pgfpathlineto{\pgfqpoint{1.829647in}{1.052033in}}%
\pgfpathlineto{\pgfqpoint{1.830061in}{1.181488in}}%
\pgfpathlineto{\pgfqpoint{1.830476in}{0.977066in}}%
\pgfpathlineto{\pgfqpoint{1.830891in}{1.176492in}}%
\pgfpathlineto{\pgfqpoint{1.831201in}{1.042627in}}%
\pgfpathlineto{\pgfqpoint{1.831305in}{1.239950in}}%
\pgfpathlineto{\pgfqpoint{1.832031in}{1.082705in}}%
\pgfpathlineto{\pgfqpoint{1.832860in}{1.171032in}}%
\pgfpathlineto{\pgfqpoint{1.832963in}{0.938347in}}%
\pgfpathlineto{\pgfqpoint{1.833896in}{1.102192in}}%
\pgfpathlineto{\pgfqpoint{1.834000in}{1.211313in}}%
\pgfpathlineto{\pgfqpoint{1.834518in}{1.037022in}}%
\pgfpathlineto{\pgfqpoint{1.834932in}{1.108267in}}%
\pgfpathlineto{\pgfqpoint{1.835347in}{0.912271in}}%
\pgfpathlineto{\pgfqpoint{1.835762in}{1.167275in}}%
\pgfpathlineto{\pgfqpoint{1.836073in}{1.029962in}}%
\pgfpathlineto{\pgfqpoint{1.836591in}{0.884022in}}%
\pgfpathlineto{\pgfqpoint{1.837109in}{1.032593in}}%
\pgfpathlineto{\pgfqpoint{1.837834in}{1.189325in}}%
\pgfpathlineto{\pgfqpoint{1.837420in}{0.966687in}}%
\pgfpathlineto{\pgfqpoint{1.838249in}{1.135014in}}%
\pgfpathlineto{\pgfqpoint{1.839078in}{0.912679in}}%
\pgfpathlineto{\pgfqpoint{1.839285in}{1.067072in}}%
\pgfpathlineto{\pgfqpoint{1.839596in}{0.941449in}}%
\pgfpathlineto{\pgfqpoint{1.840425in}{1.145465in}}%
\pgfpathlineto{\pgfqpoint{1.840633in}{0.988487in}}%
\pgfpathlineto{\pgfqpoint{1.841151in}{1.208522in}}%
\pgfpathlineto{\pgfqpoint{1.841669in}{1.030539in}}%
\pgfpathlineto{\pgfqpoint{1.842395in}{1.159570in}}%
\pgfpathlineto{\pgfqpoint{1.841980in}{0.920009in}}%
\pgfpathlineto{\pgfqpoint{1.842809in}{1.109934in}}%
\pgfpathlineto{\pgfqpoint{1.843431in}{0.888009in}}%
\pgfpathlineto{\pgfqpoint{1.843224in}{1.134000in}}%
\pgfpathlineto{\pgfqpoint{1.843846in}{1.111800in}}%
\pgfpathlineto{\pgfqpoint{1.844053in}{1.039314in}}%
\pgfpathlineto{\pgfqpoint{1.844156in}{0.891430in}}%
\pgfpathlineto{\pgfqpoint{1.844986in}{1.145679in}}%
\pgfpathlineto{\pgfqpoint{1.845193in}{0.991357in}}%
\pgfpathlineto{\pgfqpoint{1.845711in}{1.128606in}}%
\pgfpathlineto{\pgfqpoint{1.845400in}{0.876779in}}%
\pgfpathlineto{\pgfqpoint{1.846126in}{1.004535in}}%
\pgfpathlineto{\pgfqpoint{1.846229in}{0.916349in}}%
\pgfpathlineto{\pgfqpoint{1.847162in}{0.994952in}}%
\pgfpathlineto{\pgfqpoint{1.847887in}{1.118052in}}%
\pgfpathlineto{\pgfqpoint{1.847680in}{0.953947in}}%
\pgfpathlineto{\pgfqpoint{1.848198in}{0.978617in}}%
\pgfpathlineto{\pgfqpoint{1.848302in}{0.978629in}}%
\pgfpathlineto{\pgfqpoint{1.848613in}{0.884727in}}%
\pgfpathlineto{\pgfqpoint{1.848717in}{1.016509in}}%
\pgfpathlineto{\pgfqpoint{1.848820in}{1.040751in}}%
\pgfpathlineto{\pgfqpoint{1.848924in}{0.933384in}}%
\pgfpathlineto{\pgfqpoint{1.849028in}{0.816026in}}%
\pgfpathlineto{\pgfqpoint{1.849338in}{1.085935in}}%
\pgfpathlineto{\pgfqpoint{1.849960in}{0.873765in}}%
\pgfpathlineto{\pgfqpoint{1.850893in}{1.023012in}}%
\pgfpathlineto{\pgfqpoint{1.850271in}{0.841053in}}%
\pgfpathlineto{\pgfqpoint{1.851204in}{0.964127in}}%
\pgfpathlineto{\pgfqpoint{1.851308in}{0.947758in}}%
\pgfpathlineto{\pgfqpoint{1.851411in}{0.973624in}}%
\pgfpathlineto{\pgfqpoint{1.851515in}{1.145138in}}%
\pgfpathlineto{\pgfqpoint{1.851929in}{0.959226in}}%
\pgfpathlineto{\pgfqpoint{1.852551in}{1.099543in}}%
\pgfpathlineto{\pgfqpoint{1.852759in}{0.994545in}}%
\pgfpathlineto{\pgfqpoint{1.853380in}{1.262089in}}%
\pgfpathlineto{\pgfqpoint{1.853484in}{1.182110in}}%
\pgfpathlineto{\pgfqpoint{1.853588in}{1.194234in}}%
\pgfpathlineto{\pgfqpoint{1.854313in}{0.792141in}}%
\pgfpathlineto{\pgfqpoint{1.854831in}{0.942339in}}%
\pgfpathlineto{\pgfqpoint{1.855764in}{1.068280in}}%
\pgfpathlineto{\pgfqpoint{1.855453in}{0.891064in}}%
\pgfpathlineto{\pgfqpoint{1.855868in}{1.047190in}}%
\pgfpathlineto{\pgfqpoint{1.856801in}{0.787247in}}%
\pgfpathlineto{\pgfqpoint{1.856904in}{0.872859in}}%
\pgfpathlineto{\pgfqpoint{1.857319in}{0.738214in}}%
\pgfpathlineto{\pgfqpoint{1.858044in}{1.156396in}}%
\pgfpathlineto{\pgfqpoint{1.859288in}{0.853187in}}%
\pgfpathlineto{\pgfqpoint{1.860532in}{0.995526in}}%
\pgfpathlineto{\pgfqpoint{1.860635in}{0.973713in}}%
\pgfpathlineto{\pgfqpoint{1.860946in}{0.926153in}}%
\pgfpathlineto{\pgfqpoint{1.861464in}{1.017967in}}%
\pgfpathlineto{\pgfqpoint{1.861983in}{1.171995in}}%
\pgfpathlineto{\pgfqpoint{1.861879in}{0.976876in}}%
\pgfpathlineto{\pgfqpoint{1.862501in}{1.001474in}}%
\pgfpathlineto{\pgfqpoint{1.862604in}{1.004709in}}%
\pgfpathlineto{\pgfqpoint{1.863019in}{0.850955in}}%
\pgfpathlineto{\pgfqpoint{1.863744in}{1.132846in}}%
\pgfpathlineto{\pgfqpoint{1.864574in}{1.198969in}}%
\pgfpathlineto{\pgfqpoint{1.864988in}{0.999166in}}%
\pgfpathlineto{\pgfqpoint{1.865092in}{1.150080in}}%
\pgfpathlineto{\pgfqpoint{1.865506in}{0.976049in}}%
\pgfpathlineto{\pgfqpoint{1.866128in}{1.042029in}}%
\pgfpathlineto{\pgfqpoint{1.866854in}{0.993893in}}%
\pgfpathlineto{\pgfqpoint{1.867475in}{1.223440in}}%
\pgfpathlineto{\pgfqpoint{1.868615in}{1.028880in}}%
\pgfpathlineto{\pgfqpoint{1.868305in}{1.236206in}}%
\pgfpathlineto{\pgfqpoint{1.868823in}{1.055854in}}%
\pgfpathlineto{\pgfqpoint{1.869134in}{1.192303in}}%
\pgfpathlineto{\pgfqpoint{1.869652in}{0.999442in}}%
\pgfpathlineto{\pgfqpoint{1.869963in}{1.123691in}}%
\pgfpathlineto{\pgfqpoint{1.870066in}{1.021520in}}%
\pgfpathlineto{\pgfqpoint{1.870274in}{1.195862in}}%
\pgfpathlineto{\pgfqpoint{1.870999in}{1.099261in}}%
\pgfpathlineto{\pgfqpoint{1.871517in}{1.044055in}}%
\pgfpathlineto{\pgfqpoint{1.872036in}{1.158440in}}%
\pgfpathlineto{\pgfqpoint{1.872139in}{1.085521in}}%
\pgfpathlineto{\pgfqpoint{1.872657in}{1.257116in}}%
\pgfpathlineto{\pgfqpoint{1.873072in}{1.113854in}}%
\pgfpathlineto{\pgfqpoint{1.873797in}{1.010442in}}%
\pgfpathlineto{\pgfqpoint{1.874419in}{1.218932in}}%
\pgfpathlineto{\pgfqpoint{1.874730in}{0.955607in}}%
\pgfpathlineto{\pgfqpoint{1.875767in}{1.048147in}}%
\pgfpathlineto{\pgfqpoint{1.876803in}{1.281929in}}%
\pgfpathlineto{\pgfqpoint{1.876907in}{1.218568in}}%
\pgfpathlineto{\pgfqpoint{1.877632in}{0.867433in}}%
\pgfpathlineto{\pgfqpoint{1.878150in}{0.914165in}}%
\pgfpathlineto{\pgfqpoint{1.878358in}{1.132938in}}%
\pgfpathlineto{\pgfqpoint{1.879394in}{1.129849in}}%
\pgfpathlineto{\pgfqpoint{1.879705in}{1.017191in}}%
\pgfpathlineto{\pgfqpoint{1.880327in}{1.182011in}}%
\pgfpathlineto{\pgfqpoint{1.880430in}{1.180005in}}%
\pgfpathlineto{\pgfqpoint{1.880741in}{1.247881in}}%
\pgfpathlineto{\pgfqpoint{1.880845in}{1.114047in}}%
\pgfpathlineto{\pgfqpoint{1.881156in}{0.940158in}}%
\pgfpathlineto{\pgfqpoint{1.881363in}{1.162737in}}%
\pgfpathlineto{\pgfqpoint{1.881881in}{1.091041in}}%
\pgfpathlineto{\pgfqpoint{1.882503in}{1.191904in}}%
\pgfpathlineto{\pgfqpoint{1.882607in}{1.052305in}}%
\pgfpathlineto{\pgfqpoint{1.883125in}{1.140425in}}%
\pgfpathlineto{\pgfqpoint{1.884058in}{1.039697in}}%
\pgfpathlineto{\pgfqpoint{1.883643in}{1.222580in}}%
\pgfpathlineto{\pgfqpoint{1.884161in}{1.081647in}}%
\pgfpathlineto{\pgfqpoint{1.884576in}{1.382260in}}%
\pgfpathlineto{\pgfqpoint{1.885405in}{1.287879in}}%
\pgfpathlineto{\pgfqpoint{1.886545in}{1.024351in}}%
\pgfpathlineto{\pgfqpoint{1.886856in}{1.122877in}}%
\pgfpathlineto{\pgfqpoint{1.887167in}{0.919655in}}%
\pgfpathlineto{\pgfqpoint{1.887685in}{1.095821in}}%
\pgfpathlineto{\pgfqpoint{1.888100in}{0.995073in}}%
\pgfpathlineto{\pgfqpoint{1.888203in}{1.176183in}}%
\pgfpathlineto{\pgfqpoint{1.888618in}{1.118051in}}%
\pgfpathlineto{\pgfqpoint{1.888722in}{1.154591in}}%
\pgfpathlineto{\pgfqpoint{1.889240in}{1.023491in}}%
\pgfpathlineto{\pgfqpoint{1.889343in}{0.957736in}}%
\pgfpathlineto{\pgfqpoint{1.889758in}{1.227893in}}%
\pgfpathlineto{\pgfqpoint{1.890173in}{1.128303in}}%
\pgfpathlineto{\pgfqpoint{1.890276in}{1.127265in}}%
\pgfpathlineto{\pgfqpoint{1.890794in}{1.067503in}}%
\pgfpathlineto{\pgfqpoint{1.890484in}{1.198470in}}%
\pgfpathlineto{\pgfqpoint{1.890898in}{1.138848in}}%
\pgfpathlineto{\pgfqpoint{1.891313in}{1.326928in}}%
\pgfpathlineto{\pgfqpoint{1.891624in}{1.097697in}}%
\pgfpathlineto{\pgfqpoint{1.892142in}{1.292258in}}%
\pgfpathlineto{\pgfqpoint{1.893075in}{0.971131in}}%
\pgfpathlineto{\pgfqpoint{1.893489in}{1.069669in}}%
\pgfpathlineto{\pgfqpoint{1.893696in}{1.147262in}}%
\pgfpathlineto{\pgfqpoint{1.893904in}{0.947623in}}%
\pgfpathlineto{\pgfqpoint{1.894525in}{1.052768in}}%
\pgfpathlineto{\pgfqpoint{1.895666in}{0.853193in}}%
\pgfpathlineto{\pgfqpoint{1.894940in}{1.106831in}}%
\pgfpathlineto{\pgfqpoint{1.895769in}{0.931838in}}%
\pgfpathlineto{\pgfqpoint{1.896598in}{1.147009in}}%
\pgfpathlineto{\pgfqpoint{1.896909in}{1.133228in}}%
\pgfpathlineto{\pgfqpoint{1.897013in}{0.989777in}}%
\pgfpathlineto{\pgfqpoint{1.897427in}{1.138185in}}%
\pgfpathlineto{\pgfqpoint{1.898049in}{1.052094in}}%
\pgfpathlineto{\pgfqpoint{1.898671in}{1.227147in}}%
\pgfpathlineto{\pgfqpoint{1.898257in}{0.942647in}}%
\pgfpathlineto{\pgfqpoint{1.899086in}{1.066676in}}%
\pgfpathlineto{\pgfqpoint{1.899293in}{0.965553in}}%
\pgfpathlineto{\pgfqpoint{1.899604in}{1.105824in}}%
\pgfpathlineto{\pgfqpoint{1.899811in}{1.233577in}}%
\pgfpathlineto{\pgfqpoint{1.900433in}{1.036374in}}%
\pgfpathlineto{\pgfqpoint{1.900744in}{1.145336in}}%
\pgfpathlineto{\pgfqpoint{1.901573in}{0.909247in}}%
\pgfpathlineto{\pgfqpoint{1.900951in}{1.150840in}}%
\pgfpathlineto{\pgfqpoint{1.901988in}{1.089282in}}%
\pgfpathlineto{\pgfqpoint{1.902091in}{1.174690in}}%
\pgfpathlineto{\pgfqpoint{1.902817in}{0.965005in}}%
\pgfpathlineto{\pgfqpoint{1.902920in}{0.938979in}}%
\pgfpathlineto{\pgfqpoint{1.903024in}{1.055552in}}%
\pgfpathlineto{\pgfqpoint{1.903646in}{0.997114in}}%
\pgfpathlineto{\pgfqpoint{1.903853in}{1.172114in}}%
\pgfpathlineto{\pgfqpoint{1.904786in}{1.093975in}}%
\pgfpathlineto{\pgfqpoint{1.905200in}{0.931224in}}%
\pgfpathlineto{\pgfqpoint{1.905304in}{1.111555in}}%
\pgfpathlineto{\pgfqpoint{1.905822in}{0.999775in}}%
\pgfpathlineto{\pgfqpoint{1.905926in}{1.108020in}}%
\pgfpathlineto{\pgfqpoint{1.906651in}{0.919162in}}%
\pgfpathlineto{\pgfqpoint{1.906755in}{0.978778in}}%
\pgfpathlineto{\pgfqpoint{1.906859in}{0.890803in}}%
\pgfpathlineto{\pgfqpoint{1.907688in}{1.129715in}}%
\pgfpathlineto{\pgfqpoint{1.907791in}{1.129796in}}%
\pgfpathlineto{\pgfqpoint{1.907895in}{1.192801in}}%
\pgfpathlineto{\pgfqpoint{1.907999in}{1.057669in}}%
\pgfpathlineto{\pgfqpoint{1.908724in}{1.126349in}}%
\pgfpathlineto{\pgfqpoint{1.909139in}{0.980817in}}%
\pgfpathlineto{\pgfqpoint{1.909657in}{1.136366in}}%
\pgfpathlineto{\pgfqpoint{1.909761in}{1.089931in}}%
\pgfpathlineto{\pgfqpoint{1.909968in}{1.155539in}}%
\pgfpathlineto{\pgfqpoint{1.910279in}{0.969033in}}%
\pgfpathlineto{\pgfqpoint{1.910797in}{1.075484in}}%
\pgfpathlineto{\pgfqpoint{1.911833in}{0.918996in}}%
\pgfpathlineto{\pgfqpoint{1.912041in}{0.956739in}}%
\pgfpathlineto{\pgfqpoint{1.912766in}{1.184668in}}%
\pgfpathlineto{\pgfqpoint{1.913181in}{1.056616in}}%
\pgfpathlineto{\pgfqpoint{1.913699in}{0.965914in}}%
\pgfpathlineto{\pgfqpoint{1.913906in}{1.110836in}}%
\pgfpathlineto{\pgfqpoint{1.914217in}{1.028632in}}%
\pgfpathlineto{\pgfqpoint{1.914424in}{1.132699in}}%
\pgfpathlineto{\pgfqpoint{1.914632in}{0.954577in}}%
\pgfpathlineto{\pgfqpoint{1.915046in}{0.966082in}}%
\pgfpathlineto{\pgfqpoint{1.915668in}{0.909882in}}%
\pgfpathlineto{\pgfqpoint{1.915979in}{0.949186in}}%
\pgfpathlineto{\pgfqpoint{1.916186in}{1.048570in}}%
\pgfpathlineto{\pgfqpoint{1.916704in}{0.898447in}}%
\pgfpathlineto{\pgfqpoint{1.917015in}{0.898649in}}%
\pgfpathlineto{\pgfqpoint{1.917845in}{0.941702in}}%
\pgfpathlineto{\pgfqpoint{1.918259in}{0.809653in}}%
\pgfpathlineto{\pgfqpoint{1.918881in}{0.980719in}}%
\pgfpathlineto{\pgfqpoint{1.918570in}{0.798296in}}%
\pgfpathlineto{\pgfqpoint{1.919399in}{0.870613in}}%
\pgfpathlineto{\pgfqpoint{1.921368in}{1.148826in}}%
\pgfpathlineto{\pgfqpoint{1.921576in}{1.047800in}}%
\pgfpathlineto{\pgfqpoint{1.922197in}{0.949920in}}%
\pgfpathlineto{\pgfqpoint{1.922094in}{1.126137in}}%
\pgfpathlineto{\pgfqpoint{1.922405in}{1.044968in}}%
\pgfpathlineto{\pgfqpoint{1.923027in}{1.201091in}}%
\pgfpathlineto{\pgfqpoint{1.922612in}{0.989262in}}%
\pgfpathlineto{\pgfqpoint{1.923545in}{1.097404in}}%
\pgfpathlineto{\pgfqpoint{1.923856in}{0.907495in}}%
\pgfpathlineto{\pgfqpoint{1.924477in}{1.115580in}}%
\pgfpathlineto{\pgfqpoint{1.924788in}{0.975575in}}%
\pgfpathlineto{\pgfqpoint{1.926136in}{1.210561in}}%
\pgfpathlineto{\pgfqpoint{1.926239in}{1.120676in}}%
\pgfpathlineto{\pgfqpoint{1.926343in}{0.993273in}}%
\pgfpathlineto{\pgfqpoint{1.927172in}{1.268920in}}%
\pgfpathlineto{\pgfqpoint{1.929038in}{0.951639in}}%
\pgfpathlineto{\pgfqpoint{1.929141in}{1.034731in}}%
\pgfpathlineto{\pgfqpoint{1.929867in}{1.162855in}}%
\pgfpathlineto{\pgfqpoint{1.930074in}{1.007154in}}%
\pgfpathlineto{\pgfqpoint{1.930178in}{1.154797in}}%
\pgfpathlineto{\pgfqpoint{1.930281in}{0.941109in}}%
\pgfpathlineto{\pgfqpoint{1.931214in}{1.111944in}}%
\pgfpathlineto{\pgfqpoint{1.931318in}{1.178796in}}%
\pgfpathlineto{\pgfqpoint{1.932043in}{0.976299in}}%
\pgfpathlineto{\pgfqpoint{1.932250in}{1.104653in}}%
\pgfpathlineto{\pgfqpoint{1.932458in}{0.954800in}}%
\pgfpathlineto{\pgfqpoint{1.932561in}{1.229711in}}%
\pgfpathlineto{\pgfqpoint{1.933287in}{1.139752in}}%
\pgfpathlineto{\pgfqpoint{1.934012in}{0.987088in}}%
\pgfpathlineto{\pgfqpoint{1.933598in}{1.197604in}}%
\pgfpathlineto{\pgfqpoint{1.934323in}{1.136305in}}%
\pgfpathlineto{\pgfqpoint{1.934427in}{1.156505in}}%
\pgfpathlineto{\pgfqpoint{1.934531in}{1.098862in}}%
\pgfpathlineto{\pgfqpoint{1.934738in}{1.124997in}}%
\pgfpathlineto{\pgfqpoint{1.935049in}{0.973128in}}%
\pgfpathlineto{\pgfqpoint{1.935671in}{1.190000in}}%
\pgfpathlineto{\pgfqpoint{1.935878in}{1.031261in}}%
\pgfpathlineto{\pgfqpoint{1.936500in}{1.262056in}}%
\pgfpathlineto{\pgfqpoint{1.936914in}{1.022437in}}%
\pgfpathlineto{\pgfqpoint{1.937018in}{1.222005in}}%
\pgfpathlineto{\pgfqpoint{1.937329in}{0.992693in}}%
\pgfpathlineto{\pgfqpoint{1.938158in}{1.108333in}}%
\pgfpathlineto{\pgfqpoint{1.939298in}{1.228532in}}%
\pgfpathlineto{\pgfqpoint{1.939505in}{1.191912in}}%
\pgfpathlineto{\pgfqpoint{1.939713in}{1.214386in}}%
\pgfpathlineto{\pgfqpoint{1.940853in}{0.970996in}}%
\pgfpathlineto{\pgfqpoint{1.941474in}{1.324569in}}%
\pgfpathlineto{\pgfqpoint{1.942096in}{1.238601in}}%
\pgfpathlineto{\pgfqpoint{1.942614in}{0.968951in}}%
\pgfpathlineto{\pgfqpoint{1.943236in}{1.091990in}}%
\pgfpathlineto{\pgfqpoint{1.944169in}{1.227941in}}%
\pgfpathlineto{\pgfqpoint{1.943962in}{1.046393in}}%
\pgfpathlineto{\pgfqpoint{1.944480in}{1.177426in}}%
\pgfpathlineto{\pgfqpoint{1.945102in}{1.033764in}}%
\pgfpathlineto{\pgfqpoint{1.945516in}{1.179686in}}%
\pgfpathlineto{\pgfqpoint{1.945724in}{1.068727in}}%
\pgfpathlineto{\pgfqpoint{1.946346in}{0.982643in}}%
\pgfpathlineto{\pgfqpoint{1.946138in}{1.103724in}}%
\pgfpathlineto{\pgfqpoint{1.946553in}{1.040753in}}%
\pgfpathlineto{\pgfqpoint{1.946864in}{1.242514in}}%
\pgfpathlineto{\pgfqpoint{1.947175in}{0.981753in}}%
\pgfpathlineto{\pgfqpoint{1.947589in}{1.071857in}}%
\pgfpathlineto{\pgfqpoint{1.947796in}{0.924871in}}%
\pgfpathlineto{\pgfqpoint{1.948315in}{1.125904in}}%
\pgfpathlineto{\pgfqpoint{1.948729in}{1.013888in}}%
\pgfpathlineto{\pgfqpoint{1.949144in}{1.101300in}}%
\pgfpathlineto{\pgfqpoint{1.949247in}{1.005436in}}%
\pgfpathlineto{\pgfqpoint{1.949662in}{1.074339in}}%
\pgfpathlineto{\pgfqpoint{1.949973in}{0.838865in}}%
\pgfpathlineto{\pgfqpoint{1.950284in}{1.083387in}}%
\pgfpathlineto{\pgfqpoint{1.950802in}{0.995930in}}%
\pgfpathlineto{\pgfqpoint{1.951217in}{1.026145in}}%
\pgfpathlineto{\pgfqpoint{1.951320in}{0.979879in}}%
\pgfpathlineto{\pgfqpoint{1.951424in}{1.070380in}}%
\pgfpathlineto{\pgfqpoint{1.952253in}{0.920926in}}%
\pgfpathlineto{\pgfqpoint{1.952357in}{1.021285in}}%
\pgfpathlineto{\pgfqpoint{1.952978in}{0.869568in}}%
\pgfpathlineto{\pgfqpoint{1.952668in}{1.091672in}}%
\pgfpathlineto{\pgfqpoint{1.953600in}{0.937186in}}%
\pgfpathlineto{\pgfqpoint{1.953808in}{1.228369in}}%
\pgfpathlineto{\pgfqpoint{1.954533in}{0.921064in}}%
\pgfpathlineto{\pgfqpoint{1.954637in}{0.999266in}}%
\pgfpathlineto{\pgfqpoint{1.954844in}{0.810094in}}%
\pgfpathlineto{\pgfqpoint{1.955155in}{1.051866in}}%
\pgfpathlineto{\pgfqpoint{1.955880in}{0.949517in}}%
\pgfpathlineto{\pgfqpoint{1.956399in}{0.872611in}}%
\pgfpathlineto{\pgfqpoint{1.956917in}{1.063808in}}%
\pgfpathlineto{\pgfqpoint{1.957020in}{0.944621in}}%
\pgfpathlineto{\pgfqpoint{1.957850in}{1.158127in}}%
\pgfpathlineto{\pgfqpoint{1.957953in}{1.007793in}}%
\pgfpathlineto{\pgfqpoint{1.958264in}{0.986405in}}%
\pgfpathlineto{\pgfqpoint{1.958990in}{1.159505in}}%
\pgfpathlineto{\pgfqpoint{1.959404in}{0.989255in}}%
\pgfpathlineto{\pgfqpoint{1.960026in}{1.066763in}}%
\pgfpathlineto{\pgfqpoint{1.960337in}{1.163956in}}%
\pgfpathlineto{\pgfqpoint{1.960959in}{0.961917in}}%
\pgfpathlineto{\pgfqpoint{1.961166in}{1.094158in}}%
\pgfpathlineto{\pgfqpoint{1.962099in}{0.858369in}}%
\pgfpathlineto{\pgfqpoint{1.962306in}{0.957284in}}%
\pgfpathlineto{\pgfqpoint{1.962617in}{1.072285in}}%
\pgfpathlineto{\pgfqpoint{1.962721in}{0.942619in}}%
\pgfpathlineto{\pgfqpoint{1.962928in}{1.021503in}}%
\pgfpathlineto{\pgfqpoint{1.964068in}{0.859667in}}%
\pgfpathlineto{\pgfqpoint{1.964483in}{0.801621in}}%
\pgfpathlineto{\pgfqpoint{1.965208in}{1.121539in}}%
\pgfpathlineto{\pgfqpoint{1.966037in}{1.009499in}}%
\pgfpathlineto{\pgfqpoint{1.966244in}{1.113814in}}%
\pgfpathlineto{\pgfqpoint{1.966555in}{1.262382in}}%
\pgfpathlineto{\pgfqpoint{1.967074in}{1.087138in}}%
\pgfpathlineto{\pgfqpoint{1.967281in}{1.089858in}}%
\pgfpathlineto{\pgfqpoint{1.967384in}{1.013386in}}%
\pgfpathlineto{\pgfqpoint{1.968110in}{1.226609in}}%
\pgfpathlineto{\pgfqpoint{1.968214in}{1.183366in}}%
\pgfpathlineto{\pgfqpoint{1.968317in}{1.306770in}}%
\pgfpathlineto{\pgfqpoint{1.968732in}{1.059859in}}%
\pgfpathlineto{\pgfqpoint{1.969354in}{1.268726in}}%
\pgfpathlineto{\pgfqpoint{1.969665in}{1.404735in}}%
\pgfpathlineto{\pgfqpoint{1.970597in}{1.170267in}}%
\pgfpathlineto{\pgfqpoint{1.971115in}{1.335278in}}%
\pgfpathlineto{\pgfqpoint{1.971634in}{1.159022in}}%
\pgfpathlineto{\pgfqpoint{1.971737in}{1.256293in}}%
\pgfpathlineto{\pgfqpoint{1.972463in}{1.130940in}}%
\pgfpathlineto{\pgfqpoint{1.972048in}{1.293057in}}%
\pgfpathlineto{\pgfqpoint{1.972670in}{1.252561in}}%
\pgfpathlineto{\pgfqpoint{1.972774in}{1.327824in}}%
\pgfpathlineto{\pgfqpoint{1.973499in}{1.070920in}}%
\pgfpathlineto{\pgfqpoint{1.973603in}{1.145846in}}%
\pgfpathlineto{\pgfqpoint{1.973810in}{0.947232in}}%
\pgfpathlineto{\pgfqpoint{1.974432in}{1.175917in}}%
\pgfpathlineto{\pgfqpoint{1.974536in}{1.132106in}}%
\pgfpathlineto{\pgfqpoint{1.974639in}{1.314331in}}%
\pgfpathlineto{\pgfqpoint{1.975468in}{1.064010in}}%
\pgfpathlineto{\pgfqpoint{1.975572in}{1.099869in}}%
\pgfpathlineto{\pgfqpoint{1.976401in}{1.175770in}}%
\pgfpathlineto{\pgfqpoint{1.975779in}{0.957454in}}%
\pgfpathlineto{\pgfqpoint{1.976816in}{1.157944in}}%
\pgfpathlineto{\pgfqpoint{1.977023in}{1.061883in}}%
\pgfpathlineto{\pgfqpoint{1.977438in}{1.189380in}}%
\pgfpathlineto{\pgfqpoint{1.977852in}{1.141178in}}%
\pgfpathlineto{\pgfqpoint{1.978059in}{1.200877in}}%
\pgfpathlineto{\pgfqpoint{1.978267in}{1.103519in}}%
\pgfpathlineto{\pgfqpoint{1.978474in}{1.165729in}}%
\pgfpathlineto{\pgfqpoint{1.978992in}{1.044534in}}%
\pgfpathlineto{\pgfqpoint{1.979096in}{1.215019in}}%
\pgfpathlineto{\pgfqpoint{1.979614in}{1.130264in}}%
\pgfpathlineto{\pgfqpoint{1.980339in}{1.259036in}}%
\pgfpathlineto{\pgfqpoint{1.980132in}{0.998067in}}%
\pgfpathlineto{\pgfqpoint{1.980961in}{1.234832in}}%
\pgfpathlineto{\pgfqpoint{1.981583in}{1.109638in}}%
\pgfpathlineto{\pgfqpoint{1.981169in}{1.302483in}}%
\pgfpathlineto{\pgfqpoint{1.981894in}{1.113762in}}%
\pgfpathlineto{\pgfqpoint{1.981998in}{1.296023in}}%
\pgfpathlineto{\pgfqpoint{1.982930in}{1.003079in}}%
\pgfpathlineto{\pgfqpoint{1.983034in}{0.953432in}}%
\pgfpathlineto{\pgfqpoint{1.983449in}{1.177418in}}%
\pgfpathlineto{\pgfqpoint{1.983552in}{1.161492in}}%
\pgfpathlineto{\pgfqpoint{1.984174in}{1.335625in}}%
\pgfpathlineto{\pgfqpoint{1.984070in}{1.086321in}}%
\pgfpathlineto{\pgfqpoint{1.984692in}{1.253052in}}%
\pgfpathlineto{\pgfqpoint{1.985521in}{1.109959in}}%
\pgfpathlineto{\pgfqpoint{1.985314in}{1.271495in}}%
\pgfpathlineto{\pgfqpoint{1.985625in}{1.223441in}}%
\pgfpathlineto{\pgfqpoint{1.986143in}{1.491040in}}%
\pgfpathlineto{\pgfqpoint{1.986454in}{1.213755in}}%
\pgfpathlineto{\pgfqpoint{1.986558in}{1.272918in}}%
\pgfpathlineto{\pgfqpoint{1.986661in}{1.140160in}}%
\pgfpathlineto{\pgfqpoint{1.987076in}{1.353040in}}%
\pgfpathlineto{\pgfqpoint{1.987698in}{1.140332in}}%
\pgfpathlineto{\pgfqpoint{1.987802in}{1.136922in}}%
\pgfpathlineto{\pgfqpoint{1.988009in}{1.001863in}}%
\pgfpathlineto{\pgfqpoint{1.988320in}{1.274238in}}%
\pgfpathlineto{\pgfqpoint{1.988942in}{1.087397in}}%
\pgfpathlineto{\pgfqpoint{1.989149in}{1.100704in}}%
\pgfpathlineto{\pgfqpoint{1.989252in}{1.100477in}}%
\pgfpathlineto{\pgfqpoint{1.990289in}{1.302173in}}%
\pgfpathlineto{\pgfqpoint{1.990393in}{1.234306in}}%
\pgfpathlineto{\pgfqpoint{1.991014in}{1.284701in}}%
\pgfpathlineto{\pgfqpoint{1.990911in}{1.186266in}}%
\pgfpathlineto{\pgfqpoint{1.991222in}{1.238543in}}%
\pgfpathlineto{\pgfqpoint{1.992154in}{1.088371in}}%
\pgfpathlineto{\pgfqpoint{1.992362in}{1.122475in}}%
\pgfpathlineto{\pgfqpoint{1.993605in}{1.369354in}}%
\pgfpathlineto{\pgfqpoint{1.994642in}{1.028903in}}%
\pgfpathlineto{\pgfqpoint{1.994849in}{1.070979in}}%
\pgfpathlineto{\pgfqpoint{1.994953in}{1.074390in}}%
\pgfpathlineto{\pgfqpoint{1.995575in}{1.325883in}}%
\pgfpathlineto{\pgfqpoint{1.995989in}{1.127581in}}%
\pgfpathlineto{\pgfqpoint{1.996093in}{1.098403in}}%
\pgfpathlineto{\pgfqpoint{1.996196in}{1.227252in}}%
\pgfpathlineto{\pgfqpoint{1.996404in}{1.221674in}}%
\pgfpathlineto{\pgfqpoint{1.996611in}{1.081465in}}%
\pgfpathlineto{\pgfqpoint{1.997544in}{1.353493in}}%
\pgfpathlineto{\pgfqpoint{1.998891in}{0.991706in}}%
\pgfpathlineto{\pgfqpoint{1.999098in}{0.998796in}}%
\pgfpathlineto{\pgfqpoint{1.999513in}{1.201701in}}%
\pgfpathlineto{\pgfqpoint{2.000238in}{1.024031in}}%
\pgfpathlineto{\pgfqpoint{2.000446in}{0.994697in}}%
\pgfpathlineto{\pgfqpoint{2.000549in}{1.041723in}}%
\pgfpathlineto{\pgfqpoint{2.000653in}{1.129528in}}%
\pgfpathlineto{\pgfqpoint{2.000860in}{0.932170in}}%
\pgfpathlineto{\pgfqpoint{2.001689in}{1.085528in}}%
\pgfpathlineto{\pgfqpoint{2.001897in}{1.026746in}}%
\pgfpathlineto{\pgfqpoint{2.002311in}{1.221012in}}%
\pgfpathlineto{\pgfqpoint{2.002415in}{1.204657in}}%
\pgfpathlineto{\pgfqpoint{2.003037in}{1.246074in}}%
\pgfpathlineto{\pgfqpoint{2.002622in}{1.081100in}}%
\pgfpathlineto{\pgfqpoint{2.003451in}{1.201918in}}%
\pgfpathlineto{\pgfqpoint{2.004695in}{1.013599in}}%
\pgfpathlineto{\pgfqpoint{2.005006in}{0.957150in}}%
\pgfpathlineto{\pgfqpoint{2.005835in}{1.197540in}}%
\pgfpathlineto{\pgfqpoint{2.006042in}{1.006023in}}%
\pgfpathlineto{\pgfqpoint{2.006975in}{1.059263in}}%
\pgfpathlineto{\pgfqpoint{2.007908in}{1.138660in}}%
\pgfpathlineto{\pgfqpoint{2.008011in}{0.913223in}}%
\pgfpathlineto{\pgfqpoint{2.008840in}{1.334173in}}%
\pgfpathlineto{\pgfqpoint{2.009255in}{1.190996in}}%
\pgfpathlineto{\pgfqpoint{2.010084in}{1.060772in}}%
\pgfpathlineto{\pgfqpoint{2.009462in}{1.202391in}}%
\pgfpathlineto{\pgfqpoint{2.010499in}{1.098628in}}%
\pgfpathlineto{\pgfqpoint{2.010810in}{1.067851in}}%
\pgfpathlineto{\pgfqpoint{2.011535in}{1.234978in}}%
\pgfpathlineto{\pgfqpoint{2.011742in}{1.010234in}}%
\pgfpathlineto{\pgfqpoint{2.012571in}{1.241693in}}%
\pgfpathlineto{\pgfqpoint{2.012675in}{1.127199in}}%
\pgfpathlineto{\pgfqpoint{2.012779in}{1.231069in}}%
\pgfpathlineto{\pgfqpoint{2.013608in}{1.091506in}}%
\pgfpathlineto{\pgfqpoint{2.013712in}{1.172868in}}%
\pgfpathlineto{\pgfqpoint{2.014230in}{1.055903in}}%
\pgfpathlineto{\pgfqpoint{2.014333in}{1.189701in}}%
\pgfpathlineto{\pgfqpoint{2.014748in}{1.172124in}}%
\pgfpathlineto{\pgfqpoint{2.015266in}{1.299860in}}%
\pgfpathlineto{\pgfqpoint{2.015473in}{1.149007in}}%
\pgfpathlineto{\pgfqpoint{2.015784in}{1.181125in}}%
\pgfpathlineto{\pgfqpoint{2.016199in}{1.056866in}}%
\pgfpathlineto{\pgfqpoint{2.016510in}{1.307185in}}%
\pgfpathlineto{\pgfqpoint{2.016821in}{1.100858in}}%
\pgfpathlineto{\pgfqpoint{2.017339in}{1.069565in}}%
\pgfpathlineto{\pgfqpoint{2.018064in}{1.362500in}}%
\pgfpathlineto{\pgfqpoint{2.018790in}{0.998956in}}%
\pgfpathlineto{\pgfqpoint{2.019515in}{1.244259in}}%
\pgfpathlineto{\pgfqpoint{2.019619in}{1.356300in}}%
\pgfpathlineto{\pgfqpoint{2.020448in}{1.070772in}}%
\pgfpathlineto{\pgfqpoint{2.020759in}{1.198602in}}%
\pgfpathlineto{\pgfqpoint{2.021070in}{1.056404in}}%
\pgfpathlineto{\pgfqpoint{2.021485in}{1.138002in}}%
\pgfpathlineto{\pgfqpoint{2.021795in}{0.947463in}}%
\pgfpathlineto{\pgfqpoint{2.022521in}{1.163986in}}%
\pgfpathlineto{\pgfqpoint{2.022832in}{0.967100in}}%
\pgfpathlineto{\pgfqpoint{2.023039in}{0.979768in}}%
\pgfpathlineto{\pgfqpoint{2.023143in}{0.817101in}}%
\pgfpathlineto{\pgfqpoint{2.023454in}{1.083536in}}%
\pgfpathlineto{\pgfqpoint{2.023972in}{0.993120in}}%
\pgfpathlineto{\pgfqpoint{2.024179in}{0.982726in}}%
\pgfpathlineto{\pgfqpoint{2.025216in}{1.228125in}}%
\pgfpathlineto{\pgfqpoint{2.025734in}{1.302457in}}%
\pgfpathlineto{\pgfqpoint{2.025837in}{1.214138in}}%
\pgfpathlineto{\pgfqpoint{2.026874in}{1.001163in}}%
\pgfpathlineto{\pgfqpoint{2.026459in}{1.258733in}}%
\pgfpathlineto{\pgfqpoint{2.026977in}{1.013888in}}%
\pgfpathlineto{\pgfqpoint{2.027807in}{1.248542in}}%
\pgfpathlineto{\pgfqpoint{2.028117in}{1.075638in}}%
\pgfpathlineto{\pgfqpoint{2.028843in}{0.960777in}}%
\pgfpathlineto{\pgfqpoint{2.028325in}{1.137511in}}%
\pgfpathlineto{\pgfqpoint{2.029154in}{1.077954in}}%
\pgfpathlineto{\pgfqpoint{2.029983in}{0.900643in}}%
\pgfpathlineto{\pgfqpoint{2.029568in}{1.089808in}}%
\pgfpathlineto{\pgfqpoint{2.030190in}{1.037093in}}%
\pgfpathlineto{\pgfqpoint{2.031434in}{1.320550in}}%
\pgfpathlineto{\pgfqpoint{2.031538in}{1.099850in}}%
\pgfpathlineto{\pgfqpoint{2.032470in}{1.451406in}}%
\pgfpathlineto{\pgfqpoint{2.033299in}{1.173432in}}%
\pgfpathlineto{\pgfqpoint{2.033714in}{1.291014in}}%
\pgfpathlineto{\pgfqpoint{2.034336in}{1.440095in}}%
\pgfpathlineto{\pgfqpoint{2.034750in}{1.258650in}}%
\pgfpathlineto{\pgfqpoint{2.035061in}{1.368137in}}%
\pgfpathlineto{\pgfqpoint{2.035476in}{1.156632in}}%
\pgfpathlineto{\pgfqpoint{2.035683in}{1.135684in}}%
\pgfpathlineto{\pgfqpoint{2.035787in}{1.209552in}}%
\pgfpathlineto{\pgfqpoint{2.035890in}{1.282958in}}%
\pgfpathlineto{\pgfqpoint{2.036512in}{1.103920in}}%
\pgfpathlineto{\pgfqpoint{2.036720in}{1.144628in}}%
\pgfpathlineto{\pgfqpoint{2.036823in}{1.142363in}}%
\pgfpathlineto{\pgfqpoint{2.036927in}{1.005856in}}%
\pgfpathlineto{\pgfqpoint{2.037756in}{1.294140in}}%
\pgfpathlineto{\pgfqpoint{2.037860in}{1.172673in}}%
\pgfpathlineto{\pgfqpoint{2.039207in}{1.348834in}}%
\pgfpathlineto{\pgfqpoint{2.040140in}{0.954283in}}%
\pgfpathlineto{\pgfqpoint{2.040554in}{1.180615in}}%
\pgfpathlineto{\pgfqpoint{2.041072in}{1.105083in}}%
\pgfpathlineto{\pgfqpoint{2.041591in}{1.295855in}}%
\pgfpathlineto{\pgfqpoint{2.041280in}{1.097838in}}%
\pgfpathlineto{\pgfqpoint{2.042109in}{1.111628in}}%
\pgfpathlineto{\pgfqpoint{2.042731in}{0.974795in}}%
\pgfpathlineto{\pgfqpoint{2.043456in}{1.019338in}}%
\pgfpathlineto{\pgfqpoint{2.043663in}{1.229423in}}%
\pgfpathlineto{\pgfqpoint{2.044285in}{0.967821in}}%
\pgfpathlineto{\pgfqpoint{2.044596in}{1.066966in}}%
\pgfpathlineto{\pgfqpoint{2.044700in}{1.022095in}}%
\pgfpathlineto{\pgfqpoint{2.045425in}{1.144466in}}%
\pgfpathlineto{\pgfqpoint{2.045529in}{1.224596in}}%
\pgfpathlineto{\pgfqpoint{2.045840in}{0.965139in}}%
\pgfpathlineto{\pgfqpoint{2.046358in}{1.069956in}}%
\pgfpathlineto{\pgfqpoint{2.046462in}{1.031695in}}%
\pgfpathlineto{\pgfqpoint{2.046773in}{1.292971in}}%
\pgfpathlineto{\pgfqpoint{2.047291in}{1.086571in}}%
\pgfpathlineto{\pgfqpoint{2.048431in}{1.358634in}}%
\pgfpathlineto{\pgfqpoint{2.047498in}{1.076992in}}%
\pgfpathlineto{\pgfqpoint{2.048638in}{1.274707in}}%
\pgfpathlineto{\pgfqpoint{2.049260in}{1.163404in}}%
\pgfpathlineto{\pgfqpoint{2.049467in}{1.313285in}}%
\pgfpathlineto{\pgfqpoint{2.049675in}{1.239069in}}%
\pgfpathlineto{\pgfqpoint{2.050296in}{1.380604in}}%
\pgfpathlineto{\pgfqpoint{2.050815in}{1.376154in}}%
\pgfpathlineto{\pgfqpoint{2.051644in}{1.200072in}}%
\pgfpathlineto{\pgfqpoint{2.051955in}{1.316279in}}%
\pgfpathlineto{\pgfqpoint{2.052473in}{1.385223in}}%
\pgfpathlineto{\pgfqpoint{2.052162in}{1.253331in}}%
\pgfpathlineto{\pgfqpoint{2.052991in}{1.274115in}}%
\pgfpathlineto{\pgfqpoint{2.053198in}{1.394513in}}%
\pgfpathlineto{\pgfqpoint{2.053509in}{1.292878in}}%
\pgfpathlineto{\pgfqpoint{2.054131in}{1.128481in}}%
\pgfpathlineto{\pgfqpoint{2.054442in}{1.373267in}}%
\pgfpathlineto{\pgfqpoint{2.054546in}{1.217136in}}%
\pgfpathlineto{\pgfqpoint{2.054960in}{1.190851in}}%
\pgfpathlineto{\pgfqpoint{2.055686in}{1.354885in}}%
\pgfpathlineto{\pgfqpoint{2.056618in}{1.000420in}}%
\pgfpathlineto{\pgfqpoint{2.056929in}{1.065441in}}%
\pgfpathlineto{\pgfqpoint{2.057344in}{1.299613in}}%
\pgfpathlineto{\pgfqpoint{2.057759in}{0.954378in}}%
\pgfpathlineto{\pgfqpoint{2.058173in}{1.232430in}}%
\pgfpathlineto{\pgfqpoint{2.058380in}{1.190120in}}%
\pgfpathlineto{\pgfqpoint{2.058484in}{1.299186in}}%
\pgfpathlineto{\pgfqpoint{2.058588in}{1.385502in}}%
\pgfpathlineto{\pgfqpoint{2.058899in}{1.024742in}}%
\pgfpathlineto{\pgfqpoint{2.059209in}{1.183904in}}%
\pgfpathlineto{\pgfqpoint{2.060453in}{0.919825in}}%
\pgfpathlineto{\pgfqpoint{2.061593in}{1.394143in}}%
\pgfpathlineto{\pgfqpoint{2.062111in}{1.339951in}}%
\pgfpathlineto{\pgfqpoint{2.063251in}{1.160345in}}%
\pgfpathlineto{\pgfqpoint{2.063355in}{1.177966in}}%
\pgfpathlineto{\pgfqpoint{2.064391in}{1.313534in}}%
\pgfpathlineto{\pgfqpoint{2.064495in}{1.225473in}}%
\pgfpathlineto{\pgfqpoint{2.065117in}{1.297255in}}%
\pgfpathlineto{\pgfqpoint{2.065221in}{1.193241in}}%
\pgfpathlineto{\pgfqpoint{2.065739in}{1.055334in}}%
\pgfpathlineto{\pgfqpoint{2.066361in}{1.166955in}}%
\pgfpathlineto{\pgfqpoint{2.066568in}{1.141925in}}%
\pgfpathlineto{\pgfqpoint{2.066672in}{1.305496in}}%
\pgfpathlineto{\pgfqpoint{2.066775in}{1.287606in}}%
\pgfpathlineto{\pgfqpoint{2.066879in}{1.413328in}}%
\pgfpathlineto{\pgfqpoint{2.067293in}{1.141942in}}%
\pgfpathlineto{\pgfqpoint{2.067708in}{1.195909in}}%
\pgfpathlineto{\pgfqpoint{2.068226in}{1.249519in}}%
\pgfpathlineto{\pgfqpoint{2.068744in}{1.100171in}}%
\pgfpathlineto{\pgfqpoint{2.069781in}{1.330264in}}%
\pgfpathlineto{\pgfqpoint{2.068952in}{1.060688in}}%
\pgfpathlineto{\pgfqpoint{2.069884in}{1.161464in}}%
\pgfpathlineto{\pgfqpoint{2.069988in}{1.167264in}}%
\pgfpathlineto{\pgfqpoint{2.070092in}{1.127016in}}%
\pgfpathlineto{\pgfqpoint{2.070817in}{1.268472in}}%
\pgfpathlineto{\pgfqpoint{2.071128in}{1.146051in}}%
\pgfpathlineto{\pgfqpoint{2.071335in}{1.164316in}}%
\pgfpathlineto{\pgfqpoint{2.071439in}{1.117408in}}%
\pgfpathlineto{\pgfqpoint{2.071957in}{1.039460in}}%
\pgfpathlineto{\pgfqpoint{2.072164in}{1.163001in}}%
\pgfpathlineto{\pgfqpoint{2.072475in}{1.302182in}}%
\pgfpathlineto{\pgfqpoint{2.072994in}{1.098155in}}%
\pgfpathlineto{\pgfqpoint{2.073305in}{1.209862in}}%
\pgfpathlineto{\pgfqpoint{2.073926in}{1.043146in}}%
\pgfpathlineto{\pgfqpoint{2.074134in}{1.261384in}}%
\pgfpathlineto{\pgfqpoint{2.074445in}{1.177950in}}%
\pgfpathlineto{\pgfqpoint{2.074859in}{1.262093in}}%
\pgfpathlineto{\pgfqpoint{2.075688in}{1.082791in}}%
\pgfpathlineto{\pgfqpoint{2.076517in}{1.168032in}}%
\pgfpathlineto{\pgfqpoint{2.075999in}{1.013205in}}%
\pgfpathlineto{\pgfqpoint{2.076725in}{1.145329in}}%
\pgfpathlineto{\pgfqpoint{2.076828in}{0.987641in}}%
\pgfpathlineto{\pgfqpoint{2.077450in}{1.288052in}}%
\pgfpathlineto{\pgfqpoint{2.077865in}{1.081703in}}%
\pgfpathlineto{\pgfqpoint{2.078072in}{1.094269in}}%
\pgfpathlineto{\pgfqpoint{2.078176in}{1.076924in}}%
\pgfpathlineto{\pgfqpoint{2.078901in}{0.898619in}}%
\pgfpathlineto{\pgfqpoint{2.078590in}{1.200138in}}%
\pgfpathlineto{\pgfqpoint{2.079212in}{1.024364in}}%
\pgfpathlineto{\pgfqpoint{2.079316in}{1.196338in}}%
\pgfpathlineto{\pgfqpoint{2.079834in}{0.949805in}}%
\pgfpathlineto{\pgfqpoint{2.080352in}{1.103856in}}%
\pgfpathlineto{\pgfqpoint{2.080559in}{1.152578in}}%
\pgfpathlineto{\pgfqpoint{2.080767in}{0.933184in}}%
\pgfpathlineto{\pgfqpoint{2.081285in}{1.079369in}}%
\pgfpathlineto{\pgfqpoint{2.081492in}{0.904058in}}%
\pgfpathlineto{\pgfqpoint{2.082218in}{1.154837in}}%
\pgfpathlineto{\pgfqpoint{2.082321in}{1.083594in}}%
\pgfpathlineto{\pgfqpoint{2.083358in}{1.279964in}}%
\pgfpathlineto{\pgfqpoint{2.083669in}{1.200618in}}%
\pgfpathlineto{\pgfqpoint{2.084394in}{1.381905in}}%
\pgfpathlineto{\pgfqpoint{2.084083in}{1.148775in}}%
\pgfpathlineto{\pgfqpoint{2.084498in}{1.188381in}}%
\pgfpathlineto{\pgfqpoint{2.084912in}{1.049741in}}%
\pgfpathlineto{\pgfqpoint{2.084705in}{1.286161in}}%
\pgfpathlineto{\pgfqpoint{2.085534in}{1.170912in}}%
\pgfpathlineto{\pgfqpoint{2.085949in}{1.040801in}}%
\pgfpathlineto{\pgfqpoint{2.086467in}{1.218086in}}%
\pgfpathlineto{\pgfqpoint{2.087400in}{1.011746in}}%
\pgfpathlineto{\pgfqpoint{2.087607in}{1.178591in}}%
\pgfpathlineto{\pgfqpoint{2.088021in}{1.030731in}}%
\pgfpathlineto{\pgfqpoint{2.088436in}{1.236763in}}%
\pgfpathlineto{\pgfqpoint{2.088747in}{1.155119in}}%
\pgfpathlineto{\pgfqpoint{2.088954in}{1.132065in}}%
\pgfpathlineto{\pgfqpoint{2.089058in}{1.194668in}}%
\pgfpathlineto{\pgfqpoint{2.089887in}{1.382610in}}%
\pgfpathlineto{\pgfqpoint{2.089680in}{1.177003in}}%
\pgfpathlineto{\pgfqpoint{2.090301in}{1.296536in}}%
\pgfpathlineto{\pgfqpoint{2.091545in}{1.099663in}}%
\pgfpathlineto{\pgfqpoint{2.091131in}{1.299983in}}%
\pgfpathlineto{\pgfqpoint{2.091649in}{1.146214in}}%
\pgfpathlineto{\pgfqpoint{2.092374in}{1.421136in}}%
\pgfpathlineto{\pgfqpoint{2.093100in}{1.401254in}}%
\pgfpathlineto{\pgfqpoint{2.093203in}{1.413437in}}%
\pgfpathlineto{\pgfqpoint{2.093307in}{1.366258in}}%
\pgfpathlineto{\pgfqpoint{2.094343in}{1.274001in}}%
\pgfpathlineto{\pgfqpoint{2.094136in}{1.372179in}}%
\pgfpathlineto{\pgfqpoint{2.094447in}{1.277786in}}%
\pgfpathlineto{\pgfqpoint{2.094551in}{1.347821in}}%
\pgfpathlineto{\pgfqpoint{2.094758in}{1.111162in}}%
\pgfpathlineto{\pgfqpoint{2.095380in}{1.288376in}}%
\pgfpathlineto{\pgfqpoint{2.096002in}{1.096988in}}%
\pgfpathlineto{\pgfqpoint{2.095587in}{1.310859in}}%
\pgfpathlineto{\pgfqpoint{2.096520in}{1.162656in}}%
\pgfpathlineto{\pgfqpoint{2.096727in}{1.128214in}}%
\pgfpathlineto{\pgfqpoint{2.096934in}{1.171706in}}%
\pgfpathlineto{\pgfqpoint{2.097453in}{1.391085in}}%
\pgfpathlineto{\pgfqpoint{2.097245in}{1.123984in}}%
\pgfpathlineto{\pgfqpoint{2.097971in}{1.199801in}}%
\pgfpathlineto{\pgfqpoint{2.098385in}{1.104678in}}%
\pgfpathlineto{\pgfqpoint{2.099111in}{1.167760in}}%
\pgfpathlineto{\pgfqpoint{2.099525in}{1.321668in}}%
\pgfpathlineto{\pgfqpoint{2.099940in}{1.239712in}}%
\pgfpathlineto{\pgfqpoint{2.100355in}{0.980587in}}%
\pgfpathlineto{\pgfqpoint{2.101080in}{1.013735in}}%
\pgfpathlineto{\pgfqpoint{2.101806in}{1.261767in}}%
\pgfpathlineto{\pgfqpoint{2.102220in}{1.140826in}}%
\pgfpathlineto{\pgfqpoint{2.102427in}{1.008500in}}%
\pgfpathlineto{\pgfqpoint{2.102946in}{1.224204in}}%
\pgfpathlineto{\pgfqpoint{2.103256in}{1.304229in}}%
\pgfpathlineto{\pgfqpoint{2.103153in}{1.165322in}}%
\pgfpathlineto{\pgfqpoint{2.103567in}{1.180527in}}%
\pgfpathlineto{\pgfqpoint{2.104189in}{1.029291in}}%
\pgfpathlineto{\pgfqpoint{2.103878in}{1.200157in}}%
\pgfpathlineto{\pgfqpoint{2.104500in}{1.102062in}}%
\pgfpathlineto{\pgfqpoint{2.105537in}{1.340286in}}%
\pgfpathlineto{\pgfqpoint{2.104915in}{1.058980in}}%
\pgfpathlineto{\pgfqpoint{2.105744in}{1.316547in}}%
\pgfpathlineto{\pgfqpoint{2.106262in}{1.388030in}}%
\pgfpathlineto{\pgfqpoint{2.106884in}{1.144341in}}%
\pgfpathlineto{\pgfqpoint{2.107506in}{1.302340in}}%
\pgfpathlineto{\pgfqpoint{2.107920in}{1.220458in}}%
\pgfpathlineto{\pgfqpoint{2.108749in}{1.126754in}}%
\pgfpathlineto{\pgfqpoint{2.108231in}{1.263338in}}%
\pgfpathlineto{\pgfqpoint{2.109060in}{1.179188in}}%
\pgfpathlineto{\pgfqpoint{2.109579in}{1.009140in}}%
\pgfpathlineto{\pgfqpoint{2.109786in}{1.208921in}}%
\pgfpathlineto{\pgfqpoint{2.110304in}{1.093522in}}%
\pgfpathlineto{\pgfqpoint{2.110408in}{1.084309in}}%
\pgfpathlineto{\pgfqpoint{2.111030in}{1.043189in}}%
\pgfpathlineto{\pgfqpoint{2.111548in}{1.302381in}}%
\pgfpathlineto{\pgfqpoint{2.111962in}{1.091393in}}%
\pgfpathlineto{\pgfqpoint{2.112377in}{1.312278in}}%
\pgfpathlineto{\pgfqpoint{2.112791in}{1.126743in}}%
\pgfpathlineto{\pgfqpoint{2.112999in}{1.248157in}}%
\pgfpathlineto{\pgfqpoint{2.113621in}{1.035676in}}%
\pgfpathlineto{\pgfqpoint{2.113931in}{1.216868in}}%
\pgfpathlineto{\pgfqpoint{2.114139in}{1.090208in}}%
\pgfpathlineto{\pgfqpoint{2.114761in}{1.299799in}}%
\pgfpathlineto{\pgfqpoint{2.115071in}{1.141111in}}%
\pgfpathlineto{\pgfqpoint{2.115175in}{1.223705in}}%
\pgfpathlineto{\pgfqpoint{2.115590in}{1.075890in}}%
\pgfpathlineto{\pgfqpoint{2.116108in}{1.165385in}}%
\pgfpathlineto{\pgfqpoint{2.117144in}{0.980450in}}%
\pgfpathlineto{\pgfqpoint{2.116937in}{1.204379in}}%
\pgfpathlineto{\pgfqpoint{2.117248in}{1.084037in}}%
\pgfpathlineto{\pgfqpoint{2.117455in}{1.041793in}}%
\pgfpathlineto{\pgfqpoint{2.117766in}{0.982767in}}%
\pgfpathlineto{\pgfqpoint{2.118699in}{1.251735in}}%
\pgfpathlineto{\pgfqpoint{2.119528in}{1.050424in}}%
\pgfpathlineto{\pgfqpoint{2.120046in}{1.117212in}}%
\pgfpathlineto{\pgfqpoint{2.121186in}{1.362472in}}%
\pgfpathlineto{\pgfqpoint{2.121290in}{1.258979in}}%
\pgfpathlineto{\pgfqpoint{2.122326in}{1.053507in}}%
\pgfpathlineto{\pgfqpoint{2.121704in}{1.327128in}}%
\pgfpathlineto{\pgfqpoint{2.122430in}{1.134905in}}%
\pgfpathlineto{\pgfqpoint{2.122948in}{1.125626in}}%
\pgfpathlineto{\pgfqpoint{2.123674in}{1.283747in}}%
\pgfpathlineto{\pgfqpoint{2.123777in}{1.288212in}}%
\pgfpathlineto{\pgfqpoint{2.124088in}{1.048975in}}%
\pgfpathlineto{\pgfqpoint{2.125125in}{1.085925in}}%
\pgfpathlineto{\pgfqpoint{2.125228in}{1.147809in}}%
\pgfpathlineto{\pgfqpoint{2.125643in}{0.983900in}}%
\pgfpathlineto{\pgfqpoint{2.126161in}{1.047608in}}%
\pgfpathlineto{\pgfqpoint{2.126368in}{1.127004in}}%
\pgfpathlineto{\pgfqpoint{2.126472in}{1.027911in}}%
\pgfpathlineto{\pgfqpoint{2.126576in}{1.061358in}}%
\pgfpathlineto{\pgfqpoint{2.126679in}{0.844853in}}%
\pgfpathlineto{\pgfqpoint{2.127197in}{1.103215in}}%
\pgfpathlineto{\pgfqpoint{2.127716in}{0.957639in}}%
\pgfpathlineto{\pgfqpoint{2.128752in}{1.187785in}}%
\pgfpathlineto{\pgfqpoint{2.128130in}{0.869266in}}%
\pgfpathlineto{\pgfqpoint{2.128959in}{1.146985in}}%
\pgfpathlineto{\pgfqpoint{2.129063in}{0.972653in}}%
\pgfpathlineto{\pgfqpoint{2.129685in}{1.267609in}}%
\pgfpathlineto{\pgfqpoint{2.129996in}{1.188661in}}%
\pgfpathlineto{\pgfqpoint{2.130099in}{1.099626in}}%
\pgfpathlineto{\pgfqpoint{2.130825in}{1.426485in}}%
\pgfpathlineto{\pgfqpoint{2.131136in}{1.108190in}}%
\pgfpathlineto{\pgfqpoint{2.131239in}{1.320640in}}%
\pgfpathlineto{\pgfqpoint{2.132276in}{1.184329in}}%
\pgfpathlineto{\pgfqpoint{2.132794in}{1.287740in}}%
\pgfpathlineto{\pgfqpoint{2.133001in}{1.080460in}}%
\pgfpathlineto{\pgfqpoint{2.133416in}{1.283747in}}%
\pgfpathlineto{\pgfqpoint{2.134245in}{0.881363in}}%
\pgfpathlineto{\pgfqpoint{2.133623in}{1.354117in}}%
\pgfpathlineto{\pgfqpoint{2.134659in}{1.133211in}}%
\pgfpathlineto{\pgfqpoint{2.134867in}{1.179067in}}%
\pgfpathlineto{\pgfqpoint{2.135592in}{1.340320in}}%
\pgfpathlineto{\pgfqpoint{2.135178in}{1.113041in}}%
\pgfpathlineto{\pgfqpoint{2.135903in}{1.211648in}}%
\pgfpathlineto{\pgfqpoint{2.136940in}{1.114284in}}%
\pgfpathlineto{\pgfqpoint{2.136421in}{1.280397in}}%
\pgfpathlineto{\pgfqpoint{2.137043in}{1.182364in}}%
\pgfpathlineto{\pgfqpoint{2.137250in}{1.284893in}}%
\pgfpathlineto{\pgfqpoint{2.137665in}{1.147779in}}%
\pgfpathlineto{\pgfqpoint{2.138080in}{1.205940in}}%
\pgfpathlineto{\pgfqpoint{2.138494in}{1.074166in}}%
\pgfpathlineto{\pgfqpoint{2.138390in}{1.235709in}}%
\pgfpathlineto{\pgfqpoint{2.139220in}{1.148246in}}%
\pgfpathlineto{\pgfqpoint{2.140360in}{1.317684in}}%
\pgfpathlineto{\pgfqpoint{2.139427in}{1.070891in}}%
\pgfpathlineto{\pgfqpoint{2.140463in}{1.215853in}}%
\pgfpathlineto{\pgfqpoint{2.140567in}{1.025342in}}%
\pgfpathlineto{\pgfqpoint{2.141189in}{1.254406in}}%
\pgfpathlineto{\pgfqpoint{2.141500in}{1.121773in}}%
\pgfpathlineto{\pgfqpoint{2.141603in}{1.301874in}}%
\pgfpathlineto{\pgfqpoint{2.142122in}{1.071202in}}%
\pgfpathlineto{\pgfqpoint{2.142536in}{1.208248in}}%
\pgfpathlineto{\pgfqpoint{2.143054in}{1.045700in}}%
\pgfpathlineto{\pgfqpoint{2.143572in}{1.227010in}}%
\pgfpathlineto{\pgfqpoint{2.143883in}{1.048724in}}%
\pgfpathlineto{\pgfqpoint{2.143780in}{1.244970in}}%
\pgfpathlineto{\pgfqpoint{2.144713in}{1.195733in}}%
\pgfpathlineto{\pgfqpoint{2.144816in}{1.210461in}}%
\pgfpathlineto{\pgfqpoint{2.145023in}{1.097343in}}%
\pgfpathlineto{\pgfqpoint{2.145127in}{1.142645in}}%
\pgfpathlineto{\pgfqpoint{2.145438in}{0.994524in}}%
\pgfpathlineto{\pgfqpoint{2.145645in}{1.195558in}}%
\pgfpathlineto{\pgfqpoint{2.145853in}{1.167754in}}%
\pgfpathlineto{\pgfqpoint{2.145956in}{1.293141in}}%
\pgfpathlineto{\pgfqpoint{2.146060in}{1.060327in}}%
\pgfpathlineto{\pgfqpoint{2.146889in}{1.232469in}}%
\pgfpathlineto{\pgfqpoint{2.147200in}{1.110585in}}%
\pgfpathlineto{\pgfqpoint{2.147822in}{1.241113in}}%
\pgfpathlineto{\pgfqpoint{2.148029in}{1.171301in}}%
\pgfpathlineto{\pgfqpoint{2.148651in}{1.063132in}}%
\pgfpathlineto{\pgfqpoint{2.148340in}{1.183793in}}%
\pgfpathlineto{\pgfqpoint{2.148858in}{1.162444in}}%
\pgfpathlineto{\pgfqpoint{2.148962in}{1.225361in}}%
\pgfpathlineto{\pgfqpoint{2.149584in}{1.008183in}}%
\pgfpathlineto{\pgfqpoint{2.149791in}{1.131125in}}%
\pgfpathlineto{\pgfqpoint{2.150724in}{1.009398in}}%
\pgfpathlineto{\pgfqpoint{2.151449in}{0.994866in}}%
\pgfpathlineto{\pgfqpoint{2.151864in}{1.231478in}}%
\pgfpathlineto{\pgfqpoint{2.152900in}{1.110417in}}%
\pgfpathlineto{\pgfqpoint{2.152278in}{1.302053in}}%
\pgfpathlineto{\pgfqpoint{2.153211in}{1.134263in}}%
\pgfpathlineto{\pgfqpoint{2.154351in}{1.298940in}}%
\pgfpathlineto{\pgfqpoint{2.154455in}{1.251192in}}%
\pgfpathlineto{\pgfqpoint{2.154662in}{1.045321in}}%
\pgfpathlineto{\pgfqpoint{2.155595in}{1.094382in}}%
\pgfpathlineto{\pgfqpoint{2.155802in}{1.091413in}}%
\pgfpathlineto{\pgfqpoint{2.155906in}{1.094496in}}%
\pgfpathlineto{\pgfqpoint{2.156942in}{1.305651in}}%
\pgfpathlineto{\pgfqpoint{2.157253in}{1.302164in}}%
\pgfpathlineto{\pgfqpoint{2.157564in}{1.209379in}}%
\pgfpathlineto{\pgfqpoint{2.157875in}{1.317246in}}%
\pgfpathlineto{\pgfqpoint{2.158082in}{1.314153in}}%
\pgfpathlineto{\pgfqpoint{2.158186in}{1.369213in}}%
\pgfpathlineto{\pgfqpoint{2.158704in}{1.206909in}}%
\pgfpathlineto{\pgfqpoint{2.159015in}{1.091215in}}%
\pgfpathlineto{\pgfqpoint{2.159326in}{1.221503in}}%
\pgfpathlineto{\pgfqpoint{2.159844in}{1.154277in}}%
\pgfpathlineto{\pgfqpoint{2.160155in}{1.038290in}}%
\pgfpathlineto{\pgfqpoint{2.160569in}{1.280069in}}%
\pgfpathlineto{\pgfqpoint{2.160880in}{1.041684in}}%
\pgfpathlineto{\pgfqpoint{2.161502in}{1.347643in}}%
\pgfpathlineto{\pgfqpoint{2.162020in}{1.124887in}}%
\pgfpathlineto{\pgfqpoint{2.162850in}{1.324265in}}%
\pgfpathlineto{\pgfqpoint{2.162642in}{1.092313in}}%
\pgfpathlineto{\pgfqpoint{2.163264in}{1.323470in}}%
\pgfpathlineto{\pgfqpoint{2.164197in}{1.081560in}}%
\pgfpathlineto{\pgfqpoint{2.164404in}{1.261921in}}%
\pgfpathlineto{\pgfqpoint{2.164611in}{1.179915in}}%
\pgfpathlineto{\pgfqpoint{2.164922in}{1.328483in}}%
\pgfpathlineto{\pgfqpoint{2.165648in}{1.196566in}}%
\pgfpathlineto{\pgfqpoint{2.165751in}{1.165923in}}%
\pgfpathlineto{\pgfqpoint{2.165959in}{1.247026in}}%
\pgfpathlineto{\pgfqpoint{2.166581in}{1.209108in}}%
\pgfpathlineto{\pgfqpoint{2.166684in}{1.393910in}}%
\pgfpathlineto{\pgfqpoint{2.167721in}{1.270208in}}%
\pgfpathlineto{\pgfqpoint{2.168239in}{1.098811in}}%
\pgfpathlineto{\pgfqpoint{2.168342in}{1.348789in}}%
\pgfpathlineto{\pgfqpoint{2.168757in}{1.314024in}}%
\pgfpathlineto{\pgfqpoint{2.169172in}{1.364114in}}%
\pgfpathlineto{\pgfqpoint{2.168964in}{1.264097in}}%
\pgfpathlineto{\pgfqpoint{2.169482in}{1.281556in}}%
\pgfpathlineto{\pgfqpoint{2.169793in}{1.381637in}}%
\pgfpathlineto{\pgfqpoint{2.170623in}{1.196603in}}%
\pgfpathlineto{\pgfqpoint{2.171659in}{1.382893in}}%
\pgfpathlineto{\pgfqpoint{2.170933in}{1.163420in}}%
\pgfpathlineto{\pgfqpoint{2.171763in}{1.286547in}}%
\pgfpathlineto{\pgfqpoint{2.171970in}{1.212079in}}%
\pgfpathlineto{\pgfqpoint{2.172177in}{1.287897in}}%
\pgfpathlineto{\pgfqpoint{2.172281in}{1.276562in}}%
\pgfpathlineto{\pgfqpoint{2.172384in}{1.357081in}}%
\pgfpathlineto{\pgfqpoint{2.172695in}{1.187344in}}%
\pgfpathlineto{\pgfqpoint{2.173214in}{1.201883in}}%
\pgfpathlineto{\pgfqpoint{2.173524in}{1.331275in}}%
\pgfpathlineto{\pgfqpoint{2.174457in}{1.030788in}}%
\pgfpathlineto{\pgfqpoint{2.175390in}{1.247222in}}%
\pgfpathlineto{\pgfqpoint{2.175908in}{1.099663in}}%
\pgfpathlineto{\pgfqpoint{2.176012in}{0.984257in}}%
\pgfpathlineto{\pgfqpoint{2.176323in}{1.230210in}}%
\pgfpathlineto{\pgfqpoint{2.176841in}{1.224256in}}%
\pgfpathlineto{\pgfqpoint{2.177048in}{1.270661in}}%
\pgfpathlineto{\pgfqpoint{2.177255in}{1.178146in}}%
\pgfpathlineto{\pgfqpoint{2.178188in}{1.078682in}}%
\pgfpathlineto{\pgfqpoint{2.177670in}{1.320460in}}%
\pgfpathlineto{\pgfqpoint{2.178396in}{1.124250in}}%
\pgfpathlineto{\pgfqpoint{2.178706in}{1.163030in}}%
\pgfpathlineto{\pgfqpoint{2.179328in}{0.930035in}}%
\pgfpathlineto{\pgfqpoint{2.179846in}{1.110066in}}%
\pgfpathlineto{\pgfqpoint{2.180779in}{1.051303in}}%
\pgfpathlineto{\pgfqpoint{2.180987in}{1.259432in}}%
\pgfpathlineto{\pgfqpoint{2.181505in}{1.287431in}}%
\pgfpathlineto{\pgfqpoint{2.182334in}{1.069756in}}%
\pgfpathlineto{\pgfqpoint{2.182437in}{1.245490in}}%
\pgfpathlineto{\pgfqpoint{2.183370in}{1.166020in}}%
\pgfpathlineto{\pgfqpoint{2.183681in}{1.033779in}}%
\pgfpathlineto{\pgfqpoint{2.184096in}{1.234665in}}%
\pgfpathlineto{\pgfqpoint{2.184407in}{1.192388in}}%
\pgfpathlineto{\pgfqpoint{2.184510in}{1.264096in}}%
\pgfpathlineto{\pgfqpoint{2.185236in}{1.009996in}}%
\pgfpathlineto{\pgfqpoint{2.185443in}{1.192932in}}%
\pgfpathlineto{\pgfqpoint{2.185858in}{1.054163in}}%
\pgfpathlineto{\pgfqpoint{2.186687in}{1.061244in}}%
\pgfpathlineto{\pgfqpoint{2.186790in}{1.056169in}}%
\pgfpathlineto{\pgfqpoint{2.186894in}{0.832213in}}%
\pgfpathlineto{\pgfqpoint{2.187827in}{1.147448in}}%
\pgfpathlineto{\pgfqpoint{2.188863in}{1.014403in}}%
\pgfpathlineto{\pgfqpoint{2.188656in}{1.221400in}}%
\pgfpathlineto{\pgfqpoint{2.188967in}{1.051797in}}%
\pgfpathlineto{\pgfqpoint{2.189900in}{0.939228in}}%
\pgfpathlineto{\pgfqpoint{2.189485in}{1.159707in}}%
\pgfpathlineto{\pgfqpoint{2.190003in}{1.029869in}}%
\pgfpathlineto{\pgfqpoint{2.190936in}{1.246729in}}%
\pgfpathlineto{\pgfqpoint{2.191351in}{1.216503in}}%
\pgfpathlineto{\pgfqpoint{2.191558in}{1.226738in}}%
\pgfpathlineto{\pgfqpoint{2.191661in}{1.386981in}}%
\pgfpathlineto{\pgfqpoint{2.192594in}{1.223496in}}%
\pgfpathlineto{\pgfqpoint{2.193009in}{1.346303in}}%
\pgfpathlineto{\pgfqpoint{2.193216in}{1.174024in}}%
\pgfpathlineto{\pgfqpoint{2.193631in}{1.234523in}}%
\pgfpathlineto{\pgfqpoint{2.194252in}{1.166016in}}%
\pgfpathlineto{\pgfqpoint{2.194356in}{1.240614in}}%
\pgfpathlineto{\pgfqpoint{2.194460in}{1.228361in}}%
\pgfpathlineto{\pgfqpoint{2.194874in}{1.423845in}}%
\pgfpathlineto{\pgfqpoint{2.195600in}{1.306880in}}%
\pgfpathlineto{\pgfqpoint{2.195911in}{1.189764in}}%
\pgfpathlineto{\pgfqpoint{2.196636in}{1.290793in}}%
\pgfpathlineto{\pgfqpoint{2.196947in}{1.130967in}}%
\pgfpathlineto{\pgfqpoint{2.197880in}{1.391565in}}%
\pgfpathlineto{\pgfqpoint{2.198813in}{1.012397in}}%
\pgfpathlineto{\pgfqpoint{2.199227in}{1.112282in}}%
\pgfpathlineto{\pgfqpoint{2.199953in}{1.322424in}}%
\pgfpathlineto{\pgfqpoint{2.199642in}{1.057741in}}%
\pgfpathlineto{\pgfqpoint{2.200367in}{1.215198in}}%
\pgfpathlineto{\pgfqpoint{2.201196in}{0.938014in}}%
\pgfpathlineto{\pgfqpoint{2.201715in}{1.073554in}}%
\pgfpathlineto{\pgfqpoint{2.202440in}{1.173193in}}%
\pgfpathlineto{\pgfqpoint{2.202025in}{0.937609in}}%
\pgfpathlineto{\pgfqpoint{2.202647in}{1.045499in}}%
\pgfpathlineto{\pgfqpoint{2.202751in}{0.943778in}}%
\pgfpathlineto{\pgfqpoint{2.203476in}{1.303298in}}%
\pgfpathlineto{\pgfqpoint{2.204927in}{0.912199in}}%
\pgfpathlineto{\pgfqpoint{2.205031in}{1.013928in}}%
\pgfpathlineto{\pgfqpoint{2.205135in}{1.002056in}}%
\pgfpathlineto{\pgfqpoint{2.205238in}{1.011628in}}%
\pgfpathlineto{\pgfqpoint{2.205653in}{1.193733in}}%
\pgfpathlineto{\pgfqpoint{2.206275in}{0.975448in}}%
\pgfpathlineto{\pgfqpoint{2.206793in}{1.082485in}}%
\pgfpathlineto{\pgfqpoint{2.207311in}{0.952408in}}%
\pgfpathlineto{\pgfqpoint{2.207415in}{0.886928in}}%
\pgfpathlineto{\pgfqpoint{2.208037in}{1.102173in}}%
\pgfpathlineto{\pgfqpoint{2.208451in}{0.888431in}}%
\pgfpathlineto{\pgfqpoint{2.209280in}{1.134538in}}%
\pgfpathlineto{\pgfqpoint{2.209591in}{1.113650in}}%
\pgfpathlineto{\pgfqpoint{2.209798in}{0.945709in}}%
\pgfpathlineto{\pgfqpoint{2.210628in}{1.103235in}}%
\pgfpathlineto{\pgfqpoint{2.211146in}{1.178538in}}%
\pgfpathlineto{\pgfqpoint{2.211457in}{1.025362in}}%
\pgfpathlineto{\pgfqpoint{2.211664in}{1.119865in}}%
\pgfpathlineto{\pgfqpoint{2.211768in}{1.005255in}}%
\pgfpathlineto{\pgfqpoint{2.212700in}{1.185246in}}%
\pgfpathlineto{\pgfqpoint{2.213529in}{1.071509in}}%
\pgfpathlineto{\pgfqpoint{2.213115in}{1.186726in}}%
\pgfpathlineto{\pgfqpoint{2.213737in}{1.096190in}}%
\pgfpathlineto{\pgfqpoint{2.213840in}{1.199678in}}%
\pgfpathlineto{\pgfqpoint{2.214151in}{1.067710in}}%
\pgfpathlineto{\pgfqpoint{2.214773in}{1.157161in}}%
\pgfpathlineto{\pgfqpoint{2.215188in}{1.027970in}}%
\pgfpathlineto{\pgfqpoint{2.215602in}{1.245052in}}%
\pgfpathlineto{\pgfqpoint{2.215810in}{1.146456in}}%
\pgfpathlineto{\pgfqpoint{2.216742in}{1.299356in}}%
\pgfpathlineto{\pgfqpoint{2.216431in}{1.023020in}}%
\pgfpathlineto{\pgfqpoint{2.216950in}{1.202849in}}%
\pgfpathlineto{\pgfqpoint{2.217261in}{1.347852in}}%
\pgfpathlineto{\pgfqpoint{2.217779in}{1.111161in}}%
\pgfpathlineto{\pgfqpoint{2.218090in}{1.249277in}}%
\pgfpathlineto{\pgfqpoint{2.218919in}{1.134427in}}%
\pgfpathlineto{\pgfqpoint{2.218401in}{1.322333in}}%
\pgfpathlineto{\pgfqpoint{2.219022in}{1.240183in}}%
\pgfpathlineto{\pgfqpoint{2.219644in}{1.335324in}}%
\pgfpathlineto{\pgfqpoint{2.219748in}{1.154867in}}%
\pgfpathlineto{\pgfqpoint{2.219955in}{1.168419in}}%
\pgfpathlineto{\pgfqpoint{2.220473in}{1.041837in}}%
\pgfpathlineto{\pgfqpoint{2.220577in}{1.296866in}}%
\pgfpathlineto{\pgfqpoint{2.220992in}{1.174618in}}%
\pgfpathlineto{\pgfqpoint{2.221302in}{1.027686in}}%
\pgfpathlineto{\pgfqpoint{2.221406in}{1.135849in}}%
\pgfpathlineto{\pgfqpoint{2.222235in}{1.362039in}}%
\pgfpathlineto{\pgfqpoint{2.222546in}{1.248256in}}%
\pgfpathlineto{\pgfqpoint{2.223272in}{1.188674in}}%
\pgfpathlineto{\pgfqpoint{2.223064in}{1.391287in}}%
\pgfpathlineto{\pgfqpoint{2.223375in}{1.249184in}}%
\pgfpathlineto{\pgfqpoint{2.223479in}{1.369953in}}%
\pgfpathlineto{\pgfqpoint{2.224204in}{1.080966in}}%
\pgfpathlineto{\pgfqpoint{2.224308in}{1.085035in}}%
\pgfpathlineto{\pgfqpoint{2.225655in}{1.336370in}}%
\pgfpathlineto{\pgfqpoint{2.226174in}{1.036963in}}%
\pgfpathlineto{\pgfqpoint{2.226899in}{1.067780in}}%
\pgfpathlineto{\pgfqpoint{2.227003in}{1.088675in}}%
\pgfpathlineto{\pgfqpoint{2.227106in}{0.987020in}}%
\pgfpathlineto{\pgfqpoint{2.227210in}{1.077821in}}%
\pgfpathlineto{\pgfqpoint{2.227832in}{1.159750in}}%
\pgfpathlineto{\pgfqpoint{2.228454in}{0.823538in}}%
\pgfpathlineto{\pgfqpoint{2.228868in}{1.108863in}}%
\pgfpathlineto{\pgfqpoint{2.229179in}{0.818206in}}%
\pgfpathlineto{\pgfqpoint{2.229594in}{1.070951in}}%
\pgfpathlineto{\pgfqpoint{2.229697in}{1.098118in}}%
\pgfpathlineto{\pgfqpoint{2.230112in}{1.041697in}}%
\pgfpathlineto{\pgfqpoint{2.230423in}{1.084375in}}%
\pgfpathlineto{\pgfqpoint{2.231148in}{0.928186in}}%
\pgfpathlineto{\pgfqpoint{2.231459in}{0.942878in}}%
\pgfpathlineto{\pgfqpoint{2.231977in}{0.939613in}}%
\pgfpathlineto{\pgfqpoint{2.232599in}{1.195934in}}%
\pgfpathlineto{\pgfqpoint{2.233325in}{1.024154in}}%
\pgfpathlineto{\pgfqpoint{2.233739in}{1.064270in}}%
\pgfpathlineto{\pgfqpoint{2.234879in}{1.298802in}}%
\pgfpathlineto{\pgfqpoint{2.235087in}{1.221741in}}%
\pgfpathlineto{\pgfqpoint{2.235605in}{0.976830in}}%
\pgfpathlineto{\pgfqpoint{2.236019in}{1.259796in}}%
\pgfpathlineto{\pgfqpoint{2.236227in}{1.097983in}}%
\pgfpathlineto{\pgfqpoint{2.237263in}{0.928293in}}%
\pgfpathlineto{\pgfqpoint{2.236848in}{1.165945in}}%
\pgfpathlineto{\pgfqpoint{2.237367in}{1.034721in}}%
\pgfpathlineto{\pgfqpoint{2.237781in}{1.122917in}}%
\pgfpathlineto{\pgfqpoint{2.238092in}{1.006549in}}%
\pgfpathlineto{\pgfqpoint{2.238299in}{1.050451in}}%
\pgfpathlineto{\pgfqpoint{2.238403in}{1.009469in}}%
\pgfpathlineto{\pgfqpoint{2.239025in}{1.167062in}}%
\pgfpathlineto{\pgfqpoint{2.239129in}{1.178816in}}%
\pgfpathlineto{\pgfqpoint{2.239439in}{1.220990in}}%
\pgfpathlineto{\pgfqpoint{2.240269in}{0.964603in}}%
\pgfpathlineto{\pgfqpoint{2.240372in}{0.958453in}}%
\pgfpathlineto{\pgfqpoint{2.240476in}{0.998527in}}%
\pgfpathlineto{\pgfqpoint{2.240580in}{0.961663in}}%
\pgfpathlineto{\pgfqpoint{2.240890in}{1.222200in}}%
\pgfpathlineto{\pgfqpoint{2.241823in}{1.062646in}}%
\pgfpathlineto{\pgfqpoint{2.242445in}{1.140494in}}%
\pgfpathlineto{\pgfqpoint{2.242341in}{1.028067in}}%
\pgfpathlineto{\pgfqpoint{2.242860in}{1.036421in}}%
\pgfpathlineto{\pgfqpoint{2.243171in}{1.143979in}}%
\pgfpathlineto{\pgfqpoint{2.243585in}{0.919654in}}%
\pgfpathlineto{\pgfqpoint{2.243896in}{1.084628in}}%
\pgfpathlineto{\pgfqpoint{2.244518in}{1.010857in}}%
\pgfpathlineto{\pgfqpoint{2.244829in}{1.143747in}}%
\pgfpathlineto{\pgfqpoint{2.245036in}{1.040969in}}%
\pgfpathlineto{\pgfqpoint{2.245347in}{1.139569in}}%
\pgfpathlineto{\pgfqpoint{2.245451in}{0.935789in}}%
\pgfpathlineto{\pgfqpoint{2.245658in}{0.967668in}}%
\pgfpathlineto{\pgfqpoint{2.245762in}{0.805251in}}%
\pgfpathlineto{\pgfqpoint{2.246280in}{1.138160in}}%
\pgfpathlineto{\pgfqpoint{2.246798in}{0.886627in}}%
\pgfpathlineto{\pgfqpoint{2.247627in}{1.175342in}}%
\pgfpathlineto{\pgfqpoint{2.248145in}{1.154665in}}%
\pgfpathlineto{\pgfqpoint{2.248353in}{1.060412in}}%
\pgfpathlineto{\pgfqpoint{2.248871in}{1.180502in}}%
\pgfpathlineto{\pgfqpoint{2.248974in}{1.126100in}}%
\pgfpathlineto{\pgfqpoint{2.249285in}{1.246399in}}%
\pgfpathlineto{\pgfqpoint{2.249907in}{1.038981in}}%
\pgfpathlineto{\pgfqpoint{2.250011in}{1.068047in}}%
\pgfpathlineto{\pgfqpoint{2.250114in}{1.030678in}}%
\pgfpathlineto{\pgfqpoint{2.250529in}{1.206528in}}%
\pgfpathlineto{\pgfqpoint{2.250944in}{1.098915in}}%
\pgfpathlineto{\pgfqpoint{2.251047in}{1.104087in}}%
\pgfpathlineto{\pgfqpoint{2.251151in}{1.080565in}}%
\pgfpathlineto{\pgfqpoint{2.252187in}{0.869461in}}%
\pgfpathlineto{\pgfqpoint{2.251358in}{1.121130in}}%
\pgfpathlineto{\pgfqpoint{2.252394in}{0.988166in}}%
\pgfpathlineto{\pgfqpoint{2.252705in}{1.024300in}}%
\pgfpathlineto{\pgfqpoint{2.252809in}{0.973526in}}%
\pgfpathlineto{\pgfqpoint{2.252913in}{1.105353in}}%
\pgfpathlineto{\pgfqpoint{2.253431in}{0.959859in}}%
\pgfpathlineto{\pgfqpoint{2.253949in}{1.045044in}}%
\pgfpathlineto{\pgfqpoint{2.254053in}{0.919056in}}%
\pgfpathlineto{\pgfqpoint{2.254675in}{1.196130in}}%
\pgfpathlineto{\pgfqpoint{2.254985in}{1.135106in}}%
\pgfpathlineto{\pgfqpoint{2.255711in}{0.996672in}}%
\pgfpathlineto{\pgfqpoint{2.255504in}{1.197893in}}%
\pgfpathlineto{\pgfqpoint{2.256126in}{1.094109in}}%
\pgfpathlineto{\pgfqpoint{2.256851in}{1.245127in}}%
\pgfpathlineto{\pgfqpoint{2.256436in}{1.035765in}}%
\pgfpathlineto{\pgfqpoint{2.257162in}{1.145128in}}%
\pgfpathlineto{\pgfqpoint{2.257369in}{1.059564in}}%
\pgfpathlineto{\pgfqpoint{2.257473in}{1.394853in}}%
\pgfpathlineto{\pgfqpoint{2.257887in}{1.313420in}}%
\pgfpathlineto{\pgfqpoint{2.258198in}{1.435624in}}%
\pgfpathlineto{\pgfqpoint{2.258302in}{1.240431in}}%
\pgfpathlineto{\pgfqpoint{2.258613in}{1.112217in}}%
\pgfpathlineto{\pgfqpoint{2.259027in}{1.358275in}}%
\pgfpathlineto{\pgfqpoint{2.259131in}{1.275985in}}%
\pgfpathlineto{\pgfqpoint{2.259338in}{1.407144in}}%
\pgfpathlineto{\pgfqpoint{2.259546in}{1.147420in}}%
\pgfpathlineto{\pgfqpoint{2.260167in}{1.233162in}}%
\pgfpathlineto{\pgfqpoint{2.261100in}{1.119522in}}%
\pgfpathlineto{\pgfqpoint{2.260686in}{1.243985in}}%
\pgfpathlineto{\pgfqpoint{2.261308in}{1.127208in}}%
\pgfpathlineto{\pgfqpoint{2.261515in}{1.083097in}}%
\pgfpathlineto{\pgfqpoint{2.261618in}{1.189162in}}%
\pgfpathlineto{\pgfqpoint{2.261826in}{1.171219in}}%
\pgfpathlineto{\pgfqpoint{2.262240in}{1.347489in}}%
\pgfpathlineto{\pgfqpoint{2.262862in}{1.146610in}}%
\pgfpathlineto{\pgfqpoint{2.262966in}{1.326384in}}%
\pgfpathlineto{\pgfqpoint{2.263277in}{1.107767in}}%
\pgfpathlineto{\pgfqpoint{2.264106in}{1.191385in}}%
\pgfpathlineto{\pgfqpoint{2.264209in}{1.272047in}}%
\pgfpathlineto{\pgfqpoint{2.265039in}{1.056102in}}%
\pgfpathlineto{\pgfqpoint{2.265868in}{1.268702in}}%
\pgfpathlineto{\pgfqpoint{2.266282in}{1.142490in}}%
\pgfpathlineto{\pgfqpoint{2.267422in}{1.038926in}}%
\pgfpathlineto{\pgfqpoint{2.267008in}{1.228793in}}%
\pgfpathlineto{\pgfqpoint{2.267526in}{1.056930in}}%
\pgfpathlineto{\pgfqpoint{2.268459in}{1.284069in}}%
\pgfpathlineto{\pgfqpoint{2.268666in}{1.216037in}}%
\pgfpathlineto{\pgfqpoint{2.269391in}{1.289229in}}%
\pgfpathlineto{\pgfqpoint{2.269806in}{1.079505in}}%
\pgfpathlineto{\pgfqpoint{2.270946in}{1.407698in}}%
\pgfpathlineto{\pgfqpoint{2.271257in}{1.321539in}}%
\pgfpathlineto{\pgfqpoint{2.272708in}{1.071174in}}%
\pgfpathlineto{\pgfqpoint{2.272812in}{1.161759in}}%
\pgfpathlineto{\pgfqpoint{2.273744in}{1.278446in}}%
\pgfpathlineto{\pgfqpoint{2.273122in}{1.135494in}}%
\pgfpathlineto{\pgfqpoint{2.273848in}{1.241647in}}%
\pgfpathlineto{\pgfqpoint{2.274055in}{1.156201in}}%
\pgfpathlineto{\pgfqpoint{2.274781in}{1.323349in}}%
\pgfpathlineto{\pgfqpoint{2.274884in}{1.319649in}}%
\pgfpathlineto{\pgfqpoint{2.275506in}{1.138275in}}%
\pgfpathlineto{\pgfqpoint{2.275921in}{1.397071in}}%
\pgfpathlineto{\pgfqpoint{2.276335in}{1.175403in}}%
\pgfpathlineto{\pgfqpoint{2.277268in}{1.232900in}}%
\pgfpathlineto{\pgfqpoint{2.277372in}{1.318841in}}%
\pgfpathlineto{\pgfqpoint{2.278097in}{1.217098in}}%
\pgfpathlineto{\pgfqpoint{2.278408in}{1.276773in}}%
\pgfpathlineto{\pgfqpoint{2.279134in}{1.098403in}}%
\pgfpathlineto{\pgfqpoint{2.279548in}{1.157613in}}%
\pgfpathlineto{\pgfqpoint{2.280170in}{1.320615in}}%
\pgfpathlineto{\pgfqpoint{2.280274in}{1.115457in}}%
\pgfpathlineto{\pgfqpoint{2.280688in}{1.203913in}}%
\pgfpathlineto{\pgfqpoint{2.281206in}{1.021927in}}%
\pgfpathlineto{\pgfqpoint{2.281621in}{1.212380in}}%
\pgfpathlineto{\pgfqpoint{2.281828in}{1.160091in}}%
\pgfpathlineto{\pgfqpoint{2.282968in}{1.003828in}}%
\pgfpathlineto{\pgfqpoint{2.282554in}{1.233083in}}%
\pgfpathlineto{\pgfqpoint{2.283072in}{1.087723in}}%
\pgfpathlineto{\pgfqpoint{2.283797in}{1.329050in}}%
\pgfpathlineto{\pgfqpoint{2.284316in}{1.292677in}}%
\pgfpathlineto{\pgfqpoint{2.284419in}{1.215285in}}%
\pgfpathlineto{\pgfqpoint{2.284627in}{1.400079in}}%
\pgfpathlineto{\pgfqpoint{2.285248in}{1.297106in}}%
\pgfpathlineto{\pgfqpoint{2.285663in}{1.425879in}}%
\pgfpathlineto{\pgfqpoint{2.286077in}{1.230145in}}%
\pgfpathlineto{\pgfqpoint{2.286181in}{1.320334in}}%
\pgfpathlineto{\pgfqpoint{2.286492in}{1.193166in}}%
\pgfpathlineto{\pgfqpoint{2.287114in}{1.454233in}}%
\pgfpathlineto{\pgfqpoint{2.287321in}{1.280060in}}%
\pgfpathlineto{\pgfqpoint{2.288461in}{1.115663in}}%
\pgfpathlineto{\pgfqpoint{2.288565in}{1.134390in}}%
\pgfpathlineto{\pgfqpoint{2.288979in}{1.355192in}}%
\pgfpathlineto{\pgfqpoint{2.289705in}{1.339985in}}%
\pgfpathlineto{\pgfqpoint{2.290845in}{0.993272in}}%
\pgfpathlineto{\pgfqpoint{2.291052in}{1.068922in}}%
\pgfpathlineto{\pgfqpoint{2.291156in}{1.224655in}}%
\pgfpathlineto{\pgfqpoint{2.291985in}{1.009991in}}%
\pgfpathlineto{\pgfqpoint{2.292192in}{1.190577in}}%
\pgfpathlineto{\pgfqpoint{2.292918in}{1.146052in}}%
\pgfpathlineto{\pgfqpoint{2.292607in}{1.239119in}}%
\pgfpathlineto{\pgfqpoint{2.293229in}{1.165405in}}%
\pgfpathlineto{\pgfqpoint{2.293850in}{1.235084in}}%
\pgfpathlineto{\pgfqpoint{2.293643in}{1.050183in}}%
\pgfpathlineto{\pgfqpoint{2.294161in}{1.105360in}}%
\pgfpathlineto{\pgfqpoint{2.294265in}{1.089709in}}%
\pgfpathlineto{\pgfqpoint{2.294369in}{1.128136in}}%
\pgfpathlineto{\pgfqpoint{2.294783in}{1.343513in}}%
\pgfpathlineto{\pgfqpoint{2.295612in}{1.281489in}}%
\pgfpathlineto{\pgfqpoint{2.295820in}{1.423888in}}%
\pgfpathlineto{\pgfqpoint{2.296338in}{1.156247in}}%
\pgfpathlineto{\pgfqpoint{2.296545in}{1.219585in}}%
\pgfpathlineto{\pgfqpoint{2.296649in}{1.206357in}}%
\pgfpathlineto{\pgfqpoint{2.296960in}{1.282123in}}%
\pgfpathlineto{\pgfqpoint{2.297167in}{1.406722in}}%
\pgfpathlineto{\pgfqpoint{2.297685in}{1.193207in}}%
\pgfpathlineto{\pgfqpoint{2.297892in}{1.251028in}}%
\pgfpathlineto{\pgfqpoint{2.298929in}{1.094660in}}%
\pgfpathlineto{\pgfqpoint{2.298618in}{1.388456in}}%
\pgfpathlineto{\pgfqpoint{2.299032in}{1.146040in}}%
\pgfpathlineto{\pgfqpoint{2.299240in}{1.326535in}}%
\pgfpathlineto{\pgfqpoint{2.299758in}{1.106626in}}%
\pgfpathlineto{\pgfqpoint{2.300173in}{1.177575in}}%
\pgfpathlineto{\pgfqpoint{2.300380in}{0.980558in}}%
\pgfpathlineto{\pgfqpoint{2.300794in}{1.270915in}}%
\pgfpathlineto{\pgfqpoint{2.301105in}{1.242278in}}%
\pgfpathlineto{\pgfqpoint{2.301416in}{1.325169in}}%
\pgfpathlineto{\pgfqpoint{2.301934in}{1.139824in}}%
\pgfpathlineto{\pgfqpoint{2.302038in}{1.087419in}}%
\pgfpathlineto{\pgfqpoint{2.302660in}{1.303002in}}%
\pgfpathlineto{\pgfqpoint{2.302867in}{1.156526in}}%
\pgfpathlineto{\pgfqpoint{2.303696in}{1.287435in}}%
\pgfpathlineto{\pgfqpoint{2.303904in}{1.124595in}}%
\pgfpathlineto{\pgfqpoint{2.304318in}{1.249523in}}%
\pgfpathlineto{\pgfqpoint{2.304629in}{1.058646in}}%
\pgfpathlineto{\pgfqpoint{2.304940in}{1.167631in}}%
\pgfpathlineto{\pgfqpoint{2.306080in}{0.941953in}}%
\pgfpathlineto{\pgfqpoint{2.305665in}{1.258227in}}%
\pgfpathlineto{\pgfqpoint{2.306184in}{1.034831in}}%
\pgfpathlineto{\pgfqpoint{2.306909in}{1.222746in}}%
\pgfpathlineto{\pgfqpoint{2.307635in}{1.172246in}}%
\pgfpathlineto{\pgfqpoint{2.308464in}{1.006124in}}%
\pgfpathlineto{\pgfqpoint{2.307946in}{1.273631in}}%
\pgfpathlineto{\pgfqpoint{2.308982in}{1.083011in}}%
\pgfpathlineto{\pgfqpoint{2.309915in}{1.211148in}}%
\pgfpathlineto{\pgfqpoint{2.310018in}{1.010117in}}%
\pgfpathlineto{\pgfqpoint{2.310329in}{0.955770in}}%
\pgfpathlineto{\pgfqpoint{2.310226in}{1.089899in}}%
\pgfpathlineto{\pgfqpoint{2.310847in}{1.047704in}}%
\pgfpathlineto{\pgfqpoint{2.310951in}{1.123597in}}%
\pgfpathlineto{\pgfqpoint{2.311366in}{0.869685in}}%
\pgfpathlineto{\pgfqpoint{2.311780in}{1.025671in}}%
\pgfpathlineto{\pgfqpoint{2.311884in}{0.964972in}}%
\pgfpathlineto{\pgfqpoint{2.312402in}{1.162812in}}%
\pgfpathlineto{\pgfqpoint{2.312609in}{1.072525in}}%
\pgfpathlineto{\pgfqpoint{2.312713in}{1.195085in}}%
\pgfpathlineto{\pgfqpoint{2.313128in}{0.997168in}}%
\pgfpathlineto{\pgfqpoint{2.313438in}{1.058027in}}%
\pgfusepath{stroke}%
\end{pgfscope}%
\begin{pgfscope}%
\pgfsetrectcap%
\pgfsetmiterjoin%
\pgfsetlinewidth{0.803000pt}%
\definecolor{currentstroke}{rgb}{0.000000,0.000000,0.000000}%
\pgfsetstrokecolor{currentstroke}%
\pgfsetdash{}{0pt}%
\pgfpathmoveto{\pgfqpoint{0.530716in}{0.416447in}}%
\pgfpathlineto{\pgfqpoint{0.530716in}{1.788330in}}%
\pgfusepath{stroke}%
\end{pgfscope}%
\begin{pgfscope}%
\pgfsetrectcap%
\pgfsetmiterjoin%
\pgfsetlinewidth{0.803000pt}%
\definecolor{currentstroke}{rgb}{0.000000,0.000000,0.000000}%
\pgfsetstrokecolor{currentstroke}%
\pgfsetdash{}{0pt}%
\pgfpathmoveto{\pgfqpoint{2.398330in}{0.416447in}}%
\pgfpathlineto{\pgfqpoint{2.398330in}{1.788330in}}%
\pgfusepath{stroke}%
\end{pgfscope}%
\begin{pgfscope}%
\pgfsetrectcap%
\pgfsetmiterjoin%
\pgfsetlinewidth{0.803000pt}%
\definecolor{currentstroke}{rgb}{0.000000,0.000000,0.000000}%
\pgfsetstrokecolor{currentstroke}%
\pgfsetdash{}{0pt}%
\pgfpathmoveto{\pgfqpoint{0.530716in}{0.416447in}}%
\pgfpathlineto{\pgfqpoint{2.398330in}{0.416447in}}%
\pgfusepath{stroke}%
\end{pgfscope}%
\begin{pgfscope}%
\pgfsetrectcap%
\pgfsetmiterjoin%
\pgfsetlinewidth{0.803000pt}%
\definecolor{currentstroke}{rgb}{0.000000,0.000000,0.000000}%
\pgfsetstrokecolor{currentstroke}%
\pgfsetdash{}{0pt}%
\pgfpathmoveto{\pgfqpoint{0.530716in}{1.788330in}}%
\pgfpathlineto{\pgfqpoint{2.398330in}{1.788330in}}%
\pgfusepath{stroke}%
\end{pgfscope}%
\begin{pgfscope}%
\pgfsetbuttcap%
\pgfsetmiterjoin%
\definecolor{currentfill}{rgb}{1.000000,1.000000,1.000000}%
\pgfsetfillcolor{currentfill}%
\pgfsetfillopacity{0.800000}%
\pgfsetlinewidth{1.003750pt}%
\definecolor{currentstroke}{rgb}{0.800000,0.800000,0.800000}%
\pgfsetstrokecolor{currentstroke}%
\pgfsetstrokeopacity{0.800000}%
\pgfsetdash{}{0pt}%
\pgfpathmoveto{\pgfqpoint{0.608494in}{1.544552in}}%
\pgfpathlineto{\pgfqpoint{1.608827in}{1.544552in}}%
\pgfpathquadraticcurveto{\pgfqpoint{1.631049in}{1.544552in}}{\pgfqpoint{1.631049in}{1.566775in}}%
\pgfpathlineto{\pgfqpoint{1.631049in}{1.710552in}}%
\pgfpathquadraticcurveto{\pgfqpoint{1.631049in}{1.732774in}}{\pgfqpoint{1.608827in}{1.732774in}}%
\pgfpathlineto{\pgfqpoint{0.608494in}{1.732774in}}%
\pgfpathquadraticcurveto{\pgfqpoint{0.586272in}{1.732774in}}{\pgfqpoint{0.586272in}{1.710552in}}%
\pgfpathlineto{\pgfqpoint{0.586272in}{1.566775in}}%
\pgfpathquadraticcurveto{\pgfqpoint{0.586272in}{1.544552in}}{\pgfqpoint{0.608494in}{1.544552in}}%
\pgfpathlineto{\pgfqpoint{0.608494in}{1.544552in}}%
\pgfpathclose%
\pgfusepath{stroke,fill}%
\end{pgfscope}%
\begin{pgfscope}%
\pgfsetrectcap%
\pgfsetroundjoin%
\pgfsetlinewidth{1.505625pt}%
\definecolor{currentstroke}{rgb}{0.000000,0.619608,0.450980}%
\pgfsetstrokecolor{currentstroke}%
\pgfsetdash{}{0pt}%
\pgfpathmoveto{\pgfqpoint{0.630716in}{1.649441in}}%
\pgfpathlineto{\pgfqpoint{0.741827in}{1.649441in}}%
\pgfpathlineto{\pgfqpoint{0.852938in}{1.649441in}}%
\pgfusepath{stroke}%
\end{pgfscope}%
\begin{pgfscope}%
\definecolor{textcolor}{rgb}{0.000000,0.000000,0.000000}%
\pgfsetstrokecolor{textcolor}%
\pgfsetfillcolor{textcolor}%
\pgftext[x=0.941827in,y=1.610552in,left,base]{\color{textcolor}\rmfamily\fontsize{8.000000}{9.600000}\selectfont Flicker noise}%
\end{pgfscope}%
\end{pgfpicture}%
\makeatother%
\endgroup%

        } % scalebox
        \caption{Time domain}
        \label{fig:flicker_noise_time}
    \end{subfigure}
    \begin{subfigure}{0.32\linewidth}
        \centering
        \scalebox{0.75}{%
            %% Creator: Matplotlib, PGF backend
%%
%% To include the figure in your LaTeX document, write
%%   \input{<filename>.pgf}
%%
%% Make sure the required packages are loaded in your preamble
%%   \usepackage{pgf}
%%
%% Also ensure that all the required font packages are loaded; for instance,
%% the lmodern package is sometimes necessary when using math font.
%%   \usepackage{lmodern}
%%
%% Figures using additional raster images can only be included by \input if
%% they are in the same directory as the main LaTeX file. For loading figures
%% from other directories you can use the `import` package
%%   \usepackage{import}
%%
%% and then include the figures with
%%   \import{<path to file>}{<filename>.pgf}
%%
%% Matplotlib used the following preamble
%%   \usepackage{siunitx}
%%   \usepackage{fontspec}
%%   \makeatletter\@ifpackageloaded{underscore}{}{\usepackage[strings]{underscore}}\makeatother
%%
\begingroup%
\makeatletter%
\begin{pgfpicture}%
\pgfpathrectangle{\pgfpointorigin}{\pgfqpoint{2.440945in}{1.830709in}}%
\pgfusepath{use as bounding box, clip}%
\begin{pgfscope}%
\pgfsetbuttcap%
\pgfsetmiterjoin%
\definecolor{currentfill}{rgb}{1.000000,1.000000,1.000000}%
\pgfsetfillcolor{currentfill}%
\pgfsetlinewidth{0.000000pt}%
\definecolor{currentstroke}{rgb}{1.000000,1.000000,1.000000}%
\pgfsetstrokecolor{currentstroke}%
\pgfsetdash{}{0pt}%
\pgfpathmoveto{\pgfqpoint{0.000000in}{0.000000in}}%
\pgfpathlineto{\pgfqpoint{2.440945in}{0.000000in}}%
\pgfpathlineto{\pgfqpoint{2.440945in}{1.830709in}}%
\pgfpathlineto{\pgfqpoint{0.000000in}{1.830709in}}%
\pgfpathlineto{\pgfqpoint{0.000000in}{0.000000in}}%
\pgfpathclose%
\pgfusepath{fill}%
\end{pgfscope}%
\begin{pgfscope}%
\pgfsetbuttcap%
\pgfsetmiterjoin%
\definecolor{currentfill}{rgb}{1.000000,1.000000,1.000000}%
\pgfsetfillcolor{currentfill}%
\pgfsetlinewidth{0.000000pt}%
\definecolor{currentstroke}{rgb}{0.000000,0.000000,0.000000}%
\pgfsetstrokecolor{currentstroke}%
\pgfsetstrokeopacity{0.000000}%
\pgfsetdash{}{0pt}%
\pgfpathmoveto{\pgfqpoint{0.514278in}{0.417642in}}%
\pgfpathlineto{\pgfqpoint{2.399275in}{0.417642in}}%
\pgfpathlineto{\pgfqpoint{2.399275in}{1.789039in}}%
\pgfpathlineto{\pgfqpoint{0.514278in}{1.789039in}}%
\pgfpathlineto{\pgfqpoint{0.514278in}{0.417642in}}%
\pgfpathclose%
\pgfusepath{fill}%
\end{pgfscope}%
\begin{pgfscope}%
\pgfpathrectangle{\pgfqpoint{0.514278in}{0.417642in}}{\pgfqpoint{1.884996in}{1.371397in}}%
\pgfusepath{clip}%
\pgfsetrectcap%
\pgfsetroundjoin%
\pgfsetlinewidth{0.803000pt}%
\definecolor{currentstroke}{rgb}{0.450000,0.450000,0.450000}%
\pgfsetstrokecolor{currentstroke}%
\pgfsetdash{}{0pt}%
\pgfpathmoveto{\pgfqpoint{0.916836in}{0.417642in}}%
\pgfpathlineto{\pgfqpoint{0.916836in}{1.789039in}}%
\pgfusepath{stroke}%
\end{pgfscope}%
\begin{pgfscope}%
\pgfsetbuttcap%
\pgfsetroundjoin%
\definecolor{currentfill}{rgb}{0.000000,0.000000,0.000000}%
\pgfsetfillcolor{currentfill}%
\pgfsetlinewidth{0.803000pt}%
\definecolor{currentstroke}{rgb}{0.000000,0.000000,0.000000}%
\pgfsetstrokecolor{currentstroke}%
\pgfsetdash{}{0pt}%
\pgfsys@defobject{currentmarker}{\pgfqpoint{0.000000in}{-0.048611in}}{\pgfqpoint{0.000000in}{0.000000in}}{%
\pgfpathmoveto{\pgfqpoint{0.000000in}{0.000000in}}%
\pgfpathlineto{\pgfqpoint{0.000000in}{-0.048611in}}%
\pgfusepath{stroke,fill}%
}%
\begin{pgfscope}%
\pgfsys@transformshift{0.916836in}{0.417642in}%
\pgfsys@useobject{currentmarker}{}%
\end{pgfscope}%
\end{pgfscope}%
\begin{pgfscope}%
\definecolor{textcolor}{rgb}{0.000000,0.000000,0.000000}%
\pgfsetstrokecolor{textcolor}%
\pgfsetfillcolor{textcolor}%
\pgftext[x=0.916836in,y=0.320420in,,top]{\color{textcolor}\rmfamily\fontsize{8.000000}{9.600000}\selectfont \(\displaystyle {10^{-3}}\)}%
\end{pgfscope}%
\begin{pgfscope}%
\pgfpathrectangle{\pgfqpoint{0.514278in}{0.417642in}}{\pgfqpoint{1.884996in}{1.371397in}}%
\pgfusepath{clip}%
\pgfsetrectcap%
\pgfsetroundjoin%
\pgfsetlinewidth{0.803000pt}%
\definecolor{currentstroke}{rgb}{0.450000,0.450000,0.450000}%
\pgfsetstrokecolor{currentstroke}%
\pgfsetdash{}{0pt}%
\pgfpathmoveto{\pgfqpoint{1.434391in}{0.417642in}}%
\pgfpathlineto{\pgfqpoint{1.434391in}{1.789039in}}%
\pgfusepath{stroke}%
\end{pgfscope}%
\begin{pgfscope}%
\pgfsetbuttcap%
\pgfsetroundjoin%
\definecolor{currentfill}{rgb}{0.000000,0.000000,0.000000}%
\pgfsetfillcolor{currentfill}%
\pgfsetlinewidth{0.803000pt}%
\definecolor{currentstroke}{rgb}{0.000000,0.000000,0.000000}%
\pgfsetstrokecolor{currentstroke}%
\pgfsetdash{}{0pt}%
\pgfsys@defobject{currentmarker}{\pgfqpoint{0.000000in}{-0.048611in}}{\pgfqpoint{0.000000in}{0.000000in}}{%
\pgfpathmoveto{\pgfqpoint{0.000000in}{0.000000in}}%
\pgfpathlineto{\pgfqpoint{0.000000in}{-0.048611in}}%
\pgfusepath{stroke,fill}%
}%
\begin{pgfscope}%
\pgfsys@transformshift{1.434391in}{0.417642in}%
\pgfsys@useobject{currentmarker}{}%
\end{pgfscope}%
\end{pgfscope}%
\begin{pgfscope}%
\definecolor{textcolor}{rgb}{0.000000,0.000000,0.000000}%
\pgfsetstrokecolor{textcolor}%
\pgfsetfillcolor{textcolor}%
\pgftext[x=1.434391in,y=0.320420in,,top]{\color{textcolor}\rmfamily\fontsize{8.000000}{9.600000}\selectfont \(\displaystyle {10^{-2}}\)}%
\end{pgfscope}%
\begin{pgfscope}%
\pgfpathrectangle{\pgfqpoint{0.514278in}{0.417642in}}{\pgfqpoint{1.884996in}{1.371397in}}%
\pgfusepath{clip}%
\pgfsetrectcap%
\pgfsetroundjoin%
\pgfsetlinewidth{0.803000pt}%
\definecolor{currentstroke}{rgb}{0.450000,0.450000,0.450000}%
\pgfsetstrokecolor{currentstroke}%
\pgfsetdash{}{0pt}%
\pgfpathmoveto{\pgfqpoint{1.951947in}{0.417642in}}%
\pgfpathlineto{\pgfqpoint{1.951947in}{1.789039in}}%
\pgfusepath{stroke}%
\end{pgfscope}%
\begin{pgfscope}%
\pgfsetbuttcap%
\pgfsetroundjoin%
\definecolor{currentfill}{rgb}{0.000000,0.000000,0.000000}%
\pgfsetfillcolor{currentfill}%
\pgfsetlinewidth{0.803000pt}%
\definecolor{currentstroke}{rgb}{0.000000,0.000000,0.000000}%
\pgfsetstrokecolor{currentstroke}%
\pgfsetdash{}{0pt}%
\pgfsys@defobject{currentmarker}{\pgfqpoint{0.000000in}{-0.048611in}}{\pgfqpoint{0.000000in}{0.000000in}}{%
\pgfpathmoveto{\pgfqpoint{0.000000in}{0.000000in}}%
\pgfpathlineto{\pgfqpoint{0.000000in}{-0.048611in}}%
\pgfusepath{stroke,fill}%
}%
\begin{pgfscope}%
\pgfsys@transformshift{1.951947in}{0.417642in}%
\pgfsys@useobject{currentmarker}{}%
\end{pgfscope}%
\end{pgfscope}%
\begin{pgfscope}%
\definecolor{textcolor}{rgb}{0.000000,0.000000,0.000000}%
\pgfsetstrokecolor{textcolor}%
\pgfsetfillcolor{textcolor}%
\pgftext[x=1.951947in,y=0.320420in,,top]{\color{textcolor}\rmfamily\fontsize{8.000000}{9.600000}\selectfont \(\displaystyle {10^{-1}}\)}%
\end{pgfscope}%
\begin{pgfscope}%
\pgfpathrectangle{\pgfqpoint{0.514278in}{0.417642in}}{\pgfqpoint{1.884996in}{1.371397in}}%
\pgfusepath{clip}%
\pgfsetrectcap%
\pgfsetroundjoin%
\pgfsetlinewidth{0.803000pt}%
\definecolor{currentstroke}{rgb}{0.850000,0.850000,0.850000}%
\pgfsetstrokecolor{currentstroke}%
\pgfsetdash{}{0pt}%
\pgfpathmoveto{\pgfqpoint{0.555080in}{0.417642in}}%
\pgfpathlineto{\pgfqpoint{0.555080in}{1.789039in}}%
\pgfusepath{stroke}%
\end{pgfscope}%
\begin{pgfscope}%
\pgfsetbuttcap%
\pgfsetroundjoin%
\definecolor{currentfill}{rgb}{0.000000,0.000000,0.000000}%
\pgfsetfillcolor{currentfill}%
\pgfsetlinewidth{0.602250pt}%
\definecolor{currentstroke}{rgb}{0.000000,0.000000,0.000000}%
\pgfsetstrokecolor{currentstroke}%
\pgfsetdash{}{0pt}%
\pgfsys@defobject{currentmarker}{\pgfqpoint{0.000000in}{-0.027778in}}{\pgfqpoint{0.000000in}{0.000000in}}{%
\pgfpathmoveto{\pgfqpoint{0.000000in}{0.000000in}}%
\pgfpathlineto{\pgfqpoint{0.000000in}{-0.027778in}}%
\pgfusepath{stroke,fill}%
}%
\begin{pgfscope}%
\pgfsys@transformshift{0.555080in}{0.417642in}%
\pgfsys@useobject{currentmarker}{}%
\end{pgfscope}%
\end{pgfscope}%
\begin{pgfscope}%
\pgfpathrectangle{\pgfqpoint{0.514278in}{0.417642in}}{\pgfqpoint{1.884996in}{1.371397in}}%
\pgfusepath{clip}%
\pgfsetrectcap%
\pgfsetroundjoin%
\pgfsetlinewidth{0.803000pt}%
\definecolor{currentstroke}{rgb}{0.850000,0.850000,0.850000}%
\pgfsetstrokecolor{currentstroke}%
\pgfsetdash{}{0pt}%
\pgfpathmoveto{\pgfqpoint{0.646217in}{0.417642in}}%
\pgfpathlineto{\pgfqpoint{0.646217in}{1.789039in}}%
\pgfusepath{stroke}%
\end{pgfscope}%
\begin{pgfscope}%
\pgfsetbuttcap%
\pgfsetroundjoin%
\definecolor{currentfill}{rgb}{0.000000,0.000000,0.000000}%
\pgfsetfillcolor{currentfill}%
\pgfsetlinewidth{0.602250pt}%
\definecolor{currentstroke}{rgb}{0.000000,0.000000,0.000000}%
\pgfsetstrokecolor{currentstroke}%
\pgfsetdash{}{0pt}%
\pgfsys@defobject{currentmarker}{\pgfqpoint{0.000000in}{-0.027778in}}{\pgfqpoint{0.000000in}{0.000000in}}{%
\pgfpathmoveto{\pgfqpoint{0.000000in}{0.000000in}}%
\pgfpathlineto{\pgfqpoint{0.000000in}{-0.027778in}}%
\pgfusepath{stroke,fill}%
}%
\begin{pgfscope}%
\pgfsys@transformshift{0.646217in}{0.417642in}%
\pgfsys@useobject{currentmarker}{}%
\end{pgfscope}%
\end{pgfscope}%
\begin{pgfscope}%
\pgfpathrectangle{\pgfqpoint{0.514278in}{0.417642in}}{\pgfqpoint{1.884996in}{1.371397in}}%
\pgfusepath{clip}%
\pgfsetrectcap%
\pgfsetroundjoin%
\pgfsetlinewidth{0.803000pt}%
\definecolor{currentstroke}{rgb}{0.850000,0.850000,0.850000}%
\pgfsetstrokecolor{currentstroke}%
\pgfsetdash{}{0pt}%
\pgfpathmoveto{\pgfqpoint{0.710880in}{0.417642in}}%
\pgfpathlineto{\pgfqpoint{0.710880in}{1.789039in}}%
\pgfusepath{stroke}%
\end{pgfscope}%
\begin{pgfscope}%
\pgfsetbuttcap%
\pgfsetroundjoin%
\definecolor{currentfill}{rgb}{0.000000,0.000000,0.000000}%
\pgfsetfillcolor{currentfill}%
\pgfsetlinewidth{0.602250pt}%
\definecolor{currentstroke}{rgb}{0.000000,0.000000,0.000000}%
\pgfsetstrokecolor{currentstroke}%
\pgfsetdash{}{0pt}%
\pgfsys@defobject{currentmarker}{\pgfqpoint{0.000000in}{-0.027778in}}{\pgfqpoint{0.000000in}{0.000000in}}{%
\pgfpathmoveto{\pgfqpoint{0.000000in}{0.000000in}}%
\pgfpathlineto{\pgfqpoint{0.000000in}{-0.027778in}}%
\pgfusepath{stroke,fill}%
}%
\begin{pgfscope}%
\pgfsys@transformshift{0.710880in}{0.417642in}%
\pgfsys@useobject{currentmarker}{}%
\end{pgfscope}%
\end{pgfscope}%
\begin{pgfscope}%
\pgfpathrectangle{\pgfqpoint{0.514278in}{0.417642in}}{\pgfqpoint{1.884996in}{1.371397in}}%
\pgfusepath{clip}%
\pgfsetrectcap%
\pgfsetroundjoin%
\pgfsetlinewidth{0.803000pt}%
\definecolor{currentstroke}{rgb}{0.850000,0.850000,0.850000}%
\pgfsetstrokecolor{currentstroke}%
\pgfsetdash{}{0pt}%
\pgfpathmoveto{\pgfqpoint{0.761036in}{0.417642in}}%
\pgfpathlineto{\pgfqpoint{0.761036in}{1.789039in}}%
\pgfusepath{stroke}%
\end{pgfscope}%
\begin{pgfscope}%
\pgfsetbuttcap%
\pgfsetroundjoin%
\definecolor{currentfill}{rgb}{0.000000,0.000000,0.000000}%
\pgfsetfillcolor{currentfill}%
\pgfsetlinewidth{0.602250pt}%
\definecolor{currentstroke}{rgb}{0.000000,0.000000,0.000000}%
\pgfsetstrokecolor{currentstroke}%
\pgfsetdash{}{0pt}%
\pgfsys@defobject{currentmarker}{\pgfqpoint{0.000000in}{-0.027778in}}{\pgfqpoint{0.000000in}{0.000000in}}{%
\pgfpathmoveto{\pgfqpoint{0.000000in}{0.000000in}}%
\pgfpathlineto{\pgfqpoint{0.000000in}{-0.027778in}}%
\pgfusepath{stroke,fill}%
}%
\begin{pgfscope}%
\pgfsys@transformshift{0.761036in}{0.417642in}%
\pgfsys@useobject{currentmarker}{}%
\end{pgfscope}%
\end{pgfscope}%
\begin{pgfscope}%
\pgfpathrectangle{\pgfqpoint{0.514278in}{0.417642in}}{\pgfqpoint{1.884996in}{1.371397in}}%
\pgfusepath{clip}%
\pgfsetrectcap%
\pgfsetroundjoin%
\pgfsetlinewidth{0.803000pt}%
\definecolor{currentstroke}{rgb}{0.850000,0.850000,0.850000}%
\pgfsetstrokecolor{currentstroke}%
\pgfsetdash{}{0pt}%
\pgfpathmoveto{\pgfqpoint{0.802017in}{0.417642in}}%
\pgfpathlineto{\pgfqpoint{0.802017in}{1.789039in}}%
\pgfusepath{stroke}%
\end{pgfscope}%
\begin{pgfscope}%
\pgfsetbuttcap%
\pgfsetroundjoin%
\definecolor{currentfill}{rgb}{0.000000,0.000000,0.000000}%
\pgfsetfillcolor{currentfill}%
\pgfsetlinewidth{0.602250pt}%
\definecolor{currentstroke}{rgb}{0.000000,0.000000,0.000000}%
\pgfsetstrokecolor{currentstroke}%
\pgfsetdash{}{0pt}%
\pgfsys@defobject{currentmarker}{\pgfqpoint{0.000000in}{-0.027778in}}{\pgfqpoint{0.000000in}{0.000000in}}{%
\pgfpathmoveto{\pgfqpoint{0.000000in}{0.000000in}}%
\pgfpathlineto{\pgfqpoint{0.000000in}{-0.027778in}}%
\pgfusepath{stroke,fill}%
}%
\begin{pgfscope}%
\pgfsys@transformshift{0.802017in}{0.417642in}%
\pgfsys@useobject{currentmarker}{}%
\end{pgfscope}%
\end{pgfscope}%
\begin{pgfscope}%
\pgfpathrectangle{\pgfqpoint{0.514278in}{0.417642in}}{\pgfqpoint{1.884996in}{1.371397in}}%
\pgfusepath{clip}%
\pgfsetrectcap%
\pgfsetroundjoin%
\pgfsetlinewidth{0.803000pt}%
\definecolor{currentstroke}{rgb}{0.850000,0.850000,0.850000}%
\pgfsetstrokecolor{currentstroke}%
\pgfsetdash{}{0pt}%
\pgfpathmoveto{\pgfqpoint{0.836665in}{0.417642in}}%
\pgfpathlineto{\pgfqpoint{0.836665in}{1.789039in}}%
\pgfusepath{stroke}%
\end{pgfscope}%
\begin{pgfscope}%
\pgfsetbuttcap%
\pgfsetroundjoin%
\definecolor{currentfill}{rgb}{0.000000,0.000000,0.000000}%
\pgfsetfillcolor{currentfill}%
\pgfsetlinewidth{0.602250pt}%
\definecolor{currentstroke}{rgb}{0.000000,0.000000,0.000000}%
\pgfsetstrokecolor{currentstroke}%
\pgfsetdash{}{0pt}%
\pgfsys@defobject{currentmarker}{\pgfqpoint{0.000000in}{-0.027778in}}{\pgfqpoint{0.000000in}{0.000000in}}{%
\pgfpathmoveto{\pgfqpoint{0.000000in}{0.000000in}}%
\pgfpathlineto{\pgfqpoint{0.000000in}{-0.027778in}}%
\pgfusepath{stroke,fill}%
}%
\begin{pgfscope}%
\pgfsys@transformshift{0.836665in}{0.417642in}%
\pgfsys@useobject{currentmarker}{}%
\end{pgfscope}%
\end{pgfscope}%
\begin{pgfscope}%
\pgfpathrectangle{\pgfqpoint{0.514278in}{0.417642in}}{\pgfqpoint{1.884996in}{1.371397in}}%
\pgfusepath{clip}%
\pgfsetrectcap%
\pgfsetroundjoin%
\pgfsetlinewidth{0.803000pt}%
\definecolor{currentstroke}{rgb}{0.850000,0.850000,0.850000}%
\pgfsetstrokecolor{currentstroke}%
\pgfsetdash{}{0pt}%
\pgfpathmoveto{\pgfqpoint{0.866679in}{0.417642in}}%
\pgfpathlineto{\pgfqpoint{0.866679in}{1.789039in}}%
\pgfusepath{stroke}%
\end{pgfscope}%
\begin{pgfscope}%
\pgfsetbuttcap%
\pgfsetroundjoin%
\definecolor{currentfill}{rgb}{0.000000,0.000000,0.000000}%
\pgfsetfillcolor{currentfill}%
\pgfsetlinewidth{0.602250pt}%
\definecolor{currentstroke}{rgb}{0.000000,0.000000,0.000000}%
\pgfsetstrokecolor{currentstroke}%
\pgfsetdash{}{0pt}%
\pgfsys@defobject{currentmarker}{\pgfqpoint{0.000000in}{-0.027778in}}{\pgfqpoint{0.000000in}{0.000000in}}{%
\pgfpathmoveto{\pgfqpoint{0.000000in}{0.000000in}}%
\pgfpathlineto{\pgfqpoint{0.000000in}{-0.027778in}}%
\pgfusepath{stroke,fill}%
}%
\begin{pgfscope}%
\pgfsys@transformshift{0.866679in}{0.417642in}%
\pgfsys@useobject{currentmarker}{}%
\end{pgfscope}%
\end{pgfscope}%
\begin{pgfscope}%
\pgfpathrectangle{\pgfqpoint{0.514278in}{0.417642in}}{\pgfqpoint{1.884996in}{1.371397in}}%
\pgfusepath{clip}%
\pgfsetrectcap%
\pgfsetroundjoin%
\pgfsetlinewidth{0.803000pt}%
\definecolor{currentstroke}{rgb}{0.850000,0.850000,0.850000}%
\pgfsetstrokecolor{currentstroke}%
\pgfsetdash{}{0pt}%
\pgfpathmoveto{\pgfqpoint{0.893154in}{0.417642in}}%
\pgfpathlineto{\pgfqpoint{0.893154in}{1.789039in}}%
\pgfusepath{stroke}%
\end{pgfscope}%
\begin{pgfscope}%
\pgfsetbuttcap%
\pgfsetroundjoin%
\definecolor{currentfill}{rgb}{0.000000,0.000000,0.000000}%
\pgfsetfillcolor{currentfill}%
\pgfsetlinewidth{0.602250pt}%
\definecolor{currentstroke}{rgb}{0.000000,0.000000,0.000000}%
\pgfsetstrokecolor{currentstroke}%
\pgfsetdash{}{0pt}%
\pgfsys@defobject{currentmarker}{\pgfqpoint{0.000000in}{-0.027778in}}{\pgfqpoint{0.000000in}{0.000000in}}{%
\pgfpathmoveto{\pgfqpoint{0.000000in}{0.000000in}}%
\pgfpathlineto{\pgfqpoint{0.000000in}{-0.027778in}}%
\pgfusepath{stroke,fill}%
}%
\begin{pgfscope}%
\pgfsys@transformshift{0.893154in}{0.417642in}%
\pgfsys@useobject{currentmarker}{}%
\end{pgfscope}%
\end{pgfscope}%
\begin{pgfscope}%
\pgfpathrectangle{\pgfqpoint{0.514278in}{0.417642in}}{\pgfqpoint{1.884996in}{1.371397in}}%
\pgfusepath{clip}%
\pgfsetrectcap%
\pgfsetroundjoin%
\pgfsetlinewidth{0.803000pt}%
\definecolor{currentstroke}{rgb}{0.850000,0.850000,0.850000}%
\pgfsetstrokecolor{currentstroke}%
\pgfsetdash{}{0pt}%
\pgfpathmoveto{\pgfqpoint{1.072635in}{0.417642in}}%
\pgfpathlineto{\pgfqpoint{1.072635in}{1.789039in}}%
\pgfusepath{stroke}%
\end{pgfscope}%
\begin{pgfscope}%
\pgfsetbuttcap%
\pgfsetroundjoin%
\definecolor{currentfill}{rgb}{0.000000,0.000000,0.000000}%
\pgfsetfillcolor{currentfill}%
\pgfsetlinewidth{0.602250pt}%
\definecolor{currentstroke}{rgb}{0.000000,0.000000,0.000000}%
\pgfsetstrokecolor{currentstroke}%
\pgfsetdash{}{0pt}%
\pgfsys@defobject{currentmarker}{\pgfqpoint{0.000000in}{-0.027778in}}{\pgfqpoint{0.000000in}{0.000000in}}{%
\pgfpathmoveto{\pgfqpoint{0.000000in}{0.000000in}}%
\pgfpathlineto{\pgfqpoint{0.000000in}{-0.027778in}}%
\pgfusepath{stroke,fill}%
}%
\begin{pgfscope}%
\pgfsys@transformshift{1.072635in}{0.417642in}%
\pgfsys@useobject{currentmarker}{}%
\end{pgfscope}%
\end{pgfscope}%
\begin{pgfscope}%
\pgfpathrectangle{\pgfqpoint{0.514278in}{0.417642in}}{\pgfqpoint{1.884996in}{1.371397in}}%
\pgfusepath{clip}%
\pgfsetrectcap%
\pgfsetroundjoin%
\pgfsetlinewidth{0.803000pt}%
\definecolor{currentstroke}{rgb}{0.850000,0.850000,0.850000}%
\pgfsetstrokecolor{currentstroke}%
\pgfsetdash{}{0pt}%
\pgfpathmoveto{\pgfqpoint{1.163773in}{0.417642in}}%
\pgfpathlineto{\pgfqpoint{1.163773in}{1.789039in}}%
\pgfusepath{stroke}%
\end{pgfscope}%
\begin{pgfscope}%
\pgfsetbuttcap%
\pgfsetroundjoin%
\definecolor{currentfill}{rgb}{0.000000,0.000000,0.000000}%
\pgfsetfillcolor{currentfill}%
\pgfsetlinewidth{0.602250pt}%
\definecolor{currentstroke}{rgb}{0.000000,0.000000,0.000000}%
\pgfsetstrokecolor{currentstroke}%
\pgfsetdash{}{0pt}%
\pgfsys@defobject{currentmarker}{\pgfqpoint{0.000000in}{-0.027778in}}{\pgfqpoint{0.000000in}{0.000000in}}{%
\pgfpathmoveto{\pgfqpoint{0.000000in}{0.000000in}}%
\pgfpathlineto{\pgfqpoint{0.000000in}{-0.027778in}}%
\pgfusepath{stroke,fill}%
}%
\begin{pgfscope}%
\pgfsys@transformshift{1.163773in}{0.417642in}%
\pgfsys@useobject{currentmarker}{}%
\end{pgfscope}%
\end{pgfscope}%
\begin{pgfscope}%
\pgfpathrectangle{\pgfqpoint{0.514278in}{0.417642in}}{\pgfqpoint{1.884996in}{1.371397in}}%
\pgfusepath{clip}%
\pgfsetrectcap%
\pgfsetroundjoin%
\pgfsetlinewidth{0.803000pt}%
\definecolor{currentstroke}{rgb}{0.850000,0.850000,0.850000}%
\pgfsetstrokecolor{currentstroke}%
\pgfsetdash{}{0pt}%
\pgfpathmoveto{\pgfqpoint{1.228435in}{0.417642in}}%
\pgfpathlineto{\pgfqpoint{1.228435in}{1.789039in}}%
\pgfusepath{stroke}%
\end{pgfscope}%
\begin{pgfscope}%
\pgfsetbuttcap%
\pgfsetroundjoin%
\definecolor{currentfill}{rgb}{0.000000,0.000000,0.000000}%
\pgfsetfillcolor{currentfill}%
\pgfsetlinewidth{0.602250pt}%
\definecolor{currentstroke}{rgb}{0.000000,0.000000,0.000000}%
\pgfsetstrokecolor{currentstroke}%
\pgfsetdash{}{0pt}%
\pgfsys@defobject{currentmarker}{\pgfqpoint{0.000000in}{-0.027778in}}{\pgfqpoint{0.000000in}{0.000000in}}{%
\pgfpathmoveto{\pgfqpoint{0.000000in}{0.000000in}}%
\pgfpathlineto{\pgfqpoint{0.000000in}{-0.027778in}}%
\pgfusepath{stroke,fill}%
}%
\begin{pgfscope}%
\pgfsys@transformshift{1.228435in}{0.417642in}%
\pgfsys@useobject{currentmarker}{}%
\end{pgfscope}%
\end{pgfscope}%
\begin{pgfscope}%
\pgfpathrectangle{\pgfqpoint{0.514278in}{0.417642in}}{\pgfqpoint{1.884996in}{1.371397in}}%
\pgfusepath{clip}%
\pgfsetrectcap%
\pgfsetroundjoin%
\pgfsetlinewidth{0.803000pt}%
\definecolor{currentstroke}{rgb}{0.850000,0.850000,0.850000}%
\pgfsetstrokecolor{currentstroke}%
\pgfsetdash{}{0pt}%
\pgfpathmoveto{\pgfqpoint{1.278592in}{0.417642in}}%
\pgfpathlineto{\pgfqpoint{1.278592in}{1.789039in}}%
\pgfusepath{stroke}%
\end{pgfscope}%
\begin{pgfscope}%
\pgfsetbuttcap%
\pgfsetroundjoin%
\definecolor{currentfill}{rgb}{0.000000,0.000000,0.000000}%
\pgfsetfillcolor{currentfill}%
\pgfsetlinewidth{0.602250pt}%
\definecolor{currentstroke}{rgb}{0.000000,0.000000,0.000000}%
\pgfsetstrokecolor{currentstroke}%
\pgfsetdash{}{0pt}%
\pgfsys@defobject{currentmarker}{\pgfqpoint{0.000000in}{-0.027778in}}{\pgfqpoint{0.000000in}{0.000000in}}{%
\pgfpathmoveto{\pgfqpoint{0.000000in}{0.000000in}}%
\pgfpathlineto{\pgfqpoint{0.000000in}{-0.027778in}}%
\pgfusepath{stroke,fill}%
}%
\begin{pgfscope}%
\pgfsys@transformshift{1.278592in}{0.417642in}%
\pgfsys@useobject{currentmarker}{}%
\end{pgfscope}%
\end{pgfscope}%
\begin{pgfscope}%
\pgfpathrectangle{\pgfqpoint{0.514278in}{0.417642in}}{\pgfqpoint{1.884996in}{1.371397in}}%
\pgfusepath{clip}%
\pgfsetrectcap%
\pgfsetroundjoin%
\pgfsetlinewidth{0.803000pt}%
\definecolor{currentstroke}{rgb}{0.850000,0.850000,0.850000}%
\pgfsetstrokecolor{currentstroke}%
\pgfsetdash{}{0pt}%
\pgfpathmoveto{\pgfqpoint{1.319572in}{0.417642in}}%
\pgfpathlineto{\pgfqpoint{1.319572in}{1.789039in}}%
\pgfusepath{stroke}%
\end{pgfscope}%
\begin{pgfscope}%
\pgfsetbuttcap%
\pgfsetroundjoin%
\definecolor{currentfill}{rgb}{0.000000,0.000000,0.000000}%
\pgfsetfillcolor{currentfill}%
\pgfsetlinewidth{0.602250pt}%
\definecolor{currentstroke}{rgb}{0.000000,0.000000,0.000000}%
\pgfsetstrokecolor{currentstroke}%
\pgfsetdash{}{0pt}%
\pgfsys@defobject{currentmarker}{\pgfqpoint{0.000000in}{-0.027778in}}{\pgfqpoint{0.000000in}{0.000000in}}{%
\pgfpathmoveto{\pgfqpoint{0.000000in}{0.000000in}}%
\pgfpathlineto{\pgfqpoint{0.000000in}{-0.027778in}}%
\pgfusepath{stroke,fill}%
}%
\begin{pgfscope}%
\pgfsys@transformshift{1.319572in}{0.417642in}%
\pgfsys@useobject{currentmarker}{}%
\end{pgfscope}%
\end{pgfscope}%
\begin{pgfscope}%
\pgfpathrectangle{\pgfqpoint{0.514278in}{0.417642in}}{\pgfqpoint{1.884996in}{1.371397in}}%
\pgfusepath{clip}%
\pgfsetrectcap%
\pgfsetroundjoin%
\pgfsetlinewidth{0.803000pt}%
\definecolor{currentstroke}{rgb}{0.850000,0.850000,0.850000}%
\pgfsetstrokecolor{currentstroke}%
\pgfsetdash{}{0pt}%
\pgfpathmoveto{\pgfqpoint{1.354221in}{0.417642in}}%
\pgfpathlineto{\pgfqpoint{1.354221in}{1.789039in}}%
\pgfusepath{stroke}%
\end{pgfscope}%
\begin{pgfscope}%
\pgfsetbuttcap%
\pgfsetroundjoin%
\definecolor{currentfill}{rgb}{0.000000,0.000000,0.000000}%
\pgfsetfillcolor{currentfill}%
\pgfsetlinewidth{0.602250pt}%
\definecolor{currentstroke}{rgb}{0.000000,0.000000,0.000000}%
\pgfsetstrokecolor{currentstroke}%
\pgfsetdash{}{0pt}%
\pgfsys@defobject{currentmarker}{\pgfqpoint{0.000000in}{-0.027778in}}{\pgfqpoint{0.000000in}{0.000000in}}{%
\pgfpathmoveto{\pgfqpoint{0.000000in}{0.000000in}}%
\pgfpathlineto{\pgfqpoint{0.000000in}{-0.027778in}}%
\pgfusepath{stroke,fill}%
}%
\begin{pgfscope}%
\pgfsys@transformshift{1.354221in}{0.417642in}%
\pgfsys@useobject{currentmarker}{}%
\end{pgfscope}%
\end{pgfscope}%
\begin{pgfscope}%
\pgfpathrectangle{\pgfqpoint{0.514278in}{0.417642in}}{\pgfqpoint{1.884996in}{1.371397in}}%
\pgfusepath{clip}%
\pgfsetrectcap%
\pgfsetroundjoin%
\pgfsetlinewidth{0.803000pt}%
\definecolor{currentstroke}{rgb}{0.850000,0.850000,0.850000}%
\pgfsetstrokecolor{currentstroke}%
\pgfsetdash{}{0pt}%
\pgfpathmoveto{\pgfqpoint{1.384235in}{0.417642in}}%
\pgfpathlineto{\pgfqpoint{1.384235in}{1.789039in}}%
\pgfusepath{stroke}%
\end{pgfscope}%
\begin{pgfscope}%
\pgfsetbuttcap%
\pgfsetroundjoin%
\definecolor{currentfill}{rgb}{0.000000,0.000000,0.000000}%
\pgfsetfillcolor{currentfill}%
\pgfsetlinewidth{0.602250pt}%
\definecolor{currentstroke}{rgb}{0.000000,0.000000,0.000000}%
\pgfsetstrokecolor{currentstroke}%
\pgfsetdash{}{0pt}%
\pgfsys@defobject{currentmarker}{\pgfqpoint{0.000000in}{-0.027778in}}{\pgfqpoint{0.000000in}{0.000000in}}{%
\pgfpathmoveto{\pgfqpoint{0.000000in}{0.000000in}}%
\pgfpathlineto{\pgfqpoint{0.000000in}{-0.027778in}}%
\pgfusepath{stroke,fill}%
}%
\begin{pgfscope}%
\pgfsys@transformshift{1.384235in}{0.417642in}%
\pgfsys@useobject{currentmarker}{}%
\end{pgfscope}%
\end{pgfscope}%
\begin{pgfscope}%
\pgfpathrectangle{\pgfqpoint{0.514278in}{0.417642in}}{\pgfqpoint{1.884996in}{1.371397in}}%
\pgfusepath{clip}%
\pgfsetrectcap%
\pgfsetroundjoin%
\pgfsetlinewidth{0.803000pt}%
\definecolor{currentstroke}{rgb}{0.850000,0.850000,0.850000}%
\pgfsetstrokecolor{currentstroke}%
\pgfsetdash{}{0pt}%
\pgfpathmoveto{\pgfqpoint{1.410709in}{0.417642in}}%
\pgfpathlineto{\pgfqpoint{1.410709in}{1.789039in}}%
\pgfusepath{stroke}%
\end{pgfscope}%
\begin{pgfscope}%
\pgfsetbuttcap%
\pgfsetroundjoin%
\definecolor{currentfill}{rgb}{0.000000,0.000000,0.000000}%
\pgfsetfillcolor{currentfill}%
\pgfsetlinewidth{0.602250pt}%
\definecolor{currentstroke}{rgb}{0.000000,0.000000,0.000000}%
\pgfsetstrokecolor{currentstroke}%
\pgfsetdash{}{0pt}%
\pgfsys@defobject{currentmarker}{\pgfqpoint{0.000000in}{-0.027778in}}{\pgfqpoint{0.000000in}{0.000000in}}{%
\pgfpathmoveto{\pgfqpoint{0.000000in}{0.000000in}}%
\pgfpathlineto{\pgfqpoint{0.000000in}{-0.027778in}}%
\pgfusepath{stroke,fill}%
}%
\begin{pgfscope}%
\pgfsys@transformshift{1.410709in}{0.417642in}%
\pgfsys@useobject{currentmarker}{}%
\end{pgfscope}%
\end{pgfscope}%
\begin{pgfscope}%
\pgfpathrectangle{\pgfqpoint{0.514278in}{0.417642in}}{\pgfqpoint{1.884996in}{1.371397in}}%
\pgfusepath{clip}%
\pgfsetrectcap%
\pgfsetroundjoin%
\pgfsetlinewidth{0.803000pt}%
\definecolor{currentstroke}{rgb}{0.850000,0.850000,0.850000}%
\pgfsetstrokecolor{currentstroke}%
\pgfsetdash{}{0pt}%
\pgfpathmoveto{\pgfqpoint{1.590191in}{0.417642in}}%
\pgfpathlineto{\pgfqpoint{1.590191in}{1.789039in}}%
\pgfusepath{stroke}%
\end{pgfscope}%
\begin{pgfscope}%
\pgfsetbuttcap%
\pgfsetroundjoin%
\definecolor{currentfill}{rgb}{0.000000,0.000000,0.000000}%
\pgfsetfillcolor{currentfill}%
\pgfsetlinewidth{0.602250pt}%
\definecolor{currentstroke}{rgb}{0.000000,0.000000,0.000000}%
\pgfsetstrokecolor{currentstroke}%
\pgfsetdash{}{0pt}%
\pgfsys@defobject{currentmarker}{\pgfqpoint{0.000000in}{-0.027778in}}{\pgfqpoint{0.000000in}{0.000000in}}{%
\pgfpathmoveto{\pgfqpoint{0.000000in}{0.000000in}}%
\pgfpathlineto{\pgfqpoint{0.000000in}{-0.027778in}}%
\pgfusepath{stroke,fill}%
}%
\begin{pgfscope}%
\pgfsys@transformshift{1.590191in}{0.417642in}%
\pgfsys@useobject{currentmarker}{}%
\end{pgfscope}%
\end{pgfscope}%
\begin{pgfscope}%
\pgfpathrectangle{\pgfqpoint{0.514278in}{0.417642in}}{\pgfqpoint{1.884996in}{1.371397in}}%
\pgfusepath{clip}%
\pgfsetrectcap%
\pgfsetroundjoin%
\pgfsetlinewidth{0.803000pt}%
\definecolor{currentstroke}{rgb}{0.850000,0.850000,0.850000}%
\pgfsetstrokecolor{currentstroke}%
\pgfsetdash{}{0pt}%
\pgfpathmoveto{\pgfqpoint{1.681328in}{0.417642in}}%
\pgfpathlineto{\pgfqpoint{1.681328in}{1.789039in}}%
\pgfusepath{stroke}%
\end{pgfscope}%
\begin{pgfscope}%
\pgfsetbuttcap%
\pgfsetroundjoin%
\definecolor{currentfill}{rgb}{0.000000,0.000000,0.000000}%
\pgfsetfillcolor{currentfill}%
\pgfsetlinewidth{0.602250pt}%
\definecolor{currentstroke}{rgb}{0.000000,0.000000,0.000000}%
\pgfsetstrokecolor{currentstroke}%
\pgfsetdash{}{0pt}%
\pgfsys@defobject{currentmarker}{\pgfqpoint{0.000000in}{-0.027778in}}{\pgfqpoint{0.000000in}{0.000000in}}{%
\pgfpathmoveto{\pgfqpoint{0.000000in}{0.000000in}}%
\pgfpathlineto{\pgfqpoint{0.000000in}{-0.027778in}}%
\pgfusepath{stroke,fill}%
}%
\begin{pgfscope}%
\pgfsys@transformshift{1.681328in}{0.417642in}%
\pgfsys@useobject{currentmarker}{}%
\end{pgfscope}%
\end{pgfscope}%
\begin{pgfscope}%
\pgfpathrectangle{\pgfqpoint{0.514278in}{0.417642in}}{\pgfqpoint{1.884996in}{1.371397in}}%
\pgfusepath{clip}%
\pgfsetrectcap%
\pgfsetroundjoin%
\pgfsetlinewidth{0.803000pt}%
\definecolor{currentstroke}{rgb}{0.850000,0.850000,0.850000}%
\pgfsetstrokecolor{currentstroke}%
\pgfsetdash{}{0pt}%
\pgfpathmoveto{\pgfqpoint{1.745991in}{0.417642in}}%
\pgfpathlineto{\pgfqpoint{1.745991in}{1.789039in}}%
\pgfusepath{stroke}%
\end{pgfscope}%
\begin{pgfscope}%
\pgfsetbuttcap%
\pgfsetroundjoin%
\definecolor{currentfill}{rgb}{0.000000,0.000000,0.000000}%
\pgfsetfillcolor{currentfill}%
\pgfsetlinewidth{0.602250pt}%
\definecolor{currentstroke}{rgb}{0.000000,0.000000,0.000000}%
\pgfsetstrokecolor{currentstroke}%
\pgfsetdash{}{0pt}%
\pgfsys@defobject{currentmarker}{\pgfqpoint{0.000000in}{-0.027778in}}{\pgfqpoint{0.000000in}{0.000000in}}{%
\pgfpathmoveto{\pgfqpoint{0.000000in}{0.000000in}}%
\pgfpathlineto{\pgfqpoint{0.000000in}{-0.027778in}}%
\pgfusepath{stroke,fill}%
}%
\begin{pgfscope}%
\pgfsys@transformshift{1.745991in}{0.417642in}%
\pgfsys@useobject{currentmarker}{}%
\end{pgfscope}%
\end{pgfscope}%
\begin{pgfscope}%
\pgfpathrectangle{\pgfqpoint{0.514278in}{0.417642in}}{\pgfqpoint{1.884996in}{1.371397in}}%
\pgfusepath{clip}%
\pgfsetrectcap%
\pgfsetroundjoin%
\pgfsetlinewidth{0.803000pt}%
\definecolor{currentstroke}{rgb}{0.850000,0.850000,0.850000}%
\pgfsetstrokecolor{currentstroke}%
\pgfsetdash{}{0pt}%
\pgfpathmoveto{\pgfqpoint{1.796147in}{0.417642in}}%
\pgfpathlineto{\pgfqpoint{1.796147in}{1.789039in}}%
\pgfusepath{stroke}%
\end{pgfscope}%
\begin{pgfscope}%
\pgfsetbuttcap%
\pgfsetroundjoin%
\definecolor{currentfill}{rgb}{0.000000,0.000000,0.000000}%
\pgfsetfillcolor{currentfill}%
\pgfsetlinewidth{0.602250pt}%
\definecolor{currentstroke}{rgb}{0.000000,0.000000,0.000000}%
\pgfsetstrokecolor{currentstroke}%
\pgfsetdash{}{0pt}%
\pgfsys@defobject{currentmarker}{\pgfqpoint{0.000000in}{-0.027778in}}{\pgfqpoint{0.000000in}{0.000000in}}{%
\pgfpathmoveto{\pgfqpoint{0.000000in}{0.000000in}}%
\pgfpathlineto{\pgfqpoint{0.000000in}{-0.027778in}}%
\pgfusepath{stroke,fill}%
}%
\begin{pgfscope}%
\pgfsys@transformshift{1.796147in}{0.417642in}%
\pgfsys@useobject{currentmarker}{}%
\end{pgfscope}%
\end{pgfscope}%
\begin{pgfscope}%
\pgfpathrectangle{\pgfqpoint{0.514278in}{0.417642in}}{\pgfqpoint{1.884996in}{1.371397in}}%
\pgfusepath{clip}%
\pgfsetrectcap%
\pgfsetroundjoin%
\pgfsetlinewidth{0.803000pt}%
\definecolor{currentstroke}{rgb}{0.850000,0.850000,0.850000}%
\pgfsetstrokecolor{currentstroke}%
\pgfsetdash{}{0pt}%
\pgfpathmoveto{\pgfqpoint{1.837128in}{0.417642in}}%
\pgfpathlineto{\pgfqpoint{1.837128in}{1.789039in}}%
\pgfusepath{stroke}%
\end{pgfscope}%
\begin{pgfscope}%
\pgfsetbuttcap%
\pgfsetroundjoin%
\definecolor{currentfill}{rgb}{0.000000,0.000000,0.000000}%
\pgfsetfillcolor{currentfill}%
\pgfsetlinewidth{0.602250pt}%
\definecolor{currentstroke}{rgb}{0.000000,0.000000,0.000000}%
\pgfsetstrokecolor{currentstroke}%
\pgfsetdash{}{0pt}%
\pgfsys@defobject{currentmarker}{\pgfqpoint{0.000000in}{-0.027778in}}{\pgfqpoint{0.000000in}{0.000000in}}{%
\pgfpathmoveto{\pgfqpoint{0.000000in}{0.000000in}}%
\pgfpathlineto{\pgfqpoint{0.000000in}{-0.027778in}}%
\pgfusepath{stroke,fill}%
}%
\begin{pgfscope}%
\pgfsys@transformshift{1.837128in}{0.417642in}%
\pgfsys@useobject{currentmarker}{}%
\end{pgfscope}%
\end{pgfscope}%
\begin{pgfscope}%
\pgfpathrectangle{\pgfqpoint{0.514278in}{0.417642in}}{\pgfqpoint{1.884996in}{1.371397in}}%
\pgfusepath{clip}%
\pgfsetrectcap%
\pgfsetroundjoin%
\pgfsetlinewidth{0.803000pt}%
\definecolor{currentstroke}{rgb}{0.850000,0.850000,0.850000}%
\pgfsetstrokecolor{currentstroke}%
\pgfsetdash{}{0pt}%
\pgfpathmoveto{\pgfqpoint{1.871777in}{0.417642in}}%
\pgfpathlineto{\pgfqpoint{1.871777in}{1.789039in}}%
\pgfusepath{stroke}%
\end{pgfscope}%
\begin{pgfscope}%
\pgfsetbuttcap%
\pgfsetroundjoin%
\definecolor{currentfill}{rgb}{0.000000,0.000000,0.000000}%
\pgfsetfillcolor{currentfill}%
\pgfsetlinewidth{0.602250pt}%
\definecolor{currentstroke}{rgb}{0.000000,0.000000,0.000000}%
\pgfsetstrokecolor{currentstroke}%
\pgfsetdash{}{0pt}%
\pgfsys@defobject{currentmarker}{\pgfqpoint{0.000000in}{-0.027778in}}{\pgfqpoint{0.000000in}{0.000000in}}{%
\pgfpathmoveto{\pgfqpoint{0.000000in}{0.000000in}}%
\pgfpathlineto{\pgfqpoint{0.000000in}{-0.027778in}}%
\pgfusepath{stroke,fill}%
}%
\begin{pgfscope}%
\pgfsys@transformshift{1.871777in}{0.417642in}%
\pgfsys@useobject{currentmarker}{}%
\end{pgfscope}%
\end{pgfscope}%
\begin{pgfscope}%
\pgfpathrectangle{\pgfqpoint{0.514278in}{0.417642in}}{\pgfqpoint{1.884996in}{1.371397in}}%
\pgfusepath{clip}%
\pgfsetrectcap%
\pgfsetroundjoin%
\pgfsetlinewidth{0.803000pt}%
\definecolor{currentstroke}{rgb}{0.850000,0.850000,0.850000}%
\pgfsetstrokecolor{currentstroke}%
\pgfsetdash{}{0pt}%
\pgfpathmoveto{\pgfqpoint{1.901791in}{0.417642in}}%
\pgfpathlineto{\pgfqpoint{1.901791in}{1.789039in}}%
\pgfusepath{stroke}%
\end{pgfscope}%
\begin{pgfscope}%
\pgfsetbuttcap%
\pgfsetroundjoin%
\definecolor{currentfill}{rgb}{0.000000,0.000000,0.000000}%
\pgfsetfillcolor{currentfill}%
\pgfsetlinewidth{0.602250pt}%
\definecolor{currentstroke}{rgb}{0.000000,0.000000,0.000000}%
\pgfsetstrokecolor{currentstroke}%
\pgfsetdash{}{0pt}%
\pgfsys@defobject{currentmarker}{\pgfqpoint{0.000000in}{-0.027778in}}{\pgfqpoint{0.000000in}{0.000000in}}{%
\pgfpathmoveto{\pgfqpoint{0.000000in}{0.000000in}}%
\pgfpathlineto{\pgfqpoint{0.000000in}{-0.027778in}}%
\pgfusepath{stroke,fill}%
}%
\begin{pgfscope}%
\pgfsys@transformshift{1.901791in}{0.417642in}%
\pgfsys@useobject{currentmarker}{}%
\end{pgfscope}%
\end{pgfscope}%
\begin{pgfscope}%
\pgfpathrectangle{\pgfqpoint{0.514278in}{0.417642in}}{\pgfqpoint{1.884996in}{1.371397in}}%
\pgfusepath{clip}%
\pgfsetrectcap%
\pgfsetroundjoin%
\pgfsetlinewidth{0.803000pt}%
\definecolor{currentstroke}{rgb}{0.850000,0.850000,0.850000}%
\pgfsetstrokecolor{currentstroke}%
\pgfsetdash{}{0pt}%
\pgfpathmoveto{\pgfqpoint{1.928265in}{0.417642in}}%
\pgfpathlineto{\pgfqpoint{1.928265in}{1.789039in}}%
\pgfusepath{stroke}%
\end{pgfscope}%
\begin{pgfscope}%
\pgfsetbuttcap%
\pgfsetroundjoin%
\definecolor{currentfill}{rgb}{0.000000,0.000000,0.000000}%
\pgfsetfillcolor{currentfill}%
\pgfsetlinewidth{0.602250pt}%
\definecolor{currentstroke}{rgb}{0.000000,0.000000,0.000000}%
\pgfsetstrokecolor{currentstroke}%
\pgfsetdash{}{0pt}%
\pgfsys@defobject{currentmarker}{\pgfqpoint{0.000000in}{-0.027778in}}{\pgfqpoint{0.000000in}{0.000000in}}{%
\pgfpathmoveto{\pgfqpoint{0.000000in}{0.000000in}}%
\pgfpathlineto{\pgfqpoint{0.000000in}{-0.027778in}}%
\pgfusepath{stroke,fill}%
}%
\begin{pgfscope}%
\pgfsys@transformshift{1.928265in}{0.417642in}%
\pgfsys@useobject{currentmarker}{}%
\end{pgfscope}%
\end{pgfscope}%
\begin{pgfscope}%
\pgfpathrectangle{\pgfqpoint{0.514278in}{0.417642in}}{\pgfqpoint{1.884996in}{1.371397in}}%
\pgfusepath{clip}%
\pgfsetrectcap%
\pgfsetroundjoin%
\pgfsetlinewidth{0.803000pt}%
\definecolor{currentstroke}{rgb}{0.850000,0.850000,0.850000}%
\pgfsetstrokecolor{currentstroke}%
\pgfsetdash{}{0pt}%
\pgfpathmoveto{\pgfqpoint{2.107747in}{0.417642in}}%
\pgfpathlineto{\pgfqpoint{2.107747in}{1.789039in}}%
\pgfusepath{stroke}%
\end{pgfscope}%
\begin{pgfscope}%
\pgfsetbuttcap%
\pgfsetroundjoin%
\definecolor{currentfill}{rgb}{0.000000,0.000000,0.000000}%
\pgfsetfillcolor{currentfill}%
\pgfsetlinewidth{0.602250pt}%
\definecolor{currentstroke}{rgb}{0.000000,0.000000,0.000000}%
\pgfsetstrokecolor{currentstroke}%
\pgfsetdash{}{0pt}%
\pgfsys@defobject{currentmarker}{\pgfqpoint{0.000000in}{-0.027778in}}{\pgfqpoint{0.000000in}{0.000000in}}{%
\pgfpathmoveto{\pgfqpoint{0.000000in}{0.000000in}}%
\pgfpathlineto{\pgfqpoint{0.000000in}{-0.027778in}}%
\pgfusepath{stroke,fill}%
}%
\begin{pgfscope}%
\pgfsys@transformshift{2.107747in}{0.417642in}%
\pgfsys@useobject{currentmarker}{}%
\end{pgfscope}%
\end{pgfscope}%
\begin{pgfscope}%
\pgfpathrectangle{\pgfqpoint{0.514278in}{0.417642in}}{\pgfqpoint{1.884996in}{1.371397in}}%
\pgfusepath{clip}%
\pgfsetrectcap%
\pgfsetroundjoin%
\pgfsetlinewidth{0.803000pt}%
\definecolor{currentstroke}{rgb}{0.850000,0.850000,0.850000}%
\pgfsetstrokecolor{currentstroke}%
\pgfsetdash{}{0pt}%
\pgfpathmoveto{\pgfqpoint{2.198884in}{0.417642in}}%
\pgfpathlineto{\pgfqpoint{2.198884in}{1.789039in}}%
\pgfusepath{stroke}%
\end{pgfscope}%
\begin{pgfscope}%
\pgfsetbuttcap%
\pgfsetroundjoin%
\definecolor{currentfill}{rgb}{0.000000,0.000000,0.000000}%
\pgfsetfillcolor{currentfill}%
\pgfsetlinewidth{0.602250pt}%
\definecolor{currentstroke}{rgb}{0.000000,0.000000,0.000000}%
\pgfsetstrokecolor{currentstroke}%
\pgfsetdash{}{0pt}%
\pgfsys@defobject{currentmarker}{\pgfqpoint{0.000000in}{-0.027778in}}{\pgfqpoint{0.000000in}{0.000000in}}{%
\pgfpathmoveto{\pgfqpoint{0.000000in}{0.000000in}}%
\pgfpathlineto{\pgfqpoint{0.000000in}{-0.027778in}}%
\pgfusepath{stroke,fill}%
}%
\begin{pgfscope}%
\pgfsys@transformshift{2.198884in}{0.417642in}%
\pgfsys@useobject{currentmarker}{}%
\end{pgfscope}%
\end{pgfscope}%
\begin{pgfscope}%
\pgfpathrectangle{\pgfqpoint{0.514278in}{0.417642in}}{\pgfqpoint{1.884996in}{1.371397in}}%
\pgfusepath{clip}%
\pgfsetrectcap%
\pgfsetroundjoin%
\pgfsetlinewidth{0.803000pt}%
\definecolor{currentstroke}{rgb}{0.850000,0.850000,0.850000}%
\pgfsetstrokecolor{currentstroke}%
\pgfsetdash{}{0pt}%
\pgfpathmoveto{\pgfqpoint{2.263547in}{0.417642in}}%
\pgfpathlineto{\pgfqpoint{2.263547in}{1.789039in}}%
\pgfusepath{stroke}%
\end{pgfscope}%
\begin{pgfscope}%
\pgfsetbuttcap%
\pgfsetroundjoin%
\definecolor{currentfill}{rgb}{0.000000,0.000000,0.000000}%
\pgfsetfillcolor{currentfill}%
\pgfsetlinewidth{0.602250pt}%
\definecolor{currentstroke}{rgb}{0.000000,0.000000,0.000000}%
\pgfsetstrokecolor{currentstroke}%
\pgfsetdash{}{0pt}%
\pgfsys@defobject{currentmarker}{\pgfqpoint{0.000000in}{-0.027778in}}{\pgfqpoint{0.000000in}{0.000000in}}{%
\pgfpathmoveto{\pgfqpoint{0.000000in}{0.000000in}}%
\pgfpathlineto{\pgfqpoint{0.000000in}{-0.027778in}}%
\pgfusepath{stroke,fill}%
}%
\begin{pgfscope}%
\pgfsys@transformshift{2.263547in}{0.417642in}%
\pgfsys@useobject{currentmarker}{}%
\end{pgfscope}%
\end{pgfscope}%
\begin{pgfscope}%
\pgfpathrectangle{\pgfqpoint{0.514278in}{0.417642in}}{\pgfqpoint{1.884996in}{1.371397in}}%
\pgfusepath{clip}%
\pgfsetrectcap%
\pgfsetroundjoin%
\pgfsetlinewidth{0.803000pt}%
\definecolor{currentstroke}{rgb}{0.850000,0.850000,0.850000}%
\pgfsetstrokecolor{currentstroke}%
\pgfsetdash{}{0pt}%
\pgfpathmoveto{\pgfqpoint{2.313703in}{0.417642in}}%
\pgfpathlineto{\pgfqpoint{2.313703in}{1.789039in}}%
\pgfusepath{stroke}%
\end{pgfscope}%
\begin{pgfscope}%
\pgfsetbuttcap%
\pgfsetroundjoin%
\definecolor{currentfill}{rgb}{0.000000,0.000000,0.000000}%
\pgfsetfillcolor{currentfill}%
\pgfsetlinewidth{0.602250pt}%
\definecolor{currentstroke}{rgb}{0.000000,0.000000,0.000000}%
\pgfsetstrokecolor{currentstroke}%
\pgfsetdash{}{0pt}%
\pgfsys@defobject{currentmarker}{\pgfqpoint{0.000000in}{-0.027778in}}{\pgfqpoint{0.000000in}{0.000000in}}{%
\pgfpathmoveto{\pgfqpoint{0.000000in}{0.000000in}}%
\pgfpathlineto{\pgfqpoint{0.000000in}{-0.027778in}}%
\pgfusepath{stroke,fill}%
}%
\begin{pgfscope}%
\pgfsys@transformshift{2.313703in}{0.417642in}%
\pgfsys@useobject{currentmarker}{}%
\end{pgfscope}%
\end{pgfscope}%
\begin{pgfscope}%
\pgfpathrectangle{\pgfqpoint{0.514278in}{0.417642in}}{\pgfqpoint{1.884996in}{1.371397in}}%
\pgfusepath{clip}%
\pgfsetrectcap%
\pgfsetroundjoin%
\pgfsetlinewidth{0.803000pt}%
\definecolor{currentstroke}{rgb}{0.850000,0.850000,0.850000}%
\pgfsetstrokecolor{currentstroke}%
\pgfsetdash{}{0pt}%
\pgfpathmoveto{\pgfqpoint{2.354684in}{0.417642in}}%
\pgfpathlineto{\pgfqpoint{2.354684in}{1.789039in}}%
\pgfusepath{stroke}%
\end{pgfscope}%
\begin{pgfscope}%
\pgfsetbuttcap%
\pgfsetroundjoin%
\definecolor{currentfill}{rgb}{0.000000,0.000000,0.000000}%
\pgfsetfillcolor{currentfill}%
\pgfsetlinewidth{0.602250pt}%
\definecolor{currentstroke}{rgb}{0.000000,0.000000,0.000000}%
\pgfsetstrokecolor{currentstroke}%
\pgfsetdash{}{0pt}%
\pgfsys@defobject{currentmarker}{\pgfqpoint{0.000000in}{-0.027778in}}{\pgfqpoint{0.000000in}{0.000000in}}{%
\pgfpathmoveto{\pgfqpoint{0.000000in}{0.000000in}}%
\pgfpathlineto{\pgfqpoint{0.000000in}{-0.027778in}}%
\pgfusepath{stroke,fill}%
}%
\begin{pgfscope}%
\pgfsys@transformshift{2.354684in}{0.417642in}%
\pgfsys@useobject{currentmarker}{}%
\end{pgfscope}%
\end{pgfscope}%
\begin{pgfscope}%
\pgfpathrectangle{\pgfqpoint{0.514278in}{0.417642in}}{\pgfqpoint{1.884996in}{1.371397in}}%
\pgfusepath{clip}%
\pgfsetrectcap%
\pgfsetroundjoin%
\pgfsetlinewidth{0.803000pt}%
\definecolor{currentstroke}{rgb}{0.850000,0.850000,0.850000}%
\pgfsetstrokecolor{currentstroke}%
\pgfsetdash{}{0pt}%
\pgfpathmoveto{\pgfqpoint{2.389333in}{0.417642in}}%
\pgfpathlineto{\pgfqpoint{2.389333in}{1.789039in}}%
\pgfusepath{stroke}%
\end{pgfscope}%
\begin{pgfscope}%
\pgfsetbuttcap%
\pgfsetroundjoin%
\definecolor{currentfill}{rgb}{0.000000,0.000000,0.000000}%
\pgfsetfillcolor{currentfill}%
\pgfsetlinewidth{0.602250pt}%
\definecolor{currentstroke}{rgb}{0.000000,0.000000,0.000000}%
\pgfsetstrokecolor{currentstroke}%
\pgfsetdash{}{0pt}%
\pgfsys@defobject{currentmarker}{\pgfqpoint{0.000000in}{-0.027778in}}{\pgfqpoint{0.000000in}{0.000000in}}{%
\pgfpathmoveto{\pgfqpoint{0.000000in}{0.000000in}}%
\pgfpathlineto{\pgfqpoint{0.000000in}{-0.027778in}}%
\pgfusepath{stroke,fill}%
}%
\begin{pgfscope}%
\pgfsys@transformshift{2.389333in}{0.417642in}%
\pgfsys@useobject{currentmarker}{}%
\end{pgfscope}%
\end{pgfscope}%
\begin{pgfscope}%
\definecolor{textcolor}{rgb}{0.000000,0.000000,0.000000}%
\pgfsetstrokecolor{textcolor}%
\pgfsetfillcolor{textcolor}%
\pgftext[x=1.456777in,y=0.165003in,,top]{\color{textcolor}\rmfamily\fontsize{10.000000}{12.000000}\selectfont Frequency in \(\displaystyle \unit{\Hz}\)}%
\end{pgfscope}%
\begin{pgfscope}%
\pgfpathrectangle{\pgfqpoint{0.514278in}{0.417642in}}{\pgfqpoint{1.884996in}{1.371397in}}%
\pgfusepath{clip}%
\pgfsetrectcap%
\pgfsetroundjoin%
\pgfsetlinewidth{0.803000pt}%
\definecolor{currentstroke}{rgb}{0.450000,0.450000,0.450000}%
\pgfsetstrokecolor{currentstroke}%
\pgfsetdash{}{0pt}%
\pgfpathmoveto{\pgfqpoint{0.514278in}{0.640670in}}%
\pgfpathlineto{\pgfqpoint{2.399275in}{0.640670in}}%
\pgfusepath{stroke}%
\end{pgfscope}%
\begin{pgfscope}%
\pgfsetbuttcap%
\pgfsetroundjoin%
\definecolor{currentfill}{rgb}{0.000000,0.000000,0.000000}%
\pgfsetfillcolor{currentfill}%
\pgfsetlinewidth{0.803000pt}%
\definecolor{currentstroke}{rgb}{0.000000,0.000000,0.000000}%
\pgfsetstrokecolor{currentstroke}%
\pgfsetdash{}{0pt}%
\pgfsys@defobject{currentmarker}{\pgfqpoint{-0.048611in}{0.000000in}}{\pgfqpoint{-0.000000in}{0.000000in}}{%
\pgfpathmoveto{\pgfqpoint{-0.000000in}{0.000000in}}%
\pgfpathlineto{\pgfqpoint{-0.048611in}{0.000000in}}%
\pgfusepath{stroke,fill}%
}%
\begin{pgfscope}%
\pgfsys@transformshift{0.514278in}{0.640670in}%
\pgfsys@useobject{currentmarker}{}%
\end{pgfscope}%
\end{pgfscope}%
\begin{pgfscope}%
\definecolor{textcolor}{rgb}{0.000000,0.000000,0.000000}%
\pgfsetstrokecolor{textcolor}%
\pgfsetfillcolor{textcolor}%
\pgftext[x=0.241129in, y=0.601518in, left, base]{\color{textcolor}\rmfamily\fontsize{8.000000}{9.600000}\selectfont \(\displaystyle {10^{0}}\)}%
\end{pgfscope}%
\begin{pgfscope}%
\pgfpathrectangle{\pgfqpoint{0.514278in}{0.417642in}}{\pgfqpoint{1.884996in}{1.371397in}}%
\pgfusepath{clip}%
\pgfsetrectcap%
\pgfsetroundjoin%
\pgfsetlinewidth{0.803000pt}%
\definecolor{currentstroke}{rgb}{0.450000,0.450000,0.450000}%
\pgfsetstrokecolor{currentstroke}%
\pgfsetdash{}{0pt}%
\pgfpathmoveto{\pgfqpoint{0.514278in}{0.983520in}}%
\pgfpathlineto{\pgfqpoint{2.399275in}{0.983520in}}%
\pgfusepath{stroke}%
\end{pgfscope}%
\begin{pgfscope}%
\pgfsetbuttcap%
\pgfsetroundjoin%
\definecolor{currentfill}{rgb}{0.000000,0.000000,0.000000}%
\pgfsetfillcolor{currentfill}%
\pgfsetlinewidth{0.803000pt}%
\definecolor{currentstroke}{rgb}{0.000000,0.000000,0.000000}%
\pgfsetstrokecolor{currentstroke}%
\pgfsetdash{}{0pt}%
\pgfsys@defobject{currentmarker}{\pgfqpoint{-0.048611in}{0.000000in}}{\pgfqpoint{-0.000000in}{0.000000in}}{%
\pgfpathmoveto{\pgfqpoint{-0.000000in}{0.000000in}}%
\pgfpathlineto{\pgfqpoint{-0.048611in}{0.000000in}}%
\pgfusepath{stroke,fill}%
}%
\begin{pgfscope}%
\pgfsys@transformshift{0.514278in}{0.983520in}%
\pgfsys@useobject{currentmarker}{}%
\end{pgfscope}%
\end{pgfscope}%
\begin{pgfscope}%
\definecolor{textcolor}{rgb}{0.000000,0.000000,0.000000}%
\pgfsetstrokecolor{textcolor}%
\pgfsetfillcolor{textcolor}%
\pgftext[x=0.241129in, y=0.944367in, left, base]{\color{textcolor}\rmfamily\fontsize{8.000000}{9.600000}\selectfont \(\displaystyle {10^{2}}\)}%
\end{pgfscope}%
\begin{pgfscope}%
\pgfpathrectangle{\pgfqpoint{0.514278in}{0.417642in}}{\pgfqpoint{1.884996in}{1.371397in}}%
\pgfusepath{clip}%
\pgfsetrectcap%
\pgfsetroundjoin%
\pgfsetlinewidth{0.803000pt}%
\definecolor{currentstroke}{rgb}{0.450000,0.450000,0.450000}%
\pgfsetstrokecolor{currentstroke}%
\pgfsetdash{}{0pt}%
\pgfpathmoveto{\pgfqpoint{0.514278in}{1.326369in}}%
\pgfpathlineto{\pgfqpoint{2.399275in}{1.326369in}}%
\pgfusepath{stroke}%
\end{pgfscope}%
\begin{pgfscope}%
\pgfsetbuttcap%
\pgfsetroundjoin%
\definecolor{currentfill}{rgb}{0.000000,0.000000,0.000000}%
\pgfsetfillcolor{currentfill}%
\pgfsetlinewidth{0.803000pt}%
\definecolor{currentstroke}{rgb}{0.000000,0.000000,0.000000}%
\pgfsetstrokecolor{currentstroke}%
\pgfsetdash{}{0pt}%
\pgfsys@defobject{currentmarker}{\pgfqpoint{-0.048611in}{0.000000in}}{\pgfqpoint{-0.000000in}{0.000000in}}{%
\pgfpathmoveto{\pgfqpoint{-0.000000in}{0.000000in}}%
\pgfpathlineto{\pgfqpoint{-0.048611in}{0.000000in}}%
\pgfusepath{stroke,fill}%
}%
\begin{pgfscope}%
\pgfsys@transformshift{0.514278in}{1.326369in}%
\pgfsys@useobject{currentmarker}{}%
\end{pgfscope}%
\end{pgfscope}%
\begin{pgfscope}%
\definecolor{textcolor}{rgb}{0.000000,0.000000,0.000000}%
\pgfsetstrokecolor{textcolor}%
\pgfsetfillcolor{textcolor}%
\pgftext[x=0.241129in, y=1.287216in, left, base]{\color{textcolor}\rmfamily\fontsize{8.000000}{9.600000}\selectfont \(\displaystyle {10^{4}}\)}%
\end{pgfscope}%
\begin{pgfscope}%
\pgfpathrectangle{\pgfqpoint{0.514278in}{0.417642in}}{\pgfqpoint{1.884996in}{1.371397in}}%
\pgfusepath{clip}%
\pgfsetrectcap%
\pgfsetroundjoin%
\pgfsetlinewidth{0.803000pt}%
\definecolor{currentstroke}{rgb}{0.450000,0.450000,0.450000}%
\pgfsetstrokecolor{currentstroke}%
\pgfsetdash{}{0pt}%
\pgfpathmoveto{\pgfqpoint{0.514278in}{1.669218in}}%
\pgfpathlineto{\pgfqpoint{2.399275in}{1.669218in}}%
\pgfusepath{stroke}%
\end{pgfscope}%
\begin{pgfscope}%
\pgfsetbuttcap%
\pgfsetroundjoin%
\definecolor{currentfill}{rgb}{0.000000,0.000000,0.000000}%
\pgfsetfillcolor{currentfill}%
\pgfsetlinewidth{0.803000pt}%
\definecolor{currentstroke}{rgb}{0.000000,0.000000,0.000000}%
\pgfsetstrokecolor{currentstroke}%
\pgfsetdash{}{0pt}%
\pgfsys@defobject{currentmarker}{\pgfqpoint{-0.048611in}{0.000000in}}{\pgfqpoint{-0.000000in}{0.000000in}}{%
\pgfpathmoveto{\pgfqpoint{-0.000000in}{0.000000in}}%
\pgfpathlineto{\pgfqpoint{-0.048611in}{0.000000in}}%
\pgfusepath{stroke,fill}%
}%
\begin{pgfscope}%
\pgfsys@transformshift{0.514278in}{1.669218in}%
\pgfsys@useobject{currentmarker}{}%
\end{pgfscope}%
\end{pgfscope}%
\begin{pgfscope}%
\definecolor{textcolor}{rgb}{0.000000,0.000000,0.000000}%
\pgfsetstrokecolor{textcolor}%
\pgfsetfillcolor{textcolor}%
\pgftext[x=0.241129in, y=1.630065in, left, base]{\color{textcolor}\rmfamily\fontsize{8.000000}{9.600000}\selectfont \(\displaystyle {10^{6}}\)}%
\end{pgfscope}%
\begin{pgfscope}%
\definecolor{textcolor}{rgb}{0.000000,0.000000,0.000000}%
\pgfsetstrokecolor{textcolor}%
\pgfsetfillcolor{textcolor}%
\pgftext[x=0.185574in,y=1.103340in,,bottom,rotate=90.000000]{\color{textcolor}\rmfamily\fontsize{10.000000}{12.000000}\selectfont \(\displaystyle S_y(f)\) in \(\displaystyle \unit{1 \per \Hz}\)}%
\end{pgfscope}%
\begin{pgfscope}%
\pgfpathrectangle{\pgfqpoint{0.514278in}{0.417642in}}{\pgfqpoint{1.884996in}{1.371397in}}%
\pgfusepath{clip}%
\pgfsetbuttcap%
\pgfsetroundjoin%
\pgfsetlinewidth{1.505625pt}%
\definecolor{currentstroke}{rgb}{0.007843,0.619608,0.450980}%
\pgfsetstrokecolor{currentstroke}%
\pgfsetdash{{5.550000pt}{2.400000pt}}{0.000000pt}%
\pgfpathmoveto{\pgfqpoint{0.599960in}{1.235582in}}%
\pgfpathlineto{\pgfqpoint{0.755760in}{1.183978in}}%
\pgfpathlineto{\pgfqpoint{0.846897in}{1.153792in}}%
\pgfpathlineto{\pgfqpoint{0.911560in}{1.132374in}}%
\pgfpathlineto{\pgfqpoint{0.961716in}{1.115761in}}%
\pgfpathlineto{\pgfqpoint{1.002697in}{1.102188in}}%
\pgfpathlineto{\pgfqpoint{1.037345in}{1.090712in}}%
\pgfpathlineto{\pgfqpoint{1.067360in}{1.080770in}}%
\pgfpathlineto{\pgfqpoint{1.093834in}{1.072002in}}%
\pgfpathlineto{\pgfqpoint{1.117516in}{1.064158in}}%
\pgfpathlineto{\pgfqpoint{1.138939in}{1.057062in}}%
\pgfpathlineto{\pgfqpoint{1.158497in}{1.050584in}}%
\pgfpathlineto{\pgfqpoint{1.176488in}{1.044625in}}%
\pgfpathlineto{\pgfqpoint{1.193145in}{1.039108in}}%
\pgfpathlineto{\pgfqpoint{1.208653in}{1.033971in}}%
\pgfpathlineto{\pgfqpoint{1.223159in}{1.029166in}}%
\pgfpathlineto{\pgfqpoint{1.236786in}{1.024653in}}%
\pgfpathlineto{\pgfqpoint{1.249634in}{1.020398in}}%
\pgfpathlineto{\pgfqpoint{1.261786in}{1.016372in}}%
\pgfpathlineto{\pgfqpoint{1.273316in}{1.012554in}}%
\pgfpathlineto{\pgfqpoint{1.284282in}{1.008921in}}%
\pgfpathlineto{\pgfqpoint{1.294739in}{1.005458in}}%
\pgfpathlineto{\pgfqpoint{1.309564in}{1.000547in}}%
\pgfpathlineto{\pgfqpoint{1.323472in}{0.995941in}}%
\pgfpathlineto{\pgfqpoint{1.332288in}{0.993021in}}%
\pgfpathlineto{\pgfqpoint{1.340771in}{0.990211in}}%
\pgfpathlineto{\pgfqpoint{1.348945in}{0.987504in}}%
\pgfpathlineto{\pgfqpoint{1.360675in}{0.983618in}}%
\pgfpathlineto{\pgfqpoint{1.371823in}{0.979926in}}%
\pgfpathlineto{\pgfqpoint{1.382444in}{0.976408in}}%
\pgfpathlineto{\pgfqpoint{1.392586in}{0.973049in}}%
\pgfpathlineto{\pgfqpoint{1.402290in}{0.969835in}}%
\pgfpathlineto{\pgfqpoint{1.414609in}{0.965754in}}%
\pgfpathlineto{\pgfqpoint{1.426288in}{0.961886in}}%
\pgfpathlineto{\pgfqpoint{1.434666in}{0.959111in}}%
\pgfpathlineto{\pgfqpoint{1.442742in}{0.956436in}}%
\pgfpathlineto{\pgfqpoint{1.453078in}{0.953013in}}%
\pgfpathlineto{\pgfqpoint{1.465364in}{0.948943in}}%
\pgfpathlineto{\pgfqpoint{1.477013in}{0.945085in}}%
\pgfpathlineto{\pgfqpoint{1.485916in}{0.942136in}}%
\pgfpathlineto{\pgfqpoint{1.496570in}{0.938607in}}%
\pgfpathlineto{\pgfqpoint{1.506743in}{0.935238in}}%
\pgfpathlineto{\pgfqpoint{1.516475in}{0.932015in}}%
\pgfpathlineto{\pgfqpoint{1.527623in}{0.928322in}}%
\pgfpathlineto{\pgfqpoint{1.538244in}{0.924804in}}%
\pgfpathlineto{\pgfqpoint{1.548386in}{0.921445in}}%
\pgfpathlineto{\pgfqpoint{1.559667in}{0.917708in}}%
\pgfpathlineto{\pgfqpoint{1.570409in}{0.914151in}}%
\pgfpathlineto{\pgfqpoint{1.580661in}{0.910755in}}%
\pgfpathlineto{\pgfqpoint{1.590465in}{0.907507in}}%
\pgfpathlineto{\pgfqpoint{1.599860in}{0.904396in}}%
\pgfpathlineto{\pgfqpoint{1.611390in}{0.900577in}}%
\pgfpathlineto{\pgfqpoint{1.622356in}{0.896945in}}%
\pgfpathlineto{\pgfqpoint{1.631675in}{0.893858in}}%
\pgfpathlineto{\pgfqpoint{1.641716in}{0.890532in}}%
\pgfpathlineto{\pgfqpoint{1.652370in}{0.887003in}}%
\pgfpathlineto{\pgfqpoint{1.663535in}{0.883305in}}%
\pgfpathlineto{\pgfqpoint{1.674171in}{0.879782in}}%
\pgfpathlineto{\pgfqpoint{1.684327in}{0.876419in}}%
\pgfpathlineto{\pgfqpoint{1.694907in}{0.872914in}}%
\pgfpathlineto{\pgfqpoint{1.705010in}{0.869568in}}%
\pgfpathlineto{\pgfqpoint{1.715467in}{0.866105in}}%
\pgfpathlineto{\pgfqpoint{1.726209in}{0.862547in}}%
\pgfpathlineto{\pgfqpoint{1.736461in}{0.859151in}}%
\pgfpathlineto{\pgfqpoint{1.746950in}{0.855677in}}%
\pgfpathlineto{\pgfqpoint{1.756971in}{0.852358in}}%
\pgfpathlineto{\pgfqpoint{1.767189in}{0.848973in}}%
\pgfpathlineto{\pgfqpoint{1.778156in}{0.845341in}}%
\pgfpathlineto{\pgfqpoint{1.788612in}{0.841877in}}%
\pgfpathlineto{\pgfqpoint{1.798604in}{0.838568in}}%
\pgfpathlineto{\pgfqpoint{1.808690in}{0.835227in}}%
\pgfpathlineto{\pgfqpoint{1.819335in}{0.831701in}}%
\pgfpathlineto{\pgfqpoint{1.829971in}{0.828178in}}%
\pgfpathlineto{\pgfqpoint{1.840127in}{0.824815in}}%
\pgfpathlineto{\pgfqpoint{1.850275in}{0.821453in}}%
\pgfpathlineto{\pgfqpoint{1.860810in}{0.817964in}}%
\pgfpathlineto{\pgfqpoint{1.871266in}{0.814501in}}%
\pgfpathlineto{\pgfqpoint{1.881634in}{0.811067in}}%
\pgfpathlineto{\pgfqpoint{1.892260in}{0.807547in}}%
\pgfpathlineto{\pgfqpoint{1.902749in}{0.804073in}}%
\pgfpathlineto{\pgfqpoint{1.913097in}{0.800646in}}%
\pgfpathlineto{\pgfqpoint{1.923613in}{0.797163in}}%
\pgfpathlineto{\pgfqpoint{1.933956in}{0.793737in}}%
\pgfpathlineto{\pgfqpoint{1.944128in}{0.790367in}}%
\pgfpathlineto{\pgfqpoint{1.954404in}{0.786964in}}%
\pgfpathlineto{\pgfqpoint{1.965008in}{0.783452in}}%
\pgfpathlineto{\pgfqpoint{1.975629in}{0.779934in}}%
\pgfpathlineto{\pgfqpoint{1.986007in}{0.776496in}}%
\pgfpathlineto{\pgfqpoint{1.996378in}{0.773061in}}%
\pgfpathlineto{\pgfqpoint{2.006721in}{0.769635in}}%
\pgfpathlineto{\pgfqpoint{2.017021in}{0.766224in}}%
\pgfpathlineto{\pgfqpoint{2.027459in}{0.762767in}}%
\pgfpathlineto{\pgfqpoint{2.037808in}{0.759339in}}%
\pgfpathlineto{\pgfqpoint{2.048239in}{0.755884in}}%
\pgfpathlineto{\pgfqpoint{2.058720in}{0.752412in}}%
\pgfpathlineto{\pgfqpoint{2.069060in}{0.748988in}}%
\pgfpathlineto{\pgfqpoint{2.079412in}{0.745559in}}%
\pgfpathlineto{\pgfqpoint{2.089756in}{0.742133in}}%
\pgfpathlineto{\pgfqpoint{2.100212in}{0.738669in}}%
\pgfpathlineto{\pgfqpoint{2.110610in}{0.735225in}}%
\pgfpathlineto{\pgfqpoint{2.121067in}{0.731762in}}%
\pgfpathlineto{\pgfqpoint{2.131552in}{0.728289in}}%
\pgfpathlineto{\pgfqpoint{2.141925in}{0.724853in}}%
\pgfpathlineto{\pgfqpoint{2.152290in}{0.721420in}}%
\pgfpathlineto{\pgfqpoint{2.162629in}{0.717996in}}%
\pgfpathlineto{\pgfqpoint{2.173026in}{0.714552in}}%
\pgfpathlineto{\pgfqpoint{2.183455in}{0.711098in}}%
\pgfpathlineto{\pgfqpoint{2.193889in}{0.707642in}}%
\pgfpathlineto{\pgfqpoint{2.204307in}{0.704191in}}%
\pgfpathlineto{\pgfqpoint{2.214690in}{0.700752in}}%
\pgfpathlineto{\pgfqpoint{2.225104in}{0.697303in}}%
\pgfpathlineto{\pgfqpoint{2.235523in}{0.693852in}}%
\pgfpathlineto{\pgfqpoint{2.245927in}{0.690406in}}%
\pgfpathlineto{\pgfqpoint{2.256295in}{0.686971in}}%
\pgfpathlineto{\pgfqpoint{2.266681in}{0.683532in}}%
\pgfpathlineto{\pgfqpoint{2.277125in}{0.680072in}}%
\pgfpathlineto{\pgfqpoint{2.287537in}{0.676624in}}%
\pgfpathlineto{\pgfqpoint{2.297901in}{0.673191in}}%
\pgfpathlineto{\pgfqpoint{2.308315in}{0.669742in}}%
\pgfpathlineto{\pgfqpoint{2.313593in}{0.667993in}}%
\pgfusepath{stroke}%
\end{pgfscope}%
\begin{pgfscope}%
\pgfpathrectangle{\pgfqpoint{0.514278in}{0.417642in}}{\pgfqpoint{1.884996in}{1.371397in}}%
\pgfusepath{clip}%
\pgfsetbuttcap%
\pgfsetroundjoin%
\definecolor{currentfill}{rgb}{0.007843,0.619608,0.450980}%
\pgfsetfillcolor{currentfill}%
\pgfsetlinewidth{1.003750pt}%
\definecolor{currentstroke}{rgb}{0.007843,0.619608,0.450980}%
\pgfsetstrokecolor{currentstroke}%
\pgfsetdash{}{0pt}%
\pgfsys@defobject{currentmarker}{\pgfqpoint{-0.006944in}{-0.006944in}}{\pgfqpoint{0.006944in}{0.006944in}}{%
\pgfpathmoveto{\pgfqpoint{0.000000in}{-0.006944in}}%
\pgfpathcurveto{\pgfqpoint{0.001842in}{-0.006944in}}{\pgfqpoint{0.003608in}{-0.006213in}}{\pgfqpoint{0.004910in}{-0.004910in}}%
\pgfpathcurveto{\pgfqpoint{0.006213in}{-0.003608in}}{\pgfqpoint{0.006944in}{-0.001842in}}{\pgfqpoint{0.006944in}{0.000000in}}%
\pgfpathcurveto{\pgfqpoint{0.006944in}{0.001842in}}{\pgfqpoint{0.006213in}{0.003608in}}{\pgfqpoint{0.004910in}{0.004910in}}%
\pgfpathcurveto{\pgfqpoint{0.003608in}{0.006213in}}{\pgfqpoint{0.001842in}{0.006944in}}{\pgfqpoint{0.000000in}{0.006944in}}%
\pgfpathcurveto{\pgfqpoint{-0.001842in}{0.006944in}}{\pgfqpoint{-0.003608in}{0.006213in}}{\pgfqpoint{-0.004910in}{0.004910in}}%
\pgfpathcurveto{\pgfqpoint{-0.006213in}{0.003608in}}{\pgfqpoint{-0.006944in}{0.001842in}}{\pgfqpoint{-0.006944in}{0.000000in}}%
\pgfpathcurveto{\pgfqpoint{-0.006944in}{-0.001842in}}{\pgfqpoint{-0.006213in}{-0.003608in}}{\pgfqpoint{-0.004910in}{-0.004910in}}%
\pgfpathcurveto{\pgfqpoint{-0.003608in}{-0.006213in}}{\pgfqpoint{-0.001842in}{-0.006944in}}{\pgfqpoint{0.000000in}{-0.006944in}}%
\pgfpathlineto{\pgfqpoint{0.000000in}{-0.006944in}}%
\pgfpathclose%
\pgfusepath{stroke,fill}%
}%
\begin{pgfscope}%
\pgfsys@transformshift{0.599960in}{1.223928in}%
\pgfsys@useobject{currentmarker}{}%
\end{pgfscope}%
\begin{pgfscope}%
\pgfsys@transformshift{0.755760in}{1.133873in}%
\pgfsys@useobject{currentmarker}{}%
\end{pgfscope}%
\begin{pgfscope}%
\pgfsys@transformshift{0.846897in}{1.112772in}%
\pgfsys@useobject{currentmarker}{}%
\end{pgfscope}%
\begin{pgfscope}%
\pgfsys@transformshift{0.911560in}{1.087150in}%
\pgfsys@useobject{currentmarker}{}%
\end{pgfscope}%
\begin{pgfscope}%
\pgfsys@transformshift{0.961716in}{1.077979in}%
\pgfsys@useobject{currentmarker}{}%
\end{pgfscope}%
\begin{pgfscope}%
\pgfsys@transformshift{1.002697in}{1.028039in}%
\pgfsys@useobject{currentmarker}{}%
\end{pgfscope}%
\begin{pgfscope}%
\pgfsys@transformshift{1.037345in}{1.099417in}%
\pgfsys@useobject{currentmarker}{}%
\end{pgfscope}%
\begin{pgfscope}%
\pgfsys@transformshift{1.067360in}{1.110978in}%
\pgfsys@useobject{currentmarker}{}%
\end{pgfscope}%
\begin{pgfscope}%
\pgfsys@transformshift{1.093834in}{1.091707in}%
\pgfsys@useobject{currentmarker}{}%
\end{pgfscope}%
\begin{pgfscope}%
\pgfsys@transformshift{1.117516in}{1.074621in}%
\pgfsys@useobject{currentmarker}{}%
\end{pgfscope}%
\begin{pgfscope}%
\pgfsys@transformshift{1.138939in}{1.038922in}%
\pgfsys@useobject{currentmarker}{}%
\end{pgfscope}%
\begin{pgfscope}%
\pgfsys@transformshift{1.158497in}{1.072460in}%
\pgfsys@useobject{currentmarker}{}%
\end{pgfscope}%
\begin{pgfscope}%
\pgfsys@transformshift{1.176488in}{1.045045in}%
\pgfsys@useobject{currentmarker}{}%
\end{pgfscope}%
\begin{pgfscope}%
\pgfsys@transformshift{1.193145in}{0.976377in}%
\pgfsys@useobject{currentmarker}{}%
\end{pgfscope}%
\begin{pgfscope}%
\pgfsys@transformshift{1.208653in}{1.042403in}%
\pgfsys@useobject{currentmarker}{}%
\end{pgfscope}%
\begin{pgfscope}%
\pgfsys@transformshift{1.223159in}{1.066147in}%
\pgfsys@useobject{currentmarker}{}%
\end{pgfscope}%
\begin{pgfscope}%
\pgfsys@transformshift{1.236786in}{1.041717in}%
\pgfsys@useobject{currentmarker}{}%
\end{pgfscope}%
\begin{pgfscope}%
\pgfsys@transformshift{1.249634in}{0.934765in}%
\pgfsys@useobject{currentmarker}{}%
\end{pgfscope}%
\begin{pgfscope}%
\pgfsys@transformshift{1.261786in}{0.984899in}%
\pgfsys@useobject{currentmarker}{}%
\end{pgfscope}%
\begin{pgfscope}%
\pgfsys@transformshift{1.273316in}{1.050658in}%
\pgfsys@useobject{currentmarker}{}%
\end{pgfscope}%
\begin{pgfscope}%
\pgfsys@transformshift{1.284282in}{1.066325in}%
\pgfsys@useobject{currentmarker}{}%
\end{pgfscope}%
\begin{pgfscope}%
\pgfsys@transformshift{1.294739in}{1.028193in}%
\pgfsys@useobject{currentmarker}{}%
\end{pgfscope}%
\begin{pgfscope}%
\pgfsys@transformshift{1.309564in}{0.985648in}%
\pgfsys@useobject{currentmarker}{}%
\end{pgfscope}%
\begin{pgfscope}%
\pgfsys@transformshift{1.323472in}{0.981024in}%
\pgfsys@useobject{currentmarker}{}%
\end{pgfscope}%
\begin{pgfscope}%
\pgfsys@transformshift{1.332288in}{0.960523in}%
\pgfsys@useobject{currentmarker}{}%
\end{pgfscope}%
\begin{pgfscope}%
\pgfsys@transformshift{1.340771in}{0.954804in}%
\pgfsys@useobject{currentmarker}{}%
\end{pgfscope}%
\begin{pgfscope}%
\pgfsys@transformshift{1.348945in}{1.001306in}%
\pgfsys@useobject{currentmarker}{}%
\end{pgfscope}%
\begin{pgfscope}%
\pgfsys@transformshift{1.360675in}{0.988125in}%
\pgfsys@useobject{currentmarker}{}%
\end{pgfscope}%
\begin{pgfscope}%
\pgfsys@transformshift{1.371823in}{0.987385in}%
\pgfsys@useobject{currentmarker}{}%
\end{pgfscope}%
\begin{pgfscope}%
\pgfsys@transformshift{1.382444in}{0.963934in}%
\pgfsys@useobject{currentmarker}{}%
\end{pgfscope}%
\begin{pgfscope}%
\pgfsys@transformshift{1.392586in}{0.992313in}%
\pgfsys@useobject{currentmarker}{}%
\end{pgfscope}%
\begin{pgfscope}%
\pgfsys@transformshift{1.402290in}{0.971395in}%
\pgfsys@useobject{currentmarker}{}%
\end{pgfscope}%
\begin{pgfscope}%
\pgfsys@transformshift{1.414609in}{0.960609in}%
\pgfsys@useobject{currentmarker}{}%
\end{pgfscope}%
\begin{pgfscope}%
\pgfsys@transformshift{1.426288in}{0.928696in}%
\pgfsys@useobject{currentmarker}{}%
\end{pgfscope}%
\begin{pgfscope}%
\pgfsys@transformshift{1.434666in}{0.975645in}%
\pgfsys@useobject{currentmarker}{}%
\end{pgfscope}%
\begin{pgfscope}%
\pgfsys@transformshift{1.442742in}{0.996958in}%
\pgfsys@useobject{currentmarker}{}%
\end{pgfscope}%
\begin{pgfscope}%
\pgfsys@transformshift{1.453078in}{0.924735in}%
\pgfsys@useobject{currentmarker}{}%
\end{pgfscope}%
\begin{pgfscope}%
\pgfsys@transformshift{1.465364in}{0.930732in}%
\pgfsys@useobject{currentmarker}{}%
\end{pgfscope}%
\begin{pgfscope}%
\pgfsys@transformshift{1.477013in}{0.920723in}%
\pgfsys@useobject{currentmarker}{}%
\end{pgfscope}%
\begin{pgfscope}%
\pgfsys@transformshift{1.485916in}{0.857641in}%
\pgfsys@useobject{currentmarker}{}%
\end{pgfscope}%
\begin{pgfscope}%
\pgfsys@transformshift{1.496570in}{0.947202in}%
\pgfsys@useobject{currentmarker}{}%
\end{pgfscope}%
\begin{pgfscope}%
\pgfsys@transformshift{1.506743in}{0.966057in}%
\pgfsys@useobject{currentmarker}{}%
\end{pgfscope}%
\begin{pgfscope}%
\pgfsys@transformshift{1.516475in}{0.944591in}%
\pgfsys@useobject{currentmarker}{}%
\end{pgfscope}%
\begin{pgfscope}%
\pgfsys@transformshift{1.527623in}{0.885441in}%
\pgfsys@useobject{currentmarker}{}%
\end{pgfscope}%
\begin{pgfscope}%
\pgfsys@transformshift{1.538244in}{0.885741in}%
\pgfsys@useobject{currentmarker}{}%
\end{pgfscope}%
\begin{pgfscope}%
\pgfsys@transformshift{1.548386in}{0.925663in}%
\pgfsys@useobject{currentmarker}{}%
\end{pgfscope}%
\begin{pgfscope}%
\pgfsys@transformshift{1.559667in}{0.910685in}%
\pgfsys@useobject{currentmarker}{}%
\end{pgfscope}%
\begin{pgfscope}%
\pgfsys@transformshift{1.570409in}{0.909248in}%
\pgfsys@useobject{currentmarker}{}%
\end{pgfscope}%
\begin{pgfscope}%
\pgfsys@transformshift{1.580661in}{0.886260in}%
\pgfsys@useobject{currentmarker}{}%
\end{pgfscope}%
\begin{pgfscope}%
\pgfsys@transformshift{1.590465in}{0.888013in}%
\pgfsys@useobject{currentmarker}{}%
\end{pgfscope}%
\begin{pgfscope}%
\pgfsys@transformshift{1.599860in}{0.860927in}%
\pgfsys@useobject{currentmarker}{}%
\end{pgfscope}%
\begin{pgfscope}%
\pgfsys@transformshift{1.611390in}{0.869364in}%
\pgfsys@useobject{currentmarker}{}%
\end{pgfscope}%
\begin{pgfscope}%
\pgfsys@transformshift{1.622356in}{0.884202in}%
\pgfsys@useobject{currentmarker}{}%
\end{pgfscope}%
\begin{pgfscope}%
\pgfsys@transformshift{1.631675in}{0.862991in}%
\pgfsys@useobject{currentmarker}{}%
\end{pgfscope}%
\begin{pgfscope}%
\pgfsys@transformshift{1.641716in}{0.872746in}%
\pgfsys@useobject{currentmarker}{}%
\end{pgfscope}%
\begin{pgfscope}%
\pgfsys@transformshift{1.652370in}{0.892831in}%
\pgfsys@useobject{currentmarker}{}%
\end{pgfscope}%
\begin{pgfscope}%
\pgfsys@transformshift{1.663535in}{0.864944in}%
\pgfsys@useobject{currentmarker}{}%
\end{pgfscope}%
\begin{pgfscope}%
\pgfsys@transformshift{1.674171in}{0.868548in}%
\pgfsys@useobject{currentmarker}{}%
\end{pgfscope}%
\begin{pgfscope}%
\pgfsys@transformshift{1.684327in}{0.861120in}%
\pgfsys@useobject{currentmarker}{}%
\end{pgfscope}%
\begin{pgfscope}%
\pgfsys@transformshift{1.694907in}{0.866134in}%
\pgfsys@useobject{currentmarker}{}%
\end{pgfscope}%
\begin{pgfscope}%
\pgfsys@transformshift{1.705010in}{0.893510in}%
\pgfsys@useobject{currentmarker}{}%
\end{pgfscope}%
\begin{pgfscope}%
\pgfsys@transformshift{1.715467in}{0.866045in}%
\pgfsys@useobject{currentmarker}{}%
\end{pgfscope}%
\begin{pgfscope}%
\pgfsys@transformshift{1.726209in}{0.867177in}%
\pgfsys@useobject{currentmarker}{}%
\end{pgfscope}%
\begin{pgfscope}%
\pgfsys@transformshift{1.736461in}{0.869886in}%
\pgfsys@useobject{currentmarker}{}%
\end{pgfscope}%
\begin{pgfscope}%
\pgfsys@transformshift{1.746950in}{0.847004in}%
\pgfsys@useobject{currentmarker}{}%
\end{pgfscope}%
\begin{pgfscope}%
\pgfsys@transformshift{1.756971in}{0.788155in}%
\pgfsys@useobject{currentmarker}{}%
\end{pgfscope}%
\begin{pgfscope}%
\pgfsys@transformshift{1.767189in}{0.852985in}%
\pgfsys@useobject{currentmarker}{}%
\end{pgfscope}%
\begin{pgfscope}%
\pgfsys@transformshift{1.778156in}{0.861807in}%
\pgfsys@useobject{currentmarker}{}%
\end{pgfscope}%
\begin{pgfscope}%
\pgfsys@transformshift{1.788612in}{0.869935in}%
\pgfsys@useobject{currentmarker}{}%
\end{pgfscope}%
\begin{pgfscope}%
\pgfsys@transformshift{1.798604in}{0.831057in}%
\pgfsys@useobject{currentmarker}{}%
\end{pgfscope}%
\begin{pgfscope}%
\pgfsys@transformshift{1.808690in}{0.826776in}%
\pgfsys@useobject{currentmarker}{}%
\end{pgfscope}%
\begin{pgfscope}%
\pgfsys@transformshift{1.819335in}{0.807243in}%
\pgfsys@useobject{currentmarker}{}%
\end{pgfscope}%
\begin{pgfscope}%
\pgfsys@transformshift{1.829971in}{0.831039in}%
\pgfsys@useobject{currentmarker}{}%
\end{pgfscope}%
\begin{pgfscope}%
\pgfsys@transformshift{1.840127in}{0.820789in}%
\pgfsys@useobject{currentmarker}{}%
\end{pgfscope}%
\begin{pgfscope}%
\pgfsys@transformshift{1.850275in}{0.835396in}%
\pgfsys@useobject{currentmarker}{}%
\end{pgfscope}%
\begin{pgfscope}%
\pgfsys@transformshift{1.860810in}{0.795415in}%
\pgfsys@useobject{currentmarker}{}%
\end{pgfscope}%
\begin{pgfscope}%
\pgfsys@transformshift{1.871266in}{0.804096in}%
\pgfsys@useobject{currentmarker}{}%
\end{pgfscope}%
\begin{pgfscope}%
\pgfsys@transformshift{1.881634in}{0.811750in}%
\pgfsys@useobject{currentmarker}{}%
\end{pgfscope}%
\begin{pgfscope}%
\pgfsys@transformshift{1.892260in}{0.812710in}%
\pgfsys@useobject{currentmarker}{}%
\end{pgfscope}%
\begin{pgfscope}%
\pgfsys@transformshift{1.902749in}{0.830434in}%
\pgfsys@useobject{currentmarker}{}%
\end{pgfscope}%
\begin{pgfscope}%
\pgfsys@transformshift{1.913097in}{0.792247in}%
\pgfsys@useobject{currentmarker}{}%
\end{pgfscope}%
\begin{pgfscope}%
\pgfsys@transformshift{1.923613in}{0.782758in}%
\pgfsys@useobject{currentmarker}{}%
\end{pgfscope}%
\begin{pgfscope}%
\pgfsys@transformshift{1.933956in}{0.776822in}%
\pgfsys@useobject{currentmarker}{}%
\end{pgfscope}%
\begin{pgfscope}%
\pgfsys@transformshift{1.944128in}{0.791340in}%
\pgfsys@useobject{currentmarker}{}%
\end{pgfscope}%
\begin{pgfscope}%
\pgfsys@transformshift{1.954404in}{0.800167in}%
\pgfsys@useobject{currentmarker}{}%
\end{pgfscope}%
\begin{pgfscope}%
\pgfsys@transformshift{1.965008in}{0.784040in}%
\pgfsys@useobject{currentmarker}{}%
\end{pgfscope}%
\begin{pgfscope}%
\pgfsys@transformshift{1.975629in}{0.774833in}%
\pgfsys@useobject{currentmarker}{}%
\end{pgfscope}%
\begin{pgfscope}%
\pgfsys@transformshift{1.986007in}{0.778107in}%
\pgfsys@useobject{currentmarker}{}%
\end{pgfscope}%
\begin{pgfscope}%
\pgfsys@transformshift{1.996378in}{0.778122in}%
\pgfsys@useobject{currentmarker}{}%
\end{pgfscope}%
\begin{pgfscope}%
\pgfsys@transformshift{2.006721in}{0.767873in}%
\pgfsys@useobject{currentmarker}{}%
\end{pgfscope}%
\begin{pgfscope}%
\pgfsys@transformshift{2.017021in}{0.771471in}%
\pgfsys@useobject{currentmarker}{}%
\end{pgfscope}%
\begin{pgfscope}%
\pgfsys@transformshift{2.027459in}{0.768142in}%
\pgfsys@useobject{currentmarker}{}%
\end{pgfscope}%
\begin{pgfscope}%
\pgfsys@transformshift{2.037808in}{0.766022in}%
\pgfsys@useobject{currentmarker}{}%
\end{pgfscope}%
\begin{pgfscope}%
\pgfsys@transformshift{2.048239in}{0.767397in}%
\pgfsys@useobject{currentmarker}{}%
\end{pgfscope}%
\begin{pgfscope}%
\pgfsys@transformshift{2.058720in}{0.757491in}%
\pgfsys@useobject{currentmarker}{}%
\end{pgfscope}%
\begin{pgfscope}%
\pgfsys@transformshift{2.069060in}{0.747619in}%
\pgfsys@useobject{currentmarker}{}%
\end{pgfscope}%
\begin{pgfscope}%
\pgfsys@transformshift{2.079412in}{0.753371in}%
\pgfsys@useobject{currentmarker}{}%
\end{pgfscope}%
\begin{pgfscope}%
\pgfsys@transformshift{2.089756in}{0.744035in}%
\pgfsys@useobject{currentmarker}{}%
\end{pgfscope}%
\begin{pgfscope}%
\pgfsys@transformshift{2.100212in}{0.758842in}%
\pgfsys@useobject{currentmarker}{}%
\end{pgfscope}%
\begin{pgfscope}%
\pgfsys@transformshift{2.110610in}{0.746118in}%
\pgfsys@useobject{currentmarker}{}%
\end{pgfscope}%
\begin{pgfscope}%
\pgfsys@transformshift{2.121067in}{0.740097in}%
\pgfsys@useobject{currentmarker}{}%
\end{pgfscope}%
\begin{pgfscope}%
\pgfsys@transformshift{2.131552in}{0.734766in}%
\pgfsys@useobject{currentmarker}{}%
\end{pgfscope}%
\begin{pgfscope}%
\pgfsys@transformshift{2.141925in}{0.734322in}%
\pgfsys@useobject{currentmarker}{}%
\end{pgfscope}%
\begin{pgfscope}%
\pgfsys@transformshift{2.152290in}{0.727832in}%
\pgfsys@useobject{currentmarker}{}%
\end{pgfscope}%
\begin{pgfscope}%
\pgfsys@transformshift{2.162629in}{0.725064in}%
\pgfsys@useobject{currentmarker}{}%
\end{pgfscope}%
\begin{pgfscope}%
\pgfsys@transformshift{2.173026in}{0.724340in}%
\pgfsys@useobject{currentmarker}{}%
\end{pgfscope}%
\begin{pgfscope}%
\pgfsys@transformshift{2.183455in}{0.726605in}%
\pgfsys@useobject{currentmarker}{}%
\end{pgfscope}%
\begin{pgfscope}%
\pgfsys@transformshift{2.193889in}{0.723148in}%
\pgfsys@useobject{currentmarker}{}%
\end{pgfscope}%
\begin{pgfscope}%
\pgfsys@transformshift{2.204307in}{0.708706in}%
\pgfsys@useobject{currentmarker}{}%
\end{pgfscope}%
\begin{pgfscope}%
\pgfsys@transformshift{2.214690in}{0.703887in}%
\pgfsys@useobject{currentmarker}{}%
\end{pgfscope}%
\begin{pgfscope}%
\pgfsys@transformshift{2.225104in}{0.715935in}%
\pgfsys@useobject{currentmarker}{}%
\end{pgfscope}%
\begin{pgfscope}%
\pgfsys@transformshift{2.235523in}{0.717164in}%
\pgfsys@useobject{currentmarker}{}%
\end{pgfscope}%
\begin{pgfscope}%
\pgfsys@transformshift{2.245927in}{0.702726in}%
\pgfsys@useobject{currentmarker}{}%
\end{pgfscope}%
\begin{pgfscope}%
\pgfsys@transformshift{2.256295in}{0.705242in}%
\pgfsys@useobject{currentmarker}{}%
\end{pgfscope}%
\begin{pgfscope}%
\pgfsys@transformshift{2.266681in}{0.709932in}%
\pgfsys@useobject{currentmarker}{}%
\end{pgfscope}%
\begin{pgfscope}%
\pgfsys@transformshift{2.277125in}{0.705993in}%
\pgfsys@useobject{currentmarker}{}%
\end{pgfscope}%
\begin{pgfscope}%
\pgfsys@transformshift{2.287537in}{0.712018in}%
\pgfsys@useobject{currentmarker}{}%
\end{pgfscope}%
\begin{pgfscope}%
\pgfsys@transformshift{2.297901in}{0.696968in}%
\pgfsys@useobject{currentmarker}{}%
\end{pgfscope}%
\begin{pgfscope}%
\pgfsys@transformshift{2.308315in}{0.697962in}%
\pgfsys@useobject{currentmarker}{}%
\end{pgfscope}%
\begin{pgfscope}%
\pgfsys@transformshift{2.313593in}{0.684136in}%
\pgfsys@useobject{currentmarker}{}%
\end{pgfscope}%
\end{pgfscope}%
\begin{pgfscope}%
\pgfsetrectcap%
\pgfsetmiterjoin%
\pgfsetlinewidth{0.803000pt}%
\definecolor{currentstroke}{rgb}{0.000000,0.000000,0.000000}%
\pgfsetstrokecolor{currentstroke}%
\pgfsetdash{}{0pt}%
\pgfpathmoveto{\pgfqpoint{0.514278in}{0.417642in}}%
\pgfpathlineto{\pgfqpoint{0.514278in}{1.789039in}}%
\pgfusepath{stroke}%
\end{pgfscope}%
\begin{pgfscope}%
\pgfsetrectcap%
\pgfsetmiterjoin%
\pgfsetlinewidth{0.803000pt}%
\definecolor{currentstroke}{rgb}{0.000000,0.000000,0.000000}%
\pgfsetstrokecolor{currentstroke}%
\pgfsetdash{}{0pt}%
\pgfpathmoveto{\pgfqpoint{2.399275in}{0.417642in}}%
\pgfpathlineto{\pgfqpoint{2.399275in}{1.789039in}}%
\pgfusepath{stroke}%
\end{pgfscope}%
\begin{pgfscope}%
\pgfsetrectcap%
\pgfsetmiterjoin%
\pgfsetlinewidth{0.803000pt}%
\definecolor{currentstroke}{rgb}{0.000000,0.000000,0.000000}%
\pgfsetstrokecolor{currentstroke}%
\pgfsetdash{}{0pt}%
\pgfpathmoveto{\pgfqpoint{0.514278in}{0.417642in}}%
\pgfpathlineto{\pgfqpoint{2.399275in}{0.417642in}}%
\pgfusepath{stroke}%
\end{pgfscope}%
\begin{pgfscope}%
\pgfsetrectcap%
\pgfsetmiterjoin%
\pgfsetlinewidth{0.803000pt}%
\definecolor{currentstroke}{rgb}{0.000000,0.000000,0.000000}%
\pgfsetstrokecolor{currentstroke}%
\pgfsetdash{}{0pt}%
\pgfpathmoveto{\pgfqpoint{0.514278in}{1.789039in}}%
\pgfpathlineto{\pgfqpoint{2.399275in}{1.789039in}}%
\pgfusepath{stroke}%
\end{pgfscope}%
\begin{pgfscope}%
\pgfsetbuttcap%
\pgfsetmiterjoin%
\definecolor{currentfill}{rgb}{1.000000,1.000000,1.000000}%
\pgfsetfillcolor{currentfill}%
\pgfsetfillopacity{0.800000}%
\pgfsetlinewidth{1.003750pt}%
\definecolor{currentstroke}{rgb}{0.800000,0.800000,0.800000}%
\pgfsetstrokecolor{currentstroke}%
\pgfsetstrokeopacity{0.800000}%
\pgfsetdash{}{0pt}%
\pgfpathmoveto{\pgfqpoint{1.552717in}{1.517728in}}%
\pgfpathlineto{\pgfqpoint{2.321497in}{1.517728in}}%
\pgfpathquadraticcurveto{\pgfqpoint{2.343719in}{1.517728in}}{\pgfqpoint{2.343719in}{1.539950in}}%
\pgfpathlineto{\pgfqpoint{2.343719in}{1.711261in}}%
\pgfpathquadraticcurveto{\pgfqpoint{2.343719in}{1.733483in}}{\pgfqpoint{2.321497in}{1.733483in}}%
\pgfpathlineto{\pgfqpoint{1.552717in}{1.733483in}}%
\pgfpathquadraticcurveto{\pgfqpoint{1.530494in}{1.733483in}}{\pgfqpoint{1.530494in}{1.711261in}}%
\pgfpathlineto{\pgfqpoint{1.530494in}{1.539950in}}%
\pgfpathquadraticcurveto{\pgfqpoint{1.530494in}{1.517728in}}{\pgfqpoint{1.552717in}{1.517728in}}%
\pgfpathlineto{\pgfqpoint{1.552717in}{1.517728in}}%
\pgfpathclose%
\pgfusepath{stroke,fill}%
\end{pgfscope}%
\begin{pgfscope}%
\pgfsetbuttcap%
\pgfsetroundjoin%
\pgfsetlinewidth{1.505625pt}%
\definecolor{currentstroke}{rgb}{0.007843,0.619608,0.450980}%
\pgfsetstrokecolor{currentstroke}%
\pgfsetdash{{5.550000pt}{2.400000pt}}{0.000000pt}%
\pgfpathmoveto{\pgfqpoint{1.574939in}{1.628067in}}%
\pgfpathlineto{\pgfqpoint{1.686050in}{1.628067in}}%
\pgfpathlineto{\pgfqpoint{1.797161in}{1.628067in}}%
\pgfusepath{stroke}%
\end{pgfscope}%
\begin{pgfscope}%
\definecolor{textcolor}{rgb}{0.000000,0.000000,0.000000}%
\pgfsetstrokecolor{textcolor}%
\pgfsetfillcolor{textcolor}%
\pgftext[x=1.886050in,y=1.589178in,left,base]{\color{textcolor}\rmfamily\fontsize{8.000000}{9.600000}\selectfont \(\displaystyle h_{-1}f^{-1}\)}%
\end{pgfscope}%
\end{pgfpicture}%
\makeatother%
\endgroup%

        } % scalebox
        \caption{Power spectral density}
        \label{fig:flicker_noise_psd}
    \end{subfigure}
    \begin{subfigure}{0.32\linewidth}
        \centering
        \scalebox{0.75}{%
            %% Creator: Matplotlib, PGF backend
%%
%% To include the figure in your LaTeX document, write
%%   \input{<filename>.pgf}
%%
%% Make sure the required packages are loaded in your preamble
%%   \usepackage{pgf}
%%
%% Also ensure that all the required font packages are loaded; for instance,
%% the lmodern package is sometimes necessary when using math font.
%%   \usepackage{lmodern}
%%
%% Figures using additional raster images can only be included by \input if
%% they are in the same directory as the main LaTeX file. For loading figures
%% from other directories you can use the `import` package
%%   \usepackage{import}
%%
%% and then include the figures with
%%   \import{<path to file>}{<filename>.pgf}
%%
%% Matplotlib used the following preamble
%%   \usepackage{siunitx}
%%   \usepackage{fontspec}
%%   \makeatletter\@ifpackageloaded{underscore}{}{\usepackage[strings]{underscore}}\makeatother
%%
\begingroup%
\makeatletter%
\begin{pgfpicture}%
\pgfpathrectangle{\pgfpointorigin}{\pgfqpoint{2.440000in}{1.830000in}}%
\pgfusepath{use as bounding box, clip}%
\begin{pgfscope}%
\pgfsetbuttcap%
\pgfsetmiterjoin%
\definecolor{currentfill}{rgb}{1.000000,1.000000,1.000000}%
\pgfsetfillcolor{currentfill}%
\pgfsetlinewidth{0.000000pt}%
\definecolor{currentstroke}{rgb}{1.000000,1.000000,1.000000}%
\pgfsetstrokecolor{currentstroke}%
\pgfsetdash{}{0pt}%
\pgfpathmoveto{\pgfqpoint{0.000000in}{0.000000in}}%
\pgfpathlineto{\pgfqpoint{2.440000in}{0.000000in}}%
\pgfpathlineto{\pgfqpoint{2.440000in}{1.830000in}}%
\pgfpathlineto{\pgfqpoint{0.000000in}{1.830000in}}%
\pgfpathlineto{\pgfqpoint{0.000000in}{0.000000in}}%
\pgfpathclose%
\pgfusepath{fill}%
\end{pgfscope}%
\begin{pgfscope}%
\pgfsetbuttcap%
\pgfsetmiterjoin%
\definecolor{currentfill}{rgb}{1.000000,1.000000,1.000000}%
\pgfsetfillcolor{currentfill}%
\pgfsetlinewidth{0.000000pt}%
\definecolor{currentstroke}{rgb}{0.000000,0.000000,0.000000}%
\pgfsetstrokecolor{currentstroke}%
\pgfsetstrokeopacity{0.000000}%
\pgfsetdash{}{0pt}%
\pgfpathmoveto{\pgfqpoint{0.589510in}{0.417642in}}%
\pgfpathlineto{\pgfqpoint{2.398330in}{0.417642in}}%
\pgfpathlineto{\pgfqpoint{2.398330in}{1.788330in}}%
\pgfpathlineto{\pgfqpoint{0.589510in}{1.788330in}}%
\pgfpathlineto{\pgfqpoint{0.589510in}{0.417642in}}%
\pgfpathclose%
\pgfusepath{fill}%
\end{pgfscope}%
\begin{pgfscope}%
\pgfpathrectangle{\pgfqpoint{0.589510in}{0.417642in}}{\pgfqpoint{1.808820in}{1.370688in}}%
\pgfusepath{clip}%
\pgfsetrectcap%
\pgfsetroundjoin%
\pgfsetlinewidth{0.803000pt}%
\definecolor{currentstroke}{rgb}{0.450000,0.450000,0.450000}%
\pgfsetstrokecolor{currentstroke}%
\pgfsetdash{}{0pt}%
\pgfpathmoveto{\pgfqpoint{0.671729in}{0.417642in}}%
\pgfpathlineto{\pgfqpoint{0.671729in}{1.788330in}}%
\pgfusepath{stroke}%
\end{pgfscope}%
\begin{pgfscope}%
\pgfsetbuttcap%
\pgfsetroundjoin%
\definecolor{currentfill}{rgb}{0.000000,0.000000,0.000000}%
\pgfsetfillcolor{currentfill}%
\pgfsetlinewidth{0.803000pt}%
\definecolor{currentstroke}{rgb}{0.000000,0.000000,0.000000}%
\pgfsetstrokecolor{currentstroke}%
\pgfsetdash{}{0pt}%
\pgfsys@defobject{currentmarker}{\pgfqpoint{0.000000in}{-0.048611in}}{\pgfqpoint{0.000000in}{0.000000in}}{%
\pgfpathmoveto{\pgfqpoint{0.000000in}{0.000000in}}%
\pgfpathlineto{\pgfqpoint{0.000000in}{-0.048611in}}%
\pgfusepath{stroke,fill}%
}%
\begin{pgfscope}%
\pgfsys@transformshift{0.671729in}{0.417642in}%
\pgfsys@useobject{currentmarker}{}%
\end{pgfscope}%
\end{pgfscope}%
\begin{pgfscope}%
\definecolor{textcolor}{rgb}{0.000000,0.000000,0.000000}%
\pgfsetstrokecolor{textcolor}%
\pgfsetfillcolor{textcolor}%
\pgftext[x=0.671729in,y=0.320420in,,top]{\color{textcolor}\rmfamily\fontsize{8.000000}{9.600000}\selectfont \(\displaystyle {10^{0}}\)}%
\end{pgfscope}%
\begin{pgfscope}%
\pgfpathrectangle{\pgfqpoint{0.589510in}{0.417642in}}{\pgfqpoint{1.808820in}{1.370688in}}%
\pgfusepath{clip}%
\pgfsetrectcap%
\pgfsetroundjoin%
\pgfsetlinewidth{0.803000pt}%
\definecolor{currentstroke}{rgb}{0.450000,0.450000,0.450000}%
\pgfsetstrokecolor{currentstroke}%
\pgfsetdash{}{0pt}%
\pgfpathmoveto{\pgfqpoint{1.128240in}{0.417642in}}%
\pgfpathlineto{\pgfqpoint{1.128240in}{1.788330in}}%
\pgfusepath{stroke}%
\end{pgfscope}%
\begin{pgfscope}%
\pgfsetbuttcap%
\pgfsetroundjoin%
\definecolor{currentfill}{rgb}{0.000000,0.000000,0.000000}%
\pgfsetfillcolor{currentfill}%
\pgfsetlinewidth{0.803000pt}%
\definecolor{currentstroke}{rgb}{0.000000,0.000000,0.000000}%
\pgfsetstrokecolor{currentstroke}%
\pgfsetdash{}{0pt}%
\pgfsys@defobject{currentmarker}{\pgfqpoint{0.000000in}{-0.048611in}}{\pgfqpoint{0.000000in}{0.000000in}}{%
\pgfpathmoveto{\pgfqpoint{0.000000in}{0.000000in}}%
\pgfpathlineto{\pgfqpoint{0.000000in}{-0.048611in}}%
\pgfusepath{stroke,fill}%
}%
\begin{pgfscope}%
\pgfsys@transformshift{1.128240in}{0.417642in}%
\pgfsys@useobject{currentmarker}{}%
\end{pgfscope}%
\end{pgfscope}%
\begin{pgfscope}%
\definecolor{textcolor}{rgb}{0.000000,0.000000,0.000000}%
\pgfsetstrokecolor{textcolor}%
\pgfsetfillcolor{textcolor}%
\pgftext[x=1.128240in,y=0.320420in,,top]{\color{textcolor}\rmfamily\fontsize{8.000000}{9.600000}\selectfont \(\displaystyle {10^{1}}\)}%
\end{pgfscope}%
\begin{pgfscope}%
\pgfpathrectangle{\pgfqpoint{0.589510in}{0.417642in}}{\pgfqpoint{1.808820in}{1.370688in}}%
\pgfusepath{clip}%
\pgfsetrectcap%
\pgfsetroundjoin%
\pgfsetlinewidth{0.803000pt}%
\definecolor{currentstroke}{rgb}{0.450000,0.450000,0.450000}%
\pgfsetstrokecolor{currentstroke}%
\pgfsetdash{}{0pt}%
\pgfpathmoveto{\pgfqpoint{1.584752in}{0.417642in}}%
\pgfpathlineto{\pgfqpoint{1.584752in}{1.788330in}}%
\pgfusepath{stroke}%
\end{pgfscope}%
\begin{pgfscope}%
\pgfsetbuttcap%
\pgfsetroundjoin%
\definecolor{currentfill}{rgb}{0.000000,0.000000,0.000000}%
\pgfsetfillcolor{currentfill}%
\pgfsetlinewidth{0.803000pt}%
\definecolor{currentstroke}{rgb}{0.000000,0.000000,0.000000}%
\pgfsetstrokecolor{currentstroke}%
\pgfsetdash{}{0pt}%
\pgfsys@defobject{currentmarker}{\pgfqpoint{0.000000in}{-0.048611in}}{\pgfqpoint{0.000000in}{0.000000in}}{%
\pgfpathmoveto{\pgfqpoint{0.000000in}{0.000000in}}%
\pgfpathlineto{\pgfqpoint{0.000000in}{-0.048611in}}%
\pgfusepath{stroke,fill}%
}%
\begin{pgfscope}%
\pgfsys@transformshift{1.584752in}{0.417642in}%
\pgfsys@useobject{currentmarker}{}%
\end{pgfscope}%
\end{pgfscope}%
\begin{pgfscope}%
\definecolor{textcolor}{rgb}{0.000000,0.000000,0.000000}%
\pgfsetstrokecolor{textcolor}%
\pgfsetfillcolor{textcolor}%
\pgftext[x=1.584752in,y=0.320420in,,top]{\color{textcolor}\rmfamily\fontsize{8.000000}{9.600000}\selectfont \(\displaystyle {10^{2}}\)}%
\end{pgfscope}%
\begin{pgfscope}%
\pgfpathrectangle{\pgfqpoint{0.589510in}{0.417642in}}{\pgfqpoint{1.808820in}{1.370688in}}%
\pgfusepath{clip}%
\pgfsetrectcap%
\pgfsetroundjoin%
\pgfsetlinewidth{0.803000pt}%
\definecolor{currentstroke}{rgb}{0.450000,0.450000,0.450000}%
\pgfsetstrokecolor{currentstroke}%
\pgfsetdash{}{0pt}%
\pgfpathmoveto{\pgfqpoint{2.041264in}{0.417642in}}%
\pgfpathlineto{\pgfqpoint{2.041264in}{1.788330in}}%
\pgfusepath{stroke}%
\end{pgfscope}%
\begin{pgfscope}%
\pgfsetbuttcap%
\pgfsetroundjoin%
\definecolor{currentfill}{rgb}{0.000000,0.000000,0.000000}%
\pgfsetfillcolor{currentfill}%
\pgfsetlinewidth{0.803000pt}%
\definecolor{currentstroke}{rgb}{0.000000,0.000000,0.000000}%
\pgfsetstrokecolor{currentstroke}%
\pgfsetdash{}{0pt}%
\pgfsys@defobject{currentmarker}{\pgfqpoint{0.000000in}{-0.048611in}}{\pgfqpoint{0.000000in}{0.000000in}}{%
\pgfpathmoveto{\pgfqpoint{0.000000in}{0.000000in}}%
\pgfpathlineto{\pgfqpoint{0.000000in}{-0.048611in}}%
\pgfusepath{stroke,fill}%
}%
\begin{pgfscope}%
\pgfsys@transformshift{2.041264in}{0.417642in}%
\pgfsys@useobject{currentmarker}{}%
\end{pgfscope}%
\end{pgfscope}%
\begin{pgfscope}%
\definecolor{textcolor}{rgb}{0.000000,0.000000,0.000000}%
\pgfsetstrokecolor{textcolor}%
\pgfsetfillcolor{textcolor}%
\pgftext[x=2.041264in,y=0.320420in,,top]{\color{textcolor}\rmfamily\fontsize{8.000000}{9.600000}\selectfont \(\displaystyle {10^{3}}\)}%
\end{pgfscope}%
\begin{pgfscope}%
\pgfpathrectangle{\pgfqpoint{0.589510in}{0.417642in}}{\pgfqpoint{1.808820in}{1.370688in}}%
\pgfusepath{clip}%
\pgfsetrectcap%
\pgfsetroundjoin%
\pgfsetlinewidth{0.803000pt}%
\definecolor{currentstroke}{rgb}{0.850000,0.850000,0.850000}%
\pgfsetstrokecolor{currentstroke}%
\pgfsetdash{}{0pt}%
\pgfpathmoveto{\pgfqpoint{0.601014in}{0.417642in}}%
\pgfpathlineto{\pgfqpoint{0.601014in}{1.788330in}}%
\pgfusepath{stroke}%
\end{pgfscope}%
\begin{pgfscope}%
\pgfsetbuttcap%
\pgfsetroundjoin%
\definecolor{currentfill}{rgb}{0.000000,0.000000,0.000000}%
\pgfsetfillcolor{currentfill}%
\pgfsetlinewidth{0.602250pt}%
\definecolor{currentstroke}{rgb}{0.000000,0.000000,0.000000}%
\pgfsetstrokecolor{currentstroke}%
\pgfsetdash{}{0pt}%
\pgfsys@defobject{currentmarker}{\pgfqpoint{0.000000in}{-0.027778in}}{\pgfqpoint{0.000000in}{0.000000in}}{%
\pgfpathmoveto{\pgfqpoint{0.000000in}{0.000000in}}%
\pgfpathlineto{\pgfqpoint{0.000000in}{-0.027778in}}%
\pgfusepath{stroke,fill}%
}%
\begin{pgfscope}%
\pgfsys@transformshift{0.601014in}{0.417642in}%
\pgfsys@useobject{currentmarker}{}%
\end{pgfscope}%
\end{pgfscope}%
\begin{pgfscope}%
\pgfpathrectangle{\pgfqpoint{0.589510in}{0.417642in}}{\pgfqpoint{1.808820in}{1.370688in}}%
\pgfusepath{clip}%
\pgfsetrectcap%
\pgfsetroundjoin%
\pgfsetlinewidth{0.803000pt}%
\definecolor{currentstroke}{rgb}{0.850000,0.850000,0.850000}%
\pgfsetstrokecolor{currentstroke}%
\pgfsetdash{}{0pt}%
\pgfpathmoveto{\pgfqpoint{0.627488in}{0.417642in}}%
\pgfpathlineto{\pgfqpoint{0.627488in}{1.788330in}}%
\pgfusepath{stroke}%
\end{pgfscope}%
\begin{pgfscope}%
\pgfsetbuttcap%
\pgfsetroundjoin%
\definecolor{currentfill}{rgb}{0.000000,0.000000,0.000000}%
\pgfsetfillcolor{currentfill}%
\pgfsetlinewidth{0.602250pt}%
\definecolor{currentstroke}{rgb}{0.000000,0.000000,0.000000}%
\pgfsetstrokecolor{currentstroke}%
\pgfsetdash{}{0pt}%
\pgfsys@defobject{currentmarker}{\pgfqpoint{0.000000in}{-0.027778in}}{\pgfqpoint{0.000000in}{0.000000in}}{%
\pgfpathmoveto{\pgfqpoint{0.000000in}{0.000000in}}%
\pgfpathlineto{\pgfqpoint{0.000000in}{-0.027778in}}%
\pgfusepath{stroke,fill}%
}%
\begin{pgfscope}%
\pgfsys@transformshift{0.627488in}{0.417642in}%
\pgfsys@useobject{currentmarker}{}%
\end{pgfscope}%
\end{pgfscope}%
\begin{pgfscope}%
\pgfpathrectangle{\pgfqpoint{0.589510in}{0.417642in}}{\pgfqpoint{1.808820in}{1.370688in}}%
\pgfusepath{clip}%
\pgfsetrectcap%
\pgfsetroundjoin%
\pgfsetlinewidth{0.803000pt}%
\definecolor{currentstroke}{rgb}{0.850000,0.850000,0.850000}%
\pgfsetstrokecolor{currentstroke}%
\pgfsetdash{}{0pt}%
\pgfpathmoveto{\pgfqpoint{0.650840in}{0.417642in}}%
\pgfpathlineto{\pgfqpoint{0.650840in}{1.788330in}}%
\pgfusepath{stroke}%
\end{pgfscope}%
\begin{pgfscope}%
\pgfsetbuttcap%
\pgfsetroundjoin%
\definecolor{currentfill}{rgb}{0.000000,0.000000,0.000000}%
\pgfsetfillcolor{currentfill}%
\pgfsetlinewidth{0.602250pt}%
\definecolor{currentstroke}{rgb}{0.000000,0.000000,0.000000}%
\pgfsetstrokecolor{currentstroke}%
\pgfsetdash{}{0pt}%
\pgfsys@defobject{currentmarker}{\pgfqpoint{0.000000in}{-0.027778in}}{\pgfqpoint{0.000000in}{0.000000in}}{%
\pgfpathmoveto{\pgfqpoint{0.000000in}{0.000000in}}%
\pgfpathlineto{\pgfqpoint{0.000000in}{-0.027778in}}%
\pgfusepath{stroke,fill}%
}%
\begin{pgfscope}%
\pgfsys@transformshift{0.650840in}{0.417642in}%
\pgfsys@useobject{currentmarker}{}%
\end{pgfscope}%
\end{pgfscope}%
\begin{pgfscope}%
\pgfpathrectangle{\pgfqpoint{0.589510in}{0.417642in}}{\pgfqpoint{1.808820in}{1.370688in}}%
\pgfusepath{clip}%
\pgfsetrectcap%
\pgfsetroundjoin%
\pgfsetlinewidth{0.803000pt}%
\definecolor{currentstroke}{rgb}{0.850000,0.850000,0.850000}%
\pgfsetstrokecolor{currentstroke}%
\pgfsetdash{}{0pt}%
\pgfpathmoveto{\pgfqpoint{0.809153in}{0.417642in}}%
\pgfpathlineto{\pgfqpoint{0.809153in}{1.788330in}}%
\pgfusepath{stroke}%
\end{pgfscope}%
\begin{pgfscope}%
\pgfsetbuttcap%
\pgfsetroundjoin%
\definecolor{currentfill}{rgb}{0.000000,0.000000,0.000000}%
\pgfsetfillcolor{currentfill}%
\pgfsetlinewidth{0.602250pt}%
\definecolor{currentstroke}{rgb}{0.000000,0.000000,0.000000}%
\pgfsetstrokecolor{currentstroke}%
\pgfsetdash{}{0pt}%
\pgfsys@defobject{currentmarker}{\pgfqpoint{0.000000in}{-0.027778in}}{\pgfqpoint{0.000000in}{0.000000in}}{%
\pgfpathmoveto{\pgfqpoint{0.000000in}{0.000000in}}%
\pgfpathlineto{\pgfqpoint{0.000000in}{-0.027778in}}%
\pgfusepath{stroke,fill}%
}%
\begin{pgfscope}%
\pgfsys@transformshift{0.809153in}{0.417642in}%
\pgfsys@useobject{currentmarker}{}%
\end{pgfscope}%
\end{pgfscope}%
\begin{pgfscope}%
\pgfpathrectangle{\pgfqpoint{0.589510in}{0.417642in}}{\pgfqpoint{1.808820in}{1.370688in}}%
\pgfusepath{clip}%
\pgfsetrectcap%
\pgfsetroundjoin%
\pgfsetlinewidth{0.803000pt}%
\definecolor{currentstroke}{rgb}{0.850000,0.850000,0.850000}%
\pgfsetstrokecolor{currentstroke}%
\pgfsetdash{}{0pt}%
\pgfpathmoveto{\pgfqpoint{0.889540in}{0.417642in}}%
\pgfpathlineto{\pgfqpoint{0.889540in}{1.788330in}}%
\pgfusepath{stroke}%
\end{pgfscope}%
\begin{pgfscope}%
\pgfsetbuttcap%
\pgfsetroundjoin%
\definecolor{currentfill}{rgb}{0.000000,0.000000,0.000000}%
\pgfsetfillcolor{currentfill}%
\pgfsetlinewidth{0.602250pt}%
\definecolor{currentstroke}{rgb}{0.000000,0.000000,0.000000}%
\pgfsetstrokecolor{currentstroke}%
\pgfsetdash{}{0pt}%
\pgfsys@defobject{currentmarker}{\pgfqpoint{0.000000in}{-0.027778in}}{\pgfqpoint{0.000000in}{0.000000in}}{%
\pgfpathmoveto{\pgfqpoint{0.000000in}{0.000000in}}%
\pgfpathlineto{\pgfqpoint{0.000000in}{-0.027778in}}%
\pgfusepath{stroke,fill}%
}%
\begin{pgfscope}%
\pgfsys@transformshift{0.889540in}{0.417642in}%
\pgfsys@useobject{currentmarker}{}%
\end{pgfscope}%
\end{pgfscope}%
\begin{pgfscope}%
\pgfpathrectangle{\pgfqpoint{0.589510in}{0.417642in}}{\pgfqpoint{1.808820in}{1.370688in}}%
\pgfusepath{clip}%
\pgfsetrectcap%
\pgfsetroundjoin%
\pgfsetlinewidth{0.803000pt}%
\definecolor{currentstroke}{rgb}{0.850000,0.850000,0.850000}%
\pgfsetstrokecolor{currentstroke}%
\pgfsetdash{}{0pt}%
\pgfpathmoveto{\pgfqpoint{0.946576in}{0.417642in}}%
\pgfpathlineto{\pgfqpoint{0.946576in}{1.788330in}}%
\pgfusepath{stroke}%
\end{pgfscope}%
\begin{pgfscope}%
\pgfsetbuttcap%
\pgfsetroundjoin%
\definecolor{currentfill}{rgb}{0.000000,0.000000,0.000000}%
\pgfsetfillcolor{currentfill}%
\pgfsetlinewidth{0.602250pt}%
\definecolor{currentstroke}{rgb}{0.000000,0.000000,0.000000}%
\pgfsetstrokecolor{currentstroke}%
\pgfsetdash{}{0pt}%
\pgfsys@defobject{currentmarker}{\pgfqpoint{0.000000in}{-0.027778in}}{\pgfqpoint{0.000000in}{0.000000in}}{%
\pgfpathmoveto{\pgfqpoint{0.000000in}{0.000000in}}%
\pgfpathlineto{\pgfqpoint{0.000000in}{-0.027778in}}%
\pgfusepath{stroke,fill}%
}%
\begin{pgfscope}%
\pgfsys@transformshift{0.946576in}{0.417642in}%
\pgfsys@useobject{currentmarker}{}%
\end{pgfscope}%
\end{pgfscope}%
\begin{pgfscope}%
\pgfpathrectangle{\pgfqpoint{0.589510in}{0.417642in}}{\pgfqpoint{1.808820in}{1.370688in}}%
\pgfusepath{clip}%
\pgfsetrectcap%
\pgfsetroundjoin%
\pgfsetlinewidth{0.803000pt}%
\definecolor{currentstroke}{rgb}{0.850000,0.850000,0.850000}%
\pgfsetstrokecolor{currentstroke}%
\pgfsetdash{}{0pt}%
\pgfpathmoveto{\pgfqpoint{0.990817in}{0.417642in}}%
\pgfpathlineto{\pgfqpoint{0.990817in}{1.788330in}}%
\pgfusepath{stroke}%
\end{pgfscope}%
\begin{pgfscope}%
\pgfsetbuttcap%
\pgfsetroundjoin%
\definecolor{currentfill}{rgb}{0.000000,0.000000,0.000000}%
\pgfsetfillcolor{currentfill}%
\pgfsetlinewidth{0.602250pt}%
\definecolor{currentstroke}{rgb}{0.000000,0.000000,0.000000}%
\pgfsetstrokecolor{currentstroke}%
\pgfsetdash{}{0pt}%
\pgfsys@defobject{currentmarker}{\pgfqpoint{0.000000in}{-0.027778in}}{\pgfqpoint{0.000000in}{0.000000in}}{%
\pgfpathmoveto{\pgfqpoint{0.000000in}{0.000000in}}%
\pgfpathlineto{\pgfqpoint{0.000000in}{-0.027778in}}%
\pgfusepath{stroke,fill}%
}%
\begin{pgfscope}%
\pgfsys@transformshift{0.990817in}{0.417642in}%
\pgfsys@useobject{currentmarker}{}%
\end{pgfscope}%
\end{pgfscope}%
\begin{pgfscope}%
\pgfpathrectangle{\pgfqpoint{0.589510in}{0.417642in}}{\pgfqpoint{1.808820in}{1.370688in}}%
\pgfusepath{clip}%
\pgfsetrectcap%
\pgfsetroundjoin%
\pgfsetlinewidth{0.803000pt}%
\definecolor{currentstroke}{rgb}{0.850000,0.850000,0.850000}%
\pgfsetstrokecolor{currentstroke}%
\pgfsetdash{}{0pt}%
\pgfpathmoveto{\pgfqpoint{1.026964in}{0.417642in}}%
\pgfpathlineto{\pgfqpoint{1.026964in}{1.788330in}}%
\pgfusepath{stroke}%
\end{pgfscope}%
\begin{pgfscope}%
\pgfsetbuttcap%
\pgfsetroundjoin%
\definecolor{currentfill}{rgb}{0.000000,0.000000,0.000000}%
\pgfsetfillcolor{currentfill}%
\pgfsetlinewidth{0.602250pt}%
\definecolor{currentstroke}{rgb}{0.000000,0.000000,0.000000}%
\pgfsetstrokecolor{currentstroke}%
\pgfsetdash{}{0pt}%
\pgfsys@defobject{currentmarker}{\pgfqpoint{0.000000in}{-0.027778in}}{\pgfqpoint{0.000000in}{0.000000in}}{%
\pgfpathmoveto{\pgfqpoint{0.000000in}{0.000000in}}%
\pgfpathlineto{\pgfqpoint{0.000000in}{-0.027778in}}%
\pgfusepath{stroke,fill}%
}%
\begin{pgfscope}%
\pgfsys@transformshift{1.026964in}{0.417642in}%
\pgfsys@useobject{currentmarker}{}%
\end{pgfscope}%
\end{pgfscope}%
\begin{pgfscope}%
\pgfpathrectangle{\pgfqpoint{0.589510in}{0.417642in}}{\pgfqpoint{1.808820in}{1.370688in}}%
\pgfusepath{clip}%
\pgfsetrectcap%
\pgfsetroundjoin%
\pgfsetlinewidth{0.803000pt}%
\definecolor{currentstroke}{rgb}{0.850000,0.850000,0.850000}%
\pgfsetstrokecolor{currentstroke}%
\pgfsetdash{}{0pt}%
\pgfpathmoveto{\pgfqpoint{1.057526in}{0.417642in}}%
\pgfpathlineto{\pgfqpoint{1.057526in}{1.788330in}}%
\pgfusepath{stroke}%
\end{pgfscope}%
\begin{pgfscope}%
\pgfsetbuttcap%
\pgfsetroundjoin%
\definecolor{currentfill}{rgb}{0.000000,0.000000,0.000000}%
\pgfsetfillcolor{currentfill}%
\pgfsetlinewidth{0.602250pt}%
\definecolor{currentstroke}{rgb}{0.000000,0.000000,0.000000}%
\pgfsetstrokecolor{currentstroke}%
\pgfsetdash{}{0pt}%
\pgfsys@defobject{currentmarker}{\pgfqpoint{0.000000in}{-0.027778in}}{\pgfqpoint{0.000000in}{0.000000in}}{%
\pgfpathmoveto{\pgfqpoint{0.000000in}{0.000000in}}%
\pgfpathlineto{\pgfqpoint{0.000000in}{-0.027778in}}%
\pgfusepath{stroke,fill}%
}%
\begin{pgfscope}%
\pgfsys@transformshift{1.057526in}{0.417642in}%
\pgfsys@useobject{currentmarker}{}%
\end{pgfscope}%
\end{pgfscope}%
\begin{pgfscope}%
\pgfpathrectangle{\pgfqpoint{0.589510in}{0.417642in}}{\pgfqpoint{1.808820in}{1.370688in}}%
\pgfusepath{clip}%
\pgfsetrectcap%
\pgfsetroundjoin%
\pgfsetlinewidth{0.803000pt}%
\definecolor{currentstroke}{rgb}{0.850000,0.850000,0.850000}%
\pgfsetstrokecolor{currentstroke}%
\pgfsetdash{}{0pt}%
\pgfpathmoveto{\pgfqpoint{1.084000in}{0.417642in}}%
\pgfpathlineto{\pgfqpoint{1.084000in}{1.788330in}}%
\pgfusepath{stroke}%
\end{pgfscope}%
\begin{pgfscope}%
\pgfsetbuttcap%
\pgfsetroundjoin%
\definecolor{currentfill}{rgb}{0.000000,0.000000,0.000000}%
\pgfsetfillcolor{currentfill}%
\pgfsetlinewidth{0.602250pt}%
\definecolor{currentstroke}{rgb}{0.000000,0.000000,0.000000}%
\pgfsetstrokecolor{currentstroke}%
\pgfsetdash{}{0pt}%
\pgfsys@defobject{currentmarker}{\pgfqpoint{0.000000in}{-0.027778in}}{\pgfqpoint{0.000000in}{0.000000in}}{%
\pgfpathmoveto{\pgfqpoint{0.000000in}{0.000000in}}%
\pgfpathlineto{\pgfqpoint{0.000000in}{-0.027778in}}%
\pgfusepath{stroke,fill}%
}%
\begin{pgfscope}%
\pgfsys@transformshift{1.084000in}{0.417642in}%
\pgfsys@useobject{currentmarker}{}%
\end{pgfscope}%
\end{pgfscope}%
\begin{pgfscope}%
\pgfpathrectangle{\pgfqpoint{0.589510in}{0.417642in}}{\pgfqpoint{1.808820in}{1.370688in}}%
\pgfusepath{clip}%
\pgfsetrectcap%
\pgfsetroundjoin%
\pgfsetlinewidth{0.803000pt}%
\definecolor{currentstroke}{rgb}{0.850000,0.850000,0.850000}%
\pgfsetstrokecolor{currentstroke}%
\pgfsetdash{}{0pt}%
\pgfpathmoveto{\pgfqpoint{1.107352in}{0.417642in}}%
\pgfpathlineto{\pgfqpoint{1.107352in}{1.788330in}}%
\pgfusepath{stroke}%
\end{pgfscope}%
\begin{pgfscope}%
\pgfsetbuttcap%
\pgfsetroundjoin%
\definecolor{currentfill}{rgb}{0.000000,0.000000,0.000000}%
\pgfsetfillcolor{currentfill}%
\pgfsetlinewidth{0.602250pt}%
\definecolor{currentstroke}{rgb}{0.000000,0.000000,0.000000}%
\pgfsetstrokecolor{currentstroke}%
\pgfsetdash{}{0pt}%
\pgfsys@defobject{currentmarker}{\pgfqpoint{0.000000in}{-0.027778in}}{\pgfqpoint{0.000000in}{0.000000in}}{%
\pgfpathmoveto{\pgfqpoint{0.000000in}{0.000000in}}%
\pgfpathlineto{\pgfqpoint{0.000000in}{-0.027778in}}%
\pgfusepath{stroke,fill}%
}%
\begin{pgfscope}%
\pgfsys@transformshift{1.107352in}{0.417642in}%
\pgfsys@useobject{currentmarker}{}%
\end{pgfscope}%
\end{pgfscope}%
\begin{pgfscope}%
\pgfpathrectangle{\pgfqpoint{0.589510in}{0.417642in}}{\pgfqpoint{1.808820in}{1.370688in}}%
\pgfusepath{clip}%
\pgfsetrectcap%
\pgfsetroundjoin%
\pgfsetlinewidth{0.803000pt}%
\definecolor{currentstroke}{rgb}{0.850000,0.850000,0.850000}%
\pgfsetstrokecolor{currentstroke}%
\pgfsetdash{}{0pt}%
\pgfpathmoveto{\pgfqpoint{1.265664in}{0.417642in}}%
\pgfpathlineto{\pgfqpoint{1.265664in}{1.788330in}}%
\pgfusepath{stroke}%
\end{pgfscope}%
\begin{pgfscope}%
\pgfsetbuttcap%
\pgfsetroundjoin%
\definecolor{currentfill}{rgb}{0.000000,0.000000,0.000000}%
\pgfsetfillcolor{currentfill}%
\pgfsetlinewidth{0.602250pt}%
\definecolor{currentstroke}{rgb}{0.000000,0.000000,0.000000}%
\pgfsetstrokecolor{currentstroke}%
\pgfsetdash{}{0pt}%
\pgfsys@defobject{currentmarker}{\pgfqpoint{0.000000in}{-0.027778in}}{\pgfqpoint{0.000000in}{0.000000in}}{%
\pgfpathmoveto{\pgfqpoint{0.000000in}{0.000000in}}%
\pgfpathlineto{\pgfqpoint{0.000000in}{-0.027778in}}%
\pgfusepath{stroke,fill}%
}%
\begin{pgfscope}%
\pgfsys@transformshift{1.265664in}{0.417642in}%
\pgfsys@useobject{currentmarker}{}%
\end{pgfscope}%
\end{pgfscope}%
\begin{pgfscope}%
\pgfpathrectangle{\pgfqpoint{0.589510in}{0.417642in}}{\pgfqpoint{1.808820in}{1.370688in}}%
\pgfusepath{clip}%
\pgfsetrectcap%
\pgfsetroundjoin%
\pgfsetlinewidth{0.803000pt}%
\definecolor{currentstroke}{rgb}{0.850000,0.850000,0.850000}%
\pgfsetstrokecolor{currentstroke}%
\pgfsetdash{}{0pt}%
\pgfpathmoveto{\pgfqpoint{1.346052in}{0.417642in}}%
\pgfpathlineto{\pgfqpoint{1.346052in}{1.788330in}}%
\pgfusepath{stroke}%
\end{pgfscope}%
\begin{pgfscope}%
\pgfsetbuttcap%
\pgfsetroundjoin%
\definecolor{currentfill}{rgb}{0.000000,0.000000,0.000000}%
\pgfsetfillcolor{currentfill}%
\pgfsetlinewidth{0.602250pt}%
\definecolor{currentstroke}{rgb}{0.000000,0.000000,0.000000}%
\pgfsetstrokecolor{currentstroke}%
\pgfsetdash{}{0pt}%
\pgfsys@defobject{currentmarker}{\pgfqpoint{0.000000in}{-0.027778in}}{\pgfqpoint{0.000000in}{0.000000in}}{%
\pgfpathmoveto{\pgfqpoint{0.000000in}{0.000000in}}%
\pgfpathlineto{\pgfqpoint{0.000000in}{-0.027778in}}%
\pgfusepath{stroke,fill}%
}%
\begin{pgfscope}%
\pgfsys@transformshift{1.346052in}{0.417642in}%
\pgfsys@useobject{currentmarker}{}%
\end{pgfscope}%
\end{pgfscope}%
\begin{pgfscope}%
\pgfpathrectangle{\pgfqpoint{0.589510in}{0.417642in}}{\pgfqpoint{1.808820in}{1.370688in}}%
\pgfusepath{clip}%
\pgfsetrectcap%
\pgfsetroundjoin%
\pgfsetlinewidth{0.803000pt}%
\definecolor{currentstroke}{rgb}{0.850000,0.850000,0.850000}%
\pgfsetstrokecolor{currentstroke}%
\pgfsetdash{}{0pt}%
\pgfpathmoveto{\pgfqpoint{1.403088in}{0.417642in}}%
\pgfpathlineto{\pgfqpoint{1.403088in}{1.788330in}}%
\pgfusepath{stroke}%
\end{pgfscope}%
\begin{pgfscope}%
\pgfsetbuttcap%
\pgfsetroundjoin%
\definecolor{currentfill}{rgb}{0.000000,0.000000,0.000000}%
\pgfsetfillcolor{currentfill}%
\pgfsetlinewidth{0.602250pt}%
\definecolor{currentstroke}{rgb}{0.000000,0.000000,0.000000}%
\pgfsetstrokecolor{currentstroke}%
\pgfsetdash{}{0pt}%
\pgfsys@defobject{currentmarker}{\pgfqpoint{0.000000in}{-0.027778in}}{\pgfqpoint{0.000000in}{0.000000in}}{%
\pgfpathmoveto{\pgfqpoint{0.000000in}{0.000000in}}%
\pgfpathlineto{\pgfqpoint{0.000000in}{-0.027778in}}%
\pgfusepath{stroke,fill}%
}%
\begin{pgfscope}%
\pgfsys@transformshift{1.403088in}{0.417642in}%
\pgfsys@useobject{currentmarker}{}%
\end{pgfscope}%
\end{pgfscope}%
\begin{pgfscope}%
\pgfpathrectangle{\pgfqpoint{0.589510in}{0.417642in}}{\pgfqpoint{1.808820in}{1.370688in}}%
\pgfusepath{clip}%
\pgfsetrectcap%
\pgfsetroundjoin%
\pgfsetlinewidth{0.803000pt}%
\definecolor{currentstroke}{rgb}{0.850000,0.850000,0.850000}%
\pgfsetstrokecolor{currentstroke}%
\pgfsetdash{}{0pt}%
\pgfpathmoveto{\pgfqpoint{1.447328in}{0.417642in}}%
\pgfpathlineto{\pgfqpoint{1.447328in}{1.788330in}}%
\pgfusepath{stroke}%
\end{pgfscope}%
\begin{pgfscope}%
\pgfsetbuttcap%
\pgfsetroundjoin%
\definecolor{currentfill}{rgb}{0.000000,0.000000,0.000000}%
\pgfsetfillcolor{currentfill}%
\pgfsetlinewidth{0.602250pt}%
\definecolor{currentstroke}{rgb}{0.000000,0.000000,0.000000}%
\pgfsetstrokecolor{currentstroke}%
\pgfsetdash{}{0pt}%
\pgfsys@defobject{currentmarker}{\pgfqpoint{0.000000in}{-0.027778in}}{\pgfqpoint{0.000000in}{0.000000in}}{%
\pgfpathmoveto{\pgfqpoint{0.000000in}{0.000000in}}%
\pgfpathlineto{\pgfqpoint{0.000000in}{-0.027778in}}%
\pgfusepath{stroke,fill}%
}%
\begin{pgfscope}%
\pgfsys@transformshift{1.447328in}{0.417642in}%
\pgfsys@useobject{currentmarker}{}%
\end{pgfscope}%
\end{pgfscope}%
\begin{pgfscope}%
\pgfpathrectangle{\pgfqpoint{0.589510in}{0.417642in}}{\pgfqpoint{1.808820in}{1.370688in}}%
\pgfusepath{clip}%
\pgfsetrectcap%
\pgfsetroundjoin%
\pgfsetlinewidth{0.803000pt}%
\definecolor{currentstroke}{rgb}{0.850000,0.850000,0.850000}%
\pgfsetstrokecolor{currentstroke}%
\pgfsetdash{}{0pt}%
\pgfpathmoveto{\pgfqpoint{1.483475in}{0.417642in}}%
\pgfpathlineto{\pgfqpoint{1.483475in}{1.788330in}}%
\pgfusepath{stroke}%
\end{pgfscope}%
\begin{pgfscope}%
\pgfsetbuttcap%
\pgfsetroundjoin%
\definecolor{currentfill}{rgb}{0.000000,0.000000,0.000000}%
\pgfsetfillcolor{currentfill}%
\pgfsetlinewidth{0.602250pt}%
\definecolor{currentstroke}{rgb}{0.000000,0.000000,0.000000}%
\pgfsetstrokecolor{currentstroke}%
\pgfsetdash{}{0pt}%
\pgfsys@defobject{currentmarker}{\pgfqpoint{0.000000in}{-0.027778in}}{\pgfqpoint{0.000000in}{0.000000in}}{%
\pgfpathmoveto{\pgfqpoint{0.000000in}{0.000000in}}%
\pgfpathlineto{\pgfqpoint{0.000000in}{-0.027778in}}%
\pgfusepath{stroke,fill}%
}%
\begin{pgfscope}%
\pgfsys@transformshift{1.483475in}{0.417642in}%
\pgfsys@useobject{currentmarker}{}%
\end{pgfscope}%
\end{pgfscope}%
\begin{pgfscope}%
\pgfpathrectangle{\pgfqpoint{0.589510in}{0.417642in}}{\pgfqpoint{1.808820in}{1.370688in}}%
\pgfusepath{clip}%
\pgfsetrectcap%
\pgfsetroundjoin%
\pgfsetlinewidth{0.803000pt}%
\definecolor{currentstroke}{rgb}{0.850000,0.850000,0.850000}%
\pgfsetstrokecolor{currentstroke}%
\pgfsetdash{}{0pt}%
\pgfpathmoveto{\pgfqpoint{1.514037in}{0.417642in}}%
\pgfpathlineto{\pgfqpoint{1.514037in}{1.788330in}}%
\pgfusepath{stroke}%
\end{pgfscope}%
\begin{pgfscope}%
\pgfsetbuttcap%
\pgfsetroundjoin%
\definecolor{currentfill}{rgb}{0.000000,0.000000,0.000000}%
\pgfsetfillcolor{currentfill}%
\pgfsetlinewidth{0.602250pt}%
\definecolor{currentstroke}{rgb}{0.000000,0.000000,0.000000}%
\pgfsetstrokecolor{currentstroke}%
\pgfsetdash{}{0pt}%
\pgfsys@defobject{currentmarker}{\pgfqpoint{0.000000in}{-0.027778in}}{\pgfqpoint{0.000000in}{0.000000in}}{%
\pgfpathmoveto{\pgfqpoint{0.000000in}{0.000000in}}%
\pgfpathlineto{\pgfqpoint{0.000000in}{-0.027778in}}%
\pgfusepath{stroke,fill}%
}%
\begin{pgfscope}%
\pgfsys@transformshift{1.514037in}{0.417642in}%
\pgfsys@useobject{currentmarker}{}%
\end{pgfscope}%
\end{pgfscope}%
\begin{pgfscope}%
\pgfpathrectangle{\pgfqpoint{0.589510in}{0.417642in}}{\pgfqpoint{1.808820in}{1.370688in}}%
\pgfusepath{clip}%
\pgfsetrectcap%
\pgfsetroundjoin%
\pgfsetlinewidth{0.803000pt}%
\definecolor{currentstroke}{rgb}{0.850000,0.850000,0.850000}%
\pgfsetstrokecolor{currentstroke}%
\pgfsetdash{}{0pt}%
\pgfpathmoveto{\pgfqpoint{1.540511in}{0.417642in}}%
\pgfpathlineto{\pgfqpoint{1.540511in}{1.788330in}}%
\pgfusepath{stroke}%
\end{pgfscope}%
\begin{pgfscope}%
\pgfsetbuttcap%
\pgfsetroundjoin%
\definecolor{currentfill}{rgb}{0.000000,0.000000,0.000000}%
\pgfsetfillcolor{currentfill}%
\pgfsetlinewidth{0.602250pt}%
\definecolor{currentstroke}{rgb}{0.000000,0.000000,0.000000}%
\pgfsetstrokecolor{currentstroke}%
\pgfsetdash{}{0pt}%
\pgfsys@defobject{currentmarker}{\pgfqpoint{0.000000in}{-0.027778in}}{\pgfqpoint{0.000000in}{0.000000in}}{%
\pgfpathmoveto{\pgfqpoint{0.000000in}{0.000000in}}%
\pgfpathlineto{\pgfqpoint{0.000000in}{-0.027778in}}%
\pgfusepath{stroke,fill}%
}%
\begin{pgfscope}%
\pgfsys@transformshift{1.540511in}{0.417642in}%
\pgfsys@useobject{currentmarker}{}%
\end{pgfscope}%
\end{pgfscope}%
\begin{pgfscope}%
\pgfpathrectangle{\pgfqpoint{0.589510in}{0.417642in}}{\pgfqpoint{1.808820in}{1.370688in}}%
\pgfusepath{clip}%
\pgfsetrectcap%
\pgfsetroundjoin%
\pgfsetlinewidth{0.803000pt}%
\definecolor{currentstroke}{rgb}{0.850000,0.850000,0.850000}%
\pgfsetstrokecolor{currentstroke}%
\pgfsetdash{}{0pt}%
\pgfpathmoveto{\pgfqpoint{1.563863in}{0.417642in}}%
\pgfpathlineto{\pgfqpoint{1.563863in}{1.788330in}}%
\pgfusepath{stroke}%
\end{pgfscope}%
\begin{pgfscope}%
\pgfsetbuttcap%
\pgfsetroundjoin%
\definecolor{currentfill}{rgb}{0.000000,0.000000,0.000000}%
\pgfsetfillcolor{currentfill}%
\pgfsetlinewidth{0.602250pt}%
\definecolor{currentstroke}{rgb}{0.000000,0.000000,0.000000}%
\pgfsetstrokecolor{currentstroke}%
\pgfsetdash{}{0pt}%
\pgfsys@defobject{currentmarker}{\pgfqpoint{0.000000in}{-0.027778in}}{\pgfqpoint{0.000000in}{0.000000in}}{%
\pgfpathmoveto{\pgfqpoint{0.000000in}{0.000000in}}%
\pgfpathlineto{\pgfqpoint{0.000000in}{-0.027778in}}%
\pgfusepath{stroke,fill}%
}%
\begin{pgfscope}%
\pgfsys@transformshift{1.563863in}{0.417642in}%
\pgfsys@useobject{currentmarker}{}%
\end{pgfscope}%
\end{pgfscope}%
\begin{pgfscope}%
\pgfpathrectangle{\pgfqpoint{0.589510in}{0.417642in}}{\pgfqpoint{1.808820in}{1.370688in}}%
\pgfusepath{clip}%
\pgfsetrectcap%
\pgfsetroundjoin%
\pgfsetlinewidth{0.803000pt}%
\definecolor{currentstroke}{rgb}{0.850000,0.850000,0.850000}%
\pgfsetstrokecolor{currentstroke}%
\pgfsetdash{}{0pt}%
\pgfpathmoveto{\pgfqpoint{1.722176in}{0.417642in}}%
\pgfpathlineto{\pgfqpoint{1.722176in}{1.788330in}}%
\pgfusepath{stroke}%
\end{pgfscope}%
\begin{pgfscope}%
\pgfsetbuttcap%
\pgfsetroundjoin%
\definecolor{currentfill}{rgb}{0.000000,0.000000,0.000000}%
\pgfsetfillcolor{currentfill}%
\pgfsetlinewidth{0.602250pt}%
\definecolor{currentstroke}{rgb}{0.000000,0.000000,0.000000}%
\pgfsetstrokecolor{currentstroke}%
\pgfsetdash{}{0pt}%
\pgfsys@defobject{currentmarker}{\pgfqpoint{0.000000in}{-0.027778in}}{\pgfqpoint{0.000000in}{0.000000in}}{%
\pgfpathmoveto{\pgfqpoint{0.000000in}{0.000000in}}%
\pgfpathlineto{\pgfqpoint{0.000000in}{-0.027778in}}%
\pgfusepath{stroke,fill}%
}%
\begin{pgfscope}%
\pgfsys@transformshift{1.722176in}{0.417642in}%
\pgfsys@useobject{currentmarker}{}%
\end{pgfscope}%
\end{pgfscope}%
\begin{pgfscope}%
\pgfpathrectangle{\pgfqpoint{0.589510in}{0.417642in}}{\pgfqpoint{1.808820in}{1.370688in}}%
\pgfusepath{clip}%
\pgfsetrectcap%
\pgfsetroundjoin%
\pgfsetlinewidth{0.803000pt}%
\definecolor{currentstroke}{rgb}{0.850000,0.850000,0.850000}%
\pgfsetstrokecolor{currentstroke}%
\pgfsetdash{}{0pt}%
\pgfpathmoveto{\pgfqpoint{1.802563in}{0.417642in}}%
\pgfpathlineto{\pgfqpoint{1.802563in}{1.788330in}}%
\pgfusepath{stroke}%
\end{pgfscope}%
\begin{pgfscope}%
\pgfsetbuttcap%
\pgfsetroundjoin%
\definecolor{currentfill}{rgb}{0.000000,0.000000,0.000000}%
\pgfsetfillcolor{currentfill}%
\pgfsetlinewidth{0.602250pt}%
\definecolor{currentstroke}{rgb}{0.000000,0.000000,0.000000}%
\pgfsetstrokecolor{currentstroke}%
\pgfsetdash{}{0pt}%
\pgfsys@defobject{currentmarker}{\pgfqpoint{0.000000in}{-0.027778in}}{\pgfqpoint{0.000000in}{0.000000in}}{%
\pgfpathmoveto{\pgfqpoint{0.000000in}{0.000000in}}%
\pgfpathlineto{\pgfqpoint{0.000000in}{-0.027778in}}%
\pgfusepath{stroke,fill}%
}%
\begin{pgfscope}%
\pgfsys@transformshift{1.802563in}{0.417642in}%
\pgfsys@useobject{currentmarker}{}%
\end{pgfscope}%
\end{pgfscope}%
\begin{pgfscope}%
\pgfpathrectangle{\pgfqpoint{0.589510in}{0.417642in}}{\pgfqpoint{1.808820in}{1.370688in}}%
\pgfusepath{clip}%
\pgfsetrectcap%
\pgfsetroundjoin%
\pgfsetlinewidth{0.803000pt}%
\definecolor{currentstroke}{rgb}{0.850000,0.850000,0.850000}%
\pgfsetstrokecolor{currentstroke}%
\pgfsetdash{}{0pt}%
\pgfpathmoveto{\pgfqpoint{1.859599in}{0.417642in}}%
\pgfpathlineto{\pgfqpoint{1.859599in}{1.788330in}}%
\pgfusepath{stroke}%
\end{pgfscope}%
\begin{pgfscope}%
\pgfsetbuttcap%
\pgfsetroundjoin%
\definecolor{currentfill}{rgb}{0.000000,0.000000,0.000000}%
\pgfsetfillcolor{currentfill}%
\pgfsetlinewidth{0.602250pt}%
\definecolor{currentstroke}{rgb}{0.000000,0.000000,0.000000}%
\pgfsetstrokecolor{currentstroke}%
\pgfsetdash{}{0pt}%
\pgfsys@defobject{currentmarker}{\pgfqpoint{0.000000in}{-0.027778in}}{\pgfqpoint{0.000000in}{0.000000in}}{%
\pgfpathmoveto{\pgfqpoint{0.000000in}{0.000000in}}%
\pgfpathlineto{\pgfqpoint{0.000000in}{-0.027778in}}%
\pgfusepath{stroke,fill}%
}%
\begin{pgfscope}%
\pgfsys@transformshift{1.859599in}{0.417642in}%
\pgfsys@useobject{currentmarker}{}%
\end{pgfscope}%
\end{pgfscope}%
\begin{pgfscope}%
\pgfpathrectangle{\pgfqpoint{0.589510in}{0.417642in}}{\pgfqpoint{1.808820in}{1.370688in}}%
\pgfusepath{clip}%
\pgfsetrectcap%
\pgfsetroundjoin%
\pgfsetlinewidth{0.803000pt}%
\definecolor{currentstroke}{rgb}{0.850000,0.850000,0.850000}%
\pgfsetstrokecolor{currentstroke}%
\pgfsetdash{}{0pt}%
\pgfpathmoveto{\pgfqpoint{1.903840in}{0.417642in}}%
\pgfpathlineto{\pgfqpoint{1.903840in}{1.788330in}}%
\pgfusepath{stroke}%
\end{pgfscope}%
\begin{pgfscope}%
\pgfsetbuttcap%
\pgfsetroundjoin%
\definecolor{currentfill}{rgb}{0.000000,0.000000,0.000000}%
\pgfsetfillcolor{currentfill}%
\pgfsetlinewidth{0.602250pt}%
\definecolor{currentstroke}{rgb}{0.000000,0.000000,0.000000}%
\pgfsetstrokecolor{currentstroke}%
\pgfsetdash{}{0pt}%
\pgfsys@defobject{currentmarker}{\pgfqpoint{0.000000in}{-0.027778in}}{\pgfqpoint{0.000000in}{0.000000in}}{%
\pgfpathmoveto{\pgfqpoint{0.000000in}{0.000000in}}%
\pgfpathlineto{\pgfqpoint{0.000000in}{-0.027778in}}%
\pgfusepath{stroke,fill}%
}%
\begin{pgfscope}%
\pgfsys@transformshift{1.903840in}{0.417642in}%
\pgfsys@useobject{currentmarker}{}%
\end{pgfscope}%
\end{pgfscope}%
\begin{pgfscope}%
\pgfpathrectangle{\pgfqpoint{0.589510in}{0.417642in}}{\pgfqpoint{1.808820in}{1.370688in}}%
\pgfusepath{clip}%
\pgfsetrectcap%
\pgfsetroundjoin%
\pgfsetlinewidth{0.803000pt}%
\definecolor{currentstroke}{rgb}{0.850000,0.850000,0.850000}%
\pgfsetstrokecolor{currentstroke}%
\pgfsetdash{}{0pt}%
\pgfpathmoveto{\pgfqpoint{1.939987in}{0.417642in}}%
\pgfpathlineto{\pgfqpoint{1.939987in}{1.788330in}}%
\pgfusepath{stroke}%
\end{pgfscope}%
\begin{pgfscope}%
\pgfsetbuttcap%
\pgfsetroundjoin%
\definecolor{currentfill}{rgb}{0.000000,0.000000,0.000000}%
\pgfsetfillcolor{currentfill}%
\pgfsetlinewidth{0.602250pt}%
\definecolor{currentstroke}{rgb}{0.000000,0.000000,0.000000}%
\pgfsetstrokecolor{currentstroke}%
\pgfsetdash{}{0pt}%
\pgfsys@defobject{currentmarker}{\pgfqpoint{0.000000in}{-0.027778in}}{\pgfqpoint{0.000000in}{0.000000in}}{%
\pgfpathmoveto{\pgfqpoint{0.000000in}{0.000000in}}%
\pgfpathlineto{\pgfqpoint{0.000000in}{-0.027778in}}%
\pgfusepath{stroke,fill}%
}%
\begin{pgfscope}%
\pgfsys@transformshift{1.939987in}{0.417642in}%
\pgfsys@useobject{currentmarker}{}%
\end{pgfscope}%
\end{pgfscope}%
\begin{pgfscope}%
\pgfpathrectangle{\pgfqpoint{0.589510in}{0.417642in}}{\pgfqpoint{1.808820in}{1.370688in}}%
\pgfusepath{clip}%
\pgfsetrectcap%
\pgfsetroundjoin%
\pgfsetlinewidth{0.803000pt}%
\definecolor{currentstroke}{rgb}{0.850000,0.850000,0.850000}%
\pgfsetstrokecolor{currentstroke}%
\pgfsetdash{}{0pt}%
\pgfpathmoveto{\pgfqpoint{1.970549in}{0.417642in}}%
\pgfpathlineto{\pgfqpoint{1.970549in}{1.788330in}}%
\pgfusepath{stroke}%
\end{pgfscope}%
\begin{pgfscope}%
\pgfsetbuttcap%
\pgfsetroundjoin%
\definecolor{currentfill}{rgb}{0.000000,0.000000,0.000000}%
\pgfsetfillcolor{currentfill}%
\pgfsetlinewidth{0.602250pt}%
\definecolor{currentstroke}{rgb}{0.000000,0.000000,0.000000}%
\pgfsetstrokecolor{currentstroke}%
\pgfsetdash{}{0pt}%
\pgfsys@defobject{currentmarker}{\pgfqpoint{0.000000in}{-0.027778in}}{\pgfqpoint{0.000000in}{0.000000in}}{%
\pgfpathmoveto{\pgfqpoint{0.000000in}{0.000000in}}%
\pgfpathlineto{\pgfqpoint{0.000000in}{-0.027778in}}%
\pgfusepath{stroke,fill}%
}%
\begin{pgfscope}%
\pgfsys@transformshift{1.970549in}{0.417642in}%
\pgfsys@useobject{currentmarker}{}%
\end{pgfscope}%
\end{pgfscope}%
\begin{pgfscope}%
\pgfpathrectangle{\pgfqpoint{0.589510in}{0.417642in}}{\pgfqpoint{1.808820in}{1.370688in}}%
\pgfusepath{clip}%
\pgfsetrectcap%
\pgfsetroundjoin%
\pgfsetlinewidth{0.803000pt}%
\definecolor{currentstroke}{rgb}{0.850000,0.850000,0.850000}%
\pgfsetstrokecolor{currentstroke}%
\pgfsetdash{}{0pt}%
\pgfpathmoveto{\pgfqpoint{1.997023in}{0.417642in}}%
\pgfpathlineto{\pgfqpoint{1.997023in}{1.788330in}}%
\pgfusepath{stroke}%
\end{pgfscope}%
\begin{pgfscope}%
\pgfsetbuttcap%
\pgfsetroundjoin%
\definecolor{currentfill}{rgb}{0.000000,0.000000,0.000000}%
\pgfsetfillcolor{currentfill}%
\pgfsetlinewidth{0.602250pt}%
\definecolor{currentstroke}{rgb}{0.000000,0.000000,0.000000}%
\pgfsetstrokecolor{currentstroke}%
\pgfsetdash{}{0pt}%
\pgfsys@defobject{currentmarker}{\pgfqpoint{0.000000in}{-0.027778in}}{\pgfqpoint{0.000000in}{0.000000in}}{%
\pgfpathmoveto{\pgfqpoint{0.000000in}{0.000000in}}%
\pgfpathlineto{\pgfqpoint{0.000000in}{-0.027778in}}%
\pgfusepath{stroke,fill}%
}%
\begin{pgfscope}%
\pgfsys@transformshift{1.997023in}{0.417642in}%
\pgfsys@useobject{currentmarker}{}%
\end{pgfscope}%
\end{pgfscope}%
\begin{pgfscope}%
\pgfpathrectangle{\pgfqpoint{0.589510in}{0.417642in}}{\pgfqpoint{1.808820in}{1.370688in}}%
\pgfusepath{clip}%
\pgfsetrectcap%
\pgfsetroundjoin%
\pgfsetlinewidth{0.803000pt}%
\definecolor{currentstroke}{rgb}{0.850000,0.850000,0.850000}%
\pgfsetstrokecolor{currentstroke}%
\pgfsetdash{}{0pt}%
\pgfpathmoveto{\pgfqpoint{2.020375in}{0.417642in}}%
\pgfpathlineto{\pgfqpoint{2.020375in}{1.788330in}}%
\pgfusepath{stroke}%
\end{pgfscope}%
\begin{pgfscope}%
\pgfsetbuttcap%
\pgfsetroundjoin%
\definecolor{currentfill}{rgb}{0.000000,0.000000,0.000000}%
\pgfsetfillcolor{currentfill}%
\pgfsetlinewidth{0.602250pt}%
\definecolor{currentstroke}{rgb}{0.000000,0.000000,0.000000}%
\pgfsetstrokecolor{currentstroke}%
\pgfsetdash{}{0pt}%
\pgfsys@defobject{currentmarker}{\pgfqpoint{0.000000in}{-0.027778in}}{\pgfqpoint{0.000000in}{0.000000in}}{%
\pgfpathmoveto{\pgfqpoint{0.000000in}{0.000000in}}%
\pgfpathlineto{\pgfqpoint{0.000000in}{-0.027778in}}%
\pgfusepath{stroke,fill}%
}%
\begin{pgfscope}%
\pgfsys@transformshift{2.020375in}{0.417642in}%
\pgfsys@useobject{currentmarker}{}%
\end{pgfscope}%
\end{pgfscope}%
\begin{pgfscope}%
\pgfpathrectangle{\pgfqpoint{0.589510in}{0.417642in}}{\pgfqpoint{1.808820in}{1.370688in}}%
\pgfusepath{clip}%
\pgfsetrectcap%
\pgfsetroundjoin%
\pgfsetlinewidth{0.803000pt}%
\definecolor{currentstroke}{rgb}{0.850000,0.850000,0.850000}%
\pgfsetstrokecolor{currentstroke}%
\pgfsetdash{}{0pt}%
\pgfpathmoveto{\pgfqpoint{2.178687in}{0.417642in}}%
\pgfpathlineto{\pgfqpoint{2.178687in}{1.788330in}}%
\pgfusepath{stroke}%
\end{pgfscope}%
\begin{pgfscope}%
\pgfsetbuttcap%
\pgfsetroundjoin%
\definecolor{currentfill}{rgb}{0.000000,0.000000,0.000000}%
\pgfsetfillcolor{currentfill}%
\pgfsetlinewidth{0.602250pt}%
\definecolor{currentstroke}{rgb}{0.000000,0.000000,0.000000}%
\pgfsetstrokecolor{currentstroke}%
\pgfsetdash{}{0pt}%
\pgfsys@defobject{currentmarker}{\pgfqpoint{0.000000in}{-0.027778in}}{\pgfqpoint{0.000000in}{0.000000in}}{%
\pgfpathmoveto{\pgfqpoint{0.000000in}{0.000000in}}%
\pgfpathlineto{\pgfqpoint{0.000000in}{-0.027778in}}%
\pgfusepath{stroke,fill}%
}%
\begin{pgfscope}%
\pgfsys@transformshift{2.178687in}{0.417642in}%
\pgfsys@useobject{currentmarker}{}%
\end{pgfscope}%
\end{pgfscope}%
\begin{pgfscope}%
\pgfpathrectangle{\pgfqpoint{0.589510in}{0.417642in}}{\pgfqpoint{1.808820in}{1.370688in}}%
\pgfusepath{clip}%
\pgfsetrectcap%
\pgfsetroundjoin%
\pgfsetlinewidth{0.803000pt}%
\definecolor{currentstroke}{rgb}{0.850000,0.850000,0.850000}%
\pgfsetstrokecolor{currentstroke}%
\pgfsetdash{}{0pt}%
\pgfpathmoveto{\pgfqpoint{2.259075in}{0.417642in}}%
\pgfpathlineto{\pgfqpoint{2.259075in}{1.788330in}}%
\pgfusepath{stroke}%
\end{pgfscope}%
\begin{pgfscope}%
\pgfsetbuttcap%
\pgfsetroundjoin%
\definecolor{currentfill}{rgb}{0.000000,0.000000,0.000000}%
\pgfsetfillcolor{currentfill}%
\pgfsetlinewidth{0.602250pt}%
\definecolor{currentstroke}{rgb}{0.000000,0.000000,0.000000}%
\pgfsetstrokecolor{currentstroke}%
\pgfsetdash{}{0pt}%
\pgfsys@defobject{currentmarker}{\pgfqpoint{0.000000in}{-0.027778in}}{\pgfqpoint{0.000000in}{0.000000in}}{%
\pgfpathmoveto{\pgfqpoint{0.000000in}{0.000000in}}%
\pgfpathlineto{\pgfqpoint{0.000000in}{-0.027778in}}%
\pgfusepath{stroke,fill}%
}%
\begin{pgfscope}%
\pgfsys@transformshift{2.259075in}{0.417642in}%
\pgfsys@useobject{currentmarker}{}%
\end{pgfscope}%
\end{pgfscope}%
\begin{pgfscope}%
\pgfpathrectangle{\pgfqpoint{0.589510in}{0.417642in}}{\pgfqpoint{1.808820in}{1.370688in}}%
\pgfusepath{clip}%
\pgfsetrectcap%
\pgfsetroundjoin%
\pgfsetlinewidth{0.803000pt}%
\definecolor{currentstroke}{rgb}{0.850000,0.850000,0.850000}%
\pgfsetstrokecolor{currentstroke}%
\pgfsetdash{}{0pt}%
\pgfpathmoveto{\pgfqpoint{2.316111in}{0.417642in}}%
\pgfpathlineto{\pgfqpoint{2.316111in}{1.788330in}}%
\pgfusepath{stroke}%
\end{pgfscope}%
\begin{pgfscope}%
\pgfsetbuttcap%
\pgfsetroundjoin%
\definecolor{currentfill}{rgb}{0.000000,0.000000,0.000000}%
\pgfsetfillcolor{currentfill}%
\pgfsetlinewidth{0.602250pt}%
\definecolor{currentstroke}{rgb}{0.000000,0.000000,0.000000}%
\pgfsetstrokecolor{currentstroke}%
\pgfsetdash{}{0pt}%
\pgfsys@defobject{currentmarker}{\pgfqpoint{0.000000in}{-0.027778in}}{\pgfqpoint{0.000000in}{0.000000in}}{%
\pgfpathmoveto{\pgfqpoint{0.000000in}{0.000000in}}%
\pgfpathlineto{\pgfqpoint{0.000000in}{-0.027778in}}%
\pgfusepath{stroke,fill}%
}%
\begin{pgfscope}%
\pgfsys@transformshift{2.316111in}{0.417642in}%
\pgfsys@useobject{currentmarker}{}%
\end{pgfscope}%
\end{pgfscope}%
\begin{pgfscope}%
\pgfpathrectangle{\pgfqpoint{0.589510in}{0.417642in}}{\pgfqpoint{1.808820in}{1.370688in}}%
\pgfusepath{clip}%
\pgfsetrectcap%
\pgfsetroundjoin%
\pgfsetlinewidth{0.803000pt}%
\definecolor{currentstroke}{rgb}{0.850000,0.850000,0.850000}%
\pgfsetstrokecolor{currentstroke}%
\pgfsetdash{}{0pt}%
\pgfpathmoveto{\pgfqpoint{2.360351in}{0.417642in}}%
\pgfpathlineto{\pgfqpoint{2.360351in}{1.788330in}}%
\pgfusepath{stroke}%
\end{pgfscope}%
\begin{pgfscope}%
\pgfsetbuttcap%
\pgfsetroundjoin%
\definecolor{currentfill}{rgb}{0.000000,0.000000,0.000000}%
\pgfsetfillcolor{currentfill}%
\pgfsetlinewidth{0.602250pt}%
\definecolor{currentstroke}{rgb}{0.000000,0.000000,0.000000}%
\pgfsetstrokecolor{currentstroke}%
\pgfsetdash{}{0pt}%
\pgfsys@defobject{currentmarker}{\pgfqpoint{0.000000in}{-0.027778in}}{\pgfqpoint{0.000000in}{0.000000in}}{%
\pgfpathmoveto{\pgfqpoint{0.000000in}{0.000000in}}%
\pgfpathlineto{\pgfqpoint{0.000000in}{-0.027778in}}%
\pgfusepath{stroke,fill}%
}%
\begin{pgfscope}%
\pgfsys@transformshift{2.360351in}{0.417642in}%
\pgfsys@useobject{currentmarker}{}%
\end{pgfscope}%
\end{pgfscope}%
\begin{pgfscope}%
\pgfpathrectangle{\pgfqpoint{0.589510in}{0.417642in}}{\pgfqpoint{1.808820in}{1.370688in}}%
\pgfusepath{clip}%
\pgfsetrectcap%
\pgfsetroundjoin%
\pgfsetlinewidth{0.803000pt}%
\definecolor{currentstroke}{rgb}{0.850000,0.850000,0.850000}%
\pgfsetstrokecolor{currentstroke}%
\pgfsetdash{}{0pt}%
\pgfpathmoveto{\pgfqpoint{2.396499in}{0.417642in}}%
\pgfpathlineto{\pgfqpoint{2.396499in}{1.788330in}}%
\pgfusepath{stroke}%
\end{pgfscope}%
\begin{pgfscope}%
\pgfsetbuttcap%
\pgfsetroundjoin%
\definecolor{currentfill}{rgb}{0.000000,0.000000,0.000000}%
\pgfsetfillcolor{currentfill}%
\pgfsetlinewidth{0.602250pt}%
\definecolor{currentstroke}{rgb}{0.000000,0.000000,0.000000}%
\pgfsetstrokecolor{currentstroke}%
\pgfsetdash{}{0pt}%
\pgfsys@defobject{currentmarker}{\pgfqpoint{0.000000in}{-0.027778in}}{\pgfqpoint{0.000000in}{0.000000in}}{%
\pgfpathmoveto{\pgfqpoint{0.000000in}{0.000000in}}%
\pgfpathlineto{\pgfqpoint{0.000000in}{-0.027778in}}%
\pgfusepath{stroke,fill}%
}%
\begin{pgfscope}%
\pgfsys@transformshift{2.396499in}{0.417642in}%
\pgfsys@useobject{currentmarker}{}%
\end{pgfscope}%
\end{pgfscope}%
\begin{pgfscope}%
\definecolor{textcolor}{rgb}{0.000000,0.000000,0.000000}%
\pgfsetstrokecolor{textcolor}%
\pgfsetfillcolor{textcolor}%
\pgftext[x=1.493920in,y=0.165003in,,top]{\color{textcolor}\rmfamily\fontsize{10.000000}{12.000000}\selectfont \(\displaystyle \tau\) in \unit{\second}}%
\end{pgfscope}%
\begin{pgfscope}%
\pgfpathrectangle{\pgfqpoint{0.589510in}{0.417642in}}{\pgfqpoint{1.808820in}{1.370688in}}%
\pgfusepath{clip}%
\pgfsetrectcap%
\pgfsetroundjoin%
\pgfsetlinewidth{0.803000pt}%
\definecolor{currentstroke}{rgb}{0.450000,0.450000,0.450000}%
\pgfsetstrokecolor{currentstroke}%
\pgfsetdash{}{0pt}%
\pgfpathmoveto{\pgfqpoint{0.589510in}{0.417642in}}%
\pgfpathlineto{\pgfqpoint{2.398330in}{0.417642in}}%
\pgfusepath{stroke}%
\end{pgfscope}%
\begin{pgfscope}%
\pgfsetbuttcap%
\pgfsetroundjoin%
\definecolor{currentfill}{rgb}{0.000000,0.000000,0.000000}%
\pgfsetfillcolor{currentfill}%
\pgfsetlinewidth{0.803000pt}%
\definecolor{currentstroke}{rgb}{0.000000,0.000000,0.000000}%
\pgfsetstrokecolor{currentstroke}%
\pgfsetdash{}{0pt}%
\pgfsys@defobject{currentmarker}{\pgfqpoint{-0.048611in}{0.000000in}}{\pgfqpoint{-0.000000in}{0.000000in}}{%
\pgfpathmoveto{\pgfqpoint{-0.000000in}{0.000000in}}%
\pgfpathlineto{\pgfqpoint{-0.048611in}{0.000000in}}%
\pgfusepath{stroke,fill}%
}%
\begin{pgfscope}%
\pgfsys@transformshift{0.589510in}{0.417642in}%
\pgfsys@useobject{currentmarker}{}%
\end{pgfscope}%
\end{pgfscope}%
\begin{pgfscope}%
\definecolor{textcolor}{rgb}{0.000000,0.000000,0.000000}%
\pgfsetstrokecolor{textcolor}%
\pgfsetfillcolor{textcolor}%
\pgftext[x=0.236114in, y=0.378489in, left, base]{\color{textcolor}\rmfamily\fontsize{8.000000}{9.600000}\selectfont \(\displaystyle {10^{-2}}\)}%
\end{pgfscope}%
\begin{pgfscope}%
\pgfpathrectangle{\pgfqpoint{0.589510in}{0.417642in}}{\pgfqpoint{1.808820in}{1.370688in}}%
\pgfusepath{clip}%
\pgfsetrectcap%
\pgfsetroundjoin%
\pgfsetlinewidth{0.803000pt}%
\definecolor{currentstroke}{rgb}{0.450000,0.450000,0.450000}%
\pgfsetstrokecolor{currentstroke}%
\pgfsetdash{}{0pt}%
\pgfpathmoveto{\pgfqpoint{0.589510in}{0.826865in}}%
\pgfpathlineto{\pgfqpoint{2.398330in}{0.826865in}}%
\pgfusepath{stroke}%
\end{pgfscope}%
\begin{pgfscope}%
\pgfsetbuttcap%
\pgfsetroundjoin%
\definecolor{currentfill}{rgb}{0.000000,0.000000,0.000000}%
\pgfsetfillcolor{currentfill}%
\pgfsetlinewidth{0.803000pt}%
\definecolor{currentstroke}{rgb}{0.000000,0.000000,0.000000}%
\pgfsetstrokecolor{currentstroke}%
\pgfsetdash{}{0pt}%
\pgfsys@defobject{currentmarker}{\pgfqpoint{-0.048611in}{0.000000in}}{\pgfqpoint{-0.000000in}{0.000000in}}{%
\pgfpathmoveto{\pgfqpoint{-0.000000in}{0.000000in}}%
\pgfpathlineto{\pgfqpoint{-0.048611in}{0.000000in}}%
\pgfusepath{stroke,fill}%
}%
\begin{pgfscope}%
\pgfsys@transformshift{0.589510in}{0.826865in}%
\pgfsys@useobject{currentmarker}{}%
\end{pgfscope}%
\end{pgfscope}%
\begin{pgfscope}%
\definecolor{textcolor}{rgb}{0.000000,0.000000,0.000000}%
\pgfsetstrokecolor{textcolor}%
\pgfsetfillcolor{textcolor}%
\pgftext[x=0.316361in, y=0.787713in, left, base]{\color{textcolor}\rmfamily\fontsize{8.000000}{9.600000}\selectfont \(\displaystyle {10^{0}}\)}%
\end{pgfscope}%
\begin{pgfscope}%
\pgfpathrectangle{\pgfqpoint{0.589510in}{0.417642in}}{\pgfqpoint{1.808820in}{1.370688in}}%
\pgfusepath{clip}%
\pgfsetrectcap%
\pgfsetroundjoin%
\pgfsetlinewidth{0.803000pt}%
\definecolor{currentstroke}{rgb}{0.450000,0.450000,0.450000}%
\pgfsetstrokecolor{currentstroke}%
\pgfsetdash{}{0pt}%
\pgfpathmoveto{\pgfqpoint{0.589510in}{1.236089in}}%
\pgfpathlineto{\pgfqpoint{2.398330in}{1.236089in}}%
\pgfusepath{stroke}%
\end{pgfscope}%
\begin{pgfscope}%
\pgfsetbuttcap%
\pgfsetroundjoin%
\definecolor{currentfill}{rgb}{0.000000,0.000000,0.000000}%
\pgfsetfillcolor{currentfill}%
\pgfsetlinewidth{0.803000pt}%
\definecolor{currentstroke}{rgb}{0.000000,0.000000,0.000000}%
\pgfsetstrokecolor{currentstroke}%
\pgfsetdash{}{0pt}%
\pgfsys@defobject{currentmarker}{\pgfqpoint{-0.048611in}{0.000000in}}{\pgfqpoint{-0.000000in}{0.000000in}}{%
\pgfpathmoveto{\pgfqpoint{-0.000000in}{0.000000in}}%
\pgfpathlineto{\pgfqpoint{-0.048611in}{0.000000in}}%
\pgfusepath{stroke,fill}%
}%
\begin{pgfscope}%
\pgfsys@transformshift{0.589510in}{1.236089in}%
\pgfsys@useobject{currentmarker}{}%
\end{pgfscope}%
\end{pgfscope}%
\begin{pgfscope}%
\definecolor{textcolor}{rgb}{0.000000,0.000000,0.000000}%
\pgfsetstrokecolor{textcolor}%
\pgfsetfillcolor{textcolor}%
\pgftext[x=0.316361in, y=1.196936in, left, base]{\color{textcolor}\rmfamily\fontsize{8.000000}{9.600000}\selectfont \(\displaystyle {10^{2}}\)}%
\end{pgfscope}%
\begin{pgfscope}%
\pgfpathrectangle{\pgfqpoint{0.589510in}{0.417642in}}{\pgfqpoint{1.808820in}{1.370688in}}%
\pgfusepath{clip}%
\pgfsetrectcap%
\pgfsetroundjoin%
\pgfsetlinewidth{0.803000pt}%
\definecolor{currentstroke}{rgb}{0.450000,0.450000,0.450000}%
\pgfsetstrokecolor{currentstroke}%
\pgfsetdash{}{0pt}%
\pgfpathmoveto{\pgfqpoint{0.589510in}{1.645313in}}%
\pgfpathlineto{\pgfqpoint{2.398330in}{1.645313in}}%
\pgfusepath{stroke}%
\end{pgfscope}%
\begin{pgfscope}%
\pgfsetbuttcap%
\pgfsetroundjoin%
\definecolor{currentfill}{rgb}{0.000000,0.000000,0.000000}%
\pgfsetfillcolor{currentfill}%
\pgfsetlinewidth{0.803000pt}%
\definecolor{currentstroke}{rgb}{0.000000,0.000000,0.000000}%
\pgfsetstrokecolor{currentstroke}%
\pgfsetdash{}{0pt}%
\pgfsys@defobject{currentmarker}{\pgfqpoint{-0.048611in}{0.000000in}}{\pgfqpoint{-0.000000in}{0.000000in}}{%
\pgfpathmoveto{\pgfqpoint{-0.000000in}{0.000000in}}%
\pgfpathlineto{\pgfqpoint{-0.048611in}{0.000000in}}%
\pgfusepath{stroke,fill}%
}%
\begin{pgfscope}%
\pgfsys@transformshift{0.589510in}{1.645313in}%
\pgfsys@useobject{currentmarker}{}%
\end{pgfscope}%
\end{pgfscope}%
\begin{pgfscope}%
\definecolor{textcolor}{rgb}{0.000000,0.000000,0.000000}%
\pgfsetstrokecolor{textcolor}%
\pgfsetfillcolor{textcolor}%
\pgftext[x=0.316361in, y=1.606160in, left, base]{\color{textcolor}\rmfamily\fontsize{8.000000}{9.600000}\selectfont \(\displaystyle {10^{4}}\)}%
\end{pgfscope}%
\begin{pgfscope}%
\definecolor{textcolor}{rgb}{0.000000,0.000000,0.000000}%
\pgfsetstrokecolor{textcolor}%
\pgfsetfillcolor{textcolor}%
\pgftext[x=0.180559in,y=1.102986in,,bottom,rotate=90.000000]{\color{textcolor}\rmfamily\fontsize{10.000000}{12.000000}\selectfont ADEV \(\displaystyle \sigma_A(\tau)\)}%
\end{pgfscope}%
\begin{pgfscope}%
\pgfpathrectangle{\pgfqpoint{0.589510in}{0.417642in}}{\pgfqpoint{1.808820in}{1.370688in}}%
\pgfusepath{clip}%
\pgfsetbuttcap%
\pgfsetroundjoin%
\pgfsetlinewidth{1.505625pt}%
\definecolor{currentstroke}{rgb}{0.007843,0.619608,0.450980}%
\pgfsetstrokecolor{currentstroke}%
\pgfsetdash{{5.550000pt}{2.400000pt}}{0.000000pt}%
\pgfpathmoveto{\pgfqpoint{0.671729in}{0.826865in}}%
\pgfpathlineto{\pgfqpoint{0.809153in}{0.826865in}}%
\pgfpathlineto{\pgfqpoint{0.946576in}{0.826865in}}%
\pgfpathlineto{\pgfqpoint{1.128240in}{0.826865in}}%
\pgfpathlineto{\pgfqpoint{1.265664in}{0.826865in}}%
\pgfpathlineto{\pgfqpoint{1.403088in}{0.826865in}}%
\pgfpathlineto{\pgfqpoint{1.584752in}{0.826865in}}%
\pgfpathlineto{\pgfqpoint{1.722176in}{0.826865in}}%
\pgfpathlineto{\pgfqpoint{1.859599in}{0.826865in}}%
\pgfpathlineto{\pgfqpoint{2.041264in}{0.826865in}}%
\pgfpathlineto{\pgfqpoint{2.178687in}{0.826865in}}%
\pgfpathlineto{\pgfqpoint{2.316111in}{0.826865in}}%
\pgfusepath{stroke}%
\end{pgfscope}%
\begin{pgfscope}%
\pgfpathrectangle{\pgfqpoint{0.589510in}{0.417642in}}{\pgfqpoint{1.808820in}{1.370688in}}%
\pgfusepath{clip}%
\pgfsetbuttcap%
\pgfsetroundjoin%
\definecolor{currentfill}{rgb}{0.007843,0.619608,0.450980}%
\pgfsetfillcolor{currentfill}%
\pgfsetlinewidth{1.003750pt}%
\definecolor{currentstroke}{rgb}{0.007843,0.619608,0.450980}%
\pgfsetstrokecolor{currentstroke}%
\pgfsetdash{}{0pt}%
\pgfsys@defobject{currentmarker}{\pgfqpoint{-0.020833in}{-0.020833in}}{\pgfqpoint{0.020833in}{0.020833in}}{%
\pgfpathmoveto{\pgfqpoint{0.000000in}{-0.020833in}}%
\pgfpathcurveto{\pgfqpoint{0.005525in}{-0.020833in}}{\pgfqpoint{0.010825in}{-0.018638in}}{\pgfqpoint{0.014731in}{-0.014731in}}%
\pgfpathcurveto{\pgfqpoint{0.018638in}{-0.010825in}}{\pgfqpoint{0.020833in}{-0.005525in}}{\pgfqpoint{0.020833in}{0.000000in}}%
\pgfpathcurveto{\pgfqpoint{0.020833in}{0.005525in}}{\pgfqpoint{0.018638in}{0.010825in}}{\pgfqpoint{0.014731in}{0.014731in}}%
\pgfpathcurveto{\pgfqpoint{0.010825in}{0.018638in}}{\pgfqpoint{0.005525in}{0.020833in}}{\pgfqpoint{0.000000in}{0.020833in}}%
\pgfpathcurveto{\pgfqpoint{-0.005525in}{0.020833in}}{\pgfqpoint{-0.010825in}{0.018638in}}{\pgfqpoint{-0.014731in}{0.014731in}}%
\pgfpathcurveto{\pgfqpoint{-0.018638in}{0.010825in}}{\pgfqpoint{-0.020833in}{0.005525in}}{\pgfqpoint{-0.020833in}{0.000000in}}%
\pgfpathcurveto{\pgfqpoint{-0.020833in}{-0.005525in}}{\pgfqpoint{-0.018638in}{-0.010825in}}{\pgfqpoint{-0.014731in}{-0.014731in}}%
\pgfpathcurveto{\pgfqpoint{-0.010825in}{-0.018638in}}{\pgfqpoint{-0.005525in}{-0.020833in}}{\pgfqpoint{0.000000in}{-0.020833in}}%
\pgfpathlineto{\pgfqpoint{0.000000in}{-0.020833in}}%
\pgfpathclose%
\pgfusepath{stroke,fill}%
}%
\begin{pgfscope}%
\pgfsys@transformshift{0.671729in}{0.843745in}%
\pgfsys@useobject{currentmarker}{}%
\end{pgfscope}%
\begin{pgfscope}%
\pgfsys@transformshift{0.809153in}{0.833670in}%
\pgfsys@useobject{currentmarker}{}%
\end{pgfscope}%
\begin{pgfscope}%
\pgfsys@transformshift{0.946576in}{0.828583in}%
\pgfsys@useobject{currentmarker}{}%
\end{pgfscope}%
\begin{pgfscope}%
\pgfsys@transformshift{1.128240in}{0.824543in}%
\pgfsys@useobject{currentmarker}{}%
\end{pgfscope}%
\begin{pgfscope}%
\pgfsys@transformshift{1.265664in}{0.822909in}%
\pgfsys@useobject{currentmarker}{}%
\end{pgfscope}%
\begin{pgfscope}%
\pgfsys@transformshift{1.403088in}{0.826945in}%
\pgfsys@useobject{currentmarker}{}%
\end{pgfscope}%
\begin{pgfscope}%
\pgfsys@transformshift{1.584752in}{0.827471in}%
\pgfsys@useobject{currentmarker}{}%
\end{pgfscope}%
\begin{pgfscope}%
\pgfsys@transformshift{1.722176in}{0.820728in}%
\pgfsys@useobject{currentmarker}{}%
\end{pgfscope}%
\begin{pgfscope}%
\pgfsys@transformshift{1.859599in}{0.809931in}%
\pgfsys@useobject{currentmarker}{}%
\end{pgfscope}%
\begin{pgfscope}%
\pgfsys@transformshift{2.041264in}{0.819062in}%
\pgfsys@useobject{currentmarker}{}%
\end{pgfscope}%
\begin{pgfscope}%
\pgfsys@transformshift{2.178687in}{0.851857in}%
\pgfsys@useobject{currentmarker}{}%
\end{pgfscope}%
\begin{pgfscope}%
\pgfsys@transformshift{2.316111in}{0.835620in}%
\pgfsys@useobject{currentmarker}{}%
\end{pgfscope}%
\end{pgfscope}%
\begin{pgfscope}%
\pgfsetrectcap%
\pgfsetmiterjoin%
\pgfsetlinewidth{0.803000pt}%
\definecolor{currentstroke}{rgb}{0.000000,0.000000,0.000000}%
\pgfsetstrokecolor{currentstroke}%
\pgfsetdash{}{0pt}%
\pgfpathmoveto{\pgfqpoint{0.589510in}{0.417642in}}%
\pgfpathlineto{\pgfqpoint{0.589510in}{1.788330in}}%
\pgfusepath{stroke}%
\end{pgfscope}%
\begin{pgfscope}%
\pgfsetrectcap%
\pgfsetmiterjoin%
\pgfsetlinewidth{0.803000pt}%
\definecolor{currentstroke}{rgb}{0.000000,0.000000,0.000000}%
\pgfsetstrokecolor{currentstroke}%
\pgfsetdash{}{0pt}%
\pgfpathmoveto{\pgfqpoint{2.398330in}{0.417642in}}%
\pgfpathlineto{\pgfqpoint{2.398330in}{1.788330in}}%
\pgfusepath{stroke}%
\end{pgfscope}%
\begin{pgfscope}%
\pgfsetrectcap%
\pgfsetmiterjoin%
\pgfsetlinewidth{0.803000pt}%
\definecolor{currentstroke}{rgb}{0.000000,0.000000,0.000000}%
\pgfsetstrokecolor{currentstroke}%
\pgfsetdash{}{0pt}%
\pgfpathmoveto{\pgfqpoint{0.589510in}{0.417642in}}%
\pgfpathlineto{\pgfqpoint{2.398330in}{0.417642in}}%
\pgfusepath{stroke}%
\end{pgfscope}%
\begin{pgfscope}%
\pgfsetrectcap%
\pgfsetmiterjoin%
\pgfsetlinewidth{0.803000pt}%
\definecolor{currentstroke}{rgb}{0.000000,0.000000,0.000000}%
\pgfsetstrokecolor{currentstroke}%
\pgfsetdash{}{0pt}%
\pgfpathmoveto{\pgfqpoint{0.589510in}{1.788330in}}%
\pgfpathlineto{\pgfqpoint{2.398330in}{1.788330in}}%
\pgfusepath{stroke}%
\end{pgfscope}%
\begin{pgfscope}%
\pgfsetbuttcap%
\pgfsetmiterjoin%
\definecolor{currentfill}{rgb}{1.000000,1.000000,1.000000}%
\pgfsetfillcolor{currentfill}%
\pgfsetfillopacity{0.800000}%
\pgfsetlinewidth{1.003750pt}%
\definecolor{currentstroke}{rgb}{0.800000,0.800000,0.800000}%
\pgfsetstrokecolor{currentstroke}%
\pgfsetstrokeopacity{0.800000}%
\pgfsetdash{}{0pt}%
\pgfpathmoveto{\pgfqpoint{1.211763in}{1.471662in}}%
\pgfpathlineto{\pgfqpoint{2.320552in}{1.471662in}}%
\pgfpathquadraticcurveto{\pgfqpoint{2.342774in}{1.471662in}}{\pgfqpoint{2.342774in}{1.493884in}}%
\pgfpathlineto{\pgfqpoint{2.342774in}{1.710552in}}%
\pgfpathquadraticcurveto{\pgfqpoint{2.342774in}{1.732774in}}{\pgfqpoint{2.320552in}{1.732774in}}%
\pgfpathlineto{\pgfqpoint{1.211763in}{1.732774in}}%
\pgfpathquadraticcurveto{\pgfqpoint{1.189541in}{1.732774in}}{\pgfqpoint{1.189541in}{1.710552in}}%
\pgfpathlineto{\pgfqpoint{1.189541in}{1.493884in}}%
\pgfpathquadraticcurveto{\pgfqpoint{1.189541in}{1.471662in}}{\pgfqpoint{1.211763in}{1.471662in}}%
\pgfpathlineto{\pgfqpoint{1.211763in}{1.471662in}}%
\pgfpathclose%
\pgfusepath{stroke,fill}%
\end{pgfscope}%
\begin{pgfscope}%
\pgfsetbuttcap%
\pgfsetroundjoin%
\pgfsetlinewidth{1.505625pt}%
\definecolor{currentstroke}{rgb}{0.007843,0.619608,0.450980}%
\pgfsetstrokecolor{currentstroke}%
\pgfsetdash{{5.550000pt}{2.400000pt}}{0.000000pt}%
\pgfpathmoveto{\pgfqpoint{1.233985in}{1.601717in}}%
\pgfpathlineto{\pgfqpoint{1.345096in}{1.601717in}}%
\pgfpathlineto{\pgfqpoint{1.456207in}{1.601717in}}%
\pgfusepath{stroke}%
\end{pgfscope}%
\begin{pgfscope}%
\definecolor{textcolor}{rgb}{0.000000,0.000000,0.000000}%
\pgfsetstrokecolor{textcolor}%
\pgfsetfillcolor{textcolor}%
\pgftext[x=1.545096in,y=1.562828in,left,base]{\color{textcolor}\rmfamily\fontsize{8.000000}{9.600000}\selectfont \(\displaystyle \propto\sqrt{h_{-1}}\tau^{+0.0}\)}%
\end{pgfscope}%
\end{pgfpicture}%
\makeatother%
\endgroup%

        } % scalebox
        \caption{Allan deviation}
        \label{fig:flicker_noise_adev}
    \end{subfigure}
    \caption{Different representations of flicker noise.}
    \label{fig:flicker_noise_simulated}
\end{figure}

The small wiggles at longer $\tau$ are typical end-of-data errors caused by spectral leakage, because there are insufficient samples to average over \cite{adev_long_tau}. As it was discussed above, the Allan deviation can only be estimated using equation \ref{eqn:adev_estimator} given a limited number of samples.Therefore, at $\frac{\tau}{2}$ there are only $2$ samples left, so there is no averaging possible to improve the estimate of the Allan deviation, which causes the oscillations at low frequencies or large $\tau$.


\clearpage
\minisec{Random Walk}
Random walk noise can be attributed to environmental factors such as temperature \cite{random_walk_fm} and diffusion processes, the latter contributing to the ageing effect seen in semiconductors.
It is a process, where in each time step the change is randomly determined to be either a positve or negative step with equal probability and a fixed step size. Its mean is
\begin{equation}
    \langle y_n \rangle = \langle e_1 + e_2 + \dots e_n \rangle = \underbrace{\langle e_1 \rangle}_{=\,0} + \langle e_2 \rangle + \dots + \langle e_n \rangle = 0 \, ,
\end{equation}
but its variance
\begin{equation}
    \sigma_y^2 = \langle y_n^2 \rangle - \underbrace{\langle y_n \rangle}_{=\,0} = \sigma_{e_1}^2 + \sigma_{e_2}^2 + \dots \sigma_{e_n}^2 = n \sigma_e^2
\end{equation}
goes with $n$ (or $t$). It therefore not a stationary process as can also be seen in figure \ref{fig:random_walk_adev}.

The power spectral density can be calculated \cite{psd_to_adev,noise_generation} to
\begin{equation}
    S(f) = h_{-2} \frac{1}{f^2}
\end{equation}
and the Allan deviation can again be calculated from the spectral density
\begin{align}
    \sigma_A^2(\tau) &= 2 h_{-2} \int_0^\infty \frac{1}{f^2} \frac{\sin^4\left( \pi f \tau \right)}{(\pi f \tau)^2}\,df \nonumber\\
    &=\frac{2}{3} \pi^2 h_{-2}\, \tau
\end{align}

The \textit{AllanTools} library \cite{allantools} can then be used to simulate the random walk.

\begin{figure}[ht]
    \centering
    \begin{subfigure}{0.32\linewidth}
        \centering
        \scalebox{0.75}{%
            %% Creator: Matplotlib, PGF backend
%%
%% To include the figure in your LaTeX document, write
%%   \input{<filename>.pgf}
%%
%% Make sure the required packages are loaded in your preamble
%%   \usepackage{pgf}
%%
%% Also ensure that all the required font packages are loaded; for instance,
%% the lmodern package is sometimes necessary when using math font.
%%   \usepackage{lmodern}
%%
%% Figures using additional raster images can only be included by \input if
%% they are in the same directory as the main LaTeX file. For loading figures
%% from other directories you can use the `import` package
%%   \usepackage{import}
%%
%% and then include the figures with
%%   \import{<path to file>}{<filename>.pgf}
%%
%% Matplotlib used the following preamble
%%   \usepackage{siunitx}
%%   \usepackage{fontspec}
%%
\begingroup%
\makeatletter%
\begin{pgfpicture}%
\pgfpathrectangle{\pgfpointorigin}{\pgfqpoint{2.440945in}{1.830709in}}%
\pgfusepath{use as bounding box, clip}%
\begin{pgfscope}%
\pgfsetbuttcap%
\pgfsetmiterjoin%
\definecolor{currentfill}{rgb}{1.000000,1.000000,1.000000}%
\pgfsetfillcolor{currentfill}%
\pgfsetlinewidth{0.000000pt}%
\definecolor{currentstroke}{rgb}{1.000000,1.000000,1.000000}%
\pgfsetstrokecolor{currentstroke}%
\pgfsetdash{}{0pt}%
\pgfpathmoveto{\pgfqpoint{0.000000in}{0.000000in}}%
\pgfpathlineto{\pgfqpoint{2.440945in}{0.000000in}}%
\pgfpathlineto{\pgfqpoint{2.440945in}{1.830709in}}%
\pgfpathlineto{\pgfqpoint{0.000000in}{1.830709in}}%
\pgfpathlineto{\pgfqpoint{0.000000in}{0.000000in}}%
\pgfpathclose%
\pgfusepath{fill}%
\end{pgfscope}%
\begin{pgfscope}%
\pgfsetbuttcap%
\pgfsetmiterjoin%
\definecolor{currentfill}{rgb}{1.000000,1.000000,1.000000}%
\pgfsetfillcolor{currentfill}%
\pgfsetlinewidth{0.000000pt}%
\definecolor{currentstroke}{rgb}{0.000000,0.000000,0.000000}%
\pgfsetstrokecolor{currentstroke}%
\pgfsetstrokeopacity{0.000000}%
\pgfsetdash{}{0pt}%
\pgfpathmoveto{\pgfqpoint{0.589745in}{0.416447in}}%
\pgfpathlineto{\pgfqpoint{2.399275in}{0.416447in}}%
\pgfpathlineto{\pgfqpoint{2.399275in}{1.789039in}}%
\pgfpathlineto{\pgfqpoint{0.589745in}{1.789039in}}%
\pgfpathlineto{\pgfqpoint{0.589745in}{0.416447in}}%
\pgfpathclose%
\pgfusepath{fill}%
\end{pgfscope}%
\begin{pgfscope}%
\pgfpathrectangle{\pgfqpoint{0.589745in}{0.416447in}}{\pgfqpoint{1.809530in}{1.372591in}}%
\pgfusepath{clip}%
\pgfsetrectcap%
\pgfsetroundjoin%
\pgfsetlinewidth{0.803000pt}%
\definecolor{currentstroke}{rgb}{0.450000,0.450000,0.450000}%
\pgfsetstrokecolor{currentstroke}%
\pgfsetdash{}{0pt}%
\pgfpathmoveto{\pgfqpoint{0.671996in}{0.416447in}}%
\pgfpathlineto{\pgfqpoint{0.671996in}{1.789039in}}%
\pgfusepath{stroke}%
\end{pgfscope}%
\begin{pgfscope}%
\pgfsetbuttcap%
\pgfsetroundjoin%
\definecolor{currentfill}{rgb}{0.000000,0.000000,0.000000}%
\pgfsetfillcolor{currentfill}%
\pgfsetlinewidth{0.803000pt}%
\definecolor{currentstroke}{rgb}{0.000000,0.000000,0.000000}%
\pgfsetstrokecolor{currentstroke}%
\pgfsetdash{}{0pt}%
\pgfsys@defobject{currentmarker}{\pgfqpoint{0.000000in}{-0.048611in}}{\pgfqpoint{0.000000in}{0.000000in}}{%
\pgfpathmoveto{\pgfqpoint{0.000000in}{0.000000in}}%
\pgfpathlineto{\pgfqpoint{0.000000in}{-0.048611in}}%
\pgfusepath{stroke,fill}%
}%
\begin{pgfscope}%
\pgfsys@transformshift{0.671996in}{0.416447in}%
\pgfsys@useobject{currentmarker}{}%
\end{pgfscope}%
\end{pgfscope}%
\begin{pgfscope}%
\definecolor{textcolor}{rgb}{0.000000,0.000000,0.000000}%
\pgfsetstrokecolor{textcolor}%
\pgfsetfillcolor{textcolor}%
\pgftext[x=0.671996in,y=0.319225in,,top]{\color{textcolor}\rmfamily\fontsize{8.000000}{9.600000}\selectfont \(\displaystyle {0}\)}%
\end{pgfscope}%
\begin{pgfscope}%
\pgfpathrectangle{\pgfqpoint{0.589745in}{0.416447in}}{\pgfqpoint{1.809530in}{1.372591in}}%
\pgfusepath{clip}%
\pgfsetrectcap%
\pgfsetroundjoin%
\pgfsetlinewidth{0.803000pt}%
\definecolor{currentstroke}{rgb}{0.450000,0.450000,0.450000}%
\pgfsetstrokecolor{currentstroke}%
\pgfsetdash{}{0pt}%
\pgfpathmoveto{\pgfqpoint{1.174080in}{0.416447in}}%
\pgfpathlineto{\pgfqpoint{1.174080in}{1.789039in}}%
\pgfusepath{stroke}%
\end{pgfscope}%
\begin{pgfscope}%
\pgfsetbuttcap%
\pgfsetroundjoin%
\definecolor{currentfill}{rgb}{0.000000,0.000000,0.000000}%
\pgfsetfillcolor{currentfill}%
\pgfsetlinewidth{0.803000pt}%
\definecolor{currentstroke}{rgb}{0.000000,0.000000,0.000000}%
\pgfsetstrokecolor{currentstroke}%
\pgfsetdash{}{0pt}%
\pgfsys@defobject{currentmarker}{\pgfqpoint{0.000000in}{-0.048611in}}{\pgfqpoint{0.000000in}{0.000000in}}{%
\pgfpathmoveto{\pgfqpoint{0.000000in}{0.000000in}}%
\pgfpathlineto{\pgfqpoint{0.000000in}{-0.048611in}}%
\pgfusepath{stroke,fill}%
}%
\begin{pgfscope}%
\pgfsys@transformshift{1.174080in}{0.416447in}%
\pgfsys@useobject{currentmarker}{}%
\end{pgfscope}%
\end{pgfscope}%
\begin{pgfscope}%
\definecolor{textcolor}{rgb}{0.000000,0.000000,0.000000}%
\pgfsetstrokecolor{textcolor}%
\pgfsetfillcolor{textcolor}%
\pgftext[x=1.174080in,y=0.319225in,,top]{\color{textcolor}\rmfamily\fontsize{8.000000}{9.600000}\selectfont \(\displaystyle {5000}\)}%
\end{pgfscope}%
\begin{pgfscope}%
\pgfpathrectangle{\pgfqpoint{0.589745in}{0.416447in}}{\pgfqpoint{1.809530in}{1.372591in}}%
\pgfusepath{clip}%
\pgfsetrectcap%
\pgfsetroundjoin%
\pgfsetlinewidth{0.803000pt}%
\definecolor{currentstroke}{rgb}{0.450000,0.450000,0.450000}%
\pgfsetstrokecolor{currentstroke}%
\pgfsetdash{}{0pt}%
\pgfpathmoveto{\pgfqpoint{1.676164in}{0.416447in}}%
\pgfpathlineto{\pgfqpoint{1.676164in}{1.789039in}}%
\pgfusepath{stroke}%
\end{pgfscope}%
\begin{pgfscope}%
\pgfsetbuttcap%
\pgfsetroundjoin%
\definecolor{currentfill}{rgb}{0.000000,0.000000,0.000000}%
\pgfsetfillcolor{currentfill}%
\pgfsetlinewidth{0.803000pt}%
\definecolor{currentstroke}{rgb}{0.000000,0.000000,0.000000}%
\pgfsetstrokecolor{currentstroke}%
\pgfsetdash{}{0pt}%
\pgfsys@defobject{currentmarker}{\pgfqpoint{0.000000in}{-0.048611in}}{\pgfqpoint{0.000000in}{0.000000in}}{%
\pgfpathmoveto{\pgfqpoint{0.000000in}{0.000000in}}%
\pgfpathlineto{\pgfqpoint{0.000000in}{-0.048611in}}%
\pgfusepath{stroke,fill}%
}%
\begin{pgfscope}%
\pgfsys@transformshift{1.676164in}{0.416447in}%
\pgfsys@useobject{currentmarker}{}%
\end{pgfscope}%
\end{pgfscope}%
\begin{pgfscope}%
\definecolor{textcolor}{rgb}{0.000000,0.000000,0.000000}%
\pgfsetstrokecolor{textcolor}%
\pgfsetfillcolor{textcolor}%
\pgftext[x=1.676164in,y=0.319225in,,top]{\color{textcolor}\rmfamily\fontsize{8.000000}{9.600000}\selectfont \(\displaystyle {10000}\)}%
\end{pgfscope}%
\begin{pgfscope}%
\pgfpathrectangle{\pgfqpoint{0.589745in}{0.416447in}}{\pgfqpoint{1.809530in}{1.372591in}}%
\pgfusepath{clip}%
\pgfsetrectcap%
\pgfsetroundjoin%
\pgfsetlinewidth{0.803000pt}%
\definecolor{currentstroke}{rgb}{0.450000,0.450000,0.450000}%
\pgfsetstrokecolor{currentstroke}%
\pgfsetdash{}{0pt}%
\pgfpathmoveto{\pgfqpoint{2.178248in}{0.416447in}}%
\pgfpathlineto{\pgfqpoint{2.178248in}{1.789039in}}%
\pgfusepath{stroke}%
\end{pgfscope}%
\begin{pgfscope}%
\pgfsetbuttcap%
\pgfsetroundjoin%
\definecolor{currentfill}{rgb}{0.000000,0.000000,0.000000}%
\pgfsetfillcolor{currentfill}%
\pgfsetlinewidth{0.803000pt}%
\definecolor{currentstroke}{rgb}{0.000000,0.000000,0.000000}%
\pgfsetstrokecolor{currentstroke}%
\pgfsetdash{}{0pt}%
\pgfsys@defobject{currentmarker}{\pgfqpoint{0.000000in}{-0.048611in}}{\pgfqpoint{0.000000in}{0.000000in}}{%
\pgfpathmoveto{\pgfqpoint{0.000000in}{0.000000in}}%
\pgfpathlineto{\pgfqpoint{0.000000in}{-0.048611in}}%
\pgfusepath{stroke,fill}%
}%
\begin{pgfscope}%
\pgfsys@transformshift{2.178248in}{0.416447in}%
\pgfsys@useobject{currentmarker}{}%
\end{pgfscope}%
\end{pgfscope}%
\begin{pgfscope}%
\definecolor{textcolor}{rgb}{0.000000,0.000000,0.000000}%
\pgfsetstrokecolor{textcolor}%
\pgfsetfillcolor{textcolor}%
\pgftext[x=2.178248in,y=0.319225in,,top]{\color{textcolor}\rmfamily\fontsize{8.000000}{9.600000}\selectfont \(\displaystyle {15000}\)}%
\end{pgfscope}%
\begin{pgfscope}%
\definecolor{textcolor}{rgb}{0.000000,0.000000,0.000000}%
\pgfsetstrokecolor{textcolor}%
\pgfsetfillcolor{textcolor}%
\pgftext[x=1.494510in,y=0.165003in,,top]{\color{textcolor}\rmfamily\fontsize{10.000000}{12.000000}\selectfont Time in \unit{\second}}%
\end{pgfscope}%
\begin{pgfscope}%
\pgfpathrectangle{\pgfqpoint{0.589745in}{0.416447in}}{\pgfqpoint{1.809530in}{1.372591in}}%
\pgfusepath{clip}%
\pgfsetrectcap%
\pgfsetroundjoin%
\pgfsetlinewidth{0.803000pt}%
\definecolor{currentstroke}{rgb}{0.450000,0.450000,0.450000}%
\pgfsetstrokecolor{currentstroke}%
\pgfsetdash{}{0pt}%
\pgfpathmoveto{\pgfqpoint{0.589745in}{0.416447in}}%
\pgfpathlineto{\pgfqpoint{2.399275in}{0.416447in}}%
\pgfusepath{stroke}%
\end{pgfscope}%
\begin{pgfscope}%
\pgfsetbuttcap%
\pgfsetroundjoin%
\definecolor{currentfill}{rgb}{0.000000,0.000000,0.000000}%
\pgfsetfillcolor{currentfill}%
\pgfsetlinewidth{0.803000pt}%
\definecolor{currentstroke}{rgb}{0.000000,0.000000,0.000000}%
\pgfsetstrokecolor{currentstroke}%
\pgfsetdash{}{0pt}%
\pgfsys@defobject{currentmarker}{\pgfqpoint{-0.048611in}{0.000000in}}{\pgfqpoint{-0.000000in}{0.000000in}}{%
\pgfpathmoveto{\pgfqpoint{-0.000000in}{0.000000in}}%
\pgfpathlineto{\pgfqpoint{-0.048611in}{0.000000in}}%
\pgfusepath{stroke,fill}%
}%
\begin{pgfscope}%
\pgfsys@transformshift{0.589745in}{0.416447in}%
\pgfsys@useobject{currentmarker}{}%
\end{pgfscope}%
\end{pgfscope}%
\begin{pgfscope}%
\definecolor{textcolor}{rgb}{0.000000,0.000000,0.000000}%
\pgfsetstrokecolor{textcolor}%
\pgfsetfillcolor{textcolor}%
\pgftext[x=0.223614in, y=0.377892in, left, base]{\color{textcolor}\rmfamily\fontsize{8.000000}{9.600000}\selectfont \(\displaystyle {\ensuremath{-}200}\)}%
\end{pgfscope}%
\begin{pgfscope}%
\pgfpathrectangle{\pgfqpoint{0.589745in}{0.416447in}}{\pgfqpoint{1.809530in}{1.372591in}}%
\pgfusepath{clip}%
\pgfsetrectcap%
\pgfsetroundjoin%
\pgfsetlinewidth{0.803000pt}%
\definecolor{currentstroke}{rgb}{0.450000,0.450000,0.450000}%
\pgfsetstrokecolor{currentstroke}%
\pgfsetdash{}{0pt}%
\pgfpathmoveto{\pgfqpoint{0.589745in}{0.721468in}}%
\pgfpathlineto{\pgfqpoint{2.399275in}{0.721468in}}%
\pgfusepath{stroke}%
\end{pgfscope}%
\begin{pgfscope}%
\pgfsetbuttcap%
\pgfsetroundjoin%
\definecolor{currentfill}{rgb}{0.000000,0.000000,0.000000}%
\pgfsetfillcolor{currentfill}%
\pgfsetlinewidth{0.803000pt}%
\definecolor{currentstroke}{rgb}{0.000000,0.000000,0.000000}%
\pgfsetstrokecolor{currentstroke}%
\pgfsetdash{}{0pt}%
\pgfsys@defobject{currentmarker}{\pgfqpoint{-0.048611in}{0.000000in}}{\pgfqpoint{-0.000000in}{0.000000in}}{%
\pgfpathmoveto{\pgfqpoint{-0.000000in}{0.000000in}}%
\pgfpathlineto{\pgfqpoint{-0.048611in}{0.000000in}}%
\pgfusepath{stroke,fill}%
}%
\begin{pgfscope}%
\pgfsys@transformshift{0.589745in}{0.721468in}%
\pgfsys@useobject{currentmarker}{}%
\end{pgfscope}%
\end{pgfscope}%
\begin{pgfscope}%
\definecolor{textcolor}{rgb}{0.000000,0.000000,0.000000}%
\pgfsetstrokecolor{textcolor}%
\pgfsetfillcolor{textcolor}%
\pgftext[x=0.223614in, y=0.682912in, left, base]{\color{textcolor}\rmfamily\fontsize{8.000000}{9.600000}\selectfont \(\displaystyle {\ensuremath{-}100}\)}%
\end{pgfscope}%
\begin{pgfscope}%
\pgfpathrectangle{\pgfqpoint{0.589745in}{0.416447in}}{\pgfqpoint{1.809530in}{1.372591in}}%
\pgfusepath{clip}%
\pgfsetrectcap%
\pgfsetroundjoin%
\pgfsetlinewidth{0.803000pt}%
\definecolor{currentstroke}{rgb}{0.450000,0.450000,0.450000}%
\pgfsetstrokecolor{currentstroke}%
\pgfsetdash{}{0pt}%
\pgfpathmoveto{\pgfqpoint{0.589745in}{1.026488in}}%
\pgfpathlineto{\pgfqpoint{2.399275in}{1.026488in}}%
\pgfusepath{stroke}%
\end{pgfscope}%
\begin{pgfscope}%
\pgfsetbuttcap%
\pgfsetroundjoin%
\definecolor{currentfill}{rgb}{0.000000,0.000000,0.000000}%
\pgfsetfillcolor{currentfill}%
\pgfsetlinewidth{0.803000pt}%
\definecolor{currentstroke}{rgb}{0.000000,0.000000,0.000000}%
\pgfsetstrokecolor{currentstroke}%
\pgfsetdash{}{0pt}%
\pgfsys@defobject{currentmarker}{\pgfqpoint{-0.048611in}{0.000000in}}{\pgfqpoint{-0.000000in}{0.000000in}}{%
\pgfpathmoveto{\pgfqpoint{-0.000000in}{0.000000in}}%
\pgfpathlineto{\pgfqpoint{-0.048611in}{0.000000in}}%
\pgfusepath{stroke,fill}%
}%
\begin{pgfscope}%
\pgfsys@transformshift{0.589745in}{1.026488in}%
\pgfsys@useobject{currentmarker}{}%
\end{pgfscope}%
\end{pgfscope}%
\begin{pgfscope}%
\definecolor{textcolor}{rgb}{0.000000,0.000000,0.000000}%
\pgfsetstrokecolor{textcolor}%
\pgfsetfillcolor{textcolor}%
\pgftext[x=0.433494in, y=0.987932in, left, base]{\color{textcolor}\rmfamily\fontsize{8.000000}{9.600000}\selectfont \(\displaystyle {0}\)}%
\end{pgfscope}%
\begin{pgfscope}%
\pgfpathrectangle{\pgfqpoint{0.589745in}{0.416447in}}{\pgfqpoint{1.809530in}{1.372591in}}%
\pgfusepath{clip}%
\pgfsetrectcap%
\pgfsetroundjoin%
\pgfsetlinewidth{0.803000pt}%
\definecolor{currentstroke}{rgb}{0.450000,0.450000,0.450000}%
\pgfsetstrokecolor{currentstroke}%
\pgfsetdash{}{0pt}%
\pgfpathmoveto{\pgfqpoint{0.589745in}{1.331508in}}%
\pgfpathlineto{\pgfqpoint{2.399275in}{1.331508in}}%
\pgfusepath{stroke}%
\end{pgfscope}%
\begin{pgfscope}%
\pgfsetbuttcap%
\pgfsetroundjoin%
\definecolor{currentfill}{rgb}{0.000000,0.000000,0.000000}%
\pgfsetfillcolor{currentfill}%
\pgfsetlinewidth{0.803000pt}%
\definecolor{currentstroke}{rgb}{0.000000,0.000000,0.000000}%
\pgfsetstrokecolor{currentstroke}%
\pgfsetdash{}{0pt}%
\pgfsys@defobject{currentmarker}{\pgfqpoint{-0.048611in}{0.000000in}}{\pgfqpoint{-0.000000in}{0.000000in}}{%
\pgfpathmoveto{\pgfqpoint{-0.000000in}{0.000000in}}%
\pgfpathlineto{\pgfqpoint{-0.048611in}{0.000000in}}%
\pgfusepath{stroke,fill}%
}%
\begin{pgfscope}%
\pgfsys@transformshift{0.589745in}{1.331508in}%
\pgfsys@useobject{currentmarker}{}%
\end{pgfscope}%
\end{pgfscope}%
\begin{pgfscope}%
\definecolor{textcolor}{rgb}{0.000000,0.000000,0.000000}%
\pgfsetstrokecolor{textcolor}%
\pgfsetfillcolor{textcolor}%
\pgftext[x=0.315437in, y=1.292953in, left, base]{\color{textcolor}\rmfamily\fontsize{8.000000}{9.600000}\selectfont \(\displaystyle {100}\)}%
\end{pgfscope}%
\begin{pgfscope}%
\pgfpathrectangle{\pgfqpoint{0.589745in}{0.416447in}}{\pgfqpoint{1.809530in}{1.372591in}}%
\pgfusepath{clip}%
\pgfsetrectcap%
\pgfsetroundjoin%
\pgfsetlinewidth{0.803000pt}%
\definecolor{currentstroke}{rgb}{0.450000,0.450000,0.450000}%
\pgfsetstrokecolor{currentstroke}%
\pgfsetdash{}{0pt}%
\pgfpathmoveto{\pgfqpoint{0.589745in}{1.636529in}}%
\pgfpathlineto{\pgfqpoint{2.399275in}{1.636529in}}%
\pgfusepath{stroke}%
\end{pgfscope}%
\begin{pgfscope}%
\pgfsetbuttcap%
\pgfsetroundjoin%
\definecolor{currentfill}{rgb}{0.000000,0.000000,0.000000}%
\pgfsetfillcolor{currentfill}%
\pgfsetlinewidth{0.803000pt}%
\definecolor{currentstroke}{rgb}{0.000000,0.000000,0.000000}%
\pgfsetstrokecolor{currentstroke}%
\pgfsetdash{}{0pt}%
\pgfsys@defobject{currentmarker}{\pgfqpoint{-0.048611in}{0.000000in}}{\pgfqpoint{-0.000000in}{0.000000in}}{%
\pgfpathmoveto{\pgfqpoint{-0.000000in}{0.000000in}}%
\pgfpathlineto{\pgfqpoint{-0.048611in}{0.000000in}}%
\pgfusepath{stroke,fill}%
}%
\begin{pgfscope}%
\pgfsys@transformshift{0.589745in}{1.636529in}%
\pgfsys@useobject{currentmarker}{}%
\end{pgfscope}%
\end{pgfscope}%
\begin{pgfscope}%
\definecolor{textcolor}{rgb}{0.000000,0.000000,0.000000}%
\pgfsetstrokecolor{textcolor}%
\pgfsetfillcolor{textcolor}%
\pgftext[x=0.315437in, y=1.597973in, left, base]{\color{textcolor}\rmfamily\fontsize{8.000000}{9.600000}\selectfont \(\displaystyle {200}\)}%
\end{pgfscope}%
\begin{pgfscope}%
\definecolor{textcolor}{rgb}{0.000000,0.000000,0.000000}%
\pgfsetstrokecolor{textcolor}%
\pgfsetfillcolor{textcolor}%
\pgftext[x=0.168059in,y=1.102743in,,bottom,rotate=90.000000]{\color{textcolor}\rmfamily\fontsize{10.000000}{12.000000}\selectfont Ampl. in arb. unit}%
\end{pgfscope}%
\begin{pgfscope}%
\pgfpathrectangle{\pgfqpoint{0.589745in}{0.416447in}}{\pgfqpoint{1.809530in}{1.372591in}}%
\pgfusepath{clip}%
\pgfsetrectcap%
\pgfsetroundjoin%
\pgfsetlinewidth{1.505625pt}%
\definecolor{currentstroke}{rgb}{0.835294,0.368627,0.000000}%
\pgfsetstrokecolor{currentstroke}%
\pgfsetdash{}{0pt}%
\pgfpathmoveto{\pgfqpoint{0.671996in}{1.028382in}}%
\pgfpathlineto{\pgfqpoint{0.672799in}{1.050160in}}%
\pgfpathlineto{\pgfqpoint{0.673804in}{1.008388in}}%
\pgfpathlineto{\pgfqpoint{0.677117in}{0.963029in}}%
\pgfpathlineto{\pgfqpoint{0.677419in}{0.976628in}}%
\pgfpathlineto{\pgfqpoint{0.679226in}{0.994781in}}%
\pgfpathlineto{\pgfqpoint{0.682239in}{0.955876in}}%
\pgfpathlineto{\pgfqpoint{0.684448in}{0.990350in}}%
\pgfpathlineto{\pgfqpoint{0.686657in}{0.956920in}}%
\pgfpathlineto{\pgfqpoint{0.689971in}{1.008738in}}%
\pgfpathlineto{\pgfqpoint{0.691879in}{0.983409in}}%
\pgfpathlineto{\pgfqpoint{0.693284in}{1.033325in}}%
\pgfpathlineto{\pgfqpoint{0.695293in}{1.017473in}}%
\pgfpathlineto{\pgfqpoint{0.697301in}{1.038142in}}%
\pgfpathlineto{\pgfqpoint{0.698506in}{1.001274in}}%
\pgfpathlineto{\pgfqpoint{0.700414in}{1.017613in}}%
\pgfpathlineto{\pgfqpoint{0.701217in}{1.000983in}}%
\pgfpathlineto{\pgfqpoint{0.705435in}{1.059540in}}%
\pgfpathlineto{\pgfqpoint{0.706941in}{1.026195in}}%
\pgfpathlineto{\pgfqpoint{0.709452in}{1.068101in}}%
\pgfpathlineto{\pgfqpoint{0.710456in}{1.051311in}}%
\pgfpathlineto{\pgfqpoint{0.711460in}{1.072581in}}%
\pgfpathlineto{\pgfqpoint{0.713267in}{1.050411in}}%
\pgfpathlineto{\pgfqpoint{0.714874in}{1.109093in}}%
\pgfpathlineto{\pgfqpoint{0.719292in}{1.027176in}}%
\pgfpathlineto{\pgfqpoint{0.719995in}{1.068644in}}%
\pgfpathlineto{\pgfqpoint{0.722204in}{1.059532in}}%
\pgfpathlineto{\pgfqpoint{0.724414in}{1.028869in}}%
\pgfpathlineto{\pgfqpoint{0.725819in}{0.981630in}}%
\pgfpathlineto{\pgfqpoint{0.728330in}{0.995090in}}%
\pgfpathlineto{\pgfqpoint{0.729033in}{0.969724in}}%
\pgfpathlineto{\pgfqpoint{0.730639in}{0.992702in}}%
\pgfpathlineto{\pgfqpoint{0.732949in}{0.986556in}}%
\pgfpathlineto{\pgfqpoint{0.734556in}{1.038679in}}%
\pgfpathlineto{\pgfqpoint{0.736765in}{0.989943in}}%
\pgfpathlineto{\pgfqpoint{0.737970in}{1.015469in}}%
\pgfpathlineto{\pgfqpoint{0.739777in}{0.975342in}}%
\pgfpathlineto{\pgfqpoint{0.741284in}{0.998304in}}%
\pgfpathlineto{\pgfqpoint{0.742991in}{0.974466in}}%
\pgfpathlineto{\pgfqpoint{0.743895in}{0.990290in}}%
\pgfpathlineto{\pgfqpoint{0.746104in}{0.990028in}}%
\pgfpathlineto{\pgfqpoint{0.747409in}{0.938439in}}%
\pgfpathlineto{\pgfqpoint{0.749618in}{0.996588in}}%
\pgfpathlineto{\pgfqpoint{0.751627in}{0.983846in}}%
\pgfpathlineto{\pgfqpoint{0.753334in}{0.986717in}}%
\pgfpathlineto{\pgfqpoint{0.754840in}{1.033789in}}%
\pgfpathlineto{\pgfqpoint{0.755744in}{1.007055in}}%
\pgfpathlineto{\pgfqpoint{0.758355in}{1.018001in}}%
\pgfpathlineto{\pgfqpoint{0.760062in}{1.039121in}}%
\pgfpathlineto{\pgfqpoint{0.761769in}{1.111956in}}%
\pgfpathlineto{\pgfqpoint{0.765384in}{1.156039in}}%
\pgfpathlineto{\pgfqpoint{0.767492in}{1.121262in}}%
\pgfpathlineto{\pgfqpoint{0.768999in}{1.172235in}}%
\pgfpathlineto{\pgfqpoint{0.770103in}{1.126633in}}%
\pgfpathlineto{\pgfqpoint{0.771810in}{1.115731in}}%
\pgfpathlineto{\pgfqpoint{0.773718in}{1.178864in}}%
\pgfpathlineto{\pgfqpoint{0.776028in}{1.182389in}}%
\pgfpathlineto{\pgfqpoint{0.777233in}{1.215041in}}%
\pgfpathlineto{\pgfqpoint{0.780045in}{1.183131in}}%
\pgfpathlineto{\pgfqpoint{0.782455in}{1.203215in}}%
\pgfpathlineto{\pgfqpoint{0.785969in}{1.283414in}}%
\pgfpathlineto{\pgfqpoint{0.787676in}{1.252093in}}%
\pgfpathlineto{\pgfqpoint{0.789484in}{1.302433in}}%
\pgfpathlineto{\pgfqpoint{0.791793in}{1.256289in}}%
\pgfpathlineto{\pgfqpoint{0.794203in}{1.256062in}}%
\pgfpathlineto{\pgfqpoint{0.796312in}{1.307795in}}%
\pgfpathlineto{\pgfqpoint{0.797617in}{1.271131in}}%
\pgfpathlineto{\pgfqpoint{0.799224in}{1.265744in}}%
\pgfpathlineto{\pgfqpoint{0.800730in}{1.303382in}}%
\pgfpathlineto{\pgfqpoint{0.801935in}{1.287507in}}%
\pgfpathlineto{\pgfqpoint{0.803743in}{1.311649in}}%
\pgfpathlineto{\pgfqpoint{0.806856in}{1.308349in}}%
\pgfpathlineto{\pgfqpoint{0.808262in}{1.249344in}}%
\pgfpathlineto{\pgfqpoint{0.809768in}{1.234533in}}%
\pgfpathlineto{\pgfqpoint{0.811174in}{1.279436in}}%
\pgfpathlineto{\pgfqpoint{0.812479in}{1.269969in}}%
\pgfpathlineto{\pgfqpoint{0.814588in}{1.327813in}}%
\pgfpathlineto{\pgfqpoint{0.816094in}{1.314642in}}%
\pgfpathlineto{\pgfqpoint{0.817902in}{1.367676in}}%
\pgfpathlineto{\pgfqpoint{0.818906in}{1.358873in}}%
\pgfpathlineto{\pgfqpoint{0.821316in}{1.355364in}}%
\pgfpathlineto{\pgfqpoint{0.822521in}{1.419301in}}%
\pgfpathlineto{\pgfqpoint{0.825232in}{1.444840in}}%
\pgfpathlineto{\pgfqpoint{0.826437in}{1.396328in}}%
\pgfpathlineto{\pgfqpoint{0.827441in}{1.418052in}}%
\pgfpathlineto{\pgfqpoint{0.829249in}{1.408509in}}%
\pgfpathlineto{\pgfqpoint{0.832060in}{1.352404in}}%
\pgfpathlineto{\pgfqpoint{0.833667in}{1.353689in}}%
\pgfpathlineto{\pgfqpoint{0.838588in}{1.443224in}}%
\pgfpathlineto{\pgfqpoint{0.839893in}{1.424151in}}%
\pgfpathlineto{\pgfqpoint{0.843307in}{1.478097in}}%
\pgfpathlineto{\pgfqpoint{0.844512in}{1.471360in}}%
\pgfpathlineto{\pgfqpoint{0.846320in}{1.473487in}}%
\pgfpathlineto{\pgfqpoint{0.849734in}{1.531018in}}%
\pgfpathlineto{\pgfqpoint{0.851340in}{1.483831in}}%
\pgfpathlineto{\pgfqpoint{0.853951in}{1.475102in}}%
\pgfpathlineto{\pgfqpoint{0.854955in}{1.442567in}}%
\pgfpathlineto{\pgfqpoint{0.857165in}{1.432908in}}%
\pgfpathlineto{\pgfqpoint{0.858872in}{1.465357in}}%
\pgfpathlineto{\pgfqpoint{0.859876in}{1.449097in}}%
\pgfpathlineto{\pgfqpoint{0.863190in}{1.468953in}}%
\pgfpathlineto{\pgfqpoint{0.863491in}{1.448393in}}%
\pgfpathlineto{\pgfqpoint{0.865399in}{1.433213in}}%
\pgfpathlineto{\pgfqpoint{0.868311in}{1.448486in}}%
\pgfpathlineto{\pgfqpoint{0.869114in}{1.482138in}}%
\pgfpathlineto{\pgfqpoint{0.870922in}{1.468953in}}%
\pgfpathlineto{\pgfqpoint{0.872227in}{1.512453in}}%
\pgfpathlineto{\pgfqpoint{0.875139in}{1.440697in}}%
\pgfpathlineto{\pgfqpoint{0.876344in}{1.482798in}}%
\pgfpathlineto{\pgfqpoint{0.878353in}{1.444191in}}%
\pgfpathlineto{\pgfqpoint{0.880060in}{1.466788in}}%
\pgfpathlineto{\pgfqpoint{0.881566in}{1.436556in}}%
\pgfpathlineto{\pgfqpoint{0.882369in}{1.453445in}}%
\pgfpathlineto{\pgfqpoint{0.884679in}{1.416311in}}%
\pgfpathlineto{\pgfqpoint{0.886486in}{1.454866in}}%
\pgfpathlineto{\pgfqpoint{0.887792in}{1.427612in}}%
\pgfpathlineto{\pgfqpoint{0.889198in}{1.446298in}}%
\pgfpathlineto{\pgfqpoint{0.891808in}{1.428288in}}%
\pgfpathlineto{\pgfqpoint{0.893013in}{1.445594in}}%
\pgfpathlineto{\pgfqpoint{0.895423in}{1.457889in}}%
\pgfpathlineto{\pgfqpoint{0.896628in}{1.435393in}}%
\pgfpathlineto{\pgfqpoint{0.897934in}{1.453342in}}%
\pgfpathlineto{\pgfqpoint{0.901649in}{1.411022in}}%
\pgfpathlineto{\pgfqpoint{0.903155in}{1.432137in}}%
\pgfpathlineto{\pgfqpoint{0.904461in}{1.490675in}}%
\pgfpathlineto{\pgfqpoint{0.907072in}{1.504084in}}%
\pgfpathlineto{\pgfqpoint{0.909080in}{1.474221in}}%
\pgfpathlineto{\pgfqpoint{0.910386in}{1.486989in}}%
\pgfpathlineto{\pgfqpoint{0.913197in}{1.418903in}}%
\pgfpathlineto{\pgfqpoint{0.915005in}{1.450529in}}%
\pgfpathlineto{\pgfqpoint{0.917013in}{1.455788in}}%
\pgfpathlineto{\pgfqpoint{0.918218in}{1.430496in}}%
\pgfpathlineto{\pgfqpoint{0.921130in}{1.485158in}}%
\pgfpathlineto{\pgfqpoint{0.922134in}{1.464130in}}%
\pgfpathlineto{\pgfqpoint{0.923641in}{1.487643in}}%
\pgfpathlineto{\pgfqpoint{0.924846in}{1.464594in}}%
\pgfpathlineto{\pgfqpoint{0.926653in}{1.501245in}}%
\pgfpathlineto{\pgfqpoint{0.929565in}{1.467042in}}%
\pgfpathlineto{\pgfqpoint{0.931373in}{1.512224in}}%
\pgfpathlineto{\pgfqpoint{0.931774in}{1.496649in}}%
\pgfpathlineto{\pgfqpoint{0.933682in}{1.459377in}}%
\pgfpathlineto{\pgfqpoint{0.936494in}{1.521512in}}%
\pgfpathlineto{\pgfqpoint{0.937699in}{1.505337in}}%
\pgfpathlineto{\pgfqpoint{0.938904in}{1.525365in}}%
\pgfpathlineto{\pgfqpoint{0.940711in}{1.487793in}}%
\pgfpathlineto{\pgfqpoint{0.942318in}{1.488486in}}%
\pgfpathlineto{\pgfqpoint{0.943523in}{1.537586in}}%
\pgfpathlineto{\pgfqpoint{0.945230in}{1.542311in}}%
\pgfpathlineto{\pgfqpoint{0.947339in}{1.608473in}}%
\pgfpathlineto{\pgfqpoint{0.949749in}{1.603955in}}%
\pgfpathlineto{\pgfqpoint{0.952862in}{1.534210in}}%
\pgfpathlineto{\pgfqpoint{0.954167in}{1.530786in}}%
\pgfpathlineto{\pgfqpoint{0.956878in}{1.582461in}}%
\pgfpathlineto{\pgfqpoint{0.958083in}{1.555660in}}%
\pgfpathlineto{\pgfqpoint{0.960192in}{1.558678in}}%
\pgfpathlineto{\pgfqpoint{0.961698in}{1.487920in}}%
\pgfpathlineto{\pgfqpoint{0.962602in}{1.513358in}}%
\pgfpathlineto{\pgfqpoint{0.964309in}{1.494956in}}%
\pgfpathlineto{\pgfqpoint{0.966619in}{1.551913in}}%
\pgfpathlineto{\pgfqpoint{0.968025in}{1.512647in}}%
\pgfpathlineto{\pgfqpoint{0.970334in}{1.503784in}}%
\pgfpathlineto{\pgfqpoint{0.971439in}{1.539301in}}%
\pgfpathlineto{\pgfqpoint{0.972744in}{1.539295in}}%
\pgfpathlineto{\pgfqpoint{0.976058in}{1.489378in}}%
\pgfpathlineto{\pgfqpoint{0.979573in}{1.439862in}}%
\pgfpathlineto{\pgfqpoint{0.981179in}{1.486773in}}%
\pgfpathlineto{\pgfqpoint{0.983489in}{1.492912in}}%
\pgfpathlineto{\pgfqpoint{0.984493in}{1.464048in}}%
\pgfpathlineto{\pgfqpoint{0.987204in}{1.475706in}}%
\pgfpathlineto{\pgfqpoint{0.990920in}{1.375985in}}%
\pgfpathlineto{\pgfqpoint{0.992426in}{1.413554in}}%
\pgfpathlineto{\pgfqpoint{0.993832in}{1.384876in}}%
\pgfpathlineto{\pgfqpoint{0.994736in}{1.426587in}}%
\pgfpathlineto{\pgfqpoint{0.996443in}{1.408110in}}%
\pgfpathlineto{\pgfqpoint{1.001162in}{1.471423in}}%
\pgfpathlineto{\pgfqpoint{1.003070in}{1.440875in}}%
\pgfpathlineto{\pgfqpoint{1.004476in}{1.472041in}}%
\pgfpathlineto{\pgfqpoint{1.004978in}{1.461496in}}%
\pgfpathlineto{\pgfqpoint{1.008091in}{1.503035in}}%
\pgfpathlineto{\pgfqpoint{1.009698in}{1.447446in}}%
\pgfpathlineto{\pgfqpoint{1.011204in}{1.447175in}}%
\pgfpathlineto{\pgfqpoint{1.011806in}{1.474315in}}%
\pgfpathlineto{\pgfqpoint{1.014518in}{1.499617in}}%
\pgfpathlineto{\pgfqpoint{1.015723in}{1.478897in}}%
\pgfpathlineto{\pgfqpoint{1.016928in}{1.532884in}}%
\pgfpathlineto{\pgfqpoint{1.018735in}{1.487847in}}%
\pgfpathlineto{\pgfqpoint{1.020543in}{1.509778in}}%
\pgfpathlineto{\pgfqpoint{1.022451in}{1.470820in}}%
\pgfpathlineto{\pgfqpoint{1.025162in}{1.490717in}}%
\pgfpathlineto{\pgfqpoint{1.026266in}{1.457367in}}%
\pgfpathlineto{\pgfqpoint{1.028074in}{1.514357in}}%
\pgfpathlineto{\pgfqpoint{1.029580in}{1.484648in}}%
\pgfpathlineto{\pgfqpoint{1.030584in}{1.516197in}}%
\pgfpathlineto{\pgfqpoint{1.032291in}{1.505736in}}%
\pgfpathlineto{\pgfqpoint{1.034099in}{1.536619in}}%
\pgfpathlineto{\pgfqpoint{1.035806in}{1.535506in}}%
\pgfpathlineto{\pgfqpoint{1.037915in}{1.507509in}}%
\pgfpathlineto{\pgfqpoint{1.039220in}{1.507487in}}%
\pgfpathlineto{\pgfqpoint{1.041931in}{1.536696in}}%
\pgfpathlineto{\pgfqpoint{1.043538in}{1.595170in}}%
\pgfpathlineto{\pgfqpoint{1.045546in}{1.615871in}}%
\pgfpathlineto{\pgfqpoint{1.046651in}{1.576768in}}%
\pgfpathlineto{\pgfqpoint{1.048961in}{1.542125in}}%
\pgfpathlineto{\pgfqpoint{1.050467in}{1.565408in}}%
\pgfpathlineto{\pgfqpoint{1.051772in}{1.505963in}}%
\pgfpathlineto{\pgfqpoint{1.053178in}{1.509923in}}%
\pgfpathlineto{\pgfqpoint{1.055488in}{1.437115in}}%
\pgfpathlineto{\pgfqpoint{1.056693in}{1.448674in}}%
\pgfpathlineto{\pgfqpoint{1.058701in}{1.441163in}}%
\pgfpathlineto{\pgfqpoint{1.060006in}{1.409929in}}%
\pgfpathlineto{\pgfqpoint{1.062919in}{1.379515in}}%
\pgfpathlineto{\pgfqpoint{1.064124in}{1.414242in}}%
\pgfpathlineto{\pgfqpoint{1.065128in}{1.391692in}}%
\pgfpathlineto{\pgfqpoint{1.067136in}{1.428266in}}%
\pgfpathlineto{\pgfqpoint{1.068642in}{1.405846in}}%
\pgfpathlineto{\pgfqpoint{1.071755in}{1.475065in}}%
\pgfpathlineto{\pgfqpoint{1.073261in}{1.424196in}}%
\pgfpathlineto{\pgfqpoint{1.076575in}{1.399480in}}%
\pgfpathlineto{\pgfqpoint{1.076977in}{1.366641in}}%
\pgfpathlineto{\pgfqpoint{1.079688in}{1.406090in}}%
\pgfpathlineto{\pgfqpoint{1.081496in}{1.368484in}}%
\pgfpathlineto{\pgfqpoint{1.083002in}{1.390385in}}%
\pgfpathlineto{\pgfqpoint{1.083705in}{1.372606in}}%
\pgfpathlineto{\pgfqpoint{1.086215in}{1.374139in}}%
\pgfpathlineto{\pgfqpoint{1.087922in}{1.418781in}}%
\pgfpathlineto{\pgfqpoint{1.088625in}{1.396656in}}%
\pgfpathlineto{\pgfqpoint{1.090433in}{1.396206in}}%
\pgfpathlineto{\pgfqpoint{1.092843in}{1.393743in}}%
\pgfpathlineto{\pgfqpoint{1.095654in}{1.290070in}}%
\pgfpathlineto{\pgfqpoint{1.098165in}{1.294178in}}%
\pgfpathlineto{\pgfqpoint{1.100274in}{1.325778in}}%
\pgfpathlineto{\pgfqpoint{1.101378in}{1.278390in}}%
\pgfpathlineto{\pgfqpoint{1.102684in}{1.303671in}}%
\pgfpathlineto{\pgfqpoint{1.104792in}{1.238290in}}%
\pgfpathlineto{\pgfqpoint{1.107102in}{1.260167in}}%
\pgfpathlineto{\pgfqpoint{1.109010in}{1.219246in}}%
\pgfpathlineto{\pgfqpoint{1.110215in}{1.244345in}}%
\pgfpathlineto{\pgfqpoint{1.110918in}{1.221507in}}%
\pgfpathlineto{\pgfqpoint{1.113428in}{1.206574in}}%
\pgfpathlineto{\pgfqpoint{1.114633in}{1.231352in}}%
\pgfpathlineto{\pgfqpoint{1.118349in}{1.153282in}}%
\pgfpathlineto{\pgfqpoint{1.119554in}{1.153523in}}%
\pgfpathlineto{\pgfqpoint{1.121261in}{1.209198in}}%
\pgfpathlineto{\pgfqpoint{1.123470in}{1.204160in}}%
\pgfpathlineto{\pgfqpoint{1.125779in}{1.156658in}}%
\pgfpathlineto{\pgfqpoint{1.126282in}{1.188200in}}%
\pgfpathlineto{\pgfqpoint{1.128089in}{1.180588in}}%
\pgfpathlineto{\pgfqpoint{1.130599in}{1.198954in}}%
\pgfpathlineto{\pgfqpoint{1.131403in}{1.184135in}}%
\pgfpathlineto{\pgfqpoint{1.134114in}{1.226481in}}%
\pgfpathlineto{\pgfqpoint{1.136323in}{1.217661in}}%
\pgfpathlineto{\pgfqpoint{1.137729in}{1.165468in}}%
\pgfpathlineto{\pgfqpoint{1.138332in}{1.188500in}}%
\pgfpathlineto{\pgfqpoint{1.141344in}{1.178205in}}%
\pgfpathlineto{\pgfqpoint{1.142047in}{1.197090in}}%
\pgfpathlineto{\pgfqpoint{1.143955in}{1.173356in}}%
\pgfpathlineto{\pgfqpoint{1.145963in}{1.227555in}}%
\pgfpathlineto{\pgfqpoint{1.147068in}{1.203889in}}%
\pgfpathlineto{\pgfqpoint{1.148474in}{1.220746in}}%
\pgfpathlineto{\pgfqpoint{1.151587in}{1.206983in}}%
\pgfpathlineto{\pgfqpoint{1.152189in}{1.173327in}}%
\pgfpathlineto{\pgfqpoint{1.154499in}{1.127436in}}%
\pgfpathlineto{\pgfqpoint{1.156708in}{1.137973in}}%
\pgfpathlineto{\pgfqpoint{1.157612in}{1.121034in}}%
\pgfpathlineto{\pgfqpoint{1.159720in}{1.150076in}}%
\pgfpathlineto{\pgfqpoint{1.160423in}{1.122385in}}%
\pgfpathlineto{\pgfqpoint{1.162833in}{1.147802in}}%
\pgfpathlineto{\pgfqpoint{1.164942in}{1.098646in}}%
\pgfpathlineto{\pgfqpoint{1.166348in}{1.146187in}}%
\pgfpathlineto{\pgfqpoint{1.167151in}{1.137178in}}%
\pgfpathlineto{\pgfqpoint{1.169360in}{1.169407in}}%
\pgfpathlineto{\pgfqpoint{1.171971in}{1.133395in}}%
\pgfpathlineto{\pgfqpoint{1.173779in}{1.178946in}}%
\pgfpathlineto{\pgfqpoint{1.174482in}{1.155741in}}%
\pgfpathlineto{\pgfqpoint{1.176992in}{1.155095in}}%
\pgfpathlineto{\pgfqpoint{1.177996in}{1.206015in}}%
\pgfpathlineto{\pgfqpoint{1.180306in}{1.185446in}}%
\pgfpathlineto{\pgfqpoint{1.180908in}{1.198371in}}%
\pgfpathlineto{\pgfqpoint{1.182816in}{1.213476in}}%
\pgfpathlineto{\pgfqpoint{1.185025in}{1.190660in}}%
\pgfpathlineto{\pgfqpoint{1.186732in}{1.187343in}}%
\pgfpathlineto{\pgfqpoint{1.189042in}{1.215508in}}%
\pgfpathlineto{\pgfqpoint{1.190548in}{1.214918in}}%
\pgfpathlineto{\pgfqpoint{1.191352in}{1.188916in}}%
\pgfpathlineto{\pgfqpoint{1.193862in}{1.165823in}}%
\pgfpathlineto{\pgfqpoint{1.194665in}{1.194213in}}%
\pgfpathlineto{\pgfqpoint{1.196372in}{1.180284in}}%
\pgfpathlineto{\pgfqpoint{1.198481in}{1.228042in}}%
\pgfpathlineto{\pgfqpoint{1.199887in}{1.179900in}}%
\pgfpathlineto{\pgfqpoint{1.201393in}{1.216189in}}%
\pgfpathlineto{\pgfqpoint{1.204607in}{1.244571in}}%
\pgfpathlineto{\pgfqpoint{1.205711in}{1.219208in}}%
\pgfpathlineto{\pgfqpoint{1.206916in}{1.223489in}}%
\pgfpathlineto{\pgfqpoint{1.208925in}{1.174514in}}%
\pgfpathlineto{\pgfqpoint{1.211335in}{1.144399in}}%
\pgfpathlineto{\pgfqpoint{1.212740in}{1.161641in}}%
\pgfpathlineto{\pgfqpoint{1.213544in}{1.126150in}}%
\pgfpathlineto{\pgfqpoint{1.215652in}{1.115154in}}%
\pgfpathlineto{\pgfqpoint{1.218765in}{1.067327in}}%
\pgfpathlineto{\pgfqpoint{1.220975in}{1.069277in}}%
\pgfpathlineto{\pgfqpoint{1.222280in}{1.098013in}}%
\pgfpathlineto{\pgfqpoint{1.223585in}{1.101671in}}%
\pgfpathlineto{\pgfqpoint{1.226698in}{1.053406in}}%
\pgfpathlineto{\pgfqpoint{1.228405in}{1.096440in}}%
\pgfpathlineto{\pgfqpoint{1.228707in}{1.084638in}}%
\pgfpathlineto{\pgfqpoint{1.231016in}{1.113279in}}%
\pgfpathlineto{\pgfqpoint{1.232020in}{1.076230in}}%
\pgfpathlineto{\pgfqpoint{1.235334in}{1.110496in}}%
\pgfpathlineto{\pgfqpoint{1.236439in}{1.082863in}}%
\pgfpathlineto{\pgfqpoint{1.237845in}{1.111997in}}%
\pgfpathlineto{\pgfqpoint{1.238849in}{1.091113in}}%
\pgfpathlineto{\pgfqpoint{1.242062in}{1.127060in}}%
\pgfpathlineto{\pgfqpoint{1.243568in}{1.085421in}}%
\pgfpathlineto{\pgfqpoint{1.244271in}{1.093728in}}%
\pgfpathlineto{\pgfqpoint{1.246079in}{1.094918in}}%
\pgfpathlineto{\pgfqpoint{1.247384in}{1.062650in}}%
\pgfpathlineto{\pgfqpoint{1.249794in}{1.032432in}}%
\pgfpathlineto{\pgfqpoint{1.252405in}{1.017360in}}%
\pgfpathlineto{\pgfqpoint{1.253008in}{1.042951in}}%
\pgfpathlineto{\pgfqpoint{1.255518in}{1.055169in}}%
\pgfpathlineto{\pgfqpoint{1.256120in}{1.031493in}}%
\pgfpathlineto{\pgfqpoint{1.258028in}{1.020722in}}%
\pgfpathlineto{\pgfqpoint{1.261744in}{0.936060in}}%
\pgfpathlineto{\pgfqpoint{1.265058in}{0.857218in}}%
\pgfpathlineto{\pgfqpoint{1.267668in}{0.847558in}}%
\pgfpathlineto{\pgfqpoint{1.268271in}{0.875230in}}%
\pgfpathlineto{\pgfqpoint{1.269576in}{0.854532in}}%
\pgfpathlineto{\pgfqpoint{1.273995in}{0.937688in}}%
\pgfpathlineto{\pgfqpoint{1.275099in}{0.903673in}}%
\pgfpathlineto{\pgfqpoint{1.277007in}{0.948726in}}%
\pgfpathlineto{\pgfqpoint{1.278112in}{0.902784in}}%
\pgfpathlineto{\pgfqpoint{1.280723in}{0.990180in}}%
\pgfpathlineto{\pgfqpoint{1.282731in}{1.001967in}}%
\pgfpathlineto{\pgfqpoint{1.284840in}{0.975425in}}%
\pgfpathlineto{\pgfqpoint{1.285040in}{0.993391in}}%
\pgfpathlineto{\pgfqpoint{1.286647in}{0.975199in}}%
\pgfpathlineto{\pgfqpoint{1.289459in}{0.993476in}}%
\pgfpathlineto{\pgfqpoint{1.290764in}{0.963511in}}%
\pgfpathlineto{\pgfqpoint{1.293375in}{0.983902in}}%
\pgfpathlineto{\pgfqpoint{1.294580in}{0.948053in}}%
\pgfpathlineto{\pgfqpoint{1.295584in}{0.972862in}}%
\pgfpathlineto{\pgfqpoint{1.297894in}{0.946215in}}%
\pgfpathlineto{\pgfqpoint{1.299802in}{0.964695in}}%
\pgfpathlineto{\pgfqpoint{1.300404in}{0.944187in}}%
\pgfpathlineto{\pgfqpoint{1.303618in}{0.975300in}}%
\pgfpathlineto{\pgfqpoint{1.304622in}{0.952959in}}%
\pgfpathlineto{\pgfqpoint{1.306831in}{0.995901in}}%
\pgfpathlineto{\pgfqpoint{1.307132in}{0.983110in}}%
\pgfpathlineto{\pgfqpoint{1.309542in}{0.994464in}}%
\pgfpathlineto{\pgfqpoint{1.310546in}{0.963021in}}%
\pgfpathlineto{\pgfqpoint{1.312555in}{1.000053in}}%
\pgfpathlineto{\pgfqpoint{1.314764in}{0.962607in}}%
\pgfpathlineto{\pgfqpoint{1.317274in}{0.970348in}}%
\pgfpathlineto{\pgfqpoint{1.320186in}{0.883459in}}%
\pgfpathlineto{\pgfqpoint{1.322898in}{0.837076in}}%
\pgfpathlineto{\pgfqpoint{1.325709in}{0.887322in}}%
\pgfpathlineto{\pgfqpoint{1.326814in}{0.871764in}}%
\pgfpathlineto{\pgfqpoint{1.328119in}{0.898262in}}%
\pgfpathlineto{\pgfqpoint{1.329324in}{0.851760in}}%
\pgfpathlineto{\pgfqpoint{1.332136in}{0.907239in}}%
\pgfpathlineto{\pgfqpoint{1.332939in}{0.880761in}}%
\pgfpathlineto{\pgfqpoint{1.335450in}{0.864891in}}%
\pgfpathlineto{\pgfqpoint{1.336353in}{0.902078in}}%
\pgfpathlineto{\pgfqpoint{1.338663in}{0.846935in}}%
\pgfpathlineto{\pgfqpoint{1.340872in}{0.869714in}}%
\pgfpathlineto{\pgfqpoint{1.341274in}{0.840102in}}%
\pgfpathlineto{\pgfqpoint{1.343182in}{0.847296in}}%
\pgfpathlineto{\pgfqpoint{1.345291in}{0.801427in}}%
\pgfpathlineto{\pgfqpoint{1.348504in}{0.869384in}}%
\pgfpathlineto{\pgfqpoint{1.350211in}{0.798934in}}%
\pgfpathlineto{\pgfqpoint{1.351516in}{0.841236in}}%
\pgfpathlineto{\pgfqpoint{1.354629in}{0.858492in}}%
\pgfpathlineto{\pgfqpoint{1.356437in}{0.784280in}}%
\pgfpathlineto{\pgfqpoint{1.357140in}{0.803925in}}%
\pgfpathlineto{\pgfqpoint{1.358847in}{0.763057in}}%
\pgfpathlineto{\pgfqpoint{1.361357in}{0.813848in}}%
\pgfpathlineto{\pgfqpoint{1.363366in}{0.757704in}}%
\pgfpathlineto{\pgfqpoint{1.363868in}{0.787371in}}%
\pgfpathlineto{\pgfqpoint{1.365776in}{0.735035in}}%
\pgfpathlineto{\pgfqpoint{1.366981in}{0.760654in}}%
\pgfpathlineto{\pgfqpoint{1.370395in}{0.713782in}}%
\pgfpathlineto{\pgfqpoint{1.372604in}{0.783180in}}%
\pgfpathlineto{\pgfqpoint{1.374913in}{0.773549in}}%
\pgfpathlineto{\pgfqpoint{1.376821in}{0.838249in}}%
\pgfpathlineto{\pgfqpoint{1.378328in}{0.828675in}}%
\pgfpathlineto{\pgfqpoint{1.379733in}{0.783538in}}%
\pgfpathlineto{\pgfqpoint{1.382947in}{0.837344in}}%
\pgfpathlineto{\pgfqpoint{1.383549in}{0.807796in}}%
\pgfpathlineto{\pgfqpoint{1.385859in}{0.785589in}}%
\pgfpathlineto{\pgfqpoint{1.388168in}{0.799240in}}%
\pgfpathlineto{\pgfqpoint{1.389574in}{0.767216in}}%
\pgfpathlineto{\pgfqpoint{1.391382in}{0.751807in}}%
\pgfpathlineto{\pgfqpoint{1.392486in}{0.781044in}}%
\pgfpathlineto{\pgfqpoint{1.393792in}{0.760591in}}%
\pgfpathlineto{\pgfqpoint{1.395901in}{0.789110in}}%
\pgfpathlineto{\pgfqpoint{1.397106in}{0.763963in}}%
\pgfpathlineto{\pgfqpoint{1.398411in}{0.790263in}}%
\pgfpathlineto{\pgfqpoint{1.400018in}{0.753367in}}%
\pgfpathlineto{\pgfqpoint{1.401223in}{0.810095in}}%
\pgfpathlineto{\pgfqpoint{1.403833in}{0.834806in}}%
\pgfpathlineto{\pgfqpoint{1.406344in}{0.758372in}}%
\pgfpathlineto{\pgfqpoint{1.408453in}{0.809103in}}%
\pgfpathlineto{\pgfqpoint{1.409758in}{0.781256in}}%
\pgfpathlineto{\pgfqpoint{1.411566in}{0.824825in}}%
\pgfpathlineto{\pgfqpoint{1.412469in}{0.815033in}}%
\pgfpathlineto{\pgfqpoint{1.414076in}{0.841226in}}%
\pgfpathlineto{\pgfqpoint{1.416386in}{0.860335in}}%
\pgfpathlineto{\pgfqpoint{1.418796in}{0.801384in}}%
\pgfpathlineto{\pgfqpoint{1.419800in}{0.840051in}}%
\pgfpathlineto{\pgfqpoint{1.421306in}{0.800585in}}%
\pgfpathlineto{\pgfqpoint{1.422511in}{0.816833in}}%
\pgfpathlineto{\pgfqpoint{1.424921in}{0.795460in}}%
\pgfpathlineto{\pgfqpoint{1.426628in}{0.810772in}}%
\pgfpathlineto{\pgfqpoint{1.428938in}{0.761148in}}%
\pgfpathlineto{\pgfqpoint{1.430143in}{0.803163in}}%
\pgfpathlineto{\pgfqpoint{1.432954in}{0.744895in}}%
\pgfpathlineto{\pgfqpoint{1.434461in}{0.764591in}}%
\pgfpathlineto{\pgfqpoint{1.437373in}{0.702493in}}%
\pgfpathlineto{\pgfqpoint{1.438477in}{0.743120in}}%
\pgfpathlineto{\pgfqpoint{1.439984in}{0.710493in}}%
\pgfpathlineto{\pgfqpoint{1.441590in}{0.732853in}}%
\pgfpathlineto{\pgfqpoint{1.443297in}{0.678747in}}%
\pgfpathlineto{\pgfqpoint{1.445105in}{0.721444in}}%
\pgfpathlineto{\pgfqpoint{1.446310in}{0.704742in}}%
\pgfpathlineto{\pgfqpoint{1.448619in}{0.706572in}}%
\pgfpathlineto{\pgfqpoint{1.452736in}{0.792912in}}%
\pgfpathlineto{\pgfqpoint{1.454644in}{0.799048in}}%
\pgfpathlineto{\pgfqpoint{1.456954in}{0.768328in}}%
\pgfpathlineto{\pgfqpoint{1.457657in}{0.787370in}}%
\pgfpathlineto{\pgfqpoint{1.460469in}{0.801491in}}%
\pgfpathlineto{\pgfqpoint{1.461674in}{0.761410in}}%
\pgfpathlineto{\pgfqpoint{1.462577in}{0.811372in}}%
\pgfpathlineto{\pgfqpoint{1.464284in}{0.814237in}}%
\pgfpathlineto{\pgfqpoint{1.465489in}{0.768245in}}%
\pgfpathlineto{\pgfqpoint{1.467297in}{0.793749in}}%
\pgfpathlineto{\pgfqpoint{1.469707in}{0.795745in}}%
\pgfpathlineto{\pgfqpoint{1.470912in}{0.829466in}}%
\pgfpathlineto{\pgfqpoint{1.473422in}{0.824753in}}%
\pgfpathlineto{\pgfqpoint{1.474025in}{0.832572in}}%
\pgfpathlineto{\pgfqpoint{1.476134in}{0.902358in}}%
\pgfpathlineto{\pgfqpoint{1.478041in}{0.854129in}}%
\pgfpathlineto{\pgfqpoint{1.479749in}{0.881071in}}%
\pgfpathlineto{\pgfqpoint{1.480652in}{0.848032in}}%
\pgfpathlineto{\pgfqpoint{1.482359in}{0.857490in}}%
\pgfpathlineto{\pgfqpoint{1.484267in}{0.833707in}}%
\pgfpathlineto{\pgfqpoint{1.485171in}{0.864612in}}%
\pgfpathlineto{\pgfqpoint{1.486878in}{0.789323in}}%
\pgfpathlineto{\pgfqpoint{1.488585in}{0.807767in}}%
\pgfpathlineto{\pgfqpoint{1.489589in}{0.773407in}}%
\pgfpathlineto{\pgfqpoint{1.491698in}{0.773683in}}%
\pgfpathlineto{\pgfqpoint{1.493104in}{0.822970in}}%
\pgfpathlineto{\pgfqpoint{1.495213in}{0.848407in}}%
\pgfpathlineto{\pgfqpoint{1.497322in}{0.790273in}}%
\pgfpathlineto{\pgfqpoint{1.498225in}{0.814808in}}%
\pgfpathlineto{\pgfqpoint{1.500635in}{0.796700in}}%
\pgfpathlineto{\pgfqpoint{1.501439in}{0.818036in}}%
\pgfpathlineto{\pgfqpoint{1.502744in}{0.789827in}}%
\pgfpathlineto{\pgfqpoint{1.504351in}{0.785875in}}%
\pgfpathlineto{\pgfqpoint{1.505857in}{0.851034in}}%
\pgfpathlineto{\pgfqpoint{1.507464in}{0.842589in}}%
\pgfpathlineto{\pgfqpoint{1.508869in}{0.884412in}}%
\pgfpathlineto{\pgfqpoint{1.510476in}{0.868907in}}%
\pgfpathlineto{\pgfqpoint{1.512284in}{0.897647in}}%
\pgfpathlineto{\pgfqpoint{1.513890in}{0.872677in}}%
\pgfpathlineto{\pgfqpoint{1.515798in}{0.907878in}}%
\pgfpathlineto{\pgfqpoint{1.518409in}{0.872228in}}%
\pgfpathlineto{\pgfqpoint{1.519514in}{0.879258in}}%
\pgfpathlineto{\pgfqpoint{1.520819in}{0.859689in}}%
\pgfpathlineto{\pgfqpoint{1.523129in}{0.851648in}}%
\pgfpathlineto{\pgfqpoint{1.524032in}{0.874145in}}%
\pgfpathlineto{\pgfqpoint{1.525237in}{0.864680in}}%
\pgfpathlineto{\pgfqpoint{1.527346in}{0.892339in}}%
\pgfpathlineto{\pgfqpoint{1.528551in}{0.859506in}}%
\pgfpathlineto{\pgfqpoint{1.531764in}{0.918549in}}%
\pgfpathlineto{\pgfqpoint{1.533672in}{0.891412in}}%
\pgfpathlineto{\pgfqpoint{1.535982in}{0.916923in}}%
\pgfpathlineto{\pgfqpoint{1.537087in}{0.889137in}}%
\pgfpathlineto{\pgfqpoint{1.538292in}{0.922504in}}%
\pgfpathlineto{\pgfqpoint{1.540501in}{0.891754in}}%
\pgfpathlineto{\pgfqpoint{1.541103in}{0.893004in}}%
\pgfpathlineto{\pgfqpoint{1.543312in}{0.949599in}}%
\pgfpathlineto{\pgfqpoint{1.544216in}{0.938902in}}%
\pgfpathlineto{\pgfqpoint{1.548333in}{1.003202in}}%
\pgfpathlineto{\pgfqpoint{1.549538in}{0.974690in}}%
\pgfpathlineto{\pgfqpoint{1.553756in}{1.026721in}}%
\pgfpathlineto{\pgfqpoint{1.555162in}{0.984015in}}%
\pgfpathlineto{\pgfqpoint{1.555463in}{1.001669in}}%
\pgfpathlineto{\pgfqpoint{1.557973in}{1.013919in}}%
\pgfpathlineto{\pgfqpoint{1.558777in}{0.991117in}}%
\pgfpathlineto{\pgfqpoint{1.560584in}{1.006508in}}%
\pgfpathlineto{\pgfqpoint{1.561889in}{0.953867in}}%
\pgfpathlineto{\pgfqpoint{1.563697in}{0.927154in}}%
\pgfpathlineto{\pgfqpoint{1.565304in}{0.940964in}}%
\pgfpathlineto{\pgfqpoint{1.566810in}{0.920784in}}%
\pgfpathlineto{\pgfqpoint{1.569220in}{0.906119in}}%
\pgfpathlineto{\pgfqpoint{1.569923in}{0.937940in}}%
\pgfpathlineto{\pgfqpoint{1.572232in}{0.918363in}}%
\pgfpathlineto{\pgfqpoint{1.575847in}{0.989569in}}%
\pgfpathlineto{\pgfqpoint{1.577454in}{0.946394in}}%
\pgfpathlineto{\pgfqpoint{1.579864in}{0.976966in}}%
\pgfpathlineto{\pgfqpoint{1.581471in}{0.939886in}}%
\pgfpathlineto{\pgfqpoint{1.583680in}{0.951653in}}%
\pgfpathlineto{\pgfqpoint{1.584383in}{0.987396in}}%
\pgfpathlineto{\pgfqpoint{1.586190in}{0.985869in}}%
\pgfpathlineto{\pgfqpoint{1.587596in}{1.043182in}}%
\pgfpathlineto{\pgfqpoint{1.590508in}{1.033220in}}%
\pgfpathlineto{\pgfqpoint{1.592115in}{1.073534in}}%
\pgfpathlineto{\pgfqpoint{1.592416in}{1.061294in}}%
\pgfpathlineto{\pgfqpoint{1.595429in}{1.046704in}}%
\pgfpathlineto{\pgfqpoint{1.596835in}{1.007669in}}%
\pgfpathlineto{\pgfqpoint{1.597939in}{1.026317in}}%
\pgfpathlineto{\pgfqpoint{1.601454in}{0.952448in}}%
\pgfpathlineto{\pgfqpoint{1.603060in}{0.971765in}}%
\pgfpathlineto{\pgfqpoint{1.604265in}{0.947832in}}%
\pgfpathlineto{\pgfqpoint{1.606776in}{0.974973in}}%
\pgfpathlineto{\pgfqpoint{1.608382in}{0.928875in}}%
\pgfpathlineto{\pgfqpoint{1.609387in}{0.955388in}}%
\pgfpathlineto{\pgfqpoint{1.611194in}{0.954639in}}%
\pgfpathlineto{\pgfqpoint{1.612700in}{0.990138in}}%
\pgfpathlineto{\pgfqpoint{1.614407in}{0.941419in}}%
\pgfpathlineto{\pgfqpoint{1.615412in}{0.974365in}}%
\pgfpathlineto{\pgfqpoint{1.617320in}{0.958645in}}%
\pgfpathlineto{\pgfqpoint{1.618324in}{0.978972in}}%
\pgfpathlineto{\pgfqpoint{1.620633in}{0.944532in}}%
\pgfpathlineto{\pgfqpoint{1.622441in}{0.963203in}}%
\pgfpathlineto{\pgfqpoint{1.623244in}{0.936031in}}%
\pgfpathlineto{\pgfqpoint{1.625353in}{0.939660in}}%
\pgfpathlineto{\pgfqpoint{1.628165in}{1.005843in}}%
\pgfpathlineto{\pgfqpoint{1.629972in}{0.985715in}}%
\pgfpathlineto{\pgfqpoint{1.632683in}{1.006987in}}%
\pgfpathlineto{\pgfqpoint{1.634692in}{0.954303in}}%
\pgfpathlineto{\pgfqpoint{1.636399in}{0.974921in}}%
\pgfpathlineto{\pgfqpoint{1.638407in}{0.944780in}}%
\pgfpathlineto{\pgfqpoint{1.639010in}{0.974739in}}%
\pgfpathlineto{\pgfqpoint{1.640616in}{0.959250in}}%
\pgfpathlineto{\pgfqpoint{1.642424in}{0.994994in}}%
\pgfpathlineto{\pgfqpoint{1.644633in}{0.968476in}}%
\pgfpathlineto{\pgfqpoint{1.646641in}{0.996288in}}%
\pgfpathlineto{\pgfqpoint{1.647244in}{0.973017in}}%
\pgfpathlineto{\pgfqpoint{1.648951in}{0.950976in}}%
\pgfpathlineto{\pgfqpoint{1.651562in}{0.956764in}}%
\pgfpathlineto{\pgfqpoint{1.653369in}{0.910632in}}%
\pgfpathlineto{\pgfqpoint{1.654072in}{0.931619in}}%
\pgfpathlineto{\pgfqpoint{1.655378in}{0.917697in}}%
\pgfpathlineto{\pgfqpoint{1.656683in}{0.936941in}}%
\pgfpathlineto{\pgfqpoint{1.659294in}{0.888459in}}%
\pgfpathlineto{\pgfqpoint{1.660097in}{0.906527in}}%
\pgfpathlineto{\pgfqpoint{1.661704in}{0.885067in}}%
\pgfpathlineto{\pgfqpoint{1.663612in}{0.915885in}}%
\pgfpathlineto{\pgfqpoint{1.665720in}{0.896680in}}%
\pgfpathlineto{\pgfqpoint{1.667628in}{0.956787in}}%
\pgfpathlineto{\pgfqpoint{1.668231in}{0.940134in}}%
\pgfpathlineto{\pgfqpoint{1.670340in}{0.952761in}}%
\pgfpathlineto{\pgfqpoint{1.671344in}{0.928538in}}%
\pgfpathlineto{\pgfqpoint{1.673653in}{0.953887in}}%
\pgfpathlineto{\pgfqpoint{1.674959in}{0.904553in}}%
\pgfpathlineto{\pgfqpoint{1.676867in}{0.904802in}}%
\pgfpathlineto{\pgfqpoint{1.678674in}{0.948136in}}%
\pgfpathlineto{\pgfqpoint{1.679779in}{0.930173in}}%
\pgfpathlineto{\pgfqpoint{1.680984in}{0.947184in}}%
\pgfpathlineto{\pgfqpoint{1.683193in}{0.913650in}}%
\pgfpathlineto{\pgfqpoint{1.685000in}{0.928024in}}%
\pgfpathlineto{\pgfqpoint{1.686607in}{0.895741in}}%
\pgfpathlineto{\pgfqpoint{1.688314in}{0.916031in}}%
\pgfpathlineto{\pgfqpoint{1.691427in}{0.823187in}}%
\pgfpathlineto{\pgfqpoint{1.692532in}{0.808235in}}%
\pgfpathlineto{\pgfqpoint{1.696749in}{0.891675in}}%
\pgfpathlineto{\pgfqpoint{1.697854in}{0.858605in}}%
\pgfpathlineto{\pgfqpoint{1.698657in}{0.876539in}}%
\pgfpathlineto{\pgfqpoint{1.700063in}{0.849061in}}%
\pgfpathlineto{\pgfqpoint{1.702172in}{0.855250in}}%
\pgfpathlineto{\pgfqpoint{1.704682in}{0.898004in}}%
\pgfpathlineto{\pgfqpoint{1.705787in}{0.858958in}}%
\pgfpathlineto{\pgfqpoint{1.706490in}{0.885159in}}%
\pgfpathlineto{\pgfqpoint{1.709402in}{0.917078in}}%
\pgfpathlineto{\pgfqpoint{1.710506in}{0.890967in}}%
\pgfpathlineto{\pgfqpoint{1.711812in}{0.913077in}}%
\pgfpathlineto{\pgfqpoint{1.713017in}{0.881290in}}%
\pgfpathlineto{\pgfqpoint{1.714523in}{0.908561in}}%
\pgfpathlineto{\pgfqpoint{1.716330in}{0.899777in}}%
\pgfpathlineto{\pgfqpoint{1.718339in}{0.924898in}}%
\pgfpathlineto{\pgfqpoint{1.720548in}{0.912912in}}%
\pgfpathlineto{\pgfqpoint{1.722155in}{0.931263in}}%
\pgfpathlineto{\pgfqpoint{1.722858in}{0.901759in}}%
\pgfpathlineto{\pgfqpoint{1.724966in}{0.931536in}}%
\pgfpathlineto{\pgfqpoint{1.725870in}{0.895038in}}%
\pgfpathlineto{\pgfqpoint{1.728180in}{0.907900in}}%
\pgfpathlineto{\pgfqpoint{1.729585in}{0.902454in}}%
\pgfpathlineto{\pgfqpoint{1.731795in}{0.847440in}}%
\pgfpathlineto{\pgfqpoint{1.733200in}{0.875344in}}%
\pgfpathlineto{\pgfqpoint{1.734807in}{0.835755in}}%
\pgfpathlineto{\pgfqpoint{1.736313in}{0.863947in}}%
\pgfpathlineto{\pgfqpoint{1.737016in}{0.842685in}}%
\pgfpathlineto{\pgfqpoint{1.739326in}{0.892110in}}%
\pgfpathlineto{\pgfqpoint{1.741736in}{0.852284in}}%
\pgfpathlineto{\pgfqpoint{1.743041in}{0.877732in}}%
\pgfpathlineto{\pgfqpoint{1.744949in}{0.844746in}}%
\pgfpathlineto{\pgfqpoint{1.746154in}{0.873503in}}%
\pgfpathlineto{\pgfqpoint{1.747460in}{0.875588in}}%
\pgfpathlineto{\pgfqpoint{1.748765in}{0.854126in}}%
\pgfpathlineto{\pgfqpoint{1.749870in}{0.787937in}}%
\pgfpathlineto{\pgfqpoint{1.752280in}{0.754497in}}%
\pgfpathlineto{\pgfqpoint{1.753786in}{0.826013in}}%
\pgfpathlineto{\pgfqpoint{1.754690in}{0.804832in}}%
\pgfpathlineto{\pgfqpoint{1.756798in}{0.813710in}}%
\pgfpathlineto{\pgfqpoint{1.758305in}{0.810795in}}%
\pgfpathlineto{\pgfqpoint{1.760614in}{0.766831in}}%
\pgfpathlineto{\pgfqpoint{1.761116in}{0.780399in}}%
\pgfpathlineto{\pgfqpoint{1.762723in}{0.750704in}}%
\pgfpathlineto{\pgfqpoint{1.765736in}{0.771429in}}%
\pgfpathlineto{\pgfqpoint{1.767041in}{0.727234in}}%
\pgfpathlineto{\pgfqpoint{1.768848in}{0.714137in}}%
\pgfpathlineto{\pgfqpoint{1.770556in}{0.673107in}}%
\pgfpathlineto{\pgfqpoint{1.771761in}{0.702627in}}%
\pgfpathlineto{\pgfqpoint{1.772664in}{0.664298in}}%
\pgfpathlineto{\pgfqpoint{1.774673in}{0.675345in}}%
\pgfpathlineto{\pgfqpoint{1.775576in}{0.703210in}}%
\pgfpathlineto{\pgfqpoint{1.778288in}{0.695334in}}%
\pgfpathlineto{\pgfqpoint{1.779091in}{0.715745in}}%
\pgfpathlineto{\pgfqpoint{1.780698in}{0.686688in}}%
\pgfpathlineto{\pgfqpoint{1.782003in}{0.713226in}}%
\pgfpathlineto{\pgfqpoint{1.784313in}{0.673696in}}%
\pgfpathlineto{\pgfqpoint{1.785216in}{0.704127in}}%
\pgfpathlineto{\pgfqpoint{1.787526in}{0.671850in}}%
\pgfpathlineto{\pgfqpoint{1.789032in}{0.707869in}}%
\pgfpathlineto{\pgfqpoint{1.791944in}{0.660098in}}%
\pgfpathlineto{\pgfqpoint{1.794254in}{0.701232in}}%
\pgfpathlineto{\pgfqpoint{1.796162in}{0.662994in}}%
\pgfpathlineto{\pgfqpoint{1.798371in}{0.713904in}}%
\pgfpathlineto{\pgfqpoint{1.800178in}{0.699538in}}%
\pgfpathlineto{\pgfqpoint{1.801584in}{0.724509in}}%
\pgfpathlineto{\pgfqpoint{1.802890in}{0.688810in}}%
\pgfpathlineto{\pgfqpoint{1.804597in}{0.749592in}}%
\pgfpathlineto{\pgfqpoint{1.806706in}{0.732758in}}%
\pgfpathlineto{\pgfqpoint{1.808111in}{0.670335in}}%
\pgfpathlineto{\pgfqpoint{1.809316in}{0.699104in}}%
\pgfpathlineto{\pgfqpoint{1.811224in}{0.701121in}}%
\pgfpathlineto{\pgfqpoint{1.812630in}{0.748043in}}%
\pgfpathlineto{\pgfqpoint{1.815442in}{0.726855in}}%
\pgfpathlineto{\pgfqpoint{1.816948in}{0.683713in}}%
\pgfpathlineto{\pgfqpoint{1.818756in}{0.689411in}}%
\pgfpathlineto{\pgfqpoint{1.820161in}{0.633556in}}%
\pgfpathlineto{\pgfqpoint{1.821969in}{0.608616in}}%
\pgfpathlineto{\pgfqpoint{1.822371in}{0.639243in}}%
\pgfpathlineto{\pgfqpoint{1.824379in}{0.653536in}}%
\pgfpathlineto{\pgfqpoint{1.825484in}{0.612669in}}%
\pgfpathlineto{\pgfqpoint{1.827693in}{0.667109in}}%
\pgfpathlineto{\pgfqpoint{1.830002in}{0.633577in}}%
\pgfpathlineto{\pgfqpoint{1.832312in}{0.703980in}}%
\pgfpathlineto{\pgfqpoint{1.834320in}{0.687472in}}%
\pgfpathlineto{\pgfqpoint{1.836429in}{0.713556in}}%
\pgfpathlineto{\pgfqpoint{1.837634in}{0.674905in}}%
\pgfpathlineto{\pgfqpoint{1.839441in}{0.690910in}}%
\pgfpathlineto{\pgfqpoint{1.840144in}{0.739780in}}%
\pgfpathlineto{\pgfqpoint{1.842052in}{0.770399in}}%
\pgfpathlineto{\pgfqpoint{1.844563in}{0.728110in}}%
\pgfpathlineto{\pgfqpoint{1.845969in}{0.746472in}}%
\pgfpathlineto{\pgfqpoint{1.846973in}{0.718447in}}%
\pgfpathlineto{\pgfqpoint{1.848178in}{0.759496in}}%
\pgfpathlineto{\pgfqpoint{1.849784in}{0.746876in}}%
\pgfpathlineto{\pgfqpoint{1.851391in}{0.769939in}}%
\pgfpathlineto{\pgfqpoint{1.854203in}{0.761469in}}%
\pgfpathlineto{\pgfqpoint{1.855408in}{0.719810in}}%
\pgfpathlineto{\pgfqpoint{1.856211in}{0.752916in}}%
\pgfpathlineto{\pgfqpoint{1.858019in}{0.721419in}}%
\pgfpathlineto{\pgfqpoint{1.859424in}{0.755207in}}%
\pgfpathlineto{\pgfqpoint{1.861734in}{0.728233in}}%
\pgfpathlineto{\pgfqpoint{1.863140in}{0.736128in}}%
\pgfpathlineto{\pgfqpoint{1.864546in}{0.714169in}}%
\pgfpathlineto{\pgfqpoint{1.865951in}{0.715655in}}%
\pgfpathlineto{\pgfqpoint{1.868261in}{0.667608in}}%
\pgfpathlineto{\pgfqpoint{1.869868in}{0.682825in}}%
\pgfpathlineto{\pgfqpoint{1.871474in}{0.732849in}}%
\pgfpathlineto{\pgfqpoint{1.872278in}{0.685394in}}%
\pgfpathlineto{\pgfqpoint{1.875089in}{0.633366in}}%
\pgfpathlineto{\pgfqpoint{1.875792in}{0.666583in}}%
\pgfpathlineto{\pgfqpoint{1.877098in}{0.638210in}}%
\pgfpathlineto{\pgfqpoint{1.879608in}{0.664546in}}%
\pgfpathlineto{\pgfqpoint{1.880612in}{0.646551in}}%
\pgfpathlineto{\pgfqpoint{1.882119in}{0.690567in}}%
\pgfpathlineto{\pgfqpoint{1.883223in}{0.677983in}}%
\pgfpathlineto{\pgfqpoint{1.885734in}{0.726374in}}%
\pgfpathlineto{\pgfqpoint{1.887139in}{0.709147in}}%
\pgfpathlineto{\pgfqpoint{1.890051in}{0.748988in}}%
\pgfpathlineto{\pgfqpoint{1.892461in}{0.735042in}}%
\pgfpathlineto{\pgfqpoint{1.894068in}{0.762783in}}%
\pgfpathlineto{\pgfqpoint{1.894771in}{0.714070in}}%
\pgfpathlineto{\pgfqpoint{1.897583in}{0.745831in}}%
\pgfpathlineto{\pgfqpoint{1.898185in}{0.731277in}}%
\pgfpathlineto{\pgfqpoint{1.900495in}{0.743227in}}%
\pgfpathlineto{\pgfqpoint{1.902302in}{0.795378in}}%
\pgfpathlineto{\pgfqpoint{1.904009in}{0.756015in}}%
\pgfpathlineto{\pgfqpoint{1.905616in}{0.771268in}}%
\pgfpathlineto{\pgfqpoint{1.906118in}{0.749458in}}%
\pgfpathlineto{\pgfqpoint{1.909131in}{0.819571in}}%
\pgfpathlineto{\pgfqpoint{1.912344in}{0.748919in}}%
\pgfpathlineto{\pgfqpoint{1.914252in}{0.773291in}}%
\pgfpathlineto{\pgfqpoint{1.915959in}{0.753436in}}%
\pgfpathlineto{\pgfqpoint{1.917365in}{0.779093in}}%
\pgfpathlineto{\pgfqpoint{1.919674in}{0.749928in}}%
\pgfpathlineto{\pgfqpoint{1.920177in}{0.762923in}}%
\pgfpathlineto{\pgfqpoint{1.923089in}{0.726455in}}%
\pgfpathlineto{\pgfqpoint{1.924595in}{0.758595in}}%
\pgfpathlineto{\pgfqpoint{1.925298in}{0.741890in}}%
\pgfpathlineto{\pgfqpoint{1.926804in}{0.752245in}}%
\pgfpathlineto{\pgfqpoint{1.928310in}{0.725859in}}%
\pgfpathlineto{\pgfqpoint{1.930419in}{0.745444in}}%
\pgfpathlineto{\pgfqpoint{1.932427in}{0.714709in}}%
\pgfpathlineto{\pgfqpoint{1.934134in}{0.662444in}}%
\pgfpathlineto{\pgfqpoint{1.936143in}{0.651644in}}%
\pgfpathlineto{\pgfqpoint{1.937147in}{0.691141in}}%
\pgfpathlineto{\pgfqpoint{1.939055in}{0.718044in}}%
\pgfpathlineto{\pgfqpoint{1.940059in}{0.692417in}}%
\pgfpathlineto{\pgfqpoint{1.941364in}{0.704660in}}%
\pgfpathlineto{\pgfqpoint{1.943072in}{0.757742in}}%
\pgfpathlineto{\pgfqpoint{1.944578in}{0.725780in}}%
\pgfpathlineto{\pgfqpoint{1.947389in}{0.726706in}}%
\pgfpathlineto{\pgfqpoint{1.948996in}{0.751883in}}%
\pgfpathlineto{\pgfqpoint{1.950402in}{0.733273in}}%
\pgfpathlineto{\pgfqpoint{1.951808in}{0.767070in}}%
\pgfpathlineto{\pgfqpoint{1.952912in}{0.746868in}}%
\pgfpathlineto{\pgfqpoint{1.954519in}{0.781616in}}%
\pgfpathlineto{\pgfqpoint{1.956025in}{0.753822in}}%
\pgfpathlineto{\pgfqpoint{1.957230in}{0.787577in}}%
\pgfpathlineto{\pgfqpoint{1.959038in}{0.771298in}}%
\pgfpathlineto{\pgfqpoint{1.961849in}{0.786781in}}%
\pgfpathlineto{\pgfqpoint{1.962151in}{0.766369in}}%
\pgfpathlineto{\pgfqpoint{1.964561in}{0.761356in}}%
\pgfpathlineto{\pgfqpoint{1.965464in}{0.734045in}}%
\pgfpathlineto{\pgfqpoint{1.967975in}{0.707160in}}%
\pgfpathlineto{\pgfqpoint{1.968879in}{0.733075in}}%
\pgfpathlineto{\pgfqpoint{1.970385in}{0.685888in}}%
\pgfpathlineto{\pgfqpoint{1.973598in}{0.723449in}}%
\pgfpathlineto{\pgfqpoint{1.974904in}{0.730953in}}%
\pgfpathlineto{\pgfqpoint{1.978920in}{0.653650in}}%
\pgfpathlineto{\pgfqpoint{1.984042in}{0.783598in}}%
\pgfpathlineto{\pgfqpoint{1.985347in}{0.807724in}}%
\pgfpathlineto{\pgfqpoint{1.987355in}{0.817600in}}%
\pgfpathlineto{\pgfqpoint{1.988259in}{0.780046in}}%
\pgfpathlineto{\pgfqpoint{1.989364in}{0.807548in}}%
\pgfpathlineto{\pgfqpoint{1.991071in}{0.790334in}}%
\pgfpathlineto{\pgfqpoint{1.992677in}{0.806072in}}%
\pgfpathlineto{\pgfqpoint{1.994184in}{0.786636in}}%
\pgfpathlineto{\pgfqpoint{1.996192in}{0.831888in}}%
\pgfpathlineto{\pgfqpoint{1.997899in}{0.818424in}}%
\pgfpathlineto{\pgfqpoint{1.999907in}{0.881408in}}%
\pgfpathlineto{\pgfqpoint{2.001916in}{0.853554in}}%
\pgfpathlineto{\pgfqpoint{2.002217in}{0.873367in}}%
\pgfpathlineto{\pgfqpoint{2.004828in}{0.891369in}}%
\pgfpathlineto{\pgfqpoint{2.005832in}{0.868190in}}%
\pgfpathlineto{\pgfqpoint{2.007137in}{0.917343in}}%
\pgfpathlineto{\pgfqpoint{2.008744in}{0.882464in}}%
\pgfpathlineto{\pgfqpoint{2.010953in}{0.922508in}}%
\pgfpathlineto{\pgfqpoint{2.012460in}{0.873967in}}%
\pgfpathlineto{\pgfqpoint{2.014568in}{0.849402in}}%
\pgfpathlineto{\pgfqpoint{2.016275in}{0.882418in}}%
\pgfpathlineto{\pgfqpoint{2.018083in}{0.853289in}}%
\pgfpathlineto{\pgfqpoint{2.018987in}{0.872247in}}%
\pgfpathlineto{\pgfqpoint{2.021095in}{0.837927in}}%
\pgfpathlineto{\pgfqpoint{2.022100in}{0.881142in}}%
\pgfpathlineto{\pgfqpoint{2.023104in}{0.864128in}}%
\pgfpathlineto{\pgfqpoint{2.025212in}{0.859681in}}%
\pgfpathlineto{\pgfqpoint{2.028426in}{0.909856in}}%
\pgfpathlineto{\pgfqpoint{2.030133in}{0.892938in}}%
\pgfpathlineto{\pgfqpoint{2.031037in}{0.939134in}}%
\pgfpathlineto{\pgfqpoint{2.032844in}{0.948852in}}%
\pgfpathlineto{\pgfqpoint{2.035756in}{0.860870in}}%
\pgfpathlineto{\pgfqpoint{2.037162in}{0.870833in}}%
\pgfpathlineto{\pgfqpoint{2.038267in}{0.914710in}}%
\pgfpathlineto{\pgfqpoint{2.040375in}{0.914100in}}%
\pgfpathlineto{\pgfqpoint{2.042384in}{0.854339in}}%
\pgfpathlineto{\pgfqpoint{2.046802in}{0.992337in}}%
\pgfpathlineto{\pgfqpoint{2.047304in}{0.995098in}}%
\pgfpathlineto{\pgfqpoint{2.049112in}{0.950859in}}%
\pgfpathlineto{\pgfqpoint{2.051321in}{0.995100in}}%
\pgfpathlineto{\pgfqpoint{2.052225in}{0.950454in}}%
\pgfpathlineto{\pgfqpoint{2.053731in}{0.977475in}}%
\pgfpathlineto{\pgfqpoint{2.055438in}{0.931350in}}%
\pgfpathlineto{\pgfqpoint{2.057747in}{0.917556in}}%
\pgfpathlineto{\pgfqpoint{2.059655in}{0.940730in}}%
\pgfpathlineto{\pgfqpoint{2.060258in}{0.977212in}}%
\pgfpathlineto{\pgfqpoint{2.062567in}{1.019131in}}%
\pgfpathlineto{\pgfqpoint{2.063572in}{1.011488in}}%
\pgfpathlineto{\pgfqpoint{2.064877in}{1.050274in}}%
\pgfpathlineto{\pgfqpoint{2.067287in}{1.054274in}}%
\pgfpathlineto{\pgfqpoint{2.069396in}{0.994116in}}%
\pgfpathlineto{\pgfqpoint{2.070099in}{1.035483in}}%
\pgfpathlineto{\pgfqpoint{2.072107in}{0.976016in}}%
\pgfpathlineto{\pgfqpoint{2.073011in}{1.033934in}}%
\pgfpathlineto{\pgfqpoint{2.075722in}{1.049594in}}%
\pgfpathlineto{\pgfqpoint{2.077128in}{1.022735in}}%
\pgfpathlineto{\pgfqpoint{2.078132in}{1.054634in}}%
\pgfpathlineto{\pgfqpoint{2.080140in}{1.032318in}}%
\pgfpathlineto{\pgfqpoint{2.080944in}{1.061022in}}%
\pgfpathlineto{\pgfqpoint{2.083053in}{1.024424in}}%
\pgfpathlineto{\pgfqpoint{2.084358in}{1.038877in}}%
\pgfpathlineto{\pgfqpoint{2.088575in}{1.031322in}}%
\pgfpathlineto{\pgfqpoint{2.089178in}{0.995612in}}%
\pgfpathlineto{\pgfqpoint{2.091387in}{0.996096in}}%
\pgfpathlineto{\pgfqpoint{2.092391in}{0.957582in}}%
\pgfpathlineto{\pgfqpoint{2.095103in}{1.034971in}}%
\pgfpathlineto{\pgfqpoint{2.095705in}{1.008889in}}%
\pgfpathlineto{\pgfqpoint{2.098818in}{0.992236in}}%
\pgfpathlineto{\pgfqpoint{2.100425in}{1.047576in}}%
\pgfpathlineto{\pgfqpoint{2.102132in}{1.043634in}}%
\pgfpathlineto{\pgfqpoint{2.104943in}{1.111677in}}%
\pgfpathlineto{\pgfqpoint{2.108056in}{1.108287in}}%
\pgfpathlineto{\pgfqpoint{2.109161in}{1.076777in}}%
\pgfpathlineto{\pgfqpoint{2.110065in}{1.098977in}}%
\pgfpathlineto{\pgfqpoint{2.111671in}{1.052237in}}%
\pgfpathlineto{\pgfqpoint{2.113579in}{1.083155in}}%
\pgfpathlineto{\pgfqpoint{2.114985in}{1.055119in}}%
\pgfpathlineto{\pgfqpoint{2.116391in}{1.114132in}}%
\pgfpathlineto{\pgfqpoint{2.118901in}{1.104964in}}%
\pgfpathlineto{\pgfqpoint{2.119504in}{1.078705in}}%
\pgfpathlineto{\pgfqpoint{2.121010in}{1.066332in}}%
\pgfpathlineto{\pgfqpoint{2.123219in}{1.098072in}}%
\pgfpathlineto{\pgfqpoint{2.124625in}{1.100493in}}%
\pgfpathlineto{\pgfqpoint{2.126834in}{1.057208in}}%
\pgfpathlineto{\pgfqpoint{2.128340in}{1.084077in}}%
\pgfpathlineto{\pgfqpoint{2.129345in}{1.059883in}}%
\pgfpathlineto{\pgfqpoint{2.130750in}{1.111782in}}%
\pgfpathlineto{\pgfqpoint{2.133261in}{1.110320in}}%
\pgfpathlineto{\pgfqpoint{2.134667in}{1.087989in}}%
\pgfpathlineto{\pgfqpoint{2.137780in}{1.008509in}}%
\pgfpathlineto{\pgfqpoint{2.140089in}{1.087339in}}%
\pgfpathlineto{\pgfqpoint{2.140390in}{1.072291in}}%
\pgfpathlineto{\pgfqpoint{2.143001in}{1.104976in}}%
\pgfpathlineto{\pgfqpoint{2.143704in}{1.068998in}}%
\pgfpathlineto{\pgfqpoint{2.146315in}{1.099256in}}%
\pgfpathlineto{\pgfqpoint{2.148022in}{1.084112in}}%
\pgfpathlineto{\pgfqpoint{2.149328in}{1.111518in}}%
\pgfpathlineto{\pgfqpoint{2.149930in}{1.092185in}}%
\pgfpathlineto{\pgfqpoint{2.152641in}{1.106852in}}%
\pgfpathlineto{\pgfqpoint{2.154348in}{1.075782in}}%
\pgfpathlineto{\pgfqpoint{2.154750in}{1.098965in}}%
\pgfpathlineto{\pgfqpoint{2.157763in}{1.098063in}}%
\pgfpathlineto{\pgfqpoint{2.159369in}{1.060627in}}%
\pgfpathlineto{\pgfqpoint{2.160876in}{1.095286in}}%
\pgfpathlineto{\pgfqpoint{2.162482in}{1.104816in}}%
\pgfpathlineto{\pgfqpoint{2.164290in}{1.084471in}}%
\pgfpathlineto{\pgfqpoint{2.166901in}{1.146568in}}%
\pgfpathlineto{\pgfqpoint{2.168507in}{1.106111in}}%
\pgfpathlineto{\pgfqpoint{2.169813in}{1.139359in}}%
\pgfpathlineto{\pgfqpoint{2.170917in}{1.120859in}}%
\pgfpathlineto{\pgfqpoint{2.172524in}{1.125006in}}%
\pgfpathlineto{\pgfqpoint{2.175536in}{1.179375in}}%
\pgfpathlineto{\pgfqpoint{2.176340in}{1.155904in}}%
\pgfpathlineto{\pgfqpoint{2.177846in}{1.197888in}}%
\pgfpathlineto{\pgfqpoint{2.179151in}{1.181507in}}%
\pgfpathlineto{\pgfqpoint{2.181762in}{1.206155in}}%
\pgfpathlineto{\pgfqpoint{2.183871in}{1.154919in}}%
\pgfpathlineto{\pgfqpoint{2.185578in}{1.186917in}}%
\pgfpathlineto{\pgfqpoint{2.187285in}{1.125246in}}%
\pgfpathlineto{\pgfqpoint{2.189293in}{1.160679in}}%
\pgfpathlineto{\pgfqpoint{2.191402in}{1.133400in}}%
\pgfpathlineto{\pgfqpoint{2.192206in}{1.157638in}}%
\pgfpathlineto{\pgfqpoint{2.194415in}{1.103866in}}%
\pgfpathlineto{\pgfqpoint{2.196122in}{1.125116in}}%
\pgfpathlineto{\pgfqpoint{2.197728in}{1.098282in}}%
\pgfpathlineto{\pgfqpoint{2.199737in}{1.167868in}}%
\pgfpathlineto{\pgfqpoint{2.201544in}{1.161704in}}%
\pgfpathlineto{\pgfqpoint{2.202950in}{1.197011in}}%
\pgfpathlineto{\pgfqpoint{2.205661in}{1.223174in}}%
\pgfpathlineto{\pgfqpoint{2.206163in}{1.191367in}}%
\pgfpathlineto{\pgfqpoint{2.207770in}{1.185786in}}%
\pgfpathlineto{\pgfqpoint{2.209778in}{1.136783in}}%
\pgfpathlineto{\pgfqpoint{2.211486in}{1.151871in}}%
\pgfpathlineto{\pgfqpoint{2.212992in}{1.137764in}}%
\pgfpathlineto{\pgfqpoint{2.215301in}{1.069810in}}%
\pgfpathlineto{\pgfqpoint{2.216105in}{1.088246in}}%
\pgfpathlineto{\pgfqpoint{2.218515in}{1.078381in}}%
\pgfpathlineto{\pgfqpoint{2.219418in}{1.100430in}}%
\pgfpathlineto{\pgfqpoint{2.221929in}{1.087200in}}%
\pgfpathlineto{\pgfqpoint{2.222632in}{1.115621in}}%
\pgfpathlineto{\pgfqpoint{2.224941in}{1.155072in}}%
\pgfpathlineto{\pgfqpoint{2.227753in}{1.129014in}}%
\pgfpathlineto{\pgfqpoint{2.230364in}{1.194820in}}%
\pgfpathlineto{\pgfqpoint{2.231971in}{1.193783in}}%
\pgfpathlineto{\pgfqpoint{2.234682in}{1.092459in}}%
\pgfpathlineto{\pgfqpoint{2.235083in}{1.107395in}}%
\pgfpathlineto{\pgfqpoint{2.237594in}{1.069199in}}%
\pgfpathlineto{\pgfqpoint{2.239502in}{1.081984in}}%
\pgfpathlineto{\pgfqpoint{2.240908in}{1.116069in}}%
\pgfpathlineto{\pgfqpoint{2.242916in}{1.108756in}}%
\pgfpathlineto{\pgfqpoint{2.243217in}{1.088606in}}%
\pgfpathlineto{\pgfqpoint{2.246230in}{1.065813in}}%
\pgfpathlineto{\pgfqpoint{2.246933in}{1.087966in}}%
\pgfpathlineto{\pgfqpoint{2.249343in}{1.055396in}}%
\pgfpathlineto{\pgfqpoint{2.250548in}{1.072288in}}%
\pgfpathlineto{\pgfqpoint{2.251552in}{1.026070in}}%
\pgfpathlineto{\pgfqpoint{2.252857in}{1.029506in}}%
\pgfpathlineto{\pgfqpoint{2.254966in}{1.080631in}}%
\pgfpathlineto{\pgfqpoint{2.256874in}{1.067588in}}%
\pgfpathlineto{\pgfqpoint{2.257677in}{1.036677in}}%
\pgfpathlineto{\pgfqpoint{2.259485in}{1.021429in}}%
\pgfpathlineto{\pgfqpoint{2.261895in}{1.040883in}}%
\pgfpathlineto{\pgfqpoint{2.263501in}{1.115450in}}%
\pgfpathlineto{\pgfqpoint{2.264104in}{1.090197in}}%
\pgfpathlineto{\pgfqpoint{2.266715in}{1.099001in}}%
\pgfpathlineto{\pgfqpoint{2.267719in}{1.138077in}}%
\pgfpathlineto{\pgfqpoint{2.269326in}{1.135472in}}%
\pgfpathlineto{\pgfqpoint{2.270631in}{1.106404in}}%
\pgfpathlineto{\pgfqpoint{2.272539in}{1.102261in}}%
\pgfpathlineto{\pgfqpoint{2.276254in}{1.165513in}}%
\pgfpathlineto{\pgfqpoint{2.277560in}{1.138127in}}%
\pgfpathlineto{\pgfqpoint{2.279669in}{1.175828in}}%
\pgfpathlineto{\pgfqpoint{2.280271in}{1.163236in}}%
\pgfpathlineto{\pgfqpoint{2.282179in}{1.200424in}}%
\pgfpathlineto{\pgfqpoint{2.283886in}{1.179650in}}%
\pgfpathlineto{\pgfqpoint{2.285191in}{1.192929in}}%
\pgfpathlineto{\pgfqpoint{2.287501in}{1.158101in}}%
\pgfpathlineto{\pgfqpoint{2.291618in}{1.252204in}}%
\pgfpathlineto{\pgfqpoint{2.292924in}{1.208837in}}%
\pgfpathlineto{\pgfqpoint{2.294530in}{1.230181in}}%
\pgfpathlineto{\pgfqpoint{2.296237in}{1.187048in}}%
\pgfpathlineto{\pgfqpoint{2.298547in}{1.189315in}}%
\pgfpathlineto{\pgfqpoint{2.299953in}{1.241444in}}%
\pgfpathlineto{\pgfqpoint{2.302162in}{1.230870in}}%
\pgfpathlineto{\pgfqpoint{2.302664in}{1.254321in}}%
\pgfpathlineto{\pgfqpoint{2.304371in}{1.220753in}}%
\pgfpathlineto{\pgfqpoint{2.305777in}{1.246369in}}%
\pgfpathlineto{\pgfqpoint{2.308287in}{1.236272in}}%
\pgfpathlineto{\pgfqpoint{2.309894in}{1.194737in}}%
\pgfpathlineto{\pgfqpoint{2.311802in}{1.223731in}}%
\pgfpathlineto{\pgfqpoint{2.313609in}{1.213605in}}%
\pgfpathlineto{\pgfqpoint{2.315015in}{1.154837in}}%
\pgfpathlineto{\pgfqpoint{2.316321in}{1.171124in}}%
\pgfpathlineto{\pgfqpoint{2.317024in}{1.153482in}}%
\pgfpathlineto{\pgfqpoint{2.317024in}{1.153482in}}%
\pgfusepath{stroke}%
\end{pgfscope}%
\begin{pgfscope}%
\pgfsetrectcap%
\pgfsetmiterjoin%
\pgfsetlinewidth{0.803000pt}%
\definecolor{currentstroke}{rgb}{0.000000,0.000000,0.000000}%
\pgfsetstrokecolor{currentstroke}%
\pgfsetdash{}{0pt}%
\pgfpathmoveto{\pgfqpoint{0.589745in}{0.416447in}}%
\pgfpathlineto{\pgfqpoint{0.589745in}{1.789039in}}%
\pgfusepath{stroke}%
\end{pgfscope}%
\begin{pgfscope}%
\pgfsetrectcap%
\pgfsetmiterjoin%
\pgfsetlinewidth{0.803000pt}%
\definecolor{currentstroke}{rgb}{0.000000,0.000000,0.000000}%
\pgfsetstrokecolor{currentstroke}%
\pgfsetdash{}{0pt}%
\pgfpathmoveto{\pgfqpoint{2.399275in}{0.416447in}}%
\pgfpathlineto{\pgfqpoint{2.399275in}{1.789039in}}%
\pgfusepath{stroke}%
\end{pgfscope}%
\begin{pgfscope}%
\pgfsetrectcap%
\pgfsetmiterjoin%
\pgfsetlinewidth{0.803000pt}%
\definecolor{currentstroke}{rgb}{0.000000,0.000000,0.000000}%
\pgfsetstrokecolor{currentstroke}%
\pgfsetdash{}{0pt}%
\pgfpathmoveto{\pgfqpoint{0.589745in}{0.416447in}}%
\pgfpathlineto{\pgfqpoint{2.399275in}{0.416447in}}%
\pgfusepath{stroke}%
\end{pgfscope}%
\begin{pgfscope}%
\pgfsetrectcap%
\pgfsetmiterjoin%
\pgfsetlinewidth{0.803000pt}%
\definecolor{currentstroke}{rgb}{0.000000,0.000000,0.000000}%
\pgfsetstrokecolor{currentstroke}%
\pgfsetdash{}{0pt}%
\pgfpathmoveto{\pgfqpoint{0.589745in}{1.789039in}}%
\pgfpathlineto{\pgfqpoint{2.399275in}{1.789039in}}%
\pgfusepath{stroke}%
\end{pgfscope}%
\begin{pgfscope}%
\pgfsetbuttcap%
\pgfsetmiterjoin%
\definecolor{currentfill}{rgb}{1.000000,1.000000,1.000000}%
\pgfsetfillcolor{currentfill}%
\pgfsetfillopacity{0.800000}%
\pgfsetlinewidth{1.003750pt}%
\definecolor{currentstroke}{rgb}{0.800000,0.800000,0.800000}%
\pgfsetstrokecolor{currentstroke}%
\pgfsetstrokeopacity{0.800000}%
\pgfsetdash{}{0pt}%
\pgfpathmoveto{\pgfqpoint{0.667523in}{1.545261in}}%
\pgfpathlineto{\pgfqpoint{1.732745in}{1.545261in}}%
\pgfpathquadraticcurveto{\pgfqpoint{1.754967in}{1.545261in}}{\pgfqpoint{1.754967in}{1.567483in}}%
\pgfpathlineto{\pgfqpoint{1.754967in}{1.711261in}}%
\pgfpathquadraticcurveto{\pgfqpoint{1.754967in}{1.733483in}}{\pgfqpoint{1.732745in}{1.733483in}}%
\pgfpathlineto{\pgfqpoint{0.667523in}{1.733483in}}%
\pgfpathquadraticcurveto{\pgfqpoint{0.645300in}{1.733483in}}{\pgfqpoint{0.645300in}{1.711261in}}%
\pgfpathlineto{\pgfqpoint{0.645300in}{1.567483in}}%
\pgfpathquadraticcurveto{\pgfqpoint{0.645300in}{1.545261in}}{\pgfqpoint{0.667523in}{1.545261in}}%
\pgfpathlineto{\pgfqpoint{0.667523in}{1.545261in}}%
\pgfpathclose%
\pgfusepath{stroke,fill}%
\end{pgfscope}%
\begin{pgfscope}%
\pgfsetrectcap%
\pgfsetroundjoin%
\pgfsetlinewidth{1.505625pt}%
\definecolor{currentstroke}{rgb}{0.835294,0.368627,0.000000}%
\pgfsetstrokecolor{currentstroke}%
\pgfsetdash{}{0pt}%
\pgfpathmoveto{\pgfqpoint{0.689745in}{1.650150in}}%
\pgfpathlineto{\pgfqpoint{0.800856in}{1.650150in}}%
\pgfpathlineto{\pgfqpoint{0.911967in}{1.650150in}}%
\pgfusepath{stroke}%
\end{pgfscope}%
\begin{pgfscope}%
\definecolor{textcolor}{rgb}{0.000000,0.000000,0.000000}%
\pgfsetstrokecolor{textcolor}%
\pgfsetfillcolor{textcolor}%
\pgftext[x=1.000856in,y=1.611261in,left,base]{\color{textcolor}\rmfamily\fontsize{8.000000}{9.600000}\selectfont Random walk}%
\end{pgfscope}%
\end{pgfpicture}%
\makeatother%
\endgroup%

        } % scalebox
        \caption{Time domain}
        \label{fig:random_walk_time}
    \end{subfigure}
    \begin{subfigure}{0.32\linewidth}
        \centering
        \scalebox{0.75}{%
            %% Creator: Matplotlib, PGF backend
%%
%% To include the figure in your LaTeX document, write
%%   \input{<filename>.pgf}
%%
%% Make sure the required packages are loaded in your preamble
%%   \usepackage{pgf}
%%
%% Also ensure that all the required font packages are loaded; for instance,
%% the lmodern package is sometimes necessary when using math font.
%%   \usepackage{lmodern}
%%
%% Figures using additional raster images can only be included by \input if
%% they are in the same directory as the main LaTeX file. For loading figures
%% from other directories you can use the `import` package
%%   \usepackage{import}
%%
%% and then include the figures with
%%   \import{<path to file>}{<filename>.pgf}
%%
%% Matplotlib used the following preamble
%%   \usepackage{siunitx}
%%   \usepackage{fontspec}
%%   \makeatletter\@ifpackageloaded{underscore}{}{\usepackage[strings]{underscore}}\makeatother
%%
\begingroup%
\makeatletter%
\begin{pgfpicture}%
\pgfpathrectangle{\pgfpointorigin}{\pgfqpoint{2.440000in}{1.830000in}}%
\pgfusepath{use as bounding box, clip}%
\begin{pgfscope}%
\pgfsetbuttcap%
\pgfsetmiterjoin%
\definecolor{currentfill}{rgb}{1.000000,1.000000,1.000000}%
\pgfsetfillcolor{currentfill}%
\pgfsetlinewidth{0.000000pt}%
\definecolor{currentstroke}{rgb}{1.000000,1.000000,1.000000}%
\pgfsetstrokecolor{currentstroke}%
\pgfsetdash{}{0pt}%
\pgfpathmoveto{\pgfqpoint{0.000000in}{0.000000in}}%
\pgfpathlineto{\pgfqpoint{2.440000in}{0.000000in}}%
\pgfpathlineto{\pgfqpoint{2.440000in}{1.830000in}}%
\pgfpathlineto{\pgfqpoint{0.000000in}{1.830000in}}%
\pgfpathlineto{\pgfqpoint{0.000000in}{0.000000in}}%
\pgfpathclose%
\pgfusepath{fill}%
\end{pgfscope}%
\begin{pgfscope}%
\pgfsetbuttcap%
\pgfsetmiterjoin%
\definecolor{currentfill}{rgb}{1.000000,1.000000,1.000000}%
\pgfsetfillcolor{currentfill}%
\pgfsetlinewidth{0.000000pt}%
\definecolor{currentstroke}{rgb}{0.000000,0.000000,0.000000}%
\pgfsetstrokecolor{currentstroke}%
\pgfsetstrokeopacity{0.000000}%
\pgfsetdash{}{0pt}%
\pgfpathmoveto{\pgfqpoint{0.514278in}{0.417642in}}%
\pgfpathlineto{\pgfqpoint{2.398330in}{0.417642in}}%
\pgfpathlineto{\pgfqpoint{2.398330in}{1.788330in}}%
\pgfpathlineto{\pgfqpoint{0.514278in}{1.788330in}}%
\pgfpathlineto{\pgfqpoint{0.514278in}{0.417642in}}%
\pgfpathclose%
\pgfusepath{fill}%
\end{pgfscope}%
\begin{pgfscope}%
\pgfpathrectangle{\pgfqpoint{0.514278in}{0.417642in}}{\pgfqpoint{1.884052in}{1.370688in}}%
\pgfusepath{clip}%
\pgfsetrectcap%
\pgfsetroundjoin%
\pgfsetlinewidth{0.803000pt}%
\definecolor{currentstroke}{rgb}{0.450000,0.450000,0.450000}%
\pgfsetstrokecolor{currentstroke}%
\pgfsetdash{}{0pt}%
\pgfpathmoveto{\pgfqpoint{0.916624in}{0.417642in}}%
\pgfpathlineto{\pgfqpoint{0.916624in}{1.788330in}}%
\pgfusepath{stroke}%
\end{pgfscope}%
\begin{pgfscope}%
\pgfsetbuttcap%
\pgfsetroundjoin%
\definecolor{currentfill}{rgb}{0.000000,0.000000,0.000000}%
\pgfsetfillcolor{currentfill}%
\pgfsetlinewidth{0.803000pt}%
\definecolor{currentstroke}{rgb}{0.000000,0.000000,0.000000}%
\pgfsetstrokecolor{currentstroke}%
\pgfsetdash{}{0pt}%
\pgfsys@defobject{currentmarker}{\pgfqpoint{0.000000in}{-0.048611in}}{\pgfqpoint{0.000000in}{0.000000in}}{%
\pgfpathmoveto{\pgfqpoint{0.000000in}{0.000000in}}%
\pgfpathlineto{\pgfqpoint{0.000000in}{-0.048611in}}%
\pgfusepath{stroke,fill}%
}%
\begin{pgfscope}%
\pgfsys@transformshift{0.916624in}{0.417642in}%
\pgfsys@useobject{currentmarker}{}%
\end{pgfscope}%
\end{pgfscope}%
\begin{pgfscope}%
\definecolor{textcolor}{rgb}{0.000000,0.000000,0.000000}%
\pgfsetstrokecolor{textcolor}%
\pgfsetfillcolor{textcolor}%
\pgftext[x=0.916624in,y=0.320420in,,top]{\color{textcolor}\rmfamily\fontsize{8.000000}{9.600000}\selectfont \(\displaystyle {10^{-3}}\)}%
\end{pgfscope}%
\begin{pgfscope}%
\pgfpathrectangle{\pgfqpoint{0.514278in}{0.417642in}}{\pgfqpoint{1.884052in}{1.370688in}}%
\pgfusepath{clip}%
\pgfsetrectcap%
\pgfsetroundjoin%
\pgfsetlinewidth{0.803000pt}%
\definecolor{currentstroke}{rgb}{0.450000,0.450000,0.450000}%
\pgfsetstrokecolor{currentstroke}%
\pgfsetdash{}{0pt}%
\pgfpathmoveto{\pgfqpoint{1.433903in}{0.417642in}}%
\pgfpathlineto{\pgfqpoint{1.433903in}{1.788330in}}%
\pgfusepath{stroke}%
\end{pgfscope}%
\begin{pgfscope}%
\pgfsetbuttcap%
\pgfsetroundjoin%
\definecolor{currentfill}{rgb}{0.000000,0.000000,0.000000}%
\pgfsetfillcolor{currentfill}%
\pgfsetlinewidth{0.803000pt}%
\definecolor{currentstroke}{rgb}{0.000000,0.000000,0.000000}%
\pgfsetstrokecolor{currentstroke}%
\pgfsetdash{}{0pt}%
\pgfsys@defobject{currentmarker}{\pgfqpoint{0.000000in}{-0.048611in}}{\pgfqpoint{0.000000in}{0.000000in}}{%
\pgfpathmoveto{\pgfqpoint{0.000000in}{0.000000in}}%
\pgfpathlineto{\pgfqpoint{0.000000in}{-0.048611in}}%
\pgfusepath{stroke,fill}%
}%
\begin{pgfscope}%
\pgfsys@transformshift{1.433903in}{0.417642in}%
\pgfsys@useobject{currentmarker}{}%
\end{pgfscope}%
\end{pgfscope}%
\begin{pgfscope}%
\definecolor{textcolor}{rgb}{0.000000,0.000000,0.000000}%
\pgfsetstrokecolor{textcolor}%
\pgfsetfillcolor{textcolor}%
\pgftext[x=1.433903in,y=0.320420in,,top]{\color{textcolor}\rmfamily\fontsize{8.000000}{9.600000}\selectfont \(\displaystyle {10^{-2}}\)}%
\end{pgfscope}%
\begin{pgfscope}%
\pgfpathrectangle{\pgfqpoint{0.514278in}{0.417642in}}{\pgfqpoint{1.884052in}{1.370688in}}%
\pgfusepath{clip}%
\pgfsetrectcap%
\pgfsetroundjoin%
\pgfsetlinewidth{0.803000pt}%
\definecolor{currentstroke}{rgb}{0.450000,0.450000,0.450000}%
\pgfsetstrokecolor{currentstroke}%
\pgfsetdash{}{0pt}%
\pgfpathmoveto{\pgfqpoint{1.951183in}{0.417642in}}%
\pgfpathlineto{\pgfqpoint{1.951183in}{1.788330in}}%
\pgfusepath{stroke}%
\end{pgfscope}%
\begin{pgfscope}%
\pgfsetbuttcap%
\pgfsetroundjoin%
\definecolor{currentfill}{rgb}{0.000000,0.000000,0.000000}%
\pgfsetfillcolor{currentfill}%
\pgfsetlinewidth{0.803000pt}%
\definecolor{currentstroke}{rgb}{0.000000,0.000000,0.000000}%
\pgfsetstrokecolor{currentstroke}%
\pgfsetdash{}{0pt}%
\pgfsys@defobject{currentmarker}{\pgfqpoint{0.000000in}{-0.048611in}}{\pgfqpoint{0.000000in}{0.000000in}}{%
\pgfpathmoveto{\pgfqpoint{0.000000in}{0.000000in}}%
\pgfpathlineto{\pgfqpoint{0.000000in}{-0.048611in}}%
\pgfusepath{stroke,fill}%
}%
\begin{pgfscope}%
\pgfsys@transformshift{1.951183in}{0.417642in}%
\pgfsys@useobject{currentmarker}{}%
\end{pgfscope}%
\end{pgfscope}%
\begin{pgfscope}%
\definecolor{textcolor}{rgb}{0.000000,0.000000,0.000000}%
\pgfsetstrokecolor{textcolor}%
\pgfsetfillcolor{textcolor}%
\pgftext[x=1.951183in,y=0.320420in,,top]{\color{textcolor}\rmfamily\fontsize{8.000000}{9.600000}\selectfont \(\displaystyle {10^{-1}}\)}%
\end{pgfscope}%
\begin{pgfscope}%
\pgfpathrectangle{\pgfqpoint{0.514278in}{0.417642in}}{\pgfqpoint{1.884052in}{1.370688in}}%
\pgfusepath{clip}%
\pgfsetrectcap%
\pgfsetroundjoin%
\pgfsetlinewidth{0.803000pt}%
\definecolor{currentstroke}{rgb}{0.850000,0.850000,0.850000}%
\pgfsetstrokecolor{currentstroke}%
\pgfsetdash{}{0pt}%
\pgfpathmoveto{\pgfqpoint{0.555061in}{0.417642in}}%
\pgfpathlineto{\pgfqpoint{0.555061in}{1.788330in}}%
\pgfusepath{stroke}%
\end{pgfscope}%
\begin{pgfscope}%
\pgfsetbuttcap%
\pgfsetroundjoin%
\definecolor{currentfill}{rgb}{0.000000,0.000000,0.000000}%
\pgfsetfillcolor{currentfill}%
\pgfsetlinewidth{0.602250pt}%
\definecolor{currentstroke}{rgb}{0.000000,0.000000,0.000000}%
\pgfsetstrokecolor{currentstroke}%
\pgfsetdash{}{0pt}%
\pgfsys@defobject{currentmarker}{\pgfqpoint{0.000000in}{-0.027778in}}{\pgfqpoint{0.000000in}{0.000000in}}{%
\pgfpathmoveto{\pgfqpoint{0.000000in}{0.000000in}}%
\pgfpathlineto{\pgfqpoint{0.000000in}{-0.027778in}}%
\pgfusepath{stroke,fill}%
}%
\begin{pgfscope}%
\pgfsys@transformshift{0.555061in}{0.417642in}%
\pgfsys@useobject{currentmarker}{}%
\end{pgfscope}%
\end{pgfscope}%
\begin{pgfscope}%
\pgfpathrectangle{\pgfqpoint{0.514278in}{0.417642in}}{\pgfqpoint{1.884052in}{1.370688in}}%
\pgfusepath{clip}%
\pgfsetrectcap%
\pgfsetroundjoin%
\pgfsetlinewidth{0.803000pt}%
\definecolor{currentstroke}{rgb}{0.850000,0.850000,0.850000}%
\pgfsetstrokecolor{currentstroke}%
\pgfsetdash{}{0pt}%
\pgfpathmoveto{\pgfqpoint{0.646149in}{0.417642in}}%
\pgfpathlineto{\pgfqpoint{0.646149in}{1.788330in}}%
\pgfusepath{stroke}%
\end{pgfscope}%
\begin{pgfscope}%
\pgfsetbuttcap%
\pgfsetroundjoin%
\definecolor{currentfill}{rgb}{0.000000,0.000000,0.000000}%
\pgfsetfillcolor{currentfill}%
\pgfsetlinewidth{0.602250pt}%
\definecolor{currentstroke}{rgb}{0.000000,0.000000,0.000000}%
\pgfsetstrokecolor{currentstroke}%
\pgfsetdash{}{0pt}%
\pgfsys@defobject{currentmarker}{\pgfqpoint{0.000000in}{-0.027778in}}{\pgfqpoint{0.000000in}{0.000000in}}{%
\pgfpathmoveto{\pgfqpoint{0.000000in}{0.000000in}}%
\pgfpathlineto{\pgfqpoint{0.000000in}{-0.027778in}}%
\pgfusepath{stroke,fill}%
}%
\begin{pgfscope}%
\pgfsys@transformshift{0.646149in}{0.417642in}%
\pgfsys@useobject{currentmarker}{}%
\end{pgfscope}%
\end{pgfscope}%
\begin{pgfscope}%
\pgfpathrectangle{\pgfqpoint{0.514278in}{0.417642in}}{\pgfqpoint{1.884052in}{1.370688in}}%
\pgfusepath{clip}%
\pgfsetrectcap%
\pgfsetroundjoin%
\pgfsetlinewidth{0.803000pt}%
\definecolor{currentstroke}{rgb}{0.850000,0.850000,0.850000}%
\pgfsetstrokecolor{currentstroke}%
\pgfsetdash{}{0pt}%
\pgfpathmoveto{\pgfqpoint{0.710777in}{0.417642in}}%
\pgfpathlineto{\pgfqpoint{0.710777in}{1.788330in}}%
\pgfusepath{stroke}%
\end{pgfscope}%
\begin{pgfscope}%
\pgfsetbuttcap%
\pgfsetroundjoin%
\definecolor{currentfill}{rgb}{0.000000,0.000000,0.000000}%
\pgfsetfillcolor{currentfill}%
\pgfsetlinewidth{0.602250pt}%
\definecolor{currentstroke}{rgb}{0.000000,0.000000,0.000000}%
\pgfsetstrokecolor{currentstroke}%
\pgfsetdash{}{0pt}%
\pgfsys@defobject{currentmarker}{\pgfqpoint{0.000000in}{-0.027778in}}{\pgfqpoint{0.000000in}{0.000000in}}{%
\pgfpathmoveto{\pgfqpoint{0.000000in}{0.000000in}}%
\pgfpathlineto{\pgfqpoint{0.000000in}{-0.027778in}}%
\pgfusepath{stroke,fill}%
}%
\begin{pgfscope}%
\pgfsys@transformshift{0.710777in}{0.417642in}%
\pgfsys@useobject{currentmarker}{}%
\end{pgfscope}%
\end{pgfscope}%
\begin{pgfscope}%
\pgfpathrectangle{\pgfqpoint{0.514278in}{0.417642in}}{\pgfqpoint{1.884052in}{1.370688in}}%
\pgfusepath{clip}%
\pgfsetrectcap%
\pgfsetroundjoin%
\pgfsetlinewidth{0.803000pt}%
\definecolor{currentstroke}{rgb}{0.850000,0.850000,0.850000}%
\pgfsetstrokecolor{currentstroke}%
\pgfsetdash{}{0pt}%
\pgfpathmoveto{\pgfqpoint{0.760907in}{0.417642in}}%
\pgfpathlineto{\pgfqpoint{0.760907in}{1.788330in}}%
\pgfusepath{stroke}%
\end{pgfscope}%
\begin{pgfscope}%
\pgfsetbuttcap%
\pgfsetroundjoin%
\definecolor{currentfill}{rgb}{0.000000,0.000000,0.000000}%
\pgfsetfillcolor{currentfill}%
\pgfsetlinewidth{0.602250pt}%
\definecolor{currentstroke}{rgb}{0.000000,0.000000,0.000000}%
\pgfsetstrokecolor{currentstroke}%
\pgfsetdash{}{0pt}%
\pgfsys@defobject{currentmarker}{\pgfqpoint{0.000000in}{-0.027778in}}{\pgfqpoint{0.000000in}{0.000000in}}{%
\pgfpathmoveto{\pgfqpoint{0.000000in}{0.000000in}}%
\pgfpathlineto{\pgfqpoint{0.000000in}{-0.027778in}}%
\pgfusepath{stroke,fill}%
}%
\begin{pgfscope}%
\pgfsys@transformshift{0.760907in}{0.417642in}%
\pgfsys@useobject{currentmarker}{}%
\end{pgfscope}%
\end{pgfscope}%
\begin{pgfscope}%
\pgfpathrectangle{\pgfqpoint{0.514278in}{0.417642in}}{\pgfqpoint{1.884052in}{1.370688in}}%
\pgfusepath{clip}%
\pgfsetrectcap%
\pgfsetroundjoin%
\pgfsetlinewidth{0.803000pt}%
\definecolor{currentstroke}{rgb}{0.850000,0.850000,0.850000}%
\pgfsetstrokecolor{currentstroke}%
\pgfsetdash{}{0pt}%
\pgfpathmoveto{\pgfqpoint{0.801866in}{0.417642in}}%
\pgfpathlineto{\pgfqpoint{0.801866in}{1.788330in}}%
\pgfusepath{stroke}%
\end{pgfscope}%
\begin{pgfscope}%
\pgfsetbuttcap%
\pgfsetroundjoin%
\definecolor{currentfill}{rgb}{0.000000,0.000000,0.000000}%
\pgfsetfillcolor{currentfill}%
\pgfsetlinewidth{0.602250pt}%
\definecolor{currentstroke}{rgb}{0.000000,0.000000,0.000000}%
\pgfsetstrokecolor{currentstroke}%
\pgfsetdash{}{0pt}%
\pgfsys@defobject{currentmarker}{\pgfqpoint{0.000000in}{-0.027778in}}{\pgfqpoint{0.000000in}{0.000000in}}{%
\pgfpathmoveto{\pgfqpoint{0.000000in}{0.000000in}}%
\pgfpathlineto{\pgfqpoint{0.000000in}{-0.027778in}}%
\pgfusepath{stroke,fill}%
}%
\begin{pgfscope}%
\pgfsys@transformshift{0.801866in}{0.417642in}%
\pgfsys@useobject{currentmarker}{}%
\end{pgfscope}%
\end{pgfscope}%
\begin{pgfscope}%
\pgfpathrectangle{\pgfqpoint{0.514278in}{0.417642in}}{\pgfqpoint{1.884052in}{1.370688in}}%
\pgfusepath{clip}%
\pgfsetrectcap%
\pgfsetroundjoin%
\pgfsetlinewidth{0.803000pt}%
\definecolor{currentstroke}{rgb}{0.850000,0.850000,0.850000}%
\pgfsetstrokecolor{currentstroke}%
\pgfsetdash{}{0pt}%
\pgfpathmoveto{\pgfqpoint{0.836496in}{0.417642in}}%
\pgfpathlineto{\pgfqpoint{0.836496in}{1.788330in}}%
\pgfusepath{stroke}%
\end{pgfscope}%
\begin{pgfscope}%
\pgfsetbuttcap%
\pgfsetroundjoin%
\definecolor{currentfill}{rgb}{0.000000,0.000000,0.000000}%
\pgfsetfillcolor{currentfill}%
\pgfsetlinewidth{0.602250pt}%
\definecolor{currentstroke}{rgb}{0.000000,0.000000,0.000000}%
\pgfsetstrokecolor{currentstroke}%
\pgfsetdash{}{0pt}%
\pgfsys@defobject{currentmarker}{\pgfqpoint{0.000000in}{-0.027778in}}{\pgfqpoint{0.000000in}{0.000000in}}{%
\pgfpathmoveto{\pgfqpoint{0.000000in}{0.000000in}}%
\pgfpathlineto{\pgfqpoint{0.000000in}{-0.027778in}}%
\pgfusepath{stroke,fill}%
}%
\begin{pgfscope}%
\pgfsys@transformshift{0.836496in}{0.417642in}%
\pgfsys@useobject{currentmarker}{}%
\end{pgfscope}%
\end{pgfscope}%
\begin{pgfscope}%
\pgfpathrectangle{\pgfqpoint{0.514278in}{0.417642in}}{\pgfqpoint{1.884052in}{1.370688in}}%
\pgfusepath{clip}%
\pgfsetrectcap%
\pgfsetroundjoin%
\pgfsetlinewidth{0.803000pt}%
\definecolor{currentstroke}{rgb}{0.850000,0.850000,0.850000}%
\pgfsetstrokecolor{currentstroke}%
\pgfsetdash{}{0pt}%
\pgfpathmoveto{\pgfqpoint{0.866494in}{0.417642in}}%
\pgfpathlineto{\pgfqpoint{0.866494in}{1.788330in}}%
\pgfusepath{stroke}%
\end{pgfscope}%
\begin{pgfscope}%
\pgfsetbuttcap%
\pgfsetroundjoin%
\definecolor{currentfill}{rgb}{0.000000,0.000000,0.000000}%
\pgfsetfillcolor{currentfill}%
\pgfsetlinewidth{0.602250pt}%
\definecolor{currentstroke}{rgb}{0.000000,0.000000,0.000000}%
\pgfsetstrokecolor{currentstroke}%
\pgfsetdash{}{0pt}%
\pgfsys@defobject{currentmarker}{\pgfqpoint{0.000000in}{-0.027778in}}{\pgfqpoint{0.000000in}{0.000000in}}{%
\pgfpathmoveto{\pgfqpoint{0.000000in}{0.000000in}}%
\pgfpathlineto{\pgfqpoint{0.000000in}{-0.027778in}}%
\pgfusepath{stroke,fill}%
}%
\begin{pgfscope}%
\pgfsys@transformshift{0.866494in}{0.417642in}%
\pgfsys@useobject{currentmarker}{}%
\end{pgfscope}%
\end{pgfscope}%
\begin{pgfscope}%
\pgfpathrectangle{\pgfqpoint{0.514278in}{0.417642in}}{\pgfqpoint{1.884052in}{1.370688in}}%
\pgfusepath{clip}%
\pgfsetrectcap%
\pgfsetroundjoin%
\pgfsetlinewidth{0.803000pt}%
\definecolor{currentstroke}{rgb}{0.850000,0.850000,0.850000}%
\pgfsetstrokecolor{currentstroke}%
\pgfsetdash{}{0pt}%
\pgfpathmoveto{\pgfqpoint{0.892954in}{0.417642in}}%
\pgfpathlineto{\pgfqpoint{0.892954in}{1.788330in}}%
\pgfusepath{stroke}%
\end{pgfscope}%
\begin{pgfscope}%
\pgfsetbuttcap%
\pgfsetroundjoin%
\definecolor{currentfill}{rgb}{0.000000,0.000000,0.000000}%
\pgfsetfillcolor{currentfill}%
\pgfsetlinewidth{0.602250pt}%
\definecolor{currentstroke}{rgb}{0.000000,0.000000,0.000000}%
\pgfsetstrokecolor{currentstroke}%
\pgfsetdash{}{0pt}%
\pgfsys@defobject{currentmarker}{\pgfqpoint{0.000000in}{-0.027778in}}{\pgfqpoint{0.000000in}{0.000000in}}{%
\pgfpathmoveto{\pgfqpoint{0.000000in}{0.000000in}}%
\pgfpathlineto{\pgfqpoint{0.000000in}{-0.027778in}}%
\pgfusepath{stroke,fill}%
}%
\begin{pgfscope}%
\pgfsys@transformshift{0.892954in}{0.417642in}%
\pgfsys@useobject{currentmarker}{}%
\end{pgfscope}%
\end{pgfscope}%
\begin{pgfscope}%
\pgfpathrectangle{\pgfqpoint{0.514278in}{0.417642in}}{\pgfqpoint{1.884052in}{1.370688in}}%
\pgfusepath{clip}%
\pgfsetrectcap%
\pgfsetroundjoin%
\pgfsetlinewidth{0.803000pt}%
\definecolor{currentstroke}{rgb}{0.850000,0.850000,0.850000}%
\pgfsetstrokecolor{currentstroke}%
\pgfsetdash{}{0pt}%
\pgfpathmoveto{\pgfqpoint{1.072340in}{0.417642in}}%
\pgfpathlineto{\pgfqpoint{1.072340in}{1.788330in}}%
\pgfusepath{stroke}%
\end{pgfscope}%
\begin{pgfscope}%
\pgfsetbuttcap%
\pgfsetroundjoin%
\definecolor{currentfill}{rgb}{0.000000,0.000000,0.000000}%
\pgfsetfillcolor{currentfill}%
\pgfsetlinewidth{0.602250pt}%
\definecolor{currentstroke}{rgb}{0.000000,0.000000,0.000000}%
\pgfsetstrokecolor{currentstroke}%
\pgfsetdash{}{0pt}%
\pgfsys@defobject{currentmarker}{\pgfqpoint{0.000000in}{-0.027778in}}{\pgfqpoint{0.000000in}{0.000000in}}{%
\pgfpathmoveto{\pgfqpoint{0.000000in}{0.000000in}}%
\pgfpathlineto{\pgfqpoint{0.000000in}{-0.027778in}}%
\pgfusepath{stroke,fill}%
}%
\begin{pgfscope}%
\pgfsys@transformshift{1.072340in}{0.417642in}%
\pgfsys@useobject{currentmarker}{}%
\end{pgfscope}%
\end{pgfscope}%
\begin{pgfscope}%
\pgfpathrectangle{\pgfqpoint{0.514278in}{0.417642in}}{\pgfqpoint{1.884052in}{1.370688in}}%
\pgfusepath{clip}%
\pgfsetrectcap%
\pgfsetroundjoin%
\pgfsetlinewidth{0.803000pt}%
\definecolor{currentstroke}{rgb}{0.850000,0.850000,0.850000}%
\pgfsetstrokecolor{currentstroke}%
\pgfsetdash{}{0pt}%
\pgfpathmoveto{\pgfqpoint{1.163429in}{0.417642in}}%
\pgfpathlineto{\pgfqpoint{1.163429in}{1.788330in}}%
\pgfusepath{stroke}%
\end{pgfscope}%
\begin{pgfscope}%
\pgfsetbuttcap%
\pgfsetroundjoin%
\definecolor{currentfill}{rgb}{0.000000,0.000000,0.000000}%
\pgfsetfillcolor{currentfill}%
\pgfsetlinewidth{0.602250pt}%
\definecolor{currentstroke}{rgb}{0.000000,0.000000,0.000000}%
\pgfsetstrokecolor{currentstroke}%
\pgfsetdash{}{0pt}%
\pgfsys@defobject{currentmarker}{\pgfqpoint{0.000000in}{-0.027778in}}{\pgfqpoint{0.000000in}{0.000000in}}{%
\pgfpathmoveto{\pgfqpoint{0.000000in}{0.000000in}}%
\pgfpathlineto{\pgfqpoint{0.000000in}{-0.027778in}}%
\pgfusepath{stroke,fill}%
}%
\begin{pgfscope}%
\pgfsys@transformshift{1.163429in}{0.417642in}%
\pgfsys@useobject{currentmarker}{}%
\end{pgfscope}%
\end{pgfscope}%
\begin{pgfscope}%
\pgfpathrectangle{\pgfqpoint{0.514278in}{0.417642in}}{\pgfqpoint{1.884052in}{1.370688in}}%
\pgfusepath{clip}%
\pgfsetrectcap%
\pgfsetroundjoin%
\pgfsetlinewidth{0.803000pt}%
\definecolor{currentstroke}{rgb}{0.850000,0.850000,0.850000}%
\pgfsetstrokecolor{currentstroke}%
\pgfsetdash{}{0pt}%
\pgfpathmoveto{\pgfqpoint{1.228057in}{0.417642in}}%
\pgfpathlineto{\pgfqpoint{1.228057in}{1.788330in}}%
\pgfusepath{stroke}%
\end{pgfscope}%
\begin{pgfscope}%
\pgfsetbuttcap%
\pgfsetroundjoin%
\definecolor{currentfill}{rgb}{0.000000,0.000000,0.000000}%
\pgfsetfillcolor{currentfill}%
\pgfsetlinewidth{0.602250pt}%
\definecolor{currentstroke}{rgb}{0.000000,0.000000,0.000000}%
\pgfsetstrokecolor{currentstroke}%
\pgfsetdash{}{0pt}%
\pgfsys@defobject{currentmarker}{\pgfqpoint{0.000000in}{-0.027778in}}{\pgfqpoint{0.000000in}{0.000000in}}{%
\pgfpathmoveto{\pgfqpoint{0.000000in}{0.000000in}}%
\pgfpathlineto{\pgfqpoint{0.000000in}{-0.027778in}}%
\pgfusepath{stroke,fill}%
}%
\begin{pgfscope}%
\pgfsys@transformshift{1.228057in}{0.417642in}%
\pgfsys@useobject{currentmarker}{}%
\end{pgfscope}%
\end{pgfscope}%
\begin{pgfscope}%
\pgfpathrectangle{\pgfqpoint{0.514278in}{0.417642in}}{\pgfqpoint{1.884052in}{1.370688in}}%
\pgfusepath{clip}%
\pgfsetrectcap%
\pgfsetroundjoin%
\pgfsetlinewidth{0.803000pt}%
\definecolor{currentstroke}{rgb}{0.850000,0.850000,0.850000}%
\pgfsetstrokecolor{currentstroke}%
\pgfsetdash{}{0pt}%
\pgfpathmoveto{\pgfqpoint{1.278187in}{0.417642in}}%
\pgfpathlineto{\pgfqpoint{1.278187in}{1.788330in}}%
\pgfusepath{stroke}%
\end{pgfscope}%
\begin{pgfscope}%
\pgfsetbuttcap%
\pgfsetroundjoin%
\definecolor{currentfill}{rgb}{0.000000,0.000000,0.000000}%
\pgfsetfillcolor{currentfill}%
\pgfsetlinewidth{0.602250pt}%
\definecolor{currentstroke}{rgb}{0.000000,0.000000,0.000000}%
\pgfsetstrokecolor{currentstroke}%
\pgfsetdash{}{0pt}%
\pgfsys@defobject{currentmarker}{\pgfqpoint{0.000000in}{-0.027778in}}{\pgfqpoint{0.000000in}{0.000000in}}{%
\pgfpathmoveto{\pgfqpoint{0.000000in}{0.000000in}}%
\pgfpathlineto{\pgfqpoint{0.000000in}{-0.027778in}}%
\pgfusepath{stroke,fill}%
}%
\begin{pgfscope}%
\pgfsys@transformshift{1.278187in}{0.417642in}%
\pgfsys@useobject{currentmarker}{}%
\end{pgfscope}%
\end{pgfscope}%
\begin{pgfscope}%
\pgfpathrectangle{\pgfqpoint{0.514278in}{0.417642in}}{\pgfqpoint{1.884052in}{1.370688in}}%
\pgfusepath{clip}%
\pgfsetrectcap%
\pgfsetroundjoin%
\pgfsetlinewidth{0.803000pt}%
\definecolor{currentstroke}{rgb}{0.850000,0.850000,0.850000}%
\pgfsetstrokecolor{currentstroke}%
\pgfsetdash{}{0pt}%
\pgfpathmoveto{\pgfqpoint{1.319146in}{0.417642in}}%
\pgfpathlineto{\pgfqpoint{1.319146in}{1.788330in}}%
\pgfusepath{stroke}%
\end{pgfscope}%
\begin{pgfscope}%
\pgfsetbuttcap%
\pgfsetroundjoin%
\definecolor{currentfill}{rgb}{0.000000,0.000000,0.000000}%
\pgfsetfillcolor{currentfill}%
\pgfsetlinewidth{0.602250pt}%
\definecolor{currentstroke}{rgb}{0.000000,0.000000,0.000000}%
\pgfsetstrokecolor{currentstroke}%
\pgfsetdash{}{0pt}%
\pgfsys@defobject{currentmarker}{\pgfqpoint{0.000000in}{-0.027778in}}{\pgfqpoint{0.000000in}{0.000000in}}{%
\pgfpathmoveto{\pgfqpoint{0.000000in}{0.000000in}}%
\pgfpathlineto{\pgfqpoint{0.000000in}{-0.027778in}}%
\pgfusepath{stroke,fill}%
}%
\begin{pgfscope}%
\pgfsys@transformshift{1.319146in}{0.417642in}%
\pgfsys@useobject{currentmarker}{}%
\end{pgfscope}%
\end{pgfscope}%
\begin{pgfscope}%
\pgfpathrectangle{\pgfqpoint{0.514278in}{0.417642in}}{\pgfqpoint{1.884052in}{1.370688in}}%
\pgfusepath{clip}%
\pgfsetrectcap%
\pgfsetroundjoin%
\pgfsetlinewidth{0.803000pt}%
\definecolor{currentstroke}{rgb}{0.850000,0.850000,0.850000}%
\pgfsetstrokecolor{currentstroke}%
\pgfsetdash{}{0pt}%
\pgfpathmoveto{\pgfqpoint{1.353776in}{0.417642in}}%
\pgfpathlineto{\pgfqpoint{1.353776in}{1.788330in}}%
\pgfusepath{stroke}%
\end{pgfscope}%
\begin{pgfscope}%
\pgfsetbuttcap%
\pgfsetroundjoin%
\definecolor{currentfill}{rgb}{0.000000,0.000000,0.000000}%
\pgfsetfillcolor{currentfill}%
\pgfsetlinewidth{0.602250pt}%
\definecolor{currentstroke}{rgb}{0.000000,0.000000,0.000000}%
\pgfsetstrokecolor{currentstroke}%
\pgfsetdash{}{0pt}%
\pgfsys@defobject{currentmarker}{\pgfqpoint{0.000000in}{-0.027778in}}{\pgfqpoint{0.000000in}{0.000000in}}{%
\pgfpathmoveto{\pgfqpoint{0.000000in}{0.000000in}}%
\pgfpathlineto{\pgfqpoint{0.000000in}{-0.027778in}}%
\pgfusepath{stroke,fill}%
}%
\begin{pgfscope}%
\pgfsys@transformshift{1.353776in}{0.417642in}%
\pgfsys@useobject{currentmarker}{}%
\end{pgfscope}%
\end{pgfscope}%
\begin{pgfscope}%
\pgfpathrectangle{\pgfqpoint{0.514278in}{0.417642in}}{\pgfqpoint{1.884052in}{1.370688in}}%
\pgfusepath{clip}%
\pgfsetrectcap%
\pgfsetroundjoin%
\pgfsetlinewidth{0.803000pt}%
\definecolor{currentstroke}{rgb}{0.850000,0.850000,0.850000}%
\pgfsetstrokecolor{currentstroke}%
\pgfsetdash{}{0pt}%
\pgfpathmoveto{\pgfqpoint{1.383774in}{0.417642in}}%
\pgfpathlineto{\pgfqpoint{1.383774in}{1.788330in}}%
\pgfusepath{stroke}%
\end{pgfscope}%
\begin{pgfscope}%
\pgfsetbuttcap%
\pgfsetroundjoin%
\definecolor{currentfill}{rgb}{0.000000,0.000000,0.000000}%
\pgfsetfillcolor{currentfill}%
\pgfsetlinewidth{0.602250pt}%
\definecolor{currentstroke}{rgb}{0.000000,0.000000,0.000000}%
\pgfsetstrokecolor{currentstroke}%
\pgfsetdash{}{0pt}%
\pgfsys@defobject{currentmarker}{\pgfqpoint{0.000000in}{-0.027778in}}{\pgfqpoint{0.000000in}{0.000000in}}{%
\pgfpathmoveto{\pgfqpoint{0.000000in}{0.000000in}}%
\pgfpathlineto{\pgfqpoint{0.000000in}{-0.027778in}}%
\pgfusepath{stroke,fill}%
}%
\begin{pgfscope}%
\pgfsys@transformshift{1.383774in}{0.417642in}%
\pgfsys@useobject{currentmarker}{}%
\end{pgfscope}%
\end{pgfscope}%
\begin{pgfscope}%
\pgfpathrectangle{\pgfqpoint{0.514278in}{0.417642in}}{\pgfqpoint{1.884052in}{1.370688in}}%
\pgfusepath{clip}%
\pgfsetrectcap%
\pgfsetroundjoin%
\pgfsetlinewidth{0.803000pt}%
\definecolor{currentstroke}{rgb}{0.850000,0.850000,0.850000}%
\pgfsetstrokecolor{currentstroke}%
\pgfsetdash{}{0pt}%
\pgfpathmoveto{\pgfqpoint{1.410234in}{0.417642in}}%
\pgfpathlineto{\pgfqpoint{1.410234in}{1.788330in}}%
\pgfusepath{stroke}%
\end{pgfscope}%
\begin{pgfscope}%
\pgfsetbuttcap%
\pgfsetroundjoin%
\definecolor{currentfill}{rgb}{0.000000,0.000000,0.000000}%
\pgfsetfillcolor{currentfill}%
\pgfsetlinewidth{0.602250pt}%
\definecolor{currentstroke}{rgb}{0.000000,0.000000,0.000000}%
\pgfsetstrokecolor{currentstroke}%
\pgfsetdash{}{0pt}%
\pgfsys@defobject{currentmarker}{\pgfqpoint{0.000000in}{-0.027778in}}{\pgfqpoint{0.000000in}{0.000000in}}{%
\pgfpathmoveto{\pgfqpoint{0.000000in}{0.000000in}}%
\pgfpathlineto{\pgfqpoint{0.000000in}{-0.027778in}}%
\pgfusepath{stroke,fill}%
}%
\begin{pgfscope}%
\pgfsys@transformshift{1.410234in}{0.417642in}%
\pgfsys@useobject{currentmarker}{}%
\end{pgfscope}%
\end{pgfscope}%
\begin{pgfscope}%
\pgfpathrectangle{\pgfqpoint{0.514278in}{0.417642in}}{\pgfqpoint{1.884052in}{1.370688in}}%
\pgfusepath{clip}%
\pgfsetrectcap%
\pgfsetroundjoin%
\pgfsetlinewidth{0.803000pt}%
\definecolor{currentstroke}{rgb}{0.850000,0.850000,0.850000}%
\pgfsetstrokecolor{currentstroke}%
\pgfsetdash{}{0pt}%
\pgfpathmoveto{\pgfqpoint{1.589620in}{0.417642in}}%
\pgfpathlineto{\pgfqpoint{1.589620in}{1.788330in}}%
\pgfusepath{stroke}%
\end{pgfscope}%
\begin{pgfscope}%
\pgfsetbuttcap%
\pgfsetroundjoin%
\definecolor{currentfill}{rgb}{0.000000,0.000000,0.000000}%
\pgfsetfillcolor{currentfill}%
\pgfsetlinewidth{0.602250pt}%
\definecolor{currentstroke}{rgb}{0.000000,0.000000,0.000000}%
\pgfsetstrokecolor{currentstroke}%
\pgfsetdash{}{0pt}%
\pgfsys@defobject{currentmarker}{\pgfqpoint{0.000000in}{-0.027778in}}{\pgfqpoint{0.000000in}{0.000000in}}{%
\pgfpathmoveto{\pgfqpoint{0.000000in}{0.000000in}}%
\pgfpathlineto{\pgfqpoint{0.000000in}{-0.027778in}}%
\pgfusepath{stroke,fill}%
}%
\begin{pgfscope}%
\pgfsys@transformshift{1.589620in}{0.417642in}%
\pgfsys@useobject{currentmarker}{}%
\end{pgfscope}%
\end{pgfscope}%
\begin{pgfscope}%
\pgfpathrectangle{\pgfqpoint{0.514278in}{0.417642in}}{\pgfqpoint{1.884052in}{1.370688in}}%
\pgfusepath{clip}%
\pgfsetrectcap%
\pgfsetroundjoin%
\pgfsetlinewidth{0.803000pt}%
\definecolor{currentstroke}{rgb}{0.850000,0.850000,0.850000}%
\pgfsetstrokecolor{currentstroke}%
\pgfsetdash{}{0pt}%
\pgfpathmoveto{\pgfqpoint{1.680709in}{0.417642in}}%
\pgfpathlineto{\pgfqpoint{1.680709in}{1.788330in}}%
\pgfusepath{stroke}%
\end{pgfscope}%
\begin{pgfscope}%
\pgfsetbuttcap%
\pgfsetroundjoin%
\definecolor{currentfill}{rgb}{0.000000,0.000000,0.000000}%
\pgfsetfillcolor{currentfill}%
\pgfsetlinewidth{0.602250pt}%
\definecolor{currentstroke}{rgb}{0.000000,0.000000,0.000000}%
\pgfsetstrokecolor{currentstroke}%
\pgfsetdash{}{0pt}%
\pgfsys@defobject{currentmarker}{\pgfqpoint{0.000000in}{-0.027778in}}{\pgfqpoint{0.000000in}{0.000000in}}{%
\pgfpathmoveto{\pgfqpoint{0.000000in}{0.000000in}}%
\pgfpathlineto{\pgfqpoint{0.000000in}{-0.027778in}}%
\pgfusepath{stroke,fill}%
}%
\begin{pgfscope}%
\pgfsys@transformshift{1.680709in}{0.417642in}%
\pgfsys@useobject{currentmarker}{}%
\end{pgfscope}%
\end{pgfscope}%
\begin{pgfscope}%
\pgfpathrectangle{\pgfqpoint{0.514278in}{0.417642in}}{\pgfqpoint{1.884052in}{1.370688in}}%
\pgfusepath{clip}%
\pgfsetrectcap%
\pgfsetroundjoin%
\pgfsetlinewidth{0.803000pt}%
\definecolor{currentstroke}{rgb}{0.850000,0.850000,0.850000}%
\pgfsetstrokecolor{currentstroke}%
\pgfsetdash{}{0pt}%
\pgfpathmoveto{\pgfqpoint{1.745337in}{0.417642in}}%
\pgfpathlineto{\pgfqpoint{1.745337in}{1.788330in}}%
\pgfusepath{stroke}%
\end{pgfscope}%
\begin{pgfscope}%
\pgfsetbuttcap%
\pgfsetroundjoin%
\definecolor{currentfill}{rgb}{0.000000,0.000000,0.000000}%
\pgfsetfillcolor{currentfill}%
\pgfsetlinewidth{0.602250pt}%
\definecolor{currentstroke}{rgb}{0.000000,0.000000,0.000000}%
\pgfsetstrokecolor{currentstroke}%
\pgfsetdash{}{0pt}%
\pgfsys@defobject{currentmarker}{\pgfqpoint{0.000000in}{-0.027778in}}{\pgfqpoint{0.000000in}{0.000000in}}{%
\pgfpathmoveto{\pgfqpoint{0.000000in}{0.000000in}}%
\pgfpathlineto{\pgfqpoint{0.000000in}{-0.027778in}}%
\pgfusepath{stroke,fill}%
}%
\begin{pgfscope}%
\pgfsys@transformshift{1.745337in}{0.417642in}%
\pgfsys@useobject{currentmarker}{}%
\end{pgfscope}%
\end{pgfscope}%
\begin{pgfscope}%
\pgfpathrectangle{\pgfqpoint{0.514278in}{0.417642in}}{\pgfqpoint{1.884052in}{1.370688in}}%
\pgfusepath{clip}%
\pgfsetrectcap%
\pgfsetroundjoin%
\pgfsetlinewidth{0.803000pt}%
\definecolor{currentstroke}{rgb}{0.850000,0.850000,0.850000}%
\pgfsetstrokecolor{currentstroke}%
\pgfsetdash{}{0pt}%
\pgfpathmoveto{\pgfqpoint{1.795466in}{0.417642in}}%
\pgfpathlineto{\pgfqpoint{1.795466in}{1.788330in}}%
\pgfusepath{stroke}%
\end{pgfscope}%
\begin{pgfscope}%
\pgfsetbuttcap%
\pgfsetroundjoin%
\definecolor{currentfill}{rgb}{0.000000,0.000000,0.000000}%
\pgfsetfillcolor{currentfill}%
\pgfsetlinewidth{0.602250pt}%
\definecolor{currentstroke}{rgb}{0.000000,0.000000,0.000000}%
\pgfsetstrokecolor{currentstroke}%
\pgfsetdash{}{0pt}%
\pgfsys@defobject{currentmarker}{\pgfqpoint{0.000000in}{-0.027778in}}{\pgfqpoint{0.000000in}{0.000000in}}{%
\pgfpathmoveto{\pgfqpoint{0.000000in}{0.000000in}}%
\pgfpathlineto{\pgfqpoint{0.000000in}{-0.027778in}}%
\pgfusepath{stroke,fill}%
}%
\begin{pgfscope}%
\pgfsys@transformshift{1.795466in}{0.417642in}%
\pgfsys@useobject{currentmarker}{}%
\end{pgfscope}%
\end{pgfscope}%
\begin{pgfscope}%
\pgfpathrectangle{\pgfqpoint{0.514278in}{0.417642in}}{\pgfqpoint{1.884052in}{1.370688in}}%
\pgfusepath{clip}%
\pgfsetrectcap%
\pgfsetroundjoin%
\pgfsetlinewidth{0.803000pt}%
\definecolor{currentstroke}{rgb}{0.850000,0.850000,0.850000}%
\pgfsetstrokecolor{currentstroke}%
\pgfsetdash{}{0pt}%
\pgfpathmoveto{\pgfqpoint{1.836425in}{0.417642in}}%
\pgfpathlineto{\pgfqpoint{1.836425in}{1.788330in}}%
\pgfusepath{stroke}%
\end{pgfscope}%
\begin{pgfscope}%
\pgfsetbuttcap%
\pgfsetroundjoin%
\definecolor{currentfill}{rgb}{0.000000,0.000000,0.000000}%
\pgfsetfillcolor{currentfill}%
\pgfsetlinewidth{0.602250pt}%
\definecolor{currentstroke}{rgb}{0.000000,0.000000,0.000000}%
\pgfsetstrokecolor{currentstroke}%
\pgfsetdash{}{0pt}%
\pgfsys@defobject{currentmarker}{\pgfqpoint{0.000000in}{-0.027778in}}{\pgfqpoint{0.000000in}{0.000000in}}{%
\pgfpathmoveto{\pgfqpoint{0.000000in}{0.000000in}}%
\pgfpathlineto{\pgfqpoint{0.000000in}{-0.027778in}}%
\pgfusepath{stroke,fill}%
}%
\begin{pgfscope}%
\pgfsys@transformshift{1.836425in}{0.417642in}%
\pgfsys@useobject{currentmarker}{}%
\end{pgfscope}%
\end{pgfscope}%
\begin{pgfscope}%
\pgfpathrectangle{\pgfqpoint{0.514278in}{0.417642in}}{\pgfqpoint{1.884052in}{1.370688in}}%
\pgfusepath{clip}%
\pgfsetrectcap%
\pgfsetroundjoin%
\pgfsetlinewidth{0.803000pt}%
\definecolor{currentstroke}{rgb}{0.850000,0.850000,0.850000}%
\pgfsetstrokecolor{currentstroke}%
\pgfsetdash{}{0pt}%
\pgfpathmoveto{\pgfqpoint{1.871056in}{0.417642in}}%
\pgfpathlineto{\pgfqpoint{1.871056in}{1.788330in}}%
\pgfusepath{stroke}%
\end{pgfscope}%
\begin{pgfscope}%
\pgfsetbuttcap%
\pgfsetroundjoin%
\definecolor{currentfill}{rgb}{0.000000,0.000000,0.000000}%
\pgfsetfillcolor{currentfill}%
\pgfsetlinewidth{0.602250pt}%
\definecolor{currentstroke}{rgb}{0.000000,0.000000,0.000000}%
\pgfsetstrokecolor{currentstroke}%
\pgfsetdash{}{0pt}%
\pgfsys@defobject{currentmarker}{\pgfqpoint{0.000000in}{-0.027778in}}{\pgfqpoint{0.000000in}{0.000000in}}{%
\pgfpathmoveto{\pgfqpoint{0.000000in}{0.000000in}}%
\pgfpathlineto{\pgfqpoint{0.000000in}{-0.027778in}}%
\pgfusepath{stroke,fill}%
}%
\begin{pgfscope}%
\pgfsys@transformshift{1.871056in}{0.417642in}%
\pgfsys@useobject{currentmarker}{}%
\end{pgfscope}%
\end{pgfscope}%
\begin{pgfscope}%
\pgfpathrectangle{\pgfqpoint{0.514278in}{0.417642in}}{\pgfqpoint{1.884052in}{1.370688in}}%
\pgfusepath{clip}%
\pgfsetrectcap%
\pgfsetroundjoin%
\pgfsetlinewidth{0.803000pt}%
\definecolor{currentstroke}{rgb}{0.850000,0.850000,0.850000}%
\pgfsetstrokecolor{currentstroke}%
\pgfsetdash{}{0pt}%
\pgfpathmoveto{\pgfqpoint{1.901054in}{0.417642in}}%
\pgfpathlineto{\pgfqpoint{1.901054in}{1.788330in}}%
\pgfusepath{stroke}%
\end{pgfscope}%
\begin{pgfscope}%
\pgfsetbuttcap%
\pgfsetroundjoin%
\definecolor{currentfill}{rgb}{0.000000,0.000000,0.000000}%
\pgfsetfillcolor{currentfill}%
\pgfsetlinewidth{0.602250pt}%
\definecolor{currentstroke}{rgb}{0.000000,0.000000,0.000000}%
\pgfsetstrokecolor{currentstroke}%
\pgfsetdash{}{0pt}%
\pgfsys@defobject{currentmarker}{\pgfqpoint{0.000000in}{-0.027778in}}{\pgfqpoint{0.000000in}{0.000000in}}{%
\pgfpathmoveto{\pgfqpoint{0.000000in}{0.000000in}}%
\pgfpathlineto{\pgfqpoint{0.000000in}{-0.027778in}}%
\pgfusepath{stroke,fill}%
}%
\begin{pgfscope}%
\pgfsys@transformshift{1.901054in}{0.417642in}%
\pgfsys@useobject{currentmarker}{}%
\end{pgfscope}%
\end{pgfscope}%
\begin{pgfscope}%
\pgfpathrectangle{\pgfqpoint{0.514278in}{0.417642in}}{\pgfqpoint{1.884052in}{1.370688in}}%
\pgfusepath{clip}%
\pgfsetrectcap%
\pgfsetroundjoin%
\pgfsetlinewidth{0.803000pt}%
\definecolor{currentstroke}{rgb}{0.850000,0.850000,0.850000}%
\pgfsetstrokecolor{currentstroke}%
\pgfsetdash{}{0pt}%
\pgfpathmoveto{\pgfqpoint{1.927514in}{0.417642in}}%
\pgfpathlineto{\pgfqpoint{1.927514in}{1.788330in}}%
\pgfusepath{stroke}%
\end{pgfscope}%
\begin{pgfscope}%
\pgfsetbuttcap%
\pgfsetroundjoin%
\definecolor{currentfill}{rgb}{0.000000,0.000000,0.000000}%
\pgfsetfillcolor{currentfill}%
\pgfsetlinewidth{0.602250pt}%
\definecolor{currentstroke}{rgb}{0.000000,0.000000,0.000000}%
\pgfsetstrokecolor{currentstroke}%
\pgfsetdash{}{0pt}%
\pgfsys@defobject{currentmarker}{\pgfqpoint{0.000000in}{-0.027778in}}{\pgfqpoint{0.000000in}{0.000000in}}{%
\pgfpathmoveto{\pgfqpoint{0.000000in}{0.000000in}}%
\pgfpathlineto{\pgfqpoint{0.000000in}{-0.027778in}}%
\pgfusepath{stroke,fill}%
}%
\begin{pgfscope}%
\pgfsys@transformshift{1.927514in}{0.417642in}%
\pgfsys@useobject{currentmarker}{}%
\end{pgfscope}%
\end{pgfscope}%
\begin{pgfscope}%
\pgfpathrectangle{\pgfqpoint{0.514278in}{0.417642in}}{\pgfqpoint{1.884052in}{1.370688in}}%
\pgfusepath{clip}%
\pgfsetrectcap%
\pgfsetroundjoin%
\pgfsetlinewidth{0.803000pt}%
\definecolor{currentstroke}{rgb}{0.850000,0.850000,0.850000}%
\pgfsetstrokecolor{currentstroke}%
\pgfsetdash{}{0pt}%
\pgfpathmoveto{\pgfqpoint{2.106900in}{0.417642in}}%
\pgfpathlineto{\pgfqpoint{2.106900in}{1.788330in}}%
\pgfusepath{stroke}%
\end{pgfscope}%
\begin{pgfscope}%
\pgfsetbuttcap%
\pgfsetroundjoin%
\definecolor{currentfill}{rgb}{0.000000,0.000000,0.000000}%
\pgfsetfillcolor{currentfill}%
\pgfsetlinewidth{0.602250pt}%
\definecolor{currentstroke}{rgb}{0.000000,0.000000,0.000000}%
\pgfsetstrokecolor{currentstroke}%
\pgfsetdash{}{0pt}%
\pgfsys@defobject{currentmarker}{\pgfqpoint{0.000000in}{-0.027778in}}{\pgfqpoint{0.000000in}{0.000000in}}{%
\pgfpathmoveto{\pgfqpoint{0.000000in}{0.000000in}}%
\pgfpathlineto{\pgfqpoint{0.000000in}{-0.027778in}}%
\pgfusepath{stroke,fill}%
}%
\begin{pgfscope}%
\pgfsys@transformshift{2.106900in}{0.417642in}%
\pgfsys@useobject{currentmarker}{}%
\end{pgfscope}%
\end{pgfscope}%
\begin{pgfscope}%
\pgfpathrectangle{\pgfqpoint{0.514278in}{0.417642in}}{\pgfqpoint{1.884052in}{1.370688in}}%
\pgfusepath{clip}%
\pgfsetrectcap%
\pgfsetroundjoin%
\pgfsetlinewidth{0.803000pt}%
\definecolor{currentstroke}{rgb}{0.850000,0.850000,0.850000}%
\pgfsetstrokecolor{currentstroke}%
\pgfsetdash{}{0pt}%
\pgfpathmoveto{\pgfqpoint{2.197988in}{0.417642in}}%
\pgfpathlineto{\pgfqpoint{2.197988in}{1.788330in}}%
\pgfusepath{stroke}%
\end{pgfscope}%
\begin{pgfscope}%
\pgfsetbuttcap%
\pgfsetroundjoin%
\definecolor{currentfill}{rgb}{0.000000,0.000000,0.000000}%
\pgfsetfillcolor{currentfill}%
\pgfsetlinewidth{0.602250pt}%
\definecolor{currentstroke}{rgb}{0.000000,0.000000,0.000000}%
\pgfsetstrokecolor{currentstroke}%
\pgfsetdash{}{0pt}%
\pgfsys@defobject{currentmarker}{\pgfqpoint{0.000000in}{-0.027778in}}{\pgfqpoint{0.000000in}{0.000000in}}{%
\pgfpathmoveto{\pgfqpoint{0.000000in}{0.000000in}}%
\pgfpathlineto{\pgfqpoint{0.000000in}{-0.027778in}}%
\pgfusepath{stroke,fill}%
}%
\begin{pgfscope}%
\pgfsys@transformshift{2.197988in}{0.417642in}%
\pgfsys@useobject{currentmarker}{}%
\end{pgfscope}%
\end{pgfscope}%
\begin{pgfscope}%
\pgfpathrectangle{\pgfqpoint{0.514278in}{0.417642in}}{\pgfqpoint{1.884052in}{1.370688in}}%
\pgfusepath{clip}%
\pgfsetrectcap%
\pgfsetroundjoin%
\pgfsetlinewidth{0.803000pt}%
\definecolor{currentstroke}{rgb}{0.850000,0.850000,0.850000}%
\pgfsetstrokecolor{currentstroke}%
\pgfsetdash{}{0pt}%
\pgfpathmoveto{\pgfqpoint{2.262617in}{0.417642in}}%
\pgfpathlineto{\pgfqpoint{2.262617in}{1.788330in}}%
\pgfusepath{stroke}%
\end{pgfscope}%
\begin{pgfscope}%
\pgfsetbuttcap%
\pgfsetroundjoin%
\definecolor{currentfill}{rgb}{0.000000,0.000000,0.000000}%
\pgfsetfillcolor{currentfill}%
\pgfsetlinewidth{0.602250pt}%
\definecolor{currentstroke}{rgb}{0.000000,0.000000,0.000000}%
\pgfsetstrokecolor{currentstroke}%
\pgfsetdash{}{0pt}%
\pgfsys@defobject{currentmarker}{\pgfqpoint{0.000000in}{-0.027778in}}{\pgfqpoint{0.000000in}{0.000000in}}{%
\pgfpathmoveto{\pgfqpoint{0.000000in}{0.000000in}}%
\pgfpathlineto{\pgfqpoint{0.000000in}{-0.027778in}}%
\pgfusepath{stroke,fill}%
}%
\begin{pgfscope}%
\pgfsys@transformshift{2.262617in}{0.417642in}%
\pgfsys@useobject{currentmarker}{}%
\end{pgfscope}%
\end{pgfscope}%
\begin{pgfscope}%
\pgfpathrectangle{\pgfqpoint{0.514278in}{0.417642in}}{\pgfqpoint{1.884052in}{1.370688in}}%
\pgfusepath{clip}%
\pgfsetrectcap%
\pgfsetroundjoin%
\pgfsetlinewidth{0.803000pt}%
\definecolor{currentstroke}{rgb}{0.850000,0.850000,0.850000}%
\pgfsetstrokecolor{currentstroke}%
\pgfsetdash{}{0pt}%
\pgfpathmoveto{\pgfqpoint{2.312746in}{0.417642in}}%
\pgfpathlineto{\pgfqpoint{2.312746in}{1.788330in}}%
\pgfusepath{stroke}%
\end{pgfscope}%
\begin{pgfscope}%
\pgfsetbuttcap%
\pgfsetroundjoin%
\definecolor{currentfill}{rgb}{0.000000,0.000000,0.000000}%
\pgfsetfillcolor{currentfill}%
\pgfsetlinewidth{0.602250pt}%
\definecolor{currentstroke}{rgb}{0.000000,0.000000,0.000000}%
\pgfsetstrokecolor{currentstroke}%
\pgfsetdash{}{0pt}%
\pgfsys@defobject{currentmarker}{\pgfqpoint{0.000000in}{-0.027778in}}{\pgfqpoint{0.000000in}{0.000000in}}{%
\pgfpathmoveto{\pgfqpoint{0.000000in}{0.000000in}}%
\pgfpathlineto{\pgfqpoint{0.000000in}{-0.027778in}}%
\pgfusepath{stroke,fill}%
}%
\begin{pgfscope}%
\pgfsys@transformshift{2.312746in}{0.417642in}%
\pgfsys@useobject{currentmarker}{}%
\end{pgfscope}%
\end{pgfscope}%
\begin{pgfscope}%
\pgfpathrectangle{\pgfqpoint{0.514278in}{0.417642in}}{\pgfqpoint{1.884052in}{1.370688in}}%
\pgfusepath{clip}%
\pgfsetrectcap%
\pgfsetroundjoin%
\pgfsetlinewidth{0.803000pt}%
\definecolor{currentstroke}{rgb}{0.850000,0.850000,0.850000}%
\pgfsetstrokecolor{currentstroke}%
\pgfsetdash{}{0pt}%
\pgfpathmoveto{\pgfqpoint{2.353705in}{0.417642in}}%
\pgfpathlineto{\pgfqpoint{2.353705in}{1.788330in}}%
\pgfusepath{stroke}%
\end{pgfscope}%
\begin{pgfscope}%
\pgfsetbuttcap%
\pgfsetroundjoin%
\definecolor{currentfill}{rgb}{0.000000,0.000000,0.000000}%
\pgfsetfillcolor{currentfill}%
\pgfsetlinewidth{0.602250pt}%
\definecolor{currentstroke}{rgb}{0.000000,0.000000,0.000000}%
\pgfsetstrokecolor{currentstroke}%
\pgfsetdash{}{0pt}%
\pgfsys@defobject{currentmarker}{\pgfqpoint{0.000000in}{-0.027778in}}{\pgfqpoint{0.000000in}{0.000000in}}{%
\pgfpathmoveto{\pgfqpoint{0.000000in}{0.000000in}}%
\pgfpathlineto{\pgfqpoint{0.000000in}{-0.027778in}}%
\pgfusepath{stroke,fill}%
}%
\begin{pgfscope}%
\pgfsys@transformshift{2.353705in}{0.417642in}%
\pgfsys@useobject{currentmarker}{}%
\end{pgfscope}%
\end{pgfscope}%
\begin{pgfscope}%
\pgfpathrectangle{\pgfqpoint{0.514278in}{0.417642in}}{\pgfqpoint{1.884052in}{1.370688in}}%
\pgfusepath{clip}%
\pgfsetrectcap%
\pgfsetroundjoin%
\pgfsetlinewidth{0.803000pt}%
\definecolor{currentstroke}{rgb}{0.850000,0.850000,0.850000}%
\pgfsetstrokecolor{currentstroke}%
\pgfsetdash{}{0pt}%
\pgfpathmoveto{\pgfqpoint{2.388335in}{0.417642in}}%
\pgfpathlineto{\pgfqpoint{2.388335in}{1.788330in}}%
\pgfusepath{stroke}%
\end{pgfscope}%
\begin{pgfscope}%
\pgfsetbuttcap%
\pgfsetroundjoin%
\definecolor{currentfill}{rgb}{0.000000,0.000000,0.000000}%
\pgfsetfillcolor{currentfill}%
\pgfsetlinewidth{0.602250pt}%
\definecolor{currentstroke}{rgb}{0.000000,0.000000,0.000000}%
\pgfsetstrokecolor{currentstroke}%
\pgfsetdash{}{0pt}%
\pgfsys@defobject{currentmarker}{\pgfqpoint{0.000000in}{-0.027778in}}{\pgfqpoint{0.000000in}{0.000000in}}{%
\pgfpathmoveto{\pgfqpoint{0.000000in}{0.000000in}}%
\pgfpathlineto{\pgfqpoint{0.000000in}{-0.027778in}}%
\pgfusepath{stroke,fill}%
}%
\begin{pgfscope}%
\pgfsys@transformshift{2.388335in}{0.417642in}%
\pgfsys@useobject{currentmarker}{}%
\end{pgfscope}%
\end{pgfscope}%
\begin{pgfscope}%
\definecolor{textcolor}{rgb}{0.000000,0.000000,0.000000}%
\pgfsetstrokecolor{textcolor}%
\pgfsetfillcolor{textcolor}%
\pgftext[x=1.456304in,y=0.165003in,,top]{\color{textcolor}\rmfamily\fontsize{10.000000}{12.000000}\selectfont Frequency in \(\displaystyle \unit{\Hz}\)}%
\end{pgfscope}%
\begin{pgfscope}%
\pgfpathrectangle{\pgfqpoint{0.514278in}{0.417642in}}{\pgfqpoint{1.884052in}{1.370688in}}%
\pgfusepath{clip}%
\pgfsetrectcap%
\pgfsetroundjoin%
\pgfsetlinewidth{0.803000pt}%
\definecolor{currentstroke}{rgb}{0.450000,0.450000,0.450000}%
\pgfsetstrokecolor{currentstroke}%
\pgfsetdash{}{0pt}%
\pgfpathmoveto{\pgfqpoint{0.514278in}{0.640555in}}%
\pgfpathlineto{\pgfqpoint{2.398330in}{0.640555in}}%
\pgfusepath{stroke}%
\end{pgfscope}%
\begin{pgfscope}%
\pgfsetbuttcap%
\pgfsetroundjoin%
\definecolor{currentfill}{rgb}{0.000000,0.000000,0.000000}%
\pgfsetfillcolor{currentfill}%
\pgfsetlinewidth{0.803000pt}%
\definecolor{currentstroke}{rgb}{0.000000,0.000000,0.000000}%
\pgfsetstrokecolor{currentstroke}%
\pgfsetdash{}{0pt}%
\pgfsys@defobject{currentmarker}{\pgfqpoint{-0.048611in}{0.000000in}}{\pgfqpoint{-0.000000in}{0.000000in}}{%
\pgfpathmoveto{\pgfqpoint{-0.000000in}{0.000000in}}%
\pgfpathlineto{\pgfqpoint{-0.048611in}{0.000000in}}%
\pgfusepath{stroke,fill}%
}%
\begin{pgfscope}%
\pgfsys@transformshift{0.514278in}{0.640555in}%
\pgfsys@useobject{currentmarker}{}%
\end{pgfscope}%
\end{pgfscope}%
\begin{pgfscope}%
\definecolor{textcolor}{rgb}{0.000000,0.000000,0.000000}%
\pgfsetstrokecolor{textcolor}%
\pgfsetfillcolor{textcolor}%
\pgftext[x=0.241129in, y=0.601402in, left, base]{\color{textcolor}\rmfamily\fontsize{8.000000}{9.600000}\selectfont \(\displaystyle {10^{0}}\)}%
\end{pgfscope}%
\begin{pgfscope}%
\pgfpathrectangle{\pgfqpoint{0.514278in}{0.417642in}}{\pgfqpoint{1.884052in}{1.370688in}}%
\pgfusepath{clip}%
\pgfsetrectcap%
\pgfsetroundjoin%
\pgfsetlinewidth{0.803000pt}%
\definecolor{currentstroke}{rgb}{0.450000,0.450000,0.450000}%
\pgfsetstrokecolor{currentstroke}%
\pgfsetdash{}{0pt}%
\pgfpathmoveto{\pgfqpoint{0.514278in}{0.983227in}}%
\pgfpathlineto{\pgfqpoint{2.398330in}{0.983227in}}%
\pgfusepath{stroke}%
\end{pgfscope}%
\begin{pgfscope}%
\pgfsetbuttcap%
\pgfsetroundjoin%
\definecolor{currentfill}{rgb}{0.000000,0.000000,0.000000}%
\pgfsetfillcolor{currentfill}%
\pgfsetlinewidth{0.803000pt}%
\definecolor{currentstroke}{rgb}{0.000000,0.000000,0.000000}%
\pgfsetstrokecolor{currentstroke}%
\pgfsetdash{}{0pt}%
\pgfsys@defobject{currentmarker}{\pgfqpoint{-0.048611in}{0.000000in}}{\pgfqpoint{-0.000000in}{0.000000in}}{%
\pgfpathmoveto{\pgfqpoint{-0.000000in}{0.000000in}}%
\pgfpathlineto{\pgfqpoint{-0.048611in}{0.000000in}}%
\pgfusepath{stroke,fill}%
}%
\begin{pgfscope}%
\pgfsys@transformshift{0.514278in}{0.983227in}%
\pgfsys@useobject{currentmarker}{}%
\end{pgfscope}%
\end{pgfscope}%
\begin{pgfscope}%
\definecolor{textcolor}{rgb}{0.000000,0.000000,0.000000}%
\pgfsetstrokecolor{textcolor}%
\pgfsetfillcolor{textcolor}%
\pgftext[x=0.241129in, y=0.944074in, left, base]{\color{textcolor}\rmfamily\fontsize{8.000000}{9.600000}\selectfont \(\displaystyle {10^{2}}\)}%
\end{pgfscope}%
\begin{pgfscope}%
\pgfpathrectangle{\pgfqpoint{0.514278in}{0.417642in}}{\pgfqpoint{1.884052in}{1.370688in}}%
\pgfusepath{clip}%
\pgfsetrectcap%
\pgfsetroundjoin%
\pgfsetlinewidth{0.803000pt}%
\definecolor{currentstroke}{rgb}{0.450000,0.450000,0.450000}%
\pgfsetstrokecolor{currentstroke}%
\pgfsetdash{}{0pt}%
\pgfpathmoveto{\pgfqpoint{0.514278in}{1.325899in}}%
\pgfpathlineto{\pgfqpoint{2.398330in}{1.325899in}}%
\pgfusepath{stroke}%
\end{pgfscope}%
\begin{pgfscope}%
\pgfsetbuttcap%
\pgfsetroundjoin%
\definecolor{currentfill}{rgb}{0.000000,0.000000,0.000000}%
\pgfsetfillcolor{currentfill}%
\pgfsetlinewidth{0.803000pt}%
\definecolor{currentstroke}{rgb}{0.000000,0.000000,0.000000}%
\pgfsetstrokecolor{currentstroke}%
\pgfsetdash{}{0pt}%
\pgfsys@defobject{currentmarker}{\pgfqpoint{-0.048611in}{0.000000in}}{\pgfqpoint{-0.000000in}{0.000000in}}{%
\pgfpathmoveto{\pgfqpoint{-0.000000in}{0.000000in}}%
\pgfpathlineto{\pgfqpoint{-0.048611in}{0.000000in}}%
\pgfusepath{stroke,fill}%
}%
\begin{pgfscope}%
\pgfsys@transformshift{0.514278in}{1.325899in}%
\pgfsys@useobject{currentmarker}{}%
\end{pgfscope}%
\end{pgfscope}%
\begin{pgfscope}%
\definecolor{textcolor}{rgb}{0.000000,0.000000,0.000000}%
\pgfsetstrokecolor{textcolor}%
\pgfsetfillcolor{textcolor}%
\pgftext[x=0.241129in, y=1.286746in, left, base]{\color{textcolor}\rmfamily\fontsize{8.000000}{9.600000}\selectfont \(\displaystyle {10^{4}}\)}%
\end{pgfscope}%
\begin{pgfscope}%
\pgfpathrectangle{\pgfqpoint{0.514278in}{0.417642in}}{\pgfqpoint{1.884052in}{1.370688in}}%
\pgfusepath{clip}%
\pgfsetrectcap%
\pgfsetroundjoin%
\pgfsetlinewidth{0.803000pt}%
\definecolor{currentstroke}{rgb}{0.450000,0.450000,0.450000}%
\pgfsetstrokecolor{currentstroke}%
\pgfsetdash{}{0pt}%
\pgfpathmoveto{\pgfqpoint{0.514278in}{1.668571in}}%
\pgfpathlineto{\pgfqpoint{2.398330in}{1.668571in}}%
\pgfusepath{stroke}%
\end{pgfscope}%
\begin{pgfscope}%
\pgfsetbuttcap%
\pgfsetroundjoin%
\definecolor{currentfill}{rgb}{0.000000,0.000000,0.000000}%
\pgfsetfillcolor{currentfill}%
\pgfsetlinewidth{0.803000pt}%
\definecolor{currentstroke}{rgb}{0.000000,0.000000,0.000000}%
\pgfsetstrokecolor{currentstroke}%
\pgfsetdash{}{0pt}%
\pgfsys@defobject{currentmarker}{\pgfqpoint{-0.048611in}{0.000000in}}{\pgfqpoint{-0.000000in}{0.000000in}}{%
\pgfpathmoveto{\pgfqpoint{-0.000000in}{0.000000in}}%
\pgfpathlineto{\pgfqpoint{-0.048611in}{0.000000in}}%
\pgfusepath{stroke,fill}%
}%
\begin{pgfscope}%
\pgfsys@transformshift{0.514278in}{1.668571in}%
\pgfsys@useobject{currentmarker}{}%
\end{pgfscope}%
\end{pgfscope}%
\begin{pgfscope}%
\definecolor{textcolor}{rgb}{0.000000,0.000000,0.000000}%
\pgfsetstrokecolor{textcolor}%
\pgfsetfillcolor{textcolor}%
\pgftext[x=0.241129in, y=1.629418in, left, base]{\color{textcolor}\rmfamily\fontsize{8.000000}{9.600000}\selectfont \(\displaystyle {10^{6}}\)}%
\end{pgfscope}%
\begin{pgfscope}%
\definecolor{textcolor}{rgb}{0.000000,0.000000,0.000000}%
\pgfsetstrokecolor{textcolor}%
\pgfsetfillcolor{textcolor}%
\pgftext[x=0.185574in,y=1.102986in,,bottom,rotate=90.000000]{\color{textcolor}\rmfamily\fontsize{10.000000}{12.000000}\selectfont  \(\displaystyle S_y(f)\) in \(\displaystyle \unit{1 \per \Hz}\)}%
\end{pgfscope}%
\begin{pgfscope}%
\pgfpathrectangle{\pgfqpoint{0.514278in}{0.417642in}}{\pgfqpoint{1.884052in}{1.370688in}}%
\pgfusepath{clip}%
\pgfsetbuttcap%
\pgfsetroundjoin%
\pgfsetlinewidth{1.505625pt}%
\definecolor{currentstroke}{rgb}{0.835294,0.368627,0.000000}%
\pgfsetstrokecolor{currentstroke}%
\pgfsetdash{{5.550000pt}{2.400000pt}}{0.000000pt}%
\pgfpathmoveto{\pgfqpoint{0.599917in}{1.738185in}}%
\pgfpathlineto{\pgfqpoint{2.312691in}{0.603558in}}%
\pgfpathlineto{\pgfqpoint{2.312691in}{0.603558in}}%
\pgfusepath{stroke}%
\end{pgfscope}%
\begin{pgfscope}%
\pgfpathrectangle{\pgfqpoint{0.514278in}{0.417642in}}{\pgfqpoint{1.884052in}{1.370688in}}%
\pgfusepath{clip}%
\pgfsetbuttcap%
\pgfsetroundjoin%
\definecolor{currentfill}{rgb}{0.835294,0.368627,0.000000}%
\pgfsetfillcolor{currentfill}%
\pgfsetlinewidth{1.003750pt}%
\definecolor{currentstroke}{rgb}{0.835294,0.368627,0.000000}%
\pgfsetstrokecolor{currentstroke}%
\pgfsetdash{}{0pt}%
\pgfsys@defobject{currentmarker}{\pgfqpoint{-0.006944in}{-0.006944in}}{\pgfqpoint{0.006944in}{0.006944in}}{%
\pgfpathmoveto{\pgfqpoint{0.000000in}{-0.006944in}}%
\pgfpathcurveto{\pgfqpoint{0.001842in}{-0.006944in}}{\pgfqpoint{0.003608in}{-0.006213in}}{\pgfqpoint{0.004910in}{-0.004910in}}%
\pgfpathcurveto{\pgfqpoint{0.006213in}{-0.003608in}}{\pgfqpoint{0.006944in}{-0.001842in}}{\pgfqpoint{0.006944in}{0.000000in}}%
\pgfpathcurveto{\pgfqpoint{0.006944in}{0.001842in}}{\pgfqpoint{0.006213in}{0.003608in}}{\pgfqpoint{0.004910in}{0.004910in}}%
\pgfpathcurveto{\pgfqpoint{0.003608in}{0.006213in}}{\pgfqpoint{0.001842in}{0.006944in}}{\pgfqpoint{0.000000in}{0.006944in}}%
\pgfpathcurveto{\pgfqpoint{-0.001842in}{0.006944in}}{\pgfqpoint{-0.003608in}{0.006213in}}{\pgfqpoint{-0.004910in}{0.004910in}}%
\pgfpathcurveto{\pgfqpoint{-0.006213in}{0.003608in}}{\pgfqpoint{-0.006944in}{0.001842in}}{\pgfqpoint{-0.006944in}{0.000000in}}%
\pgfpathcurveto{\pgfqpoint{-0.006944in}{-0.001842in}}{\pgfqpoint{-0.006213in}{-0.003608in}}{\pgfqpoint{-0.004910in}{-0.004910in}}%
\pgfpathcurveto{\pgfqpoint{-0.003608in}{-0.006213in}}{\pgfqpoint{-0.001842in}{-0.006944in}}{\pgfqpoint{0.000000in}{-0.006944in}}%
\pgfpathlineto{\pgfqpoint{0.000000in}{-0.006944in}}%
\pgfpathclose%
\pgfusepath{stroke,fill}%
}%
\begin{pgfscope}%
\pgfsys@transformshift{-226.701573in}{1.699596in}%
\pgfsys@useobject{currentmarker}{}%
\end{pgfscope}%
\begin{pgfscope}%
\pgfsys@transformshift{0.599917in}{1.767760in}%
\pgfsys@useobject{currentmarker}{}%
\end{pgfscope}%
\begin{pgfscope}%
\pgfsys@transformshift{0.755634in}{1.633926in}%
\pgfsys@useobject{currentmarker}{}%
\end{pgfscope}%
\begin{pgfscope}%
\pgfsys@transformshift{0.846722in}{1.542294in}%
\pgfsys@useobject{currentmarker}{}%
\end{pgfscope}%
\begin{pgfscope}%
\pgfsys@transformshift{0.911351in}{1.484904in}%
\pgfsys@useobject{currentmarker}{}%
\end{pgfscope}%
\begin{pgfscope}%
\pgfsys@transformshift{0.961480in}{1.467589in}%
\pgfsys@useobject{currentmarker}{}%
\end{pgfscope}%
\begin{pgfscope}%
\pgfsys@transformshift{1.002439in}{1.396364in}%
\pgfsys@useobject{currentmarker}{}%
\end{pgfscope}%
\begin{pgfscope}%
\pgfsys@transformshift{1.037069in}{1.455179in}%
\pgfsys@useobject{currentmarker}{}%
\end{pgfscope}%
\begin{pgfscope}%
\pgfsys@transformshift{1.067067in}{1.459825in}%
\pgfsys@useobject{currentmarker}{}%
\end{pgfscope}%
\begin{pgfscope}%
\pgfsys@transformshift{1.093527in}{1.432011in}%
\pgfsys@useobject{currentmarker}{}%
\end{pgfscope}%
\begin{pgfscope}%
\pgfsys@transformshift{1.117197in}{1.407185in}%
\pgfsys@useobject{currentmarker}{}%
\end{pgfscope}%
\begin{pgfscope}%
\pgfsys@transformshift{1.138608in}{1.363179in}%
\pgfsys@useobject{currentmarker}{}%
\end{pgfscope}%
\begin{pgfscope}%
\pgfsys@transformshift{1.158156in}{1.390307in}%
\pgfsys@useobject{currentmarker}{}%
\end{pgfscope}%
\begin{pgfscope}%
\pgfsys@transformshift{1.176137in}{1.358330in}%
\pgfsys@useobject{currentmarker}{}%
\end{pgfscope}%
\begin{pgfscope}%
\pgfsys@transformshift{1.192786in}{1.282640in}%
\pgfsys@useobject{currentmarker}{}%
\end{pgfscope}%
\begin{pgfscope}%
\pgfsys@transformshift{1.208285in}{1.342931in}%
\pgfsys@useobject{currentmarker}{}%
\end{pgfscope}%
\begin{pgfscope}%
\pgfsys@transformshift{1.222784in}{1.362450in}%
\pgfsys@useobject{currentmarker}{}%
\end{pgfscope}%
\begin{pgfscope}%
\pgfsys@transformshift{1.236403in}{1.334753in}%
\pgfsys@useobject{currentmarker}{}%
\end{pgfscope}%
\begin{pgfscope}%
\pgfsys@transformshift{1.249244in}{1.223887in}%
\pgfsys@useobject{currentmarker}{}%
\end{pgfscope}%
\begin{pgfscope}%
\pgfsys@transformshift{1.261390in}{1.267876in}%
\pgfsys@useobject{currentmarker}{}%
\end{pgfscope}%
\begin{pgfscope}%
\pgfsys@transformshift{1.272914in}{1.329337in}%
\pgfsys@useobject{currentmarker}{}%
\end{pgfscope}%
\begin{pgfscope}%
\pgfsys@transformshift{1.283874in}{1.343124in}%
\pgfsys@useobject{currentmarker}{}%
\end{pgfscope}%
\begin{pgfscope}%
\pgfsys@transformshift{1.294325in}{1.301182in}%
\pgfsys@useobject{currentmarker}{}%
\end{pgfscope}%
\begin{pgfscope}%
\pgfsys@transformshift{1.304311in}{1.275590in}%
\pgfsys@useobject{currentmarker}{}%
\end{pgfscope}%
\begin{pgfscope}%
\pgfsys@transformshift{1.313872in}{1.224897in}%
\pgfsys@useobject{currentmarker}{}%
\end{pgfscope}%
\begin{pgfscope}%
\pgfsys@transformshift{1.323043in}{1.244226in}%
\pgfsys@useobject{currentmarker}{}%
\end{pgfscope}%
\begin{pgfscope}%
\pgfsys@transformshift{1.331854in}{1.221070in}%
\pgfsys@useobject{currentmarker}{}%
\end{pgfscope}%
\begin{pgfscope}%
\pgfsys@transformshift{1.340333in}{1.212021in}%
\pgfsys@useobject{currentmarker}{}%
\end{pgfscope}%
\begin{pgfscope}%
\pgfsys@transformshift{1.348503in}{1.255859in}%
\pgfsys@useobject{currentmarker}{}%
\end{pgfscope}%
\begin{pgfscope}%
\pgfsys@transformshift{1.356386in}{1.259043in}%
\pgfsys@useobject{currentmarker}{}%
\end{pgfscope}%
\begin{pgfscope}%
\pgfsys@transformshift{1.364002in}{1.213137in}%
\pgfsys@useobject{currentmarker}{}%
\end{pgfscope}%
\begin{pgfscope}%
\pgfsys@transformshift{1.371368in}{1.234630in}%
\pgfsys@useobject{currentmarker}{}%
\end{pgfscope}%
\begin{pgfscope}%
\pgfsys@transformshift{1.378501in}{1.216343in}%
\pgfsys@useobject{currentmarker}{}%
\end{pgfscope}%
\begin{pgfscope}%
\pgfsys@transformshift{1.385414in}{1.197913in}%
\pgfsys@useobject{currentmarker}{}%
\end{pgfscope}%
\begin{pgfscope}%
\pgfsys@transformshift{1.392120in}{1.232938in}%
\pgfsys@useobject{currentmarker}{}%
\end{pgfscope}%
\begin{pgfscope}%
\pgfsys@transformshift{1.398632in}{1.191697in}%
\pgfsys@useobject{currentmarker}{}%
\end{pgfscope}%
\begin{pgfscope}%
\pgfsys@transformshift{1.404961in}{1.221783in}%
\pgfsys@useobject{currentmarker}{}%
\end{pgfscope}%
\begin{pgfscope}%
\pgfsys@transformshift{1.411116in}{1.208746in}%
\pgfsys@useobject{currentmarker}{}%
\end{pgfscope}%
\begin{pgfscope}%
\pgfsys@transformshift{1.417107in}{1.176183in}%
\pgfsys@useobject{currentmarker}{}%
\end{pgfscope}%
\begin{pgfscope}%
\pgfsys@transformshift{1.422943in}{1.131458in}%
\pgfsys@useobject{currentmarker}{}%
\end{pgfscope}%
\begin{pgfscope}%
\pgfsys@transformshift{1.428630in}{1.176332in}%
\pgfsys@useobject{currentmarker}{}%
\end{pgfscope}%
\begin{pgfscope}%
\pgfsys@transformshift{1.434178in}{1.202293in}%
\pgfsys@useobject{currentmarker}{}%
\end{pgfscope}%
\begin{pgfscope}%
\pgfsys@transformshift{1.439591in}{1.213066in}%
\pgfsys@useobject{currentmarker}{}%
\end{pgfscope}%
\begin{pgfscope}%
\pgfsys@transformshift{1.444877in}{1.227325in}%
\pgfsys@useobject{currentmarker}{}%
\end{pgfscope}%
\begin{pgfscope}%
\pgfsys@transformshift{1.450042in}{1.175888in}%
\pgfsys@useobject{currentmarker}{}%
\end{pgfscope}%
\begin{pgfscope}%
\pgfsys@transformshift{1.455090in}{1.095359in}%
\pgfsys@useobject{currentmarker}{}%
\end{pgfscope}%
\begin{pgfscope}%
\pgfsys@transformshift{1.460028in}{1.120846in}%
\pgfsys@useobject{currentmarker}{}%
\end{pgfscope}%
\begin{pgfscope}%
\pgfsys@transformshift{1.464859in}{1.142584in}%
\pgfsys@useobject{currentmarker}{}%
\end{pgfscope}%
\begin{pgfscope}%
\pgfsys@transformshift{1.469589in}{1.168559in}%
\pgfsys@useobject{currentmarker}{}%
\end{pgfscope}%
\begin{pgfscope}%
\pgfsys@transformshift{1.474221in}{1.154005in}%
\pgfsys@useobject{currentmarker}{}%
\end{pgfscope}%
\begin{pgfscope}%
\pgfsys@transformshift{1.478760in}{1.105622in}%
\pgfsys@useobject{currentmarker}{}%
\end{pgfscope}%
\begin{pgfscope}%
\pgfsys@transformshift{1.483209in}{1.045925in}%
\pgfsys@useobject{currentmarker}{}%
\end{pgfscope}%
\begin{pgfscope}%
\pgfsys@transformshift{1.487571in}{1.082888in}%
\pgfsys@useobject{currentmarker}{}%
\end{pgfscope}%
\begin{pgfscope}%
\pgfsys@transformshift{1.491850in}{1.148928in}%
\pgfsys@useobject{currentmarker}{}%
\end{pgfscope}%
\begin{pgfscope}%
\pgfsys@transformshift{1.496049in}{1.150320in}%
\pgfsys@useobject{currentmarker}{}%
\end{pgfscope}%
\begin{pgfscope}%
\pgfsys@transformshift{1.500172in}{1.159522in}%
\pgfsys@useobject{currentmarker}{}%
\end{pgfscope}%
\begin{pgfscope}%
\pgfsys@transformshift{1.504219in}{1.163355in}%
\pgfsys@useobject{currentmarker}{}%
\end{pgfscope}%
\begin{pgfscope}%
\pgfsys@transformshift{1.508196in}{1.173362in}%
\pgfsys@useobject{currentmarker}{}%
\end{pgfscope}%
\begin{pgfscope}%
\pgfsys@transformshift{1.512103in}{1.164352in}%
\pgfsys@useobject{currentmarker}{}%
\end{pgfscope}%
\begin{pgfscope}%
\pgfsys@transformshift{1.515943in}{1.141812in}%
\pgfsys@useobject{currentmarker}{}%
\end{pgfscope}%
\begin{pgfscope}%
\pgfsys@transformshift{1.519719in}{1.120098in}%
\pgfsys@useobject{currentmarker}{}%
\end{pgfscope}%
\begin{pgfscope}%
\pgfsys@transformshift{1.523432in}{1.112526in}%
\pgfsys@useobject{currentmarker}{}%
\end{pgfscope}%
\begin{pgfscope}%
\pgfsys@transformshift{1.527085in}{1.055989in}%
\pgfsys@useobject{currentmarker}{}%
\end{pgfscope}%
\begin{pgfscope}%
\pgfsys@transformshift{1.530680in}{1.062603in}%
\pgfsys@useobject{currentmarker}{}%
\end{pgfscope}%
\begin{pgfscope}%
\pgfsys@transformshift{1.534217in}{1.096690in}%
\pgfsys@useobject{currentmarker}{}%
\end{pgfscope}%
\begin{pgfscope}%
\pgfsys@transformshift{1.537700in}{1.068067in}%
\pgfsys@useobject{currentmarker}{}%
\end{pgfscope}%
\begin{pgfscope}%
\pgfsys@transformshift{1.541130in}{1.065379in}%
\pgfsys@useobject{currentmarker}{}%
\end{pgfscope}%
\begin{pgfscope}%
\pgfsys@transformshift{1.544509in}{1.095491in}%
\pgfsys@useobject{currentmarker}{}%
\end{pgfscope}%
\begin{pgfscope}%
\pgfsys@transformshift{1.547837in}{1.127129in}%
\pgfsys@useobject{currentmarker}{}%
\end{pgfscope}%
\begin{pgfscope}%
\pgfsys@transformshift{1.551117in}{1.117168in}%
\pgfsys@useobject{currentmarker}{}%
\end{pgfscope}%
\begin{pgfscope}%
\pgfsys@transformshift{1.554349in}{1.048005in}%
\pgfsys@useobject{currentmarker}{}%
\end{pgfscope}%
\begin{pgfscope}%
\pgfsys@transformshift{1.557536in}{1.069736in}%
\pgfsys@useobject{currentmarker}{}%
\end{pgfscope}%
\begin{pgfscope}%
\pgfsys@transformshift{1.560678in}{1.118483in}%
\pgfsys@useobject{currentmarker}{}%
\end{pgfscope}%
\begin{pgfscope}%
\pgfsys@transformshift{1.563776in}{1.120149in}%
\pgfsys@useobject{currentmarker}{}%
\end{pgfscope}%
\begin{pgfscope}%
\pgfsys@transformshift{1.566833in}{1.081543in}%
\pgfsys@useobject{currentmarker}{}%
\end{pgfscope}%
\begin{pgfscope}%
\pgfsys@transformshift{1.569848in}{1.084658in}%
\pgfsys@useobject{currentmarker}{}%
\end{pgfscope}%
\begin{pgfscope}%
\pgfsys@transformshift{1.572824in}{1.103899in}%
\pgfsys@useobject{currentmarker}{}%
\end{pgfscope}%
\begin{pgfscope}%
\pgfsys@transformshift{1.575761in}{1.081469in}%
\pgfsys@useobject{currentmarker}{}%
\end{pgfscope}%
\begin{pgfscope}%
\pgfsys@transformshift{1.578659in}{1.031169in}%
\pgfsys@useobject{currentmarker}{}%
\end{pgfscope}%
\begin{pgfscope}%
\pgfsys@transformshift{1.581521in}{1.037846in}%
\pgfsys@useobject{currentmarker}{}%
\end{pgfscope}%
\begin{pgfscope}%
\pgfsys@transformshift{1.584347in}{1.089595in}%
\pgfsys@useobject{currentmarker}{}%
\end{pgfscope}%
\begin{pgfscope}%
\pgfsys@transformshift{1.587138in}{1.077670in}%
\pgfsys@useobject{currentmarker}{}%
\end{pgfscope}%
\begin{pgfscope}%
\pgfsys@transformshift{1.589894in}{1.064079in}%
\pgfsys@useobject{currentmarker}{}%
\end{pgfscope}%
\begin{pgfscope}%
\pgfsys@transformshift{1.592617in}{1.043820in}%
\pgfsys@useobject{currentmarker}{}%
\end{pgfscope}%
\begin{pgfscope}%
\pgfsys@transformshift{1.595308in}{1.024211in}%
\pgfsys@useobject{currentmarker}{}%
\end{pgfscope}%
\begin{pgfscope}%
\pgfsys@transformshift{1.597966in}{1.021445in}%
\pgfsys@useobject{currentmarker}{}%
\end{pgfscope}%
\begin{pgfscope}%
\pgfsys@transformshift{1.600594in}{1.057507in}%
\pgfsys@useobject{currentmarker}{}%
\end{pgfscope}%
\begin{pgfscope}%
\pgfsys@transformshift{1.603191in}{1.020981in}%
\pgfsys@useobject{currentmarker}{}%
\end{pgfscope}%
\begin{pgfscope}%
\pgfsys@transformshift{1.605759in}{1.041287in}%
\pgfsys@useobject{currentmarker}{}%
\end{pgfscope}%
\begin{pgfscope}%
\pgfsys@transformshift{1.608297in}{0.994575in}%
\pgfsys@useobject{currentmarker}{}%
\end{pgfscope}%
\begin{pgfscope}%
\pgfsys@transformshift{1.610807in}{0.997938in}%
\pgfsys@useobject{currentmarker}{}%
\end{pgfscope}%
\begin{pgfscope}%
\pgfsys@transformshift{1.613290in}{1.066445in}%
\pgfsys@useobject{currentmarker}{}%
\end{pgfscope}%
\begin{pgfscope}%
\pgfsys@transformshift{1.615745in}{1.056958in}%
\pgfsys@useobject{currentmarker}{}%
\end{pgfscope}%
\begin{pgfscope}%
\pgfsys@transformshift{1.618173in}{1.063108in}%
\pgfsys@useobject{currentmarker}{}%
\end{pgfscope}%
\begin{pgfscope}%
\pgfsys@transformshift{1.620576in}{1.051093in}%
\pgfsys@useobject{currentmarker}{}%
\end{pgfscope}%
\begin{pgfscope}%
\pgfsys@transformshift{1.622953in}{1.030392in}%
\pgfsys@useobject{currentmarker}{}%
\end{pgfscope}%
\begin{pgfscope}%
\pgfsys@transformshift{1.625306in}{1.046458in}%
\pgfsys@useobject{currentmarker}{}%
\end{pgfscope}%
\begin{pgfscope}%
\pgfsys@transformshift{1.627634in}{1.031185in}%
\pgfsys@useobject{currentmarker}{}%
\end{pgfscope}%
\begin{pgfscope}%
\pgfsys@transformshift{1.629938in}{1.014596in}%
\pgfsys@useobject{currentmarker}{}%
\end{pgfscope}%
\begin{pgfscope}%
\pgfsys@transformshift{1.632219in}{1.033804in}%
\pgfsys@useobject{currentmarker}{}%
\end{pgfscope}%
\begin{pgfscope}%
\pgfsys@transformshift{1.634477in}{1.016011in}%
\pgfsys@useobject{currentmarker}{}%
\end{pgfscope}%
\begin{pgfscope}%
\pgfsys@transformshift{1.636712in}{0.987823in}%
\pgfsys@useobject{currentmarker}{}%
\end{pgfscope}%
\begin{pgfscope}%
\pgfsys@transformshift{1.638925in}{1.021058in}%
\pgfsys@useobject{currentmarker}{}%
\end{pgfscope}%
\begin{pgfscope}%
\pgfsys@transformshift{1.641117in}{1.027398in}%
\pgfsys@useobject{currentmarker}{}%
\end{pgfscope}%
\begin{pgfscope}%
\pgfsys@transformshift{1.643288in}{1.065967in}%
\pgfsys@useobject{currentmarker}{}%
\end{pgfscope}%
\begin{pgfscope}%
\pgfsys@transformshift{1.645437in}{1.029419in}%
\pgfsys@useobject{currentmarker}{}%
\end{pgfscope}%
\begin{pgfscope}%
\pgfsys@transformshift{1.647567in}{1.031442in}%
\pgfsys@useobject{currentmarker}{}%
\end{pgfscope}%
\begin{pgfscope}%
\pgfsys@transformshift{1.649676in}{1.042601in}%
\pgfsys@useobject{currentmarker}{}%
\end{pgfscope}%
\begin{pgfscope}%
\pgfsys@transformshift{1.651766in}{1.046952in}%
\pgfsys@useobject{currentmarker}{}%
\end{pgfscope}%
\begin{pgfscope}%
\pgfsys@transformshift{1.653837in}{1.062189in}%
\pgfsys@useobject{currentmarker}{}%
\end{pgfscope}%
\begin{pgfscope}%
\pgfsys@transformshift{1.655888in}{1.049449in}%
\pgfsys@useobject{currentmarker}{}%
\end{pgfscope}%
\begin{pgfscope}%
\pgfsys@transformshift{1.657921in}{1.013267in}%
\pgfsys@useobject{currentmarker}{}%
\end{pgfscope}%
\begin{pgfscope}%
\pgfsys@transformshift{1.659936in}{1.039645in}%
\pgfsys@useobject{currentmarker}{}%
\end{pgfscope}%
\begin{pgfscope}%
\pgfsys@transformshift{1.661933in}{1.020337in}%
\pgfsys@useobject{currentmarker}{}%
\end{pgfscope}%
\begin{pgfscope}%
\pgfsys@transformshift{1.663912in}{1.018009in}%
\pgfsys@useobject{currentmarker}{}%
\end{pgfscope}%
\begin{pgfscope}%
\pgfsys@transformshift{1.665874in}{1.015334in}%
\pgfsys@useobject{currentmarker}{}%
\end{pgfscope}%
\begin{pgfscope}%
\pgfsys@transformshift{1.667819in}{0.975119in}%
\pgfsys@useobject{currentmarker}{}%
\end{pgfscope}%
\begin{pgfscope}%
\pgfsys@transformshift{1.669748in}{0.996764in}%
\pgfsys@useobject{currentmarker}{}%
\end{pgfscope}%
\begin{pgfscope}%
\pgfsys@transformshift{1.671660in}{1.042020in}%
\pgfsys@useobject{currentmarker}{}%
\end{pgfscope}%
\begin{pgfscope}%
\pgfsys@transformshift{1.673556in}{1.029084in}%
\pgfsys@useobject{currentmarker}{}%
\end{pgfscope}%
\begin{pgfscope}%
\pgfsys@transformshift{1.675435in}{0.999451in}%
\pgfsys@useobject{currentmarker}{}%
\end{pgfscope}%
\begin{pgfscope}%
\pgfsys@transformshift{1.677300in}{1.000164in}%
\pgfsys@useobject{currentmarker}{}%
\end{pgfscope}%
\begin{pgfscope}%
\pgfsys@transformshift{1.679149in}{1.017880in}%
\pgfsys@useobject{currentmarker}{}%
\end{pgfscope}%
\begin{pgfscope}%
\pgfsys@transformshift{1.680983in}{0.984477in}%
\pgfsys@useobject{currentmarker}{}%
\end{pgfscope}%
\begin{pgfscope}%
\pgfsys@transformshift{1.682802in}{1.013991in}%
\pgfsys@useobject{currentmarker}{}%
\end{pgfscope}%
\begin{pgfscope}%
\pgfsys@transformshift{1.684606in}{1.005423in}%
\pgfsys@useobject{currentmarker}{}%
\end{pgfscope}%
\begin{pgfscope}%
\pgfsys@transformshift{1.686396in}{1.010247in}%
\pgfsys@useobject{currentmarker}{}%
\end{pgfscope}%
\begin{pgfscope}%
\pgfsys@transformshift{1.688172in}{0.993147in}%
\pgfsys@useobject{currentmarker}{}%
\end{pgfscope}%
\begin{pgfscope}%
\pgfsys@transformshift{1.689934in}{1.005775in}%
\pgfsys@useobject{currentmarker}{}%
\end{pgfscope}%
\begin{pgfscope}%
\pgfsys@transformshift{1.691682in}{1.003888in}%
\pgfsys@useobject{currentmarker}{}%
\end{pgfscope}%
\begin{pgfscope}%
\pgfsys@transformshift{1.693417in}{0.976093in}%
\pgfsys@useobject{currentmarker}{}%
\end{pgfscope}%
\begin{pgfscope}%
\pgfsys@transformshift{1.695139in}{1.014675in}%
\pgfsys@useobject{currentmarker}{}%
\end{pgfscope}%
\begin{pgfscope}%
\pgfsys@transformshift{1.696847in}{1.028608in}%
\pgfsys@useobject{currentmarker}{}%
\end{pgfscope}%
\begin{pgfscope}%
\pgfsys@transformshift{1.698543in}{1.000062in}%
\pgfsys@useobject{currentmarker}{}%
\end{pgfscope}%
\begin{pgfscope}%
\pgfsys@transformshift{1.700225in}{1.002268in}%
\pgfsys@useobject{currentmarker}{}%
\end{pgfscope}%
\begin{pgfscope}%
\pgfsys@transformshift{1.701896in}{1.033007in}%
\pgfsys@useobject{currentmarker}{}%
\end{pgfscope}%
\begin{pgfscope}%
\pgfsys@transformshift{1.703554in}{1.021678in}%
\pgfsys@useobject{currentmarker}{}%
\end{pgfscope}%
\begin{pgfscope}%
\pgfsys@transformshift{1.705199in}{1.058368in}%
\pgfsys@useobject{currentmarker}{}%
\end{pgfscope}%
\begin{pgfscope}%
\pgfsys@transformshift{1.706833in}{1.042920in}%
\pgfsys@useobject{currentmarker}{}%
\end{pgfscope}%
\begin{pgfscope}%
\pgfsys@transformshift{1.708455in}{1.010079in}%
\pgfsys@useobject{currentmarker}{}%
\end{pgfscope}%
\begin{pgfscope}%
\pgfsys@transformshift{1.710066in}{1.016787in}%
\pgfsys@useobject{currentmarker}{}%
\end{pgfscope}%
\begin{pgfscope}%
\pgfsys@transformshift{1.711665in}{0.982408in}%
\pgfsys@useobject{currentmarker}{}%
\end{pgfscope}%
\begin{pgfscope}%
\pgfsys@transformshift{1.713252in}{1.052763in}%
\pgfsys@useobject{currentmarker}{}%
\end{pgfscope}%
\begin{pgfscope}%
\pgfsys@transformshift{1.714829in}{1.004967in}%
\pgfsys@useobject{currentmarker}{}%
\end{pgfscope}%
\begin{pgfscope}%
\pgfsys@transformshift{1.716394in}{1.010402in}%
\pgfsys@useobject{currentmarker}{}%
\end{pgfscope}%
\begin{pgfscope}%
\pgfsys@transformshift{1.717949in}{0.941492in}%
\pgfsys@useobject{currentmarker}{}%
\end{pgfscope}%
\begin{pgfscope}%
\pgfsys@transformshift{1.719493in}{0.898942in}%
\pgfsys@useobject{currentmarker}{}%
\end{pgfscope}%
\begin{pgfscope}%
\pgfsys@transformshift{1.721026in}{0.958275in}%
\pgfsys@useobject{currentmarker}{}%
\end{pgfscope}%
\begin{pgfscope}%
\pgfsys@transformshift{1.722550in}{0.991722in}%
\pgfsys@useobject{currentmarker}{}%
\end{pgfscope}%
\begin{pgfscope}%
\pgfsys@transformshift{1.724062in}{1.004222in}%
\pgfsys@useobject{currentmarker}{}%
\end{pgfscope}%
\begin{pgfscope}%
\pgfsys@transformshift{1.725565in}{1.027335in}%
\pgfsys@useobject{currentmarker}{}%
\end{pgfscope}%
\begin{pgfscope}%
\pgfsys@transformshift{1.727058in}{0.990523in}%
\pgfsys@useobject{currentmarker}{}%
\end{pgfscope}%
\begin{pgfscope}%
\pgfsys@transformshift{1.728541in}{0.989322in}%
\pgfsys@useobject{currentmarker}{}%
\end{pgfscope}%
\begin{pgfscope}%
\pgfsys@transformshift{1.730014in}{1.001958in}%
\pgfsys@useobject{currentmarker}{}%
\end{pgfscope}%
\begin{pgfscope}%
\pgfsys@transformshift{1.731477in}{0.952554in}%
\pgfsys@useobject{currentmarker}{}%
\end{pgfscope}%
\begin{pgfscope}%
\pgfsys@transformshift{1.732931in}{0.975463in}%
\pgfsys@useobject{currentmarker}{}%
\end{pgfscope}%
\begin{pgfscope}%
\pgfsys@transformshift{1.734376in}{0.988815in}%
\pgfsys@useobject{currentmarker}{}%
\end{pgfscope}%
\begin{pgfscope}%
\pgfsys@transformshift{1.735812in}{1.019218in}%
\pgfsys@useobject{currentmarker}{}%
\end{pgfscope}%
\begin{pgfscope}%
\pgfsys@transformshift{1.737238in}{1.034054in}%
\pgfsys@useobject{currentmarker}{}%
\end{pgfscope}%
\begin{pgfscope}%
\pgfsys@transformshift{1.738655in}{1.013071in}%
\pgfsys@useobject{currentmarker}{}%
\end{pgfscope}%
\begin{pgfscope}%
\pgfsys@transformshift{1.740064in}{0.949076in}%
\pgfsys@useobject{currentmarker}{}%
\end{pgfscope}%
\begin{pgfscope}%
\pgfsys@transformshift{1.741463in}{0.878230in}%
\pgfsys@useobject{currentmarker}{}%
\end{pgfscope}%
\begin{pgfscope}%
\pgfsys@transformshift{1.742854in}{0.939312in}%
\pgfsys@useobject{currentmarker}{}%
\end{pgfscope}%
\begin{pgfscope}%
\pgfsys@transformshift{1.744237in}{0.970803in}%
\pgfsys@useobject{currentmarker}{}%
\end{pgfscope}%
\begin{pgfscope}%
\pgfsys@transformshift{1.745611in}{1.004233in}%
\pgfsys@useobject{currentmarker}{}%
\end{pgfscope}%
\begin{pgfscope}%
\pgfsys@transformshift{1.746977in}{1.022592in}%
\pgfsys@useobject{currentmarker}{}%
\end{pgfscope}%
\begin{pgfscope}%
\pgfsys@transformshift{1.748334in}{0.985141in}%
\pgfsys@useobject{currentmarker}{}%
\end{pgfscope}%
\begin{pgfscope}%
\pgfsys@transformshift{1.749683in}{0.932980in}%
\pgfsys@useobject{currentmarker}{}%
\end{pgfscope}%
\begin{pgfscope}%
\pgfsys@transformshift{1.751025in}{0.933641in}%
\pgfsys@useobject{currentmarker}{}%
\end{pgfscope}%
\begin{pgfscope}%
\pgfsys@transformshift{1.752358in}{0.896854in}%
\pgfsys@useobject{currentmarker}{}%
\end{pgfscope}%
\begin{pgfscope}%
\pgfsys@transformshift{1.753683in}{0.908551in}%
\pgfsys@useobject{currentmarker}{}%
\end{pgfscope}%
\begin{pgfscope}%
\pgfsys@transformshift{1.755001in}{0.918177in}%
\pgfsys@useobject{currentmarker}{}%
\end{pgfscope}%
\begin{pgfscope}%
\pgfsys@transformshift{1.756311in}{0.944832in}%
\pgfsys@useobject{currentmarker}{}%
\end{pgfscope}%
\begin{pgfscope}%
\pgfsys@transformshift{1.757613in}{0.924943in}%
\pgfsys@useobject{currentmarker}{}%
\end{pgfscope}%
\begin{pgfscope}%
\pgfsys@transformshift{1.758908in}{0.873133in}%
\pgfsys@useobject{currentmarker}{}%
\end{pgfscope}%
\begin{pgfscope}%
\pgfsys@transformshift{1.760195in}{0.854998in}%
\pgfsys@useobject{currentmarker}{}%
\end{pgfscope}%
\begin{pgfscope}%
\pgfsys@transformshift{1.761475in}{0.927751in}%
\pgfsys@useobject{currentmarker}{}%
\end{pgfscope}%
\begin{pgfscope}%
\pgfsys@transformshift{1.762748in}{0.981997in}%
\pgfsys@useobject{currentmarker}{}%
\end{pgfscope}%
\begin{pgfscope}%
\pgfsys@transformshift{1.764014in}{0.959718in}%
\pgfsys@useobject{currentmarker}{}%
\end{pgfscope}%
\begin{pgfscope}%
\pgfsys@transformshift{1.765272in}{0.951327in}%
\pgfsys@useobject{currentmarker}{}%
\end{pgfscope}%
\begin{pgfscope}%
\pgfsys@transformshift{1.766524in}{0.992409in}%
\pgfsys@useobject{currentmarker}{}%
\end{pgfscope}%
\begin{pgfscope}%
\pgfsys@transformshift{1.767769in}{0.998812in}%
\pgfsys@useobject{currentmarker}{}%
\end{pgfscope}%
\begin{pgfscope}%
\pgfsys@transformshift{1.769006in}{0.977150in}%
\pgfsys@useobject{currentmarker}{}%
\end{pgfscope}%
\begin{pgfscope}%
\pgfsys@transformshift{1.770237in}{0.951963in}%
\pgfsys@useobject{currentmarker}{}%
\end{pgfscope}%
\begin{pgfscope}%
\pgfsys@transformshift{1.771462in}{0.957431in}%
\pgfsys@useobject{currentmarker}{}%
\end{pgfscope}%
\begin{pgfscope}%
\pgfsys@transformshift{1.772679in}{0.988688in}%
\pgfsys@useobject{currentmarker}{}%
\end{pgfscope}%
\begin{pgfscope}%
\pgfsys@transformshift{1.773890in}{0.959751in}%
\pgfsys@useobject{currentmarker}{}%
\end{pgfscope}%
\begin{pgfscope}%
\pgfsys@transformshift{1.775095in}{0.953720in}%
\pgfsys@useobject{currentmarker}{}%
\end{pgfscope}%
\begin{pgfscope}%
\pgfsys@transformshift{1.776293in}{0.996163in}%
\pgfsys@useobject{currentmarker}{}%
\end{pgfscope}%
\begin{pgfscope}%
\pgfsys@transformshift{1.777485in}{0.978138in}%
\pgfsys@useobject{currentmarker}{}%
\end{pgfscope}%
\begin{pgfscope}%
\pgfsys@transformshift{1.778670in}{0.942278in}%
\pgfsys@useobject{currentmarker}{}%
\end{pgfscope}%
\begin{pgfscope}%
\pgfsys@transformshift{1.779849in}{0.998451in}%
\pgfsys@useobject{currentmarker}{}%
\end{pgfscope}%
\begin{pgfscope}%
\pgfsys@transformshift{1.781023in}{0.987817in}%
\pgfsys@useobject{currentmarker}{}%
\end{pgfscope}%
\begin{pgfscope}%
\pgfsys@transformshift{1.782190in}{0.941817in}%
\pgfsys@useobject{currentmarker}{}%
\end{pgfscope}%
\begin{pgfscope}%
\pgfsys@transformshift{1.783351in}{0.954538in}%
\pgfsys@useobject{currentmarker}{}%
\end{pgfscope}%
\begin{pgfscope}%
\pgfsys@transformshift{1.784506in}{0.981581in}%
\pgfsys@useobject{currentmarker}{}%
\end{pgfscope}%
\begin{pgfscope}%
\pgfsys@transformshift{1.785655in}{0.902075in}%
\pgfsys@useobject{currentmarker}{}%
\end{pgfscope}%
\begin{pgfscope}%
\pgfsys@transformshift{1.786798in}{0.890611in}%
\pgfsys@useobject{currentmarker}{}%
\end{pgfscope}%
\begin{pgfscope}%
\pgfsys@transformshift{1.787936in}{0.961528in}%
\pgfsys@useobject{currentmarker}{}%
\end{pgfscope}%
\begin{pgfscope}%
\pgfsys@transformshift{1.789067in}{0.989306in}%
\pgfsys@useobject{currentmarker}{}%
\end{pgfscope}%
\begin{pgfscope}%
\pgfsys@transformshift{1.790193in}{0.993728in}%
\pgfsys@useobject{currentmarker}{}%
\end{pgfscope}%
\begin{pgfscope}%
\pgfsys@transformshift{1.791314in}{1.019221in}%
\pgfsys@useobject{currentmarker}{}%
\end{pgfscope}%
\begin{pgfscope}%
\pgfsys@transformshift{1.792429in}{1.019675in}%
\pgfsys@useobject{currentmarker}{}%
\end{pgfscope}%
\begin{pgfscope}%
\pgfsys@transformshift{1.793538in}{0.968329in}%
\pgfsys@useobject{currentmarker}{}%
\end{pgfscope}%
\begin{pgfscope}%
\pgfsys@transformshift{1.794642in}{0.927693in}%
\pgfsys@useobject{currentmarker}{}%
\end{pgfscope}%
\begin{pgfscope}%
\pgfsys@transformshift{1.795741in}{0.886227in}%
\pgfsys@useobject{currentmarker}{}%
\end{pgfscope}%
\begin{pgfscope}%
\pgfsys@transformshift{1.796834in}{0.911023in}%
\pgfsys@useobject{currentmarker}{}%
\end{pgfscope}%
\begin{pgfscope}%
\pgfsys@transformshift{1.797922in}{0.964434in}%
\pgfsys@useobject{currentmarker}{}%
\end{pgfscope}%
\begin{pgfscope}%
\pgfsys@transformshift{1.799004in}{0.942858in}%
\pgfsys@useobject{currentmarker}{}%
\end{pgfscope}%
\begin{pgfscope}%
\pgfsys@transformshift{1.800082in}{0.935556in}%
\pgfsys@useobject{currentmarker}{}%
\end{pgfscope}%
\begin{pgfscope}%
\pgfsys@transformshift{1.801154in}{0.955890in}%
\pgfsys@useobject{currentmarker}{}%
\end{pgfscope}%
\begin{pgfscope}%
\pgfsys@transformshift{1.802221in}{0.903562in}%
\pgfsys@useobject{currentmarker}{}%
\end{pgfscope}%
\begin{pgfscope}%
\pgfsys@transformshift{1.803284in}{0.918587in}%
\pgfsys@useobject{currentmarker}{}%
\end{pgfscope}%
\begin{pgfscope}%
\pgfsys@transformshift{1.804341in}{0.910437in}%
\pgfsys@useobject{currentmarker}{}%
\end{pgfscope}%
\begin{pgfscope}%
\pgfsys@transformshift{1.805393in}{0.914167in}%
\pgfsys@useobject{currentmarker}{}%
\end{pgfscope}%
\begin{pgfscope}%
\pgfsys@transformshift{1.806440in}{0.928218in}%
\pgfsys@useobject{currentmarker}{}%
\end{pgfscope}%
\begin{pgfscope}%
\pgfsys@transformshift{1.807483in}{0.928742in}%
\pgfsys@useobject{currentmarker}{}%
\end{pgfscope}%
\begin{pgfscope}%
\pgfsys@transformshift{1.808520in}{0.940043in}%
\pgfsys@useobject{currentmarker}{}%
\end{pgfscope}%
\begin{pgfscope}%
\pgfsys@transformshift{1.809553in}{0.965731in}%
\pgfsys@useobject{currentmarker}{}%
\end{pgfscope}%
\begin{pgfscope}%
\pgfsys@transformshift{1.810581in}{0.934750in}%
\pgfsys@useobject{currentmarker}{}%
\end{pgfscope}%
\begin{pgfscope}%
\pgfsys@transformshift{1.811605in}{0.932998in}%
\pgfsys@useobject{currentmarker}{}%
\end{pgfscope}%
\begin{pgfscope}%
\pgfsys@transformshift{1.812624in}{0.903523in}%
\pgfsys@useobject{currentmarker}{}%
\end{pgfscope}%
\begin{pgfscope}%
\pgfsys@transformshift{1.813638in}{0.894253in}%
\pgfsys@useobject{currentmarker}{}%
\end{pgfscope}%
\begin{pgfscope}%
\pgfsys@transformshift{1.814648in}{0.919857in}%
\pgfsys@useobject{currentmarker}{}%
\end{pgfscope}%
\begin{pgfscope}%
\pgfsys@transformshift{1.815653in}{0.943210in}%
\pgfsys@useobject{currentmarker}{}%
\end{pgfscope}%
\begin{pgfscope}%
\pgfsys@transformshift{1.816653in}{0.938640in}%
\pgfsys@useobject{currentmarker}{}%
\end{pgfscope}%
\begin{pgfscope}%
\pgfsys@transformshift{1.817650in}{0.922136in}%
\pgfsys@useobject{currentmarker}{}%
\end{pgfscope}%
\begin{pgfscope}%
\pgfsys@transformshift{1.818642in}{0.885553in}%
\pgfsys@useobject{currentmarker}{}%
\end{pgfscope}%
\begin{pgfscope}%
\pgfsys@transformshift{1.819629in}{0.882626in}%
\pgfsys@useobject{currentmarker}{}%
\end{pgfscope}%
\begin{pgfscope}%
\pgfsys@transformshift{1.820612in}{0.923199in}%
\pgfsys@useobject{currentmarker}{}%
\end{pgfscope}%
\begin{pgfscope}%
\pgfsys@transformshift{1.821591in}{0.871260in}%
\pgfsys@useobject{currentmarker}{}%
\end{pgfscope}%
\begin{pgfscope}%
\pgfsys@transformshift{1.822566in}{0.835198in}%
\pgfsys@useobject{currentmarker}{}%
\end{pgfscope}%
\begin{pgfscope}%
\pgfsys@transformshift{1.823536in}{0.896678in}%
\pgfsys@useobject{currentmarker}{}%
\end{pgfscope}%
\begin{pgfscope}%
\pgfsys@transformshift{1.824502in}{0.917862in}%
\pgfsys@useobject{currentmarker}{}%
\end{pgfscope}%
\begin{pgfscope}%
\pgfsys@transformshift{1.825464in}{0.900156in}%
\pgfsys@useobject{currentmarker}{}%
\end{pgfscope}%
\begin{pgfscope}%
\pgfsys@transformshift{1.826423in}{0.954410in}%
\pgfsys@useobject{currentmarker}{}%
\end{pgfscope}%
\begin{pgfscope}%
\pgfsys@transformshift{1.827376in}{0.967181in}%
\pgfsys@useobject{currentmarker}{}%
\end{pgfscope}%
\begin{pgfscope}%
\pgfsys@transformshift{1.828326in}{0.915909in}%
\pgfsys@useobject{currentmarker}{}%
\end{pgfscope}%
\begin{pgfscope}%
\pgfsys@transformshift{1.829272in}{0.922509in}%
\pgfsys@useobject{currentmarker}{}%
\end{pgfscope}%
\begin{pgfscope}%
\pgfsys@transformshift{1.830214in}{0.911325in}%
\pgfsys@useobject{currentmarker}{}%
\end{pgfscope}%
\begin{pgfscope}%
\pgfsys@transformshift{1.831152in}{0.930041in}%
\pgfsys@useobject{currentmarker}{}%
\end{pgfscope}%
\begin{pgfscope}%
\pgfsys@transformshift{1.832086in}{0.940203in}%
\pgfsys@useobject{currentmarker}{}%
\end{pgfscope}%
\begin{pgfscope}%
\pgfsys@transformshift{1.833017in}{0.926372in}%
\pgfsys@useobject{currentmarker}{}%
\end{pgfscope}%
\begin{pgfscope}%
\pgfsys@transformshift{1.833943in}{0.866135in}%
\pgfsys@useobject{currentmarker}{}%
\end{pgfscope}%
\begin{pgfscope}%
\pgfsys@transformshift{1.834866in}{0.878424in}%
\pgfsys@useobject{currentmarker}{}%
\end{pgfscope}%
\begin{pgfscope}%
\pgfsys@transformshift{1.835784in}{0.889490in}%
\pgfsys@useobject{currentmarker}{}%
\end{pgfscope}%
\begin{pgfscope}%
\pgfsys@transformshift{1.836699in}{0.877095in}%
\pgfsys@useobject{currentmarker}{}%
\end{pgfscope}%
\begin{pgfscope}%
\pgfsys@transformshift{1.837611in}{0.892343in}%
\pgfsys@useobject{currentmarker}{}%
\end{pgfscope}%
\begin{pgfscope}%
\pgfsys@transformshift{1.838518in}{0.940050in}%
\pgfsys@useobject{currentmarker}{}%
\end{pgfscope}%
\begin{pgfscope}%
\pgfsys@transformshift{1.839423in}{0.948179in}%
\pgfsys@useobject{currentmarker}{}%
\end{pgfscope}%
\begin{pgfscope}%
\pgfsys@transformshift{1.840323in}{0.882358in}%
\pgfsys@useobject{currentmarker}{}%
\end{pgfscope}%
\begin{pgfscope}%
\pgfsys@transformshift{1.841220in}{0.892817in}%
\pgfsys@useobject{currentmarker}{}%
\end{pgfscope}%
\begin{pgfscope}%
\pgfsys@transformshift{1.842113in}{0.925810in}%
\pgfsys@useobject{currentmarker}{}%
\end{pgfscope}%
\begin{pgfscope}%
\pgfsys@transformshift{1.843003in}{0.938740in}%
\pgfsys@useobject{currentmarker}{}%
\end{pgfscope}%
\begin{pgfscope}%
\pgfsys@transformshift{1.843889in}{0.929383in}%
\pgfsys@useobject{currentmarker}{}%
\end{pgfscope}%
\begin{pgfscope}%
\pgfsys@transformshift{1.844772in}{0.946552in}%
\pgfsys@useobject{currentmarker}{}%
\end{pgfscope}%
\begin{pgfscope}%
\pgfsys@transformshift{1.845651in}{0.954592in}%
\pgfsys@useobject{currentmarker}{}%
\end{pgfscope}%
\begin{pgfscope}%
\pgfsys@transformshift{1.846527in}{0.925979in}%
\pgfsys@useobject{currentmarker}{}%
\end{pgfscope}%
\begin{pgfscope}%
\pgfsys@transformshift{1.847399in}{0.918996in}%
\pgfsys@useobject{currentmarker}{}%
\end{pgfscope}%
\begin{pgfscope}%
\pgfsys@transformshift{1.848268in}{0.936402in}%
\pgfsys@useobject{currentmarker}{}%
\end{pgfscope}%
\begin{pgfscope}%
\pgfsys@transformshift{1.849134in}{0.893017in}%
\pgfsys@useobject{currentmarker}{}%
\end{pgfscope}%
\begin{pgfscope}%
\pgfsys@transformshift{1.849996in}{0.925095in}%
\pgfsys@useobject{currentmarker}{}%
\end{pgfscope}%
\begin{pgfscope}%
\pgfsys@transformshift{1.850855in}{0.953066in}%
\pgfsys@useobject{currentmarker}{}%
\end{pgfscope}%
\begin{pgfscope}%
\pgfsys@transformshift{1.851711in}{0.909335in}%
\pgfsys@useobject{currentmarker}{}%
\end{pgfscope}%
\begin{pgfscope}%
\pgfsys@transformshift{1.852564in}{0.909027in}%
\pgfsys@useobject{currentmarker}{}%
\end{pgfscope}%
\begin{pgfscope}%
\pgfsys@transformshift{1.853413in}{0.906683in}%
\pgfsys@useobject{currentmarker}{}%
\end{pgfscope}%
\begin{pgfscope}%
\pgfsys@transformshift{1.854259in}{0.882094in}%
\pgfsys@useobject{currentmarker}{}%
\end{pgfscope}%
\begin{pgfscope}%
\pgfsys@transformshift{1.855102in}{0.845605in}%
\pgfsys@useobject{currentmarker}{}%
\end{pgfscope}%
\begin{pgfscope}%
\pgfsys@transformshift{1.855942in}{0.822602in}%
\pgfsys@useobject{currentmarker}{}%
\end{pgfscope}%
\begin{pgfscope}%
\pgfsys@transformshift{1.856779in}{0.857045in}%
\pgfsys@useobject{currentmarker}{}%
\end{pgfscope}%
\begin{pgfscope}%
\pgfsys@transformshift{1.857612in}{0.882969in}%
\pgfsys@useobject{currentmarker}{}%
\end{pgfscope}%
\begin{pgfscope}%
\pgfsys@transformshift{1.858443in}{0.874862in}%
\pgfsys@useobject{currentmarker}{}%
\end{pgfscope}%
\begin{pgfscope}%
\pgfsys@transformshift{1.859270in}{0.892196in}%
\pgfsys@useobject{currentmarker}{}%
\end{pgfscope}%
\begin{pgfscope}%
\pgfsys@transformshift{1.860095in}{0.927409in}%
\pgfsys@useobject{currentmarker}{}%
\end{pgfscope}%
\begin{pgfscope}%
\pgfsys@transformshift{1.860916in}{0.862143in}%
\pgfsys@useobject{currentmarker}{}%
\end{pgfscope}%
\begin{pgfscope}%
\pgfsys@transformshift{1.861735in}{0.856254in}%
\pgfsys@useobject{currentmarker}{}%
\end{pgfscope}%
\begin{pgfscope}%
\pgfsys@transformshift{1.862550in}{0.867133in}%
\pgfsys@useobject{currentmarker}{}%
\end{pgfscope}%
\begin{pgfscope}%
\pgfsys@transformshift{1.863362in}{0.908099in}%
\pgfsys@useobject{currentmarker}{}%
\end{pgfscope}%
\begin{pgfscope}%
\pgfsys@transformshift{1.864172in}{0.897139in}%
\pgfsys@useobject{currentmarker}{}%
\end{pgfscope}%
\begin{pgfscope}%
\pgfsys@transformshift{1.864979in}{0.897752in}%
\pgfsys@useobject{currentmarker}{}%
\end{pgfscope}%
\begin{pgfscope}%
\pgfsys@transformshift{1.865782in}{0.892636in}%
\pgfsys@useobject{currentmarker}{}%
\end{pgfscope}%
\begin{pgfscope}%
\pgfsys@transformshift{1.866583in}{0.866430in}%
\pgfsys@useobject{currentmarker}{}%
\end{pgfscope}%
\begin{pgfscope}%
\pgfsys@transformshift{1.867381in}{0.902646in}%
\pgfsys@useobject{currentmarker}{}%
\end{pgfscope}%
\begin{pgfscope}%
\pgfsys@transformshift{1.868177in}{0.896592in}%
\pgfsys@useobject{currentmarker}{}%
\end{pgfscope}%
\begin{pgfscope}%
\pgfsys@transformshift{1.868969in}{0.931603in}%
\pgfsys@useobject{currentmarker}{}%
\end{pgfscope}%
\begin{pgfscope}%
\pgfsys@transformshift{1.869759in}{0.905184in}%
\pgfsys@useobject{currentmarker}{}%
\end{pgfscope}%
\begin{pgfscope}%
\pgfsys@transformshift{1.870546in}{0.815348in}%
\pgfsys@useobject{currentmarker}{}%
\end{pgfscope}%
\begin{pgfscope}%
\pgfsys@transformshift{1.871330in}{0.877727in}%
\pgfsys@useobject{currentmarker}{}%
\end{pgfscope}%
\begin{pgfscope}%
\pgfsys@transformshift{1.872111in}{0.900657in}%
\pgfsys@useobject{currentmarker}{}%
\end{pgfscope}%
\begin{pgfscope}%
\pgfsys@transformshift{1.872890in}{0.860195in}%
\pgfsys@useobject{currentmarker}{}%
\end{pgfscope}%
\begin{pgfscope}%
\pgfsys@transformshift{1.873666in}{0.841754in}%
\pgfsys@useobject{currentmarker}{}%
\end{pgfscope}%
\begin{pgfscope}%
\pgfsys@transformshift{1.874439in}{0.852890in}%
\pgfsys@useobject{currentmarker}{}%
\end{pgfscope}%
\begin{pgfscope}%
\pgfsys@transformshift{1.875210in}{0.907077in}%
\pgfsys@useobject{currentmarker}{}%
\end{pgfscope}%
\begin{pgfscope}%
\pgfsys@transformshift{1.875978in}{0.925623in}%
\pgfsys@useobject{currentmarker}{}%
\end{pgfscope}%
\begin{pgfscope}%
\pgfsys@transformshift{1.876743in}{0.919749in}%
\pgfsys@useobject{currentmarker}{}%
\end{pgfscope}%
\begin{pgfscope}%
\pgfsys@transformshift{1.877506in}{0.856119in}%
\pgfsys@useobject{currentmarker}{}%
\end{pgfscope}%
\begin{pgfscope}%
\pgfsys@transformshift{1.878266in}{0.852607in}%
\pgfsys@useobject{currentmarker}{}%
\end{pgfscope}%
\begin{pgfscope}%
\pgfsys@transformshift{1.879024in}{0.861972in}%
\pgfsys@useobject{currentmarker}{}%
\end{pgfscope}%
\begin{pgfscope}%
\pgfsys@transformshift{1.879779in}{0.895745in}%
\pgfsys@useobject{currentmarker}{}%
\end{pgfscope}%
\begin{pgfscope}%
\pgfsys@transformshift{1.880532in}{0.895235in}%
\pgfsys@useobject{currentmarker}{}%
\end{pgfscope}%
\begin{pgfscope}%
\pgfsys@transformshift{1.881282in}{0.862184in}%
\pgfsys@useobject{currentmarker}{}%
\end{pgfscope}%
\begin{pgfscope}%
\pgfsys@transformshift{1.882029in}{0.898176in}%
\pgfsys@useobject{currentmarker}{}%
\end{pgfscope}%
\begin{pgfscope}%
\pgfsys@transformshift{1.882774in}{0.894241in}%
\pgfsys@useobject{currentmarker}{}%
\end{pgfscope}%
\begin{pgfscope}%
\pgfsys@transformshift{1.883517in}{0.862124in}%
\pgfsys@useobject{currentmarker}{}%
\end{pgfscope}%
\begin{pgfscope}%
\pgfsys@transformshift{1.884257in}{0.913113in}%
\pgfsys@useobject{currentmarker}{}%
\end{pgfscope}%
\begin{pgfscope}%
\pgfsys@transformshift{1.884995in}{0.911665in}%
\pgfsys@useobject{currentmarker}{}%
\end{pgfscope}%
\begin{pgfscope}%
\pgfsys@transformshift{1.885730in}{0.868371in}%
\pgfsys@useobject{currentmarker}{}%
\end{pgfscope}%
\begin{pgfscope}%
\pgfsys@transformshift{1.886463in}{0.851790in}%
\pgfsys@useobject{currentmarker}{}%
\end{pgfscope}%
\begin{pgfscope}%
\pgfsys@transformshift{1.887194in}{0.830603in}%
\pgfsys@useobject{currentmarker}{}%
\end{pgfscope}%
\begin{pgfscope}%
\pgfsys@transformshift{1.887922in}{0.917230in}%
\pgfsys@useobject{currentmarker}{}%
\end{pgfscope}%
\begin{pgfscope}%
\pgfsys@transformshift{1.888648in}{0.929372in}%
\pgfsys@useobject{currentmarker}{}%
\end{pgfscope}%
\begin{pgfscope}%
\pgfsys@transformshift{1.889372in}{0.884223in}%
\pgfsys@useobject{currentmarker}{}%
\end{pgfscope}%
\begin{pgfscope}%
\pgfsys@transformshift{1.890093in}{0.866289in}%
\pgfsys@useobject{currentmarker}{}%
\end{pgfscope}%
\begin{pgfscope}%
\pgfsys@transformshift{1.890812in}{0.864875in}%
\pgfsys@useobject{currentmarker}{}%
\end{pgfscope}%
\begin{pgfscope}%
\pgfsys@transformshift{1.891528in}{0.920091in}%
\pgfsys@useobject{currentmarker}{}%
\end{pgfscope}%
\begin{pgfscope}%
\pgfsys@transformshift{1.892243in}{0.897719in}%
\pgfsys@useobject{currentmarker}{}%
\end{pgfscope}%
\begin{pgfscope}%
\pgfsys@transformshift{1.892955in}{0.860372in}%
\pgfsys@useobject{currentmarker}{}%
\end{pgfscope}%
\begin{pgfscope}%
\pgfsys@transformshift{1.893664in}{0.914282in}%
\pgfsys@useobject{currentmarker}{}%
\end{pgfscope}%
\begin{pgfscope}%
\pgfsys@transformshift{1.894372in}{0.909851in}%
\pgfsys@useobject{currentmarker}{}%
\end{pgfscope}%
\begin{pgfscope}%
\pgfsys@transformshift{1.895077in}{0.897261in}%
\pgfsys@useobject{currentmarker}{}%
\end{pgfscope}%
\begin{pgfscope}%
\pgfsys@transformshift{1.895780in}{0.820830in}%
\pgfsys@useobject{currentmarker}{}%
\end{pgfscope}%
\begin{pgfscope}%
\pgfsys@transformshift{1.896481in}{0.860261in}%
\pgfsys@useobject{currentmarker}{}%
\end{pgfscope}%
\begin{pgfscope}%
\pgfsys@transformshift{1.897180in}{0.869633in}%
\pgfsys@useobject{currentmarker}{}%
\end{pgfscope}%
\begin{pgfscope}%
\pgfsys@transformshift{1.897877in}{0.876122in}%
\pgfsys@useobject{currentmarker}{}%
\end{pgfscope}%
\begin{pgfscope}%
\pgfsys@transformshift{1.898571in}{0.870369in}%
\pgfsys@useobject{currentmarker}{}%
\end{pgfscope}%
\begin{pgfscope}%
\pgfsys@transformshift{1.899264in}{0.869295in}%
\pgfsys@useobject{currentmarker}{}%
\end{pgfscope}%
\begin{pgfscope}%
\pgfsys@transformshift{1.899954in}{0.919133in}%
\pgfsys@useobject{currentmarker}{}%
\end{pgfscope}%
\begin{pgfscope}%
\pgfsys@transformshift{1.900642in}{0.931295in}%
\pgfsys@useobject{currentmarker}{}%
\end{pgfscope}%
\begin{pgfscope}%
\pgfsys@transformshift{1.901328in}{0.881504in}%
\pgfsys@useobject{currentmarker}{}%
\end{pgfscope}%
\begin{pgfscope}%
\pgfsys@transformshift{1.902012in}{0.857289in}%
\pgfsys@useobject{currentmarker}{}%
\end{pgfscope}%
\begin{pgfscope}%
\pgfsys@transformshift{1.902693in}{0.882424in}%
\pgfsys@useobject{currentmarker}{}%
\end{pgfscope}%
\begin{pgfscope}%
\pgfsys@transformshift{1.903373in}{0.962077in}%
\pgfsys@useobject{currentmarker}{}%
\end{pgfscope}%
\begin{pgfscope}%
\pgfsys@transformshift{1.904051in}{0.951743in}%
\pgfsys@useobject{currentmarker}{}%
\end{pgfscope}%
\begin{pgfscope}%
\pgfsys@transformshift{1.904726in}{0.913622in}%
\pgfsys@useobject{currentmarker}{}%
\end{pgfscope}%
\begin{pgfscope}%
\pgfsys@transformshift{1.905400in}{0.915250in}%
\pgfsys@useobject{currentmarker}{}%
\end{pgfscope}%
\begin{pgfscope}%
\pgfsys@transformshift{1.906072in}{0.866016in}%
\pgfsys@useobject{currentmarker}{}%
\end{pgfscope}%
\begin{pgfscope}%
\pgfsys@transformshift{1.906741in}{0.848849in}%
\pgfsys@useobject{currentmarker}{}%
\end{pgfscope}%
\begin{pgfscope}%
\pgfsys@transformshift{1.907409in}{0.851163in}%
\pgfsys@useobject{currentmarker}{}%
\end{pgfscope}%
\begin{pgfscope}%
\pgfsys@transformshift{1.908075in}{0.828793in}%
\pgfsys@useobject{currentmarker}{}%
\end{pgfscope}%
\begin{pgfscope}%
\pgfsys@transformshift{1.908738in}{0.822618in}%
\pgfsys@useobject{currentmarker}{}%
\end{pgfscope}%
\begin{pgfscope}%
\pgfsys@transformshift{1.909400in}{0.834806in}%
\pgfsys@useobject{currentmarker}{}%
\end{pgfscope}%
\begin{pgfscope}%
\pgfsys@transformshift{1.910060in}{0.827132in}%
\pgfsys@useobject{currentmarker}{}%
\end{pgfscope}%
\begin{pgfscope}%
\pgfsys@transformshift{1.910717in}{0.829656in}%
\pgfsys@useobject{currentmarker}{}%
\end{pgfscope}%
\begin{pgfscope}%
\pgfsys@transformshift{1.911373in}{0.883877in}%
\pgfsys@useobject{currentmarker}{}%
\end{pgfscope}%
\begin{pgfscope}%
\pgfsys@transformshift{1.912027in}{0.878445in}%
\pgfsys@useobject{currentmarker}{}%
\end{pgfscope}%
\begin{pgfscope}%
\pgfsys@transformshift{1.912680in}{0.913145in}%
\pgfsys@useobject{currentmarker}{}%
\end{pgfscope}%
\begin{pgfscope}%
\pgfsys@transformshift{1.913330in}{0.923664in}%
\pgfsys@useobject{currentmarker}{}%
\end{pgfscope}%
\begin{pgfscope}%
\pgfsys@transformshift{1.913978in}{0.832269in}%
\pgfsys@useobject{currentmarker}{}%
\end{pgfscope}%
\begin{pgfscope}%
\pgfsys@transformshift{1.914625in}{0.881998in}%
\pgfsys@useobject{currentmarker}{}%
\end{pgfscope}%
\begin{pgfscope}%
\pgfsys@transformshift{1.915269in}{0.870560in}%
\pgfsys@useobject{currentmarker}{}%
\end{pgfscope}%
\begin{pgfscope}%
\pgfsys@transformshift{1.915912in}{0.852447in}%
\pgfsys@useobject{currentmarker}{}%
\end{pgfscope}%
\begin{pgfscope}%
\pgfsys@transformshift{1.916553in}{0.831464in}%
\pgfsys@useobject{currentmarker}{}%
\end{pgfscope}%
\begin{pgfscope}%
\pgfsys@transformshift{1.917192in}{0.760373in}%
\pgfsys@useobject{currentmarker}{}%
\end{pgfscope}%
\begin{pgfscope}%
\pgfsys@transformshift{1.917829in}{0.852771in}%
\pgfsys@useobject{currentmarker}{}%
\end{pgfscope}%
\begin{pgfscope}%
\pgfsys@transformshift{1.918465in}{0.883449in}%
\pgfsys@useobject{currentmarker}{}%
\end{pgfscope}%
\begin{pgfscope}%
\pgfsys@transformshift{1.919099in}{0.855862in}%
\pgfsys@useobject{currentmarker}{}%
\end{pgfscope}%
\begin{pgfscope}%
\pgfsys@transformshift{1.919731in}{0.854415in}%
\pgfsys@useobject{currentmarker}{}%
\end{pgfscope}%
\begin{pgfscope}%
\pgfsys@transformshift{1.920361in}{0.855146in}%
\pgfsys@useobject{currentmarker}{}%
\end{pgfscope}%
\begin{pgfscope}%
\pgfsys@transformshift{1.920989in}{0.845043in}%
\pgfsys@useobject{currentmarker}{}%
\end{pgfscope}%
\begin{pgfscope}%
\pgfsys@transformshift{1.921616in}{0.797923in}%
\pgfsys@useobject{currentmarker}{}%
\end{pgfscope}%
\begin{pgfscope}%
\pgfsys@transformshift{1.922241in}{0.851903in}%
\pgfsys@useobject{currentmarker}{}%
\end{pgfscope}%
\begin{pgfscope}%
\pgfsys@transformshift{1.922864in}{0.799815in}%
\pgfsys@useobject{currentmarker}{}%
\end{pgfscope}%
\begin{pgfscope}%
\pgfsys@transformshift{1.923485in}{0.862288in}%
\pgfsys@useobject{currentmarker}{}%
\end{pgfscope}%
\begin{pgfscope}%
\pgfsys@transformshift{1.924105in}{0.893396in}%
\pgfsys@useobject{currentmarker}{}%
\end{pgfscope}%
\begin{pgfscope}%
\pgfsys@transformshift{1.924723in}{0.863905in}%
\pgfsys@useobject{currentmarker}{}%
\end{pgfscope}%
\begin{pgfscope}%
\pgfsys@transformshift{1.925339in}{0.849477in}%
\pgfsys@useobject{currentmarker}{}%
\end{pgfscope}%
\begin{pgfscope}%
\pgfsys@transformshift{1.925954in}{0.811275in}%
\pgfsys@useobject{currentmarker}{}%
\end{pgfscope}%
\begin{pgfscope}%
\pgfsys@transformshift{1.926567in}{0.846296in}%
\pgfsys@useobject{currentmarker}{}%
\end{pgfscope}%
\begin{pgfscope}%
\pgfsys@transformshift{1.927178in}{0.817540in}%
\pgfsys@useobject{currentmarker}{}%
\end{pgfscope}%
\begin{pgfscope}%
\pgfsys@transformshift{1.927788in}{0.799489in}%
\pgfsys@useobject{currentmarker}{}%
\end{pgfscope}%
\begin{pgfscope}%
\pgfsys@transformshift{1.928396in}{0.781782in}%
\pgfsys@useobject{currentmarker}{}%
\end{pgfscope}%
\begin{pgfscope}%
\pgfsys@transformshift{1.929002in}{0.820781in}%
\pgfsys@useobject{currentmarker}{}%
\end{pgfscope}%
\begin{pgfscope}%
\pgfsys@transformshift{1.929607in}{0.871187in}%
\pgfsys@useobject{currentmarker}{}%
\end{pgfscope}%
\begin{pgfscope}%
\pgfsys@transformshift{1.930210in}{0.849613in}%
\pgfsys@useobject{currentmarker}{}%
\end{pgfscope}%
\begin{pgfscope}%
\pgfsys@transformshift{1.930811in}{0.817072in}%
\pgfsys@useobject{currentmarker}{}%
\end{pgfscope}%
\begin{pgfscope}%
\pgfsys@transformshift{1.931411in}{0.837572in}%
\pgfsys@useobject{currentmarker}{}%
\end{pgfscope}%
\begin{pgfscope}%
\pgfsys@transformshift{1.932010in}{0.834399in}%
\pgfsys@useobject{currentmarker}{}%
\end{pgfscope}%
\begin{pgfscope}%
\pgfsys@transformshift{1.932606in}{0.865978in}%
\pgfsys@useobject{currentmarker}{}%
\end{pgfscope}%
\begin{pgfscope}%
\pgfsys@transformshift{1.933201in}{0.844577in}%
\pgfsys@useobject{currentmarker}{}%
\end{pgfscope}%
\begin{pgfscope}%
\pgfsys@transformshift{1.933795in}{0.832464in}%
\pgfsys@useobject{currentmarker}{}%
\end{pgfscope}%
\begin{pgfscope}%
\pgfsys@transformshift{1.934387in}{0.848728in}%
\pgfsys@useobject{currentmarker}{}%
\end{pgfscope}%
\begin{pgfscope}%
\pgfsys@transformshift{1.934977in}{0.825778in}%
\pgfsys@useobject{currentmarker}{}%
\end{pgfscope}%
\begin{pgfscope}%
\pgfsys@transformshift{1.935566in}{0.754126in}%
\pgfsys@useobject{currentmarker}{}%
\end{pgfscope}%
\begin{pgfscope}%
\pgfsys@transformshift{1.936154in}{0.841103in}%
\pgfsys@useobject{currentmarker}{}%
\end{pgfscope}%
\begin{pgfscope}%
\pgfsys@transformshift{1.936739in}{0.841987in}%
\pgfsys@useobject{currentmarker}{}%
\end{pgfscope}%
\begin{pgfscope}%
\pgfsys@transformshift{1.937324in}{0.836999in}%
\pgfsys@useobject{currentmarker}{}%
\end{pgfscope}%
\begin{pgfscope}%
\pgfsys@transformshift{1.937906in}{0.875378in}%
\pgfsys@useobject{currentmarker}{}%
\end{pgfscope}%
\begin{pgfscope}%
\pgfsys@transformshift{1.938488in}{0.881792in}%
\pgfsys@useobject{currentmarker}{}%
\end{pgfscope}%
\begin{pgfscope}%
\pgfsys@transformshift{1.939067in}{0.889746in}%
\pgfsys@useobject{currentmarker}{}%
\end{pgfscope}%
\begin{pgfscope}%
\pgfsys@transformshift{1.939646in}{0.810120in}%
\pgfsys@useobject{currentmarker}{}%
\end{pgfscope}%
\begin{pgfscope}%
\pgfsys@transformshift{1.940222in}{0.817668in}%
\pgfsys@useobject{currentmarker}{}%
\end{pgfscope}%
\begin{pgfscope}%
\pgfsys@transformshift{1.940798in}{0.804678in}%
\pgfsys@useobject{currentmarker}{}%
\end{pgfscope}%
\begin{pgfscope}%
\pgfsys@transformshift{1.941371in}{0.831522in}%
\pgfsys@useobject{currentmarker}{}%
\end{pgfscope}%
\begin{pgfscope}%
\pgfsys@transformshift{1.941944in}{0.847745in}%
\pgfsys@useobject{currentmarker}{}%
\end{pgfscope}%
\begin{pgfscope}%
\pgfsys@transformshift{1.942515in}{0.852349in}%
\pgfsys@useobject{currentmarker}{}%
\end{pgfscope}%
\begin{pgfscope}%
\pgfsys@transformshift{1.943084in}{0.863436in}%
\pgfsys@useobject{currentmarker}{}%
\end{pgfscope}%
\begin{pgfscope}%
\pgfsys@transformshift{1.943652in}{0.855830in}%
\pgfsys@useobject{currentmarker}{}%
\end{pgfscope}%
\begin{pgfscope}%
\pgfsys@transformshift{1.944219in}{0.852371in}%
\pgfsys@useobject{currentmarker}{}%
\end{pgfscope}%
\begin{pgfscope}%
\pgfsys@transformshift{1.944784in}{0.848238in}%
\pgfsys@useobject{currentmarker}{}%
\end{pgfscope}%
\begin{pgfscope}%
\pgfsys@transformshift{1.945348in}{0.858159in}%
\pgfsys@useobject{currentmarker}{}%
\end{pgfscope}%
\begin{pgfscope}%
\pgfsys@transformshift{1.945910in}{0.862303in}%
\pgfsys@useobject{currentmarker}{}%
\end{pgfscope}%
\begin{pgfscope}%
\pgfsys@transformshift{1.946471in}{0.837512in}%
\pgfsys@useobject{currentmarker}{}%
\end{pgfscope}%
\begin{pgfscope}%
\pgfsys@transformshift{1.947031in}{0.821151in}%
\pgfsys@useobject{currentmarker}{}%
\end{pgfscope}%
\begin{pgfscope}%
\pgfsys@transformshift{1.947589in}{0.818575in}%
\pgfsys@useobject{currentmarker}{}%
\end{pgfscope}%
\begin{pgfscope}%
\pgfsys@transformshift{1.948145in}{0.881364in}%
\pgfsys@useobject{currentmarker}{}%
\end{pgfscope}%
\begin{pgfscope}%
\pgfsys@transformshift{1.948701in}{0.869961in}%
\pgfsys@useobject{currentmarker}{}%
\end{pgfscope}%
\begin{pgfscope}%
\pgfsys@transformshift{1.949255in}{0.854023in}%
\pgfsys@useobject{currentmarker}{}%
\end{pgfscope}%
\begin{pgfscope}%
\pgfsys@transformshift{1.949807in}{0.840069in}%
\pgfsys@useobject{currentmarker}{}%
\end{pgfscope}%
\begin{pgfscope}%
\pgfsys@transformshift{1.950359in}{0.861918in}%
\pgfsys@useobject{currentmarker}{}%
\end{pgfscope}%
\begin{pgfscope}%
\pgfsys@transformshift{1.950909in}{0.837782in}%
\pgfsys@useobject{currentmarker}{}%
\end{pgfscope}%
\begin{pgfscope}%
\pgfsys@transformshift{1.951457in}{0.837257in}%
\pgfsys@useobject{currentmarker}{}%
\end{pgfscope}%
\begin{pgfscope}%
\pgfsys@transformshift{1.952005in}{0.804909in}%
\pgfsys@useobject{currentmarker}{}%
\end{pgfscope}%
\begin{pgfscope}%
\pgfsys@transformshift{1.952550in}{0.835141in}%
\pgfsys@useobject{currentmarker}{}%
\end{pgfscope}%
\begin{pgfscope}%
\pgfsys@transformshift{1.953095in}{0.878779in}%
\pgfsys@useobject{currentmarker}{}%
\end{pgfscope}%
\begin{pgfscope}%
\pgfsys@transformshift{1.953638in}{0.876278in}%
\pgfsys@useobject{currentmarker}{}%
\end{pgfscope}%
\begin{pgfscope}%
\pgfsys@transformshift{1.954180in}{0.857814in}%
\pgfsys@useobject{currentmarker}{}%
\end{pgfscope}%
\begin{pgfscope}%
\pgfsys@transformshift{1.954721in}{0.894881in}%
\pgfsys@useobject{currentmarker}{}%
\end{pgfscope}%
\begin{pgfscope}%
\pgfsys@transformshift{1.955260in}{0.880440in}%
\pgfsys@useobject{currentmarker}{}%
\end{pgfscope}%
\begin{pgfscope}%
\pgfsys@transformshift{1.955799in}{0.833254in}%
\pgfsys@useobject{currentmarker}{}%
\end{pgfscope}%
\begin{pgfscope}%
\pgfsys@transformshift{1.956335in}{0.852752in}%
\pgfsys@useobject{currentmarker}{}%
\end{pgfscope}%
\begin{pgfscope}%
\pgfsys@transformshift{1.956871in}{0.872189in}%
\pgfsys@useobject{currentmarker}{}%
\end{pgfscope}%
\begin{pgfscope}%
\pgfsys@transformshift{1.957405in}{0.863373in}%
\pgfsys@useobject{currentmarker}{}%
\end{pgfscope}%
\begin{pgfscope}%
\pgfsys@transformshift{1.957938in}{0.803570in}%
\pgfsys@useobject{currentmarker}{}%
\end{pgfscope}%
\begin{pgfscope}%
\pgfsys@transformshift{1.958470in}{0.836460in}%
\pgfsys@useobject{currentmarker}{}%
\end{pgfscope}%
\begin{pgfscope}%
\pgfsys@transformshift{1.959000in}{0.879784in}%
\pgfsys@useobject{currentmarker}{}%
\end{pgfscope}%
\begin{pgfscope}%
\pgfsys@transformshift{1.959529in}{0.843010in}%
\pgfsys@useobject{currentmarker}{}%
\end{pgfscope}%
\begin{pgfscope}%
\pgfsys@transformshift{1.960057in}{0.816646in}%
\pgfsys@useobject{currentmarker}{}%
\end{pgfscope}%
\begin{pgfscope}%
\pgfsys@transformshift{1.960584in}{0.773861in}%
\pgfsys@useobject{currentmarker}{}%
\end{pgfscope}%
\begin{pgfscope}%
\pgfsys@transformshift{1.961110in}{0.805317in}%
\pgfsys@useobject{currentmarker}{}%
\end{pgfscope}%
\begin{pgfscope}%
\pgfsys@transformshift{1.961634in}{0.826616in}%
\pgfsys@useobject{currentmarker}{}%
\end{pgfscope}%
\begin{pgfscope}%
\pgfsys@transformshift{1.962157in}{0.815432in}%
\pgfsys@useobject{currentmarker}{}%
\end{pgfscope}%
\begin{pgfscope}%
\pgfsys@transformshift{1.962679in}{0.835468in}%
\pgfsys@useobject{currentmarker}{}%
\end{pgfscope}%
\begin{pgfscope}%
\pgfsys@transformshift{1.963199in}{0.858071in}%
\pgfsys@useobject{currentmarker}{}%
\end{pgfscope}%
\begin{pgfscope}%
\pgfsys@transformshift{1.963719in}{0.831046in}%
\pgfsys@useobject{currentmarker}{}%
\end{pgfscope}%
\begin{pgfscope}%
\pgfsys@transformshift{1.964237in}{0.819070in}%
\pgfsys@useobject{currentmarker}{}%
\end{pgfscope}%
\begin{pgfscope}%
\pgfsys@transformshift{1.964754in}{0.856447in}%
\pgfsys@useobject{currentmarker}{}%
\end{pgfscope}%
\begin{pgfscope}%
\pgfsys@transformshift{1.965270in}{0.860059in}%
\pgfsys@useobject{currentmarker}{}%
\end{pgfscope}%
\begin{pgfscope}%
\pgfsys@transformshift{1.965785in}{0.810766in}%
\pgfsys@useobject{currentmarker}{}%
\end{pgfscope}%
\begin{pgfscope}%
\pgfsys@transformshift{1.966298in}{0.813258in}%
\pgfsys@useobject{currentmarker}{}%
\end{pgfscope}%
\begin{pgfscope}%
\pgfsys@transformshift{1.966810in}{0.863284in}%
\pgfsys@useobject{currentmarker}{}%
\end{pgfscope}%
\begin{pgfscope}%
\pgfsys@transformshift{1.967322in}{0.874811in}%
\pgfsys@useobject{currentmarker}{}%
\end{pgfscope}%
\begin{pgfscope}%
\pgfsys@transformshift{1.967832in}{0.849959in}%
\pgfsys@useobject{currentmarker}{}%
\end{pgfscope}%
\begin{pgfscope}%
\pgfsys@transformshift{1.968340in}{0.800901in}%
\pgfsys@useobject{currentmarker}{}%
\end{pgfscope}%
\begin{pgfscope}%
\pgfsys@transformshift{1.968848in}{0.795933in}%
\pgfsys@useobject{currentmarker}{}%
\end{pgfscope}%
\begin{pgfscope}%
\pgfsys@transformshift{1.969355in}{0.831290in}%
\pgfsys@useobject{currentmarker}{}%
\end{pgfscope}%
\begin{pgfscope}%
\pgfsys@transformshift{1.969860in}{0.804019in}%
\pgfsys@useobject{currentmarker}{}%
\end{pgfscope}%
\begin{pgfscope}%
\pgfsys@transformshift{1.970364in}{0.836466in}%
\pgfsys@useobject{currentmarker}{}%
\end{pgfscope}%
\begin{pgfscope}%
\pgfsys@transformshift{1.970868in}{0.818526in}%
\pgfsys@useobject{currentmarker}{}%
\end{pgfscope}%
\begin{pgfscope}%
\pgfsys@transformshift{1.971370in}{0.796743in}%
\pgfsys@useobject{currentmarker}{}%
\end{pgfscope}%
\begin{pgfscope}%
\pgfsys@transformshift{1.971870in}{0.787864in}%
\pgfsys@useobject{currentmarker}{}%
\end{pgfscope}%
\begin{pgfscope}%
\pgfsys@transformshift{1.972370in}{0.782755in}%
\pgfsys@useobject{currentmarker}{}%
\end{pgfscope}%
\begin{pgfscope}%
\pgfsys@transformshift{1.972869in}{0.852076in}%
\pgfsys@useobject{currentmarker}{}%
\end{pgfscope}%
\begin{pgfscope}%
\pgfsys@transformshift{1.973366in}{0.823058in}%
\pgfsys@useobject{currentmarker}{}%
\end{pgfscope}%
\begin{pgfscope}%
\pgfsys@transformshift{1.973863in}{0.796900in}%
\pgfsys@useobject{currentmarker}{}%
\end{pgfscope}%
\begin{pgfscope}%
\pgfsys@transformshift{1.974358in}{0.822127in}%
\pgfsys@useobject{currentmarker}{}%
\end{pgfscope}%
\begin{pgfscope}%
\pgfsys@transformshift{1.974853in}{0.831811in}%
\pgfsys@useobject{currentmarker}{}%
\end{pgfscope}%
\begin{pgfscope}%
\pgfsys@transformshift{1.975346in}{0.852858in}%
\pgfsys@useobject{currentmarker}{}%
\end{pgfscope}%
\begin{pgfscope}%
\pgfsys@transformshift{1.975838in}{0.845188in}%
\pgfsys@useobject{currentmarker}{}%
\end{pgfscope}%
\begin{pgfscope}%
\pgfsys@transformshift{1.976329in}{0.855269in}%
\pgfsys@useobject{currentmarker}{}%
\end{pgfscope}%
\begin{pgfscope}%
\pgfsys@transformshift{1.976819in}{0.840067in}%
\pgfsys@useobject{currentmarker}{}%
\end{pgfscope}%
\begin{pgfscope}%
\pgfsys@transformshift{1.977308in}{0.828964in}%
\pgfsys@useobject{currentmarker}{}%
\end{pgfscope}%
\begin{pgfscope}%
\pgfsys@transformshift{1.977796in}{0.827034in}%
\pgfsys@useobject{currentmarker}{}%
\end{pgfscope}%
\begin{pgfscope}%
\pgfsys@transformshift{1.978282in}{0.807477in}%
\pgfsys@useobject{currentmarker}{}%
\end{pgfscope}%
\begin{pgfscope}%
\pgfsys@transformshift{1.978768in}{0.797191in}%
\pgfsys@useobject{currentmarker}{}%
\end{pgfscope}%
\begin{pgfscope}%
\pgfsys@transformshift{1.979253in}{0.825363in}%
\pgfsys@useobject{currentmarker}{}%
\end{pgfscope}%
\begin{pgfscope}%
\pgfsys@transformshift{1.979736in}{0.802362in}%
\pgfsys@useobject{currentmarker}{}%
\end{pgfscope}%
\begin{pgfscope}%
\pgfsys@transformshift{1.980219in}{0.810875in}%
\pgfsys@useobject{currentmarker}{}%
\end{pgfscope}%
\begin{pgfscope}%
\pgfsys@transformshift{1.980701in}{0.817469in}%
\pgfsys@useobject{currentmarker}{}%
\end{pgfscope}%
\begin{pgfscope}%
\pgfsys@transformshift{1.981181in}{0.857488in}%
\pgfsys@useobject{currentmarker}{}%
\end{pgfscope}%
\begin{pgfscope}%
\pgfsys@transformshift{1.981661in}{0.849323in}%
\pgfsys@useobject{currentmarker}{}%
\end{pgfscope}%
\begin{pgfscope}%
\pgfsys@transformshift{1.982139in}{0.848861in}%
\pgfsys@useobject{currentmarker}{}%
\end{pgfscope}%
\begin{pgfscope}%
\pgfsys@transformshift{1.982617in}{0.846113in}%
\pgfsys@useobject{currentmarker}{}%
\end{pgfscope}%
\begin{pgfscope}%
\pgfsys@transformshift{1.983093in}{0.824861in}%
\pgfsys@useobject{currentmarker}{}%
\end{pgfscope}%
\begin{pgfscope}%
\pgfsys@transformshift{1.983569in}{0.824708in}%
\pgfsys@useobject{currentmarker}{}%
\end{pgfscope}%
\begin{pgfscope}%
\pgfsys@transformshift{1.984043in}{0.798602in}%
\pgfsys@useobject{currentmarker}{}%
\end{pgfscope}%
\begin{pgfscope}%
\pgfsys@transformshift{1.984517in}{0.824694in}%
\pgfsys@useobject{currentmarker}{}%
\end{pgfscope}%
\begin{pgfscope}%
\pgfsys@transformshift{1.984989in}{0.839707in}%
\pgfsys@useobject{currentmarker}{}%
\end{pgfscope}%
\begin{pgfscope}%
\pgfsys@transformshift{1.985460in}{0.820366in}%
\pgfsys@useobject{currentmarker}{}%
\end{pgfscope}%
\begin{pgfscope}%
\pgfsys@transformshift{1.985931in}{0.791033in}%
\pgfsys@useobject{currentmarker}{}%
\end{pgfscope}%
\begin{pgfscope}%
\pgfsys@transformshift{1.986400in}{0.803211in}%
\pgfsys@useobject{currentmarker}{}%
\end{pgfscope}%
\begin{pgfscope}%
\pgfsys@transformshift{1.986869in}{0.770530in}%
\pgfsys@useobject{currentmarker}{}%
\end{pgfscope}%
\begin{pgfscope}%
\pgfsys@transformshift{1.987336in}{0.799032in}%
\pgfsys@useobject{currentmarker}{}%
\end{pgfscope}%
\begin{pgfscope}%
\pgfsys@transformshift{1.987803in}{0.830753in}%
\pgfsys@useobject{currentmarker}{}%
\end{pgfscope}%
\begin{pgfscope}%
\pgfsys@transformshift{1.988269in}{0.802493in}%
\pgfsys@useobject{currentmarker}{}%
\end{pgfscope}%
\begin{pgfscope}%
\pgfsys@transformshift{1.988733in}{0.762829in}%
\pgfsys@useobject{currentmarker}{}%
\end{pgfscope}%
\begin{pgfscope}%
\pgfsys@transformshift{1.989197in}{0.786480in}%
\pgfsys@useobject{currentmarker}{}%
\end{pgfscope}%
\begin{pgfscope}%
\pgfsys@transformshift{1.989660in}{0.849395in}%
\pgfsys@useobject{currentmarker}{}%
\end{pgfscope}%
\begin{pgfscope}%
\pgfsys@transformshift{1.990121in}{0.861601in}%
\pgfsys@useobject{currentmarker}{}%
\end{pgfscope}%
\begin{pgfscope}%
\pgfsys@transformshift{1.990582in}{0.844780in}%
\pgfsys@useobject{currentmarker}{}%
\end{pgfscope}%
\begin{pgfscope}%
\pgfsys@transformshift{1.991042in}{0.789673in}%
\pgfsys@useobject{currentmarker}{}%
\end{pgfscope}%
\begin{pgfscope}%
\pgfsys@transformshift{1.991501in}{0.773997in}%
\pgfsys@useobject{currentmarker}{}%
\end{pgfscope}%
\begin{pgfscope}%
\pgfsys@transformshift{1.991959in}{0.806476in}%
\pgfsys@useobject{currentmarker}{}%
\end{pgfscope}%
\begin{pgfscope}%
\pgfsys@transformshift{1.992416in}{0.823982in}%
\pgfsys@useobject{currentmarker}{}%
\end{pgfscope}%
\begin{pgfscope}%
\pgfsys@transformshift{1.992872in}{0.816315in}%
\pgfsys@useobject{currentmarker}{}%
\end{pgfscope}%
\begin{pgfscope}%
\pgfsys@transformshift{1.993328in}{0.840357in}%
\pgfsys@useobject{currentmarker}{}%
\end{pgfscope}%
\begin{pgfscope}%
\pgfsys@transformshift{1.993782in}{0.823913in}%
\pgfsys@useobject{currentmarker}{}%
\end{pgfscope}%
\begin{pgfscope}%
\pgfsys@transformshift{1.994235in}{0.804318in}%
\pgfsys@useobject{currentmarker}{}%
\end{pgfscope}%
\begin{pgfscope}%
\pgfsys@transformshift{1.994688in}{0.799733in}%
\pgfsys@useobject{currentmarker}{}%
\end{pgfscope}%
\begin{pgfscope}%
\pgfsys@transformshift{1.995139in}{0.863067in}%
\pgfsys@useobject{currentmarker}{}%
\end{pgfscope}%
\begin{pgfscope}%
\pgfsys@transformshift{1.995590in}{0.860490in}%
\pgfsys@useobject{currentmarker}{}%
\end{pgfscope}%
\begin{pgfscope}%
\pgfsys@transformshift{1.996040in}{0.809483in}%
\pgfsys@useobject{currentmarker}{}%
\end{pgfscope}%
\begin{pgfscope}%
\pgfsys@transformshift{1.996488in}{0.797447in}%
\pgfsys@useobject{currentmarker}{}%
\end{pgfscope}%
\begin{pgfscope}%
\pgfsys@transformshift{1.996936in}{0.815548in}%
\pgfsys@useobject{currentmarker}{}%
\end{pgfscope}%
\begin{pgfscope}%
\pgfsys@transformshift{1.997384in}{0.832481in}%
\pgfsys@useobject{currentmarker}{}%
\end{pgfscope}%
\begin{pgfscope}%
\pgfsys@transformshift{1.997830in}{0.805600in}%
\pgfsys@useobject{currentmarker}{}%
\end{pgfscope}%
\begin{pgfscope}%
\pgfsys@transformshift{1.998275in}{0.812903in}%
\pgfsys@useobject{currentmarker}{}%
\end{pgfscope}%
\begin{pgfscope}%
\pgfsys@transformshift{1.998719in}{0.843620in}%
\pgfsys@useobject{currentmarker}{}%
\end{pgfscope}%
\begin{pgfscope}%
\pgfsys@transformshift{1.999163in}{0.824199in}%
\pgfsys@useobject{currentmarker}{}%
\end{pgfscope}%
\begin{pgfscope}%
\pgfsys@transformshift{1.999606in}{0.765688in}%
\pgfsys@useobject{currentmarker}{}%
\end{pgfscope}%
\begin{pgfscope}%
\pgfsys@transformshift{2.000047in}{0.815875in}%
\pgfsys@useobject{currentmarker}{}%
\end{pgfscope}%
\begin{pgfscope}%
\pgfsys@transformshift{2.000488in}{0.816073in}%
\pgfsys@useobject{currentmarker}{}%
\end{pgfscope}%
\begin{pgfscope}%
\pgfsys@transformshift{2.000928in}{0.835400in}%
\pgfsys@useobject{currentmarker}{}%
\end{pgfscope}%
\begin{pgfscope}%
\pgfsys@transformshift{2.001368in}{0.803598in}%
\pgfsys@useobject{currentmarker}{}%
\end{pgfscope}%
\begin{pgfscope}%
\pgfsys@transformshift{2.001806in}{0.804704in}%
\pgfsys@useobject{currentmarker}{}%
\end{pgfscope}%
\begin{pgfscope}%
\pgfsys@transformshift{2.002243in}{0.848122in}%
\pgfsys@useobject{currentmarker}{}%
\end{pgfscope}%
\begin{pgfscope}%
\pgfsys@transformshift{2.002680in}{0.736194in}%
\pgfsys@useobject{currentmarker}{}%
\end{pgfscope}%
\begin{pgfscope}%
\pgfsys@transformshift{2.003116in}{0.748053in}%
\pgfsys@useobject{currentmarker}{}%
\end{pgfscope}%
\begin{pgfscope}%
\pgfsys@transformshift{2.003551in}{0.785164in}%
\pgfsys@useobject{currentmarker}{}%
\end{pgfscope}%
\begin{pgfscope}%
\pgfsys@transformshift{2.003985in}{0.846940in}%
\pgfsys@useobject{currentmarker}{}%
\end{pgfscope}%
\begin{pgfscope}%
\pgfsys@transformshift{2.004418in}{0.826073in}%
\pgfsys@useobject{currentmarker}{}%
\end{pgfscope}%
\begin{pgfscope}%
\pgfsys@transformshift{2.004851in}{0.732683in}%
\pgfsys@useobject{currentmarker}{}%
\end{pgfscope}%
\begin{pgfscope}%
\pgfsys@transformshift{2.005282in}{0.764420in}%
\pgfsys@useobject{currentmarker}{}%
\end{pgfscope}%
\begin{pgfscope}%
\pgfsys@transformshift{2.005713in}{0.750013in}%
\pgfsys@useobject{currentmarker}{}%
\end{pgfscope}%
\begin{pgfscope}%
\pgfsys@transformshift{2.006143in}{0.841089in}%
\pgfsys@useobject{currentmarker}{}%
\end{pgfscope}%
\begin{pgfscope}%
\pgfsys@transformshift{2.006572in}{0.796806in}%
\pgfsys@useobject{currentmarker}{}%
\end{pgfscope}%
\begin{pgfscope}%
\pgfsys@transformshift{2.007000in}{0.777390in}%
\pgfsys@useobject{currentmarker}{}%
\end{pgfscope}%
\begin{pgfscope}%
\pgfsys@transformshift{2.007428in}{0.816028in}%
\pgfsys@useobject{currentmarker}{}%
\end{pgfscope}%
\begin{pgfscope}%
\pgfsys@transformshift{2.007855in}{0.842749in}%
\pgfsys@useobject{currentmarker}{}%
\end{pgfscope}%
\begin{pgfscope}%
\pgfsys@transformshift{2.008280in}{0.837164in}%
\pgfsys@useobject{currentmarker}{}%
\end{pgfscope}%
\begin{pgfscope}%
\pgfsys@transformshift{2.008706in}{0.795140in}%
\pgfsys@useobject{currentmarker}{}%
\end{pgfscope}%
\begin{pgfscope}%
\pgfsys@transformshift{2.009130in}{0.760137in}%
\pgfsys@useobject{currentmarker}{}%
\end{pgfscope}%
\begin{pgfscope}%
\pgfsys@transformshift{2.009553in}{0.793978in}%
\pgfsys@useobject{currentmarker}{}%
\end{pgfscope}%
\begin{pgfscope}%
\pgfsys@transformshift{2.009976in}{0.836378in}%
\pgfsys@useobject{currentmarker}{}%
\end{pgfscope}%
\begin{pgfscope}%
\pgfsys@transformshift{2.010398in}{0.815180in}%
\pgfsys@useobject{currentmarker}{}%
\end{pgfscope}%
\begin{pgfscope}%
\pgfsys@transformshift{2.010819in}{0.780410in}%
\pgfsys@useobject{currentmarker}{}%
\end{pgfscope}%
\begin{pgfscope}%
\pgfsys@transformshift{2.011239in}{0.798333in}%
\pgfsys@useobject{currentmarker}{}%
\end{pgfscope}%
\begin{pgfscope}%
\pgfsys@transformshift{2.011659in}{0.824686in}%
\pgfsys@useobject{currentmarker}{}%
\end{pgfscope}%
\begin{pgfscope}%
\pgfsys@transformshift{2.012078in}{0.834952in}%
\pgfsys@useobject{currentmarker}{}%
\end{pgfscope}%
\begin{pgfscope}%
\pgfsys@transformshift{2.012495in}{0.791375in}%
\pgfsys@useobject{currentmarker}{}%
\end{pgfscope}%
\begin{pgfscope}%
\pgfsys@transformshift{2.012913in}{0.794212in}%
\pgfsys@useobject{currentmarker}{}%
\end{pgfscope}%
\begin{pgfscope}%
\pgfsys@transformshift{2.013329in}{0.783574in}%
\pgfsys@useobject{currentmarker}{}%
\end{pgfscope}%
\begin{pgfscope}%
\pgfsys@transformshift{2.013745in}{0.809414in}%
\pgfsys@useobject{currentmarker}{}%
\end{pgfscope}%
\begin{pgfscope}%
\pgfsys@transformshift{2.014160in}{0.802448in}%
\pgfsys@useobject{currentmarker}{}%
\end{pgfscope}%
\begin{pgfscope}%
\pgfsys@transformshift{2.014574in}{0.824433in}%
\pgfsys@useobject{currentmarker}{}%
\end{pgfscope}%
\begin{pgfscope}%
\pgfsys@transformshift{2.014987in}{0.829897in}%
\pgfsys@useobject{currentmarker}{}%
\end{pgfscope}%
\begin{pgfscope}%
\pgfsys@transformshift{2.015400in}{0.794305in}%
\pgfsys@useobject{currentmarker}{}%
\end{pgfscope}%
\begin{pgfscope}%
\pgfsys@transformshift{2.015811in}{0.769556in}%
\pgfsys@useobject{currentmarker}{}%
\end{pgfscope}%
\begin{pgfscope}%
\pgfsys@transformshift{2.016223in}{0.808688in}%
\pgfsys@useobject{currentmarker}{}%
\end{pgfscope}%
\begin{pgfscope}%
\pgfsys@transformshift{2.016633in}{0.812639in}%
\pgfsys@useobject{currentmarker}{}%
\end{pgfscope}%
\begin{pgfscope}%
\pgfsys@transformshift{2.017042in}{0.787931in}%
\pgfsys@useobject{currentmarker}{}%
\end{pgfscope}%
\begin{pgfscope}%
\pgfsys@transformshift{2.017451in}{0.772141in}%
\pgfsys@useobject{currentmarker}{}%
\end{pgfscope}%
\begin{pgfscope}%
\pgfsys@transformshift{2.017859in}{0.783414in}%
\pgfsys@useobject{currentmarker}{}%
\end{pgfscope}%
\begin{pgfscope}%
\pgfsys@transformshift{2.018267in}{0.762160in}%
\pgfsys@useobject{currentmarker}{}%
\end{pgfscope}%
\begin{pgfscope}%
\pgfsys@transformshift{2.018673in}{0.804974in}%
\pgfsys@useobject{currentmarker}{}%
\end{pgfscope}%
\begin{pgfscope}%
\pgfsys@transformshift{2.019079in}{0.846118in}%
\pgfsys@useobject{currentmarker}{}%
\end{pgfscope}%
\begin{pgfscope}%
\pgfsys@transformshift{2.019484in}{0.821338in}%
\pgfsys@useobject{currentmarker}{}%
\end{pgfscope}%
\begin{pgfscope}%
\pgfsys@transformshift{2.019889in}{0.811291in}%
\pgfsys@useobject{currentmarker}{}%
\end{pgfscope}%
\begin{pgfscope}%
\pgfsys@transformshift{2.020292in}{0.822839in}%
\pgfsys@useobject{currentmarker}{}%
\end{pgfscope}%
\begin{pgfscope}%
\pgfsys@transformshift{2.020695in}{0.818493in}%
\pgfsys@useobject{currentmarker}{}%
\end{pgfscope}%
\begin{pgfscope}%
\pgfsys@transformshift{2.021098in}{0.804460in}%
\pgfsys@useobject{currentmarker}{}%
\end{pgfscope}%
\begin{pgfscope}%
\pgfsys@transformshift{2.021499in}{0.817500in}%
\pgfsys@useobject{currentmarker}{}%
\end{pgfscope}%
\begin{pgfscope}%
\pgfsys@transformshift{2.021900in}{0.803444in}%
\pgfsys@useobject{currentmarker}{}%
\end{pgfscope}%
\begin{pgfscope}%
\pgfsys@transformshift{2.022300in}{0.787036in}%
\pgfsys@useobject{currentmarker}{}%
\end{pgfscope}%
\begin{pgfscope}%
\pgfsys@transformshift{2.022699in}{0.777913in}%
\pgfsys@useobject{currentmarker}{}%
\end{pgfscope}%
\begin{pgfscope}%
\pgfsys@transformshift{2.023098in}{0.817688in}%
\pgfsys@useobject{currentmarker}{}%
\end{pgfscope}%
\begin{pgfscope}%
\pgfsys@transformshift{2.023496in}{0.833024in}%
\pgfsys@useobject{currentmarker}{}%
\end{pgfscope}%
\begin{pgfscope}%
\pgfsys@transformshift{2.023893in}{0.814919in}%
\pgfsys@useobject{currentmarker}{}%
\end{pgfscope}%
\begin{pgfscope}%
\pgfsys@transformshift{2.024290in}{0.834457in}%
\pgfsys@useobject{currentmarker}{}%
\end{pgfscope}%
\begin{pgfscope}%
\pgfsys@transformshift{2.024686in}{0.786378in}%
\pgfsys@useobject{currentmarker}{}%
\end{pgfscope}%
\begin{pgfscope}%
\pgfsys@transformshift{2.025081in}{0.734385in}%
\pgfsys@useobject{currentmarker}{}%
\end{pgfscope}%
\begin{pgfscope}%
\pgfsys@transformshift{2.025475in}{0.760606in}%
\pgfsys@useobject{currentmarker}{}%
\end{pgfscope}%
\begin{pgfscope}%
\pgfsys@transformshift{2.025869in}{0.822223in}%
\pgfsys@useobject{currentmarker}{}%
\end{pgfscope}%
\begin{pgfscope}%
\pgfsys@transformshift{2.026262in}{0.846939in}%
\pgfsys@useobject{currentmarker}{}%
\end{pgfscope}%
\begin{pgfscope}%
\pgfsys@transformshift{2.026655in}{0.818413in}%
\pgfsys@useobject{currentmarker}{}%
\end{pgfscope}%
\begin{pgfscope}%
\pgfsys@transformshift{2.027046in}{0.772538in}%
\pgfsys@useobject{currentmarker}{}%
\end{pgfscope}%
\begin{pgfscope}%
\pgfsys@transformshift{2.027437in}{0.764981in}%
\pgfsys@useobject{currentmarker}{}%
\end{pgfscope}%
\begin{pgfscope}%
\pgfsys@transformshift{2.027828in}{0.817610in}%
\pgfsys@useobject{currentmarker}{}%
\end{pgfscope}%
\begin{pgfscope}%
\pgfsys@transformshift{2.028217in}{0.795965in}%
\pgfsys@useobject{currentmarker}{}%
\end{pgfscope}%
\begin{pgfscope}%
\pgfsys@transformshift{2.028606in}{0.797624in}%
\pgfsys@useobject{currentmarker}{}%
\end{pgfscope}%
\begin{pgfscope}%
\pgfsys@transformshift{2.028995in}{0.767084in}%
\pgfsys@useobject{currentmarker}{}%
\end{pgfscope}%
\begin{pgfscope}%
\pgfsys@transformshift{2.029382in}{0.758665in}%
\pgfsys@useobject{currentmarker}{}%
\end{pgfscope}%
\begin{pgfscope}%
\pgfsys@transformshift{2.029769in}{0.827693in}%
\pgfsys@useobject{currentmarker}{}%
\end{pgfscope}%
\begin{pgfscope}%
\pgfsys@transformshift{2.030156in}{0.814346in}%
\pgfsys@useobject{currentmarker}{}%
\end{pgfscope}%
\begin{pgfscope}%
\pgfsys@transformshift{2.030541in}{0.754448in}%
\pgfsys@useobject{currentmarker}{}%
\end{pgfscope}%
\begin{pgfscope}%
\pgfsys@transformshift{2.030926in}{0.745627in}%
\pgfsys@useobject{currentmarker}{}%
\end{pgfscope}%
\begin{pgfscope}%
\pgfsys@transformshift{2.031311in}{0.784945in}%
\pgfsys@useobject{currentmarker}{}%
\end{pgfscope}%
\begin{pgfscope}%
\pgfsys@transformshift{2.031694in}{0.815758in}%
\pgfsys@useobject{currentmarker}{}%
\end{pgfscope}%
\begin{pgfscope}%
\pgfsys@transformshift{2.032078in}{0.810155in}%
\pgfsys@useobject{currentmarker}{}%
\end{pgfscope}%
\begin{pgfscope}%
\pgfsys@transformshift{2.032460in}{0.790675in}%
\pgfsys@useobject{currentmarker}{}%
\end{pgfscope}%
\begin{pgfscope}%
\pgfsys@transformshift{2.032842in}{0.764484in}%
\pgfsys@useobject{currentmarker}{}%
\end{pgfscope}%
\begin{pgfscope}%
\pgfsys@transformshift{2.033223in}{0.811195in}%
\pgfsys@useobject{currentmarker}{}%
\end{pgfscope}%
\begin{pgfscope}%
\pgfsys@transformshift{2.033603in}{0.765842in}%
\pgfsys@useobject{currentmarker}{}%
\end{pgfscope}%
\begin{pgfscope}%
\pgfsys@transformshift{2.033983in}{0.794832in}%
\pgfsys@useobject{currentmarker}{}%
\end{pgfscope}%
\begin{pgfscope}%
\pgfsys@transformshift{2.034362in}{0.809335in}%
\pgfsys@useobject{currentmarker}{}%
\end{pgfscope}%
\begin{pgfscope}%
\pgfsys@transformshift{2.034741in}{0.744305in}%
\pgfsys@useobject{currentmarker}{}%
\end{pgfscope}%
\begin{pgfscope}%
\pgfsys@transformshift{2.035119in}{0.796932in}%
\pgfsys@useobject{currentmarker}{}%
\end{pgfscope}%
\begin{pgfscope}%
\pgfsys@transformshift{2.035496in}{0.807257in}%
\pgfsys@useobject{currentmarker}{}%
\end{pgfscope}%
\begin{pgfscope}%
\pgfsys@transformshift{2.035872in}{0.794995in}%
\pgfsys@useobject{currentmarker}{}%
\end{pgfscope}%
\begin{pgfscope}%
\pgfsys@transformshift{2.036248in}{0.837130in}%
\pgfsys@useobject{currentmarker}{}%
\end{pgfscope}%
\begin{pgfscope}%
\pgfsys@transformshift{2.036624in}{0.832889in}%
\pgfsys@useobject{currentmarker}{}%
\end{pgfscope}%
\begin{pgfscope}%
\pgfsys@transformshift{2.036998in}{0.791830in}%
\pgfsys@useobject{currentmarker}{}%
\end{pgfscope}%
\begin{pgfscope}%
\pgfsys@transformshift{2.037373in}{0.784453in}%
\pgfsys@useobject{currentmarker}{}%
\end{pgfscope}%
\begin{pgfscope}%
\pgfsys@transformshift{2.037746in}{0.797190in}%
\pgfsys@useobject{currentmarker}{}%
\end{pgfscope}%
\begin{pgfscope}%
\pgfsys@transformshift{2.038119in}{0.795050in}%
\pgfsys@useobject{currentmarker}{}%
\end{pgfscope}%
\begin{pgfscope}%
\pgfsys@transformshift{2.038491in}{0.798083in}%
\pgfsys@useobject{currentmarker}{}%
\end{pgfscope}%
\begin{pgfscope}%
\pgfsys@transformshift{2.038863in}{0.744900in}%
\pgfsys@useobject{currentmarker}{}%
\end{pgfscope}%
\begin{pgfscope}%
\pgfsys@transformshift{2.039234in}{0.779797in}%
\pgfsys@useobject{currentmarker}{}%
\end{pgfscope}%
\begin{pgfscope}%
\pgfsys@transformshift{2.039604in}{0.776970in}%
\pgfsys@useobject{currentmarker}{}%
\end{pgfscope}%
\begin{pgfscope}%
\pgfsys@transformshift{2.039974in}{0.765209in}%
\pgfsys@useobject{currentmarker}{}%
\end{pgfscope}%
\begin{pgfscope}%
\pgfsys@transformshift{2.040343in}{0.809163in}%
\pgfsys@useobject{currentmarker}{}%
\end{pgfscope}%
\begin{pgfscope}%
\pgfsys@transformshift{2.040712in}{0.776786in}%
\pgfsys@useobject{currentmarker}{}%
\end{pgfscope}%
\begin{pgfscope}%
\pgfsys@transformshift{2.041080in}{0.780541in}%
\pgfsys@useobject{currentmarker}{}%
\end{pgfscope}%
\begin{pgfscope}%
\pgfsys@transformshift{2.041447in}{0.800843in}%
\pgfsys@useobject{currentmarker}{}%
\end{pgfscope}%
\begin{pgfscope}%
\pgfsys@transformshift{2.041814in}{0.825914in}%
\pgfsys@useobject{currentmarker}{}%
\end{pgfscope}%
\begin{pgfscope}%
\pgfsys@transformshift{2.042180in}{0.864052in}%
\pgfsys@useobject{currentmarker}{}%
\end{pgfscope}%
\begin{pgfscope}%
\pgfsys@transformshift{2.042546in}{0.770022in}%
\pgfsys@useobject{currentmarker}{}%
\end{pgfscope}%
\begin{pgfscope}%
\pgfsys@transformshift{2.042911in}{0.778317in}%
\pgfsys@useobject{currentmarker}{}%
\end{pgfscope}%
\begin{pgfscope}%
\pgfsys@transformshift{2.043275in}{0.836923in}%
\pgfsys@useobject{currentmarker}{}%
\end{pgfscope}%
\begin{pgfscope}%
\pgfsys@transformshift{2.043639in}{0.803202in}%
\pgfsys@useobject{currentmarker}{}%
\end{pgfscope}%
\begin{pgfscope}%
\pgfsys@transformshift{2.044002in}{0.765283in}%
\pgfsys@useobject{currentmarker}{}%
\end{pgfscope}%
\begin{pgfscope}%
\pgfsys@transformshift{2.044365in}{0.813224in}%
\pgfsys@useobject{currentmarker}{}%
\end{pgfscope}%
\begin{pgfscope}%
\pgfsys@transformshift{2.044727in}{0.792821in}%
\pgfsys@useobject{currentmarker}{}%
\end{pgfscope}%
\begin{pgfscope}%
\pgfsys@transformshift{2.045088in}{0.757780in}%
\pgfsys@useobject{currentmarker}{}%
\end{pgfscope}%
\begin{pgfscope}%
\pgfsys@transformshift{2.045449in}{0.763163in}%
\pgfsys@useobject{currentmarker}{}%
\end{pgfscope}%
\begin{pgfscope}%
\pgfsys@transformshift{2.045809in}{0.783541in}%
\pgfsys@useobject{currentmarker}{}%
\end{pgfscope}%
\begin{pgfscope}%
\pgfsys@transformshift{2.046169in}{0.792656in}%
\pgfsys@useobject{currentmarker}{}%
\end{pgfscope}%
\begin{pgfscope}%
\pgfsys@transformshift{2.046528in}{0.771310in}%
\pgfsys@useobject{currentmarker}{}%
\end{pgfscope}%
\begin{pgfscope}%
\pgfsys@transformshift{2.046887in}{0.742986in}%
\pgfsys@useobject{currentmarker}{}%
\end{pgfscope}%
\begin{pgfscope}%
\pgfsys@transformshift{2.047245in}{0.788321in}%
\pgfsys@useobject{currentmarker}{}%
\end{pgfscope}%
\begin{pgfscope}%
\pgfsys@transformshift{2.047602in}{0.795158in}%
\pgfsys@useobject{currentmarker}{}%
\end{pgfscope}%
\begin{pgfscope}%
\pgfsys@transformshift{2.047959in}{0.795058in}%
\pgfsys@useobject{currentmarker}{}%
\end{pgfscope}%
\begin{pgfscope}%
\pgfsys@transformshift{2.048316in}{0.795739in}%
\pgfsys@useobject{currentmarker}{}%
\end{pgfscope}%
\begin{pgfscope}%
\pgfsys@transformshift{2.048671in}{0.792409in}%
\pgfsys@useobject{currentmarker}{}%
\end{pgfscope}%
\begin{pgfscope}%
\pgfsys@transformshift{2.049027in}{0.776553in}%
\pgfsys@useobject{currentmarker}{}%
\end{pgfscope}%
\begin{pgfscope}%
\pgfsys@transformshift{2.049381in}{0.784012in}%
\pgfsys@useobject{currentmarker}{}%
\end{pgfscope}%
\begin{pgfscope}%
\pgfsys@transformshift{2.049735in}{0.772137in}%
\pgfsys@useobject{currentmarker}{}%
\end{pgfscope}%
\begin{pgfscope}%
\pgfsys@transformshift{2.050089in}{0.791689in}%
\pgfsys@useobject{currentmarker}{}%
\end{pgfscope}%
\begin{pgfscope}%
\pgfsys@transformshift{2.050442in}{0.807334in}%
\pgfsys@useobject{currentmarker}{}%
\end{pgfscope}%
\begin{pgfscope}%
\pgfsys@transformshift{2.050794in}{0.752869in}%
\pgfsys@useobject{currentmarker}{}%
\end{pgfscope}%
\begin{pgfscope}%
\pgfsys@transformshift{2.051146in}{0.725112in}%
\pgfsys@useobject{currentmarker}{}%
\end{pgfscope}%
\begin{pgfscope}%
\pgfsys@transformshift{2.051497in}{0.812152in}%
\pgfsys@useobject{currentmarker}{}%
\end{pgfscope}%
\begin{pgfscope}%
\pgfsys@transformshift{2.051848in}{0.825674in}%
\pgfsys@useobject{currentmarker}{}%
\end{pgfscope}%
\begin{pgfscope}%
\pgfsys@transformshift{2.052198in}{0.767659in}%
\pgfsys@useobject{currentmarker}{}%
\end{pgfscope}%
\begin{pgfscope}%
\pgfsys@transformshift{2.052548in}{0.826811in}%
\pgfsys@useobject{currentmarker}{}%
\end{pgfscope}%
\begin{pgfscope}%
\pgfsys@transformshift{2.052897in}{0.828595in}%
\pgfsys@useobject{currentmarker}{}%
\end{pgfscope}%
\begin{pgfscope}%
\pgfsys@transformshift{2.053245in}{0.791855in}%
\pgfsys@useobject{currentmarker}{}%
\end{pgfscope}%
\begin{pgfscope}%
\pgfsys@transformshift{2.053593in}{0.729114in}%
\pgfsys@useobject{currentmarker}{}%
\end{pgfscope}%
\begin{pgfscope}%
\pgfsys@transformshift{2.053941in}{0.723759in}%
\pgfsys@useobject{currentmarker}{}%
\end{pgfscope}%
\begin{pgfscope}%
\pgfsys@transformshift{2.054288in}{0.748747in}%
\pgfsys@useobject{currentmarker}{}%
\end{pgfscope}%
\begin{pgfscope}%
\pgfsys@transformshift{2.054634in}{0.762472in}%
\pgfsys@useobject{currentmarker}{}%
\end{pgfscope}%
\begin{pgfscope}%
\pgfsys@transformshift{2.054980in}{0.785257in}%
\pgfsys@useobject{currentmarker}{}%
\end{pgfscope}%
\begin{pgfscope}%
\pgfsys@transformshift{2.055326in}{0.795376in}%
\pgfsys@useobject{currentmarker}{}%
\end{pgfscope}%
\begin{pgfscope}%
\pgfsys@transformshift{2.055670in}{0.804972in}%
\pgfsys@useobject{currentmarker}{}%
\end{pgfscope}%
\begin{pgfscope}%
\pgfsys@transformshift{2.056015in}{0.789617in}%
\pgfsys@useobject{currentmarker}{}%
\end{pgfscope}%
\begin{pgfscope}%
\pgfsys@transformshift{2.056358in}{0.813792in}%
\pgfsys@useobject{currentmarker}{}%
\end{pgfscope}%
\begin{pgfscope}%
\pgfsys@transformshift{2.056702in}{0.788919in}%
\pgfsys@useobject{currentmarker}{}%
\end{pgfscope}%
\begin{pgfscope}%
\pgfsys@transformshift{2.057044in}{0.727923in}%
\pgfsys@useobject{currentmarker}{}%
\end{pgfscope}%
\begin{pgfscope}%
\pgfsys@transformshift{2.057387in}{0.737615in}%
\pgfsys@useobject{currentmarker}{}%
\end{pgfscope}%
\begin{pgfscope}%
\pgfsys@transformshift{2.057728in}{0.782174in}%
\pgfsys@useobject{currentmarker}{}%
\end{pgfscope}%
\begin{pgfscope}%
\pgfsys@transformshift{2.058069in}{0.781884in}%
\pgfsys@useobject{currentmarker}{}%
\end{pgfscope}%
\begin{pgfscope}%
\pgfsys@transformshift{2.058410in}{0.721037in}%
\pgfsys@useobject{currentmarker}{}%
\end{pgfscope}%
\begin{pgfscope}%
\pgfsys@transformshift{2.058750in}{0.765744in}%
\pgfsys@useobject{currentmarker}{}%
\end{pgfscope}%
\begin{pgfscope}%
\pgfsys@transformshift{2.059090in}{0.808250in}%
\pgfsys@useobject{currentmarker}{}%
\end{pgfscope}%
\begin{pgfscope}%
\pgfsys@transformshift{2.059429in}{0.822864in}%
\pgfsys@useobject{currentmarker}{}%
\end{pgfscope}%
\begin{pgfscope}%
\pgfsys@transformshift{2.059767in}{0.763653in}%
\pgfsys@useobject{currentmarker}{}%
\end{pgfscope}%
\begin{pgfscope}%
\pgfsys@transformshift{2.060106in}{0.761226in}%
\pgfsys@useobject{currentmarker}{}%
\end{pgfscope}%
\begin{pgfscope}%
\pgfsys@transformshift{2.060443in}{0.788990in}%
\pgfsys@useobject{currentmarker}{}%
\end{pgfscope}%
\begin{pgfscope}%
\pgfsys@transformshift{2.060780in}{0.773484in}%
\pgfsys@useobject{currentmarker}{}%
\end{pgfscope}%
\begin{pgfscope}%
\pgfsys@transformshift{2.061117in}{0.790140in}%
\pgfsys@useobject{currentmarker}{}%
\end{pgfscope}%
\begin{pgfscope}%
\pgfsys@transformshift{2.061453in}{0.820679in}%
\pgfsys@useobject{currentmarker}{}%
\end{pgfscope}%
\begin{pgfscope}%
\pgfsys@transformshift{2.061788in}{0.758347in}%
\pgfsys@useobject{currentmarker}{}%
\end{pgfscope}%
\begin{pgfscope}%
\pgfsys@transformshift{2.062123in}{0.758432in}%
\pgfsys@useobject{currentmarker}{}%
\end{pgfscope}%
\begin{pgfscope}%
\pgfsys@transformshift{2.062458in}{0.787997in}%
\pgfsys@useobject{currentmarker}{}%
\end{pgfscope}%
\begin{pgfscope}%
\pgfsys@transformshift{2.062792in}{0.721743in}%
\pgfsys@useobject{currentmarker}{}%
\end{pgfscope}%
\begin{pgfscope}%
\pgfsys@transformshift{2.063126in}{0.800623in}%
\pgfsys@useobject{currentmarker}{}%
\end{pgfscope}%
\begin{pgfscope}%
\pgfsys@transformshift{2.063459in}{0.781085in}%
\pgfsys@useobject{currentmarker}{}%
\end{pgfscope}%
\begin{pgfscope}%
\pgfsys@transformshift{2.063791in}{0.756461in}%
\pgfsys@useobject{currentmarker}{}%
\end{pgfscope}%
\begin{pgfscope}%
\pgfsys@transformshift{2.064123in}{0.715672in}%
\pgfsys@useobject{currentmarker}{}%
\end{pgfscope}%
\begin{pgfscope}%
\pgfsys@transformshift{2.064455in}{0.772093in}%
\pgfsys@useobject{currentmarker}{}%
\end{pgfscope}%
\begin{pgfscope}%
\pgfsys@transformshift{2.064786in}{0.787890in}%
\pgfsys@useobject{currentmarker}{}%
\end{pgfscope}%
\begin{pgfscope}%
\pgfsys@transformshift{2.065117in}{0.787823in}%
\pgfsys@useobject{currentmarker}{}%
\end{pgfscope}%
\begin{pgfscope}%
\pgfsys@transformshift{2.065447in}{0.776960in}%
\pgfsys@useobject{currentmarker}{}%
\end{pgfscope}%
\begin{pgfscope}%
\pgfsys@transformshift{2.065776in}{0.764442in}%
\pgfsys@useobject{currentmarker}{}%
\end{pgfscope}%
\begin{pgfscope}%
\pgfsys@transformshift{2.066106in}{0.746445in}%
\pgfsys@useobject{currentmarker}{}%
\end{pgfscope}%
\begin{pgfscope}%
\pgfsys@transformshift{2.066434in}{0.785003in}%
\pgfsys@useobject{currentmarker}{}%
\end{pgfscope}%
\begin{pgfscope}%
\pgfsys@transformshift{2.066762in}{0.785275in}%
\pgfsys@useobject{currentmarker}{}%
\end{pgfscope}%
\begin{pgfscope}%
\pgfsys@transformshift{2.067090in}{0.766141in}%
\pgfsys@useobject{currentmarker}{}%
\end{pgfscope}%
\begin{pgfscope}%
\pgfsys@transformshift{2.067417in}{0.737749in}%
\pgfsys@useobject{currentmarker}{}%
\end{pgfscope}%
\begin{pgfscope}%
\pgfsys@transformshift{2.067744in}{0.729141in}%
\pgfsys@useobject{currentmarker}{}%
\end{pgfscope}%
\begin{pgfscope}%
\pgfsys@transformshift{2.068070in}{0.738890in}%
\pgfsys@useobject{currentmarker}{}%
\end{pgfscope}%
\begin{pgfscope}%
\pgfsys@transformshift{2.068396in}{0.748667in}%
\pgfsys@useobject{currentmarker}{}%
\end{pgfscope}%
\begin{pgfscope}%
\pgfsys@transformshift{2.068722in}{0.756254in}%
\pgfsys@useobject{currentmarker}{}%
\end{pgfscope}%
\begin{pgfscope}%
\pgfsys@transformshift{2.069046in}{0.778096in}%
\pgfsys@useobject{currentmarker}{}%
\end{pgfscope}%
\begin{pgfscope}%
\pgfsys@transformshift{2.069371in}{0.758567in}%
\pgfsys@useobject{currentmarker}{}%
\end{pgfscope}%
\begin{pgfscope}%
\pgfsys@transformshift{2.069695in}{0.745304in}%
\pgfsys@useobject{currentmarker}{}%
\end{pgfscope}%
\begin{pgfscope}%
\pgfsys@transformshift{2.070018in}{0.727923in}%
\pgfsys@useobject{currentmarker}{}%
\end{pgfscope}%
\begin{pgfscope}%
\pgfsys@transformshift{2.070341in}{0.760419in}%
\pgfsys@useobject{currentmarker}{}%
\end{pgfscope}%
\begin{pgfscope}%
\pgfsys@transformshift{2.070664in}{0.758535in}%
\pgfsys@useobject{currentmarker}{}%
\end{pgfscope}%
\begin{pgfscope}%
\pgfsys@transformshift{2.070986in}{0.752122in}%
\pgfsys@useobject{currentmarker}{}%
\end{pgfscope}%
\begin{pgfscope}%
\pgfsys@transformshift{2.071308in}{0.781840in}%
\pgfsys@useobject{currentmarker}{}%
\end{pgfscope}%
\begin{pgfscope}%
\pgfsys@transformshift{2.071629in}{0.800878in}%
\pgfsys@useobject{currentmarker}{}%
\end{pgfscope}%
\begin{pgfscope}%
\pgfsys@transformshift{2.071949in}{0.790472in}%
\pgfsys@useobject{currentmarker}{}%
\end{pgfscope}%
\begin{pgfscope}%
\pgfsys@transformshift{2.072270in}{0.747592in}%
\pgfsys@useobject{currentmarker}{}%
\end{pgfscope}%
\begin{pgfscope}%
\pgfsys@transformshift{2.072589in}{0.785789in}%
\pgfsys@useobject{currentmarker}{}%
\end{pgfscope}%
\begin{pgfscope}%
\pgfsys@transformshift{2.072909in}{0.787160in}%
\pgfsys@useobject{currentmarker}{}%
\end{pgfscope}%
\begin{pgfscope}%
\pgfsys@transformshift{2.073228in}{0.753818in}%
\pgfsys@useobject{currentmarker}{}%
\end{pgfscope}%
\begin{pgfscope}%
\pgfsys@transformshift{2.073546in}{0.765796in}%
\pgfsys@useobject{currentmarker}{}%
\end{pgfscope}%
\begin{pgfscope}%
\pgfsys@transformshift{2.073864in}{0.725444in}%
\pgfsys@useobject{currentmarker}{}%
\end{pgfscope}%
\begin{pgfscope}%
\pgfsys@transformshift{2.074182in}{0.767389in}%
\pgfsys@useobject{currentmarker}{}%
\end{pgfscope}%
\begin{pgfscope}%
\pgfsys@transformshift{2.074499in}{0.757386in}%
\pgfsys@useobject{currentmarker}{}%
\end{pgfscope}%
\begin{pgfscope}%
\pgfsys@transformshift{2.074815in}{0.739924in}%
\pgfsys@useobject{currentmarker}{}%
\end{pgfscope}%
\begin{pgfscope}%
\pgfsys@transformshift{2.075131in}{0.712449in}%
\pgfsys@useobject{currentmarker}{}%
\end{pgfscope}%
\begin{pgfscope}%
\pgfsys@transformshift{2.075447in}{0.771152in}%
\pgfsys@useobject{currentmarker}{}%
\end{pgfscope}%
\begin{pgfscope}%
\pgfsys@transformshift{2.075763in}{0.791699in}%
\pgfsys@useobject{currentmarker}{}%
\end{pgfscope}%
\begin{pgfscope}%
\pgfsys@transformshift{2.076077in}{0.778587in}%
\pgfsys@useobject{currentmarker}{}%
\end{pgfscope}%
\begin{pgfscope}%
\pgfsys@transformshift{2.076392in}{0.731044in}%
\pgfsys@useobject{currentmarker}{}%
\end{pgfscope}%
\begin{pgfscope}%
\pgfsys@transformshift{2.076706in}{0.719593in}%
\pgfsys@useobject{currentmarker}{}%
\end{pgfscope}%
\begin{pgfscope}%
\pgfsys@transformshift{2.077019in}{0.815000in}%
\pgfsys@useobject{currentmarker}{}%
\end{pgfscope}%
\begin{pgfscope}%
\pgfsys@transformshift{2.077332in}{0.810825in}%
\pgfsys@useobject{currentmarker}{}%
\end{pgfscope}%
\begin{pgfscope}%
\pgfsys@transformshift{2.077645in}{0.777372in}%
\pgfsys@useobject{currentmarker}{}%
\end{pgfscope}%
\begin{pgfscope}%
\pgfsys@transformshift{2.077957in}{0.761272in}%
\pgfsys@useobject{currentmarker}{}%
\end{pgfscope}%
\begin{pgfscope}%
\pgfsys@transformshift{2.078269in}{0.766970in}%
\pgfsys@useobject{currentmarker}{}%
\end{pgfscope}%
\begin{pgfscope}%
\pgfsys@transformshift{2.078581in}{0.785769in}%
\pgfsys@useobject{currentmarker}{}%
\end{pgfscope}%
\begin{pgfscope}%
\pgfsys@transformshift{2.078891in}{0.740198in}%
\pgfsys@useobject{currentmarker}{}%
\end{pgfscope}%
\begin{pgfscope}%
\pgfsys@transformshift{2.079202in}{0.800970in}%
\pgfsys@useobject{currentmarker}{}%
\end{pgfscope}%
\begin{pgfscope}%
\pgfsys@transformshift{2.079512in}{0.793799in}%
\pgfsys@useobject{currentmarker}{}%
\end{pgfscope}%
\begin{pgfscope}%
\pgfsys@transformshift{2.079822in}{0.807703in}%
\pgfsys@useobject{currentmarker}{}%
\end{pgfscope}%
\begin{pgfscope}%
\pgfsys@transformshift{2.080131in}{0.802201in}%
\pgfsys@useobject{currentmarker}{}%
\end{pgfscope}%
\begin{pgfscope}%
\pgfsys@transformshift{2.080440in}{0.746714in}%
\pgfsys@useobject{currentmarker}{}%
\end{pgfscope}%
\begin{pgfscope}%
\pgfsys@transformshift{2.080748in}{0.735326in}%
\pgfsys@useobject{currentmarker}{}%
\end{pgfscope}%
\begin{pgfscope}%
\pgfsys@transformshift{2.081056in}{0.751767in}%
\pgfsys@useobject{currentmarker}{}%
\end{pgfscope}%
\begin{pgfscope}%
\pgfsys@transformshift{2.081364in}{0.751211in}%
\pgfsys@useobject{currentmarker}{}%
\end{pgfscope}%
\begin{pgfscope}%
\pgfsys@transformshift{2.081671in}{0.733680in}%
\pgfsys@useobject{currentmarker}{}%
\end{pgfscope}%
\begin{pgfscope}%
\pgfsys@transformshift{2.081977in}{0.776505in}%
\pgfsys@useobject{currentmarker}{}%
\end{pgfscope}%
\begin{pgfscope}%
\pgfsys@transformshift{2.082284in}{0.786109in}%
\pgfsys@useobject{currentmarker}{}%
\end{pgfscope}%
\begin{pgfscope}%
\pgfsys@transformshift{2.082589in}{0.784557in}%
\pgfsys@useobject{currentmarker}{}%
\end{pgfscope}%
\begin{pgfscope}%
\pgfsys@transformshift{2.082895in}{0.781083in}%
\pgfsys@useobject{currentmarker}{}%
\end{pgfscope}%
\begin{pgfscope}%
\pgfsys@transformshift{2.083200in}{0.724494in}%
\pgfsys@useobject{currentmarker}{}%
\end{pgfscope}%
\begin{pgfscope}%
\pgfsys@transformshift{2.083505in}{0.754370in}%
\pgfsys@useobject{currentmarker}{}%
\end{pgfscope}%
\begin{pgfscope}%
\pgfsys@transformshift{2.083809in}{0.767576in}%
\pgfsys@useobject{currentmarker}{}%
\end{pgfscope}%
\begin{pgfscope}%
\pgfsys@transformshift{2.084113in}{0.722638in}%
\pgfsys@useobject{currentmarker}{}%
\end{pgfscope}%
\begin{pgfscope}%
\pgfsys@transformshift{2.084416in}{0.665345in}%
\pgfsys@useobject{currentmarker}{}%
\end{pgfscope}%
\begin{pgfscope}%
\pgfsys@transformshift{2.084719in}{0.773847in}%
\pgfsys@useobject{currentmarker}{}%
\end{pgfscope}%
\begin{pgfscope}%
\pgfsys@transformshift{2.085021in}{0.791958in}%
\pgfsys@useobject{currentmarker}{}%
\end{pgfscope}%
\begin{pgfscope}%
\pgfsys@transformshift{2.085324in}{0.770189in}%
\pgfsys@useobject{currentmarker}{}%
\end{pgfscope}%
\begin{pgfscope}%
\pgfsys@transformshift{2.085625in}{0.756130in}%
\pgfsys@useobject{currentmarker}{}%
\end{pgfscope}%
\begin{pgfscope}%
\pgfsys@transformshift{2.085927in}{0.759117in}%
\pgfsys@useobject{currentmarker}{}%
\end{pgfscope}%
\begin{pgfscope}%
\pgfsys@transformshift{2.086228in}{0.711966in}%
\pgfsys@useobject{currentmarker}{}%
\end{pgfscope}%
\begin{pgfscope}%
\pgfsys@transformshift{2.086528in}{0.743672in}%
\pgfsys@useobject{currentmarker}{}%
\end{pgfscope}%
\begin{pgfscope}%
\pgfsys@transformshift{2.086828in}{0.796596in}%
\pgfsys@useobject{currentmarker}{}%
\end{pgfscope}%
\begin{pgfscope}%
\pgfsys@transformshift{2.087128in}{0.758227in}%
\pgfsys@useobject{currentmarker}{}%
\end{pgfscope}%
\begin{pgfscope}%
\pgfsys@transformshift{2.087427in}{0.765954in}%
\pgfsys@useobject{currentmarker}{}%
\end{pgfscope}%
\begin{pgfscope}%
\pgfsys@transformshift{2.087726in}{0.793542in}%
\pgfsys@useobject{currentmarker}{}%
\end{pgfscope}%
\begin{pgfscope}%
\pgfsys@transformshift{2.088025in}{0.778983in}%
\pgfsys@useobject{currentmarker}{}%
\end{pgfscope}%
\begin{pgfscope}%
\pgfsys@transformshift{2.088323in}{0.776103in}%
\pgfsys@useobject{currentmarker}{}%
\end{pgfscope}%
\begin{pgfscope}%
\pgfsys@transformshift{2.088621in}{0.768464in}%
\pgfsys@useobject{currentmarker}{}%
\end{pgfscope}%
\begin{pgfscope}%
\pgfsys@transformshift{2.088918in}{0.717334in}%
\pgfsys@useobject{currentmarker}{}%
\end{pgfscope}%
\begin{pgfscope}%
\pgfsys@transformshift{2.089215in}{0.752356in}%
\pgfsys@useobject{currentmarker}{}%
\end{pgfscope}%
\begin{pgfscope}%
\pgfsys@transformshift{2.089512in}{0.752851in}%
\pgfsys@useobject{currentmarker}{}%
\end{pgfscope}%
\begin{pgfscope}%
\pgfsys@transformshift{2.089808in}{0.736706in}%
\pgfsys@useobject{currentmarker}{}%
\end{pgfscope}%
\begin{pgfscope}%
\pgfsys@transformshift{2.090104in}{0.786102in}%
\pgfsys@useobject{currentmarker}{}%
\end{pgfscope}%
\begin{pgfscope}%
\pgfsys@transformshift{2.090399in}{0.794158in}%
\pgfsys@useobject{currentmarker}{}%
\end{pgfscope}%
\begin{pgfscope}%
\pgfsys@transformshift{2.090694in}{0.705120in}%
\pgfsys@useobject{currentmarker}{}%
\end{pgfscope}%
\begin{pgfscope}%
\pgfsys@transformshift{2.090989in}{0.751445in}%
\pgfsys@useobject{currentmarker}{}%
\end{pgfscope}%
\begin{pgfscope}%
\pgfsys@transformshift{2.091283in}{0.710301in}%
\pgfsys@useobject{currentmarker}{}%
\end{pgfscope}%
\begin{pgfscope}%
\pgfsys@transformshift{2.091577in}{0.766715in}%
\pgfsys@useobject{currentmarker}{}%
\end{pgfscope}%
\begin{pgfscope}%
\pgfsys@transformshift{2.091870in}{0.763879in}%
\pgfsys@useobject{currentmarker}{}%
\end{pgfscope}%
\begin{pgfscope}%
\pgfsys@transformshift{2.092163in}{0.745374in}%
\pgfsys@useobject{currentmarker}{}%
\end{pgfscope}%
\begin{pgfscope}%
\pgfsys@transformshift{2.092456in}{0.768062in}%
\pgfsys@useobject{currentmarker}{}%
\end{pgfscope}%
\begin{pgfscope}%
\pgfsys@transformshift{2.092748in}{0.764653in}%
\pgfsys@useobject{currentmarker}{}%
\end{pgfscope}%
\begin{pgfscope}%
\pgfsys@transformshift{2.093040in}{0.711543in}%
\pgfsys@useobject{currentmarker}{}%
\end{pgfscope}%
\begin{pgfscope}%
\pgfsys@transformshift{2.093332in}{0.672255in}%
\pgfsys@useobject{currentmarker}{}%
\end{pgfscope}%
\begin{pgfscope}%
\pgfsys@transformshift{2.093623in}{0.748229in}%
\pgfsys@useobject{currentmarker}{}%
\end{pgfscope}%
\begin{pgfscope}%
\pgfsys@transformshift{2.093914in}{0.771720in}%
\pgfsys@useobject{currentmarker}{}%
\end{pgfscope}%
\begin{pgfscope}%
\pgfsys@transformshift{2.094204in}{0.776900in}%
\pgfsys@useobject{currentmarker}{}%
\end{pgfscope}%
\begin{pgfscope}%
\pgfsys@transformshift{2.094494in}{0.778248in}%
\pgfsys@useobject{currentmarker}{}%
\end{pgfscope}%
\begin{pgfscope}%
\pgfsys@transformshift{2.094784in}{0.752275in}%
\pgfsys@useobject{currentmarker}{}%
\end{pgfscope}%
\begin{pgfscope}%
\pgfsys@transformshift{2.095073in}{0.675818in}%
\pgfsys@useobject{currentmarker}{}%
\end{pgfscope}%
\begin{pgfscope}%
\pgfsys@transformshift{2.095362in}{0.728916in}%
\pgfsys@useobject{currentmarker}{}%
\end{pgfscope}%
\begin{pgfscope}%
\pgfsys@transformshift{2.095651in}{0.756213in}%
\pgfsys@useobject{currentmarker}{}%
\end{pgfscope}%
\begin{pgfscope}%
\pgfsys@transformshift{2.095939in}{0.780391in}%
\pgfsys@useobject{currentmarker}{}%
\end{pgfscope}%
\begin{pgfscope}%
\pgfsys@transformshift{2.096227in}{0.779043in}%
\pgfsys@useobject{currentmarker}{}%
\end{pgfscope}%
\begin{pgfscope}%
\pgfsys@transformshift{2.096514in}{0.762338in}%
\pgfsys@useobject{currentmarker}{}%
\end{pgfscope}%
\begin{pgfscope}%
\pgfsys@transformshift{2.096801in}{0.774066in}%
\pgfsys@useobject{currentmarker}{}%
\end{pgfscope}%
\begin{pgfscope}%
\pgfsys@transformshift{2.097088in}{0.786978in}%
\pgfsys@useobject{currentmarker}{}%
\end{pgfscope}%
\begin{pgfscope}%
\pgfsys@transformshift{2.097375in}{0.810369in}%
\pgfsys@useobject{currentmarker}{}%
\end{pgfscope}%
\begin{pgfscope}%
\pgfsys@transformshift{2.097661in}{0.802044in}%
\pgfsys@useobject{currentmarker}{}%
\end{pgfscope}%
\begin{pgfscope}%
\pgfsys@transformshift{2.097946in}{0.801201in}%
\pgfsys@useobject{currentmarker}{}%
\end{pgfscope}%
\begin{pgfscope}%
\pgfsys@transformshift{2.098231in}{0.766326in}%
\pgfsys@useobject{currentmarker}{}%
\end{pgfscope}%
\begin{pgfscope}%
\pgfsys@transformshift{2.098516in}{0.744037in}%
\pgfsys@useobject{currentmarker}{}%
\end{pgfscope}%
\begin{pgfscope}%
\pgfsys@transformshift{2.098801in}{0.762782in}%
\pgfsys@useobject{currentmarker}{}%
\end{pgfscope}%
\begin{pgfscope}%
\pgfsys@transformshift{2.099085in}{0.792498in}%
\pgfsys@useobject{currentmarker}{}%
\end{pgfscope}%
\begin{pgfscope}%
\pgfsys@transformshift{2.099369in}{0.805887in}%
\pgfsys@useobject{currentmarker}{}%
\end{pgfscope}%
\begin{pgfscope}%
\pgfsys@transformshift{2.099652in}{0.759720in}%
\pgfsys@useobject{currentmarker}{}%
\end{pgfscope}%
\begin{pgfscope}%
\pgfsys@transformshift{2.099936in}{0.744603in}%
\pgfsys@useobject{currentmarker}{}%
\end{pgfscope}%
\begin{pgfscope}%
\pgfsys@transformshift{2.100218in}{0.765481in}%
\pgfsys@useobject{currentmarker}{}%
\end{pgfscope}%
\begin{pgfscope}%
\pgfsys@transformshift{2.100501in}{0.773291in}%
\pgfsys@useobject{currentmarker}{}%
\end{pgfscope}%
\begin{pgfscope}%
\pgfsys@transformshift{2.100783in}{0.755713in}%
\pgfsys@useobject{currentmarker}{}%
\end{pgfscope}%
\begin{pgfscope}%
\pgfsys@transformshift{2.101064in}{0.785410in}%
\pgfsys@useobject{currentmarker}{}%
\end{pgfscope}%
\begin{pgfscope}%
\pgfsys@transformshift{2.101346in}{0.763174in}%
\pgfsys@useobject{currentmarker}{}%
\end{pgfscope}%
\begin{pgfscope}%
\pgfsys@transformshift{2.101627in}{0.737039in}%
\pgfsys@useobject{currentmarker}{}%
\end{pgfscope}%
\begin{pgfscope}%
\pgfsys@transformshift{2.101907in}{0.767780in}%
\pgfsys@useobject{currentmarker}{}%
\end{pgfscope}%
\begin{pgfscope}%
\pgfsys@transformshift{2.102188in}{0.759880in}%
\pgfsys@useobject{currentmarker}{}%
\end{pgfscope}%
\begin{pgfscope}%
\pgfsys@transformshift{2.102468in}{0.725297in}%
\pgfsys@useobject{currentmarker}{}%
\end{pgfscope}%
\begin{pgfscope}%
\pgfsys@transformshift{2.102747in}{0.717293in}%
\pgfsys@useobject{currentmarker}{}%
\end{pgfscope}%
\begin{pgfscope}%
\pgfsys@transformshift{2.103026in}{0.806572in}%
\pgfsys@useobject{currentmarker}{}%
\end{pgfscope}%
\begin{pgfscope}%
\pgfsys@transformshift{2.103305in}{0.795638in}%
\pgfsys@useobject{currentmarker}{}%
\end{pgfscope}%
\begin{pgfscope}%
\pgfsys@transformshift{2.103584in}{0.770456in}%
\pgfsys@useobject{currentmarker}{}%
\end{pgfscope}%
\begin{pgfscope}%
\pgfsys@transformshift{2.103862in}{0.771636in}%
\pgfsys@useobject{currentmarker}{}%
\end{pgfscope}%
\begin{pgfscope}%
\pgfsys@transformshift{2.104140in}{0.765007in}%
\pgfsys@useobject{currentmarker}{}%
\end{pgfscope}%
\begin{pgfscope}%
\pgfsys@transformshift{2.104417in}{0.674943in}%
\pgfsys@useobject{currentmarker}{}%
\end{pgfscope}%
\begin{pgfscope}%
\pgfsys@transformshift{2.104695in}{0.738306in}%
\pgfsys@useobject{currentmarker}{}%
\end{pgfscope}%
\begin{pgfscope}%
\pgfsys@transformshift{2.104972in}{0.763139in}%
\pgfsys@useobject{currentmarker}{}%
\end{pgfscope}%
\begin{pgfscope}%
\pgfsys@transformshift{2.105248in}{0.717427in}%
\pgfsys@useobject{currentmarker}{}%
\end{pgfscope}%
\begin{pgfscope}%
\pgfsys@transformshift{2.105524in}{0.707985in}%
\pgfsys@useobject{currentmarker}{}%
\end{pgfscope}%
\begin{pgfscope}%
\pgfsys@transformshift{2.105800in}{0.738873in}%
\pgfsys@useobject{currentmarker}{}%
\end{pgfscope}%
\begin{pgfscope}%
\pgfsys@transformshift{2.106075in}{0.771415in}%
\pgfsys@useobject{currentmarker}{}%
\end{pgfscope}%
\begin{pgfscope}%
\pgfsys@transformshift{2.106351in}{0.750304in}%
\pgfsys@useobject{currentmarker}{}%
\end{pgfscope}%
\begin{pgfscope}%
\pgfsys@transformshift{2.106625in}{0.761546in}%
\pgfsys@useobject{currentmarker}{}%
\end{pgfscope}%
\begin{pgfscope}%
\pgfsys@transformshift{2.106900in}{0.758811in}%
\pgfsys@useobject{currentmarker}{}%
\end{pgfscope}%
\begin{pgfscope}%
\pgfsys@transformshift{2.107174in}{0.747350in}%
\pgfsys@useobject{currentmarker}{}%
\end{pgfscope}%
\begin{pgfscope}%
\pgfsys@transformshift{2.107448in}{0.756182in}%
\pgfsys@useobject{currentmarker}{}%
\end{pgfscope}%
\begin{pgfscope}%
\pgfsys@transformshift{2.107721in}{0.733143in}%
\pgfsys@useobject{currentmarker}{}%
\end{pgfscope}%
\begin{pgfscope}%
\pgfsys@transformshift{2.107994in}{0.733651in}%
\pgfsys@useobject{currentmarker}{}%
\end{pgfscope}%
\begin{pgfscope}%
\pgfsys@transformshift{2.108267in}{0.744592in}%
\pgfsys@useobject{currentmarker}{}%
\end{pgfscope}%
\begin{pgfscope}%
\pgfsys@transformshift{2.108540in}{0.757944in}%
\pgfsys@useobject{currentmarker}{}%
\end{pgfscope}%
\begin{pgfscope}%
\pgfsys@transformshift{2.108812in}{0.721864in}%
\pgfsys@useobject{currentmarker}{}%
\end{pgfscope}%
\begin{pgfscope}%
\pgfsys@transformshift{2.109084in}{0.742827in}%
\pgfsys@useobject{currentmarker}{}%
\end{pgfscope}%
\begin{pgfscope}%
\pgfsys@transformshift{2.109355in}{0.796482in}%
\pgfsys@useobject{currentmarker}{}%
\end{pgfscope}%
\begin{pgfscope}%
\pgfsys@transformshift{2.109626in}{0.791489in}%
\pgfsys@useobject{currentmarker}{}%
\end{pgfscope}%
\begin{pgfscope}%
\pgfsys@transformshift{2.109897in}{0.743602in}%
\pgfsys@useobject{currentmarker}{}%
\end{pgfscope}%
\begin{pgfscope}%
\pgfsys@transformshift{2.110168in}{0.728055in}%
\pgfsys@useobject{currentmarker}{}%
\end{pgfscope}%
\begin{pgfscope}%
\pgfsys@transformshift{2.110438in}{0.750642in}%
\pgfsys@useobject{currentmarker}{}%
\end{pgfscope}%
\begin{pgfscope}%
\pgfsys@transformshift{2.110708in}{0.718617in}%
\pgfsys@useobject{currentmarker}{}%
\end{pgfscope}%
\begin{pgfscope}%
\pgfsys@transformshift{2.110977in}{0.668139in}%
\pgfsys@useobject{currentmarker}{}%
\end{pgfscope}%
\begin{pgfscope}%
\pgfsys@transformshift{2.111246in}{0.724750in}%
\pgfsys@useobject{currentmarker}{}%
\end{pgfscope}%
\begin{pgfscope}%
\pgfsys@transformshift{2.111515in}{0.775464in}%
\pgfsys@useobject{currentmarker}{}%
\end{pgfscope}%
\begin{pgfscope}%
\pgfsys@transformshift{2.111784in}{0.765697in}%
\pgfsys@useobject{currentmarker}{}%
\end{pgfscope}%
\begin{pgfscope}%
\pgfsys@transformshift{2.112052in}{0.766776in}%
\pgfsys@useobject{currentmarker}{}%
\end{pgfscope}%
\begin{pgfscope}%
\pgfsys@transformshift{2.112320in}{0.763934in}%
\pgfsys@useobject{currentmarker}{}%
\end{pgfscope}%
\begin{pgfscope}%
\pgfsys@transformshift{2.112588in}{0.745045in}%
\pgfsys@useobject{currentmarker}{}%
\end{pgfscope}%
\begin{pgfscope}%
\pgfsys@transformshift{2.112855in}{0.770596in}%
\pgfsys@useobject{currentmarker}{}%
\end{pgfscope}%
\begin{pgfscope}%
\pgfsys@transformshift{2.113122in}{0.737559in}%
\pgfsys@useobject{currentmarker}{}%
\end{pgfscope}%
\begin{pgfscope}%
\pgfsys@transformshift{2.113388in}{0.756427in}%
\pgfsys@useobject{currentmarker}{}%
\end{pgfscope}%
\begin{pgfscope}%
\pgfsys@transformshift{2.113655in}{0.745014in}%
\pgfsys@useobject{currentmarker}{}%
\end{pgfscope}%
\begin{pgfscope}%
\pgfsys@transformshift{2.113921in}{0.769913in}%
\pgfsys@useobject{currentmarker}{}%
\end{pgfscope}%
\begin{pgfscope}%
\pgfsys@transformshift{2.114187in}{0.808915in}%
\pgfsys@useobject{currentmarker}{}%
\end{pgfscope}%
\begin{pgfscope}%
\pgfsys@transformshift{2.114452in}{0.784133in}%
\pgfsys@useobject{currentmarker}{}%
\end{pgfscope}%
\begin{pgfscope}%
\pgfsys@transformshift{2.114717in}{0.722321in}%
\pgfsys@useobject{currentmarker}{}%
\end{pgfscope}%
\begin{pgfscope}%
\pgfsys@transformshift{2.114982in}{0.695509in}%
\pgfsys@useobject{currentmarker}{}%
\end{pgfscope}%
\begin{pgfscope}%
\pgfsys@transformshift{2.115246in}{0.722933in}%
\pgfsys@useobject{currentmarker}{}%
\end{pgfscope}%
\begin{pgfscope}%
\pgfsys@transformshift{2.115510in}{0.771400in}%
\pgfsys@useobject{currentmarker}{}%
\end{pgfscope}%
\begin{pgfscope}%
\pgfsys@transformshift{2.115774in}{0.745901in}%
\pgfsys@useobject{currentmarker}{}%
\end{pgfscope}%
\begin{pgfscope}%
\pgfsys@transformshift{2.116038in}{0.675524in}%
\pgfsys@useobject{currentmarker}{}%
\end{pgfscope}%
\begin{pgfscope}%
\pgfsys@transformshift{2.116301in}{0.690438in}%
\pgfsys@useobject{currentmarker}{}%
\end{pgfscope}%
\begin{pgfscope}%
\pgfsys@transformshift{2.116564in}{0.694552in}%
\pgfsys@useobject{currentmarker}{}%
\end{pgfscope}%
\begin{pgfscope}%
\pgfsys@transformshift{2.116826in}{0.735613in}%
\pgfsys@useobject{currentmarker}{}%
\end{pgfscope}%
\begin{pgfscope}%
\pgfsys@transformshift{2.117089in}{0.745796in}%
\pgfsys@useobject{currentmarker}{}%
\end{pgfscope}%
\begin{pgfscope}%
\pgfsys@transformshift{2.117351in}{0.757476in}%
\pgfsys@useobject{currentmarker}{}%
\end{pgfscope}%
\begin{pgfscope}%
\pgfsys@transformshift{2.117612in}{0.775756in}%
\pgfsys@useobject{currentmarker}{}%
\end{pgfscope}%
\begin{pgfscope}%
\pgfsys@transformshift{2.117874in}{0.744567in}%
\pgfsys@useobject{currentmarker}{}%
\end{pgfscope}%
\begin{pgfscope}%
\pgfsys@transformshift{2.118135in}{0.734748in}%
\pgfsys@useobject{currentmarker}{}%
\end{pgfscope}%
\begin{pgfscope}%
\pgfsys@transformshift{2.118396in}{0.725906in}%
\pgfsys@useobject{currentmarker}{}%
\end{pgfscope}%
\begin{pgfscope}%
\pgfsys@transformshift{2.118656in}{0.732736in}%
\pgfsys@useobject{currentmarker}{}%
\end{pgfscope}%
\begin{pgfscope}%
\pgfsys@transformshift{2.118916in}{0.719722in}%
\pgfsys@useobject{currentmarker}{}%
\end{pgfscope}%
\begin{pgfscope}%
\pgfsys@transformshift{2.119176in}{0.706597in}%
\pgfsys@useobject{currentmarker}{}%
\end{pgfscope}%
\begin{pgfscope}%
\pgfsys@transformshift{2.119436in}{0.717115in}%
\pgfsys@useobject{currentmarker}{}%
\end{pgfscope}%
\begin{pgfscope}%
\pgfsys@transformshift{2.119695in}{0.710174in}%
\pgfsys@useobject{currentmarker}{}%
\end{pgfscope}%
\begin{pgfscope}%
\pgfsys@transformshift{2.119954in}{0.701083in}%
\pgfsys@useobject{currentmarker}{}%
\end{pgfscope}%
\begin{pgfscope}%
\pgfsys@transformshift{2.120213in}{0.738727in}%
\pgfsys@useobject{currentmarker}{}%
\end{pgfscope}%
\begin{pgfscope}%
\pgfsys@transformshift{2.120471in}{0.754360in}%
\pgfsys@useobject{currentmarker}{}%
\end{pgfscope}%
\begin{pgfscope}%
\pgfsys@transformshift{2.120729in}{0.754275in}%
\pgfsys@useobject{currentmarker}{}%
\end{pgfscope}%
\begin{pgfscope}%
\pgfsys@transformshift{2.120987in}{0.733179in}%
\pgfsys@useobject{currentmarker}{}%
\end{pgfscope}%
\begin{pgfscope}%
\pgfsys@transformshift{2.121244in}{0.755059in}%
\pgfsys@useobject{currentmarker}{}%
\end{pgfscope}%
\begin{pgfscope}%
\pgfsys@transformshift{2.121501in}{0.791621in}%
\pgfsys@useobject{currentmarker}{}%
\end{pgfscope}%
\begin{pgfscope}%
\pgfsys@transformshift{2.121758in}{0.787327in}%
\pgfsys@useobject{currentmarker}{}%
\end{pgfscope}%
\begin{pgfscope}%
\pgfsys@transformshift{2.122015in}{0.743552in}%
\pgfsys@useobject{currentmarker}{}%
\end{pgfscope}%
\begin{pgfscope}%
\pgfsys@transformshift{2.122271in}{0.738259in}%
\pgfsys@useobject{currentmarker}{}%
\end{pgfscope}%
\begin{pgfscope}%
\pgfsys@transformshift{2.122527in}{0.782943in}%
\pgfsys@useobject{currentmarker}{}%
\end{pgfscope}%
\begin{pgfscope}%
\pgfsys@transformshift{2.122783in}{0.781467in}%
\pgfsys@useobject{currentmarker}{}%
\end{pgfscope}%
\begin{pgfscope}%
\pgfsys@transformshift{2.123038in}{0.771210in}%
\pgfsys@useobject{currentmarker}{}%
\end{pgfscope}%
\begin{pgfscope}%
\pgfsys@transformshift{2.123293in}{0.709657in}%
\pgfsys@useobject{currentmarker}{}%
\end{pgfscope}%
\begin{pgfscope}%
\pgfsys@transformshift{2.123548in}{0.739582in}%
\pgfsys@useobject{currentmarker}{}%
\end{pgfscope}%
\begin{pgfscope}%
\pgfsys@transformshift{2.123803in}{0.757784in}%
\pgfsys@useobject{currentmarker}{}%
\end{pgfscope}%
\begin{pgfscope}%
\pgfsys@transformshift{2.124057in}{0.748578in}%
\pgfsys@useobject{currentmarker}{}%
\end{pgfscope}%
\begin{pgfscope}%
\pgfsys@transformshift{2.124311in}{0.722612in}%
\pgfsys@useobject{currentmarker}{}%
\end{pgfscope}%
\begin{pgfscope}%
\pgfsys@transformshift{2.124565in}{0.731054in}%
\pgfsys@useobject{currentmarker}{}%
\end{pgfscope}%
\begin{pgfscope}%
\pgfsys@transformshift{2.124818in}{0.761786in}%
\pgfsys@useobject{currentmarker}{}%
\end{pgfscope}%
\begin{pgfscope}%
\pgfsys@transformshift{2.125071in}{0.768314in}%
\pgfsys@useobject{currentmarker}{}%
\end{pgfscope}%
\begin{pgfscope}%
\pgfsys@transformshift{2.125324in}{0.758108in}%
\pgfsys@useobject{currentmarker}{}%
\end{pgfscope}%
\begin{pgfscope}%
\pgfsys@transformshift{2.125577in}{0.719701in}%
\pgfsys@useobject{currentmarker}{}%
\end{pgfscope}%
\begin{pgfscope}%
\pgfsys@transformshift{2.125829in}{0.762119in}%
\pgfsys@useobject{currentmarker}{}%
\end{pgfscope}%
\begin{pgfscope}%
\pgfsys@transformshift{2.126081in}{0.734965in}%
\pgfsys@useobject{currentmarker}{}%
\end{pgfscope}%
\begin{pgfscope}%
\pgfsys@transformshift{2.126333in}{0.745933in}%
\pgfsys@useobject{currentmarker}{}%
\end{pgfscope}%
\begin{pgfscope}%
\pgfsys@transformshift{2.126584in}{0.773927in}%
\pgfsys@useobject{currentmarker}{}%
\end{pgfscope}%
\begin{pgfscope}%
\pgfsys@transformshift{2.126835in}{0.749195in}%
\pgfsys@useobject{currentmarker}{}%
\end{pgfscope}%
\begin{pgfscope}%
\pgfsys@transformshift{2.127086in}{0.726377in}%
\pgfsys@useobject{currentmarker}{}%
\end{pgfscope}%
\begin{pgfscope}%
\pgfsys@transformshift{2.127337in}{0.739536in}%
\pgfsys@useobject{currentmarker}{}%
\end{pgfscope}%
\begin{pgfscope}%
\pgfsys@transformshift{2.127587in}{0.742369in}%
\pgfsys@useobject{currentmarker}{}%
\end{pgfscope}%
\begin{pgfscope}%
\pgfsys@transformshift{2.127837in}{0.692092in}%
\pgfsys@useobject{currentmarker}{}%
\end{pgfscope}%
\begin{pgfscope}%
\pgfsys@transformshift{2.128087in}{0.583372in}%
\pgfsys@useobject{currentmarker}{}%
\end{pgfscope}%
\begin{pgfscope}%
\pgfsys@transformshift{2.128336in}{0.687074in}%
\pgfsys@useobject{currentmarker}{}%
\end{pgfscope}%
\begin{pgfscope}%
\pgfsys@transformshift{2.128586in}{0.700639in}%
\pgfsys@useobject{currentmarker}{}%
\end{pgfscope}%
\begin{pgfscope}%
\pgfsys@transformshift{2.128835in}{0.712495in}%
\pgfsys@useobject{currentmarker}{}%
\end{pgfscope}%
\begin{pgfscope}%
\pgfsys@transformshift{2.129083in}{0.733748in}%
\pgfsys@useobject{currentmarker}{}%
\end{pgfscope}%
\begin{pgfscope}%
\pgfsys@transformshift{2.129332in}{0.757149in}%
\pgfsys@useobject{currentmarker}{}%
\end{pgfscope}%
\begin{pgfscope}%
\pgfsys@transformshift{2.129580in}{0.755118in}%
\pgfsys@useobject{currentmarker}{}%
\end{pgfscope}%
\begin{pgfscope}%
\pgfsys@transformshift{2.129827in}{0.740099in}%
\pgfsys@useobject{currentmarker}{}%
\end{pgfscope}%
\begin{pgfscope}%
\pgfsys@transformshift{2.130075in}{0.757521in}%
\pgfsys@useobject{currentmarker}{}%
\end{pgfscope}%
\begin{pgfscope}%
\pgfsys@transformshift{2.130322in}{0.714419in}%
\pgfsys@useobject{currentmarker}{}%
\end{pgfscope}%
\begin{pgfscope}%
\pgfsys@transformshift{2.130569in}{0.692725in}%
\pgfsys@useobject{currentmarker}{}%
\end{pgfscope}%
\begin{pgfscope}%
\pgfsys@transformshift{2.130816in}{0.671637in}%
\pgfsys@useobject{currentmarker}{}%
\end{pgfscope}%
\begin{pgfscope}%
\pgfsys@transformshift{2.131062in}{0.712582in}%
\pgfsys@useobject{currentmarker}{}%
\end{pgfscope}%
\begin{pgfscope}%
\pgfsys@transformshift{2.131309in}{0.722571in}%
\pgfsys@useobject{currentmarker}{}%
\end{pgfscope}%
\begin{pgfscope}%
\pgfsys@transformshift{2.131555in}{0.728320in}%
\pgfsys@useobject{currentmarker}{}%
\end{pgfscope}%
\begin{pgfscope}%
\pgfsys@transformshift{2.131800in}{0.745514in}%
\pgfsys@useobject{currentmarker}{}%
\end{pgfscope}%
\begin{pgfscope}%
\pgfsys@transformshift{2.132046in}{0.718325in}%
\pgfsys@useobject{currentmarker}{}%
\end{pgfscope}%
\begin{pgfscope}%
\pgfsys@transformshift{2.132291in}{0.739599in}%
\pgfsys@useobject{currentmarker}{}%
\end{pgfscope}%
\begin{pgfscope}%
\pgfsys@transformshift{2.132536in}{0.764963in}%
\pgfsys@useobject{currentmarker}{}%
\end{pgfscope}%
\begin{pgfscope}%
\pgfsys@transformshift{2.132780in}{0.741206in}%
\pgfsys@useobject{currentmarker}{}%
\end{pgfscope}%
\begin{pgfscope}%
\pgfsys@transformshift{2.133025in}{0.693069in}%
\pgfsys@useobject{currentmarker}{}%
\end{pgfscope}%
\begin{pgfscope}%
\pgfsys@transformshift{2.133269in}{0.729560in}%
\pgfsys@useobject{currentmarker}{}%
\end{pgfscope}%
\begin{pgfscope}%
\pgfsys@transformshift{2.133512in}{0.698952in}%
\pgfsys@useobject{currentmarker}{}%
\end{pgfscope}%
\begin{pgfscope}%
\pgfsys@transformshift{2.133756in}{0.714272in}%
\pgfsys@useobject{currentmarker}{}%
\end{pgfscope}%
\begin{pgfscope}%
\pgfsys@transformshift{2.133999in}{0.739100in}%
\pgfsys@useobject{currentmarker}{}%
\end{pgfscope}%
\begin{pgfscope}%
\pgfsys@transformshift{2.134242in}{0.788243in}%
\pgfsys@useobject{currentmarker}{}%
\end{pgfscope}%
\begin{pgfscope}%
\pgfsys@transformshift{2.134485in}{0.766811in}%
\pgfsys@useobject{currentmarker}{}%
\end{pgfscope}%
\begin{pgfscope}%
\pgfsys@transformshift{2.134727in}{0.742117in}%
\pgfsys@useobject{currentmarker}{}%
\end{pgfscope}%
\begin{pgfscope}%
\pgfsys@transformshift{2.134970in}{0.772503in}%
\pgfsys@useobject{currentmarker}{}%
\end{pgfscope}%
\begin{pgfscope}%
\pgfsys@transformshift{2.135212in}{0.755034in}%
\pgfsys@useobject{currentmarker}{}%
\end{pgfscope}%
\begin{pgfscope}%
\pgfsys@transformshift{2.135453in}{0.756957in}%
\pgfsys@useobject{currentmarker}{}%
\end{pgfscope}%
\begin{pgfscope}%
\pgfsys@transformshift{2.135695in}{0.747327in}%
\pgfsys@useobject{currentmarker}{}%
\end{pgfscope}%
\begin{pgfscope}%
\pgfsys@transformshift{2.135936in}{0.745524in}%
\pgfsys@useobject{currentmarker}{}%
\end{pgfscope}%
\begin{pgfscope}%
\pgfsys@transformshift{2.136177in}{0.743153in}%
\pgfsys@useobject{currentmarker}{}%
\end{pgfscope}%
\begin{pgfscope}%
\pgfsys@transformshift{2.136417in}{0.784306in}%
\pgfsys@useobject{currentmarker}{}%
\end{pgfscope}%
\begin{pgfscope}%
\pgfsys@transformshift{2.136658in}{0.756540in}%
\pgfsys@useobject{currentmarker}{}%
\end{pgfscope}%
\begin{pgfscope}%
\pgfsys@transformshift{2.136898in}{0.734460in}%
\pgfsys@useobject{currentmarker}{}%
\end{pgfscope}%
\begin{pgfscope}%
\pgfsys@transformshift{2.137138in}{0.741341in}%
\pgfsys@useobject{currentmarker}{}%
\end{pgfscope}%
\begin{pgfscope}%
\pgfsys@transformshift{2.137377in}{0.710754in}%
\pgfsys@useobject{currentmarker}{}%
\end{pgfscope}%
\begin{pgfscope}%
\pgfsys@transformshift{2.137617in}{0.694449in}%
\pgfsys@useobject{currentmarker}{}%
\end{pgfscope}%
\begin{pgfscope}%
\pgfsys@transformshift{2.137856in}{0.678078in}%
\pgfsys@useobject{currentmarker}{}%
\end{pgfscope}%
\begin{pgfscope}%
\pgfsys@transformshift{2.138095in}{0.723370in}%
\pgfsys@useobject{currentmarker}{}%
\end{pgfscope}%
\begin{pgfscope}%
\pgfsys@transformshift{2.138333in}{0.715188in}%
\pgfsys@useobject{currentmarker}{}%
\end{pgfscope}%
\begin{pgfscope}%
\pgfsys@transformshift{2.138572in}{0.727830in}%
\pgfsys@useobject{currentmarker}{}%
\end{pgfscope}%
\begin{pgfscope}%
\pgfsys@transformshift{2.138810in}{0.747185in}%
\pgfsys@useobject{currentmarker}{}%
\end{pgfscope}%
\begin{pgfscope}%
\pgfsys@transformshift{2.139048in}{0.741907in}%
\pgfsys@useobject{currentmarker}{}%
\end{pgfscope}%
\begin{pgfscope}%
\pgfsys@transformshift{2.139285in}{0.745801in}%
\pgfsys@useobject{currentmarker}{}%
\end{pgfscope}%
\begin{pgfscope}%
\pgfsys@transformshift{2.139523in}{0.733556in}%
\pgfsys@useobject{currentmarker}{}%
\end{pgfscope}%
\begin{pgfscope}%
\pgfsys@transformshift{2.139760in}{0.769085in}%
\pgfsys@useobject{currentmarker}{}%
\end{pgfscope}%
\begin{pgfscope}%
\pgfsys@transformshift{2.139997in}{0.747306in}%
\pgfsys@useobject{currentmarker}{}%
\end{pgfscope}%
\begin{pgfscope}%
\pgfsys@transformshift{2.140233in}{0.717067in}%
\pgfsys@useobject{currentmarker}{}%
\end{pgfscope}%
\begin{pgfscope}%
\pgfsys@transformshift{2.140470in}{0.682604in}%
\pgfsys@useobject{currentmarker}{}%
\end{pgfscope}%
\begin{pgfscope}%
\pgfsys@transformshift{2.140706in}{0.677891in}%
\pgfsys@useobject{currentmarker}{}%
\end{pgfscope}%
\begin{pgfscope}%
\pgfsys@transformshift{2.140942in}{0.696633in}%
\pgfsys@useobject{currentmarker}{}%
\end{pgfscope}%
\begin{pgfscope}%
\pgfsys@transformshift{2.141177in}{0.686876in}%
\pgfsys@useobject{currentmarker}{}%
\end{pgfscope}%
\begin{pgfscope}%
\pgfsys@transformshift{2.141412in}{0.755205in}%
\pgfsys@useobject{currentmarker}{}%
\end{pgfscope}%
\begin{pgfscope}%
\pgfsys@transformshift{2.141648in}{0.759415in}%
\pgfsys@useobject{currentmarker}{}%
\end{pgfscope}%
\begin{pgfscope}%
\pgfsys@transformshift{2.141882in}{0.724966in}%
\pgfsys@useobject{currentmarker}{}%
\end{pgfscope}%
\begin{pgfscope}%
\pgfsys@transformshift{2.142117in}{0.723721in}%
\pgfsys@useobject{currentmarker}{}%
\end{pgfscope}%
\begin{pgfscope}%
\pgfsys@transformshift{2.142351in}{0.717700in}%
\pgfsys@useobject{currentmarker}{}%
\end{pgfscope}%
\begin{pgfscope}%
\pgfsys@transformshift{2.142586in}{0.678488in}%
\pgfsys@useobject{currentmarker}{}%
\end{pgfscope}%
\begin{pgfscope}%
\pgfsys@transformshift{2.142820in}{0.688144in}%
\pgfsys@useobject{currentmarker}{}%
\end{pgfscope}%
\begin{pgfscope}%
\pgfsys@transformshift{2.143053in}{0.729649in}%
\pgfsys@useobject{currentmarker}{}%
\end{pgfscope}%
\begin{pgfscope}%
\pgfsys@transformshift{2.143287in}{0.738245in}%
\pgfsys@useobject{currentmarker}{}%
\end{pgfscope}%
\begin{pgfscope}%
\pgfsys@transformshift{2.143520in}{0.751378in}%
\pgfsys@useobject{currentmarker}{}%
\end{pgfscope}%
\begin{pgfscope}%
\pgfsys@transformshift{2.143753in}{0.742542in}%
\pgfsys@useobject{currentmarker}{}%
\end{pgfscope}%
\begin{pgfscope}%
\pgfsys@transformshift{2.143985in}{0.727847in}%
\pgfsys@useobject{currentmarker}{}%
\end{pgfscope}%
\begin{pgfscope}%
\pgfsys@transformshift{2.144218in}{0.782439in}%
\pgfsys@useobject{currentmarker}{}%
\end{pgfscope}%
\begin{pgfscope}%
\pgfsys@transformshift{2.144450in}{0.777934in}%
\pgfsys@useobject{currentmarker}{}%
\end{pgfscope}%
\begin{pgfscope}%
\pgfsys@transformshift{2.144682in}{0.727502in}%
\pgfsys@useobject{currentmarker}{}%
\end{pgfscope}%
\begin{pgfscope}%
\pgfsys@transformshift{2.144914in}{0.724140in}%
\pgfsys@useobject{currentmarker}{}%
\end{pgfscope}%
\begin{pgfscope}%
\pgfsys@transformshift{2.145145in}{0.671094in}%
\pgfsys@useobject{currentmarker}{}%
\end{pgfscope}%
\begin{pgfscope}%
\pgfsys@transformshift{2.145376in}{0.679968in}%
\pgfsys@useobject{currentmarker}{}%
\end{pgfscope}%
\begin{pgfscope}%
\pgfsys@transformshift{2.145607in}{0.718964in}%
\pgfsys@useobject{currentmarker}{}%
\end{pgfscope}%
\begin{pgfscope}%
\pgfsys@transformshift{2.145838in}{0.738431in}%
\pgfsys@useobject{currentmarker}{}%
\end{pgfscope}%
\begin{pgfscope}%
\pgfsys@transformshift{2.146069in}{0.765222in}%
\pgfsys@useobject{currentmarker}{}%
\end{pgfscope}%
\begin{pgfscope}%
\pgfsys@transformshift{2.146299in}{0.742933in}%
\pgfsys@useobject{currentmarker}{}%
\end{pgfscope}%
\begin{pgfscope}%
\pgfsys@transformshift{2.146529in}{0.735651in}%
\pgfsys@useobject{currentmarker}{}%
\end{pgfscope}%
\begin{pgfscope}%
\pgfsys@transformshift{2.146759in}{0.708925in}%
\pgfsys@useobject{currentmarker}{}%
\end{pgfscope}%
\begin{pgfscope}%
\pgfsys@transformshift{2.146988in}{0.657875in}%
\pgfsys@useobject{currentmarker}{}%
\end{pgfscope}%
\begin{pgfscope}%
\pgfsys@transformshift{2.147218in}{0.649352in}%
\pgfsys@useobject{currentmarker}{}%
\end{pgfscope}%
\begin{pgfscope}%
\pgfsys@transformshift{2.147447in}{0.691531in}%
\pgfsys@useobject{currentmarker}{}%
\end{pgfscope}%
\begin{pgfscope}%
\pgfsys@transformshift{2.147676in}{0.698453in}%
\pgfsys@useobject{currentmarker}{}%
\end{pgfscope}%
\begin{pgfscope}%
\pgfsys@transformshift{2.147904in}{0.670853in}%
\pgfsys@useobject{currentmarker}{}%
\end{pgfscope}%
\begin{pgfscope}%
\pgfsys@transformshift{2.148133in}{0.702732in}%
\pgfsys@useobject{currentmarker}{}%
\end{pgfscope}%
\begin{pgfscope}%
\pgfsys@transformshift{2.148361in}{0.719129in}%
\pgfsys@useobject{currentmarker}{}%
\end{pgfscope}%
\begin{pgfscope}%
\pgfsys@transformshift{2.148589in}{0.720260in}%
\pgfsys@useobject{currentmarker}{}%
\end{pgfscope}%
\begin{pgfscope}%
\pgfsys@transformshift{2.148817in}{0.737484in}%
\pgfsys@useobject{currentmarker}{}%
\end{pgfscope}%
\begin{pgfscope}%
\pgfsys@transformshift{2.149044in}{0.687334in}%
\pgfsys@useobject{currentmarker}{}%
\end{pgfscope}%
\begin{pgfscope}%
\pgfsys@transformshift{2.149271in}{0.672022in}%
\pgfsys@useobject{currentmarker}{}%
\end{pgfscope}%
\begin{pgfscope}%
\pgfsys@transformshift{2.149499in}{0.718057in}%
\pgfsys@useobject{currentmarker}{}%
\end{pgfscope}%
\begin{pgfscope}%
\pgfsys@transformshift{2.149725in}{0.735866in}%
\pgfsys@useobject{currentmarker}{}%
\end{pgfscope}%
\begin{pgfscope}%
\pgfsys@transformshift{2.149952in}{0.719864in}%
\pgfsys@useobject{currentmarker}{}%
\end{pgfscope}%
\begin{pgfscope}%
\pgfsys@transformshift{2.150178in}{0.715389in}%
\pgfsys@useobject{currentmarker}{}%
\end{pgfscope}%
\begin{pgfscope}%
\pgfsys@transformshift{2.150404in}{0.732510in}%
\pgfsys@useobject{currentmarker}{}%
\end{pgfscope}%
\begin{pgfscope}%
\pgfsys@transformshift{2.150630in}{0.759267in}%
\pgfsys@useobject{currentmarker}{}%
\end{pgfscope}%
\begin{pgfscope}%
\pgfsys@transformshift{2.150856in}{0.772014in}%
\pgfsys@useobject{currentmarker}{}%
\end{pgfscope}%
\begin{pgfscope}%
\pgfsys@transformshift{2.151081in}{0.739052in}%
\pgfsys@useobject{currentmarker}{}%
\end{pgfscope}%
\begin{pgfscope}%
\pgfsys@transformshift{2.151307in}{0.744961in}%
\pgfsys@useobject{currentmarker}{}%
\end{pgfscope}%
\begin{pgfscope}%
\pgfsys@transformshift{2.151532in}{0.756346in}%
\pgfsys@useobject{currentmarker}{}%
\end{pgfscope}%
\begin{pgfscope}%
\pgfsys@transformshift{2.151756in}{0.786215in}%
\pgfsys@useobject{currentmarker}{}%
\end{pgfscope}%
\begin{pgfscope}%
\pgfsys@transformshift{2.151981in}{0.775746in}%
\pgfsys@useobject{currentmarker}{}%
\end{pgfscope}%
\begin{pgfscope}%
\pgfsys@transformshift{2.152205in}{0.736112in}%
\pgfsys@useobject{currentmarker}{}%
\end{pgfscope}%
\begin{pgfscope}%
\pgfsys@transformshift{2.152429in}{0.726876in}%
\pgfsys@useobject{currentmarker}{}%
\end{pgfscope}%
\begin{pgfscope}%
\pgfsys@transformshift{2.152653in}{0.734080in}%
\pgfsys@useobject{currentmarker}{}%
\end{pgfscope}%
\begin{pgfscope}%
\pgfsys@transformshift{2.152877in}{0.739071in}%
\pgfsys@useobject{currentmarker}{}%
\end{pgfscope}%
\begin{pgfscope}%
\pgfsys@transformshift{2.153100in}{0.745069in}%
\pgfsys@useobject{currentmarker}{}%
\end{pgfscope}%
\begin{pgfscope}%
\pgfsys@transformshift{2.153323in}{0.745703in}%
\pgfsys@useobject{currentmarker}{}%
\end{pgfscope}%
\begin{pgfscope}%
\pgfsys@transformshift{2.153546in}{0.724801in}%
\pgfsys@useobject{currentmarker}{}%
\end{pgfscope}%
\begin{pgfscope}%
\pgfsys@transformshift{2.153769in}{0.715722in}%
\pgfsys@useobject{currentmarker}{}%
\end{pgfscope}%
\begin{pgfscope}%
\pgfsys@transformshift{2.153992in}{0.698345in}%
\pgfsys@useobject{currentmarker}{}%
\end{pgfscope}%
\begin{pgfscope}%
\pgfsys@transformshift{2.154214in}{0.657475in}%
\pgfsys@useobject{currentmarker}{}%
\end{pgfscope}%
\begin{pgfscope}%
\pgfsys@transformshift{2.154436in}{0.664817in}%
\pgfsys@useobject{currentmarker}{}%
\end{pgfscope}%
\begin{pgfscope}%
\pgfsys@transformshift{2.154658in}{0.680556in}%
\pgfsys@useobject{currentmarker}{}%
\end{pgfscope}%
\begin{pgfscope}%
\pgfsys@transformshift{2.154880in}{0.672700in}%
\pgfsys@useobject{currentmarker}{}%
\end{pgfscope}%
\begin{pgfscope}%
\pgfsys@transformshift{2.155101in}{0.647495in}%
\pgfsys@useobject{currentmarker}{}%
\end{pgfscope}%
\begin{pgfscope}%
\pgfsys@transformshift{2.155322in}{0.682551in}%
\pgfsys@useobject{currentmarker}{}%
\end{pgfscope}%
\begin{pgfscope}%
\pgfsys@transformshift{2.155543in}{0.699665in}%
\pgfsys@useobject{currentmarker}{}%
\end{pgfscope}%
\begin{pgfscope}%
\pgfsys@transformshift{2.155764in}{0.728117in}%
\pgfsys@useobject{currentmarker}{}%
\end{pgfscope}%
\begin{pgfscope}%
\pgfsys@transformshift{2.155985in}{0.738862in}%
\pgfsys@useobject{currentmarker}{}%
\end{pgfscope}%
\begin{pgfscope}%
\pgfsys@transformshift{2.156205in}{0.756020in}%
\pgfsys@useobject{currentmarker}{}%
\end{pgfscope}%
\begin{pgfscope}%
\pgfsys@transformshift{2.156425in}{0.728128in}%
\pgfsys@useobject{currentmarker}{}%
\end{pgfscope}%
\begin{pgfscope}%
\pgfsys@transformshift{2.156645in}{0.706383in}%
\pgfsys@useobject{currentmarker}{}%
\end{pgfscope}%
\begin{pgfscope}%
\pgfsys@transformshift{2.156865in}{0.744930in}%
\pgfsys@useobject{currentmarker}{}%
\end{pgfscope}%
\begin{pgfscope}%
\pgfsys@transformshift{2.157084in}{0.736139in}%
\pgfsys@useobject{currentmarker}{}%
\end{pgfscope}%
\begin{pgfscope}%
\pgfsys@transformshift{2.157304in}{0.678594in}%
\pgfsys@useobject{currentmarker}{}%
\end{pgfscope}%
\begin{pgfscope}%
\pgfsys@transformshift{2.157523in}{0.720768in}%
\pgfsys@useobject{currentmarker}{}%
\end{pgfscope}%
\begin{pgfscope}%
\pgfsys@transformshift{2.157741in}{0.745489in}%
\pgfsys@useobject{currentmarker}{}%
\end{pgfscope}%
\begin{pgfscope}%
\pgfsys@transformshift{2.157960in}{0.757569in}%
\pgfsys@useobject{currentmarker}{}%
\end{pgfscope}%
\begin{pgfscope}%
\pgfsys@transformshift{2.158179in}{0.743203in}%
\pgfsys@useobject{currentmarker}{}%
\end{pgfscope}%
\begin{pgfscope}%
\pgfsys@transformshift{2.158397in}{0.707806in}%
\pgfsys@useobject{currentmarker}{}%
\end{pgfscope}%
\begin{pgfscope}%
\pgfsys@transformshift{2.158615in}{0.706890in}%
\pgfsys@useobject{currentmarker}{}%
\end{pgfscope}%
\begin{pgfscope}%
\pgfsys@transformshift{2.158833in}{0.696165in}%
\pgfsys@useobject{currentmarker}{}%
\end{pgfscope}%
\begin{pgfscope}%
\pgfsys@transformshift{2.159050in}{0.712572in}%
\pgfsys@useobject{currentmarker}{}%
\end{pgfscope}%
\begin{pgfscope}%
\pgfsys@transformshift{2.159268in}{0.746496in}%
\pgfsys@useobject{currentmarker}{}%
\end{pgfscope}%
\begin{pgfscope}%
\pgfsys@transformshift{2.159485in}{0.726192in}%
\pgfsys@useobject{currentmarker}{}%
\end{pgfscope}%
\begin{pgfscope}%
\pgfsys@transformshift{2.159702in}{0.727406in}%
\pgfsys@useobject{currentmarker}{}%
\end{pgfscope}%
\begin{pgfscope}%
\pgfsys@transformshift{2.159918in}{0.738646in}%
\pgfsys@useobject{currentmarker}{}%
\end{pgfscope}%
\begin{pgfscope}%
\pgfsys@transformshift{2.160135in}{0.740001in}%
\pgfsys@useobject{currentmarker}{}%
\end{pgfscope}%
\begin{pgfscope}%
\pgfsys@transformshift{2.160351in}{0.723584in}%
\pgfsys@useobject{currentmarker}{}%
\end{pgfscope}%
\begin{pgfscope}%
\pgfsys@transformshift{2.160567in}{0.656126in}%
\pgfsys@useobject{currentmarker}{}%
\end{pgfscope}%
\begin{pgfscope}%
\pgfsys@transformshift{2.160783in}{0.701592in}%
\pgfsys@useobject{currentmarker}{}%
\end{pgfscope}%
\begin{pgfscope}%
\pgfsys@transformshift{2.160999in}{0.663691in}%
\pgfsys@useobject{currentmarker}{}%
\end{pgfscope}%
\begin{pgfscope}%
\pgfsys@transformshift{2.161214in}{0.692026in}%
\pgfsys@useobject{currentmarker}{}%
\end{pgfscope}%
\begin{pgfscope}%
\pgfsys@transformshift{2.161430in}{0.770786in}%
\pgfsys@useobject{currentmarker}{}%
\end{pgfscope}%
\begin{pgfscope}%
\pgfsys@transformshift{2.161645in}{0.759329in}%
\pgfsys@useobject{currentmarker}{}%
\end{pgfscope}%
\begin{pgfscope}%
\pgfsys@transformshift{2.161860in}{0.710355in}%
\pgfsys@useobject{currentmarker}{}%
\end{pgfscope}%
\begin{pgfscope}%
\pgfsys@transformshift{2.162074in}{0.739331in}%
\pgfsys@useobject{currentmarker}{}%
\end{pgfscope}%
\begin{pgfscope}%
\pgfsys@transformshift{2.162289in}{0.708310in}%
\pgfsys@useobject{currentmarker}{}%
\end{pgfscope}%
\begin{pgfscope}%
\pgfsys@transformshift{2.162503in}{0.669930in}%
\pgfsys@useobject{currentmarker}{}%
\end{pgfscope}%
\begin{pgfscope}%
\pgfsys@transformshift{2.162717in}{0.728552in}%
\pgfsys@useobject{currentmarker}{}%
\end{pgfscope}%
\begin{pgfscope}%
\pgfsys@transformshift{2.162931in}{0.755829in}%
\pgfsys@useobject{currentmarker}{}%
\end{pgfscope}%
\begin{pgfscope}%
\pgfsys@transformshift{2.163145in}{0.713167in}%
\pgfsys@useobject{currentmarker}{}%
\end{pgfscope}%
\begin{pgfscope}%
\pgfsys@transformshift{2.163358in}{0.672702in}%
\pgfsys@useobject{currentmarker}{}%
\end{pgfscope}%
\begin{pgfscope}%
\pgfsys@transformshift{2.163571in}{0.708122in}%
\pgfsys@useobject{currentmarker}{}%
\end{pgfscope}%
\begin{pgfscope}%
\pgfsys@transformshift{2.163784in}{0.744918in}%
\pgfsys@useobject{currentmarker}{}%
\end{pgfscope}%
\begin{pgfscope}%
\pgfsys@transformshift{2.163997in}{0.745216in}%
\pgfsys@useobject{currentmarker}{}%
\end{pgfscope}%
\begin{pgfscope}%
\pgfsys@transformshift{2.164210in}{0.711256in}%
\pgfsys@useobject{currentmarker}{}%
\end{pgfscope}%
\begin{pgfscope}%
\pgfsys@transformshift{2.164422in}{0.651359in}%
\pgfsys@useobject{currentmarker}{}%
\end{pgfscope}%
\begin{pgfscope}%
\pgfsys@transformshift{2.164635in}{0.623871in}%
\pgfsys@useobject{currentmarker}{}%
\end{pgfscope}%
\begin{pgfscope}%
\pgfsys@transformshift{2.164847in}{0.634790in}%
\pgfsys@useobject{currentmarker}{}%
\end{pgfscope}%
\begin{pgfscope}%
\pgfsys@transformshift{2.165058in}{0.667293in}%
\pgfsys@useobject{currentmarker}{}%
\end{pgfscope}%
\begin{pgfscope}%
\pgfsys@transformshift{2.165270in}{0.696907in}%
\pgfsys@useobject{currentmarker}{}%
\end{pgfscope}%
\begin{pgfscope}%
\pgfsys@transformshift{2.165481in}{0.698575in}%
\pgfsys@useobject{currentmarker}{}%
\end{pgfscope}%
\begin{pgfscope}%
\pgfsys@transformshift{2.165693in}{0.673732in}%
\pgfsys@useobject{currentmarker}{}%
\end{pgfscope}%
\begin{pgfscope}%
\pgfsys@transformshift{2.165904in}{0.717383in}%
\pgfsys@useobject{currentmarker}{}%
\end{pgfscope}%
\begin{pgfscope}%
\pgfsys@transformshift{2.166115in}{0.734440in}%
\pgfsys@useobject{currentmarker}{}%
\end{pgfscope}%
\begin{pgfscope}%
\pgfsys@transformshift{2.166325in}{0.721465in}%
\pgfsys@useobject{currentmarker}{}%
\end{pgfscope}%
\begin{pgfscope}%
\pgfsys@transformshift{2.166536in}{0.727223in}%
\pgfsys@useobject{currentmarker}{}%
\end{pgfscope}%
\begin{pgfscope}%
\pgfsys@transformshift{2.166746in}{0.726318in}%
\pgfsys@useobject{currentmarker}{}%
\end{pgfscope}%
\begin{pgfscope}%
\pgfsys@transformshift{2.166956in}{0.705061in}%
\pgfsys@useobject{currentmarker}{}%
\end{pgfscope}%
\begin{pgfscope}%
\pgfsys@transformshift{2.167166in}{0.713422in}%
\pgfsys@useobject{currentmarker}{}%
\end{pgfscope}%
\begin{pgfscope}%
\pgfsys@transformshift{2.167375in}{0.721215in}%
\pgfsys@useobject{currentmarker}{}%
\end{pgfscope}%
\begin{pgfscope}%
\pgfsys@transformshift{2.167585in}{0.669433in}%
\pgfsys@useobject{currentmarker}{}%
\end{pgfscope}%
\begin{pgfscope}%
\pgfsys@transformshift{2.167794in}{0.703672in}%
\pgfsys@useobject{currentmarker}{}%
\end{pgfscope}%
\begin{pgfscope}%
\pgfsys@transformshift{2.168003in}{0.720923in}%
\pgfsys@useobject{currentmarker}{}%
\end{pgfscope}%
\begin{pgfscope}%
\pgfsys@transformshift{2.168212in}{0.773505in}%
\pgfsys@useobject{currentmarker}{}%
\end{pgfscope}%
\begin{pgfscope}%
\pgfsys@transformshift{2.168421in}{0.763206in}%
\pgfsys@useobject{currentmarker}{}%
\end{pgfscope}%
\begin{pgfscope}%
\pgfsys@transformshift{2.168629in}{0.720212in}%
\pgfsys@useobject{currentmarker}{}%
\end{pgfscope}%
\begin{pgfscope}%
\pgfsys@transformshift{2.168838in}{0.711242in}%
\pgfsys@useobject{currentmarker}{}%
\end{pgfscope}%
\begin{pgfscope}%
\pgfsys@transformshift{2.169046in}{0.668423in}%
\pgfsys@useobject{currentmarker}{}%
\end{pgfscope}%
\begin{pgfscope}%
\pgfsys@transformshift{2.169254in}{0.699231in}%
\pgfsys@useobject{currentmarker}{}%
\end{pgfscope}%
\begin{pgfscope}%
\pgfsys@transformshift{2.169461in}{0.696676in}%
\pgfsys@useobject{currentmarker}{}%
\end{pgfscope}%
\begin{pgfscope}%
\pgfsys@transformshift{2.169669in}{0.681883in}%
\pgfsys@useobject{currentmarker}{}%
\end{pgfscope}%
\begin{pgfscope}%
\pgfsys@transformshift{2.169876in}{0.688953in}%
\pgfsys@useobject{currentmarker}{}%
\end{pgfscope}%
\begin{pgfscope}%
\pgfsys@transformshift{2.170083in}{0.736737in}%
\pgfsys@useobject{currentmarker}{}%
\end{pgfscope}%
\begin{pgfscope}%
\pgfsys@transformshift{2.170290in}{0.726701in}%
\pgfsys@useobject{currentmarker}{}%
\end{pgfscope}%
\begin{pgfscope}%
\pgfsys@transformshift{2.170497in}{0.708945in}%
\pgfsys@useobject{currentmarker}{}%
\end{pgfscope}%
\begin{pgfscope}%
\pgfsys@transformshift{2.170704in}{0.754924in}%
\pgfsys@useobject{currentmarker}{}%
\end{pgfscope}%
\begin{pgfscope}%
\pgfsys@transformshift{2.170910in}{0.756449in}%
\pgfsys@useobject{currentmarker}{}%
\end{pgfscope}%
\begin{pgfscope}%
\pgfsys@transformshift{2.171116in}{0.669675in}%
\pgfsys@useobject{currentmarker}{}%
\end{pgfscope}%
\begin{pgfscope}%
\pgfsys@transformshift{2.171322in}{0.648596in}%
\pgfsys@useobject{currentmarker}{}%
\end{pgfscope}%
\begin{pgfscope}%
\pgfsys@transformshift{2.171528in}{0.720611in}%
\pgfsys@useobject{currentmarker}{}%
\end{pgfscope}%
\begin{pgfscope}%
\pgfsys@transformshift{2.171734in}{0.736635in}%
\pgfsys@useobject{currentmarker}{}%
\end{pgfscope}%
\begin{pgfscope}%
\pgfsys@transformshift{2.171939in}{0.700356in}%
\pgfsys@useobject{currentmarker}{}%
\end{pgfscope}%
\begin{pgfscope}%
\pgfsys@transformshift{2.172144in}{0.676777in}%
\pgfsys@useobject{currentmarker}{}%
\end{pgfscope}%
\begin{pgfscope}%
\pgfsys@transformshift{2.172350in}{0.722993in}%
\pgfsys@useobject{currentmarker}{}%
\end{pgfscope}%
\begin{pgfscope}%
\pgfsys@transformshift{2.172554in}{0.708050in}%
\pgfsys@useobject{currentmarker}{}%
\end{pgfscope}%
\begin{pgfscope}%
\pgfsys@transformshift{2.172759in}{0.671673in}%
\pgfsys@useobject{currentmarker}{}%
\end{pgfscope}%
\begin{pgfscope}%
\pgfsys@transformshift{2.172964in}{0.677384in}%
\pgfsys@useobject{currentmarker}{}%
\end{pgfscope}%
\begin{pgfscope}%
\pgfsys@transformshift{2.173168in}{0.733607in}%
\pgfsys@useobject{currentmarker}{}%
\end{pgfscope}%
\begin{pgfscope}%
\pgfsys@transformshift{2.173372in}{0.764656in}%
\pgfsys@useobject{currentmarker}{}%
\end{pgfscope}%
\begin{pgfscope}%
\pgfsys@transformshift{2.173576in}{0.706235in}%
\pgfsys@useobject{currentmarker}{}%
\end{pgfscope}%
\begin{pgfscope}%
\pgfsys@transformshift{2.173780in}{0.733161in}%
\pgfsys@useobject{currentmarker}{}%
\end{pgfscope}%
\begin{pgfscope}%
\pgfsys@transformshift{2.173983in}{0.742567in}%
\pgfsys@useobject{currentmarker}{}%
\end{pgfscope}%
\begin{pgfscope}%
\pgfsys@transformshift{2.174187in}{0.710698in}%
\pgfsys@useobject{currentmarker}{}%
\end{pgfscope}%
\begin{pgfscope}%
\pgfsys@transformshift{2.174390in}{0.650475in}%
\pgfsys@useobject{currentmarker}{}%
\end{pgfscope}%
\begin{pgfscope}%
\pgfsys@transformshift{2.174593in}{0.657691in}%
\pgfsys@useobject{currentmarker}{}%
\end{pgfscope}%
\begin{pgfscope}%
\pgfsys@transformshift{2.174796in}{0.691387in}%
\pgfsys@useobject{currentmarker}{}%
\end{pgfscope}%
\begin{pgfscope}%
\pgfsys@transformshift{2.174999in}{0.701248in}%
\pgfsys@useobject{currentmarker}{}%
\end{pgfscope}%
\begin{pgfscope}%
\pgfsys@transformshift{2.175201in}{0.698631in}%
\pgfsys@useobject{currentmarker}{}%
\end{pgfscope}%
\begin{pgfscope}%
\pgfsys@transformshift{2.175403in}{0.715662in}%
\pgfsys@useobject{currentmarker}{}%
\end{pgfscope}%
\begin{pgfscope}%
\pgfsys@transformshift{2.175605in}{0.734154in}%
\pgfsys@useobject{currentmarker}{}%
\end{pgfscope}%
\begin{pgfscope}%
\pgfsys@transformshift{2.175807in}{0.743521in}%
\pgfsys@useobject{currentmarker}{}%
\end{pgfscope}%
\begin{pgfscope}%
\pgfsys@transformshift{2.176009in}{0.739611in}%
\pgfsys@useobject{currentmarker}{}%
\end{pgfscope}%
\begin{pgfscope}%
\pgfsys@transformshift{2.176211in}{0.717308in}%
\pgfsys@useobject{currentmarker}{}%
\end{pgfscope}%
\begin{pgfscope}%
\pgfsys@transformshift{2.176412in}{0.728441in}%
\pgfsys@useobject{currentmarker}{}%
\end{pgfscope}%
\begin{pgfscope}%
\pgfsys@transformshift{2.176613in}{0.694546in}%
\pgfsys@useobject{currentmarker}{}%
\end{pgfscope}%
\begin{pgfscope}%
\pgfsys@transformshift{2.176814in}{0.673112in}%
\pgfsys@useobject{currentmarker}{}%
\end{pgfscope}%
\begin{pgfscope}%
\pgfsys@transformshift{2.177015in}{0.696367in}%
\pgfsys@useobject{currentmarker}{}%
\end{pgfscope}%
\begin{pgfscope}%
\pgfsys@transformshift{2.177216in}{0.705838in}%
\pgfsys@useobject{currentmarker}{}%
\end{pgfscope}%
\begin{pgfscope}%
\pgfsys@transformshift{2.177416in}{0.678078in}%
\pgfsys@useobject{currentmarker}{}%
\end{pgfscope}%
\begin{pgfscope}%
\pgfsys@transformshift{2.177617in}{0.681173in}%
\pgfsys@useobject{currentmarker}{}%
\end{pgfscope}%
\begin{pgfscope}%
\pgfsys@transformshift{2.177817in}{0.761714in}%
\pgfsys@useobject{currentmarker}{}%
\end{pgfscope}%
\begin{pgfscope}%
\pgfsys@transformshift{2.178017in}{0.732977in}%
\pgfsys@useobject{currentmarker}{}%
\end{pgfscope}%
\begin{pgfscope}%
\pgfsys@transformshift{2.178217in}{0.721785in}%
\pgfsys@useobject{currentmarker}{}%
\end{pgfscope}%
\begin{pgfscope}%
\pgfsys@transformshift{2.178416in}{0.685483in}%
\pgfsys@useobject{currentmarker}{}%
\end{pgfscope}%
\begin{pgfscope}%
\pgfsys@transformshift{2.178616in}{0.684326in}%
\pgfsys@useobject{currentmarker}{}%
\end{pgfscope}%
\begin{pgfscope}%
\pgfsys@transformshift{2.178815in}{0.718565in}%
\pgfsys@useobject{currentmarker}{}%
\end{pgfscope}%
\begin{pgfscope}%
\pgfsys@transformshift{2.179014in}{0.685271in}%
\pgfsys@useobject{currentmarker}{}%
\end{pgfscope}%
\begin{pgfscope}%
\pgfsys@transformshift{2.179213in}{0.664190in}%
\pgfsys@useobject{currentmarker}{}%
\end{pgfscope}%
\begin{pgfscope}%
\pgfsys@transformshift{2.179411in}{0.685953in}%
\pgfsys@useobject{currentmarker}{}%
\end{pgfscope}%
\begin{pgfscope}%
\pgfsys@transformshift{2.179610in}{0.738743in}%
\pgfsys@useobject{currentmarker}{}%
\end{pgfscope}%
\begin{pgfscope}%
\pgfsys@transformshift{2.179808in}{0.713178in}%
\pgfsys@useobject{currentmarker}{}%
\end{pgfscope}%
\begin{pgfscope}%
\pgfsys@transformshift{2.180007in}{0.715660in}%
\pgfsys@useobject{currentmarker}{}%
\end{pgfscope}%
\begin{pgfscope}%
\pgfsys@transformshift{2.180205in}{0.728192in}%
\pgfsys@useobject{currentmarker}{}%
\end{pgfscope}%
\begin{pgfscope}%
\pgfsys@transformshift{2.180402in}{0.695799in}%
\pgfsys@useobject{currentmarker}{}%
\end{pgfscope}%
\begin{pgfscope}%
\pgfsys@transformshift{2.180600in}{0.717860in}%
\pgfsys@useobject{currentmarker}{}%
\end{pgfscope}%
\begin{pgfscope}%
\pgfsys@transformshift{2.180798in}{0.734015in}%
\pgfsys@useobject{currentmarker}{}%
\end{pgfscope}%
\begin{pgfscope}%
\pgfsys@transformshift{2.180995in}{0.731213in}%
\pgfsys@useobject{currentmarker}{}%
\end{pgfscope}%
\begin{pgfscope}%
\pgfsys@transformshift{2.181192in}{0.710442in}%
\pgfsys@useobject{currentmarker}{}%
\end{pgfscope}%
\begin{pgfscope}%
\pgfsys@transformshift{2.181389in}{0.692507in}%
\pgfsys@useobject{currentmarker}{}%
\end{pgfscope}%
\begin{pgfscope}%
\pgfsys@transformshift{2.181586in}{0.695097in}%
\pgfsys@useobject{currentmarker}{}%
\end{pgfscope}%
\begin{pgfscope}%
\pgfsys@transformshift{2.181782in}{0.727766in}%
\pgfsys@useobject{currentmarker}{}%
\end{pgfscope}%
\begin{pgfscope}%
\pgfsys@transformshift{2.181979in}{0.672138in}%
\pgfsys@useobject{currentmarker}{}%
\end{pgfscope}%
\begin{pgfscope}%
\pgfsys@transformshift{2.182175in}{0.674989in}%
\pgfsys@useobject{currentmarker}{}%
\end{pgfscope}%
\begin{pgfscope}%
\pgfsys@transformshift{2.182371in}{0.742624in}%
\pgfsys@useobject{currentmarker}{}%
\end{pgfscope}%
\begin{pgfscope}%
\pgfsys@transformshift{2.182567in}{0.755067in}%
\pgfsys@useobject{currentmarker}{}%
\end{pgfscope}%
\begin{pgfscope}%
\pgfsys@transformshift{2.182763in}{0.717957in}%
\pgfsys@useobject{currentmarker}{}%
\end{pgfscope}%
\begin{pgfscope}%
\pgfsys@transformshift{2.182959in}{0.711630in}%
\pgfsys@useobject{currentmarker}{}%
\end{pgfscope}%
\begin{pgfscope}%
\pgfsys@transformshift{2.183154in}{0.715877in}%
\pgfsys@useobject{currentmarker}{}%
\end{pgfscope}%
\begin{pgfscope}%
\pgfsys@transformshift{2.183349in}{0.689947in}%
\pgfsys@useobject{currentmarker}{}%
\end{pgfscope}%
\begin{pgfscope}%
\pgfsys@transformshift{2.183544in}{0.668804in}%
\pgfsys@useobject{currentmarker}{}%
\end{pgfscope}%
\begin{pgfscope}%
\pgfsys@transformshift{2.183739in}{0.661677in}%
\pgfsys@useobject{currentmarker}{}%
\end{pgfscope}%
\begin{pgfscope}%
\pgfsys@transformshift{2.183934in}{0.680868in}%
\pgfsys@useobject{currentmarker}{}%
\end{pgfscope}%
\begin{pgfscope}%
\pgfsys@transformshift{2.184129in}{0.714945in}%
\pgfsys@useobject{currentmarker}{}%
\end{pgfscope}%
\begin{pgfscope}%
\pgfsys@transformshift{2.184323in}{0.715940in}%
\pgfsys@useobject{currentmarker}{}%
\end{pgfscope}%
\begin{pgfscope}%
\pgfsys@transformshift{2.184517in}{0.727992in}%
\pgfsys@useobject{currentmarker}{}%
\end{pgfscope}%
\begin{pgfscope}%
\pgfsys@transformshift{2.184711in}{0.730353in}%
\pgfsys@useobject{currentmarker}{}%
\end{pgfscope}%
\begin{pgfscope}%
\pgfsys@transformshift{2.184905in}{0.739834in}%
\pgfsys@useobject{currentmarker}{}%
\end{pgfscope}%
\begin{pgfscope}%
\pgfsys@transformshift{2.185099in}{0.713620in}%
\pgfsys@useobject{currentmarker}{}%
\end{pgfscope}%
\begin{pgfscope}%
\pgfsys@transformshift{2.185293in}{0.685257in}%
\pgfsys@useobject{currentmarker}{}%
\end{pgfscope}%
\begin{pgfscope}%
\pgfsys@transformshift{2.185486in}{0.681571in}%
\pgfsys@useobject{currentmarker}{}%
\end{pgfscope}%
\begin{pgfscope}%
\pgfsys@transformshift{2.185679in}{0.649910in}%
\pgfsys@useobject{currentmarker}{}%
\end{pgfscope}%
\begin{pgfscope}%
\pgfsys@transformshift{2.185872in}{0.652157in}%
\pgfsys@useobject{currentmarker}{}%
\end{pgfscope}%
\begin{pgfscope}%
\pgfsys@transformshift{2.186065in}{0.687677in}%
\pgfsys@useobject{currentmarker}{}%
\end{pgfscope}%
\begin{pgfscope}%
\pgfsys@transformshift{2.186258in}{0.717005in}%
\pgfsys@useobject{currentmarker}{}%
\end{pgfscope}%
\begin{pgfscope}%
\pgfsys@transformshift{2.186451in}{0.718910in}%
\pgfsys@useobject{currentmarker}{}%
\end{pgfscope}%
\begin{pgfscope}%
\pgfsys@transformshift{2.186643in}{0.732227in}%
\pgfsys@useobject{currentmarker}{}%
\end{pgfscope}%
\begin{pgfscope}%
\pgfsys@transformshift{2.186835in}{0.755536in}%
\pgfsys@useobject{currentmarker}{}%
\end{pgfscope}%
\begin{pgfscope}%
\pgfsys@transformshift{2.187028in}{0.753894in}%
\pgfsys@useobject{currentmarker}{}%
\end{pgfscope}%
\begin{pgfscope}%
\pgfsys@transformshift{2.187219in}{0.772044in}%
\pgfsys@useobject{currentmarker}{}%
\end{pgfscope}%
\begin{pgfscope}%
\pgfsys@transformshift{2.187411in}{0.751588in}%
\pgfsys@useobject{currentmarker}{}%
\end{pgfscope}%
\begin{pgfscope}%
\pgfsys@transformshift{2.187603in}{0.707936in}%
\pgfsys@useobject{currentmarker}{}%
\end{pgfscope}%
\begin{pgfscope}%
\pgfsys@transformshift{2.187794in}{0.736030in}%
\pgfsys@useobject{currentmarker}{}%
\end{pgfscope}%
\begin{pgfscope}%
\pgfsys@transformshift{2.187986in}{0.728339in}%
\pgfsys@useobject{currentmarker}{}%
\end{pgfscope}%
\begin{pgfscope}%
\pgfsys@transformshift{2.188177in}{0.699380in}%
\pgfsys@useobject{currentmarker}{}%
\end{pgfscope}%
\begin{pgfscope}%
\pgfsys@transformshift{2.188368in}{0.666017in}%
\pgfsys@useobject{currentmarker}{}%
\end{pgfscope}%
\begin{pgfscope}%
\pgfsys@transformshift{2.188558in}{0.680523in}%
\pgfsys@useobject{currentmarker}{}%
\end{pgfscope}%
\begin{pgfscope}%
\pgfsys@transformshift{2.188749in}{0.722615in}%
\pgfsys@useobject{currentmarker}{}%
\end{pgfscope}%
\begin{pgfscope}%
\pgfsys@transformshift{2.188939in}{0.705168in}%
\pgfsys@useobject{currentmarker}{}%
\end{pgfscope}%
\begin{pgfscope}%
\pgfsys@transformshift{2.189130in}{0.689519in}%
\pgfsys@useobject{currentmarker}{}%
\end{pgfscope}%
\begin{pgfscope}%
\pgfsys@transformshift{2.189320in}{0.688651in}%
\pgfsys@useobject{currentmarker}{}%
\end{pgfscope}%
\begin{pgfscope}%
\pgfsys@transformshift{2.189510in}{0.710229in}%
\pgfsys@useobject{currentmarker}{}%
\end{pgfscope}%
\begin{pgfscope}%
\pgfsys@transformshift{2.189700in}{0.716728in}%
\pgfsys@useobject{currentmarker}{}%
\end{pgfscope}%
\begin{pgfscope}%
\pgfsys@transformshift{2.189889in}{0.727247in}%
\pgfsys@useobject{currentmarker}{}%
\end{pgfscope}%
\begin{pgfscope}%
\pgfsys@transformshift{2.190079in}{0.704509in}%
\pgfsys@useobject{currentmarker}{}%
\end{pgfscope}%
\begin{pgfscope}%
\pgfsys@transformshift{2.190268in}{0.704351in}%
\pgfsys@useobject{currentmarker}{}%
\end{pgfscope}%
\begin{pgfscope}%
\pgfsys@transformshift{2.190457in}{0.662984in}%
\pgfsys@useobject{currentmarker}{}%
\end{pgfscope}%
\begin{pgfscope}%
\pgfsys@transformshift{2.190646in}{0.592196in}%
\pgfsys@useobject{currentmarker}{}%
\end{pgfscope}%
\begin{pgfscope}%
\pgfsys@transformshift{2.190835in}{0.662557in}%
\pgfsys@useobject{currentmarker}{}%
\end{pgfscope}%
\begin{pgfscope}%
\pgfsys@transformshift{2.191024in}{0.708106in}%
\pgfsys@useobject{currentmarker}{}%
\end{pgfscope}%
\begin{pgfscope}%
\pgfsys@transformshift{2.191213in}{0.702896in}%
\pgfsys@useobject{currentmarker}{}%
\end{pgfscope}%
\begin{pgfscope}%
\pgfsys@transformshift{2.191401in}{0.683284in}%
\pgfsys@useobject{currentmarker}{}%
\end{pgfscope}%
\begin{pgfscope}%
\pgfsys@transformshift{2.191589in}{0.655938in}%
\pgfsys@useobject{currentmarker}{}%
\end{pgfscope}%
\begin{pgfscope}%
\pgfsys@transformshift{2.191777in}{0.678317in}%
\pgfsys@useobject{currentmarker}{}%
\end{pgfscope}%
\begin{pgfscope}%
\pgfsys@transformshift{2.191965in}{0.715542in}%
\pgfsys@useobject{currentmarker}{}%
\end{pgfscope}%
\begin{pgfscope}%
\pgfsys@transformshift{2.192153in}{0.680849in}%
\pgfsys@useobject{currentmarker}{}%
\end{pgfscope}%
\begin{pgfscope}%
\pgfsys@transformshift{2.192340in}{0.724524in}%
\pgfsys@useobject{currentmarker}{}%
\end{pgfscope}%
\begin{pgfscope}%
\pgfsys@transformshift{2.192528in}{0.754026in}%
\pgfsys@useobject{currentmarker}{}%
\end{pgfscope}%
\begin{pgfscope}%
\pgfsys@transformshift{2.192715in}{0.712912in}%
\pgfsys@useobject{currentmarker}{}%
\end{pgfscope}%
\begin{pgfscope}%
\pgfsys@transformshift{2.192902in}{0.701435in}%
\pgfsys@useobject{currentmarker}{}%
\end{pgfscope}%
\begin{pgfscope}%
\pgfsys@transformshift{2.193089in}{0.722462in}%
\pgfsys@useobject{currentmarker}{}%
\end{pgfscope}%
\begin{pgfscope}%
\pgfsys@transformshift{2.193276in}{0.723242in}%
\pgfsys@useobject{currentmarker}{}%
\end{pgfscope}%
\begin{pgfscope}%
\pgfsys@transformshift{2.193463in}{0.695910in}%
\pgfsys@useobject{currentmarker}{}%
\end{pgfscope}%
\begin{pgfscope}%
\pgfsys@transformshift{2.193649in}{0.704246in}%
\pgfsys@useobject{currentmarker}{}%
\end{pgfscope}%
\begin{pgfscope}%
\pgfsys@transformshift{2.193836in}{0.719939in}%
\pgfsys@useobject{currentmarker}{}%
\end{pgfscope}%
\begin{pgfscope}%
\pgfsys@transformshift{2.194022in}{0.701344in}%
\pgfsys@useobject{currentmarker}{}%
\end{pgfscope}%
\begin{pgfscope}%
\pgfsys@transformshift{2.194208in}{0.698389in}%
\pgfsys@useobject{currentmarker}{}%
\end{pgfscope}%
\begin{pgfscope}%
\pgfsys@transformshift{2.194394in}{0.700112in}%
\pgfsys@useobject{currentmarker}{}%
\end{pgfscope}%
\begin{pgfscope}%
\pgfsys@transformshift{2.194580in}{0.717927in}%
\pgfsys@useobject{currentmarker}{}%
\end{pgfscope}%
\begin{pgfscope}%
\pgfsys@transformshift{2.194765in}{0.760663in}%
\pgfsys@useobject{currentmarker}{}%
\end{pgfscope}%
\begin{pgfscope}%
\pgfsys@transformshift{2.194951in}{0.742220in}%
\pgfsys@useobject{currentmarker}{}%
\end{pgfscope}%
\begin{pgfscope}%
\pgfsys@transformshift{2.195136in}{0.712015in}%
\pgfsys@useobject{currentmarker}{}%
\end{pgfscope}%
\begin{pgfscope}%
\pgfsys@transformshift{2.195321in}{0.719726in}%
\pgfsys@useobject{currentmarker}{}%
\end{pgfscope}%
\begin{pgfscope}%
\pgfsys@transformshift{2.195506in}{0.718760in}%
\pgfsys@useobject{currentmarker}{}%
\end{pgfscope}%
\begin{pgfscope}%
\pgfsys@transformshift{2.195691in}{0.631928in}%
\pgfsys@useobject{currentmarker}{}%
\end{pgfscope}%
\begin{pgfscope}%
\pgfsys@transformshift{2.195875in}{0.687643in}%
\pgfsys@useobject{currentmarker}{}%
\end{pgfscope}%
\begin{pgfscope}%
\pgfsys@transformshift{2.196060in}{0.693508in}%
\pgfsys@useobject{currentmarker}{}%
\end{pgfscope}%
\begin{pgfscope}%
\pgfsys@transformshift{2.196244in}{0.722062in}%
\pgfsys@useobject{currentmarker}{}%
\end{pgfscope}%
\begin{pgfscope}%
\pgfsys@transformshift{2.196429in}{0.704035in}%
\pgfsys@useobject{currentmarker}{}%
\end{pgfscope}%
\begin{pgfscope}%
\pgfsys@transformshift{2.196613in}{0.699288in}%
\pgfsys@useobject{currentmarker}{}%
\end{pgfscope}%
\begin{pgfscope}%
\pgfsys@transformshift{2.196797in}{0.711622in}%
\pgfsys@useobject{currentmarker}{}%
\end{pgfscope}%
\begin{pgfscope}%
\pgfsys@transformshift{2.196980in}{0.692619in}%
\pgfsys@useobject{currentmarker}{}%
\end{pgfscope}%
\begin{pgfscope}%
\pgfsys@transformshift{2.197164in}{0.760304in}%
\pgfsys@useobject{currentmarker}{}%
\end{pgfscope}%
\begin{pgfscope}%
\pgfsys@transformshift{2.197347in}{0.746247in}%
\pgfsys@useobject{currentmarker}{}%
\end{pgfscope}%
\begin{pgfscope}%
\pgfsys@transformshift{2.197531in}{0.729386in}%
\pgfsys@useobject{currentmarker}{}%
\end{pgfscope}%
\begin{pgfscope}%
\pgfsys@transformshift{2.197714in}{0.722816in}%
\pgfsys@useobject{currentmarker}{}%
\end{pgfscope}%
\begin{pgfscope}%
\pgfsys@transformshift{2.197897in}{0.701341in}%
\pgfsys@useobject{currentmarker}{}%
\end{pgfscope}%
\begin{pgfscope}%
\pgfsys@transformshift{2.198080in}{0.669729in}%
\pgfsys@useobject{currentmarker}{}%
\end{pgfscope}%
\begin{pgfscope}%
\pgfsys@transformshift{2.198262in}{0.706661in}%
\pgfsys@useobject{currentmarker}{}%
\end{pgfscope}%
\begin{pgfscope}%
\pgfsys@transformshift{2.198445in}{0.715244in}%
\pgfsys@useobject{currentmarker}{}%
\end{pgfscope}%
\begin{pgfscope}%
\pgfsys@transformshift{2.198627in}{0.706333in}%
\pgfsys@useobject{currentmarker}{}%
\end{pgfscope}%
\begin{pgfscope}%
\pgfsys@transformshift{2.198810in}{0.690146in}%
\pgfsys@useobject{currentmarker}{}%
\end{pgfscope}%
\begin{pgfscope}%
\pgfsys@transformshift{2.198992in}{0.690807in}%
\pgfsys@useobject{currentmarker}{}%
\end{pgfscope}%
\begin{pgfscope}%
\pgfsys@transformshift{2.199174in}{0.700704in}%
\pgfsys@useobject{currentmarker}{}%
\end{pgfscope}%
\begin{pgfscope}%
\pgfsys@transformshift{2.199356in}{0.699940in}%
\pgfsys@useobject{currentmarker}{}%
\end{pgfscope}%
\begin{pgfscope}%
\pgfsys@transformshift{2.199537in}{0.709103in}%
\pgfsys@useobject{currentmarker}{}%
\end{pgfscope}%
\begin{pgfscope}%
\pgfsys@transformshift{2.199719in}{0.701384in}%
\pgfsys@useobject{currentmarker}{}%
\end{pgfscope}%
\begin{pgfscope}%
\pgfsys@transformshift{2.199900in}{0.684330in}%
\pgfsys@useobject{currentmarker}{}%
\end{pgfscope}%
\begin{pgfscope}%
\pgfsys@transformshift{2.200081in}{0.686280in}%
\pgfsys@useobject{currentmarker}{}%
\end{pgfscope}%
\begin{pgfscope}%
\pgfsys@transformshift{2.200263in}{0.709757in}%
\pgfsys@useobject{currentmarker}{}%
\end{pgfscope}%
\begin{pgfscope}%
\pgfsys@transformshift{2.200444in}{0.709704in}%
\pgfsys@useobject{currentmarker}{}%
\end{pgfscope}%
\begin{pgfscope}%
\pgfsys@transformshift{2.200624in}{0.698326in}%
\pgfsys@useobject{currentmarker}{}%
\end{pgfscope}%
\begin{pgfscope}%
\pgfsys@transformshift{2.200805in}{0.660877in}%
\pgfsys@useobject{currentmarker}{}%
\end{pgfscope}%
\begin{pgfscope}%
\pgfsys@transformshift{2.200986in}{0.684121in}%
\pgfsys@useobject{currentmarker}{}%
\end{pgfscope}%
\begin{pgfscope}%
\pgfsys@transformshift{2.201166in}{0.711412in}%
\pgfsys@useobject{currentmarker}{}%
\end{pgfscope}%
\begin{pgfscope}%
\pgfsys@transformshift{2.201346in}{0.647488in}%
\pgfsys@useobject{currentmarker}{}%
\end{pgfscope}%
\begin{pgfscope}%
\pgfsys@transformshift{2.201526in}{0.623108in}%
\pgfsys@useobject{currentmarker}{}%
\end{pgfscope}%
\begin{pgfscope}%
\pgfsys@transformshift{2.201706in}{0.636178in}%
\pgfsys@useobject{currentmarker}{}%
\end{pgfscope}%
\begin{pgfscope}%
\pgfsys@transformshift{2.201886in}{0.650636in}%
\pgfsys@useobject{currentmarker}{}%
\end{pgfscope}%
\begin{pgfscope}%
\pgfsys@transformshift{2.202066in}{0.681083in}%
\pgfsys@useobject{currentmarker}{}%
\end{pgfscope}%
\begin{pgfscope}%
\pgfsys@transformshift{2.202245in}{0.646135in}%
\pgfsys@useobject{currentmarker}{}%
\end{pgfscope}%
\begin{pgfscope}%
\pgfsys@transformshift{2.202424in}{0.657928in}%
\pgfsys@useobject{currentmarker}{}%
\end{pgfscope}%
\begin{pgfscope}%
\pgfsys@transformshift{2.202604in}{0.673066in}%
\pgfsys@useobject{currentmarker}{}%
\end{pgfscope}%
\begin{pgfscope}%
\pgfsys@transformshift{2.202783in}{0.690799in}%
\pgfsys@useobject{currentmarker}{}%
\end{pgfscope}%
\begin{pgfscope}%
\pgfsys@transformshift{2.202962in}{0.703526in}%
\pgfsys@useobject{currentmarker}{}%
\end{pgfscope}%
\begin{pgfscope}%
\pgfsys@transformshift{2.203140in}{0.686310in}%
\pgfsys@useobject{currentmarker}{}%
\end{pgfscope}%
\begin{pgfscope}%
\pgfsys@transformshift{2.203319in}{0.658153in}%
\pgfsys@useobject{currentmarker}{}%
\end{pgfscope}%
\begin{pgfscope}%
\pgfsys@transformshift{2.203498in}{0.684317in}%
\pgfsys@useobject{currentmarker}{}%
\end{pgfscope}%
\begin{pgfscope}%
\pgfsys@transformshift{2.203676in}{0.743454in}%
\pgfsys@useobject{currentmarker}{}%
\end{pgfscope}%
\begin{pgfscope}%
\pgfsys@transformshift{2.203854in}{0.738394in}%
\pgfsys@useobject{currentmarker}{}%
\end{pgfscope}%
\begin{pgfscope}%
\pgfsys@transformshift{2.204032in}{0.718487in}%
\pgfsys@useobject{currentmarker}{}%
\end{pgfscope}%
\begin{pgfscope}%
\pgfsys@transformshift{2.204210in}{0.654393in}%
\pgfsys@useobject{currentmarker}{}%
\end{pgfscope}%
\begin{pgfscope}%
\pgfsys@transformshift{2.204388in}{0.665074in}%
\pgfsys@useobject{currentmarker}{}%
\end{pgfscope}%
\begin{pgfscope}%
\pgfsys@transformshift{2.204566in}{0.707318in}%
\pgfsys@useobject{currentmarker}{}%
\end{pgfscope}%
\begin{pgfscope}%
\pgfsys@transformshift{2.204743in}{0.721140in}%
\pgfsys@useobject{currentmarker}{}%
\end{pgfscope}%
\begin{pgfscope}%
\pgfsys@transformshift{2.204921in}{0.629012in}%
\pgfsys@useobject{currentmarker}{}%
\end{pgfscope}%
\begin{pgfscope}%
\pgfsys@transformshift{2.205098in}{0.731811in}%
\pgfsys@useobject{currentmarker}{}%
\end{pgfscope}%
\begin{pgfscope}%
\pgfsys@transformshift{2.205275in}{0.717617in}%
\pgfsys@useobject{currentmarker}{}%
\end{pgfscope}%
\begin{pgfscope}%
\pgfsys@transformshift{2.205452in}{0.698380in}%
\pgfsys@useobject{currentmarker}{}%
\end{pgfscope}%
\begin{pgfscope}%
\pgfsys@transformshift{2.205629in}{0.710271in}%
\pgfsys@useobject{currentmarker}{}%
\end{pgfscope}%
\begin{pgfscope}%
\pgfsys@transformshift{2.205805in}{0.687162in}%
\pgfsys@useobject{currentmarker}{}%
\end{pgfscope}%
\begin{pgfscope}%
\pgfsys@transformshift{2.205982in}{0.665303in}%
\pgfsys@useobject{currentmarker}{}%
\end{pgfscope}%
\begin{pgfscope}%
\pgfsys@transformshift{2.206158in}{0.722979in}%
\pgfsys@useobject{currentmarker}{}%
\end{pgfscope}%
\begin{pgfscope}%
\pgfsys@transformshift{2.206335in}{0.722266in}%
\pgfsys@useobject{currentmarker}{}%
\end{pgfscope}%
\begin{pgfscope}%
\pgfsys@transformshift{2.206511in}{0.682677in}%
\pgfsys@useobject{currentmarker}{}%
\end{pgfscope}%
\begin{pgfscope}%
\pgfsys@transformshift{2.206687in}{0.659216in}%
\pgfsys@useobject{currentmarker}{}%
\end{pgfscope}%
\begin{pgfscope}%
\pgfsys@transformshift{2.206863in}{0.691140in}%
\pgfsys@useobject{currentmarker}{}%
\end{pgfscope}%
\begin{pgfscope}%
\pgfsys@transformshift{2.207038in}{0.690426in}%
\pgfsys@useobject{currentmarker}{}%
\end{pgfscope}%
\begin{pgfscope}%
\pgfsys@transformshift{2.207214in}{0.672113in}%
\pgfsys@useobject{currentmarker}{}%
\end{pgfscope}%
\begin{pgfscope}%
\pgfsys@transformshift{2.207389in}{0.724117in}%
\pgfsys@useobject{currentmarker}{}%
\end{pgfscope}%
\begin{pgfscope}%
\pgfsys@transformshift{2.207565in}{0.727842in}%
\pgfsys@useobject{currentmarker}{}%
\end{pgfscope}%
\begin{pgfscope}%
\pgfsys@transformshift{2.207740in}{0.654429in}%
\pgfsys@useobject{currentmarker}{}%
\end{pgfscope}%
\begin{pgfscope}%
\pgfsys@transformshift{2.207915in}{0.645144in}%
\pgfsys@useobject{currentmarker}{}%
\end{pgfscope}%
\begin{pgfscope}%
\pgfsys@transformshift{2.208090in}{0.645824in}%
\pgfsys@useobject{currentmarker}{}%
\end{pgfscope}%
\begin{pgfscope}%
\pgfsys@transformshift{2.208264in}{0.636517in}%
\pgfsys@useobject{currentmarker}{}%
\end{pgfscope}%
\begin{pgfscope}%
\pgfsys@transformshift{2.208439in}{0.674486in}%
\pgfsys@useobject{currentmarker}{}%
\end{pgfscope}%
\begin{pgfscope}%
\pgfsys@transformshift{2.208614in}{0.614702in}%
\pgfsys@useobject{currentmarker}{}%
\end{pgfscope}%
\begin{pgfscope}%
\pgfsys@transformshift{2.208788in}{0.655269in}%
\pgfsys@useobject{currentmarker}{}%
\end{pgfscope}%
\begin{pgfscope}%
\pgfsys@transformshift{2.208962in}{0.665779in}%
\pgfsys@useobject{currentmarker}{}%
\end{pgfscope}%
\begin{pgfscope}%
\pgfsys@transformshift{2.209136in}{0.690538in}%
\pgfsys@useobject{currentmarker}{}%
\end{pgfscope}%
\begin{pgfscope}%
\pgfsys@transformshift{2.209310in}{0.706981in}%
\pgfsys@useobject{currentmarker}{}%
\end{pgfscope}%
\begin{pgfscope}%
\pgfsys@transformshift{2.209484in}{0.701144in}%
\pgfsys@useobject{currentmarker}{}%
\end{pgfscope}%
\begin{pgfscope}%
\pgfsys@transformshift{2.209658in}{0.699004in}%
\pgfsys@useobject{currentmarker}{}%
\end{pgfscope}%
\begin{pgfscope}%
\pgfsys@transformshift{2.209831in}{0.670171in}%
\pgfsys@useobject{currentmarker}{}%
\end{pgfscope}%
\begin{pgfscope}%
\pgfsys@transformshift{2.210005in}{0.691461in}%
\pgfsys@useobject{currentmarker}{}%
\end{pgfscope}%
\begin{pgfscope}%
\pgfsys@transformshift{2.210178in}{0.664451in}%
\pgfsys@useobject{currentmarker}{}%
\end{pgfscope}%
\begin{pgfscope}%
\pgfsys@transformshift{2.210351in}{0.637785in}%
\pgfsys@useobject{currentmarker}{}%
\end{pgfscope}%
\begin{pgfscope}%
\pgfsys@transformshift{2.210524in}{0.658172in}%
\pgfsys@useobject{currentmarker}{}%
\end{pgfscope}%
\begin{pgfscope}%
\pgfsys@transformshift{2.210697in}{0.662022in}%
\pgfsys@useobject{currentmarker}{}%
\end{pgfscope}%
\begin{pgfscope}%
\pgfsys@transformshift{2.210870in}{0.655558in}%
\pgfsys@useobject{currentmarker}{}%
\end{pgfscope}%
\begin{pgfscope}%
\pgfsys@transformshift{2.211042in}{0.665558in}%
\pgfsys@useobject{currentmarker}{}%
\end{pgfscope}%
\begin{pgfscope}%
\pgfsys@transformshift{2.211215in}{0.703880in}%
\pgfsys@useobject{currentmarker}{}%
\end{pgfscope}%
\begin{pgfscope}%
\pgfsys@transformshift{2.211387in}{0.698403in}%
\pgfsys@useobject{currentmarker}{}%
\end{pgfscope}%
\begin{pgfscope}%
\pgfsys@transformshift{2.211559in}{0.654520in}%
\pgfsys@useobject{currentmarker}{}%
\end{pgfscope}%
\begin{pgfscope}%
\pgfsys@transformshift{2.211731in}{0.651175in}%
\pgfsys@useobject{currentmarker}{}%
\end{pgfscope}%
\begin{pgfscope}%
\pgfsys@transformshift{2.211903in}{0.642618in}%
\pgfsys@useobject{currentmarker}{}%
\end{pgfscope}%
\begin{pgfscope}%
\pgfsys@transformshift{2.212075in}{0.604438in}%
\pgfsys@useobject{currentmarker}{}%
\end{pgfscope}%
\begin{pgfscope}%
\pgfsys@transformshift{2.212247in}{0.676310in}%
\pgfsys@useobject{currentmarker}{}%
\end{pgfscope}%
\begin{pgfscope}%
\pgfsys@transformshift{2.212418in}{0.692149in}%
\pgfsys@useobject{currentmarker}{}%
\end{pgfscope}%
\begin{pgfscope}%
\pgfsys@transformshift{2.212590in}{0.678494in}%
\pgfsys@useobject{currentmarker}{}%
\end{pgfscope}%
\begin{pgfscope}%
\pgfsys@transformshift{2.212761in}{0.663655in}%
\pgfsys@useobject{currentmarker}{}%
\end{pgfscope}%
\begin{pgfscope}%
\pgfsys@transformshift{2.212932in}{0.709323in}%
\pgfsys@useobject{currentmarker}{}%
\end{pgfscope}%
\begin{pgfscope}%
\pgfsys@transformshift{2.213103in}{0.707774in}%
\pgfsys@useobject{currentmarker}{}%
\end{pgfscope}%
\begin{pgfscope}%
\pgfsys@transformshift{2.213274in}{0.685196in}%
\pgfsys@useobject{currentmarker}{}%
\end{pgfscope}%
\begin{pgfscope}%
\pgfsys@transformshift{2.213445in}{0.715490in}%
\pgfsys@useobject{currentmarker}{}%
\end{pgfscope}%
\begin{pgfscope}%
\pgfsys@transformshift{2.213616in}{0.685996in}%
\pgfsys@useobject{currentmarker}{}%
\end{pgfscope}%
\begin{pgfscope}%
\pgfsys@transformshift{2.213786in}{0.690508in}%
\pgfsys@useobject{currentmarker}{}%
\end{pgfscope}%
\begin{pgfscope}%
\pgfsys@transformshift{2.213957in}{0.689029in}%
\pgfsys@useobject{currentmarker}{}%
\end{pgfscope}%
\begin{pgfscope}%
\pgfsys@transformshift{2.214127in}{0.695985in}%
\pgfsys@useobject{currentmarker}{}%
\end{pgfscope}%
\begin{pgfscope}%
\pgfsys@transformshift{2.214297in}{0.651903in}%
\pgfsys@useobject{currentmarker}{}%
\end{pgfscope}%
\begin{pgfscope}%
\pgfsys@transformshift{2.214467in}{0.661715in}%
\pgfsys@useobject{currentmarker}{}%
\end{pgfscope}%
\begin{pgfscope}%
\pgfsys@transformshift{2.214637in}{0.727454in}%
\pgfsys@useobject{currentmarker}{}%
\end{pgfscope}%
\begin{pgfscope}%
\pgfsys@transformshift{2.214807in}{0.703116in}%
\pgfsys@useobject{currentmarker}{}%
\end{pgfscope}%
\begin{pgfscope}%
\pgfsys@transformshift{2.214976in}{0.672621in}%
\pgfsys@useobject{currentmarker}{}%
\end{pgfscope}%
\begin{pgfscope}%
\pgfsys@transformshift{2.215146in}{0.670440in}%
\pgfsys@useobject{currentmarker}{}%
\end{pgfscope}%
\begin{pgfscope}%
\pgfsys@transformshift{2.215315in}{0.688599in}%
\pgfsys@useobject{currentmarker}{}%
\end{pgfscope}%
\begin{pgfscope}%
\pgfsys@transformshift{2.215484in}{0.676720in}%
\pgfsys@useobject{currentmarker}{}%
\end{pgfscope}%
\begin{pgfscope}%
\pgfsys@transformshift{2.215653in}{0.725742in}%
\pgfsys@useobject{currentmarker}{}%
\end{pgfscope}%
\begin{pgfscope}%
\pgfsys@transformshift{2.215822in}{0.729649in}%
\pgfsys@useobject{currentmarker}{}%
\end{pgfscope}%
\begin{pgfscope}%
\pgfsys@transformshift{2.215991in}{0.687232in}%
\pgfsys@useobject{currentmarker}{}%
\end{pgfscope}%
\begin{pgfscope}%
\pgfsys@transformshift{2.216160in}{0.659842in}%
\pgfsys@useobject{currentmarker}{}%
\end{pgfscope}%
\begin{pgfscope}%
\pgfsys@transformshift{2.216328in}{0.694122in}%
\pgfsys@useobject{currentmarker}{}%
\end{pgfscope}%
\begin{pgfscope}%
\pgfsys@transformshift{2.216497in}{0.681836in}%
\pgfsys@useobject{currentmarker}{}%
\end{pgfscope}%
\begin{pgfscope}%
\pgfsys@transformshift{2.216665in}{0.722855in}%
\pgfsys@useobject{currentmarker}{}%
\end{pgfscope}%
\begin{pgfscope}%
\pgfsys@transformshift{2.216833in}{0.713131in}%
\pgfsys@useobject{currentmarker}{}%
\end{pgfscope}%
\begin{pgfscope}%
\pgfsys@transformshift{2.217002in}{0.686684in}%
\pgfsys@useobject{currentmarker}{}%
\end{pgfscope}%
\begin{pgfscope}%
\pgfsys@transformshift{2.217170in}{0.670298in}%
\pgfsys@useobject{currentmarker}{}%
\end{pgfscope}%
\begin{pgfscope}%
\pgfsys@transformshift{2.217337in}{0.691544in}%
\pgfsys@useobject{currentmarker}{}%
\end{pgfscope}%
\begin{pgfscope}%
\pgfsys@transformshift{2.217505in}{0.707442in}%
\pgfsys@useobject{currentmarker}{}%
\end{pgfscope}%
\begin{pgfscope}%
\pgfsys@transformshift{2.217673in}{0.701476in}%
\pgfsys@useobject{currentmarker}{}%
\end{pgfscope}%
\begin{pgfscope}%
\pgfsys@transformshift{2.217840in}{0.742355in}%
\pgfsys@useobject{currentmarker}{}%
\end{pgfscope}%
\begin{pgfscope}%
\pgfsys@transformshift{2.218007in}{0.697654in}%
\pgfsys@useobject{currentmarker}{}%
\end{pgfscope}%
\begin{pgfscope}%
\pgfsys@transformshift{2.218175in}{0.662512in}%
\pgfsys@useobject{currentmarker}{}%
\end{pgfscope}%
\begin{pgfscope}%
\pgfsys@transformshift{2.218342in}{0.647427in}%
\pgfsys@useobject{currentmarker}{}%
\end{pgfscope}%
\begin{pgfscope}%
\pgfsys@transformshift{2.218509in}{0.586686in}%
\pgfsys@useobject{currentmarker}{}%
\end{pgfscope}%
\begin{pgfscope}%
\pgfsys@transformshift{2.218676in}{0.672205in}%
\pgfsys@useobject{currentmarker}{}%
\end{pgfscope}%
\begin{pgfscope}%
\pgfsys@transformshift{2.218842in}{0.658616in}%
\pgfsys@useobject{currentmarker}{}%
\end{pgfscope}%
\begin{pgfscope}%
\pgfsys@transformshift{2.219009in}{0.697068in}%
\pgfsys@useobject{currentmarker}{}%
\end{pgfscope}%
\begin{pgfscope}%
\pgfsys@transformshift{2.219175in}{0.679150in}%
\pgfsys@useobject{currentmarker}{}%
\end{pgfscope}%
\begin{pgfscope}%
\pgfsys@transformshift{2.219342in}{0.718839in}%
\pgfsys@useobject{currentmarker}{}%
\end{pgfscope}%
\begin{pgfscope}%
\pgfsys@transformshift{2.219508in}{0.751922in}%
\pgfsys@useobject{currentmarker}{}%
\end{pgfscope}%
\begin{pgfscope}%
\pgfsys@transformshift{2.219674in}{0.661995in}%
\pgfsys@useobject{currentmarker}{}%
\end{pgfscope}%
\begin{pgfscope}%
\pgfsys@transformshift{2.219840in}{0.678927in}%
\pgfsys@useobject{currentmarker}{}%
\end{pgfscope}%
\begin{pgfscope}%
\pgfsys@transformshift{2.220006in}{0.716956in}%
\pgfsys@useobject{currentmarker}{}%
\end{pgfscope}%
\begin{pgfscope}%
\pgfsys@transformshift{2.220172in}{0.730325in}%
\pgfsys@useobject{currentmarker}{}%
\end{pgfscope}%
\begin{pgfscope}%
\pgfsys@transformshift{2.220337in}{0.698822in}%
\pgfsys@useobject{currentmarker}{}%
\end{pgfscope}%
\begin{pgfscope}%
\pgfsys@transformshift{2.220503in}{0.652438in}%
\pgfsys@useobject{currentmarker}{}%
\end{pgfscope}%
\begin{pgfscope}%
\pgfsys@transformshift{2.220668in}{0.658983in}%
\pgfsys@useobject{currentmarker}{}%
\end{pgfscope}%
\begin{pgfscope}%
\pgfsys@transformshift{2.220833in}{0.662532in}%
\pgfsys@useobject{currentmarker}{}%
\end{pgfscope}%
\begin{pgfscope}%
\pgfsys@transformshift{2.220998in}{0.661507in}%
\pgfsys@useobject{currentmarker}{}%
\end{pgfscope}%
\begin{pgfscope}%
\pgfsys@transformshift{2.221163in}{0.691979in}%
\pgfsys@useobject{currentmarker}{}%
\end{pgfscope}%
\begin{pgfscope}%
\pgfsys@transformshift{2.221328in}{0.684149in}%
\pgfsys@useobject{currentmarker}{}%
\end{pgfscope}%
\begin{pgfscope}%
\pgfsys@transformshift{2.221493in}{0.676064in}%
\pgfsys@useobject{currentmarker}{}%
\end{pgfscope}%
\begin{pgfscope}%
\pgfsys@transformshift{2.221658in}{0.737285in}%
\pgfsys@useobject{currentmarker}{}%
\end{pgfscope}%
\begin{pgfscope}%
\pgfsys@transformshift{2.221822in}{0.695028in}%
\pgfsys@useobject{currentmarker}{}%
\end{pgfscope}%
\begin{pgfscope}%
\pgfsys@transformshift{2.221987in}{0.644811in}%
\pgfsys@useobject{currentmarker}{}%
\end{pgfscope}%
\begin{pgfscope}%
\pgfsys@transformshift{2.222151in}{0.661033in}%
\pgfsys@useobject{currentmarker}{}%
\end{pgfscope}%
\begin{pgfscope}%
\pgfsys@transformshift{2.222315in}{0.656753in}%
\pgfsys@useobject{currentmarker}{}%
\end{pgfscope}%
\begin{pgfscope}%
\pgfsys@transformshift{2.222479in}{0.659268in}%
\pgfsys@useobject{currentmarker}{}%
\end{pgfscope}%
\begin{pgfscope}%
\pgfsys@transformshift{2.222643in}{0.716246in}%
\pgfsys@useobject{currentmarker}{}%
\end{pgfscope}%
\begin{pgfscope}%
\pgfsys@transformshift{2.222807in}{0.726481in}%
\pgfsys@useobject{currentmarker}{}%
\end{pgfscope}%
\begin{pgfscope}%
\pgfsys@transformshift{2.222971in}{0.715042in}%
\pgfsys@useobject{currentmarker}{}%
\end{pgfscope}%
\begin{pgfscope}%
\pgfsys@transformshift{2.223134in}{0.700755in}%
\pgfsys@useobject{currentmarker}{}%
\end{pgfscope}%
\begin{pgfscope}%
\pgfsys@transformshift{2.223298in}{0.723611in}%
\pgfsys@useobject{currentmarker}{}%
\end{pgfscope}%
\begin{pgfscope}%
\pgfsys@transformshift{2.223461in}{0.713462in}%
\pgfsys@useobject{currentmarker}{}%
\end{pgfscope}%
\begin{pgfscope}%
\pgfsys@transformshift{2.223624in}{0.656997in}%
\pgfsys@useobject{currentmarker}{}%
\end{pgfscope}%
\begin{pgfscope}%
\pgfsys@transformshift{2.223787in}{0.685790in}%
\pgfsys@useobject{currentmarker}{}%
\end{pgfscope}%
\begin{pgfscope}%
\pgfsys@transformshift{2.223950in}{0.677426in}%
\pgfsys@useobject{currentmarker}{}%
\end{pgfscope}%
\begin{pgfscope}%
\pgfsys@transformshift{2.224113in}{0.701467in}%
\pgfsys@useobject{currentmarker}{}%
\end{pgfscope}%
\begin{pgfscope}%
\pgfsys@transformshift{2.224276in}{0.737747in}%
\pgfsys@useobject{currentmarker}{}%
\end{pgfscope}%
\begin{pgfscope}%
\pgfsys@transformshift{2.224438in}{0.756439in}%
\pgfsys@useobject{currentmarker}{}%
\end{pgfscope}%
\begin{pgfscope}%
\pgfsys@transformshift{2.224601in}{0.703901in}%
\pgfsys@useobject{currentmarker}{}%
\end{pgfscope}%
\begin{pgfscope}%
\pgfsys@transformshift{2.224763in}{0.681114in}%
\pgfsys@useobject{currentmarker}{}%
\end{pgfscope}%
\begin{pgfscope}%
\pgfsys@transformshift{2.224925in}{0.683200in}%
\pgfsys@useobject{currentmarker}{}%
\end{pgfscope}%
\begin{pgfscope}%
\pgfsys@transformshift{2.225088in}{0.702248in}%
\pgfsys@useobject{currentmarker}{}%
\end{pgfscope}%
\begin{pgfscope}%
\pgfsys@transformshift{2.225250in}{0.700620in}%
\pgfsys@useobject{currentmarker}{}%
\end{pgfscope}%
\begin{pgfscope}%
\pgfsys@transformshift{2.225412in}{0.645747in}%
\pgfsys@useobject{currentmarker}{}%
\end{pgfscope}%
\begin{pgfscope}%
\pgfsys@transformshift{2.225573in}{0.719457in}%
\pgfsys@useobject{currentmarker}{}%
\end{pgfscope}%
\begin{pgfscope}%
\pgfsys@transformshift{2.225735in}{0.720150in}%
\pgfsys@useobject{currentmarker}{}%
\end{pgfscope}%
\begin{pgfscope}%
\pgfsys@transformshift{2.225897in}{0.705578in}%
\pgfsys@useobject{currentmarker}{}%
\end{pgfscope}%
\begin{pgfscope}%
\pgfsys@transformshift{2.226058in}{0.677766in}%
\pgfsys@useobject{currentmarker}{}%
\end{pgfscope}%
\begin{pgfscope}%
\pgfsys@transformshift{2.226219in}{0.658175in}%
\pgfsys@useobject{currentmarker}{}%
\end{pgfscope}%
\begin{pgfscope}%
\pgfsys@transformshift{2.226381in}{0.684897in}%
\pgfsys@useobject{currentmarker}{}%
\end{pgfscope}%
\begin{pgfscope}%
\pgfsys@transformshift{2.226542in}{0.700898in}%
\pgfsys@useobject{currentmarker}{}%
\end{pgfscope}%
\begin{pgfscope}%
\pgfsys@transformshift{2.226703in}{0.620834in}%
\pgfsys@useobject{currentmarker}{}%
\end{pgfscope}%
\begin{pgfscope}%
\pgfsys@transformshift{2.226863in}{0.638387in}%
\pgfsys@useobject{currentmarker}{}%
\end{pgfscope}%
\begin{pgfscope}%
\pgfsys@transformshift{2.227024in}{0.654146in}%
\pgfsys@useobject{currentmarker}{}%
\end{pgfscope}%
\begin{pgfscope}%
\pgfsys@transformshift{2.227185in}{0.670980in}%
\pgfsys@useobject{currentmarker}{}%
\end{pgfscope}%
\begin{pgfscope}%
\pgfsys@transformshift{2.227345in}{0.664999in}%
\pgfsys@useobject{currentmarker}{}%
\end{pgfscope}%
\begin{pgfscope}%
\pgfsys@transformshift{2.227506in}{0.677720in}%
\pgfsys@useobject{currentmarker}{}%
\end{pgfscope}%
\begin{pgfscope}%
\pgfsys@transformshift{2.227666in}{0.683698in}%
\pgfsys@useobject{currentmarker}{}%
\end{pgfscope}%
\begin{pgfscope}%
\pgfsys@transformshift{2.227826in}{0.663858in}%
\pgfsys@useobject{currentmarker}{}%
\end{pgfscope}%
\begin{pgfscope}%
\pgfsys@transformshift{2.227986in}{0.658817in}%
\pgfsys@useobject{currentmarker}{}%
\end{pgfscope}%
\begin{pgfscope}%
\pgfsys@transformshift{2.228146in}{0.649575in}%
\pgfsys@useobject{currentmarker}{}%
\end{pgfscope}%
\begin{pgfscope}%
\pgfsys@transformshift{2.228306in}{0.694644in}%
\pgfsys@useobject{currentmarker}{}%
\end{pgfscope}%
\begin{pgfscope}%
\pgfsys@transformshift{2.228466in}{0.664077in}%
\pgfsys@useobject{currentmarker}{}%
\end{pgfscope}%
\begin{pgfscope}%
\pgfsys@transformshift{2.228625in}{0.692586in}%
\pgfsys@useobject{currentmarker}{}%
\end{pgfscope}%
\begin{pgfscope}%
\pgfsys@transformshift{2.228785in}{0.683773in}%
\pgfsys@useobject{currentmarker}{}%
\end{pgfscope}%
\begin{pgfscope}%
\pgfsys@transformshift{2.228944in}{0.722397in}%
\pgfsys@useobject{currentmarker}{}%
\end{pgfscope}%
\begin{pgfscope}%
\pgfsys@transformshift{2.229104in}{0.737867in}%
\pgfsys@useobject{currentmarker}{}%
\end{pgfscope}%
\begin{pgfscope}%
\pgfsys@transformshift{2.229263in}{0.748168in}%
\pgfsys@useobject{currentmarker}{}%
\end{pgfscope}%
\begin{pgfscope}%
\pgfsys@transformshift{2.229422in}{0.755335in}%
\pgfsys@useobject{currentmarker}{}%
\end{pgfscope}%
\begin{pgfscope}%
\pgfsys@transformshift{2.229581in}{0.745430in}%
\pgfsys@useobject{currentmarker}{}%
\end{pgfscope}%
\begin{pgfscope}%
\pgfsys@transformshift{2.229740in}{0.752728in}%
\pgfsys@useobject{currentmarker}{}%
\end{pgfscope}%
\begin{pgfscope}%
\pgfsys@transformshift{2.229898in}{0.722353in}%
\pgfsys@useobject{currentmarker}{}%
\end{pgfscope}%
\begin{pgfscope}%
\pgfsys@transformshift{2.230057in}{0.676680in}%
\pgfsys@useobject{currentmarker}{}%
\end{pgfscope}%
\begin{pgfscope}%
\pgfsys@transformshift{2.230215in}{0.694917in}%
\pgfsys@useobject{currentmarker}{}%
\end{pgfscope}%
\begin{pgfscope}%
\pgfsys@transformshift{2.230374in}{0.727930in}%
\pgfsys@useobject{currentmarker}{}%
\end{pgfscope}%
\begin{pgfscope}%
\pgfsys@transformshift{2.230532in}{0.660147in}%
\pgfsys@useobject{currentmarker}{}%
\end{pgfscope}%
\begin{pgfscope}%
\pgfsys@transformshift{2.230690in}{0.707979in}%
\pgfsys@useobject{currentmarker}{}%
\end{pgfscope}%
\begin{pgfscope}%
\pgfsys@transformshift{2.230848in}{0.695276in}%
\pgfsys@useobject{currentmarker}{}%
\end{pgfscope}%
\begin{pgfscope}%
\pgfsys@transformshift{2.231006in}{0.649705in}%
\pgfsys@useobject{currentmarker}{}%
\end{pgfscope}%
\begin{pgfscope}%
\pgfsys@transformshift{2.231164in}{0.704161in}%
\pgfsys@useobject{currentmarker}{}%
\end{pgfscope}%
\begin{pgfscope}%
\pgfsys@transformshift{2.231322in}{0.713497in}%
\pgfsys@useobject{currentmarker}{}%
\end{pgfscope}%
\begin{pgfscope}%
\pgfsys@transformshift{2.231479in}{0.672658in}%
\pgfsys@useobject{currentmarker}{}%
\end{pgfscope}%
\begin{pgfscope}%
\pgfsys@transformshift{2.231637in}{0.661265in}%
\pgfsys@useobject{currentmarker}{}%
\end{pgfscope}%
\begin{pgfscope}%
\pgfsys@transformshift{2.231794in}{0.689194in}%
\pgfsys@useobject{currentmarker}{}%
\end{pgfscope}%
\begin{pgfscope}%
\pgfsys@transformshift{2.231951in}{0.689508in}%
\pgfsys@useobject{currentmarker}{}%
\end{pgfscope}%
\begin{pgfscope}%
\pgfsys@transformshift{2.232109in}{0.687662in}%
\pgfsys@useobject{currentmarker}{}%
\end{pgfscope}%
\begin{pgfscope}%
\pgfsys@transformshift{2.232266in}{0.683328in}%
\pgfsys@useobject{currentmarker}{}%
\end{pgfscope}%
\begin{pgfscope}%
\pgfsys@transformshift{2.232423in}{0.661486in}%
\pgfsys@useobject{currentmarker}{}%
\end{pgfscope}%
\begin{pgfscope}%
\pgfsys@transformshift{2.232579in}{0.679501in}%
\pgfsys@useobject{currentmarker}{}%
\end{pgfscope}%
\begin{pgfscope}%
\pgfsys@transformshift{2.232736in}{0.681998in}%
\pgfsys@useobject{currentmarker}{}%
\end{pgfscope}%
\begin{pgfscope}%
\pgfsys@transformshift{2.232893in}{0.711393in}%
\pgfsys@useobject{currentmarker}{}%
\end{pgfscope}%
\begin{pgfscope}%
\pgfsys@transformshift{2.233049in}{0.704705in}%
\pgfsys@useobject{currentmarker}{}%
\end{pgfscope}%
\begin{pgfscope}%
\pgfsys@transformshift{2.233206in}{0.668906in}%
\pgfsys@useobject{currentmarker}{}%
\end{pgfscope}%
\begin{pgfscope}%
\pgfsys@transformshift{2.233362in}{0.699601in}%
\pgfsys@useobject{currentmarker}{}%
\end{pgfscope}%
\begin{pgfscope}%
\pgfsys@transformshift{2.233518in}{0.675201in}%
\pgfsys@useobject{currentmarker}{}%
\end{pgfscope}%
\begin{pgfscope}%
\pgfsys@transformshift{2.233674in}{0.667717in}%
\pgfsys@useobject{currentmarker}{}%
\end{pgfscope}%
\begin{pgfscope}%
\pgfsys@transformshift{2.233830in}{0.657621in}%
\pgfsys@useobject{currentmarker}{}%
\end{pgfscope}%
\begin{pgfscope}%
\pgfsys@transformshift{2.233986in}{0.666936in}%
\pgfsys@useobject{currentmarker}{}%
\end{pgfscope}%
\begin{pgfscope}%
\pgfsys@transformshift{2.234142in}{0.675027in}%
\pgfsys@useobject{currentmarker}{}%
\end{pgfscope}%
\begin{pgfscope}%
\pgfsys@transformshift{2.234297in}{0.717864in}%
\pgfsys@useobject{currentmarker}{}%
\end{pgfscope}%
\begin{pgfscope}%
\pgfsys@transformshift{2.234453in}{0.721882in}%
\pgfsys@useobject{currentmarker}{}%
\end{pgfscope}%
\begin{pgfscope}%
\pgfsys@transformshift{2.234608in}{0.728866in}%
\pgfsys@useobject{currentmarker}{}%
\end{pgfscope}%
\begin{pgfscope}%
\pgfsys@transformshift{2.234763in}{0.706888in}%
\pgfsys@useobject{currentmarker}{}%
\end{pgfscope}%
\begin{pgfscope}%
\pgfsys@transformshift{2.234919in}{0.679311in}%
\pgfsys@useobject{currentmarker}{}%
\end{pgfscope}%
\begin{pgfscope}%
\pgfsys@transformshift{2.235074in}{0.629745in}%
\pgfsys@useobject{currentmarker}{}%
\end{pgfscope}%
\begin{pgfscope}%
\pgfsys@transformshift{2.235229in}{0.732518in}%
\pgfsys@useobject{currentmarker}{}%
\end{pgfscope}%
\begin{pgfscope}%
\pgfsys@transformshift{2.235384in}{0.734973in}%
\pgfsys@useobject{currentmarker}{}%
\end{pgfscope}%
\begin{pgfscope}%
\pgfsys@transformshift{2.235538in}{0.636745in}%
\pgfsys@useobject{currentmarker}{}%
\end{pgfscope}%
\begin{pgfscope}%
\pgfsys@transformshift{2.235693in}{0.643513in}%
\pgfsys@useobject{currentmarker}{}%
\end{pgfscope}%
\begin{pgfscope}%
\pgfsys@transformshift{2.235848in}{0.658157in}%
\pgfsys@useobject{currentmarker}{}%
\end{pgfscope}%
\begin{pgfscope}%
\pgfsys@transformshift{2.236002in}{0.645197in}%
\pgfsys@useobject{currentmarker}{}%
\end{pgfscope}%
\begin{pgfscope}%
\pgfsys@transformshift{2.236156in}{0.705928in}%
\pgfsys@useobject{currentmarker}{}%
\end{pgfscope}%
\begin{pgfscope}%
\pgfsys@transformshift{2.236311in}{0.690141in}%
\pgfsys@useobject{currentmarker}{}%
\end{pgfscope}%
\begin{pgfscope}%
\pgfsys@transformshift{2.236465in}{0.685857in}%
\pgfsys@useobject{currentmarker}{}%
\end{pgfscope}%
\begin{pgfscope}%
\pgfsys@transformshift{2.236619in}{0.639104in}%
\pgfsys@useobject{currentmarker}{}%
\end{pgfscope}%
\begin{pgfscope}%
\pgfsys@transformshift{2.236773in}{0.668784in}%
\pgfsys@useobject{currentmarker}{}%
\end{pgfscope}%
\begin{pgfscope}%
\pgfsys@transformshift{2.236927in}{0.650435in}%
\pgfsys@useobject{currentmarker}{}%
\end{pgfscope}%
\begin{pgfscope}%
\pgfsys@transformshift{2.237080in}{0.660918in}%
\pgfsys@useobject{currentmarker}{}%
\end{pgfscope}%
\begin{pgfscope}%
\pgfsys@transformshift{2.237234in}{0.651859in}%
\pgfsys@useobject{currentmarker}{}%
\end{pgfscope}%
\begin{pgfscope}%
\pgfsys@transformshift{2.237387in}{0.670603in}%
\pgfsys@useobject{currentmarker}{}%
\end{pgfscope}%
\begin{pgfscope}%
\pgfsys@transformshift{2.237541in}{0.705735in}%
\pgfsys@useobject{currentmarker}{}%
\end{pgfscope}%
\begin{pgfscope}%
\pgfsys@transformshift{2.237694in}{0.675600in}%
\pgfsys@useobject{currentmarker}{}%
\end{pgfscope}%
\begin{pgfscope}%
\pgfsys@transformshift{2.237847in}{0.666518in}%
\pgfsys@useobject{currentmarker}{}%
\end{pgfscope}%
\begin{pgfscope}%
\pgfsys@transformshift{2.238000in}{0.697076in}%
\pgfsys@useobject{currentmarker}{}%
\end{pgfscope}%
\begin{pgfscope}%
\pgfsys@transformshift{2.238153in}{0.654820in}%
\pgfsys@useobject{currentmarker}{}%
\end{pgfscope}%
\begin{pgfscope}%
\pgfsys@transformshift{2.238306in}{0.653475in}%
\pgfsys@useobject{currentmarker}{}%
\end{pgfscope}%
\begin{pgfscope}%
\pgfsys@transformshift{2.238459in}{0.700927in}%
\pgfsys@useobject{currentmarker}{}%
\end{pgfscope}%
\begin{pgfscope}%
\pgfsys@transformshift{2.238612in}{0.676631in}%
\pgfsys@useobject{currentmarker}{}%
\end{pgfscope}%
\begin{pgfscope}%
\pgfsys@transformshift{2.238764in}{0.698705in}%
\pgfsys@useobject{currentmarker}{}%
\end{pgfscope}%
\begin{pgfscope}%
\pgfsys@transformshift{2.238917in}{0.716267in}%
\pgfsys@useobject{currentmarker}{}%
\end{pgfscope}%
\begin{pgfscope}%
\pgfsys@transformshift{2.239069in}{0.735411in}%
\pgfsys@useobject{currentmarker}{}%
\end{pgfscope}%
\begin{pgfscope}%
\pgfsys@transformshift{2.239221in}{0.689838in}%
\pgfsys@useobject{currentmarker}{}%
\end{pgfscope}%
\begin{pgfscope}%
\pgfsys@transformshift{2.239373in}{0.697758in}%
\pgfsys@useobject{currentmarker}{}%
\end{pgfscope}%
\begin{pgfscope}%
\pgfsys@transformshift{2.239525in}{0.713092in}%
\pgfsys@useobject{currentmarker}{}%
\end{pgfscope}%
\begin{pgfscope}%
\pgfsys@transformshift{2.239677in}{0.742158in}%
\pgfsys@useobject{currentmarker}{}%
\end{pgfscope}%
\begin{pgfscope}%
\pgfsys@transformshift{2.239829in}{0.725152in}%
\pgfsys@useobject{currentmarker}{}%
\end{pgfscope}%
\begin{pgfscope}%
\pgfsys@transformshift{2.239981in}{0.653973in}%
\pgfsys@useobject{currentmarker}{}%
\end{pgfscope}%
\begin{pgfscope}%
\pgfsys@transformshift{2.240133in}{0.638739in}%
\pgfsys@useobject{currentmarker}{}%
\end{pgfscope}%
\begin{pgfscope}%
\pgfsys@transformshift{2.240284in}{0.643816in}%
\pgfsys@useobject{currentmarker}{}%
\end{pgfscope}%
\begin{pgfscope}%
\pgfsys@transformshift{2.240436in}{0.598254in}%
\pgfsys@useobject{currentmarker}{}%
\end{pgfscope}%
\begin{pgfscope}%
\pgfsys@transformshift{2.240587in}{0.653877in}%
\pgfsys@useobject{currentmarker}{}%
\end{pgfscope}%
\begin{pgfscope}%
\pgfsys@transformshift{2.240738in}{0.641419in}%
\pgfsys@useobject{currentmarker}{}%
\end{pgfscope}%
\begin{pgfscope}%
\pgfsys@transformshift{2.240889in}{0.698568in}%
\pgfsys@useobject{currentmarker}{}%
\end{pgfscope}%
\begin{pgfscope}%
\pgfsys@transformshift{2.241040in}{0.689027in}%
\pgfsys@useobject{currentmarker}{}%
\end{pgfscope}%
\begin{pgfscope}%
\pgfsys@transformshift{2.241191in}{0.717154in}%
\pgfsys@useobject{currentmarker}{}%
\end{pgfscope}%
\begin{pgfscope}%
\pgfsys@transformshift{2.241342in}{0.742262in}%
\pgfsys@useobject{currentmarker}{}%
\end{pgfscope}%
\begin{pgfscope}%
\pgfsys@transformshift{2.241493in}{0.724503in}%
\pgfsys@useobject{currentmarker}{}%
\end{pgfscope}%
\begin{pgfscope}%
\pgfsys@transformshift{2.241643in}{0.702501in}%
\pgfsys@useobject{currentmarker}{}%
\end{pgfscope}%
\begin{pgfscope}%
\pgfsys@transformshift{2.241794in}{0.706021in}%
\pgfsys@useobject{currentmarker}{}%
\end{pgfscope}%
\begin{pgfscope}%
\pgfsys@transformshift{2.241944in}{0.685025in}%
\pgfsys@useobject{currentmarker}{}%
\end{pgfscope}%
\begin{pgfscope}%
\pgfsys@transformshift{2.242095in}{0.645130in}%
\pgfsys@useobject{currentmarker}{}%
\end{pgfscope}%
\begin{pgfscope}%
\pgfsys@transformshift{2.242245in}{0.637856in}%
\pgfsys@useobject{currentmarker}{}%
\end{pgfscope}%
\begin{pgfscope}%
\pgfsys@transformshift{2.242395in}{0.683687in}%
\pgfsys@useobject{currentmarker}{}%
\end{pgfscope}%
\begin{pgfscope}%
\pgfsys@transformshift{2.242545in}{0.667367in}%
\pgfsys@useobject{currentmarker}{}%
\end{pgfscope}%
\begin{pgfscope}%
\pgfsys@transformshift{2.242695in}{0.631481in}%
\pgfsys@useobject{currentmarker}{}%
\end{pgfscope}%
\begin{pgfscope}%
\pgfsys@transformshift{2.242845in}{0.653579in}%
\pgfsys@useobject{currentmarker}{}%
\end{pgfscope}%
\begin{pgfscope}%
\pgfsys@transformshift{2.242994in}{0.639135in}%
\pgfsys@useobject{currentmarker}{}%
\end{pgfscope}%
\begin{pgfscope}%
\pgfsys@transformshift{2.243144in}{0.647941in}%
\pgfsys@useobject{currentmarker}{}%
\end{pgfscope}%
\begin{pgfscope}%
\pgfsys@transformshift{2.243294in}{0.671264in}%
\pgfsys@useobject{currentmarker}{}%
\end{pgfscope}%
\begin{pgfscope}%
\pgfsys@transformshift{2.243443in}{0.715717in}%
\pgfsys@useobject{currentmarker}{}%
\end{pgfscope}%
\begin{pgfscope}%
\pgfsys@transformshift{2.243592in}{0.693670in}%
\pgfsys@useobject{currentmarker}{}%
\end{pgfscope}%
\begin{pgfscope}%
\pgfsys@transformshift{2.243742in}{0.694179in}%
\pgfsys@useobject{currentmarker}{}%
\end{pgfscope}%
\begin{pgfscope}%
\pgfsys@transformshift{2.243891in}{0.704584in}%
\pgfsys@useobject{currentmarker}{}%
\end{pgfscope}%
\begin{pgfscope}%
\pgfsys@transformshift{2.244040in}{0.692247in}%
\pgfsys@useobject{currentmarker}{}%
\end{pgfscope}%
\begin{pgfscope}%
\pgfsys@transformshift{2.244189in}{0.659932in}%
\pgfsys@useobject{currentmarker}{}%
\end{pgfscope}%
\begin{pgfscope}%
\pgfsys@transformshift{2.244337in}{0.646508in}%
\pgfsys@useobject{currentmarker}{}%
\end{pgfscope}%
\begin{pgfscope}%
\pgfsys@transformshift{2.244486in}{0.688065in}%
\pgfsys@useobject{currentmarker}{}%
\end{pgfscope}%
\begin{pgfscope}%
\pgfsys@transformshift{2.244635in}{0.667971in}%
\pgfsys@useobject{currentmarker}{}%
\end{pgfscope}%
\begin{pgfscope}%
\pgfsys@transformshift{2.244783in}{0.658482in}%
\pgfsys@useobject{currentmarker}{}%
\end{pgfscope}%
\begin{pgfscope}%
\pgfsys@transformshift{2.244932in}{0.640259in}%
\pgfsys@useobject{currentmarker}{}%
\end{pgfscope}%
\begin{pgfscope}%
\pgfsys@transformshift{2.245080in}{0.616577in}%
\pgfsys@useobject{currentmarker}{}%
\end{pgfscope}%
\begin{pgfscope}%
\pgfsys@transformshift{2.245228in}{0.659901in}%
\pgfsys@useobject{currentmarker}{}%
\end{pgfscope}%
\begin{pgfscope}%
\pgfsys@transformshift{2.245377in}{0.670329in}%
\pgfsys@useobject{currentmarker}{}%
\end{pgfscope}%
\begin{pgfscope}%
\pgfsys@transformshift{2.245525in}{0.711482in}%
\pgfsys@useobject{currentmarker}{}%
\end{pgfscope}%
\begin{pgfscope}%
\pgfsys@transformshift{2.245672in}{0.734372in}%
\pgfsys@useobject{currentmarker}{}%
\end{pgfscope}%
\begin{pgfscope}%
\pgfsys@transformshift{2.245820in}{0.687755in}%
\pgfsys@useobject{currentmarker}{}%
\end{pgfscope}%
\begin{pgfscope}%
\pgfsys@transformshift{2.245968in}{0.617986in}%
\pgfsys@useobject{currentmarker}{}%
\end{pgfscope}%
\begin{pgfscope}%
\pgfsys@transformshift{2.246116in}{0.632775in}%
\pgfsys@useobject{currentmarker}{}%
\end{pgfscope}%
\begin{pgfscope}%
\pgfsys@transformshift{2.246263in}{0.677654in}%
\pgfsys@useobject{currentmarker}{}%
\end{pgfscope}%
\begin{pgfscope}%
\pgfsys@transformshift{2.246411in}{0.668243in}%
\pgfsys@useobject{currentmarker}{}%
\end{pgfscope}%
\begin{pgfscope}%
\pgfsys@transformshift{2.246558in}{0.698607in}%
\pgfsys@useobject{currentmarker}{}%
\end{pgfscope}%
\begin{pgfscope}%
\pgfsys@transformshift{2.246705in}{0.670122in}%
\pgfsys@useobject{currentmarker}{}%
\end{pgfscope}%
\begin{pgfscope}%
\pgfsys@transformshift{2.246853in}{0.679737in}%
\pgfsys@useobject{currentmarker}{}%
\end{pgfscope}%
\begin{pgfscope}%
\pgfsys@transformshift{2.247000in}{0.724229in}%
\pgfsys@useobject{currentmarker}{}%
\end{pgfscope}%
\begin{pgfscope}%
\pgfsys@transformshift{2.247147in}{0.698804in}%
\pgfsys@useobject{currentmarker}{}%
\end{pgfscope}%
\begin{pgfscope}%
\pgfsys@transformshift{2.247293in}{0.651864in}%
\pgfsys@useobject{currentmarker}{}%
\end{pgfscope}%
\begin{pgfscope}%
\pgfsys@transformshift{2.247440in}{0.645893in}%
\pgfsys@useobject{currentmarker}{}%
\end{pgfscope}%
\begin{pgfscope}%
\pgfsys@transformshift{2.247587in}{0.656626in}%
\pgfsys@useobject{currentmarker}{}%
\end{pgfscope}%
\begin{pgfscope}%
\pgfsys@transformshift{2.247734in}{0.680341in}%
\pgfsys@useobject{currentmarker}{}%
\end{pgfscope}%
\begin{pgfscope}%
\pgfsys@transformshift{2.247880in}{0.687878in}%
\pgfsys@useobject{currentmarker}{}%
\end{pgfscope}%
\begin{pgfscope}%
\pgfsys@transformshift{2.248026in}{0.699692in}%
\pgfsys@useobject{currentmarker}{}%
\end{pgfscope}%
\begin{pgfscope}%
\pgfsys@transformshift{2.248173in}{0.638129in}%
\pgfsys@useobject{currentmarker}{}%
\end{pgfscope}%
\begin{pgfscope}%
\pgfsys@transformshift{2.248319in}{0.656535in}%
\pgfsys@useobject{currentmarker}{}%
\end{pgfscope}%
\begin{pgfscope}%
\pgfsys@transformshift{2.248465in}{0.654489in}%
\pgfsys@useobject{currentmarker}{}%
\end{pgfscope}%
\begin{pgfscope}%
\pgfsys@transformshift{2.248611in}{0.634004in}%
\pgfsys@useobject{currentmarker}{}%
\end{pgfscope}%
\begin{pgfscope}%
\pgfsys@transformshift{2.248757in}{0.651584in}%
\pgfsys@useobject{currentmarker}{}%
\end{pgfscope}%
\begin{pgfscope}%
\pgfsys@transformshift{2.248903in}{0.671580in}%
\pgfsys@useobject{currentmarker}{}%
\end{pgfscope}%
\begin{pgfscope}%
\pgfsys@transformshift{2.249049in}{0.691352in}%
\pgfsys@useobject{currentmarker}{}%
\end{pgfscope}%
\begin{pgfscope}%
\pgfsys@transformshift{2.249194in}{0.689325in}%
\pgfsys@useobject{currentmarker}{}%
\end{pgfscope}%
\begin{pgfscope}%
\pgfsys@transformshift{2.249340in}{0.700260in}%
\pgfsys@useobject{currentmarker}{}%
\end{pgfscope}%
\begin{pgfscope}%
\pgfsys@transformshift{2.249485in}{0.686621in}%
\pgfsys@useobject{currentmarker}{}%
\end{pgfscope}%
\begin{pgfscope}%
\pgfsys@transformshift{2.249631in}{0.652202in}%
\pgfsys@useobject{currentmarker}{}%
\end{pgfscope}%
\begin{pgfscope}%
\pgfsys@transformshift{2.249776in}{0.680983in}%
\pgfsys@useobject{currentmarker}{}%
\end{pgfscope}%
\begin{pgfscope}%
\pgfsys@transformshift{2.249921in}{0.690999in}%
\pgfsys@useobject{currentmarker}{}%
\end{pgfscope}%
\begin{pgfscope}%
\pgfsys@transformshift{2.250066in}{0.597394in}%
\pgfsys@useobject{currentmarker}{}%
\end{pgfscope}%
\begin{pgfscope}%
\pgfsys@transformshift{2.250211in}{0.615952in}%
\pgfsys@useobject{currentmarker}{}%
\end{pgfscope}%
\begin{pgfscope}%
\pgfsys@transformshift{2.250356in}{0.637657in}%
\pgfsys@useobject{currentmarker}{}%
\end{pgfscope}%
\begin{pgfscope}%
\pgfsys@transformshift{2.250501in}{0.709751in}%
\pgfsys@useobject{currentmarker}{}%
\end{pgfscope}%
\begin{pgfscope}%
\pgfsys@transformshift{2.250645in}{0.710290in}%
\pgfsys@useobject{currentmarker}{}%
\end{pgfscope}%
\begin{pgfscope}%
\pgfsys@transformshift{2.250790in}{0.683359in}%
\pgfsys@useobject{currentmarker}{}%
\end{pgfscope}%
\begin{pgfscope}%
\pgfsys@transformshift{2.250935in}{0.653658in}%
\pgfsys@useobject{currentmarker}{}%
\end{pgfscope}%
\begin{pgfscope}%
\pgfsys@transformshift{2.251079in}{0.689323in}%
\pgfsys@useobject{currentmarker}{}%
\end{pgfscope}%
\begin{pgfscope}%
\pgfsys@transformshift{2.251223in}{0.701257in}%
\pgfsys@useobject{currentmarker}{}%
\end{pgfscope}%
\begin{pgfscope}%
\pgfsys@transformshift{2.251368in}{0.679560in}%
\pgfsys@useobject{currentmarker}{}%
\end{pgfscope}%
\begin{pgfscope}%
\pgfsys@transformshift{2.251512in}{0.679012in}%
\pgfsys@useobject{currentmarker}{}%
\end{pgfscope}%
\begin{pgfscope}%
\pgfsys@transformshift{2.251656in}{0.693315in}%
\pgfsys@useobject{currentmarker}{}%
\end{pgfscope}%
\begin{pgfscope}%
\pgfsys@transformshift{2.251800in}{0.707314in}%
\pgfsys@useobject{currentmarker}{}%
\end{pgfscope}%
\begin{pgfscope}%
\pgfsys@transformshift{2.251944in}{0.695550in}%
\pgfsys@useobject{currentmarker}{}%
\end{pgfscope}%
\begin{pgfscope}%
\pgfsys@transformshift{2.252087in}{0.711409in}%
\pgfsys@useobject{currentmarker}{}%
\end{pgfscope}%
\begin{pgfscope}%
\pgfsys@transformshift{2.252231in}{0.710220in}%
\pgfsys@useobject{currentmarker}{}%
\end{pgfscope}%
\begin{pgfscope}%
\pgfsys@transformshift{2.252375in}{0.657986in}%
\pgfsys@useobject{currentmarker}{}%
\end{pgfscope}%
\begin{pgfscope}%
\pgfsys@transformshift{2.252518in}{0.654439in}%
\pgfsys@useobject{currentmarker}{}%
\end{pgfscope}%
\begin{pgfscope}%
\pgfsys@transformshift{2.252662in}{0.638055in}%
\pgfsys@useobject{currentmarker}{}%
\end{pgfscope}%
\begin{pgfscope}%
\pgfsys@transformshift{2.252805in}{0.676442in}%
\pgfsys@useobject{currentmarker}{}%
\end{pgfscope}%
\begin{pgfscope}%
\pgfsys@transformshift{2.252948in}{0.663980in}%
\pgfsys@useobject{currentmarker}{}%
\end{pgfscope}%
\begin{pgfscope}%
\pgfsys@transformshift{2.253091in}{0.676592in}%
\pgfsys@useobject{currentmarker}{}%
\end{pgfscope}%
\begin{pgfscope}%
\pgfsys@transformshift{2.253234in}{0.695366in}%
\pgfsys@useobject{currentmarker}{}%
\end{pgfscope}%
\begin{pgfscope}%
\pgfsys@transformshift{2.253377in}{0.660301in}%
\pgfsys@useobject{currentmarker}{}%
\end{pgfscope}%
\begin{pgfscope}%
\pgfsys@transformshift{2.253520in}{0.692952in}%
\pgfsys@useobject{currentmarker}{}%
\end{pgfscope}%
\begin{pgfscope}%
\pgfsys@transformshift{2.253663in}{0.724638in}%
\pgfsys@useobject{currentmarker}{}%
\end{pgfscope}%
\begin{pgfscope}%
\pgfsys@transformshift{2.253806in}{0.694144in}%
\pgfsys@useobject{currentmarker}{}%
\end{pgfscope}%
\begin{pgfscope}%
\pgfsys@transformshift{2.253948in}{0.662542in}%
\pgfsys@useobject{currentmarker}{}%
\end{pgfscope}%
\begin{pgfscope}%
\pgfsys@transformshift{2.254091in}{0.687305in}%
\pgfsys@useobject{currentmarker}{}%
\end{pgfscope}%
\begin{pgfscope}%
\pgfsys@transformshift{2.254233in}{0.679545in}%
\pgfsys@useobject{currentmarker}{}%
\end{pgfscope}%
\begin{pgfscope}%
\pgfsys@transformshift{2.254375in}{0.654846in}%
\pgfsys@useobject{currentmarker}{}%
\end{pgfscope}%
\begin{pgfscope}%
\pgfsys@transformshift{2.254518in}{0.640337in}%
\pgfsys@useobject{currentmarker}{}%
\end{pgfscope}%
\begin{pgfscope}%
\pgfsys@transformshift{2.254660in}{0.682372in}%
\pgfsys@useobject{currentmarker}{}%
\end{pgfscope}%
\begin{pgfscope}%
\pgfsys@transformshift{2.254802in}{0.714596in}%
\pgfsys@useobject{currentmarker}{}%
\end{pgfscope}%
\begin{pgfscope}%
\pgfsys@transformshift{2.254944in}{0.678984in}%
\pgfsys@useobject{currentmarker}{}%
\end{pgfscope}%
\begin{pgfscope}%
\pgfsys@transformshift{2.255086in}{0.672408in}%
\pgfsys@useobject{currentmarker}{}%
\end{pgfscope}%
\begin{pgfscope}%
\pgfsys@transformshift{2.255227in}{0.686225in}%
\pgfsys@useobject{currentmarker}{}%
\end{pgfscope}%
\begin{pgfscope}%
\pgfsys@transformshift{2.255369in}{0.689422in}%
\pgfsys@useobject{currentmarker}{}%
\end{pgfscope}%
\begin{pgfscope}%
\pgfsys@transformshift{2.255511in}{0.663292in}%
\pgfsys@useobject{currentmarker}{}%
\end{pgfscope}%
\begin{pgfscope}%
\pgfsys@transformshift{2.255652in}{0.700407in}%
\pgfsys@useobject{currentmarker}{}%
\end{pgfscope}%
\begin{pgfscope}%
\pgfsys@transformshift{2.255794in}{0.681218in}%
\pgfsys@useobject{currentmarker}{}%
\end{pgfscope}%
\begin{pgfscope}%
\pgfsys@transformshift{2.255935in}{0.699670in}%
\pgfsys@useobject{currentmarker}{}%
\end{pgfscope}%
\begin{pgfscope}%
\pgfsys@transformshift{2.256076in}{0.654761in}%
\pgfsys@useobject{currentmarker}{}%
\end{pgfscope}%
\begin{pgfscope}%
\pgfsys@transformshift{2.256217in}{0.662021in}%
\pgfsys@useobject{currentmarker}{}%
\end{pgfscope}%
\begin{pgfscope}%
\pgfsys@transformshift{2.256358in}{0.655766in}%
\pgfsys@useobject{currentmarker}{}%
\end{pgfscope}%
\begin{pgfscope}%
\pgfsys@transformshift{2.256499in}{0.630917in}%
\pgfsys@useobject{currentmarker}{}%
\end{pgfscope}%
\begin{pgfscope}%
\pgfsys@transformshift{2.256640in}{0.689770in}%
\pgfsys@useobject{currentmarker}{}%
\end{pgfscope}%
\begin{pgfscope}%
\pgfsys@transformshift{2.256781in}{0.654405in}%
\pgfsys@useobject{currentmarker}{}%
\end{pgfscope}%
\begin{pgfscope}%
\pgfsys@transformshift{2.256922in}{0.653953in}%
\pgfsys@useobject{currentmarker}{}%
\end{pgfscope}%
\begin{pgfscope}%
\pgfsys@transformshift{2.257062in}{0.696136in}%
\pgfsys@useobject{currentmarker}{}%
\end{pgfscope}%
\begin{pgfscope}%
\pgfsys@transformshift{2.257203in}{0.697490in}%
\pgfsys@useobject{currentmarker}{}%
\end{pgfscope}%
\begin{pgfscope}%
\pgfsys@transformshift{2.257343in}{0.672456in}%
\pgfsys@useobject{currentmarker}{}%
\end{pgfscope}%
\begin{pgfscope}%
\pgfsys@transformshift{2.257484in}{0.636366in}%
\pgfsys@useobject{currentmarker}{}%
\end{pgfscope}%
\begin{pgfscope}%
\pgfsys@transformshift{2.257624in}{0.620733in}%
\pgfsys@useobject{currentmarker}{}%
\end{pgfscope}%
\begin{pgfscope}%
\pgfsys@transformshift{2.257764in}{0.646307in}%
\pgfsys@useobject{currentmarker}{}%
\end{pgfscope}%
\begin{pgfscope}%
\pgfsys@transformshift{2.257904in}{0.643135in}%
\pgfsys@useobject{currentmarker}{}%
\end{pgfscope}%
\begin{pgfscope}%
\pgfsys@transformshift{2.258044in}{0.634727in}%
\pgfsys@useobject{currentmarker}{}%
\end{pgfscope}%
\begin{pgfscope}%
\pgfsys@transformshift{2.258184in}{0.674726in}%
\pgfsys@useobject{currentmarker}{}%
\end{pgfscope}%
\begin{pgfscope}%
\pgfsys@transformshift{2.258324in}{0.627905in}%
\pgfsys@useobject{currentmarker}{}%
\end{pgfscope}%
\begin{pgfscope}%
\pgfsys@transformshift{2.258464in}{0.590248in}%
\pgfsys@useobject{currentmarker}{}%
\end{pgfscope}%
\begin{pgfscope}%
\pgfsys@transformshift{2.258604in}{0.691105in}%
\pgfsys@useobject{currentmarker}{}%
\end{pgfscope}%
\begin{pgfscope}%
\pgfsys@transformshift{2.258743in}{0.707850in}%
\pgfsys@useobject{currentmarker}{}%
\end{pgfscope}%
\begin{pgfscope}%
\pgfsys@transformshift{2.258883in}{0.691628in}%
\pgfsys@useobject{currentmarker}{}%
\end{pgfscope}%
\begin{pgfscope}%
\pgfsys@transformshift{2.259022in}{0.727400in}%
\pgfsys@useobject{currentmarker}{}%
\end{pgfscope}%
\begin{pgfscope}%
\pgfsys@transformshift{2.259161in}{0.701337in}%
\pgfsys@useobject{currentmarker}{}%
\end{pgfscope}%
\begin{pgfscope}%
\pgfsys@transformshift{2.259301in}{0.675359in}%
\pgfsys@useobject{currentmarker}{}%
\end{pgfscope}%
\begin{pgfscope}%
\pgfsys@transformshift{2.259440in}{0.627953in}%
\pgfsys@useobject{currentmarker}{}%
\end{pgfscope}%
\begin{pgfscope}%
\pgfsys@transformshift{2.259579in}{0.629988in}%
\pgfsys@useobject{currentmarker}{}%
\end{pgfscope}%
\begin{pgfscope}%
\pgfsys@transformshift{2.259718in}{0.662198in}%
\pgfsys@useobject{currentmarker}{}%
\end{pgfscope}%
\begin{pgfscope}%
\pgfsys@transformshift{2.259857in}{0.668037in}%
\pgfsys@useobject{currentmarker}{}%
\end{pgfscope}%
\begin{pgfscope}%
\pgfsys@transformshift{2.259995in}{0.658715in}%
\pgfsys@useobject{currentmarker}{}%
\end{pgfscope}%
\begin{pgfscope}%
\pgfsys@transformshift{2.260134in}{0.688499in}%
\pgfsys@useobject{currentmarker}{}%
\end{pgfscope}%
\begin{pgfscope}%
\pgfsys@transformshift{2.260273in}{0.732446in}%
\pgfsys@useobject{currentmarker}{}%
\end{pgfscope}%
\begin{pgfscope}%
\pgfsys@transformshift{2.260411in}{0.687231in}%
\pgfsys@useobject{currentmarker}{}%
\end{pgfscope}%
\begin{pgfscope}%
\pgfsys@transformshift{2.260550in}{0.715975in}%
\pgfsys@useobject{currentmarker}{}%
\end{pgfscope}%
\begin{pgfscope}%
\pgfsys@transformshift{2.260688in}{0.718137in}%
\pgfsys@useobject{currentmarker}{}%
\end{pgfscope}%
\begin{pgfscope}%
\pgfsys@transformshift{2.260827in}{0.681819in}%
\pgfsys@useobject{currentmarker}{}%
\end{pgfscope}%
\begin{pgfscope}%
\pgfsys@transformshift{2.260965in}{0.639055in}%
\pgfsys@useobject{currentmarker}{}%
\end{pgfscope}%
\begin{pgfscope}%
\pgfsys@transformshift{2.261103in}{0.692238in}%
\pgfsys@useobject{currentmarker}{}%
\end{pgfscope}%
\begin{pgfscope}%
\pgfsys@transformshift{2.261241in}{0.725174in}%
\pgfsys@useobject{currentmarker}{}%
\end{pgfscope}%
\begin{pgfscope}%
\pgfsys@transformshift{2.261379in}{0.700106in}%
\pgfsys@useobject{currentmarker}{}%
\end{pgfscope}%
\begin{pgfscope}%
\pgfsys@transformshift{2.261517in}{0.679129in}%
\pgfsys@useobject{currentmarker}{}%
\end{pgfscope}%
\begin{pgfscope}%
\pgfsys@transformshift{2.261654in}{0.703659in}%
\pgfsys@useobject{currentmarker}{}%
\end{pgfscope}%
\begin{pgfscope}%
\pgfsys@transformshift{2.261792in}{0.687547in}%
\pgfsys@useobject{currentmarker}{}%
\end{pgfscope}%
\begin{pgfscope}%
\pgfsys@transformshift{2.261930in}{0.634297in}%
\pgfsys@useobject{currentmarker}{}%
\end{pgfscope}%
\begin{pgfscope}%
\pgfsys@transformshift{2.262067in}{0.681215in}%
\pgfsys@useobject{currentmarker}{}%
\end{pgfscope}%
\begin{pgfscope}%
\pgfsys@transformshift{2.262205in}{0.672779in}%
\pgfsys@useobject{currentmarker}{}%
\end{pgfscope}%
\begin{pgfscope}%
\pgfsys@transformshift{2.262342in}{0.659315in}%
\pgfsys@useobject{currentmarker}{}%
\end{pgfscope}%
\begin{pgfscope}%
\pgfsys@transformshift{2.262479in}{0.654322in}%
\pgfsys@useobject{currentmarker}{}%
\end{pgfscope}%
\begin{pgfscope}%
\pgfsys@transformshift{2.262617in}{0.681712in}%
\pgfsys@useobject{currentmarker}{}%
\end{pgfscope}%
\begin{pgfscope}%
\pgfsys@transformshift{2.262754in}{0.663223in}%
\pgfsys@useobject{currentmarker}{}%
\end{pgfscope}%
\begin{pgfscope}%
\pgfsys@transformshift{2.262891in}{0.636156in}%
\pgfsys@useobject{currentmarker}{}%
\end{pgfscope}%
\begin{pgfscope}%
\pgfsys@transformshift{2.263028in}{0.703908in}%
\pgfsys@useobject{currentmarker}{}%
\end{pgfscope}%
\begin{pgfscope}%
\pgfsys@transformshift{2.263165in}{0.692141in}%
\pgfsys@useobject{currentmarker}{}%
\end{pgfscope}%
\begin{pgfscope}%
\pgfsys@transformshift{2.263301in}{0.696151in}%
\pgfsys@useobject{currentmarker}{}%
\end{pgfscope}%
\begin{pgfscope}%
\pgfsys@transformshift{2.263438in}{0.680734in}%
\pgfsys@useobject{currentmarker}{}%
\end{pgfscope}%
\begin{pgfscope}%
\pgfsys@transformshift{2.263575in}{0.656500in}%
\pgfsys@useobject{currentmarker}{}%
\end{pgfscope}%
\begin{pgfscope}%
\pgfsys@transformshift{2.263711in}{0.657229in}%
\pgfsys@useobject{currentmarker}{}%
\end{pgfscope}%
\begin{pgfscope}%
\pgfsys@transformshift{2.263848in}{0.624708in}%
\pgfsys@useobject{currentmarker}{}%
\end{pgfscope}%
\begin{pgfscope}%
\pgfsys@transformshift{2.263984in}{0.593841in}%
\pgfsys@useobject{currentmarker}{}%
\end{pgfscope}%
\begin{pgfscope}%
\pgfsys@transformshift{2.264120in}{0.672873in}%
\pgfsys@useobject{currentmarker}{}%
\end{pgfscope}%
\begin{pgfscope}%
\pgfsys@transformshift{2.264256in}{0.645601in}%
\pgfsys@useobject{currentmarker}{}%
\end{pgfscope}%
\begin{pgfscope}%
\pgfsys@transformshift{2.264392in}{0.653257in}%
\pgfsys@useobject{currentmarker}{}%
\end{pgfscope}%
\begin{pgfscope}%
\pgfsys@transformshift{2.264529in}{0.721783in}%
\pgfsys@useobject{currentmarker}{}%
\end{pgfscope}%
\begin{pgfscope}%
\pgfsys@transformshift{2.264664in}{0.733285in}%
\pgfsys@useobject{currentmarker}{}%
\end{pgfscope}%
\begin{pgfscope}%
\pgfsys@transformshift{2.264800in}{0.722527in}%
\pgfsys@useobject{currentmarker}{}%
\end{pgfscope}%
\begin{pgfscope}%
\pgfsys@transformshift{2.264936in}{0.677941in}%
\pgfsys@useobject{currentmarker}{}%
\end{pgfscope}%
\begin{pgfscope}%
\pgfsys@transformshift{2.265072in}{0.714649in}%
\pgfsys@useobject{currentmarker}{}%
\end{pgfscope}%
\begin{pgfscope}%
\pgfsys@transformshift{2.265207in}{0.709469in}%
\pgfsys@useobject{currentmarker}{}%
\end{pgfscope}%
\begin{pgfscope}%
\pgfsys@transformshift{2.265343in}{0.612561in}%
\pgfsys@useobject{currentmarker}{}%
\end{pgfscope}%
\begin{pgfscope}%
\pgfsys@transformshift{2.265478in}{0.656899in}%
\pgfsys@useobject{currentmarker}{}%
\end{pgfscope}%
\begin{pgfscope}%
\pgfsys@transformshift{2.265614in}{0.705286in}%
\pgfsys@useobject{currentmarker}{}%
\end{pgfscope}%
\begin{pgfscope}%
\pgfsys@transformshift{2.265749in}{0.669695in}%
\pgfsys@useobject{currentmarker}{}%
\end{pgfscope}%
\begin{pgfscope}%
\pgfsys@transformshift{2.265884in}{0.648200in}%
\pgfsys@useobject{currentmarker}{}%
\end{pgfscope}%
\begin{pgfscope}%
\pgfsys@transformshift{2.266019in}{0.658233in}%
\pgfsys@useobject{currentmarker}{}%
\end{pgfscope}%
\begin{pgfscope}%
\pgfsys@transformshift{2.266154in}{0.676541in}%
\pgfsys@useobject{currentmarker}{}%
\end{pgfscope}%
\begin{pgfscope}%
\pgfsys@transformshift{2.266289in}{0.670689in}%
\pgfsys@useobject{currentmarker}{}%
\end{pgfscope}%
\begin{pgfscope}%
\pgfsys@transformshift{2.266424in}{0.676695in}%
\pgfsys@useobject{currentmarker}{}%
\end{pgfscope}%
\begin{pgfscope}%
\pgfsys@transformshift{2.266559in}{0.690134in}%
\pgfsys@useobject{currentmarker}{}%
\end{pgfscope}%
\begin{pgfscope}%
\pgfsys@transformshift{2.266694in}{0.717867in}%
\pgfsys@useobject{currentmarker}{}%
\end{pgfscope}%
\begin{pgfscope}%
\pgfsys@transformshift{2.266828in}{0.702821in}%
\pgfsys@useobject{currentmarker}{}%
\end{pgfscope}%
\begin{pgfscope}%
\pgfsys@transformshift{2.266963in}{0.704742in}%
\pgfsys@useobject{currentmarker}{}%
\end{pgfscope}%
\begin{pgfscope}%
\pgfsys@transformshift{2.267098in}{0.696943in}%
\pgfsys@useobject{currentmarker}{}%
\end{pgfscope}%
\begin{pgfscope}%
\pgfsys@transformshift{2.267232in}{0.703939in}%
\pgfsys@useobject{currentmarker}{}%
\end{pgfscope}%
\begin{pgfscope}%
\pgfsys@transformshift{2.267366in}{0.689822in}%
\pgfsys@useobject{currentmarker}{}%
\end{pgfscope}%
\begin{pgfscope}%
\pgfsys@transformshift{2.267501in}{0.645271in}%
\pgfsys@useobject{currentmarker}{}%
\end{pgfscope}%
\begin{pgfscope}%
\pgfsys@transformshift{2.267635in}{0.621526in}%
\pgfsys@useobject{currentmarker}{}%
\end{pgfscope}%
\begin{pgfscope}%
\pgfsys@transformshift{2.267769in}{0.675530in}%
\pgfsys@useobject{currentmarker}{}%
\end{pgfscope}%
\begin{pgfscope}%
\pgfsys@transformshift{2.267903in}{0.704719in}%
\pgfsys@useobject{currentmarker}{}%
\end{pgfscope}%
\begin{pgfscope}%
\pgfsys@transformshift{2.268037in}{0.608201in}%
\pgfsys@useobject{currentmarker}{}%
\end{pgfscope}%
\begin{pgfscope}%
\pgfsys@transformshift{2.268171in}{0.622905in}%
\pgfsys@useobject{currentmarker}{}%
\end{pgfscope}%
\begin{pgfscope}%
\pgfsys@transformshift{2.268304in}{0.697939in}%
\pgfsys@useobject{currentmarker}{}%
\end{pgfscope}%
\begin{pgfscope}%
\pgfsys@transformshift{2.268438in}{0.669543in}%
\pgfsys@useobject{currentmarker}{}%
\end{pgfscope}%
\begin{pgfscope}%
\pgfsys@transformshift{2.268572in}{0.695268in}%
\pgfsys@useobject{currentmarker}{}%
\end{pgfscope}%
\begin{pgfscope}%
\pgfsys@transformshift{2.268705in}{0.705057in}%
\pgfsys@useobject{currentmarker}{}%
\end{pgfscope}%
\begin{pgfscope}%
\pgfsys@transformshift{2.268839in}{0.684384in}%
\pgfsys@useobject{currentmarker}{}%
\end{pgfscope}%
\begin{pgfscope}%
\pgfsys@transformshift{2.268972in}{0.656603in}%
\pgfsys@useobject{currentmarker}{}%
\end{pgfscope}%
\begin{pgfscope}%
\pgfsys@transformshift{2.269105in}{0.630561in}%
\pgfsys@useobject{currentmarker}{}%
\end{pgfscope}%
\begin{pgfscope}%
\pgfsys@transformshift{2.269238in}{0.717822in}%
\pgfsys@useobject{currentmarker}{}%
\end{pgfscope}%
\begin{pgfscope}%
\pgfsys@transformshift{2.269371in}{0.737095in}%
\pgfsys@useobject{currentmarker}{}%
\end{pgfscope}%
\begin{pgfscope}%
\pgfsys@transformshift{2.269505in}{0.682879in}%
\pgfsys@useobject{currentmarker}{}%
\end{pgfscope}%
\begin{pgfscope}%
\pgfsys@transformshift{2.269638in}{0.665629in}%
\pgfsys@useobject{currentmarker}{}%
\end{pgfscope}%
\begin{pgfscope}%
\pgfsys@transformshift{2.269770in}{0.641309in}%
\pgfsys@useobject{currentmarker}{}%
\end{pgfscope}%
\begin{pgfscope}%
\pgfsys@transformshift{2.269903in}{0.675560in}%
\pgfsys@useobject{currentmarker}{}%
\end{pgfscope}%
\begin{pgfscope}%
\pgfsys@transformshift{2.270036in}{0.703418in}%
\pgfsys@useobject{currentmarker}{}%
\end{pgfscope}%
\begin{pgfscope}%
\pgfsys@transformshift{2.270169in}{0.618859in}%
\pgfsys@useobject{currentmarker}{}%
\end{pgfscope}%
\begin{pgfscope}%
\pgfsys@transformshift{2.270301in}{0.677442in}%
\pgfsys@useobject{currentmarker}{}%
\end{pgfscope}%
\begin{pgfscope}%
\pgfsys@transformshift{2.270434in}{0.684949in}%
\pgfsys@useobject{currentmarker}{}%
\end{pgfscope}%
\begin{pgfscope}%
\pgfsys@transformshift{2.270566in}{0.625947in}%
\pgfsys@useobject{currentmarker}{}%
\end{pgfscope}%
\begin{pgfscope}%
\pgfsys@transformshift{2.270698in}{0.662892in}%
\pgfsys@useobject{currentmarker}{}%
\end{pgfscope}%
\begin{pgfscope}%
\pgfsys@transformshift{2.270831in}{0.680748in}%
\pgfsys@useobject{currentmarker}{}%
\end{pgfscope}%
\begin{pgfscope}%
\pgfsys@transformshift{2.270963in}{0.710043in}%
\pgfsys@useobject{currentmarker}{}%
\end{pgfscope}%
\begin{pgfscope}%
\pgfsys@transformshift{2.271095in}{0.710242in}%
\pgfsys@useobject{currentmarker}{}%
\end{pgfscope}%
\begin{pgfscope}%
\pgfsys@transformshift{2.271227in}{0.639190in}%
\pgfsys@useobject{currentmarker}{}%
\end{pgfscope}%
\begin{pgfscope}%
\pgfsys@transformshift{2.271359in}{0.663887in}%
\pgfsys@useobject{currentmarker}{}%
\end{pgfscope}%
\begin{pgfscope}%
\pgfsys@transformshift{2.271491in}{0.706935in}%
\pgfsys@useobject{currentmarker}{}%
\end{pgfscope}%
\begin{pgfscope}%
\pgfsys@transformshift{2.271623in}{0.686287in}%
\pgfsys@useobject{currentmarker}{}%
\end{pgfscope}%
\begin{pgfscope}%
\pgfsys@transformshift{2.271754in}{0.687468in}%
\pgfsys@useobject{currentmarker}{}%
\end{pgfscope}%
\begin{pgfscope}%
\pgfsys@transformshift{2.271886in}{0.694717in}%
\pgfsys@useobject{currentmarker}{}%
\end{pgfscope}%
\begin{pgfscope}%
\pgfsys@transformshift{2.272018in}{0.689393in}%
\pgfsys@useobject{currentmarker}{}%
\end{pgfscope}%
\begin{pgfscope}%
\pgfsys@transformshift{2.272149in}{0.699422in}%
\pgfsys@useobject{currentmarker}{}%
\end{pgfscope}%
\begin{pgfscope}%
\pgfsys@transformshift{2.272281in}{0.689321in}%
\pgfsys@useobject{currentmarker}{}%
\end{pgfscope}%
\begin{pgfscope}%
\pgfsys@transformshift{2.272412in}{0.632581in}%
\pgfsys@useobject{currentmarker}{}%
\end{pgfscope}%
\begin{pgfscope}%
\pgfsys@transformshift{2.272543in}{0.621202in}%
\pgfsys@useobject{currentmarker}{}%
\end{pgfscope}%
\begin{pgfscope}%
\pgfsys@transformshift{2.272674in}{0.640186in}%
\pgfsys@useobject{currentmarker}{}%
\end{pgfscope}%
\begin{pgfscope}%
\pgfsys@transformshift{2.272805in}{0.663630in}%
\pgfsys@useobject{currentmarker}{}%
\end{pgfscope}%
\begin{pgfscope}%
\pgfsys@transformshift{2.272936in}{0.634707in}%
\pgfsys@useobject{currentmarker}{}%
\end{pgfscope}%
\begin{pgfscope}%
\pgfsys@transformshift{2.273067in}{0.634922in}%
\pgfsys@useobject{currentmarker}{}%
\end{pgfscope}%
\begin{pgfscope}%
\pgfsys@transformshift{2.273198in}{0.660453in}%
\pgfsys@useobject{currentmarker}{}%
\end{pgfscope}%
\begin{pgfscope}%
\pgfsys@transformshift{2.273329in}{0.694990in}%
\pgfsys@useobject{currentmarker}{}%
\end{pgfscope}%
\begin{pgfscope}%
\pgfsys@transformshift{2.273460in}{0.678135in}%
\pgfsys@useobject{currentmarker}{}%
\end{pgfscope}%
\begin{pgfscope}%
\pgfsys@transformshift{2.273590in}{0.640511in}%
\pgfsys@useobject{currentmarker}{}%
\end{pgfscope}%
\begin{pgfscope}%
\pgfsys@transformshift{2.273721in}{0.682518in}%
\pgfsys@useobject{currentmarker}{}%
\end{pgfscope}%
\begin{pgfscope}%
\pgfsys@transformshift{2.273852in}{0.659285in}%
\pgfsys@useobject{currentmarker}{}%
\end{pgfscope}%
\begin{pgfscope}%
\pgfsys@transformshift{2.273982in}{0.684687in}%
\pgfsys@useobject{currentmarker}{}%
\end{pgfscope}%
\begin{pgfscope}%
\pgfsys@transformshift{2.274112in}{0.683009in}%
\pgfsys@useobject{currentmarker}{}%
\end{pgfscope}%
\begin{pgfscope}%
\pgfsys@transformshift{2.274243in}{0.674116in}%
\pgfsys@useobject{currentmarker}{}%
\end{pgfscope}%
\begin{pgfscope}%
\pgfsys@transformshift{2.274373in}{0.638475in}%
\pgfsys@useobject{currentmarker}{}%
\end{pgfscope}%
\begin{pgfscope}%
\pgfsys@transformshift{2.274503in}{0.694480in}%
\pgfsys@useobject{currentmarker}{}%
\end{pgfscope}%
\begin{pgfscope}%
\pgfsys@transformshift{2.274633in}{0.709923in}%
\pgfsys@useobject{currentmarker}{}%
\end{pgfscope}%
\begin{pgfscope}%
\pgfsys@transformshift{2.274763in}{0.652359in}%
\pgfsys@useobject{currentmarker}{}%
\end{pgfscope}%
\begin{pgfscope}%
\pgfsys@transformshift{2.274893in}{0.657751in}%
\pgfsys@useobject{currentmarker}{}%
\end{pgfscope}%
\begin{pgfscope}%
\pgfsys@transformshift{2.275023in}{0.659568in}%
\pgfsys@useobject{currentmarker}{}%
\end{pgfscope}%
\begin{pgfscope}%
\pgfsys@transformshift{2.275152in}{0.628808in}%
\pgfsys@useobject{currentmarker}{}%
\end{pgfscope}%
\begin{pgfscope}%
\pgfsys@transformshift{2.275282in}{0.639788in}%
\pgfsys@useobject{currentmarker}{}%
\end{pgfscope}%
\begin{pgfscope}%
\pgfsys@transformshift{2.275412in}{0.671710in}%
\pgfsys@useobject{currentmarker}{}%
\end{pgfscope}%
\begin{pgfscope}%
\pgfsys@transformshift{2.275541in}{0.739571in}%
\pgfsys@useobject{currentmarker}{}%
\end{pgfscope}%
\begin{pgfscope}%
\pgfsys@transformshift{2.275671in}{0.705023in}%
\pgfsys@useobject{currentmarker}{}%
\end{pgfscope}%
\begin{pgfscope}%
\pgfsys@transformshift{2.275800in}{0.617422in}%
\pgfsys@useobject{currentmarker}{}%
\end{pgfscope}%
\begin{pgfscope}%
\pgfsys@transformshift{2.275929in}{0.621267in}%
\pgfsys@useobject{currentmarker}{}%
\end{pgfscope}%
\begin{pgfscope}%
\pgfsys@transformshift{2.276058in}{0.654245in}%
\pgfsys@useobject{currentmarker}{}%
\end{pgfscope}%
\begin{pgfscope}%
\pgfsys@transformshift{2.276188in}{0.675421in}%
\pgfsys@useobject{currentmarker}{}%
\end{pgfscope}%
\begin{pgfscope}%
\pgfsys@transformshift{2.276317in}{0.662926in}%
\pgfsys@useobject{currentmarker}{}%
\end{pgfscope}%
\begin{pgfscope}%
\pgfsys@transformshift{2.276446in}{0.593707in}%
\pgfsys@useobject{currentmarker}{}%
\end{pgfscope}%
\begin{pgfscope}%
\pgfsys@transformshift{2.276575in}{0.699276in}%
\pgfsys@useobject{currentmarker}{}%
\end{pgfscope}%
\begin{pgfscope}%
\pgfsys@transformshift{2.276703in}{0.757518in}%
\pgfsys@useobject{currentmarker}{}%
\end{pgfscope}%
\begin{pgfscope}%
\pgfsys@transformshift{2.276832in}{0.761603in}%
\pgfsys@useobject{currentmarker}{}%
\end{pgfscope}%
\begin{pgfscope}%
\pgfsys@transformshift{2.276961in}{0.685873in}%
\pgfsys@useobject{currentmarker}{}%
\end{pgfscope}%
\begin{pgfscope}%
\pgfsys@transformshift{2.277090in}{0.612118in}%
\pgfsys@useobject{currentmarker}{}%
\end{pgfscope}%
\begin{pgfscope}%
\pgfsys@transformshift{2.277218in}{0.645655in}%
\pgfsys@useobject{currentmarker}{}%
\end{pgfscope}%
\begin{pgfscope}%
\pgfsys@transformshift{2.277347in}{0.656503in}%
\pgfsys@useobject{currentmarker}{}%
\end{pgfscope}%
\begin{pgfscope}%
\pgfsys@transformshift{2.277475in}{0.655345in}%
\pgfsys@useobject{currentmarker}{}%
\end{pgfscope}%
\begin{pgfscope}%
\pgfsys@transformshift{2.277603in}{0.669952in}%
\pgfsys@useobject{currentmarker}{}%
\end{pgfscope}%
\begin{pgfscope}%
\pgfsys@transformshift{2.277732in}{0.685307in}%
\pgfsys@useobject{currentmarker}{}%
\end{pgfscope}%
\begin{pgfscope}%
\pgfsys@transformshift{2.277860in}{0.703907in}%
\pgfsys@useobject{currentmarker}{}%
\end{pgfscope}%
\begin{pgfscope}%
\pgfsys@transformshift{2.277988in}{0.640740in}%
\pgfsys@useobject{currentmarker}{}%
\end{pgfscope}%
\begin{pgfscope}%
\pgfsys@transformshift{2.278116in}{0.651054in}%
\pgfsys@useobject{currentmarker}{}%
\end{pgfscope}%
\begin{pgfscope}%
\pgfsys@transformshift{2.278244in}{0.675960in}%
\pgfsys@useobject{currentmarker}{}%
\end{pgfscope}%
\begin{pgfscope}%
\pgfsys@transformshift{2.278372in}{0.691841in}%
\pgfsys@useobject{currentmarker}{}%
\end{pgfscope}%
\begin{pgfscope}%
\pgfsys@transformshift{2.278500in}{0.673071in}%
\pgfsys@useobject{currentmarker}{}%
\end{pgfscope}%
\begin{pgfscope}%
\pgfsys@transformshift{2.278627in}{0.653595in}%
\pgfsys@useobject{currentmarker}{}%
\end{pgfscope}%
\begin{pgfscope}%
\pgfsys@transformshift{2.278755in}{0.675640in}%
\pgfsys@useobject{currentmarker}{}%
\end{pgfscope}%
\begin{pgfscope}%
\pgfsys@transformshift{2.278883in}{0.690476in}%
\pgfsys@useobject{currentmarker}{}%
\end{pgfscope}%
\begin{pgfscope}%
\pgfsys@transformshift{2.279010in}{0.693903in}%
\pgfsys@useobject{currentmarker}{}%
\end{pgfscope}%
\begin{pgfscope}%
\pgfsys@transformshift{2.279138in}{0.646390in}%
\pgfsys@useobject{currentmarker}{}%
\end{pgfscope}%
\begin{pgfscope}%
\pgfsys@transformshift{2.279265in}{0.631137in}%
\pgfsys@useobject{currentmarker}{}%
\end{pgfscope}%
\begin{pgfscope}%
\pgfsys@transformshift{2.279392in}{0.665355in}%
\pgfsys@useobject{currentmarker}{}%
\end{pgfscope}%
\begin{pgfscope}%
\pgfsys@transformshift{2.279520in}{0.703147in}%
\pgfsys@useobject{currentmarker}{}%
\end{pgfscope}%
\begin{pgfscope}%
\pgfsys@transformshift{2.279647in}{0.660187in}%
\pgfsys@useobject{currentmarker}{}%
\end{pgfscope}%
\begin{pgfscope}%
\pgfsys@transformshift{2.279774in}{0.617738in}%
\pgfsys@useobject{currentmarker}{}%
\end{pgfscope}%
\begin{pgfscope}%
\pgfsys@transformshift{2.279901in}{0.612342in}%
\pgfsys@useobject{currentmarker}{}%
\end{pgfscope}%
\begin{pgfscope}%
\pgfsys@transformshift{2.280028in}{0.659620in}%
\pgfsys@useobject{currentmarker}{}%
\end{pgfscope}%
\begin{pgfscope}%
\pgfsys@transformshift{2.280155in}{0.647861in}%
\pgfsys@useobject{currentmarker}{}%
\end{pgfscope}%
\begin{pgfscope}%
\pgfsys@transformshift{2.280282in}{0.675484in}%
\pgfsys@useobject{currentmarker}{}%
\end{pgfscope}%
\begin{pgfscope}%
\pgfsys@transformshift{2.280408in}{0.680495in}%
\pgfsys@useobject{currentmarker}{}%
\end{pgfscope}%
\begin{pgfscope}%
\pgfsys@transformshift{2.280535in}{0.693669in}%
\pgfsys@useobject{currentmarker}{}%
\end{pgfscope}%
\begin{pgfscope}%
\pgfsys@transformshift{2.280662in}{0.684703in}%
\pgfsys@useobject{currentmarker}{}%
\end{pgfscope}%
\begin{pgfscope}%
\pgfsys@transformshift{2.280788in}{0.684902in}%
\pgfsys@useobject{currentmarker}{}%
\end{pgfscope}%
\begin{pgfscope}%
\pgfsys@transformshift{2.280915in}{0.713158in}%
\pgfsys@useobject{currentmarker}{}%
\end{pgfscope}%
\begin{pgfscope}%
\pgfsys@transformshift{2.281041in}{0.689498in}%
\pgfsys@useobject{currentmarker}{}%
\end{pgfscope}%
\begin{pgfscope}%
\pgfsys@transformshift{2.281167in}{0.678358in}%
\pgfsys@useobject{currentmarker}{}%
\end{pgfscope}%
\begin{pgfscope}%
\pgfsys@transformshift{2.281294in}{0.685504in}%
\pgfsys@useobject{currentmarker}{}%
\end{pgfscope}%
\begin{pgfscope}%
\pgfsys@transformshift{2.281420in}{0.654222in}%
\pgfsys@useobject{currentmarker}{}%
\end{pgfscope}%
\begin{pgfscope}%
\pgfsys@transformshift{2.281546in}{0.628392in}%
\pgfsys@useobject{currentmarker}{}%
\end{pgfscope}%
\begin{pgfscope}%
\pgfsys@transformshift{2.281672in}{0.700803in}%
\pgfsys@useobject{currentmarker}{}%
\end{pgfscope}%
\begin{pgfscope}%
\pgfsys@transformshift{2.281798in}{0.685082in}%
\pgfsys@useobject{currentmarker}{}%
\end{pgfscope}%
\begin{pgfscope}%
\pgfsys@transformshift{2.281924in}{0.649362in}%
\pgfsys@useobject{currentmarker}{}%
\end{pgfscope}%
\begin{pgfscope}%
\pgfsys@transformshift{2.282050in}{0.642260in}%
\pgfsys@useobject{currentmarker}{}%
\end{pgfscope}%
\begin{pgfscope}%
\pgfsys@transformshift{2.282175in}{0.704298in}%
\pgfsys@useobject{currentmarker}{}%
\end{pgfscope}%
\begin{pgfscope}%
\pgfsys@transformshift{2.282301in}{0.703003in}%
\pgfsys@useobject{currentmarker}{}%
\end{pgfscope}%
\begin{pgfscope}%
\pgfsys@transformshift{2.282427in}{0.705209in}%
\pgfsys@useobject{currentmarker}{}%
\end{pgfscope}%
\begin{pgfscope}%
\pgfsys@transformshift{2.282552in}{0.677271in}%
\pgfsys@useobject{currentmarker}{}%
\end{pgfscope}%
\begin{pgfscope}%
\pgfsys@transformshift{2.282678in}{0.674395in}%
\pgfsys@useobject{currentmarker}{}%
\end{pgfscope}%
\begin{pgfscope}%
\pgfsys@transformshift{2.282803in}{0.643454in}%
\pgfsys@useobject{currentmarker}{}%
\end{pgfscope}%
\begin{pgfscope}%
\pgfsys@transformshift{2.282928in}{0.682233in}%
\pgfsys@useobject{currentmarker}{}%
\end{pgfscope}%
\begin{pgfscope}%
\pgfsys@transformshift{2.283054in}{0.685461in}%
\pgfsys@useobject{currentmarker}{}%
\end{pgfscope}%
\begin{pgfscope}%
\pgfsys@transformshift{2.283179in}{0.660369in}%
\pgfsys@useobject{currentmarker}{}%
\end{pgfscope}%
\begin{pgfscope}%
\pgfsys@transformshift{2.283304in}{0.672619in}%
\pgfsys@useobject{currentmarker}{}%
\end{pgfscope}%
\begin{pgfscope}%
\pgfsys@transformshift{2.283429in}{0.712954in}%
\pgfsys@useobject{currentmarker}{}%
\end{pgfscope}%
\begin{pgfscope}%
\pgfsys@transformshift{2.283554in}{0.682941in}%
\pgfsys@useobject{currentmarker}{}%
\end{pgfscope}%
\begin{pgfscope}%
\pgfsys@transformshift{2.283679in}{0.667730in}%
\pgfsys@useobject{currentmarker}{}%
\end{pgfscope}%
\begin{pgfscope}%
\pgfsys@transformshift{2.283804in}{0.708436in}%
\pgfsys@useobject{currentmarker}{}%
\end{pgfscope}%
\begin{pgfscope}%
\pgfsys@transformshift{2.283928in}{0.675085in}%
\pgfsys@useobject{currentmarker}{}%
\end{pgfscope}%
\begin{pgfscope}%
\pgfsys@transformshift{2.284053in}{0.683896in}%
\pgfsys@useobject{currentmarker}{}%
\end{pgfscope}%
\begin{pgfscope}%
\pgfsys@transformshift{2.284178in}{0.684381in}%
\pgfsys@useobject{currentmarker}{}%
\end{pgfscope}%
\begin{pgfscope}%
\pgfsys@transformshift{2.284302in}{0.695623in}%
\pgfsys@useobject{currentmarker}{}%
\end{pgfscope}%
\begin{pgfscope}%
\pgfsys@transformshift{2.284427in}{0.647550in}%
\pgfsys@useobject{currentmarker}{}%
\end{pgfscope}%
\begin{pgfscope}%
\pgfsys@transformshift{2.284551in}{0.667445in}%
\pgfsys@useobject{currentmarker}{}%
\end{pgfscope}%
\begin{pgfscope}%
\pgfsys@transformshift{2.284676in}{0.635571in}%
\pgfsys@useobject{currentmarker}{}%
\end{pgfscope}%
\begin{pgfscope}%
\pgfsys@transformshift{2.284800in}{0.602407in}%
\pgfsys@useobject{currentmarker}{}%
\end{pgfscope}%
\begin{pgfscope}%
\pgfsys@transformshift{2.284924in}{0.614202in}%
\pgfsys@useobject{currentmarker}{}%
\end{pgfscope}%
\begin{pgfscope}%
\pgfsys@transformshift{2.285048in}{0.620568in}%
\pgfsys@useobject{currentmarker}{}%
\end{pgfscope}%
\begin{pgfscope}%
\pgfsys@transformshift{2.285172in}{0.681380in}%
\pgfsys@useobject{currentmarker}{}%
\end{pgfscope}%
\begin{pgfscope}%
\pgfsys@transformshift{2.285296in}{0.707333in}%
\pgfsys@useobject{currentmarker}{}%
\end{pgfscope}%
\begin{pgfscope}%
\pgfsys@transformshift{2.285420in}{0.691262in}%
\pgfsys@useobject{currentmarker}{}%
\end{pgfscope}%
\begin{pgfscope}%
\pgfsys@transformshift{2.285544in}{0.670071in}%
\pgfsys@useobject{currentmarker}{}%
\end{pgfscope}%
\begin{pgfscope}%
\pgfsys@transformshift{2.285668in}{0.695140in}%
\pgfsys@useobject{currentmarker}{}%
\end{pgfscope}%
\begin{pgfscope}%
\pgfsys@transformshift{2.285792in}{0.690953in}%
\pgfsys@useobject{currentmarker}{}%
\end{pgfscope}%
\begin{pgfscope}%
\pgfsys@transformshift{2.285915in}{0.692511in}%
\pgfsys@useobject{currentmarker}{}%
\end{pgfscope}%
\begin{pgfscope}%
\pgfsys@transformshift{2.286039in}{0.670758in}%
\pgfsys@useobject{currentmarker}{}%
\end{pgfscope}%
\begin{pgfscope}%
\pgfsys@transformshift{2.286163in}{0.700373in}%
\pgfsys@useobject{currentmarker}{}%
\end{pgfscope}%
\begin{pgfscope}%
\pgfsys@transformshift{2.286286in}{0.717317in}%
\pgfsys@useobject{currentmarker}{}%
\end{pgfscope}%
\begin{pgfscope}%
\pgfsys@transformshift{2.286409in}{0.645983in}%
\pgfsys@useobject{currentmarker}{}%
\end{pgfscope}%
\begin{pgfscope}%
\pgfsys@transformshift{2.286533in}{0.668079in}%
\pgfsys@useobject{currentmarker}{}%
\end{pgfscope}%
\begin{pgfscope}%
\pgfsys@transformshift{2.286656in}{0.684820in}%
\pgfsys@useobject{currentmarker}{}%
\end{pgfscope}%
\begin{pgfscope}%
\pgfsys@transformshift{2.286779in}{0.691099in}%
\pgfsys@useobject{currentmarker}{}%
\end{pgfscope}%
\begin{pgfscope}%
\pgfsys@transformshift{2.286902in}{0.709386in}%
\pgfsys@useobject{currentmarker}{}%
\end{pgfscope}%
\begin{pgfscope}%
\pgfsys@transformshift{2.287025in}{0.679216in}%
\pgfsys@useobject{currentmarker}{}%
\end{pgfscope}%
\begin{pgfscope}%
\pgfsys@transformshift{2.287148in}{0.650438in}%
\pgfsys@useobject{currentmarker}{}%
\end{pgfscope}%
\begin{pgfscope}%
\pgfsys@transformshift{2.287271in}{0.628079in}%
\pgfsys@useobject{currentmarker}{}%
\end{pgfscope}%
\begin{pgfscope}%
\pgfsys@transformshift{2.287394in}{0.659483in}%
\pgfsys@useobject{currentmarker}{}%
\end{pgfscope}%
\begin{pgfscope}%
\pgfsys@transformshift{2.287517in}{0.707804in}%
\pgfsys@useobject{currentmarker}{}%
\end{pgfscope}%
\begin{pgfscope}%
\pgfsys@transformshift{2.287640in}{0.675398in}%
\pgfsys@useobject{currentmarker}{}%
\end{pgfscope}%
\begin{pgfscope}%
\pgfsys@transformshift{2.287762in}{0.672865in}%
\pgfsys@useobject{currentmarker}{}%
\end{pgfscope}%
\begin{pgfscope}%
\pgfsys@transformshift{2.287885in}{0.692482in}%
\pgfsys@useobject{currentmarker}{}%
\end{pgfscope}%
\begin{pgfscope}%
\pgfsys@transformshift{2.288007in}{0.679088in}%
\pgfsys@useobject{currentmarker}{}%
\end{pgfscope}%
\begin{pgfscope}%
\pgfsys@transformshift{2.288130in}{0.671585in}%
\pgfsys@useobject{currentmarker}{}%
\end{pgfscope}%
\begin{pgfscope}%
\pgfsys@transformshift{2.288252in}{0.674258in}%
\pgfsys@useobject{currentmarker}{}%
\end{pgfscope}%
\begin{pgfscope}%
\pgfsys@transformshift{2.288375in}{0.641660in}%
\pgfsys@useobject{currentmarker}{}%
\end{pgfscope}%
\begin{pgfscope}%
\pgfsys@transformshift{2.288497in}{0.690724in}%
\pgfsys@useobject{currentmarker}{}%
\end{pgfscope}%
\begin{pgfscope}%
\pgfsys@transformshift{2.288619in}{0.707782in}%
\pgfsys@useobject{currentmarker}{}%
\end{pgfscope}%
\begin{pgfscope}%
\pgfsys@transformshift{2.288741in}{0.676124in}%
\pgfsys@useobject{currentmarker}{}%
\end{pgfscope}%
\begin{pgfscope}%
\pgfsys@transformshift{2.288863in}{0.680647in}%
\pgfsys@useobject{currentmarker}{}%
\end{pgfscope}%
\begin{pgfscope}%
\pgfsys@transformshift{2.288985in}{0.666883in}%
\pgfsys@useobject{currentmarker}{}%
\end{pgfscope}%
\begin{pgfscope}%
\pgfsys@transformshift{2.289107in}{0.649389in}%
\pgfsys@useobject{currentmarker}{}%
\end{pgfscope}%
\begin{pgfscope}%
\pgfsys@transformshift{2.289229in}{0.672225in}%
\pgfsys@useobject{currentmarker}{}%
\end{pgfscope}%
\begin{pgfscope}%
\pgfsys@transformshift{2.289351in}{0.696815in}%
\pgfsys@useobject{currentmarker}{}%
\end{pgfscope}%
\begin{pgfscope}%
\pgfsys@transformshift{2.289473in}{0.686586in}%
\pgfsys@useobject{currentmarker}{}%
\end{pgfscope}%
\begin{pgfscope}%
\pgfsys@transformshift{2.289594in}{0.634180in}%
\pgfsys@useobject{currentmarker}{}%
\end{pgfscope}%
\begin{pgfscope}%
\pgfsys@transformshift{2.289716in}{0.661545in}%
\pgfsys@useobject{currentmarker}{}%
\end{pgfscope}%
\begin{pgfscope}%
\pgfsys@transformshift{2.289837in}{0.735895in}%
\pgfsys@useobject{currentmarker}{}%
\end{pgfscope}%
\begin{pgfscope}%
\pgfsys@transformshift{2.289959in}{0.751341in}%
\pgfsys@useobject{currentmarker}{}%
\end{pgfscope}%
\begin{pgfscope}%
\pgfsys@transformshift{2.290080in}{0.681450in}%
\pgfsys@useobject{currentmarker}{}%
\end{pgfscope}%
\begin{pgfscope}%
\pgfsys@transformshift{2.290202in}{0.695147in}%
\pgfsys@useobject{currentmarker}{}%
\end{pgfscope}%
\begin{pgfscope}%
\pgfsys@transformshift{2.290323in}{0.704108in}%
\pgfsys@useobject{currentmarker}{}%
\end{pgfscope}%
\begin{pgfscope}%
\pgfsys@transformshift{2.290444in}{0.643409in}%
\pgfsys@useobject{currentmarker}{}%
\end{pgfscope}%
\begin{pgfscope}%
\pgfsys@transformshift{2.290565in}{0.703679in}%
\pgfsys@useobject{currentmarker}{}%
\end{pgfscope}%
\begin{pgfscope}%
\pgfsys@transformshift{2.290686in}{0.693934in}%
\pgfsys@useobject{currentmarker}{}%
\end{pgfscope}%
\begin{pgfscope}%
\pgfsys@transformshift{2.290807in}{0.679671in}%
\pgfsys@useobject{currentmarker}{}%
\end{pgfscope}%
\begin{pgfscope}%
\pgfsys@transformshift{2.290928in}{0.681857in}%
\pgfsys@useobject{currentmarker}{}%
\end{pgfscope}%
\begin{pgfscope}%
\pgfsys@transformshift{2.291049in}{0.691992in}%
\pgfsys@useobject{currentmarker}{}%
\end{pgfscope}%
\begin{pgfscope}%
\pgfsys@transformshift{2.291170in}{0.674166in}%
\pgfsys@useobject{currentmarker}{}%
\end{pgfscope}%
\begin{pgfscope}%
\pgfsys@transformshift{2.291291in}{0.711859in}%
\pgfsys@useobject{currentmarker}{}%
\end{pgfscope}%
\begin{pgfscope}%
\pgfsys@transformshift{2.291411in}{0.668748in}%
\pgfsys@useobject{currentmarker}{}%
\end{pgfscope}%
\begin{pgfscope}%
\pgfsys@transformshift{2.291532in}{0.676178in}%
\pgfsys@useobject{currentmarker}{}%
\end{pgfscope}%
\begin{pgfscope}%
\pgfsys@transformshift{2.291653in}{0.698768in}%
\pgfsys@useobject{currentmarker}{}%
\end{pgfscope}%
\begin{pgfscope}%
\pgfsys@transformshift{2.291773in}{0.671036in}%
\pgfsys@useobject{currentmarker}{}%
\end{pgfscope}%
\begin{pgfscope}%
\pgfsys@transformshift{2.291893in}{0.657567in}%
\pgfsys@useobject{currentmarker}{}%
\end{pgfscope}%
\begin{pgfscope}%
\pgfsys@transformshift{2.292014in}{0.654721in}%
\pgfsys@useobject{currentmarker}{}%
\end{pgfscope}%
\begin{pgfscope}%
\pgfsys@transformshift{2.292134in}{0.675216in}%
\pgfsys@useobject{currentmarker}{}%
\end{pgfscope}%
\begin{pgfscope}%
\pgfsys@transformshift{2.292254in}{0.660652in}%
\pgfsys@useobject{currentmarker}{}%
\end{pgfscope}%
\begin{pgfscope}%
\pgfsys@transformshift{2.292374in}{0.650192in}%
\pgfsys@useobject{currentmarker}{}%
\end{pgfscope}%
\begin{pgfscope}%
\pgfsys@transformshift{2.292495in}{0.644563in}%
\pgfsys@useobject{currentmarker}{}%
\end{pgfscope}%
\begin{pgfscope}%
\pgfsys@transformshift{2.292615in}{0.683611in}%
\pgfsys@useobject{currentmarker}{}%
\end{pgfscope}%
\begin{pgfscope}%
\pgfsys@transformshift{2.292735in}{0.681995in}%
\pgfsys@useobject{currentmarker}{}%
\end{pgfscope}%
\begin{pgfscope}%
\pgfsys@transformshift{2.292855in}{0.667117in}%
\pgfsys@useobject{currentmarker}{}%
\end{pgfscope}%
\begin{pgfscope}%
\pgfsys@transformshift{2.292974in}{0.635375in}%
\pgfsys@useobject{currentmarker}{}%
\end{pgfscope}%
\begin{pgfscope}%
\pgfsys@transformshift{2.293094in}{0.627388in}%
\pgfsys@useobject{currentmarker}{}%
\end{pgfscope}%
\begin{pgfscope}%
\pgfsys@transformshift{2.293214in}{0.681910in}%
\pgfsys@useobject{currentmarker}{}%
\end{pgfscope}%
\begin{pgfscope}%
\pgfsys@transformshift{2.293334in}{0.682722in}%
\pgfsys@useobject{currentmarker}{}%
\end{pgfscope}%
\begin{pgfscope}%
\pgfsys@transformshift{2.293453in}{0.647924in}%
\pgfsys@useobject{currentmarker}{}%
\end{pgfscope}%
\begin{pgfscope}%
\pgfsys@transformshift{2.293573in}{0.694397in}%
\pgfsys@useobject{currentmarker}{}%
\end{pgfscope}%
\begin{pgfscope}%
\pgfsys@transformshift{2.293692in}{0.677924in}%
\pgfsys@useobject{currentmarker}{}%
\end{pgfscope}%
\begin{pgfscope}%
\pgfsys@transformshift{2.293812in}{0.647918in}%
\pgfsys@useobject{currentmarker}{}%
\end{pgfscope}%
\begin{pgfscope}%
\pgfsys@transformshift{2.293931in}{0.673262in}%
\pgfsys@useobject{currentmarker}{}%
\end{pgfscope}%
\begin{pgfscope}%
\pgfsys@transformshift{2.294050in}{0.682434in}%
\pgfsys@useobject{currentmarker}{}%
\end{pgfscope}%
\begin{pgfscope}%
\pgfsys@transformshift{2.294169in}{0.692400in}%
\pgfsys@useobject{currentmarker}{}%
\end{pgfscope}%
\begin{pgfscope}%
\pgfsys@transformshift{2.294288in}{0.710708in}%
\pgfsys@useobject{currentmarker}{}%
\end{pgfscope}%
\begin{pgfscope}%
\pgfsys@transformshift{2.294408in}{0.706811in}%
\pgfsys@useobject{currentmarker}{}%
\end{pgfscope}%
\begin{pgfscope}%
\pgfsys@transformshift{2.294527in}{0.672568in}%
\pgfsys@useobject{currentmarker}{}%
\end{pgfscope}%
\begin{pgfscope}%
\pgfsys@transformshift{2.294646in}{0.646445in}%
\pgfsys@useobject{currentmarker}{}%
\end{pgfscope}%
\begin{pgfscope}%
\pgfsys@transformshift{2.294764in}{0.661729in}%
\pgfsys@useobject{currentmarker}{}%
\end{pgfscope}%
\begin{pgfscope}%
\pgfsys@transformshift{2.294883in}{0.648276in}%
\pgfsys@useobject{currentmarker}{}%
\end{pgfscope}%
\begin{pgfscope}%
\pgfsys@transformshift{2.295002in}{0.638861in}%
\pgfsys@useobject{currentmarker}{}%
\end{pgfscope}%
\begin{pgfscope}%
\pgfsys@transformshift{2.295121in}{0.627525in}%
\pgfsys@useobject{currentmarker}{}%
\end{pgfscope}%
\begin{pgfscope}%
\pgfsys@transformshift{2.295239in}{0.680816in}%
\pgfsys@useobject{currentmarker}{}%
\end{pgfscope}%
\begin{pgfscope}%
\pgfsys@transformshift{2.295358in}{0.709967in}%
\pgfsys@useobject{currentmarker}{}%
\end{pgfscope}%
\begin{pgfscope}%
\pgfsys@transformshift{2.295476in}{0.700293in}%
\pgfsys@useobject{currentmarker}{}%
\end{pgfscope}%
\begin{pgfscope}%
\pgfsys@transformshift{2.295595in}{0.656387in}%
\pgfsys@useobject{currentmarker}{}%
\end{pgfscope}%
\begin{pgfscope}%
\pgfsys@transformshift{2.295713in}{0.662081in}%
\pgfsys@useobject{currentmarker}{}%
\end{pgfscope}%
\begin{pgfscope}%
\pgfsys@transformshift{2.295832in}{0.661520in}%
\pgfsys@useobject{currentmarker}{}%
\end{pgfscope}%
\begin{pgfscope}%
\pgfsys@transformshift{2.295950in}{0.680124in}%
\pgfsys@useobject{currentmarker}{}%
\end{pgfscope}%
\begin{pgfscope}%
\pgfsys@transformshift{2.296068in}{0.656054in}%
\pgfsys@useobject{currentmarker}{}%
\end{pgfscope}%
\begin{pgfscope}%
\pgfsys@transformshift{2.296186in}{0.652909in}%
\pgfsys@useobject{currentmarker}{}%
\end{pgfscope}%
\begin{pgfscope}%
\pgfsys@transformshift{2.296304in}{0.641789in}%
\pgfsys@useobject{currentmarker}{}%
\end{pgfscope}%
\begin{pgfscope}%
\pgfsys@transformshift{2.296422in}{0.603751in}%
\pgfsys@useobject{currentmarker}{}%
\end{pgfscope}%
\begin{pgfscope}%
\pgfsys@transformshift{2.296540in}{0.631273in}%
\pgfsys@useobject{currentmarker}{}%
\end{pgfscope}%
\begin{pgfscope}%
\pgfsys@transformshift{2.296658in}{0.691810in}%
\pgfsys@useobject{currentmarker}{}%
\end{pgfscope}%
\begin{pgfscope}%
\pgfsys@transformshift{2.296776in}{0.698668in}%
\pgfsys@useobject{currentmarker}{}%
\end{pgfscope}%
\begin{pgfscope}%
\pgfsys@transformshift{2.296894in}{0.674084in}%
\pgfsys@useobject{currentmarker}{}%
\end{pgfscope}%
\begin{pgfscope}%
\pgfsys@transformshift{2.297012in}{0.665989in}%
\pgfsys@useobject{currentmarker}{}%
\end{pgfscope}%
\begin{pgfscope}%
\pgfsys@transformshift{2.297129in}{0.609636in}%
\pgfsys@useobject{currentmarker}{}%
\end{pgfscope}%
\begin{pgfscope}%
\pgfsys@transformshift{2.297247in}{0.638547in}%
\pgfsys@useobject{currentmarker}{}%
\end{pgfscope}%
\begin{pgfscope}%
\pgfsys@transformshift{2.297364in}{0.690769in}%
\pgfsys@useobject{currentmarker}{}%
\end{pgfscope}%
\begin{pgfscope}%
\pgfsys@transformshift{2.297482in}{0.684271in}%
\pgfsys@useobject{currentmarker}{}%
\end{pgfscope}%
\begin{pgfscope}%
\pgfsys@transformshift{2.297599in}{0.648709in}%
\pgfsys@useobject{currentmarker}{}%
\end{pgfscope}%
\begin{pgfscope}%
\pgfsys@transformshift{2.297717in}{0.639670in}%
\pgfsys@useobject{currentmarker}{}%
\end{pgfscope}%
\begin{pgfscope}%
\pgfsys@transformshift{2.297834in}{0.648369in}%
\pgfsys@useobject{currentmarker}{}%
\end{pgfscope}%
\begin{pgfscope}%
\pgfsys@transformshift{2.297951in}{0.663953in}%
\pgfsys@useobject{currentmarker}{}%
\end{pgfscope}%
\begin{pgfscope}%
\pgfsys@transformshift{2.298068in}{0.654222in}%
\pgfsys@useobject{currentmarker}{}%
\end{pgfscope}%
\begin{pgfscope}%
\pgfsys@transformshift{2.298185in}{0.660754in}%
\pgfsys@useobject{currentmarker}{}%
\end{pgfscope}%
\begin{pgfscope}%
\pgfsys@transformshift{2.298302in}{0.679540in}%
\pgfsys@useobject{currentmarker}{}%
\end{pgfscope}%
\begin{pgfscope}%
\pgfsys@transformshift{2.298419in}{0.623709in}%
\pgfsys@useobject{currentmarker}{}%
\end{pgfscope}%
\begin{pgfscope}%
\pgfsys@transformshift{2.298536in}{0.636966in}%
\pgfsys@useobject{currentmarker}{}%
\end{pgfscope}%
\begin{pgfscope}%
\pgfsys@transformshift{2.298653in}{0.630659in}%
\pgfsys@useobject{currentmarker}{}%
\end{pgfscope}%
\begin{pgfscope}%
\pgfsys@transformshift{2.298770in}{0.626511in}%
\pgfsys@useobject{currentmarker}{}%
\end{pgfscope}%
\begin{pgfscope}%
\pgfsys@transformshift{2.298887in}{0.677257in}%
\pgfsys@useobject{currentmarker}{}%
\end{pgfscope}%
\begin{pgfscope}%
\pgfsys@transformshift{2.299003in}{0.660927in}%
\pgfsys@useobject{currentmarker}{}%
\end{pgfscope}%
\begin{pgfscope}%
\pgfsys@transformshift{2.299120in}{0.658376in}%
\pgfsys@useobject{currentmarker}{}%
\end{pgfscope}%
\begin{pgfscope}%
\pgfsys@transformshift{2.299236in}{0.690972in}%
\pgfsys@useobject{currentmarker}{}%
\end{pgfscope}%
\begin{pgfscope}%
\pgfsys@transformshift{2.299353in}{0.692994in}%
\pgfsys@useobject{currentmarker}{}%
\end{pgfscope}%
\begin{pgfscope}%
\pgfsys@transformshift{2.299469in}{0.692625in}%
\pgfsys@useobject{currentmarker}{}%
\end{pgfscope}%
\begin{pgfscope}%
\pgfsys@transformshift{2.299586in}{0.701255in}%
\pgfsys@useobject{currentmarker}{}%
\end{pgfscope}%
\begin{pgfscope}%
\pgfsys@transformshift{2.299702in}{0.679299in}%
\pgfsys@useobject{currentmarker}{}%
\end{pgfscope}%
\begin{pgfscope}%
\pgfsys@transformshift{2.299818in}{0.674972in}%
\pgfsys@useobject{currentmarker}{}%
\end{pgfscope}%
\begin{pgfscope}%
\pgfsys@transformshift{2.299934in}{0.639771in}%
\pgfsys@useobject{currentmarker}{}%
\end{pgfscope}%
\begin{pgfscope}%
\pgfsys@transformshift{2.300051in}{0.687561in}%
\pgfsys@useobject{currentmarker}{}%
\end{pgfscope}%
\begin{pgfscope}%
\pgfsys@transformshift{2.300167in}{0.683616in}%
\pgfsys@useobject{currentmarker}{}%
\end{pgfscope}%
\begin{pgfscope}%
\pgfsys@transformshift{2.300283in}{0.642821in}%
\pgfsys@useobject{currentmarker}{}%
\end{pgfscope}%
\begin{pgfscope}%
\pgfsys@transformshift{2.300399in}{0.669478in}%
\pgfsys@useobject{currentmarker}{}%
\end{pgfscope}%
\begin{pgfscope}%
\pgfsys@transformshift{2.300515in}{0.668758in}%
\pgfsys@useobject{currentmarker}{}%
\end{pgfscope}%
\begin{pgfscope}%
\pgfsys@transformshift{2.300630in}{0.589486in}%
\pgfsys@useobject{currentmarker}{}%
\end{pgfscope}%
\begin{pgfscope}%
\pgfsys@transformshift{2.300746in}{0.648213in}%
\pgfsys@useobject{currentmarker}{}%
\end{pgfscope}%
\begin{pgfscope}%
\pgfsys@transformshift{2.300862in}{0.651395in}%
\pgfsys@useobject{currentmarker}{}%
\end{pgfscope}%
\begin{pgfscope}%
\pgfsys@transformshift{2.300977in}{0.631733in}%
\pgfsys@useobject{currentmarker}{}%
\end{pgfscope}%
\begin{pgfscope}%
\pgfsys@transformshift{2.301093in}{0.668036in}%
\pgfsys@useobject{currentmarker}{}%
\end{pgfscope}%
\begin{pgfscope}%
\pgfsys@transformshift{2.301209in}{0.669826in}%
\pgfsys@useobject{currentmarker}{}%
\end{pgfscope}%
\begin{pgfscope}%
\pgfsys@transformshift{2.301324in}{0.682665in}%
\pgfsys@useobject{currentmarker}{}%
\end{pgfscope}%
\begin{pgfscope}%
\pgfsys@transformshift{2.301439in}{0.629785in}%
\pgfsys@useobject{currentmarker}{}%
\end{pgfscope}%
\begin{pgfscope}%
\pgfsys@transformshift{2.301555in}{0.690100in}%
\pgfsys@useobject{currentmarker}{}%
\end{pgfscope}%
\begin{pgfscope}%
\pgfsys@transformshift{2.301670in}{0.704950in}%
\pgfsys@useobject{currentmarker}{}%
\end{pgfscope}%
\begin{pgfscope}%
\pgfsys@transformshift{2.301785in}{0.656226in}%
\pgfsys@useobject{currentmarker}{}%
\end{pgfscope}%
\begin{pgfscope}%
\pgfsys@transformshift{2.301901in}{0.635383in}%
\pgfsys@useobject{currentmarker}{}%
\end{pgfscope}%
\begin{pgfscope}%
\pgfsys@transformshift{2.302016in}{0.590936in}%
\pgfsys@useobject{currentmarker}{}%
\end{pgfscope}%
\begin{pgfscope}%
\pgfsys@transformshift{2.302131in}{0.626375in}%
\pgfsys@useobject{currentmarker}{}%
\end{pgfscope}%
\begin{pgfscope}%
\pgfsys@transformshift{2.302246in}{0.681330in}%
\pgfsys@useobject{currentmarker}{}%
\end{pgfscope}%
\begin{pgfscope}%
\pgfsys@transformshift{2.302361in}{0.670729in}%
\pgfsys@useobject{currentmarker}{}%
\end{pgfscope}%
\begin{pgfscope}%
\pgfsys@transformshift{2.302476in}{0.722981in}%
\pgfsys@useobject{currentmarker}{}%
\end{pgfscope}%
\begin{pgfscope}%
\pgfsys@transformshift{2.302590in}{0.702981in}%
\pgfsys@useobject{currentmarker}{}%
\end{pgfscope}%
\begin{pgfscope}%
\pgfsys@transformshift{2.302705in}{0.655000in}%
\pgfsys@useobject{currentmarker}{}%
\end{pgfscope}%
\begin{pgfscope}%
\pgfsys@transformshift{2.302820in}{0.680883in}%
\pgfsys@useobject{currentmarker}{}%
\end{pgfscope}%
\begin{pgfscope}%
\pgfsys@transformshift{2.302934in}{0.670715in}%
\pgfsys@useobject{currentmarker}{}%
\end{pgfscope}%
\begin{pgfscope}%
\pgfsys@transformshift{2.303049in}{0.657788in}%
\pgfsys@useobject{currentmarker}{}%
\end{pgfscope}%
\begin{pgfscope}%
\pgfsys@transformshift{2.303164in}{0.674681in}%
\pgfsys@useobject{currentmarker}{}%
\end{pgfscope}%
\begin{pgfscope}%
\pgfsys@transformshift{2.303278in}{0.653235in}%
\pgfsys@useobject{currentmarker}{}%
\end{pgfscope}%
\begin{pgfscope}%
\pgfsys@transformshift{2.303392in}{0.673450in}%
\pgfsys@useobject{currentmarker}{}%
\end{pgfscope}%
\begin{pgfscope}%
\pgfsys@transformshift{2.303507in}{0.647342in}%
\pgfsys@useobject{currentmarker}{}%
\end{pgfscope}%
\begin{pgfscope}%
\pgfsys@transformshift{2.303621in}{0.672778in}%
\pgfsys@useobject{currentmarker}{}%
\end{pgfscope}%
\begin{pgfscope}%
\pgfsys@transformshift{2.303735in}{0.663151in}%
\pgfsys@useobject{currentmarker}{}%
\end{pgfscope}%
\begin{pgfscope}%
\pgfsys@transformshift{2.303850in}{0.660736in}%
\pgfsys@useobject{currentmarker}{}%
\end{pgfscope}%
\begin{pgfscope}%
\pgfsys@transformshift{2.303964in}{0.661568in}%
\pgfsys@useobject{currentmarker}{}%
\end{pgfscope}%
\begin{pgfscope}%
\pgfsys@transformshift{2.304078in}{0.681546in}%
\pgfsys@useobject{currentmarker}{}%
\end{pgfscope}%
\begin{pgfscope}%
\pgfsys@transformshift{2.304192in}{0.662191in}%
\pgfsys@useobject{currentmarker}{}%
\end{pgfscope}%
\begin{pgfscope}%
\pgfsys@transformshift{2.304306in}{0.663871in}%
\pgfsys@useobject{currentmarker}{}%
\end{pgfscope}%
\begin{pgfscope}%
\pgfsys@transformshift{2.304420in}{0.636050in}%
\pgfsys@useobject{currentmarker}{}%
\end{pgfscope}%
\begin{pgfscope}%
\pgfsys@transformshift{2.304533in}{0.598852in}%
\pgfsys@useobject{currentmarker}{}%
\end{pgfscope}%
\begin{pgfscope}%
\pgfsys@transformshift{2.304647in}{0.626567in}%
\pgfsys@useobject{currentmarker}{}%
\end{pgfscope}%
\begin{pgfscope}%
\pgfsys@transformshift{2.304761in}{0.641927in}%
\pgfsys@useobject{currentmarker}{}%
\end{pgfscope}%
\begin{pgfscope}%
\pgfsys@transformshift{2.304875in}{0.652052in}%
\pgfsys@useobject{currentmarker}{}%
\end{pgfscope}%
\begin{pgfscope}%
\pgfsys@transformshift{2.304988in}{0.664075in}%
\pgfsys@useobject{currentmarker}{}%
\end{pgfscope}%
\begin{pgfscope}%
\pgfsys@transformshift{2.305102in}{0.685435in}%
\pgfsys@useobject{currentmarker}{}%
\end{pgfscope}%
\begin{pgfscope}%
\pgfsys@transformshift{2.305215in}{0.682574in}%
\pgfsys@useobject{currentmarker}{}%
\end{pgfscope}%
\begin{pgfscope}%
\pgfsys@transformshift{2.305329in}{0.634659in}%
\pgfsys@useobject{currentmarker}{}%
\end{pgfscope}%
\begin{pgfscope}%
\pgfsys@transformshift{2.305442in}{0.635693in}%
\pgfsys@useobject{currentmarker}{}%
\end{pgfscope}%
\begin{pgfscope}%
\pgfsys@transformshift{2.305555in}{0.665719in}%
\pgfsys@useobject{currentmarker}{}%
\end{pgfscope}%
\begin{pgfscope}%
\pgfsys@transformshift{2.305669in}{0.701529in}%
\pgfsys@useobject{currentmarker}{}%
\end{pgfscope}%
\begin{pgfscope}%
\pgfsys@transformshift{2.305782in}{0.661021in}%
\pgfsys@useobject{currentmarker}{}%
\end{pgfscope}%
\begin{pgfscope}%
\pgfsys@transformshift{2.305895in}{0.699278in}%
\pgfsys@useobject{currentmarker}{}%
\end{pgfscope}%
\begin{pgfscope}%
\pgfsys@transformshift{2.306008in}{0.648960in}%
\pgfsys@useobject{currentmarker}{}%
\end{pgfscope}%
\begin{pgfscope}%
\pgfsys@transformshift{2.306121in}{0.650623in}%
\pgfsys@useobject{currentmarker}{}%
\end{pgfscope}%
\begin{pgfscope}%
\pgfsys@transformshift{2.306234in}{0.651458in}%
\pgfsys@useobject{currentmarker}{}%
\end{pgfscope}%
\begin{pgfscope}%
\pgfsys@transformshift{2.306347in}{0.613825in}%
\pgfsys@useobject{currentmarker}{}%
\end{pgfscope}%
\begin{pgfscope}%
\pgfsys@transformshift{2.306460in}{0.586209in}%
\pgfsys@useobject{currentmarker}{}%
\end{pgfscope}%
\begin{pgfscope}%
\pgfsys@transformshift{2.306573in}{0.670083in}%
\pgfsys@useobject{currentmarker}{}%
\end{pgfscope}%
\begin{pgfscope}%
\pgfsys@transformshift{2.306685in}{0.672948in}%
\pgfsys@useobject{currentmarker}{}%
\end{pgfscope}%
\begin{pgfscope}%
\pgfsys@transformshift{2.306798in}{0.645266in}%
\pgfsys@useobject{currentmarker}{}%
\end{pgfscope}%
\begin{pgfscope}%
\pgfsys@transformshift{2.306911in}{0.642729in}%
\pgfsys@useobject{currentmarker}{}%
\end{pgfscope}%
\begin{pgfscope}%
\pgfsys@transformshift{2.307023in}{0.668373in}%
\pgfsys@useobject{currentmarker}{}%
\end{pgfscope}%
\begin{pgfscope}%
\pgfsys@transformshift{2.307136in}{0.699892in}%
\pgfsys@useobject{currentmarker}{}%
\end{pgfscope}%
\begin{pgfscope}%
\pgfsys@transformshift{2.307248in}{0.719898in}%
\pgfsys@useobject{currentmarker}{}%
\end{pgfscope}%
\begin{pgfscope}%
\pgfsys@transformshift{2.307361in}{0.689919in}%
\pgfsys@useobject{currentmarker}{}%
\end{pgfscope}%
\begin{pgfscope}%
\pgfsys@transformshift{2.307473in}{0.687589in}%
\pgfsys@useobject{currentmarker}{}%
\end{pgfscope}%
\begin{pgfscope}%
\pgfsys@transformshift{2.307585in}{0.683352in}%
\pgfsys@useobject{currentmarker}{}%
\end{pgfscope}%
\begin{pgfscope}%
\pgfsys@transformshift{2.307698in}{0.662579in}%
\pgfsys@useobject{currentmarker}{}%
\end{pgfscope}%
\begin{pgfscope}%
\pgfsys@transformshift{2.307810in}{0.678877in}%
\pgfsys@useobject{currentmarker}{}%
\end{pgfscope}%
\begin{pgfscope}%
\pgfsys@transformshift{2.307922in}{0.610988in}%
\pgfsys@useobject{currentmarker}{}%
\end{pgfscope}%
\begin{pgfscope}%
\pgfsys@transformshift{2.308034in}{0.633629in}%
\pgfsys@useobject{currentmarker}{}%
\end{pgfscope}%
\begin{pgfscope}%
\pgfsys@transformshift{2.308146in}{0.634300in}%
\pgfsys@useobject{currentmarker}{}%
\end{pgfscope}%
\begin{pgfscope}%
\pgfsys@transformshift{2.308258in}{0.655192in}%
\pgfsys@useobject{currentmarker}{}%
\end{pgfscope}%
\begin{pgfscope}%
\pgfsys@transformshift{2.308370in}{0.666752in}%
\pgfsys@useobject{currentmarker}{}%
\end{pgfscope}%
\begin{pgfscope}%
\pgfsys@transformshift{2.308482in}{0.648257in}%
\pgfsys@useobject{currentmarker}{}%
\end{pgfscope}%
\begin{pgfscope}%
\pgfsys@transformshift{2.308594in}{0.666169in}%
\pgfsys@useobject{currentmarker}{}%
\end{pgfscope}%
\begin{pgfscope}%
\pgfsys@transformshift{2.308705in}{0.673702in}%
\pgfsys@useobject{currentmarker}{}%
\end{pgfscope}%
\begin{pgfscope}%
\pgfsys@transformshift{2.308817in}{0.616075in}%
\pgfsys@useobject{currentmarker}{}%
\end{pgfscope}%
\begin{pgfscope}%
\pgfsys@transformshift{2.308929in}{0.711626in}%
\pgfsys@useobject{currentmarker}{}%
\end{pgfscope}%
\begin{pgfscope}%
\pgfsys@transformshift{2.309040in}{0.692222in}%
\pgfsys@useobject{currentmarker}{}%
\end{pgfscope}%
\begin{pgfscope}%
\pgfsys@transformshift{2.309152in}{0.689041in}%
\pgfsys@useobject{currentmarker}{}%
\end{pgfscope}%
\begin{pgfscope}%
\pgfsys@transformshift{2.309263in}{0.700235in}%
\pgfsys@useobject{currentmarker}{}%
\end{pgfscope}%
\begin{pgfscope}%
\pgfsys@transformshift{2.309375in}{0.649461in}%
\pgfsys@useobject{currentmarker}{}%
\end{pgfscope}%
\begin{pgfscope}%
\pgfsys@transformshift{2.309486in}{0.677203in}%
\pgfsys@useobject{currentmarker}{}%
\end{pgfscope}%
\begin{pgfscope}%
\pgfsys@transformshift{2.309597in}{0.691380in}%
\pgfsys@useobject{currentmarker}{}%
\end{pgfscope}%
\begin{pgfscope}%
\pgfsys@transformshift{2.309708in}{0.638386in}%
\pgfsys@useobject{currentmarker}{}%
\end{pgfscope}%
\begin{pgfscope}%
\pgfsys@transformshift{2.309820in}{0.626961in}%
\pgfsys@useobject{currentmarker}{}%
\end{pgfscope}%
\begin{pgfscope}%
\pgfsys@transformshift{2.309931in}{0.677970in}%
\pgfsys@useobject{currentmarker}{}%
\end{pgfscope}%
\begin{pgfscope}%
\pgfsys@transformshift{2.310042in}{0.633506in}%
\pgfsys@useobject{currentmarker}{}%
\end{pgfscope}%
\begin{pgfscope}%
\pgfsys@transformshift{2.310153in}{0.605629in}%
\pgfsys@useobject{currentmarker}{}%
\end{pgfscope}%
\begin{pgfscope}%
\pgfsys@transformshift{2.310264in}{0.670713in}%
\pgfsys@useobject{currentmarker}{}%
\end{pgfscope}%
\begin{pgfscope}%
\pgfsys@transformshift{2.310375in}{0.670991in}%
\pgfsys@useobject{currentmarker}{}%
\end{pgfscope}%
\begin{pgfscope}%
\pgfsys@transformshift{2.310486in}{0.680648in}%
\pgfsys@useobject{currentmarker}{}%
\end{pgfscope}%
\begin{pgfscope}%
\pgfsys@transformshift{2.310596in}{0.602741in}%
\pgfsys@useobject{currentmarker}{}%
\end{pgfscope}%
\begin{pgfscope}%
\pgfsys@transformshift{2.310707in}{0.631625in}%
\pgfsys@useobject{currentmarker}{}%
\end{pgfscope}%
\begin{pgfscope}%
\pgfsys@transformshift{2.310818in}{0.629701in}%
\pgfsys@useobject{currentmarker}{}%
\end{pgfscope}%
\begin{pgfscope}%
\pgfsys@transformshift{2.310928in}{0.643290in}%
\pgfsys@useobject{currentmarker}{}%
\end{pgfscope}%
\begin{pgfscope}%
\pgfsys@transformshift{2.311039in}{0.642079in}%
\pgfsys@useobject{currentmarker}{}%
\end{pgfscope}%
\begin{pgfscope}%
\pgfsys@transformshift{2.311150in}{0.652611in}%
\pgfsys@useobject{currentmarker}{}%
\end{pgfscope}%
\begin{pgfscope}%
\pgfsys@transformshift{2.311260in}{0.602453in}%
\pgfsys@useobject{currentmarker}{}%
\end{pgfscope}%
\begin{pgfscope}%
\pgfsys@transformshift{2.311370in}{0.654620in}%
\pgfsys@useobject{currentmarker}{}%
\end{pgfscope}%
\begin{pgfscope}%
\pgfsys@transformshift{2.311481in}{0.641068in}%
\pgfsys@useobject{currentmarker}{}%
\end{pgfscope}%
\begin{pgfscope}%
\pgfsys@transformshift{2.311591in}{0.647847in}%
\pgfsys@useobject{currentmarker}{}%
\end{pgfscope}%
\begin{pgfscope}%
\pgfsys@transformshift{2.311701in}{0.647023in}%
\pgfsys@useobject{currentmarker}{}%
\end{pgfscope}%
\begin{pgfscope}%
\pgfsys@transformshift{2.311812in}{0.683638in}%
\pgfsys@useobject{currentmarker}{}%
\end{pgfscope}%
\begin{pgfscope}%
\pgfsys@transformshift{2.311922in}{0.675673in}%
\pgfsys@useobject{currentmarker}{}%
\end{pgfscope}%
\begin{pgfscope}%
\pgfsys@transformshift{2.312032in}{0.677050in}%
\pgfsys@useobject{currentmarker}{}%
\end{pgfscope}%
\begin{pgfscope}%
\pgfsys@transformshift{2.312142in}{0.715185in}%
\pgfsys@useobject{currentmarker}{}%
\end{pgfscope}%
\begin{pgfscope}%
\pgfsys@transformshift{2.312252in}{0.704471in}%
\pgfsys@useobject{currentmarker}{}%
\end{pgfscope}%
\begin{pgfscope}%
\pgfsys@transformshift{2.312362in}{0.719565in}%
\pgfsys@useobject{currentmarker}{}%
\end{pgfscope}%
\begin{pgfscope}%
\pgfsys@transformshift{2.312472in}{0.703206in}%
\pgfsys@useobject{currentmarker}{}%
\end{pgfscope}%
\begin{pgfscope}%
\pgfsys@transformshift{2.312582in}{0.667330in}%
\pgfsys@useobject{currentmarker}{}%
\end{pgfscope}%
\begin{pgfscope}%
\pgfsys@transformshift{2.312691in}{0.635971in}%
\pgfsys@useobject{currentmarker}{}%
\end{pgfscope}%
\end{pgfscope}%
\begin{pgfscope}%
\pgfsetrectcap%
\pgfsetmiterjoin%
\pgfsetlinewidth{0.803000pt}%
\definecolor{currentstroke}{rgb}{0.000000,0.000000,0.000000}%
\pgfsetstrokecolor{currentstroke}%
\pgfsetdash{}{0pt}%
\pgfpathmoveto{\pgfqpoint{0.514278in}{0.417642in}}%
\pgfpathlineto{\pgfqpoint{0.514278in}{1.788330in}}%
\pgfusepath{stroke}%
\end{pgfscope}%
\begin{pgfscope}%
\pgfsetrectcap%
\pgfsetmiterjoin%
\pgfsetlinewidth{0.803000pt}%
\definecolor{currentstroke}{rgb}{0.000000,0.000000,0.000000}%
\pgfsetstrokecolor{currentstroke}%
\pgfsetdash{}{0pt}%
\pgfpathmoveto{\pgfqpoint{2.398330in}{0.417642in}}%
\pgfpathlineto{\pgfqpoint{2.398330in}{1.788330in}}%
\pgfusepath{stroke}%
\end{pgfscope}%
\begin{pgfscope}%
\pgfsetrectcap%
\pgfsetmiterjoin%
\pgfsetlinewidth{0.803000pt}%
\definecolor{currentstroke}{rgb}{0.000000,0.000000,0.000000}%
\pgfsetstrokecolor{currentstroke}%
\pgfsetdash{}{0pt}%
\pgfpathmoveto{\pgfqpoint{0.514278in}{0.417642in}}%
\pgfpathlineto{\pgfqpoint{2.398330in}{0.417642in}}%
\pgfusepath{stroke}%
\end{pgfscope}%
\begin{pgfscope}%
\pgfsetrectcap%
\pgfsetmiterjoin%
\pgfsetlinewidth{0.803000pt}%
\definecolor{currentstroke}{rgb}{0.000000,0.000000,0.000000}%
\pgfsetstrokecolor{currentstroke}%
\pgfsetdash{}{0pt}%
\pgfpathmoveto{\pgfqpoint{0.514278in}{1.788330in}}%
\pgfpathlineto{\pgfqpoint{2.398330in}{1.788330in}}%
\pgfusepath{stroke}%
\end{pgfscope}%
\begin{pgfscope}%
\pgfsetbuttcap%
\pgfsetmiterjoin%
\definecolor{currentfill}{rgb}{1.000000,1.000000,1.000000}%
\pgfsetfillcolor{currentfill}%
\pgfsetfillopacity{0.800000}%
\pgfsetlinewidth{1.003750pt}%
\definecolor{currentstroke}{rgb}{0.800000,0.800000,0.800000}%
\pgfsetstrokecolor{currentstroke}%
\pgfsetstrokeopacity{0.800000}%
\pgfsetdash{}{0pt}%
\pgfpathmoveto{\pgfqpoint{1.551772in}{1.517019in}}%
\pgfpathlineto{\pgfqpoint{2.320552in}{1.517019in}}%
\pgfpathquadraticcurveto{\pgfqpoint{2.342774in}{1.517019in}}{\pgfqpoint{2.342774in}{1.539241in}}%
\pgfpathlineto{\pgfqpoint{2.342774in}{1.710552in}}%
\pgfpathquadraticcurveto{\pgfqpoint{2.342774in}{1.732774in}}{\pgfqpoint{2.320552in}{1.732774in}}%
\pgfpathlineto{\pgfqpoint{1.551772in}{1.732774in}}%
\pgfpathquadraticcurveto{\pgfqpoint{1.529549in}{1.732774in}}{\pgfqpoint{1.529549in}{1.710552in}}%
\pgfpathlineto{\pgfqpoint{1.529549in}{1.539241in}}%
\pgfpathquadraticcurveto{\pgfqpoint{1.529549in}{1.517019in}}{\pgfqpoint{1.551772in}{1.517019in}}%
\pgfpathlineto{\pgfqpoint{1.551772in}{1.517019in}}%
\pgfpathclose%
\pgfusepath{stroke,fill}%
\end{pgfscope}%
\begin{pgfscope}%
\pgfsetbuttcap%
\pgfsetroundjoin%
\pgfsetlinewidth{1.505625pt}%
\definecolor{currentstroke}{rgb}{0.835294,0.368627,0.000000}%
\pgfsetstrokecolor{currentstroke}%
\pgfsetdash{{5.550000pt}{2.400000pt}}{0.000000pt}%
\pgfpathmoveto{\pgfqpoint{1.573994in}{1.627358in}}%
\pgfpathlineto{\pgfqpoint{1.685105in}{1.627358in}}%
\pgfpathlineto{\pgfqpoint{1.796216in}{1.627358in}}%
\pgfusepath{stroke}%
\end{pgfscope}%
\begin{pgfscope}%
\definecolor{textcolor}{rgb}{0.000000,0.000000,0.000000}%
\pgfsetstrokecolor{textcolor}%
\pgfsetfillcolor{textcolor}%
\pgftext[x=1.885105in,y=1.588469in,left,base]{\color{textcolor}\rmfamily\fontsize{8.000000}{9.600000}\selectfont \(\displaystyle h_{-2}f^{-2}\)}%
\end{pgfscope}%
\end{pgfpicture}%
\makeatother%
\endgroup%

        } % scalebox
        \caption{Power spectral density}
        \label{fig:random_walk_psd}
    \end{subfigure}
    \begin{subfigure}{0.32\linewidth}
        \centering
        \scalebox{0.75}{%
            %% Creator: Matplotlib, PGF backend
%%
%% To include the figure in your LaTeX document, write
%%   \input{<filename>.pgf}
%%
%% Make sure the required packages are loaded in your preamble
%%   \usepackage{pgf}
%%
%% Also ensure that all the required font packages are loaded; for instance,
%% the lmodern package is sometimes necessary when using math font.
%%   \usepackage{lmodern}
%%
%% Figures using additional raster images can only be included by \input if
%% they are in the same directory as the main LaTeX file. For loading figures
%% from other directories you can use the `import` package
%%   \usepackage{import}
%%
%% and then include the figures with
%%   \import{<path to file>}{<filename>.pgf}
%%
%% Matplotlib used the following preamble
%%   \usepackage{siunitx}
%%   \sisetup{per-mode = symbol}%
%%   \usepackage{fontspec}
%%   \makeatletter\@ifpackageloaded{underscore}{}{\usepackage[strings]{underscore}}\makeatother
%%
\begingroup%
\makeatletter%
\begin{pgfpicture}%
\pgfpathrectangle{\pgfpointorigin}{\pgfqpoint{2.440945in}{1.830709in}}%
\pgfusepath{use as bounding box, clip}%
\begin{pgfscope}%
\pgfsetbuttcap%
\pgfsetmiterjoin%
\definecolor{currentfill}{rgb}{1.000000,1.000000,1.000000}%
\pgfsetfillcolor{currentfill}%
\pgfsetlinewidth{0.000000pt}%
\definecolor{currentstroke}{rgb}{1.000000,1.000000,1.000000}%
\pgfsetstrokecolor{currentstroke}%
\pgfsetdash{}{0pt}%
\pgfpathmoveto{\pgfqpoint{0.000000in}{0.000000in}}%
\pgfpathlineto{\pgfqpoint{2.440945in}{0.000000in}}%
\pgfpathlineto{\pgfqpoint{2.440945in}{1.830709in}}%
\pgfpathlineto{\pgfqpoint{0.000000in}{1.830709in}}%
\pgfpathlineto{\pgfqpoint{0.000000in}{0.000000in}}%
\pgfpathclose%
\pgfusepath{fill}%
\end{pgfscope}%
\begin{pgfscope}%
\pgfsetbuttcap%
\pgfsetmiterjoin%
\definecolor{currentfill}{rgb}{1.000000,1.000000,1.000000}%
\pgfsetfillcolor{currentfill}%
\pgfsetlinewidth{0.000000pt}%
\definecolor{currentstroke}{rgb}{0.000000,0.000000,0.000000}%
\pgfsetstrokecolor{currentstroke}%
\pgfsetstrokeopacity{0.000000}%
\pgfsetdash{}{0pt}%
\pgfpathmoveto{\pgfqpoint{0.589510in}{0.417642in}}%
\pgfpathlineto{\pgfqpoint{2.399275in}{0.417642in}}%
\pgfpathlineto{\pgfqpoint{2.399275in}{1.789039in}}%
\pgfpathlineto{\pgfqpoint{0.589510in}{1.789039in}}%
\pgfpathlineto{\pgfqpoint{0.589510in}{0.417642in}}%
\pgfpathclose%
\pgfusepath{fill}%
\end{pgfscope}%
\begin{pgfscope}%
\pgfpathrectangle{\pgfqpoint{0.589510in}{0.417642in}}{\pgfqpoint{1.809765in}{1.371397in}}%
\pgfusepath{clip}%
\pgfsetrectcap%
\pgfsetroundjoin%
\pgfsetlinewidth{0.803000pt}%
\definecolor{currentstroke}{rgb}{0.450000,0.450000,0.450000}%
\pgfsetstrokecolor{currentstroke}%
\pgfsetdash{}{0pt}%
\pgfpathmoveto{\pgfqpoint{0.671772in}{0.417642in}}%
\pgfpathlineto{\pgfqpoint{0.671772in}{1.789039in}}%
\pgfusepath{stroke}%
\end{pgfscope}%
\begin{pgfscope}%
\pgfsetbuttcap%
\pgfsetroundjoin%
\definecolor{currentfill}{rgb}{0.000000,0.000000,0.000000}%
\pgfsetfillcolor{currentfill}%
\pgfsetlinewidth{0.803000pt}%
\definecolor{currentstroke}{rgb}{0.000000,0.000000,0.000000}%
\pgfsetstrokecolor{currentstroke}%
\pgfsetdash{}{0pt}%
\pgfsys@defobject{currentmarker}{\pgfqpoint{0.000000in}{-0.048611in}}{\pgfqpoint{0.000000in}{0.000000in}}{%
\pgfpathmoveto{\pgfqpoint{0.000000in}{0.000000in}}%
\pgfpathlineto{\pgfqpoint{0.000000in}{-0.048611in}}%
\pgfusepath{stroke,fill}%
}%
\begin{pgfscope}%
\pgfsys@transformshift{0.671772in}{0.417642in}%
\pgfsys@useobject{currentmarker}{}%
\end{pgfscope}%
\end{pgfscope}%
\begin{pgfscope}%
\definecolor{textcolor}{rgb}{0.000000,0.000000,0.000000}%
\pgfsetstrokecolor{textcolor}%
\pgfsetfillcolor{textcolor}%
\pgftext[x=0.671772in,y=0.320420in,,top]{\color{textcolor}\rmfamily\fontsize{8.000000}{9.600000}\selectfont \(\displaystyle {10^{0}}\)}%
\end{pgfscope}%
\begin{pgfscope}%
\pgfpathrectangle{\pgfqpoint{0.589510in}{0.417642in}}{\pgfqpoint{1.809765in}{1.371397in}}%
\pgfusepath{clip}%
\pgfsetrectcap%
\pgfsetroundjoin%
\pgfsetlinewidth{0.803000pt}%
\definecolor{currentstroke}{rgb}{0.450000,0.450000,0.450000}%
\pgfsetstrokecolor{currentstroke}%
\pgfsetdash{}{0pt}%
\pgfpathmoveto{\pgfqpoint{1.128522in}{0.417642in}}%
\pgfpathlineto{\pgfqpoint{1.128522in}{1.789039in}}%
\pgfusepath{stroke}%
\end{pgfscope}%
\begin{pgfscope}%
\pgfsetbuttcap%
\pgfsetroundjoin%
\definecolor{currentfill}{rgb}{0.000000,0.000000,0.000000}%
\pgfsetfillcolor{currentfill}%
\pgfsetlinewidth{0.803000pt}%
\definecolor{currentstroke}{rgb}{0.000000,0.000000,0.000000}%
\pgfsetstrokecolor{currentstroke}%
\pgfsetdash{}{0pt}%
\pgfsys@defobject{currentmarker}{\pgfqpoint{0.000000in}{-0.048611in}}{\pgfqpoint{0.000000in}{0.000000in}}{%
\pgfpathmoveto{\pgfqpoint{0.000000in}{0.000000in}}%
\pgfpathlineto{\pgfqpoint{0.000000in}{-0.048611in}}%
\pgfusepath{stroke,fill}%
}%
\begin{pgfscope}%
\pgfsys@transformshift{1.128522in}{0.417642in}%
\pgfsys@useobject{currentmarker}{}%
\end{pgfscope}%
\end{pgfscope}%
\begin{pgfscope}%
\definecolor{textcolor}{rgb}{0.000000,0.000000,0.000000}%
\pgfsetstrokecolor{textcolor}%
\pgfsetfillcolor{textcolor}%
\pgftext[x=1.128522in,y=0.320420in,,top]{\color{textcolor}\rmfamily\fontsize{8.000000}{9.600000}\selectfont \(\displaystyle {10^{1}}\)}%
\end{pgfscope}%
\begin{pgfscope}%
\pgfpathrectangle{\pgfqpoint{0.589510in}{0.417642in}}{\pgfqpoint{1.809765in}{1.371397in}}%
\pgfusepath{clip}%
\pgfsetrectcap%
\pgfsetroundjoin%
\pgfsetlinewidth{0.803000pt}%
\definecolor{currentstroke}{rgb}{0.450000,0.450000,0.450000}%
\pgfsetstrokecolor{currentstroke}%
\pgfsetdash{}{0pt}%
\pgfpathmoveto{\pgfqpoint{1.585272in}{0.417642in}}%
\pgfpathlineto{\pgfqpoint{1.585272in}{1.789039in}}%
\pgfusepath{stroke}%
\end{pgfscope}%
\begin{pgfscope}%
\pgfsetbuttcap%
\pgfsetroundjoin%
\definecolor{currentfill}{rgb}{0.000000,0.000000,0.000000}%
\pgfsetfillcolor{currentfill}%
\pgfsetlinewidth{0.803000pt}%
\definecolor{currentstroke}{rgb}{0.000000,0.000000,0.000000}%
\pgfsetstrokecolor{currentstroke}%
\pgfsetdash{}{0pt}%
\pgfsys@defobject{currentmarker}{\pgfqpoint{0.000000in}{-0.048611in}}{\pgfqpoint{0.000000in}{0.000000in}}{%
\pgfpathmoveto{\pgfqpoint{0.000000in}{0.000000in}}%
\pgfpathlineto{\pgfqpoint{0.000000in}{-0.048611in}}%
\pgfusepath{stroke,fill}%
}%
\begin{pgfscope}%
\pgfsys@transformshift{1.585272in}{0.417642in}%
\pgfsys@useobject{currentmarker}{}%
\end{pgfscope}%
\end{pgfscope}%
\begin{pgfscope}%
\definecolor{textcolor}{rgb}{0.000000,0.000000,0.000000}%
\pgfsetstrokecolor{textcolor}%
\pgfsetfillcolor{textcolor}%
\pgftext[x=1.585272in,y=0.320420in,,top]{\color{textcolor}\rmfamily\fontsize{8.000000}{9.600000}\selectfont \(\displaystyle {10^{2}}\)}%
\end{pgfscope}%
\begin{pgfscope}%
\pgfpathrectangle{\pgfqpoint{0.589510in}{0.417642in}}{\pgfqpoint{1.809765in}{1.371397in}}%
\pgfusepath{clip}%
\pgfsetrectcap%
\pgfsetroundjoin%
\pgfsetlinewidth{0.803000pt}%
\definecolor{currentstroke}{rgb}{0.450000,0.450000,0.450000}%
\pgfsetstrokecolor{currentstroke}%
\pgfsetdash{}{0pt}%
\pgfpathmoveto{\pgfqpoint{2.042022in}{0.417642in}}%
\pgfpathlineto{\pgfqpoint{2.042022in}{1.789039in}}%
\pgfusepath{stroke}%
\end{pgfscope}%
\begin{pgfscope}%
\pgfsetbuttcap%
\pgfsetroundjoin%
\definecolor{currentfill}{rgb}{0.000000,0.000000,0.000000}%
\pgfsetfillcolor{currentfill}%
\pgfsetlinewidth{0.803000pt}%
\definecolor{currentstroke}{rgb}{0.000000,0.000000,0.000000}%
\pgfsetstrokecolor{currentstroke}%
\pgfsetdash{}{0pt}%
\pgfsys@defobject{currentmarker}{\pgfqpoint{0.000000in}{-0.048611in}}{\pgfqpoint{0.000000in}{0.000000in}}{%
\pgfpathmoveto{\pgfqpoint{0.000000in}{0.000000in}}%
\pgfpathlineto{\pgfqpoint{0.000000in}{-0.048611in}}%
\pgfusepath{stroke,fill}%
}%
\begin{pgfscope}%
\pgfsys@transformshift{2.042022in}{0.417642in}%
\pgfsys@useobject{currentmarker}{}%
\end{pgfscope}%
\end{pgfscope}%
\begin{pgfscope}%
\definecolor{textcolor}{rgb}{0.000000,0.000000,0.000000}%
\pgfsetstrokecolor{textcolor}%
\pgfsetfillcolor{textcolor}%
\pgftext[x=2.042022in,y=0.320420in,,top]{\color{textcolor}\rmfamily\fontsize{8.000000}{9.600000}\selectfont \(\displaystyle {10^{3}}\)}%
\end{pgfscope}%
\begin{pgfscope}%
\pgfpathrectangle{\pgfqpoint{0.589510in}{0.417642in}}{\pgfqpoint{1.809765in}{1.371397in}}%
\pgfusepath{clip}%
\pgfsetrectcap%
\pgfsetroundjoin%
\pgfsetlinewidth{0.803000pt}%
\definecolor{currentstroke}{rgb}{0.850000,0.850000,0.850000}%
\pgfsetstrokecolor{currentstroke}%
\pgfsetdash{}{0pt}%
\pgfpathmoveto{\pgfqpoint{0.601020in}{0.417642in}}%
\pgfpathlineto{\pgfqpoint{0.601020in}{1.789039in}}%
\pgfusepath{stroke}%
\end{pgfscope}%
\begin{pgfscope}%
\pgfsetbuttcap%
\pgfsetroundjoin%
\definecolor{currentfill}{rgb}{0.000000,0.000000,0.000000}%
\pgfsetfillcolor{currentfill}%
\pgfsetlinewidth{0.602250pt}%
\definecolor{currentstroke}{rgb}{0.000000,0.000000,0.000000}%
\pgfsetstrokecolor{currentstroke}%
\pgfsetdash{}{0pt}%
\pgfsys@defobject{currentmarker}{\pgfqpoint{0.000000in}{-0.027778in}}{\pgfqpoint{0.000000in}{0.000000in}}{%
\pgfpathmoveto{\pgfqpoint{0.000000in}{0.000000in}}%
\pgfpathlineto{\pgfqpoint{0.000000in}{-0.027778in}}%
\pgfusepath{stroke,fill}%
}%
\begin{pgfscope}%
\pgfsys@transformshift{0.601020in}{0.417642in}%
\pgfsys@useobject{currentmarker}{}%
\end{pgfscope}%
\end{pgfscope}%
\begin{pgfscope}%
\pgfpathrectangle{\pgfqpoint{0.589510in}{0.417642in}}{\pgfqpoint{1.809765in}{1.371397in}}%
\pgfusepath{clip}%
\pgfsetrectcap%
\pgfsetroundjoin%
\pgfsetlinewidth{0.803000pt}%
\definecolor{currentstroke}{rgb}{0.850000,0.850000,0.850000}%
\pgfsetstrokecolor{currentstroke}%
\pgfsetdash{}{0pt}%
\pgfpathmoveto{\pgfqpoint{0.627508in}{0.417642in}}%
\pgfpathlineto{\pgfqpoint{0.627508in}{1.789039in}}%
\pgfusepath{stroke}%
\end{pgfscope}%
\begin{pgfscope}%
\pgfsetbuttcap%
\pgfsetroundjoin%
\definecolor{currentfill}{rgb}{0.000000,0.000000,0.000000}%
\pgfsetfillcolor{currentfill}%
\pgfsetlinewidth{0.602250pt}%
\definecolor{currentstroke}{rgb}{0.000000,0.000000,0.000000}%
\pgfsetstrokecolor{currentstroke}%
\pgfsetdash{}{0pt}%
\pgfsys@defobject{currentmarker}{\pgfqpoint{0.000000in}{-0.027778in}}{\pgfqpoint{0.000000in}{0.000000in}}{%
\pgfpathmoveto{\pgfqpoint{0.000000in}{0.000000in}}%
\pgfpathlineto{\pgfqpoint{0.000000in}{-0.027778in}}%
\pgfusepath{stroke,fill}%
}%
\begin{pgfscope}%
\pgfsys@transformshift{0.627508in}{0.417642in}%
\pgfsys@useobject{currentmarker}{}%
\end{pgfscope}%
\end{pgfscope}%
\begin{pgfscope}%
\pgfpathrectangle{\pgfqpoint{0.589510in}{0.417642in}}{\pgfqpoint{1.809765in}{1.371397in}}%
\pgfusepath{clip}%
\pgfsetrectcap%
\pgfsetroundjoin%
\pgfsetlinewidth{0.803000pt}%
\definecolor{currentstroke}{rgb}{0.850000,0.850000,0.850000}%
\pgfsetstrokecolor{currentstroke}%
\pgfsetdash{}{0pt}%
\pgfpathmoveto{\pgfqpoint{0.650872in}{0.417642in}}%
\pgfpathlineto{\pgfqpoint{0.650872in}{1.789039in}}%
\pgfusepath{stroke}%
\end{pgfscope}%
\begin{pgfscope}%
\pgfsetbuttcap%
\pgfsetroundjoin%
\definecolor{currentfill}{rgb}{0.000000,0.000000,0.000000}%
\pgfsetfillcolor{currentfill}%
\pgfsetlinewidth{0.602250pt}%
\definecolor{currentstroke}{rgb}{0.000000,0.000000,0.000000}%
\pgfsetstrokecolor{currentstroke}%
\pgfsetdash{}{0pt}%
\pgfsys@defobject{currentmarker}{\pgfqpoint{0.000000in}{-0.027778in}}{\pgfqpoint{0.000000in}{0.000000in}}{%
\pgfpathmoveto{\pgfqpoint{0.000000in}{0.000000in}}%
\pgfpathlineto{\pgfqpoint{0.000000in}{-0.027778in}}%
\pgfusepath{stroke,fill}%
}%
\begin{pgfscope}%
\pgfsys@transformshift{0.650872in}{0.417642in}%
\pgfsys@useobject{currentmarker}{}%
\end{pgfscope}%
\end{pgfscope}%
\begin{pgfscope}%
\pgfpathrectangle{\pgfqpoint{0.589510in}{0.417642in}}{\pgfqpoint{1.809765in}{1.371397in}}%
\pgfusepath{clip}%
\pgfsetrectcap%
\pgfsetroundjoin%
\pgfsetlinewidth{0.803000pt}%
\definecolor{currentstroke}{rgb}{0.850000,0.850000,0.850000}%
\pgfsetstrokecolor{currentstroke}%
\pgfsetdash{}{0pt}%
\pgfpathmoveto{\pgfqpoint{0.809267in}{0.417642in}}%
\pgfpathlineto{\pgfqpoint{0.809267in}{1.789039in}}%
\pgfusepath{stroke}%
\end{pgfscope}%
\begin{pgfscope}%
\pgfsetbuttcap%
\pgfsetroundjoin%
\definecolor{currentfill}{rgb}{0.000000,0.000000,0.000000}%
\pgfsetfillcolor{currentfill}%
\pgfsetlinewidth{0.602250pt}%
\definecolor{currentstroke}{rgb}{0.000000,0.000000,0.000000}%
\pgfsetstrokecolor{currentstroke}%
\pgfsetdash{}{0pt}%
\pgfsys@defobject{currentmarker}{\pgfqpoint{0.000000in}{-0.027778in}}{\pgfqpoint{0.000000in}{0.000000in}}{%
\pgfpathmoveto{\pgfqpoint{0.000000in}{0.000000in}}%
\pgfpathlineto{\pgfqpoint{0.000000in}{-0.027778in}}%
\pgfusepath{stroke,fill}%
}%
\begin{pgfscope}%
\pgfsys@transformshift{0.809267in}{0.417642in}%
\pgfsys@useobject{currentmarker}{}%
\end{pgfscope}%
\end{pgfscope}%
\begin{pgfscope}%
\pgfpathrectangle{\pgfqpoint{0.589510in}{0.417642in}}{\pgfqpoint{1.809765in}{1.371397in}}%
\pgfusepath{clip}%
\pgfsetrectcap%
\pgfsetroundjoin%
\pgfsetlinewidth{0.803000pt}%
\definecolor{currentstroke}{rgb}{0.850000,0.850000,0.850000}%
\pgfsetstrokecolor{currentstroke}%
\pgfsetdash{}{0pt}%
\pgfpathmoveto{\pgfqpoint{0.889697in}{0.417642in}}%
\pgfpathlineto{\pgfqpoint{0.889697in}{1.789039in}}%
\pgfusepath{stroke}%
\end{pgfscope}%
\begin{pgfscope}%
\pgfsetbuttcap%
\pgfsetroundjoin%
\definecolor{currentfill}{rgb}{0.000000,0.000000,0.000000}%
\pgfsetfillcolor{currentfill}%
\pgfsetlinewidth{0.602250pt}%
\definecolor{currentstroke}{rgb}{0.000000,0.000000,0.000000}%
\pgfsetstrokecolor{currentstroke}%
\pgfsetdash{}{0pt}%
\pgfsys@defobject{currentmarker}{\pgfqpoint{0.000000in}{-0.027778in}}{\pgfqpoint{0.000000in}{0.000000in}}{%
\pgfpathmoveto{\pgfqpoint{0.000000in}{0.000000in}}%
\pgfpathlineto{\pgfqpoint{0.000000in}{-0.027778in}}%
\pgfusepath{stroke,fill}%
}%
\begin{pgfscope}%
\pgfsys@transformshift{0.889697in}{0.417642in}%
\pgfsys@useobject{currentmarker}{}%
\end{pgfscope}%
\end{pgfscope}%
\begin{pgfscope}%
\pgfpathrectangle{\pgfqpoint{0.589510in}{0.417642in}}{\pgfqpoint{1.809765in}{1.371397in}}%
\pgfusepath{clip}%
\pgfsetrectcap%
\pgfsetroundjoin%
\pgfsetlinewidth{0.803000pt}%
\definecolor{currentstroke}{rgb}{0.850000,0.850000,0.850000}%
\pgfsetstrokecolor{currentstroke}%
\pgfsetdash{}{0pt}%
\pgfpathmoveto{\pgfqpoint{0.946763in}{0.417642in}}%
\pgfpathlineto{\pgfqpoint{0.946763in}{1.789039in}}%
\pgfusepath{stroke}%
\end{pgfscope}%
\begin{pgfscope}%
\pgfsetbuttcap%
\pgfsetroundjoin%
\definecolor{currentfill}{rgb}{0.000000,0.000000,0.000000}%
\pgfsetfillcolor{currentfill}%
\pgfsetlinewidth{0.602250pt}%
\definecolor{currentstroke}{rgb}{0.000000,0.000000,0.000000}%
\pgfsetstrokecolor{currentstroke}%
\pgfsetdash{}{0pt}%
\pgfsys@defobject{currentmarker}{\pgfqpoint{0.000000in}{-0.027778in}}{\pgfqpoint{0.000000in}{0.000000in}}{%
\pgfpathmoveto{\pgfqpoint{0.000000in}{0.000000in}}%
\pgfpathlineto{\pgfqpoint{0.000000in}{-0.027778in}}%
\pgfusepath{stroke,fill}%
}%
\begin{pgfscope}%
\pgfsys@transformshift{0.946763in}{0.417642in}%
\pgfsys@useobject{currentmarker}{}%
\end{pgfscope}%
\end{pgfscope}%
\begin{pgfscope}%
\pgfpathrectangle{\pgfqpoint{0.589510in}{0.417642in}}{\pgfqpoint{1.809765in}{1.371397in}}%
\pgfusepath{clip}%
\pgfsetrectcap%
\pgfsetroundjoin%
\pgfsetlinewidth{0.803000pt}%
\definecolor{currentstroke}{rgb}{0.850000,0.850000,0.850000}%
\pgfsetstrokecolor{currentstroke}%
\pgfsetdash{}{0pt}%
\pgfpathmoveto{\pgfqpoint{0.991026in}{0.417642in}}%
\pgfpathlineto{\pgfqpoint{0.991026in}{1.789039in}}%
\pgfusepath{stroke}%
\end{pgfscope}%
\begin{pgfscope}%
\pgfsetbuttcap%
\pgfsetroundjoin%
\definecolor{currentfill}{rgb}{0.000000,0.000000,0.000000}%
\pgfsetfillcolor{currentfill}%
\pgfsetlinewidth{0.602250pt}%
\definecolor{currentstroke}{rgb}{0.000000,0.000000,0.000000}%
\pgfsetstrokecolor{currentstroke}%
\pgfsetdash{}{0pt}%
\pgfsys@defobject{currentmarker}{\pgfqpoint{0.000000in}{-0.027778in}}{\pgfqpoint{0.000000in}{0.000000in}}{%
\pgfpathmoveto{\pgfqpoint{0.000000in}{0.000000in}}%
\pgfpathlineto{\pgfqpoint{0.000000in}{-0.027778in}}%
\pgfusepath{stroke,fill}%
}%
\begin{pgfscope}%
\pgfsys@transformshift{0.991026in}{0.417642in}%
\pgfsys@useobject{currentmarker}{}%
\end{pgfscope}%
\end{pgfscope}%
\begin{pgfscope}%
\pgfpathrectangle{\pgfqpoint{0.589510in}{0.417642in}}{\pgfqpoint{1.809765in}{1.371397in}}%
\pgfusepath{clip}%
\pgfsetrectcap%
\pgfsetroundjoin%
\pgfsetlinewidth{0.803000pt}%
\definecolor{currentstroke}{rgb}{0.850000,0.850000,0.850000}%
\pgfsetstrokecolor{currentstroke}%
\pgfsetdash{}{0pt}%
\pgfpathmoveto{\pgfqpoint{1.027192in}{0.417642in}}%
\pgfpathlineto{\pgfqpoint{1.027192in}{1.789039in}}%
\pgfusepath{stroke}%
\end{pgfscope}%
\begin{pgfscope}%
\pgfsetbuttcap%
\pgfsetroundjoin%
\definecolor{currentfill}{rgb}{0.000000,0.000000,0.000000}%
\pgfsetfillcolor{currentfill}%
\pgfsetlinewidth{0.602250pt}%
\definecolor{currentstroke}{rgb}{0.000000,0.000000,0.000000}%
\pgfsetstrokecolor{currentstroke}%
\pgfsetdash{}{0pt}%
\pgfsys@defobject{currentmarker}{\pgfqpoint{0.000000in}{-0.027778in}}{\pgfqpoint{0.000000in}{0.000000in}}{%
\pgfpathmoveto{\pgfqpoint{0.000000in}{0.000000in}}%
\pgfpathlineto{\pgfqpoint{0.000000in}{-0.027778in}}%
\pgfusepath{stroke,fill}%
}%
\begin{pgfscope}%
\pgfsys@transformshift{1.027192in}{0.417642in}%
\pgfsys@useobject{currentmarker}{}%
\end{pgfscope}%
\end{pgfscope}%
\begin{pgfscope}%
\pgfpathrectangle{\pgfqpoint{0.589510in}{0.417642in}}{\pgfqpoint{1.809765in}{1.371397in}}%
\pgfusepath{clip}%
\pgfsetrectcap%
\pgfsetroundjoin%
\pgfsetlinewidth{0.803000pt}%
\definecolor{currentstroke}{rgb}{0.850000,0.850000,0.850000}%
\pgfsetstrokecolor{currentstroke}%
\pgfsetdash{}{0pt}%
\pgfpathmoveto{\pgfqpoint{1.057770in}{0.417642in}}%
\pgfpathlineto{\pgfqpoint{1.057770in}{1.789039in}}%
\pgfusepath{stroke}%
\end{pgfscope}%
\begin{pgfscope}%
\pgfsetbuttcap%
\pgfsetroundjoin%
\definecolor{currentfill}{rgb}{0.000000,0.000000,0.000000}%
\pgfsetfillcolor{currentfill}%
\pgfsetlinewidth{0.602250pt}%
\definecolor{currentstroke}{rgb}{0.000000,0.000000,0.000000}%
\pgfsetstrokecolor{currentstroke}%
\pgfsetdash{}{0pt}%
\pgfsys@defobject{currentmarker}{\pgfqpoint{0.000000in}{-0.027778in}}{\pgfqpoint{0.000000in}{0.000000in}}{%
\pgfpathmoveto{\pgfqpoint{0.000000in}{0.000000in}}%
\pgfpathlineto{\pgfqpoint{0.000000in}{-0.027778in}}%
\pgfusepath{stroke,fill}%
}%
\begin{pgfscope}%
\pgfsys@transformshift{1.057770in}{0.417642in}%
\pgfsys@useobject{currentmarker}{}%
\end{pgfscope}%
\end{pgfscope}%
\begin{pgfscope}%
\pgfpathrectangle{\pgfqpoint{0.589510in}{0.417642in}}{\pgfqpoint{1.809765in}{1.371397in}}%
\pgfusepath{clip}%
\pgfsetrectcap%
\pgfsetroundjoin%
\pgfsetlinewidth{0.803000pt}%
\definecolor{currentstroke}{rgb}{0.850000,0.850000,0.850000}%
\pgfsetstrokecolor{currentstroke}%
\pgfsetdash{}{0pt}%
\pgfpathmoveto{\pgfqpoint{1.084258in}{0.417642in}}%
\pgfpathlineto{\pgfqpoint{1.084258in}{1.789039in}}%
\pgfusepath{stroke}%
\end{pgfscope}%
\begin{pgfscope}%
\pgfsetbuttcap%
\pgfsetroundjoin%
\definecolor{currentfill}{rgb}{0.000000,0.000000,0.000000}%
\pgfsetfillcolor{currentfill}%
\pgfsetlinewidth{0.602250pt}%
\definecolor{currentstroke}{rgb}{0.000000,0.000000,0.000000}%
\pgfsetstrokecolor{currentstroke}%
\pgfsetdash{}{0pt}%
\pgfsys@defobject{currentmarker}{\pgfqpoint{0.000000in}{-0.027778in}}{\pgfqpoint{0.000000in}{0.000000in}}{%
\pgfpathmoveto{\pgfqpoint{0.000000in}{0.000000in}}%
\pgfpathlineto{\pgfqpoint{0.000000in}{-0.027778in}}%
\pgfusepath{stroke,fill}%
}%
\begin{pgfscope}%
\pgfsys@transformshift{1.084258in}{0.417642in}%
\pgfsys@useobject{currentmarker}{}%
\end{pgfscope}%
\end{pgfscope}%
\begin{pgfscope}%
\pgfpathrectangle{\pgfqpoint{0.589510in}{0.417642in}}{\pgfqpoint{1.809765in}{1.371397in}}%
\pgfusepath{clip}%
\pgfsetrectcap%
\pgfsetroundjoin%
\pgfsetlinewidth{0.803000pt}%
\definecolor{currentstroke}{rgb}{0.850000,0.850000,0.850000}%
\pgfsetstrokecolor{currentstroke}%
\pgfsetdash{}{0pt}%
\pgfpathmoveto{\pgfqpoint{1.107622in}{0.417642in}}%
\pgfpathlineto{\pgfqpoint{1.107622in}{1.789039in}}%
\pgfusepath{stroke}%
\end{pgfscope}%
\begin{pgfscope}%
\pgfsetbuttcap%
\pgfsetroundjoin%
\definecolor{currentfill}{rgb}{0.000000,0.000000,0.000000}%
\pgfsetfillcolor{currentfill}%
\pgfsetlinewidth{0.602250pt}%
\definecolor{currentstroke}{rgb}{0.000000,0.000000,0.000000}%
\pgfsetstrokecolor{currentstroke}%
\pgfsetdash{}{0pt}%
\pgfsys@defobject{currentmarker}{\pgfqpoint{0.000000in}{-0.027778in}}{\pgfqpoint{0.000000in}{0.000000in}}{%
\pgfpathmoveto{\pgfqpoint{0.000000in}{0.000000in}}%
\pgfpathlineto{\pgfqpoint{0.000000in}{-0.027778in}}%
\pgfusepath{stroke,fill}%
}%
\begin{pgfscope}%
\pgfsys@transformshift{1.107622in}{0.417642in}%
\pgfsys@useobject{currentmarker}{}%
\end{pgfscope}%
\end{pgfscope}%
\begin{pgfscope}%
\pgfpathrectangle{\pgfqpoint{0.589510in}{0.417642in}}{\pgfqpoint{1.809765in}{1.371397in}}%
\pgfusepath{clip}%
\pgfsetrectcap%
\pgfsetroundjoin%
\pgfsetlinewidth{0.803000pt}%
\definecolor{currentstroke}{rgb}{0.850000,0.850000,0.850000}%
\pgfsetstrokecolor{currentstroke}%
\pgfsetdash{}{0pt}%
\pgfpathmoveto{\pgfqpoint{1.266017in}{0.417642in}}%
\pgfpathlineto{\pgfqpoint{1.266017in}{1.789039in}}%
\pgfusepath{stroke}%
\end{pgfscope}%
\begin{pgfscope}%
\pgfsetbuttcap%
\pgfsetroundjoin%
\definecolor{currentfill}{rgb}{0.000000,0.000000,0.000000}%
\pgfsetfillcolor{currentfill}%
\pgfsetlinewidth{0.602250pt}%
\definecolor{currentstroke}{rgb}{0.000000,0.000000,0.000000}%
\pgfsetstrokecolor{currentstroke}%
\pgfsetdash{}{0pt}%
\pgfsys@defobject{currentmarker}{\pgfqpoint{0.000000in}{-0.027778in}}{\pgfqpoint{0.000000in}{0.000000in}}{%
\pgfpathmoveto{\pgfqpoint{0.000000in}{0.000000in}}%
\pgfpathlineto{\pgfqpoint{0.000000in}{-0.027778in}}%
\pgfusepath{stroke,fill}%
}%
\begin{pgfscope}%
\pgfsys@transformshift{1.266017in}{0.417642in}%
\pgfsys@useobject{currentmarker}{}%
\end{pgfscope}%
\end{pgfscope}%
\begin{pgfscope}%
\pgfpathrectangle{\pgfqpoint{0.589510in}{0.417642in}}{\pgfqpoint{1.809765in}{1.371397in}}%
\pgfusepath{clip}%
\pgfsetrectcap%
\pgfsetroundjoin%
\pgfsetlinewidth{0.803000pt}%
\definecolor{currentstroke}{rgb}{0.850000,0.850000,0.850000}%
\pgfsetstrokecolor{currentstroke}%
\pgfsetdash{}{0pt}%
\pgfpathmoveto{\pgfqpoint{1.346447in}{0.417642in}}%
\pgfpathlineto{\pgfqpoint{1.346447in}{1.789039in}}%
\pgfusepath{stroke}%
\end{pgfscope}%
\begin{pgfscope}%
\pgfsetbuttcap%
\pgfsetroundjoin%
\definecolor{currentfill}{rgb}{0.000000,0.000000,0.000000}%
\pgfsetfillcolor{currentfill}%
\pgfsetlinewidth{0.602250pt}%
\definecolor{currentstroke}{rgb}{0.000000,0.000000,0.000000}%
\pgfsetstrokecolor{currentstroke}%
\pgfsetdash{}{0pt}%
\pgfsys@defobject{currentmarker}{\pgfqpoint{0.000000in}{-0.027778in}}{\pgfqpoint{0.000000in}{0.000000in}}{%
\pgfpathmoveto{\pgfqpoint{0.000000in}{0.000000in}}%
\pgfpathlineto{\pgfqpoint{0.000000in}{-0.027778in}}%
\pgfusepath{stroke,fill}%
}%
\begin{pgfscope}%
\pgfsys@transformshift{1.346447in}{0.417642in}%
\pgfsys@useobject{currentmarker}{}%
\end{pgfscope}%
\end{pgfscope}%
\begin{pgfscope}%
\pgfpathrectangle{\pgfqpoint{0.589510in}{0.417642in}}{\pgfqpoint{1.809765in}{1.371397in}}%
\pgfusepath{clip}%
\pgfsetrectcap%
\pgfsetroundjoin%
\pgfsetlinewidth{0.803000pt}%
\definecolor{currentstroke}{rgb}{0.850000,0.850000,0.850000}%
\pgfsetstrokecolor{currentstroke}%
\pgfsetdash{}{0pt}%
\pgfpathmoveto{\pgfqpoint{1.403513in}{0.417642in}}%
\pgfpathlineto{\pgfqpoint{1.403513in}{1.789039in}}%
\pgfusepath{stroke}%
\end{pgfscope}%
\begin{pgfscope}%
\pgfsetbuttcap%
\pgfsetroundjoin%
\definecolor{currentfill}{rgb}{0.000000,0.000000,0.000000}%
\pgfsetfillcolor{currentfill}%
\pgfsetlinewidth{0.602250pt}%
\definecolor{currentstroke}{rgb}{0.000000,0.000000,0.000000}%
\pgfsetstrokecolor{currentstroke}%
\pgfsetdash{}{0pt}%
\pgfsys@defobject{currentmarker}{\pgfqpoint{0.000000in}{-0.027778in}}{\pgfqpoint{0.000000in}{0.000000in}}{%
\pgfpathmoveto{\pgfqpoint{0.000000in}{0.000000in}}%
\pgfpathlineto{\pgfqpoint{0.000000in}{-0.027778in}}%
\pgfusepath{stroke,fill}%
}%
\begin{pgfscope}%
\pgfsys@transformshift{1.403513in}{0.417642in}%
\pgfsys@useobject{currentmarker}{}%
\end{pgfscope}%
\end{pgfscope}%
\begin{pgfscope}%
\pgfpathrectangle{\pgfqpoint{0.589510in}{0.417642in}}{\pgfqpoint{1.809765in}{1.371397in}}%
\pgfusepath{clip}%
\pgfsetrectcap%
\pgfsetroundjoin%
\pgfsetlinewidth{0.803000pt}%
\definecolor{currentstroke}{rgb}{0.850000,0.850000,0.850000}%
\pgfsetstrokecolor{currentstroke}%
\pgfsetdash{}{0pt}%
\pgfpathmoveto{\pgfqpoint{1.447776in}{0.417642in}}%
\pgfpathlineto{\pgfqpoint{1.447776in}{1.789039in}}%
\pgfusepath{stroke}%
\end{pgfscope}%
\begin{pgfscope}%
\pgfsetbuttcap%
\pgfsetroundjoin%
\definecolor{currentfill}{rgb}{0.000000,0.000000,0.000000}%
\pgfsetfillcolor{currentfill}%
\pgfsetlinewidth{0.602250pt}%
\definecolor{currentstroke}{rgb}{0.000000,0.000000,0.000000}%
\pgfsetstrokecolor{currentstroke}%
\pgfsetdash{}{0pt}%
\pgfsys@defobject{currentmarker}{\pgfqpoint{0.000000in}{-0.027778in}}{\pgfqpoint{0.000000in}{0.000000in}}{%
\pgfpathmoveto{\pgfqpoint{0.000000in}{0.000000in}}%
\pgfpathlineto{\pgfqpoint{0.000000in}{-0.027778in}}%
\pgfusepath{stroke,fill}%
}%
\begin{pgfscope}%
\pgfsys@transformshift{1.447776in}{0.417642in}%
\pgfsys@useobject{currentmarker}{}%
\end{pgfscope}%
\end{pgfscope}%
\begin{pgfscope}%
\pgfpathrectangle{\pgfqpoint{0.589510in}{0.417642in}}{\pgfqpoint{1.809765in}{1.371397in}}%
\pgfusepath{clip}%
\pgfsetrectcap%
\pgfsetroundjoin%
\pgfsetlinewidth{0.803000pt}%
\definecolor{currentstroke}{rgb}{0.850000,0.850000,0.850000}%
\pgfsetstrokecolor{currentstroke}%
\pgfsetdash{}{0pt}%
\pgfpathmoveto{\pgfqpoint{1.483942in}{0.417642in}}%
\pgfpathlineto{\pgfqpoint{1.483942in}{1.789039in}}%
\pgfusepath{stroke}%
\end{pgfscope}%
\begin{pgfscope}%
\pgfsetbuttcap%
\pgfsetroundjoin%
\definecolor{currentfill}{rgb}{0.000000,0.000000,0.000000}%
\pgfsetfillcolor{currentfill}%
\pgfsetlinewidth{0.602250pt}%
\definecolor{currentstroke}{rgb}{0.000000,0.000000,0.000000}%
\pgfsetstrokecolor{currentstroke}%
\pgfsetdash{}{0pt}%
\pgfsys@defobject{currentmarker}{\pgfqpoint{0.000000in}{-0.027778in}}{\pgfqpoint{0.000000in}{0.000000in}}{%
\pgfpathmoveto{\pgfqpoint{0.000000in}{0.000000in}}%
\pgfpathlineto{\pgfqpoint{0.000000in}{-0.027778in}}%
\pgfusepath{stroke,fill}%
}%
\begin{pgfscope}%
\pgfsys@transformshift{1.483942in}{0.417642in}%
\pgfsys@useobject{currentmarker}{}%
\end{pgfscope}%
\end{pgfscope}%
\begin{pgfscope}%
\pgfpathrectangle{\pgfqpoint{0.589510in}{0.417642in}}{\pgfqpoint{1.809765in}{1.371397in}}%
\pgfusepath{clip}%
\pgfsetrectcap%
\pgfsetroundjoin%
\pgfsetlinewidth{0.803000pt}%
\definecolor{currentstroke}{rgb}{0.850000,0.850000,0.850000}%
\pgfsetstrokecolor{currentstroke}%
\pgfsetdash{}{0pt}%
\pgfpathmoveto{\pgfqpoint{1.514520in}{0.417642in}}%
\pgfpathlineto{\pgfqpoint{1.514520in}{1.789039in}}%
\pgfusepath{stroke}%
\end{pgfscope}%
\begin{pgfscope}%
\pgfsetbuttcap%
\pgfsetroundjoin%
\definecolor{currentfill}{rgb}{0.000000,0.000000,0.000000}%
\pgfsetfillcolor{currentfill}%
\pgfsetlinewidth{0.602250pt}%
\definecolor{currentstroke}{rgb}{0.000000,0.000000,0.000000}%
\pgfsetstrokecolor{currentstroke}%
\pgfsetdash{}{0pt}%
\pgfsys@defobject{currentmarker}{\pgfqpoint{0.000000in}{-0.027778in}}{\pgfqpoint{0.000000in}{0.000000in}}{%
\pgfpathmoveto{\pgfqpoint{0.000000in}{0.000000in}}%
\pgfpathlineto{\pgfqpoint{0.000000in}{-0.027778in}}%
\pgfusepath{stroke,fill}%
}%
\begin{pgfscope}%
\pgfsys@transformshift{1.514520in}{0.417642in}%
\pgfsys@useobject{currentmarker}{}%
\end{pgfscope}%
\end{pgfscope}%
\begin{pgfscope}%
\pgfpathrectangle{\pgfqpoint{0.589510in}{0.417642in}}{\pgfqpoint{1.809765in}{1.371397in}}%
\pgfusepath{clip}%
\pgfsetrectcap%
\pgfsetroundjoin%
\pgfsetlinewidth{0.803000pt}%
\definecolor{currentstroke}{rgb}{0.850000,0.850000,0.850000}%
\pgfsetstrokecolor{currentstroke}%
\pgfsetdash{}{0pt}%
\pgfpathmoveto{\pgfqpoint{1.541008in}{0.417642in}}%
\pgfpathlineto{\pgfqpoint{1.541008in}{1.789039in}}%
\pgfusepath{stroke}%
\end{pgfscope}%
\begin{pgfscope}%
\pgfsetbuttcap%
\pgfsetroundjoin%
\definecolor{currentfill}{rgb}{0.000000,0.000000,0.000000}%
\pgfsetfillcolor{currentfill}%
\pgfsetlinewidth{0.602250pt}%
\definecolor{currentstroke}{rgb}{0.000000,0.000000,0.000000}%
\pgfsetstrokecolor{currentstroke}%
\pgfsetdash{}{0pt}%
\pgfsys@defobject{currentmarker}{\pgfqpoint{0.000000in}{-0.027778in}}{\pgfqpoint{0.000000in}{0.000000in}}{%
\pgfpathmoveto{\pgfqpoint{0.000000in}{0.000000in}}%
\pgfpathlineto{\pgfqpoint{0.000000in}{-0.027778in}}%
\pgfusepath{stroke,fill}%
}%
\begin{pgfscope}%
\pgfsys@transformshift{1.541008in}{0.417642in}%
\pgfsys@useobject{currentmarker}{}%
\end{pgfscope}%
\end{pgfscope}%
\begin{pgfscope}%
\pgfpathrectangle{\pgfqpoint{0.589510in}{0.417642in}}{\pgfqpoint{1.809765in}{1.371397in}}%
\pgfusepath{clip}%
\pgfsetrectcap%
\pgfsetroundjoin%
\pgfsetlinewidth{0.803000pt}%
\definecolor{currentstroke}{rgb}{0.850000,0.850000,0.850000}%
\pgfsetstrokecolor{currentstroke}%
\pgfsetdash{}{0pt}%
\pgfpathmoveto{\pgfqpoint{1.564372in}{0.417642in}}%
\pgfpathlineto{\pgfqpoint{1.564372in}{1.789039in}}%
\pgfusepath{stroke}%
\end{pgfscope}%
\begin{pgfscope}%
\pgfsetbuttcap%
\pgfsetroundjoin%
\definecolor{currentfill}{rgb}{0.000000,0.000000,0.000000}%
\pgfsetfillcolor{currentfill}%
\pgfsetlinewidth{0.602250pt}%
\definecolor{currentstroke}{rgb}{0.000000,0.000000,0.000000}%
\pgfsetstrokecolor{currentstroke}%
\pgfsetdash{}{0pt}%
\pgfsys@defobject{currentmarker}{\pgfqpoint{0.000000in}{-0.027778in}}{\pgfqpoint{0.000000in}{0.000000in}}{%
\pgfpathmoveto{\pgfqpoint{0.000000in}{0.000000in}}%
\pgfpathlineto{\pgfqpoint{0.000000in}{-0.027778in}}%
\pgfusepath{stroke,fill}%
}%
\begin{pgfscope}%
\pgfsys@transformshift{1.564372in}{0.417642in}%
\pgfsys@useobject{currentmarker}{}%
\end{pgfscope}%
\end{pgfscope}%
\begin{pgfscope}%
\pgfpathrectangle{\pgfqpoint{0.589510in}{0.417642in}}{\pgfqpoint{1.809765in}{1.371397in}}%
\pgfusepath{clip}%
\pgfsetrectcap%
\pgfsetroundjoin%
\pgfsetlinewidth{0.803000pt}%
\definecolor{currentstroke}{rgb}{0.850000,0.850000,0.850000}%
\pgfsetstrokecolor{currentstroke}%
\pgfsetdash{}{0pt}%
\pgfpathmoveto{\pgfqpoint{1.722767in}{0.417642in}}%
\pgfpathlineto{\pgfqpoint{1.722767in}{1.789039in}}%
\pgfusepath{stroke}%
\end{pgfscope}%
\begin{pgfscope}%
\pgfsetbuttcap%
\pgfsetroundjoin%
\definecolor{currentfill}{rgb}{0.000000,0.000000,0.000000}%
\pgfsetfillcolor{currentfill}%
\pgfsetlinewidth{0.602250pt}%
\definecolor{currentstroke}{rgb}{0.000000,0.000000,0.000000}%
\pgfsetstrokecolor{currentstroke}%
\pgfsetdash{}{0pt}%
\pgfsys@defobject{currentmarker}{\pgfqpoint{0.000000in}{-0.027778in}}{\pgfqpoint{0.000000in}{0.000000in}}{%
\pgfpathmoveto{\pgfqpoint{0.000000in}{0.000000in}}%
\pgfpathlineto{\pgfqpoint{0.000000in}{-0.027778in}}%
\pgfusepath{stroke,fill}%
}%
\begin{pgfscope}%
\pgfsys@transformshift{1.722767in}{0.417642in}%
\pgfsys@useobject{currentmarker}{}%
\end{pgfscope}%
\end{pgfscope}%
\begin{pgfscope}%
\pgfpathrectangle{\pgfqpoint{0.589510in}{0.417642in}}{\pgfqpoint{1.809765in}{1.371397in}}%
\pgfusepath{clip}%
\pgfsetrectcap%
\pgfsetroundjoin%
\pgfsetlinewidth{0.803000pt}%
\definecolor{currentstroke}{rgb}{0.850000,0.850000,0.850000}%
\pgfsetstrokecolor{currentstroke}%
\pgfsetdash{}{0pt}%
\pgfpathmoveto{\pgfqpoint{1.803197in}{0.417642in}}%
\pgfpathlineto{\pgfqpoint{1.803197in}{1.789039in}}%
\pgfusepath{stroke}%
\end{pgfscope}%
\begin{pgfscope}%
\pgfsetbuttcap%
\pgfsetroundjoin%
\definecolor{currentfill}{rgb}{0.000000,0.000000,0.000000}%
\pgfsetfillcolor{currentfill}%
\pgfsetlinewidth{0.602250pt}%
\definecolor{currentstroke}{rgb}{0.000000,0.000000,0.000000}%
\pgfsetstrokecolor{currentstroke}%
\pgfsetdash{}{0pt}%
\pgfsys@defobject{currentmarker}{\pgfqpoint{0.000000in}{-0.027778in}}{\pgfqpoint{0.000000in}{0.000000in}}{%
\pgfpathmoveto{\pgfqpoint{0.000000in}{0.000000in}}%
\pgfpathlineto{\pgfqpoint{0.000000in}{-0.027778in}}%
\pgfusepath{stroke,fill}%
}%
\begin{pgfscope}%
\pgfsys@transformshift{1.803197in}{0.417642in}%
\pgfsys@useobject{currentmarker}{}%
\end{pgfscope}%
\end{pgfscope}%
\begin{pgfscope}%
\pgfpathrectangle{\pgfqpoint{0.589510in}{0.417642in}}{\pgfqpoint{1.809765in}{1.371397in}}%
\pgfusepath{clip}%
\pgfsetrectcap%
\pgfsetroundjoin%
\pgfsetlinewidth{0.803000pt}%
\definecolor{currentstroke}{rgb}{0.850000,0.850000,0.850000}%
\pgfsetstrokecolor{currentstroke}%
\pgfsetdash{}{0pt}%
\pgfpathmoveto{\pgfqpoint{1.860263in}{0.417642in}}%
\pgfpathlineto{\pgfqpoint{1.860263in}{1.789039in}}%
\pgfusepath{stroke}%
\end{pgfscope}%
\begin{pgfscope}%
\pgfsetbuttcap%
\pgfsetroundjoin%
\definecolor{currentfill}{rgb}{0.000000,0.000000,0.000000}%
\pgfsetfillcolor{currentfill}%
\pgfsetlinewidth{0.602250pt}%
\definecolor{currentstroke}{rgb}{0.000000,0.000000,0.000000}%
\pgfsetstrokecolor{currentstroke}%
\pgfsetdash{}{0pt}%
\pgfsys@defobject{currentmarker}{\pgfqpoint{0.000000in}{-0.027778in}}{\pgfqpoint{0.000000in}{0.000000in}}{%
\pgfpathmoveto{\pgfqpoint{0.000000in}{0.000000in}}%
\pgfpathlineto{\pgfqpoint{0.000000in}{-0.027778in}}%
\pgfusepath{stroke,fill}%
}%
\begin{pgfscope}%
\pgfsys@transformshift{1.860263in}{0.417642in}%
\pgfsys@useobject{currentmarker}{}%
\end{pgfscope}%
\end{pgfscope}%
\begin{pgfscope}%
\pgfpathrectangle{\pgfqpoint{0.589510in}{0.417642in}}{\pgfqpoint{1.809765in}{1.371397in}}%
\pgfusepath{clip}%
\pgfsetrectcap%
\pgfsetroundjoin%
\pgfsetlinewidth{0.803000pt}%
\definecolor{currentstroke}{rgb}{0.850000,0.850000,0.850000}%
\pgfsetstrokecolor{currentstroke}%
\pgfsetdash{}{0pt}%
\pgfpathmoveto{\pgfqpoint{1.904526in}{0.417642in}}%
\pgfpathlineto{\pgfqpoint{1.904526in}{1.789039in}}%
\pgfusepath{stroke}%
\end{pgfscope}%
\begin{pgfscope}%
\pgfsetbuttcap%
\pgfsetroundjoin%
\definecolor{currentfill}{rgb}{0.000000,0.000000,0.000000}%
\pgfsetfillcolor{currentfill}%
\pgfsetlinewidth{0.602250pt}%
\definecolor{currentstroke}{rgb}{0.000000,0.000000,0.000000}%
\pgfsetstrokecolor{currentstroke}%
\pgfsetdash{}{0pt}%
\pgfsys@defobject{currentmarker}{\pgfqpoint{0.000000in}{-0.027778in}}{\pgfqpoint{0.000000in}{0.000000in}}{%
\pgfpathmoveto{\pgfqpoint{0.000000in}{0.000000in}}%
\pgfpathlineto{\pgfqpoint{0.000000in}{-0.027778in}}%
\pgfusepath{stroke,fill}%
}%
\begin{pgfscope}%
\pgfsys@transformshift{1.904526in}{0.417642in}%
\pgfsys@useobject{currentmarker}{}%
\end{pgfscope}%
\end{pgfscope}%
\begin{pgfscope}%
\pgfpathrectangle{\pgfqpoint{0.589510in}{0.417642in}}{\pgfqpoint{1.809765in}{1.371397in}}%
\pgfusepath{clip}%
\pgfsetrectcap%
\pgfsetroundjoin%
\pgfsetlinewidth{0.803000pt}%
\definecolor{currentstroke}{rgb}{0.850000,0.850000,0.850000}%
\pgfsetstrokecolor{currentstroke}%
\pgfsetdash{}{0pt}%
\pgfpathmoveto{\pgfqpoint{1.940693in}{0.417642in}}%
\pgfpathlineto{\pgfqpoint{1.940693in}{1.789039in}}%
\pgfusepath{stroke}%
\end{pgfscope}%
\begin{pgfscope}%
\pgfsetbuttcap%
\pgfsetroundjoin%
\definecolor{currentfill}{rgb}{0.000000,0.000000,0.000000}%
\pgfsetfillcolor{currentfill}%
\pgfsetlinewidth{0.602250pt}%
\definecolor{currentstroke}{rgb}{0.000000,0.000000,0.000000}%
\pgfsetstrokecolor{currentstroke}%
\pgfsetdash{}{0pt}%
\pgfsys@defobject{currentmarker}{\pgfqpoint{0.000000in}{-0.027778in}}{\pgfqpoint{0.000000in}{0.000000in}}{%
\pgfpathmoveto{\pgfqpoint{0.000000in}{0.000000in}}%
\pgfpathlineto{\pgfqpoint{0.000000in}{-0.027778in}}%
\pgfusepath{stroke,fill}%
}%
\begin{pgfscope}%
\pgfsys@transformshift{1.940693in}{0.417642in}%
\pgfsys@useobject{currentmarker}{}%
\end{pgfscope}%
\end{pgfscope}%
\begin{pgfscope}%
\pgfpathrectangle{\pgfqpoint{0.589510in}{0.417642in}}{\pgfqpoint{1.809765in}{1.371397in}}%
\pgfusepath{clip}%
\pgfsetrectcap%
\pgfsetroundjoin%
\pgfsetlinewidth{0.803000pt}%
\definecolor{currentstroke}{rgb}{0.850000,0.850000,0.850000}%
\pgfsetstrokecolor{currentstroke}%
\pgfsetdash{}{0pt}%
\pgfpathmoveto{\pgfqpoint{1.971270in}{0.417642in}}%
\pgfpathlineto{\pgfqpoint{1.971270in}{1.789039in}}%
\pgfusepath{stroke}%
\end{pgfscope}%
\begin{pgfscope}%
\pgfsetbuttcap%
\pgfsetroundjoin%
\definecolor{currentfill}{rgb}{0.000000,0.000000,0.000000}%
\pgfsetfillcolor{currentfill}%
\pgfsetlinewidth{0.602250pt}%
\definecolor{currentstroke}{rgb}{0.000000,0.000000,0.000000}%
\pgfsetstrokecolor{currentstroke}%
\pgfsetdash{}{0pt}%
\pgfsys@defobject{currentmarker}{\pgfqpoint{0.000000in}{-0.027778in}}{\pgfqpoint{0.000000in}{0.000000in}}{%
\pgfpathmoveto{\pgfqpoint{0.000000in}{0.000000in}}%
\pgfpathlineto{\pgfqpoint{0.000000in}{-0.027778in}}%
\pgfusepath{stroke,fill}%
}%
\begin{pgfscope}%
\pgfsys@transformshift{1.971270in}{0.417642in}%
\pgfsys@useobject{currentmarker}{}%
\end{pgfscope}%
\end{pgfscope}%
\begin{pgfscope}%
\pgfpathrectangle{\pgfqpoint{0.589510in}{0.417642in}}{\pgfqpoint{1.809765in}{1.371397in}}%
\pgfusepath{clip}%
\pgfsetrectcap%
\pgfsetroundjoin%
\pgfsetlinewidth{0.803000pt}%
\definecolor{currentstroke}{rgb}{0.850000,0.850000,0.850000}%
\pgfsetstrokecolor{currentstroke}%
\pgfsetdash{}{0pt}%
\pgfpathmoveto{\pgfqpoint{1.997758in}{0.417642in}}%
\pgfpathlineto{\pgfqpoint{1.997758in}{1.789039in}}%
\pgfusepath{stroke}%
\end{pgfscope}%
\begin{pgfscope}%
\pgfsetbuttcap%
\pgfsetroundjoin%
\definecolor{currentfill}{rgb}{0.000000,0.000000,0.000000}%
\pgfsetfillcolor{currentfill}%
\pgfsetlinewidth{0.602250pt}%
\definecolor{currentstroke}{rgb}{0.000000,0.000000,0.000000}%
\pgfsetstrokecolor{currentstroke}%
\pgfsetdash{}{0pt}%
\pgfsys@defobject{currentmarker}{\pgfqpoint{0.000000in}{-0.027778in}}{\pgfqpoint{0.000000in}{0.000000in}}{%
\pgfpathmoveto{\pgfqpoint{0.000000in}{0.000000in}}%
\pgfpathlineto{\pgfqpoint{0.000000in}{-0.027778in}}%
\pgfusepath{stroke,fill}%
}%
\begin{pgfscope}%
\pgfsys@transformshift{1.997758in}{0.417642in}%
\pgfsys@useobject{currentmarker}{}%
\end{pgfscope}%
\end{pgfscope}%
\begin{pgfscope}%
\pgfpathrectangle{\pgfqpoint{0.589510in}{0.417642in}}{\pgfqpoint{1.809765in}{1.371397in}}%
\pgfusepath{clip}%
\pgfsetrectcap%
\pgfsetroundjoin%
\pgfsetlinewidth{0.803000pt}%
\definecolor{currentstroke}{rgb}{0.850000,0.850000,0.850000}%
\pgfsetstrokecolor{currentstroke}%
\pgfsetdash{}{0pt}%
\pgfpathmoveto{\pgfqpoint{2.021122in}{0.417642in}}%
\pgfpathlineto{\pgfqpoint{2.021122in}{1.789039in}}%
\pgfusepath{stroke}%
\end{pgfscope}%
\begin{pgfscope}%
\pgfsetbuttcap%
\pgfsetroundjoin%
\definecolor{currentfill}{rgb}{0.000000,0.000000,0.000000}%
\pgfsetfillcolor{currentfill}%
\pgfsetlinewidth{0.602250pt}%
\definecolor{currentstroke}{rgb}{0.000000,0.000000,0.000000}%
\pgfsetstrokecolor{currentstroke}%
\pgfsetdash{}{0pt}%
\pgfsys@defobject{currentmarker}{\pgfqpoint{0.000000in}{-0.027778in}}{\pgfqpoint{0.000000in}{0.000000in}}{%
\pgfpathmoveto{\pgfqpoint{0.000000in}{0.000000in}}%
\pgfpathlineto{\pgfqpoint{0.000000in}{-0.027778in}}%
\pgfusepath{stroke,fill}%
}%
\begin{pgfscope}%
\pgfsys@transformshift{2.021122in}{0.417642in}%
\pgfsys@useobject{currentmarker}{}%
\end{pgfscope}%
\end{pgfscope}%
\begin{pgfscope}%
\pgfpathrectangle{\pgfqpoint{0.589510in}{0.417642in}}{\pgfqpoint{1.809765in}{1.371397in}}%
\pgfusepath{clip}%
\pgfsetrectcap%
\pgfsetroundjoin%
\pgfsetlinewidth{0.803000pt}%
\definecolor{currentstroke}{rgb}{0.850000,0.850000,0.850000}%
\pgfsetstrokecolor{currentstroke}%
\pgfsetdash{}{0pt}%
\pgfpathmoveto{\pgfqpoint{2.179517in}{0.417642in}}%
\pgfpathlineto{\pgfqpoint{2.179517in}{1.789039in}}%
\pgfusepath{stroke}%
\end{pgfscope}%
\begin{pgfscope}%
\pgfsetbuttcap%
\pgfsetroundjoin%
\definecolor{currentfill}{rgb}{0.000000,0.000000,0.000000}%
\pgfsetfillcolor{currentfill}%
\pgfsetlinewidth{0.602250pt}%
\definecolor{currentstroke}{rgb}{0.000000,0.000000,0.000000}%
\pgfsetstrokecolor{currentstroke}%
\pgfsetdash{}{0pt}%
\pgfsys@defobject{currentmarker}{\pgfqpoint{0.000000in}{-0.027778in}}{\pgfqpoint{0.000000in}{0.000000in}}{%
\pgfpathmoveto{\pgfqpoint{0.000000in}{0.000000in}}%
\pgfpathlineto{\pgfqpoint{0.000000in}{-0.027778in}}%
\pgfusepath{stroke,fill}%
}%
\begin{pgfscope}%
\pgfsys@transformshift{2.179517in}{0.417642in}%
\pgfsys@useobject{currentmarker}{}%
\end{pgfscope}%
\end{pgfscope}%
\begin{pgfscope}%
\pgfpathrectangle{\pgfqpoint{0.589510in}{0.417642in}}{\pgfqpoint{1.809765in}{1.371397in}}%
\pgfusepath{clip}%
\pgfsetrectcap%
\pgfsetroundjoin%
\pgfsetlinewidth{0.803000pt}%
\definecolor{currentstroke}{rgb}{0.850000,0.850000,0.850000}%
\pgfsetstrokecolor{currentstroke}%
\pgfsetdash{}{0pt}%
\pgfpathmoveto{\pgfqpoint{2.259947in}{0.417642in}}%
\pgfpathlineto{\pgfqpoint{2.259947in}{1.789039in}}%
\pgfusepath{stroke}%
\end{pgfscope}%
\begin{pgfscope}%
\pgfsetbuttcap%
\pgfsetroundjoin%
\definecolor{currentfill}{rgb}{0.000000,0.000000,0.000000}%
\pgfsetfillcolor{currentfill}%
\pgfsetlinewidth{0.602250pt}%
\definecolor{currentstroke}{rgb}{0.000000,0.000000,0.000000}%
\pgfsetstrokecolor{currentstroke}%
\pgfsetdash{}{0pt}%
\pgfsys@defobject{currentmarker}{\pgfqpoint{0.000000in}{-0.027778in}}{\pgfqpoint{0.000000in}{0.000000in}}{%
\pgfpathmoveto{\pgfqpoint{0.000000in}{0.000000in}}%
\pgfpathlineto{\pgfqpoint{0.000000in}{-0.027778in}}%
\pgfusepath{stroke,fill}%
}%
\begin{pgfscope}%
\pgfsys@transformshift{2.259947in}{0.417642in}%
\pgfsys@useobject{currentmarker}{}%
\end{pgfscope}%
\end{pgfscope}%
\begin{pgfscope}%
\pgfpathrectangle{\pgfqpoint{0.589510in}{0.417642in}}{\pgfqpoint{1.809765in}{1.371397in}}%
\pgfusepath{clip}%
\pgfsetrectcap%
\pgfsetroundjoin%
\pgfsetlinewidth{0.803000pt}%
\definecolor{currentstroke}{rgb}{0.850000,0.850000,0.850000}%
\pgfsetstrokecolor{currentstroke}%
\pgfsetdash{}{0pt}%
\pgfpathmoveto{\pgfqpoint{2.317013in}{0.417642in}}%
\pgfpathlineto{\pgfqpoint{2.317013in}{1.789039in}}%
\pgfusepath{stroke}%
\end{pgfscope}%
\begin{pgfscope}%
\pgfsetbuttcap%
\pgfsetroundjoin%
\definecolor{currentfill}{rgb}{0.000000,0.000000,0.000000}%
\pgfsetfillcolor{currentfill}%
\pgfsetlinewidth{0.602250pt}%
\definecolor{currentstroke}{rgb}{0.000000,0.000000,0.000000}%
\pgfsetstrokecolor{currentstroke}%
\pgfsetdash{}{0pt}%
\pgfsys@defobject{currentmarker}{\pgfqpoint{0.000000in}{-0.027778in}}{\pgfqpoint{0.000000in}{0.000000in}}{%
\pgfpathmoveto{\pgfqpoint{0.000000in}{0.000000in}}%
\pgfpathlineto{\pgfqpoint{0.000000in}{-0.027778in}}%
\pgfusepath{stroke,fill}%
}%
\begin{pgfscope}%
\pgfsys@transformshift{2.317013in}{0.417642in}%
\pgfsys@useobject{currentmarker}{}%
\end{pgfscope}%
\end{pgfscope}%
\begin{pgfscope}%
\pgfpathrectangle{\pgfqpoint{0.589510in}{0.417642in}}{\pgfqpoint{1.809765in}{1.371397in}}%
\pgfusepath{clip}%
\pgfsetrectcap%
\pgfsetroundjoin%
\pgfsetlinewidth{0.803000pt}%
\definecolor{currentstroke}{rgb}{0.850000,0.850000,0.850000}%
\pgfsetstrokecolor{currentstroke}%
\pgfsetdash{}{0pt}%
\pgfpathmoveto{\pgfqpoint{2.361277in}{0.417642in}}%
\pgfpathlineto{\pgfqpoint{2.361277in}{1.789039in}}%
\pgfusepath{stroke}%
\end{pgfscope}%
\begin{pgfscope}%
\pgfsetbuttcap%
\pgfsetroundjoin%
\definecolor{currentfill}{rgb}{0.000000,0.000000,0.000000}%
\pgfsetfillcolor{currentfill}%
\pgfsetlinewidth{0.602250pt}%
\definecolor{currentstroke}{rgb}{0.000000,0.000000,0.000000}%
\pgfsetstrokecolor{currentstroke}%
\pgfsetdash{}{0pt}%
\pgfsys@defobject{currentmarker}{\pgfqpoint{0.000000in}{-0.027778in}}{\pgfqpoint{0.000000in}{0.000000in}}{%
\pgfpathmoveto{\pgfqpoint{0.000000in}{0.000000in}}%
\pgfpathlineto{\pgfqpoint{0.000000in}{-0.027778in}}%
\pgfusepath{stroke,fill}%
}%
\begin{pgfscope}%
\pgfsys@transformshift{2.361277in}{0.417642in}%
\pgfsys@useobject{currentmarker}{}%
\end{pgfscope}%
\end{pgfscope}%
\begin{pgfscope}%
\pgfpathrectangle{\pgfqpoint{0.589510in}{0.417642in}}{\pgfqpoint{1.809765in}{1.371397in}}%
\pgfusepath{clip}%
\pgfsetrectcap%
\pgfsetroundjoin%
\pgfsetlinewidth{0.803000pt}%
\definecolor{currentstroke}{rgb}{0.850000,0.850000,0.850000}%
\pgfsetstrokecolor{currentstroke}%
\pgfsetdash{}{0pt}%
\pgfpathmoveto{\pgfqpoint{2.397443in}{0.417642in}}%
\pgfpathlineto{\pgfqpoint{2.397443in}{1.789039in}}%
\pgfusepath{stroke}%
\end{pgfscope}%
\begin{pgfscope}%
\pgfsetbuttcap%
\pgfsetroundjoin%
\definecolor{currentfill}{rgb}{0.000000,0.000000,0.000000}%
\pgfsetfillcolor{currentfill}%
\pgfsetlinewidth{0.602250pt}%
\definecolor{currentstroke}{rgb}{0.000000,0.000000,0.000000}%
\pgfsetstrokecolor{currentstroke}%
\pgfsetdash{}{0pt}%
\pgfsys@defobject{currentmarker}{\pgfqpoint{0.000000in}{-0.027778in}}{\pgfqpoint{0.000000in}{0.000000in}}{%
\pgfpathmoveto{\pgfqpoint{0.000000in}{0.000000in}}%
\pgfpathlineto{\pgfqpoint{0.000000in}{-0.027778in}}%
\pgfusepath{stroke,fill}%
}%
\begin{pgfscope}%
\pgfsys@transformshift{2.397443in}{0.417642in}%
\pgfsys@useobject{currentmarker}{}%
\end{pgfscope}%
\end{pgfscope}%
\begin{pgfscope}%
\definecolor{textcolor}{rgb}{0.000000,0.000000,0.000000}%
\pgfsetstrokecolor{textcolor}%
\pgfsetfillcolor{textcolor}%
\pgftext[x=1.494392in,y=0.165003in,,top]{\color{textcolor}\rmfamily\fontsize{10.000000}{12.000000}\selectfont \(\displaystyle \tau\) in \unit{\second}}%
\end{pgfscope}%
\begin{pgfscope}%
\pgfpathrectangle{\pgfqpoint{0.589510in}{0.417642in}}{\pgfqpoint{1.809765in}{1.371397in}}%
\pgfusepath{clip}%
\pgfsetrectcap%
\pgfsetroundjoin%
\pgfsetlinewidth{0.803000pt}%
\definecolor{currentstroke}{rgb}{0.450000,0.450000,0.450000}%
\pgfsetstrokecolor{currentstroke}%
\pgfsetdash{}{0pt}%
\pgfpathmoveto{\pgfqpoint{0.589510in}{0.417642in}}%
\pgfpathlineto{\pgfqpoint{2.399275in}{0.417642in}}%
\pgfusepath{stroke}%
\end{pgfscope}%
\begin{pgfscope}%
\pgfsetbuttcap%
\pgfsetroundjoin%
\definecolor{currentfill}{rgb}{0.000000,0.000000,0.000000}%
\pgfsetfillcolor{currentfill}%
\pgfsetlinewidth{0.803000pt}%
\definecolor{currentstroke}{rgb}{0.000000,0.000000,0.000000}%
\pgfsetstrokecolor{currentstroke}%
\pgfsetdash{}{0pt}%
\pgfsys@defobject{currentmarker}{\pgfqpoint{-0.048611in}{0.000000in}}{\pgfqpoint{-0.000000in}{0.000000in}}{%
\pgfpathmoveto{\pgfqpoint{-0.000000in}{0.000000in}}%
\pgfpathlineto{\pgfqpoint{-0.048611in}{0.000000in}}%
\pgfusepath{stroke,fill}%
}%
\begin{pgfscope}%
\pgfsys@transformshift{0.589510in}{0.417642in}%
\pgfsys@useobject{currentmarker}{}%
\end{pgfscope}%
\end{pgfscope}%
\begin{pgfscope}%
\definecolor{textcolor}{rgb}{0.000000,0.000000,0.000000}%
\pgfsetstrokecolor{textcolor}%
\pgfsetfillcolor{textcolor}%
\pgftext[x=0.236114in, y=0.378489in, left, base]{\color{textcolor}\rmfamily\fontsize{8.000000}{9.600000}\selectfont \(\displaystyle {10^{-2}}\)}%
\end{pgfscope}%
\begin{pgfscope}%
\pgfpathrectangle{\pgfqpoint{0.589510in}{0.417642in}}{\pgfqpoint{1.809765in}{1.371397in}}%
\pgfusepath{clip}%
\pgfsetrectcap%
\pgfsetroundjoin%
\pgfsetlinewidth{0.803000pt}%
\definecolor{currentstroke}{rgb}{0.450000,0.450000,0.450000}%
\pgfsetstrokecolor{currentstroke}%
\pgfsetdash{}{0pt}%
\pgfpathmoveto{\pgfqpoint{0.589510in}{0.827077in}}%
\pgfpathlineto{\pgfqpoint{2.399275in}{0.827077in}}%
\pgfusepath{stroke}%
\end{pgfscope}%
\begin{pgfscope}%
\pgfsetbuttcap%
\pgfsetroundjoin%
\definecolor{currentfill}{rgb}{0.000000,0.000000,0.000000}%
\pgfsetfillcolor{currentfill}%
\pgfsetlinewidth{0.803000pt}%
\definecolor{currentstroke}{rgb}{0.000000,0.000000,0.000000}%
\pgfsetstrokecolor{currentstroke}%
\pgfsetdash{}{0pt}%
\pgfsys@defobject{currentmarker}{\pgfqpoint{-0.048611in}{0.000000in}}{\pgfqpoint{-0.000000in}{0.000000in}}{%
\pgfpathmoveto{\pgfqpoint{-0.000000in}{0.000000in}}%
\pgfpathlineto{\pgfqpoint{-0.048611in}{0.000000in}}%
\pgfusepath{stroke,fill}%
}%
\begin{pgfscope}%
\pgfsys@transformshift{0.589510in}{0.827077in}%
\pgfsys@useobject{currentmarker}{}%
\end{pgfscope}%
\end{pgfscope}%
\begin{pgfscope}%
\definecolor{textcolor}{rgb}{0.000000,0.000000,0.000000}%
\pgfsetstrokecolor{textcolor}%
\pgfsetfillcolor{textcolor}%
\pgftext[x=0.316361in, y=0.787924in, left, base]{\color{textcolor}\rmfamily\fontsize{8.000000}{9.600000}\selectfont \(\displaystyle {10^{0}}\)}%
\end{pgfscope}%
\begin{pgfscope}%
\pgfpathrectangle{\pgfqpoint{0.589510in}{0.417642in}}{\pgfqpoint{1.809765in}{1.371397in}}%
\pgfusepath{clip}%
\pgfsetrectcap%
\pgfsetroundjoin%
\pgfsetlinewidth{0.803000pt}%
\definecolor{currentstroke}{rgb}{0.450000,0.450000,0.450000}%
\pgfsetstrokecolor{currentstroke}%
\pgfsetdash{}{0pt}%
\pgfpathmoveto{\pgfqpoint{0.589510in}{1.236512in}}%
\pgfpathlineto{\pgfqpoint{2.399275in}{1.236512in}}%
\pgfusepath{stroke}%
\end{pgfscope}%
\begin{pgfscope}%
\pgfsetbuttcap%
\pgfsetroundjoin%
\definecolor{currentfill}{rgb}{0.000000,0.000000,0.000000}%
\pgfsetfillcolor{currentfill}%
\pgfsetlinewidth{0.803000pt}%
\definecolor{currentstroke}{rgb}{0.000000,0.000000,0.000000}%
\pgfsetstrokecolor{currentstroke}%
\pgfsetdash{}{0pt}%
\pgfsys@defobject{currentmarker}{\pgfqpoint{-0.048611in}{0.000000in}}{\pgfqpoint{-0.000000in}{0.000000in}}{%
\pgfpathmoveto{\pgfqpoint{-0.000000in}{0.000000in}}%
\pgfpathlineto{\pgfqpoint{-0.048611in}{0.000000in}}%
\pgfusepath{stroke,fill}%
}%
\begin{pgfscope}%
\pgfsys@transformshift{0.589510in}{1.236512in}%
\pgfsys@useobject{currentmarker}{}%
\end{pgfscope}%
\end{pgfscope}%
\begin{pgfscope}%
\definecolor{textcolor}{rgb}{0.000000,0.000000,0.000000}%
\pgfsetstrokecolor{textcolor}%
\pgfsetfillcolor{textcolor}%
\pgftext[x=0.316361in, y=1.197359in, left, base]{\color{textcolor}\rmfamily\fontsize{8.000000}{9.600000}\selectfont \(\displaystyle {10^{2}}\)}%
\end{pgfscope}%
\begin{pgfscope}%
\pgfpathrectangle{\pgfqpoint{0.589510in}{0.417642in}}{\pgfqpoint{1.809765in}{1.371397in}}%
\pgfusepath{clip}%
\pgfsetrectcap%
\pgfsetroundjoin%
\pgfsetlinewidth{0.803000pt}%
\definecolor{currentstroke}{rgb}{0.450000,0.450000,0.450000}%
\pgfsetstrokecolor{currentstroke}%
\pgfsetdash{}{0pt}%
\pgfpathmoveto{\pgfqpoint{0.589510in}{1.645947in}}%
\pgfpathlineto{\pgfqpoint{2.399275in}{1.645947in}}%
\pgfusepath{stroke}%
\end{pgfscope}%
\begin{pgfscope}%
\pgfsetbuttcap%
\pgfsetroundjoin%
\definecolor{currentfill}{rgb}{0.000000,0.000000,0.000000}%
\pgfsetfillcolor{currentfill}%
\pgfsetlinewidth{0.803000pt}%
\definecolor{currentstroke}{rgb}{0.000000,0.000000,0.000000}%
\pgfsetstrokecolor{currentstroke}%
\pgfsetdash{}{0pt}%
\pgfsys@defobject{currentmarker}{\pgfqpoint{-0.048611in}{0.000000in}}{\pgfqpoint{-0.000000in}{0.000000in}}{%
\pgfpathmoveto{\pgfqpoint{-0.000000in}{0.000000in}}%
\pgfpathlineto{\pgfqpoint{-0.048611in}{0.000000in}}%
\pgfusepath{stroke,fill}%
}%
\begin{pgfscope}%
\pgfsys@transformshift{0.589510in}{1.645947in}%
\pgfsys@useobject{currentmarker}{}%
\end{pgfscope}%
\end{pgfscope}%
\begin{pgfscope}%
\definecolor{textcolor}{rgb}{0.000000,0.000000,0.000000}%
\pgfsetstrokecolor{textcolor}%
\pgfsetfillcolor{textcolor}%
\pgftext[x=0.316361in, y=1.606795in, left, base]{\color{textcolor}\rmfamily\fontsize{8.000000}{9.600000}\selectfont \(\displaystyle {10^{4}}\)}%
\end{pgfscope}%
\begin{pgfscope}%
\definecolor{textcolor}{rgb}{0.000000,0.000000,0.000000}%
\pgfsetstrokecolor{textcolor}%
\pgfsetfillcolor{textcolor}%
\pgftext[x=0.180559in,y=1.103340in,,bottom,rotate=90.000000]{\color{textcolor}\rmfamily\fontsize{10.000000}{12.000000}\selectfont ADEV \(\displaystyle \sigma_A(\tau)\)}%
\end{pgfscope}%
\begin{pgfscope}%
\pgfpathrectangle{\pgfqpoint{0.589510in}{0.417642in}}{\pgfqpoint{1.809765in}{1.371397in}}%
\pgfusepath{clip}%
\pgfsetbuttcap%
\pgfsetroundjoin%
\pgfsetlinewidth{1.505625pt}%
\definecolor{currentstroke}{rgb}{0.835294,0.368627,0.000000}%
\pgfsetstrokecolor{currentstroke}%
\pgfsetdash{{5.550000pt}{2.400000pt}}{0.000000pt}%
\pgfpathmoveto{\pgfqpoint{0.671772in}{0.827077in}}%
\pgfpathlineto{\pgfqpoint{0.809267in}{0.857890in}}%
\pgfpathlineto{\pgfqpoint{0.946763in}{0.888703in}}%
\pgfpathlineto{\pgfqpoint{1.128522in}{0.929436in}}%
\pgfpathlineto{\pgfqpoint{1.266017in}{0.960249in}}%
\pgfpathlineto{\pgfqpoint{1.403513in}{0.991062in}}%
\pgfpathlineto{\pgfqpoint{1.585272in}{1.031795in}}%
\pgfpathlineto{\pgfqpoint{1.722767in}{1.062608in}}%
\pgfpathlineto{\pgfqpoint{1.860263in}{1.093421in}}%
\pgfpathlineto{\pgfqpoint{2.042022in}{1.134153in}}%
\pgfpathlineto{\pgfqpoint{2.179517in}{1.164966in}}%
\pgfpathlineto{\pgfqpoint{2.317013in}{1.195780in}}%
\pgfusepath{stroke}%
\end{pgfscope}%
\begin{pgfscope}%
\pgfpathrectangle{\pgfqpoint{0.589510in}{0.417642in}}{\pgfqpoint{1.809765in}{1.371397in}}%
\pgfusepath{clip}%
\pgfsetbuttcap%
\pgfsetroundjoin%
\definecolor{currentfill}{rgb}{0.835294,0.368627,0.000000}%
\pgfsetfillcolor{currentfill}%
\pgfsetlinewidth{1.003750pt}%
\definecolor{currentstroke}{rgb}{0.835294,0.368627,0.000000}%
\pgfsetstrokecolor{currentstroke}%
\pgfsetdash{}{0pt}%
\pgfsys@defobject{currentmarker}{\pgfqpoint{-0.020833in}{-0.020833in}}{\pgfqpoint{0.020833in}{0.020833in}}{%
\pgfpathmoveto{\pgfqpoint{0.000000in}{-0.020833in}}%
\pgfpathcurveto{\pgfqpoint{0.005525in}{-0.020833in}}{\pgfqpoint{0.010825in}{-0.018638in}}{\pgfqpoint{0.014731in}{-0.014731in}}%
\pgfpathcurveto{\pgfqpoint{0.018638in}{-0.010825in}}{\pgfqpoint{0.020833in}{-0.005525in}}{\pgfqpoint{0.020833in}{0.000000in}}%
\pgfpathcurveto{\pgfqpoint{0.020833in}{0.005525in}}{\pgfqpoint{0.018638in}{0.010825in}}{\pgfqpoint{0.014731in}{0.014731in}}%
\pgfpathcurveto{\pgfqpoint{0.010825in}{0.018638in}}{\pgfqpoint{0.005525in}{0.020833in}}{\pgfqpoint{0.000000in}{0.020833in}}%
\pgfpathcurveto{\pgfqpoint{-0.005525in}{0.020833in}}{\pgfqpoint{-0.010825in}{0.018638in}}{\pgfqpoint{-0.014731in}{0.014731in}}%
\pgfpathcurveto{\pgfqpoint{-0.018638in}{0.010825in}}{\pgfqpoint{-0.020833in}{0.005525in}}{\pgfqpoint{-0.020833in}{0.000000in}}%
\pgfpathcurveto{\pgfqpoint{-0.020833in}{-0.005525in}}{\pgfqpoint{-0.018638in}{-0.010825in}}{\pgfqpoint{-0.014731in}{-0.014731in}}%
\pgfpathcurveto{\pgfqpoint{-0.010825in}{-0.018638in}}{\pgfqpoint{-0.005525in}{-0.020833in}}{\pgfqpoint{0.000000in}{-0.020833in}}%
\pgfpathlineto{\pgfqpoint{0.000000in}{-0.020833in}}%
\pgfpathclose%
\pgfusepath{stroke,fill}%
}%
\begin{pgfscope}%
\pgfsys@transformshift{0.671772in}{0.845380in}%
\pgfsys@useobject{currentmarker}{}%
\end{pgfscope}%
\begin{pgfscope}%
\pgfsys@transformshift{0.809267in}{0.862796in}%
\pgfsys@useobject{currentmarker}{}%
\end{pgfscope}%
\begin{pgfscope}%
\pgfsys@transformshift{0.946763in}{0.888667in}%
\pgfsys@useobject{currentmarker}{}%
\end{pgfscope}%
\begin{pgfscope}%
\pgfsys@transformshift{1.128522in}{0.926846in}%
\pgfsys@useobject{currentmarker}{}%
\end{pgfscope}%
\begin{pgfscope}%
\pgfsys@transformshift{1.266017in}{0.958050in}%
\pgfsys@useobject{currentmarker}{}%
\end{pgfscope}%
\begin{pgfscope}%
\pgfsys@transformshift{1.403513in}{0.990530in}%
\pgfsys@useobject{currentmarker}{}%
\end{pgfscope}%
\begin{pgfscope}%
\pgfsys@transformshift{1.585272in}{1.029061in}%
\pgfsys@useobject{currentmarker}{}%
\end{pgfscope}%
\begin{pgfscope}%
\pgfsys@transformshift{1.722767in}{1.056186in}%
\pgfsys@useobject{currentmarker}{}%
\end{pgfscope}%
\begin{pgfscope}%
\pgfsys@transformshift{1.860263in}{1.087384in}%
\pgfsys@useobject{currentmarker}{}%
\end{pgfscope}%
\begin{pgfscope}%
\pgfsys@transformshift{2.042022in}{1.143452in}%
\pgfsys@useobject{currentmarker}{}%
\end{pgfscope}%
\begin{pgfscope}%
\pgfsys@transformshift{2.179517in}{1.178795in}%
\pgfsys@useobject{currentmarker}{}%
\end{pgfscope}%
\begin{pgfscope}%
\pgfsys@transformshift{2.317013in}{1.197055in}%
\pgfsys@useobject{currentmarker}{}%
\end{pgfscope}%
\end{pgfscope}%
\begin{pgfscope}%
\pgfsetrectcap%
\pgfsetmiterjoin%
\pgfsetlinewidth{0.803000pt}%
\definecolor{currentstroke}{rgb}{0.000000,0.000000,0.000000}%
\pgfsetstrokecolor{currentstroke}%
\pgfsetdash{}{0pt}%
\pgfpathmoveto{\pgfqpoint{0.589510in}{0.417642in}}%
\pgfpathlineto{\pgfqpoint{0.589510in}{1.789039in}}%
\pgfusepath{stroke}%
\end{pgfscope}%
\begin{pgfscope}%
\pgfsetrectcap%
\pgfsetmiterjoin%
\pgfsetlinewidth{0.803000pt}%
\definecolor{currentstroke}{rgb}{0.000000,0.000000,0.000000}%
\pgfsetstrokecolor{currentstroke}%
\pgfsetdash{}{0pt}%
\pgfpathmoveto{\pgfqpoint{2.399275in}{0.417642in}}%
\pgfpathlineto{\pgfqpoint{2.399275in}{1.789039in}}%
\pgfusepath{stroke}%
\end{pgfscope}%
\begin{pgfscope}%
\pgfsetrectcap%
\pgfsetmiterjoin%
\pgfsetlinewidth{0.803000pt}%
\definecolor{currentstroke}{rgb}{0.000000,0.000000,0.000000}%
\pgfsetstrokecolor{currentstroke}%
\pgfsetdash{}{0pt}%
\pgfpathmoveto{\pgfqpoint{0.589510in}{0.417642in}}%
\pgfpathlineto{\pgfqpoint{2.399275in}{0.417642in}}%
\pgfusepath{stroke}%
\end{pgfscope}%
\begin{pgfscope}%
\pgfsetrectcap%
\pgfsetmiterjoin%
\pgfsetlinewidth{0.803000pt}%
\definecolor{currentstroke}{rgb}{0.000000,0.000000,0.000000}%
\pgfsetstrokecolor{currentstroke}%
\pgfsetdash{}{0pt}%
\pgfpathmoveto{\pgfqpoint{0.589510in}{1.789039in}}%
\pgfpathlineto{\pgfqpoint{2.399275in}{1.789039in}}%
\pgfusepath{stroke}%
\end{pgfscope}%
\begin{pgfscope}%
\pgfsetbuttcap%
\pgfsetmiterjoin%
\definecolor{currentfill}{rgb}{1.000000,1.000000,1.000000}%
\pgfsetfillcolor{currentfill}%
\pgfsetfillopacity{0.800000}%
\pgfsetlinewidth{1.003750pt}%
\definecolor{currentstroke}{rgb}{0.800000,0.800000,0.800000}%
\pgfsetstrokecolor{currentstroke}%
\pgfsetstrokeopacity{0.800000}%
\pgfsetdash{}{0pt}%
\pgfpathmoveto{\pgfqpoint{1.212708in}{1.472371in}}%
\pgfpathlineto{\pgfqpoint{2.321497in}{1.472371in}}%
\pgfpathquadraticcurveto{\pgfqpoint{2.343719in}{1.472371in}}{\pgfqpoint{2.343719in}{1.494593in}}%
\pgfpathlineto{\pgfqpoint{2.343719in}{1.711261in}}%
\pgfpathquadraticcurveto{\pgfqpoint{2.343719in}{1.733483in}}{\pgfqpoint{2.321497in}{1.733483in}}%
\pgfpathlineto{\pgfqpoint{1.212708in}{1.733483in}}%
\pgfpathquadraticcurveto{\pgfqpoint{1.190486in}{1.733483in}}{\pgfqpoint{1.190486in}{1.711261in}}%
\pgfpathlineto{\pgfqpoint{1.190486in}{1.494593in}}%
\pgfpathquadraticcurveto{\pgfqpoint{1.190486in}{1.472371in}}{\pgfqpoint{1.212708in}{1.472371in}}%
\pgfpathlineto{\pgfqpoint{1.212708in}{1.472371in}}%
\pgfpathclose%
\pgfusepath{stroke,fill}%
\end{pgfscope}%
\begin{pgfscope}%
\pgfsetbuttcap%
\pgfsetroundjoin%
\pgfsetlinewidth{1.505625pt}%
\definecolor{currentstroke}{rgb}{0.835294,0.368627,0.000000}%
\pgfsetstrokecolor{currentstroke}%
\pgfsetdash{{5.550000pt}{2.400000pt}}{0.000000pt}%
\pgfpathmoveto{\pgfqpoint{1.234930in}{1.602426in}}%
\pgfpathlineto{\pgfqpoint{1.346041in}{1.602426in}}%
\pgfpathlineto{\pgfqpoint{1.457152in}{1.602426in}}%
\pgfusepath{stroke}%
\end{pgfscope}%
\begin{pgfscope}%
\definecolor{textcolor}{rgb}{0.000000,0.000000,0.000000}%
\pgfsetstrokecolor{textcolor}%
\pgfsetfillcolor{textcolor}%
\pgftext[x=1.546041in,y=1.563537in,left,base]{\color{textcolor}\rmfamily\fontsize{8.000000}{9.600000}\selectfont \(\displaystyle \propto\sqrt{h_{-2}}\tau^{+0.5}\)}%
\end{pgfscope}%
\end{pgfpicture}%
\makeatother%
\endgroup%

        } % scalebox
        \caption{Allan deviation}
        \label{fig:random_walk_adev}
    \end{subfigure}
    \caption{Different representations of random walk noise.}
    \label{fig:random_walk_noise_simulated}
\end{figure}


\clearpage
\minisec{Drift}
Finally, the last feature of the Allan deviation plot, that needs to be discussed is drift. Drift happens at very long time scales and descriped a linear dependence of measurand on the time. This is also part of the ageing effect. \citeauthor{adev_drift} discussed the effect of drift \cite{adev_drift} on the Allan variance and found the following relationship:
\begin{align}
    \sigma_A^2(\tau) = \frac{D^2}{2} \tau^2
\end{align}
with slope of the drift $D$.

\begin{figure}[ht]
    \centering
    \begin{subfigure}{0.32\linewidth}
        \centering
        \scalebox{0.75}{%
            %% Creator: Matplotlib, PGF backend
%%
%% To include the figure in your LaTeX document, write
%%   \input{<filename>.pgf}
%%
%% Make sure the required packages are loaded in your preamble
%%   \usepackage{pgf}
%%
%% Also ensure that all the required font packages are loaded; for instance,
%% the lmodern package is sometimes necessary when using math font.
%%   \usepackage{lmodern}
%%
%% Figures using additional raster images can only be included by \input if
%% they are in the same directory as the main LaTeX file. For loading figures
%% from other directories you can use the `import` package
%%   \usepackage{import}
%%
%% and then include the figures with
%%   \import{<path to file>}{<filename>.pgf}
%%
%% Matplotlib used the following preamble
%%   \usepackage{siunitx}
%%   \usepackage{fontspec}
%%   \makeatletter\@ifpackageloaded{underscore}{}{\usepackage[strings]{underscore}}\makeatother
%%
\begingroup%
\makeatletter%
\begin{pgfpicture}%
\pgfpathrectangle{\pgfpointorigin}{\pgfqpoint{2.440000in}{1.830000in}}%
\pgfusepath{use as bounding box, clip}%
\begin{pgfscope}%
\pgfsetbuttcap%
\pgfsetmiterjoin%
\definecolor{currentfill}{rgb}{1.000000,1.000000,1.000000}%
\pgfsetfillcolor{currentfill}%
\pgfsetlinewidth{0.000000pt}%
\definecolor{currentstroke}{rgb}{1.000000,1.000000,1.000000}%
\pgfsetstrokecolor{currentstroke}%
\pgfsetdash{}{0pt}%
\pgfpathmoveto{\pgfqpoint{0.000000in}{0.000000in}}%
\pgfpathlineto{\pgfqpoint{2.440000in}{0.000000in}}%
\pgfpathlineto{\pgfqpoint{2.440000in}{1.830000in}}%
\pgfpathlineto{\pgfqpoint{0.000000in}{1.830000in}}%
\pgfpathlineto{\pgfqpoint{0.000000in}{0.000000in}}%
\pgfpathclose%
\pgfusepath{fill}%
\end{pgfscope}%
\begin{pgfscope}%
\pgfsetbuttcap%
\pgfsetmiterjoin%
\definecolor{currentfill}{rgb}{1.000000,1.000000,1.000000}%
\pgfsetfillcolor{currentfill}%
\pgfsetlinewidth{0.000000pt}%
\definecolor{currentstroke}{rgb}{0.000000,0.000000,0.000000}%
\pgfsetstrokecolor{currentstroke}%
\pgfsetstrokeopacity{0.000000}%
\pgfsetdash{}{0pt}%
\pgfpathmoveto{\pgfqpoint{0.615980in}{0.416447in}}%
\pgfpathlineto{\pgfqpoint{2.398330in}{0.416447in}}%
\pgfpathlineto{\pgfqpoint{2.398330in}{1.788330in}}%
\pgfpathlineto{\pgfqpoint{0.615980in}{1.788330in}}%
\pgfpathlineto{\pgfqpoint{0.615980in}{0.416447in}}%
\pgfpathclose%
\pgfusepath{fill}%
\end{pgfscope}%
\begin{pgfscope}%
\pgfpathrectangle{\pgfqpoint{0.615980in}{0.416447in}}{\pgfqpoint{1.782350in}{1.371883in}}%
\pgfusepath{clip}%
\pgfsetrectcap%
\pgfsetroundjoin%
\pgfsetlinewidth{0.803000pt}%
\definecolor{currentstroke}{rgb}{0.450000,0.450000,0.450000}%
\pgfsetstrokecolor{currentstroke}%
\pgfsetdash{}{0pt}%
\pgfpathmoveto{\pgfqpoint{0.696996in}{0.416447in}}%
\pgfpathlineto{\pgfqpoint{0.696996in}{1.788330in}}%
\pgfusepath{stroke}%
\end{pgfscope}%
\begin{pgfscope}%
\pgfsetbuttcap%
\pgfsetroundjoin%
\definecolor{currentfill}{rgb}{0.000000,0.000000,0.000000}%
\pgfsetfillcolor{currentfill}%
\pgfsetlinewidth{0.803000pt}%
\definecolor{currentstroke}{rgb}{0.000000,0.000000,0.000000}%
\pgfsetstrokecolor{currentstroke}%
\pgfsetdash{}{0pt}%
\pgfsys@defobject{currentmarker}{\pgfqpoint{0.000000in}{-0.048611in}}{\pgfqpoint{0.000000in}{0.000000in}}{%
\pgfpathmoveto{\pgfqpoint{0.000000in}{0.000000in}}%
\pgfpathlineto{\pgfqpoint{0.000000in}{-0.048611in}}%
\pgfusepath{stroke,fill}%
}%
\begin{pgfscope}%
\pgfsys@transformshift{0.696996in}{0.416447in}%
\pgfsys@useobject{currentmarker}{}%
\end{pgfscope}%
\end{pgfscope}%
\begin{pgfscope}%
\definecolor{textcolor}{rgb}{0.000000,0.000000,0.000000}%
\pgfsetstrokecolor{textcolor}%
\pgfsetfillcolor{textcolor}%
\pgftext[x=0.696996in,y=0.319225in,,top]{\color{textcolor}\rmfamily\fontsize{8.000000}{9.600000}\selectfont \(\displaystyle {0}\)}%
\end{pgfscope}%
\begin{pgfscope}%
\pgfpathrectangle{\pgfqpoint{0.615980in}{0.416447in}}{\pgfqpoint{1.782350in}{1.371883in}}%
\pgfusepath{clip}%
\pgfsetrectcap%
\pgfsetroundjoin%
\pgfsetlinewidth{0.803000pt}%
\definecolor{currentstroke}{rgb}{0.450000,0.450000,0.450000}%
\pgfsetstrokecolor{currentstroke}%
\pgfsetdash{}{0pt}%
\pgfpathmoveto{\pgfqpoint{1.191538in}{0.416447in}}%
\pgfpathlineto{\pgfqpoint{1.191538in}{1.788330in}}%
\pgfusepath{stroke}%
\end{pgfscope}%
\begin{pgfscope}%
\pgfsetbuttcap%
\pgfsetroundjoin%
\definecolor{currentfill}{rgb}{0.000000,0.000000,0.000000}%
\pgfsetfillcolor{currentfill}%
\pgfsetlinewidth{0.803000pt}%
\definecolor{currentstroke}{rgb}{0.000000,0.000000,0.000000}%
\pgfsetstrokecolor{currentstroke}%
\pgfsetdash{}{0pt}%
\pgfsys@defobject{currentmarker}{\pgfqpoint{0.000000in}{-0.048611in}}{\pgfqpoint{0.000000in}{0.000000in}}{%
\pgfpathmoveto{\pgfqpoint{0.000000in}{0.000000in}}%
\pgfpathlineto{\pgfqpoint{0.000000in}{-0.048611in}}%
\pgfusepath{stroke,fill}%
}%
\begin{pgfscope}%
\pgfsys@transformshift{1.191538in}{0.416447in}%
\pgfsys@useobject{currentmarker}{}%
\end{pgfscope}%
\end{pgfscope}%
\begin{pgfscope}%
\definecolor{textcolor}{rgb}{0.000000,0.000000,0.000000}%
\pgfsetstrokecolor{textcolor}%
\pgfsetfillcolor{textcolor}%
\pgftext[x=1.191538in,y=0.319225in,,top]{\color{textcolor}\rmfamily\fontsize{8.000000}{9.600000}\selectfont \(\displaystyle {5000}\)}%
\end{pgfscope}%
\begin{pgfscope}%
\pgfpathrectangle{\pgfqpoint{0.615980in}{0.416447in}}{\pgfqpoint{1.782350in}{1.371883in}}%
\pgfusepath{clip}%
\pgfsetrectcap%
\pgfsetroundjoin%
\pgfsetlinewidth{0.803000pt}%
\definecolor{currentstroke}{rgb}{0.450000,0.450000,0.450000}%
\pgfsetstrokecolor{currentstroke}%
\pgfsetdash{}{0pt}%
\pgfpathmoveto{\pgfqpoint{1.686080in}{0.416447in}}%
\pgfpathlineto{\pgfqpoint{1.686080in}{1.788330in}}%
\pgfusepath{stroke}%
\end{pgfscope}%
\begin{pgfscope}%
\pgfsetbuttcap%
\pgfsetroundjoin%
\definecolor{currentfill}{rgb}{0.000000,0.000000,0.000000}%
\pgfsetfillcolor{currentfill}%
\pgfsetlinewidth{0.803000pt}%
\definecolor{currentstroke}{rgb}{0.000000,0.000000,0.000000}%
\pgfsetstrokecolor{currentstroke}%
\pgfsetdash{}{0pt}%
\pgfsys@defobject{currentmarker}{\pgfqpoint{0.000000in}{-0.048611in}}{\pgfqpoint{0.000000in}{0.000000in}}{%
\pgfpathmoveto{\pgfqpoint{0.000000in}{0.000000in}}%
\pgfpathlineto{\pgfqpoint{0.000000in}{-0.048611in}}%
\pgfusepath{stroke,fill}%
}%
\begin{pgfscope}%
\pgfsys@transformshift{1.686080in}{0.416447in}%
\pgfsys@useobject{currentmarker}{}%
\end{pgfscope}%
\end{pgfscope}%
\begin{pgfscope}%
\definecolor{textcolor}{rgb}{0.000000,0.000000,0.000000}%
\pgfsetstrokecolor{textcolor}%
\pgfsetfillcolor{textcolor}%
\pgftext[x=1.686080in,y=0.319225in,,top]{\color{textcolor}\rmfamily\fontsize{8.000000}{9.600000}\selectfont \(\displaystyle {10000}\)}%
\end{pgfscope}%
\begin{pgfscope}%
\pgfpathrectangle{\pgfqpoint{0.615980in}{0.416447in}}{\pgfqpoint{1.782350in}{1.371883in}}%
\pgfusepath{clip}%
\pgfsetrectcap%
\pgfsetroundjoin%
\pgfsetlinewidth{0.803000pt}%
\definecolor{currentstroke}{rgb}{0.450000,0.450000,0.450000}%
\pgfsetstrokecolor{currentstroke}%
\pgfsetdash{}{0pt}%
\pgfpathmoveto{\pgfqpoint{2.180623in}{0.416447in}}%
\pgfpathlineto{\pgfqpoint{2.180623in}{1.788330in}}%
\pgfusepath{stroke}%
\end{pgfscope}%
\begin{pgfscope}%
\pgfsetbuttcap%
\pgfsetroundjoin%
\definecolor{currentfill}{rgb}{0.000000,0.000000,0.000000}%
\pgfsetfillcolor{currentfill}%
\pgfsetlinewidth{0.803000pt}%
\definecolor{currentstroke}{rgb}{0.000000,0.000000,0.000000}%
\pgfsetstrokecolor{currentstroke}%
\pgfsetdash{}{0pt}%
\pgfsys@defobject{currentmarker}{\pgfqpoint{0.000000in}{-0.048611in}}{\pgfqpoint{0.000000in}{0.000000in}}{%
\pgfpathmoveto{\pgfqpoint{0.000000in}{0.000000in}}%
\pgfpathlineto{\pgfqpoint{0.000000in}{-0.048611in}}%
\pgfusepath{stroke,fill}%
}%
\begin{pgfscope}%
\pgfsys@transformshift{2.180623in}{0.416447in}%
\pgfsys@useobject{currentmarker}{}%
\end{pgfscope}%
\end{pgfscope}%
\begin{pgfscope}%
\definecolor{textcolor}{rgb}{0.000000,0.000000,0.000000}%
\pgfsetstrokecolor{textcolor}%
\pgfsetfillcolor{textcolor}%
\pgftext[x=2.180623in,y=0.319225in,,top]{\color{textcolor}\rmfamily\fontsize{8.000000}{9.600000}\selectfont \(\displaystyle {15000}\)}%
\end{pgfscope}%
\begin{pgfscope}%
\definecolor{textcolor}{rgb}{0.000000,0.000000,0.000000}%
\pgfsetstrokecolor{textcolor}%
\pgfsetfillcolor{textcolor}%
\pgftext[x=1.507155in,y=0.165003in,,top]{\color{textcolor}\rmfamily\fontsize{10.000000}{12.000000}\selectfont Time in \(\displaystyle \unit{\second}\)}%
\end{pgfscope}%
\begin{pgfscope}%
\pgfpathrectangle{\pgfqpoint{0.615980in}{0.416447in}}{\pgfqpoint{1.782350in}{1.371883in}}%
\pgfusepath{clip}%
\pgfsetrectcap%
\pgfsetroundjoin%
\pgfsetlinewidth{0.803000pt}%
\definecolor{currentstroke}{rgb}{0.450000,0.450000,0.450000}%
\pgfsetstrokecolor{currentstroke}%
\pgfsetdash{}{0pt}%
\pgfpathmoveto{\pgfqpoint{0.615980in}{0.478806in}}%
\pgfpathlineto{\pgfqpoint{2.398330in}{0.478806in}}%
\pgfusepath{stroke}%
\end{pgfscope}%
\begin{pgfscope}%
\pgfsetbuttcap%
\pgfsetroundjoin%
\definecolor{currentfill}{rgb}{0.000000,0.000000,0.000000}%
\pgfsetfillcolor{currentfill}%
\pgfsetlinewidth{0.803000pt}%
\definecolor{currentstroke}{rgb}{0.000000,0.000000,0.000000}%
\pgfsetstrokecolor{currentstroke}%
\pgfsetdash{}{0pt}%
\pgfsys@defobject{currentmarker}{\pgfqpoint{-0.048611in}{0.000000in}}{\pgfqpoint{-0.000000in}{0.000000in}}{%
\pgfpathmoveto{\pgfqpoint{-0.000000in}{0.000000in}}%
\pgfpathlineto{\pgfqpoint{-0.048611in}{0.000000in}}%
\pgfusepath{stroke,fill}%
}%
\begin{pgfscope}%
\pgfsys@transformshift{0.615980in}{0.478806in}%
\pgfsys@useobject{currentmarker}{}%
\end{pgfscope}%
\end{pgfscope}%
\begin{pgfscope}%
\definecolor{textcolor}{rgb}{0.000000,0.000000,0.000000}%
\pgfsetstrokecolor{textcolor}%
\pgfsetfillcolor{textcolor}%
\pgftext[x=0.459729in, y=0.440250in, left, base]{\color{textcolor}\rmfamily\fontsize{8.000000}{9.600000}\selectfont \(\displaystyle {0}\)}%
\end{pgfscope}%
\begin{pgfscope}%
\pgfpathrectangle{\pgfqpoint{0.615980in}{0.416447in}}{\pgfqpoint{1.782350in}{1.371883in}}%
\pgfusepath{clip}%
\pgfsetrectcap%
\pgfsetroundjoin%
\pgfsetlinewidth{0.803000pt}%
\definecolor{currentstroke}{rgb}{0.450000,0.450000,0.450000}%
\pgfsetstrokecolor{currentstroke}%
\pgfsetdash{}{0pt}%
\pgfpathmoveto{\pgfqpoint{0.615980in}{0.747967in}}%
\pgfpathlineto{\pgfqpoint{2.398330in}{0.747967in}}%
\pgfusepath{stroke}%
\end{pgfscope}%
\begin{pgfscope}%
\pgfsetbuttcap%
\pgfsetroundjoin%
\definecolor{currentfill}{rgb}{0.000000,0.000000,0.000000}%
\pgfsetfillcolor{currentfill}%
\pgfsetlinewidth{0.803000pt}%
\definecolor{currentstroke}{rgb}{0.000000,0.000000,0.000000}%
\pgfsetstrokecolor{currentstroke}%
\pgfsetdash{}{0pt}%
\pgfsys@defobject{currentmarker}{\pgfqpoint{-0.048611in}{0.000000in}}{\pgfqpoint{-0.000000in}{0.000000in}}{%
\pgfpathmoveto{\pgfqpoint{-0.000000in}{0.000000in}}%
\pgfpathlineto{\pgfqpoint{-0.048611in}{0.000000in}}%
\pgfusepath{stroke,fill}%
}%
\begin{pgfscope}%
\pgfsys@transformshift{0.615980in}{0.747967in}%
\pgfsys@useobject{currentmarker}{}%
\end{pgfscope}%
\end{pgfscope}%
\begin{pgfscope}%
\definecolor{textcolor}{rgb}{0.000000,0.000000,0.000000}%
\pgfsetstrokecolor{textcolor}%
\pgfsetfillcolor{textcolor}%
\pgftext[x=0.282643in, y=0.709411in, left, base]{\color{textcolor}\rmfamily\fontsize{8.000000}{9.600000}\selectfont \(\displaystyle {5000}\)}%
\end{pgfscope}%
\begin{pgfscope}%
\pgfpathrectangle{\pgfqpoint{0.615980in}{0.416447in}}{\pgfqpoint{1.782350in}{1.371883in}}%
\pgfusepath{clip}%
\pgfsetrectcap%
\pgfsetroundjoin%
\pgfsetlinewidth{0.803000pt}%
\definecolor{currentstroke}{rgb}{0.450000,0.450000,0.450000}%
\pgfsetstrokecolor{currentstroke}%
\pgfsetdash{}{0pt}%
\pgfpathmoveto{\pgfqpoint{0.615980in}{1.017128in}}%
\pgfpathlineto{\pgfqpoint{2.398330in}{1.017128in}}%
\pgfusepath{stroke}%
\end{pgfscope}%
\begin{pgfscope}%
\pgfsetbuttcap%
\pgfsetroundjoin%
\definecolor{currentfill}{rgb}{0.000000,0.000000,0.000000}%
\pgfsetfillcolor{currentfill}%
\pgfsetlinewidth{0.803000pt}%
\definecolor{currentstroke}{rgb}{0.000000,0.000000,0.000000}%
\pgfsetstrokecolor{currentstroke}%
\pgfsetdash{}{0pt}%
\pgfsys@defobject{currentmarker}{\pgfqpoint{-0.048611in}{0.000000in}}{\pgfqpoint{-0.000000in}{0.000000in}}{%
\pgfpathmoveto{\pgfqpoint{-0.000000in}{0.000000in}}%
\pgfpathlineto{\pgfqpoint{-0.048611in}{0.000000in}}%
\pgfusepath{stroke,fill}%
}%
\begin{pgfscope}%
\pgfsys@transformshift{0.615980in}{1.017128in}%
\pgfsys@useobject{currentmarker}{}%
\end{pgfscope}%
\end{pgfscope}%
\begin{pgfscope}%
\definecolor{textcolor}{rgb}{0.000000,0.000000,0.000000}%
\pgfsetstrokecolor{textcolor}%
\pgfsetfillcolor{textcolor}%
\pgftext[x=0.223614in, y=0.978572in, left, base]{\color{textcolor}\rmfamily\fontsize{8.000000}{9.600000}\selectfont \(\displaystyle {10000}\)}%
\end{pgfscope}%
\begin{pgfscope}%
\pgfpathrectangle{\pgfqpoint{0.615980in}{0.416447in}}{\pgfqpoint{1.782350in}{1.371883in}}%
\pgfusepath{clip}%
\pgfsetrectcap%
\pgfsetroundjoin%
\pgfsetlinewidth{0.803000pt}%
\definecolor{currentstroke}{rgb}{0.450000,0.450000,0.450000}%
\pgfsetstrokecolor{currentstroke}%
\pgfsetdash{}{0pt}%
\pgfpathmoveto{\pgfqpoint{0.615980in}{1.286289in}}%
\pgfpathlineto{\pgfqpoint{2.398330in}{1.286289in}}%
\pgfusepath{stroke}%
\end{pgfscope}%
\begin{pgfscope}%
\pgfsetbuttcap%
\pgfsetroundjoin%
\definecolor{currentfill}{rgb}{0.000000,0.000000,0.000000}%
\pgfsetfillcolor{currentfill}%
\pgfsetlinewidth{0.803000pt}%
\definecolor{currentstroke}{rgb}{0.000000,0.000000,0.000000}%
\pgfsetstrokecolor{currentstroke}%
\pgfsetdash{}{0pt}%
\pgfsys@defobject{currentmarker}{\pgfqpoint{-0.048611in}{0.000000in}}{\pgfqpoint{-0.000000in}{0.000000in}}{%
\pgfpathmoveto{\pgfqpoint{-0.000000in}{0.000000in}}%
\pgfpathlineto{\pgfqpoint{-0.048611in}{0.000000in}}%
\pgfusepath{stroke,fill}%
}%
\begin{pgfscope}%
\pgfsys@transformshift{0.615980in}{1.286289in}%
\pgfsys@useobject{currentmarker}{}%
\end{pgfscope}%
\end{pgfscope}%
\begin{pgfscope}%
\definecolor{textcolor}{rgb}{0.000000,0.000000,0.000000}%
\pgfsetstrokecolor{textcolor}%
\pgfsetfillcolor{textcolor}%
\pgftext[x=0.223614in, y=1.247734in, left, base]{\color{textcolor}\rmfamily\fontsize{8.000000}{9.600000}\selectfont \(\displaystyle {15000}\)}%
\end{pgfscope}%
\begin{pgfscope}%
\pgfpathrectangle{\pgfqpoint{0.615980in}{0.416447in}}{\pgfqpoint{1.782350in}{1.371883in}}%
\pgfusepath{clip}%
\pgfsetrectcap%
\pgfsetroundjoin%
\pgfsetlinewidth{0.803000pt}%
\definecolor{currentstroke}{rgb}{0.450000,0.450000,0.450000}%
\pgfsetstrokecolor{currentstroke}%
\pgfsetdash{}{0pt}%
\pgfpathmoveto{\pgfqpoint{0.615980in}{1.555450in}}%
\pgfpathlineto{\pgfqpoint{2.398330in}{1.555450in}}%
\pgfusepath{stroke}%
\end{pgfscope}%
\begin{pgfscope}%
\pgfsetbuttcap%
\pgfsetroundjoin%
\definecolor{currentfill}{rgb}{0.000000,0.000000,0.000000}%
\pgfsetfillcolor{currentfill}%
\pgfsetlinewidth{0.803000pt}%
\definecolor{currentstroke}{rgb}{0.000000,0.000000,0.000000}%
\pgfsetstrokecolor{currentstroke}%
\pgfsetdash{}{0pt}%
\pgfsys@defobject{currentmarker}{\pgfqpoint{-0.048611in}{0.000000in}}{\pgfqpoint{-0.000000in}{0.000000in}}{%
\pgfpathmoveto{\pgfqpoint{-0.000000in}{0.000000in}}%
\pgfpathlineto{\pgfqpoint{-0.048611in}{0.000000in}}%
\pgfusepath{stroke,fill}%
}%
\begin{pgfscope}%
\pgfsys@transformshift{0.615980in}{1.555450in}%
\pgfsys@useobject{currentmarker}{}%
\end{pgfscope}%
\end{pgfscope}%
\begin{pgfscope}%
\definecolor{textcolor}{rgb}{0.000000,0.000000,0.000000}%
\pgfsetstrokecolor{textcolor}%
\pgfsetfillcolor{textcolor}%
\pgftext[x=0.223614in, y=1.516895in, left, base]{\color{textcolor}\rmfamily\fontsize{8.000000}{9.600000}\selectfont \(\displaystyle {20000}\)}%
\end{pgfscope}%
\begin{pgfscope}%
\definecolor{textcolor}{rgb}{0.000000,0.000000,0.000000}%
\pgfsetstrokecolor{textcolor}%
\pgfsetfillcolor{textcolor}%
\pgftext[x=0.168059in,y=1.102389in,,bottom,rotate=90.000000]{\color{textcolor}\rmfamily\fontsize{10.000000}{12.000000}\selectfont Ampl. in arb. unit}%
\end{pgfscope}%
\begin{pgfscope}%
\pgfpathrectangle{\pgfqpoint{0.615980in}{0.416447in}}{\pgfqpoint{1.782350in}{1.371883in}}%
\pgfusepath{clip}%
\pgfsetrectcap%
\pgfsetroundjoin%
\pgfsetlinewidth{1.505625pt}%
\definecolor{currentstroke}{rgb}{0.800000,0.474510,0.654902}%
\pgfsetstrokecolor{currentstroke}%
\pgfsetdash{}{0pt}%
\pgfpathmoveto{\pgfqpoint{0.696996in}{0.478806in}}%
\pgfpathlineto{\pgfqpoint{2.317314in}{1.725972in}}%
\pgfpathlineto{\pgfqpoint{2.317314in}{1.725972in}}%
\pgfusepath{stroke}%
\end{pgfscope}%
\begin{pgfscope}%
\pgfsetrectcap%
\pgfsetmiterjoin%
\pgfsetlinewidth{0.803000pt}%
\definecolor{currentstroke}{rgb}{0.000000,0.000000,0.000000}%
\pgfsetstrokecolor{currentstroke}%
\pgfsetdash{}{0pt}%
\pgfpathmoveto{\pgfqpoint{0.615980in}{0.416447in}}%
\pgfpathlineto{\pgfqpoint{0.615980in}{1.788330in}}%
\pgfusepath{stroke}%
\end{pgfscope}%
\begin{pgfscope}%
\pgfsetrectcap%
\pgfsetmiterjoin%
\pgfsetlinewidth{0.803000pt}%
\definecolor{currentstroke}{rgb}{0.000000,0.000000,0.000000}%
\pgfsetstrokecolor{currentstroke}%
\pgfsetdash{}{0pt}%
\pgfpathmoveto{\pgfqpoint{2.398330in}{0.416447in}}%
\pgfpathlineto{\pgfqpoint{2.398330in}{1.788330in}}%
\pgfusepath{stroke}%
\end{pgfscope}%
\begin{pgfscope}%
\pgfsetrectcap%
\pgfsetmiterjoin%
\pgfsetlinewidth{0.803000pt}%
\definecolor{currentstroke}{rgb}{0.000000,0.000000,0.000000}%
\pgfsetstrokecolor{currentstroke}%
\pgfsetdash{}{0pt}%
\pgfpathmoveto{\pgfqpoint{0.615980in}{0.416447in}}%
\pgfpathlineto{\pgfqpoint{2.398330in}{0.416447in}}%
\pgfusepath{stroke}%
\end{pgfscope}%
\begin{pgfscope}%
\pgfsetrectcap%
\pgfsetmiterjoin%
\pgfsetlinewidth{0.803000pt}%
\definecolor{currentstroke}{rgb}{0.000000,0.000000,0.000000}%
\pgfsetstrokecolor{currentstroke}%
\pgfsetdash{}{0pt}%
\pgfpathmoveto{\pgfqpoint{0.615980in}{1.788330in}}%
\pgfpathlineto{\pgfqpoint{2.398330in}{1.788330in}}%
\pgfusepath{stroke}%
\end{pgfscope}%
\begin{pgfscope}%
\pgfsetbuttcap%
\pgfsetmiterjoin%
\definecolor{currentfill}{rgb}{1.000000,1.000000,1.000000}%
\pgfsetfillcolor{currentfill}%
\pgfsetfillopacity{0.800000}%
\pgfsetlinewidth{1.003750pt}%
\definecolor{currentstroke}{rgb}{0.800000,0.800000,0.800000}%
\pgfsetstrokecolor{currentstroke}%
\pgfsetstrokeopacity{0.800000}%
\pgfsetdash{}{0pt}%
\pgfpathmoveto{\pgfqpoint{0.693757in}{1.544552in}}%
\pgfpathlineto{\pgfqpoint{1.644535in}{1.544552in}}%
\pgfpathquadraticcurveto{\pgfqpoint{1.666757in}{1.544552in}}{\pgfqpoint{1.666757in}{1.566775in}}%
\pgfpathlineto{\pgfqpoint{1.666757in}{1.710552in}}%
\pgfpathquadraticcurveto{\pgfqpoint{1.666757in}{1.732774in}}{\pgfqpoint{1.644535in}{1.732774in}}%
\pgfpathlineto{\pgfqpoint{0.693757in}{1.732774in}}%
\pgfpathquadraticcurveto{\pgfqpoint{0.671535in}{1.732774in}}{\pgfqpoint{0.671535in}{1.710552in}}%
\pgfpathlineto{\pgfqpoint{0.671535in}{1.566775in}}%
\pgfpathquadraticcurveto{\pgfqpoint{0.671535in}{1.544552in}}{\pgfqpoint{0.693757in}{1.544552in}}%
\pgfpathlineto{\pgfqpoint{0.693757in}{1.544552in}}%
\pgfpathclose%
\pgfusepath{stroke,fill}%
\end{pgfscope}%
\begin{pgfscope}%
\pgfsetrectcap%
\pgfsetroundjoin%
\pgfsetlinewidth{1.505625pt}%
\definecolor{currentstroke}{rgb}{0.800000,0.474510,0.654902}%
\pgfsetstrokecolor{currentstroke}%
\pgfsetdash{}{0pt}%
\pgfpathmoveto{\pgfqpoint{0.715980in}{1.649441in}}%
\pgfpathlineto{\pgfqpoint{0.827091in}{1.649441in}}%
\pgfpathlineto{\pgfqpoint{0.938202in}{1.649441in}}%
\pgfusepath{stroke}%
\end{pgfscope}%
\begin{pgfscope}%
\definecolor{textcolor}{rgb}{0.000000,0.000000,0.000000}%
\pgfsetstrokecolor{textcolor}%
\pgfsetfillcolor{textcolor}%
\pgftext[x=1.027091in,y=1.610552in,left,base]{\color{textcolor}\rmfamily\fontsize{8.000000}{9.600000}\selectfont Linear drift}%
\end{pgfscope}%
\end{pgfpicture}%
\makeatother%
\endgroup%

        } % scalebox
        \caption{Time domain}
        \label{fig:drift_time}
    \end{subfigure}
    \begin{subfigure}{0.32\linewidth}
        \centering
        \scalebox{0.75}{%
            %% Creator: Matplotlib, PGF backend
%%
%% To include the figure in your LaTeX document, write
%%   \input{<filename>.pgf}
%%
%% Make sure the required packages are loaded in your preamble
%%   \usepackage{pgf}
%%
%% Also ensure that all the required font packages are loaded; for instance,
%% the lmodern package is sometimes necessary when using math font.
%%   \usepackage{lmodern}
%%
%% Figures using additional raster images can only be included by \input if
%% they are in the same directory as the main LaTeX file. For loading figures
%% from other directories you can use the `import` package
%%   \usepackage{import}
%%
%% and then include the figures with
%%   \import{<path to file>}{<filename>.pgf}
%%
%% Matplotlib used the following preamble
%%   \usepackage{siunitx}
%%   \sisetup{per-mode = symbol}%
%%   \usepackage{fontspec}
%%   \makeatletter\@ifpackageloaded{underscore}{}{\usepackage[strings]{underscore}}\makeatother
%%
\begingroup%
\makeatletter%
\begin{pgfpicture}%
\pgfpathrectangle{\pgfpointorigin}{\pgfqpoint{2.440945in}{1.830709in}}%
\pgfusepath{use as bounding box, clip}%
\begin{pgfscope}%
\pgfsetbuttcap%
\pgfsetmiterjoin%
\definecolor{currentfill}{rgb}{1.000000,1.000000,1.000000}%
\pgfsetfillcolor{currentfill}%
\pgfsetlinewidth{0.000000pt}%
\definecolor{currentstroke}{rgb}{1.000000,1.000000,1.000000}%
\pgfsetstrokecolor{currentstroke}%
\pgfsetdash{}{0pt}%
\pgfpathmoveto{\pgfqpoint{0.000000in}{0.000000in}}%
\pgfpathlineto{\pgfqpoint{2.440945in}{0.000000in}}%
\pgfpathlineto{\pgfqpoint{2.440945in}{1.830709in}}%
\pgfpathlineto{\pgfqpoint{0.000000in}{1.830709in}}%
\pgfpathlineto{\pgfqpoint{0.000000in}{0.000000in}}%
\pgfpathclose%
\pgfusepath{fill}%
\end{pgfscope}%
\begin{pgfscope}%
\pgfsetbuttcap%
\pgfsetmiterjoin%
\definecolor{currentfill}{rgb}{1.000000,1.000000,1.000000}%
\pgfsetfillcolor{currentfill}%
\pgfsetlinewidth{0.000000pt}%
\definecolor{currentstroke}{rgb}{0.000000,0.000000,0.000000}%
\pgfsetstrokecolor{currentstroke}%
\pgfsetstrokeopacity{0.000000}%
\pgfsetdash{}{0pt}%
\pgfpathmoveto{\pgfqpoint{0.589510in}{0.417642in}}%
\pgfpathlineto{\pgfqpoint{2.399275in}{0.417642in}}%
\pgfpathlineto{\pgfqpoint{2.399275in}{1.789039in}}%
\pgfpathlineto{\pgfqpoint{0.589510in}{1.789039in}}%
\pgfpathlineto{\pgfqpoint{0.589510in}{0.417642in}}%
\pgfpathclose%
\pgfusepath{fill}%
\end{pgfscope}%
\begin{pgfscope}%
\pgfpathrectangle{\pgfqpoint{0.589510in}{0.417642in}}{\pgfqpoint{1.809765in}{1.371397in}}%
\pgfusepath{clip}%
\pgfsetrectcap%
\pgfsetroundjoin%
\pgfsetlinewidth{0.803000pt}%
\definecolor{currentstroke}{rgb}{0.450000,0.450000,0.450000}%
\pgfsetstrokecolor{currentstroke}%
\pgfsetdash{}{0pt}%
\pgfpathmoveto{\pgfqpoint{0.671772in}{0.417642in}}%
\pgfpathlineto{\pgfqpoint{0.671772in}{1.789039in}}%
\pgfusepath{stroke}%
\end{pgfscope}%
\begin{pgfscope}%
\pgfsetbuttcap%
\pgfsetroundjoin%
\definecolor{currentfill}{rgb}{0.000000,0.000000,0.000000}%
\pgfsetfillcolor{currentfill}%
\pgfsetlinewidth{0.803000pt}%
\definecolor{currentstroke}{rgb}{0.000000,0.000000,0.000000}%
\pgfsetstrokecolor{currentstroke}%
\pgfsetdash{}{0pt}%
\pgfsys@defobject{currentmarker}{\pgfqpoint{0.000000in}{-0.048611in}}{\pgfqpoint{0.000000in}{0.000000in}}{%
\pgfpathmoveto{\pgfqpoint{0.000000in}{0.000000in}}%
\pgfpathlineto{\pgfqpoint{0.000000in}{-0.048611in}}%
\pgfusepath{stroke,fill}%
}%
\begin{pgfscope}%
\pgfsys@transformshift{0.671772in}{0.417642in}%
\pgfsys@useobject{currentmarker}{}%
\end{pgfscope}%
\end{pgfscope}%
\begin{pgfscope}%
\definecolor{textcolor}{rgb}{0.000000,0.000000,0.000000}%
\pgfsetstrokecolor{textcolor}%
\pgfsetfillcolor{textcolor}%
\pgftext[x=0.671772in,y=0.320420in,,top]{\color{textcolor}\rmfamily\fontsize{8.000000}{9.600000}\selectfont \(\displaystyle {10^{0}}\)}%
\end{pgfscope}%
\begin{pgfscope}%
\pgfpathrectangle{\pgfqpoint{0.589510in}{0.417642in}}{\pgfqpoint{1.809765in}{1.371397in}}%
\pgfusepath{clip}%
\pgfsetrectcap%
\pgfsetroundjoin%
\pgfsetlinewidth{0.803000pt}%
\definecolor{currentstroke}{rgb}{0.450000,0.450000,0.450000}%
\pgfsetstrokecolor{currentstroke}%
\pgfsetdash{}{0pt}%
\pgfpathmoveto{\pgfqpoint{1.128522in}{0.417642in}}%
\pgfpathlineto{\pgfqpoint{1.128522in}{1.789039in}}%
\pgfusepath{stroke}%
\end{pgfscope}%
\begin{pgfscope}%
\pgfsetbuttcap%
\pgfsetroundjoin%
\definecolor{currentfill}{rgb}{0.000000,0.000000,0.000000}%
\pgfsetfillcolor{currentfill}%
\pgfsetlinewidth{0.803000pt}%
\definecolor{currentstroke}{rgb}{0.000000,0.000000,0.000000}%
\pgfsetstrokecolor{currentstroke}%
\pgfsetdash{}{0pt}%
\pgfsys@defobject{currentmarker}{\pgfqpoint{0.000000in}{-0.048611in}}{\pgfqpoint{0.000000in}{0.000000in}}{%
\pgfpathmoveto{\pgfqpoint{0.000000in}{0.000000in}}%
\pgfpathlineto{\pgfqpoint{0.000000in}{-0.048611in}}%
\pgfusepath{stroke,fill}%
}%
\begin{pgfscope}%
\pgfsys@transformshift{1.128522in}{0.417642in}%
\pgfsys@useobject{currentmarker}{}%
\end{pgfscope}%
\end{pgfscope}%
\begin{pgfscope}%
\definecolor{textcolor}{rgb}{0.000000,0.000000,0.000000}%
\pgfsetstrokecolor{textcolor}%
\pgfsetfillcolor{textcolor}%
\pgftext[x=1.128522in,y=0.320420in,,top]{\color{textcolor}\rmfamily\fontsize{8.000000}{9.600000}\selectfont \(\displaystyle {10^{1}}\)}%
\end{pgfscope}%
\begin{pgfscope}%
\pgfpathrectangle{\pgfqpoint{0.589510in}{0.417642in}}{\pgfqpoint{1.809765in}{1.371397in}}%
\pgfusepath{clip}%
\pgfsetrectcap%
\pgfsetroundjoin%
\pgfsetlinewidth{0.803000pt}%
\definecolor{currentstroke}{rgb}{0.450000,0.450000,0.450000}%
\pgfsetstrokecolor{currentstroke}%
\pgfsetdash{}{0pt}%
\pgfpathmoveto{\pgfqpoint{1.585272in}{0.417642in}}%
\pgfpathlineto{\pgfqpoint{1.585272in}{1.789039in}}%
\pgfusepath{stroke}%
\end{pgfscope}%
\begin{pgfscope}%
\pgfsetbuttcap%
\pgfsetroundjoin%
\definecolor{currentfill}{rgb}{0.000000,0.000000,0.000000}%
\pgfsetfillcolor{currentfill}%
\pgfsetlinewidth{0.803000pt}%
\definecolor{currentstroke}{rgb}{0.000000,0.000000,0.000000}%
\pgfsetstrokecolor{currentstroke}%
\pgfsetdash{}{0pt}%
\pgfsys@defobject{currentmarker}{\pgfqpoint{0.000000in}{-0.048611in}}{\pgfqpoint{0.000000in}{0.000000in}}{%
\pgfpathmoveto{\pgfqpoint{0.000000in}{0.000000in}}%
\pgfpathlineto{\pgfqpoint{0.000000in}{-0.048611in}}%
\pgfusepath{stroke,fill}%
}%
\begin{pgfscope}%
\pgfsys@transformshift{1.585272in}{0.417642in}%
\pgfsys@useobject{currentmarker}{}%
\end{pgfscope}%
\end{pgfscope}%
\begin{pgfscope}%
\definecolor{textcolor}{rgb}{0.000000,0.000000,0.000000}%
\pgfsetstrokecolor{textcolor}%
\pgfsetfillcolor{textcolor}%
\pgftext[x=1.585272in,y=0.320420in,,top]{\color{textcolor}\rmfamily\fontsize{8.000000}{9.600000}\selectfont \(\displaystyle {10^{2}}\)}%
\end{pgfscope}%
\begin{pgfscope}%
\pgfpathrectangle{\pgfqpoint{0.589510in}{0.417642in}}{\pgfqpoint{1.809765in}{1.371397in}}%
\pgfusepath{clip}%
\pgfsetrectcap%
\pgfsetroundjoin%
\pgfsetlinewidth{0.803000pt}%
\definecolor{currentstroke}{rgb}{0.450000,0.450000,0.450000}%
\pgfsetstrokecolor{currentstroke}%
\pgfsetdash{}{0pt}%
\pgfpathmoveto{\pgfqpoint{2.042022in}{0.417642in}}%
\pgfpathlineto{\pgfqpoint{2.042022in}{1.789039in}}%
\pgfusepath{stroke}%
\end{pgfscope}%
\begin{pgfscope}%
\pgfsetbuttcap%
\pgfsetroundjoin%
\definecolor{currentfill}{rgb}{0.000000,0.000000,0.000000}%
\pgfsetfillcolor{currentfill}%
\pgfsetlinewidth{0.803000pt}%
\definecolor{currentstroke}{rgb}{0.000000,0.000000,0.000000}%
\pgfsetstrokecolor{currentstroke}%
\pgfsetdash{}{0pt}%
\pgfsys@defobject{currentmarker}{\pgfqpoint{0.000000in}{-0.048611in}}{\pgfqpoint{0.000000in}{0.000000in}}{%
\pgfpathmoveto{\pgfqpoint{0.000000in}{0.000000in}}%
\pgfpathlineto{\pgfqpoint{0.000000in}{-0.048611in}}%
\pgfusepath{stroke,fill}%
}%
\begin{pgfscope}%
\pgfsys@transformshift{2.042022in}{0.417642in}%
\pgfsys@useobject{currentmarker}{}%
\end{pgfscope}%
\end{pgfscope}%
\begin{pgfscope}%
\definecolor{textcolor}{rgb}{0.000000,0.000000,0.000000}%
\pgfsetstrokecolor{textcolor}%
\pgfsetfillcolor{textcolor}%
\pgftext[x=2.042022in,y=0.320420in,,top]{\color{textcolor}\rmfamily\fontsize{8.000000}{9.600000}\selectfont \(\displaystyle {10^{3}}\)}%
\end{pgfscope}%
\begin{pgfscope}%
\pgfpathrectangle{\pgfqpoint{0.589510in}{0.417642in}}{\pgfqpoint{1.809765in}{1.371397in}}%
\pgfusepath{clip}%
\pgfsetrectcap%
\pgfsetroundjoin%
\pgfsetlinewidth{0.803000pt}%
\definecolor{currentstroke}{rgb}{0.850000,0.850000,0.850000}%
\pgfsetstrokecolor{currentstroke}%
\pgfsetdash{}{0pt}%
\pgfpathmoveto{\pgfqpoint{0.601020in}{0.417642in}}%
\pgfpathlineto{\pgfqpoint{0.601020in}{1.789039in}}%
\pgfusepath{stroke}%
\end{pgfscope}%
\begin{pgfscope}%
\pgfsetbuttcap%
\pgfsetroundjoin%
\definecolor{currentfill}{rgb}{0.000000,0.000000,0.000000}%
\pgfsetfillcolor{currentfill}%
\pgfsetlinewidth{0.602250pt}%
\definecolor{currentstroke}{rgb}{0.000000,0.000000,0.000000}%
\pgfsetstrokecolor{currentstroke}%
\pgfsetdash{}{0pt}%
\pgfsys@defobject{currentmarker}{\pgfqpoint{0.000000in}{-0.027778in}}{\pgfqpoint{0.000000in}{0.000000in}}{%
\pgfpathmoveto{\pgfqpoint{0.000000in}{0.000000in}}%
\pgfpathlineto{\pgfqpoint{0.000000in}{-0.027778in}}%
\pgfusepath{stroke,fill}%
}%
\begin{pgfscope}%
\pgfsys@transformshift{0.601020in}{0.417642in}%
\pgfsys@useobject{currentmarker}{}%
\end{pgfscope}%
\end{pgfscope}%
\begin{pgfscope}%
\pgfpathrectangle{\pgfqpoint{0.589510in}{0.417642in}}{\pgfqpoint{1.809765in}{1.371397in}}%
\pgfusepath{clip}%
\pgfsetrectcap%
\pgfsetroundjoin%
\pgfsetlinewidth{0.803000pt}%
\definecolor{currentstroke}{rgb}{0.850000,0.850000,0.850000}%
\pgfsetstrokecolor{currentstroke}%
\pgfsetdash{}{0pt}%
\pgfpathmoveto{\pgfqpoint{0.627508in}{0.417642in}}%
\pgfpathlineto{\pgfqpoint{0.627508in}{1.789039in}}%
\pgfusepath{stroke}%
\end{pgfscope}%
\begin{pgfscope}%
\pgfsetbuttcap%
\pgfsetroundjoin%
\definecolor{currentfill}{rgb}{0.000000,0.000000,0.000000}%
\pgfsetfillcolor{currentfill}%
\pgfsetlinewidth{0.602250pt}%
\definecolor{currentstroke}{rgb}{0.000000,0.000000,0.000000}%
\pgfsetstrokecolor{currentstroke}%
\pgfsetdash{}{0pt}%
\pgfsys@defobject{currentmarker}{\pgfqpoint{0.000000in}{-0.027778in}}{\pgfqpoint{0.000000in}{0.000000in}}{%
\pgfpathmoveto{\pgfqpoint{0.000000in}{0.000000in}}%
\pgfpathlineto{\pgfqpoint{0.000000in}{-0.027778in}}%
\pgfusepath{stroke,fill}%
}%
\begin{pgfscope}%
\pgfsys@transformshift{0.627508in}{0.417642in}%
\pgfsys@useobject{currentmarker}{}%
\end{pgfscope}%
\end{pgfscope}%
\begin{pgfscope}%
\pgfpathrectangle{\pgfqpoint{0.589510in}{0.417642in}}{\pgfqpoint{1.809765in}{1.371397in}}%
\pgfusepath{clip}%
\pgfsetrectcap%
\pgfsetroundjoin%
\pgfsetlinewidth{0.803000pt}%
\definecolor{currentstroke}{rgb}{0.850000,0.850000,0.850000}%
\pgfsetstrokecolor{currentstroke}%
\pgfsetdash{}{0pt}%
\pgfpathmoveto{\pgfqpoint{0.650872in}{0.417642in}}%
\pgfpathlineto{\pgfqpoint{0.650872in}{1.789039in}}%
\pgfusepath{stroke}%
\end{pgfscope}%
\begin{pgfscope}%
\pgfsetbuttcap%
\pgfsetroundjoin%
\definecolor{currentfill}{rgb}{0.000000,0.000000,0.000000}%
\pgfsetfillcolor{currentfill}%
\pgfsetlinewidth{0.602250pt}%
\definecolor{currentstroke}{rgb}{0.000000,0.000000,0.000000}%
\pgfsetstrokecolor{currentstroke}%
\pgfsetdash{}{0pt}%
\pgfsys@defobject{currentmarker}{\pgfqpoint{0.000000in}{-0.027778in}}{\pgfqpoint{0.000000in}{0.000000in}}{%
\pgfpathmoveto{\pgfqpoint{0.000000in}{0.000000in}}%
\pgfpathlineto{\pgfqpoint{0.000000in}{-0.027778in}}%
\pgfusepath{stroke,fill}%
}%
\begin{pgfscope}%
\pgfsys@transformshift{0.650872in}{0.417642in}%
\pgfsys@useobject{currentmarker}{}%
\end{pgfscope}%
\end{pgfscope}%
\begin{pgfscope}%
\pgfpathrectangle{\pgfqpoint{0.589510in}{0.417642in}}{\pgfqpoint{1.809765in}{1.371397in}}%
\pgfusepath{clip}%
\pgfsetrectcap%
\pgfsetroundjoin%
\pgfsetlinewidth{0.803000pt}%
\definecolor{currentstroke}{rgb}{0.850000,0.850000,0.850000}%
\pgfsetstrokecolor{currentstroke}%
\pgfsetdash{}{0pt}%
\pgfpathmoveto{\pgfqpoint{0.809267in}{0.417642in}}%
\pgfpathlineto{\pgfqpoint{0.809267in}{1.789039in}}%
\pgfusepath{stroke}%
\end{pgfscope}%
\begin{pgfscope}%
\pgfsetbuttcap%
\pgfsetroundjoin%
\definecolor{currentfill}{rgb}{0.000000,0.000000,0.000000}%
\pgfsetfillcolor{currentfill}%
\pgfsetlinewidth{0.602250pt}%
\definecolor{currentstroke}{rgb}{0.000000,0.000000,0.000000}%
\pgfsetstrokecolor{currentstroke}%
\pgfsetdash{}{0pt}%
\pgfsys@defobject{currentmarker}{\pgfqpoint{0.000000in}{-0.027778in}}{\pgfqpoint{0.000000in}{0.000000in}}{%
\pgfpathmoveto{\pgfqpoint{0.000000in}{0.000000in}}%
\pgfpathlineto{\pgfqpoint{0.000000in}{-0.027778in}}%
\pgfusepath{stroke,fill}%
}%
\begin{pgfscope}%
\pgfsys@transformshift{0.809267in}{0.417642in}%
\pgfsys@useobject{currentmarker}{}%
\end{pgfscope}%
\end{pgfscope}%
\begin{pgfscope}%
\pgfpathrectangle{\pgfqpoint{0.589510in}{0.417642in}}{\pgfqpoint{1.809765in}{1.371397in}}%
\pgfusepath{clip}%
\pgfsetrectcap%
\pgfsetroundjoin%
\pgfsetlinewidth{0.803000pt}%
\definecolor{currentstroke}{rgb}{0.850000,0.850000,0.850000}%
\pgfsetstrokecolor{currentstroke}%
\pgfsetdash{}{0pt}%
\pgfpathmoveto{\pgfqpoint{0.889697in}{0.417642in}}%
\pgfpathlineto{\pgfqpoint{0.889697in}{1.789039in}}%
\pgfusepath{stroke}%
\end{pgfscope}%
\begin{pgfscope}%
\pgfsetbuttcap%
\pgfsetroundjoin%
\definecolor{currentfill}{rgb}{0.000000,0.000000,0.000000}%
\pgfsetfillcolor{currentfill}%
\pgfsetlinewidth{0.602250pt}%
\definecolor{currentstroke}{rgb}{0.000000,0.000000,0.000000}%
\pgfsetstrokecolor{currentstroke}%
\pgfsetdash{}{0pt}%
\pgfsys@defobject{currentmarker}{\pgfqpoint{0.000000in}{-0.027778in}}{\pgfqpoint{0.000000in}{0.000000in}}{%
\pgfpathmoveto{\pgfqpoint{0.000000in}{0.000000in}}%
\pgfpathlineto{\pgfqpoint{0.000000in}{-0.027778in}}%
\pgfusepath{stroke,fill}%
}%
\begin{pgfscope}%
\pgfsys@transformshift{0.889697in}{0.417642in}%
\pgfsys@useobject{currentmarker}{}%
\end{pgfscope}%
\end{pgfscope}%
\begin{pgfscope}%
\pgfpathrectangle{\pgfqpoint{0.589510in}{0.417642in}}{\pgfqpoint{1.809765in}{1.371397in}}%
\pgfusepath{clip}%
\pgfsetrectcap%
\pgfsetroundjoin%
\pgfsetlinewidth{0.803000pt}%
\definecolor{currentstroke}{rgb}{0.850000,0.850000,0.850000}%
\pgfsetstrokecolor{currentstroke}%
\pgfsetdash{}{0pt}%
\pgfpathmoveto{\pgfqpoint{0.946763in}{0.417642in}}%
\pgfpathlineto{\pgfqpoint{0.946763in}{1.789039in}}%
\pgfusepath{stroke}%
\end{pgfscope}%
\begin{pgfscope}%
\pgfsetbuttcap%
\pgfsetroundjoin%
\definecolor{currentfill}{rgb}{0.000000,0.000000,0.000000}%
\pgfsetfillcolor{currentfill}%
\pgfsetlinewidth{0.602250pt}%
\definecolor{currentstroke}{rgb}{0.000000,0.000000,0.000000}%
\pgfsetstrokecolor{currentstroke}%
\pgfsetdash{}{0pt}%
\pgfsys@defobject{currentmarker}{\pgfqpoint{0.000000in}{-0.027778in}}{\pgfqpoint{0.000000in}{0.000000in}}{%
\pgfpathmoveto{\pgfqpoint{0.000000in}{0.000000in}}%
\pgfpathlineto{\pgfqpoint{0.000000in}{-0.027778in}}%
\pgfusepath{stroke,fill}%
}%
\begin{pgfscope}%
\pgfsys@transformshift{0.946763in}{0.417642in}%
\pgfsys@useobject{currentmarker}{}%
\end{pgfscope}%
\end{pgfscope}%
\begin{pgfscope}%
\pgfpathrectangle{\pgfqpoint{0.589510in}{0.417642in}}{\pgfqpoint{1.809765in}{1.371397in}}%
\pgfusepath{clip}%
\pgfsetrectcap%
\pgfsetroundjoin%
\pgfsetlinewidth{0.803000pt}%
\definecolor{currentstroke}{rgb}{0.850000,0.850000,0.850000}%
\pgfsetstrokecolor{currentstroke}%
\pgfsetdash{}{0pt}%
\pgfpathmoveto{\pgfqpoint{0.991026in}{0.417642in}}%
\pgfpathlineto{\pgfqpoint{0.991026in}{1.789039in}}%
\pgfusepath{stroke}%
\end{pgfscope}%
\begin{pgfscope}%
\pgfsetbuttcap%
\pgfsetroundjoin%
\definecolor{currentfill}{rgb}{0.000000,0.000000,0.000000}%
\pgfsetfillcolor{currentfill}%
\pgfsetlinewidth{0.602250pt}%
\definecolor{currentstroke}{rgb}{0.000000,0.000000,0.000000}%
\pgfsetstrokecolor{currentstroke}%
\pgfsetdash{}{0pt}%
\pgfsys@defobject{currentmarker}{\pgfqpoint{0.000000in}{-0.027778in}}{\pgfqpoint{0.000000in}{0.000000in}}{%
\pgfpathmoveto{\pgfqpoint{0.000000in}{0.000000in}}%
\pgfpathlineto{\pgfqpoint{0.000000in}{-0.027778in}}%
\pgfusepath{stroke,fill}%
}%
\begin{pgfscope}%
\pgfsys@transformshift{0.991026in}{0.417642in}%
\pgfsys@useobject{currentmarker}{}%
\end{pgfscope}%
\end{pgfscope}%
\begin{pgfscope}%
\pgfpathrectangle{\pgfqpoint{0.589510in}{0.417642in}}{\pgfqpoint{1.809765in}{1.371397in}}%
\pgfusepath{clip}%
\pgfsetrectcap%
\pgfsetroundjoin%
\pgfsetlinewidth{0.803000pt}%
\definecolor{currentstroke}{rgb}{0.850000,0.850000,0.850000}%
\pgfsetstrokecolor{currentstroke}%
\pgfsetdash{}{0pt}%
\pgfpathmoveto{\pgfqpoint{1.027192in}{0.417642in}}%
\pgfpathlineto{\pgfqpoint{1.027192in}{1.789039in}}%
\pgfusepath{stroke}%
\end{pgfscope}%
\begin{pgfscope}%
\pgfsetbuttcap%
\pgfsetroundjoin%
\definecolor{currentfill}{rgb}{0.000000,0.000000,0.000000}%
\pgfsetfillcolor{currentfill}%
\pgfsetlinewidth{0.602250pt}%
\definecolor{currentstroke}{rgb}{0.000000,0.000000,0.000000}%
\pgfsetstrokecolor{currentstroke}%
\pgfsetdash{}{0pt}%
\pgfsys@defobject{currentmarker}{\pgfqpoint{0.000000in}{-0.027778in}}{\pgfqpoint{0.000000in}{0.000000in}}{%
\pgfpathmoveto{\pgfqpoint{0.000000in}{0.000000in}}%
\pgfpathlineto{\pgfqpoint{0.000000in}{-0.027778in}}%
\pgfusepath{stroke,fill}%
}%
\begin{pgfscope}%
\pgfsys@transformshift{1.027192in}{0.417642in}%
\pgfsys@useobject{currentmarker}{}%
\end{pgfscope}%
\end{pgfscope}%
\begin{pgfscope}%
\pgfpathrectangle{\pgfqpoint{0.589510in}{0.417642in}}{\pgfqpoint{1.809765in}{1.371397in}}%
\pgfusepath{clip}%
\pgfsetrectcap%
\pgfsetroundjoin%
\pgfsetlinewidth{0.803000pt}%
\definecolor{currentstroke}{rgb}{0.850000,0.850000,0.850000}%
\pgfsetstrokecolor{currentstroke}%
\pgfsetdash{}{0pt}%
\pgfpathmoveto{\pgfqpoint{1.057770in}{0.417642in}}%
\pgfpathlineto{\pgfqpoint{1.057770in}{1.789039in}}%
\pgfusepath{stroke}%
\end{pgfscope}%
\begin{pgfscope}%
\pgfsetbuttcap%
\pgfsetroundjoin%
\definecolor{currentfill}{rgb}{0.000000,0.000000,0.000000}%
\pgfsetfillcolor{currentfill}%
\pgfsetlinewidth{0.602250pt}%
\definecolor{currentstroke}{rgb}{0.000000,0.000000,0.000000}%
\pgfsetstrokecolor{currentstroke}%
\pgfsetdash{}{0pt}%
\pgfsys@defobject{currentmarker}{\pgfqpoint{0.000000in}{-0.027778in}}{\pgfqpoint{0.000000in}{0.000000in}}{%
\pgfpathmoveto{\pgfqpoint{0.000000in}{0.000000in}}%
\pgfpathlineto{\pgfqpoint{0.000000in}{-0.027778in}}%
\pgfusepath{stroke,fill}%
}%
\begin{pgfscope}%
\pgfsys@transformshift{1.057770in}{0.417642in}%
\pgfsys@useobject{currentmarker}{}%
\end{pgfscope}%
\end{pgfscope}%
\begin{pgfscope}%
\pgfpathrectangle{\pgfqpoint{0.589510in}{0.417642in}}{\pgfqpoint{1.809765in}{1.371397in}}%
\pgfusepath{clip}%
\pgfsetrectcap%
\pgfsetroundjoin%
\pgfsetlinewidth{0.803000pt}%
\definecolor{currentstroke}{rgb}{0.850000,0.850000,0.850000}%
\pgfsetstrokecolor{currentstroke}%
\pgfsetdash{}{0pt}%
\pgfpathmoveto{\pgfqpoint{1.084258in}{0.417642in}}%
\pgfpathlineto{\pgfqpoint{1.084258in}{1.789039in}}%
\pgfusepath{stroke}%
\end{pgfscope}%
\begin{pgfscope}%
\pgfsetbuttcap%
\pgfsetroundjoin%
\definecolor{currentfill}{rgb}{0.000000,0.000000,0.000000}%
\pgfsetfillcolor{currentfill}%
\pgfsetlinewidth{0.602250pt}%
\definecolor{currentstroke}{rgb}{0.000000,0.000000,0.000000}%
\pgfsetstrokecolor{currentstroke}%
\pgfsetdash{}{0pt}%
\pgfsys@defobject{currentmarker}{\pgfqpoint{0.000000in}{-0.027778in}}{\pgfqpoint{0.000000in}{0.000000in}}{%
\pgfpathmoveto{\pgfqpoint{0.000000in}{0.000000in}}%
\pgfpathlineto{\pgfqpoint{0.000000in}{-0.027778in}}%
\pgfusepath{stroke,fill}%
}%
\begin{pgfscope}%
\pgfsys@transformshift{1.084258in}{0.417642in}%
\pgfsys@useobject{currentmarker}{}%
\end{pgfscope}%
\end{pgfscope}%
\begin{pgfscope}%
\pgfpathrectangle{\pgfqpoint{0.589510in}{0.417642in}}{\pgfqpoint{1.809765in}{1.371397in}}%
\pgfusepath{clip}%
\pgfsetrectcap%
\pgfsetroundjoin%
\pgfsetlinewidth{0.803000pt}%
\definecolor{currentstroke}{rgb}{0.850000,0.850000,0.850000}%
\pgfsetstrokecolor{currentstroke}%
\pgfsetdash{}{0pt}%
\pgfpathmoveto{\pgfqpoint{1.107622in}{0.417642in}}%
\pgfpathlineto{\pgfqpoint{1.107622in}{1.789039in}}%
\pgfusepath{stroke}%
\end{pgfscope}%
\begin{pgfscope}%
\pgfsetbuttcap%
\pgfsetroundjoin%
\definecolor{currentfill}{rgb}{0.000000,0.000000,0.000000}%
\pgfsetfillcolor{currentfill}%
\pgfsetlinewidth{0.602250pt}%
\definecolor{currentstroke}{rgb}{0.000000,0.000000,0.000000}%
\pgfsetstrokecolor{currentstroke}%
\pgfsetdash{}{0pt}%
\pgfsys@defobject{currentmarker}{\pgfqpoint{0.000000in}{-0.027778in}}{\pgfqpoint{0.000000in}{0.000000in}}{%
\pgfpathmoveto{\pgfqpoint{0.000000in}{0.000000in}}%
\pgfpathlineto{\pgfqpoint{0.000000in}{-0.027778in}}%
\pgfusepath{stroke,fill}%
}%
\begin{pgfscope}%
\pgfsys@transformshift{1.107622in}{0.417642in}%
\pgfsys@useobject{currentmarker}{}%
\end{pgfscope}%
\end{pgfscope}%
\begin{pgfscope}%
\pgfpathrectangle{\pgfqpoint{0.589510in}{0.417642in}}{\pgfqpoint{1.809765in}{1.371397in}}%
\pgfusepath{clip}%
\pgfsetrectcap%
\pgfsetroundjoin%
\pgfsetlinewidth{0.803000pt}%
\definecolor{currentstroke}{rgb}{0.850000,0.850000,0.850000}%
\pgfsetstrokecolor{currentstroke}%
\pgfsetdash{}{0pt}%
\pgfpathmoveto{\pgfqpoint{1.266017in}{0.417642in}}%
\pgfpathlineto{\pgfqpoint{1.266017in}{1.789039in}}%
\pgfusepath{stroke}%
\end{pgfscope}%
\begin{pgfscope}%
\pgfsetbuttcap%
\pgfsetroundjoin%
\definecolor{currentfill}{rgb}{0.000000,0.000000,0.000000}%
\pgfsetfillcolor{currentfill}%
\pgfsetlinewidth{0.602250pt}%
\definecolor{currentstroke}{rgb}{0.000000,0.000000,0.000000}%
\pgfsetstrokecolor{currentstroke}%
\pgfsetdash{}{0pt}%
\pgfsys@defobject{currentmarker}{\pgfqpoint{0.000000in}{-0.027778in}}{\pgfqpoint{0.000000in}{0.000000in}}{%
\pgfpathmoveto{\pgfqpoint{0.000000in}{0.000000in}}%
\pgfpathlineto{\pgfqpoint{0.000000in}{-0.027778in}}%
\pgfusepath{stroke,fill}%
}%
\begin{pgfscope}%
\pgfsys@transformshift{1.266017in}{0.417642in}%
\pgfsys@useobject{currentmarker}{}%
\end{pgfscope}%
\end{pgfscope}%
\begin{pgfscope}%
\pgfpathrectangle{\pgfqpoint{0.589510in}{0.417642in}}{\pgfqpoint{1.809765in}{1.371397in}}%
\pgfusepath{clip}%
\pgfsetrectcap%
\pgfsetroundjoin%
\pgfsetlinewidth{0.803000pt}%
\definecolor{currentstroke}{rgb}{0.850000,0.850000,0.850000}%
\pgfsetstrokecolor{currentstroke}%
\pgfsetdash{}{0pt}%
\pgfpathmoveto{\pgfqpoint{1.346447in}{0.417642in}}%
\pgfpathlineto{\pgfqpoint{1.346447in}{1.789039in}}%
\pgfusepath{stroke}%
\end{pgfscope}%
\begin{pgfscope}%
\pgfsetbuttcap%
\pgfsetroundjoin%
\definecolor{currentfill}{rgb}{0.000000,0.000000,0.000000}%
\pgfsetfillcolor{currentfill}%
\pgfsetlinewidth{0.602250pt}%
\definecolor{currentstroke}{rgb}{0.000000,0.000000,0.000000}%
\pgfsetstrokecolor{currentstroke}%
\pgfsetdash{}{0pt}%
\pgfsys@defobject{currentmarker}{\pgfqpoint{0.000000in}{-0.027778in}}{\pgfqpoint{0.000000in}{0.000000in}}{%
\pgfpathmoveto{\pgfqpoint{0.000000in}{0.000000in}}%
\pgfpathlineto{\pgfqpoint{0.000000in}{-0.027778in}}%
\pgfusepath{stroke,fill}%
}%
\begin{pgfscope}%
\pgfsys@transformshift{1.346447in}{0.417642in}%
\pgfsys@useobject{currentmarker}{}%
\end{pgfscope}%
\end{pgfscope}%
\begin{pgfscope}%
\pgfpathrectangle{\pgfqpoint{0.589510in}{0.417642in}}{\pgfqpoint{1.809765in}{1.371397in}}%
\pgfusepath{clip}%
\pgfsetrectcap%
\pgfsetroundjoin%
\pgfsetlinewidth{0.803000pt}%
\definecolor{currentstroke}{rgb}{0.850000,0.850000,0.850000}%
\pgfsetstrokecolor{currentstroke}%
\pgfsetdash{}{0pt}%
\pgfpathmoveto{\pgfqpoint{1.403513in}{0.417642in}}%
\pgfpathlineto{\pgfqpoint{1.403513in}{1.789039in}}%
\pgfusepath{stroke}%
\end{pgfscope}%
\begin{pgfscope}%
\pgfsetbuttcap%
\pgfsetroundjoin%
\definecolor{currentfill}{rgb}{0.000000,0.000000,0.000000}%
\pgfsetfillcolor{currentfill}%
\pgfsetlinewidth{0.602250pt}%
\definecolor{currentstroke}{rgb}{0.000000,0.000000,0.000000}%
\pgfsetstrokecolor{currentstroke}%
\pgfsetdash{}{0pt}%
\pgfsys@defobject{currentmarker}{\pgfqpoint{0.000000in}{-0.027778in}}{\pgfqpoint{0.000000in}{0.000000in}}{%
\pgfpathmoveto{\pgfqpoint{0.000000in}{0.000000in}}%
\pgfpathlineto{\pgfqpoint{0.000000in}{-0.027778in}}%
\pgfusepath{stroke,fill}%
}%
\begin{pgfscope}%
\pgfsys@transformshift{1.403513in}{0.417642in}%
\pgfsys@useobject{currentmarker}{}%
\end{pgfscope}%
\end{pgfscope}%
\begin{pgfscope}%
\pgfpathrectangle{\pgfqpoint{0.589510in}{0.417642in}}{\pgfqpoint{1.809765in}{1.371397in}}%
\pgfusepath{clip}%
\pgfsetrectcap%
\pgfsetroundjoin%
\pgfsetlinewidth{0.803000pt}%
\definecolor{currentstroke}{rgb}{0.850000,0.850000,0.850000}%
\pgfsetstrokecolor{currentstroke}%
\pgfsetdash{}{0pt}%
\pgfpathmoveto{\pgfqpoint{1.447776in}{0.417642in}}%
\pgfpathlineto{\pgfqpoint{1.447776in}{1.789039in}}%
\pgfusepath{stroke}%
\end{pgfscope}%
\begin{pgfscope}%
\pgfsetbuttcap%
\pgfsetroundjoin%
\definecolor{currentfill}{rgb}{0.000000,0.000000,0.000000}%
\pgfsetfillcolor{currentfill}%
\pgfsetlinewidth{0.602250pt}%
\definecolor{currentstroke}{rgb}{0.000000,0.000000,0.000000}%
\pgfsetstrokecolor{currentstroke}%
\pgfsetdash{}{0pt}%
\pgfsys@defobject{currentmarker}{\pgfqpoint{0.000000in}{-0.027778in}}{\pgfqpoint{0.000000in}{0.000000in}}{%
\pgfpathmoveto{\pgfqpoint{0.000000in}{0.000000in}}%
\pgfpathlineto{\pgfqpoint{0.000000in}{-0.027778in}}%
\pgfusepath{stroke,fill}%
}%
\begin{pgfscope}%
\pgfsys@transformshift{1.447776in}{0.417642in}%
\pgfsys@useobject{currentmarker}{}%
\end{pgfscope}%
\end{pgfscope}%
\begin{pgfscope}%
\pgfpathrectangle{\pgfqpoint{0.589510in}{0.417642in}}{\pgfqpoint{1.809765in}{1.371397in}}%
\pgfusepath{clip}%
\pgfsetrectcap%
\pgfsetroundjoin%
\pgfsetlinewidth{0.803000pt}%
\definecolor{currentstroke}{rgb}{0.850000,0.850000,0.850000}%
\pgfsetstrokecolor{currentstroke}%
\pgfsetdash{}{0pt}%
\pgfpathmoveto{\pgfqpoint{1.483942in}{0.417642in}}%
\pgfpathlineto{\pgfqpoint{1.483942in}{1.789039in}}%
\pgfusepath{stroke}%
\end{pgfscope}%
\begin{pgfscope}%
\pgfsetbuttcap%
\pgfsetroundjoin%
\definecolor{currentfill}{rgb}{0.000000,0.000000,0.000000}%
\pgfsetfillcolor{currentfill}%
\pgfsetlinewidth{0.602250pt}%
\definecolor{currentstroke}{rgb}{0.000000,0.000000,0.000000}%
\pgfsetstrokecolor{currentstroke}%
\pgfsetdash{}{0pt}%
\pgfsys@defobject{currentmarker}{\pgfqpoint{0.000000in}{-0.027778in}}{\pgfqpoint{0.000000in}{0.000000in}}{%
\pgfpathmoveto{\pgfqpoint{0.000000in}{0.000000in}}%
\pgfpathlineto{\pgfqpoint{0.000000in}{-0.027778in}}%
\pgfusepath{stroke,fill}%
}%
\begin{pgfscope}%
\pgfsys@transformshift{1.483942in}{0.417642in}%
\pgfsys@useobject{currentmarker}{}%
\end{pgfscope}%
\end{pgfscope}%
\begin{pgfscope}%
\pgfpathrectangle{\pgfqpoint{0.589510in}{0.417642in}}{\pgfqpoint{1.809765in}{1.371397in}}%
\pgfusepath{clip}%
\pgfsetrectcap%
\pgfsetroundjoin%
\pgfsetlinewidth{0.803000pt}%
\definecolor{currentstroke}{rgb}{0.850000,0.850000,0.850000}%
\pgfsetstrokecolor{currentstroke}%
\pgfsetdash{}{0pt}%
\pgfpathmoveto{\pgfqpoint{1.514520in}{0.417642in}}%
\pgfpathlineto{\pgfqpoint{1.514520in}{1.789039in}}%
\pgfusepath{stroke}%
\end{pgfscope}%
\begin{pgfscope}%
\pgfsetbuttcap%
\pgfsetroundjoin%
\definecolor{currentfill}{rgb}{0.000000,0.000000,0.000000}%
\pgfsetfillcolor{currentfill}%
\pgfsetlinewidth{0.602250pt}%
\definecolor{currentstroke}{rgb}{0.000000,0.000000,0.000000}%
\pgfsetstrokecolor{currentstroke}%
\pgfsetdash{}{0pt}%
\pgfsys@defobject{currentmarker}{\pgfqpoint{0.000000in}{-0.027778in}}{\pgfqpoint{0.000000in}{0.000000in}}{%
\pgfpathmoveto{\pgfqpoint{0.000000in}{0.000000in}}%
\pgfpathlineto{\pgfqpoint{0.000000in}{-0.027778in}}%
\pgfusepath{stroke,fill}%
}%
\begin{pgfscope}%
\pgfsys@transformshift{1.514520in}{0.417642in}%
\pgfsys@useobject{currentmarker}{}%
\end{pgfscope}%
\end{pgfscope}%
\begin{pgfscope}%
\pgfpathrectangle{\pgfqpoint{0.589510in}{0.417642in}}{\pgfqpoint{1.809765in}{1.371397in}}%
\pgfusepath{clip}%
\pgfsetrectcap%
\pgfsetroundjoin%
\pgfsetlinewidth{0.803000pt}%
\definecolor{currentstroke}{rgb}{0.850000,0.850000,0.850000}%
\pgfsetstrokecolor{currentstroke}%
\pgfsetdash{}{0pt}%
\pgfpathmoveto{\pgfqpoint{1.541008in}{0.417642in}}%
\pgfpathlineto{\pgfqpoint{1.541008in}{1.789039in}}%
\pgfusepath{stroke}%
\end{pgfscope}%
\begin{pgfscope}%
\pgfsetbuttcap%
\pgfsetroundjoin%
\definecolor{currentfill}{rgb}{0.000000,0.000000,0.000000}%
\pgfsetfillcolor{currentfill}%
\pgfsetlinewidth{0.602250pt}%
\definecolor{currentstroke}{rgb}{0.000000,0.000000,0.000000}%
\pgfsetstrokecolor{currentstroke}%
\pgfsetdash{}{0pt}%
\pgfsys@defobject{currentmarker}{\pgfqpoint{0.000000in}{-0.027778in}}{\pgfqpoint{0.000000in}{0.000000in}}{%
\pgfpathmoveto{\pgfqpoint{0.000000in}{0.000000in}}%
\pgfpathlineto{\pgfqpoint{0.000000in}{-0.027778in}}%
\pgfusepath{stroke,fill}%
}%
\begin{pgfscope}%
\pgfsys@transformshift{1.541008in}{0.417642in}%
\pgfsys@useobject{currentmarker}{}%
\end{pgfscope}%
\end{pgfscope}%
\begin{pgfscope}%
\pgfpathrectangle{\pgfqpoint{0.589510in}{0.417642in}}{\pgfqpoint{1.809765in}{1.371397in}}%
\pgfusepath{clip}%
\pgfsetrectcap%
\pgfsetroundjoin%
\pgfsetlinewidth{0.803000pt}%
\definecolor{currentstroke}{rgb}{0.850000,0.850000,0.850000}%
\pgfsetstrokecolor{currentstroke}%
\pgfsetdash{}{0pt}%
\pgfpathmoveto{\pgfqpoint{1.564372in}{0.417642in}}%
\pgfpathlineto{\pgfqpoint{1.564372in}{1.789039in}}%
\pgfusepath{stroke}%
\end{pgfscope}%
\begin{pgfscope}%
\pgfsetbuttcap%
\pgfsetroundjoin%
\definecolor{currentfill}{rgb}{0.000000,0.000000,0.000000}%
\pgfsetfillcolor{currentfill}%
\pgfsetlinewidth{0.602250pt}%
\definecolor{currentstroke}{rgb}{0.000000,0.000000,0.000000}%
\pgfsetstrokecolor{currentstroke}%
\pgfsetdash{}{0pt}%
\pgfsys@defobject{currentmarker}{\pgfqpoint{0.000000in}{-0.027778in}}{\pgfqpoint{0.000000in}{0.000000in}}{%
\pgfpathmoveto{\pgfqpoint{0.000000in}{0.000000in}}%
\pgfpathlineto{\pgfqpoint{0.000000in}{-0.027778in}}%
\pgfusepath{stroke,fill}%
}%
\begin{pgfscope}%
\pgfsys@transformshift{1.564372in}{0.417642in}%
\pgfsys@useobject{currentmarker}{}%
\end{pgfscope}%
\end{pgfscope}%
\begin{pgfscope}%
\pgfpathrectangle{\pgfqpoint{0.589510in}{0.417642in}}{\pgfqpoint{1.809765in}{1.371397in}}%
\pgfusepath{clip}%
\pgfsetrectcap%
\pgfsetroundjoin%
\pgfsetlinewidth{0.803000pt}%
\definecolor{currentstroke}{rgb}{0.850000,0.850000,0.850000}%
\pgfsetstrokecolor{currentstroke}%
\pgfsetdash{}{0pt}%
\pgfpathmoveto{\pgfqpoint{1.722767in}{0.417642in}}%
\pgfpathlineto{\pgfqpoint{1.722767in}{1.789039in}}%
\pgfusepath{stroke}%
\end{pgfscope}%
\begin{pgfscope}%
\pgfsetbuttcap%
\pgfsetroundjoin%
\definecolor{currentfill}{rgb}{0.000000,0.000000,0.000000}%
\pgfsetfillcolor{currentfill}%
\pgfsetlinewidth{0.602250pt}%
\definecolor{currentstroke}{rgb}{0.000000,0.000000,0.000000}%
\pgfsetstrokecolor{currentstroke}%
\pgfsetdash{}{0pt}%
\pgfsys@defobject{currentmarker}{\pgfqpoint{0.000000in}{-0.027778in}}{\pgfqpoint{0.000000in}{0.000000in}}{%
\pgfpathmoveto{\pgfqpoint{0.000000in}{0.000000in}}%
\pgfpathlineto{\pgfqpoint{0.000000in}{-0.027778in}}%
\pgfusepath{stroke,fill}%
}%
\begin{pgfscope}%
\pgfsys@transformshift{1.722767in}{0.417642in}%
\pgfsys@useobject{currentmarker}{}%
\end{pgfscope}%
\end{pgfscope}%
\begin{pgfscope}%
\pgfpathrectangle{\pgfqpoint{0.589510in}{0.417642in}}{\pgfqpoint{1.809765in}{1.371397in}}%
\pgfusepath{clip}%
\pgfsetrectcap%
\pgfsetroundjoin%
\pgfsetlinewidth{0.803000pt}%
\definecolor{currentstroke}{rgb}{0.850000,0.850000,0.850000}%
\pgfsetstrokecolor{currentstroke}%
\pgfsetdash{}{0pt}%
\pgfpathmoveto{\pgfqpoint{1.803197in}{0.417642in}}%
\pgfpathlineto{\pgfqpoint{1.803197in}{1.789039in}}%
\pgfusepath{stroke}%
\end{pgfscope}%
\begin{pgfscope}%
\pgfsetbuttcap%
\pgfsetroundjoin%
\definecolor{currentfill}{rgb}{0.000000,0.000000,0.000000}%
\pgfsetfillcolor{currentfill}%
\pgfsetlinewidth{0.602250pt}%
\definecolor{currentstroke}{rgb}{0.000000,0.000000,0.000000}%
\pgfsetstrokecolor{currentstroke}%
\pgfsetdash{}{0pt}%
\pgfsys@defobject{currentmarker}{\pgfqpoint{0.000000in}{-0.027778in}}{\pgfqpoint{0.000000in}{0.000000in}}{%
\pgfpathmoveto{\pgfqpoint{0.000000in}{0.000000in}}%
\pgfpathlineto{\pgfqpoint{0.000000in}{-0.027778in}}%
\pgfusepath{stroke,fill}%
}%
\begin{pgfscope}%
\pgfsys@transformshift{1.803197in}{0.417642in}%
\pgfsys@useobject{currentmarker}{}%
\end{pgfscope}%
\end{pgfscope}%
\begin{pgfscope}%
\pgfpathrectangle{\pgfqpoint{0.589510in}{0.417642in}}{\pgfqpoint{1.809765in}{1.371397in}}%
\pgfusepath{clip}%
\pgfsetrectcap%
\pgfsetroundjoin%
\pgfsetlinewidth{0.803000pt}%
\definecolor{currentstroke}{rgb}{0.850000,0.850000,0.850000}%
\pgfsetstrokecolor{currentstroke}%
\pgfsetdash{}{0pt}%
\pgfpathmoveto{\pgfqpoint{1.860263in}{0.417642in}}%
\pgfpathlineto{\pgfqpoint{1.860263in}{1.789039in}}%
\pgfusepath{stroke}%
\end{pgfscope}%
\begin{pgfscope}%
\pgfsetbuttcap%
\pgfsetroundjoin%
\definecolor{currentfill}{rgb}{0.000000,0.000000,0.000000}%
\pgfsetfillcolor{currentfill}%
\pgfsetlinewidth{0.602250pt}%
\definecolor{currentstroke}{rgb}{0.000000,0.000000,0.000000}%
\pgfsetstrokecolor{currentstroke}%
\pgfsetdash{}{0pt}%
\pgfsys@defobject{currentmarker}{\pgfqpoint{0.000000in}{-0.027778in}}{\pgfqpoint{0.000000in}{0.000000in}}{%
\pgfpathmoveto{\pgfqpoint{0.000000in}{0.000000in}}%
\pgfpathlineto{\pgfqpoint{0.000000in}{-0.027778in}}%
\pgfusepath{stroke,fill}%
}%
\begin{pgfscope}%
\pgfsys@transformshift{1.860263in}{0.417642in}%
\pgfsys@useobject{currentmarker}{}%
\end{pgfscope}%
\end{pgfscope}%
\begin{pgfscope}%
\pgfpathrectangle{\pgfqpoint{0.589510in}{0.417642in}}{\pgfqpoint{1.809765in}{1.371397in}}%
\pgfusepath{clip}%
\pgfsetrectcap%
\pgfsetroundjoin%
\pgfsetlinewidth{0.803000pt}%
\definecolor{currentstroke}{rgb}{0.850000,0.850000,0.850000}%
\pgfsetstrokecolor{currentstroke}%
\pgfsetdash{}{0pt}%
\pgfpathmoveto{\pgfqpoint{1.904526in}{0.417642in}}%
\pgfpathlineto{\pgfqpoint{1.904526in}{1.789039in}}%
\pgfusepath{stroke}%
\end{pgfscope}%
\begin{pgfscope}%
\pgfsetbuttcap%
\pgfsetroundjoin%
\definecolor{currentfill}{rgb}{0.000000,0.000000,0.000000}%
\pgfsetfillcolor{currentfill}%
\pgfsetlinewidth{0.602250pt}%
\definecolor{currentstroke}{rgb}{0.000000,0.000000,0.000000}%
\pgfsetstrokecolor{currentstroke}%
\pgfsetdash{}{0pt}%
\pgfsys@defobject{currentmarker}{\pgfqpoint{0.000000in}{-0.027778in}}{\pgfqpoint{0.000000in}{0.000000in}}{%
\pgfpathmoveto{\pgfqpoint{0.000000in}{0.000000in}}%
\pgfpathlineto{\pgfqpoint{0.000000in}{-0.027778in}}%
\pgfusepath{stroke,fill}%
}%
\begin{pgfscope}%
\pgfsys@transformshift{1.904526in}{0.417642in}%
\pgfsys@useobject{currentmarker}{}%
\end{pgfscope}%
\end{pgfscope}%
\begin{pgfscope}%
\pgfpathrectangle{\pgfqpoint{0.589510in}{0.417642in}}{\pgfqpoint{1.809765in}{1.371397in}}%
\pgfusepath{clip}%
\pgfsetrectcap%
\pgfsetroundjoin%
\pgfsetlinewidth{0.803000pt}%
\definecolor{currentstroke}{rgb}{0.850000,0.850000,0.850000}%
\pgfsetstrokecolor{currentstroke}%
\pgfsetdash{}{0pt}%
\pgfpathmoveto{\pgfqpoint{1.940693in}{0.417642in}}%
\pgfpathlineto{\pgfqpoint{1.940693in}{1.789039in}}%
\pgfusepath{stroke}%
\end{pgfscope}%
\begin{pgfscope}%
\pgfsetbuttcap%
\pgfsetroundjoin%
\definecolor{currentfill}{rgb}{0.000000,0.000000,0.000000}%
\pgfsetfillcolor{currentfill}%
\pgfsetlinewidth{0.602250pt}%
\definecolor{currentstroke}{rgb}{0.000000,0.000000,0.000000}%
\pgfsetstrokecolor{currentstroke}%
\pgfsetdash{}{0pt}%
\pgfsys@defobject{currentmarker}{\pgfqpoint{0.000000in}{-0.027778in}}{\pgfqpoint{0.000000in}{0.000000in}}{%
\pgfpathmoveto{\pgfqpoint{0.000000in}{0.000000in}}%
\pgfpathlineto{\pgfqpoint{0.000000in}{-0.027778in}}%
\pgfusepath{stroke,fill}%
}%
\begin{pgfscope}%
\pgfsys@transformshift{1.940693in}{0.417642in}%
\pgfsys@useobject{currentmarker}{}%
\end{pgfscope}%
\end{pgfscope}%
\begin{pgfscope}%
\pgfpathrectangle{\pgfqpoint{0.589510in}{0.417642in}}{\pgfqpoint{1.809765in}{1.371397in}}%
\pgfusepath{clip}%
\pgfsetrectcap%
\pgfsetroundjoin%
\pgfsetlinewidth{0.803000pt}%
\definecolor{currentstroke}{rgb}{0.850000,0.850000,0.850000}%
\pgfsetstrokecolor{currentstroke}%
\pgfsetdash{}{0pt}%
\pgfpathmoveto{\pgfqpoint{1.971270in}{0.417642in}}%
\pgfpathlineto{\pgfqpoint{1.971270in}{1.789039in}}%
\pgfusepath{stroke}%
\end{pgfscope}%
\begin{pgfscope}%
\pgfsetbuttcap%
\pgfsetroundjoin%
\definecolor{currentfill}{rgb}{0.000000,0.000000,0.000000}%
\pgfsetfillcolor{currentfill}%
\pgfsetlinewidth{0.602250pt}%
\definecolor{currentstroke}{rgb}{0.000000,0.000000,0.000000}%
\pgfsetstrokecolor{currentstroke}%
\pgfsetdash{}{0pt}%
\pgfsys@defobject{currentmarker}{\pgfqpoint{0.000000in}{-0.027778in}}{\pgfqpoint{0.000000in}{0.000000in}}{%
\pgfpathmoveto{\pgfqpoint{0.000000in}{0.000000in}}%
\pgfpathlineto{\pgfqpoint{0.000000in}{-0.027778in}}%
\pgfusepath{stroke,fill}%
}%
\begin{pgfscope}%
\pgfsys@transformshift{1.971270in}{0.417642in}%
\pgfsys@useobject{currentmarker}{}%
\end{pgfscope}%
\end{pgfscope}%
\begin{pgfscope}%
\pgfpathrectangle{\pgfqpoint{0.589510in}{0.417642in}}{\pgfqpoint{1.809765in}{1.371397in}}%
\pgfusepath{clip}%
\pgfsetrectcap%
\pgfsetroundjoin%
\pgfsetlinewidth{0.803000pt}%
\definecolor{currentstroke}{rgb}{0.850000,0.850000,0.850000}%
\pgfsetstrokecolor{currentstroke}%
\pgfsetdash{}{0pt}%
\pgfpathmoveto{\pgfqpoint{1.997758in}{0.417642in}}%
\pgfpathlineto{\pgfqpoint{1.997758in}{1.789039in}}%
\pgfusepath{stroke}%
\end{pgfscope}%
\begin{pgfscope}%
\pgfsetbuttcap%
\pgfsetroundjoin%
\definecolor{currentfill}{rgb}{0.000000,0.000000,0.000000}%
\pgfsetfillcolor{currentfill}%
\pgfsetlinewidth{0.602250pt}%
\definecolor{currentstroke}{rgb}{0.000000,0.000000,0.000000}%
\pgfsetstrokecolor{currentstroke}%
\pgfsetdash{}{0pt}%
\pgfsys@defobject{currentmarker}{\pgfqpoint{0.000000in}{-0.027778in}}{\pgfqpoint{0.000000in}{0.000000in}}{%
\pgfpathmoveto{\pgfqpoint{0.000000in}{0.000000in}}%
\pgfpathlineto{\pgfqpoint{0.000000in}{-0.027778in}}%
\pgfusepath{stroke,fill}%
}%
\begin{pgfscope}%
\pgfsys@transformshift{1.997758in}{0.417642in}%
\pgfsys@useobject{currentmarker}{}%
\end{pgfscope}%
\end{pgfscope}%
\begin{pgfscope}%
\pgfpathrectangle{\pgfqpoint{0.589510in}{0.417642in}}{\pgfqpoint{1.809765in}{1.371397in}}%
\pgfusepath{clip}%
\pgfsetrectcap%
\pgfsetroundjoin%
\pgfsetlinewidth{0.803000pt}%
\definecolor{currentstroke}{rgb}{0.850000,0.850000,0.850000}%
\pgfsetstrokecolor{currentstroke}%
\pgfsetdash{}{0pt}%
\pgfpathmoveto{\pgfqpoint{2.021122in}{0.417642in}}%
\pgfpathlineto{\pgfqpoint{2.021122in}{1.789039in}}%
\pgfusepath{stroke}%
\end{pgfscope}%
\begin{pgfscope}%
\pgfsetbuttcap%
\pgfsetroundjoin%
\definecolor{currentfill}{rgb}{0.000000,0.000000,0.000000}%
\pgfsetfillcolor{currentfill}%
\pgfsetlinewidth{0.602250pt}%
\definecolor{currentstroke}{rgb}{0.000000,0.000000,0.000000}%
\pgfsetstrokecolor{currentstroke}%
\pgfsetdash{}{0pt}%
\pgfsys@defobject{currentmarker}{\pgfqpoint{0.000000in}{-0.027778in}}{\pgfqpoint{0.000000in}{0.000000in}}{%
\pgfpathmoveto{\pgfqpoint{0.000000in}{0.000000in}}%
\pgfpathlineto{\pgfqpoint{0.000000in}{-0.027778in}}%
\pgfusepath{stroke,fill}%
}%
\begin{pgfscope}%
\pgfsys@transformshift{2.021122in}{0.417642in}%
\pgfsys@useobject{currentmarker}{}%
\end{pgfscope}%
\end{pgfscope}%
\begin{pgfscope}%
\pgfpathrectangle{\pgfqpoint{0.589510in}{0.417642in}}{\pgfqpoint{1.809765in}{1.371397in}}%
\pgfusepath{clip}%
\pgfsetrectcap%
\pgfsetroundjoin%
\pgfsetlinewidth{0.803000pt}%
\definecolor{currentstroke}{rgb}{0.850000,0.850000,0.850000}%
\pgfsetstrokecolor{currentstroke}%
\pgfsetdash{}{0pt}%
\pgfpathmoveto{\pgfqpoint{2.179517in}{0.417642in}}%
\pgfpathlineto{\pgfqpoint{2.179517in}{1.789039in}}%
\pgfusepath{stroke}%
\end{pgfscope}%
\begin{pgfscope}%
\pgfsetbuttcap%
\pgfsetroundjoin%
\definecolor{currentfill}{rgb}{0.000000,0.000000,0.000000}%
\pgfsetfillcolor{currentfill}%
\pgfsetlinewidth{0.602250pt}%
\definecolor{currentstroke}{rgb}{0.000000,0.000000,0.000000}%
\pgfsetstrokecolor{currentstroke}%
\pgfsetdash{}{0pt}%
\pgfsys@defobject{currentmarker}{\pgfqpoint{0.000000in}{-0.027778in}}{\pgfqpoint{0.000000in}{0.000000in}}{%
\pgfpathmoveto{\pgfqpoint{0.000000in}{0.000000in}}%
\pgfpathlineto{\pgfqpoint{0.000000in}{-0.027778in}}%
\pgfusepath{stroke,fill}%
}%
\begin{pgfscope}%
\pgfsys@transformshift{2.179517in}{0.417642in}%
\pgfsys@useobject{currentmarker}{}%
\end{pgfscope}%
\end{pgfscope}%
\begin{pgfscope}%
\pgfpathrectangle{\pgfqpoint{0.589510in}{0.417642in}}{\pgfqpoint{1.809765in}{1.371397in}}%
\pgfusepath{clip}%
\pgfsetrectcap%
\pgfsetroundjoin%
\pgfsetlinewidth{0.803000pt}%
\definecolor{currentstroke}{rgb}{0.850000,0.850000,0.850000}%
\pgfsetstrokecolor{currentstroke}%
\pgfsetdash{}{0pt}%
\pgfpathmoveto{\pgfqpoint{2.259947in}{0.417642in}}%
\pgfpathlineto{\pgfqpoint{2.259947in}{1.789039in}}%
\pgfusepath{stroke}%
\end{pgfscope}%
\begin{pgfscope}%
\pgfsetbuttcap%
\pgfsetroundjoin%
\definecolor{currentfill}{rgb}{0.000000,0.000000,0.000000}%
\pgfsetfillcolor{currentfill}%
\pgfsetlinewidth{0.602250pt}%
\definecolor{currentstroke}{rgb}{0.000000,0.000000,0.000000}%
\pgfsetstrokecolor{currentstroke}%
\pgfsetdash{}{0pt}%
\pgfsys@defobject{currentmarker}{\pgfqpoint{0.000000in}{-0.027778in}}{\pgfqpoint{0.000000in}{0.000000in}}{%
\pgfpathmoveto{\pgfqpoint{0.000000in}{0.000000in}}%
\pgfpathlineto{\pgfqpoint{0.000000in}{-0.027778in}}%
\pgfusepath{stroke,fill}%
}%
\begin{pgfscope}%
\pgfsys@transformshift{2.259947in}{0.417642in}%
\pgfsys@useobject{currentmarker}{}%
\end{pgfscope}%
\end{pgfscope}%
\begin{pgfscope}%
\pgfpathrectangle{\pgfqpoint{0.589510in}{0.417642in}}{\pgfqpoint{1.809765in}{1.371397in}}%
\pgfusepath{clip}%
\pgfsetrectcap%
\pgfsetroundjoin%
\pgfsetlinewidth{0.803000pt}%
\definecolor{currentstroke}{rgb}{0.850000,0.850000,0.850000}%
\pgfsetstrokecolor{currentstroke}%
\pgfsetdash{}{0pt}%
\pgfpathmoveto{\pgfqpoint{2.317013in}{0.417642in}}%
\pgfpathlineto{\pgfqpoint{2.317013in}{1.789039in}}%
\pgfusepath{stroke}%
\end{pgfscope}%
\begin{pgfscope}%
\pgfsetbuttcap%
\pgfsetroundjoin%
\definecolor{currentfill}{rgb}{0.000000,0.000000,0.000000}%
\pgfsetfillcolor{currentfill}%
\pgfsetlinewidth{0.602250pt}%
\definecolor{currentstroke}{rgb}{0.000000,0.000000,0.000000}%
\pgfsetstrokecolor{currentstroke}%
\pgfsetdash{}{0pt}%
\pgfsys@defobject{currentmarker}{\pgfqpoint{0.000000in}{-0.027778in}}{\pgfqpoint{0.000000in}{0.000000in}}{%
\pgfpathmoveto{\pgfqpoint{0.000000in}{0.000000in}}%
\pgfpathlineto{\pgfqpoint{0.000000in}{-0.027778in}}%
\pgfusepath{stroke,fill}%
}%
\begin{pgfscope}%
\pgfsys@transformshift{2.317013in}{0.417642in}%
\pgfsys@useobject{currentmarker}{}%
\end{pgfscope}%
\end{pgfscope}%
\begin{pgfscope}%
\pgfpathrectangle{\pgfqpoint{0.589510in}{0.417642in}}{\pgfqpoint{1.809765in}{1.371397in}}%
\pgfusepath{clip}%
\pgfsetrectcap%
\pgfsetroundjoin%
\pgfsetlinewidth{0.803000pt}%
\definecolor{currentstroke}{rgb}{0.850000,0.850000,0.850000}%
\pgfsetstrokecolor{currentstroke}%
\pgfsetdash{}{0pt}%
\pgfpathmoveto{\pgfqpoint{2.361277in}{0.417642in}}%
\pgfpathlineto{\pgfqpoint{2.361277in}{1.789039in}}%
\pgfusepath{stroke}%
\end{pgfscope}%
\begin{pgfscope}%
\pgfsetbuttcap%
\pgfsetroundjoin%
\definecolor{currentfill}{rgb}{0.000000,0.000000,0.000000}%
\pgfsetfillcolor{currentfill}%
\pgfsetlinewidth{0.602250pt}%
\definecolor{currentstroke}{rgb}{0.000000,0.000000,0.000000}%
\pgfsetstrokecolor{currentstroke}%
\pgfsetdash{}{0pt}%
\pgfsys@defobject{currentmarker}{\pgfqpoint{0.000000in}{-0.027778in}}{\pgfqpoint{0.000000in}{0.000000in}}{%
\pgfpathmoveto{\pgfqpoint{0.000000in}{0.000000in}}%
\pgfpathlineto{\pgfqpoint{0.000000in}{-0.027778in}}%
\pgfusepath{stroke,fill}%
}%
\begin{pgfscope}%
\pgfsys@transformshift{2.361277in}{0.417642in}%
\pgfsys@useobject{currentmarker}{}%
\end{pgfscope}%
\end{pgfscope}%
\begin{pgfscope}%
\pgfpathrectangle{\pgfqpoint{0.589510in}{0.417642in}}{\pgfqpoint{1.809765in}{1.371397in}}%
\pgfusepath{clip}%
\pgfsetrectcap%
\pgfsetroundjoin%
\pgfsetlinewidth{0.803000pt}%
\definecolor{currentstroke}{rgb}{0.850000,0.850000,0.850000}%
\pgfsetstrokecolor{currentstroke}%
\pgfsetdash{}{0pt}%
\pgfpathmoveto{\pgfqpoint{2.397443in}{0.417642in}}%
\pgfpathlineto{\pgfqpoint{2.397443in}{1.789039in}}%
\pgfusepath{stroke}%
\end{pgfscope}%
\begin{pgfscope}%
\pgfsetbuttcap%
\pgfsetroundjoin%
\definecolor{currentfill}{rgb}{0.000000,0.000000,0.000000}%
\pgfsetfillcolor{currentfill}%
\pgfsetlinewidth{0.602250pt}%
\definecolor{currentstroke}{rgb}{0.000000,0.000000,0.000000}%
\pgfsetstrokecolor{currentstroke}%
\pgfsetdash{}{0pt}%
\pgfsys@defobject{currentmarker}{\pgfqpoint{0.000000in}{-0.027778in}}{\pgfqpoint{0.000000in}{0.000000in}}{%
\pgfpathmoveto{\pgfqpoint{0.000000in}{0.000000in}}%
\pgfpathlineto{\pgfqpoint{0.000000in}{-0.027778in}}%
\pgfusepath{stroke,fill}%
}%
\begin{pgfscope}%
\pgfsys@transformshift{2.397443in}{0.417642in}%
\pgfsys@useobject{currentmarker}{}%
\end{pgfscope}%
\end{pgfscope}%
\begin{pgfscope}%
\definecolor{textcolor}{rgb}{0.000000,0.000000,0.000000}%
\pgfsetstrokecolor{textcolor}%
\pgfsetfillcolor{textcolor}%
\pgftext[x=1.494392in,y=0.165003in,,top]{\color{textcolor}\rmfamily\fontsize{10.000000}{12.000000}\selectfont \(\displaystyle \tau\) in \unit{\second}}%
\end{pgfscope}%
\begin{pgfscope}%
\pgfpathrectangle{\pgfqpoint{0.589510in}{0.417642in}}{\pgfqpoint{1.809765in}{1.371397in}}%
\pgfusepath{clip}%
\pgfsetrectcap%
\pgfsetroundjoin%
\pgfsetlinewidth{0.803000pt}%
\definecolor{currentstroke}{rgb}{0.450000,0.450000,0.450000}%
\pgfsetstrokecolor{currentstroke}%
\pgfsetdash{}{0pt}%
\pgfpathmoveto{\pgfqpoint{0.589510in}{0.417642in}}%
\pgfpathlineto{\pgfqpoint{2.399275in}{0.417642in}}%
\pgfusepath{stroke}%
\end{pgfscope}%
\begin{pgfscope}%
\pgfsetbuttcap%
\pgfsetroundjoin%
\definecolor{currentfill}{rgb}{0.000000,0.000000,0.000000}%
\pgfsetfillcolor{currentfill}%
\pgfsetlinewidth{0.803000pt}%
\definecolor{currentstroke}{rgb}{0.000000,0.000000,0.000000}%
\pgfsetstrokecolor{currentstroke}%
\pgfsetdash{}{0pt}%
\pgfsys@defobject{currentmarker}{\pgfqpoint{-0.048611in}{0.000000in}}{\pgfqpoint{-0.000000in}{0.000000in}}{%
\pgfpathmoveto{\pgfqpoint{-0.000000in}{0.000000in}}%
\pgfpathlineto{\pgfqpoint{-0.048611in}{0.000000in}}%
\pgfusepath{stroke,fill}%
}%
\begin{pgfscope}%
\pgfsys@transformshift{0.589510in}{0.417642in}%
\pgfsys@useobject{currentmarker}{}%
\end{pgfscope}%
\end{pgfscope}%
\begin{pgfscope}%
\definecolor{textcolor}{rgb}{0.000000,0.000000,0.000000}%
\pgfsetstrokecolor{textcolor}%
\pgfsetfillcolor{textcolor}%
\pgftext[x=0.236114in, y=0.378489in, left, base]{\color{textcolor}\rmfamily\fontsize{8.000000}{9.600000}\selectfont \(\displaystyle {10^{-2}}\)}%
\end{pgfscope}%
\begin{pgfscope}%
\pgfpathrectangle{\pgfqpoint{0.589510in}{0.417642in}}{\pgfqpoint{1.809765in}{1.371397in}}%
\pgfusepath{clip}%
\pgfsetrectcap%
\pgfsetroundjoin%
\pgfsetlinewidth{0.803000pt}%
\definecolor{currentstroke}{rgb}{0.450000,0.450000,0.450000}%
\pgfsetstrokecolor{currentstroke}%
\pgfsetdash{}{0pt}%
\pgfpathmoveto{\pgfqpoint{0.589510in}{0.827077in}}%
\pgfpathlineto{\pgfqpoint{2.399275in}{0.827077in}}%
\pgfusepath{stroke}%
\end{pgfscope}%
\begin{pgfscope}%
\pgfsetbuttcap%
\pgfsetroundjoin%
\definecolor{currentfill}{rgb}{0.000000,0.000000,0.000000}%
\pgfsetfillcolor{currentfill}%
\pgfsetlinewidth{0.803000pt}%
\definecolor{currentstroke}{rgb}{0.000000,0.000000,0.000000}%
\pgfsetstrokecolor{currentstroke}%
\pgfsetdash{}{0pt}%
\pgfsys@defobject{currentmarker}{\pgfqpoint{-0.048611in}{0.000000in}}{\pgfqpoint{-0.000000in}{0.000000in}}{%
\pgfpathmoveto{\pgfqpoint{-0.000000in}{0.000000in}}%
\pgfpathlineto{\pgfqpoint{-0.048611in}{0.000000in}}%
\pgfusepath{stroke,fill}%
}%
\begin{pgfscope}%
\pgfsys@transformshift{0.589510in}{0.827077in}%
\pgfsys@useobject{currentmarker}{}%
\end{pgfscope}%
\end{pgfscope}%
\begin{pgfscope}%
\definecolor{textcolor}{rgb}{0.000000,0.000000,0.000000}%
\pgfsetstrokecolor{textcolor}%
\pgfsetfillcolor{textcolor}%
\pgftext[x=0.316361in, y=0.787924in, left, base]{\color{textcolor}\rmfamily\fontsize{8.000000}{9.600000}\selectfont \(\displaystyle {10^{0}}\)}%
\end{pgfscope}%
\begin{pgfscope}%
\pgfpathrectangle{\pgfqpoint{0.589510in}{0.417642in}}{\pgfqpoint{1.809765in}{1.371397in}}%
\pgfusepath{clip}%
\pgfsetrectcap%
\pgfsetroundjoin%
\pgfsetlinewidth{0.803000pt}%
\definecolor{currentstroke}{rgb}{0.450000,0.450000,0.450000}%
\pgfsetstrokecolor{currentstroke}%
\pgfsetdash{}{0pt}%
\pgfpathmoveto{\pgfqpoint{0.589510in}{1.236512in}}%
\pgfpathlineto{\pgfqpoint{2.399275in}{1.236512in}}%
\pgfusepath{stroke}%
\end{pgfscope}%
\begin{pgfscope}%
\pgfsetbuttcap%
\pgfsetroundjoin%
\definecolor{currentfill}{rgb}{0.000000,0.000000,0.000000}%
\pgfsetfillcolor{currentfill}%
\pgfsetlinewidth{0.803000pt}%
\definecolor{currentstroke}{rgb}{0.000000,0.000000,0.000000}%
\pgfsetstrokecolor{currentstroke}%
\pgfsetdash{}{0pt}%
\pgfsys@defobject{currentmarker}{\pgfqpoint{-0.048611in}{0.000000in}}{\pgfqpoint{-0.000000in}{0.000000in}}{%
\pgfpathmoveto{\pgfqpoint{-0.000000in}{0.000000in}}%
\pgfpathlineto{\pgfqpoint{-0.048611in}{0.000000in}}%
\pgfusepath{stroke,fill}%
}%
\begin{pgfscope}%
\pgfsys@transformshift{0.589510in}{1.236512in}%
\pgfsys@useobject{currentmarker}{}%
\end{pgfscope}%
\end{pgfscope}%
\begin{pgfscope}%
\definecolor{textcolor}{rgb}{0.000000,0.000000,0.000000}%
\pgfsetstrokecolor{textcolor}%
\pgfsetfillcolor{textcolor}%
\pgftext[x=0.316361in, y=1.197359in, left, base]{\color{textcolor}\rmfamily\fontsize{8.000000}{9.600000}\selectfont \(\displaystyle {10^{2}}\)}%
\end{pgfscope}%
\begin{pgfscope}%
\pgfpathrectangle{\pgfqpoint{0.589510in}{0.417642in}}{\pgfqpoint{1.809765in}{1.371397in}}%
\pgfusepath{clip}%
\pgfsetrectcap%
\pgfsetroundjoin%
\pgfsetlinewidth{0.803000pt}%
\definecolor{currentstroke}{rgb}{0.450000,0.450000,0.450000}%
\pgfsetstrokecolor{currentstroke}%
\pgfsetdash{}{0pt}%
\pgfpathmoveto{\pgfqpoint{0.589510in}{1.645947in}}%
\pgfpathlineto{\pgfqpoint{2.399275in}{1.645947in}}%
\pgfusepath{stroke}%
\end{pgfscope}%
\begin{pgfscope}%
\pgfsetbuttcap%
\pgfsetroundjoin%
\definecolor{currentfill}{rgb}{0.000000,0.000000,0.000000}%
\pgfsetfillcolor{currentfill}%
\pgfsetlinewidth{0.803000pt}%
\definecolor{currentstroke}{rgb}{0.000000,0.000000,0.000000}%
\pgfsetstrokecolor{currentstroke}%
\pgfsetdash{}{0pt}%
\pgfsys@defobject{currentmarker}{\pgfqpoint{-0.048611in}{0.000000in}}{\pgfqpoint{-0.000000in}{0.000000in}}{%
\pgfpathmoveto{\pgfqpoint{-0.000000in}{0.000000in}}%
\pgfpathlineto{\pgfqpoint{-0.048611in}{0.000000in}}%
\pgfusepath{stroke,fill}%
}%
\begin{pgfscope}%
\pgfsys@transformshift{0.589510in}{1.645947in}%
\pgfsys@useobject{currentmarker}{}%
\end{pgfscope}%
\end{pgfscope}%
\begin{pgfscope}%
\definecolor{textcolor}{rgb}{0.000000,0.000000,0.000000}%
\pgfsetstrokecolor{textcolor}%
\pgfsetfillcolor{textcolor}%
\pgftext[x=0.316361in, y=1.606795in, left, base]{\color{textcolor}\rmfamily\fontsize{8.000000}{9.600000}\selectfont \(\displaystyle {10^{4}}\)}%
\end{pgfscope}%
\begin{pgfscope}%
\definecolor{textcolor}{rgb}{0.000000,0.000000,0.000000}%
\pgfsetstrokecolor{textcolor}%
\pgfsetfillcolor{textcolor}%
\pgftext[x=0.180559in,y=1.103340in,,bottom,rotate=90.000000]{\color{textcolor}\rmfamily\fontsize{10.000000}{12.000000}\selectfont ADEV \(\displaystyle \sigma_A(\tau)\)}%
\end{pgfscope}%
\begin{pgfscope}%
\pgfpathrectangle{\pgfqpoint{0.589510in}{0.417642in}}{\pgfqpoint{1.809765in}{1.371397in}}%
\pgfusepath{clip}%
\pgfsetbuttcap%
\pgfsetroundjoin%
\pgfsetlinewidth{1.505625pt}%
\definecolor{currentstroke}{rgb}{0.800000,0.470588,0.737255}%
\pgfsetstrokecolor{currentstroke}%
\pgfsetdash{{5.550000pt}{2.400000pt}}{0.000000pt}%
\pgfpathmoveto{\pgfqpoint{0.671772in}{0.827077in}}%
\pgfpathlineto{\pgfqpoint{0.809267in}{0.888703in}}%
\pgfpathlineto{\pgfqpoint{0.946763in}{0.950329in}}%
\pgfpathlineto{\pgfqpoint{1.128522in}{1.031795in}}%
\pgfpathlineto{\pgfqpoint{1.266017in}{1.093421in}}%
\pgfpathlineto{\pgfqpoint{1.403513in}{1.155047in}}%
\pgfpathlineto{\pgfqpoint{1.585272in}{1.236512in}}%
\pgfpathlineto{\pgfqpoint{1.722767in}{1.298138in}}%
\pgfpathlineto{\pgfqpoint{1.860263in}{1.359764in}}%
\pgfpathlineto{\pgfqpoint{2.042022in}{1.441230in}}%
\pgfpathlineto{\pgfqpoint{2.179517in}{1.502856in}}%
\pgfpathlineto{\pgfqpoint{2.317013in}{1.564482in}}%
\pgfusepath{stroke}%
\end{pgfscope}%
\begin{pgfscope}%
\pgfpathrectangle{\pgfqpoint{0.589510in}{0.417642in}}{\pgfqpoint{1.809765in}{1.371397in}}%
\pgfusepath{clip}%
\pgfsetbuttcap%
\pgfsetroundjoin%
\definecolor{currentfill}{rgb}{0.800000,0.470588,0.737255}%
\pgfsetfillcolor{currentfill}%
\pgfsetlinewidth{1.003750pt}%
\definecolor{currentstroke}{rgb}{0.800000,0.470588,0.737255}%
\pgfsetstrokecolor{currentstroke}%
\pgfsetdash{}{0pt}%
\pgfsys@defobject{currentmarker}{\pgfqpoint{-0.020833in}{-0.020833in}}{\pgfqpoint{0.020833in}{0.020833in}}{%
\pgfpathmoveto{\pgfqpoint{0.000000in}{-0.020833in}}%
\pgfpathcurveto{\pgfqpoint{0.005525in}{-0.020833in}}{\pgfqpoint{0.010825in}{-0.018638in}}{\pgfqpoint{0.014731in}{-0.014731in}}%
\pgfpathcurveto{\pgfqpoint{0.018638in}{-0.010825in}}{\pgfqpoint{0.020833in}{-0.005525in}}{\pgfqpoint{0.020833in}{0.000000in}}%
\pgfpathcurveto{\pgfqpoint{0.020833in}{0.005525in}}{\pgfqpoint{0.018638in}{0.010825in}}{\pgfqpoint{0.014731in}{0.014731in}}%
\pgfpathcurveto{\pgfqpoint{0.010825in}{0.018638in}}{\pgfqpoint{0.005525in}{0.020833in}}{\pgfqpoint{0.000000in}{0.020833in}}%
\pgfpathcurveto{\pgfqpoint{-0.005525in}{0.020833in}}{\pgfqpoint{-0.010825in}{0.018638in}}{\pgfqpoint{-0.014731in}{0.014731in}}%
\pgfpathcurveto{\pgfqpoint{-0.018638in}{0.010825in}}{\pgfqpoint{-0.020833in}{0.005525in}}{\pgfqpoint{-0.020833in}{0.000000in}}%
\pgfpathcurveto{\pgfqpoint{-0.020833in}{-0.005525in}}{\pgfqpoint{-0.018638in}{-0.010825in}}{\pgfqpoint{-0.014731in}{-0.014731in}}%
\pgfpathcurveto{\pgfqpoint{-0.010825in}{-0.018638in}}{\pgfqpoint{-0.005525in}{-0.020833in}}{\pgfqpoint{0.000000in}{-0.020833in}}%
\pgfpathlineto{\pgfqpoint{0.000000in}{-0.020833in}}%
\pgfpathclose%
\pgfusepath{stroke,fill}%
}%
\begin{pgfscope}%
\pgfsys@transformshift{0.671772in}{0.827077in}%
\pgfsys@useobject{currentmarker}{}%
\end{pgfscope}%
\begin{pgfscope}%
\pgfsys@transformshift{0.809267in}{0.888703in}%
\pgfsys@useobject{currentmarker}{}%
\end{pgfscope}%
\begin{pgfscope}%
\pgfsys@transformshift{0.946763in}{0.950329in}%
\pgfsys@useobject{currentmarker}{}%
\end{pgfscope}%
\begin{pgfscope}%
\pgfsys@transformshift{1.128522in}{1.031795in}%
\pgfsys@useobject{currentmarker}{}%
\end{pgfscope}%
\begin{pgfscope}%
\pgfsys@transformshift{1.266017in}{1.093421in}%
\pgfsys@useobject{currentmarker}{}%
\end{pgfscope}%
\begin{pgfscope}%
\pgfsys@transformshift{1.403513in}{1.155047in}%
\pgfsys@useobject{currentmarker}{}%
\end{pgfscope}%
\begin{pgfscope}%
\pgfsys@transformshift{1.585272in}{1.236512in}%
\pgfsys@useobject{currentmarker}{}%
\end{pgfscope}%
\begin{pgfscope}%
\pgfsys@transformshift{1.722767in}{1.298138in}%
\pgfsys@useobject{currentmarker}{}%
\end{pgfscope}%
\begin{pgfscope}%
\pgfsys@transformshift{1.860263in}{1.359764in}%
\pgfsys@useobject{currentmarker}{}%
\end{pgfscope}%
\begin{pgfscope}%
\pgfsys@transformshift{2.042022in}{1.441230in}%
\pgfsys@useobject{currentmarker}{}%
\end{pgfscope}%
\begin{pgfscope}%
\pgfsys@transformshift{2.179517in}{1.502856in}%
\pgfsys@useobject{currentmarker}{}%
\end{pgfscope}%
\begin{pgfscope}%
\pgfsys@transformshift{2.317013in}{1.564482in}%
\pgfsys@useobject{currentmarker}{}%
\end{pgfscope}%
\end{pgfscope}%
\begin{pgfscope}%
\pgfsetrectcap%
\pgfsetmiterjoin%
\pgfsetlinewidth{0.803000pt}%
\definecolor{currentstroke}{rgb}{0.000000,0.000000,0.000000}%
\pgfsetstrokecolor{currentstroke}%
\pgfsetdash{}{0pt}%
\pgfpathmoveto{\pgfqpoint{0.589510in}{0.417642in}}%
\pgfpathlineto{\pgfqpoint{0.589510in}{1.789039in}}%
\pgfusepath{stroke}%
\end{pgfscope}%
\begin{pgfscope}%
\pgfsetrectcap%
\pgfsetmiterjoin%
\pgfsetlinewidth{0.803000pt}%
\definecolor{currentstroke}{rgb}{0.000000,0.000000,0.000000}%
\pgfsetstrokecolor{currentstroke}%
\pgfsetdash{}{0pt}%
\pgfpathmoveto{\pgfqpoint{2.399275in}{0.417642in}}%
\pgfpathlineto{\pgfqpoint{2.399275in}{1.789039in}}%
\pgfusepath{stroke}%
\end{pgfscope}%
\begin{pgfscope}%
\pgfsetrectcap%
\pgfsetmiterjoin%
\pgfsetlinewidth{0.803000pt}%
\definecolor{currentstroke}{rgb}{0.000000,0.000000,0.000000}%
\pgfsetstrokecolor{currentstroke}%
\pgfsetdash{}{0pt}%
\pgfpathmoveto{\pgfqpoint{0.589510in}{0.417642in}}%
\pgfpathlineto{\pgfqpoint{2.399275in}{0.417642in}}%
\pgfusepath{stroke}%
\end{pgfscope}%
\begin{pgfscope}%
\pgfsetrectcap%
\pgfsetmiterjoin%
\pgfsetlinewidth{0.803000pt}%
\definecolor{currentstroke}{rgb}{0.000000,0.000000,0.000000}%
\pgfsetstrokecolor{currentstroke}%
\pgfsetdash{}{0pt}%
\pgfpathmoveto{\pgfqpoint{0.589510in}{1.789039in}}%
\pgfpathlineto{\pgfqpoint{2.399275in}{1.789039in}}%
\pgfusepath{stroke}%
\end{pgfscope}%
\begin{pgfscope}%
\pgfsetbuttcap%
\pgfsetmiterjoin%
\definecolor{currentfill}{rgb}{1.000000,1.000000,1.000000}%
\pgfsetfillcolor{currentfill}%
\pgfsetfillopacity{0.800000}%
\pgfsetlinewidth{1.003750pt}%
\definecolor{currentstroke}{rgb}{0.800000,0.800000,0.800000}%
\pgfsetstrokecolor{currentstroke}%
\pgfsetstrokeopacity{0.800000}%
\pgfsetdash{}{0pt}%
\pgfpathmoveto{\pgfqpoint{0.667288in}{1.544733in}}%
\pgfpathlineto{\pgfqpoint{1.448613in}{1.544733in}}%
\pgfpathquadraticcurveto{\pgfqpoint{1.470835in}{1.544733in}}{\pgfqpoint{1.470835in}{1.566956in}}%
\pgfpathlineto{\pgfqpoint{1.470835in}{1.711261in}}%
\pgfpathquadraticcurveto{\pgfqpoint{1.470835in}{1.733483in}}{\pgfqpoint{1.448613in}{1.733483in}}%
\pgfpathlineto{\pgfqpoint{0.667288in}{1.733483in}}%
\pgfpathquadraticcurveto{\pgfqpoint{0.645065in}{1.733483in}}{\pgfqpoint{0.645065in}{1.711261in}}%
\pgfpathlineto{\pgfqpoint{0.645065in}{1.566956in}}%
\pgfpathquadraticcurveto{\pgfqpoint{0.645065in}{1.544733in}}{\pgfqpoint{0.667288in}{1.544733in}}%
\pgfpathlineto{\pgfqpoint{0.667288in}{1.544733in}}%
\pgfpathclose%
\pgfusepath{stroke,fill}%
\end{pgfscope}%
\begin{pgfscope}%
\pgfsetbuttcap%
\pgfsetroundjoin%
\pgfsetlinewidth{1.505625pt}%
\definecolor{currentstroke}{rgb}{0.800000,0.470588,0.737255}%
\pgfsetstrokecolor{currentstroke}%
\pgfsetdash{{5.550000pt}{2.400000pt}}{0.000000pt}%
\pgfpathmoveto{\pgfqpoint{0.689510in}{1.649622in}}%
\pgfpathlineto{\pgfqpoint{0.800621in}{1.649622in}}%
\pgfpathlineto{\pgfqpoint{0.911732in}{1.649622in}}%
\pgfusepath{stroke}%
\end{pgfscope}%
\begin{pgfscope}%
\definecolor{textcolor}{rgb}{0.000000,0.000000,0.000000}%
\pgfsetstrokecolor{textcolor}%
\pgfsetfillcolor{textcolor}%
\pgftext[x=1.000621in,y=1.610733in,left,base]{\color{textcolor}\rmfamily\fontsize{8.000000}{9.600000}\selectfont \(\displaystyle \propto D\tau^{+1}\)}%
\end{pgfscope}%
\end{pgfpicture}%
\makeatother%
\endgroup%

        } % scalebox
        \caption{Allan deviation}
        \label{fig:drift_adev}
    \end{subfigure}
    \caption{Different representations of linear drift.}
    \label{fig:drift_noise_simulated}
\end{figure}

The coefficients given here are using the assumption, that the Allan deviation is the appropriate measure for the sample data. This might not always be the case, because the Allan deviation assumes a dead time of $\theta = 0$. This problem was extensively discussed by \citeauthor{psd_to_adev} \cite{psd_to_adev} and even special models were developed to account for the algorithms of modern frequency counters \cite{adev_frequency_counter}. It is therefore important to discuss typical measurement settings for voltmeter to estimate errors that arise from those settings. Typical settings, that affect the dead time of a voltmeter are auto-zeroing and line synchronization. Auto-zeroing is typically done by adding additional measurements to the normal input integratiion cycle. These measurements are a zero measurement to correct for offset drift and a measurement of the reference voltage to correct for gain errors. The implementation details and type of measurements are manufacturer dependant and must be determinded for every multimeter used.

% check \cite{psd_to_adev} Appendix II for details on dead time
% Compare PSD in Generation-Recombination Noise, Allan Variance, and Low-Frequency Gain Instabilities in Microwave Amplifiers to our controller. The hump look similar. Due to popcorn noise

\clearpage
\section{Temperature Controller}

\subsection{Tuning of a PID controller}
The number of emperical algorithms to determine a set of PID parameters ($\mathrm{K_p, K_i, K_d}$) are numerous. In this work only the most common algorithms and a few notable exceptions will be presented.
\subsection{Design}

\chapter{Preparation}
\section{Grounding and Shielding}
Add parts from "references\\Grounding and Shielding.pdf"
\section{Laser Current Driver}
\subsection{Design}
\subsubsection{Simulation}
\paragraph{Op Amp Stability}
\subsection{Noise Considerations}
\subsection{Voltage Reference}
\subsection{MOSFET Selection}

\section{LabKraken}
\subsection{Design Goals}
LabKraken is a designed to be a asynchronous, resilient data aquisition suite, that scales to thousands of sensors and accross different networks.
\subsection{Hardware}
\subsection{Software Architecture}
LabKraken needs to scale to thousands of sensors, which need to be served concurrently. This problem is commonly referered to as the C10K problem as dubbed by Dan Kegel back in 1999 \cite{10kProblem} and refers to serving \num{10000} concurrent connections via network sockets. While today millions of concurrent connections can be handled by servers, handling \num{10000} can still be challenging, especially, if the data sources are heterogeneous as is typical for sensor networks of different sensors from different manufacturers.

In order to meet the design goals, an asynchronous architecture was chosen and several different architectures were implemented over time. All in all four complete rewrites of the software were made to arrive at the architecture presented here. The reason for the rewrites is mostly historic and can be explained by the history of the programming language Python, which was used to write the code. The first first version was written for Python 2.6 and exclusively supported sensors made Tinkerforge. In 2015, Python 3.5 was released, which supported a new syntax for asynchronous coroutines. The software was rewritten from scratch to support this new syntax, because it made the code a lot more verbose and easier to follow. With the release of Python 3.7 in 2018 asynchronous generator expressions where mature enough to be used in productions and the programm was again rewritten to use the new syntax. In 2021 a new approach was taken and the programm was once more rewritten with a functional programming style. I will discuss each approach in the next sections to highlight the improvements, that were made over time. Each of these sections discusses the same programm, but written in different styles to show the differences.

\subsubsection{Threaded Design}
The first version of LabKraken used a threaded design approach, because the original libraries of the Tinkerforge sensors are built around threads. The following simplified example shows some code to connect to a temperature sensor over the network and read its data.

\inputpython{source/lab_kraken_threads.py}{1}{26}

\subsubsection{Device Identifiers}
Every sensor network needs device identifiers. Preferably those identifiers should be unique. Typically a device has some kind of internal indetifier. Here are a few examples of the sensors used in our network:

\begin{table}[h]
\centering
\begin{tabularx}{0.95\textwidth}{|l|p{6.5cm}|X|}
    \hline
    Device Type& Identifiers& Example\\
    \hline
    GPIB (SCPI)& \textit{*IDN?} returns \newline \$manufacturer,\$name,\$serial,\$revision& \\
    \hline
    Tinkerforge& Each sensor has a base58 encoded integer device id& QE9 (163684)\\
    \hline
    Labnode& Universal Unique Identifier (UUID) & cc2f2159-e2fb-4ed9-\newline8021-7771890b37ad\\
    \hline
\end{tabularx}
\end{table}

As it can be seen above, these identifiers do not guarantee to uniquely identify a device within a network. The Tinkerforge id is the weakest, as it is a \qty{32}{\bit} integer (4.294.967.295 options), which might easily collide with another id from a different manufacturer. The tinkerforge id is presented as a base58 encoded string. An encoder/decoder example can be found in the TinkerforgeAsync library \cite{TinkerforgeAsync}.

The id string returned by a SCPI device is slightly better, but again does not guarantee uniqueness. As it is shown in the example the same device might return a different id defpending on its settings. This typically done by manufacturers for compatibility reasons.

The only reasonably unique id is the universal unique identifier (UUID) or globally unique identifier (GUID), as dubbed by Microsoft, used in the Labnodes. Their id can be used for networks with participant numbers going into the millions.

Calculating the probability of a collision between two random UUIDs is called the birthday problem \cite{BirthdayProblem} in probability theory. A randomly generated version 4 UUID of variant 1 as defined in RFC 4122 \cite{RFC-UUID} has \qty{122}{\bit} of entropy, that is out of \qty{128}{\bit}, \qty{4}{\bit} are reserved for the UUID version and \qty{2}{\bit} for the variant. This gives the probability of at least one collision in $n$ devices out of $M = 2^{122}$ possibilities:
\begin{align}
    p(n) &= 1 - 1 \cdot \left(1 - \frac{1}{M}\right) \cdot \left(1 - \frac{2}{M}\right) \dots \left(1 - \frac{n-1}{M}\right) \nonumber\\
    &= 1 - \prod_{k=1}^{n-1} \left(1 - \frac{k}{M} \right)
\end{align}
Using the Taylor series $e^x = 1+x \dots$, assuming $n \ll M$ and approximating we can simplify this to:
\begin{align}
    p(n) &\approx 1 - \left(e^\frac{-1}{M} \cdot e^\frac{-2}{M} \dots e^\frac{-(n-1)}{M} \right) \nonumber\\
    &\approx 1 - \left(e^\frac{-n(n-1)/2}{M} \right) \nonumber\\
    &\approx 1 - \left(1 - \frac{n^2}{2 M} \right) = \frac{n^2}{2 M}
\end{align}
For one million devices, this gives a probability of about \num{2e-25}, which is negligible.

In the Kraken implementation, all devices, except for the Labnodes, will be mapped to UUIDs using the underlying configuration database. It is up to the user to ensure the uniqueness of the non-UUID ids reported by the devices to ensure proper mapping.


\subsubsection{Limitations} % FIXME: Different title
There is one inherent limitation to the ethernet bus for instrumentation. The ethernet bus is inherently asynchronous and multiple controllers can talk to the device at the same time. Not only that, but different processes within the same controller can talk to the same device. This makes deterministic statements about the device state challenging.

While it is impossible to rule out the possibility of multiple controllers on a network, care was taken to synchronize the workers within Kraken.
\subsection{Databases}
\subsubsection{Cardinality}
\begin{itemize}
 \item TimescaleDB vs Influx
 \item Example Sensors vs. Experiment
\end{itemize}


\section{Short Introduction to Control Theory}
This section will give a very brief introduction into some basic concepts of control theory. Many systems require control over one or more process variables. For example, temperature control of a room or a device, or creating a current from a voltage. All of this requires control over a process and is established trough feedback, which allows a controller to sense the state of the system.

The focus of this section lies on the principels feedback and control and will be detailed in the following sections.

\subsection{Transfer Functions}

\subsection{Open and Closed Loop Systems}
To understand feedback, one needs to take a look at dynamical systems. There are two types of systems: open and closed loop systems. A system is called open loop, if the output of a system does not influece its input as in figure \ref{fig:open_loop}. On the other hand, if the output is connected to the input of the system it is called closed loop system, an example is shown in figure \ref{fig:closed_loop}. $G(s)$ is called the transfer function of the system, while $R(s)$ is the input, $Y(s)$ is the output and $s$ the Laplace variable.

\begin{figure}[ht]
    \centering
    \begin{subfigure}{0.4\linewidth}
        \import{figures/}{open_loop.tex}
        \caption{Open loop system.}
        \label{fig:open_loop}
    \end{subfigure}
    \begin{subfigure}{0.4\linewidth}
        \import{figures/}{closed_loop.tex}
        \caption{Closed loop system.}
        \label{fig:closed_loop}
    \end{subfigure}
\end{figure}

It is convenient to express the transfer function as its Laplace transform. The unilateral Laplace transform is definded as:
\begin{equation}
    \mathscr{L}\left( f(t) \right) = F(s) = \int_0^\infty f(t) e^{-st}\,dt.
\end{equation}

with $f: \mathbb{R}^+ \to \mathbb{R}$, that is integrable and grows no faster than $e^{s_0t}$ for $s_0 \in \mathbb{R}$. The latter property is important for deriving the rules of differentiation and integration.

To understand the benefits of using the Laplace representation for transfer function a few useful properties must be discussed. First of all the Laplace transform is linear:
\begin{align}
    \mathscr{L}\left(a \cdot f(t) + b \cdot g(t) \right) &= \int_0^\infty (a \cdot f(t) + b \cdot g(t)) e^{-st}\,dt \nonumber\\
    &= a \int_0^\infty f(t) e^{-st}\,dt + b \int_0^\infty g(t) e^{-st}\,dt \nonumber\\
    &= a \mathscr{L}\left(f(t)\right) + b \mathscr{L}\left(g(t)\right)
\end{align}

Another interesting property is the derivative and integral of a function $f$:

\begin{align}
    \mathscr{L}\left(\frac{df}{dt}\right) &= \int_0^\infty \underbracket{f'(t)}_{v'(t)} \underbracket{\vphantom{f'(t)}e^{-st}}_{u(t)}\,dt \nonumber\\
    &= \left[e^{-st} f(t) \right]_0^\infty - \int_0^\infty (-s)f'(t)\,dt \nonumber\\
    &= -f(0) + s \int_0^\infty f'(t)\,dt \nonumber\\
    &= s F(s) - f(0)
\end{align}

\begin{align}
    \mathscr{L} \left( \int_0^t f(\tau)\,d\tau \right) &= \int_0^\infty \left(\int_0^t f(\tau)\,d\tau e^{-st} \right)\,dt \nonumber\\
    &= \int_0^\infty \underbracket{e^{-st}\vphantom{\int_0^t}}_{v'(t)} \underbracket{\int_0^t f(t)\,d\tau}_{u(t)}\,dt \nonumber\\
    &= \left[\frac{-1}{s} e^{-st} \int_0^t f(t)\,d\tau \right]_0^\infty - \int_0^\infty \frac{-1}{s} e^{-s\tau} f(\tau)\,d\tau \nonumber\\
    &= 0 + \frac{1}{s} \int_0^\infty e^{-s\tau} f(\tau)\,d\tau \nonumber\\
    &= \frac{1}{s} F(s) \label{eqn:lapace_integration}
\end{align}

If the initial state $f(0)$ can be chosen to be $0$, the differentiation becomes a simple multiplication by $s$, while the integration becomes a division by $s$. Finally, the most important aspect is, that a simple relation between the input $r(t)$ and the ouput $y(t)$ of a system can be given. The relation between input and the ouput of a system as shown in figure \ref{fig:open_loop} is given by the convolution, see e.g. \cite{pid_basics}. Assuming the system has an initial state of $0$ for $t<0$, hence $r(t<0) = 0$ and $g(t<0) = 0$, one can calculate:

\begin{equation}
    y(t) = (r \ast g)(t) = \int_0^\infty r(\tau) g(t-\tau)\,d\tau
    \label{eqn:convolution}
\end{equation}

Applying the Laplace transformation, greatly simplifies this:
\begin{align}
    Y(s) &= \int_0^\infty e^{-st} y(t)\,dt \nonumber\\
    \overset{\ref{eqn:convolution}}&{=} \int_0^\infty \underbrace{e^{-st}}_{e^{-s(t-\tau)}e^{-s\tau}} \int_0^\infty r(\tau) g(t-\tau)\,d\tau\,dt \nonumber\\
    &= \int_0^\infty \int_0^t e^{-s(t-\tau)} e^{-s\tau} g(t-\tau) r(\tau)\,d\tau\,dt \nonumber\\
    &= \int_0^\infty e^{-s\tau} r(\tau)\,d\tau \int_0^\infty e^{-st} g(t)\,dt \nonumber\\
    &= R(s) \cdot G(s)
\end{align}

This formula is a lot simpler than the convolution of $r(t)$ and $g(t)$, therefore the use of the Laplace transform has become very popular in control theory.

Another property that is heavily used in control theory is the time delay of functions. To show this property, let $f(t-\theta)$ be
\begin{equation}
    g(t) \coloneqq \begin{cases} f(t-\theta), & t \geq \theta \\ 0, & t < \theta \end{cases} \label{eqn:delayed_f}
\end{equation}

The reason for this definition is, that the system must be causal. This means, it is impossible to get data from the future ($t<\theta$). An example is shown in figure \ref{fig:heaviside}.

\begin{figure}[ht]
    \centering
    \begin{subfigure}{0.4\linewidth}
        \scalebox{0.75}{%
            \import{figures/}{laplace_no_delay.tex}
        } % scalebox
        \caption{Original signal $f(t)$.}
        \label{fig:heaviside}
    \end{subfigure}
    \begin{subfigure}{0.4\linewidth}
        \scalebox{0.75}{%
            \import{figures/}{laplace_time_delay.tex}
        } % scalebox
        \caption{Delayed signal $f(t-2)$.}
        \label{fig:heaviside_delayed}
    \end{subfigure}
\end{figure}

The Laplace transform of a delayed signal can be calculated as follows:

\begin{align}
    \mathscr{L}\left( g(t) \right) &= \int_0^\infty f(t-\theta) e^{-st}\,dt \nonumber\\
    \overset{\ref{eqn:delayed_f}}&{=} \int_\theta^\infty f(t-\theta) e^{-st}\,dt \nonumber\\
    \overset{u \coloneqq t-\theta}&{=} \int_0^\infty f(u) e^{-s(u+\theta)}\,du \nonumber\\
    &= e^{-s\theta} \int_0^\infty f(u) e^{-su} \nonumber\\
    &= e^{-s\theta} F(s) \label{eqn:laplace_delayed}
\end{align}

To satisfy the causaulity requirement, the Heaviside function $H(t)$ can be used:
\begin{align}
    \mathscr{L}\left( f(t-\theta) H(t-\theta) \right) = e^{-s\theta} F(s) \label{eqn:laplace_causality}
\end{align}

Lastly, the Laplace transform of $e^{at}$, which is commonly used in differential equations:
\begin{align}
    \mathscr{L}\left(e^{at} \right) &= \int_0^\infty e^{(a-s)t}\,dt = \frac{1}{a-s} \left[e^{(a-s)t} \right]_0^\infty = \frac{1}{s-a} \label{eqn:laplace_exponential}
\end{align}


Using these tools, it is possible calculate the transfer function of a temperature controller. This is done in the next section.

\subsection{A Model for Temperature Control}
\begin{figure}[h]
    \centering
    \scalebox{1}{%
        \import{figures/}{first_order_model.tex}
    } % scalebox
    \caption{Simple temperature model of a generic system.}
    \label{fig:first_order_model_room}
\end{figure}

In order to describe a closed-loop system, one has to first create a model for the process and the controller involved. A simple model can be derived from the idea, that the system at temperature $T_{system}$ has a thermal capacitance $C_{system}$, an influx of heat $\dot Q_{load}$ from a thermal load and a controller removing heat from the system through a heat exchanger with a resistance of $R_{force}$. Additionally, there is some leakage through the walls of the system to the ambient environment via $R_{leakage}$. The analogy of thermodynamics with electrondynamics allows to create the model in figure \ref{fig:first_order_model_room}. Since this this model is to be used for a temperature controller, an assumption to simplify it can be made.

The controller will keep $T_{system}$ constant and if the ambient temperature and $\dot Q_{load}$ is \textit{reasonably stable}, it is easy to see, that a constant thermal flux must flow through $R$ since it cannot pass through the thermal capacitance $C$. \textit{Reasonably stable} means that it can be treated as constant with respect to the temperature controller time constants. This will be further discussed in section \ref{} with regards to system stability. If this assumption holds, the thermal flux from the system load will only cause a constant offset of $T_{in}$, since the heat must be removed by the controller, and the model can be simplified further:

\begin{figure}[h]
    \centering
    \scalebox{1}{%
        \import{figures/}{first_order_model_kirchhoff.tex}
    } % scalebox
\end{figure}

Neglecting the constant thermal flux from the system load and exploiting the analogy of thermodynamics and electrondynamics again, using Kirchhoff's second law, we find:

\begin{align}
    \sum T_i &= 0 \nonumber\\
    T_{in}(t) - \dot{Q}(t) R - \frac 1 C \int \dot{Q}(t)\,dt &= 0 \label{eqn:first_order_model_kirchhoff}
\end{align}

Taking the Laplace transform, applying equation \ref{eqn:lapace_integration} and using $T_{out} = \frac{1}{sC} \dot Q(s)$ to replace $\dot Q$, equation \ref{eqn:first_order_model_kirchhoff} can be written as:
\begin{align*}
    T_{in}(s) - \dot{Q}(s) R - \frac{1}{sC} \dot{Q}(s) &= 0\\
    \dot{Q}(s) = \frac{T_{in}(s)}{R-\frac{1}{sC}} &= \frac{T_{out}}{\frac{1}{sC}}
\end{align*}

This allows to calculate the transfer function of the process $P$:
\begin{align}
    P(s) &= \frac{T_{out}}{T_{in}} = \frac{\frac{1}{sC}}{R-\frac{1}{sC}} \nonumber\\
    &= \frac{1}{sRC + 1} \nonumber\\
    &= \frac{K}{1 + s\tau} \label{eqn:first_order_model}
\end{align}
with the system gain $K$ and the time constant $\tau$. In case of the $RC$ circuit, the gain is $1$, but other systems may a gain or attenuation of $K \neq 1$ in the sensor.

Equation \ref{eqn:first_order_model} is called the transfer function of a first-order model, because its origin is a differential equation of first order. This model describes homogeneous systems, like a room, very well, as can be seen in section \ref{}, but in order to derive the transfer function including the controller and the sensor some more work is required.

Expanding on figure \ref{fig:closed_loop} and equation \ref{eqn:convolution} the closed-loop transfer function becomes:
\begin{equation}
    G(s) = P(s) \cdot S(s)
\end{equation}

and the block diagram becomes

\begin{figure}[ht]
    \centering
    \import{figures/}{open_loop_full.tex}
    \caption{Open loop system with sensor.}
\end{figure}

The transfer funciton of the sensor can, in the most simple case, be modeled as a delay line with delay $\theta$ and $f(t-\theta) = H(t-\theta)$. Using equation \ref{eqn:laplace_delayed} $S(s)$ can be written as
\begin{equation}
    S(s) = e^{-\theta s} .
\end{equation}

The full process model including the time delay is:
\begin{equation}
    G(s) = \frac{K}{1 + s\tau} e^{-\theta s} \label{eqn:first_order_plus_dead_time_model}
\end{equation}

This is called a first-order plus dead-time model (FOPDT) or first-order plus time-delay model (FOPTD). To fit experimental data to this model it is more convenient to transform the transfer function \ref{eqn:first_order_plus_dead_time_model} into the time domain. To calculate the output response an input $U(s)$ is required. In principal any function can do, but a step function is typically used, for example by \citeauthor{ziegler_nichols} \cite{ziegler_nichols} and many others \cite{tuning_rules,pessen_integral,simc,smic2,pid_controllers_for_time_delay_systems,pi_stabilization_of_fopdt_systems, pid_basics}. It is both simple to calculate and apply to a real system. Using equations \ref{eqn:laplace_delayed} and \ref{eqn:laplace_exponential}, the Heaviside $H(t)$ step function transforms as
\begin{equation}
    \mathscr{L} \left(u(t) \right) = U(s) = \mathscr{L} \left( \Delta u H(t) \right) = \frac{\Delta u}{s}
\end{equation}

with the step size $\Delta u$. The output $Y(s)$ can then be calculated analytically.

\begin{align}
    Y(s) &= \frac{\Delta u}{s} \frac{K}{1 + s\tau} e^{-\theta s} \nonumber\\
    &=  K \Delta u \frac{1}{s (1 + s\tau)} e^{-\theta s} \nonumber\\
    &= K \Delta u \left(\frac{1}{s} - \frac{\tau}{s\tau+1} \right) e^{-\theta s} \nonumber\\
    &= K \Delta u \left(\frac{1}{s} - \frac{1}{s+\frac{1}{\tau}} \right) e^{-\theta s}
\end{align}

To derive $y(t)$, the inverse Laplace transform of $Y(s)$ is required. Unfortunately, this is not as simple as the Laplace transform. Fortunately, using \ref{eqn:laplace_exponential} while making sure causaulity is guaranteed as shown in \ref{eqn:laplace_causality}, the simple first order model can easily be transformed back into the time domain.

\begin{align}
    \mathscr{L}^{-1} \left(Y(s)\right) = y(t) &= K \Delta u \mathscr{L}^{-1} \left(\frac{1}{s} e^{-\theta s} \right)  - K \mathscr{L}^{-1} \left( \frac{1}{s+\frac{1}{\tau}} e^{-\theta s} \right) \nonumber\\
    \overset{\ref{eqn:laplace_exponential}}&{=} K \Delta u \cdot 1 \cdot H(t-\theta) - \left(e^{-\frac{t-\theta}{\tau}} \right) H(t-\theta) \nonumber\\
    &= K \Delta u \left(1- e^{-\frac{t-\theta}{\tau}} \right) H(t-\theta)
\end{align}

The time domain solution of the FOPDT model can now be used extract the parameters $\tau$, $\theta$ and $K$ from a real physical system using a fit to the measurement data. The parameter $\Delta u$ is already known, since it is an input parameter. A simulation of the step response of a first-order model with time delay is shown in figure \ref{fig:fopdt}. Here it can be clearly seen, that the output does not change until the time delay $\theta$ has passed and the Heaviside function changes from $0$ to $1$.

\begin{figure}[ht]
    \centering
    %% Creator: Matplotlib, PGF backend
%%
%% To include the figure in your LaTeX document, write
%%   \input{<filename>.pgf}
%%
%% Make sure the required packages are loaded in your preamble
%%   \usepackage{pgf}
%%
%% Also ensure that all the required font packages are loaded; for instance,
%% the lmodern package is sometimes necessary when using math font.
%%   \usepackage{lmodern}
%%
%% Figures using additional raster images can only be included by \input if
%% they are in the same directory as the main LaTeX file. For loading figures
%% from other directories you can use the `import` package
%%   \usepackage{import}
%%
%% and then include the figures with
%%   \import{<path to file>}{<filename>.pgf}
%%
%% Matplotlib used the following preamble
%%   \usepackage{fontspec}
%%   \setmainfont{DejaVuSerif.ttf}[Path=\detokenize{/home/maat/Documents/Uni/Physik/Phd/Thesis/data/env/lib/python3.10/site-packages/matplotlib/mpl-data/fonts/ttf/}]
%%   \setsansfont{DejaVuSans.ttf}[Path=\detokenize{/home/maat/Documents/Uni/Physik/Phd/Thesis/data/env/lib/python3.10/site-packages/matplotlib/mpl-data/fonts/ttf/}]
%%   \setmonofont{DejaVuSansMono.ttf}[Path=\detokenize{/home/maat/Documents/Uni/Physik/Phd/Thesis/data/env/lib/python3.10/site-packages/matplotlib/mpl-data/fonts/ttf/}]
%%
\begingroup%
\makeatletter%
\begin{pgfpicture}%
\pgfpathrectangle{\pgfpointorigin}{\pgfqpoint{5.208662in}{3.219130in}}%
\pgfusepath{use as bounding box, clip}%
\begin{pgfscope}%
\pgfsetbuttcap%
\pgfsetmiterjoin%
\definecolor{currentfill}{rgb}{1.000000,1.000000,1.000000}%
\pgfsetfillcolor{currentfill}%
\pgfsetlinewidth{0.000000pt}%
\definecolor{currentstroke}{rgb}{1.000000,1.000000,1.000000}%
\pgfsetstrokecolor{currentstroke}%
\pgfsetdash{}{0pt}%
\pgfpathmoveto{\pgfqpoint{0.000000in}{0.000000in}}%
\pgfpathlineto{\pgfqpoint{5.208662in}{0.000000in}}%
\pgfpathlineto{\pgfqpoint{5.208662in}{3.219130in}}%
\pgfpathlineto{\pgfqpoint{0.000000in}{3.219130in}}%
\pgfpathlineto{\pgfqpoint{0.000000in}{0.000000in}}%
\pgfpathclose%
\pgfusepath{fill}%
\end{pgfscope}%
\begin{pgfscope}%
\pgfsetbuttcap%
\pgfsetmiterjoin%
\definecolor{currentfill}{rgb}{1.000000,1.000000,1.000000}%
\pgfsetfillcolor{currentfill}%
\pgfsetlinewidth{0.000000pt}%
\definecolor{currentstroke}{rgb}{0.000000,0.000000,0.000000}%
\pgfsetstrokecolor{currentstroke}%
\pgfsetstrokeopacity{0.000000}%
\pgfsetdash{}{0pt}%
\pgfpathmoveto{\pgfqpoint{0.779028in}{0.582778in}}%
\pgfpathlineto{\pgfqpoint{4.970537in}{0.582778in}}%
\pgfpathlineto{\pgfqpoint{4.970537in}{3.014130in}}%
\pgfpathlineto{\pgfqpoint{0.779028in}{3.014130in}}%
\pgfpathlineto{\pgfqpoint{0.779028in}{0.582778in}}%
\pgfpathclose%
\pgfusepath{fill}%
\end{pgfscope}%
\begin{pgfscope}%
\pgfsetbuttcap%
\pgfsetroundjoin%
\definecolor{currentfill}{rgb}{0.000000,0.000000,0.000000}%
\pgfsetfillcolor{currentfill}%
\pgfsetlinewidth{0.803000pt}%
\definecolor{currentstroke}{rgb}{0.000000,0.000000,0.000000}%
\pgfsetstrokecolor{currentstroke}%
\pgfsetdash{}{0pt}%
\pgfsys@defobject{currentmarker}{\pgfqpoint{0.000000in}{-0.048611in}}{\pgfqpoint{0.000000in}{0.000000in}}{%
\pgfpathmoveto{\pgfqpoint{0.000000in}{0.000000in}}%
\pgfpathlineto{\pgfqpoint{0.000000in}{-0.048611in}}%
\pgfusepath{stroke,fill}%
}%
\begin{pgfscope}%
\pgfsys@transformshift{0.779028in}{0.582778in}%
\pgfsys@useobject{currentmarker}{}%
\end{pgfscope}%
\end{pgfscope}%
\begin{pgfscope}%
\definecolor{textcolor}{rgb}{0.000000,0.000000,0.000000}%
\pgfsetstrokecolor{textcolor}%
\pgfsetfillcolor{textcolor}%
\pgftext[x=0.779028in,y=0.485556in,,top]{\color{textcolor}\sffamily\fontsize{10.000000}{12.000000}\selectfont 0}%
\end{pgfscope}%
\begin{pgfscope}%
\pgfsetbuttcap%
\pgfsetroundjoin%
\definecolor{currentfill}{rgb}{0.000000,0.000000,0.000000}%
\pgfsetfillcolor{currentfill}%
\pgfsetlinewidth{0.803000pt}%
\definecolor{currentstroke}{rgb}{0.000000,0.000000,0.000000}%
\pgfsetstrokecolor{currentstroke}%
\pgfsetdash{}{0pt}%
\pgfsys@defobject{currentmarker}{\pgfqpoint{0.000000in}{-0.048611in}}{\pgfqpoint{0.000000in}{0.000000in}}{%
\pgfpathmoveto{\pgfqpoint{0.000000in}{0.000000in}}%
\pgfpathlineto{\pgfqpoint{0.000000in}{-0.048611in}}%
\pgfusepath{stroke,fill}%
}%
\begin{pgfscope}%
\pgfsys@transformshift{1.617330in}{0.582778in}%
\pgfsys@useobject{currentmarker}{}%
\end{pgfscope}%
\end{pgfscope}%
\begin{pgfscope}%
\definecolor{textcolor}{rgb}{0.000000,0.000000,0.000000}%
\pgfsetstrokecolor{textcolor}%
\pgfsetfillcolor{textcolor}%
\pgftext[x=1.617330in,y=0.485556in,,top]{\color{textcolor}\sffamily\fontsize{10.000000}{12.000000}\selectfont 2}%
\end{pgfscope}%
\begin{pgfscope}%
\pgfsetbuttcap%
\pgfsetroundjoin%
\definecolor{currentfill}{rgb}{0.000000,0.000000,0.000000}%
\pgfsetfillcolor{currentfill}%
\pgfsetlinewidth{0.803000pt}%
\definecolor{currentstroke}{rgb}{0.000000,0.000000,0.000000}%
\pgfsetstrokecolor{currentstroke}%
\pgfsetdash{}{0pt}%
\pgfsys@defobject{currentmarker}{\pgfqpoint{0.000000in}{-0.048611in}}{\pgfqpoint{0.000000in}{0.000000in}}{%
\pgfpathmoveto{\pgfqpoint{0.000000in}{0.000000in}}%
\pgfpathlineto{\pgfqpoint{0.000000in}{-0.048611in}}%
\pgfusepath{stroke,fill}%
}%
\begin{pgfscope}%
\pgfsys@transformshift{2.455631in}{0.582778in}%
\pgfsys@useobject{currentmarker}{}%
\end{pgfscope}%
\end{pgfscope}%
\begin{pgfscope}%
\definecolor{textcolor}{rgb}{0.000000,0.000000,0.000000}%
\pgfsetstrokecolor{textcolor}%
\pgfsetfillcolor{textcolor}%
\pgftext[x=2.455631in,y=0.485556in,,top]{\color{textcolor}\sffamily\fontsize{10.000000}{12.000000}\selectfont 4}%
\end{pgfscope}%
\begin{pgfscope}%
\pgfsetbuttcap%
\pgfsetroundjoin%
\definecolor{currentfill}{rgb}{0.000000,0.000000,0.000000}%
\pgfsetfillcolor{currentfill}%
\pgfsetlinewidth{0.803000pt}%
\definecolor{currentstroke}{rgb}{0.000000,0.000000,0.000000}%
\pgfsetstrokecolor{currentstroke}%
\pgfsetdash{}{0pt}%
\pgfsys@defobject{currentmarker}{\pgfqpoint{0.000000in}{-0.048611in}}{\pgfqpoint{0.000000in}{0.000000in}}{%
\pgfpathmoveto{\pgfqpoint{0.000000in}{0.000000in}}%
\pgfpathlineto{\pgfqpoint{0.000000in}{-0.048611in}}%
\pgfusepath{stroke,fill}%
}%
\begin{pgfscope}%
\pgfsys@transformshift{3.293933in}{0.582778in}%
\pgfsys@useobject{currentmarker}{}%
\end{pgfscope}%
\end{pgfscope}%
\begin{pgfscope}%
\definecolor{textcolor}{rgb}{0.000000,0.000000,0.000000}%
\pgfsetstrokecolor{textcolor}%
\pgfsetfillcolor{textcolor}%
\pgftext[x=3.293933in,y=0.485556in,,top]{\color{textcolor}\sffamily\fontsize{10.000000}{12.000000}\selectfont 6}%
\end{pgfscope}%
\begin{pgfscope}%
\pgfsetbuttcap%
\pgfsetroundjoin%
\definecolor{currentfill}{rgb}{0.000000,0.000000,0.000000}%
\pgfsetfillcolor{currentfill}%
\pgfsetlinewidth{0.803000pt}%
\definecolor{currentstroke}{rgb}{0.000000,0.000000,0.000000}%
\pgfsetstrokecolor{currentstroke}%
\pgfsetdash{}{0pt}%
\pgfsys@defobject{currentmarker}{\pgfqpoint{0.000000in}{-0.048611in}}{\pgfqpoint{0.000000in}{0.000000in}}{%
\pgfpathmoveto{\pgfqpoint{0.000000in}{0.000000in}}%
\pgfpathlineto{\pgfqpoint{0.000000in}{-0.048611in}}%
\pgfusepath{stroke,fill}%
}%
\begin{pgfscope}%
\pgfsys@transformshift{4.132235in}{0.582778in}%
\pgfsys@useobject{currentmarker}{}%
\end{pgfscope}%
\end{pgfscope}%
\begin{pgfscope}%
\definecolor{textcolor}{rgb}{0.000000,0.000000,0.000000}%
\pgfsetstrokecolor{textcolor}%
\pgfsetfillcolor{textcolor}%
\pgftext[x=4.132235in,y=0.485556in,,top]{\color{textcolor}\sffamily\fontsize{10.000000}{12.000000}\selectfont 8}%
\end{pgfscope}%
\begin{pgfscope}%
\pgfsetbuttcap%
\pgfsetroundjoin%
\definecolor{currentfill}{rgb}{0.000000,0.000000,0.000000}%
\pgfsetfillcolor{currentfill}%
\pgfsetlinewidth{0.803000pt}%
\definecolor{currentstroke}{rgb}{0.000000,0.000000,0.000000}%
\pgfsetstrokecolor{currentstroke}%
\pgfsetdash{}{0pt}%
\pgfsys@defobject{currentmarker}{\pgfqpoint{0.000000in}{-0.048611in}}{\pgfqpoint{0.000000in}{0.000000in}}{%
\pgfpathmoveto{\pgfqpoint{0.000000in}{0.000000in}}%
\pgfpathlineto{\pgfqpoint{0.000000in}{-0.048611in}}%
\pgfusepath{stroke,fill}%
}%
\begin{pgfscope}%
\pgfsys@transformshift{4.970537in}{0.582778in}%
\pgfsys@useobject{currentmarker}{}%
\end{pgfscope}%
\end{pgfscope}%
\begin{pgfscope}%
\definecolor{textcolor}{rgb}{0.000000,0.000000,0.000000}%
\pgfsetstrokecolor{textcolor}%
\pgfsetfillcolor{textcolor}%
\pgftext[x=4.970537in,y=0.485556in,,top]{\color{textcolor}\sffamily\fontsize{10.000000}{12.000000}\selectfont 10}%
\end{pgfscope}%
\begin{pgfscope}%
\definecolor{textcolor}{rgb}{0.000000,0.000000,0.000000}%
\pgfsetstrokecolor{textcolor}%
\pgfsetfillcolor{textcolor}%
\pgftext[x=2.874782in,y=0.295587in,,top]{\color{textcolor}\sffamily\fontsize{10.000000}{12.000000}\selectfont Time}%
\end{pgfscope}%
\begin{pgfscope}%
\pgfsetbuttcap%
\pgfsetroundjoin%
\definecolor{currentfill}{rgb}{0.000000,0.000000,0.000000}%
\pgfsetfillcolor{currentfill}%
\pgfsetlinewidth{0.803000pt}%
\definecolor{currentstroke}{rgb}{0.000000,0.000000,0.000000}%
\pgfsetstrokecolor{currentstroke}%
\pgfsetdash{}{0pt}%
\pgfsys@defobject{currentmarker}{\pgfqpoint{-0.048611in}{0.000000in}}{\pgfqpoint{-0.000000in}{0.000000in}}{%
\pgfpathmoveto{\pgfqpoint{-0.000000in}{0.000000in}}%
\pgfpathlineto{\pgfqpoint{-0.048611in}{0.000000in}}%
\pgfusepath{stroke,fill}%
}%
\begin{pgfscope}%
\pgfsys@transformshift{0.779028in}{0.582778in}%
\pgfsys@useobject{currentmarker}{}%
\end{pgfscope}%
\end{pgfscope}%
\begin{pgfscope}%
\definecolor{textcolor}{rgb}{0.000000,0.000000,0.000000}%
\pgfsetstrokecolor{textcolor}%
\pgfsetfillcolor{textcolor}%
\pgftext[x=0.352901in, y=0.530016in, left, base]{\color{textcolor}\sffamily\fontsize{10.000000}{12.000000}\selectfont \ensuremath{-}1.0}%
\end{pgfscope}%
\begin{pgfscope}%
\pgfsetbuttcap%
\pgfsetroundjoin%
\definecolor{currentfill}{rgb}{0.000000,0.000000,0.000000}%
\pgfsetfillcolor{currentfill}%
\pgfsetlinewidth{0.803000pt}%
\definecolor{currentstroke}{rgb}{0.000000,0.000000,0.000000}%
\pgfsetstrokecolor{currentstroke}%
\pgfsetdash{}{0pt}%
\pgfsys@defobject{currentmarker}{\pgfqpoint{-0.048611in}{0.000000in}}{\pgfqpoint{-0.000000in}{0.000000in}}{%
\pgfpathmoveto{\pgfqpoint{-0.000000in}{0.000000in}}%
\pgfpathlineto{\pgfqpoint{-0.048611in}{0.000000in}}%
\pgfusepath{stroke,fill}%
}%
\begin{pgfscope}%
\pgfsys@transformshift{0.779028in}{1.069048in}%
\pgfsys@useobject{currentmarker}{}%
\end{pgfscope}%
\end{pgfscope}%
\begin{pgfscope}%
\definecolor{textcolor}{rgb}{0.000000,0.000000,0.000000}%
\pgfsetstrokecolor{textcolor}%
\pgfsetfillcolor{textcolor}%
\pgftext[x=0.352901in, y=1.016287in, left, base]{\color{textcolor}\sffamily\fontsize{10.000000}{12.000000}\selectfont \ensuremath{-}0.5}%
\end{pgfscope}%
\begin{pgfscope}%
\pgfsetbuttcap%
\pgfsetroundjoin%
\definecolor{currentfill}{rgb}{0.000000,0.000000,0.000000}%
\pgfsetfillcolor{currentfill}%
\pgfsetlinewidth{0.803000pt}%
\definecolor{currentstroke}{rgb}{0.000000,0.000000,0.000000}%
\pgfsetstrokecolor{currentstroke}%
\pgfsetdash{}{0pt}%
\pgfsys@defobject{currentmarker}{\pgfqpoint{-0.048611in}{0.000000in}}{\pgfqpoint{-0.000000in}{0.000000in}}{%
\pgfpathmoveto{\pgfqpoint{-0.000000in}{0.000000in}}%
\pgfpathlineto{\pgfqpoint{-0.048611in}{0.000000in}}%
\pgfusepath{stroke,fill}%
}%
\begin{pgfscope}%
\pgfsys@transformshift{0.779028in}{1.555319in}%
\pgfsys@useobject{currentmarker}{}%
\end{pgfscope}%
\end{pgfscope}%
\begin{pgfscope}%
\definecolor{textcolor}{rgb}{0.000000,0.000000,0.000000}%
\pgfsetstrokecolor{textcolor}%
\pgfsetfillcolor{textcolor}%
\pgftext[x=0.460926in, y=1.502557in, left, base]{\color{textcolor}\sffamily\fontsize{10.000000}{12.000000}\selectfont 0.0}%
\end{pgfscope}%
\begin{pgfscope}%
\pgfsetbuttcap%
\pgfsetroundjoin%
\definecolor{currentfill}{rgb}{0.000000,0.000000,0.000000}%
\pgfsetfillcolor{currentfill}%
\pgfsetlinewidth{0.803000pt}%
\definecolor{currentstroke}{rgb}{0.000000,0.000000,0.000000}%
\pgfsetstrokecolor{currentstroke}%
\pgfsetdash{}{0pt}%
\pgfsys@defobject{currentmarker}{\pgfqpoint{-0.048611in}{0.000000in}}{\pgfqpoint{-0.000000in}{0.000000in}}{%
\pgfpathmoveto{\pgfqpoint{-0.000000in}{0.000000in}}%
\pgfpathlineto{\pgfqpoint{-0.048611in}{0.000000in}}%
\pgfusepath{stroke,fill}%
}%
\begin{pgfscope}%
\pgfsys@transformshift{0.779028in}{2.041589in}%
\pgfsys@useobject{currentmarker}{}%
\end{pgfscope}%
\end{pgfscope}%
\begin{pgfscope}%
\definecolor{textcolor}{rgb}{0.000000,0.000000,0.000000}%
\pgfsetstrokecolor{textcolor}%
\pgfsetfillcolor{textcolor}%
\pgftext[x=0.460926in, y=1.988828in, left, base]{\color{textcolor}\sffamily\fontsize{10.000000}{12.000000}\selectfont 0.5}%
\end{pgfscope}%
\begin{pgfscope}%
\pgfsetbuttcap%
\pgfsetroundjoin%
\definecolor{currentfill}{rgb}{0.000000,0.000000,0.000000}%
\pgfsetfillcolor{currentfill}%
\pgfsetlinewidth{0.803000pt}%
\definecolor{currentstroke}{rgb}{0.000000,0.000000,0.000000}%
\pgfsetstrokecolor{currentstroke}%
\pgfsetdash{}{0pt}%
\pgfsys@defobject{currentmarker}{\pgfqpoint{-0.048611in}{0.000000in}}{\pgfqpoint{-0.000000in}{0.000000in}}{%
\pgfpathmoveto{\pgfqpoint{-0.000000in}{0.000000in}}%
\pgfpathlineto{\pgfqpoint{-0.048611in}{0.000000in}}%
\pgfusepath{stroke,fill}%
}%
\begin{pgfscope}%
\pgfsys@transformshift{0.779028in}{2.527860in}%
\pgfsys@useobject{currentmarker}{}%
\end{pgfscope}%
\end{pgfscope}%
\begin{pgfscope}%
\definecolor{textcolor}{rgb}{0.000000,0.000000,0.000000}%
\pgfsetstrokecolor{textcolor}%
\pgfsetfillcolor{textcolor}%
\pgftext[x=0.460926in, y=2.475098in, left, base]{\color{textcolor}\sffamily\fontsize{10.000000}{12.000000}\selectfont 1.0}%
\end{pgfscope}%
\begin{pgfscope}%
\pgfsetbuttcap%
\pgfsetroundjoin%
\definecolor{currentfill}{rgb}{0.000000,0.000000,0.000000}%
\pgfsetfillcolor{currentfill}%
\pgfsetlinewidth{0.803000pt}%
\definecolor{currentstroke}{rgb}{0.000000,0.000000,0.000000}%
\pgfsetstrokecolor{currentstroke}%
\pgfsetdash{}{0pt}%
\pgfsys@defobject{currentmarker}{\pgfqpoint{-0.048611in}{0.000000in}}{\pgfqpoint{-0.000000in}{0.000000in}}{%
\pgfpathmoveto{\pgfqpoint{-0.000000in}{0.000000in}}%
\pgfpathlineto{\pgfqpoint{-0.048611in}{0.000000in}}%
\pgfusepath{stroke,fill}%
}%
\begin{pgfscope}%
\pgfsys@transformshift{0.779028in}{3.014130in}%
\pgfsys@useobject{currentmarker}{}%
\end{pgfscope}%
\end{pgfscope}%
\begin{pgfscope}%
\definecolor{textcolor}{rgb}{0.000000,0.000000,0.000000}%
\pgfsetstrokecolor{textcolor}%
\pgfsetfillcolor{textcolor}%
\pgftext[x=0.460926in, y=2.961369in, left, base]{\color{textcolor}\sffamily\fontsize{10.000000}{12.000000}\selectfont 1.5}%
\end{pgfscope}%
\begin{pgfscope}%
\definecolor{textcolor}{rgb}{0.000000,0.000000,0.000000}%
\pgfsetstrokecolor{textcolor}%
\pgfsetfillcolor{textcolor}%
\pgftext[x=0.297346in,y=1.798454in,,bottom,rotate=90.000000]{\color{textcolor}\sffamily\fontsize{10.000000}{12.000000}\selectfont Process Output}%
\end{pgfscope}%
\begin{pgfscope}%
\pgfpathrectangle{\pgfqpoint{0.779028in}{0.582778in}}{\pgfqpoint{4.191509in}{2.431352in}}%
\pgfusepath{clip}%
\pgfsetbuttcap%
\pgfsetroundjoin%
\pgfsetlinewidth{2.007500pt}%
\definecolor{currentstroke}{rgb}{0.000000,0.419608,0.643137}%
\pgfsetstrokecolor{currentstroke}%
\pgfsetstrokeopacity{0.700000}%
\pgfsetdash{{7.400000pt}{3.200000pt}}{0.000000pt}%
\pgfpathmoveto{\pgfqpoint{1.870301in}{0.572778in}}%
\pgfpathlineto{\pgfqpoint{1.910735in}{0.664918in}}%
\pgfpathlineto{\pgfqpoint{1.952650in}{0.755775in}}%
\pgfpathlineto{\pgfqpoint{1.994565in}{0.842200in}}%
\pgfpathlineto{\pgfqpoint{2.036481in}{0.924411in}}%
\pgfpathlineto{\pgfqpoint{2.078396in}{1.002612in}}%
\pgfpathlineto{\pgfqpoint{2.120311in}{1.076999in}}%
\pgfpathlineto{\pgfqpoint{2.162226in}{1.147758in}}%
\pgfpathlineto{\pgfqpoint{2.204141in}{1.215067in}}%
\pgfpathlineto{\pgfqpoint{2.246056in}{1.279092in}}%
\pgfpathlineto{\pgfqpoint{2.287971in}{1.339995in}}%
\pgfpathlineto{\pgfqpoint{2.329886in}{1.397928in}}%
\pgfpathlineto{\pgfqpoint{2.371801in}{1.453036in}}%
\pgfpathlineto{\pgfqpoint{2.413716in}{1.505455in}}%
\pgfpathlineto{\pgfqpoint{2.455631in}{1.555319in}}%
\pgfpathlineto{\pgfqpoint{2.497547in}{1.602750in}}%
\pgfpathlineto{\pgfqpoint{2.539462in}{1.647868in}}%
\pgfpathlineto{\pgfqpoint{2.581377in}{1.690786in}}%
\pgfpathlineto{\pgfqpoint{2.623292in}{1.731610in}}%
\pgfpathlineto{\pgfqpoint{2.665207in}{1.770444in}}%
\pgfpathlineto{\pgfqpoint{2.707122in}{1.807384in}}%
\pgfpathlineto{\pgfqpoint{2.749037in}{1.842522in}}%
\pgfpathlineto{\pgfqpoint{2.790952in}{1.875946in}}%
\pgfpathlineto{\pgfqpoint{2.832867in}{1.907740in}}%
\pgfpathlineto{\pgfqpoint{2.874782in}{1.937984in}}%
\pgfpathlineto{\pgfqpoint{2.916697in}{1.966752in}}%
\pgfpathlineto{\pgfqpoint{2.958613in}{1.994118in}}%
\pgfpathlineto{\pgfqpoint{3.000528in}{2.020149in}}%
\pgfpathlineto{\pgfqpoint{3.042443in}{2.044910in}}%
\pgfpathlineto{\pgfqpoint{3.084358in}{2.068464in}}%
\pgfpathlineto{\pgfqpoint{3.126273in}{2.090869in}}%
\pgfpathlineto{\pgfqpoint{3.168188in}{2.112181in}}%
\pgfpathlineto{\pgfqpoint{3.210103in}{2.132454in}}%
\pgfpathlineto{\pgfqpoint{3.252018in}{2.151738in}}%
\pgfpathlineto{\pgfqpoint{3.293933in}{2.170082in}}%
\pgfpathlineto{\pgfqpoint{3.335848in}{2.187531in}}%
\pgfpathlineto{\pgfqpoint{3.377763in}{2.204129in}}%
\pgfpathlineto{\pgfqpoint{3.419679in}{2.219917in}}%
\pgfpathlineto{\pgfqpoint{3.461594in}{2.234936in}}%
\pgfpathlineto{\pgfqpoint{3.503509in}{2.249222in}}%
\pgfpathlineto{\pgfqpoint{3.545424in}{2.262811in}}%
\pgfpathlineto{\pgfqpoint{3.587339in}{2.275738in}}%
\pgfpathlineto{\pgfqpoint{3.629254in}{2.288034in}}%
\pgfpathlineto{\pgfqpoint{3.671169in}{2.299730in}}%
\pgfpathlineto{\pgfqpoint{3.713084in}{2.310856in}}%
\pgfpathlineto{\pgfqpoint{3.754999in}{2.321440in}}%
\pgfpathlineto{\pgfqpoint{3.796914in}{2.331507in}}%
\pgfpathlineto{\pgfqpoint{3.838829in}{2.341083in}}%
\pgfpathlineto{\pgfqpoint{3.880745in}{2.350192in}}%
\pgfpathlineto{\pgfqpoint{3.922660in}{2.358857in}}%
\pgfpathlineto{\pgfqpoint{3.964575in}{2.367100in}}%
\pgfpathlineto{\pgfqpoint{4.006490in}{2.374940in}}%
\pgfpathlineto{\pgfqpoint{4.048405in}{2.382398in}}%
\pgfpathlineto{\pgfqpoint{4.090320in}{2.389492in}}%
\pgfpathlineto{\pgfqpoint{4.132235in}{2.396241in}}%
\pgfpathlineto{\pgfqpoint{4.174150in}{2.402660in}}%
\pgfpathlineto{\pgfqpoint{4.216065in}{2.408766in}}%
\pgfpathlineto{\pgfqpoint{4.257980in}{2.414574in}}%
\pgfpathlineto{\pgfqpoint{4.299895in}{2.420099in}}%
\pgfpathlineto{\pgfqpoint{4.341811in}{2.425355in}}%
\pgfpathlineto{\pgfqpoint{4.383726in}{2.430354in}}%
\pgfpathlineto{\pgfqpoint{4.425641in}{2.435109in}}%
\pgfpathlineto{\pgfqpoint{4.467556in}{2.439633in}}%
\pgfpathlineto{\pgfqpoint{4.509471in}{2.443936in}}%
\pgfpathlineto{\pgfqpoint{4.551386in}{2.448029in}}%
\pgfpathlineto{\pgfqpoint{4.593301in}{2.451922in}}%
\pgfpathlineto{\pgfqpoint{4.635216in}{2.455626in}}%
\pgfpathlineto{\pgfqpoint{4.677131in}{2.459148in}}%
\pgfpathlineto{\pgfqpoint{4.719046in}{2.462500in}}%
\pgfpathlineto{\pgfqpoint{4.760961in}{2.465687in}}%
\pgfpathlineto{\pgfqpoint{4.802877in}{2.468719in}}%
\pgfpathlineto{\pgfqpoint{4.844792in}{2.471604in}}%
\pgfpathlineto{\pgfqpoint{4.886707in}{2.474347in}}%
\pgfpathlineto{\pgfqpoint{4.928622in}{2.476957in}}%
\pgfpathlineto{\pgfqpoint{4.970537in}{2.479440in}}%
\pgfusepath{stroke}%
\end{pgfscope}%
\begin{pgfscope}%
\pgfpathrectangle{\pgfqpoint{0.779028in}{0.582778in}}{\pgfqpoint{4.191509in}{2.431352in}}%
\pgfusepath{clip}%
\pgfsetbuttcap%
\pgfsetroundjoin%
\pgfsetlinewidth{2.007500pt}%
\definecolor{currentstroke}{rgb}{1.000000,0.501961,0.054902}%
\pgfsetstrokecolor{currentstroke}%
\pgfsetstrokeopacity{0.700000}%
\pgfsetdash{{2.000000pt}{3.300000pt}}{0.000000pt}%
\pgfpathmoveto{\pgfqpoint{0.779028in}{1.555319in}}%
\pgfpathlineto{\pgfqpoint{2.455631in}{1.555319in}}%
\pgfpathlineto{\pgfqpoint{2.455673in}{2.527860in}}%
\pgfpathlineto{\pgfqpoint{4.970537in}{2.527860in}}%
\pgfusepath{stroke}%
\end{pgfscope}%
\begin{pgfscope}%
\pgfpathrectangle{\pgfqpoint{0.779028in}{0.582778in}}{\pgfqpoint{4.191509in}{2.431352in}}%
\pgfusepath{clip}%
\pgfsetrectcap%
\pgfsetroundjoin%
\pgfsetlinewidth{3.011250pt}%
\definecolor{currentstroke}{rgb}{0.349020,0.349020,0.349020}%
\pgfsetstrokecolor{currentstroke}%
\pgfsetdash{}{0pt}%
\pgfpathmoveto{\pgfqpoint{0.779028in}{1.555319in}}%
\pgfpathlineto{\pgfqpoint{0.820943in}{1.555319in}}%
\pgfpathlineto{\pgfqpoint{0.862858in}{1.555319in}}%
\pgfpathlineto{\pgfqpoint{0.904773in}{1.555319in}}%
\pgfpathlineto{\pgfqpoint{0.946688in}{1.555319in}}%
\pgfpathlineto{\pgfqpoint{0.988603in}{1.555319in}}%
\pgfpathlineto{\pgfqpoint{1.030518in}{1.555319in}}%
\pgfpathlineto{\pgfqpoint{1.072433in}{1.555319in}}%
\pgfpathlineto{\pgfqpoint{1.114349in}{1.555319in}}%
\pgfpathlineto{\pgfqpoint{1.156264in}{1.555319in}}%
\pgfpathlineto{\pgfqpoint{1.198179in}{1.555319in}}%
\pgfpathlineto{\pgfqpoint{1.240094in}{1.555319in}}%
\pgfpathlineto{\pgfqpoint{1.282009in}{1.555319in}}%
\pgfpathlineto{\pgfqpoint{1.323924in}{1.555319in}}%
\pgfpathlineto{\pgfqpoint{1.365839in}{1.555319in}}%
\pgfpathlineto{\pgfqpoint{1.407754in}{1.555319in}}%
\pgfpathlineto{\pgfqpoint{1.449669in}{1.555319in}}%
\pgfpathlineto{\pgfqpoint{1.491584in}{1.555319in}}%
\pgfpathlineto{\pgfqpoint{1.533499in}{1.555319in}}%
\pgfpathlineto{\pgfqpoint{1.575415in}{1.555319in}}%
\pgfpathlineto{\pgfqpoint{1.617330in}{1.555319in}}%
\pgfpathlineto{\pgfqpoint{1.659245in}{1.555319in}}%
\pgfpathlineto{\pgfqpoint{1.701160in}{1.555319in}}%
\pgfpathlineto{\pgfqpoint{1.743075in}{1.555319in}}%
\pgfpathlineto{\pgfqpoint{1.784990in}{1.555319in}}%
\pgfpathlineto{\pgfqpoint{1.826905in}{1.555319in}}%
\pgfpathlineto{\pgfqpoint{1.868820in}{1.555319in}}%
\pgfpathlineto{\pgfqpoint{1.910735in}{1.555319in}}%
\pgfpathlineto{\pgfqpoint{1.952650in}{1.555319in}}%
\pgfpathlineto{\pgfqpoint{1.994565in}{1.555319in}}%
\pgfpathlineto{\pgfqpoint{2.036481in}{1.555319in}}%
\pgfpathlineto{\pgfqpoint{2.078396in}{1.555319in}}%
\pgfpathlineto{\pgfqpoint{2.120311in}{1.555319in}}%
\pgfpathlineto{\pgfqpoint{2.162226in}{1.555319in}}%
\pgfpathlineto{\pgfqpoint{2.204141in}{1.555319in}}%
\pgfpathlineto{\pgfqpoint{2.246056in}{1.555319in}}%
\pgfpathlineto{\pgfqpoint{2.287971in}{1.555319in}}%
\pgfpathlineto{\pgfqpoint{2.329886in}{1.555319in}}%
\pgfpathlineto{\pgfqpoint{2.371801in}{1.555319in}}%
\pgfpathlineto{\pgfqpoint{2.413716in}{1.555319in}}%
\pgfpathlineto{\pgfqpoint{2.455631in}{1.555319in}}%
\pgfpathlineto{\pgfqpoint{2.497547in}{1.602750in}}%
\pgfpathlineto{\pgfqpoint{2.539462in}{1.647868in}}%
\pgfpathlineto{\pgfqpoint{2.581377in}{1.690786in}}%
\pgfpathlineto{\pgfqpoint{2.623292in}{1.731610in}}%
\pgfpathlineto{\pgfqpoint{2.665207in}{1.770444in}}%
\pgfpathlineto{\pgfqpoint{2.707122in}{1.807384in}}%
\pgfpathlineto{\pgfqpoint{2.749037in}{1.842522in}}%
\pgfpathlineto{\pgfqpoint{2.790952in}{1.875946in}}%
\pgfpathlineto{\pgfqpoint{2.832867in}{1.907740in}}%
\pgfpathlineto{\pgfqpoint{2.874782in}{1.937984in}}%
\pgfpathlineto{\pgfqpoint{2.916697in}{1.966752in}}%
\pgfpathlineto{\pgfqpoint{2.958613in}{1.994118in}}%
\pgfpathlineto{\pgfqpoint{3.000528in}{2.020149in}}%
\pgfpathlineto{\pgfqpoint{3.042443in}{2.044910in}}%
\pgfpathlineto{\pgfqpoint{3.084358in}{2.068464in}}%
\pgfpathlineto{\pgfqpoint{3.126273in}{2.090869in}}%
\pgfpathlineto{\pgfqpoint{3.168188in}{2.112181in}}%
\pgfpathlineto{\pgfqpoint{3.210103in}{2.132454in}}%
\pgfpathlineto{\pgfqpoint{3.252018in}{2.151738in}}%
\pgfpathlineto{\pgfqpoint{3.293933in}{2.170082in}}%
\pgfpathlineto{\pgfqpoint{3.335848in}{2.187531in}}%
\pgfpathlineto{\pgfqpoint{3.377763in}{2.204129in}}%
\pgfpathlineto{\pgfqpoint{3.419679in}{2.219917in}}%
\pgfpathlineto{\pgfqpoint{3.461594in}{2.234936in}}%
\pgfpathlineto{\pgfqpoint{3.503509in}{2.249222in}}%
\pgfpathlineto{\pgfqpoint{3.545424in}{2.262811in}}%
\pgfpathlineto{\pgfqpoint{3.587339in}{2.275738in}}%
\pgfpathlineto{\pgfqpoint{3.629254in}{2.288034in}}%
\pgfpathlineto{\pgfqpoint{3.671169in}{2.299730in}}%
\pgfpathlineto{\pgfqpoint{3.713084in}{2.310856in}}%
\pgfpathlineto{\pgfqpoint{3.754999in}{2.321440in}}%
\pgfpathlineto{\pgfqpoint{3.796914in}{2.331507in}}%
\pgfpathlineto{\pgfqpoint{3.838829in}{2.341083in}}%
\pgfpathlineto{\pgfqpoint{3.880745in}{2.350192in}}%
\pgfpathlineto{\pgfqpoint{3.922660in}{2.358857in}}%
\pgfpathlineto{\pgfqpoint{3.964575in}{2.367099in}}%
\pgfpathlineto{\pgfqpoint{4.006490in}{2.374940in}}%
\pgfpathlineto{\pgfqpoint{4.048405in}{2.382398in}}%
\pgfpathlineto{\pgfqpoint{4.090320in}{2.389492in}}%
\pgfpathlineto{\pgfqpoint{4.132235in}{2.396240in}}%
\pgfpathlineto{\pgfqpoint{4.174150in}{2.402659in}}%
\pgfpathlineto{\pgfqpoint{4.216065in}{2.408765in}}%
\pgfpathlineto{\pgfqpoint{4.257980in}{2.414574in}}%
\pgfpathlineto{\pgfqpoint{4.299895in}{2.420099in}}%
\pgfpathlineto{\pgfqpoint{4.341811in}{2.425354in}}%
\pgfpathlineto{\pgfqpoint{4.383726in}{2.430354in}}%
\pgfpathlineto{\pgfqpoint{4.425641in}{2.435109in}}%
\pgfpathlineto{\pgfqpoint{4.467556in}{2.439632in}}%
\pgfpathlineto{\pgfqpoint{4.509471in}{2.443935in}}%
\pgfpathlineto{\pgfqpoint{4.551386in}{2.448028in}}%
\pgfpathlineto{\pgfqpoint{4.593301in}{2.451922in}}%
\pgfpathlineto{\pgfqpoint{4.635216in}{2.455625in}}%
\pgfpathlineto{\pgfqpoint{4.677131in}{2.459148in}}%
\pgfpathlineto{\pgfqpoint{4.719046in}{2.462499in}}%
\pgfpathlineto{\pgfqpoint{4.760961in}{2.465687in}}%
\pgfpathlineto{\pgfqpoint{4.802877in}{2.468719in}}%
\pgfpathlineto{\pgfqpoint{4.844792in}{2.471604in}}%
\pgfpathlineto{\pgfqpoint{4.886707in}{2.474347in}}%
\pgfpathlineto{\pgfqpoint{4.928622in}{2.476957in}}%
\pgfpathlineto{\pgfqpoint{4.970537in}{2.479440in}}%
\pgfusepath{stroke}%
\end{pgfscope}%
\begin{pgfscope}%
\pgfsetrectcap%
\pgfsetmiterjoin%
\pgfsetlinewidth{0.803000pt}%
\definecolor{currentstroke}{rgb}{0.000000,0.000000,0.000000}%
\pgfsetstrokecolor{currentstroke}%
\pgfsetdash{}{0pt}%
\pgfpathmoveto{\pgfqpoint{0.779028in}{0.582778in}}%
\pgfpathlineto{\pgfqpoint{0.779028in}{3.014130in}}%
\pgfusepath{stroke}%
\end{pgfscope}%
\begin{pgfscope}%
\pgfsetrectcap%
\pgfsetmiterjoin%
\pgfsetlinewidth{0.803000pt}%
\definecolor{currentstroke}{rgb}{0.000000,0.000000,0.000000}%
\pgfsetstrokecolor{currentstroke}%
\pgfsetdash{}{0pt}%
\pgfpathmoveto{\pgfqpoint{4.970537in}{0.582778in}}%
\pgfpathlineto{\pgfqpoint{4.970537in}{3.014130in}}%
\pgfusepath{stroke}%
\end{pgfscope}%
\begin{pgfscope}%
\pgfsetrectcap%
\pgfsetmiterjoin%
\pgfsetlinewidth{0.803000pt}%
\definecolor{currentstroke}{rgb}{0.000000,0.000000,0.000000}%
\pgfsetstrokecolor{currentstroke}%
\pgfsetdash{}{0pt}%
\pgfpathmoveto{\pgfqpoint{0.779028in}{0.582778in}}%
\pgfpathlineto{\pgfqpoint{4.970537in}{0.582778in}}%
\pgfusepath{stroke}%
\end{pgfscope}%
\begin{pgfscope}%
\pgfsetrectcap%
\pgfsetmiterjoin%
\pgfsetlinewidth{0.803000pt}%
\definecolor{currentstroke}{rgb}{0.000000,0.000000,0.000000}%
\pgfsetstrokecolor{currentstroke}%
\pgfsetdash{}{0pt}%
\pgfpathmoveto{\pgfqpoint{0.779028in}{3.014130in}}%
\pgfpathlineto{\pgfqpoint{4.970537in}{3.014130in}}%
\pgfusepath{stroke}%
\end{pgfscope}%
\begin{pgfscope}%
\pgfsetbuttcap%
\pgfsetmiterjoin%
\definecolor{currentfill}{rgb}{1.000000,1.000000,1.000000}%
\pgfsetfillcolor{currentfill}%
\pgfsetfillopacity{0.800000}%
\pgfsetlinewidth{1.003750pt}%
\definecolor{currentstroke}{rgb}{0.800000,0.800000,0.800000}%
\pgfsetstrokecolor{currentstroke}%
\pgfsetstrokeopacity{0.800000}%
\pgfsetdash{}{0pt}%
\pgfpathmoveto{\pgfqpoint{0.876250in}{2.252837in}}%
\pgfpathlineto{\pgfqpoint{1.915452in}{2.252837in}}%
\pgfpathquadraticcurveto{\pgfqpoint{1.943230in}{2.252837in}}{\pgfqpoint{1.943230in}{2.280615in}}%
\pgfpathlineto{\pgfqpoint{1.943230in}{2.916908in}}%
\pgfpathquadraticcurveto{\pgfqpoint{1.943230in}{2.944686in}}{\pgfqpoint{1.915452in}{2.944686in}}%
\pgfpathlineto{\pgfqpoint{0.876250in}{2.944686in}}%
\pgfpathquadraticcurveto{\pgfqpoint{0.848472in}{2.944686in}}{\pgfqpoint{0.848472in}{2.916908in}}%
\pgfpathlineto{\pgfqpoint{0.848472in}{2.280615in}}%
\pgfpathquadraticcurveto{\pgfqpoint{0.848472in}{2.252837in}}{\pgfqpoint{0.876250in}{2.252837in}}%
\pgfpathlineto{\pgfqpoint{0.876250in}{2.252837in}}%
\pgfpathclose%
\pgfusepath{stroke,fill}%
\end{pgfscope}%
\begin{pgfscope}%
\pgfsetbuttcap%
\pgfsetroundjoin%
\pgfsetlinewidth{2.007500pt}%
\definecolor{currentstroke}{rgb}{0.000000,0.419608,0.643137}%
\pgfsetstrokecolor{currentstroke}%
\pgfsetstrokeopacity{0.700000}%
\pgfsetdash{{7.400000pt}{3.200000pt}}{0.000000pt}%
\pgfpathmoveto{\pgfqpoint{0.904028in}{2.805273in}}%
\pgfpathlineto{\pgfqpoint{1.042917in}{2.805273in}}%
\pgfpathlineto{\pgfqpoint{1.181806in}{2.805273in}}%
\pgfusepath{stroke}%
\end{pgfscope}%
\begin{pgfscope}%
\definecolor{textcolor}{rgb}{0.000000,0.000000,0.000000}%
\pgfsetstrokecolor{textcolor}%
\pgfsetfillcolor{textcolor}%
\pgftext[x=1.292917in,y=2.756661in,left,base]{\color{textcolor}\sffamily\fontsize{10.000000}{12.000000}\selectfont \(\displaystyle 1-e^{-\frac{t-\theta}{\tau} }\)}%
\end{pgfscope}%
\begin{pgfscope}%
\pgfsetbuttcap%
\pgfsetroundjoin%
\pgfsetlinewidth{2.007500pt}%
\definecolor{currentstroke}{rgb}{1.000000,0.501961,0.054902}%
\pgfsetstrokecolor{currentstroke}%
\pgfsetstrokeopacity{0.700000}%
\pgfsetdash{{2.000000pt}{3.300000pt}}{0.000000pt}%
\pgfpathmoveto{\pgfqpoint{0.904028in}{2.601415in}}%
\pgfpathlineto{\pgfqpoint{1.042917in}{2.601415in}}%
\pgfpathlineto{\pgfqpoint{1.181806in}{2.601415in}}%
\pgfusepath{stroke}%
\end{pgfscope}%
\begin{pgfscope}%
\definecolor{textcolor}{rgb}{0.000000,0.000000,0.000000}%
\pgfsetstrokecolor{textcolor}%
\pgfsetfillcolor{textcolor}%
\pgftext[x=1.292917in,y=2.552804in,left,base]{\color{textcolor}\sffamily\fontsize{10.000000}{12.000000}\selectfont \(\displaystyle H(t- \theta)\)}%
\end{pgfscope}%
\begin{pgfscope}%
\pgfsetrectcap%
\pgfsetroundjoin%
\pgfsetlinewidth{3.011250pt}%
\definecolor{currentstroke}{rgb}{0.349020,0.349020,0.349020}%
\pgfsetstrokecolor{currentstroke}%
\pgfsetdash{}{0pt}%
\pgfpathmoveto{\pgfqpoint{0.904028in}{2.391726in}}%
\pgfpathlineto{\pgfqpoint{1.042917in}{2.391726in}}%
\pgfpathlineto{\pgfqpoint{1.181806in}{2.391726in}}%
\pgfusepath{stroke}%
\end{pgfscope}%
\begin{pgfscope}%
\definecolor{textcolor}{rgb}{0.000000,0.000000,0.000000}%
\pgfsetstrokecolor{textcolor}%
\pgfsetfillcolor{textcolor}%
\pgftext[x=1.292917in,y=2.343115in,left,base]{\color{textcolor}\sffamily\fontsize{10.000000}{12.000000}\selectfont \(\displaystyle y(t)\)}%
\end{pgfscope}%
\end{pgfpicture}%
\makeatother%
\endgroup%

    \caption{Time domain plot of a first-order plus dead time model, showing induvidual components of the model and the composite function $y(t)$. Model parameters: $K= \Delta u = 1$, $\tau=2$, $\theta=4$.}
    \label{fig:fopdt}
\end{figure}

%\cite{pi_first_order_system}



% https://apmonitor.com/pdc/index.php/Main/FirstOrderPlusDeadTime

\subsection{PID tuning rules}

%\subsubsection{SIMC}
We use $\tau_c = \tau$ as suggested by \cite{simc,smic2} for “\textit{tightest possible subject to maintaining smooth control}“.

\section{Allan Deviation}
The Allan variance \cite{adev} $\sigma_y^2(\tau)$ is a two-sample variance and used as a measure of stability. The Allan deviation $\sigma_y(\tau)$ is the square root of the variance. Originally, the Allan variance was used to quantify the performance of oscillators, namely the frequency stability, but it can be used evaluate any quantity. In order to define the Allan variance, a few terms need to be defined first. A single measurement value of the time series $y(t)$ can be written as
\begin{equation}
    \bar y_k(t) = \frac 1 \tau \int_{t_{k}}^{t_{k}+\tau} y(t)\,dt .
\end{equation}
This is the $k$-th measurement with a measurement time or integration time $\tau$. The latter term is frequently used for DMMs. $t_k$ is the sampling inverval including the dead time $\theta$
\begin{equation}
    t_{k+1} = t_k + T
\end{equation}
with
\begin{equation}
    T = \tau + \theta .
\end{equation}

Using this, the standard deviation over $N$ sampled is defined as \cite{adev,psd_to_adev}
\begin{equation}
    \sigma_y^2(N,T,\tau) = \left\langle \frac{1}{N-1} \left(\sum _{n=0}^{N-1}\bar y_n^2(t)-\frac{1}{N}\left(\sum _{n=0}^{N-1} \bar y_n(t)\right)^2\right)\right\rangle
\end{equation}
The $\langle \; \rangle$ denotes the (infinite time) average over all measurands $y_k$. Hence for all $k$.

The Allan variance is a special case of this definition with zero dead-time ($\theta=0$) and only 2 samples:
\begin{align}
    \sigma_y^2(\tau) &= \sigma_y^2(N=2,T=\tau,\tau) \label{eqn:allan_coefficients}\\
    &= \left\langle \frac{\left(\bar y_{k+1} - \bar y_k \right)^2}{2} \right\rangle
\end{align}
In practice, no experiment can take an infinite number of samples, so typically the Allan variance is estimated using a number of samples $m$:
\begin{equation}
    \sigma_y^2(\tau) \approx \frac1 m \sum_{k=1}^m \frac{\left(\bar y_{k+1} - \bar y_{k} \right)^2}{2} \label{eqn:adev_estimator}
\end{equation}

It can be shown \cite{psd_to_adev}, that \ref{eqn:adev_estimator} is indeed more usefull than $\sigma_y^2(N\to\infty,T=\tau,\tau)$, becuase $\sigma_y^2(\tau)$ even for $m \to \infty$ converges for processes, that do not have a convergent $\sigma_y^2(N\to\infty,T=\tau,\tau)$.

Additionally, the Allan variance is mathematically related to the two-sided power spectral density $S_y(f)$ \cite{psd_to_adev}:
\begin{equation}
    \sigma_y^2(\tau) = 2 \int_0^\infty S_y(f) \frac{\sin^4\left( \pi f t \right)}{(\pi f \tau)^2}\,df
\end{equation}

and therefore all processes, that can be seen in the power spectral density can also be seen in the allan deviation. The inverse transform, however, is not always possible as shown by \citeauthor{inverse_adev} \cite{inverse_adev}.

Distinguishing different noise processes using the Allan deviation will be elaborated in the next section.

\subsection{Identifying Noise in Allan Deviation Plots}
It was already mentioned by \citeauthor{adev} in \cite{adev}, that types of noise, whose spectral density follows a power law
\begin{equation}
    S(f) = C \cdot f^\alpha
\end{equation}
can be easily identified in the Allan deviation plot. The most common power coefficients encountered in experimental data can be found in table \ref{tab:adev_alpha_mu} and warants a further discussion.
% TODO: put in (4) from Generation-Recombination Noise, Allan Variance, and Low-Frequency Gain Instabilities in Microwave Amplifiers

\begin{table}[ht]
    \centering
    \begin{tabular}{lccc}
        \toprule
        Amplitude noise type& Power-law coefficient $\alpha$& Allan deviation& $\sigma_y(N=2,T=\tau+\theta,\tau)$\\
        \midrule
            White noise & 0& $\propto \tau^{-1/2}$ \cite{adev_noise_types}& \\
            Flicker/Burst noise& -1& $\propto \tau^0$ \cite{adev_noise_types}& \\
            Random walk noise& -2& $\propto \tau^{1/2}$ \cite{adev_noise_types}& \\
            Drift & --& $\propto \tau^1$ \cite{adev_drift}& \\
        \bottomrule
    \end{tabular}
    \caption{Power law representations using the Allan variance.}
    \label{tab:adev_alpha_mu}
\end{table}

In the previous section in equation \ref{eqn:allan_coefficients}, the Allan deviation was defined as the two-sample variance without dead-time. The effect of dead-time will be discussed later. The more important aspect is the general shape of the Allan deviation of the different power laws and will be discussed first. Linear drift, which can not be considered noise, but deterministic process has a very characteristic shape in the Allan deviation plot and will therefore be discussed here as well.

\minisec{White Noise}
White noise is probably the most common type of noise found in measurement data. Johnson noise found in resistors, caused by the random fluctuation of the charge carriers, is one example of mostly white noise up to bandwidth of \qty{100}{\MHz}, from where on quantum corrections are required \cite{nist_johnson_noise}. Amplifiers also tend to have a white noise spectrum at higher frequencies. For these reasons, white noise typically makes up for a considerabe amount of noise in a measurement, unless one measures at very low frequencies. White noise is a series of uncorrelated random events and therefore characterised by a uniform power spectral density, which means there is the same power in a given bandwidth at all frequencies. Another one of its important and often used properties is, that the variance of two uncorrelated variables adds:
\begin{equation}
    \sigma_{x+y}^2  = \sigma_x^2 + \sigma_y^2 + \underbrace{2\,\mathrm{Cov}(x,y)}_{\text{uncorrelated} = 0}\ = \sigma_x^2 + \sigma_y^2
\end{equation}

This allows the simple addition of variances from different sources, but it must be stressed here, that this property is only valid for uncorrelated sources like white noise, although it is usually incorrectly applied to all measurements in disreagard of the dominant noise present. This Unfortunately obscures rather than clarifies the uncertainties involved.

In order to demonstrate the effect of white noise in Allan deviation plots, it was simulated using Python and the excellent AllanTools library \cite{allantools}. The noise generator chosen in the AllanTools library is based on the work of \citeauthor{noise_generation} \cite{noise_generation}. The full Python program code is published online \cite{}. For better comparison, all noise densities are normalized to give an Allan deviation of $\sigma_y(\tau_0)=1$.

Figure \ref{fig:white_noise_simulated} shows a sample of white noise in three different forms. Figure \ref{fig:white_noise_time} is the time series representation. From this sample, the power spectral density was calculated and is shown in figure \ref{fig:white_noise_psd}. The dashed line shows the expectation value of the power spectral density and the Allan deviation.

\begin{figure}[ht]
    \centering
    \begin{subfigure}{0.3\linewidth}
        \scalebox{0.75}{%
            %% Creator: Matplotlib, PGF backend
%%
%% To include the figure in your LaTeX document, write
%%   \input{<filename>.pgf}
%%
%% Make sure the required packages are loaded in your preamble
%%   \usepackage{pgf}
%%
%% Also ensure that all the required font packages are loaded; for instance,
%% the lmodern package is sometimes necessary when using math font.
%%   \usepackage{lmodern}
%%
%% Figures using additional raster images can only be included by \input if
%% they are in the same directory as the main LaTeX file. For loading figures
%% from other directories you can use the `import` package
%%   \usepackage{import}
%%
%% and then include the figures with
%%   \import{<path to file>}{<filename>.pgf}
%%
%% Matplotlib used the following preamble
%%   \usepackage{siunitx}
%%   \usepackage{fontspec}
%%
\begingroup%
\makeatletter%
\begin{pgfpicture}%
\pgfpathrectangle{\pgfpointorigin}{\pgfqpoint{2.390000in}{1.830000in}}%
\pgfusepath{use as bounding box, clip}%
\begin{pgfscope}%
\pgfsetbuttcap%
\pgfsetmiterjoin%
\definecolor{currentfill}{rgb}{1.000000,1.000000,1.000000}%
\pgfsetfillcolor{currentfill}%
\pgfsetlinewidth{0.000000pt}%
\definecolor{currentstroke}{rgb}{1.000000,1.000000,1.000000}%
\pgfsetstrokecolor{currentstroke}%
\pgfsetdash{}{0pt}%
\pgfpathmoveto{\pgfqpoint{0.000000in}{0.000000in}}%
\pgfpathlineto{\pgfqpoint{2.390000in}{0.000000in}}%
\pgfpathlineto{\pgfqpoint{2.390000in}{1.830000in}}%
\pgfpathlineto{\pgfqpoint{0.000000in}{1.830000in}}%
\pgfpathlineto{\pgfqpoint{0.000000in}{0.000000in}}%
\pgfpathclose%
\pgfusepath{fill}%
\end{pgfscope}%
\begin{pgfscope}%
\pgfsetbuttcap%
\pgfsetmiterjoin%
\definecolor{currentfill}{rgb}{1.000000,1.000000,1.000000}%
\pgfsetfillcolor{currentfill}%
\pgfsetlinewidth{0.000000pt}%
\definecolor{currentstroke}{rgb}{0.000000,0.000000,0.000000}%
\pgfsetstrokecolor{currentstroke}%
\pgfsetstrokeopacity{0.000000}%
\pgfsetdash{}{0pt}%
\pgfpathmoveto{\pgfqpoint{0.471688in}{0.416447in}}%
\pgfpathlineto{\pgfqpoint{2.348330in}{0.416447in}}%
\pgfpathlineto{\pgfqpoint{2.348330in}{1.773646in}}%
\pgfpathlineto{\pgfqpoint{0.471688in}{1.773646in}}%
\pgfpathlineto{\pgfqpoint{0.471688in}{0.416447in}}%
\pgfpathclose%
\pgfusepath{fill}%
\end{pgfscope}%
\begin{pgfscope}%
\pgfpathrectangle{\pgfqpoint{0.471688in}{0.416447in}}{\pgfqpoint{1.876642in}{1.357199in}}%
\pgfusepath{clip}%
\pgfsetrectcap%
\pgfsetroundjoin%
\pgfsetlinewidth{0.803000pt}%
\definecolor{currentstroke}{rgb}{0.450000,0.450000,0.450000}%
\pgfsetstrokecolor{currentstroke}%
\pgfsetdash{}{0pt}%
\pgfpathmoveto{\pgfqpoint{0.556989in}{0.416447in}}%
\pgfpathlineto{\pgfqpoint{0.556989in}{1.773646in}}%
\pgfusepath{stroke}%
\end{pgfscope}%
\begin{pgfscope}%
\pgfsetbuttcap%
\pgfsetroundjoin%
\definecolor{currentfill}{rgb}{0.000000,0.000000,0.000000}%
\pgfsetfillcolor{currentfill}%
\pgfsetlinewidth{0.803000pt}%
\definecolor{currentstroke}{rgb}{0.000000,0.000000,0.000000}%
\pgfsetstrokecolor{currentstroke}%
\pgfsetdash{}{0pt}%
\pgfsys@defobject{currentmarker}{\pgfqpoint{0.000000in}{-0.048611in}}{\pgfqpoint{0.000000in}{0.000000in}}{%
\pgfpathmoveto{\pgfqpoint{0.000000in}{0.000000in}}%
\pgfpathlineto{\pgfqpoint{0.000000in}{-0.048611in}}%
\pgfusepath{stroke,fill}%
}%
\begin{pgfscope}%
\pgfsys@transformshift{0.556989in}{0.416447in}%
\pgfsys@useobject{currentmarker}{}%
\end{pgfscope}%
\end{pgfscope}%
\begin{pgfscope}%
\definecolor{textcolor}{rgb}{0.000000,0.000000,0.000000}%
\pgfsetstrokecolor{textcolor}%
\pgfsetfillcolor{textcolor}%
\pgftext[x=0.556989in,y=0.319225in,,top]{\color{textcolor}\rmfamily\fontsize{8.000000}{9.600000}\selectfont \(\displaystyle {0}\)}%
\end{pgfscope}%
\begin{pgfscope}%
\pgfpathrectangle{\pgfqpoint{0.471688in}{0.416447in}}{\pgfqpoint{1.876642in}{1.357199in}}%
\pgfusepath{clip}%
\pgfsetrectcap%
\pgfsetroundjoin%
\pgfsetlinewidth{0.803000pt}%
\definecolor{currentstroke}{rgb}{0.450000,0.450000,0.450000}%
\pgfsetstrokecolor{currentstroke}%
\pgfsetdash{}{0pt}%
\pgfpathmoveto{\pgfqpoint{1.077695in}{0.416447in}}%
\pgfpathlineto{\pgfqpoint{1.077695in}{1.773646in}}%
\pgfusepath{stroke}%
\end{pgfscope}%
\begin{pgfscope}%
\pgfsetbuttcap%
\pgfsetroundjoin%
\definecolor{currentfill}{rgb}{0.000000,0.000000,0.000000}%
\pgfsetfillcolor{currentfill}%
\pgfsetlinewidth{0.803000pt}%
\definecolor{currentstroke}{rgb}{0.000000,0.000000,0.000000}%
\pgfsetstrokecolor{currentstroke}%
\pgfsetdash{}{0pt}%
\pgfsys@defobject{currentmarker}{\pgfqpoint{0.000000in}{-0.048611in}}{\pgfqpoint{0.000000in}{0.000000in}}{%
\pgfpathmoveto{\pgfqpoint{0.000000in}{0.000000in}}%
\pgfpathlineto{\pgfqpoint{0.000000in}{-0.048611in}}%
\pgfusepath{stroke,fill}%
}%
\begin{pgfscope}%
\pgfsys@transformshift{1.077695in}{0.416447in}%
\pgfsys@useobject{currentmarker}{}%
\end{pgfscope}%
\end{pgfscope}%
\begin{pgfscope}%
\definecolor{textcolor}{rgb}{0.000000,0.000000,0.000000}%
\pgfsetstrokecolor{textcolor}%
\pgfsetfillcolor{textcolor}%
\pgftext[x=1.077695in,y=0.319225in,,top]{\color{textcolor}\rmfamily\fontsize{8.000000}{9.600000}\selectfont \(\displaystyle {5000}\)}%
\end{pgfscope}%
\begin{pgfscope}%
\pgfpathrectangle{\pgfqpoint{0.471688in}{0.416447in}}{\pgfqpoint{1.876642in}{1.357199in}}%
\pgfusepath{clip}%
\pgfsetrectcap%
\pgfsetroundjoin%
\pgfsetlinewidth{0.803000pt}%
\definecolor{currentstroke}{rgb}{0.450000,0.450000,0.450000}%
\pgfsetstrokecolor{currentstroke}%
\pgfsetdash{}{0pt}%
\pgfpathmoveto{\pgfqpoint{1.598400in}{0.416447in}}%
\pgfpathlineto{\pgfqpoint{1.598400in}{1.773646in}}%
\pgfusepath{stroke}%
\end{pgfscope}%
\begin{pgfscope}%
\pgfsetbuttcap%
\pgfsetroundjoin%
\definecolor{currentfill}{rgb}{0.000000,0.000000,0.000000}%
\pgfsetfillcolor{currentfill}%
\pgfsetlinewidth{0.803000pt}%
\definecolor{currentstroke}{rgb}{0.000000,0.000000,0.000000}%
\pgfsetstrokecolor{currentstroke}%
\pgfsetdash{}{0pt}%
\pgfsys@defobject{currentmarker}{\pgfqpoint{0.000000in}{-0.048611in}}{\pgfqpoint{0.000000in}{0.000000in}}{%
\pgfpathmoveto{\pgfqpoint{0.000000in}{0.000000in}}%
\pgfpathlineto{\pgfqpoint{0.000000in}{-0.048611in}}%
\pgfusepath{stroke,fill}%
}%
\begin{pgfscope}%
\pgfsys@transformshift{1.598400in}{0.416447in}%
\pgfsys@useobject{currentmarker}{}%
\end{pgfscope}%
\end{pgfscope}%
\begin{pgfscope}%
\definecolor{textcolor}{rgb}{0.000000,0.000000,0.000000}%
\pgfsetstrokecolor{textcolor}%
\pgfsetfillcolor{textcolor}%
\pgftext[x=1.598400in,y=0.319225in,,top]{\color{textcolor}\rmfamily\fontsize{8.000000}{9.600000}\selectfont \(\displaystyle {10000}\)}%
\end{pgfscope}%
\begin{pgfscope}%
\pgfpathrectangle{\pgfqpoint{0.471688in}{0.416447in}}{\pgfqpoint{1.876642in}{1.357199in}}%
\pgfusepath{clip}%
\pgfsetrectcap%
\pgfsetroundjoin%
\pgfsetlinewidth{0.803000pt}%
\definecolor{currentstroke}{rgb}{0.450000,0.450000,0.450000}%
\pgfsetstrokecolor{currentstroke}%
\pgfsetdash{}{0pt}%
\pgfpathmoveto{\pgfqpoint{2.119105in}{0.416447in}}%
\pgfpathlineto{\pgfqpoint{2.119105in}{1.773646in}}%
\pgfusepath{stroke}%
\end{pgfscope}%
\begin{pgfscope}%
\pgfsetbuttcap%
\pgfsetroundjoin%
\definecolor{currentfill}{rgb}{0.000000,0.000000,0.000000}%
\pgfsetfillcolor{currentfill}%
\pgfsetlinewidth{0.803000pt}%
\definecolor{currentstroke}{rgb}{0.000000,0.000000,0.000000}%
\pgfsetstrokecolor{currentstroke}%
\pgfsetdash{}{0pt}%
\pgfsys@defobject{currentmarker}{\pgfqpoint{0.000000in}{-0.048611in}}{\pgfqpoint{0.000000in}{0.000000in}}{%
\pgfpathmoveto{\pgfqpoint{0.000000in}{0.000000in}}%
\pgfpathlineto{\pgfqpoint{0.000000in}{-0.048611in}}%
\pgfusepath{stroke,fill}%
}%
\begin{pgfscope}%
\pgfsys@transformshift{2.119105in}{0.416447in}%
\pgfsys@useobject{currentmarker}{}%
\end{pgfscope}%
\end{pgfscope}%
\begin{pgfscope}%
\definecolor{textcolor}{rgb}{0.000000,0.000000,0.000000}%
\pgfsetstrokecolor{textcolor}%
\pgfsetfillcolor{textcolor}%
\pgftext[x=2.119105in,y=0.319225in,,top]{\color{textcolor}\rmfamily\fontsize{8.000000}{9.600000}\selectfont \(\displaystyle {15000}\)}%
\end{pgfscope}%
\begin{pgfscope}%
\definecolor{textcolor}{rgb}{0.000000,0.000000,0.000000}%
\pgfsetstrokecolor{textcolor}%
\pgfsetfillcolor{textcolor}%
\pgftext[x=1.410009in,y=0.165003in,,top]{\color{textcolor}\rmfamily\fontsize{10.000000}{12.000000}\selectfont Time in \unit{s}}%
\end{pgfscope}%
\begin{pgfscope}%
\pgfpathrectangle{\pgfqpoint{0.471688in}{0.416447in}}{\pgfqpoint{1.876642in}{1.357199in}}%
\pgfusepath{clip}%
\pgfsetrectcap%
\pgfsetroundjoin%
\pgfsetlinewidth{0.803000pt}%
\definecolor{currentstroke}{rgb}{0.450000,0.450000,0.450000}%
\pgfsetstrokecolor{currentstroke}%
\pgfsetdash{}{0pt}%
\pgfpathmoveto{\pgfqpoint{0.471688in}{0.466742in}}%
\pgfpathlineto{\pgfqpoint{2.348330in}{0.466742in}}%
\pgfusepath{stroke}%
\end{pgfscope}%
\begin{pgfscope}%
\pgfsetbuttcap%
\pgfsetroundjoin%
\definecolor{currentfill}{rgb}{0.000000,0.000000,0.000000}%
\pgfsetfillcolor{currentfill}%
\pgfsetlinewidth{0.803000pt}%
\definecolor{currentstroke}{rgb}{0.000000,0.000000,0.000000}%
\pgfsetstrokecolor{currentstroke}%
\pgfsetdash{}{0pt}%
\pgfsys@defobject{currentmarker}{\pgfqpoint{-0.048611in}{0.000000in}}{\pgfqpoint{-0.000000in}{0.000000in}}{%
\pgfpathmoveto{\pgfqpoint{-0.000000in}{0.000000in}}%
\pgfpathlineto{\pgfqpoint{-0.048611in}{0.000000in}}%
\pgfusepath{stroke,fill}%
}%
\begin{pgfscope}%
\pgfsys@transformshift{0.471688in}{0.466742in}%
\pgfsys@useobject{currentmarker}{}%
\end{pgfscope}%
\end{pgfscope}%
\begin{pgfscope}%
\definecolor{textcolor}{rgb}{0.000000,0.000000,0.000000}%
\pgfsetstrokecolor{textcolor}%
\pgfsetfillcolor{textcolor}%
\pgftext[x=0.223614in, y=0.428187in, left, base]{\color{textcolor}\rmfamily\fontsize{8.000000}{9.600000}\selectfont \(\displaystyle {\ensuremath{-}4}\)}%
\end{pgfscope}%
\begin{pgfscope}%
\pgfpathrectangle{\pgfqpoint{0.471688in}{0.416447in}}{\pgfqpoint{1.876642in}{1.357199in}}%
\pgfusepath{clip}%
\pgfsetrectcap%
\pgfsetroundjoin%
\pgfsetlinewidth{0.803000pt}%
\definecolor{currentstroke}{rgb}{0.450000,0.450000,0.450000}%
\pgfsetstrokecolor{currentstroke}%
\pgfsetdash{}{0pt}%
\pgfpathmoveto{\pgfqpoint{0.471688in}{0.760456in}}%
\pgfpathlineto{\pgfqpoint{2.348330in}{0.760456in}}%
\pgfusepath{stroke}%
\end{pgfscope}%
\begin{pgfscope}%
\pgfsetbuttcap%
\pgfsetroundjoin%
\definecolor{currentfill}{rgb}{0.000000,0.000000,0.000000}%
\pgfsetfillcolor{currentfill}%
\pgfsetlinewidth{0.803000pt}%
\definecolor{currentstroke}{rgb}{0.000000,0.000000,0.000000}%
\pgfsetstrokecolor{currentstroke}%
\pgfsetdash{}{0pt}%
\pgfsys@defobject{currentmarker}{\pgfqpoint{-0.048611in}{0.000000in}}{\pgfqpoint{-0.000000in}{0.000000in}}{%
\pgfpathmoveto{\pgfqpoint{-0.000000in}{0.000000in}}%
\pgfpathlineto{\pgfqpoint{-0.048611in}{0.000000in}}%
\pgfusepath{stroke,fill}%
}%
\begin{pgfscope}%
\pgfsys@transformshift{0.471688in}{0.760456in}%
\pgfsys@useobject{currentmarker}{}%
\end{pgfscope}%
\end{pgfscope}%
\begin{pgfscope}%
\definecolor{textcolor}{rgb}{0.000000,0.000000,0.000000}%
\pgfsetstrokecolor{textcolor}%
\pgfsetfillcolor{textcolor}%
\pgftext[x=0.223614in, y=0.721901in, left, base]{\color{textcolor}\rmfamily\fontsize{8.000000}{9.600000}\selectfont \(\displaystyle {\ensuremath{-}2}\)}%
\end{pgfscope}%
\begin{pgfscope}%
\pgfpathrectangle{\pgfqpoint{0.471688in}{0.416447in}}{\pgfqpoint{1.876642in}{1.357199in}}%
\pgfusepath{clip}%
\pgfsetrectcap%
\pgfsetroundjoin%
\pgfsetlinewidth{0.803000pt}%
\definecolor{currentstroke}{rgb}{0.450000,0.450000,0.450000}%
\pgfsetstrokecolor{currentstroke}%
\pgfsetdash{}{0pt}%
\pgfpathmoveto{\pgfqpoint{0.471688in}{1.054170in}}%
\pgfpathlineto{\pgfqpoint{2.348330in}{1.054170in}}%
\pgfusepath{stroke}%
\end{pgfscope}%
\begin{pgfscope}%
\pgfsetbuttcap%
\pgfsetroundjoin%
\definecolor{currentfill}{rgb}{0.000000,0.000000,0.000000}%
\pgfsetfillcolor{currentfill}%
\pgfsetlinewidth{0.803000pt}%
\definecolor{currentstroke}{rgb}{0.000000,0.000000,0.000000}%
\pgfsetstrokecolor{currentstroke}%
\pgfsetdash{}{0pt}%
\pgfsys@defobject{currentmarker}{\pgfqpoint{-0.048611in}{0.000000in}}{\pgfqpoint{-0.000000in}{0.000000in}}{%
\pgfpathmoveto{\pgfqpoint{-0.000000in}{0.000000in}}%
\pgfpathlineto{\pgfqpoint{-0.048611in}{0.000000in}}%
\pgfusepath{stroke,fill}%
}%
\begin{pgfscope}%
\pgfsys@transformshift{0.471688in}{1.054170in}%
\pgfsys@useobject{currentmarker}{}%
\end{pgfscope}%
\end{pgfscope}%
\begin{pgfscope}%
\definecolor{textcolor}{rgb}{0.000000,0.000000,0.000000}%
\pgfsetstrokecolor{textcolor}%
\pgfsetfillcolor{textcolor}%
\pgftext[x=0.315437in, y=1.015615in, left, base]{\color{textcolor}\rmfamily\fontsize{8.000000}{9.600000}\selectfont \(\displaystyle {0}\)}%
\end{pgfscope}%
\begin{pgfscope}%
\pgfpathrectangle{\pgfqpoint{0.471688in}{0.416447in}}{\pgfqpoint{1.876642in}{1.357199in}}%
\pgfusepath{clip}%
\pgfsetrectcap%
\pgfsetroundjoin%
\pgfsetlinewidth{0.803000pt}%
\definecolor{currentstroke}{rgb}{0.450000,0.450000,0.450000}%
\pgfsetstrokecolor{currentstroke}%
\pgfsetdash{}{0pt}%
\pgfpathmoveto{\pgfqpoint{0.471688in}{1.347884in}}%
\pgfpathlineto{\pgfqpoint{2.348330in}{1.347884in}}%
\pgfusepath{stroke}%
\end{pgfscope}%
\begin{pgfscope}%
\pgfsetbuttcap%
\pgfsetroundjoin%
\definecolor{currentfill}{rgb}{0.000000,0.000000,0.000000}%
\pgfsetfillcolor{currentfill}%
\pgfsetlinewidth{0.803000pt}%
\definecolor{currentstroke}{rgb}{0.000000,0.000000,0.000000}%
\pgfsetstrokecolor{currentstroke}%
\pgfsetdash{}{0pt}%
\pgfsys@defobject{currentmarker}{\pgfqpoint{-0.048611in}{0.000000in}}{\pgfqpoint{-0.000000in}{0.000000in}}{%
\pgfpathmoveto{\pgfqpoint{-0.000000in}{0.000000in}}%
\pgfpathlineto{\pgfqpoint{-0.048611in}{0.000000in}}%
\pgfusepath{stroke,fill}%
}%
\begin{pgfscope}%
\pgfsys@transformshift{0.471688in}{1.347884in}%
\pgfsys@useobject{currentmarker}{}%
\end{pgfscope}%
\end{pgfscope}%
\begin{pgfscope}%
\definecolor{textcolor}{rgb}{0.000000,0.000000,0.000000}%
\pgfsetstrokecolor{textcolor}%
\pgfsetfillcolor{textcolor}%
\pgftext[x=0.315437in, y=1.309329in, left, base]{\color{textcolor}\rmfamily\fontsize{8.000000}{9.600000}\selectfont \(\displaystyle {2}\)}%
\end{pgfscope}%
\begin{pgfscope}%
\pgfpathrectangle{\pgfqpoint{0.471688in}{0.416447in}}{\pgfqpoint{1.876642in}{1.357199in}}%
\pgfusepath{clip}%
\pgfsetrectcap%
\pgfsetroundjoin%
\pgfsetlinewidth{0.803000pt}%
\definecolor{currentstroke}{rgb}{0.450000,0.450000,0.450000}%
\pgfsetstrokecolor{currentstroke}%
\pgfsetdash{}{0pt}%
\pgfpathmoveto{\pgfqpoint{0.471688in}{1.641598in}}%
\pgfpathlineto{\pgfqpoint{2.348330in}{1.641598in}}%
\pgfusepath{stroke}%
\end{pgfscope}%
\begin{pgfscope}%
\pgfsetbuttcap%
\pgfsetroundjoin%
\definecolor{currentfill}{rgb}{0.000000,0.000000,0.000000}%
\pgfsetfillcolor{currentfill}%
\pgfsetlinewidth{0.803000pt}%
\definecolor{currentstroke}{rgb}{0.000000,0.000000,0.000000}%
\pgfsetstrokecolor{currentstroke}%
\pgfsetdash{}{0pt}%
\pgfsys@defobject{currentmarker}{\pgfqpoint{-0.048611in}{0.000000in}}{\pgfqpoint{-0.000000in}{0.000000in}}{%
\pgfpathmoveto{\pgfqpoint{-0.000000in}{0.000000in}}%
\pgfpathlineto{\pgfqpoint{-0.048611in}{0.000000in}}%
\pgfusepath{stroke,fill}%
}%
\begin{pgfscope}%
\pgfsys@transformshift{0.471688in}{1.641598in}%
\pgfsys@useobject{currentmarker}{}%
\end{pgfscope}%
\end{pgfscope}%
\begin{pgfscope}%
\definecolor{textcolor}{rgb}{0.000000,0.000000,0.000000}%
\pgfsetstrokecolor{textcolor}%
\pgfsetfillcolor{textcolor}%
\pgftext[x=0.315437in, y=1.603043in, left, base]{\color{textcolor}\rmfamily\fontsize{8.000000}{9.600000}\selectfont \(\displaystyle {4}\)}%
\end{pgfscope}%
\begin{pgfscope}%
\definecolor{textcolor}{rgb}{0.000000,0.000000,0.000000}%
\pgfsetstrokecolor{textcolor}%
\pgfsetfillcolor{textcolor}%
\pgftext[x=0.168059in,y=1.095047in,,bottom,rotate=90.000000]{\color{textcolor}\rmfamily\fontsize{10.000000}{12.000000}\selectfont Amplitude in arb. unit}%
\end{pgfscope}%
\begin{pgfscope}%
\pgfpathrectangle{\pgfqpoint{0.471688in}{0.416447in}}{\pgfqpoint{1.876642in}{1.357199in}}%
\pgfusepath{clip}%
\pgfsetrectcap%
\pgfsetroundjoin%
\pgfsetlinewidth{1.505625pt}%
\definecolor{currentstroke}{rgb}{0.000000,0.447059,0.698039}%
\pgfsetstrokecolor{currentstroke}%
\pgfsetdash{}{0pt}%
\pgfpathmoveto{\pgfqpoint{0.556989in}{1.033865in}}%
\pgfpathlineto{\pgfqpoint{0.557510in}{1.286089in}}%
\pgfpathlineto{\pgfqpoint{0.557718in}{0.985225in}}%
\pgfpathlineto{\pgfqpoint{0.558135in}{1.089704in}}%
\pgfpathlineto{\pgfqpoint{0.558239in}{0.773192in}}%
\pgfpathlineto{\pgfqpoint{0.558968in}{1.269411in}}%
\pgfpathlineto{\pgfqpoint{0.559281in}{0.844936in}}%
\pgfpathlineto{\pgfqpoint{0.560114in}{1.326190in}}%
\pgfpathlineto{\pgfqpoint{0.560426in}{1.174967in}}%
\pgfpathlineto{\pgfqpoint{0.560739in}{0.766379in}}%
\pgfpathlineto{\pgfqpoint{0.561572in}{0.948456in}}%
\pgfpathlineto{\pgfqpoint{0.561780in}{1.209416in}}%
\pgfpathlineto{\pgfqpoint{0.561988in}{0.795256in}}%
\pgfpathlineto{\pgfqpoint{0.562613in}{1.190935in}}%
\pgfpathlineto{\pgfqpoint{0.563446in}{0.878499in}}%
\pgfpathlineto{\pgfqpoint{0.563654in}{1.253344in}}%
\pgfpathlineto{\pgfqpoint{0.563759in}{1.043595in}}%
\pgfpathlineto{\pgfqpoint{0.564488in}{1.283949in}}%
\pgfpathlineto{\pgfqpoint{0.564592in}{0.669442in}}%
\pgfpathlineto{\pgfqpoint{0.564800in}{1.066954in}}%
\pgfpathlineto{\pgfqpoint{0.565112in}{0.762282in}}%
\pgfpathlineto{\pgfqpoint{0.565425in}{1.271209in}}%
\pgfpathlineto{\pgfqpoint{0.565737in}{0.980484in}}%
\pgfpathlineto{\pgfqpoint{0.566362in}{1.196423in}}%
\pgfpathlineto{\pgfqpoint{0.566466in}{0.951069in}}%
\pgfpathlineto{\pgfqpoint{0.566675in}{0.996587in}}%
\pgfpathlineto{\pgfqpoint{0.566779in}{0.839243in}}%
\pgfpathlineto{\pgfqpoint{0.566883in}{1.097658in}}%
\pgfpathlineto{\pgfqpoint{0.567716in}{1.030484in}}%
\pgfpathlineto{\pgfqpoint{0.568653in}{1.415915in}}%
\pgfpathlineto{\pgfqpoint{0.568341in}{0.772385in}}%
\pgfpathlineto{\pgfqpoint{0.568862in}{1.098455in}}%
\pgfpathlineto{\pgfqpoint{0.569695in}{0.848299in}}%
\pgfpathlineto{\pgfqpoint{0.569590in}{1.260181in}}%
\pgfpathlineto{\pgfqpoint{0.569799in}{1.140354in}}%
\pgfpathlineto{\pgfqpoint{0.569903in}{1.375854in}}%
\pgfpathlineto{\pgfqpoint{0.570424in}{0.826445in}}%
\pgfpathlineto{\pgfqpoint{0.570736in}{1.123721in}}%
\pgfpathlineto{\pgfqpoint{0.571673in}{0.818100in}}%
\pgfpathlineto{\pgfqpoint{0.570944in}{1.281789in}}%
\pgfpathlineto{\pgfqpoint{0.571777in}{1.081285in}}%
\pgfpathlineto{\pgfqpoint{0.571986in}{1.168987in}}%
\pgfpathlineto{\pgfqpoint{0.572194in}{0.860252in}}%
\pgfpathlineto{\pgfqpoint{0.573027in}{0.949263in}}%
\pgfpathlineto{\pgfqpoint{0.573131in}{1.328172in}}%
\pgfpathlineto{\pgfqpoint{0.573340in}{0.879219in}}%
\pgfpathlineto{\pgfqpoint{0.574173in}{1.174896in}}%
\pgfpathlineto{\pgfqpoint{0.574277in}{1.332728in}}%
\pgfpathlineto{\pgfqpoint{0.574589in}{0.923539in}}%
\pgfpathlineto{\pgfqpoint{0.575110in}{1.175648in}}%
\pgfpathlineto{\pgfqpoint{0.575839in}{0.896902in}}%
\pgfpathlineto{\pgfqpoint{0.575527in}{1.453646in}}%
\pgfpathlineto{\pgfqpoint{0.576256in}{1.123669in}}%
\pgfpathlineto{\pgfqpoint{0.576568in}{0.831704in}}%
\pgfpathlineto{\pgfqpoint{0.576776in}{1.179939in}}%
\pgfpathlineto{\pgfqpoint{0.577401in}{1.076746in}}%
\pgfpathlineto{\pgfqpoint{0.577505in}{1.062719in}}%
\pgfpathlineto{\pgfqpoint{0.578130in}{0.851850in}}%
\pgfpathlineto{\pgfqpoint{0.577922in}{1.213224in}}%
\pgfpathlineto{\pgfqpoint{0.578547in}{1.129809in}}%
\pgfpathlineto{\pgfqpoint{0.578651in}{1.619971in}}%
\pgfpathlineto{\pgfqpoint{0.579380in}{0.940676in}}%
\pgfpathlineto{\pgfqpoint{0.579588in}{0.982891in}}%
\pgfpathlineto{\pgfqpoint{0.579796in}{1.394094in}}%
\pgfpathlineto{\pgfqpoint{0.579900in}{0.779949in}}%
\pgfpathlineto{\pgfqpoint{0.580734in}{1.153974in}}%
\pgfpathlineto{\pgfqpoint{0.581463in}{0.756764in}}%
\pgfpathlineto{\pgfqpoint{0.581254in}{1.369024in}}%
\pgfpathlineto{\pgfqpoint{0.581879in}{0.937783in}}%
\pgfpathlineto{\pgfqpoint{0.582712in}{1.313440in}}%
\pgfpathlineto{\pgfqpoint{0.582296in}{0.877898in}}%
\pgfpathlineto{\pgfqpoint{0.582816in}{1.113645in}}%
\pgfpathlineto{\pgfqpoint{0.583337in}{0.831040in}}%
\pgfpathlineto{\pgfqpoint{0.583129in}{1.365824in}}%
\pgfpathlineto{\pgfqpoint{0.583962in}{0.918044in}}%
\pgfpathlineto{\pgfqpoint{0.584587in}{1.293901in}}%
\pgfpathlineto{\pgfqpoint{0.584170in}{0.578167in}}%
\pgfpathlineto{\pgfqpoint{0.585003in}{1.265831in}}%
\pgfpathlineto{\pgfqpoint{0.585108in}{0.843304in}}%
\pgfpathlineto{\pgfqpoint{0.586045in}{1.070841in}}%
\pgfpathlineto{\pgfqpoint{0.586461in}{1.367421in}}%
\pgfpathlineto{\pgfqpoint{0.586565in}{0.767493in}}%
\pgfpathlineto{\pgfqpoint{0.587086in}{1.023606in}}%
\pgfpathlineto{\pgfqpoint{0.587607in}{0.952412in}}%
\pgfpathlineto{\pgfqpoint{0.587711in}{1.186283in}}%
\pgfpathlineto{\pgfqpoint{0.587815in}{1.099299in}}%
\pgfpathlineto{\pgfqpoint{0.588128in}{0.932427in}}%
\pgfpathlineto{\pgfqpoint{0.588752in}{1.241804in}}%
\pgfpathlineto{\pgfqpoint{0.588857in}{0.967294in}}%
\pgfpathlineto{\pgfqpoint{0.589586in}{1.245889in}}%
\pgfpathlineto{\pgfqpoint{0.589898in}{1.008606in}}%
\pgfpathlineto{\pgfqpoint{0.590523in}{1.361452in}}%
\pgfpathlineto{\pgfqpoint{0.590731in}{0.875858in}}%
\pgfpathlineto{\pgfqpoint{0.591148in}{1.146447in}}%
\pgfpathlineto{\pgfqpoint{0.591356in}{0.922402in}}%
\pgfpathlineto{\pgfqpoint{0.591668in}{1.197374in}}%
\pgfpathlineto{\pgfqpoint{0.592293in}{0.933422in}}%
\pgfpathlineto{\pgfqpoint{0.593439in}{1.270837in}}%
\pgfpathlineto{\pgfqpoint{0.592918in}{0.841509in}}%
\pgfpathlineto{\pgfqpoint{0.593543in}{1.180124in}}%
\pgfpathlineto{\pgfqpoint{0.593855in}{0.906942in}}%
\pgfpathlineto{\pgfqpoint{0.594480in}{1.279264in}}%
\pgfpathlineto{\pgfqpoint{0.594688in}{1.113165in}}%
\pgfpathlineto{\pgfqpoint{0.595313in}{0.940648in}}%
\pgfpathlineto{\pgfqpoint{0.595834in}{1.370380in}}%
\pgfpathlineto{\pgfqpoint{0.596667in}{0.742261in}}%
\pgfpathlineto{\pgfqpoint{0.596251in}{1.375758in}}%
\pgfpathlineto{\pgfqpoint{0.596980in}{1.076257in}}%
\pgfpathlineto{\pgfqpoint{0.597188in}{1.329699in}}%
\pgfpathlineto{\pgfqpoint{0.597500in}{0.922232in}}%
\pgfpathlineto{\pgfqpoint{0.597604in}{1.126412in}}%
\pgfpathlineto{\pgfqpoint{0.598125in}{0.802585in}}%
\pgfpathlineto{\pgfqpoint{0.597813in}{1.323133in}}%
\pgfpathlineto{\pgfqpoint{0.598646in}{0.966148in}}%
\pgfpathlineto{\pgfqpoint{0.599167in}{0.897383in}}%
\pgfpathlineto{\pgfqpoint{0.599583in}{1.158676in}}%
\pgfpathlineto{\pgfqpoint{0.599791in}{0.828875in}}%
\pgfpathlineto{\pgfqpoint{0.600208in}{1.281968in}}%
\pgfpathlineto{\pgfqpoint{0.600520in}{1.064086in}}%
\pgfpathlineto{\pgfqpoint{0.600625in}{1.356806in}}%
\pgfpathlineto{\pgfqpoint{0.601249in}{0.912465in}}%
\pgfpathlineto{\pgfqpoint{0.601458in}{1.209607in}}%
\pgfpathlineto{\pgfqpoint{0.601770in}{0.754695in}}%
\pgfpathlineto{\pgfqpoint{0.602291in}{1.293344in}}%
\pgfpathlineto{\pgfqpoint{0.602603in}{1.046013in}}%
\pgfpathlineto{\pgfqpoint{0.603228in}{1.152516in}}%
\pgfpathlineto{\pgfqpoint{0.602916in}{0.750552in}}%
\pgfpathlineto{\pgfqpoint{0.603540in}{0.978705in}}%
\pgfpathlineto{\pgfqpoint{0.603853in}{1.194440in}}%
\pgfpathlineto{\pgfqpoint{0.604686in}{0.878254in}}%
\pgfpathlineto{\pgfqpoint{0.604790in}{1.342704in}}%
\pgfpathlineto{\pgfqpoint{0.605727in}{0.969607in}}%
\pgfpathlineto{\pgfqpoint{0.606665in}{1.506326in}}%
\pgfpathlineto{\pgfqpoint{0.605936in}{0.716117in}}%
\pgfpathlineto{\pgfqpoint{0.606873in}{1.035385in}}%
\pgfpathlineto{\pgfqpoint{0.607081in}{0.818252in}}%
\pgfpathlineto{\pgfqpoint{0.607706in}{1.301939in}}%
\pgfpathlineto{\pgfqpoint{0.607914in}{1.053000in}}%
\pgfpathlineto{\pgfqpoint{0.608852in}{0.851097in}}%
\pgfpathlineto{\pgfqpoint{0.609060in}{1.334582in}}%
\pgfpathlineto{\pgfqpoint{0.609164in}{0.848781in}}%
\pgfpathlineto{\pgfqpoint{0.610206in}{1.019079in}}%
\pgfpathlineto{\pgfqpoint{0.611039in}{0.809006in}}%
\pgfpathlineto{\pgfqpoint{0.610518in}{1.165105in}}%
\pgfpathlineto{\pgfqpoint{0.611247in}{0.956859in}}%
\pgfpathlineto{\pgfqpoint{0.612080in}{1.147952in}}%
\pgfpathlineto{\pgfqpoint{0.611559in}{0.789111in}}%
\pgfpathlineto{\pgfqpoint{0.612288in}{1.044466in}}%
\pgfpathlineto{\pgfqpoint{0.612392in}{0.876324in}}%
\pgfpathlineto{\pgfqpoint{0.612913in}{1.201952in}}%
\pgfpathlineto{\pgfqpoint{0.613330in}{1.131976in}}%
\pgfpathlineto{\pgfqpoint{0.613434in}{1.265875in}}%
\pgfpathlineto{\pgfqpoint{0.613538in}{0.691192in}}%
\pgfpathlineto{\pgfqpoint{0.614371in}{1.225667in}}%
\pgfpathlineto{\pgfqpoint{0.614788in}{0.982562in}}%
\pgfpathlineto{\pgfqpoint{0.615308in}{1.358958in}}%
\pgfpathlineto{\pgfqpoint{0.615517in}{1.006292in}}%
\pgfpathlineto{\pgfqpoint{0.616350in}{1.300367in}}%
\pgfpathlineto{\pgfqpoint{0.615829in}{0.754857in}}%
\pgfpathlineto{\pgfqpoint{0.616662in}{1.175974in}}%
\pgfpathlineto{\pgfqpoint{0.616766in}{0.729450in}}%
\pgfpathlineto{\pgfqpoint{0.617600in}{1.387638in}}%
\pgfpathlineto{\pgfqpoint{0.617704in}{1.080879in}}%
\pgfpathlineto{\pgfqpoint{0.617808in}{1.090623in}}%
\pgfpathlineto{\pgfqpoint{0.618745in}{1.206540in}}%
\pgfpathlineto{\pgfqpoint{0.619057in}{0.910432in}}%
\pgfpathlineto{\pgfqpoint{0.619891in}{1.329791in}}%
\pgfpathlineto{\pgfqpoint{0.620099in}{0.871384in}}%
\pgfpathlineto{\pgfqpoint{0.620203in}{0.792953in}}%
\pgfpathlineto{\pgfqpoint{0.620307in}{1.273875in}}%
\pgfpathlineto{\pgfqpoint{0.620724in}{0.888884in}}%
\pgfpathlineto{\pgfqpoint{0.620828in}{1.413346in}}%
\pgfpathlineto{\pgfqpoint{0.621869in}{1.288141in}}%
\pgfpathlineto{\pgfqpoint{0.622494in}{0.805620in}}%
\pgfpathlineto{\pgfqpoint{0.623015in}{0.811143in}}%
\pgfpathlineto{\pgfqpoint{0.623640in}{1.299278in}}%
\pgfpathlineto{\pgfqpoint{0.624056in}{1.014119in}}%
\pgfpathlineto{\pgfqpoint{0.624160in}{0.658114in}}%
\pgfpathlineto{\pgfqpoint{0.624994in}{1.432086in}}%
\pgfpathlineto{\pgfqpoint{0.625098in}{1.062867in}}%
\pgfpathlineto{\pgfqpoint{0.625410in}{1.083261in}}%
\pgfpathlineto{\pgfqpoint{0.626035in}{0.949484in}}%
\pgfpathlineto{\pgfqpoint{0.626347in}{1.275042in}}%
\pgfpathlineto{\pgfqpoint{0.626452in}{0.664857in}}%
\pgfpathlineto{\pgfqpoint{0.626972in}{0.999622in}}%
\pgfpathlineto{\pgfqpoint{0.627076in}{0.847467in}}%
\pgfpathlineto{\pgfqpoint{0.627389in}{1.311504in}}%
\pgfpathlineto{\pgfqpoint{0.627909in}{0.977141in}}%
\pgfpathlineto{\pgfqpoint{0.628847in}{1.224426in}}%
\pgfpathlineto{\pgfqpoint{0.628118in}{0.874692in}}%
\pgfpathlineto{\pgfqpoint{0.629055in}{1.144625in}}%
\pgfpathlineto{\pgfqpoint{0.629159in}{1.141271in}}%
\pgfpathlineto{\pgfqpoint{0.629888in}{1.208225in}}%
\pgfpathlineto{\pgfqpoint{0.630201in}{0.825568in}}%
\pgfpathlineto{\pgfqpoint{0.630513in}{1.311874in}}%
\pgfpathlineto{\pgfqpoint{0.630617in}{0.748424in}}%
\pgfpathlineto{\pgfqpoint{0.631346in}{1.216198in}}%
\pgfpathlineto{\pgfqpoint{0.631971in}{0.803020in}}%
\pgfpathlineto{\pgfqpoint{0.632492in}{1.056878in}}%
\pgfpathlineto{\pgfqpoint{0.633012in}{0.910489in}}%
\pgfpathlineto{\pgfqpoint{0.633117in}{1.114125in}}%
\pgfpathlineto{\pgfqpoint{0.633950in}{0.783825in}}%
\pgfpathlineto{\pgfqpoint{0.633741in}{1.298757in}}%
\pgfpathlineto{\pgfqpoint{0.634158in}{0.962411in}}%
\pgfpathlineto{\pgfqpoint{0.634887in}{0.847592in}}%
\pgfpathlineto{\pgfqpoint{0.635303in}{1.208905in}}%
\pgfpathlineto{\pgfqpoint{0.635408in}{0.914744in}}%
\pgfpathlineto{\pgfqpoint{0.635512in}{1.440754in}}%
\pgfpathlineto{\pgfqpoint{0.636345in}{1.040063in}}%
\pgfpathlineto{\pgfqpoint{0.636970in}{1.320341in}}%
\pgfpathlineto{\pgfqpoint{0.636553in}{0.950904in}}%
\pgfpathlineto{\pgfqpoint{0.637490in}{1.145488in}}%
\pgfpathlineto{\pgfqpoint{0.637595in}{0.823572in}}%
\pgfpathlineto{\pgfqpoint{0.638532in}{1.136380in}}%
\pgfpathlineto{\pgfqpoint{0.638636in}{1.212891in}}%
\pgfpathlineto{\pgfqpoint{0.639053in}{0.810247in}}%
\pgfpathlineto{\pgfqpoint{0.639365in}{1.094054in}}%
\pgfpathlineto{\pgfqpoint{0.639469in}{0.866671in}}%
\pgfpathlineto{\pgfqpoint{0.639677in}{1.208833in}}%
\pgfpathlineto{\pgfqpoint{0.640511in}{0.986282in}}%
\pgfpathlineto{\pgfqpoint{0.640719in}{1.008766in}}%
\pgfpathlineto{\pgfqpoint{0.641448in}{1.266669in}}%
\pgfpathlineto{\pgfqpoint{0.641135in}{0.922791in}}%
\pgfpathlineto{\pgfqpoint{0.641656in}{1.205704in}}%
\pgfpathlineto{\pgfqpoint{0.641760in}{0.836005in}}%
\pgfpathlineto{\pgfqpoint{0.642489in}{1.391780in}}%
\pgfpathlineto{\pgfqpoint{0.642698in}{0.988745in}}%
\pgfpathlineto{\pgfqpoint{0.642906in}{1.286142in}}%
\pgfpathlineto{\pgfqpoint{0.643322in}{0.856728in}}%
\pgfpathlineto{\pgfqpoint{0.643843in}{1.088566in}}%
\pgfpathlineto{\pgfqpoint{0.643947in}{1.281872in}}%
\pgfpathlineto{\pgfqpoint{0.644572in}{0.889291in}}%
\pgfpathlineto{\pgfqpoint{0.644884in}{1.126490in}}%
\pgfpathlineto{\pgfqpoint{0.644989in}{1.096637in}}%
\pgfpathlineto{\pgfqpoint{0.645093in}{1.414748in}}%
\pgfpathlineto{\pgfqpoint{0.646030in}{0.931933in}}%
\pgfpathlineto{\pgfqpoint{0.646759in}{1.317908in}}%
\pgfpathlineto{\pgfqpoint{0.646551in}{0.831573in}}%
\pgfpathlineto{\pgfqpoint{0.647176in}{1.103288in}}%
\pgfpathlineto{\pgfqpoint{0.647384in}{1.349383in}}%
\pgfpathlineto{\pgfqpoint{0.647696in}{0.851608in}}%
\pgfpathlineto{\pgfqpoint{0.648009in}{1.317714in}}%
\pgfpathlineto{\pgfqpoint{0.648634in}{0.976209in}}%
\pgfpathlineto{\pgfqpoint{0.648529in}{1.425268in}}%
\pgfpathlineto{\pgfqpoint{0.649154in}{1.139924in}}%
\pgfpathlineto{\pgfqpoint{0.649258in}{1.001406in}}%
\pgfpathlineto{\pgfqpoint{0.649987in}{1.257150in}}%
\pgfpathlineto{\pgfqpoint{0.650300in}{1.075716in}}%
\pgfpathlineto{\pgfqpoint{0.651237in}{0.854466in}}%
\pgfpathlineto{\pgfqpoint{0.651550in}{1.371860in}}%
\pgfpathlineto{\pgfqpoint{0.652487in}{0.988608in}}%
\pgfpathlineto{\pgfqpoint{0.652695in}{1.211841in}}%
\pgfpathlineto{\pgfqpoint{0.653216in}{0.830480in}}%
\pgfpathlineto{\pgfqpoint{0.653528in}{1.284079in}}%
\pgfpathlineto{\pgfqpoint{0.653736in}{0.972635in}}%
\pgfpathlineto{\pgfqpoint{0.653841in}{1.330431in}}%
\pgfpathlineto{\pgfqpoint{0.654049in}{0.731260in}}%
\pgfpathlineto{\pgfqpoint{0.654882in}{1.182297in}}%
\pgfpathlineto{\pgfqpoint{0.655403in}{0.830136in}}%
\pgfpathlineto{\pgfqpoint{0.655194in}{1.325362in}}%
\pgfpathlineto{\pgfqpoint{0.656028in}{0.870373in}}%
\pgfpathlineto{\pgfqpoint{0.656132in}{1.266197in}}%
\pgfpathlineto{\pgfqpoint{0.657173in}{1.158293in}}%
\pgfpathlineto{\pgfqpoint{0.657590in}{1.318651in}}%
\pgfpathlineto{\pgfqpoint{0.658423in}{0.776678in}}%
\pgfpathlineto{\pgfqpoint{0.659568in}{1.274955in}}%
\pgfpathlineto{\pgfqpoint{0.658631in}{0.698207in}}%
\pgfpathlineto{\pgfqpoint{0.659673in}{1.183017in}}%
\pgfpathlineto{\pgfqpoint{0.660089in}{0.754327in}}%
\pgfpathlineto{\pgfqpoint{0.660610in}{1.318173in}}%
\pgfpathlineto{\pgfqpoint{0.660818in}{0.970289in}}%
\pgfpathlineto{\pgfqpoint{0.661026in}{1.259676in}}%
\pgfpathlineto{\pgfqpoint{0.661339in}{0.959163in}}%
\pgfpathlineto{\pgfqpoint{0.661859in}{1.208305in}}%
\pgfpathlineto{\pgfqpoint{0.662380in}{0.952958in}}%
\pgfpathlineto{\pgfqpoint{0.662276in}{1.358937in}}%
\pgfpathlineto{\pgfqpoint{0.663005in}{1.116462in}}%
\pgfpathlineto{\pgfqpoint{0.663109in}{1.130952in}}%
\pgfpathlineto{\pgfqpoint{0.663422in}{1.368778in}}%
\pgfpathlineto{\pgfqpoint{0.664255in}{0.809127in}}%
\pgfpathlineto{\pgfqpoint{0.665192in}{1.301467in}}%
\pgfpathlineto{\pgfqpoint{0.665400in}{1.212582in}}%
\pgfpathlineto{\pgfqpoint{0.665921in}{0.822666in}}%
\pgfpathlineto{\pgfqpoint{0.666025in}{1.271011in}}%
\pgfpathlineto{\pgfqpoint{0.666650in}{0.973354in}}%
\pgfpathlineto{\pgfqpoint{0.666754in}{0.955538in}}%
\pgfpathlineto{\pgfqpoint{0.666858in}{1.050418in}}%
\pgfpathlineto{\pgfqpoint{0.666962in}{1.226394in}}%
\pgfpathlineto{\pgfqpoint{0.667379in}{0.635842in}}%
\pgfpathlineto{\pgfqpoint{0.667796in}{0.889739in}}%
\pgfpathlineto{\pgfqpoint{0.667900in}{0.864037in}}%
\pgfpathlineto{\pgfqpoint{0.668004in}{1.224646in}}%
\pgfpathlineto{\pgfqpoint{0.668941in}{0.825575in}}%
\pgfpathlineto{\pgfqpoint{0.669045in}{0.991299in}}%
\pgfpathlineto{\pgfqpoint{0.669670in}{1.359155in}}%
\pgfpathlineto{\pgfqpoint{0.670087in}{1.089238in}}%
\pgfpathlineto{\pgfqpoint{0.670816in}{1.161764in}}%
\pgfpathlineto{\pgfqpoint{0.671128in}{0.760721in}}%
\pgfpathlineto{\pgfqpoint{0.671649in}{1.360863in}}%
\pgfpathlineto{\pgfqpoint{0.671545in}{0.628835in}}%
\pgfpathlineto{\pgfqpoint{0.672378in}{1.335259in}}%
\pgfpathlineto{\pgfqpoint{0.672482in}{1.026167in}}%
\pgfpathlineto{\pgfqpoint{0.673107in}{1.407129in}}%
\pgfpathlineto{\pgfqpoint{0.673523in}{1.216525in}}%
\pgfpathlineto{\pgfqpoint{0.673627in}{1.228494in}}%
\pgfpathlineto{\pgfqpoint{0.673732in}{1.147972in}}%
\pgfpathlineto{\pgfqpoint{0.674044in}{0.885821in}}%
\pgfpathlineto{\pgfqpoint{0.673940in}{1.294051in}}%
\pgfpathlineto{\pgfqpoint{0.674773in}{1.291155in}}%
\pgfpathlineto{\pgfqpoint{0.675294in}{0.873888in}}%
\pgfpathlineto{\pgfqpoint{0.676023in}{1.028754in}}%
\pgfpathlineto{\pgfqpoint{0.676439in}{1.288335in}}%
\pgfpathlineto{\pgfqpoint{0.676335in}{0.842689in}}%
\pgfpathlineto{\pgfqpoint{0.676647in}{0.908577in}}%
\pgfpathlineto{\pgfqpoint{0.677376in}{1.384956in}}%
\pgfpathlineto{\pgfqpoint{0.677689in}{0.687115in}}%
\pgfpathlineto{\pgfqpoint{0.678210in}{1.412465in}}%
\pgfpathlineto{\pgfqpoint{0.678834in}{1.095465in}}%
\pgfpathlineto{\pgfqpoint{0.679355in}{0.805548in}}%
\pgfpathlineto{\pgfqpoint{0.679772in}{1.080589in}}%
\pgfpathlineto{\pgfqpoint{0.679876in}{1.258596in}}%
\pgfpathlineto{\pgfqpoint{0.680084in}{0.823531in}}%
\pgfpathlineto{\pgfqpoint{0.680813in}{1.069645in}}%
\pgfpathlineto{\pgfqpoint{0.681230in}{0.820228in}}%
\pgfpathlineto{\pgfqpoint{0.681750in}{1.254504in}}%
\pgfpathlineto{\pgfqpoint{0.681959in}{0.991109in}}%
\pgfpathlineto{\pgfqpoint{0.682792in}{1.311108in}}%
\pgfpathlineto{\pgfqpoint{0.682688in}{0.848188in}}%
\pgfpathlineto{\pgfqpoint{0.683000in}{0.952412in}}%
\pgfpathlineto{\pgfqpoint{0.683104in}{0.948667in}}%
\pgfpathlineto{\pgfqpoint{0.683833in}{1.369201in}}%
\pgfpathlineto{\pgfqpoint{0.683937in}{0.879763in}}%
\pgfpathlineto{\pgfqpoint{0.684250in}{1.114940in}}%
\pgfpathlineto{\pgfqpoint{0.685083in}{0.853005in}}%
\pgfpathlineto{\pgfqpoint{0.684875in}{1.378801in}}%
\pgfpathlineto{\pgfqpoint{0.685187in}{1.041206in}}%
\pgfpathlineto{\pgfqpoint{0.685291in}{1.433019in}}%
\pgfpathlineto{\pgfqpoint{0.685395in}{0.936145in}}%
\pgfpathlineto{\pgfqpoint{0.686228in}{0.968746in}}%
\pgfpathlineto{\pgfqpoint{0.686957in}{0.825676in}}%
\pgfpathlineto{\pgfqpoint{0.687062in}{1.102762in}}%
\pgfpathlineto{\pgfqpoint{0.687686in}{1.299799in}}%
\pgfpathlineto{\pgfqpoint{0.687270in}{0.761376in}}%
\pgfpathlineto{\pgfqpoint{0.688103in}{1.136165in}}%
\pgfpathlineto{\pgfqpoint{0.688624in}{0.846556in}}%
\pgfpathlineto{\pgfqpoint{0.689040in}{1.235699in}}%
\pgfpathlineto{\pgfqpoint{0.689249in}{1.143627in}}%
\pgfpathlineto{\pgfqpoint{0.689353in}{0.893757in}}%
\pgfpathlineto{\pgfqpoint{0.689769in}{1.394780in}}%
\pgfpathlineto{\pgfqpoint{0.690290in}{1.031779in}}%
\pgfpathlineto{\pgfqpoint{0.691123in}{0.738003in}}%
\pgfpathlineto{\pgfqpoint{0.691331in}{1.420285in}}%
\pgfpathlineto{\pgfqpoint{0.692269in}{0.956057in}}%
\pgfpathlineto{\pgfqpoint{0.692477in}{1.028476in}}%
\pgfpathlineto{\pgfqpoint{0.692789in}{0.922224in}}%
\pgfpathlineto{\pgfqpoint{0.693102in}{1.213642in}}%
\pgfpathlineto{\pgfqpoint{0.693310in}{1.050495in}}%
\pgfpathlineto{\pgfqpoint{0.693518in}{1.258332in}}%
\pgfpathlineto{\pgfqpoint{0.693831in}{0.861518in}}%
\pgfpathlineto{\pgfqpoint{0.694456in}{1.131959in}}%
\pgfpathlineto{\pgfqpoint{0.694872in}{0.764059in}}%
\pgfpathlineto{\pgfqpoint{0.695393in}{1.278972in}}%
\pgfpathlineto{\pgfqpoint{0.696018in}{0.964088in}}%
\pgfpathlineto{\pgfqpoint{0.696643in}{1.118338in}}%
\pgfpathlineto{\pgfqpoint{0.696747in}{1.229025in}}%
\pgfpathlineto{\pgfqpoint{0.697059in}{0.679109in}}%
\pgfpathlineto{\pgfqpoint{0.697476in}{0.882331in}}%
\pgfpathlineto{\pgfqpoint{0.697996in}{0.673658in}}%
\pgfpathlineto{\pgfqpoint{0.697788in}{1.165658in}}%
\pgfpathlineto{\pgfqpoint{0.698309in}{0.836982in}}%
\pgfpathlineto{\pgfqpoint{0.698934in}{1.229895in}}%
\pgfpathlineto{\pgfqpoint{0.699454in}{0.924231in}}%
\pgfpathlineto{\pgfqpoint{0.700496in}{1.322333in}}%
\pgfpathlineto{\pgfqpoint{0.699871in}{0.817208in}}%
\pgfpathlineto{\pgfqpoint{0.700600in}{1.153729in}}%
\pgfpathlineto{\pgfqpoint{0.701537in}{0.943526in}}%
\pgfpathlineto{\pgfqpoint{0.700808in}{1.370986in}}%
\pgfpathlineto{\pgfqpoint{0.701745in}{0.954643in}}%
\pgfpathlineto{\pgfqpoint{0.701850in}{1.349850in}}%
\pgfpathlineto{\pgfqpoint{0.702266in}{0.856335in}}%
\pgfpathlineto{\pgfqpoint{0.702891in}{1.228299in}}%
\pgfpathlineto{\pgfqpoint{0.703308in}{0.845367in}}%
\pgfpathlineto{\pgfqpoint{0.703412in}{1.274652in}}%
\pgfpathlineto{\pgfqpoint{0.703932in}{1.121121in}}%
\pgfpathlineto{\pgfqpoint{0.704037in}{1.372116in}}%
\pgfpathlineto{\pgfqpoint{0.704141in}{0.959665in}}%
\pgfpathlineto{\pgfqpoint{0.704974in}{1.000574in}}%
\pgfpathlineto{\pgfqpoint{0.705286in}{0.765397in}}%
\pgfpathlineto{\pgfqpoint{0.705390in}{1.356139in}}%
\pgfpathlineto{\pgfqpoint{0.705807in}{1.099315in}}%
\pgfpathlineto{\pgfqpoint{0.706536in}{1.290220in}}%
\pgfpathlineto{\pgfqpoint{0.706119in}{0.968541in}}%
\pgfpathlineto{\pgfqpoint{0.706953in}{1.226503in}}%
\pgfpathlineto{\pgfqpoint{0.707161in}{0.863721in}}%
\pgfpathlineto{\pgfqpoint{0.707994in}{1.304153in}}%
\pgfpathlineto{\pgfqpoint{0.708098in}{1.089527in}}%
\pgfpathlineto{\pgfqpoint{0.708202in}{1.436246in}}%
\pgfpathlineto{\pgfqpoint{0.708411in}{0.795590in}}%
\pgfpathlineto{\pgfqpoint{0.709140in}{1.023474in}}%
\pgfpathlineto{\pgfqpoint{0.709764in}{0.878547in}}%
\pgfpathlineto{\pgfqpoint{0.709556in}{1.299408in}}%
\pgfpathlineto{\pgfqpoint{0.710181in}{1.034489in}}%
\pgfpathlineto{\pgfqpoint{0.710493in}{1.149375in}}%
\pgfpathlineto{\pgfqpoint{0.710389in}{0.938903in}}%
\pgfpathlineto{\pgfqpoint{0.710702in}{1.115782in}}%
\pgfpathlineto{\pgfqpoint{0.711743in}{0.795583in}}%
\pgfpathlineto{\pgfqpoint{0.711222in}{1.307787in}}%
\pgfpathlineto{\pgfqpoint{0.711847in}{0.830117in}}%
\pgfpathlineto{\pgfqpoint{0.712160in}{1.429860in}}%
\pgfpathlineto{\pgfqpoint{0.712993in}{1.141187in}}%
\pgfpathlineto{\pgfqpoint{0.713097in}{1.168478in}}%
\pgfpathlineto{\pgfqpoint{0.713305in}{0.934012in}}%
\pgfpathlineto{\pgfqpoint{0.713513in}{1.029178in}}%
\pgfpathlineto{\pgfqpoint{0.713618in}{0.987611in}}%
\pgfpathlineto{\pgfqpoint{0.713826in}{1.194464in}}%
\pgfpathlineto{\pgfqpoint{0.713930in}{1.067154in}}%
\pgfpathlineto{\pgfqpoint{0.714763in}{1.276431in}}%
\pgfpathlineto{\pgfqpoint{0.714451in}{0.780056in}}%
\pgfpathlineto{\pgfqpoint{0.714867in}{1.146809in}}%
\pgfpathlineto{\pgfqpoint{0.714971in}{0.903761in}}%
\pgfpathlineto{\pgfqpoint{0.715805in}{1.330935in}}%
\pgfpathlineto{\pgfqpoint{0.716013in}{0.982379in}}%
\pgfpathlineto{\pgfqpoint{0.716846in}{1.129596in}}%
\pgfpathlineto{\pgfqpoint{0.717158in}{0.632358in}}%
\pgfpathlineto{\pgfqpoint{0.717367in}{1.314438in}}%
\pgfpathlineto{\pgfqpoint{0.718304in}{1.039462in}}%
\pgfpathlineto{\pgfqpoint{0.718512in}{0.925232in}}%
\pgfpathlineto{\pgfqpoint{0.718825in}{1.108121in}}%
\pgfpathlineto{\pgfqpoint{0.719137in}{1.273382in}}%
\pgfpathlineto{\pgfqpoint{0.719033in}{0.936218in}}%
\pgfpathlineto{\pgfqpoint{0.719449in}{0.944437in}}%
\pgfpathlineto{\pgfqpoint{0.719554in}{0.698154in}}%
\pgfpathlineto{\pgfqpoint{0.719658in}{1.183999in}}%
\pgfpathlineto{\pgfqpoint{0.720491in}{0.869969in}}%
\pgfpathlineto{\pgfqpoint{0.721116in}{1.225221in}}%
\pgfpathlineto{\pgfqpoint{0.721012in}{0.816977in}}%
\pgfpathlineto{\pgfqpoint{0.721532in}{1.222856in}}%
\pgfpathlineto{\pgfqpoint{0.722574in}{0.625150in}}%
\pgfpathlineto{\pgfqpoint{0.723199in}{1.356328in}}%
\pgfpathlineto{\pgfqpoint{0.723719in}{0.959306in}}%
\pgfpathlineto{\pgfqpoint{0.724032in}{0.914167in}}%
\pgfpathlineto{\pgfqpoint{0.724136in}{1.072185in}}%
\pgfpathlineto{\pgfqpoint{0.725073in}{1.523101in}}%
\pgfpathlineto{\pgfqpoint{0.724657in}{0.824605in}}%
\pgfpathlineto{\pgfqpoint{0.725177in}{1.098044in}}%
\pgfpathlineto{\pgfqpoint{0.725698in}{0.842949in}}%
\pgfpathlineto{\pgfqpoint{0.726010in}{1.329277in}}%
\pgfpathlineto{\pgfqpoint{0.726219in}{0.988478in}}%
\pgfpathlineto{\pgfqpoint{0.727260in}{0.913904in}}%
\pgfpathlineto{\pgfqpoint{0.727468in}{1.357067in}}%
\pgfpathlineto{\pgfqpoint{0.727572in}{0.897396in}}%
\pgfpathlineto{\pgfqpoint{0.728510in}{1.228546in}}%
\pgfpathlineto{\pgfqpoint{0.728614in}{1.434466in}}%
\pgfpathlineto{\pgfqpoint{0.729030in}{1.002305in}}%
\pgfpathlineto{\pgfqpoint{0.729447in}{1.247669in}}%
\pgfpathlineto{\pgfqpoint{0.729968in}{0.742561in}}%
\pgfpathlineto{\pgfqpoint{0.730488in}{0.980041in}}%
\pgfpathlineto{\pgfqpoint{0.731217in}{1.291400in}}%
\pgfpathlineto{\pgfqpoint{0.731113in}{0.918708in}}%
\pgfpathlineto{\pgfqpoint{0.731634in}{1.200779in}}%
\pgfpathlineto{\pgfqpoint{0.732259in}{0.881020in}}%
\pgfpathlineto{\pgfqpoint{0.732363in}{1.228589in}}%
\pgfpathlineto{\pgfqpoint{0.732780in}{1.016148in}}%
\pgfpathlineto{\pgfqpoint{0.733404in}{1.338916in}}%
\pgfpathlineto{\pgfqpoint{0.733196in}{0.856504in}}%
\pgfpathlineto{\pgfqpoint{0.733925in}{1.105663in}}%
\pgfpathlineto{\pgfqpoint{0.734446in}{1.018740in}}%
\pgfpathlineto{\pgfqpoint{0.734654in}{1.333181in}}%
\pgfpathlineto{\pgfqpoint{0.735175in}{0.735099in}}%
\pgfpathlineto{\pgfqpoint{0.735800in}{1.158358in}}%
\pgfpathlineto{\pgfqpoint{0.736424in}{0.916795in}}%
\pgfpathlineto{\pgfqpoint{0.736529in}{1.213654in}}%
\pgfpathlineto{\pgfqpoint{0.736841in}{1.108109in}}%
\pgfpathlineto{\pgfqpoint{0.736945in}{1.324121in}}%
\pgfpathlineto{\pgfqpoint{0.737570in}{0.816804in}}%
\pgfpathlineto{\pgfqpoint{0.737987in}{1.305386in}}%
\pgfpathlineto{\pgfqpoint{0.739132in}{0.657182in}}%
\pgfpathlineto{\pgfqpoint{0.740174in}{1.295011in}}%
\pgfpathlineto{\pgfqpoint{0.740278in}{1.163157in}}%
\pgfpathlineto{\pgfqpoint{0.740486in}{0.818910in}}%
\pgfpathlineto{\pgfqpoint{0.740798in}{1.372989in}}%
\pgfpathlineto{\pgfqpoint{0.741215in}{1.060574in}}%
\pgfpathlineto{\pgfqpoint{0.741319in}{1.303101in}}%
\pgfpathlineto{\pgfqpoint{0.741736in}{0.860316in}}%
\pgfpathlineto{\pgfqpoint{0.742152in}{0.952483in}}%
\pgfpathlineto{\pgfqpoint{0.742465in}{0.833192in}}%
\pgfpathlineto{\pgfqpoint{0.742569in}{1.165790in}}%
\pgfpathlineto{\pgfqpoint{0.743089in}{1.101043in}}%
\pgfpathlineto{\pgfqpoint{0.743194in}{1.251025in}}%
\pgfpathlineto{\pgfqpoint{0.743298in}{0.778788in}}%
\pgfpathlineto{\pgfqpoint{0.744131in}{1.212190in}}%
\pgfpathlineto{\pgfqpoint{0.744756in}{0.789367in}}%
\pgfpathlineto{\pgfqpoint{0.744339in}{1.333374in}}%
\pgfpathlineto{\pgfqpoint{0.745172in}{0.821961in}}%
\pgfpathlineto{\pgfqpoint{0.745589in}{1.290936in}}%
\pgfpathlineto{\pgfqpoint{0.746005in}{0.738161in}}%
\pgfpathlineto{\pgfqpoint{0.746214in}{1.023164in}}%
\pgfpathlineto{\pgfqpoint{0.746943in}{1.119072in}}%
\pgfpathlineto{\pgfqpoint{0.747047in}{0.894038in}}%
\pgfpathlineto{\pgfqpoint{0.747151in}{1.261376in}}%
\pgfpathlineto{\pgfqpoint{0.748088in}{0.886206in}}%
\pgfpathlineto{\pgfqpoint{0.748192in}{1.070113in}}%
\pgfpathlineto{\pgfqpoint{0.748609in}{0.777808in}}%
\pgfpathlineto{\pgfqpoint{0.748921in}{1.077646in}}%
\pgfpathlineto{\pgfqpoint{0.749026in}{0.903249in}}%
\pgfpathlineto{\pgfqpoint{0.749650in}{1.314887in}}%
\pgfpathlineto{\pgfqpoint{0.749755in}{0.878876in}}%
\pgfpathlineto{\pgfqpoint{0.750171in}{1.258664in}}%
\pgfpathlineto{\pgfqpoint{0.750900in}{0.797386in}}%
\pgfpathlineto{\pgfqpoint{0.751212in}{1.205807in}}%
\pgfpathlineto{\pgfqpoint{0.751317in}{1.219635in}}%
\pgfpathlineto{\pgfqpoint{0.752150in}{1.259136in}}%
\pgfpathlineto{\pgfqpoint{0.752462in}{0.806946in}}%
\pgfpathlineto{\pgfqpoint{0.752775in}{1.291430in}}%
\pgfpathlineto{\pgfqpoint{0.753712in}{1.134396in}}%
\pgfpathlineto{\pgfqpoint{0.753920in}{0.990084in}}%
\pgfpathlineto{\pgfqpoint{0.754441in}{1.288943in}}%
\pgfpathlineto{\pgfqpoint{0.754857in}{1.005147in}}%
\pgfpathlineto{\pgfqpoint{0.755899in}{1.381852in}}%
\pgfpathlineto{\pgfqpoint{0.755378in}{0.823476in}}%
\pgfpathlineto{\pgfqpoint{0.756003in}{1.147873in}}%
\pgfpathlineto{\pgfqpoint{0.756107in}{1.127723in}}%
\pgfpathlineto{\pgfqpoint{0.756836in}{0.763496in}}%
\pgfpathlineto{\pgfqpoint{0.756940in}{1.164006in}}%
\pgfpathlineto{\pgfqpoint{0.757149in}{1.089305in}}%
\pgfpathlineto{\pgfqpoint{0.757461in}{0.682580in}}%
\pgfpathlineto{\pgfqpoint{0.758294in}{1.442511in}}%
\pgfpathlineto{\pgfqpoint{0.758398in}{0.833449in}}%
\pgfpathlineto{\pgfqpoint{0.759440in}{0.999890in}}%
\pgfpathlineto{\pgfqpoint{0.759960in}{0.793788in}}%
\pgfpathlineto{\pgfqpoint{0.760481in}{1.256016in}}%
\pgfpathlineto{\pgfqpoint{0.760585in}{0.740532in}}%
\pgfpathlineto{\pgfqpoint{0.760689in}{1.514971in}}%
\pgfpathlineto{\pgfqpoint{0.761627in}{1.025190in}}%
\pgfpathlineto{\pgfqpoint{0.761731in}{1.031955in}}%
\pgfpathlineto{\pgfqpoint{0.761835in}{1.025547in}}%
\pgfpathlineto{\pgfqpoint{0.761939in}{1.220672in}}%
\pgfpathlineto{\pgfqpoint{0.762147in}{0.622354in}}%
\pgfpathlineto{\pgfqpoint{0.762980in}{1.113949in}}%
\pgfpathlineto{\pgfqpoint{0.763189in}{0.851564in}}%
\pgfpathlineto{\pgfqpoint{0.763501in}{1.194143in}}%
\pgfpathlineto{\pgfqpoint{0.764022in}{1.117601in}}%
\pgfpathlineto{\pgfqpoint{0.764334in}{1.228771in}}%
\pgfpathlineto{\pgfqpoint{0.764751in}{1.050276in}}%
\pgfpathlineto{\pgfqpoint{0.765584in}{0.776080in}}%
\pgfpathlineto{\pgfqpoint{0.765688in}{1.085494in}}%
\pgfpathlineto{\pgfqpoint{0.765896in}{0.934175in}}%
\pgfpathlineto{\pgfqpoint{0.766105in}{1.191859in}}%
\pgfpathlineto{\pgfqpoint{0.766209in}{0.818089in}}%
\pgfpathlineto{\pgfqpoint{0.766834in}{1.015325in}}%
\pgfpathlineto{\pgfqpoint{0.767667in}{0.610734in}}%
\pgfpathlineto{\pgfqpoint{0.767458in}{1.253993in}}%
\pgfpathlineto{\pgfqpoint{0.767771in}{1.081170in}}%
\pgfpathlineto{\pgfqpoint{0.768396in}{0.911198in}}%
\pgfpathlineto{\pgfqpoint{0.768812in}{1.459446in}}%
\pgfpathlineto{\pgfqpoint{0.769229in}{0.794369in}}%
\pgfpathlineto{\pgfqpoint{0.769958in}{1.054201in}}%
\pgfpathlineto{\pgfqpoint{0.770062in}{1.052805in}}%
\pgfpathlineto{\pgfqpoint{0.770791in}{0.856520in}}%
\pgfpathlineto{\pgfqpoint{0.770374in}{1.175342in}}%
\pgfpathlineto{\pgfqpoint{0.770999in}{0.984234in}}%
\pgfpathlineto{\pgfqpoint{0.771103in}{1.182581in}}%
\pgfpathlineto{\pgfqpoint{0.771937in}{0.968588in}}%
\pgfpathlineto{\pgfqpoint{0.772041in}{0.696013in}}%
\pgfpathlineto{\pgfqpoint{0.772249in}{1.263111in}}%
\pgfpathlineto{\pgfqpoint{0.772978in}{1.007753in}}%
\pgfpathlineto{\pgfqpoint{0.773186in}{1.230969in}}%
\pgfpathlineto{\pgfqpoint{0.773811in}{0.716960in}}%
\pgfpathlineto{\pgfqpoint{0.773915in}{0.973320in}}%
\pgfpathlineto{\pgfqpoint{0.774019in}{0.874900in}}%
\pgfpathlineto{\pgfqpoint{0.774436in}{1.267475in}}%
\pgfpathlineto{\pgfqpoint{0.774853in}{0.886925in}}%
\pgfpathlineto{\pgfqpoint{0.775373in}{0.825990in}}%
\pgfpathlineto{\pgfqpoint{0.775894in}{1.219448in}}%
\pgfpathlineto{\pgfqpoint{0.775998in}{0.879497in}}%
\pgfpathlineto{\pgfqpoint{0.776102in}{1.295408in}}%
\pgfpathlineto{\pgfqpoint{0.776935in}{0.906992in}}%
\pgfpathlineto{\pgfqpoint{0.777664in}{1.408480in}}%
\pgfpathlineto{\pgfqpoint{0.777456in}{0.677874in}}%
\pgfpathlineto{\pgfqpoint{0.778081in}{1.057470in}}%
\pgfpathlineto{\pgfqpoint{0.778393in}{1.217816in}}%
\pgfpathlineto{\pgfqpoint{0.778289in}{0.880760in}}%
\pgfpathlineto{\pgfqpoint{0.779018in}{1.141082in}}%
\pgfpathlineto{\pgfqpoint{0.779643in}{0.838654in}}%
\pgfpathlineto{\pgfqpoint{0.779435in}{1.281655in}}%
\pgfpathlineto{\pgfqpoint{0.780164in}{1.021392in}}%
\pgfpathlineto{\pgfqpoint{0.780893in}{1.298169in}}%
\pgfpathlineto{\pgfqpoint{0.780684in}{0.812857in}}%
\pgfpathlineto{\pgfqpoint{0.781205in}{0.869106in}}%
\pgfpathlineto{\pgfqpoint{0.782247in}{1.286284in}}%
\pgfpathlineto{\pgfqpoint{0.782351in}{0.905097in}}%
\pgfpathlineto{\pgfqpoint{0.782559in}{0.869486in}}%
\pgfpathlineto{\pgfqpoint{0.783392in}{1.361034in}}%
\pgfpathlineto{\pgfqpoint{0.784329in}{0.729343in}}%
\pgfpathlineto{\pgfqpoint{0.784538in}{0.845052in}}%
\pgfpathlineto{\pgfqpoint{0.785162in}{1.351454in}}%
\pgfpathlineto{\pgfqpoint{0.784954in}{0.819475in}}%
\pgfpathlineto{\pgfqpoint{0.785683in}{1.056982in}}%
\pgfpathlineto{\pgfqpoint{0.786308in}{0.864244in}}%
\pgfpathlineto{\pgfqpoint{0.785996in}{1.319152in}}%
\pgfpathlineto{\pgfqpoint{0.786516in}{1.123547in}}%
\pgfpathlineto{\pgfqpoint{0.786620in}{1.272123in}}%
\pgfpathlineto{\pgfqpoint{0.787037in}{0.870539in}}%
\pgfpathlineto{\pgfqpoint{0.787558in}{1.093216in}}%
\pgfpathlineto{\pgfqpoint{0.787870in}{1.362529in}}%
\pgfpathlineto{\pgfqpoint{0.788495in}{0.958805in}}%
\pgfpathlineto{\pgfqpoint{0.788703in}{1.262649in}}%
\pgfpathlineto{\pgfqpoint{0.789328in}{0.764847in}}%
\pgfpathlineto{\pgfqpoint{0.789641in}{1.161488in}}%
\pgfpathlineto{\pgfqpoint{0.790474in}{0.824451in}}%
\pgfpathlineto{\pgfqpoint{0.790370in}{1.344425in}}%
\pgfpathlineto{\pgfqpoint{0.790890in}{0.858804in}}%
\pgfpathlineto{\pgfqpoint{0.791307in}{1.276094in}}%
\pgfpathlineto{\pgfqpoint{0.791411in}{0.841055in}}%
\pgfpathlineto{\pgfqpoint{0.791932in}{0.992379in}}%
\pgfpathlineto{\pgfqpoint{0.792036in}{0.671723in}}%
\pgfpathlineto{\pgfqpoint{0.792452in}{1.235654in}}%
\pgfpathlineto{\pgfqpoint{0.792973in}{1.101113in}}%
\pgfpathlineto{\pgfqpoint{0.793910in}{1.153941in}}%
\pgfpathlineto{\pgfqpoint{0.794014in}{0.885462in}}%
\pgfpathlineto{\pgfqpoint{0.794535in}{1.229534in}}%
\pgfpathlineto{\pgfqpoint{0.794327in}{0.800724in}}%
\pgfpathlineto{\pgfqpoint{0.795056in}{1.017230in}}%
\pgfpathlineto{\pgfqpoint{0.795785in}{0.932214in}}%
\pgfpathlineto{\pgfqpoint{0.795368in}{1.309540in}}%
\pgfpathlineto{\pgfqpoint{0.795889in}{1.130905in}}%
\pgfpathlineto{\pgfqpoint{0.795993in}{1.115615in}}%
\pgfpathlineto{\pgfqpoint{0.796097in}{1.260005in}}%
\pgfpathlineto{\pgfqpoint{0.796201in}{1.149694in}}%
\pgfpathlineto{\pgfqpoint{0.796306in}{0.833432in}}%
\pgfpathlineto{\pgfqpoint{0.796930in}{1.510884in}}%
\pgfpathlineto{\pgfqpoint{0.797243in}{1.132666in}}%
\pgfpathlineto{\pgfqpoint{0.797555in}{1.317810in}}%
\pgfpathlineto{\pgfqpoint{0.797451in}{1.007078in}}%
\pgfpathlineto{\pgfqpoint{0.798076in}{1.177375in}}%
\pgfpathlineto{\pgfqpoint{0.798388in}{0.892605in}}%
\pgfpathlineto{\pgfqpoint{0.798805in}{1.300081in}}%
\pgfpathlineto{\pgfqpoint{0.799222in}{0.997322in}}%
\pgfpathlineto{\pgfqpoint{0.799950in}{1.184929in}}%
\pgfpathlineto{\pgfqpoint{0.799742in}{0.831260in}}%
\pgfpathlineto{\pgfqpoint{0.800367in}{1.055208in}}%
\pgfpathlineto{\pgfqpoint{0.800992in}{0.737937in}}%
\pgfpathlineto{\pgfqpoint{0.801096in}{1.215295in}}%
\pgfpathlineto{\pgfqpoint{0.801304in}{0.927466in}}%
\pgfpathlineto{\pgfqpoint{0.802033in}{1.232153in}}%
\pgfpathlineto{\pgfqpoint{0.801929in}{0.853570in}}%
\pgfpathlineto{\pgfqpoint{0.802346in}{1.075228in}}%
\pgfpathlineto{\pgfqpoint{0.802450in}{0.793318in}}%
\pgfpathlineto{\pgfqpoint{0.803179in}{1.279409in}}%
\pgfpathlineto{\pgfqpoint{0.803491in}{0.890201in}}%
\pgfpathlineto{\pgfqpoint{0.804220in}{1.266893in}}%
\pgfpathlineto{\pgfqpoint{0.804429in}{0.859431in}}%
\pgfpathlineto{\pgfqpoint{0.804637in}{1.092394in}}%
\pgfpathlineto{\pgfqpoint{0.805678in}{1.206204in}}%
\pgfpathlineto{\pgfqpoint{0.805782in}{0.783045in}}%
\pgfpathlineto{\pgfqpoint{0.806616in}{1.084609in}}%
\pgfpathlineto{\pgfqpoint{0.807032in}{0.973887in}}%
\pgfpathlineto{\pgfqpoint{0.807240in}{1.299929in}}%
\pgfpathlineto{\pgfqpoint{0.807449in}{0.863318in}}%
\pgfpathlineto{\pgfqpoint{0.807969in}{1.280026in}}%
\pgfpathlineto{\pgfqpoint{0.808802in}{0.859369in}}%
\pgfpathlineto{\pgfqpoint{0.809115in}{1.020483in}}%
\pgfpathlineto{\pgfqpoint{0.809740in}{1.219290in}}%
\pgfpathlineto{\pgfqpoint{0.809323in}{0.829316in}}%
\pgfpathlineto{\pgfqpoint{0.809948in}{1.137087in}}%
\pgfpathlineto{\pgfqpoint{0.810677in}{1.175481in}}%
\pgfpathlineto{\pgfqpoint{0.810885in}{0.818233in}}%
\pgfpathlineto{\pgfqpoint{0.810989in}{1.311109in}}%
\pgfpathlineto{\pgfqpoint{0.811927in}{0.952760in}}%
\pgfpathlineto{\pgfqpoint{0.812031in}{0.919621in}}%
\pgfpathlineto{\pgfqpoint{0.812239in}{1.044305in}}%
\pgfpathlineto{\pgfqpoint{0.812343in}{0.949056in}}%
\pgfpathlineto{\pgfqpoint{0.812552in}{1.272583in}}%
\pgfpathlineto{\pgfqpoint{0.813385in}{0.872411in}}%
\pgfpathlineto{\pgfqpoint{0.813489in}{1.181812in}}%
\pgfpathlineto{\pgfqpoint{0.813593in}{1.184475in}}%
\pgfpathlineto{\pgfqpoint{0.814426in}{0.851354in}}%
\pgfpathlineto{\pgfqpoint{0.814218in}{1.401322in}}%
\pgfpathlineto{\pgfqpoint{0.814739in}{1.020266in}}%
\pgfpathlineto{\pgfqpoint{0.815363in}{1.346785in}}%
\pgfpathlineto{\pgfqpoint{0.814947in}{0.857587in}}%
\pgfpathlineto{\pgfqpoint{0.815780in}{0.997318in}}%
\pgfpathlineto{\pgfqpoint{0.816509in}{1.266178in}}%
\pgfpathlineto{\pgfqpoint{0.816092in}{0.772253in}}%
\pgfpathlineto{\pgfqpoint{0.816821in}{1.013203in}}%
\pgfpathlineto{\pgfqpoint{0.816925in}{1.013086in}}%
\pgfpathlineto{\pgfqpoint{0.817863in}{1.487266in}}%
\pgfpathlineto{\pgfqpoint{0.817654in}{0.894808in}}%
\pgfpathlineto{\pgfqpoint{0.817967in}{1.236960in}}%
\pgfpathlineto{\pgfqpoint{0.819008in}{0.728251in}}%
\pgfpathlineto{\pgfqpoint{0.819112in}{1.020423in}}%
\pgfpathlineto{\pgfqpoint{0.819217in}{0.929135in}}%
\pgfpathlineto{\pgfqpoint{0.819425in}{1.492577in}}%
\pgfpathlineto{\pgfqpoint{0.820050in}{1.143291in}}%
\pgfpathlineto{\pgfqpoint{0.820154in}{1.356188in}}%
\pgfpathlineto{\pgfqpoint{0.820258in}{0.888091in}}%
\pgfpathlineto{\pgfqpoint{0.821091in}{1.189052in}}%
\pgfpathlineto{\pgfqpoint{0.822028in}{0.851858in}}%
\pgfpathlineto{\pgfqpoint{0.821820in}{1.410551in}}%
\pgfpathlineto{\pgfqpoint{0.822133in}{0.971229in}}%
\pgfpathlineto{\pgfqpoint{0.822341in}{1.360873in}}%
\pgfpathlineto{\pgfqpoint{0.823070in}{0.667092in}}%
\pgfpathlineto{\pgfqpoint{0.823278in}{1.102313in}}%
\pgfpathlineto{\pgfqpoint{0.823382in}{1.189906in}}%
\pgfpathlineto{\pgfqpoint{0.823486in}{0.905273in}}%
\pgfpathlineto{\pgfqpoint{0.824320in}{1.180240in}}%
\pgfpathlineto{\pgfqpoint{0.824632in}{0.841435in}}%
\pgfpathlineto{\pgfqpoint{0.824944in}{1.439052in}}%
\pgfpathlineto{\pgfqpoint{0.825361in}{1.095384in}}%
\pgfpathlineto{\pgfqpoint{0.825465in}{1.312436in}}%
\pgfpathlineto{\pgfqpoint{0.826194in}{0.860430in}}%
\pgfpathlineto{\pgfqpoint{0.826298in}{1.024361in}}%
\pgfpathlineto{\pgfqpoint{0.827131in}{0.724199in}}%
\pgfpathlineto{\pgfqpoint{0.826611in}{1.235897in}}%
\pgfpathlineto{\pgfqpoint{0.827235in}{0.965072in}}%
\pgfpathlineto{\pgfqpoint{0.827444in}{1.230410in}}%
\pgfpathlineto{\pgfqpoint{0.827652in}{0.778585in}}%
\pgfpathlineto{\pgfqpoint{0.828277in}{0.965359in}}%
\pgfpathlineto{\pgfqpoint{0.828381in}{0.730268in}}%
\pgfpathlineto{\pgfqpoint{0.828589in}{1.241395in}}%
\pgfpathlineto{\pgfqpoint{0.829318in}{0.946352in}}%
\pgfpathlineto{\pgfqpoint{0.829631in}{1.267968in}}%
\pgfpathlineto{\pgfqpoint{0.830256in}{0.927714in}}%
\pgfpathlineto{\pgfqpoint{0.830464in}{1.156662in}}%
\pgfpathlineto{\pgfqpoint{0.831193in}{1.303821in}}%
\pgfpathlineto{\pgfqpoint{0.831609in}{0.902537in}}%
\pgfpathlineto{\pgfqpoint{0.831818in}{1.319880in}}%
\pgfpathlineto{\pgfqpoint{0.832547in}{0.847753in}}%
\pgfpathlineto{\pgfqpoint{0.832755in}{1.186676in}}%
\pgfpathlineto{\pgfqpoint{0.832963in}{1.277288in}}%
\pgfpathlineto{\pgfqpoint{0.833900in}{0.881526in}}%
\pgfpathlineto{\pgfqpoint{0.834525in}{1.178731in}}%
\pgfpathlineto{\pgfqpoint{0.834421in}{0.853397in}}%
\pgfpathlineto{\pgfqpoint{0.835150in}{1.092418in}}%
\pgfpathlineto{\pgfqpoint{0.835463in}{0.765945in}}%
\pgfpathlineto{\pgfqpoint{0.835983in}{1.153907in}}%
\pgfpathlineto{\pgfqpoint{0.836192in}{1.005789in}}%
\pgfpathlineto{\pgfqpoint{0.836816in}{1.165310in}}%
\pgfpathlineto{\pgfqpoint{0.836712in}{0.828661in}}%
\pgfpathlineto{\pgfqpoint{0.837233in}{1.048398in}}%
\pgfpathlineto{\pgfqpoint{0.837337in}{1.034419in}}%
\pgfpathlineto{\pgfqpoint{0.837441in}{1.103200in}}%
\pgfpathlineto{\pgfqpoint{0.838274in}{1.349686in}}%
\pgfpathlineto{\pgfqpoint{0.837754in}{0.861462in}}%
\pgfpathlineto{\pgfqpoint{0.838587in}{1.207649in}}%
\pgfpathlineto{\pgfqpoint{0.839628in}{0.745767in}}%
\pgfpathlineto{\pgfqpoint{0.839003in}{1.283769in}}%
\pgfpathlineto{\pgfqpoint{0.839732in}{1.146031in}}%
\pgfpathlineto{\pgfqpoint{0.840149in}{1.341152in}}%
\pgfpathlineto{\pgfqpoint{0.839941in}{0.967179in}}%
\pgfpathlineto{\pgfqpoint{0.840253in}{0.979898in}}%
\pgfpathlineto{\pgfqpoint{0.840357in}{0.898662in}}%
\pgfpathlineto{\pgfqpoint{0.840566in}{1.342390in}}%
\pgfpathlineto{\pgfqpoint{0.841086in}{0.982900in}}%
\pgfpathlineto{\pgfqpoint{0.841607in}{1.345984in}}%
\pgfpathlineto{\pgfqpoint{0.841503in}{0.924963in}}%
\pgfpathlineto{\pgfqpoint{0.842232in}{1.115096in}}%
\pgfpathlineto{\pgfqpoint{0.842440in}{1.060561in}}%
\pgfpathlineto{\pgfqpoint{0.842544in}{1.136137in}}%
\pgfpathlineto{\pgfqpoint{0.842648in}{0.682687in}}%
\pgfpathlineto{\pgfqpoint{0.843273in}{1.311974in}}%
\pgfpathlineto{\pgfqpoint{0.843586in}{1.032019in}}%
\pgfpathlineto{\pgfqpoint{0.843794in}{0.987841in}}%
\pgfpathlineto{\pgfqpoint{0.843898in}{1.082811in}}%
\pgfpathlineto{\pgfqpoint{0.844106in}{0.888132in}}%
\pgfpathlineto{\pgfqpoint{0.844523in}{1.185660in}}%
\pgfpathlineto{\pgfqpoint{0.844939in}{1.049234in}}%
\pgfpathlineto{\pgfqpoint{0.845044in}{1.120952in}}%
\pgfpathlineto{\pgfqpoint{0.845668in}{0.826745in}}%
\pgfpathlineto{\pgfqpoint{0.845981in}{1.016549in}}%
\pgfpathlineto{\pgfqpoint{0.846085in}{1.018705in}}%
\pgfpathlineto{\pgfqpoint{0.846397in}{1.182599in}}%
\pgfpathlineto{\pgfqpoint{0.846606in}{0.877472in}}%
\pgfpathlineto{\pgfqpoint{0.846918in}{1.002488in}}%
\pgfpathlineto{\pgfqpoint{0.847126in}{0.863827in}}%
\pgfpathlineto{\pgfqpoint{0.847439in}{1.141555in}}%
\pgfpathlineto{\pgfqpoint{0.847960in}{0.929573in}}%
\pgfpathlineto{\pgfqpoint{0.848584in}{1.449661in}}%
\pgfpathlineto{\pgfqpoint{0.848168in}{0.791903in}}%
\pgfpathlineto{\pgfqpoint{0.849105in}{1.042334in}}%
\pgfpathlineto{\pgfqpoint{0.849209in}{0.931830in}}%
\pgfpathlineto{\pgfqpoint{0.849938in}{1.323294in}}%
\pgfpathlineto{\pgfqpoint{0.850042in}{1.173734in}}%
\pgfpathlineto{\pgfqpoint{0.850459in}{0.771078in}}%
\pgfpathlineto{\pgfqpoint{0.850667in}{1.178288in}}%
\pgfpathlineto{\pgfqpoint{0.850980in}{1.046474in}}%
\pgfpathlineto{\pgfqpoint{0.851084in}{1.304430in}}%
\pgfpathlineto{\pgfqpoint{0.851604in}{0.803945in}}%
\pgfpathlineto{\pgfqpoint{0.852021in}{1.018428in}}%
\pgfpathlineto{\pgfqpoint{0.852125in}{1.010137in}}%
\pgfpathlineto{\pgfqpoint{0.852438in}{1.380014in}}%
\pgfpathlineto{\pgfqpoint{0.852750in}{0.831441in}}%
\pgfpathlineto{\pgfqpoint{0.853167in}{0.893024in}}%
\pgfpathlineto{\pgfqpoint{0.854104in}{1.233036in}}%
\pgfpathlineto{\pgfqpoint{0.853896in}{0.891914in}}%
\pgfpathlineto{\pgfqpoint{0.854520in}{1.227421in}}%
\pgfpathlineto{\pgfqpoint{0.854937in}{0.892291in}}%
\pgfpathlineto{\pgfqpoint{0.855562in}{1.132712in}}%
\pgfpathlineto{\pgfqpoint{0.855666in}{1.234655in}}%
\pgfpathlineto{\pgfqpoint{0.855978in}{0.747676in}}%
\pgfpathlineto{\pgfqpoint{0.856395in}{0.916638in}}%
\pgfpathlineto{\pgfqpoint{0.856812in}{0.725470in}}%
\pgfpathlineto{\pgfqpoint{0.857124in}{1.121385in}}%
\pgfpathlineto{\pgfqpoint{0.857228in}{0.650680in}}%
\pgfpathlineto{\pgfqpoint{0.857540in}{1.258058in}}%
\pgfpathlineto{\pgfqpoint{0.858165in}{1.085803in}}%
\pgfpathlineto{\pgfqpoint{0.858269in}{1.128813in}}%
\pgfpathlineto{\pgfqpoint{0.858374in}{1.630766in}}%
\pgfpathlineto{\pgfqpoint{0.858478in}{0.748104in}}%
\pgfpathlineto{\pgfqpoint{0.859311in}{0.831363in}}%
\pgfpathlineto{\pgfqpoint{0.860456in}{1.325652in}}%
\pgfpathlineto{\pgfqpoint{0.860977in}{0.948605in}}%
\pgfpathlineto{\pgfqpoint{0.861602in}{1.037101in}}%
\pgfpathlineto{\pgfqpoint{0.861810in}{1.285928in}}%
\pgfpathlineto{\pgfqpoint{0.862227in}{0.923579in}}%
\pgfpathlineto{\pgfqpoint{0.862539in}{1.270442in}}%
\pgfpathlineto{\pgfqpoint{0.863789in}{0.614901in}}%
\pgfpathlineto{\pgfqpoint{0.864622in}{1.310783in}}%
\pgfpathlineto{\pgfqpoint{0.864935in}{1.220944in}}%
\pgfpathlineto{\pgfqpoint{0.865559in}{1.233789in}}%
\pgfpathlineto{\pgfqpoint{0.866184in}{0.773342in}}%
\pgfpathlineto{\pgfqpoint{0.866497in}{1.346016in}}%
\pgfpathlineto{\pgfqpoint{0.867434in}{1.161742in}}%
\pgfpathlineto{\pgfqpoint{0.867642in}{0.811139in}}%
\pgfpathlineto{\pgfqpoint{0.868059in}{0.888192in}}%
\pgfpathlineto{\pgfqpoint{0.868163in}{1.408342in}}%
\pgfpathlineto{\pgfqpoint{0.868996in}{0.871809in}}%
\pgfpathlineto{\pgfqpoint{0.869100in}{1.103247in}}%
\pgfpathlineto{\pgfqpoint{0.869308in}{0.773995in}}%
\pgfpathlineto{\pgfqpoint{0.869621in}{1.331391in}}%
\pgfpathlineto{\pgfqpoint{0.870037in}{0.826938in}}%
\pgfpathlineto{\pgfqpoint{0.870454in}{1.248873in}}%
\pgfpathlineto{\pgfqpoint{0.871079in}{0.638248in}}%
\pgfpathlineto{\pgfqpoint{0.871183in}{0.987915in}}%
\pgfpathlineto{\pgfqpoint{0.871391in}{1.230437in}}%
\pgfpathlineto{\pgfqpoint{0.871808in}{0.779586in}}%
\pgfpathlineto{\pgfqpoint{0.872329in}{1.059228in}}%
\pgfpathlineto{\pgfqpoint{0.872433in}{0.688742in}}%
\pgfpathlineto{\pgfqpoint{0.872641in}{1.120771in}}%
\pgfpathlineto{\pgfqpoint{0.873370in}{1.023828in}}%
\pgfpathlineto{\pgfqpoint{0.873995in}{0.753122in}}%
\pgfpathlineto{\pgfqpoint{0.874411in}{1.229994in}}%
\pgfpathlineto{\pgfqpoint{0.874515in}{0.852730in}}%
\pgfpathlineto{\pgfqpoint{0.874932in}{1.308958in}}%
\pgfpathlineto{\pgfqpoint{0.875557in}{1.100791in}}%
\pgfpathlineto{\pgfqpoint{0.875973in}{0.831760in}}%
\pgfpathlineto{\pgfqpoint{0.876182in}{1.311895in}}%
\pgfpathlineto{\pgfqpoint{0.876598in}{0.867760in}}%
\pgfpathlineto{\pgfqpoint{0.877119in}{1.321166in}}%
\pgfpathlineto{\pgfqpoint{0.877744in}{1.074783in}}%
\pgfpathlineto{\pgfqpoint{0.877848in}{0.765292in}}%
\pgfpathlineto{\pgfqpoint{0.878369in}{1.233185in}}%
\pgfpathlineto{\pgfqpoint{0.878785in}{1.165739in}}%
\pgfpathlineto{\pgfqpoint{0.879098in}{0.927303in}}%
\pgfpathlineto{\pgfqpoint{0.879202in}{1.242069in}}%
\pgfpathlineto{\pgfqpoint{0.879827in}{1.179815in}}%
\pgfpathlineto{\pgfqpoint{0.879931in}{1.296408in}}%
\pgfpathlineto{\pgfqpoint{0.880347in}{0.864263in}}%
\pgfpathlineto{\pgfqpoint{0.880556in}{1.120345in}}%
\pgfpathlineto{\pgfqpoint{0.880660in}{0.779575in}}%
\pgfpathlineto{\pgfqpoint{0.881597in}{1.208180in}}%
\pgfpathlineto{\pgfqpoint{0.881701in}{1.297279in}}%
\pgfpathlineto{\pgfqpoint{0.881909in}{0.964269in}}%
\pgfpathlineto{\pgfqpoint{0.882326in}{1.162794in}}%
\pgfpathlineto{\pgfqpoint{0.882430in}{0.773768in}}%
\pgfpathlineto{\pgfqpoint{0.883159in}{1.273713in}}%
\pgfpathlineto{\pgfqpoint{0.883367in}{1.119228in}}%
\pgfpathlineto{\pgfqpoint{0.883472in}{1.193912in}}%
\pgfpathlineto{\pgfqpoint{0.883576in}{0.904205in}}%
\pgfpathlineto{\pgfqpoint{0.883888in}{1.178525in}}%
\pgfpathlineto{\pgfqpoint{0.883992in}{0.741772in}}%
\pgfpathlineto{\pgfqpoint{0.885034in}{0.999950in}}%
\pgfpathlineto{\pgfqpoint{0.885138in}{0.627259in}}%
\pgfpathlineto{\pgfqpoint{0.885763in}{1.249291in}}%
\pgfpathlineto{\pgfqpoint{0.886075in}{1.121123in}}%
\pgfpathlineto{\pgfqpoint{0.886492in}{0.796902in}}%
\pgfpathlineto{\pgfqpoint{0.887117in}{1.143496in}}%
\pgfpathlineto{\pgfqpoint{0.887221in}{0.860740in}}%
\pgfpathlineto{\pgfqpoint{0.888054in}{1.197859in}}%
\pgfpathlineto{\pgfqpoint{0.887637in}{0.838763in}}%
\pgfpathlineto{\pgfqpoint{0.888366in}{1.005592in}}%
\pgfpathlineto{\pgfqpoint{0.888679in}{0.878414in}}%
\pgfpathlineto{\pgfqpoint{0.888575in}{1.249575in}}%
\pgfpathlineto{\pgfqpoint{0.888887in}{1.185307in}}%
\pgfpathlineto{\pgfqpoint{0.889304in}{1.389247in}}%
\pgfpathlineto{\pgfqpoint{0.889616in}{0.778078in}}%
\pgfpathlineto{\pgfqpoint{0.889720in}{1.158837in}}%
\pgfpathlineto{\pgfqpoint{0.889824in}{0.777617in}}%
\pgfpathlineto{\pgfqpoint{0.890449in}{1.241981in}}%
\pgfpathlineto{\pgfqpoint{0.890761in}{0.940700in}}%
\pgfpathlineto{\pgfqpoint{0.891699in}{1.315840in}}%
\pgfpathlineto{\pgfqpoint{0.891803in}{0.810884in}}%
\pgfpathlineto{\pgfqpoint{0.892428in}{1.255443in}}%
\pgfpathlineto{\pgfqpoint{0.893053in}{1.168623in}}%
\pgfpathlineto{\pgfqpoint{0.893365in}{0.756666in}}%
\pgfpathlineto{\pgfqpoint{0.893573in}{1.334876in}}%
\pgfpathlineto{\pgfqpoint{0.894094in}{1.081296in}}%
\pgfpathlineto{\pgfqpoint{0.894406in}{1.475415in}}%
\pgfpathlineto{\pgfqpoint{0.894511in}{0.809125in}}%
\pgfpathlineto{\pgfqpoint{0.895135in}{1.087800in}}%
\pgfpathlineto{\pgfqpoint{0.895240in}{0.781324in}}%
\pgfpathlineto{\pgfqpoint{0.896177in}{1.228451in}}%
\pgfpathlineto{\pgfqpoint{0.896281in}{1.265106in}}%
\pgfpathlineto{\pgfqpoint{0.896385in}{1.019202in}}%
\pgfpathlineto{\pgfqpoint{0.896489in}{1.060926in}}%
\pgfpathlineto{\pgfqpoint{0.896593in}{0.921321in}}%
\pgfpathlineto{\pgfqpoint{0.896906in}{1.339252in}}%
\pgfpathlineto{\pgfqpoint{0.897531in}{0.925970in}}%
\pgfpathlineto{\pgfqpoint{0.897947in}{1.253227in}}%
\pgfpathlineto{\pgfqpoint{0.897843in}{0.919609in}}%
\pgfpathlineto{\pgfqpoint{0.898572in}{1.058581in}}%
\pgfpathlineto{\pgfqpoint{0.898989in}{1.274958in}}%
\pgfpathlineto{\pgfqpoint{0.899613in}{0.688818in}}%
\pgfpathlineto{\pgfqpoint{0.900134in}{1.300360in}}%
\pgfpathlineto{\pgfqpoint{0.900759in}{1.085983in}}%
\pgfpathlineto{\pgfqpoint{0.901592in}{0.880604in}}%
\pgfpathlineto{\pgfqpoint{0.901384in}{1.312813in}}%
\pgfpathlineto{\pgfqpoint{0.901696in}{1.120179in}}%
\pgfpathlineto{\pgfqpoint{0.902529in}{1.394885in}}%
\pgfpathlineto{\pgfqpoint{0.901905in}{0.905146in}}%
\pgfpathlineto{\pgfqpoint{0.902634in}{1.082974in}}%
\pgfpathlineto{\pgfqpoint{0.902842in}{0.826695in}}%
\pgfpathlineto{\pgfqpoint{0.903050in}{1.216283in}}%
\pgfpathlineto{\pgfqpoint{0.903779in}{0.827169in}}%
\pgfpathlineto{\pgfqpoint{0.904612in}{1.317979in}}%
\pgfpathlineto{\pgfqpoint{0.904821in}{1.018531in}}%
\pgfpathlineto{\pgfqpoint{0.904925in}{0.903441in}}%
\pgfpathlineto{\pgfqpoint{0.905237in}{1.293322in}}%
\pgfpathlineto{\pgfqpoint{0.905862in}{1.004106in}}%
\pgfpathlineto{\pgfqpoint{0.906591in}{1.121456in}}%
\pgfpathlineto{\pgfqpoint{0.906487in}{0.741551in}}%
\pgfpathlineto{\pgfqpoint{0.906799in}{0.884795in}}%
\pgfpathlineto{\pgfqpoint{0.907007in}{0.790725in}}%
\pgfpathlineto{\pgfqpoint{0.907216in}{0.950046in}}%
\pgfpathlineto{\pgfqpoint{0.907320in}{1.341020in}}%
\pgfpathlineto{\pgfqpoint{0.908257in}{0.807060in}}%
\pgfpathlineto{\pgfqpoint{0.908882in}{1.372124in}}%
\pgfpathlineto{\pgfqpoint{0.909403in}{0.971351in}}%
\pgfpathlineto{\pgfqpoint{0.909923in}{0.822577in}}%
\pgfpathlineto{\pgfqpoint{0.909715in}{1.180708in}}%
\pgfpathlineto{\pgfqpoint{0.910236in}{1.043058in}}%
\pgfpathlineto{\pgfqpoint{0.910757in}{1.387749in}}%
\pgfpathlineto{\pgfqpoint{0.910444in}{0.850628in}}%
\pgfpathlineto{\pgfqpoint{0.911277in}{1.039654in}}%
\pgfpathlineto{\pgfqpoint{0.911902in}{1.354766in}}%
\pgfpathlineto{\pgfqpoint{0.911486in}{0.682292in}}%
\pgfpathlineto{\pgfqpoint{0.912215in}{1.346644in}}%
\pgfpathlineto{\pgfqpoint{0.913464in}{0.793351in}}%
\pgfpathlineto{\pgfqpoint{0.914506in}{1.387893in}}%
\pgfpathlineto{\pgfqpoint{0.914610in}{1.184732in}}%
\pgfpathlineto{\pgfqpoint{0.914922in}{0.864973in}}%
\pgfpathlineto{\pgfqpoint{0.915443in}{1.287471in}}%
\pgfpathlineto{\pgfqpoint{0.916276in}{0.869845in}}%
\pgfpathlineto{\pgfqpoint{0.916693in}{1.355664in}}%
\pgfpathlineto{\pgfqpoint{0.916588in}{0.623960in}}%
\pgfpathlineto{\pgfqpoint{0.917526in}{1.204483in}}%
\pgfpathlineto{\pgfqpoint{0.917734in}{0.857863in}}%
\pgfpathlineto{\pgfqpoint{0.918775in}{0.865520in}}%
\pgfpathlineto{\pgfqpoint{0.919400in}{0.671881in}}%
\pgfpathlineto{\pgfqpoint{0.919921in}{1.278890in}}%
\pgfpathlineto{\pgfqpoint{0.920233in}{0.819552in}}%
\pgfpathlineto{\pgfqpoint{0.920858in}{1.338361in}}%
\pgfpathlineto{\pgfqpoint{0.921171in}{1.027273in}}%
\pgfpathlineto{\pgfqpoint{0.921379in}{1.227556in}}%
\pgfpathlineto{\pgfqpoint{0.921900in}{0.852614in}}%
\pgfpathlineto{\pgfqpoint{0.922629in}{1.305595in}}%
\pgfpathlineto{\pgfqpoint{0.922733in}{0.797518in}}%
\pgfpathlineto{\pgfqpoint{0.923045in}{0.937337in}}%
\pgfpathlineto{\pgfqpoint{0.923566in}{0.688016in}}%
\pgfpathlineto{\pgfqpoint{0.924191in}{1.357857in}}%
\pgfpathlineto{\pgfqpoint{0.924295in}{0.785175in}}%
\pgfpathlineto{\pgfqpoint{0.924920in}{1.361960in}}%
\pgfpathlineto{\pgfqpoint{0.925336in}{1.046970in}}%
\pgfpathlineto{\pgfqpoint{0.925440in}{1.030620in}}%
\pgfpathlineto{\pgfqpoint{0.925545in}{1.102758in}}%
\pgfpathlineto{\pgfqpoint{0.926274in}{1.269542in}}%
\pgfpathlineto{\pgfqpoint{0.926378in}{0.890927in}}%
\pgfpathlineto{\pgfqpoint{0.926482in}{1.048909in}}%
\pgfpathlineto{\pgfqpoint{0.927003in}{0.783942in}}%
\pgfpathlineto{\pgfqpoint{0.926898in}{1.423010in}}%
\pgfpathlineto{\pgfqpoint{0.927523in}{0.863371in}}%
\pgfpathlineto{\pgfqpoint{0.927836in}{0.732565in}}%
\pgfpathlineto{\pgfqpoint{0.928773in}{1.246844in}}%
\pgfpathlineto{\pgfqpoint{0.929085in}{0.867653in}}%
\pgfpathlineto{\pgfqpoint{0.929710in}{1.366208in}}%
\pgfpathlineto{\pgfqpoint{0.929814in}{1.249687in}}%
\pgfpathlineto{\pgfqpoint{0.929919in}{1.336190in}}%
\pgfpathlineto{\pgfqpoint{0.930231in}{0.864210in}}%
\pgfpathlineto{\pgfqpoint{0.930543in}{1.236212in}}%
\pgfpathlineto{\pgfqpoint{0.931064in}{0.836904in}}%
\pgfpathlineto{\pgfqpoint{0.930960in}{1.392476in}}%
\pgfpathlineto{\pgfqpoint{0.931689in}{1.020764in}}%
\pgfpathlineto{\pgfqpoint{0.932626in}{0.855976in}}%
\pgfpathlineto{\pgfqpoint{0.932001in}{1.355200in}}%
\pgfpathlineto{\pgfqpoint{0.932730in}{0.993193in}}%
\pgfpathlineto{\pgfqpoint{0.933668in}{1.229630in}}%
\pgfpathlineto{\pgfqpoint{0.933459in}{0.909398in}}%
\pgfpathlineto{\pgfqpoint{0.933772in}{1.185556in}}%
\pgfpathlineto{\pgfqpoint{0.934084in}{0.763654in}}%
\pgfpathlineto{\pgfqpoint{0.934292in}{1.332871in}}%
\pgfpathlineto{\pgfqpoint{0.934813in}{0.961246in}}%
\pgfpathlineto{\pgfqpoint{0.935230in}{1.163459in}}%
\pgfpathlineto{\pgfqpoint{0.935021in}{0.961021in}}%
\pgfpathlineto{\pgfqpoint{0.935855in}{1.121402in}}%
\pgfpathlineto{\pgfqpoint{0.936479in}{0.890565in}}%
\pgfpathlineto{\pgfqpoint{0.936792in}{1.169141in}}%
\pgfpathlineto{\pgfqpoint{0.936896in}{0.990857in}}%
\pgfpathlineto{\pgfqpoint{0.937729in}{0.747756in}}%
\pgfpathlineto{\pgfqpoint{0.937937in}{1.379096in}}%
\pgfpathlineto{\pgfqpoint{0.938562in}{0.869444in}}%
\pgfpathlineto{\pgfqpoint{0.939083in}{1.205353in}}%
\pgfpathlineto{\pgfqpoint{0.939187in}{1.182276in}}%
\pgfpathlineto{\pgfqpoint{0.939291in}{0.886310in}}%
\pgfpathlineto{\pgfqpoint{0.939916in}{1.229181in}}%
\pgfpathlineto{\pgfqpoint{0.940333in}{1.080413in}}%
\pgfpathlineto{\pgfqpoint{0.940749in}{1.419138in}}%
\pgfpathlineto{\pgfqpoint{0.940645in}{0.929263in}}%
\pgfpathlineto{\pgfqpoint{0.941478in}{1.209452in}}%
\pgfpathlineto{\pgfqpoint{0.942415in}{0.919602in}}%
\pgfpathlineto{\pgfqpoint{0.942311in}{1.267497in}}%
\pgfpathlineto{\pgfqpoint{0.942832in}{0.977134in}}%
\pgfpathlineto{\pgfqpoint{0.943873in}{1.530441in}}%
\pgfpathlineto{\pgfqpoint{0.943561in}{0.740984in}}%
\pgfpathlineto{\pgfqpoint{0.943978in}{1.393104in}}%
\pgfpathlineto{\pgfqpoint{0.945227in}{0.714028in}}%
\pgfpathlineto{\pgfqpoint{0.945331in}{1.180176in}}%
\pgfpathlineto{\pgfqpoint{0.946373in}{0.949214in}}%
\pgfpathlineto{\pgfqpoint{0.947206in}{1.264008in}}%
\pgfpathlineto{\pgfqpoint{0.946998in}{0.734453in}}%
\pgfpathlineto{\pgfqpoint{0.947623in}{1.238159in}}%
\pgfpathlineto{\pgfqpoint{0.947935in}{0.765020in}}%
\pgfpathlineto{\pgfqpoint{0.948143in}{1.354064in}}%
\pgfpathlineto{\pgfqpoint{0.948664in}{0.934773in}}%
\pgfpathlineto{\pgfqpoint{0.949185in}{1.245600in}}%
\pgfpathlineto{\pgfqpoint{0.949080in}{0.910931in}}%
\pgfpathlineto{\pgfqpoint{0.949705in}{0.939328in}}%
\pgfpathlineto{\pgfqpoint{0.949809in}{0.942603in}}%
\pgfpathlineto{\pgfqpoint{0.950330in}{0.749356in}}%
\pgfpathlineto{\pgfqpoint{0.950018in}{1.123420in}}%
\pgfpathlineto{\pgfqpoint{0.950851in}{0.882020in}}%
\pgfpathlineto{\pgfqpoint{0.951476in}{1.245566in}}%
\pgfpathlineto{\pgfqpoint{0.951996in}{1.055328in}}%
\pgfpathlineto{\pgfqpoint{0.952725in}{0.819159in}}%
\pgfpathlineto{\pgfqpoint{0.952309in}{1.085318in}}%
\pgfpathlineto{\pgfqpoint{0.953038in}{0.923058in}}%
\pgfpathlineto{\pgfqpoint{0.953559in}{1.173242in}}%
\pgfpathlineto{\pgfqpoint{0.953350in}{0.765082in}}%
\pgfpathlineto{\pgfqpoint{0.954079in}{0.868470in}}%
\pgfpathlineto{\pgfqpoint{0.954183in}{0.790840in}}%
\pgfpathlineto{\pgfqpoint{0.954808in}{1.093718in}}%
\pgfpathlineto{\pgfqpoint{0.954912in}{1.047578in}}%
\pgfpathlineto{\pgfqpoint{0.955537in}{1.227739in}}%
\pgfpathlineto{\pgfqpoint{0.955329in}{0.881441in}}%
\pgfpathlineto{\pgfqpoint{0.955746in}{1.024428in}}%
\pgfpathlineto{\pgfqpoint{0.956058in}{0.845197in}}%
\pgfpathlineto{\pgfqpoint{0.956370in}{1.235929in}}%
\pgfpathlineto{\pgfqpoint{0.956891in}{0.949933in}}%
\pgfpathlineto{\pgfqpoint{0.957203in}{1.290431in}}%
\pgfpathlineto{\pgfqpoint{0.957724in}{0.843702in}}%
\pgfpathlineto{\pgfqpoint{0.958037in}{1.244506in}}%
\pgfpathlineto{\pgfqpoint{0.958453in}{0.849189in}}%
\pgfpathlineto{\pgfqpoint{0.959182in}{1.022507in}}%
\pgfpathlineto{\pgfqpoint{0.959390in}{0.947191in}}%
\pgfpathlineto{\pgfqpoint{0.959495in}{1.101095in}}%
\pgfpathlineto{\pgfqpoint{0.959599in}{1.297016in}}%
\pgfpathlineto{\pgfqpoint{0.960224in}{0.837274in}}%
\pgfpathlineto{\pgfqpoint{0.960536in}{1.193687in}}%
\pgfpathlineto{\pgfqpoint{0.961577in}{0.731654in}}%
\pgfpathlineto{\pgfqpoint{0.961682in}{1.030361in}}%
\pgfpathlineto{\pgfqpoint{0.961890in}{1.178942in}}%
\pgfpathlineto{\pgfqpoint{0.962202in}{0.936693in}}%
\pgfpathlineto{\pgfqpoint{0.962306in}{0.773967in}}%
\pgfpathlineto{\pgfqpoint{0.962619in}{1.258809in}}%
\pgfpathlineto{\pgfqpoint{0.963140in}{1.035070in}}%
\pgfpathlineto{\pgfqpoint{0.963244in}{1.299285in}}%
\pgfpathlineto{\pgfqpoint{0.964077in}{0.846328in}}%
\pgfpathlineto{\pgfqpoint{0.964285in}{1.202418in}}%
\pgfpathlineto{\pgfqpoint{0.965118in}{1.326664in}}%
\pgfpathlineto{\pgfqpoint{0.965431in}{0.817500in}}%
\pgfpathlineto{\pgfqpoint{0.966264in}{1.278572in}}%
\pgfpathlineto{\pgfqpoint{0.966680in}{1.111197in}}%
\pgfpathlineto{\pgfqpoint{0.966784in}{1.237669in}}%
\pgfpathlineto{\pgfqpoint{0.966889in}{0.858443in}}%
\pgfpathlineto{\pgfqpoint{0.967409in}{0.966500in}}%
\pgfpathlineto{\pgfqpoint{0.967513in}{0.703137in}}%
\pgfpathlineto{\pgfqpoint{0.967722in}{1.188320in}}%
\pgfpathlineto{\pgfqpoint{0.968347in}{0.931650in}}%
\pgfpathlineto{\pgfqpoint{0.968451in}{1.374920in}}%
\pgfpathlineto{\pgfqpoint{0.969388in}{1.331795in}}%
\pgfpathlineto{\pgfqpoint{0.970013in}{0.776662in}}%
\pgfpathlineto{\pgfqpoint{0.970117in}{1.350128in}}%
\pgfpathlineto{\pgfqpoint{0.970534in}{1.105212in}}%
\pgfpathlineto{\pgfqpoint{0.971054in}{1.266717in}}%
\pgfpathlineto{\pgfqpoint{0.971263in}{0.900021in}}%
\pgfpathlineto{\pgfqpoint{0.971575in}{1.517072in}}%
\pgfpathlineto{\pgfqpoint{0.971887in}{0.748400in}}%
\pgfpathlineto{\pgfqpoint{0.972304in}{0.940413in}}%
\pgfpathlineto{\pgfqpoint{0.973033in}{0.749029in}}%
\pgfpathlineto{\pgfqpoint{0.972720in}{1.164475in}}%
\pgfpathlineto{\pgfqpoint{0.973137in}{1.007132in}}%
\pgfpathlineto{\pgfqpoint{0.973241in}{1.295512in}}%
\pgfpathlineto{\pgfqpoint{0.974074in}{0.657011in}}%
\pgfpathlineto{\pgfqpoint{0.974178in}{0.961667in}}%
\pgfpathlineto{\pgfqpoint{0.974595in}{0.847754in}}%
\pgfpathlineto{\pgfqpoint{0.975532in}{1.372841in}}%
\pgfpathlineto{\pgfqpoint{0.975741in}{0.782214in}}%
\pgfpathlineto{\pgfqpoint{0.976678in}{0.846187in}}%
\pgfpathlineto{\pgfqpoint{0.977094in}{1.194669in}}%
\pgfpathlineto{\pgfqpoint{0.976886in}{0.637232in}}%
\pgfpathlineto{\pgfqpoint{0.977719in}{1.166645in}}%
\pgfpathlineto{\pgfqpoint{0.977823in}{0.816866in}}%
\pgfpathlineto{\pgfqpoint{0.978344in}{1.467522in}}%
\pgfpathlineto{\pgfqpoint{0.978761in}{1.080788in}}%
\pgfpathlineto{\pgfqpoint{0.979281in}{1.268904in}}%
\pgfpathlineto{\pgfqpoint{0.979073in}{0.904421in}}%
\pgfpathlineto{\pgfqpoint{0.979802in}{1.095183in}}%
\pgfpathlineto{\pgfqpoint{0.980635in}{1.208956in}}%
\pgfpathlineto{\pgfqpoint{0.980948in}{0.828365in}}%
\pgfpathlineto{\pgfqpoint{0.981468in}{0.759361in}}%
\pgfpathlineto{\pgfqpoint{0.981989in}{1.300696in}}%
\pgfpathlineto{\pgfqpoint{0.982093in}{0.773433in}}%
\pgfpathlineto{\pgfqpoint{0.983135in}{1.086708in}}%
\pgfpathlineto{\pgfqpoint{0.984072in}{1.252585in}}%
\pgfpathlineto{\pgfqpoint{0.983864in}{0.833852in}}%
\pgfpathlineto{\pgfqpoint{0.984176in}{1.184314in}}%
\pgfpathlineto{\pgfqpoint{0.985113in}{1.283089in}}%
\pgfpathlineto{\pgfqpoint{0.985217in}{0.829411in}}%
\pgfpathlineto{\pgfqpoint{0.985322in}{1.293495in}}%
\pgfpathlineto{\pgfqpoint{0.985634in}{0.776609in}}%
\pgfpathlineto{\pgfqpoint{0.986363in}{1.099355in}}%
\pgfpathlineto{\pgfqpoint{0.986467in}{0.893286in}}%
\pgfpathlineto{\pgfqpoint{0.986675in}{1.318326in}}%
\pgfpathlineto{\pgfqpoint{0.987509in}{0.997154in}}%
\pgfpathlineto{\pgfqpoint{0.988133in}{1.294775in}}%
\pgfpathlineto{\pgfqpoint{0.988342in}{1.099939in}}%
\pgfpathlineto{\pgfqpoint{0.988550in}{0.667829in}}%
\pgfpathlineto{\pgfqpoint{0.989175in}{1.219793in}}%
\pgfpathlineto{\pgfqpoint{0.989383in}{0.944479in}}%
\pgfpathlineto{\pgfqpoint{0.989487in}{1.199583in}}%
\pgfpathlineto{\pgfqpoint{0.990424in}{0.784488in}}%
\pgfpathlineto{\pgfqpoint{0.990945in}{1.199027in}}%
\pgfpathlineto{\pgfqpoint{0.991674in}{1.126351in}}%
\pgfpathlineto{\pgfqpoint{0.992403in}{0.929876in}}%
\pgfpathlineto{\pgfqpoint{0.991987in}{1.320482in}}%
\pgfpathlineto{\pgfqpoint{0.992716in}{1.106962in}}%
\pgfpathlineto{\pgfqpoint{0.992820in}{1.399149in}}%
\pgfpathlineto{\pgfqpoint{0.993549in}{0.697838in}}%
\pgfpathlineto{\pgfqpoint{0.993861in}{1.215317in}}%
\pgfpathlineto{\pgfqpoint{0.994069in}{0.612478in}}%
\pgfpathlineto{\pgfqpoint{0.995007in}{0.634948in}}%
\pgfpathlineto{\pgfqpoint{0.995944in}{1.326832in}}%
\pgfpathlineto{\pgfqpoint{0.996152in}{1.101819in}}%
\pgfpathlineto{\pgfqpoint{0.996361in}{0.702895in}}%
\pgfpathlineto{\pgfqpoint{0.996985in}{1.328619in}}%
\pgfpathlineto{\pgfqpoint{0.997194in}{1.094788in}}%
\pgfpathlineto{\pgfqpoint{0.997402in}{0.823475in}}%
\pgfpathlineto{\pgfqpoint{0.997506in}{1.253246in}}%
\pgfpathlineto{\pgfqpoint{0.997923in}{0.880958in}}%
\pgfpathlineto{\pgfqpoint{0.998235in}{1.267594in}}%
\pgfpathlineto{\pgfqpoint{0.998443in}{0.783675in}}%
\pgfpathlineto{\pgfqpoint{0.999068in}{1.078298in}}%
\pgfpathlineto{\pgfqpoint{0.999172in}{1.376701in}}%
\pgfpathlineto{\pgfqpoint{0.999797in}{0.893858in}}%
\pgfpathlineto{\pgfqpoint{1.000214in}{1.263368in}}%
\pgfpathlineto{\pgfqpoint{1.001255in}{0.761113in}}%
\pgfpathlineto{\pgfqpoint{1.001359in}{0.889508in}}%
\pgfpathlineto{\pgfqpoint{1.001984in}{0.757255in}}%
\pgfpathlineto{\pgfqpoint{1.002609in}{1.361571in}}%
\pgfpathlineto{\pgfqpoint{1.002713in}{0.832140in}}%
\pgfpathlineto{\pgfqpoint{1.003755in}{0.990976in}}%
\pgfpathlineto{\pgfqpoint{1.004171in}{0.829895in}}%
\pgfpathlineto{\pgfqpoint{1.004067in}{1.231369in}}%
\pgfpathlineto{\pgfqpoint{1.004900in}{0.911114in}}%
\pgfpathlineto{\pgfqpoint{1.005421in}{1.161663in}}%
\pgfpathlineto{\pgfqpoint{1.005837in}{0.797801in}}%
\pgfpathlineto{\pgfqpoint{1.006046in}{1.110480in}}%
\pgfpathlineto{\pgfqpoint{1.006358in}{0.888590in}}%
\pgfpathlineto{\pgfqpoint{1.006983in}{1.233377in}}%
\pgfpathlineto{\pgfqpoint{1.007191in}{0.942189in}}%
\pgfpathlineto{\pgfqpoint{1.007608in}{1.303744in}}%
\pgfpathlineto{\pgfqpoint{1.007920in}{0.884743in}}%
\pgfpathlineto{\pgfqpoint{1.008233in}{1.123639in}}%
\pgfpathlineto{\pgfqpoint{1.008649in}{0.751145in}}%
\pgfpathlineto{\pgfqpoint{1.009066in}{1.340298in}}%
\pgfpathlineto{\pgfqpoint{1.009378in}{1.058182in}}%
\pgfpathlineto{\pgfqpoint{1.009482in}{1.047214in}}%
\pgfpathlineto{\pgfqpoint{1.009899in}{0.755137in}}%
\pgfpathlineto{\pgfqpoint{1.010420in}{1.159301in}}%
\pgfpathlineto{\pgfqpoint{1.010524in}{1.087512in}}%
\pgfpathlineto{\pgfqpoint{1.011044in}{0.933070in}}%
\pgfpathlineto{\pgfqpoint{1.011253in}{1.229735in}}%
\pgfpathlineto{\pgfqpoint{1.011357in}{0.966551in}}%
\pgfpathlineto{\pgfqpoint{1.011461in}{1.362267in}}%
\pgfpathlineto{\pgfqpoint{1.011982in}{0.798313in}}%
\pgfpathlineto{\pgfqpoint{1.012502in}{1.149041in}}%
\pgfpathlineto{\pgfqpoint{1.012607in}{0.705876in}}%
\pgfpathlineto{\pgfqpoint{1.012711in}{1.354063in}}%
\pgfpathlineto{\pgfqpoint{1.013648in}{0.953385in}}%
\pgfpathlineto{\pgfqpoint{1.013752in}{0.843773in}}%
\pgfpathlineto{\pgfqpoint{1.014481in}{1.177481in}}%
\pgfpathlineto{\pgfqpoint{1.014793in}{0.912571in}}%
\pgfpathlineto{\pgfqpoint{1.014898in}{1.212784in}}%
\pgfpathlineto{\pgfqpoint{1.015939in}{1.133607in}}%
\pgfpathlineto{\pgfqpoint{1.016043in}{1.182842in}}%
\pgfpathlineto{\pgfqpoint{1.016356in}{0.898731in}}%
\pgfpathlineto{\pgfqpoint{1.016460in}{0.687313in}}%
\pgfpathlineto{\pgfqpoint{1.017189in}{1.255321in}}%
\pgfpathlineto{\pgfqpoint{1.017293in}{0.894676in}}%
\pgfpathlineto{\pgfqpoint{1.017397in}{1.238535in}}%
\pgfpathlineto{\pgfqpoint{1.018334in}{0.726610in}}%
\pgfpathlineto{\pgfqpoint{1.018438in}{1.087461in}}%
\pgfpathlineto{\pgfqpoint{1.019272in}{0.854895in}}%
\pgfpathlineto{\pgfqpoint{1.018855in}{1.226499in}}%
\pgfpathlineto{\pgfqpoint{1.019480in}{1.135219in}}%
\pgfpathlineto{\pgfqpoint{1.019584in}{1.154067in}}%
\pgfpathlineto{\pgfqpoint{1.019688in}{0.769745in}}%
\pgfpathlineto{\pgfqpoint{1.020209in}{1.273973in}}%
\pgfpathlineto{\pgfqpoint{1.020625in}{1.097725in}}%
\pgfpathlineto{\pgfqpoint{1.021459in}{1.325762in}}%
\pgfpathlineto{\pgfqpoint{1.021146in}{0.826320in}}%
\pgfpathlineto{\pgfqpoint{1.021667in}{1.224941in}}%
\pgfpathlineto{\pgfqpoint{1.021875in}{0.884343in}}%
\pgfpathlineto{\pgfqpoint{1.022604in}{1.244348in}}%
\pgfpathlineto{\pgfqpoint{1.022708in}{1.134093in}}%
\pgfpathlineto{\pgfqpoint{1.022812in}{1.178382in}}%
\pgfpathlineto{\pgfqpoint{1.023021in}{0.986444in}}%
\pgfpathlineto{\pgfqpoint{1.023125in}{1.165277in}}%
\pgfpathlineto{\pgfqpoint{1.023958in}{0.819969in}}%
\pgfpathlineto{\pgfqpoint{1.023333in}{1.208942in}}%
\pgfpathlineto{\pgfqpoint{1.024166in}{0.969299in}}%
\pgfpathlineto{\pgfqpoint{1.024999in}{1.300505in}}%
\pgfpathlineto{\pgfqpoint{1.024583in}{0.913545in}}%
\pgfpathlineto{\pgfqpoint{1.025208in}{1.212313in}}%
\pgfpathlineto{\pgfqpoint{1.025312in}{0.735538in}}%
\pgfpathlineto{\pgfqpoint{1.026249in}{1.093178in}}%
\pgfpathlineto{\pgfqpoint{1.026666in}{1.209518in}}%
\pgfpathlineto{\pgfqpoint{1.026561in}{0.879852in}}%
\pgfpathlineto{\pgfqpoint{1.026978in}{1.028042in}}%
\pgfpathlineto{\pgfqpoint{1.027082in}{0.920118in}}%
\pgfpathlineto{\pgfqpoint{1.027707in}{1.211944in}}%
\pgfpathlineto{\pgfqpoint{1.027811in}{1.127070in}}%
\pgfpathlineto{\pgfqpoint{1.027915in}{1.358908in}}%
\pgfpathlineto{\pgfqpoint{1.028228in}{0.850352in}}%
\pgfpathlineto{\pgfqpoint{1.028853in}{1.127721in}}%
\pgfpathlineto{\pgfqpoint{1.029894in}{0.786731in}}%
\pgfpathlineto{\pgfqpoint{1.029790in}{1.339836in}}%
\pgfpathlineto{\pgfqpoint{1.029998in}{0.966399in}}%
\pgfpathlineto{\pgfqpoint{1.030727in}{1.258769in}}%
\pgfpathlineto{\pgfqpoint{1.031039in}{0.912283in}}%
\pgfpathlineto{\pgfqpoint{1.031144in}{1.042512in}}%
\pgfpathlineto{\pgfqpoint{1.031664in}{1.150421in}}%
\pgfpathlineto{\pgfqpoint{1.031768in}{0.897005in}}%
\pgfpathlineto{\pgfqpoint{1.031873in}{1.306624in}}%
\pgfpathlineto{\pgfqpoint{1.032497in}{0.873656in}}%
\pgfpathlineto{\pgfqpoint{1.032810in}{0.966583in}}%
\pgfpathlineto{\pgfqpoint{1.033539in}{1.321530in}}%
\pgfpathlineto{\pgfqpoint{1.033435in}{0.881325in}}%
\pgfpathlineto{\pgfqpoint{1.033851in}{1.084933in}}%
\pgfpathlineto{\pgfqpoint{1.034789in}{0.895588in}}%
\pgfpathlineto{\pgfqpoint{1.034372in}{1.229387in}}%
\pgfpathlineto{\pgfqpoint{1.034893in}{0.964061in}}%
\pgfpathlineto{\pgfqpoint{1.036142in}{1.294291in}}%
\pgfpathlineto{\pgfqpoint{1.035205in}{0.735260in}}%
\pgfpathlineto{\pgfqpoint{1.036247in}{1.246802in}}%
\pgfpathlineto{\pgfqpoint{1.037080in}{0.757530in}}%
\pgfpathlineto{\pgfqpoint{1.037184in}{1.265930in}}%
\pgfpathlineto{\pgfqpoint{1.037392in}{1.062374in}}%
\pgfpathlineto{\pgfqpoint{1.037705in}{0.637463in}}%
\pgfpathlineto{\pgfqpoint{1.037600in}{1.222473in}}%
\pgfpathlineto{\pgfqpoint{1.038017in}{0.987551in}}%
\pgfpathlineto{\pgfqpoint{1.038121in}{1.307679in}}%
\pgfpathlineto{\pgfqpoint{1.039058in}{0.827285in}}%
\pgfpathlineto{\pgfqpoint{1.039371in}{1.177202in}}%
\pgfpathlineto{\pgfqpoint{1.039579in}{0.626191in}}%
\pgfpathlineto{\pgfqpoint{1.040204in}{1.073059in}}%
\pgfpathlineto{\pgfqpoint{1.040933in}{0.816745in}}%
\pgfpathlineto{\pgfqpoint{1.040620in}{1.291535in}}%
\pgfpathlineto{\pgfqpoint{1.041141in}{1.017659in}}%
\pgfpathlineto{\pgfqpoint{1.041662in}{1.370637in}}%
\pgfpathlineto{\pgfqpoint{1.041870in}{0.915975in}}%
\pgfpathlineto{\pgfqpoint{1.042183in}{1.271807in}}%
\pgfpathlineto{\pgfqpoint{1.042912in}{1.298739in}}%
\pgfpathlineto{\pgfqpoint{1.043328in}{0.814597in}}%
\pgfpathlineto{\pgfqpoint{1.044474in}{1.270202in}}%
\pgfpathlineto{\pgfqpoint{1.045099in}{0.842167in}}%
\pgfpathlineto{\pgfqpoint{1.045619in}{1.131549in}}%
\pgfpathlineto{\pgfqpoint{1.046452in}{0.841932in}}%
\pgfpathlineto{\pgfqpoint{1.046036in}{1.241056in}}%
\pgfpathlineto{\pgfqpoint{1.046661in}{1.139051in}}%
\pgfpathlineto{\pgfqpoint{1.046765in}{1.138960in}}%
\pgfpathlineto{\pgfqpoint{1.047390in}{0.964830in}}%
\pgfpathlineto{\pgfqpoint{1.046973in}{1.260789in}}%
\pgfpathlineto{\pgfqpoint{1.047910in}{1.048605in}}%
\pgfpathlineto{\pgfqpoint{1.048014in}{1.404920in}}%
\pgfpathlineto{\pgfqpoint{1.048743in}{1.009590in}}%
\pgfpathlineto{\pgfqpoint{1.048848in}{1.042287in}}%
\pgfpathlineto{\pgfqpoint{1.048952in}{0.588571in}}%
\pgfpathlineto{\pgfqpoint{1.049264in}{1.288065in}}%
\pgfpathlineto{\pgfqpoint{1.049889in}{1.096898in}}%
\pgfpathlineto{\pgfqpoint{1.050306in}{0.853707in}}%
\pgfpathlineto{\pgfqpoint{1.050930in}{1.289779in}}%
\pgfpathlineto{\pgfqpoint{1.052180in}{0.834288in}}%
\pgfpathlineto{\pgfqpoint{1.052284in}{1.242285in}}%
\pgfpathlineto{\pgfqpoint{1.053326in}{1.017050in}}%
\pgfpathlineto{\pgfqpoint{1.054055in}{0.713416in}}%
\pgfpathlineto{\pgfqpoint{1.054367in}{1.230759in}}%
\pgfpathlineto{\pgfqpoint{1.054575in}{0.587649in}}%
\pgfpathlineto{\pgfqpoint{1.055096in}{1.458793in}}%
\pgfpathlineto{\pgfqpoint{1.055513in}{0.747195in}}%
\pgfpathlineto{\pgfqpoint{1.056658in}{1.162724in}}%
\pgfpathlineto{\pgfqpoint{1.056762in}{1.128630in}}%
\pgfpathlineto{\pgfqpoint{1.057595in}{1.263780in}}%
\pgfpathlineto{\pgfqpoint{1.057908in}{0.773077in}}%
\pgfpathlineto{\pgfqpoint{1.058429in}{1.248082in}}%
\pgfpathlineto{\pgfqpoint{1.059053in}{0.929977in}}%
\pgfpathlineto{\pgfqpoint{1.059678in}{1.189794in}}%
\pgfpathlineto{\pgfqpoint{1.059991in}{0.879949in}}%
\pgfpathlineto{\pgfqpoint{1.060199in}{1.145183in}}%
\pgfpathlineto{\pgfqpoint{1.060303in}{0.773480in}}%
\pgfpathlineto{\pgfqpoint{1.060720in}{1.252553in}}%
\pgfpathlineto{\pgfqpoint{1.061240in}{1.033368in}}%
\pgfpathlineto{\pgfqpoint{1.062074in}{1.428510in}}%
\pgfpathlineto{\pgfqpoint{1.061969in}{0.668698in}}%
\pgfpathlineto{\pgfqpoint{1.062490in}{1.398352in}}%
\pgfpathlineto{\pgfqpoint{1.063531in}{0.806765in}}%
\pgfpathlineto{\pgfqpoint{1.063740in}{1.048835in}}%
\pgfpathlineto{\pgfqpoint{1.063844in}{1.075749in}}%
\pgfpathlineto{\pgfqpoint{1.063948in}{0.977492in}}%
\pgfpathlineto{\pgfqpoint{1.064052in}{1.506029in}}%
\pgfpathlineto{\pgfqpoint{1.064677in}{0.857371in}}%
\pgfpathlineto{\pgfqpoint{1.065094in}{1.200114in}}%
\pgfpathlineto{\pgfqpoint{1.065823in}{1.290540in}}%
\pgfpathlineto{\pgfqpoint{1.065302in}{0.928991in}}%
\pgfpathlineto{\pgfqpoint{1.066031in}{1.147567in}}%
\pgfpathlineto{\pgfqpoint{1.066760in}{0.806202in}}%
\pgfpathlineto{\pgfqpoint{1.066968in}{1.152698in}}%
\pgfpathlineto{\pgfqpoint{1.067176in}{0.996621in}}%
\pgfpathlineto{\pgfqpoint{1.067905in}{0.884772in}}%
\pgfpathlineto{\pgfqpoint{1.067697in}{1.190490in}}%
\pgfpathlineto{\pgfqpoint{1.068218in}{0.978061in}}%
\pgfpathlineto{\pgfqpoint{1.069259in}{1.298501in}}%
\pgfpathlineto{\pgfqpoint{1.068634in}{0.867380in}}%
\pgfpathlineto{\pgfqpoint{1.069363in}{1.098997in}}%
\pgfpathlineto{\pgfqpoint{1.069572in}{1.261854in}}%
\pgfpathlineto{\pgfqpoint{1.069780in}{0.980561in}}%
\pgfpathlineto{\pgfqpoint{1.069988in}{1.089204in}}%
\pgfpathlineto{\pgfqpoint{1.070405in}{0.732369in}}%
\pgfpathlineto{\pgfqpoint{1.070509in}{1.279520in}}%
\pgfpathlineto{\pgfqpoint{1.071030in}{1.234803in}}%
\pgfpathlineto{\pgfqpoint{1.071238in}{0.848560in}}%
\pgfpathlineto{\pgfqpoint{1.071759in}{1.310897in}}%
\pgfpathlineto{\pgfqpoint{1.072383in}{1.089446in}}%
\pgfpathlineto{\pgfqpoint{1.072488in}{1.238202in}}%
\pgfpathlineto{\pgfqpoint{1.072904in}{0.850142in}}%
\pgfpathlineto{\pgfqpoint{1.073425in}{1.007360in}}%
\pgfpathlineto{\pgfqpoint{1.073529in}{0.902935in}}%
\pgfpathlineto{\pgfqpoint{1.074154in}{1.218539in}}%
\pgfpathlineto{\pgfqpoint{1.074258in}{1.010927in}}%
\pgfpathlineto{\pgfqpoint{1.074466in}{1.278678in}}%
\pgfpathlineto{\pgfqpoint{1.074779in}{0.747842in}}%
\pgfpathlineto{\pgfqpoint{1.075299in}{1.111387in}}%
\pgfpathlineto{\pgfqpoint{1.075404in}{0.728086in}}%
\pgfpathlineto{\pgfqpoint{1.076028in}{1.294399in}}%
\pgfpathlineto{\pgfqpoint{1.076341in}{1.082455in}}%
\pgfpathlineto{\pgfqpoint{1.077278in}{1.511323in}}%
\pgfpathlineto{\pgfqpoint{1.076549in}{0.745865in}}%
\pgfpathlineto{\pgfqpoint{1.077382in}{1.172836in}}%
\pgfpathlineto{\pgfqpoint{1.077799in}{0.790468in}}%
\pgfpathlineto{\pgfqpoint{1.078215in}{1.208147in}}%
\pgfpathlineto{\pgfqpoint{1.078528in}{1.033386in}}%
\pgfpathlineto{\pgfqpoint{1.079361in}{1.354079in}}%
\pgfpathlineto{\pgfqpoint{1.079257in}{0.745073in}}%
\pgfpathlineto{\pgfqpoint{1.079569in}{1.112641in}}%
\pgfpathlineto{\pgfqpoint{1.079986in}{0.819288in}}%
\pgfpathlineto{\pgfqpoint{1.080090in}{1.388577in}}%
\pgfpathlineto{\pgfqpoint{1.080402in}{0.989515in}}%
\pgfpathlineto{\pgfqpoint{1.081235in}{1.405799in}}%
\pgfpathlineto{\pgfqpoint{1.080715in}{0.872635in}}%
\pgfpathlineto{\pgfqpoint{1.081548in}{1.292649in}}%
\pgfpathlineto{\pgfqpoint{1.082381in}{0.749517in}}%
\pgfpathlineto{\pgfqpoint{1.082693in}{1.026755in}}%
\pgfpathlineto{\pgfqpoint{1.082798in}{1.161238in}}%
\pgfpathlineto{\pgfqpoint{1.083214in}{0.875507in}}%
\pgfpathlineto{\pgfqpoint{1.083735in}{0.991919in}}%
\pgfpathlineto{\pgfqpoint{1.083943in}{1.207917in}}%
\pgfpathlineto{\pgfqpoint{1.084047in}{1.004937in}}%
\pgfpathlineto{\pgfqpoint{1.084151in}{0.857226in}}%
\pgfpathlineto{\pgfqpoint{1.084568in}{1.290255in}}%
\pgfpathlineto{\pgfqpoint{1.084985in}{1.021910in}}%
\pgfpathlineto{\pgfqpoint{1.085818in}{1.271236in}}%
\pgfpathlineto{\pgfqpoint{1.085193in}{0.825753in}}%
\pgfpathlineto{\pgfqpoint{1.086130in}{1.066575in}}%
\pgfpathlineto{\pgfqpoint{1.086963in}{0.786779in}}%
\pgfpathlineto{\pgfqpoint{1.086443in}{1.395565in}}%
\pgfpathlineto{\pgfqpoint{1.087172in}{0.849752in}}%
\pgfpathlineto{\pgfqpoint{1.087276in}{1.304430in}}%
\pgfpathlineto{\pgfqpoint{1.088317in}{1.065643in}}%
\pgfpathlineto{\pgfqpoint{1.089150in}{1.427096in}}%
\pgfpathlineto{\pgfqpoint{1.088838in}{0.896492in}}%
\pgfpathlineto{\pgfqpoint{1.089254in}{1.048312in}}%
\pgfpathlineto{\pgfqpoint{1.089879in}{0.821780in}}%
\pgfpathlineto{\pgfqpoint{1.089671in}{1.299916in}}%
\pgfpathlineto{\pgfqpoint{1.090296in}{1.016337in}}%
\pgfpathlineto{\pgfqpoint{1.091233in}{1.178094in}}%
\pgfpathlineto{\pgfqpoint{1.090608in}{0.953774in}}%
\pgfpathlineto{\pgfqpoint{1.091441in}{1.120783in}}%
\pgfpathlineto{\pgfqpoint{1.091545in}{0.821810in}}%
\pgfpathlineto{\pgfqpoint{1.092379in}{1.206003in}}%
\pgfpathlineto{\pgfqpoint{1.092483in}{0.994332in}}%
\pgfpathlineto{\pgfqpoint{1.093212in}{1.388391in}}%
\pgfpathlineto{\pgfqpoint{1.093316in}{0.929015in}}%
\pgfpathlineto{\pgfqpoint{1.093524in}{0.976453in}}%
\pgfpathlineto{\pgfqpoint{1.093628in}{0.926509in}}%
\pgfpathlineto{\pgfqpoint{1.093837in}{1.074342in}}%
\pgfpathlineto{\pgfqpoint{1.093941in}{1.482725in}}%
\pgfpathlineto{\pgfqpoint{1.094357in}{0.749495in}}%
\pgfpathlineto{\pgfqpoint{1.094878in}{0.980218in}}%
\pgfpathlineto{\pgfqpoint{1.095399in}{0.867135in}}%
\pgfpathlineto{\pgfqpoint{1.095815in}{1.318723in}}%
\pgfpathlineto{\pgfqpoint{1.096857in}{0.767058in}}%
\pgfpathlineto{\pgfqpoint{1.096961in}{0.999989in}}%
\pgfpathlineto{\pgfqpoint{1.097273in}{1.308405in}}%
\pgfpathlineto{\pgfqpoint{1.097794in}{0.782988in}}%
\pgfpathlineto{\pgfqpoint{1.098002in}{1.205351in}}%
\pgfpathlineto{\pgfqpoint{1.098835in}{1.403942in}}%
\pgfpathlineto{\pgfqpoint{1.099252in}{0.903514in}}%
\pgfpathlineto{\pgfqpoint{1.099981in}{1.128229in}}%
\pgfpathlineto{\pgfqpoint{1.099564in}{0.895783in}}%
\pgfpathlineto{\pgfqpoint{1.100293in}{0.978700in}}%
\pgfpathlineto{\pgfqpoint{1.100397in}{0.878096in}}%
\pgfpathlineto{\pgfqpoint{1.100502in}{1.316277in}}%
\pgfpathlineto{\pgfqpoint{1.100814in}{0.917771in}}%
\pgfpathlineto{\pgfqpoint{1.100918in}{1.485292in}}%
\pgfpathlineto{\pgfqpoint{1.101022in}{0.896545in}}%
\pgfpathlineto{\pgfqpoint{1.101855in}{1.016436in}}%
\pgfpathlineto{\pgfqpoint{1.102064in}{1.357904in}}%
\pgfpathlineto{\pgfqpoint{1.102168in}{0.848893in}}%
\pgfpathlineto{\pgfqpoint{1.103001in}{1.196924in}}%
\pgfpathlineto{\pgfqpoint{1.103313in}{0.825252in}}%
\pgfpathlineto{\pgfqpoint{1.104251in}{0.993595in}}%
\pgfpathlineto{\pgfqpoint{1.104459in}{0.815233in}}%
\pgfpathlineto{\pgfqpoint{1.104563in}{1.351202in}}%
\pgfpathlineto{\pgfqpoint{1.104980in}{1.115478in}}%
\pgfpathlineto{\pgfqpoint{1.105500in}{0.857990in}}%
\pgfpathlineto{\pgfqpoint{1.105813in}{1.434223in}}%
\pgfpathlineto{\pgfqpoint{1.106333in}{0.838219in}}%
\pgfpathlineto{\pgfqpoint{1.106958in}{1.204977in}}%
\pgfpathlineto{\pgfqpoint{1.108000in}{0.834449in}}%
\pgfpathlineto{\pgfqpoint{1.107167in}{1.291901in}}%
\pgfpathlineto{\pgfqpoint{1.108104in}{0.919324in}}%
\pgfpathlineto{\pgfqpoint{1.108208in}{1.248425in}}%
\pgfpathlineto{\pgfqpoint{1.109145in}{1.135466in}}%
\pgfpathlineto{\pgfqpoint{1.109770in}{0.910435in}}%
\pgfpathlineto{\pgfqpoint{1.109978in}{1.235351in}}%
\pgfpathlineto{\pgfqpoint{1.110291in}{0.957731in}}%
\pgfpathlineto{\pgfqpoint{1.111124in}{1.213392in}}%
\pgfpathlineto{\pgfqpoint{1.110499in}{0.832410in}}%
\pgfpathlineto{\pgfqpoint{1.111436in}{1.052822in}}%
\pgfpathlineto{\pgfqpoint{1.111853in}{0.797626in}}%
\pgfpathlineto{\pgfqpoint{1.112269in}{1.229416in}}%
\pgfpathlineto{\pgfqpoint{1.112686in}{0.936460in}}%
\pgfpathlineto{\pgfqpoint{1.113519in}{1.300135in}}%
\pgfpathlineto{\pgfqpoint{1.113207in}{0.804234in}}%
\pgfpathlineto{\pgfqpoint{1.113623in}{1.013814in}}%
\pgfpathlineto{\pgfqpoint{1.113727in}{0.730902in}}%
\pgfpathlineto{\pgfqpoint{1.114352in}{1.224856in}}%
\pgfpathlineto{\pgfqpoint{1.114665in}{0.814498in}}%
\pgfpathlineto{\pgfqpoint{1.114873in}{0.788794in}}%
\pgfpathlineto{\pgfqpoint{1.116019in}{1.287517in}}%
\pgfpathlineto{\pgfqpoint{1.116852in}{0.723825in}}%
\pgfpathlineto{\pgfqpoint{1.116748in}{1.431886in}}%
\pgfpathlineto{\pgfqpoint{1.117164in}{0.968051in}}%
\pgfpathlineto{\pgfqpoint{1.117372in}{0.920711in}}%
\pgfpathlineto{\pgfqpoint{1.117477in}{1.301555in}}%
\pgfpathlineto{\pgfqpoint{1.118101in}{0.734321in}}%
\pgfpathlineto{\pgfqpoint{1.118518in}{1.076102in}}%
\pgfpathlineto{\pgfqpoint{1.119039in}{0.776205in}}%
\pgfpathlineto{\pgfqpoint{1.118726in}{1.172119in}}%
\pgfpathlineto{\pgfqpoint{1.119664in}{0.929116in}}%
\pgfpathlineto{\pgfqpoint{1.119976in}{1.223095in}}%
\pgfpathlineto{\pgfqpoint{1.119872in}{0.829254in}}%
\pgfpathlineto{\pgfqpoint{1.120809in}{1.148834in}}%
\pgfpathlineto{\pgfqpoint{1.121017in}{0.763986in}}%
\pgfpathlineto{\pgfqpoint{1.121538in}{1.190156in}}%
\pgfpathlineto{\pgfqpoint{1.121850in}{1.150493in}}%
\pgfpathlineto{\pgfqpoint{1.121955in}{1.195727in}}%
\pgfpathlineto{\pgfqpoint{1.122163in}{0.940914in}}%
\pgfpathlineto{\pgfqpoint{1.122267in}{0.793650in}}%
\pgfpathlineto{\pgfqpoint{1.122788in}{1.155198in}}%
\pgfpathlineto{\pgfqpoint{1.122996in}{1.045793in}}%
\pgfpathlineto{\pgfqpoint{1.123100in}{1.228047in}}%
\pgfpathlineto{\pgfqpoint{1.123621in}{0.929252in}}%
\pgfpathlineto{\pgfqpoint{1.123933in}{1.021585in}}%
\pgfpathlineto{\pgfqpoint{1.124350in}{0.847551in}}%
\pgfpathlineto{\pgfqpoint{1.124142in}{1.313838in}}%
\pgfpathlineto{\pgfqpoint{1.124871in}{1.197509in}}%
\pgfpathlineto{\pgfqpoint{1.124975in}{1.259695in}}%
\pgfpathlineto{\pgfqpoint{1.125287in}{0.901981in}}%
\pgfpathlineto{\pgfqpoint{1.125704in}{1.214271in}}%
\pgfpathlineto{\pgfqpoint{1.125912in}{0.765255in}}%
\pgfpathlineto{\pgfqpoint{1.126537in}{1.276724in}}%
\pgfpathlineto{\pgfqpoint{1.126849in}{0.980713in}}%
\pgfpathlineto{\pgfqpoint{1.127578in}{1.341986in}}%
\pgfpathlineto{\pgfqpoint{1.127891in}{1.059572in}}%
\pgfpathlineto{\pgfqpoint{1.127995in}{0.704018in}}%
\pgfpathlineto{\pgfqpoint{1.128724in}{1.204203in}}%
\pgfpathlineto{\pgfqpoint{1.128828in}{0.998547in}}%
\pgfpathlineto{\pgfqpoint{1.128932in}{1.360024in}}%
\pgfpathlineto{\pgfqpoint{1.129453in}{0.731910in}}%
\pgfpathlineto{\pgfqpoint{1.129869in}{1.078461in}}%
\pgfpathlineto{\pgfqpoint{1.130390in}{0.682362in}}%
\pgfpathlineto{\pgfqpoint{1.130182in}{1.177792in}}%
\pgfpathlineto{\pgfqpoint{1.130911in}{1.147807in}}%
\pgfpathlineto{\pgfqpoint{1.131223in}{0.831099in}}%
\pgfpathlineto{\pgfqpoint{1.131119in}{1.229743in}}%
\pgfpathlineto{\pgfqpoint{1.132265in}{0.964348in}}%
\pgfpathlineto{\pgfqpoint{1.132889in}{0.944149in}}%
\pgfpathlineto{\pgfqpoint{1.133306in}{1.421445in}}%
\pgfpathlineto{\pgfqpoint{1.134347in}{0.808747in}}%
\pgfpathlineto{\pgfqpoint{1.134452in}{1.353153in}}%
\pgfpathlineto{\pgfqpoint{1.135493in}{1.065269in}}%
\pgfpathlineto{\pgfqpoint{1.136534in}{1.213908in}}%
\pgfpathlineto{\pgfqpoint{1.135910in}{0.900627in}}%
\pgfpathlineto{\pgfqpoint{1.136639in}{1.155739in}}%
\pgfpathlineto{\pgfqpoint{1.137472in}{0.766721in}}%
\pgfpathlineto{\pgfqpoint{1.137888in}{1.056473in}}%
\pgfpathlineto{\pgfqpoint{1.137992in}{1.333238in}}%
\pgfpathlineto{\pgfqpoint{1.138721in}{0.651908in}}%
\pgfpathlineto{\pgfqpoint{1.138930in}{1.280378in}}%
\pgfpathlineto{\pgfqpoint{1.139554in}{0.798232in}}%
\pgfpathlineto{\pgfqpoint{1.140075in}{1.026532in}}%
\pgfpathlineto{\pgfqpoint{1.140179in}{1.038958in}}%
\pgfpathlineto{\pgfqpoint{1.140283in}{1.022761in}}%
\pgfpathlineto{\pgfqpoint{1.141221in}{1.313595in}}%
\pgfpathlineto{\pgfqpoint{1.140700in}{0.867522in}}%
\pgfpathlineto{\pgfqpoint{1.141325in}{1.206697in}}%
\pgfpathlineto{\pgfqpoint{1.141637in}{0.679164in}}%
\pgfpathlineto{\pgfqpoint{1.142470in}{1.124170in}}%
\pgfpathlineto{\pgfqpoint{1.142575in}{1.306163in}}%
\pgfpathlineto{\pgfqpoint{1.142991in}{0.935085in}}%
\pgfpathlineto{\pgfqpoint{1.143512in}{1.098000in}}%
\pgfpathlineto{\pgfqpoint{1.143824in}{1.360325in}}%
\pgfpathlineto{\pgfqpoint{1.144449in}{0.638120in}}%
\pgfpathlineto{\pgfqpoint{1.144970in}{1.284469in}}%
\pgfpathlineto{\pgfqpoint{1.145595in}{1.219860in}}%
\pgfpathlineto{\pgfqpoint{1.146636in}{0.837035in}}%
\pgfpathlineto{\pgfqpoint{1.146115in}{1.256608in}}%
\pgfpathlineto{\pgfqpoint{1.146740in}{0.982558in}}%
\pgfpathlineto{\pgfqpoint{1.147157in}{0.800069in}}%
\pgfpathlineto{\pgfqpoint{1.147365in}{1.420933in}}%
\pgfpathlineto{\pgfqpoint{1.147573in}{0.960018in}}%
\pgfpathlineto{\pgfqpoint{1.147677in}{1.468943in}}%
\pgfpathlineto{\pgfqpoint{1.147782in}{0.860471in}}%
\pgfpathlineto{\pgfqpoint{1.148615in}{0.906296in}}%
\pgfpathlineto{\pgfqpoint{1.149031in}{1.238497in}}%
\pgfpathlineto{\pgfqpoint{1.149552in}{0.797689in}}%
\pgfpathlineto{\pgfqpoint{1.149656in}{1.102875in}}%
\pgfpathlineto{\pgfqpoint{1.149760in}{0.900388in}}%
\pgfpathlineto{\pgfqpoint{1.149864in}{1.305447in}}%
\pgfpathlineto{\pgfqpoint{1.150802in}{0.965489in}}%
\pgfpathlineto{\pgfqpoint{1.151427in}{1.333013in}}%
\pgfpathlineto{\pgfqpoint{1.151010in}{0.747857in}}%
\pgfpathlineto{\pgfqpoint{1.151947in}{1.112117in}}%
\pgfpathlineto{\pgfqpoint{1.152676in}{1.307081in}}%
\pgfpathlineto{\pgfqpoint{1.153197in}{0.729869in}}%
\pgfpathlineto{\pgfqpoint{1.154342in}{1.261536in}}%
\pgfpathlineto{\pgfqpoint{1.155071in}{0.794373in}}%
\pgfpathlineto{\pgfqpoint{1.155488in}{1.051542in}}%
\pgfpathlineto{\pgfqpoint{1.156009in}{0.855037in}}%
\pgfpathlineto{\pgfqpoint{1.155905in}{1.258444in}}%
\pgfpathlineto{\pgfqpoint{1.156529in}{0.978001in}}%
\pgfpathlineto{\pgfqpoint{1.156634in}{1.179469in}}%
\pgfpathlineto{\pgfqpoint{1.157050in}{0.847863in}}%
\pgfpathlineto{\pgfqpoint{1.157571in}{1.084287in}}%
\pgfpathlineto{\pgfqpoint{1.157675in}{0.815765in}}%
\pgfpathlineto{\pgfqpoint{1.157779in}{1.331501in}}%
\pgfpathlineto{\pgfqpoint{1.158612in}{1.058457in}}%
\pgfpathlineto{\pgfqpoint{1.159133in}{1.362309in}}%
\pgfpathlineto{\pgfqpoint{1.158821in}{0.873163in}}%
\pgfpathlineto{\pgfqpoint{1.159550in}{1.339403in}}%
\pgfpathlineto{\pgfqpoint{1.159966in}{0.686063in}}%
\pgfpathlineto{\pgfqpoint{1.160487in}{1.509177in}}%
\pgfpathlineto{\pgfqpoint{1.160695in}{1.129086in}}%
\pgfpathlineto{\pgfqpoint{1.161528in}{0.715781in}}%
\pgfpathlineto{\pgfqpoint{1.161216in}{1.321288in}}%
\pgfpathlineto{\pgfqpoint{1.161632in}{1.001154in}}%
\pgfpathlineto{\pgfqpoint{1.161736in}{1.451129in}}%
\pgfpathlineto{\pgfqpoint{1.162361in}{0.696528in}}%
\pgfpathlineto{\pgfqpoint{1.162674in}{1.080039in}}%
\pgfpathlineto{\pgfqpoint{1.163507in}{0.808263in}}%
\pgfpathlineto{\pgfqpoint{1.163299in}{1.229945in}}%
\pgfpathlineto{\pgfqpoint{1.163715in}{1.059219in}}%
\pgfpathlineto{\pgfqpoint{1.163923in}{1.254209in}}%
\pgfpathlineto{\pgfqpoint{1.164028in}{1.019446in}}%
\pgfpathlineto{\pgfqpoint{1.164132in}{1.036603in}}%
\pgfpathlineto{\pgfqpoint{1.164757in}{1.266292in}}%
\pgfpathlineto{\pgfqpoint{1.165381in}{0.742254in}}%
\pgfpathlineto{\pgfqpoint{1.165694in}{1.484061in}}%
\pgfpathlineto{\pgfqpoint{1.166423in}{0.910458in}}%
\pgfpathlineto{\pgfqpoint{1.166735in}{0.727623in}}%
\pgfpathlineto{\pgfqpoint{1.166944in}{1.009969in}}%
\pgfpathlineto{\pgfqpoint{1.167048in}{0.961439in}}%
\pgfpathlineto{\pgfqpoint{1.167152in}{1.236773in}}%
\pgfpathlineto{\pgfqpoint{1.167464in}{0.790720in}}%
\pgfpathlineto{\pgfqpoint{1.168089in}{0.997404in}}%
\pgfpathlineto{\pgfqpoint{1.168402in}{1.107093in}}%
\pgfpathlineto{\pgfqpoint{1.168506in}{0.920436in}}%
\pgfpathlineto{\pgfqpoint{1.169339in}{1.258662in}}%
\pgfpathlineto{\pgfqpoint{1.169235in}{0.701175in}}%
\pgfpathlineto{\pgfqpoint{1.169547in}{0.833815in}}%
\pgfpathlineto{\pgfqpoint{1.169651in}{0.837002in}}%
\pgfpathlineto{\pgfqpoint{1.169755in}{1.299754in}}%
\pgfpathlineto{\pgfqpoint{1.170068in}{0.721385in}}%
\pgfpathlineto{\pgfqpoint{1.170797in}{1.202016in}}%
\pgfpathlineto{\pgfqpoint{1.171213in}{1.214510in}}%
\pgfpathlineto{\pgfqpoint{1.172046in}{0.628356in}}%
\pgfpathlineto{\pgfqpoint{1.172151in}{1.299619in}}%
\pgfpathlineto{\pgfqpoint{1.173192in}{1.149024in}}%
\pgfpathlineto{\pgfqpoint{1.173817in}{0.796821in}}%
\pgfpathlineto{\pgfqpoint{1.173400in}{1.238077in}}%
\pgfpathlineto{\pgfqpoint{1.174233in}{1.227607in}}%
\pgfpathlineto{\pgfqpoint{1.174754in}{0.714337in}}%
\pgfpathlineto{\pgfqpoint{1.174858in}{1.299927in}}%
\pgfpathlineto{\pgfqpoint{1.175275in}{1.236943in}}%
\pgfpathlineto{\pgfqpoint{1.175379in}{1.381882in}}%
\pgfpathlineto{\pgfqpoint{1.175796in}{0.827359in}}%
\pgfpathlineto{\pgfqpoint{1.176212in}{1.225393in}}%
\pgfpathlineto{\pgfqpoint{1.176316in}{0.809394in}}%
\pgfpathlineto{\pgfqpoint{1.177358in}{0.988723in}}%
\pgfpathlineto{\pgfqpoint{1.177670in}{1.363348in}}%
\pgfpathlineto{\pgfqpoint{1.177878in}{0.852167in}}%
\pgfpathlineto{\pgfqpoint{1.178399in}{1.103802in}}%
\pgfpathlineto{\pgfqpoint{1.178920in}{0.737880in}}%
\pgfpathlineto{\pgfqpoint{1.179232in}{1.296970in}}%
\pgfpathlineto{\pgfqpoint{1.179440in}{1.089395in}}%
\pgfpathlineto{\pgfqpoint{1.179857in}{1.346617in}}%
\pgfpathlineto{\pgfqpoint{1.179649in}{0.907542in}}%
\pgfpathlineto{\pgfqpoint{1.180169in}{0.970587in}}%
\pgfpathlineto{\pgfqpoint{1.180690in}{1.133145in}}%
\pgfpathlineto{\pgfqpoint{1.180898in}{0.907562in}}%
\pgfpathlineto{\pgfqpoint{1.181315in}{1.223445in}}%
\pgfpathlineto{\pgfqpoint{1.181419in}{0.875386in}}%
\pgfpathlineto{\pgfqpoint{1.181940in}{0.915822in}}%
\pgfpathlineto{\pgfqpoint{1.182565in}{1.221597in}}%
\pgfpathlineto{\pgfqpoint{1.182461in}{0.821310in}}%
\pgfpathlineto{\pgfqpoint{1.183190in}{1.124515in}}%
\pgfpathlineto{\pgfqpoint{1.183294in}{0.927681in}}%
\pgfpathlineto{\pgfqpoint{1.184023in}{1.420165in}}%
\pgfpathlineto{\pgfqpoint{1.184231in}{1.007505in}}%
\pgfpathlineto{\pgfqpoint{1.184439in}{1.211530in}}%
\pgfpathlineto{\pgfqpoint{1.184648in}{0.765515in}}%
\pgfpathlineto{\pgfqpoint{1.185168in}{0.923358in}}%
\pgfpathlineto{\pgfqpoint{1.185585in}{0.805252in}}%
\pgfpathlineto{\pgfqpoint{1.185897in}{1.194284in}}%
\pgfpathlineto{\pgfqpoint{1.186105in}{1.377552in}}%
\pgfpathlineto{\pgfqpoint{1.186418in}{0.993013in}}%
\pgfpathlineto{\pgfqpoint{1.186626in}{1.279284in}}%
\pgfpathlineto{\pgfqpoint{1.186730in}{0.856646in}}%
\pgfpathlineto{\pgfqpoint{1.187668in}{1.204792in}}%
\pgfpathlineto{\pgfqpoint{1.187772in}{1.442168in}}%
\pgfpathlineto{\pgfqpoint{1.188605in}{0.931444in}}%
\pgfpathlineto{\pgfqpoint{1.188709in}{1.215305in}}%
\pgfpathlineto{\pgfqpoint{1.189126in}{0.832628in}}%
\pgfpathlineto{\pgfqpoint{1.189438in}{1.304904in}}%
\pgfpathlineto{\pgfqpoint{1.189855in}{0.857886in}}%
\pgfpathlineto{\pgfqpoint{1.190063in}{1.281599in}}%
\pgfpathlineto{\pgfqpoint{1.190792in}{0.826865in}}%
\pgfpathlineto{\pgfqpoint{1.191000in}{1.048247in}}%
\pgfpathlineto{\pgfqpoint{1.191104in}{1.163632in}}%
\pgfpathlineto{\pgfqpoint{1.191729in}{0.886263in}}%
\pgfpathlineto{\pgfqpoint{1.191937in}{1.088394in}}%
\pgfpathlineto{\pgfqpoint{1.192354in}{0.887720in}}%
\pgfpathlineto{\pgfqpoint{1.192771in}{1.397497in}}%
\pgfpathlineto{\pgfqpoint{1.193083in}{0.947443in}}%
\pgfpathlineto{\pgfqpoint{1.193187in}{1.207042in}}%
\pgfpathlineto{\pgfqpoint{1.193291in}{0.868358in}}%
\pgfpathlineto{\pgfqpoint{1.194229in}{1.037359in}}%
\pgfpathlineto{\pgfqpoint{1.194645in}{1.278911in}}%
\pgfpathlineto{\pgfqpoint{1.194749in}{0.939902in}}%
\pgfpathlineto{\pgfqpoint{1.194957in}{1.274177in}}%
\pgfpathlineto{\pgfqpoint{1.195166in}{0.840485in}}%
\pgfpathlineto{\pgfqpoint{1.195478in}{1.328588in}}%
\pgfpathlineto{\pgfqpoint{1.196103in}{1.060607in}}%
\pgfpathlineto{\pgfqpoint{1.196207in}{1.194537in}}%
\pgfpathlineto{\pgfqpoint{1.196311in}{0.983054in}}%
\pgfpathlineto{\pgfqpoint{1.197040in}{0.999572in}}%
\pgfpathlineto{\pgfqpoint{1.197144in}{0.837173in}}%
\pgfpathlineto{\pgfqpoint{1.197353in}{1.332052in}}%
\pgfpathlineto{\pgfqpoint{1.198082in}{0.919353in}}%
\pgfpathlineto{\pgfqpoint{1.198811in}{1.286122in}}%
\pgfpathlineto{\pgfqpoint{1.198290in}{0.824395in}}%
\pgfpathlineto{\pgfqpoint{1.199331in}{1.118554in}}%
\pgfpathlineto{\pgfqpoint{1.199748in}{0.860247in}}%
\pgfpathlineto{\pgfqpoint{1.200269in}{1.287879in}}%
\pgfpathlineto{\pgfqpoint{1.200373in}{1.182922in}}%
\pgfpathlineto{\pgfqpoint{1.200477in}{1.195763in}}%
\pgfpathlineto{\pgfqpoint{1.200685in}{0.800228in}}%
\pgfpathlineto{\pgfqpoint{1.200789in}{1.340970in}}%
\pgfpathlineto{\pgfqpoint{1.201623in}{0.896096in}}%
\pgfpathlineto{\pgfqpoint{1.202768in}{1.390754in}}%
\pgfpathlineto{\pgfqpoint{1.201935in}{0.655804in}}%
\pgfpathlineto{\pgfqpoint{1.202872in}{1.328420in}}%
\pgfpathlineto{\pgfqpoint{1.203914in}{0.862582in}}%
\pgfpathlineto{\pgfqpoint{1.204018in}{0.922940in}}%
\pgfpathlineto{\pgfqpoint{1.204122in}{0.813122in}}%
\pgfpathlineto{\pgfqpoint{1.204538in}{1.255379in}}%
\pgfpathlineto{\pgfqpoint{1.204851in}{1.041381in}}%
\pgfpathlineto{\pgfqpoint{1.204955in}{1.330243in}}%
\pgfpathlineto{\pgfqpoint{1.205267in}{0.844750in}}%
\pgfpathlineto{\pgfqpoint{1.205892in}{1.027144in}}%
\pgfpathlineto{\pgfqpoint{1.205996in}{0.688816in}}%
\pgfpathlineto{\pgfqpoint{1.206934in}{1.249865in}}%
\pgfpathlineto{\pgfqpoint{1.207559in}{0.805357in}}%
\pgfpathlineto{\pgfqpoint{1.207871in}{1.001621in}}%
\pgfpathlineto{\pgfqpoint{1.207975in}{1.441562in}}%
\pgfpathlineto{\pgfqpoint{1.208288in}{0.730307in}}%
\pgfpathlineto{\pgfqpoint{1.208912in}{1.167726in}}%
\pgfpathlineto{\pgfqpoint{1.209537in}{1.196441in}}%
\pgfpathlineto{\pgfqpoint{1.210162in}{0.895284in}}%
\pgfpathlineto{\pgfqpoint{1.211203in}{1.203806in}}%
\pgfpathlineto{\pgfqpoint{1.210683in}{0.876591in}}%
\pgfpathlineto{\pgfqpoint{1.211308in}{0.966906in}}%
\pgfpathlineto{\pgfqpoint{1.211412in}{0.963685in}}%
\pgfpathlineto{\pgfqpoint{1.211516in}{0.901410in}}%
\pgfpathlineto{\pgfqpoint{1.211724in}{1.266940in}}%
\pgfpathlineto{\pgfqpoint{1.211932in}{1.018715in}}%
\pgfpathlineto{\pgfqpoint{1.212037in}{1.269578in}}%
\pgfpathlineto{\pgfqpoint{1.212349in}{0.879276in}}%
\pgfpathlineto{\pgfqpoint{1.212974in}{1.021627in}}%
\pgfpathlineto{\pgfqpoint{1.213390in}{0.768189in}}%
\pgfpathlineto{\pgfqpoint{1.213286in}{1.219779in}}%
\pgfpathlineto{\pgfqpoint{1.214015in}{1.026987in}}%
\pgfpathlineto{\pgfqpoint{1.214640in}{1.395569in}}%
\pgfpathlineto{\pgfqpoint{1.214224in}{0.872128in}}%
\pgfpathlineto{\pgfqpoint{1.215057in}{1.235305in}}%
\pgfpathlineto{\pgfqpoint{1.215577in}{0.846702in}}%
\pgfpathlineto{\pgfqpoint{1.215890in}{1.258601in}}%
\pgfpathlineto{\pgfqpoint{1.216306in}{0.989074in}}%
\pgfpathlineto{\pgfqpoint{1.216723in}{1.155635in}}%
\pgfpathlineto{\pgfqpoint{1.216827in}{0.988847in}}%
\pgfpathlineto{\pgfqpoint{1.217035in}{1.047544in}}%
\pgfpathlineto{\pgfqpoint{1.217973in}{0.882424in}}%
\pgfpathlineto{\pgfqpoint{1.217869in}{1.241077in}}%
\pgfpathlineto{\pgfqpoint{1.218077in}{0.948291in}}%
\pgfpathlineto{\pgfqpoint{1.218181in}{1.462428in}}%
\pgfpathlineto{\pgfqpoint{1.218598in}{0.749131in}}%
\pgfpathlineto{\pgfqpoint{1.219222in}{1.186101in}}%
\pgfpathlineto{\pgfqpoint{1.219431in}{0.978893in}}%
\pgfpathlineto{\pgfqpoint{1.219743in}{1.316655in}}%
\pgfpathlineto{\pgfqpoint{1.219951in}{0.716183in}}%
\pgfpathlineto{\pgfqpoint{1.220889in}{1.106695in}}%
\pgfpathlineto{\pgfqpoint{1.221409in}{0.842930in}}%
\pgfpathlineto{\pgfqpoint{1.221305in}{1.356374in}}%
\pgfpathlineto{\pgfqpoint{1.221722in}{1.118707in}}%
\pgfpathlineto{\pgfqpoint{1.221826in}{1.240379in}}%
\pgfpathlineto{\pgfqpoint{1.222347in}{0.768995in}}%
\pgfpathlineto{\pgfqpoint{1.222659in}{0.956562in}}%
\pgfpathlineto{\pgfqpoint{1.222867in}{1.173017in}}%
\pgfpathlineto{\pgfqpoint{1.223076in}{0.865372in}}%
\pgfpathlineto{\pgfqpoint{1.223909in}{1.250514in}}%
\pgfpathlineto{\pgfqpoint{1.223284in}{0.843606in}}%
\pgfpathlineto{\pgfqpoint{1.224221in}{0.989886in}}%
\pgfpathlineto{\pgfqpoint{1.225158in}{1.270914in}}%
\pgfpathlineto{\pgfqpoint{1.224638in}{0.789277in}}%
\pgfpathlineto{\pgfqpoint{1.225367in}{1.020528in}}%
\pgfpathlineto{\pgfqpoint{1.225471in}{0.887189in}}%
\pgfpathlineto{\pgfqpoint{1.226200in}{1.274580in}}%
\pgfpathlineto{\pgfqpoint{1.226304in}{1.252839in}}%
\pgfpathlineto{\pgfqpoint{1.226616in}{0.669679in}}%
\pgfpathlineto{\pgfqpoint{1.227554in}{0.775190in}}%
\pgfpathlineto{\pgfqpoint{1.228387in}{0.723332in}}%
\pgfpathlineto{\pgfqpoint{1.228699in}{1.196647in}}%
\pgfpathlineto{\pgfqpoint{1.229012in}{0.832927in}}%
\pgfpathlineto{\pgfqpoint{1.229324in}{1.200436in}}%
\pgfpathlineto{\pgfqpoint{1.229845in}{1.029454in}}%
\pgfpathlineto{\pgfqpoint{1.230157in}{1.172994in}}%
\pgfpathlineto{\pgfqpoint{1.230053in}{0.876802in}}%
\pgfpathlineto{\pgfqpoint{1.230886in}{1.073169in}}%
\pgfpathlineto{\pgfqpoint{1.230990in}{0.704542in}}%
\pgfpathlineto{\pgfqpoint{1.231303in}{1.206488in}}%
\pgfpathlineto{\pgfqpoint{1.231928in}{0.924049in}}%
\pgfpathlineto{\pgfqpoint{1.232240in}{1.421082in}}%
\pgfpathlineto{\pgfqpoint{1.232448in}{0.847546in}}%
\pgfpathlineto{\pgfqpoint{1.233073in}{1.151617in}}%
\pgfpathlineto{\pgfqpoint{1.233802in}{1.234404in}}%
\pgfpathlineto{\pgfqpoint{1.234115in}{0.852972in}}%
\pgfpathlineto{\pgfqpoint{1.234844in}{1.281982in}}%
\pgfpathlineto{\pgfqpoint{1.235260in}{1.075527in}}%
\pgfpathlineto{\pgfqpoint{1.235677in}{0.657165in}}%
\pgfpathlineto{\pgfqpoint{1.235468in}{1.302703in}}%
\pgfpathlineto{\pgfqpoint{1.236197in}{1.172691in}}%
\pgfpathlineto{\pgfqpoint{1.236301in}{1.218604in}}%
\pgfpathlineto{\pgfqpoint{1.236510in}{1.004340in}}%
\pgfpathlineto{\pgfqpoint{1.236718in}{1.057049in}}%
\pgfpathlineto{\pgfqpoint{1.237135in}{1.280395in}}%
\pgfpathlineto{\pgfqpoint{1.237864in}{0.903898in}}%
\pgfpathlineto{\pgfqpoint{1.238072in}{0.661391in}}%
\pgfpathlineto{\pgfqpoint{1.239113in}{1.224060in}}%
\pgfpathlineto{\pgfqpoint{1.239530in}{0.848150in}}%
\pgfpathlineto{\pgfqpoint{1.239946in}{1.260065in}}%
\pgfpathlineto{\pgfqpoint{1.240259in}{1.025782in}}%
\pgfpathlineto{\pgfqpoint{1.241196in}{0.860188in}}%
\pgfpathlineto{\pgfqpoint{1.241300in}{1.368886in}}%
\pgfpathlineto{\pgfqpoint{1.242029in}{0.790845in}}%
\pgfpathlineto{\pgfqpoint{1.242446in}{0.827504in}}%
\pgfpathlineto{\pgfqpoint{1.242550in}{1.288287in}}%
\pgfpathlineto{\pgfqpoint{1.243487in}{0.962786in}}%
\pgfpathlineto{\pgfqpoint{1.243591in}{0.783914in}}%
\pgfpathlineto{\pgfqpoint{1.244112in}{1.309840in}}%
\pgfpathlineto{\pgfqpoint{1.244424in}{1.188195in}}%
\pgfpathlineto{\pgfqpoint{1.244737in}{0.804269in}}%
\pgfpathlineto{\pgfqpoint{1.245362in}{1.195521in}}%
\pgfpathlineto{\pgfqpoint{1.245466in}{0.874097in}}%
\pgfpathlineto{\pgfqpoint{1.245987in}{1.401547in}}%
\pgfpathlineto{\pgfqpoint{1.246299in}{0.704881in}}%
\pgfpathlineto{\pgfqpoint{1.246611in}{1.048419in}}%
\pgfpathlineto{\pgfqpoint{1.246820in}{0.753185in}}%
\pgfpathlineto{\pgfqpoint{1.247028in}{1.267911in}}%
\pgfpathlineto{\pgfqpoint{1.247757in}{0.942506in}}%
\pgfpathlineto{\pgfqpoint{1.248903in}{1.318290in}}%
\pgfpathlineto{\pgfqpoint{1.248069in}{0.861470in}}%
\pgfpathlineto{\pgfqpoint{1.249007in}{1.135368in}}%
\pgfpathlineto{\pgfqpoint{1.249944in}{1.320974in}}%
\pgfpathlineto{\pgfqpoint{1.250048in}{0.867766in}}%
\pgfpathlineto{\pgfqpoint{1.250152in}{1.279766in}}%
\pgfpathlineto{\pgfqpoint{1.250777in}{0.656549in}}%
\pgfpathlineto{\pgfqpoint{1.251194in}{1.022603in}}%
\pgfpathlineto{\pgfqpoint{1.251923in}{1.191637in}}%
\pgfpathlineto{\pgfqpoint{1.251610in}{0.956409in}}%
\pgfpathlineto{\pgfqpoint{1.252131in}{1.058646in}}%
\pgfpathlineto{\pgfqpoint{1.252756in}{1.159199in}}%
\pgfpathlineto{\pgfqpoint{1.253381in}{0.796372in}}%
\pgfpathlineto{\pgfqpoint{1.254630in}{1.229090in}}%
\pgfpathlineto{\pgfqpoint{1.254839in}{0.728180in}}%
\pgfpathlineto{\pgfqpoint{1.255463in}{1.268913in}}%
\pgfpathlineto{\pgfqpoint{1.255672in}{1.211681in}}%
\pgfpathlineto{\pgfqpoint{1.255880in}{1.264809in}}%
\pgfpathlineto{\pgfqpoint{1.255984in}{1.095989in}}%
\pgfpathlineto{\pgfqpoint{1.257026in}{0.688084in}}%
\pgfpathlineto{\pgfqpoint{1.256505in}{1.310375in}}%
\pgfpathlineto{\pgfqpoint{1.257130in}{0.980868in}}%
\pgfpathlineto{\pgfqpoint{1.257650in}{1.316152in}}%
\pgfpathlineto{\pgfqpoint{1.257546in}{0.900729in}}%
\pgfpathlineto{\pgfqpoint{1.258275in}{1.129021in}}%
\pgfpathlineto{\pgfqpoint{1.258588in}{1.485255in}}%
\pgfpathlineto{\pgfqpoint{1.258692in}{0.903213in}}%
\pgfpathlineto{\pgfqpoint{1.258796in}{0.802956in}}%
\pgfpathlineto{\pgfqpoint{1.259525in}{1.130274in}}%
\pgfpathlineto{\pgfqpoint{1.259629in}{0.979833in}}%
\pgfpathlineto{\pgfqpoint{1.259837in}{1.185495in}}%
\pgfpathlineto{\pgfqpoint{1.260254in}{0.769038in}}%
\pgfpathlineto{\pgfqpoint{1.260358in}{0.703930in}}%
\pgfpathlineto{\pgfqpoint{1.260566in}{1.126426in}}%
\pgfpathlineto{\pgfqpoint{1.261504in}{1.367796in}}%
\pgfpathlineto{\pgfqpoint{1.261295in}{0.742247in}}%
\pgfpathlineto{\pgfqpoint{1.261712in}{1.296569in}}%
\pgfpathlineto{\pgfqpoint{1.262128in}{0.728502in}}%
\pgfpathlineto{\pgfqpoint{1.262857in}{1.030064in}}%
\pgfpathlineto{\pgfqpoint{1.262962in}{1.230682in}}%
\pgfpathlineto{\pgfqpoint{1.263691in}{0.887906in}}%
\pgfpathlineto{\pgfqpoint{1.263899in}{1.155124in}}%
\pgfpathlineto{\pgfqpoint{1.264315in}{0.964990in}}%
\pgfpathlineto{\pgfqpoint{1.264732in}{1.285088in}}%
\pgfpathlineto{\pgfqpoint{1.264836in}{1.012738in}}%
\pgfpathlineto{\pgfqpoint{1.265357in}{1.299843in}}%
\pgfpathlineto{\pgfqpoint{1.265253in}{0.708989in}}%
\pgfpathlineto{\pgfqpoint{1.265878in}{0.999006in}}%
\pgfpathlineto{\pgfqpoint{1.266398in}{0.744759in}}%
\pgfpathlineto{\pgfqpoint{1.266086in}{1.119032in}}%
\pgfpathlineto{\pgfqpoint{1.266919in}{1.053822in}}%
\pgfpathlineto{\pgfqpoint{1.267752in}{0.834540in}}%
\pgfpathlineto{\pgfqpoint{1.267127in}{1.346715in}}%
\pgfpathlineto{\pgfqpoint{1.268065in}{0.990175in}}%
\pgfpathlineto{\pgfqpoint{1.268793in}{1.190467in}}%
\pgfpathlineto{\pgfqpoint{1.268898in}{0.794163in}}%
\pgfpathlineto{\pgfqpoint{1.269106in}{1.016280in}}%
\pgfpathlineto{\pgfqpoint{1.269418in}{1.328951in}}%
\pgfpathlineto{\pgfqpoint{1.269522in}{0.929212in}}%
\pgfpathlineto{\pgfqpoint{1.270043in}{1.425708in}}%
\pgfpathlineto{\pgfqpoint{1.270564in}{0.800729in}}%
\pgfpathlineto{\pgfqpoint{1.271501in}{1.338098in}}%
\pgfpathlineto{\pgfqpoint{1.271814in}{1.205530in}}%
\pgfpathlineto{\pgfqpoint{1.271918in}{1.214949in}}%
\pgfpathlineto{\pgfqpoint{1.272543in}{0.855924in}}%
\pgfpathlineto{\pgfqpoint{1.273063in}{0.946453in}}%
\pgfpathlineto{\pgfqpoint{1.273376in}{1.081631in}}%
\pgfpathlineto{\pgfqpoint{1.273272in}{0.815188in}}%
\pgfpathlineto{\pgfqpoint{1.273896in}{0.934501in}}%
\pgfpathlineto{\pgfqpoint{1.274001in}{0.837371in}}%
\pgfpathlineto{\pgfqpoint{1.274105in}{1.319332in}}%
\pgfpathlineto{\pgfqpoint{1.274417in}{1.202619in}}%
\pgfpathlineto{\pgfqpoint{1.274521in}{1.572437in}}%
\pgfpathlineto{\pgfqpoint{1.275354in}{0.837629in}}%
\pgfpathlineto{\pgfqpoint{1.275459in}{1.193216in}}%
\pgfpathlineto{\pgfqpoint{1.276500in}{0.744326in}}%
\pgfpathlineto{\pgfqpoint{1.277229in}{1.430198in}}%
\pgfpathlineto{\pgfqpoint{1.277645in}{0.961526in}}%
\pgfpathlineto{\pgfqpoint{1.278687in}{1.179464in}}%
\pgfpathlineto{\pgfqpoint{1.277958in}{0.755705in}}%
\pgfpathlineto{\pgfqpoint{1.278791in}{1.090515in}}%
\pgfpathlineto{\pgfqpoint{1.279520in}{0.863093in}}%
\pgfpathlineto{\pgfqpoint{1.279208in}{1.266608in}}%
\pgfpathlineto{\pgfqpoint{1.279624in}{0.991172in}}%
\pgfpathlineto{\pgfqpoint{1.279832in}{1.245083in}}%
\pgfpathlineto{\pgfqpoint{1.280249in}{0.784171in}}%
\pgfpathlineto{\pgfqpoint{1.280770in}{1.226165in}}%
\pgfpathlineto{\pgfqpoint{1.281186in}{0.796541in}}%
\pgfpathlineto{\pgfqpoint{1.281499in}{1.445824in}}%
\pgfpathlineto{\pgfqpoint{1.281811in}{1.088255in}}%
\pgfpathlineto{\pgfqpoint{1.282332in}{0.969219in}}%
\pgfpathlineto{\pgfqpoint{1.282644in}{1.175533in}}%
\pgfpathlineto{\pgfqpoint{1.283686in}{0.767564in}}%
\pgfpathlineto{\pgfqpoint{1.282957in}{1.334962in}}%
\pgfpathlineto{\pgfqpoint{1.283790in}{0.860989in}}%
\pgfpathlineto{\pgfqpoint{1.284519in}{1.361262in}}%
\pgfpathlineto{\pgfqpoint{1.284415in}{0.848223in}}%
\pgfpathlineto{\pgfqpoint{1.284935in}{1.195498in}}%
\pgfpathlineto{\pgfqpoint{1.285039in}{1.193081in}}%
\pgfpathlineto{\pgfqpoint{1.285144in}{1.211839in}}%
\pgfpathlineto{\pgfqpoint{1.285977in}{0.793105in}}%
\pgfpathlineto{\pgfqpoint{1.286185in}{1.019846in}}%
\pgfpathlineto{\pgfqpoint{1.286497in}{1.290304in}}%
\pgfpathlineto{\pgfqpoint{1.287226in}{1.161682in}}%
\pgfpathlineto{\pgfqpoint{1.288060in}{0.815455in}}%
\pgfpathlineto{\pgfqpoint{1.287955in}{1.355038in}}%
\pgfpathlineto{\pgfqpoint{1.288268in}{1.058964in}}%
\pgfpathlineto{\pgfqpoint{1.288372in}{1.221203in}}%
\pgfpathlineto{\pgfqpoint{1.288997in}{0.853842in}}%
\pgfpathlineto{\pgfqpoint{1.289205in}{1.127734in}}%
\pgfpathlineto{\pgfqpoint{1.289830in}{0.718488in}}%
\pgfpathlineto{\pgfqpoint{1.289413in}{1.289928in}}%
\pgfpathlineto{\pgfqpoint{1.290351in}{0.830605in}}%
\pgfpathlineto{\pgfqpoint{1.291288in}{1.208377in}}%
\pgfpathlineto{\pgfqpoint{1.291496in}{1.075170in}}%
\pgfpathlineto{\pgfqpoint{1.291600in}{0.800694in}}%
\pgfpathlineto{\pgfqpoint{1.291913in}{1.340998in}}%
\pgfpathlineto{\pgfqpoint{1.292642in}{0.916249in}}%
\pgfpathlineto{\pgfqpoint{1.293891in}{1.356673in}}%
\pgfpathlineto{\pgfqpoint{1.293058in}{0.748599in}}%
\pgfpathlineto{\pgfqpoint{1.293996in}{1.136223in}}%
\pgfpathlineto{\pgfqpoint{1.294725in}{0.745098in}}%
\pgfpathlineto{\pgfqpoint{1.294308in}{1.236783in}}%
\pgfpathlineto{\pgfqpoint{1.295037in}{1.061962in}}%
\pgfpathlineto{\pgfqpoint{1.295870in}{1.452270in}}%
\pgfpathlineto{\pgfqpoint{1.295766in}{0.802159in}}%
\pgfpathlineto{\pgfqpoint{1.295974in}{1.044809in}}%
\pgfpathlineto{\pgfqpoint{1.296703in}{1.147164in}}%
\pgfpathlineto{\pgfqpoint{1.297120in}{0.706490in}}%
\pgfpathlineto{\pgfqpoint{1.297953in}{1.311639in}}%
\pgfpathlineto{\pgfqpoint{1.298265in}{1.155096in}}%
\pgfpathlineto{\pgfqpoint{1.298890in}{0.770136in}}%
\pgfpathlineto{\pgfqpoint{1.298786in}{1.251279in}}%
\pgfpathlineto{\pgfqpoint{1.299411in}{1.043252in}}%
\pgfpathlineto{\pgfqpoint{1.300036in}{0.958342in}}%
\pgfpathlineto{\pgfqpoint{1.299619in}{1.141158in}}%
\pgfpathlineto{\pgfqpoint{1.300244in}{0.976631in}}%
\pgfpathlineto{\pgfqpoint{1.300452in}{0.581141in}}%
\pgfpathlineto{\pgfqpoint{1.301285in}{1.254834in}}%
\pgfpathlineto{\pgfqpoint{1.302014in}{0.749974in}}%
\pgfpathlineto{\pgfqpoint{1.302431in}{0.994977in}}%
\pgfpathlineto{\pgfqpoint{1.303160in}{1.332888in}}%
\pgfpathlineto{\pgfqpoint{1.302848in}{0.857457in}}%
\pgfpathlineto{\pgfqpoint{1.303472in}{1.059832in}}%
\pgfpathlineto{\pgfqpoint{1.304306in}{0.912484in}}%
\pgfpathlineto{\pgfqpoint{1.303993in}{1.256042in}}%
\pgfpathlineto{\pgfqpoint{1.304514in}{0.947801in}}%
\pgfpathlineto{\pgfqpoint{1.305347in}{1.365747in}}%
\pgfpathlineto{\pgfqpoint{1.305243in}{0.750866in}}%
\pgfpathlineto{\pgfqpoint{1.305451in}{1.012588in}}%
\pgfpathlineto{\pgfqpoint{1.305555in}{0.756779in}}%
\pgfpathlineto{\pgfqpoint{1.305972in}{1.199734in}}%
\pgfpathlineto{\pgfqpoint{1.306388in}{1.038845in}}%
\pgfpathlineto{\pgfqpoint{1.306493in}{1.314914in}}%
\pgfpathlineto{\pgfqpoint{1.307117in}{0.798100in}}%
\pgfpathlineto{\pgfqpoint{1.307430in}{1.272928in}}%
\pgfpathlineto{\pgfqpoint{1.307951in}{0.864586in}}%
\pgfpathlineto{\pgfqpoint{1.307742in}{1.401842in}}%
\pgfpathlineto{\pgfqpoint{1.308575in}{0.890493in}}%
\pgfpathlineto{\pgfqpoint{1.308888in}{0.720126in}}%
\pgfpathlineto{\pgfqpoint{1.308784in}{1.216699in}}%
\pgfpathlineto{\pgfqpoint{1.308992in}{1.024192in}}%
\pgfpathlineto{\pgfqpoint{1.309096in}{1.512024in}}%
\pgfpathlineto{\pgfqpoint{1.309304in}{0.889578in}}%
\pgfpathlineto{\pgfqpoint{1.310033in}{1.090167in}}%
\pgfpathlineto{\pgfqpoint{1.310346in}{1.243085in}}%
\pgfpathlineto{\pgfqpoint{1.310762in}{0.907764in}}%
\pgfpathlineto{\pgfqpoint{1.310971in}{0.989024in}}%
\pgfpathlineto{\pgfqpoint{1.311700in}{0.772594in}}%
\pgfpathlineto{\pgfqpoint{1.311283in}{1.156817in}}%
\pgfpathlineto{\pgfqpoint{1.311804in}{0.838125in}}%
\pgfpathlineto{\pgfqpoint{1.312845in}{1.414318in}}%
\pgfpathlineto{\pgfqpoint{1.312949in}{1.345481in}}%
\pgfpathlineto{\pgfqpoint{1.313991in}{0.801918in}}%
\pgfpathlineto{\pgfqpoint{1.314095in}{1.088850in}}%
\pgfpathlineto{\pgfqpoint{1.314616in}{0.899280in}}%
\pgfpathlineto{\pgfqpoint{1.315240in}{1.343026in}}%
\pgfpathlineto{\pgfqpoint{1.316074in}{0.620318in}}%
\pgfpathlineto{\pgfqpoint{1.316386in}{0.910507in}}%
\pgfpathlineto{\pgfqpoint{1.317011in}{1.218870in}}%
\pgfpathlineto{\pgfqpoint{1.316803in}{0.898678in}}%
\pgfpathlineto{\pgfqpoint{1.317532in}{1.121174in}}%
\pgfpathlineto{\pgfqpoint{1.317636in}{0.607041in}}%
\pgfpathlineto{\pgfqpoint{1.317844in}{1.265111in}}%
\pgfpathlineto{\pgfqpoint{1.318573in}{0.889995in}}%
\pgfpathlineto{\pgfqpoint{1.318677in}{1.280212in}}%
\pgfpathlineto{\pgfqpoint{1.319094in}{0.788467in}}%
\pgfpathlineto{\pgfqpoint{1.319718in}{1.158085in}}%
\pgfpathlineto{\pgfqpoint{1.320239in}{1.486378in}}%
\pgfpathlineto{\pgfqpoint{1.320031in}{0.852149in}}%
\pgfpathlineto{\pgfqpoint{1.320656in}{1.089681in}}%
\pgfpathlineto{\pgfqpoint{1.320760in}{1.098434in}}%
\pgfpathlineto{\pgfqpoint{1.320864in}{1.048798in}}%
\pgfpathlineto{\pgfqpoint{1.320968in}{0.798962in}}%
\pgfpathlineto{\pgfqpoint{1.321176in}{1.231966in}}%
\pgfpathlineto{\pgfqpoint{1.322010in}{0.865076in}}%
\pgfpathlineto{\pgfqpoint{1.322218in}{1.262627in}}%
\pgfpathlineto{\pgfqpoint{1.323051in}{1.020238in}}%
\pgfpathlineto{\pgfqpoint{1.323155in}{0.879135in}}%
\pgfpathlineto{\pgfqpoint{1.323259in}{1.339591in}}%
\pgfpathlineto{\pgfqpoint{1.323884in}{1.166400in}}%
\pgfpathlineto{\pgfqpoint{1.323988in}{1.307934in}}%
\pgfpathlineto{\pgfqpoint{1.324197in}{0.865332in}}%
\pgfpathlineto{\pgfqpoint{1.324613in}{0.996391in}}%
\pgfpathlineto{\pgfqpoint{1.324717in}{0.773791in}}%
\pgfpathlineto{\pgfqpoint{1.324926in}{1.343823in}}%
\pgfpathlineto{\pgfqpoint{1.325655in}{1.008252in}}%
\pgfpathlineto{\pgfqpoint{1.325759in}{0.937274in}}%
\pgfpathlineto{\pgfqpoint{1.326383in}{1.176529in}}%
\pgfpathlineto{\pgfqpoint{1.326488in}{1.242558in}}%
\pgfpathlineto{\pgfqpoint{1.326696in}{0.862114in}}%
\pgfpathlineto{\pgfqpoint{1.326904in}{1.060641in}}%
\pgfpathlineto{\pgfqpoint{1.327008in}{0.784575in}}%
\pgfpathlineto{\pgfqpoint{1.327112in}{1.258105in}}%
\pgfpathlineto{\pgfqpoint{1.327946in}{0.940768in}}%
\pgfpathlineto{\pgfqpoint{1.328362in}{0.778812in}}%
\pgfpathlineto{\pgfqpoint{1.328987in}{1.390295in}}%
\pgfpathlineto{\pgfqpoint{1.330028in}{0.790688in}}%
\pgfpathlineto{\pgfqpoint{1.330133in}{1.133439in}}%
\pgfpathlineto{\pgfqpoint{1.330549in}{0.825521in}}%
\pgfpathlineto{\pgfqpoint{1.331278in}{0.940717in}}%
\pgfpathlineto{\pgfqpoint{1.331799in}{1.464226in}}%
\pgfpathlineto{\pgfqpoint{1.332111in}{0.792673in}}%
\pgfpathlineto{\pgfqpoint{1.332528in}{1.235859in}}%
\pgfpathlineto{\pgfqpoint{1.333673in}{0.856208in}}%
\pgfpathlineto{\pgfqpoint{1.334194in}{1.364229in}}%
\pgfpathlineto{\pgfqpoint{1.334923in}{1.009239in}}%
\pgfpathlineto{\pgfqpoint{1.335444in}{0.724390in}}%
\pgfpathlineto{\pgfqpoint{1.335652in}{1.335931in}}%
\pgfpathlineto{\pgfqpoint{1.335964in}{1.097568in}}%
\pgfpathlineto{\pgfqpoint{1.337006in}{0.940473in}}%
\pgfpathlineto{\pgfqpoint{1.336485in}{1.234637in}}%
\pgfpathlineto{\pgfqpoint{1.337110in}{0.965671in}}%
\pgfpathlineto{\pgfqpoint{1.337631in}{1.366176in}}%
\pgfpathlineto{\pgfqpoint{1.337735in}{0.646816in}}%
\pgfpathlineto{\pgfqpoint{1.338256in}{1.068831in}}%
\pgfpathlineto{\pgfqpoint{1.338464in}{0.724929in}}%
\pgfpathlineto{\pgfqpoint{1.338985in}{1.310416in}}%
\pgfpathlineto{\pgfqpoint{1.339089in}{1.407338in}}%
\pgfpathlineto{\pgfqpoint{1.339401in}{0.881286in}}%
\pgfpathlineto{\pgfqpoint{1.339714in}{1.017847in}}%
\pgfpathlineto{\pgfqpoint{1.339818in}{0.704798in}}%
\pgfpathlineto{\pgfqpoint{1.340651in}{1.203368in}}%
\pgfpathlineto{\pgfqpoint{1.340859in}{0.836060in}}%
\pgfpathlineto{\pgfqpoint{1.341067in}{1.239335in}}%
\pgfpathlineto{\pgfqpoint{1.341796in}{0.742821in}}%
\pgfpathlineto{\pgfqpoint{1.343046in}{1.300853in}}%
\pgfpathlineto{\pgfqpoint{1.344192in}{0.829085in}}%
\pgfpathlineto{\pgfqpoint{1.343254in}{1.335733in}}%
\pgfpathlineto{\pgfqpoint{1.344296in}{0.888429in}}%
\pgfpathlineto{\pgfqpoint{1.345025in}{1.226953in}}%
\pgfpathlineto{\pgfqpoint{1.344712in}{0.801418in}}%
\pgfpathlineto{\pgfqpoint{1.345337in}{1.058546in}}%
\pgfpathlineto{\pgfqpoint{1.345858in}{0.738000in}}%
\pgfpathlineto{\pgfqpoint{1.345545in}{1.272069in}}%
\pgfpathlineto{\pgfqpoint{1.346170in}{0.844326in}}%
\pgfpathlineto{\pgfqpoint{1.346587in}{1.254531in}}%
\pgfpathlineto{\pgfqpoint{1.347212in}{1.079099in}}%
\pgfpathlineto{\pgfqpoint{1.347941in}{0.823166in}}%
\pgfpathlineto{\pgfqpoint{1.347732in}{1.266664in}}%
\pgfpathlineto{\pgfqpoint{1.348253in}{1.091579in}}%
\pgfpathlineto{\pgfqpoint{1.348982in}{0.922343in}}%
\pgfpathlineto{\pgfqpoint{1.348566in}{1.125449in}}%
\pgfpathlineto{\pgfqpoint{1.349295in}{1.046999in}}%
\pgfpathlineto{\pgfqpoint{1.349399in}{1.088749in}}%
\pgfpathlineto{\pgfqpoint{1.349503in}{0.512508in}}%
\pgfpathlineto{\pgfqpoint{1.349815in}{1.275199in}}%
\pgfpathlineto{\pgfqpoint{1.350440in}{0.936351in}}%
\pgfpathlineto{\pgfqpoint{1.350857in}{1.453902in}}%
\pgfpathlineto{\pgfqpoint{1.350648in}{0.936228in}}%
\pgfpathlineto{\pgfqpoint{1.351169in}{1.126423in}}%
\pgfpathlineto{\pgfqpoint{1.351273in}{0.818280in}}%
\pgfpathlineto{\pgfqpoint{1.351898in}{1.326155in}}%
\pgfpathlineto{\pgfqpoint{1.352210in}{1.001063in}}%
\pgfpathlineto{\pgfqpoint{1.352523in}{0.750415in}}%
\pgfpathlineto{\pgfqpoint{1.353252in}{1.354417in}}%
\pgfpathlineto{\pgfqpoint{1.354293in}{0.740513in}}%
\pgfpathlineto{\pgfqpoint{1.354397in}{1.165750in}}%
\pgfpathlineto{\pgfqpoint{1.355231in}{0.902952in}}%
\pgfpathlineto{\pgfqpoint{1.354606in}{1.255698in}}%
\pgfpathlineto{\pgfqpoint{1.355543in}{1.048111in}}%
\pgfpathlineto{\pgfqpoint{1.355647in}{1.164276in}}%
\pgfpathlineto{\pgfqpoint{1.355751in}{0.758268in}}%
\pgfpathlineto{\pgfqpoint{1.356272in}{1.022664in}}%
\pgfpathlineto{\pgfqpoint{1.356376in}{0.525326in}}%
\pgfpathlineto{\pgfqpoint{1.357001in}{1.280665in}}%
\pgfpathlineto{\pgfqpoint{1.357313in}{0.980761in}}%
\pgfpathlineto{\pgfqpoint{1.357938in}{1.273428in}}%
\pgfpathlineto{\pgfqpoint{1.357834in}{0.966498in}}%
\pgfpathlineto{\pgfqpoint{1.358355in}{1.205676in}}%
\pgfpathlineto{\pgfqpoint{1.358459in}{0.847261in}}%
\pgfpathlineto{\pgfqpoint{1.359396in}{1.285599in}}%
\pgfpathlineto{\pgfqpoint{1.359500in}{1.047151in}}%
\pgfpathlineto{\pgfqpoint{1.359917in}{0.858388in}}%
\pgfpathlineto{\pgfqpoint{1.360229in}{1.244960in}}%
\pgfpathlineto{\pgfqpoint{1.360438in}{0.949305in}}%
\pgfpathlineto{\pgfqpoint{1.361271in}{1.294606in}}%
\pgfpathlineto{\pgfqpoint{1.361375in}{0.843966in}}%
\pgfpathlineto{\pgfqpoint{1.361479in}{0.865371in}}%
\pgfpathlineto{\pgfqpoint{1.361583in}{0.816872in}}%
\pgfpathlineto{\pgfqpoint{1.361687in}{1.108896in}}%
\pgfpathlineto{\pgfqpoint{1.361791in}{1.078621in}}%
\pgfpathlineto{\pgfqpoint{1.362520in}{1.452202in}}%
\pgfpathlineto{\pgfqpoint{1.362416in}{0.808660in}}%
\pgfpathlineto{\pgfqpoint{1.362937in}{1.158352in}}%
\pgfpathlineto{\pgfqpoint{1.363458in}{0.935232in}}%
\pgfpathlineto{\pgfqpoint{1.363562in}{1.262566in}}%
\pgfpathlineto{\pgfqpoint{1.364083in}{1.100717in}}%
\pgfpathlineto{\pgfqpoint{1.364707in}{1.394532in}}%
\pgfpathlineto{\pgfqpoint{1.364812in}{0.847499in}}%
\pgfpathlineto{\pgfqpoint{1.364916in}{0.863535in}}%
\pgfpathlineto{\pgfqpoint{1.365645in}{1.435826in}}%
\pgfpathlineto{\pgfqpoint{1.365749in}{0.800980in}}%
\pgfpathlineto{\pgfqpoint{1.366061in}{1.185070in}}%
\pgfpathlineto{\pgfqpoint{1.366790in}{0.904542in}}%
\pgfpathlineto{\pgfqpoint{1.366686in}{1.206910in}}%
\pgfpathlineto{\pgfqpoint{1.367103in}{0.997467in}}%
\pgfpathlineto{\pgfqpoint{1.367415in}{1.437006in}}%
\pgfpathlineto{\pgfqpoint{1.367832in}{0.960504in}}%
\pgfpathlineto{\pgfqpoint{1.368144in}{1.070908in}}%
\pgfpathlineto{\pgfqpoint{1.369081in}{0.784878in}}%
\pgfpathlineto{\pgfqpoint{1.368665in}{1.229457in}}%
\pgfpathlineto{\pgfqpoint{1.369290in}{0.954324in}}%
\pgfpathlineto{\pgfqpoint{1.369394in}{0.937277in}}%
\pgfpathlineto{\pgfqpoint{1.369498in}{1.082351in}}%
\pgfpathlineto{\pgfqpoint{1.369706in}{0.972623in}}%
\pgfpathlineto{\pgfqpoint{1.369914in}{1.307216in}}%
\pgfpathlineto{\pgfqpoint{1.370123in}{0.771798in}}%
\pgfpathlineto{\pgfqpoint{1.370748in}{1.076406in}}%
\pgfpathlineto{\pgfqpoint{1.370956in}{0.922532in}}%
\pgfpathlineto{\pgfqpoint{1.371685in}{1.195521in}}%
\pgfpathlineto{\pgfqpoint{1.371789in}{1.225346in}}%
\pgfpathlineto{\pgfqpoint{1.371997in}{1.021725in}}%
\pgfpathlineto{\pgfqpoint{1.372101in}{1.082238in}}%
\pgfpathlineto{\pgfqpoint{1.372206in}{0.885328in}}%
\pgfpathlineto{\pgfqpoint{1.372414in}{1.163518in}}%
\pgfpathlineto{\pgfqpoint{1.373247in}{1.001049in}}%
\pgfpathlineto{\pgfqpoint{1.373976in}{0.923241in}}%
\pgfpathlineto{\pgfqpoint{1.374393in}{1.348241in}}%
\pgfpathlineto{\pgfqpoint{1.375226in}{0.635623in}}%
\pgfpathlineto{\pgfqpoint{1.375434in}{1.355843in}}%
\pgfpathlineto{\pgfqpoint{1.375642in}{0.846181in}}%
\pgfpathlineto{\pgfqpoint{1.376684in}{1.474947in}}%
\pgfpathlineto{\pgfqpoint{1.376892in}{1.468927in}}%
\pgfpathlineto{\pgfqpoint{1.377517in}{0.860308in}}%
\pgfpathlineto{\pgfqpoint{1.378037in}{1.123965in}}%
\pgfpathlineto{\pgfqpoint{1.378142in}{1.226090in}}%
\pgfpathlineto{\pgfqpoint{1.378246in}{0.811089in}}%
\pgfpathlineto{\pgfqpoint{1.378871in}{0.905456in}}%
\pgfpathlineto{\pgfqpoint{1.378975in}{0.769939in}}%
\pgfpathlineto{\pgfqpoint{1.379183in}{1.249554in}}%
\pgfpathlineto{\pgfqpoint{1.379912in}{0.881779in}}%
\pgfpathlineto{\pgfqpoint{1.380016in}{1.217569in}}%
\pgfpathlineto{\pgfqpoint{1.381058in}{1.004551in}}%
\pgfpathlineto{\pgfqpoint{1.381682in}{1.246642in}}%
\pgfpathlineto{\pgfqpoint{1.381474in}{0.952969in}}%
\pgfpathlineto{\pgfqpoint{1.381787in}{1.226155in}}%
\pgfpathlineto{\pgfqpoint{1.382620in}{0.836146in}}%
\pgfpathlineto{\pgfqpoint{1.382828in}{1.251830in}}%
\pgfpathlineto{\pgfqpoint{1.382932in}{1.028522in}}%
\pgfpathlineto{\pgfqpoint{1.383453in}{1.409385in}}%
\pgfpathlineto{\pgfqpoint{1.383349in}{0.728358in}}%
\pgfpathlineto{\pgfqpoint{1.383869in}{1.096469in}}%
\pgfpathlineto{\pgfqpoint{1.383973in}{0.842120in}}%
\pgfpathlineto{\pgfqpoint{1.384390in}{1.218792in}}%
\pgfpathlineto{\pgfqpoint{1.384911in}{1.197778in}}%
\pgfpathlineto{\pgfqpoint{1.385119in}{0.850071in}}%
\pgfpathlineto{\pgfqpoint{1.385536in}{1.371252in}}%
\pgfpathlineto{\pgfqpoint{1.386056in}{0.939601in}}%
\pgfpathlineto{\pgfqpoint{1.386889in}{1.288759in}}%
\pgfpathlineto{\pgfqpoint{1.387098in}{0.958013in}}%
\pgfpathlineto{\pgfqpoint{1.387618in}{1.160354in}}%
\pgfpathlineto{\pgfqpoint{1.387514in}{0.909095in}}%
\pgfpathlineto{\pgfqpoint{1.387723in}{1.066729in}}%
\pgfpathlineto{\pgfqpoint{1.387827in}{0.802207in}}%
\pgfpathlineto{\pgfqpoint{1.388243in}{1.331624in}}%
\pgfpathlineto{\pgfqpoint{1.388764in}{1.101863in}}%
\pgfpathlineto{\pgfqpoint{1.389493in}{1.378703in}}%
\pgfpathlineto{\pgfqpoint{1.389389in}{0.950278in}}%
\pgfpathlineto{\pgfqpoint{1.389597in}{1.102192in}}%
\pgfpathlineto{\pgfqpoint{1.389701in}{0.931526in}}%
\pgfpathlineto{\pgfqpoint{1.389910in}{1.359307in}}%
\pgfpathlineto{\pgfqpoint{1.390639in}{1.069109in}}%
\pgfpathlineto{\pgfqpoint{1.390743in}{1.274164in}}%
\pgfpathlineto{\pgfqpoint{1.391680in}{0.946535in}}%
\pgfpathlineto{\pgfqpoint{1.391784in}{0.945462in}}%
\pgfpathlineto{\pgfqpoint{1.391888in}{1.365653in}}%
\pgfpathlineto{\pgfqpoint{1.392513in}{0.593257in}}%
\pgfpathlineto{\pgfqpoint{1.392825in}{1.062968in}}%
\pgfpathlineto{\pgfqpoint{1.392930in}{0.815510in}}%
\pgfpathlineto{\pgfqpoint{1.393659in}{1.345737in}}%
\pgfpathlineto{\pgfqpoint{1.393867in}{0.848048in}}%
\pgfpathlineto{\pgfqpoint{1.394075in}{1.385192in}}%
\pgfpathlineto{\pgfqpoint{1.394388in}{0.842416in}}%
\pgfpathlineto{\pgfqpoint{1.394908in}{0.950386in}}%
\pgfpathlineto{\pgfqpoint{1.395012in}{0.864266in}}%
\pgfpathlineto{\pgfqpoint{1.395741in}{1.122671in}}%
\pgfpathlineto{\pgfqpoint{1.395846in}{1.065779in}}%
\pgfpathlineto{\pgfqpoint{1.396470in}{0.833437in}}%
\pgfpathlineto{\pgfqpoint{1.396575in}{1.200395in}}%
\pgfpathlineto{\pgfqpoint{1.396783in}{1.067433in}}%
\pgfpathlineto{\pgfqpoint{1.396887in}{1.225172in}}%
\pgfpathlineto{\pgfqpoint{1.397616in}{0.937435in}}%
\pgfpathlineto{\pgfqpoint{1.397720in}{0.749830in}}%
\pgfpathlineto{\pgfqpoint{1.398241in}{1.250111in}}%
\pgfpathlineto{\pgfqpoint{1.398553in}{0.976495in}}%
\pgfpathlineto{\pgfqpoint{1.399491in}{1.178429in}}%
\pgfpathlineto{\pgfqpoint{1.398970in}{0.762705in}}%
\pgfpathlineto{\pgfqpoint{1.399595in}{1.038976in}}%
\pgfpathlineto{\pgfqpoint{1.400428in}{0.702289in}}%
\pgfpathlineto{\pgfqpoint{1.399907in}{1.441384in}}%
\pgfpathlineto{\pgfqpoint{1.400948in}{0.763271in}}%
\pgfpathlineto{\pgfqpoint{1.401261in}{1.216708in}}%
\pgfpathlineto{\pgfqpoint{1.402094in}{0.905233in}}%
\pgfpathlineto{\pgfqpoint{1.402511in}{1.221624in}}%
\pgfpathlineto{\pgfqpoint{1.402615in}{0.880845in}}%
\pgfpathlineto{\pgfqpoint{1.403135in}{1.183260in}}%
\pgfpathlineto{\pgfqpoint{1.403969in}{0.829438in}}%
\pgfpathlineto{\pgfqpoint{1.403864in}{1.323295in}}%
\pgfpathlineto{\pgfqpoint{1.404177in}{0.920750in}}%
\pgfpathlineto{\pgfqpoint{1.404385in}{0.816677in}}%
\pgfpathlineto{\pgfqpoint{1.405322in}{1.298635in}}%
\pgfpathlineto{\pgfqpoint{1.405947in}{0.830963in}}%
\pgfpathlineto{\pgfqpoint{1.406572in}{0.942274in}}%
\pgfpathlineto{\pgfqpoint{1.407197in}{1.493686in}}%
\pgfpathlineto{\pgfqpoint{1.407093in}{0.794508in}}%
\pgfpathlineto{\pgfqpoint{1.407822in}{1.239776in}}%
\pgfpathlineto{\pgfqpoint{1.408759in}{0.578131in}}%
\pgfpathlineto{\pgfqpoint{1.408342in}{1.289156in}}%
\pgfpathlineto{\pgfqpoint{1.408967in}{1.053902in}}%
\pgfpathlineto{\pgfqpoint{1.409071in}{1.074728in}}%
\pgfpathlineto{\pgfqpoint{1.409176in}{1.062086in}}%
\pgfpathlineto{\pgfqpoint{1.409280in}{0.859399in}}%
\pgfpathlineto{\pgfqpoint{1.409696in}{1.193182in}}%
\pgfpathlineto{\pgfqpoint{1.410217in}{1.069987in}}%
\pgfpathlineto{\pgfqpoint{1.410634in}{1.378206in}}%
\pgfpathlineto{\pgfqpoint{1.410842in}{0.490731in}}%
\pgfpathlineto{\pgfqpoint{1.411258in}{0.995656in}}%
\pgfpathlineto{\pgfqpoint{1.411363in}{0.988578in}}%
\pgfpathlineto{\pgfqpoint{1.411571in}{0.825156in}}%
\pgfpathlineto{\pgfqpoint{1.412508in}{1.272323in}}%
\pgfpathlineto{\pgfqpoint{1.412612in}{0.753490in}}%
\pgfpathlineto{\pgfqpoint{1.413550in}{0.990635in}}%
\pgfpathlineto{\pgfqpoint{1.413654in}{1.375160in}}%
\pgfpathlineto{\pgfqpoint{1.414174in}{0.753902in}}%
\pgfpathlineto{\pgfqpoint{1.414591in}{1.188409in}}%
\pgfpathlineto{\pgfqpoint{1.415528in}{0.855971in}}%
\pgfpathlineto{\pgfqpoint{1.414799in}{1.291473in}}%
\pgfpathlineto{\pgfqpoint{1.415737in}{0.995453in}}%
\pgfpathlineto{\pgfqpoint{1.415841in}{1.550163in}}%
\pgfpathlineto{\pgfqpoint{1.416257in}{0.778037in}}%
\pgfpathlineto{\pgfqpoint{1.416778in}{1.087815in}}%
\pgfpathlineto{\pgfqpoint{1.417507in}{0.728761in}}%
\pgfpathlineto{\pgfqpoint{1.417194in}{1.306524in}}%
\pgfpathlineto{\pgfqpoint{1.417923in}{0.943922in}}%
\pgfpathlineto{\pgfqpoint{1.418757in}{1.415420in}}%
\pgfpathlineto{\pgfqpoint{1.418132in}{0.837313in}}%
\pgfpathlineto{\pgfqpoint{1.419069in}{1.220944in}}%
\pgfpathlineto{\pgfqpoint{1.419486in}{0.621949in}}%
\pgfpathlineto{\pgfqpoint{1.419277in}{1.256023in}}%
\pgfpathlineto{\pgfqpoint{1.420215in}{1.041192in}}%
\pgfpathlineto{\pgfqpoint{1.420319in}{1.378826in}}%
\pgfpathlineto{\pgfqpoint{1.421048in}{1.026675in}}%
\pgfpathlineto{\pgfqpoint{1.421256in}{1.267677in}}%
\pgfpathlineto{\pgfqpoint{1.421881in}{0.845021in}}%
\pgfpathlineto{\pgfqpoint{1.421985in}{1.334903in}}%
\pgfpathlineto{\pgfqpoint{1.422297in}{0.921609in}}%
\pgfpathlineto{\pgfqpoint{1.422402in}{1.450439in}}%
\pgfpathlineto{\pgfqpoint{1.423339in}{0.758956in}}%
\pgfpathlineto{\pgfqpoint{1.423547in}{1.319634in}}%
\pgfpathlineto{\pgfqpoint{1.424797in}{1.205636in}}%
\pgfpathlineto{\pgfqpoint{1.425838in}{0.761933in}}%
\pgfpathlineto{\pgfqpoint{1.425630in}{1.433586in}}%
\pgfpathlineto{\pgfqpoint{1.426046in}{0.998890in}}%
\pgfpathlineto{\pgfqpoint{1.426880in}{1.376294in}}%
\pgfpathlineto{\pgfqpoint{1.426463in}{0.832824in}}%
\pgfpathlineto{\pgfqpoint{1.427088in}{1.045831in}}%
\pgfpathlineto{\pgfqpoint{1.427713in}{0.859783in}}%
\pgfpathlineto{\pgfqpoint{1.427921in}{1.183044in}}%
\pgfpathlineto{\pgfqpoint{1.428025in}{1.105814in}}%
\pgfpathlineto{\pgfqpoint{1.428338in}{1.293023in}}%
\pgfpathlineto{\pgfqpoint{1.428546in}{0.997480in}}%
\pgfpathlineto{\pgfqpoint{1.428858in}{0.695736in}}%
\pgfpathlineto{\pgfqpoint{1.429379in}{1.173074in}}%
\pgfpathlineto{\pgfqpoint{1.429483in}{1.088052in}}%
\pgfpathlineto{\pgfqpoint{1.429691in}{1.099160in}}%
\pgfpathlineto{\pgfqpoint{1.429796in}{1.067773in}}%
\pgfpathlineto{\pgfqpoint{1.430212in}{1.293566in}}%
\pgfpathlineto{\pgfqpoint{1.430108in}{0.840022in}}%
\pgfpathlineto{\pgfqpoint{1.430837in}{1.285345in}}%
\pgfpathlineto{\pgfqpoint{1.431045in}{0.931195in}}%
\pgfpathlineto{\pgfqpoint{1.431983in}{1.029311in}}%
\pgfpathlineto{\pgfqpoint{1.432295in}{1.355532in}}%
\pgfpathlineto{\pgfqpoint{1.432191in}{0.706875in}}%
\pgfpathlineto{\pgfqpoint{1.433024in}{1.092466in}}%
\pgfpathlineto{\pgfqpoint{1.433857in}{1.192762in}}%
\pgfpathlineto{\pgfqpoint{1.434065in}{0.926586in}}%
\pgfpathlineto{\pgfqpoint{1.434378in}{1.176135in}}%
\pgfpathlineto{\pgfqpoint{1.434586in}{0.792847in}}%
\pgfpathlineto{\pgfqpoint{1.435211in}{1.011853in}}%
\pgfpathlineto{\pgfqpoint{1.435940in}{1.259685in}}%
\pgfpathlineto{\pgfqpoint{1.435836in}{0.937323in}}%
\pgfpathlineto{\pgfqpoint{1.436044in}{1.019433in}}%
\pgfpathlineto{\pgfqpoint{1.436877in}{0.796362in}}%
\pgfpathlineto{\pgfqpoint{1.436565in}{1.378666in}}%
\pgfpathlineto{\pgfqpoint{1.437190in}{0.962409in}}%
\pgfpathlineto{\pgfqpoint{1.437814in}{1.287603in}}%
\pgfpathlineto{\pgfqpoint{1.438023in}{0.736661in}}%
\pgfpathlineto{\pgfqpoint{1.439064in}{1.279971in}}%
\pgfpathlineto{\pgfqpoint{1.439168in}{0.957482in}}%
\pgfpathlineto{\pgfqpoint{1.439272in}{0.966863in}}%
\pgfpathlineto{\pgfqpoint{1.439793in}{1.379360in}}%
\pgfpathlineto{\pgfqpoint{1.439689in}{0.826316in}}%
\pgfpathlineto{\pgfqpoint{1.440210in}{1.335062in}}%
\pgfpathlineto{\pgfqpoint{1.440314in}{0.859607in}}%
\pgfpathlineto{\pgfqpoint{1.440418in}{1.345491in}}%
\pgfpathlineto{\pgfqpoint{1.441355in}{0.972625in}}%
\pgfpathlineto{\pgfqpoint{1.441876in}{0.792783in}}%
\pgfpathlineto{\pgfqpoint{1.441772in}{1.231722in}}%
\pgfpathlineto{\pgfqpoint{1.441980in}{1.210451in}}%
\pgfpathlineto{\pgfqpoint{1.442084in}{1.422113in}}%
\pgfpathlineto{\pgfqpoint{1.442501in}{0.852622in}}%
\pgfpathlineto{\pgfqpoint{1.443021in}{1.141336in}}%
\pgfpathlineto{\pgfqpoint{1.443750in}{0.917979in}}%
\pgfpathlineto{\pgfqpoint{1.443855in}{1.243386in}}%
\pgfpathlineto{\pgfqpoint{1.443959in}{0.961023in}}%
\pgfpathlineto{\pgfqpoint{1.444063in}{1.219950in}}%
\pgfpathlineto{\pgfqpoint{1.444584in}{0.797940in}}%
\pgfpathlineto{\pgfqpoint{1.445000in}{1.202552in}}%
\pgfpathlineto{\pgfqpoint{1.445313in}{0.842298in}}%
\pgfpathlineto{\pgfqpoint{1.445937in}{1.303478in}}%
\pgfpathlineto{\pgfqpoint{1.446042in}{1.102939in}}%
\pgfpathlineto{\pgfqpoint{1.446771in}{1.337059in}}%
\pgfpathlineto{\pgfqpoint{1.446354in}{0.854145in}}%
\pgfpathlineto{\pgfqpoint{1.446979in}{1.227292in}}%
\pgfpathlineto{\pgfqpoint{1.447291in}{0.858996in}}%
\pgfpathlineto{\pgfqpoint{1.448020in}{1.369663in}}%
\pgfpathlineto{\pgfqpoint{1.448124in}{1.040380in}}%
\pgfpathlineto{\pgfqpoint{1.448958in}{1.234872in}}%
\pgfpathlineto{\pgfqpoint{1.448749in}{0.708398in}}%
\pgfpathlineto{\pgfqpoint{1.449166in}{1.119645in}}%
\pgfpathlineto{\pgfqpoint{1.449686in}{1.232553in}}%
\pgfpathlineto{\pgfqpoint{1.450207in}{0.722892in}}%
\pgfpathlineto{\pgfqpoint{1.451040in}{1.327919in}}%
\pgfpathlineto{\pgfqpoint{1.451457in}{1.267822in}}%
\pgfpathlineto{\pgfqpoint{1.451561in}{0.813853in}}%
\pgfpathlineto{\pgfqpoint{1.451665in}{1.338465in}}%
\pgfpathlineto{\pgfqpoint{1.452602in}{0.942246in}}%
\pgfpathlineto{\pgfqpoint{1.452707in}{1.433939in}}%
\pgfpathlineto{\pgfqpoint{1.453331in}{0.685712in}}%
\pgfpathlineto{\pgfqpoint{1.453644in}{1.139601in}}%
\pgfpathlineto{\pgfqpoint{1.454165in}{0.834291in}}%
\pgfpathlineto{\pgfqpoint{1.454269in}{1.160864in}}%
\pgfpathlineto{\pgfqpoint{1.454373in}{1.156046in}}%
\pgfpathlineto{\pgfqpoint{1.454477in}{1.286864in}}%
\pgfpathlineto{\pgfqpoint{1.454581in}{0.938151in}}%
\pgfpathlineto{\pgfqpoint{1.455414in}{1.149173in}}%
\pgfpathlineto{\pgfqpoint{1.455623in}{0.859174in}}%
\pgfpathlineto{\pgfqpoint{1.455831in}{1.276720in}}%
\pgfpathlineto{\pgfqpoint{1.456560in}{0.963696in}}%
\pgfpathlineto{\pgfqpoint{1.456664in}{1.291645in}}%
\pgfpathlineto{\pgfqpoint{1.457289in}{0.730657in}}%
\pgfpathlineto{\pgfqpoint{1.457601in}{0.888927in}}%
\pgfpathlineto{\pgfqpoint{1.457705in}{0.894288in}}%
\pgfpathlineto{\pgfqpoint{1.458226in}{0.881731in}}%
\pgfpathlineto{\pgfqpoint{1.458747in}{1.392815in}}%
\pgfpathlineto{\pgfqpoint{1.458851in}{0.848667in}}%
\pgfpathlineto{\pgfqpoint{1.459788in}{1.218788in}}%
\pgfpathlineto{\pgfqpoint{1.459892in}{1.284681in}}%
\pgfpathlineto{\pgfqpoint{1.460205in}{1.003900in}}%
\pgfpathlineto{\pgfqpoint{1.460621in}{1.116365in}}%
\pgfpathlineto{\pgfqpoint{1.461038in}{0.806382in}}%
\pgfpathlineto{\pgfqpoint{1.461559in}{1.256853in}}%
\pgfpathlineto{\pgfqpoint{1.462183in}{1.036612in}}%
\pgfpathlineto{\pgfqpoint{1.462496in}{1.398642in}}%
\pgfpathlineto{\pgfqpoint{1.463121in}{0.842226in}}%
\pgfpathlineto{\pgfqpoint{1.463641in}{1.114256in}}%
\pgfpathlineto{\pgfqpoint{1.463746in}{1.261950in}}%
\pgfpathlineto{\pgfqpoint{1.464162in}{0.882661in}}%
\pgfpathlineto{\pgfqpoint{1.464683in}{1.075681in}}%
\pgfpathlineto{\pgfqpoint{1.465620in}{1.517911in}}%
\pgfpathlineto{\pgfqpoint{1.464891in}{0.936598in}}%
\pgfpathlineto{\pgfqpoint{1.465828in}{1.114844in}}%
\pgfpathlineto{\pgfqpoint{1.466349in}{0.778015in}}%
\pgfpathlineto{\pgfqpoint{1.466766in}{1.019031in}}%
\pgfpathlineto{\pgfqpoint{1.467495in}{1.355410in}}%
\pgfpathlineto{\pgfqpoint{1.466974in}{0.836470in}}%
\pgfpathlineto{\pgfqpoint{1.467807in}{1.014163in}}%
\pgfpathlineto{\pgfqpoint{1.467911in}{0.767125in}}%
\pgfpathlineto{\pgfqpoint{1.468640in}{1.240569in}}%
\pgfpathlineto{\pgfqpoint{1.468848in}{1.097358in}}%
\pgfpathlineto{\pgfqpoint{1.468953in}{1.078157in}}%
\pgfpathlineto{\pgfqpoint{1.469057in}{1.268834in}}%
\pgfpathlineto{\pgfqpoint{1.469265in}{0.863953in}}%
\pgfpathlineto{\pgfqpoint{1.470098in}{1.168694in}}%
\pgfpathlineto{\pgfqpoint{1.470723in}{0.851905in}}%
\pgfpathlineto{\pgfqpoint{1.470306in}{1.284518in}}%
\pgfpathlineto{\pgfqpoint{1.471244in}{0.946279in}}%
\pgfpathlineto{\pgfqpoint{1.471452in}{1.227983in}}%
\pgfpathlineto{\pgfqpoint{1.471660in}{0.877367in}}%
\pgfpathlineto{\pgfqpoint{1.472181in}{0.949882in}}%
\pgfpathlineto{\pgfqpoint{1.472285in}{0.743613in}}%
\pgfpathlineto{\pgfqpoint{1.473118in}{1.238652in}}%
\pgfpathlineto{\pgfqpoint{1.473222in}{1.228594in}}%
\pgfpathlineto{\pgfqpoint{1.473847in}{0.758391in}}%
\pgfpathlineto{\pgfqpoint{1.474368in}{1.075355in}}%
\pgfpathlineto{\pgfqpoint{1.474993in}{1.014759in}}%
\pgfpathlineto{\pgfqpoint{1.475513in}{1.251358in}}%
\pgfpathlineto{\pgfqpoint{1.475930in}{0.853535in}}%
\pgfpathlineto{\pgfqpoint{1.476659in}{0.982699in}}%
\pgfpathlineto{\pgfqpoint{1.477284in}{1.262734in}}%
\pgfpathlineto{\pgfqpoint{1.477180in}{0.855965in}}%
\pgfpathlineto{\pgfqpoint{1.477700in}{1.043858in}}%
\pgfpathlineto{\pgfqpoint{1.478013in}{0.945276in}}%
\pgfpathlineto{\pgfqpoint{1.478429in}{1.152430in}}%
\pgfpathlineto{\pgfqpoint{1.478534in}{1.328656in}}%
\pgfpathlineto{\pgfqpoint{1.478846in}{0.760201in}}%
\pgfpathlineto{\pgfqpoint{1.479263in}{1.073643in}}%
\pgfpathlineto{\pgfqpoint{1.479471in}{0.787642in}}%
\pgfpathlineto{\pgfqpoint{1.480096in}{1.240958in}}%
\pgfpathlineto{\pgfqpoint{1.480304in}{0.929713in}}%
\pgfpathlineto{\pgfqpoint{1.480616in}{1.209539in}}%
\pgfpathlineto{\pgfqpoint{1.480825in}{0.794504in}}%
\pgfpathlineto{\pgfqpoint{1.481450in}{1.137371in}}%
\pgfpathlineto{\pgfqpoint{1.481658in}{0.737481in}}%
\pgfpathlineto{\pgfqpoint{1.482074in}{1.309406in}}%
\pgfpathlineto{\pgfqpoint{1.482595in}{1.015485in}}%
\pgfpathlineto{\pgfqpoint{1.482699in}{1.009780in}}%
\pgfpathlineto{\pgfqpoint{1.483220in}{1.354589in}}%
\pgfpathlineto{\pgfqpoint{1.483116in}{0.889221in}}%
\pgfpathlineto{\pgfqpoint{1.483845in}{1.085623in}}%
\pgfpathlineto{\pgfqpoint{1.484470in}{0.828347in}}%
\pgfpathlineto{\pgfqpoint{1.484365in}{1.161979in}}%
\pgfpathlineto{\pgfqpoint{1.484990in}{1.021417in}}%
\pgfpathlineto{\pgfqpoint{1.485407in}{1.313076in}}%
\pgfpathlineto{\pgfqpoint{1.485615in}{0.802284in}}%
\pgfpathlineto{\pgfqpoint{1.485928in}{1.111851in}}%
\pgfpathlineto{\pgfqpoint{1.486448in}{0.855724in}}%
\pgfpathlineto{\pgfqpoint{1.486344in}{1.148578in}}%
\pgfpathlineto{\pgfqpoint{1.486969in}{1.080129in}}%
\pgfpathlineto{\pgfqpoint{1.487281in}{1.146232in}}%
\pgfpathlineto{\pgfqpoint{1.487490in}{0.908998in}}%
\pgfpathlineto{\pgfqpoint{1.487594in}{1.383266in}}%
\pgfpathlineto{\pgfqpoint{1.488635in}{1.081314in}}%
\pgfpathlineto{\pgfqpoint{1.488844in}{1.136385in}}%
\pgfpathlineto{\pgfqpoint{1.488948in}{1.089070in}}%
\pgfpathlineto{\pgfqpoint{1.489052in}{0.876364in}}%
\pgfpathlineto{\pgfqpoint{1.489468in}{1.208484in}}%
\pgfpathlineto{\pgfqpoint{1.489989in}{1.121786in}}%
\pgfpathlineto{\pgfqpoint{1.490510in}{0.827268in}}%
\pgfpathlineto{\pgfqpoint{1.490406in}{1.152134in}}%
\pgfpathlineto{\pgfqpoint{1.490822in}{1.132547in}}%
\pgfpathlineto{\pgfqpoint{1.490926in}{1.273421in}}%
\pgfpathlineto{\pgfqpoint{1.491551in}{0.920023in}}%
\pgfpathlineto{\pgfqpoint{1.491864in}{1.146042in}}%
\pgfpathlineto{\pgfqpoint{1.492697in}{0.954552in}}%
\pgfpathlineto{\pgfqpoint{1.492593in}{1.298932in}}%
\pgfpathlineto{\pgfqpoint{1.492801in}{0.963979in}}%
\pgfpathlineto{\pgfqpoint{1.493113in}{1.289299in}}%
\pgfpathlineto{\pgfqpoint{1.493426in}{0.787875in}}%
\pgfpathlineto{\pgfqpoint{1.493842in}{1.287445in}}%
\pgfpathlineto{\pgfqpoint{1.494467in}{0.768901in}}%
\pgfpathlineto{\pgfqpoint{1.494988in}{1.098984in}}%
\pgfpathlineto{\pgfqpoint{1.495092in}{1.222776in}}%
\pgfpathlineto{\pgfqpoint{1.495404in}{0.683966in}}%
\pgfpathlineto{\pgfqpoint{1.496133in}{1.151959in}}%
\pgfpathlineto{\pgfqpoint{1.496446in}{0.902085in}}%
\pgfpathlineto{\pgfqpoint{1.496342in}{1.290326in}}%
\pgfpathlineto{\pgfqpoint{1.497383in}{1.018990in}}%
\pgfpathlineto{\pgfqpoint{1.498008in}{1.251978in}}%
\pgfpathlineto{\pgfqpoint{1.497591in}{0.880912in}}%
\pgfpathlineto{\pgfqpoint{1.498529in}{1.113398in}}%
\pgfpathlineto{\pgfqpoint{1.498633in}{0.786060in}}%
\pgfpathlineto{\pgfqpoint{1.498945in}{1.241895in}}%
\pgfpathlineto{\pgfqpoint{1.499674in}{1.004949in}}%
\pgfpathlineto{\pgfqpoint{1.499778in}{0.797973in}}%
\pgfpathlineto{\pgfqpoint{1.500299in}{1.223301in}}%
\pgfpathlineto{\pgfqpoint{1.500716in}{0.938436in}}%
\pgfpathlineto{\pgfqpoint{1.501861in}{1.355890in}}%
\pgfpathlineto{\pgfqpoint{1.501653in}{0.819549in}}%
\pgfpathlineto{\pgfqpoint{1.501965in}{1.280742in}}%
\pgfpathlineto{\pgfqpoint{1.502382in}{0.806902in}}%
\pgfpathlineto{\pgfqpoint{1.502590in}{1.289975in}}%
\pgfpathlineto{\pgfqpoint{1.503007in}{0.994826in}}%
\pgfpathlineto{\pgfqpoint{1.503111in}{1.353102in}}%
\pgfpathlineto{\pgfqpoint{1.503736in}{0.884322in}}%
\pgfpathlineto{\pgfqpoint{1.504048in}{1.077008in}}%
\pgfpathlineto{\pgfqpoint{1.504152in}{0.868710in}}%
\pgfpathlineto{\pgfqpoint{1.504881in}{1.256560in}}%
\pgfpathlineto{\pgfqpoint{1.505090in}{1.003770in}}%
\pgfpathlineto{\pgfqpoint{1.505610in}{1.325578in}}%
\pgfpathlineto{\pgfqpoint{1.506235in}{1.291048in}}%
\pgfpathlineto{\pgfqpoint{1.506652in}{0.732263in}}%
\pgfpathlineto{\pgfqpoint{1.507381in}{0.934570in}}%
\pgfpathlineto{\pgfqpoint{1.507797in}{1.349875in}}%
\pgfpathlineto{\pgfqpoint{1.508110in}{0.830482in}}%
\pgfpathlineto{\pgfqpoint{1.508422in}{1.208417in}}%
\pgfpathlineto{\pgfqpoint{1.509255in}{0.803811in}}%
\pgfpathlineto{\pgfqpoint{1.509568in}{0.978395in}}%
\pgfpathlineto{\pgfqpoint{1.510609in}{1.259454in}}%
\pgfpathlineto{\pgfqpoint{1.510713in}{1.229102in}}%
\pgfpathlineto{\pgfqpoint{1.511442in}{0.870008in}}%
\pgfpathlineto{\pgfqpoint{1.511859in}{0.892822in}}%
\pgfpathlineto{\pgfqpoint{1.512484in}{1.238860in}}%
\pgfpathlineto{\pgfqpoint{1.512171in}{0.885105in}}%
\pgfpathlineto{\pgfqpoint{1.513004in}{0.997119in}}%
\pgfpathlineto{\pgfqpoint{1.513213in}{1.103215in}}%
\pgfpathlineto{\pgfqpoint{1.513421in}{1.044461in}}%
\pgfpathlineto{\pgfqpoint{1.514046in}{0.788609in}}%
\pgfpathlineto{\pgfqpoint{1.513837in}{1.290653in}}%
\pgfpathlineto{\pgfqpoint{1.514462in}{0.829516in}}%
\pgfpathlineto{\pgfqpoint{1.514983in}{1.298569in}}%
\pgfpathlineto{\pgfqpoint{1.514775in}{0.688870in}}%
\pgfpathlineto{\pgfqpoint{1.515608in}{1.208744in}}%
\pgfpathlineto{\pgfqpoint{1.515712in}{0.783522in}}%
\pgfpathlineto{\pgfqpoint{1.516233in}{1.232334in}}%
\pgfpathlineto{\pgfqpoint{1.516649in}{1.231359in}}%
\pgfpathlineto{\pgfqpoint{1.517899in}{0.794519in}}%
\pgfpathlineto{\pgfqpoint{1.518524in}{1.227021in}}%
\pgfpathlineto{\pgfqpoint{1.518940in}{0.931051in}}%
\pgfpathlineto{\pgfqpoint{1.519044in}{0.876083in}}%
\pgfpathlineto{\pgfqpoint{1.519149in}{1.169004in}}%
\pgfpathlineto{\pgfqpoint{1.519669in}{0.942050in}}%
\pgfpathlineto{\pgfqpoint{1.520294in}{1.300479in}}%
\pgfpathlineto{\pgfqpoint{1.520086in}{0.808100in}}%
\pgfpathlineto{\pgfqpoint{1.520711in}{0.984356in}}%
\pgfpathlineto{\pgfqpoint{1.520919in}{0.893234in}}%
\pgfpathlineto{\pgfqpoint{1.521023in}{1.460725in}}%
\pgfpathlineto{\pgfqpoint{1.521960in}{0.771327in}}%
\pgfpathlineto{\pgfqpoint{1.522065in}{1.316639in}}%
\pgfpathlineto{\pgfqpoint{1.522794in}{0.780032in}}%
\pgfpathlineto{\pgfqpoint{1.523210in}{1.009078in}}%
\pgfpathlineto{\pgfqpoint{1.523314in}{1.222111in}}%
\pgfpathlineto{\pgfqpoint{1.523835in}{0.683855in}}%
\pgfpathlineto{\pgfqpoint{1.524251in}{1.185123in}}%
\pgfpathlineto{\pgfqpoint{1.524564in}{0.824051in}}%
\pgfpathlineto{\pgfqpoint{1.524772in}{1.340991in}}%
\pgfpathlineto{\pgfqpoint{1.525293in}{0.898056in}}%
\pgfpathlineto{\pgfqpoint{1.525709in}{0.794466in}}%
\pgfpathlineto{\pgfqpoint{1.526438in}{1.336652in}}%
\pgfpathlineto{\pgfqpoint{1.526959in}{0.828520in}}%
\pgfpathlineto{\pgfqpoint{1.527584in}{0.994530in}}%
\pgfpathlineto{\pgfqpoint{1.528521in}{1.261340in}}%
\pgfpathlineto{\pgfqpoint{1.527792in}{0.817033in}}%
\pgfpathlineto{\pgfqpoint{1.528625in}{1.008687in}}%
\pgfpathlineto{\pgfqpoint{1.528730in}{0.969803in}}%
\pgfpathlineto{\pgfqpoint{1.528834in}{1.182186in}}%
\pgfpathlineto{\pgfqpoint{1.529042in}{1.026660in}}%
\pgfpathlineto{\pgfqpoint{1.529771in}{1.259294in}}%
\pgfpathlineto{\pgfqpoint{1.529563in}{0.916919in}}%
\pgfpathlineto{\pgfqpoint{1.530083in}{1.030414in}}%
\pgfpathlineto{\pgfqpoint{1.530292in}{1.164279in}}%
\pgfpathlineto{\pgfqpoint{1.530812in}{0.906257in}}%
\pgfpathlineto{\pgfqpoint{1.530917in}{1.078876in}}%
\pgfpathlineto{\pgfqpoint{1.531021in}{0.736768in}}%
\pgfpathlineto{\pgfqpoint{1.531645in}{1.263288in}}%
\pgfpathlineto{\pgfqpoint{1.531958in}{1.074630in}}%
\pgfpathlineto{\pgfqpoint{1.532374in}{1.250308in}}%
\pgfpathlineto{\pgfqpoint{1.532687in}{0.833028in}}%
\pgfpathlineto{\pgfqpoint{1.532999in}{1.102786in}}%
\pgfpathlineto{\pgfqpoint{1.533312in}{0.750930in}}%
\pgfpathlineto{\pgfqpoint{1.533416in}{1.233601in}}%
\pgfpathlineto{\pgfqpoint{1.534145in}{0.947968in}}%
\pgfpathlineto{\pgfqpoint{1.534457in}{1.310655in}}%
\pgfpathlineto{\pgfqpoint{1.534353in}{0.937314in}}%
\pgfpathlineto{\pgfqpoint{1.535290in}{1.157238in}}%
\pgfpathlineto{\pgfqpoint{1.535395in}{1.147705in}}%
\pgfpathlineto{\pgfqpoint{1.536019in}{0.821081in}}%
\pgfpathlineto{\pgfqpoint{1.536332in}{1.319309in}}%
\pgfpathlineto{\pgfqpoint{1.536540in}{0.991525in}}%
\pgfpathlineto{\pgfqpoint{1.536644in}{0.996984in}}%
\pgfpathlineto{\pgfqpoint{1.537477in}{1.308827in}}%
\pgfpathlineto{\pgfqpoint{1.537269in}{0.874360in}}%
\pgfpathlineto{\pgfqpoint{1.537686in}{1.069996in}}%
\pgfpathlineto{\pgfqpoint{1.538311in}{1.181759in}}%
\pgfpathlineto{\pgfqpoint{1.538727in}{0.763538in}}%
\pgfpathlineto{\pgfqpoint{1.538831in}{1.351823in}}%
\pgfpathlineto{\pgfqpoint{1.539873in}{1.038896in}}%
\pgfpathlineto{\pgfqpoint{1.539977in}{0.795965in}}%
\pgfpathlineto{\pgfqpoint{1.540393in}{1.264789in}}%
\pgfpathlineto{\pgfqpoint{1.540914in}{1.162863in}}%
\pgfpathlineto{\pgfqpoint{1.541539in}{0.946627in}}%
\pgfpathlineto{\pgfqpoint{1.541851in}{1.207704in}}%
\pgfpathlineto{\pgfqpoint{1.542060in}{1.110159in}}%
\pgfpathlineto{\pgfqpoint{1.542789in}{0.781066in}}%
\pgfpathlineto{\pgfqpoint{1.542684in}{1.264710in}}%
\pgfpathlineto{\pgfqpoint{1.543101in}{1.114728in}}%
\pgfpathlineto{\pgfqpoint{1.543830in}{1.287790in}}%
\pgfpathlineto{\pgfqpoint{1.543518in}{0.741830in}}%
\pgfpathlineto{\pgfqpoint{1.544142in}{1.139911in}}%
\pgfpathlineto{\pgfqpoint{1.544559in}{0.835045in}}%
\pgfpathlineto{\pgfqpoint{1.544663in}{1.244381in}}%
\pgfpathlineto{\pgfqpoint{1.545288in}{1.047555in}}%
\pgfpathlineto{\pgfqpoint{1.545705in}{0.876556in}}%
\pgfpathlineto{\pgfqpoint{1.545809in}{1.236688in}}%
\pgfpathlineto{\pgfqpoint{1.546017in}{1.063847in}}%
\pgfpathlineto{\pgfqpoint{1.546642in}{1.325568in}}%
\pgfpathlineto{\pgfqpoint{1.546538in}{0.903679in}}%
\pgfpathlineto{\pgfqpoint{1.547163in}{1.175937in}}%
\pgfpathlineto{\pgfqpoint{1.547267in}{0.703024in}}%
\pgfpathlineto{\pgfqpoint{1.547891in}{1.325543in}}%
\pgfpathlineto{\pgfqpoint{1.548204in}{0.994738in}}%
\pgfpathlineto{\pgfqpoint{1.548620in}{1.373115in}}%
\pgfpathlineto{\pgfqpoint{1.548933in}{0.879566in}}%
\pgfpathlineto{\pgfqpoint{1.549349in}{1.173016in}}%
\pgfpathlineto{\pgfqpoint{1.549454in}{1.235146in}}%
\pgfpathlineto{\pgfqpoint{1.549558in}{0.830116in}}%
\pgfpathlineto{\pgfqpoint{1.550078in}{1.203750in}}%
\pgfpathlineto{\pgfqpoint{1.550807in}{1.261036in}}%
\pgfpathlineto{\pgfqpoint{1.551328in}{0.752601in}}%
\pgfpathlineto{\pgfqpoint{1.551953in}{1.349129in}}%
\pgfpathlineto{\pgfqpoint{1.552474in}{1.007602in}}%
\pgfpathlineto{\pgfqpoint{1.553099in}{0.909305in}}%
\pgfpathlineto{\pgfqpoint{1.552786in}{1.258396in}}%
\pgfpathlineto{\pgfqpoint{1.553203in}{1.130070in}}%
\pgfpathlineto{\pgfqpoint{1.553307in}{1.370845in}}%
\pgfpathlineto{\pgfqpoint{1.553828in}{0.651004in}}%
\pgfpathlineto{\pgfqpoint{1.554244in}{1.164782in}}%
\pgfpathlineto{\pgfqpoint{1.554557in}{1.269528in}}%
\pgfpathlineto{\pgfqpoint{1.555390in}{0.751768in}}%
\pgfpathlineto{\pgfqpoint{1.555806in}{1.238086in}}%
\pgfpathlineto{\pgfqpoint{1.556535in}{1.006563in}}%
\pgfpathlineto{\pgfqpoint{1.557264in}{0.642690in}}%
\pgfpathlineto{\pgfqpoint{1.557160in}{1.278074in}}%
\pgfpathlineto{\pgfqpoint{1.557577in}{1.127468in}}%
\pgfpathlineto{\pgfqpoint{1.557993in}{0.779072in}}%
\pgfpathlineto{\pgfqpoint{1.558514in}{1.205001in}}%
\pgfpathlineto{\pgfqpoint{1.558826in}{0.979289in}}%
\pgfpathlineto{\pgfqpoint{1.559451in}{1.406860in}}%
\pgfpathlineto{\pgfqpoint{1.559139in}{0.873975in}}%
\pgfpathlineto{\pgfqpoint{1.559868in}{1.039880in}}%
\pgfpathlineto{\pgfqpoint{1.560284in}{0.847584in}}%
\pgfpathlineto{\pgfqpoint{1.560180in}{1.044154in}}%
\pgfpathlineto{\pgfqpoint{1.560388in}{1.042974in}}%
\pgfpathlineto{\pgfqpoint{1.560493in}{1.273998in}}%
\pgfpathlineto{\pgfqpoint{1.560805in}{0.804346in}}%
\pgfpathlineto{\pgfqpoint{1.561430in}{1.127878in}}%
\pgfpathlineto{\pgfqpoint{1.562159in}{1.277161in}}%
\pgfpathlineto{\pgfqpoint{1.562575in}{0.783672in}}%
\pgfpathlineto{\pgfqpoint{1.562680in}{1.245795in}}%
\pgfpathlineto{\pgfqpoint{1.563617in}{0.997985in}}%
\pgfpathlineto{\pgfqpoint{1.564242in}{0.819026in}}%
\pgfpathlineto{\pgfqpoint{1.564346in}{1.146043in}}%
\pgfpathlineto{\pgfqpoint{1.564658in}{1.287591in}}%
\pgfpathlineto{\pgfqpoint{1.565075in}{0.850844in}}%
\pgfpathlineto{\pgfqpoint{1.565283in}{0.940267in}}%
\pgfpathlineto{\pgfqpoint{1.565387in}{0.939650in}}%
\pgfpathlineto{\pgfqpoint{1.566220in}{1.425773in}}%
\pgfpathlineto{\pgfqpoint{1.565595in}{0.777641in}}%
\pgfpathlineto{\pgfqpoint{1.566429in}{1.020278in}}%
\pgfpathlineto{\pgfqpoint{1.566949in}{0.835459in}}%
\pgfpathlineto{\pgfqpoint{1.566845in}{1.230924in}}%
\pgfpathlineto{\pgfqpoint{1.567366in}{1.055132in}}%
\pgfpathlineto{\pgfqpoint{1.567678in}{1.282960in}}%
\pgfpathlineto{\pgfqpoint{1.567887in}{0.856658in}}%
\pgfpathlineto{\pgfqpoint{1.568199in}{1.153735in}}%
\pgfpathlineto{\pgfqpoint{1.568303in}{0.865499in}}%
\pgfpathlineto{\pgfqpoint{1.569240in}{1.089083in}}%
\pgfpathlineto{\pgfqpoint{1.569865in}{0.852229in}}%
\pgfpathlineto{\pgfqpoint{1.570074in}{1.111104in}}%
\pgfpathlineto{\pgfqpoint{1.570178in}{0.866570in}}%
\pgfpathlineto{\pgfqpoint{1.570594in}{1.253546in}}%
\pgfpathlineto{\pgfqpoint{1.571219in}{0.934606in}}%
\pgfpathlineto{\pgfqpoint{1.571427in}{0.839715in}}%
\pgfpathlineto{\pgfqpoint{1.571740in}{1.149769in}}%
\pgfpathlineto{\pgfqpoint{1.572052in}{0.954003in}}%
\pgfpathlineto{\pgfqpoint{1.572156in}{1.223401in}}%
\pgfpathlineto{\pgfqpoint{1.572469in}{0.896204in}}%
\pgfpathlineto{\pgfqpoint{1.573094in}{1.035569in}}%
\pgfpathlineto{\pgfqpoint{1.573302in}{0.845717in}}%
\pgfpathlineto{\pgfqpoint{1.573927in}{1.191198in}}%
\pgfpathlineto{\pgfqpoint{1.574031in}{1.063502in}}%
\pgfpathlineto{\pgfqpoint{1.574864in}{1.287291in}}%
\pgfpathlineto{\pgfqpoint{1.574343in}{0.478138in}}%
\pgfpathlineto{\pgfqpoint{1.575176in}{1.177088in}}%
\pgfpathlineto{\pgfqpoint{1.575801in}{0.826899in}}%
\pgfpathlineto{\pgfqpoint{1.575489in}{1.211524in}}%
\pgfpathlineto{\pgfqpoint{1.576010in}{1.082512in}}%
\pgfpathlineto{\pgfqpoint{1.576114in}{1.300370in}}%
\pgfpathlineto{\pgfqpoint{1.576218in}{0.824343in}}%
\pgfpathlineto{\pgfqpoint{1.577051in}{1.046606in}}%
\pgfpathlineto{\pgfqpoint{1.577676in}{1.392769in}}%
\pgfpathlineto{\pgfqpoint{1.577363in}{0.880426in}}%
\pgfpathlineto{\pgfqpoint{1.577884in}{1.174971in}}%
\pgfpathlineto{\pgfqpoint{1.578301in}{0.749863in}}%
\pgfpathlineto{\pgfqpoint{1.578821in}{1.218751in}}%
\pgfpathlineto{\pgfqpoint{1.578926in}{1.034597in}}%
\pgfpathlineto{\pgfqpoint{1.579446in}{1.216953in}}%
\pgfpathlineto{\pgfqpoint{1.579342in}{0.894193in}}%
\pgfpathlineto{\pgfqpoint{1.579967in}{1.028304in}}%
\pgfpathlineto{\pgfqpoint{1.580175in}{0.806271in}}%
\pgfpathlineto{\pgfqpoint{1.580384in}{1.130906in}}%
\pgfpathlineto{\pgfqpoint{1.580904in}{1.002204in}}%
\pgfpathlineto{\pgfqpoint{1.581008in}{1.372192in}}%
\pgfpathlineto{\pgfqpoint{1.581946in}{0.870400in}}%
\pgfpathlineto{\pgfqpoint{1.582883in}{1.155028in}}%
\pgfpathlineto{\pgfqpoint{1.583195in}{1.141061in}}%
\pgfpathlineto{\pgfqpoint{1.584028in}{0.896565in}}%
\pgfpathlineto{\pgfqpoint{1.583612in}{1.355828in}}%
\pgfpathlineto{\pgfqpoint{1.584237in}{1.049799in}}%
\pgfpathlineto{\pgfqpoint{1.584862in}{0.873855in}}%
\pgfpathlineto{\pgfqpoint{1.585382in}{1.332629in}}%
\pgfpathlineto{\pgfqpoint{1.585486in}{0.854322in}}%
\pgfpathlineto{\pgfqpoint{1.586528in}{1.004898in}}%
\pgfpathlineto{\pgfqpoint{1.587673in}{1.291031in}}%
\pgfpathlineto{\pgfqpoint{1.586736in}{0.889801in}}%
\pgfpathlineto{\pgfqpoint{1.587778in}{1.193188in}}%
\pgfpathlineto{\pgfqpoint{1.587882in}{1.191431in}}%
\pgfpathlineto{\pgfqpoint{1.588402in}{0.869385in}}%
\pgfpathlineto{\pgfqpoint{1.588611in}{1.395262in}}%
\pgfpathlineto{\pgfqpoint{1.588923in}{1.086235in}}%
\pgfpathlineto{\pgfqpoint{1.589548in}{1.247435in}}%
\pgfpathlineto{\pgfqpoint{1.589131in}{0.948749in}}%
\pgfpathlineto{\pgfqpoint{1.589860in}{1.068190in}}%
\pgfpathlineto{\pgfqpoint{1.590173in}{0.801733in}}%
\pgfpathlineto{\pgfqpoint{1.590277in}{1.207923in}}%
\pgfpathlineto{\pgfqpoint{1.591006in}{0.950557in}}%
\pgfpathlineto{\pgfqpoint{1.591110in}{0.929923in}}%
\pgfpathlineto{\pgfqpoint{1.592256in}{1.351141in}}%
\pgfpathlineto{\pgfqpoint{1.591943in}{0.773503in}}%
\pgfpathlineto{\pgfqpoint{1.592360in}{1.323092in}}%
\pgfpathlineto{\pgfqpoint{1.592672in}{0.710079in}}%
\pgfpathlineto{\pgfqpoint{1.593505in}{1.091491in}}%
\pgfpathlineto{\pgfqpoint{1.594547in}{1.287548in}}%
\pgfpathlineto{\pgfqpoint{1.594130in}{0.896935in}}%
\pgfpathlineto{\pgfqpoint{1.594651in}{1.127765in}}%
\pgfpathlineto{\pgfqpoint{1.595172in}{0.920150in}}%
\pgfpathlineto{\pgfqpoint{1.595588in}{1.276705in}}%
\pgfpathlineto{\pgfqpoint{1.595692in}{0.974607in}}%
\pgfpathlineto{\pgfqpoint{1.595796in}{1.300049in}}%
\pgfpathlineto{\pgfqpoint{1.596109in}{0.795708in}}%
\pgfpathlineto{\pgfqpoint{1.596734in}{0.996496in}}%
\pgfpathlineto{\pgfqpoint{1.596838in}{0.999619in}}%
\pgfpathlineto{\pgfqpoint{1.597254in}{1.264708in}}%
\pgfpathlineto{\pgfqpoint{1.597879in}{0.760699in}}%
\pgfpathlineto{\pgfqpoint{1.598712in}{1.229984in}}%
\pgfpathlineto{\pgfqpoint{1.599025in}{0.939376in}}%
\pgfpathlineto{\pgfqpoint{1.599129in}{0.929690in}}%
\pgfpathlineto{\pgfqpoint{1.600274in}{1.317709in}}%
\pgfpathlineto{\pgfqpoint{1.599545in}{0.860936in}}%
\pgfpathlineto{\pgfqpoint{1.600379in}{1.281041in}}%
\pgfpathlineto{\pgfqpoint{1.601212in}{0.803835in}}%
\pgfpathlineto{\pgfqpoint{1.601524in}{1.193043in}}%
\pgfpathlineto{\pgfqpoint{1.601837in}{0.763860in}}%
\pgfpathlineto{\pgfqpoint{1.602357in}{1.345781in}}%
\pgfpathlineto{\pgfqpoint{1.602878in}{0.932530in}}%
\pgfpathlineto{\pgfqpoint{1.603607in}{1.147699in}}%
\pgfpathlineto{\pgfqpoint{1.603503in}{0.793955in}}%
\pgfpathlineto{\pgfqpoint{1.604024in}{1.117839in}}%
\pgfpathlineto{\pgfqpoint{1.604232in}{0.807161in}}%
\pgfpathlineto{\pgfqpoint{1.604857in}{1.315823in}}%
\pgfpathlineto{\pgfqpoint{1.605169in}{1.013589in}}%
\pgfpathlineto{\pgfqpoint{1.605794in}{1.199245in}}%
\pgfpathlineto{\pgfqpoint{1.605690in}{0.890212in}}%
\pgfpathlineto{\pgfqpoint{1.606210in}{0.957709in}}%
\pgfpathlineto{\pgfqpoint{1.606731in}{0.862134in}}%
\pgfpathlineto{\pgfqpoint{1.606627in}{1.325660in}}%
\pgfpathlineto{\pgfqpoint{1.606835in}{1.094359in}}%
\pgfpathlineto{\pgfqpoint{1.606939in}{1.284274in}}%
\pgfpathlineto{\pgfqpoint{1.607356in}{0.851248in}}%
\pgfpathlineto{\pgfqpoint{1.607773in}{0.886891in}}%
\pgfpathlineto{\pgfqpoint{1.607877in}{0.693171in}}%
\pgfpathlineto{\pgfqpoint{1.608189in}{1.289013in}}%
\pgfpathlineto{\pgfqpoint{1.608606in}{0.968528in}}%
\pgfpathlineto{\pgfqpoint{1.608710in}{1.260600in}}%
\pgfpathlineto{\pgfqpoint{1.608918in}{0.745389in}}%
\pgfpathlineto{\pgfqpoint{1.609647in}{1.247063in}}%
\pgfpathlineto{\pgfqpoint{1.609855in}{1.327941in}}%
\pgfpathlineto{\pgfqpoint{1.610689in}{0.889262in}}%
\pgfpathlineto{\pgfqpoint{1.610793in}{1.421779in}}%
\pgfpathlineto{\pgfqpoint{1.611522in}{0.654359in}}%
\pgfpathlineto{\pgfqpoint{1.611834in}{1.101803in}}%
\pgfpathlineto{\pgfqpoint{1.612563in}{0.784956in}}%
\pgfpathlineto{\pgfqpoint{1.612667in}{1.203665in}}%
\pgfpathlineto{\pgfqpoint{1.612876in}{1.027930in}}%
\pgfpathlineto{\pgfqpoint{1.613396in}{1.218347in}}%
\pgfpathlineto{\pgfqpoint{1.613709in}{0.799741in}}%
\pgfpathlineto{\pgfqpoint{1.613917in}{1.059194in}}%
\pgfpathlineto{\pgfqpoint{1.614854in}{1.232174in}}%
\pgfpathlineto{\pgfqpoint{1.614958in}{0.834885in}}%
\pgfpathlineto{\pgfqpoint{1.616000in}{1.314095in}}%
\pgfpathlineto{\pgfqpoint{1.615375in}{0.690436in}}%
\pgfpathlineto{\pgfqpoint{1.616208in}{1.148163in}}%
\pgfpathlineto{\pgfqpoint{1.617041in}{0.845724in}}%
\pgfpathlineto{\pgfqpoint{1.616833in}{1.275766in}}%
\pgfpathlineto{\pgfqpoint{1.617249in}{1.162369in}}%
\pgfpathlineto{\pgfqpoint{1.617458in}{0.961763in}}%
\pgfpathlineto{\pgfqpoint{1.617562in}{1.482128in}}%
\pgfpathlineto{\pgfqpoint{1.617666in}{0.866622in}}%
\pgfpathlineto{\pgfqpoint{1.618603in}{1.137238in}}%
\pgfpathlineto{\pgfqpoint{1.618916in}{1.243832in}}%
\pgfpathlineto{\pgfqpoint{1.619541in}{1.054081in}}%
\pgfpathlineto{\pgfqpoint{1.619749in}{1.276823in}}%
\pgfpathlineto{\pgfqpoint{1.619853in}{0.832776in}}%
\pgfpathlineto{\pgfqpoint{1.620478in}{1.037704in}}%
\pgfpathlineto{\pgfqpoint{1.620894in}{0.651720in}}%
\pgfpathlineto{\pgfqpoint{1.621207in}{1.247460in}}%
\pgfpathlineto{\pgfqpoint{1.621519in}{1.117372in}}%
\pgfpathlineto{\pgfqpoint{1.621623in}{1.241362in}}%
\pgfpathlineto{\pgfqpoint{1.622352in}{0.878304in}}%
\pgfpathlineto{\pgfqpoint{1.622456in}{0.922907in}}%
\pgfpathlineto{\pgfqpoint{1.622665in}{0.835317in}}%
\pgfpathlineto{\pgfqpoint{1.622769in}{1.075672in}}%
\pgfpathlineto{\pgfqpoint{1.623290in}{1.301728in}}%
\pgfpathlineto{\pgfqpoint{1.623081in}{0.833041in}}%
\pgfpathlineto{\pgfqpoint{1.623810in}{0.977829in}}%
\pgfpathlineto{\pgfqpoint{1.624435in}{1.200202in}}%
\pgfpathlineto{\pgfqpoint{1.624852in}{0.855663in}}%
\pgfpathlineto{\pgfqpoint{1.625477in}{1.185597in}}%
\pgfpathlineto{\pgfqpoint{1.625997in}{0.991521in}}%
\pgfpathlineto{\pgfqpoint{1.626622in}{1.343296in}}%
\pgfpathlineto{\pgfqpoint{1.626830in}{0.942471in}}%
\pgfpathlineto{\pgfqpoint{1.627351in}{1.228878in}}%
\pgfpathlineto{\pgfqpoint{1.628601in}{0.849748in}}%
\pgfpathlineto{\pgfqpoint{1.628705in}{0.924978in}}%
\pgfpathlineto{\pgfqpoint{1.629538in}{1.334299in}}%
\pgfpathlineto{\pgfqpoint{1.629122in}{0.708219in}}%
\pgfpathlineto{\pgfqpoint{1.629746in}{1.331381in}}%
\pgfpathlineto{\pgfqpoint{1.630059in}{0.867098in}}%
\pgfpathlineto{\pgfqpoint{1.630892in}{1.173259in}}%
\pgfpathlineto{\pgfqpoint{1.631308in}{0.809108in}}%
\pgfpathlineto{\pgfqpoint{1.631517in}{1.356758in}}%
\pgfpathlineto{\pgfqpoint{1.632142in}{0.861171in}}%
\pgfpathlineto{\pgfqpoint{1.632871in}{1.332539in}}%
\pgfpathlineto{\pgfqpoint{1.633391in}{1.292597in}}%
\pgfpathlineto{\pgfqpoint{1.633808in}{0.760206in}}%
\pgfpathlineto{\pgfqpoint{1.634537in}{1.168891in}}%
\pgfpathlineto{\pgfqpoint{1.635474in}{0.761757in}}%
\pgfpathlineto{\pgfqpoint{1.635266in}{1.431997in}}%
\pgfpathlineto{\pgfqpoint{1.635682in}{1.072159in}}%
\pgfpathlineto{\pgfqpoint{1.635787in}{1.066999in}}%
\pgfpathlineto{\pgfqpoint{1.636307in}{1.149010in}}%
\pgfpathlineto{\pgfqpoint{1.636620in}{0.853055in}}%
\pgfpathlineto{\pgfqpoint{1.636724in}{1.313347in}}%
\pgfpathlineto{\pgfqpoint{1.637765in}{1.032636in}}%
\pgfpathlineto{\pgfqpoint{1.637869in}{1.399153in}}%
\pgfpathlineto{\pgfqpoint{1.638494in}{0.873610in}}%
\pgfpathlineto{\pgfqpoint{1.638807in}{1.139212in}}%
\pgfpathlineto{\pgfqpoint{1.638911in}{0.688326in}}%
\pgfpathlineto{\pgfqpoint{1.639640in}{1.274242in}}%
\pgfpathlineto{\pgfqpoint{1.639848in}{0.779587in}}%
\pgfpathlineto{\pgfqpoint{1.640160in}{1.230159in}}%
\pgfpathlineto{\pgfqpoint{1.640994in}{1.054600in}}%
\pgfpathlineto{\pgfqpoint{1.641618in}{1.241035in}}%
\pgfpathlineto{\pgfqpoint{1.641202in}{0.896935in}}%
\pgfpathlineto{\pgfqpoint{1.642139in}{1.112550in}}%
\pgfpathlineto{\pgfqpoint{1.642972in}{0.870567in}}%
\pgfpathlineto{\pgfqpoint{1.642868in}{1.216662in}}%
\pgfpathlineto{\pgfqpoint{1.643076in}{0.992124in}}%
\pgfpathlineto{\pgfqpoint{1.643910in}{1.291568in}}%
\pgfpathlineto{\pgfqpoint{1.643701in}{0.831438in}}%
\pgfpathlineto{\pgfqpoint{1.644118in}{1.007426in}}%
\pgfpathlineto{\pgfqpoint{1.644430in}{0.968535in}}%
\pgfpathlineto{\pgfqpoint{1.644534in}{1.125652in}}%
\pgfpathlineto{\pgfqpoint{1.644639in}{1.083147in}}%
\pgfpathlineto{\pgfqpoint{1.644847in}{0.899832in}}%
\pgfpathlineto{\pgfqpoint{1.645472in}{1.228404in}}%
\pgfpathlineto{\pgfqpoint{1.646201in}{0.868157in}}%
\pgfpathlineto{\pgfqpoint{1.645888in}{1.440755in}}%
\pgfpathlineto{\pgfqpoint{1.646617in}{1.014828in}}%
\pgfpathlineto{\pgfqpoint{1.647034in}{1.233893in}}%
\pgfpathlineto{\pgfqpoint{1.647554in}{0.914192in}}%
\pgfpathlineto{\pgfqpoint{1.648075in}{1.333294in}}%
\pgfpathlineto{\pgfqpoint{1.648492in}{0.857724in}}%
\pgfpathlineto{\pgfqpoint{1.648700in}{1.137774in}}%
\pgfpathlineto{\pgfqpoint{1.649012in}{1.195478in}}%
\pgfpathlineto{\pgfqpoint{1.649950in}{0.835739in}}%
\pgfpathlineto{\pgfqpoint{1.650366in}{1.280524in}}%
\pgfpathlineto{\pgfqpoint{1.650991in}{0.996651in}}%
\pgfpathlineto{\pgfqpoint{1.651095in}{0.777709in}}%
\pgfpathlineto{\pgfqpoint{1.651928in}{1.196055in}}%
\pgfpathlineto{\pgfqpoint{1.652033in}{0.888465in}}%
\pgfpathlineto{\pgfqpoint{1.652345in}{1.365114in}}%
\pgfpathlineto{\pgfqpoint{1.652657in}{0.776312in}}%
\pgfpathlineto{\pgfqpoint{1.653074in}{0.971478in}}%
\pgfpathlineto{\pgfqpoint{1.653907in}{0.815825in}}%
\pgfpathlineto{\pgfqpoint{1.653491in}{1.280441in}}%
\pgfpathlineto{\pgfqpoint{1.654011in}{0.875965in}}%
\pgfpathlineto{\pgfqpoint{1.654428in}{1.328577in}}%
\pgfpathlineto{\pgfqpoint{1.654844in}{0.641410in}}%
\pgfpathlineto{\pgfqpoint{1.655157in}{1.106203in}}%
\pgfpathlineto{\pgfqpoint{1.655886in}{0.839449in}}%
\pgfpathlineto{\pgfqpoint{1.655990in}{1.202435in}}%
\pgfpathlineto{\pgfqpoint{1.656094in}{0.831745in}}%
\pgfpathlineto{\pgfqpoint{1.657031in}{1.391937in}}%
\pgfpathlineto{\pgfqpoint{1.657135in}{0.842499in}}%
\pgfpathlineto{\pgfqpoint{1.657552in}{1.293447in}}%
\pgfpathlineto{\pgfqpoint{1.657656in}{0.782762in}}%
\pgfpathlineto{\pgfqpoint{1.658281in}{1.026791in}}%
\pgfpathlineto{\pgfqpoint{1.658489in}{1.111246in}}%
\pgfpathlineto{\pgfqpoint{1.658698in}{0.894297in}}%
\pgfpathlineto{\pgfqpoint{1.658802in}{1.465572in}}%
\pgfpathlineto{\pgfqpoint{1.659010in}{0.728499in}}%
\pgfpathlineto{\pgfqpoint{1.659739in}{1.013146in}}%
\pgfpathlineto{\pgfqpoint{1.659843in}{0.999697in}}%
\pgfpathlineto{\pgfqpoint{1.659947in}{1.042247in}}%
\pgfpathlineto{\pgfqpoint{1.660780in}{1.228755in}}%
\pgfpathlineto{\pgfqpoint{1.660885in}{0.956504in}}%
\pgfpathlineto{\pgfqpoint{1.661197in}{1.081033in}}%
\pgfpathlineto{\pgfqpoint{1.661301in}{0.781937in}}%
\pgfpathlineto{\pgfqpoint{1.662030in}{1.257294in}}%
\pgfpathlineto{\pgfqpoint{1.662447in}{1.109044in}}%
\pgfpathlineto{\pgfqpoint{1.662759in}{0.776654in}}%
\pgfpathlineto{\pgfqpoint{1.663176in}{1.252833in}}%
\pgfpathlineto{\pgfqpoint{1.663384in}{1.012237in}}%
\pgfpathlineto{\pgfqpoint{1.663800in}{1.320058in}}%
\pgfpathlineto{\pgfqpoint{1.664217in}{0.829686in}}%
\pgfpathlineto{\pgfqpoint{1.664425in}{0.890393in}}%
\pgfpathlineto{\pgfqpoint{1.665571in}{1.342532in}}%
\pgfpathlineto{\pgfqpoint{1.664634in}{0.879301in}}%
\pgfpathlineto{\pgfqpoint{1.665779in}{1.000882in}}%
\pgfpathlineto{\pgfqpoint{1.666300in}{0.809597in}}%
\pgfpathlineto{\pgfqpoint{1.666508in}{1.267780in}}%
\pgfpathlineto{\pgfqpoint{1.666716in}{0.923961in}}%
\pgfpathlineto{\pgfqpoint{1.667029in}{1.260404in}}%
\pgfpathlineto{\pgfqpoint{1.667341in}{0.802297in}}%
\pgfpathlineto{\pgfqpoint{1.667862in}{1.063831in}}%
\pgfpathlineto{\pgfqpoint{1.668487in}{0.837238in}}%
\pgfpathlineto{\pgfqpoint{1.668695in}{1.216108in}}%
\pgfpathlineto{\pgfqpoint{1.668903in}{0.976370in}}%
\pgfpathlineto{\pgfqpoint{1.669737in}{0.813582in}}%
\pgfpathlineto{\pgfqpoint{1.669945in}{1.232077in}}%
\pgfpathlineto{\pgfqpoint{1.670882in}{1.369723in}}%
\pgfpathlineto{\pgfqpoint{1.671090in}{0.748072in}}%
\pgfpathlineto{\pgfqpoint{1.672340in}{1.364922in}}%
\pgfpathlineto{\pgfqpoint{1.673381in}{0.763631in}}%
\pgfpathlineto{\pgfqpoint{1.673486in}{1.058016in}}%
\pgfpathlineto{\pgfqpoint{1.673590in}{1.064639in}}%
\pgfpathlineto{\pgfqpoint{1.673694in}{1.148006in}}%
\pgfpathlineto{\pgfqpoint{1.674110in}{0.717678in}}%
\pgfpathlineto{\pgfqpoint{1.674527in}{0.978079in}}%
\pgfpathlineto{\pgfqpoint{1.674631in}{0.975126in}}%
\pgfpathlineto{\pgfqpoint{1.674839in}{0.579445in}}%
\pgfpathlineto{\pgfqpoint{1.674944in}{1.275676in}}%
\pgfpathlineto{\pgfqpoint{1.675673in}{0.996569in}}%
\pgfpathlineto{\pgfqpoint{1.675777in}{1.300623in}}%
\pgfpathlineto{\pgfqpoint{1.676610in}{0.723963in}}%
\pgfpathlineto{\pgfqpoint{1.676714in}{0.888942in}}%
\pgfpathlineto{\pgfqpoint{1.677755in}{1.289275in}}%
\pgfpathlineto{\pgfqpoint{1.677964in}{1.191173in}}%
\pgfpathlineto{\pgfqpoint{1.678068in}{0.610874in}}%
\pgfpathlineto{\pgfqpoint{1.678380in}{1.348826in}}%
\pgfpathlineto{\pgfqpoint{1.679109in}{0.823277in}}%
\pgfpathlineto{\pgfqpoint{1.679942in}{1.286254in}}%
\pgfpathlineto{\pgfqpoint{1.679630in}{0.814637in}}%
\pgfpathlineto{\pgfqpoint{1.680255in}{1.042302in}}%
\pgfpathlineto{\pgfqpoint{1.681192in}{0.774590in}}%
\pgfpathlineto{\pgfqpoint{1.680984in}{1.179442in}}%
\pgfpathlineto{\pgfqpoint{1.681296in}{0.983256in}}%
\pgfpathlineto{\pgfqpoint{1.681504in}{1.233748in}}%
\pgfpathlineto{\pgfqpoint{1.682129in}{0.957121in}}%
\pgfpathlineto{\pgfqpoint{1.682338in}{1.052423in}}%
\pgfpathlineto{\pgfqpoint{1.682650in}{0.949433in}}%
\pgfpathlineto{\pgfqpoint{1.682546in}{1.128776in}}%
\pgfpathlineto{\pgfqpoint{1.683379in}{1.083846in}}%
\pgfpathlineto{\pgfqpoint{1.683900in}{0.823072in}}%
\pgfpathlineto{\pgfqpoint{1.683587in}{1.304615in}}%
\pgfpathlineto{\pgfqpoint{1.684629in}{0.966746in}}%
\pgfpathlineto{\pgfqpoint{1.685149in}{1.295802in}}%
\pgfpathlineto{\pgfqpoint{1.685358in}{0.952945in}}%
\pgfpathlineto{\pgfqpoint{1.685670in}{0.966864in}}%
\pgfpathlineto{\pgfqpoint{1.685774in}{0.866245in}}%
\pgfpathlineto{\pgfqpoint{1.686399in}{1.240692in}}%
\pgfpathlineto{\pgfqpoint{1.686503in}{1.244351in}}%
\pgfpathlineto{\pgfqpoint{1.687545in}{0.860989in}}%
\pgfpathlineto{\pgfqpoint{1.688378in}{1.270301in}}%
\pgfpathlineto{\pgfqpoint{1.687961in}{0.857826in}}%
\pgfpathlineto{\pgfqpoint{1.688586in}{0.937469in}}%
\pgfpathlineto{\pgfqpoint{1.688690in}{0.949708in}}%
\pgfpathlineto{\pgfqpoint{1.689315in}{1.262570in}}%
\pgfpathlineto{\pgfqpoint{1.689419in}{0.767859in}}%
\pgfpathlineto{\pgfqpoint{1.689732in}{0.857569in}}%
\pgfpathlineto{\pgfqpoint{1.690773in}{1.247408in}}%
\pgfpathlineto{\pgfqpoint{1.691085in}{1.123683in}}%
\pgfpathlineto{\pgfqpoint{1.691294in}{1.131209in}}%
\pgfpathlineto{\pgfqpoint{1.692231in}{0.831768in}}%
\pgfpathlineto{\pgfqpoint{1.692543in}{0.803749in}}%
\pgfpathlineto{\pgfqpoint{1.693377in}{1.134961in}}%
\pgfpathlineto{\pgfqpoint{1.694210in}{0.854199in}}%
\pgfpathlineto{\pgfqpoint{1.694314in}{1.302698in}}%
\pgfpathlineto{\pgfqpoint{1.694522in}{1.059365in}}%
\pgfpathlineto{\pgfqpoint{1.695355in}{0.882939in}}%
\pgfpathlineto{\pgfqpoint{1.695251in}{1.325971in}}%
\pgfpathlineto{\pgfqpoint{1.695668in}{0.896986in}}%
\pgfpathlineto{\pgfqpoint{1.695980in}{0.688145in}}%
\pgfpathlineto{\pgfqpoint{1.696709in}{1.231892in}}%
\pgfpathlineto{\pgfqpoint{1.697126in}{1.382119in}}%
\pgfpathlineto{\pgfqpoint{1.697855in}{0.737901in}}%
\pgfpathlineto{\pgfqpoint{1.698896in}{1.147461in}}%
\pgfpathlineto{\pgfqpoint{1.698271in}{0.617519in}}%
\pgfpathlineto{\pgfqpoint{1.699000in}{0.903459in}}%
\pgfpathlineto{\pgfqpoint{1.699521in}{1.323028in}}%
\pgfpathlineto{\pgfqpoint{1.700146in}{1.214138in}}%
\pgfpathlineto{\pgfqpoint{1.700354in}{0.710161in}}%
\pgfpathlineto{\pgfqpoint{1.700979in}{1.298724in}}%
\pgfpathlineto{\pgfqpoint{1.701395in}{1.039874in}}%
\pgfpathlineto{\pgfqpoint{1.702124in}{1.227353in}}%
\pgfpathlineto{\pgfqpoint{1.701812in}{0.861422in}}%
\pgfpathlineto{\pgfqpoint{1.702333in}{1.012346in}}%
\pgfpathlineto{\pgfqpoint{1.702437in}{0.881146in}}%
\pgfpathlineto{\pgfqpoint{1.702645in}{1.217292in}}%
\pgfpathlineto{\pgfqpoint{1.703374in}{1.042629in}}%
\pgfpathlineto{\pgfqpoint{1.703895in}{1.192454in}}%
\pgfpathlineto{\pgfqpoint{1.703791in}{0.940107in}}%
\pgfpathlineto{\pgfqpoint{1.704207in}{0.944405in}}%
\pgfpathlineto{\pgfqpoint{1.704624in}{0.815687in}}%
\pgfpathlineto{\pgfqpoint{1.704415in}{1.246254in}}%
\pgfpathlineto{\pgfqpoint{1.704936in}{1.072436in}}%
\pgfpathlineto{\pgfqpoint{1.705040in}{1.288864in}}%
\pgfpathlineto{\pgfqpoint{1.705769in}{0.809262in}}%
\pgfpathlineto{\pgfqpoint{1.705873in}{1.199642in}}%
\pgfpathlineto{\pgfqpoint{1.706707in}{0.887775in}}%
\pgfpathlineto{\pgfqpoint{1.706915in}{1.212045in}}%
\pgfpathlineto{\pgfqpoint{1.707019in}{1.053544in}}%
\pgfpathlineto{\pgfqpoint{1.707540in}{1.373025in}}%
\pgfpathlineto{\pgfqpoint{1.707748in}{0.808566in}}%
\pgfpathlineto{\pgfqpoint{1.708060in}{1.175541in}}%
\pgfpathlineto{\pgfqpoint{1.708894in}{0.791441in}}%
\pgfpathlineto{\pgfqpoint{1.709102in}{1.240830in}}%
\pgfpathlineto{\pgfqpoint{1.709206in}{1.006137in}}%
\pgfpathlineto{\pgfqpoint{1.709310in}{1.010414in}}%
\pgfpathlineto{\pgfqpoint{1.709518in}{1.270708in}}%
\pgfpathlineto{\pgfqpoint{1.709831in}{0.816435in}}%
\pgfpathlineto{\pgfqpoint{1.710352in}{0.972970in}}%
\pgfpathlineto{\pgfqpoint{1.710560in}{0.881315in}}%
\pgfpathlineto{\pgfqpoint{1.710872in}{1.308309in}}%
\pgfpathlineto{\pgfqpoint{1.711081in}{0.991023in}}%
\pgfpathlineto{\pgfqpoint{1.711705in}{1.289837in}}%
\pgfpathlineto{\pgfqpoint{1.711601in}{0.621397in}}%
\pgfpathlineto{\pgfqpoint{1.712122in}{1.078171in}}%
\pgfpathlineto{\pgfqpoint{1.712434in}{0.862214in}}%
\pgfpathlineto{\pgfqpoint{1.712330in}{1.144719in}}%
\pgfpathlineto{\pgfqpoint{1.713163in}{1.088934in}}%
\pgfpathlineto{\pgfqpoint{1.713996in}{1.196053in}}%
\pgfpathlineto{\pgfqpoint{1.713372in}{0.778833in}}%
\pgfpathlineto{\pgfqpoint{1.714101in}{1.058997in}}%
\pgfpathlineto{\pgfqpoint{1.714205in}{0.997957in}}%
\pgfpathlineto{\pgfqpoint{1.714725in}{1.214679in}}%
\pgfpathlineto{\pgfqpoint{1.715038in}{1.062190in}}%
\pgfpathlineto{\pgfqpoint{1.715350in}{1.209956in}}%
\pgfpathlineto{\pgfqpoint{1.715663in}{0.894689in}}%
\pgfpathlineto{\pgfqpoint{1.715871in}{1.197523in}}%
\pgfpathlineto{\pgfqpoint{1.715975in}{0.822678in}}%
\pgfpathlineto{\pgfqpoint{1.716288in}{1.224913in}}%
\pgfpathlineto{\pgfqpoint{1.717017in}{0.883017in}}%
\pgfpathlineto{\pgfqpoint{1.717121in}{1.270147in}}%
\pgfpathlineto{\pgfqpoint{1.717225in}{0.859543in}}%
\pgfpathlineto{\pgfqpoint{1.718058in}{1.043305in}}%
\pgfpathlineto{\pgfqpoint{1.718370in}{0.690331in}}%
\pgfpathlineto{\pgfqpoint{1.718579in}{1.217593in}}%
\pgfpathlineto{\pgfqpoint{1.718891in}{1.150443in}}%
\pgfpathlineto{\pgfqpoint{1.719412in}{1.400696in}}%
\pgfpathlineto{\pgfqpoint{1.719308in}{0.768890in}}%
\pgfpathlineto{\pgfqpoint{1.719933in}{1.038312in}}%
\pgfpathlineto{\pgfqpoint{1.720037in}{1.037578in}}%
\pgfpathlineto{\pgfqpoint{1.720870in}{1.191250in}}%
\pgfpathlineto{\pgfqpoint{1.720349in}{0.880025in}}%
\pgfpathlineto{\pgfqpoint{1.721078in}{1.080409in}}%
\pgfpathlineto{\pgfqpoint{1.721182in}{0.558443in}}%
\pgfpathlineto{\pgfqpoint{1.721807in}{1.201522in}}%
\pgfpathlineto{\pgfqpoint{1.722119in}{0.800664in}}%
\pgfpathlineto{\pgfqpoint{1.722953in}{1.315179in}}%
\pgfpathlineto{\pgfqpoint{1.722848in}{0.773255in}}%
\pgfpathlineto{\pgfqpoint{1.723265in}{1.187345in}}%
\pgfpathlineto{\pgfqpoint{1.723369in}{0.840958in}}%
\pgfpathlineto{\pgfqpoint{1.724202in}{1.300485in}}%
\pgfpathlineto{\pgfqpoint{1.724411in}{1.042208in}}%
\pgfpathlineto{\pgfqpoint{1.724827in}{1.217469in}}%
\pgfpathlineto{\pgfqpoint{1.725348in}{0.918611in}}%
\pgfpathlineto{\pgfqpoint{1.725452in}{1.014593in}}%
\pgfpathlineto{\pgfqpoint{1.725660in}{0.810156in}}%
\pgfpathlineto{\pgfqpoint{1.725764in}{1.180516in}}%
\pgfpathlineto{\pgfqpoint{1.725869in}{0.647769in}}%
\pgfpathlineto{\pgfqpoint{1.725973in}{1.452860in}}%
\pgfpathlineto{\pgfqpoint{1.726806in}{1.040326in}}%
\pgfpathlineto{\pgfqpoint{1.727014in}{0.861270in}}%
\pgfpathlineto{\pgfqpoint{1.727847in}{1.244688in}}%
\pgfpathlineto{\pgfqpoint{1.729097in}{0.653338in}}%
\pgfpathlineto{\pgfqpoint{1.730034in}{1.254809in}}%
\pgfpathlineto{\pgfqpoint{1.730347in}{1.214915in}}%
\pgfpathlineto{\pgfqpoint{1.730555in}{1.328813in}}%
\pgfpathlineto{\pgfqpoint{1.731492in}{0.910193in}}%
\pgfpathlineto{\pgfqpoint{1.731596in}{1.367757in}}%
\pgfpathlineto{\pgfqpoint{1.731700in}{1.004746in}}%
\pgfpathlineto{\pgfqpoint{1.731805in}{1.008907in}}%
\pgfpathlineto{\pgfqpoint{1.731909in}{1.003248in}}%
\pgfpathlineto{\pgfqpoint{1.732325in}{0.853594in}}%
\pgfpathlineto{\pgfqpoint{1.732429in}{0.938841in}}%
\pgfpathlineto{\pgfqpoint{1.732534in}{1.340389in}}%
\pgfpathlineto{\pgfqpoint{1.733054in}{0.794025in}}%
\pgfpathlineto{\pgfqpoint{1.733471in}{1.010030in}}%
\pgfpathlineto{\pgfqpoint{1.733992in}{0.734676in}}%
\pgfpathlineto{\pgfqpoint{1.733679in}{1.086377in}}%
\pgfpathlineto{\pgfqpoint{1.734616in}{0.888077in}}%
\pgfpathlineto{\pgfqpoint{1.735345in}{1.187338in}}%
\pgfpathlineto{\pgfqpoint{1.735033in}{0.861981in}}%
\pgfpathlineto{\pgfqpoint{1.735866in}{1.155605in}}%
\pgfpathlineto{\pgfqpoint{1.736595in}{0.937157in}}%
\pgfpathlineto{\pgfqpoint{1.736074in}{1.289685in}}%
\pgfpathlineto{\pgfqpoint{1.737012in}{1.128717in}}%
\pgfpathlineto{\pgfqpoint{1.737636in}{1.550219in}}%
\pgfpathlineto{\pgfqpoint{1.737220in}{0.844903in}}%
\pgfpathlineto{\pgfqpoint{1.737845in}{1.042667in}}%
\pgfpathlineto{\pgfqpoint{1.737949in}{0.978470in}}%
\pgfpathlineto{\pgfqpoint{1.738574in}{1.251588in}}%
\pgfpathlineto{\pgfqpoint{1.738678in}{1.257495in}}%
\pgfpathlineto{\pgfqpoint{1.738886in}{0.857197in}}%
\pgfpathlineto{\pgfqpoint{1.739094in}{1.282428in}}%
\pgfpathlineto{\pgfqpoint{1.739823in}{1.097092in}}%
\pgfpathlineto{\pgfqpoint{1.740657in}{1.306535in}}%
\pgfpathlineto{\pgfqpoint{1.740761in}{0.864772in}}%
\pgfpathlineto{\pgfqpoint{1.740865in}{1.238290in}}%
\pgfpathlineto{\pgfqpoint{1.741073in}{0.790418in}}%
\pgfpathlineto{\pgfqpoint{1.742010in}{1.124492in}}%
\pgfpathlineto{\pgfqpoint{1.742219in}{0.844205in}}%
\pgfpathlineto{\pgfqpoint{1.742635in}{1.179143in}}%
\pgfpathlineto{\pgfqpoint{1.743052in}{0.851412in}}%
\pgfpathlineto{\pgfqpoint{1.743260in}{1.131295in}}%
\pgfpathlineto{\pgfqpoint{1.743364in}{0.703398in}}%
\pgfpathlineto{\pgfqpoint{1.744197in}{0.948963in}}%
\pgfpathlineto{\pgfqpoint{1.744302in}{0.960331in}}%
\pgfpathlineto{\pgfqpoint{1.744406in}{0.895526in}}%
\pgfpathlineto{\pgfqpoint{1.745030in}{1.278889in}}%
\pgfpathlineto{\pgfqpoint{1.745239in}{0.756553in}}%
\pgfpathlineto{\pgfqpoint{1.745551in}{1.059221in}}%
\pgfpathlineto{\pgfqpoint{1.746072in}{1.177770in}}%
\pgfpathlineto{\pgfqpoint{1.746801in}{0.566441in}}%
\pgfpathlineto{\pgfqpoint{1.747946in}{1.241599in}}%
\pgfpathlineto{\pgfqpoint{1.748051in}{1.108685in}}%
\pgfpathlineto{\pgfqpoint{1.748675in}{0.850720in}}%
\pgfpathlineto{\pgfqpoint{1.748988in}{1.175409in}}%
\pgfpathlineto{\pgfqpoint{1.749092in}{1.195314in}}%
\pgfpathlineto{\pgfqpoint{1.749509in}{0.787959in}}%
\pgfpathlineto{\pgfqpoint{1.749717in}{1.412686in}}%
\pgfpathlineto{\pgfqpoint{1.750238in}{0.998105in}}%
\pgfpathlineto{\pgfqpoint{1.750758in}{1.537001in}}%
\pgfpathlineto{\pgfqpoint{1.750654in}{0.832417in}}%
\pgfpathlineto{\pgfqpoint{1.751383in}{1.243168in}}%
\pgfpathlineto{\pgfqpoint{1.752216in}{0.866379in}}%
\pgfpathlineto{\pgfqpoint{1.751904in}{1.263280in}}%
\pgfpathlineto{\pgfqpoint{1.752529in}{1.103562in}}%
\pgfpathlineto{\pgfqpoint{1.753049in}{0.704245in}}%
\pgfpathlineto{\pgfqpoint{1.753362in}{1.231501in}}%
\pgfpathlineto{\pgfqpoint{1.753570in}{1.080673in}}%
\pgfpathlineto{\pgfqpoint{1.753674in}{1.087440in}}%
\pgfpathlineto{\pgfqpoint{1.754195in}{0.956682in}}%
\pgfpathlineto{\pgfqpoint{1.754924in}{1.327746in}}%
\pgfpathlineto{\pgfqpoint{1.755861in}{0.849374in}}%
\pgfpathlineto{\pgfqpoint{1.755965in}{1.048291in}}%
\pgfpathlineto{\pgfqpoint{1.756069in}{1.279304in}}%
\pgfpathlineto{\pgfqpoint{1.756694in}{0.627013in}}%
\pgfpathlineto{\pgfqpoint{1.757007in}{0.933789in}}%
\pgfpathlineto{\pgfqpoint{1.757944in}{0.823535in}}%
\pgfpathlineto{\pgfqpoint{1.758256in}{1.282850in}}%
\pgfpathlineto{\pgfqpoint{1.759402in}{0.836238in}}%
\pgfpathlineto{\pgfqpoint{1.759506in}{1.267396in}}%
\pgfpathlineto{\pgfqpoint{1.760443in}{0.758426in}}%
\pgfpathlineto{\pgfqpoint{1.760548in}{1.071548in}}%
\pgfpathlineto{\pgfqpoint{1.760964in}{0.880152in}}%
\pgfpathlineto{\pgfqpoint{1.761068in}{1.289866in}}%
\pgfpathlineto{\pgfqpoint{1.761693in}{0.954887in}}%
\pgfpathlineto{\pgfqpoint{1.762630in}{1.232749in}}%
\pgfpathlineto{\pgfqpoint{1.762734in}{0.948533in}}%
\pgfpathlineto{\pgfqpoint{1.763359in}{1.213255in}}%
\pgfpathlineto{\pgfqpoint{1.763776in}{0.912049in}}%
\pgfpathlineto{\pgfqpoint{1.763880in}{1.137071in}}%
\pgfpathlineto{\pgfqpoint{1.764609in}{1.424745in}}%
\pgfpathlineto{\pgfqpoint{1.764921in}{0.781614in}}%
\pgfpathlineto{\pgfqpoint{1.766067in}{1.436363in}}%
\pgfpathlineto{\pgfqpoint{1.767317in}{0.911544in}}%
\pgfpathlineto{\pgfqpoint{1.767837in}{1.422533in}}%
\pgfpathlineto{\pgfqpoint{1.767525in}{0.786160in}}%
\pgfpathlineto{\pgfqpoint{1.768462in}{1.197688in}}%
\pgfpathlineto{\pgfqpoint{1.769400in}{0.618482in}}%
\pgfpathlineto{\pgfqpoint{1.769295in}{1.275855in}}%
\pgfpathlineto{\pgfqpoint{1.769504in}{1.129051in}}%
\pgfpathlineto{\pgfqpoint{1.769816in}{1.044644in}}%
\pgfpathlineto{\pgfqpoint{1.770128in}{1.273159in}}%
\pgfpathlineto{\pgfqpoint{1.770545in}{0.742242in}}%
\pgfpathlineto{\pgfqpoint{1.770441in}{1.300681in}}%
\pgfpathlineto{\pgfqpoint{1.771274in}{0.967346in}}%
\pgfpathlineto{\pgfqpoint{1.771378in}{1.281506in}}%
\pgfpathlineto{\pgfqpoint{1.772107in}{0.806954in}}%
\pgfpathlineto{\pgfqpoint{1.772315in}{1.055297in}}%
\pgfpathlineto{\pgfqpoint{1.772628in}{0.751421in}}%
\pgfpathlineto{\pgfqpoint{1.772524in}{1.363317in}}%
\pgfpathlineto{\pgfqpoint{1.773357in}{1.012063in}}%
\pgfpathlineto{\pgfqpoint{1.773461in}{1.340825in}}%
\pgfpathlineto{\pgfqpoint{1.773982in}{0.642976in}}%
\pgfpathlineto{\pgfqpoint{1.774502in}{1.272090in}}%
\pgfpathlineto{\pgfqpoint{1.775544in}{0.810211in}}%
\pgfpathlineto{\pgfqpoint{1.776377in}{1.348167in}}%
\pgfpathlineto{\pgfqpoint{1.776689in}{1.097201in}}%
\pgfpathlineto{\pgfqpoint{1.777314in}{1.310304in}}%
\pgfpathlineto{\pgfqpoint{1.776898in}{0.808723in}}%
\pgfpathlineto{\pgfqpoint{1.777418in}{1.022777in}}%
\pgfpathlineto{\pgfqpoint{1.777731in}{0.863190in}}%
\pgfpathlineto{\pgfqpoint{1.777835in}{1.264746in}}%
\pgfpathlineto{\pgfqpoint{1.778460in}{0.902895in}}%
\pgfpathlineto{\pgfqpoint{1.778564in}{1.413349in}}%
\pgfpathlineto{\pgfqpoint{1.779293in}{0.874192in}}%
\pgfpathlineto{\pgfqpoint{1.779501in}{1.051477in}}%
\pgfpathlineto{\pgfqpoint{1.779605in}{1.036928in}}%
\pgfpathlineto{\pgfqpoint{1.779814in}{1.037125in}}%
\pgfpathlineto{\pgfqpoint{1.780647in}{1.270711in}}%
\pgfpathlineto{\pgfqpoint{1.780230in}{0.608608in}}%
\pgfpathlineto{\pgfqpoint{1.780855in}{1.015726in}}%
\pgfpathlineto{\pgfqpoint{1.780959in}{0.966151in}}%
\pgfpathlineto{\pgfqpoint{1.781272in}{1.307988in}}%
\pgfpathlineto{\pgfqpoint{1.781584in}{1.064187in}}%
\pgfpathlineto{\pgfqpoint{1.781688in}{1.174298in}}%
\pgfpathlineto{\pgfqpoint{1.781792in}{0.798979in}}%
\pgfpathlineto{\pgfqpoint{1.782521in}{0.895148in}}%
\pgfpathlineto{\pgfqpoint{1.782625in}{0.771432in}}%
\pgfpathlineto{\pgfqpoint{1.783042in}{1.245515in}}%
\pgfpathlineto{\pgfqpoint{1.783563in}{0.930473in}}%
\pgfpathlineto{\pgfqpoint{1.784500in}{1.289709in}}%
\pgfpathlineto{\pgfqpoint{1.783875in}{0.784540in}}%
\pgfpathlineto{\pgfqpoint{1.784604in}{1.153942in}}%
\pgfpathlineto{\pgfqpoint{1.785333in}{0.780099in}}%
\pgfpathlineto{\pgfqpoint{1.785541in}{1.308130in}}%
\pgfpathlineto{\pgfqpoint{1.785646in}{0.835195in}}%
\pgfpathlineto{\pgfqpoint{1.786583in}{1.211802in}}%
\pgfpathlineto{\pgfqpoint{1.786791in}{0.935255in}}%
\pgfpathlineto{\pgfqpoint{1.786895in}{0.870419in}}%
\pgfpathlineto{\pgfqpoint{1.787103in}{1.284412in}}%
\pgfpathlineto{\pgfqpoint{1.787624in}{1.071883in}}%
\pgfpathlineto{\pgfqpoint{1.788457in}{1.301003in}}%
\pgfpathlineto{\pgfqpoint{1.787937in}{0.846447in}}%
\pgfpathlineto{\pgfqpoint{1.788561in}{1.053840in}}%
\pgfpathlineto{\pgfqpoint{1.789186in}{0.818651in}}%
\pgfpathlineto{\pgfqpoint{1.789082in}{1.117786in}}%
\pgfpathlineto{\pgfqpoint{1.789603in}{1.104624in}}%
\pgfpathlineto{\pgfqpoint{1.789707in}{1.329560in}}%
\pgfpathlineto{\pgfqpoint{1.790019in}{0.910907in}}%
\pgfpathlineto{\pgfqpoint{1.790540in}{1.205870in}}%
\pgfpathlineto{\pgfqpoint{1.791477in}{0.712666in}}%
\pgfpathlineto{\pgfqpoint{1.791165in}{1.235128in}}%
\pgfpathlineto{\pgfqpoint{1.791582in}{1.019319in}}%
\pgfpathlineto{\pgfqpoint{1.792311in}{1.294593in}}%
\pgfpathlineto{\pgfqpoint{1.792519in}{0.888359in}}%
\pgfpathlineto{\pgfqpoint{1.792623in}{1.044390in}}%
\pgfpathlineto{\pgfqpoint{1.792727in}{0.727504in}}%
\pgfpathlineto{\pgfqpoint{1.793040in}{1.300249in}}%
\pgfpathlineto{\pgfqpoint{1.793664in}{1.025564in}}%
\pgfpathlineto{\pgfqpoint{1.793977in}{0.962591in}}%
\pgfpathlineto{\pgfqpoint{1.793873in}{1.098110in}}%
\pgfpathlineto{\pgfqpoint{1.794497in}{1.060121in}}%
\pgfpathlineto{\pgfqpoint{1.795226in}{1.352460in}}%
\pgfpathlineto{\pgfqpoint{1.795331in}{0.920137in}}%
\pgfpathlineto{\pgfqpoint{1.795539in}{0.997265in}}%
\pgfpathlineto{\pgfqpoint{1.795851in}{1.095179in}}%
\pgfpathlineto{\pgfqpoint{1.796372in}{0.720940in}}%
\pgfpathlineto{\pgfqpoint{1.796684in}{1.433104in}}%
\pgfpathlineto{\pgfqpoint{1.796997in}{0.952890in}}%
\pgfpathlineto{\pgfqpoint{1.797934in}{1.177683in}}%
\pgfpathlineto{\pgfqpoint{1.797309in}{0.861090in}}%
\pgfpathlineto{\pgfqpoint{1.798038in}{1.138523in}}%
\pgfpathlineto{\pgfqpoint{1.798871in}{1.412483in}}%
\pgfpathlineto{\pgfqpoint{1.799184in}{0.931928in}}%
\pgfpathlineto{\pgfqpoint{1.799392in}{1.373921in}}%
\pgfpathlineto{\pgfqpoint{1.800746in}{1.350495in}}%
\pgfpathlineto{\pgfqpoint{1.801683in}{0.614524in}}%
\pgfpathlineto{\pgfqpoint{1.801892in}{0.982071in}}%
\pgfpathlineto{\pgfqpoint{1.801996in}{1.257548in}}%
\pgfpathlineto{\pgfqpoint{1.802620in}{0.877550in}}%
\pgfpathlineto{\pgfqpoint{1.802933in}{1.060994in}}%
\pgfpathlineto{\pgfqpoint{1.803037in}{1.065608in}}%
\pgfpathlineto{\pgfqpoint{1.803141in}{1.280931in}}%
\pgfpathlineto{\pgfqpoint{1.803349in}{0.656545in}}%
\pgfpathlineto{\pgfqpoint{1.804078in}{1.084053in}}%
\pgfpathlineto{\pgfqpoint{1.804703in}{0.683583in}}%
\pgfpathlineto{\pgfqpoint{1.804391in}{1.382685in}}%
\pgfpathlineto{\pgfqpoint{1.805120in}{1.065532in}}%
\pgfpathlineto{\pgfqpoint{1.805328in}{1.428076in}}%
\pgfpathlineto{\pgfqpoint{1.805641in}{0.841066in}}%
\pgfpathlineto{\pgfqpoint{1.806265in}{1.177239in}}%
\pgfpathlineto{\pgfqpoint{1.806682in}{0.778229in}}%
\pgfpathlineto{\pgfqpoint{1.806578in}{1.226700in}}%
\pgfpathlineto{\pgfqpoint{1.807411in}{1.000118in}}%
\pgfpathlineto{\pgfqpoint{1.807619in}{0.964365in}}%
\pgfpathlineto{\pgfqpoint{1.807932in}{1.178525in}}%
\pgfpathlineto{\pgfqpoint{1.808348in}{0.947339in}}%
\pgfpathlineto{\pgfqpoint{1.808765in}{1.131804in}}%
\pgfpathlineto{\pgfqpoint{1.808869in}{1.134441in}}%
\pgfpathlineto{\pgfqpoint{1.809390in}{1.413763in}}%
\pgfpathlineto{\pgfqpoint{1.809494in}{0.848039in}}%
\pgfpathlineto{\pgfqpoint{1.809702in}{1.117507in}}%
\pgfpathlineto{\pgfqpoint{1.810431in}{0.711102in}}%
\pgfpathlineto{\pgfqpoint{1.810535in}{1.308932in}}%
\pgfpathlineto{\pgfqpoint{1.810848in}{1.039278in}}%
\pgfpathlineto{\pgfqpoint{1.810952in}{1.039778in}}%
\pgfpathlineto{\pgfqpoint{1.811160in}{1.299607in}}%
\pgfpathlineto{\pgfqpoint{1.811681in}{0.986188in}}%
\pgfpathlineto{\pgfqpoint{1.811993in}{1.213149in}}%
\pgfpathlineto{\pgfqpoint{1.812410in}{0.852853in}}%
\pgfpathlineto{\pgfqpoint{1.812514in}{1.288396in}}%
\pgfpathlineto{\pgfqpoint{1.813139in}{1.086362in}}%
\pgfpathlineto{\pgfqpoint{1.813764in}{1.271935in}}%
\pgfpathlineto{\pgfqpoint{1.813555in}{1.004676in}}%
\pgfpathlineto{\pgfqpoint{1.814076in}{1.097349in}}%
\pgfpathlineto{\pgfqpoint{1.814180in}{0.833421in}}%
\pgfpathlineto{\pgfqpoint{1.814701in}{1.300365in}}%
\pgfpathlineto{\pgfqpoint{1.815117in}{0.935625in}}%
\pgfpathlineto{\pgfqpoint{1.815430in}{1.283017in}}%
\pgfpathlineto{\pgfqpoint{1.815846in}{0.757511in}}%
\pgfpathlineto{\pgfqpoint{1.816367in}{1.151351in}}%
\pgfpathlineto{\pgfqpoint{1.816992in}{0.814017in}}%
\pgfpathlineto{\pgfqpoint{1.816575in}{1.251710in}}%
\pgfpathlineto{\pgfqpoint{1.817617in}{1.033564in}}%
\pgfpathlineto{\pgfqpoint{1.817721in}{1.350499in}}%
\pgfpathlineto{\pgfqpoint{1.818242in}{0.987179in}}%
\pgfpathlineto{\pgfqpoint{1.818658in}{1.079091in}}%
\pgfpathlineto{\pgfqpoint{1.819075in}{1.193023in}}%
\pgfpathlineto{\pgfqpoint{1.818971in}{0.933559in}}%
\pgfpathlineto{\pgfqpoint{1.819491in}{1.186155in}}%
\pgfpathlineto{\pgfqpoint{1.820324in}{0.901009in}}%
\pgfpathlineto{\pgfqpoint{1.820116in}{1.341964in}}%
\pgfpathlineto{\pgfqpoint{1.820533in}{0.997924in}}%
\pgfpathlineto{\pgfqpoint{1.821470in}{1.289559in}}%
\pgfpathlineto{\pgfqpoint{1.821158in}{0.884385in}}%
\pgfpathlineto{\pgfqpoint{1.821782in}{1.211798in}}%
\pgfpathlineto{\pgfqpoint{1.822199in}{0.729977in}}%
\pgfpathlineto{\pgfqpoint{1.822407in}{1.227181in}}%
\pgfpathlineto{\pgfqpoint{1.823240in}{0.993211in}}%
\pgfpathlineto{\pgfqpoint{1.824282in}{1.424912in}}%
\pgfpathlineto{\pgfqpoint{1.823865in}{0.952787in}}%
\pgfpathlineto{\pgfqpoint{1.824386in}{1.108633in}}%
\pgfpathlineto{\pgfqpoint{1.824490in}{0.755184in}}%
\pgfpathlineto{\pgfqpoint{1.825219in}{1.110587in}}%
\pgfpathlineto{\pgfqpoint{1.825427in}{1.098663in}}%
\pgfpathlineto{\pgfqpoint{1.826156in}{0.844449in}}%
\pgfpathlineto{\pgfqpoint{1.825844in}{1.336463in}}%
\pgfpathlineto{\pgfqpoint{1.826365in}{1.120702in}}%
\pgfpathlineto{\pgfqpoint{1.827198in}{0.929580in}}%
\pgfpathlineto{\pgfqpoint{1.827510in}{1.300712in}}%
\pgfpathlineto{\pgfqpoint{1.828656in}{0.736789in}}%
\pgfpathlineto{\pgfqpoint{1.828760in}{1.234899in}}%
\pgfpathlineto{\pgfqpoint{1.829801in}{1.119661in}}%
\pgfpathlineto{\pgfqpoint{1.830114in}{0.854701in}}%
\pgfpathlineto{\pgfqpoint{1.830010in}{1.250241in}}%
\pgfpathlineto{\pgfqpoint{1.830947in}{1.031187in}}%
\pgfpathlineto{\pgfqpoint{1.831572in}{0.894590in}}%
\pgfpathlineto{\pgfqpoint{1.831155in}{1.392187in}}%
\pgfpathlineto{\pgfqpoint{1.831780in}{1.149494in}}%
\pgfpathlineto{\pgfqpoint{1.832092in}{1.486700in}}%
\pgfpathlineto{\pgfqpoint{1.832301in}{0.956791in}}%
\pgfpathlineto{\pgfqpoint{1.832613in}{1.190155in}}%
\pgfpathlineto{\pgfqpoint{1.833655in}{0.768190in}}%
\pgfpathlineto{\pgfqpoint{1.832821in}{1.401977in}}%
\pgfpathlineto{\pgfqpoint{1.833759in}{1.041051in}}%
\pgfpathlineto{\pgfqpoint{1.833863in}{1.217482in}}%
\pgfpathlineto{\pgfqpoint{1.834696in}{0.757372in}}%
\pgfpathlineto{\pgfqpoint{1.834800in}{1.079128in}}%
\pgfpathlineto{\pgfqpoint{1.835217in}{1.205003in}}%
\pgfpathlineto{\pgfqpoint{1.835633in}{0.877306in}}%
\pgfpathlineto{\pgfqpoint{1.835737in}{1.356194in}}%
\pgfpathlineto{\pgfqpoint{1.836466in}{0.845980in}}%
\pgfpathlineto{\pgfqpoint{1.836675in}{0.964213in}}%
\pgfpathlineto{\pgfqpoint{1.836883in}{0.862668in}}%
\pgfpathlineto{\pgfqpoint{1.836987in}{1.219595in}}%
\pgfpathlineto{\pgfqpoint{1.837091in}{1.289434in}}%
\pgfpathlineto{\pgfqpoint{1.837612in}{0.939829in}}%
\pgfpathlineto{\pgfqpoint{1.837820in}{1.087516in}}%
\pgfpathlineto{\pgfqpoint{1.838237in}{0.931008in}}%
\pgfpathlineto{\pgfqpoint{1.838028in}{1.244372in}}%
\pgfpathlineto{\pgfqpoint{1.838653in}{1.135188in}}%
\pgfpathlineto{\pgfqpoint{1.838757in}{1.332742in}}%
\pgfpathlineto{\pgfqpoint{1.839174in}{0.861790in}}%
\pgfpathlineto{\pgfqpoint{1.839695in}{1.225806in}}%
\pgfpathlineto{\pgfqpoint{1.840632in}{0.697290in}}%
\pgfpathlineto{\pgfqpoint{1.840944in}{0.924360in}}%
\pgfpathlineto{\pgfqpoint{1.841257in}{1.152665in}}%
\pgfpathlineto{\pgfqpoint{1.841361in}{0.708187in}}%
\pgfpathlineto{\pgfqpoint{1.842090in}{1.006055in}}%
\pgfpathlineto{\pgfqpoint{1.842507in}{1.179004in}}%
\pgfpathlineto{\pgfqpoint{1.843236in}{0.706263in}}%
\pgfpathlineto{\pgfqpoint{1.844069in}{1.280346in}}%
\pgfpathlineto{\pgfqpoint{1.844485in}{1.164304in}}%
\pgfpathlineto{\pgfqpoint{1.844589in}{0.806441in}}%
\pgfpathlineto{\pgfqpoint{1.845527in}{0.930124in}}%
\pgfpathlineto{\pgfqpoint{1.846256in}{1.402877in}}%
\pgfpathlineto{\pgfqpoint{1.845839in}{0.786066in}}%
\pgfpathlineto{\pgfqpoint{1.846672in}{1.042982in}}%
\pgfpathlineto{\pgfqpoint{1.846880in}{0.847570in}}%
\pgfpathlineto{\pgfqpoint{1.847193in}{1.234595in}}%
\pgfpathlineto{\pgfqpoint{1.847297in}{1.377585in}}%
\pgfpathlineto{\pgfqpoint{1.848026in}{0.859217in}}%
\pgfpathlineto{\pgfqpoint{1.848859in}{1.180568in}}%
\pgfpathlineto{\pgfqpoint{1.848651in}{0.720339in}}%
\pgfpathlineto{\pgfqpoint{1.849067in}{0.931348in}}%
\pgfpathlineto{\pgfqpoint{1.849172in}{0.780069in}}%
\pgfpathlineto{\pgfqpoint{1.849276in}{1.287658in}}%
\pgfpathlineto{\pgfqpoint{1.850005in}{1.079587in}}%
\pgfpathlineto{\pgfqpoint{1.850109in}{1.242747in}}%
\pgfpathlineto{\pgfqpoint{1.850421in}{0.827899in}}%
\pgfpathlineto{\pgfqpoint{1.851046in}{1.066660in}}%
\pgfpathlineto{\pgfqpoint{1.851567in}{0.817727in}}%
\pgfpathlineto{\pgfqpoint{1.851463in}{1.280039in}}%
\pgfpathlineto{\pgfqpoint{1.852087in}{1.022931in}}%
\pgfpathlineto{\pgfqpoint{1.852921in}{1.342314in}}%
\pgfpathlineto{\pgfqpoint{1.852712in}{0.842740in}}%
\pgfpathlineto{\pgfqpoint{1.853025in}{0.937859in}}%
\pgfpathlineto{\pgfqpoint{1.853129in}{0.878914in}}%
\pgfpathlineto{\pgfqpoint{1.853545in}{1.261099in}}%
\pgfpathlineto{\pgfqpoint{1.853962in}{0.980362in}}%
\pgfpathlineto{\pgfqpoint{1.854587in}{1.374940in}}%
\pgfpathlineto{\pgfqpoint{1.854274in}{0.790833in}}%
\pgfpathlineto{\pgfqpoint{1.855003in}{1.011779in}}%
\pgfpathlineto{\pgfqpoint{1.855108in}{1.011577in}}%
\pgfpathlineto{\pgfqpoint{1.855316in}{1.278306in}}%
\pgfpathlineto{\pgfqpoint{1.855628in}{0.900467in}}%
\pgfpathlineto{\pgfqpoint{1.856149in}{1.021746in}}%
\pgfpathlineto{\pgfqpoint{1.856461in}{0.793029in}}%
\pgfpathlineto{\pgfqpoint{1.856357in}{1.132589in}}%
\pgfpathlineto{\pgfqpoint{1.856774in}{0.919009in}}%
\pgfpathlineto{\pgfqpoint{1.857295in}{1.252934in}}%
\pgfpathlineto{\pgfqpoint{1.857815in}{0.894168in}}%
\pgfpathlineto{\pgfqpoint{1.857919in}{0.843492in}}%
\pgfpathlineto{\pgfqpoint{1.858232in}{1.189170in}}%
\pgfpathlineto{\pgfqpoint{1.858544in}{0.884152in}}%
\pgfpathlineto{\pgfqpoint{1.859169in}{1.226021in}}%
\pgfpathlineto{\pgfqpoint{1.859482in}{0.830123in}}%
\pgfpathlineto{\pgfqpoint{1.859586in}{1.135254in}}%
\pgfpathlineto{\pgfqpoint{1.859690in}{0.939562in}}%
\pgfpathlineto{\pgfqpoint{1.860419in}{1.363589in}}%
\pgfpathlineto{\pgfqpoint{1.860731in}{1.054640in}}%
\pgfpathlineto{\pgfqpoint{1.861356in}{0.896858in}}%
\pgfpathlineto{\pgfqpoint{1.861460in}{1.249962in}}%
\pgfpathlineto{\pgfqpoint{1.861877in}{0.972330in}}%
\pgfpathlineto{\pgfqpoint{1.862085in}{1.213964in}}%
\pgfpathlineto{\pgfqpoint{1.862293in}{0.828981in}}%
\pgfpathlineto{\pgfqpoint{1.862918in}{0.947768in}}%
\pgfpathlineto{\pgfqpoint{1.863751in}{1.224329in}}%
\pgfpathlineto{\pgfqpoint{1.863335in}{0.942829in}}%
\pgfpathlineto{\pgfqpoint{1.863855in}{1.128787in}}%
\pgfpathlineto{\pgfqpoint{1.863960in}{0.820543in}}%
\pgfpathlineto{\pgfqpoint{1.864168in}{1.143099in}}%
\pgfpathlineto{\pgfqpoint{1.865001in}{1.019941in}}%
\pgfpathlineto{\pgfqpoint{1.865730in}{0.808629in}}%
\pgfpathlineto{\pgfqpoint{1.865522in}{1.132493in}}%
\pgfpathlineto{\pgfqpoint{1.865938in}{0.897010in}}%
\pgfpathlineto{\pgfqpoint{1.866563in}{1.283147in}}%
\pgfpathlineto{\pgfqpoint{1.866251in}{0.810568in}}%
\pgfpathlineto{\pgfqpoint{1.867084in}{0.996705in}}%
\pgfpathlineto{\pgfqpoint{1.867292in}{1.291018in}}%
\pgfpathlineto{\pgfqpoint{1.867396in}{0.951481in}}%
\pgfpathlineto{\pgfqpoint{1.867709in}{1.093897in}}%
\pgfpathlineto{\pgfqpoint{1.867813in}{0.818229in}}%
\pgfpathlineto{\pgfqpoint{1.868333in}{1.490264in}}%
\pgfpathlineto{\pgfqpoint{1.868750in}{1.164848in}}%
\pgfpathlineto{\pgfqpoint{1.868854in}{1.169256in}}%
\pgfpathlineto{\pgfqpoint{1.869896in}{0.752959in}}%
\pgfpathlineto{\pgfqpoint{1.869062in}{1.238059in}}%
\pgfpathlineto{\pgfqpoint{1.870104in}{1.086269in}}%
\pgfpathlineto{\pgfqpoint{1.870729in}{1.300291in}}%
\pgfpathlineto{\pgfqpoint{1.870312in}{0.829686in}}%
\pgfpathlineto{\pgfqpoint{1.871145in}{0.983168in}}%
\pgfpathlineto{\pgfqpoint{1.871666in}{1.117178in}}%
\pgfpathlineto{\pgfqpoint{1.871562in}{0.711123in}}%
\pgfpathlineto{\pgfqpoint{1.871978in}{1.086443in}}%
\pgfpathlineto{\pgfqpoint{1.872083in}{0.843641in}}%
\pgfpathlineto{\pgfqpoint{1.872187in}{1.337497in}}%
\pgfpathlineto{\pgfqpoint{1.873124in}{1.001516in}}%
\pgfpathlineto{\pgfqpoint{1.873749in}{1.290950in}}%
\pgfpathlineto{\pgfqpoint{1.874061in}{0.790543in}}%
\pgfpathlineto{\pgfqpoint{1.874790in}{1.291313in}}%
\pgfpathlineto{\pgfqpoint{1.875207in}{1.144692in}}%
\pgfpathlineto{\pgfqpoint{1.876248in}{0.815921in}}%
\pgfpathlineto{\pgfqpoint{1.876352in}{1.037523in}}%
\pgfpathlineto{\pgfqpoint{1.877185in}{1.206715in}}%
\pgfpathlineto{\pgfqpoint{1.876769in}{0.821366in}}%
\pgfpathlineto{\pgfqpoint{1.877602in}{1.155142in}}%
\pgfpathlineto{\pgfqpoint{1.878019in}{0.692838in}}%
\pgfpathlineto{\pgfqpoint{1.877914in}{1.287218in}}%
\pgfpathlineto{\pgfqpoint{1.878643in}{1.002312in}}%
\pgfpathlineto{\pgfqpoint{1.879060in}{1.263187in}}%
\pgfpathlineto{\pgfqpoint{1.879268in}{0.850555in}}%
\pgfpathlineto{\pgfqpoint{1.879581in}{1.128412in}}%
\pgfpathlineto{\pgfqpoint{1.880206in}{0.858378in}}%
\pgfpathlineto{\pgfqpoint{1.880310in}{1.536466in}}%
\pgfpathlineto{\pgfqpoint{1.880726in}{1.035893in}}%
\pgfpathlineto{\pgfqpoint{1.881351in}{1.338793in}}%
\pgfpathlineto{\pgfqpoint{1.881247in}{0.841612in}}%
\pgfpathlineto{\pgfqpoint{1.881664in}{1.022910in}}%
\pgfpathlineto{\pgfqpoint{1.882080in}{1.200338in}}%
\pgfpathlineto{\pgfqpoint{1.882601in}{0.796027in}}%
\pgfpathlineto{\pgfqpoint{1.883434in}{1.388137in}}%
\pgfpathlineto{\pgfqpoint{1.883746in}{1.275075in}}%
\pgfpathlineto{\pgfqpoint{1.884267in}{1.343858in}}%
\pgfpathlineto{\pgfqpoint{1.884684in}{0.843718in}}%
\pgfpathlineto{\pgfqpoint{1.884788in}{1.376028in}}%
\pgfpathlineto{\pgfqpoint{1.885100in}{0.798536in}}%
\pgfpathlineto{\pgfqpoint{1.885829in}{1.254155in}}%
\pgfpathlineto{\pgfqpoint{1.885933in}{0.968003in}}%
\pgfpathlineto{\pgfqpoint{1.886975in}{1.157155in}}%
\pgfpathlineto{\pgfqpoint{1.887079in}{1.160358in}}%
\pgfpathlineto{\pgfqpoint{1.887600in}{0.645473in}}%
\pgfpathlineto{\pgfqpoint{1.887495in}{1.240504in}}%
\pgfpathlineto{\pgfqpoint{1.888224in}{1.000009in}}%
\pgfpathlineto{\pgfqpoint{1.888537in}{0.755411in}}%
\pgfpathlineto{\pgfqpoint{1.889266in}{1.421788in}}%
\pgfpathlineto{\pgfqpoint{1.889682in}{0.722754in}}%
\pgfpathlineto{\pgfqpoint{1.890411in}{0.796421in}}%
\pgfpathlineto{\pgfqpoint{1.890932in}{1.264105in}}%
\pgfpathlineto{\pgfqpoint{1.891661in}{1.134580in}}%
\pgfpathlineto{\pgfqpoint{1.891974in}{1.264409in}}%
\pgfpathlineto{\pgfqpoint{1.892703in}{0.838596in}}%
\pgfpathlineto{\pgfqpoint{1.893327in}{1.255699in}}%
\pgfpathlineto{\pgfqpoint{1.893848in}{1.112127in}}%
\pgfpathlineto{\pgfqpoint{1.893952in}{1.106789in}}%
\pgfpathlineto{\pgfqpoint{1.894889in}{0.823133in}}%
\pgfpathlineto{\pgfqpoint{1.894473in}{1.370973in}}%
\pgfpathlineto{\pgfqpoint{1.894994in}{0.889678in}}%
\pgfpathlineto{\pgfqpoint{1.895202in}{1.312066in}}%
\pgfpathlineto{\pgfqpoint{1.895618in}{0.788118in}}%
\pgfpathlineto{\pgfqpoint{1.896139in}{1.295010in}}%
\pgfpathlineto{\pgfqpoint{1.896452in}{0.840558in}}%
\pgfpathlineto{\pgfqpoint{1.897285in}{1.070876in}}%
\pgfpathlineto{\pgfqpoint{1.897389in}{1.088193in}}%
\pgfpathlineto{\pgfqpoint{1.897493in}{1.064795in}}%
\pgfpathlineto{\pgfqpoint{1.897805in}{0.728990in}}%
\pgfpathlineto{\pgfqpoint{1.898118in}{1.178052in}}%
\pgfpathlineto{\pgfqpoint{1.898534in}{1.079197in}}%
\pgfpathlineto{\pgfqpoint{1.898743in}{1.021458in}}%
\pgfpathlineto{\pgfqpoint{1.898847in}{1.108348in}}%
\pgfpathlineto{\pgfqpoint{1.899055in}{1.071906in}}%
\pgfpathlineto{\pgfqpoint{1.899472in}{0.858755in}}%
\pgfpathlineto{\pgfqpoint{1.899784in}{1.200147in}}%
\pgfpathlineto{\pgfqpoint{1.900097in}{0.870795in}}%
\pgfpathlineto{\pgfqpoint{1.900513in}{1.271546in}}%
\pgfpathlineto{\pgfqpoint{1.900826in}{0.794305in}}%
\pgfpathlineto{\pgfqpoint{1.901242in}{1.060377in}}%
\pgfpathlineto{\pgfqpoint{1.901346in}{1.078458in}}%
\pgfpathlineto{\pgfqpoint{1.901450in}{0.950380in}}%
\pgfpathlineto{\pgfqpoint{1.901659in}{1.471625in}}%
\pgfpathlineto{\pgfqpoint{1.902388in}{0.887400in}}%
\pgfpathlineto{\pgfqpoint{1.902492in}{1.113683in}}%
\pgfpathlineto{\pgfqpoint{1.902700in}{0.675530in}}%
\pgfpathlineto{\pgfqpoint{1.903012in}{1.251939in}}%
\pgfpathlineto{\pgfqpoint{1.903533in}{0.962369in}}%
\pgfpathlineto{\pgfqpoint{1.904366in}{1.248406in}}%
\pgfpathlineto{\pgfqpoint{1.904262in}{0.757779in}}%
\pgfpathlineto{\pgfqpoint{1.904575in}{1.111745in}}%
\pgfpathlineto{\pgfqpoint{1.905304in}{1.345443in}}%
\pgfpathlineto{\pgfqpoint{1.905616in}{0.829902in}}%
\pgfpathlineto{\pgfqpoint{1.905720in}{1.365174in}}%
\pgfpathlineto{\pgfqpoint{1.906657in}{1.207633in}}%
\pgfpathlineto{\pgfqpoint{1.906970in}{0.840221in}}%
\pgfpathlineto{\pgfqpoint{1.906866in}{1.288943in}}%
\pgfpathlineto{\pgfqpoint{1.907803in}{1.020634in}}%
\pgfpathlineto{\pgfqpoint{1.908011in}{1.222538in}}%
\pgfpathlineto{\pgfqpoint{1.908636in}{0.865986in}}%
\pgfpathlineto{\pgfqpoint{1.908740in}{1.043640in}}%
\pgfpathlineto{\pgfqpoint{1.909157in}{0.774756in}}%
\pgfpathlineto{\pgfqpoint{1.909053in}{1.173791in}}%
\pgfpathlineto{\pgfqpoint{1.909782in}{0.967560in}}%
\pgfpathlineto{\pgfqpoint{1.910511in}{1.280754in}}%
\pgfpathlineto{\pgfqpoint{1.909990in}{0.715686in}}%
\pgfpathlineto{\pgfqpoint{1.910823in}{1.131628in}}%
\pgfpathlineto{\pgfqpoint{1.910927in}{0.817583in}}%
\pgfpathlineto{\pgfqpoint{1.911760in}{1.212545in}}%
\pgfpathlineto{\pgfqpoint{1.911969in}{0.870317in}}%
\pgfpathlineto{\pgfqpoint{1.912385in}{0.798338in}}%
\pgfpathlineto{\pgfqpoint{1.913114in}{1.398411in}}%
\pgfpathlineto{\pgfqpoint{1.913635in}{0.967970in}}%
\pgfpathlineto{\pgfqpoint{1.914156in}{1.235716in}}%
\pgfpathlineto{\pgfqpoint{1.914260in}{1.358157in}}%
\pgfpathlineto{\pgfqpoint{1.914885in}{0.922538in}}%
\pgfpathlineto{\pgfqpoint{1.915093in}{1.309394in}}%
\pgfpathlineto{\pgfqpoint{1.915301in}{0.857216in}}%
\pgfpathlineto{\pgfqpoint{1.916134in}{1.089205in}}%
\pgfpathlineto{\pgfqpoint{1.917072in}{1.372917in}}%
\pgfpathlineto{\pgfqpoint{1.916655in}{0.894810in}}%
\pgfpathlineto{\pgfqpoint{1.917280in}{1.223076in}}%
\pgfpathlineto{\pgfqpoint{1.917800in}{0.908391in}}%
\pgfpathlineto{\pgfqpoint{1.917592in}{1.406399in}}%
\pgfpathlineto{\pgfqpoint{1.918529in}{0.960740in}}%
\pgfpathlineto{\pgfqpoint{1.918738in}{1.326588in}}%
\pgfpathlineto{\pgfqpoint{1.919363in}{0.944353in}}%
\pgfpathlineto{\pgfqpoint{1.919675in}{1.231614in}}%
\pgfpathlineto{\pgfqpoint{1.920404in}{0.887239in}}%
\pgfpathlineto{\pgfqpoint{1.919987in}{1.339451in}}%
\pgfpathlineto{\pgfqpoint{1.920821in}{0.954422in}}%
\pgfpathlineto{\pgfqpoint{1.921654in}{0.741563in}}%
\pgfpathlineto{\pgfqpoint{1.921862in}{1.174380in}}%
\pgfpathlineto{\pgfqpoint{1.922591in}{1.447693in}}%
\pgfpathlineto{\pgfqpoint{1.923008in}{0.811194in}}%
\pgfpathlineto{\pgfqpoint{1.923841in}{1.208829in}}%
\pgfpathlineto{\pgfqpoint{1.923737in}{0.691336in}}%
\pgfpathlineto{\pgfqpoint{1.924153in}{1.192373in}}%
\pgfpathlineto{\pgfqpoint{1.924986in}{0.928964in}}%
\pgfpathlineto{\pgfqpoint{1.924361in}{1.232773in}}%
\pgfpathlineto{\pgfqpoint{1.925299in}{1.059100in}}%
\pgfpathlineto{\pgfqpoint{1.925403in}{1.214153in}}%
\pgfpathlineto{\pgfqpoint{1.925715in}{0.966108in}}%
\pgfpathlineto{\pgfqpoint{1.926340in}{1.170501in}}%
\pgfpathlineto{\pgfqpoint{1.926757in}{0.804741in}}%
\pgfpathlineto{\pgfqpoint{1.927069in}{1.331112in}}%
\pgfpathlineto{\pgfqpoint{1.927486in}{1.000332in}}%
\pgfpathlineto{\pgfqpoint{1.928319in}{1.500717in}}%
\pgfpathlineto{\pgfqpoint{1.928110in}{0.803435in}}%
\pgfpathlineto{\pgfqpoint{1.928527in}{1.222733in}}%
\pgfpathlineto{\pgfqpoint{1.929568in}{0.889401in}}%
\pgfpathlineto{\pgfqpoint{1.929152in}{1.346252in}}%
\pgfpathlineto{\pgfqpoint{1.929673in}{1.012188in}}%
\pgfpathlineto{\pgfqpoint{1.929985in}{1.412637in}}%
\pgfpathlineto{\pgfqpoint{1.930506in}{0.927108in}}%
\pgfpathlineto{\pgfqpoint{1.930714in}{0.978103in}}%
\pgfpathlineto{\pgfqpoint{1.930818in}{0.791922in}}%
\pgfpathlineto{\pgfqpoint{1.931651in}{1.268682in}}%
\pgfpathlineto{\pgfqpoint{1.931755in}{0.964476in}}%
\pgfpathlineto{\pgfqpoint{1.932172in}{1.508926in}}%
\pgfpathlineto{\pgfqpoint{1.932068in}{0.860142in}}%
\pgfpathlineto{\pgfqpoint{1.932797in}{1.073281in}}%
\pgfpathlineto{\pgfqpoint{1.932901in}{0.833276in}}%
\pgfpathlineto{\pgfqpoint{1.933734in}{1.265070in}}%
\pgfpathlineto{\pgfqpoint{1.933838in}{1.186084in}}%
\pgfpathlineto{\pgfqpoint{1.934671in}{0.818494in}}%
\pgfpathlineto{\pgfqpoint{1.934151in}{1.557730in}}%
\pgfpathlineto{\pgfqpoint{1.934880in}{1.110025in}}%
\pgfpathlineto{\pgfqpoint{1.934984in}{1.279614in}}%
\pgfpathlineto{\pgfqpoint{1.935296in}{0.736566in}}%
\pgfpathlineto{\pgfqpoint{1.935817in}{0.940096in}}%
\pgfpathlineto{\pgfqpoint{1.936025in}{0.758252in}}%
\pgfpathlineto{\pgfqpoint{1.936129in}{1.053742in}}%
\pgfpathlineto{\pgfqpoint{1.936233in}{0.937005in}}%
\pgfpathlineto{\pgfqpoint{1.936338in}{1.439715in}}%
\pgfpathlineto{\pgfqpoint{1.936650in}{0.823737in}}%
\pgfpathlineto{\pgfqpoint{1.937379in}{1.326608in}}%
\pgfpathlineto{\pgfqpoint{1.938004in}{1.007067in}}%
\pgfpathlineto{\pgfqpoint{1.938629in}{1.025594in}}%
\pgfpathlineto{\pgfqpoint{1.939045in}{1.251154in}}%
\pgfpathlineto{\pgfqpoint{1.938837in}{0.984630in}}%
\pgfpathlineto{\pgfqpoint{1.939254in}{1.074038in}}%
\pgfpathlineto{\pgfqpoint{1.939358in}{0.841636in}}%
\pgfpathlineto{\pgfqpoint{1.940295in}{0.995618in}}%
\pgfpathlineto{\pgfqpoint{1.940920in}{1.304936in}}%
\pgfpathlineto{\pgfqpoint{1.940712in}{0.956961in}}%
\pgfpathlineto{\pgfqpoint{1.941649in}{1.211108in}}%
\pgfpathlineto{\pgfqpoint{1.942378in}{0.780596in}}%
\pgfpathlineto{\pgfqpoint{1.942690in}{0.813868in}}%
\pgfpathlineto{\pgfqpoint{1.943627in}{1.396793in}}%
\pgfpathlineto{\pgfqpoint{1.943836in}{1.033428in}}%
\pgfpathlineto{\pgfqpoint{1.944565in}{1.276548in}}%
\pgfpathlineto{\pgfqpoint{1.944669in}{0.707813in}}%
\pgfpathlineto{\pgfqpoint{1.945398in}{1.404365in}}%
\pgfpathlineto{\pgfqpoint{1.945814in}{0.966026in}}%
\pgfpathlineto{\pgfqpoint{1.946127in}{0.752705in}}%
\pgfpathlineto{\pgfqpoint{1.946439in}{1.028418in}}%
\pgfpathlineto{\pgfqpoint{1.946648in}{1.107399in}}%
\pgfpathlineto{\pgfqpoint{1.946856in}{0.968781in}}%
\pgfpathlineto{\pgfqpoint{1.946960in}{0.765837in}}%
\pgfpathlineto{\pgfqpoint{1.947481in}{1.192006in}}%
\pgfpathlineto{\pgfqpoint{1.948001in}{0.832784in}}%
\pgfpathlineto{\pgfqpoint{1.948730in}{1.210936in}}%
\pgfpathlineto{\pgfqpoint{1.948939in}{0.723370in}}%
\pgfpathlineto{\pgfqpoint{1.949147in}{1.031232in}}%
\pgfpathlineto{\pgfqpoint{1.949355in}{0.962023in}}%
\pgfpathlineto{\pgfqpoint{1.949459in}{1.086103in}}%
\pgfpathlineto{\pgfqpoint{1.949668in}{1.050524in}}%
\pgfpathlineto{\pgfqpoint{1.950084in}{1.256851in}}%
\pgfpathlineto{\pgfqpoint{1.949980in}{0.982712in}}%
\pgfpathlineto{\pgfqpoint{1.950605in}{1.190511in}}%
\pgfpathlineto{\pgfqpoint{1.950709in}{0.822232in}}%
\pgfpathlineto{\pgfqpoint{1.951021in}{1.300550in}}%
\pgfpathlineto{\pgfqpoint{1.951646in}{0.904546in}}%
\pgfpathlineto{\pgfqpoint{1.952271in}{1.226557in}}%
\pgfpathlineto{\pgfqpoint{1.952375in}{0.899743in}}%
\pgfpathlineto{\pgfqpoint{1.952688in}{0.940813in}}%
\pgfpathlineto{\pgfqpoint{1.953000in}{0.833141in}}%
\pgfpathlineto{\pgfqpoint{1.952896in}{1.069602in}}%
\pgfpathlineto{\pgfqpoint{1.953104in}{0.906144in}}%
\pgfpathlineto{\pgfqpoint{1.953208in}{1.355829in}}%
\pgfpathlineto{\pgfqpoint{1.954042in}{0.754810in}}%
\pgfpathlineto{\pgfqpoint{1.954250in}{1.078236in}}%
\pgfpathlineto{\pgfqpoint{1.954875in}{1.204666in}}%
\pgfpathlineto{\pgfqpoint{1.954458in}{0.920269in}}%
\pgfpathlineto{\pgfqpoint{1.954979in}{1.187654in}}%
\pgfpathlineto{\pgfqpoint{1.956020in}{1.225595in}}%
\pgfpathlineto{\pgfqpoint{1.956124in}{0.640009in}}%
\pgfpathlineto{\pgfqpoint{1.956958in}{1.522141in}}%
\pgfpathlineto{\pgfqpoint{1.957270in}{0.920119in}}%
\pgfpathlineto{\pgfqpoint{1.957582in}{1.174903in}}%
\pgfpathlineto{\pgfqpoint{1.957791in}{0.889026in}}%
\pgfpathlineto{\pgfqpoint{1.958416in}{1.164382in}}%
\pgfpathlineto{\pgfqpoint{1.958936in}{0.891573in}}%
\pgfpathlineto{\pgfqpoint{1.959144in}{1.247660in}}%
\pgfpathlineto{\pgfqpoint{1.959561in}{1.120623in}}%
\pgfpathlineto{\pgfqpoint{1.959769in}{0.756942in}}%
\pgfpathlineto{\pgfqpoint{1.959978in}{1.201737in}}%
\pgfpathlineto{\pgfqpoint{1.960082in}{0.833110in}}%
\pgfpathlineto{\pgfqpoint{1.960707in}{1.364089in}}%
\pgfpathlineto{\pgfqpoint{1.961123in}{0.957095in}}%
\pgfpathlineto{\pgfqpoint{1.961227in}{1.171545in}}%
\pgfpathlineto{\pgfqpoint{1.961748in}{0.932898in}}%
\pgfpathlineto{\pgfqpoint{1.962269in}{1.093130in}}%
\pgfpathlineto{\pgfqpoint{1.962373in}{0.844264in}}%
\pgfpathlineto{\pgfqpoint{1.963310in}{1.252573in}}%
\pgfpathlineto{\pgfqpoint{1.964039in}{0.799120in}}%
\pgfpathlineto{\pgfqpoint{1.963727in}{1.299098in}}%
\pgfpathlineto{\pgfqpoint{1.964560in}{0.981007in}}%
\pgfpathlineto{\pgfqpoint{1.964664in}{1.449262in}}%
\pgfpathlineto{\pgfqpoint{1.965497in}{0.851456in}}%
\pgfpathlineto{\pgfqpoint{1.965705in}{1.315781in}}%
\pgfpathlineto{\pgfqpoint{1.966955in}{0.601474in}}%
\pgfpathlineto{\pgfqpoint{1.967788in}{1.363430in}}%
\pgfpathlineto{\pgfqpoint{1.968101in}{1.079077in}}%
\pgfpathlineto{\pgfqpoint{1.968309in}{1.165957in}}%
\pgfpathlineto{\pgfqpoint{1.969246in}{0.777218in}}%
\pgfpathlineto{\pgfqpoint{1.969663in}{1.142223in}}%
\pgfpathlineto{\pgfqpoint{1.969975in}{0.747670in}}%
\pgfpathlineto{\pgfqpoint{1.970392in}{0.914493in}}%
\pgfpathlineto{\pgfqpoint{1.970704in}{1.299721in}}%
\pgfpathlineto{\pgfqpoint{1.970912in}{0.784007in}}%
\pgfpathlineto{\pgfqpoint{1.971225in}{0.917616in}}%
\pgfpathlineto{\pgfqpoint{1.971329in}{0.618245in}}%
\pgfpathlineto{\pgfqpoint{1.971433in}{1.252361in}}%
\pgfpathlineto{\pgfqpoint{1.972162in}{0.918976in}}%
\pgfpathlineto{\pgfqpoint{1.972266in}{1.276848in}}%
\pgfpathlineto{\pgfqpoint{1.972370in}{0.823570in}}%
\pgfpathlineto{\pgfqpoint{1.973308in}{1.062340in}}%
\pgfpathlineto{\pgfqpoint{1.973412in}{1.320979in}}%
\pgfpathlineto{\pgfqpoint{1.974141in}{0.855166in}}%
\pgfpathlineto{\pgfqpoint{1.974349in}{0.934103in}}%
\pgfpathlineto{\pgfqpoint{1.974453in}{0.899233in}}%
\pgfpathlineto{\pgfqpoint{1.974557in}{0.948392in}}%
\pgfpathlineto{\pgfqpoint{1.974662in}{1.419156in}}%
\pgfpathlineto{\pgfqpoint{1.975078in}{0.783402in}}%
\pgfpathlineto{\pgfqpoint{1.975599in}{1.162634in}}%
\pgfpathlineto{\pgfqpoint{1.976015in}{1.260685in}}%
\pgfpathlineto{\pgfqpoint{1.976744in}{0.726923in}}%
\pgfpathlineto{\pgfqpoint{1.977161in}{1.300591in}}%
\pgfpathlineto{\pgfqpoint{1.977890in}{0.984577in}}%
\pgfpathlineto{\pgfqpoint{1.979035in}{1.355934in}}%
\pgfpathlineto{\pgfqpoint{1.978098in}{0.753946in}}%
\pgfpathlineto{\pgfqpoint{1.979140in}{1.222911in}}%
\pgfpathlineto{\pgfqpoint{1.979764in}{0.732018in}}%
\pgfpathlineto{\pgfqpoint{1.980077in}{1.312107in}}%
\pgfpathlineto{\pgfqpoint{1.980285in}{1.058993in}}%
\pgfpathlineto{\pgfqpoint{1.980702in}{1.417191in}}%
\pgfpathlineto{\pgfqpoint{1.981118in}{1.002214in}}%
\pgfpathlineto{\pgfqpoint{1.981327in}{1.261156in}}%
\pgfpathlineto{\pgfqpoint{1.981535in}{0.805245in}}%
\pgfpathlineto{\pgfqpoint{1.982056in}{1.270256in}}%
\pgfpathlineto{\pgfqpoint{1.982472in}{1.172241in}}%
\pgfpathlineto{\pgfqpoint{1.982576in}{1.343251in}}%
\pgfpathlineto{\pgfqpoint{1.983409in}{0.912929in}}%
\pgfpathlineto{\pgfqpoint{1.983513in}{1.117986in}}%
\pgfpathlineto{\pgfqpoint{1.983722in}{0.835512in}}%
\pgfpathlineto{\pgfqpoint{1.984138in}{1.223381in}}%
\pgfpathlineto{\pgfqpoint{1.984659in}{0.962718in}}%
\pgfpathlineto{\pgfqpoint{1.985180in}{0.728149in}}%
\pgfpathlineto{\pgfqpoint{1.985284in}{1.193178in}}%
\pgfpathlineto{\pgfqpoint{1.985596in}{0.829834in}}%
\pgfpathlineto{\pgfqpoint{1.986013in}{1.393710in}}%
\pgfpathlineto{\pgfqpoint{1.986742in}{1.094107in}}%
\pgfpathlineto{\pgfqpoint{1.987679in}{0.779698in}}%
\pgfpathlineto{\pgfqpoint{1.986950in}{1.276113in}}%
\pgfpathlineto{\pgfqpoint{1.988096in}{0.871712in}}%
\pgfpathlineto{\pgfqpoint{1.988616in}{1.380240in}}%
\pgfpathlineto{\pgfqpoint{1.988408in}{0.705008in}}%
\pgfpathlineto{\pgfqpoint{1.989241in}{1.048735in}}%
\pgfpathlineto{\pgfqpoint{1.989554in}{0.835771in}}%
\pgfpathlineto{\pgfqpoint{1.989866in}{1.333321in}}%
\pgfpathlineto{\pgfqpoint{1.990387in}{0.886344in}}%
\pgfpathlineto{\pgfqpoint{1.991220in}{1.136911in}}%
\pgfpathlineto{\pgfqpoint{1.991012in}{0.763492in}}%
\pgfpathlineto{\pgfqpoint{1.991636in}{0.974755in}}%
\pgfpathlineto{\pgfqpoint{1.991741in}{0.960662in}}%
\pgfpathlineto{\pgfqpoint{1.991845in}{1.079698in}}%
\pgfpathlineto{\pgfqpoint{1.991949in}{1.393071in}}%
\pgfpathlineto{\pgfqpoint{1.992574in}{0.705989in}}%
\pgfpathlineto{\pgfqpoint{1.992886in}{1.010981in}}%
\pgfpathlineto{\pgfqpoint{1.992990in}{0.933624in}}%
\pgfpathlineto{\pgfqpoint{1.993199in}{1.187273in}}%
\pgfpathlineto{\pgfqpoint{1.993719in}{1.163606in}}%
\pgfpathlineto{\pgfqpoint{1.994240in}{1.412764in}}%
\pgfpathlineto{\pgfqpoint{1.993928in}{0.671523in}}%
\pgfpathlineto{\pgfqpoint{1.994552in}{1.048984in}}%
\pgfpathlineto{\pgfqpoint{1.995073in}{1.339599in}}%
\pgfpathlineto{\pgfqpoint{1.995594in}{0.901162in}}%
\pgfpathlineto{\pgfqpoint{1.996531in}{1.355449in}}%
\pgfpathlineto{\pgfqpoint{1.996739in}{1.223521in}}%
\pgfpathlineto{\pgfqpoint{1.997052in}{0.860572in}}%
\pgfpathlineto{\pgfqpoint{1.997677in}{1.294261in}}%
\pgfpathlineto{\pgfqpoint{1.997885in}{1.043797in}}%
\pgfpathlineto{\pgfqpoint{1.997989in}{1.032328in}}%
\pgfpathlineto{\pgfqpoint{1.998093in}{1.124751in}}%
\pgfpathlineto{\pgfqpoint{1.998614in}{1.315873in}}%
\pgfpathlineto{\pgfqpoint{1.998718in}{0.999997in}}%
\pgfpathlineto{\pgfqpoint{1.999135in}{1.221155in}}%
\pgfpathlineto{\pgfqpoint{1.999864in}{0.908906in}}%
\pgfpathlineto{\pgfqpoint{1.999447in}{1.272170in}}%
\pgfpathlineto{\pgfqpoint{2.000176in}{1.061026in}}%
\pgfpathlineto{\pgfqpoint{2.000384in}{1.280544in}}%
\pgfpathlineto{\pgfqpoint{2.000488in}{0.928448in}}%
\pgfpathlineto{\pgfqpoint{2.001217in}{1.153118in}}%
\pgfpathlineto{\pgfqpoint{2.001946in}{0.714364in}}%
\pgfpathlineto{\pgfqpoint{2.001530in}{1.268458in}}%
\pgfpathlineto{\pgfqpoint{2.002363in}{0.884604in}}%
\pgfpathlineto{\pgfqpoint{2.002675in}{1.343849in}}%
\pgfpathlineto{\pgfqpoint{2.002884in}{0.856744in}}%
\pgfpathlineto{\pgfqpoint{2.004029in}{1.226907in}}%
\pgfpathlineto{\pgfqpoint{2.004967in}{0.746588in}}%
\pgfpathlineto{\pgfqpoint{2.005175in}{0.990875in}}%
\pgfpathlineto{\pgfqpoint{2.005383in}{1.248396in}}%
\pgfpathlineto{\pgfqpoint{2.005487in}{0.927956in}}%
\pgfpathlineto{\pgfqpoint{2.006425in}{1.350704in}}%
\pgfpathlineto{\pgfqpoint{2.006112in}{0.626552in}}%
\pgfpathlineto{\pgfqpoint{2.006633in}{1.101698in}}%
\pgfpathlineto{\pgfqpoint{2.007258in}{0.746037in}}%
\pgfpathlineto{\pgfqpoint{2.006945in}{1.369654in}}%
\pgfpathlineto{\pgfqpoint{2.007883in}{1.010777in}}%
\pgfpathlineto{\pgfqpoint{2.008403in}{1.231621in}}%
\pgfpathlineto{\pgfqpoint{2.008195in}{0.901832in}}%
\pgfpathlineto{\pgfqpoint{2.008716in}{0.922164in}}%
\pgfpathlineto{\pgfqpoint{2.008820in}{0.620381in}}%
\pgfpathlineto{\pgfqpoint{2.009236in}{1.536702in}}%
\pgfpathlineto{\pgfqpoint{2.009757in}{1.105264in}}%
\pgfpathlineto{\pgfqpoint{2.009965in}{1.322854in}}%
\pgfpathlineto{\pgfqpoint{2.010278in}{0.844272in}}%
\pgfpathlineto{\pgfqpoint{2.010798in}{1.107058in}}%
\pgfpathlineto{\pgfqpoint{2.010903in}{1.115477in}}%
\pgfpathlineto{\pgfqpoint{2.011007in}{1.066087in}}%
\pgfpathlineto{\pgfqpoint{2.011111in}{1.080416in}}%
\pgfpathlineto{\pgfqpoint{2.011632in}{0.905758in}}%
\pgfpathlineto{\pgfqpoint{2.011319in}{1.200433in}}%
\pgfpathlineto{\pgfqpoint{2.012256in}{1.005589in}}%
\pgfpathlineto{\pgfqpoint{2.012777in}{1.247294in}}%
\pgfpathlineto{\pgfqpoint{2.013298in}{0.958043in}}%
\pgfpathlineto{\pgfqpoint{2.013402in}{1.070374in}}%
\pgfpathlineto{\pgfqpoint{2.013506in}{1.216862in}}%
\pgfpathlineto{\pgfqpoint{2.014131in}{0.847710in}}%
\pgfpathlineto{\pgfqpoint{2.014443in}{1.115360in}}%
\pgfpathlineto{\pgfqpoint{2.015589in}{0.858538in}}%
\pgfpathlineto{\pgfqpoint{2.015277in}{1.418471in}}%
\pgfpathlineto{\pgfqpoint{2.015693in}{0.935280in}}%
\pgfpathlineto{\pgfqpoint{2.016006in}{1.336030in}}%
\pgfpathlineto{\pgfqpoint{2.016110in}{0.871574in}}%
\pgfpathlineto{\pgfqpoint{2.016734in}{0.993032in}}%
\pgfpathlineto{\pgfqpoint{2.017047in}{0.919847in}}%
\pgfpathlineto{\pgfqpoint{2.017255in}{1.205122in}}%
\pgfpathlineto{\pgfqpoint{2.017359in}{0.785692in}}%
\pgfpathlineto{\pgfqpoint{2.017463in}{1.291691in}}%
\pgfpathlineto{\pgfqpoint{2.018401in}{0.977536in}}%
\pgfpathlineto{\pgfqpoint{2.018505in}{0.920359in}}%
\pgfpathlineto{\pgfqpoint{2.018609in}{1.234017in}}%
\pgfpathlineto{\pgfqpoint{2.019130in}{0.967971in}}%
\pgfpathlineto{\pgfqpoint{2.019234in}{1.244177in}}%
\pgfpathlineto{\pgfqpoint{2.019963in}{0.913122in}}%
\pgfpathlineto{\pgfqpoint{2.020171in}{0.930115in}}%
\pgfpathlineto{\pgfqpoint{2.020275in}{0.935120in}}%
\pgfpathlineto{\pgfqpoint{2.020379in}{0.916264in}}%
\pgfpathlineto{\pgfqpoint{2.020900in}{1.369959in}}%
\pgfpathlineto{\pgfqpoint{2.021004in}{0.906622in}}%
\pgfpathlineto{\pgfqpoint{2.021629in}{1.139343in}}%
\pgfpathlineto{\pgfqpoint{2.021733in}{1.152545in}}%
\pgfpathlineto{\pgfqpoint{2.021942in}{0.908832in}}%
\pgfpathlineto{\pgfqpoint{2.022566in}{1.386444in}}%
\pgfpathlineto{\pgfqpoint{2.022775in}{0.990506in}}%
\pgfpathlineto{\pgfqpoint{2.023295in}{1.323230in}}%
\pgfpathlineto{\pgfqpoint{2.023191in}{0.948647in}}%
\pgfpathlineto{\pgfqpoint{2.023816in}{1.255768in}}%
\pgfpathlineto{\pgfqpoint{2.024441in}{0.835173in}}%
\pgfpathlineto{\pgfqpoint{2.024857in}{1.155028in}}%
\pgfpathlineto{\pgfqpoint{2.024962in}{1.189192in}}%
\pgfpathlineto{\pgfqpoint{2.025066in}{1.042659in}}%
\pgfpathlineto{\pgfqpoint{2.025170in}{1.088027in}}%
\pgfpathlineto{\pgfqpoint{2.026211in}{0.727276in}}%
\pgfpathlineto{\pgfqpoint{2.025795in}{1.388665in}}%
\pgfpathlineto{\pgfqpoint{2.026315in}{0.938513in}}%
\pgfpathlineto{\pgfqpoint{2.027044in}{1.289420in}}%
\pgfpathlineto{\pgfqpoint{2.026732in}{0.679681in}}%
\pgfpathlineto{\pgfqpoint{2.027253in}{0.926381in}}%
\pgfpathlineto{\pgfqpoint{2.027357in}{0.660596in}}%
\pgfpathlineto{\pgfqpoint{2.027773in}{1.368950in}}%
\pgfpathlineto{\pgfqpoint{2.028294in}{0.763493in}}%
\pgfpathlineto{\pgfqpoint{2.028607in}{1.253792in}}%
\pgfpathlineto{\pgfqpoint{2.029231in}{0.702891in}}%
\pgfpathlineto{\pgfqpoint{2.029440in}{1.031775in}}%
\pgfpathlineto{\pgfqpoint{2.029960in}{0.770523in}}%
\pgfpathlineto{\pgfqpoint{2.029648in}{1.166361in}}%
\pgfpathlineto{\pgfqpoint{2.030065in}{0.888374in}}%
\pgfpathlineto{\pgfqpoint{2.030169in}{1.405601in}}%
\pgfpathlineto{\pgfqpoint{2.031210in}{1.163181in}}%
\pgfpathlineto{\pgfqpoint{2.031835in}{1.294289in}}%
\pgfpathlineto{\pgfqpoint{2.031939in}{0.901032in}}%
\pgfpathlineto{\pgfqpoint{2.032252in}{1.130537in}}%
\pgfpathlineto{\pgfqpoint{2.033293in}{0.778514in}}%
\pgfpathlineto{\pgfqpoint{2.032460in}{1.360553in}}%
\pgfpathlineto{\pgfqpoint{2.033397in}{0.818251in}}%
\pgfpathlineto{\pgfqpoint{2.033709in}{1.288000in}}%
\pgfpathlineto{\pgfqpoint{2.034438in}{0.767052in}}%
\pgfpathlineto{\pgfqpoint{2.034647in}{1.262747in}}%
\pgfpathlineto{\pgfqpoint{2.035688in}{0.848667in}}%
\pgfpathlineto{\pgfqpoint{2.035896in}{0.859093in}}%
\pgfpathlineto{\pgfqpoint{2.036834in}{1.305261in}}%
\pgfpathlineto{\pgfqpoint{2.037146in}{1.159263in}}%
\pgfpathlineto{\pgfqpoint{2.037667in}{1.264615in}}%
\pgfpathlineto{\pgfqpoint{2.038188in}{0.886858in}}%
\pgfpathlineto{\pgfqpoint{2.038396in}{1.319161in}}%
\pgfpathlineto{\pgfqpoint{2.039021in}{0.870595in}}%
\pgfpathlineto{\pgfqpoint{2.039333in}{1.160762in}}%
\pgfpathlineto{\pgfqpoint{2.040062in}{0.797763in}}%
\pgfpathlineto{\pgfqpoint{2.039646in}{1.200147in}}%
\pgfpathlineto{\pgfqpoint{2.040375in}{1.104070in}}%
\pgfpathlineto{\pgfqpoint{2.040895in}{1.398235in}}%
\pgfpathlineto{\pgfqpoint{2.041103in}{0.900824in}}%
\pgfpathlineto{\pgfqpoint{2.041520in}{1.177689in}}%
\pgfpathlineto{\pgfqpoint{2.041624in}{1.234618in}}%
\pgfpathlineto{\pgfqpoint{2.041937in}{0.951595in}}%
\pgfpathlineto{\pgfqpoint{2.042353in}{1.103213in}}%
\pgfpathlineto{\pgfqpoint{2.043082in}{1.193364in}}%
\pgfpathlineto{\pgfqpoint{2.043290in}{0.746772in}}%
\pgfpathlineto{\pgfqpoint{2.043395in}{1.275643in}}%
\pgfpathlineto{\pgfqpoint{2.044436in}{0.998660in}}%
\pgfpathlineto{\pgfqpoint{2.044540in}{0.758221in}}%
\pgfpathlineto{\pgfqpoint{2.044957in}{1.241925in}}%
\pgfpathlineto{\pgfqpoint{2.045477in}{1.064079in}}%
\pgfpathlineto{\pgfqpoint{2.045998in}{1.408337in}}%
\pgfpathlineto{\pgfqpoint{2.045790in}{0.925325in}}%
\pgfpathlineto{\pgfqpoint{2.046311in}{1.270851in}}%
\pgfpathlineto{\pgfqpoint{2.046415in}{0.841704in}}%
\pgfpathlineto{\pgfqpoint{2.047456in}{1.020269in}}%
\pgfpathlineto{\pgfqpoint{2.048393in}{1.313123in}}%
\pgfpathlineto{\pgfqpoint{2.048185in}{0.889780in}}%
\pgfpathlineto{\pgfqpoint{2.048498in}{1.016382in}}%
\pgfpathlineto{\pgfqpoint{2.048602in}{0.640750in}}%
\pgfpathlineto{\pgfqpoint{2.049018in}{1.250068in}}%
\pgfpathlineto{\pgfqpoint{2.049539in}{0.896959in}}%
\pgfpathlineto{\pgfqpoint{2.049643in}{0.916847in}}%
\pgfpathlineto{\pgfqpoint{2.050268in}{1.269199in}}%
\pgfpathlineto{\pgfqpoint{2.050684in}{0.875601in}}%
\pgfpathlineto{\pgfqpoint{2.050789in}{1.018070in}}%
\pgfpathlineto{\pgfqpoint{2.050997in}{0.786081in}}%
\pgfpathlineto{\pgfqpoint{2.051309in}{1.229803in}}%
\pgfpathlineto{\pgfqpoint{2.051726in}{0.963807in}}%
\pgfpathlineto{\pgfqpoint{2.051830in}{1.327009in}}%
\pgfpathlineto{\pgfqpoint{2.052663in}{0.842927in}}%
\pgfpathlineto{\pgfqpoint{2.052871in}{1.229831in}}%
\pgfpathlineto{\pgfqpoint{2.053288in}{0.889375in}}%
\pgfpathlineto{\pgfqpoint{2.053705in}{1.346913in}}%
\pgfpathlineto{\pgfqpoint{2.054017in}{0.968941in}}%
\pgfpathlineto{\pgfqpoint{2.054121in}{1.340149in}}%
\pgfpathlineto{\pgfqpoint{2.055058in}{0.929582in}}%
\pgfpathlineto{\pgfqpoint{2.055475in}{0.891532in}}%
\pgfpathlineto{\pgfqpoint{2.056100in}{1.267230in}}%
\pgfpathlineto{\pgfqpoint{2.056621in}{0.863710in}}%
\pgfpathlineto{\pgfqpoint{2.057245in}{0.866103in}}%
\pgfpathlineto{\pgfqpoint{2.058287in}{1.221589in}}%
\pgfpathlineto{\pgfqpoint{2.058183in}{0.798380in}}%
\pgfpathlineto{\pgfqpoint{2.058391in}{1.196715in}}%
\pgfpathlineto{\pgfqpoint{2.058495in}{0.867184in}}%
\pgfpathlineto{\pgfqpoint{2.059120in}{1.348317in}}%
\pgfpathlineto{\pgfqpoint{2.059432in}{0.942180in}}%
\pgfpathlineto{\pgfqpoint{2.060161in}{1.367291in}}%
\pgfpathlineto{\pgfqpoint{2.059641in}{0.811548in}}%
\pgfpathlineto{\pgfqpoint{2.060474in}{1.012745in}}%
\pgfpathlineto{\pgfqpoint{2.060578in}{0.887424in}}%
\pgfpathlineto{\pgfqpoint{2.060682in}{1.238614in}}%
\pgfpathlineto{\pgfqpoint{2.061515in}{1.078395in}}%
\pgfpathlineto{\pgfqpoint{2.061619in}{1.220427in}}%
\pgfpathlineto{\pgfqpoint{2.062244in}{0.731607in}}%
\pgfpathlineto{\pgfqpoint{2.062557in}{1.119724in}}%
\pgfpathlineto{\pgfqpoint{2.063598in}{0.784604in}}%
\pgfpathlineto{\pgfqpoint{2.063181in}{1.321812in}}%
\pgfpathlineto{\pgfqpoint{2.063702in}{1.027853in}}%
\pgfpathlineto{\pgfqpoint{2.063806in}{1.195079in}}%
\pgfpathlineto{\pgfqpoint{2.064015in}{0.837589in}}%
\pgfpathlineto{\pgfqpoint{2.064744in}{1.165289in}}%
\pgfpathlineto{\pgfqpoint{2.065577in}{1.278440in}}%
\pgfpathlineto{\pgfqpoint{2.065785in}{0.767543in}}%
\pgfpathlineto{\pgfqpoint{2.066618in}{1.394514in}}%
\pgfpathlineto{\pgfqpoint{2.066410in}{0.729869in}}%
\pgfpathlineto{\pgfqpoint{2.066930in}{1.226418in}}%
\pgfpathlineto{\pgfqpoint{2.067868in}{0.882569in}}%
\pgfpathlineto{\pgfqpoint{2.068180in}{0.915581in}}%
\pgfpathlineto{\pgfqpoint{2.068805in}{1.250240in}}%
\pgfpathlineto{\pgfqpoint{2.069534in}{1.250217in}}%
\pgfpathlineto{\pgfqpoint{2.070575in}{0.832620in}}%
\pgfpathlineto{\pgfqpoint{2.069846in}{1.251480in}}%
\pgfpathlineto{\pgfqpoint{2.070784in}{0.919309in}}%
\pgfpathlineto{\pgfqpoint{2.070888in}{0.835363in}}%
\pgfpathlineto{\pgfqpoint{2.071200in}{1.254048in}}%
\pgfpathlineto{\pgfqpoint{2.071409in}{1.085213in}}%
\pgfpathlineto{\pgfqpoint{2.072346in}{1.296366in}}%
\pgfpathlineto{\pgfqpoint{2.071617in}{0.827356in}}%
\pgfpathlineto{\pgfqpoint{2.072450in}{1.180601in}}%
\pgfpathlineto{\pgfqpoint{2.072762in}{0.807147in}}%
\pgfpathlineto{\pgfqpoint{2.073387in}{1.273320in}}%
\pgfpathlineto{\pgfqpoint{2.073596in}{0.885493in}}%
\pgfpathlineto{\pgfqpoint{2.073908in}{1.118908in}}%
\pgfpathlineto{\pgfqpoint{2.074116in}{0.773432in}}%
\pgfpathlineto{\pgfqpoint{2.074741in}{1.080985in}}%
\pgfpathlineto{\pgfqpoint{2.075366in}{0.565209in}}%
\pgfpathlineto{\pgfqpoint{2.075053in}{1.154075in}}%
\pgfpathlineto{\pgfqpoint{2.075782in}{1.100536in}}%
\pgfpathlineto{\pgfqpoint{2.076511in}{1.233689in}}%
\pgfpathlineto{\pgfqpoint{2.076303in}{0.858027in}}%
\pgfpathlineto{\pgfqpoint{2.076616in}{1.063302in}}%
\pgfpathlineto{\pgfqpoint{2.077449in}{1.331416in}}%
\pgfpathlineto{\pgfqpoint{2.077761in}{0.740897in}}%
\pgfpathlineto{\pgfqpoint{2.078386in}{1.378437in}}%
\pgfpathlineto{\pgfqpoint{2.078907in}{1.271041in}}%
\pgfpathlineto{\pgfqpoint{2.079844in}{0.780389in}}%
\pgfpathlineto{\pgfqpoint{2.079948in}{1.383818in}}%
\pgfpathlineto{\pgfqpoint{2.080052in}{0.989600in}}%
\pgfpathlineto{\pgfqpoint{2.080365in}{1.280075in}}%
\pgfpathlineto{\pgfqpoint{2.080573in}{0.879158in}}%
\pgfpathlineto{\pgfqpoint{2.081094in}{1.146353in}}%
\pgfpathlineto{\pgfqpoint{2.081719in}{0.762841in}}%
\pgfpathlineto{\pgfqpoint{2.081510in}{1.354484in}}%
\pgfpathlineto{\pgfqpoint{2.082239in}{0.918810in}}%
\pgfpathlineto{\pgfqpoint{2.082343in}{1.409323in}}%
\pgfpathlineto{\pgfqpoint{2.082968in}{0.576836in}}%
\pgfpathlineto{\pgfqpoint{2.083385in}{1.120641in}}%
\pgfpathlineto{\pgfqpoint{2.083905in}{0.924960in}}%
\pgfpathlineto{\pgfqpoint{2.083697in}{1.231799in}}%
\pgfpathlineto{\pgfqpoint{2.084218in}{1.068964in}}%
\pgfpathlineto{\pgfqpoint{2.084322in}{1.373959in}}%
\pgfpathlineto{\pgfqpoint{2.084530in}{0.922020in}}%
\pgfpathlineto{\pgfqpoint{2.085259in}{0.994607in}}%
\pgfpathlineto{\pgfqpoint{2.085572in}{0.899817in}}%
\pgfpathlineto{\pgfqpoint{2.085468in}{1.226093in}}%
\pgfpathlineto{\pgfqpoint{2.085884in}{1.191983in}}%
\pgfpathlineto{\pgfqpoint{2.085988in}{1.233406in}}%
\pgfpathlineto{\pgfqpoint{2.086092in}{0.944436in}}%
\pgfpathlineto{\pgfqpoint{2.086509in}{1.230871in}}%
\pgfpathlineto{\pgfqpoint{2.087238in}{0.762470in}}%
\pgfpathlineto{\pgfqpoint{2.087655in}{0.841903in}}%
\pgfpathlineto{\pgfqpoint{2.088279in}{1.296766in}}%
\pgfpathlineto{\pgfqpoint{2.088175in}{0.831179in}}%
\pgfpathlineto{\pgfqpoint{2.088696in}{1.052465in}}%
\pgfpathlineto{\pgfqpoint{2.089321in}{0.640283in}}%
\pgfpathlineto{\pgfqpoint{2.089008in}{1.287609in}}%
\pgfpathlineto{\pgfqpoint{2.089737in}{1.131790in}}%
\pgfpathlineto{\pgfqpoint{2.089946in}{1.199603in}}%
\pgfpathlineto{\pgfqpoint{2.090050in}{1.119626in}}%
\pgfpathlineto{\pgfqpoint{2.090883in}{0.704194in}}%
\pgfpathlineto{\pgfqpoint{2.090362in}{1.392542in}}%
\pgfpathlineto{\pgfqpoint{2.091091in}{1.048606in}}%
\pgfpathlineto{\pgfqpoint{2.091195in}{1.242367in}}%
\pgfpathlineto{\pgfqpoint{2.091404in}{0.812369in}}%
\pgfpathlineto{\pgfqpoint{2.092237in}{1.146987in}}%
\pgfpathlineto{\pgfqpoint{2.092445in}{1.069051in}}%
\pgfpathlineto{\pgfqpoint{2.092549in}{1.156520in}}%
\pgfpathlineto{\pgfqpoint{2.092653in}{0.698956in}}%
\pgfpathlineto{\pgfqpoint{2.092966in}{1.224951in}}%
\pgfpathlineto{\pgfqpoint{2.093695in}{0.934772in}}%
\pgfpathlineto{\pgfqpoint{2.093799in}{0.940556in}}%
\pgfpathlineto{\pgfqpoint{2.094215in}{0.736825in}}%
\pgfpathlineto{\pgfqpoint{2.094736in}{1.345715in}}%
\pgfpathlineto{\pgfqpoint{2.094840in}{0.711119in}}%
\pgfpathlineto{\pgfqpoint{2.095882in}{1.036061in}}%
\pgfpathlineto{\pgfqpoint{2.096194in}{0.977396in}}%
\pgfpathlineto{\pgfqpoint{2.096090in}{1.209256in}}%
\pgfpathlineto{\pgfqpoint{2.096923in}{1.002815in}}%
\pgfpathlineto{\pgfqpoint{2.097756in}{1.201302in}}%
\pgfpathlineto{\pgfqpoint{2.097444in}{0.831351in}}%
\pgfpathlineto{\pgfqpoint{2.097860in}{1.054194in}}%
\pgfpathlineto{\pgfqpoint{2.098381in}{0.750136in}}%
\pgfpathlineto{\pgfqpoint{2.098277in}{1.163271in}}%
\pgfpathlineto{\pgfqpoint{2.099006in}{0.947027in}}%
\pgfpathlineto{\pgfqpoint{2.099631in}{1.313803in}}%
\pgfpathlineto{\pgfqpoint{2.099527in}{0.921572in}}%
\pgfpathlineto{\pgfqpoint{2.099735in}{1.195262in}}%
\pgfpathlineto{\pgfqpoint{2.099839in}{0.780794in}}%
\pgfpathlineto{\pgfqpoint{2.100672in}{1.253415in}}%
\pgfpathlineto{\pgfqpoint{2.100776in}{1.120592in}}%
\pgfpathlineto{\pgfqpoint{2.100880in}{1.120048in}}%
\pgfpathlineto{\pgfqpoint{2.100985in}{1.282927in}}%
\pgfpathlineto{\pgfqpoint{2.101714in}{0.850998in}}%
\pgfpathlineto{\pgfqpoint{2.101922in}{1.093125in}}%
\pgfpathlineto{\pgfqpoint{2.102859in}{0.913470in}}%
\pgfpathlineto{\pgfqpoint{2.102547in}{1.257336in}}%
\pgfpathlineto{\pgfqpoint{2.102963in}{1.152715in}}%
\pgfpathlineto{\pgfqpoint{2.103380in}{0.787439in}}%
\pgfpathlineto{\pgfqpoint{2.103172in}{1.242697in}}%
\pgfpathlineto{\pgfqpoint{2.104213in}{0.927082in}}%
\pgfpathlineto{\pgfqpoint{2.105150in}{1.280015in}}%
\pgfpathlineto{\pgfqpoint{2.105463in}{1.102168in}}%
\pgfpathlineto{\pgfqpoint{2.106400in}{0.984237in}}%
\pgfpathlineto{\pgfqpoint{2.106608in}{1.200508in}}%
\pgfpathlineto{\pgfqpoint{2.107025in}{1.240928in}}%
\pgfpathlineto{\pgfqpoint{2.107129in}{1.032147in}}%
\pgfpathlineto{\pgfqpoint{2.107650in}{0.790386in}}%
\pgfpathlineto{\pgfqpoint{2.107337in}{1.245250in}}%
\pgfpathlineto{\pgfqpoint{2.108066in}{1.140557in}}%
\pgfpathlineto{\pgfqpoint{2.108170in}{1.160190in}}%
\pgfpathlineto{\pgfqpoint{2.109003in}{0.839068in}}%
\pgfpathlineto{\pgfqpoint{2.109108in}{1.220553in}}%
\pgfpathlineto{\pgfqpoint{2.109212in}{1.129077in}}%
\pgfpathlineto{\pgfqpoint{2.109420in}{1.243678in}}%
\pgfpathlineto{\pgfqpoint{2.109524in}{0.932982in}}%
\pgfpathlineto{\pgfqpoint{2.109628in}{1.001538in}}%
\pgfpathlineto{\pgfqpoint{2.109837in}{1.515364in}}%
\pgfpathlineto{\pgfqpoint{2.110774in}{0.727155in}}%
\pgfpathlineto{\pgfqpoint{2.111711in}{1.411912in}}%
\pgfpathlineto{\pgfqpoint{2.112024in}{1.339329in}}%
\pgfpathlineto{\pgfqpoint{2.113065in}{0.866690in}}%
\pgfpathlineto{\pgfqpoint{2.113169in}{1.048108in}}%
\pgfpathlineto{\pgfqpoint{2.113794in}{1.328504in}}%
\pgfpathlineto{\pgfqpoint{2.114106in}{0.943351in}}%
\pgfpathlineto{\pgfqpoint{2.114211in}{1.128053in}}%
\pgfpathlineto{\pgfqpoint{2.114627in}{0.960471in}}%
\pgfpathlineto{\pgfqpoint{2.114835in}{1.146534in}}%
\pgfpathlineto{\pgfqpoint{2.115460in}{1.056810in}}%
\pgfpathlineto{\pgfqpoint{2.115564in}{1.450397in}}%
\pgfpathlineto{\pgfqpoint{2.116502in}{0.913192in}}%
\pgfpathlineto{\pgfqpoint{2.117231in}{1.468919in}}%
\pgfpathlineto{\pgfqpoint{2.117335in}{0.811275in}}%
\pgfpathlineto{\pgfqpoint{2.117751in}{1.179196in}}%
\pgfpathlineto{\pgfqpoint{2.118480in}{0.920582in}}%
\pgfpathlineto{\pgfqpoint{2.117960in}{1.252587in}}%
\pgfpathlineto{\pgfqpoint{2.118897in}{1.031586in}}%
\pgfpathlineto{\pgfqpoint{2.119105in}{1.049375in}}%
\pgfpathlineto{\pgfqpoint{2.119313in}{1.193224in}}%
\pgfpathlineto{\pgfqpoint{2.119418in}{0.944436in}}%
\pgfpathlineto{\pgfqpoint{2.119522in}{0.929878in}}%
\pgfpathlineto{\pgfqpoint{2.120147in}{1.282528in}}%
\pgfpathlineto{\pgfqpoint{2.119834in}{0.868685in}}%
\pgfpathlineto{\pgfqpoint{2.120563in}{1.193133in}}%
\pgfpathlineto{\pgfqpoint{2.121605in}{0.870848in}}%
\pgfpathlineto{\pgfqpoint{2.121292in}{1.246771in}}%
\pgfpathlineto{\pgfqpoint{2.121709in}{1.068001in}}%
\pgfpathlineto{\pgfqpoint{2.122438in}{1.310291in}}%
\pgfpathlineto{\pgfqpoint{2.122229in}{0.809846in}}%
\pgfpathlineto{\pgfqpoint{2.122750in}{1.060547in}}%
\pgfpathlineto{\pgfqpoint{2.123375in}{0.805389in}}%
\pgfpathlineto{\pgfqpoint{2.123583in}{1.082633in}}%
\pgfpathlineto{\pgfqpoint{2.123791in}{0.980560in}}%
\pgfpathlineto{\pgfqpoint{2.124000in}{1.206906in}}%
\pgfpathlineto{\pgfqpoint{2.124729in}{0.872515in}}%
\pgfpathlineto{\pgfqpoint{2.124833in}{0.992509in}}%
\pgfpathlineto{\pgfqpoint{2.124937in}{0.749448in}}%
\pgfpathlineto{\pgfqpoint{2.125041in}{1.334987in}}%
\pgfpathlineto{\pgfqpoint{2.125874in}{1.151765in}}%
\pgfpathlineto{\pgfqpoint{2.125978in}{1.167951in}}%
\pgfpathlineto{\pgfqpoint{2.126603in}{1.291489in}}%
\pgfpathlineto{\pgfqpoint{2.127124in}{0.790375in}}%
\pgfpathlineto{\pgfqpoint{2.128061in}{1.161702in}}%
\pgfpathlineto{\pgfqpoint{2.127749in}{0.699457in}}%
\pgfpathlineto{\pgfqpoint{2.128165in}{0.972646in}}%
\pgfpathlineto{\pgfqpoint{2.128270in}{0.737394in}}%
\pgfpathlineto{\pgfqpoint{2.128582in}{1.229033in}}%
\pgfpathlineto{\pgfqpoint{2.129103in}{1.164928in}}%
\pgfpathlineto{\pgfqpoint{2.129207in}{1.154600in}}%
\pgfpathlineto{\pgfqpoint{2.129728in}{0.831378in}}%
\pgfpathlineto{\pgfqpoint{2.129936in}{1.327606in}}%
\pgfpathlineto{\pgfqpoint{2.130352in}{1.005796in}}%
\pgfpathlineto{\pgfqpoint{2.130873in}{0.917160in}}%
\pgfpathlineto{\pgfqpoint{2.131394in}{1.362400in}}%
\pgfpathlineto{\pgfqpoint{2.132435in}{0.824798in}}%
\pgfpathlineto{\pgfqpoint{2.132539in}{0.985452in}}%
\pgfpathlineto{\pgfqpoint{2.132643in}{0.885791in}}%
\pgfpathlineto{\pgfqpoint{2.133060in}{1.315642in}}%
\pgfpathlineto{\pgfqpoint{2.133372in}{1.099103in}}%
\pgfpathlineto{\pgfqpoint{2.133789in}{0.801129in}}%
\pgfpathlineto{\pgfqpoint{2.134414in}{1.275588in}}%
\pgfpathlineto{\pgfqpoint{2.134830in}{0.864816in}}%
\pgfpathlineto{\pgfqpoint{2.135559in}{1.111572in}}%
\pgfpathlineto{\pgfqpoint{2.135872in}{0.520316in}}%
\pgfpathlineto{\pgfqpoint{2.135976in}{1.361410in}}%
\pgfpathlineto{\pgfqpoint{2.136705in}{1.033378in}}%
\pgfpathlineto{\pgfqpoint{2.137122in}{1.377742in}}%
\pgfpathlineto{\pgfqpoint{2.136913in}{0.887642in}}%
\pgfpathlineto{\pgfqpoint{2.137746in}{0.995526in}}%
\pgfpathlineto{\pgfqpoint{2.138371in}{1.230020in}}%
\pgfpathlineto{\pgfqpoint{2.138892in}{0.733849in}}%
\pgfpathlineto{\pgfqpoint{2.139621in}{1.301734in}}%
\pgfpathlineto{\pgfqpoint{2.140037in}{0.864700in}}%
\pgfpathlineto{\pgfqpoint{2.140662in}{1.450812in}}%
\pgfpathlineto{\pgfqpoint{2.141183in}{1.192116in}}%
\pgfpathlineto{\pgfqpoint{2.142224in}{0.886668in}}%
\pgfpathlineto{\pgfqpoint{2.141391in}{1.299311in}}%
\pgfpathlineto{\pgfqpoint{2.142433in}{1.042836in}}%
\pgfpathlineto{\pgfqpoint{2.143266in}{0.994900in}}%
\pgfpathlineto{\pgfqpoint{2.143578in}{1.385590in}}%
\pgfpathlineto{\pgfqpoint{2.143891in}{1.446378in}}%
\pgfpathlineto{\pgfqpoint{2.144724in}{0.859605in}}%
\pgfpathlineto{\pgfqpoint{2.145765in}{1.210993in}}%
\pgfpathlineto{\pgfqpoint{2.145974in}{0.829343in}}%
\pgfpathlineto{\pgfqpoint{2.146598in}{1.323569in}}%
\pgfpathlineto{\pgfqpoint{2.146807in}{1.262928in}}%
\pgfpathlineto{\pgfqpoint{2.147640in}{0.622969in}}%
\pgfpathlineto{\pgfqpoint{2.148265in}{0.913345in}}%
\pgfpathlineto{\pgfqpoint{2.148889in}{1.317306in}}%
\pgfpathlineto{\pgfqpoint{2.148681in}{0.838872in}}%
\pgfpathlineto{\pgfqpoint{2.149306in}{1.184409in}}%
\pgfpathlineto{\pgfqpoint{2.150243in}{0.697143in}}%
\pgfpathlineto{\pgfqpoint{2.150347in}{1.213942in}}%
\pgfpathlineto{\pgfqpoint{2.151076in}{0.599752in}}%
\pgfpathlineto{\pgfqpoint{2.151389in}{1.025343in}}%
\pgfpathlineto{\pgfqpoint{2.151805in}{0.772812in}}%
\pgfpathlineto{\pgfqpoint{2.152534in}{1.443031in}}%
\pgfpathlineto{\pgfqpoint{2.153784in}{0.750497in}}%
\pgfpathlineto{\pgfqpoint{2.154721in}{1.276119in}}%
\pgfpathlineto{\pgfqpoint{2.155138in}{1.189881in}}%
\pgfpathlineto{\pgfqpoint{2.155346in}{0.812515in}}%
\pgfpathlineto{\pgfqpoint{2.156283in}{0.941216in}}%
\pgfpathlineto{\pgfqpoint{2.156492in}{0.817288in}}%
\pgfpathlineto{\pgfqpoint{2.156596in}{1.064374in}}%
\pgfpathlineto{\pgfqpoint{2.156700in}{0.970652in}}%
\pgfpathlineto{\pgfqpoint{2.157117in}{1.320409in}}%
\pgfpathlineto{\pgfqpoint{2.156908in}{0.841610in}}%
\pgfpathlineto{\pgfqpoint{2.157846in}{1.073802in}}%
\pgfpathlineto{\pgfqpoint{2.158783in}{0.790644in}}%
\pgfpathlineto{\pgfqpoint{2.158158in}{1.224146in}}%
\pgfpathlineto{\pgfqpoint{2.158887in}{0.889503in}}%
\pgfpathlineto{\pgfqpoint{2.158991in}{1.184603in}}%
\pgfpathlineto{\pgfqpoint{2.159199in}{0.826750in}}%
\pgfpathlineto{\pgfqpoint{2.160033in}{1.056777in}}%
\pgfpathlineto{\pgfqpoint{2.160241in}{1.296725in}}%
\pgfpathlineto{\pgfqpoint{2.160866in}{0.857995in}}%
\pgfpathlineto{\pgfqpoint{2.160970in}{1.325512in}}%
\pgfpathlineto{\pgfqpoint{2.162011in}{1.082902in}}%
\pgfpathlineto{\pgfqpoint{2.162324in}{0.931729in}}%
\pgfpathlineto{\pgfqpoint{2.162949in}{1.258878in}}%
\pgfpathlineto{\pgfqpoint{2.163053in}{0.933972in}}%
\pgfpathlineto{\pgfqpoint{2.163678in}{1.228706in}}%
\pgfpathlineto{\pgfqpoint{2.164094in}{1.070719in}}%
\pgfpathlineto{\pgfqpoint{2.164198in}{0.878592in}}%
\pgfpathlineto{\pgfqpoint{2.164511in}{1.320664in}}%
\pgfpathlineto{\pgfqpoint{2.165240in}{0.959782in}}%
\pgfpathlineto{\pgfqpoint{2.165864in}{1.324085in}}%
\pgfpathlineto{\pgfqpoint{2.165552in}{0.762067in}}%
\pgfpathlineto{\pgfqpoint{2.166385in}{1.294763in}}%
\pgfpathlineto{\pgfqpoint{2.166906in}{0.826797in}}%
\pgfpathlineto{\pgfqpoint{2.167635in}{0.965017in}}%
\pgfpathlineto{\pgfqpoint{2.167739in}{0.967645in}}%
\pgfpathlineto{\pgfqpoint{2.168051in}{0.907293in}}%
\pgfpathlineto{\pgfqpoint{2.168156in}{1.213274in}}%
\pgfpathlineto{\pgfqpoint{2.168364in}{0.925081in}}%
\pgfpathlineto{\pgfqpoint{2.168780in}{1.290514in}}%
\pgfpathlineto{\pgfqpoint{2.168885in}{0.806644in}}%
\pgfpathlineto{\pgfqpoint{2.169718in}{1.476250in}}%
\pgfpathlineto{\pgfqpoint{2.169926in}{1.070573in}}%
\pgfpathlineto{\pgfqpoint{2.170343in}{0.776827in}}%
\pgfpathlineto{\pgfqpoint{2.170655in}{1.339663in}}%
\pgfpathlineto{\pgfqpoint{2.170967in}{1.031297in}}%
\pgfpathlineto{\pgfqpoint{2.171072in}{1.041460in}}%
\pgfpathlineto{\pgfqpoint{2.171488in}{0.897567in}}%
\pgfpathlineto{\pgfqpoint{2.172217in}{1.301752in}}%
\pgfpathlineto{\pgfqpoint{2.173363in}{0.719092in}}%
\pgfpathlineto{\pgfqpoint{2.172634in}{1.330821in}}%
\pgfpathlineto{\pgfqpoint{2.173467in}{0.890887in}}%
\pgfpathlineto{\pgfqpoint{2.174716in}{1.340146in}}%
\pgfpathlineto{\pgfqpoint{2.174508in}{0.837466in}}%
\pgfpathlineto{\pgfqpoint{2.174821in}{1.218937in}}%
\pgfpathlineto{\pgfqpoint{2.175341in}{0.715048in}}%
\pgfpathlineto{\pgfqpoint{2.175966in}{1.052836in}}%
\pgfpathlineto{\pgfqpoint{2.176487in}{0.634706in}}%
\pgfpathlineto{\pgfqpoint{2.176695in}{1.189159in}}%
\pgfpathlineto{\pgfqpoint{2.176903in}{1.159383in}}%
\pgfpathlineto{\pgfqpoint{2.177008in}{1.223898in}}%
\pgfpathlineto{\pgfqpoint{2.177112in}{0.568382in}}%
\pgfpathlineto{\pgfqpoint{2.177737in}{1.336846in}}%
\pgfpathlineto{\pgfqpoint{2.178153in}{0.790472in}}%
\pgfpathlineto{\pgfqpoint{2.178466in}{1.318666in}}%
\pgfpathlineto{\pgfqpoint{2.178361in}{0.716479in}}%
\pgfpathlineto{\pgfqpoint{2.179611in}{1.070405in}}%
\pgfpathlineto{\pgfqpoint{2.179715in}{0.837149in}}%
\pgfpathlineto{\pgfqpoint{2.179924in}{1.213193in}}%
\pgfpathlineto{\pgfqpoint{2.180652in}{0.956490in}}%
\pgfpathlineto{\pgfqpoint{2.181694in}{1.368843in}}%
\pgfpathlineto{\pgfqpoint{2.181173in}{0.699870in}}%
\pgfpathlineto{\pgfqpoint{2.181798in}{1.265585in}}%
\pgfpathlineto{\pgfqpoint{2.182423in}{0.903657in}}%
\pgfpathlineto{\pgfqpoint{2.182944in}{0.940589in}}%
\pgfpathlineto{\pgfqpoint{2.183360in}{1.425280in}}%
\pgfpathlineto{\pgfqpoint{2.184089in}{1.415881in}}%
\pgfpathlineto{\pgfqpoint{2.184818in}{0.685527in}}%
\pgfpathlineto{\pgfqpoint{2.185131in}{1.160506in}}%
\pgfpathlineto{\pgfqpoint{2.185235in}{1.349486in}}%
\pgfpathlineto{\pgfqpoint{2.185547in}{0.882800in}}%
\pgfpathlineto{\pgfqpoint{2.186172in}{1.060519in}}%
\pgfpathlineto{\pgfqpoint{2.186484in}{0.697344in}}%
\pgfpathlineto{\pgfqpoint{2.186901in}{1.128233in}}%
\pgfpathlineto{\pgfqpoint{2.187318in}{0.847937in}}%
\pgfpathlineto{\pgfqpoint{2.188255in}{1.186458in}}%
\pgfpathlineto{\pgfqpoint{2.188359in}{1.156094in}}%
\pgfpathlineto{\pgfqpoint{2.188463in}{0.744889in}}%
\pgfpathlineto{\pgfqpoint{2.188671in}{1.278466in}}%
\pgfpathlineto{\pgfqpoint{2.189504in}{0.913347in}}%
\pgfpathlineto{\pgfqpoint{2.190129in}{1.403747in}}%
\pgfpathlineto{\pgfqpoint{2.190858in}{1.169267in}}%
\pgfpathlineto{\pgfqpoint{2.190962in}{0.848850in}}%
\pgfpathlineto{\pgfqpoint{2.191691in}{1.226566in}}%
\pgfpathlineto{\pgfqpoint{2.191900in}{1.100894in}}%
\pgfpathlineto{\pgfqpoint{2.192004in}{1.114917in}}%
\pgfpathlineto{\pgfqpoint{2.192837in}{0.854783in}}%
\pgfpathlineto{\pgfqpoint{2.192941in}{1.287852in}}%
\pgfpathlineto{\pgfqpoint{2.193045in}{1.133881in}}%
\pgfpathlineto{\pgfqpoint{2.194191in}{0.900018in}}%
\pgfpathlineto{\pgfqpoint{2.193358in}{1.187460in}}%
\pgfpathlineto{\pgfqpoint{2.194295in}{0.988565in}}%
\pgfpathlineto{\pgfqpoint{2.194607in}{1.255506in}}%
\pgfpathlineto{\pgfqpoint{2.194712in}{0.701428in}}%
\pgfpathlineto{\pgfqpoint{2.194920in}{0.979337in}}%
\pgfpathlineto{\pgfqpoint{2.195024in}{0.636517in}}%
\pgfpathlineto{\pgfqpoint{2.195441in}{1.285196in}}%
\pgfpathlineto{\pgfqpoint{2.196065in}{0.849370in}}%
\pgfpathlineto{\pgfqpoint{2.196690in}{1.271878in}}%
\pgfpathlineto{\pgfqpoint{2.197211in}{1.075576in}}%
\pgfpathlineto{\pgfqpoint{2.197836in}{1.199083in}}%
\pgfpathlineto{\pgfqpoint{2.197627in}{0.927873in}}%
\pgfpathlineto{\pgfqpoint{2.198148in}{1.162414in}}%
\pgfpathlineto{\pgfqpoint{2.199190in}{0.869381in}}%
\pgfpathlineto{\pgfqpoint{2.198565in}{1.275858in}}%
\pgfpathlineto{\pgfqpoint{2.199398in}{0.936126in}}%
\pgfpathlineto{\pgfqpoint{2.199814in}{1.305635in}}%
\pgfpathlineto{\pgfqpoint{2.199710in}{0.873989in}}%
\pgfpathlineto{\pgfqpoint{2.200439in}{0.994799in}}%
\pgfpathlineto{\pgfqpoint{2.200856in}{0.843188in}}%
\pgfpathlineto{\pgfqpoint{2.200648in}{1.160494in}}%
\pgfpathlineto{\pgfqpoint{2.201168in}{0.884806in}}%
\pgfpathlineto{\pgfqpoint{2.201481in}{0.713984in}}%
\pgfpathlineto{\pgfqpoint{2.202210in}{1.270948in}}%
\pgfpathlineto{\pgfqpoint{2.203355in}{0.790247in}}%
\pgfpathlineto{\pgfqpoint{2.203980in}{1.342172in}}%
\pgfpathlineto{\pgfqpoint{2.204501in}{1.169787in}}%
\pgfpathlineto{\pgfqpoint{2.204605in}{1.191375in}}%
\pgfpathlineto{\pgfqpoint{2.204917in}{0.825255in}}%
\pgfpathlineto{\pgfqpoint{2.205230in}{1.336861in}}%
\pgfpathlineto{\pgfqpoint{2.205750in}{0.881109in}}%
\pgfpathlineto{\pgfqpoint{2.206792in}{1.711955in}}%
\pgfpathlineto{\pgfqpoint{2.207000in}{1.180629in}}%
\pgfpathlineto{\pgfqpoint{2.207625in}{0.836956in}}%
\pgfpathlineto{\pgfqpoint{2.207417in}{1.330500in}}%
\pgfpathlineto{\pgfqpoint{2.208146in}{0.970931in}}%
\pgfpathlineto{\pgfqpoint{2.208250in}{1.358546in}}%
\pgfpathlineto{\pgfqpoint{2.208875in}{0.759103in}}%
\pgfpathlineto{\pgfqpoint{2.209187in}{1.084497in}}%
\pgfpathlineto{\pgfqpoint{2.209604in}{0.876103in}}%
\pgfpathlineto{\pgfqpoint{2.209500in}{1.131077in}}%
\pgfpathlineto{\pgfqpoint{2.209916in}{1.083539in}}%
\pgfpathlineto{\pgfqpoint{2.210124in}{0.874359in}}%
\pgfpathlineto{\pgfqpoint{2.210958in}{1.214546in}}%
\pgfpathlineto{\pgfqpoint{2.211999in}{0.865110in}}%
\pgfpathlineto{\pgfqpoint{2.211270in}{1.344317in}}%
\pgfpathlineto{\pgfqpoint{2.212103in}{0.939616in}}%
\pgfpathlineto{\pgfqpoint{2.212207in}{0.965370in}}%
\pgfpathlineto{\pgfqpoint{2.212311in}{1.383220in}}%
\pgfpathlineto{\pgfqpoint{2.213145in}{0.851450in}}%
\pgfpathlineto{\pgfqpoint{2.213353in}{1.254678in}}%
\pgfpathlineto{\pgfqpoint{2.214394in}{0.820640in}}%
\pgfpathlineto{\pgfqpoint{2.214498in}{1.051620in}}%
\pgfpathlineto{\pgfqpoint{2.215436in}{0.912821in}}%
\pgfpathlineto{\pgfqpoint{2.215019in}{1.257147in}}%
\pgfpathlineto{\pgfqpoint{2.215540in}{1.066217in}}%
\pgfpathlineto{\pgfqpoint{2.215852in}{0.881506in}}%
\pgfpathlineto{\pgfqpoint{2.216165in}{1.155259in}}%
\pgfpathlineto{\pgfqpoint{2.216269in}{1.151846in}}%
\pgfpathlineto{\pgfqpoint{2.217206in}{1.284068in}}%
\pgfpathlineto{\pgfqpoint{2.216477in}{0.869234in}}%
\pgfpathlineto{\pgfqpoint{2.217310in}{1.134777in}}%
\pgfpathlineto{\pgfqpoint{2.217727in}{0.917217in}}%
\pgfpathlineto{\pgfqpoint{2.217831in}{1.327149in}}%
\pgfpathlineto{\pgfqpoint{2.218352in}{1.130136in}}%
\pgfpathlineto{\pgfqpoint{2.218456in}{1.146578in}}%
\pgfpathlineto{\pgfqpoint{2.219185in}{0.823113in}}%
\pgfpathlineto{\pgfqpoint{2.219497in}{1.233140in}}%
\pgfpathlineto{\pgfqpoint{2.219601in}{0.996478in}}%
\pgfpathlineto{\pgfqpoint{2.220330in}{1.457848in}}%
\pgfpathlineto{\pgfqpoint{2.220122in}{0.897320in}}%
\pgfpathlineto{\pgfqpoint{2.220643in}{1.169967in}}%
\pgfpathlineto{\pgfqpoint{2.221059in}{0.760023in}}%
\pgfpathlineto{\pgfqpoint{2.221268in}{1.231489in}}%
\pgfpathlineto{\pgfqpoint{2.221788in}{0.874678in}}%
\pgfpathlineto{\pgfqpoint{2.222725in}{1.228983in}}%
\pgfpathlineto{\pgfqpoint{2.222101in}{0.871528in}}%
\pgfpathlineto{\pgfqpoint{2.222934in}{1.150316in}}%
\pgfpathlineto{\pgfqpoint{2.223142in}{1.183303in}}%
\pgfpathlineto{\pgfqpoint{2.223246in}{1.059943in}}%
\pgfpathlineto{\pgfqpoint{2.223350in}{0.920189in}}%
\pgfpathlineto{\pgfqpoint{2.223663in}{1.321225in}}%
\pgfpathlineto{\pgfqpoint{2.224183in}{1.276717in}}%
\pgfpathlineto{\pgfqpoint{2.224392in}{0.871877in}}%
\pgfpathlineto{\pgfqpoint{2.225225in}{1.056793in}}%
\pgfpathlineto{\pgfqpoint{2.225329in}{1.417688in}}%
\pgfpathlineto{\pgfqpoint{2.225746in}{0.911002in}}%
\pgfpathlineto{\pgfqpoint{2.226266in}{1.147141in}}%
\pgfpathlineto{\pgfqpoint{2.226370in}{0.881026in}}%
\pgfpathlineto{\pgfqpoint{2.226579in}{1.238575in}}%
\pgfpathlineto{\pgfqpoint{2.227308in}{1.082021in}}%
\pgfpathlineto{\pgfqpoint{2.227412in}{1.205806in}}%
\pgfpathlineto{\pgfqpoint{2.228037in}{0.860823in}}%
\pgfpathlineto{\pgfqpoint{2.228141in}{1.153147in}}%
\pgfpathlineto{\pgfqpoint{2.228557in}{0.826905in}}%
\pgfpathlineto{\pgfqpoint{2.228349in}{1.242879in}}%
\pgfpathlineto{\pgfqpoint{2.229286in}{0.916557in}}%
\pgfpathlineto{\pgfqpoint{2.229599in}{1.301634in}}%
\pgfpathlineto{\pgfqpoint{2.229703in}{0.757530in}}%
\pgfpathlineto{\pgfqpoint{2.230328in}{1.140007in}}%
\pgfpathlineto{\pgfqpoint{2.230536in}{0.850290in}}%
\pgfpathlineto{\pgfqpoint{2.230953in}{1.169081in}}%
\pgfpathlineto{\pgfqpoint{2.231473in}{1.023282in}}%
\pgfpathlineto{\pgfqpoint{2.232411in}{0.788761in}}%
\pgfpathlineto{\pgfqpoint{2.231994in}{1.242334in}}%
\pgfpathlineto{\pgfqpoint{2.232515in}{1.075335in}}%
\pgfpathlineto{\pgfqpoint{2.233244in}{1.417925in}}%
\pgfpathlineto{\pgfqpoint{2.233348in}{0.871691in}}%
\pgfpathlineto{\pgfqpoint{2.233452in}{1.131751in}}%
\pgfpathlineto{\pgfqpoint{2.234389in}{0.967460in}}%
\pgfpathlineto{\pgfqpoint{2.234077in}{1.380370in}}%
\pgfpathlineto{\pgfqpoint{2.234598in}{1.028224in}}%
\pgfpathlineto{\pgfqpoint{2.235014in}{1.279394in}}%
\pgfpathlineto{\pgfqpoint{2.235535in}{0.934866in}}%
\pgfpathlineto{\pgfqpoint{2.235639in}{1.193562in}}%
\pgfpathlineto{\pgfqpoint{2.235743in}{0.876214in}}%
\pgfpathlineto{\pgfqpoint{2.236576in}{1.268702in}}%
\pgfpathlineto{\pgfqpoint{2.236785in}{0.883993in}}%
\pgfpathlineto{\pgfqpoint{2.236889in}{0.899600in}}%
\pgfpathlineto{\pgfqpoint{2.236993in}{0.747702in}}%
\pgfpathlineto{\pgfqpoint{2.237514in}{1.126529in}}%
\pgfpathlineto{\pgfqpoint{2.237930in}{0.922221in}}%
\pgfpathlineto{\pgfqpoint{2.238347in}{1.314248in}}%
\pgfpathlineto{\pgfqpoint{2.238659in}{0.890994in}}%
\pgfpathlineto{\pgfqpoint{2.239180in}{1.231638in}}%
\pgfpathlineto{\pgfqpoint{2.240325in}{0.645362in}}%
\pgfpathlineto{\pgfqpoint{2.241575in}{1.307645in}}%
\pgfpathlineto{\pgfqpoint{2.242200in}{0.943150in}}%
\pgfpathlineto{\pgfqpoint{2.242721in}{1.053190in}}%
\pgfpathlineto{\pgfqpoint{2.243345in}{1.322717in}}%
\pgfpathlineto{\pgfqpoint{2.243033in}{0.832685in}}%
\pgfpathlineto{\pgfqpoint{2.243658in}{0.975404in}}%
\pgfpathlineto{\pgfqpoint{2.243762in}{0.972805in}}%
\pgfpathlineto{\pgfqpoint{2.244283in}{1.458501in}}%
\pgfpathlineto{\pgfqpoint{2.244179in}{0.936166in}}%
\pgfpathlineto{\pgfqpoint{2.244803in}{1.298381in}}%
\pgfpathlineto{\pgfqpoint{2.245637in}{0.847263in}}%
\pgfpathlineto{\pgfqpoint{2.245324in}{1.333458in}}%
\pgfpathlineto{\pgfqpoint{2.245949in}{1.128133in}}%
\pgfpathlineto{\pgfqpoint{2.247094in}{0.886129in}}%
\pgfpathlineto{\pgfqpoint{2.246678in}{1.321455in}}%
\pgfpathlineto{\pgfqpoint{2.247199in}{0.954498in}}%
\pgfpathlineto{\pgfqpoint{2.248136in}{1.349193in}}%
\pgfpathlineto{\pgfqpoint{2.247615in}{0.835382in}}%
\pgfpathlineto{\pgfqpoint{2.248240in}{1.038777in}}%
\pgfpathlineto{\pgfqpoint{2.248344in}{1.033821in}}%
\pgfpathlineto{\pgfqpoint{2.248448in}{0.677124in}}%
\pgfpathlineto{\pgfqpoint{2.248761in}{1.283849in}}%
\pgfpathlineto{\pgfqpoint{2.249386in}{1.090487in}}%
\pgfpathlineto{\pgfqpoint{2.250219in}{1.288800in}}%
\pgfpathlineto{\pgfqpoint{2.249906in}{0.676726in}}%
\pgfpathlineto{\pgfqpoint{2.250323in}{1.180074in}}%
\pgfpathlineto{\pgfqpoint{2.250948in}{1.216534in}}%
\pgfpathlineto{\pgfqpoint{2.251573in}{0.857863in}}%
\pgfpathlineto{\pgfqpoint{2.252093in}{1.277481in}}%
\pgfpathlineto{\pgfqpoint{2.252718in}{1.048082in}}%
\pgfpathlineto{\pgfqpoint{2.253239in}{1.304375in}}%
\pgfpathlineto{\pgfqpoint{2.253760in}{0.841685in}}%
\pgfpathlineto{\pgfqpoint{2.253864in}{1.267647in}}%
\pgfpathlineto{\pgfqpoint{2.254176in}{0.819171in}}%
\pgfpathlineto{\pgfqpoint{2.254905in}{1.039398in}}%
\pgfpathlineto{\pgfqpoint{2.255634in}{0.760890in}}%
\pgfpathlineto{\pgfqpoint{2.255217in}{1.454870in}}%
\pgfpathlineto{\pgfqpoint{2.255842in}{1.079777in}}%
\pgfpathlineto{\pgfqpoint{2.256363in}{1.274311in}}%
\pgfpathlineto{\pgfqpoint{2.256571in}{0.871138in}}%
\pgfpathlineto{\pgfqpoint{2.256988in}{1.167522in}}%
\pgfpathlineto{\pgfqpoint{2.257092in}{1.176316in}}%
\pgfpathlineto{\pgfqpoint{2.258029in}{0.649650in}}%
\pgfpathlineto{\pgfqpoint{2.257404in}{1.274303in}}%
\pgfpathlineto{\pgfqpoint{2.258238in}{0.992976in}}%
\pgfpathlineto{\pgfqpoint{2.258654in}{1.258594in}}%
\pgfpathlineto{\pgfqpoint{2.258446in}{0.920915in}}%
\pgfpathlineto{\pgfqpoint{2.259487in}{1.148398in}}%
\pgfpathlineto{\pgfqpoint{2.259591in}{0.677102in}}%
\pgfpathlineto{\pgfqpoint{2.260529in}{1.202392in}}%
\pgfpathlineto{\pgfqpoint{2.260633in}{0.913415in}}%
\pgfpathlineto{\pgfqpoint{2.260945in}{0.698574in}}%
\pgfpathlineto{\pgfqpoint{2.261987in}{1.221856in}}%
\pgfpathlineto{\pgfqpoint{2.262507in}{0.868053in}}%
\pgfpathlineto{\pgfqpoint{2.262299in}{1.294224in}}%
\pgfpathlineto{\pgfqpoint{2.263028in}{1.093974in}}%
\pgfusepath{stroke}%
\end{pgfscope}%
\begin{pgfscope}%
\pgfsetrectcap%
\pgfsetmiterjoin%
\pgfsetlinewidth{0.803000pt}%
\definecolor{currentstroke}{rgb}{0.000000,0.000000,0.000000}%
\pgfsetstrokecolor{currentstroke}%
\pgfsetdash{}{0pt}%
\pgfpathmoveto{\pgfqpoint{0.471688in}{0.416447in}}%
\pgfpathlineto{\pgfqpoint{0.471688in}{1.773646in}}%
\pgfusepath{stroke}%
\end{pgfscope}%
\begin{pgfscope}%
\pgfsetrectcap%
\pgfsetmiterjoin%
\pgfsetlinewidth{0.803000pt}%
\definecolor{currentstroke}{rgb}{0.000000,0.000000,0.000000}%
\pgfsetstrokecolor{currentstroke}%
\pgfsetdash{}{0pt}%
\pgfpathmoveto{\pgfqpoint{2.348330in}{0.416447in}}%
\pgfpathlineto{\pgfqpoint{2.348330in}{1.773646in}}%
\pgfusepath{stroke}%
\end{pgfscope}%
\begin{pgfscope}%
\pgfsetrectcap%
\pgfsetmiterjoin%
\pgfsetlinewidth{0.803000pt}%
\definecolor{currentstroke}{rgb}{0.000000,0.000000,0.000000}%
\pgfsetstrokecolor{currentstroke}%
\pgfsetdash{}{0pt}%
\pgfpathmoveto{\pgfqpoint{0.471688in}{0.416447in}}%
\pgfpathlineto{\pgfqpoint{2.348330in}{0.416447in}}%
\pgfusepath{stroke}%
\end{pgfscope}%
\begin{pgfscope}%
\pgfsetrectcap%
\pgfsetmiterjoin%
\pgfsetlinewidth{0.803000pt}%
\definecolor{currentstroke}{rgb}{0.000000,0.000000,0.000000}%
\pgfsetstrokecolor{currentstroke}%
\pgfsetdash{}{0pt}%
\pgfpathmoveto{\pgfqpoint{0.471688in}{1.773646in}}%
\pgfpathlineto{\pgfqpoint{2.348330in}{1.773646in}}%
\pgfusepath{stroke}%
\end{pgfscope}%
\begin{pgfscope}%
\pgfsetbuttcap%
\pgfsetmiterjoin%
\definecolor{currentfill}{rgb}{1.000000,1.000000,1.000000}%
\pgfsetfillcolor{currentfill}%
\pgfsetfillopacity{0.800000}%
\pgfsetlinewidth{1.003750pt}%
\definecolor{currentstroke}{rgb}{0.800000,0.800000,0.800000}%
\pgfsetstrokecolor{currentstroke}%
\pgfsetstrokeopacity{0.800000}%
\pgfsetdash{}{0pt}%
\pgfpathmoveto{\pgfqpoint{0.549465in}{1.529869in}}%
\pgfpathlineto{\pgfqpoint{1.518576in}{1.529869in}}%
\pgfpathquadraticcurveto{\pgfqpoint{1.540799in}{1.529869in}}{\pgfqpoint{1.540799in}{1.552091in}}%
\pgfpathlineto{\pgfqpoint{1.540799in}{1.695868in}}%
\pgfpathquadraticcurveto{\pgfqpoint{1.540799in}{1.718091in}}{\pgfqpoint{1.518576in}{1.718091in}}%
\pgfpathlineto{\pgfqpoint{0.549465in}{1.718091in}}%
\pgfpathquadraticcurveto{\pgfqpoint{0.527243in}{1.718091in}}{\pgfqpoint{0.527243in}{1.695868in}}%
\pgfpathlineto{\pgfqpoint{0.527243in}{1.552091in}}%
\pgfpathquadraticcurveto{\pgfqpoint{0.527243in}{1.529869in}}{\pgfqpoint{0.549465in}{1.529869in}}%
\pgfpathlineto{\pgfqpoint{0.549465in}{1.529869in}}%
\pgfpathclose%
\pgfusepath{stroke,fill}%
\end{pgfscope}%
\begin{pgfscope}%
\pgfsetrectcap%
\pgfsetroundjoin%
\pgfsetlinewidth{1.505625pt}%
\definecolor{currentstroke}{rgb}{0.000000,0.447059,0.698039}%
\pgfsetstrokecolor{currentstroke}%
\pgfsetdash{}{0pt}%
\pgfpathmoveto{\pgfqpoint{0.571688in}{1.634757in}}%
\pgfpathlineto{\pgfqpoint{0.682799in}{1.634757in}}%
\pgfpathlineto{\pgfqpoint{0.793910in}{1.634757in}}%
\pgfusepath{stroke}%
\end{pgfscope}%
\begin{pgfscope}%
\definecolor{textcolor}{rgb}{0.000000,0.000000,0.000000}%
\pgfsetstrokecolor{textcolor}%
\pgfsetfillcolor{textcolor}%
\pgftext[x=0.882799in,y=1.595868in,left,base]{\color{textcolor}\rmfamily\fontsize{8.000000}{9.600000}\selectfont White noise}%
\end{pgfscope}%
\end{pgfpicture}%
\makeatother%
\endgroup%

        } % scalebox
        \caption{Time domain}
        \label{fig:white_noise_time}
    \end{subfigure}
    \begin{subfigure}{0.3\linewidth}
        \scalebox{0.75}{%
            %% Creator: Matplotlib, PGF backend
%%
%% To include the figure in your LaTeX document, write
%%   \input{<filename>.pgf}
%%
%% Make sure the required packages are loaded in your preamble
%%   \usepackage{pgf}
%%
%% Also ensure that all the required font packages are loaded; for instance,
%% the lmodern package is sometimes necessary when using math font.
%%   \usepackage{lmodern}
%%
%% Figures using additional raster images can only be included by \input if
%% they are in the same directory as the main LaTeX file. For loading figures
%% from other directories you can use the `import` package
%%   \usepackage{import}
%%
%% and then include the figures with
%%   \import{<path to file>}{<filename>.pgf}
%%
%% Matplotlib used the following preamble
%%   \usepackage{siunitx}
%%   \usepackage{fontspec}
%%   \makeatletter\@ifpackageloaded{underscore}{}{\usepackage[strings]{underscore}}\makeatother
%%
\begingroup%
\makeatletter%
\begin{pgfpicture}%
\pgfpathrectangle{\pgfpointorigin}{\pgfqpoint{2.440000in}{1.830000in}}%
\pgfusepath{use as bounding box, clip}%
\begin{pgfscope}%
\pgfsetbuttcap%
\pgfsetmiterjoin%
\definecolor{currentfill}{rgb}{1.000000,1.000000,1.000000}%
\pgfsetfillcolor{currentfill}%
\pgfsetlinewidth{0.000000pt}%
\definecolor{currentstroke}{rgb}{1.000000,1.000000,1.000000}%
\pgfsetstrokecolor{currentstroke}%
\pgfsetdash{}{0pt}%
\pgfpathmoveto{\pgfqpoint{0.000000in}{0.000000in}}%
\pgfpathlineto{\pgfqpoint{2.440000in}{0.000000in}}%
\pgfpathlineto{\pgfqpoint{2.440000in}{1.830000in}}%
\pgfpathlineto{\pgfqpoint{0.000000in}{1.830000in}}%
\pgfpathlineto{\pgfqpoint{0.000000in}{0.000000in}}%
\pgfpathclose%
\pgfusepath{fill}%
\end{pgfscope}%
\begin{pgfscope}%
\pgfsetbuttcap%
\pgfsetmiterjoin%
\definecolor{currentfill}{rgb}{1.000000,1.000000,1.000000}%
\pgfsetfillcolor{currentfill}%
\pgfsetlinewidth{0.000000pt}%
\definecolor{currentstroke}{rgb}{0.000000,0.000000,0.000000}%
\pgfsetstrokecolor{currentstroke}%
\pgfsetstrokeopacity{0.000000}%
\pgfsetdash{}{0pt}%
\pgfpathmoveto{\pgfqpoint{0.514278in}{0.417642in}}%
\pgfpathlineto{\pgfqpoint{2.398330in}{0.417642in}}%
\pgfpathlineto{\pgfqpoint{2.398330in}{1.788330in}}%
\pgfpathlineto{\pgfqpoint{0.514278in}{1.788330in}}%
\pgfpathlineto{\pgfqpoint{0.514278in}{0.417642in}}%
\pgfpathclose%
\pgfusepath{fill}%
\end{pgfscope}%
\begin{pgfscope}%
\pgfpathrectangle{\pgfqpoint{0.514278in}{0.417642in}}{\pgfqpoint{1.884052in}{1.370688in}}%
\pgfusepath{clip}%
\pgfsetrectcap%
\pgfsetroundjoin%
\pgfsetlinewidth{0.803000pt}%
\definecolor{currentstroke}{rgb}{0.450000,0.450000,0.450000}%
\pgfsetstrokecolor{currentstroke}%
\pgfsetdash{}{0pt}%
\pgfpathmoveto{\pgfqpoint{0.916624in}{0.417642in}}%
\pgfpathlineto{\pgfqpoint{0.916624in}{1.788330in}}%
\pgfusepath{stroke}%
\end{pgfscope}%
\begin{pgfscope}%
\pgfsetbuttcap%
\pgfsetroundjoin%
\definecolor{currentfill}{rgb}{0.000000,0.000000,0.000000}%
\pgfsetfillcolor{currentfill}%
\pgfsetlinewidth{0.803000pt}%
\definecolor{currentstroke}{rgb}{0.000000,0.000000,0.000000}%
\pgfsetstrokecolor{currentstroke}%
\pgfsetdash{}{0pt}%
\pgfsys@defobject{currentmarker}{\pgfqpoint{0.000000in}{-0.048611in}}{\pgfqpoint{0.000000in}{0.000000in}}{%
\pgfpathmoveto{\pgfqpoint{0.000000in}{0.000000in}}%
\pgfpathlineto{\pgfqpoint{0.000000in}{-0.048611in}}%
\pgfusepath{stroke,fill}%
}%
\begin{pgfscope}%
\pgfsys@transformshift{0.916624in}{0.417642in}%
\pgfsys@useobject{currentmarker}{}%
\end{pgfscope}%
\end{pgfscope}%
\begin{pgfscope}%
\definecolor{textcolor}{rgb}{0.000000,0.000000,0.000000}%
\pgfsetstrokecolor{textcolor}%
\pgfsetfillcolor{textcolor}%
\pgftext[x=0.916624in,y=0.320420in,,top]{\color{textcolor}\rmfamily\fontsize{8.000000}{9.600000}\selectfont \(\displaystyle {10^{-3}}\)}%
\end{pgfscope}%
\begin{pgfscope}%
\pgfpathrectangle{\pgfqpoint{0.514278in}{0.417642in}}{\pgfqpoint{1.884052in}{1.370688in}}%
\pgfusepath{clip}%
\pgfsetrectcap%
\pgfsetroundjoin%
\pgfsetlinewidth{0.803000pt}%
\definecolor{currentstroke}{rgb}{0.450000,0.450000,0.450000}%
\pgfsetstrokecolor{currentstroke}%
\pgfsetdash{}{0pt}%
\pgfpathmoveto{\pgfqpoint{1.433903in}{0.417642in}}%
\pgfpathlineto{\pgfqpoint{1.433903in}{1.788330in}}%
\pgfusepath{stroke}%
\end{pgfscope}%
\begin{pgfscope}%
\pgfsetbuttcap%
\pgfsetroundjoin%
\definecolor{currentfill}{rgb}{0.000000,0.000000,0.000000}%
\pgfsetfillcolor{currentfill}%
\pgfsetlinewidth{0.803000pt}%
\definecolor{currentstroke}{rgb}{0.000000,0.000000,0.000000}%
\pgfsetstrokecolor{currentstroke}%
\pgfsetdash{}{0pt}%
\pgfsys@defobject{currentmarker}{\pgfqpoint{0.000000in}{-0.048611in}}{\pgfqpoint{0.000000in}{0.000000in}}{%
\pgfpathmoveto{\pgfqpoint{0.000000in}{0.000000in}}%
\pgfpathlineto{\pgfqpoint{0.000000in}{-0.048611in}}%
\pgfusepath{stroke,fill}%
}%
\begin{pgfscope}%
\pgfsys@transformshift{1.433903in}{0.417642in}%
\pgfsys@useobject{currentmarker}{}%
\end{pgfscope}%
\end{pgfscope}%
\begin{pgfscope}%
\definecolor{textcolor}{rgb}{0.000000,0.000000,0.000000}%
\pgfsetstrokecolor{textcolor}%
\pgfsetfillcolor{textcolor}%
\pgftext[x=1.433903in,y=0.320420in,,top]{\color{textcolor}\rmfamily\fontsize{8.000000}{9.600000}\selectfont \(\displaystyle {10^{-2}}\)}%
\end{pgfscope}%
\begin{pgfscope}%
\pgfpathrectangle{\pgfqpoint{0.514278in}{0.417642in}}{\pgfqpoint{1.884052in}{1.370688in}}%
\pgfusepath{clip}%
\pgfsetrectcap%
\pgfsetroundjoin%
\pgfsetlinewidth{0.803000pt}%
\definecolor{currentstroke}{rgb}{0.450000,0.450000,0.450000}%
\pgfsetstrokecolor{currentstroke}%
\pgfsetdash{}{0pt}%
\pgfpathmoveto{\pgfqpoint{1.951183in}{0.417642in}}%
\pgfpathlineto{\pgfqpoint{1.951183in}{1.788330in}}%
\pgfusepath{stroke}%
\end{pgfscope}%
\begin{pgfscope}%
\pgfsetbuttcap%
\pgfsetroundjoin%
\definecolor{currentfill}{rgb}{0.000000,0.000000,0.000000}%
\pgfsetfillcolor{currentfill}%
\pgfsetlinewidth{0.803000pt}%
\definecolor{currentstroke}{rgb}{0.000000,0.000000,0.000000}%
\pgfsetstrokecolor{currentstroke}%
\pgfsetdash{}{0pt}%
\pgfsys@defobject{currentmarker}{\pgfqpoint{0.000000in}{-0.048611in}}{\pgfqpoint{0.000000in}{0.000000in}}{%
\pgfpathmoveto{\pgfqpoint{0.000000in}{0.000000in}}%
\pgfpathlineto{\pgfqpoint{0.000000in}{-0.048611in}}%
\pgfusepath{stroke,fill}%
}%
\begin{pgfscope}%
\pgfsys@transformshift{1.951183in}{0.417642in}%
\pgfsys@useobject{currentmarker}{}%
\end{pgfscope}%
\end{pgfscope}%
\begin{pgfscope}%
\definecolor{textcolor}{rgb}{0.000000,0.000000,0.000000}%
\pgfsetstrokecolor{textcolor}%
\pgfsetfillcolor{textcolor}%
\pgftext[x=1.951183in,y=0.320420in,,top]{\color{textcolor}\rmfamily\fontsize{8.000000}{9.600000}\selectfont \(\displaystyle {10^{-1}}\)}%
\end{pgfscope}%
\begin{pgfscope}%
\pgfpathrectangle{\pgfqpoint{0.514278in}{0.417642in}}{\pgfqpoint{1.884052in}{1.370688in}}%
\pgfusepath{clip}%
\pgfsetrectcap%
\pgfsetroundjoin%
\pgfsetlinewidth{0.803000pt}%
\definecolor{currentstroke}{rgb}{0.850000,0.850000,0.850000}%
\pgfsetstrokecolor{currentstroke}%
\pgfsetdash{}{0pt}%
\pgfpathmoveto{\pgfqpoint{0.555061in}{0.417642in}}%
\pgfpathlineto{\pgfqpoint{0.555061in}{1.788330in}}%
\pgfusepath{stroke}%
\end{pgfscope}%
\begin{pgfscope}%
\pgfsetbuttcap%
\pgfsetroundjoin%
\definecolor{currentfill}{rgb}{0.000000,0.000000,0.000000}%
\pgfsetfillcolor{currentfill}%
\pgfsetlinewidth{0.602250pt}%
\definecolor{currentstroke}{rgb}{0.000000,0.000000,0.000000}%
\pgfsetstrokecolor{currentstroke}%
\pgfsetdash{}{0pt}%
\pgfsys@defobject{currentmarker}{\pgfqpoint{0.000000in}{-0.027778in}}{\pgfqpoint{0.000000in}{0.000000in}}{%
\pgfpathmoveto{\pgfqpoint{0.000000in}{0.000000in}}%
\pgfpathlineto{\pgfqpoint{0.000000in}{-0.027778in}}%
\pgfusepath{stroke,fill}%
}%
\begin{pgfscope}%
\pgfsys@transformshift{0.555061in}{0.417642in}%
\pgfsys@useobject{currentmarker}{}%
\end{pgfscope}%
\end{pgfscope}%
\begin{pgfscope}%
\pgfpathrectangle{\pgfqpoint{0.514278in}{0.417642in}}{\pgfqpoint{1.884052in}{1.370688in}}%
\pgfusepath{clip}%
\pgfsetrectcap%
\pgfsetroundjoin%
\pgfsetlinewidth{0.803000pt}%
\definecolor{currentstroke}{rgb}{0.850000,0.850000,0.850000}%
\pgfsetstrokecolor{currentstroke}%
\pgfsetdash{}{0pt}%
\pgfpathmoveto{\pgfqpoint{0.646149in}{0.417642in}}%
\pgfpathlineto{\pgfqpoint{0.646149in}{1.788330in}}%
\pgfusepath{stroke}%
\end{pgfscope}%
\begin{pgfscope}%
\pgfsetbuttcap%
\pgfsetroundjoin%
\definecolor{currentfill}{rgb}{0.000000,0.000000,0.000000}%
\pgfsetfillcolor{currentfill}%
\pgfsetlinewidth{0.602250pt}%
\definecolor{currentstroke}{rgb}{0.000000,0.000000,0.000000}%
\pgfsetstrokecolor{currentstroke}%
\pgfsetdash{}{0pt}%
\pgfsys@defobject{currentmarker}{\pgfqpoint{0.000000in}{-0.027778in}}{\pgfqpoint{0.000000in}{0.000000in}}{%
\pgfpathmoveto{\pgfqpoint{0.000000in}{0.000000in}}%
\pgfpathlineto{\pgfqpoint{0.000000in}{-0.027778in}}%
\pgfusepath{stroke,fill}%
}%
\begin{pgfscope}%
\pgfsys@transformshift{0.646149in}{0.417642in}%
\pgfsys@useobject{currentmarker}{}%
\end{pgfscope}%
\end{pgfscope}%
\begin{pgfscope}%
\pgfpathrectangle{\pgfqpoint{0.514278in}{0.417642in}}{\pgfqpoint{1.884052in}{1.370688in}}%
\pgfusepath{clip}%
\pgfsetrectcap%
\pgfsetroundjoin%
\pgfsetlinewidth{0.803000pt}%
\definecolor{currentstroke}{rgb}{0.850000,0.850000,0.850000}%
\pgfsetstrokecolor{currentstroke}%
\pgfsetdash{}{0pt}%
\pgfpathmoveto{\pgfqpoint{0.710777in}{0.417642in}}%
\pgfpathlineto{\pgfqpoint{0.710777in}{1.788330in}}%
\pgfusepath{stroke}%
\end{pgfscope}%
\begin{pgfscope}%
\pgfsetbuttcap%
\pgfsetroundjoin%
\definecolor{currentfill}{rgb}{0.000000,0.000000,0.000000}%
\pgfsetfillcolor{currentfill}%
\pgfsetlinewidth{0.602250pt}%
\definecolor{currentstroke}{rgb}{0.000000,0.000000,0.000000}%
\pgfsetstrokecolor{currentstroke}%
\pgfsetdash{}{0pt}%
\pgfsys@defobject{currentmarker}{\pgfqpoint{0.000000in}{-0.027778in}}{\pgfqpoint{0.000000in}{0.000000in}}{%
\pgfpathmoveto{\pgfqpoint{0.000000in}{0.000000in}}%
\pgfpathlineto{\pgfqpoint{0.000000in}{-0.027778in}}%
\pgfusepath{stroke,fill}%
}%
\begin{pgfscope}%
\pgfsys@transformshift{0.710777in}{0.417642in}%
\pgfsys@useobject{currentmarker}{}%
\end{pgfscope}%
\end{pgfscope}%
\begin{pgfscope}%
\pgfpathrectangle{\pgfqpoint{0.514278in}{0.417642in}}{\pgfqpoint{1.884052in}{1.370688in}}%
\pgfusepath{clip}%
\pgfsetrectcap%
\pgfsetroundjoin%
\pgfsetlinewidth{0.803000pt}%
\definecolor{currentstroke}{rgb}{0.850000,0.850000,0.850000}%
\pgfsetstrokecolor{currentstroke}%
\pgfsetdash{}{0pt}%
\pgfpathmoveto{\pgfqpoint{0.760907in}{0.417642in}}%
\pgfpathlineto{\pgfqpoint{0.760907in}{1.788330in}}%
\pgfusepath{stroke}%
\end{pgfscope}%
\begin{pgfscope}%
\pgfsetbuttcap%
\pgfsetroundjoin%
\definecolor{currentfill}{rgb}{0.000000,0.000000,0.000000}%
\pgfsetfillcolor{currentfill}%
\pgfsetlinewidth{0.602250pt}%
\definecolor{currentstroke}{rgb}{0.000000,0.000000,0.000000}%
\pgfsetstrokecolor{currentstroke}%
\pgfsetdash{}{0pt}%
\pgfsys@defobject{currentmarker}{\pgfqpoint{0.000000in}{-0.027778in}}{\pgfqpoint{0.000000in}{0.000000in}}{%
\pgfpathmoveto{\pgfqpoint{0.000000in}{0.000000in}}%
\pgfpathlineto{\pgfqpoint{0.000000in}{-0.027778in}}%
\pgfusepath{stroke,fill}%
}%
\begin{pgfscope}%
\pgfsys@transformshift{0.760907in}{0.417642in}%
\pgfsys@useobject{currentmarker}{}%
\end{pgfscope}%
\end{pgfscope}%
\begin{pgfscope}%
\pgfpathrectangle{\pgfqpoint{0.514278in}{0.417642in}}{\pgfqpoint{1.884052in}{1.370688in}}%
\pgfusepath{clip}%
\pgfsetrectcap%
\pgfsetroundjoin%
\pgfsetlinewidth{0.803000pt}%
\definecolor{currentstroke}{rgb}{0.850000,0.850000,0.850000}%
\pgfsetstrokecolor{currentstroke}%
\pgfsetdash{}{0pt}%
\pgfpathmoveto{\pgfqpoint{0.801866in}{0.417642in}}%
\pgfpathlineto{\pgfqpoint{0.801866in}{1.788330in}}%
\pgfusepath{stroke}%
\end{pgfscope}%
\begin{pgfscope}%
\pgfsetbuttcap%
\pgfsetroundjoin%
\definecolor{currentfill}{rgb}{0.000000,0.000000,0.000000}%
\pgfsetfillcolor{currentfill}%
\pgfsetlinewidth{0.602250pt}%
\definecolor{currentstroke}{rgb}{0.000000,0.000000,0.000000}%
\pgfsetstrokecolor{currentstroke}%
\pgfsetdash{}{0pt}%
\pgfsys@defobject{currentmarker}{\pgfqpoint{0.000000in}{-0.027778in}}{\pgfqpoint{0.000000in}{0.000000in}}{%
\pgfpathmoveto{\pgfqpoint{0.000000in}{0.000000in}}%
\pgfpathlineto{\pgfqpoint{0.000000in}{-0.027778in}}%
\pgfusepath{stroke,fill}%
}%
\begin{pgfscope}%
\pgfsys@transformshift{0.801866in}{0.417642in}%
\pgfsys@useobject{currentmarker}{}%
\end{pgfscope}%
\end{pgfscope}%
\begin{pgfscope}%
\pgfpathrectangle{\pgfqpoint{0.514278in}{0.417642in}}{\pgfqpoint{1.884052in}{1.370688in}}%
\pgfusepath{clip}%
\pgfsetrectcap%
\pgfsetroundjoin%
\pgfsetlinewidth{0.803000pt}%
\definecolor{currentstroke}{rgb}{0.850000,0.850000,0.850000}%
\pgfsetstrokecolor{currentstroke}%
\pgfsetdash{}{0pt}%
\pgfpathmoveto{\pgfqpoint{0.836496in}{0.417642in}}%
\pgfpathlineto{\pgfqpoint{0.836496in}{1.788330in}}%
\pgfusepath{stroke}%
\end{pgfscope}%
\begin{pgfscope}%
\pgfsetbuttcap%
\pgfsetroundjoin%
\definecolor{currentfill}{rgb}{0.000000,0.000000,0.000000}%
\pgfsetfillcolor{currentfill}%
\pgfsetlinewidth{0.602250pt}%
\definecolor{currentstroke}{rgb}{0.000000,0.000000,0.000000}%
\pgfsetstrokecolor{currentstroke}%
\pgfsetdash{}{0pt}%
\pgfsys@defobject{currentmarker}{\pgfqpoint{0.000000in}{-0.027778in}}{\pgfqpoint{0.000000in}{0.000000in}}{%
\pgfpathmoveto{\pgfqpoint{0.000000in}{0.000000in}}%
\pgfpathlineto{\pgfqpoint{0.000000in}{-0.027778in}}%
\pgfusepath{stroke,fill}%
}%
\begin{pgfscope}%
\pgfsys@transformshift{0.836496in}{0.417642in}%
\pgfsys@useobject{currentmarker}{}%
\end{pgfscope}%
\end{pgfscope}%
\begin{pgfscope}%
\pgfpathrectangle{\pgfqpoint{0.514278in}{0.417642in}}{\pgfqpoint{1.884052in}{1.370688in}}%
\pgfusepath{clip}%
\pgfsetrectcap%
\pgfsetroundjoin%
\pgfsetlinewidth{0.803000pt}%
\definecolor{currentstroke}{rgb}{0.850000,0.850000,0.850000}%
\pgfsetstrokecolor{currentstroke}%
\pgfsetdash{}{0pt}%
\pgfpathmoveto{\pgfqpoint{0.866494in}{0.417642in}}%
\pgfpathlineto{\pgfqpoint{0.866494in}{1.788330in}}%
\pgfusepath{stroke}%
\end{pgfscope}%
\begin{pgfscope}%
\pgfsetbuttcap%
\pgfsetroundjoin%
\definecolor{currentfill}{rgb}{0.000000,0.000000,0.000000}%
\pgfsetfillcolor{currentfill}%
\pgfsetlinewidth{0.602250pt}%
\definecolor{currentstroke}{rgb}{0.000000,0.000000,0.000000}%
\pgfsetstrokecolor{currentstroke}%
\pgfsetdash{}{0pt}%
\pgfsys@defobject{currentmarker}{\pgfqpoint{0.000000in}{-0.027778in}}{\pgfqpoint{0.000000in}{0.000000in}}{%
\pgfpathmoveto{\pgfqpoint{0.000000in}{0.000000in}}%
\pgfpathlineto{\pgfqpoint{0.000000in}{-0.027778in}}%
\pgfusepath{stroke,fill}%
}%
\begin{pgfscope}%
\pgfsys@transformshift{0.866494in}{0.417642in}%
\pgfsys@useobject{currentmarker}{}%
\end{pgfscope}%
\end{pgfscope}%
\begin{pgfscope}%
\pgfpathrectangle{\pgfqpoint{0.514278in}{0.417642in}}{\pgfqpoint{1.884052in}{1.370688in}}%
\pgfusepath{clip}%
\pgfsetrectcap%
\pgfsetroundjoin%
\pgfsetlinewidth{0.803000pt}%
\definecolor{currentstroke}{rgb}{0.850000,0.850000,0.850000}%
\pgfsetstrokecolor{currentstroke}%
\pgfsetdash{}{0pt}%
\pgfpathmoveto{\pgfqpoint{0.892954in}{0.417642in}}%
\pgfpathlineto{\pgfqpoint{0.892954in}{1.788330in}}%
\pgfusepath{stroke}%
\end{pgfscope}%
\begin{pgfscope}%
\pgfsetbuttcap%
\pgfsetroundjoin%
\definecolor{currentfill}{rgb}{0.000000,0.000000,0.000000}%
\pgfsetfillcolor{currentfill}%
\pgfsetlinewidth{0.602250pt}%
\definecolor{currentstroke}{rgb}{0.000000,0.000000,0.000000}%
\pgfsetstrokecolor{currentstroke}%
\pgfsetdash{}{0pt}%
\pgfsys@defobject{currentmarker}{\pgfqpoint{0.000000in}{-0.027778in}}{\pgfqpoint{0.000000in}{0.000000in}}{%
\pgfpathmoveto{\pgfqpoint{0.000000in}{0.000000in}}%
\pgfpathlineto{\pgfqpoint{0.000000in}{-0.027778in}}%
\pgfusepath{stroke,fill}%
}%
\begin{pgfscope}%
\pgfsys@transformshift{0.892954in}{0.417642in}%
\pgfsys@useobject{currentmarker}{}%
\end{pgfscope}%
\end{pgfscope}%
\begin{pgfscope}%
\pgfpathrectangle{\pgfqpoint{0.514278in}{0.417642in}}{\pgfqpoint{1.884052in}{1.370688in}}%
\pgfusepath{clip}%
\pgfsetrectcap%
\pgfsetroundjoin%
\pgfsetlinewidth{0.803000pt}%
\definecolor{currentstroke}{rgb}{0.850000,0.850000,0.850000}%
\pgfsetstrokecolor{currentstroke}%
\pgfsetdash{}{0pt}%
\pgfpathmoveto{\pgfqpoint{1.072340in}{0.417642in}}%
\pgfpathlineto{\pgfqpoint{1.072340in}{1.788330in}}%
\pgfusepath{stroke}%
\end{pgfscope}%
\begin{pgfscope}%
\pgfsetbuttcap%
\pgfsetroundjoin%
\definecolor{currentfill}{rgb}{0.000000,0.000000,0.000000}%
\pgfsetfillcolor{currentfill}%
\pgfsetlinewidth{0.602250pt}%
\definecolor{currentstroke}{rgb}{0.000000,0.000000,0.000000}%
\pgfsetstrokecolor{currentstroke}%
\pgfsetdash{}{0pt}%
\pgfsys@defobject{currentmarker}{\pgfqpoint{0.000000in}{-0.027778in}}{\pgfqpoint{0.000000in}{0.000000in}}{%
\pgfpathmoveto{\pgfqpoint{0.000000in}{0.000000in}}%
\pgfpathlineto{\pgfqpoint{0.000000in}{-0.027778in}}%
\pgfusepath{stroke,fill}%
}%
\begin{pgfscope}%
\pgfsys@transformshift{1.072340in}{0.417642in}%
\pgfsys@useobject{currentmarker}{}%
\end{pgfscope}%
\end{pgfscope}%
\begin{pgfscope}%
\pgfpathrectangle{\pgfqpoint{0.514278in}{0.417642in}}{\pgfqpoint{1.884052in}{1.370688in}}%
\pgfusepath{clip}%
\pgfsetrectcap%
\pgfsetroundjoin%
\pgfsetlinewidth{0.803000pt}%
\definecolor{currentstroke}{rgb}{0.850000,0.850000,0.850000}%
\pgfsetstrokecolor{currentstroke}%
\pgfsetdash{}{0pt}%
\pgfpathmoveto{\pgfqpoint{1.163429in}{0.417642in}}%
\pgfpathlineto{\pgfqpoint{1.163429in}{1.788330in}}%
\pgfusepath{stroke}%
\end{pgfscope}%
\begin{pgfscope}%
\pgfsetbuttcap%
\pgfsetroundjoin%
\definecolor{currentfill}{rgb}{0.000000,0.000000,0.000000}%
\pgfsetfillcolor{currentfill}%
\pgfsetlinewidth{0.602250pt}%
\definecolor{currentstroke}{rgb}{0.000000,0.000000,0.000000}%
\pgfsetstrokecolor{currentstroke}%
\pgfsetdash{}{0pt}%
\pgfsys@defobject{currentmarker}{\pgfqpoint{0.000000in}{-0.027778in}}{\pgfqpoint{0.000000in}{0.000000in}}{%
\pgfpathmoveto{\pgfqpoint{0.000000in}{0.000000in}}%
\pgfpathlineto{\pgfqpoint{0.000000in}{-0.027778in}}%
\pgfusepath{stroke,fill}%
}%
\begin{pgfscope}%
\pgfsys@transformshift{1.163429in}{0.417642in}%
\pgfsys@useobject{currentmarker}{}%
\end{pgfscope}%
\end{pgfscope}%
\begin{pgfscope}%
\pgfpathrectangle{\pgfqpoint{0.514278in}{0.417642in}}{\pgfqpoint{1.884052in}{1.370688in}}%
\pgfusepath{clip}%
\pgfsetrectcap%
\pgfsetroundjoin%
\pgfsetlinewidth{0.803000pt}%
\definecolor{currentstroke}{rgb}{0.850000,0.850000,0.850000}%
\pgfsetstrokecolor{currentstroke}%
\pgfsetdash{}{0pt}%
\pgfpathmoveto{\pgfqpoint{1.228057in}{0.417642in}}%
\pgfpathlineto{\pgfqpoint{1.228057in}{1.788330in}}%
\pgfusepath{stroke}%
\end{pgfscope}%
\begin{pgfscope}%
\pgfsetbuttcap%
\pgfsetroundjoin%
\definecolor{currentfill}{rgb}{0.000000,0.000000,0.000000}%
\pgfsetfillcolor{currentfill}%
\pgfsetlinewidth{0.602250pt}%
\definecolor{currentstroke}{rgb}{0.000000,0.000000,0.000000}%
\pgfsetstrokecolor{currentstroke}%
\pgfsetdash{}{0pt}%
\pgfsys@defobject{currentmarker}{\pgfqpoint{0.000000in}{-0.027778in}}{\pgfqpoint{0.000000in}{0.000000in}}{%
\pgfpathmoveto{\pgfqpoint{0.000000in}{0.000000in}}%
\pgfpathlineto{\pgfqpoint{0.000000in}{-0.027778in}}%
\pgfusepath{stroke,fill}%
}%
\begin{pgfscope}%
\pgfsys@transformshift{1.228057in}{0.417642in}%
\pgfsys@useobject{currentmarker}{}%
\end{pgfscope}%
\end{pgfscope}%
\begin{pgfscope}%
\pgfpathrectangle{\pgfqpoint{0.514278in}{0.417642in}}{\pgfqpoint{1.884052in}{1.370688in}}%
\pgfusepath{clip}%
\pgfsetrectcap%
\pgfsetroundjoin%
\pgfsetlinewidth{0.803000pt}%
\definecolor{currentstroke}{rgb}{0.850000,0.850000,0.850000}%
\pgfsetstrokecolor{currentstroke}%
\pgfsetdash{}{0pt}%
\pgfpathmoveto{\pgfqpoint{1.278187in}{0.417642in}}%
\pgfpathlineto{\pgfqpoint{1.278187in}{1.788330in}}%
\pgfusepath{stroke}%
\end{pgfscope}%
\begin{pgfscope}%
\pgfsetbuttcap%
\pgfsetroundjoin%
\definecolor{currentfill}{rgb}{0.000000,0.000000,0.000000}%
\pgfsetfillcolor{currentfill}%
\pgfsetlinewidth{0.602250pt}%
\definecolor{currentstroke}{rgb}{0.000000,0.000000,0.000000}%
\pgfsetstrokecolor{currentstroke}%
\pgfsetdash{}{0pt}%
\pgfsys@defobject{currentmarker}{\pgfqpoint{0.000000in}{-0.027778in}}{\pgfqpoint{0.000000in}{0.000000in}}{%
\pgfpathmoveto{\pgfqpoint{0.000000in}{0.000000in}}%
\pgfpathlineto{\pgfqpoint{0.000000in}{-0.027778in}}%
\pgfusepath{stroke,fill}%
}%
\begin{pgfscope}%
\pgfsys@transformshift{1.278187in}{0.417642in}%
\pgfsys@useobject{currentmarker}{}%
\end{pgfscope}%
\end{pgfscope}%
\begin{pgfscope}%
\pgfpathrectangle{\pgfqpoint{0.514278in}{0.417642in}}{\pgfqpoint{1.884052in}{1.370688in}}%
\pgfusepath{clip}%
\pgfsetrectcap%
\pgfsetroundjoin%
\pgfsetlinewidth{0.803000pt}%
\definecolor{currentstroke}{rgb}{0.850000,0.850000,0.850000}%
\pgfsetstrokecolor{currentstroke}%
\pgfsetdash{}{0pt}%
\pgfpathmoveto{\pgfqpoint{1.319146in}{0.417642in}}%
\pgfpathlineto{\pgfqpoint{1.319146in}{1.788330in}}%
\pgfusepath{stroke}%
\end{pgfscope}%
\begin{pgfscope}%
\pgfsetbuttcap%
\pgfsetroundjoin%
\definecolor{currentfill}{rgb}{0.000000,0.000000,0.000000}%
\pgfsetfillcolor{currentfill}%
\pgfsetlinewidth{0.602250pt}%
\definecolor{currentstroke}{rgb}{0.000000,0.000000,0.000000}%
\pgfsetstrokecolor{currentstroke}%
\pgfsetdash{}{0pt}%
\pgfsys@defobject{currentmarker}{\pgfqpoint{0.000000in}{-0.027778in}}{\pgfqpoint{0.000000in}{0.000000in}}{%
\pgfpathmoveto{\pgfqpoint{0.000000in}{0.000000in}}%
\pgfpathlineto{\pgfqpoint{0.000000in}{-0.027778in}}%
\pgfusepath{stroke,fill}%
}%
\begin{pgfscope}%
\pgfsys@transformshift{1.319146in}{0.417642in}%
\pgfsys@useobject{currentmarker}{}%
\end{pgfscope}%
\end{pgfscope}%
\begin{pgfscope}%
\pgfpathrectangle{\pgfqpoint{0.514278in}{0.417642in}}{\pgfqpoint{1.884052in}{1.370688in}}%
\pgfusepath{clip}%
\pgfsetrectcap%
\pgfsetroundjoin%
\pgfsetlinewidth{0.803000pt}%
\definecolor{currentstroke}{rgb}{0.850000,0.850000,0.850000}%
\pgfsetstrokecolor{currentstroke}%
\pgfsetdash{}{0pt}%
\pgfpathmoveto{\pgfqpoint{1.353776in}{0.417642in}}%
\pgfpathlineto{\pgfqpoint{1.353776in}{1.788330in}}%
\pgfusepath{stroke}%
\end{pgfscope}%
\begin{pgfscope}%
\pgfsetbuttcap%
\pgfsetroundjoin%
\definecolor{currentfill}{rgb}{0.000000,0.000000,0.000000}%
\pgfsetfillcolor{currentfill}%
\pgfsetlinewidth{0.602250pt}%
\definecolor{currentstroke}{rgb}{0.000000,0.000000,0.000000}%
\pgfsetstrokecolor{currentstroke}%
\pgfsetdash{}{0pt}%
\pgfsys@defobject{currentmarker}{\pgfqpoint{0.000000in}{-0.027778in}}{\pgfqpoint{0.000000in}{0.000000in}}{%
\pgfpathmoveto{\pgfqpoint{0.000000in}{0.000000in}}%
\pgfpathlineto{\pgfqpoint{0.000000in}{-0.027778in}}%
\pgfusepath{stroke,fill}%
}%
\begin{pgfscope}%
\pgfsys@transformshift{1.353776in}{0.417642in}%
\pgfsys@useobject{currentmarker}{}%
\end{pgfscope}%
\end{pgfscope}%
\begin{pgfscope}%
\pgfpathrectangle{\pgfqpoint{0.514278in}{0.417642in}}{\pgfqpoint{1.884052in}{1.370688in}}%
\pgfusepath{clip}%
\pgfsetrectcap%
\pgfsetroundjoin%
\pgfsetlinewidth{0.803000pt}%
\definecolor{currentstroke}{rgb}{0.850000,0.850000,0.850000}%
\pgfsetstrokecolor{currentstroke}%
\pgfsetdash{}{0pt}%
\pgfpathmoveto{\pgfqpoint{1.383774in}{0.417642in}}%
\pgfpathlineto{\pgfqpoint{1.383774in}{1.788330in}}%
\pgfusepath{stroke}%
\end{pgfscope}%
\begin{pgfscope}%
\pgfsetbuttcap%
\pgfsetroundjoin%
\definecolor{currentfill}{rgb}{0.000000,0.000000,0.000000}%
\pgfsetfillcolor{currentfill}%
\pgfsetlinewidth{0.602250pt}%
\definecolor{currentstroke}{rgb}{0.000000,0.000000,0.000000}%
\pgfsetstrokecolor{currentstroke}%
\pgfsetdash{}{0pt}%
\pgfsys@defobject{currentmarker}{\pgfqpoint{0.000000in}{-0.027778in}}{\pgfqpoint{0.000000in}{0.000000in}}{%
\pgfpathmoveto{\pgfqpoint{0.000000in}{0.000000in}}%
\pgfpathlineto{\pgfqpoint{0.000000in}{-0.027778in}}%
\pgfusepath{stroke,fill}%
}%
\begin{pgfscope}%
\pgfsys@transformshift{1.383774in}{0.417642in}%
\pgfsys@useobject{currentmarker}{}%
\end{pgfscope}%
\end{pgfscope}%
\begin{pgfscope}%
\pgfpathrectangle{\pgfqpoint{0.514278in}{0.417642in}}{\pgfqpoint{1.884052in}{1.370688in}}%
\pgfusepath{clip}%
\pgfsetrectcap%
\pgfsetroundjoin%
\pgfsetlinewidth{0.803000pt}%
\definecolor{currentstroke}{rgb}{0.850000,0.850000,0.850000}%
\pgfsetstrokecolor{currentstroke}%
\pgfsetdash{}{0pt}%
\pgfpathmoveto{\pgfqpoint{1.410234in}{0.417642in}}%
\pgfpathlineto{\pgfqpoint{1.410234in}{1.788330in}}%
\pgfusepath{stroke}%
\end{pgfscope}%
\begin{pgfscope}%
\pgfsetbuttcap%
\pgfsetroundjoin%
\definecolor{currentfill}{rgb}{0.000000,0.000000,0.000000}%
\pgfsetfillcolor{currentfill}%
\pgfsetlinewidth{0.602250pt}%
\definecolor{currentstroke}{rgb}{0.000000,0.000000,0.000000}%
\pgfsetstrokecolor{currentstroke}%
\pgfsetdash{}{0pt}%
\pgfsys@defobject{currentmarker}{\pgfqpoint{0.000000in}{-0.027778in}}{\pgfqpoint{0.000000in}{0.000000in}}{%
\pgfpathmoveto{\pgfqpoint{0.000000in}{0.000000in}}%
\pgfpathlineto{\pgfqpoint{0.000000in}{-0.027778in}}%
\pgfusepath{stroke,fill}%
}%
\begin{pgfscope}%
\pgfsys@transformshift{1.410234in}{0.417642in}%
\pgfsys@useobject{currentmarker}{}%
\end{pgfscope}%
\end{pgfscope}%
\begin{pgfscope}%
\pgfpathrectangle{\pgfqpoint{0.514278in}{0.417642in}}{\pgfqpoint{1.884052in}{1.370688in}}%
\pgfusepath{clip}%
\pgfsetrectcap%
\pgfsetroundjoin%
\pgfsetlinewidth{0.803000pt}%
\definecolor{currentstroke}{rgb}{0.850000,0.850000,0.850000}%
\pgfsetstrokecolor{currentstroke}%
\pgfsetdash{}{0pt}%
\pgfpathmoveto{\pgfqpoint{1.589620in}{0.417642in}}%
\pgfpathlineto{\pgfqpoint{1.589620in}{1.788330in}}%
\pgfusepath{stroke}%
\end{pgfscope}%
\begin{pgfscope}%
\pgfsetbuttcap%
\pgfsetroundjoin%
\definecolor{currentfill}{rgb}{0.000000,0.000000,0.000000}%
\pgfsetfillcolor{currentfill}%
\pgfsetlinewidth{0.602250pt}%
\definecolor{currentstroke}{rgb}{0.000000,0.000000,0.000000}%
\pgfsetstrokecolor{currentstroke}%
\pgfsetdash{}{0pt}%
\pgfsys@defobject{currentmarker}{\pgfqpoint{0.000000in}{-0.027778in}}{\pgfqpoint{0.000000in}{0.000000in}}{%
\pgfpathmoveto{\pgfqpoint{0.000000in}{0.000000in}}%
\pgfpathlineto{\pgfqpoint{0.000000in}{-0.027778in}}%
\pgfusepath{stroke,fill}%
}%
\begin{pgfscope}%
\pgfsys@transformshift{1.589620in}{0.417642in}%
\pgfsys@useobject{currentmarker}{}%
\end{pgfscope}%
\end{pgfscope}%
\begin{pgfscope}%
\pgfpathrectangle{\pgfqpoint{0.514278in}{0.417642in}}{\pgfqpoint{1.884052in}{1.370688in}}%
\pgfusepath{clip}%
\pgfsetrectcap%
\pgfsetroundjoin%
\pgfsetlinewidth{0.803000pt}%
\definecolor{currentstroke}{rgb}{0.850000,0.850000,0.850000}%
\pgfsetstrokecolor{currentstroke}%
\pgfsetdash{}{0pt}%
\pgfpathmoveto{\pgfqpoint{1.680709in}{0.417642in}}%
\pgfpathlineto{\pgfqpoint{1.680709in}{1.788330in}}%
\pgfusepath{stroke}%
\end{pgfscope}%
\begin{pgfscope}%
\pgfsetbuttcap%
\pgfsetroundjoin%
\definecolor{currentfill}{rgb}{0.000000,0.000000,0.000000}%
\pgfsetfillcolor{currentfill}%
\pgfsetlinewidth{0.602250pt}%
\definecolor{currentstroke}{rgb}{0.000000,0.000000,0.000000}%
\pgfsetstrokecolor{currentstroke}%
\pgfsetdash{}{0pt}%
\pgfsys@defobject{currentmarker}{\pgfqpoint{0.000000in}{-0.027778in}}{\pgfqpoint{0.000000in}{0.000000in}}{%
\pgfpathmoveto{\pgfqpoint{0.000000in}{0.000000in}}%
\pgfpathlineto{\pgfqpoint{0.000000in}{-0.027778in}}%
\pgfusepath{stroke,fill}%
}%
\begin{pgfscope}%
\pgfsys@transformshift{1.680709in}{0.417642in}%
\pgfsys@useobject{currentmarker}{}%
\end{pgfscope}%
\end{pgfscope}%
\begin{pgfscope}%
\pgfpathrectangle{\pgfqpoint{0.514278in}{0.417642in}}{\pgfqpoint{1.884052in}{1.370688in}}%
\pgfusepath{clip}%
\pgfsetrectcap%
\pgfsetroundjoin%
\pgfsetlinewidth{0.803000pt}%
\definecolor{currentstroke}{rgb}{0.850000,0.850000,0.850000}%
\pgfsetstrokecolor{currentstroke}%
\pgfsetdash{}{0pt}%
\pgfpathmoveto{\pgfqpoint{1.745337in}{0.417642in}}%
\pgfpathlineto{\pgfqpoint{1.745337in}{1.788330in}}%
\pgfusepath{stroke}%
\end{pgfscope}%
\begin{pgfscope}%
\pgfsetbuttcap%
\pgfsetroundjoin%
\definecolor{currentfill}{rgb}{0.000000,0.000000,0.000000}%
\pgfsetfillcolor{currentfill}%
\pgfsetlinewidth{0.602250pt}%
\definecolor{currentstroke}{rgb}{0.000000,0.000000,0.000000}%
\pgfsetstrokecolor{currentstroke}%
\pgfsetdash{}{0pt}%
\pgfsys@defobject{currentmarker}{\pgfqpoint{0.000000in}{-0.027778in}}{\pgfqpoint{0.000000in}{0.000000in}}{%
\pgfpathmoveto{\pgfqpoint{0.000000in}{0.000000in}}%
\pgfpathlineto{\pgfqpoint{0.000000in}{-0.027778in}}%
\pgfusepath{stroke,fill}%
}%
\begin{pgfscope}%
\pgfsys@transformshift{1.745337in}{0.417642in}%
\pgfsys@useobject{currentmarker}{}%
\end{pgfscope}%
\end{pgfscope}%
\begin{pgfscope}%
\pgfpathrectangle{\pgfqpoint{0.514278in}{0.417642in}}{\pgfqpoint{1.884052in}{1.370688in}}%
\pgfusepath{clip}%
\pgfsetrectcap%
\pgfsetroundjoin%
\pgfsetlinewidth{0.803000pt}%
\definecolor{currentstroke}{rgb}{0.850000,0.850000,0.850000}%
\pgfsetstrokecolor{currentstroke}%
\pgfsetdash{}{0pt}%
\pgfpathmoveto{\pgfqpoint{1.795466in}{0.417642in}}%
\pgfpathlineto{\pgfqpoint{1.795466in}{1.788330in}}%
\pgfusepath{stroke}%
\end{pgfscope}%
\begin{pgfscope}%
\pgfsetbuttcap%
\pgfsetroundjoin%
\definecolor{currentfill}{rgb}{0.000000,0.000000,0.000000}%
\pgfsetfillcolor{currentfill}%
\pgfsetlinewidth{0.602250pt}%
\definecolor{currentstroke}{rgb}{0.000000,0.000000,0.000000}%
\pgfsetstrokecolor{currentstroke}%
\pgfsetdash{}{0pt}%
\pgfsys@defobject{currentmarker}{\pgfqpoint{0.000000in}{-0.027778in}}{\pgfqpoint{0.000000in}{0.000000in}}{%
\pgfpathmoveto{\pgfqpoint{0.000000in}{0.000000in}}%
\pgfpathlineto{\pgfqpoint{0.000000in}{-0.027778in}}%
\pgfusepath{stroke,fill}%
}%
\begin{pgfscope}%
\pgfsys@transformshift{1.795466in}{0.417642in}%
\pgfsys@useobject{currentmarker}{}%
\end{pgfscope}%
\end{pgfscope}%
\begin{pgfscope}%
\pgfpathrectangle{\pgfqpoint{0.514278in}{0.417642in}}{\pgfqpoint{1.884052in}{1.370688in}}%
\pgfusepath{clip}%
\pgfsetrectcap%
\pgfsetroundjoin%
\pgfsetlinewidth{0.803000pt}%
\definecolor{currentstroke}{rgb}{0.850000,0.850000,0.850000}%
\pgfsetstrokecolor{currentstroke}%
\pgfsetdash{}{0pt}%
\pgfpathmoveto{\pgfqpoint{1.836425in}{0.417642in}}%
\pgfpathlineto{\pgfqpoint{1.836425in}{1.788330in}}%
\pgfusepath{stroke}%
\end{pgfscope}%
\begin{pgfscope}%
\pgfsetbuttcap%
\pgfsetroundjoin%
\definecolor{currentfill}{rgb}{0.000000,0.000000,0.000000}%
\pgfsetfillcolor{currentfill}%
\pgfsetlinewidth{0.602250pt}%
\definecolor{currentstroke}{rgb}{0.000000,0.000000,0.000000}%
\pgfsetstrokecolor{currentstroke}%
\pgfsetdash{}{0pt}%
\pgfsys@defobject{currentmarker}{\pgfqpoint{0.000000in}{-0.027778in}}{\pgfqpoint{0.000000in}{0.000000in}}{%
\pgfpathmoveto{\pgfqpoint{0.000000in}{0.000000in}}%
\pgfpathlineto{\pgfqpoint{0.000000in}{-0.027778in}}%
\pgfusepath{stroke,fill}%
}%
\begin{pgfscope}%
\pgfsys@transformshift{1.836425in}{0.417642in}%
\pgfsys@useobject{currentmarker}{}%
\end{pgfscope}%
\end{pgfscope}%
\begin{pgfscope}%
\pgfpathrectangle{\pgfqpoint{0.514278in}{0.417642in}}{\pgfqpoint{1.884052in}{1.370688in}}%
\pgfusepath{clip}%
\pgfsetrectcap%
\pgfsetroundjoin%
\pgfsetlinewidth{0.803000pt}%
\definecolor{currentstroke}{rgb}{0.850000,0.850000,0.850000}%
\pgfsetstrokecolor{currentstroke}%
\pgfsetdash{}{0pt}%
\pgfpathmoveto{\pgfqpoint{1.871056in}{0.417642in}}%
\pgfpathlineto{\pgfqpoint{1.871056in}{1.788330in}}%
\pgfusepath{stroke}%
\end{pgfscope}%
\begin{pgfscope}%
\pgfsetbuttcap%
\pgfsetroundjoin%
\definecolor{currentfill}{rgb}{0.000000,0.000000,0.000000}%
\pgfsetfillcolor{currentfill}%
\pgfsetlinewidth{0.602250pt}%
\definecolor{currentstroke}{rgb}{0.000000,0.000000,0.000000}%
\pgfsetstrokecolor{currentstroke}%
\pgfsetdash{}{0pt}%
\pgfsys@defobject{currentmarker}{\pgfqpoint{0.000000in}{-0.027778in}}{\pgfqpoint{0.000000in}{0.000000in}}{%
\pgfpathmoveto{\pgfqpoint{0.000000in}{0.000000in}}%
\pgfpathlineto{\pgfqpoint{0.000000in}{-0.027778in}}%
\pgfusepath{stroke,fill}%
}%
\begin{pgfscope}%
\pgfsys@transformshift{1.871056in}{0.417642in}%
\pgfsys@useobject{currentmarker}{}%
\end{pgfscope}%
\end{pgfscope}%
\begin{pgfscope}%
\pgfpathrectangle{\pgfqpoint{0.514278in}{0.417642in}}{\pgfqpoint{1.884052in}{1.370688in}}%
\pgfusepath{clip}%
\pgfsetrectcap%
\pgfsetroundjoin%
\pgfsetlinewidth{0.803000pt}%
\definecolor{currentstroke}{rgb}{0.850000,0.850000,0.850000}%
\pgfsetstrokecolor{currentstroke}%
\pgfsetdash{}{0pt}%
\pgfpathmoveto{\pgfqpoint{1.901054in}{0.417642in}}%
\pgfpathlineto{\pgfqpoint{1.901054in}{1.788330in}}%
\pgfusepath{stroke}%
\end{pgfscope}%
\begin{pgfscope}%
\pgfsetbuttcap%
\pgfsetroundjoin%
\definecolor{currentfill}{rgb}{0.000000,0.000000,0.000000}%
\pgfsetfillcolor{currentfill}%
\pgfsetlinewidth{0.602250pt}%
\definecolor{currentstroke}{rgb}{0.000000,0.000000,0.000000}%
\pgfsetstrokecolor{currentstroke}%
\pgfsetdash{}{0pt}%
\pgfsys@defobject{currentmarker}{\pgfqpoint{0.000000in}{-0.027778in}}{\pgfqpoint{0.000000in}{0.000000in}}{%
\pgfpathmoveto{\pgfqpoint{0.000000in}{0.000000in}}%
\pgfpathlineto{\pgfqpoint{0.000000in}{-0.027778in}}%
\pgfusepath{stroke,fill}%
}%
\begin{pgfscope}%
\pgfsys@transformshift{1.901054in}{0.417642in}%
\pgfsys@useobject{currentmarker}{}%
\end{pgfscope}%
\end{pgfscope}%
\begin{pgfscope}%
\pgfpathrectangle{\pgfqpoint{0.514278in}{0.417642in}}{\pgfqpoint{1.884052in}{1.370688in}}%
\pgfusepath{clip}%
\pgfsetrectcap%
\pgfsetroundjoin%
\pgfsetlinewidth{0.803000pt}%
\definecolor{currentstroke}{rgb}{0.850000,0.850000,0.850000}%
\pgfsetstrokecolor{currentstroke}%
\pgfsetdash{}{0pt}%
\pgfpathmoveto{\pgfqpoint{1.927514in}{0.417642in}}%
\pgfpathlineto{\pgfqpoint{1.927514in}{1.788330in}}%
\pgfusepath{stroke}%
\end{pgfscope}%
\begin{pgfscope}%
\pgfsetbuttcap%
\pgfsetroundjoin%
\definecolor{currentfill}{rgb}{0.000000,0.000000,0.000000}%
\pgfsetfillcolor{currentfill}%
\pgfsetlinewidth{0.602250pt}%
\definecolor{currentstroke}{rgb}{0.000000,0.000000,0.000000}%
\pgfsetstrokecolor{currentstroke}%
\pgfsetdash{}{0pt}%
\pgfsys@defobject{currentmarker}{\pgfqpoint{0.000000in}{-0.027778in}}{\pgfqpoint{0.000000in}{0.000000in}}{%
\pgfpathmoveto{\pgfqpoint{0.000000in}{0.000000in}}%
\pgfpathlineto{\pgfqpoint{0.000000in}{-0.027778in}}%
\pgfusepath{stroke,fill}%
}%
\begin{pgfscope}%
\pgfsys@transformshift{1.927514in}{0.417642in}%
\pgfsys@useobject{currentmarker}{}%
\end{pgfscope}%
\end{pgfscope}%
\begin{pgfscope}%
\pgfpathrectangle{\pgfqpoint{0.514278in}{0.417642in}}{\pgfqpoint{1.884052in}{1.370688in}}%
\pgfusepath{clip}%
\pgfsetrectcap%
\pgfsetroundjoin%
\pgfsetlinewidth{0.803000pt}%
\definecolor{currentstroke}{rgb}{0.850000,0.850000,0.850000}%
\pgfsetstrokecolor{currentstroke}%
\pgfsetdash{}{0pt}%
\pgfpathmoveto{\pgfqpoint{2.106900in}{0.417642in}}%
\pgfpathlineto{\pgfqpoint{2.106900in}{1.788330in}}%
\pgfusepath{stroke}%
\end{pgfscope}%
\begin{pgfscope}%
\pgfsetbuttcap%
\pgfsetroundjoin%
\definecolor{currentfill}{rgb}{0.000000,0.000000,0.000000}%
\pgfsetfillcolor{currentfill}%
\pgfsetlinewidth{0.602250pt}%
\definecolor{currentstroke}{rgb}{0.000000,0.000000,0.000000}%
\pgfsetstrokecolor{currentstroke}%
\pgfsetdash{}{0pt}%
\pgfsys@defobject{currentmarker}{\pgfqpoint{0.000000in}{-0.027778in}}{\pgfqpoint{0.000000in}{0.000000in}}{%
\pgfpathmoveto{\pgfqpoint{0.000000in}{0.000000in}}%
\pgfpathlineto{\pgfqpoint{0.000000in}{-0.027778in}}%
\pgfusepath{stroke,fill}%
}%
\begin{pgfscope}%
\pgfsys@transformshift{2.106900in}{0.417642in}%
\pgfsys@useobject{currentmarker}{}%
\end{pgfscope}%
\end{pgfscope}%
\begin{pgfscope}%
\pgfpathrectangle{\pgfqpoint{0.514278in}{0.417642in}}{\pgfqpoint{1.884052in}{1.370688in}}%
\pgfusepath{clip}%
\pgfsetrectcap%
\pgfsetroundjoin%
\pgfsetlinewidth{0.803000pt}%
\definecolor{currentstroke}{rgb}{0.850000,0.850000,0.850000}%
\pgfsetstrokecolor{currentstroke}%
\pgfsetdash{}{0pt}%
\pgfpathmoveto{\pgfqpoint{2.197988in}{0.417642in}}%
\pgfpathlineto{\pgfqpoint{2.197988in}{1.788330in}}%
\pgfusepath{stroke}%
\end{pgfscope}%
\begin{pgfscope}%
\pgfsetbuttcap%
\pgfsetroundjoin%
\definecolor{currentfill}{rgb}{0.000000,0.000000,0.000000}%
\pgfsetfillcolor{currentfill}%
\pgfsetlinewidth{0.602250pt}%
\definecolor{currentstroke}{rgb}{0.000000,0.000000,0.000000}%
\pgfsetstrokecolor{currentstroke}%
\pgfsetdash{}{0pt}%
\pgfsys@defobject{currentmarker}{\pgfqpoint{0.000000in}{-0.027778in}}{\pgfqpoint{0.000000in}{0.000000in}}{%
\pgfpathmoveto{\pgfqpoint{0.000000in}{0.000000in}}%
\pgfpathlineto{\pgfqpoint{0.000000in}{-0.027778in}}%
\pgfusepath{stroke,fill}%
}%
\begin{pgfscope}%
\pgfsys@transformshift{2.197988in}{0.417642in}%
\pgfsys@useobject{currentmarker}{}%
\end{pgfscope}%
\end{pgfscope}%
\begin{pgfscope}%
\pgfpathrectangle{\pgfqpoint{0.514278in}{0.417642in}}{\pgfqpoint{1.884052in}{1.370688in}}%
\pgfusepath{clip}%
\pgfsetrectcap%
\pgfsetroundjoin%
\pgfsetlinewidth{0.803000pt}%
\definecolor{currentstroke}{rgb}{0.850000,0.850000,0.850000}%
\pgfsetstrokecolor{currentstroke}%
\pgfsetdash{}{0pt}%
\pgfpathmoveto{\pgfqpoint{2.262617in}{0.417642in}}%
\pgfpathlineto{\pgfqpoint{2.262617in}{1.788330in}}%
\pgfusepath{stroke}%
\end{pgfscope}%
\begin{pgfscope}%
\pgfsetbuttcap%
\pgfsetroundjoin%
\definecolor{currentfill}{rgb}{0.000000,0.000000,0.000000}%
\pgfsetfillcolor{currentfill}%
\pgfsetlinewidth{0.602250pt}%
\definecolor{currentstroke}{rgb}{0.000000,0.000000,0.000000}%
\pgfsetstrokecolor{currentstroke}%
\pgfsetdash{}{0pt}%
\pgfsys@defobject{currentmarker}{\pgfqpoint{0.000000in}{-0.027778in}}{\pgfqpoint{0.000000in}{0.000000in}}{%
\pgfpathmoveto{\pgfqpoint{0.000000in}{0.000000in}}%
\pgfpathlineto{\pgfqpoint{0.000000in}{-0.027778in}}%
\pgfusepath{stroke,fill}%
}%
\begin{pgfscope}%
\pgfsys@transformshift{2.262617in}{0.417642in}%
\pgfsys@useobject{currentmarker}{}%
\end{pgfscope}%
\end{pgfscope}%
\begin{pgfscope}%
\pgfpathrectangle{\pgfqpoint{0.514278in}{0.417642in}}{\pgfqpoint{1.884052in}{1.370688in}}%
\pgfusepath{clip}%
\pgfsetrectcap%
\pgfsetroundjoin%
\pgfsetlinewidth{0.803000pt}%
\definecolor{currentstroke}{rgb}{0.850000,0.850000,0.850000}%
\pgfsetstrokecolor{currentstroke}%
\pgfsetdash{}{0pt}%
\pgfpathmoveto{\pgfqpoint{2.312746in}{0.417642in}}%
\pgfpathlineto{\pgfqpoint{2.312746in}{1.788330in}}%
\pgfusepath{stroke}%
\end{pgfscope}%
\begin{pgfscope}%
\pgfsetbuttcap%
\pgfsetroundjoin%
\definecolor{currentfill}{rgb}{0.000000,0.000000,0.000000}%
\pgfsetfillcolor{currentfill}%
\pgfsetlinewidth{0.602250pt}%
\definecolor{currentstroke}{rgb}{0.000000,0.000000,0.000000}%
\pgfsetstrokecolor{currentstroke}%
\pgfsetdash{}{0pt}%
\pgfsys@defobject{currentmarker}{\pgfqpoint{0.000000in}{-0.027778in}}{\pgfqpoint{0.000000in}{0.000000in}}{%
\pgfpathmoveto{\pgfqpoint{0.000000in}{0.000000in}}%
\pgfpathlineto{\pgfqpoint{0.000000in}{-0.027778in}}%
\pgfusepath{stroke,fill}%
}%
\begin{pgfscope}%
\pgfsys@transformshift{2.312746in}{0.417642in}%
\pgfsys@useobject{currentmarker}{}%
\end{pgfscope}%
\end{pgfscope}%
\begin{pgfscope}%
\pgfpathrectangle{\pgfqpoint{0.514278in}{0.417642in}}{\pgfqpoint{1.884052in}{1.370688in}}%
\pgfusepath{clip}%
\pgfsetrectcap%
\pgfsetroundjoin%
\pgfsetlinewidth{0.803000pt}%
\definecolor{currentstroke}{rgb}{0.850000,0.850000,0.850000}%
\pgfsetstrokecolor{currentstroke}%
\pgfsetdash{}{0pt}%
\pgfpathmoveto{\pgfqpoint{2.353705in}{0.417642in}}%
\pgfpathlineto{\pgfqpoint{2.353705in}{1.788330in}}%
\pgfusepath{stroke}%
\end{pgfscope}%
\begin{pgfscope}%
\pgfsetbuttcap%
\pgfsetroundjoin%
\definecolor{currentfill}{rgb}{0.000000,0.000000,0.000000}%
\pgfsetfillcolor{currentfill}%
\pgfsetlinewidth{0.602250pt}%
\definecolor{currentstroke}{rgb}{0.000000,0.000000,0.000000}%
\pgfsetstrokecolor{currentstroke}%
\pgfsetdash{}{0pt}%
\pgfsys@defobject{currentmarker}{\pgfqpoint{0.000000in}{-0.027778in}}{\pgfqpoint{0.000000in}{0.000000in}}{%
\pgfpathmoveto{\pgfqpoint{0.000000in}{0.000000in}}%
\pgfpathlineto{\pgfqpoint{0.000000in}{-0.027778in}}%
\pgfusepath{stroke,fill}%
}%
\begin{pgfscope}%
\pgfsys@transformshift{2.353705in}{0.417642in}%
\pgfsys@useobject{currentmarker}{}%
\end{pgfscope}%
\end{pgfscope}%
\begin{pgfscope}%
\pgfpathrectangle{\pgfqpoint{0.514278in}{0.417642in}}{\pgfqpoint{1.884052in}{1.370688in}}%
\pgfusepath{clip}%
\pgfsetrectcap%
\pgfsetroundjoin%
\pgfsetlinewidth{0.803000pt}%
\definecolor{currentstroke}{rgb}{0.850000,0.850000,0.850000}%
\pgfsetstrokecolor{currentstroke}%
\pgfsetdash{}{0pt}%
\pgfpathmoveto{\pgfqpoint{2.388335in}{0.417642in}}%
\pgfpathlineto{\pgfqpoint{2.388335in}{1.788330in}}%
\pgfusepath{stroke}%
\end{pgfscope}%
\begin{pgfscope}%
\pgfsetbuttcap%
\pgfsetroundjoin%
\definecolor{currentfill}{rgb}{0.000000,0.000000,0.000000}%
\pgfsetfillcolor{currentfill}%
\pgfsetlinewidth{0.602250pt}%
\definecolor{currentstroke}{rgb}{0.000000,0.000000,0.000000}%
\pgfsetstrokecolor{currentstroke}%
\pgfsetdash{}{0pt}%
\pgfsys@defobject{currentmarker}{\pgfqpoint{0.000000in}{-0.027778in}}{\pgfqpoint{0.000000in}{0.000000in}}{%
\pgfpathmoveto{\pgfqpoint{0.000000in}{0.000000in}}%
\pgfpathlineto{\pgfqpoint{0.000000in}{-0.027778in}}%
\pgfusepath{stroke,fill}%
}%
\begin{pgfscope}%
\pgfsys@transformshift{2.388335in}{0.417642in}%
\pgfsys@useobject{currentmarker}{}%
\end{pgfscope}%
\end{pgfscope}%
\begin{pgfscope}%
\definecolor{textcolor}{rgb}{0.000000,0.000000,0.000000}%
\pgfsetstrokecolor{textcolor}%
\pgfsetfillcolor{textcolor}%
\pgftext[x=1.456304in,y=0.165003in,,top]{\color{textcolor}\rmfamily\fontsize{10.000000}{12.000000}\selectfont Frequency in \(\displaystyle \unit{\Hz}\)}%
\end{pgfscope}%
\begin{pgfscope}%
\pgfpathrectangle{\pgfqpoint{0.514278in}{0.417642in}}{\pgfqpoint{1.884052in}{1.370688in}}%
\pgfusepath{clip}%
\pgfsetrectcap%
\pgfsetroundjoin%
\pgfsetlinewidth{0.803000pt}%
\definecolor{currentstroke}{rgb}{0.450000,0.450000,0.450000}%
\pgfsetstrokecolor{currentstroke}%
\pgfsetdash{}{0pt}%
\pgfpathmoveto{\pgfqpoint{0.514278in}{0.640555in}}%
\pgfpathlineto{\pgfqpoint{2.398330in}{0.640555in}}%
\pgfusepath{stroke}%
\end{pgfscope}%
\begin{pgfscope}%
\pgfsetbuttcap%
\pgfsetroundjoin%
\definecolor{currentfill}{rgb}{0.000000,0.000000,0.000000}%
\pgfsetfillcolor{currentfill}%
\pgfsetlinewidth{0.803000pt}%
\definecolor{currentstroke}{rgb}{0.000000,0.000000,0.000000}%
\pgfsetstrokecolor{currentstroke}%
\pgfsetdash{}{0pt}%
\pgfsys@defobject{currentmarker}{\pgfqpoint{-0.048611in}{0.000000in}}{\pgfqpoint{-0.000000in}{0.000000in}}{%
\pgfpathmoveto{\pgfqpoint{-0.000000in}{0.000000in}}%
\pgfpathlineto{\pgfqpoint{-0.048611in}{0.000000in}}%
\pgfusepath{stroke,fill}%
}%
\begin{pgfscope}%
\pgfsys@transformshift{0.514278in}{0.640555in}%
\pgfsys@useobject{currentmarker}{}%
\end{pgfscope}%
\end{pgfscope}%
\begin{pgfscope}%
\definecolor{textcolor}{rgb}{0.000000,0.000000,0.000000}%
\pgfsetstrokecolor{textcolor}%
\pgfsetfillcolor{textcolor}%
\pgftext[x=0.241129in, y=0.601402in, left, base]{\color{textcolor}\rmfamily\fontsize{8.000000}{9.600000}\selectfont \(\displaystyle {10^{0}}\)}%
\end{pgfscope}%
\begin{pgfscope}%
\pgfpathrectangle{\pgfqpoint{0.514278in}{0.417642in}}{\pgfqpoint{1.884052in}{1.370688in}}%
\pgfusepath{clip}%
\pgfsetrectcap%
\pgfsetroundjoin%
\pgfsetlinewidth{0.803000pt}%
\definecolor{currentstroke}{rgb}{0.450000,0.450000,0.450000}%
\pgfsetstrokecolor{currentstroke}%
\pgfsetdash{}{0pt}%
\pgfpathmoveto{\pgfqpoint{0.514278in}{0.983227in}}%
\pgfpathlineto{\pgfqpoint{2.398330in}{0.983227in}}%
\pgfusepath{stroke}%
\end{pgfscope}%
\begin{pgfscope}%
\pgfsetbuttcap%
\pgfsetroundjoin%
\definecolor{currentfill}{rgb}{0.000000,0.000000,0.000000}%
\pgfsetfillcolor{currentfill}%
\pgfsetlinewidth{0.803000pt}%
\definecolor{currentstroke}{rgb}{0.000000,0.000000,0.000000}%
\pgfsetstrokecolor{currentstroke}%
\pgfsetdash{}{0pt}%
\pgfsys@defobject{currentmarker}{\pgfqpoint{-0.048611in}{0.000000in}}{\pgfqpoint{-0.000000in}{0.000000in}}{%
\pgfpathmoveto{\pgfqpoint{-0.000000in}{0.000000in}}%
\pgfpathlineto{\pgfqpoint{-0.048611in}{0.000000in}}%
\pgfusepath{stroke,fill}%
}%
\begin{pgfscope}%
\pgfsys@transformshift{0.514278in}{0.983227in}%
\pgfsys@useobject{currentmarker}{}%
\end{pgfscope}%
\end{pgfscope}%
\begin{pgfscope}%
\definecolor{textcolor}{rgb}{0.000000,0.000000,0.000000}%
\pgfsetstrokecolor{textcolor}%
\pgfsetfillcolor{textcolor}%
\pgftext[x=0.241129in, y=0.944074in, left, base]{\color{textcolor}\rmfamily\fontsize{8.000000}{9.600000}\selectfont \(\displaystyle {10^{2}}\)}%
\end{pgfscope}%
\begin{pgfscope}%
\pgfpathrectangle{\pgfqpoint{0.514278in}{0.417642in}}{\pgfqpoint{1.884052in}{1.370688in}}%
\pgfusepath{clip}%
\pgfsetrectcap%
\pgfsetroundjoin%
\pgfsetlinewidth{0.803000pt}%
\definecolor{currentstroke}{rgb}{0.450000,0.450000,0.450000}%
\pgfsetstrokecolor{currentstroke}%
\pgfsetdash{}{0pt}%
\pgfpathmoveto{\pgfqpoint{0.514278in}{1.325899in}}%
\pgfpathlineto{\pgfqpoint{2.398330in}{1.325899in}}%
\pgfusepath{stroke}%
\end{pgfscope}%
\begin{pgfscope}%
\pgfsetbuttcap%
\pgfsetroundjoin%
\definecolor{currentfill}{rgb}{0.000000,0.000000,0.000000}%
\pgfsetfillcolor{currentfill}%
\pgfsetlinewidth{0.803000pt}%
\definecolor{currentstroke}{rgb}{0.000000,0.000000,0.000000}%
\pgfsetstrokecolor{currentstroke}%
\pgfsetdash{}{0pt}%
\pgfsys@defobject{currentmarker}{\pgfqpoint{-0.048611in}{0.000000in}}{\pgfqpoint{-0.000000in}{0.000000in}}{%
\pgfpathmoveto{\pgfqpoint{-0.000000in}{0.000000in}}%
\pgfpathlineto{\pgfqpoint{-0.048611in}{0.000000in}}%
\pgfusepath{stroke,fill}%
}%
\begin{pgfscope}%
\pgfsys@transformshift{0.514278in}{1.325899in}%
\pgfsys@useobject{currentmarker}{}%
\end{pgfscope}%
\end{pgfscope}%
\begin{pgfscope}%
\definecolor{textcolor}{rgb}{0.000000,0.000000,0.000000}%
\pgfsetstrokecolor{textcolor}%
\pgfsetfillcolor{textcolor}%
\pgftext[x=0.241129in, y=1.286746in, left, base]{\color{textcolor}\rmfamily\fontsize{8.000000}{9.600000}\selectfont \(\displaystyle {10^{4}}\)}%
\end{pgfscope}%
\begin{pgfscope}%
\pgfpathrectangle{\pgfqpoint{0.514278in}{0.417642in}}{\pgfqpoint{1.884052in}{1.370688in}}%
\pgfusepath{clip}%
\pgfsetrectcap%
\pgfsetroundjoin%
\pgfsetlinewidth{0.803000pt}%
\definecolor{currentstroke}{rgb}{0.450000,0.450000,0.450000}%
\pgfsetstrokecolor{currentstroke}%
\pgfsetdash{}{0pt}%
\pgfpathmoveto{\pgfqpoint{0.514278in}{1.668571in}}%
\pgfpathlineto{\pgfqpoint{2.398330in}{1.668571in}}%
\pgfusepath{stroke}%
\end{pgfscope}%
\begin{pgfscope}%
\pgfsetbuttcap%
\pgfsetroundjoin%
\definecolor{currentfill}{rgb}{0.000000,0.000000,0.000000}%
\pgfsetfillcolor{currentfill}%
\pgfsetlinewidth{0.803000pt}%
\definecolor{currentstroke}{rgb}{0.000000,0.000000,0.000000}%
\pgfsetstrokecolor{currentstroke}%
\pgfsetdash{}{0pt}%
\pgfsys@defobject{currentmarker}{\pgfqpoint{-0.048611in}{0.000000in}}{\pgfqpoint{-0.000000in}{0.000000in}}{%
\pgfpathmoveto{\pgfqpoint{-0.000000in}{0.000000in}}%
\pgfpathlineto{\pgfqpoint{-0.048611in}{0.000000in}}%
\pgfusepath{stroke,fill}%
}%
\begin{pgfscope}%
\pgfsys@transformshift{0.514278in}{1.668571in}%
\pgfsys@useobject{currentmarker}{}%
\end{pgfscope}%
\end{pgfscope}%
\begin{pgfscope}%
\definecolor{textcolor}{rgb}{0.000000,0.000000,0.000000}%
\pgfsetstrokecolor{textcolor}%
\pgfsetfillcolor{textcolor}%
\pgftext[x=0.241129in, y=1.629418in, left, base]{\color{textcolor}\rmfamily\fontsize{8.000000}{9.600000}\selectfont \(\displaystyle {10^{6}}\)}%
\end{pgfscope}%
\begin{pgfscope}%
\definecolor{textcolor}{rgb}{0.000000,0.000000,0.000000}%
\pgfsetstrokecolor{textcolor}%
\pgfsetfillcolor{textcolor}%
\pgftext[x=0.185574in,y=1.102986in,,bottom,rotate=90.000000]{\color{textcolor}\rmfamily\fontsize{10.000000}{12.000000}\selectfont  \(\displaystyle S_y(f)\) in \(\displaystyle \unit{1 \per \Hz}\)}%
\end{pgfscope}%
\begin{pgfscope}%
\pgfpathrectangle{\pgfqpoint{0.514278in}{0.417642in}}{\pgfqpoint{1.884052in}{1.370688in}}%
\pgfusepath{clip}%
\pgfsetbuttcap%
\pgfsetroundjoin%
\pgfsetlinewidth{1.505625pt}%
\definecolor{currentstroke}{rgb}{0.000000,0.447059,0.698039}%
\pgfsetstrokecolor{currentstroke}%
\pgfsetdash{{5.550000pt}{2.400000pt}}{0.000000pt}%
\pgfpathmoveto{\pgfqpoint{0.599917in}{0.692133in}}%
\pgfpathlineto{\pgfqpoint{2.312691in}{0.692133in}}%
\pgfpathlineto{\pgfqpoint{2.312691in}{0.692133in}}%
\pgfusepath{stroke}%
\end{pgfscope}%
\begin{pgfscope}%
\pgfpathrectangle{\pgfqpoint{0.514278in}{0.417642in}}{\pgfqpoint{1.884052in}{1.370688in}}%
\pgfusepath{clip}%
\pgfsetbuttcap%
\pgfsetroundjoin%
\definecolor{currentfill}{rgb}{0.000000,0.447059,0.698039}%
\pgfsetfillcolor{currentfill}%
\pgfsetlinewidth{1.003750pt}%
\definecolor{currentstroke}{rgb}{0.000000,0.447059,0.698039}%
\pgfsetstrokecolor{currentstroke}%
\pgfsetdash{}{0pt}%
\pgfsys@defobject{currentmarker}{\pgfqpoint{-0.006944in}{-0.006944in}}{\pgfqpoint{0.006944in}{0.006944in}}{%
\pgfpathmoveto{\pgfqpoint{0.000000in}{-0.006944in}}%
\pgfpathcurveto{\pgfqpoint{0.001842in}{-0.006944in}}{\pgfqpoint{0.003608in}{-0.006213in}}{\pgfqpoint{0.004910in}{-0.004910in}}%
\pgfpathcurveto{\pgfqpoint{0.006213in}{-0.003608in}}{\pgfqpoint{0.006944in}{-0.001842in}}{\pgfqpoint{0.006944in}{0.000000in}}%
\pgfpathcurveto{\pgfqpoint{0.006944in}{0.001842in}}{\pgfqpoint{0.006213in}{0.003608in}}{\pgfqpoint{0.004910in}{0.004910in}}%
\pgfpathcurveto{\pgfqpoint{0.003608in}{0.006213in}}{\pgfqpoint{0.001842in}{0.006944in}}{\pgfqpoint{0.000000in}{0.006944in}}%
\pgfpathcurveto{\pgfqpoint{-0.001842in}{0.006944in}}{\pgfqpoint{-0.003608in}{0.006213in}}{\pgfqpoint{-0.004910in}{0.004910in}}%
\pgfpathcurveto{\pgfqpoint{-0.006213in}{0.003608in}}{\pgfqpoint{-0.006944in}{0.001842in}}{\pgfqpoint{-0.006944in}{0.000000in}}%
\pgfpathcurveto{\pgfqpoint{-0.006944in}{-0.001842in}}{\pgfqpoint{-0.006213in}{-0.003608in}}{\pgfqpoint{-0.004910in}{-0.004910in}}%
\pgfpathcurveto{\pgfqpoint{-0.003608in}{-0.006213in}}{\pgfqpoint{-0.001842in}{-0.006944in}}{\pgfqpoint{0.000000in}{-0.006944in}}%
\pgfpathlineto{\pgfqpoint{0.000000in}{-0.006944in}}%
\pgfpathclose%
\pgfusepath{stroke,fill}%
}%
\begin{pgfscope}%
\pgfsys@transformshift{-226.701573in}{0.624720in}%
\pgfsys@useobject{currentmarker}{}%
\end{pgfscope}%
\begin{pgfscope}%
\pgfsys@transformshift{0.599917in}{0.663678in}%
\pgfsys@useobject{currentmarker}{}%
\end{pgfscope}%
\begin{pgfscope}%
\pgfsys@transformshift{0.755634in}{0.618518in}%
\pgfsys@useobject{currentmarker}{}%
\end{pgfscope}%
\begin{pgfscope}%
\pgfsys@transformshift{0.846722in}{0.651871in}%
\pgfsys@useobject{currentmarker}{}%
\end{pgfscope}%
\begin{pgfscope}%
\pgfsys@transformshift{0.911351in}{0.646106in}%
\pgfsys@useobject{currentmarker}{}%
\end{pgfscope}%
\begin{pgfscope}%
\pgfsys@transformshift{0.961480in}{0.650617in}%
\pgfsys@useobject{currentmarker}{}%
\end{pgfscope}%
\begin{pgfscope}%
\pgfsys@transformshift{1.002439in}{0.619497in}%
\pgfsys@useobject{currentmarker}{}%
\end{pgfscope}%
\begin{pgfscope}%
\pgfsys@transformshift{1.037069in}{0.703448in}%
\pgfsys@useobject{currentmarker}{}%
\end{pgfscope}%
\begin{pgfscope}%
\pgfsys@transformshift{1.067067in}{0.721385in}%
\pgfsys@useobject{currentmarker}{}%
\end{pgfscope}%
\begin{pgfscope}%
\pgfsys@transformshift{1.093527in}{0.711558in}%
\pgfsys@useobject{currentmarker}{}%
\end{pgfscope}%
\begin{pgfscope}%
\pgfsys@transformshift{1.117197in}{0.701313in}%
\pgfsys@useobject{currentmarker}{}%
\end{pgfscope}%
\begin{pgfscope}%
\pgfsys@transformshift{1.138608in}{0.674215in}%
\pgfsys@useobject{currentmarker}{}%
\end{pgfscope}%
\begin{pgfscope}%
\pgfsys@transformshift{1.158156in}{0.714359in}%
\pgfsys@useobject{currentmarker}{}%
\end{pgfscope}%
\begin{pgfscope}%
\pgfsys@transformshift{1.176137in}{0.690859in}%
\pgfsys@useobject{currentmarker}{}%
\end{pgfscope}%
\begin{pgfscope}%
\pgfsys@transformshift{1.192786in}{0.630191in}%
\pgfsys@useobject{currentmarker}{}%
\end{pgfscope}%
\begin{pgfscope}%
\pgfsys@transformshift{1.208285in}{0.701226in}%
\pgfsys@useobject{currentmarker}{}%
\end{pgfscope}%
\begin{pgfscope}%
\pgfsys@transformshift{1.222784in}{0.729309in}%
\pgfsys@useobject{currentmarker}{}%
\end{pgfscope}%
\begin{pgfscope}%
\pgfsys@transformshift{1.236403in}{0.708058in}%
\pgfsys@useobject{currentmarker}{}%
\end{pgfscope}%
\begin{pgfscope}%
\pgfsys@transformshift{1.249244in}{0.605219in}%
\pgfsys@useobject{currentmarker}{}%
\end{pgfscope}%
\begin{pgfscope}%
\pgfsys@transformshift{1.261390in}{0.661534in}%
\pgfsys@useobject{currentmarker}{}%
\end{pgfscope}%
\begin{pgfscope}%
\pgfsys@transformshift{1.272914in}{0.731254in}%
\pgfsys@useobject{currentmarker}{}%
\end{pgfscope}%
\begin{pgfscope}%
\pgfsys@transformshift{1.283874in}{0.748974in}%
\pgfsys@useobject{currentmarker}{}%
\end{pgfscope}%
\begin{pgfscope}%
\pgfsys@transformshift{1.294325in}{0.714574in}%
\pgfsys@useobject{currentmarker}{}%
\end{pgfscope}%
\begin{pgfscope}%
\pgfsys@transformshift{1.304311in}{0.694932in}%
\pgfsys@useobject{currentmarker}{}%
\end{pgfscope}%
\begin{pgfscope}%
\pgfsys@transformshift{1.313872in}{0.651474in}%
\pgfsys@useobject{currentmarker}{}%
\end{pgfscope}%
\begin{pgfscope}%
\pgfsys@transformshift{1.323043in}{0.677275in}%
\pgfsys@useobject{currentmarker}{}%
\end{pgfscope}%
\begin{pgfscope}%
\pgfsys@transformshift{1.331854in}{0.659492in}%
\pgfsys@useobject{currentmarker}{}%
\end{pgfscope}%
\begin{pgfscope}%
\pgfsys@transformshift{1.340333in}{0.657028in}%
\pgfsys@useobject{currentmarker}{}%
\end{pgfscope}%
\begin{pgfscope}%
\pgfsys@transformshift{1.348503in}{0.706173in}%
\pgfsys@useobject{currentmarker}{}%
\end{pgfscope}%
\begin{pgfscope}%
\pgfsys@transformshift{1.356386in}{0.713395in}%
\pgfsys@useobject{currentmarker}{}%
\end{pgfscope}%
\begin{pgfscope}%
\pgfsys@transformshift{1.364002in}{0.673664in}%
\pgfsys@useobject{currentmarker}{}%
\end{pgfscope}%
\begin{pgfscope}%
\pgfsys@transformshift{1.371368in}{0.699535in}%
\pgfsys@useobject{currentmarker}{}%
\end{pgfscope}%
\begin{pgfscope}%
\pgfsys@transformshift{1.378501in}{0.685568in}%
\pgfsys@useobject{currentmarker}{}%
\end{pgfscope}%
\begin{pgfscope}%
\pgfsys@transformshift{1.385414in}{0.673357in}%
\pgfsys@useobject{currentmarker}{}%
\end{pgfscope}%
\begin{pgfscope}%
\pgfsys@transformshift{1.392120in}{0.711047in}%
\pgfsys@useobject{currentmarker}{}%
\end{pgfscope}%
\begin{pgfscope}%
\pgfsys@transformshift{1.398632in}{0.674407in}%
\pgfsys@useobject{currentmarker}{}%
\end{pgfscope}%
\begin{pgfscope}%
\pgfsys@transformshift{1.404961in}{0.709521in}%
\pgfsys@useobject{currentmarker}{}%
\end{pgfscope}%
\begin{pgfscope}%
\pgfsys@transformshift{1.411116in}{0.699352in}%
\pgfsys@useobject{currentmarker}{}%
\end{pgfscope}%
\begin{pgfscope}%
\pgfsys@transformshift{1.417107in}{0.670861in}%
\pgfsys@useobject{currentmarker}{}%
\end{pgfscope}%
\begin{pgfscope}%
\pgfsys@transformshift{1.422943in}{0.630255in}%
\pgfsys@useobject{currentmarker}{}%
\end{pgfscope}%
\begin{pgfscope}%
\pgfsys@transformshift{1.428630in}{0.680685in}%
\pgfsys@useobject{currentmarker}{}%
\end{pgfscope}%
\begin{pgfscope}%
\pgfsys@transformshift{1.434178in}{0.708422in}%
\pgfsys@useobject{currentmarker}{}%
\end{pgfscope}%
\begin{pgfscope}%
\pgfsys@transformshift{1.439591in}{0.724003in}%
\pgfsys@useobject{currentmarker}{}%
\end{pgfscope}%
\begin{pgfscope}%
\pgfsys@transformshift{1.444877in}{0.740658in}%
\pgfsys@useobject{currentmarker}{}%
\end{pgfscope}%
\begin{pgfscope}%
\pgfsys@transformshift{1.450042in}{0.691679in}%
\pgfsys@useobject{currentmarker}{}%
\end{pgfscope}%
\begin{pgfscope}%
\pgfsys@transformshift{1.455090in}{0.615527in}%
\pgfsys@useobject{currentmarker}{}%
\end{pgfscope}%
\begin{pgfscope}%
\pgfsys@transformshift{1.460028in}{0.645323in}%
\pgfsys@useobject{currentmarker}{}%
\end{pgfscope}%
\begin{pgfscope}%
\pgfsys@transformshift{1.464859in}{0.669445in}%
\pgfsys@useobject{currentmarker}{}%
\end{pgfscope}%
\begin{pgfscope}%
\pgfsys@transformshift{1.469589in}{0.698805in}%
\pgfsys@useobject{currentmarker}{}%
\end{pgfscope}%
\begin{pgfscope}%
\pgfsys@transformshift{1.474221in}{0.686711in}%
\pgfsys@useobject{currentmarker}{}%
\end{pgfscope}%
\begin{pgfscope}%
\pgfsys@transformshift{1.478760in}{0.640861in}%
\pgfsys@useobject{currentmarker}{}%
\end{pgfscope}%
\begin{pgfscope}%
\pgfsys@transformshift{1.483209in}{0.584360in}%
\pgfsys@useobject{currentmarker}{}%
\end{pgfscope}%
\begin{pgfscope}%
\pgfsys@transformshift{1.487571in}{0.626187in}%
\pgfsys@useobject{currentmarker}{}%
\end{pgfscope}%
\begin{pgfscope}%
\pgfsys@transformshift{1.491850in}{0.693923in}%
\pgfsys@useobject{currentmarker}{}%
\end{pgfscope}%
\begin{pgfscope}%
\pgfsys@transformshift{1.496049in}{0.697953in}%
\pgfsys@useobject{currentmarker}{}%
\end{pgfscope}%
\begin{pgfscope}%
\pgfsys@transformshift{1.500172in}{0.709626in}%
\pgfsys@useobject{currentmarker}{}%
\end{pgfscope}%
\begin{pgfscope}%
\pgfsys@transformshift{1.504219in}{0.716719in}%
\pgfsys@useobject{currentmarker}{}%
\end{pgfscope}%
\begin{pgfscope}%
\pgfsys@transformshift{1.508196in}{0.728854in}%
\pgfsys@useobject{currentmarker}{}%
\end{pgfscope}%
\begin{pgfscope}%
\pgfsys@transformshift{1.512103in}{0.722589in}%
\pgfsys@useobject{currentmarker}{}%
\end{pgfscope}%
\begin{pgfscope}%
\pgfsys@transformshift{1.515943in}{0.701960in}%
\pgfsys@useobject{currentmarker}{}%
\end{pgfscope}%
\begin{pgfscope}%
\pgfsys@transformshift{1.519719in}{0.683477in}%
\pgfsys@useobject{currentmarker}{}%
\end{pgfscope}%
\begin{pgfscope}%
\pgfsys@transformshift{1.523432in}{0.677670in}%
\pgfsys@useobject{currentmarker}{}%
\end{pgfscope}%
\begin{pgfscope}%
\pgfsys@transformshift{1.527085in}{0.623546in}%
\pgfsys@useobject{currentmarker}{}%
\end{pgfscope}%
\begin{pgfscope}%
\pgfsys@transformshift{1.530680in}{0.633649in}%
\pgfsys@useobject{currentmarker}{}%
\end{pgfscope}%
\begin{pgfscope}%
\pgfsys@transformshift{1.534217in}{0.669744in}%
\pgfsys@useobject{currentmarker}{}%
\end{pgfscope}%
\begin{pgfscope}%
\pgfsys@transformshift{1.537700in}{0.642444in}%
\pgfsys@useobject{currentmarker}{}%
\end{pgfscope}%
\begin{pgfscope}%
\pgfsys@transformshift{1.541130in}{0.643139in}%
\pgfsys@useobject{currentmarker}{}%
\end{pgfscope}%
\begin{pgfscope}%
\pgfsys@transformshift{1.544509in}{0.675283in}%
\pgfsys@useobject{currentmarker}{}%
\end{pgfscope}%
\begin{pgfscope}%
\pgfsys@transformshift{1.547837in}{0.709297in}%
\pgfsys@useobject{currentmarker}{}%
\end{pgfscope}%
\begin{pgfscope}%
\pgfsys@transformshift{1.551117in}{0.700611in}%
\pgfsys@useobject{currentmarker}{}%
\end{pgfscope}%
\begin{pgfscope}%
\pgfsys@transformshift{1.554349in}{0.634289in}%
\pgfsys@useobject{currentmarker}{}%
\end{pgfscope}%
\begin{pgfscope}%
\pgfsys@transformshift{1.557536in}{0.658252in}%
\pgfsys@useobject{currentmarker}{}%
\end{pgfscope}%
\begin{pgfscope}%
\pgfsys@transformshift{1.560678in}{0.709045in}%
\pgfsys@useobject{currentmarker}{}%
\end{pgfscope}%
\begin{pgfscope}%
\pgfsys@transformshift{1.563776in}{0.712386in}%
\pgfsys@useobject{currentmarker}{}%
\end{pgfscope}%
\begin{pgfscope}%
\pgfsys@transformshift{1.566833in}{0.675398in}%
\pgfsys@useobject{currentmarker}{}%
\end{pgfscope}%
\begin{pgfscope}%
\pgfsys@transformshift{1.569848in}{0.681361in}%
\pgfsys@useobject{currentmarker}{}%
\end{pgfscope}%
\begin{pgfscope}%
\pgfsys@transformshift{1.572824in}{0.702359in}%
\pgfsys@useobject{currentmarker}{}%
\end{pgfscope}%
\begin{pgfscope}%
\pgfsys@transformshift{1.575761in}{0.681252in}%
\pgfsys@useobject{currentmarker}{}%
\end{pgfscope}%
\begin{pgfscope}%
\pgfsys@transformshift{1.578659in}{0.633352in}%
\pgfsys@useobject{currentmarker}{}%
\end{pgfscope}%
\begin{pgfscope}%
\pgfsys@transformshift{1.581521in}{0.642407in}%
\pgfsys@useobject{currentmarker}{}%
\end{pgfscope}%
\begin{pgfscope}%
\pgfsys@transformshift{1.584347in}{0.695630in}%
\pgfsys@useobject{currentmarker}{}%
\end{pgfscope}%
\begin{pgfscope}%
\pgfsys@transformshift{1.587138in}{0.685366in}%
\pgfsys@useobject{currentmarker}{}%
\end{pgfscope}%
\begin{pgfscope}%
\pgfsys@transformshift{1.589894in}{0.673601in}%
\pgfsys@useobject{currentmarker}{}%
\end{pgfscope}%
\begin{pgfscope}%
\pgfsys@transformshift{1.592617in}{0.655245in}%
\pgfsys@useobject{currentmarker}{}%
\end{pgfscope}%
\begin{pgfscope}%
\pgfsys@transformshift{1.595308in}{0.637186in}%
\pgfsys@useobject{currentmarker}{}%
\end{pgfscope}%
\begin{pgfscope}%
\pgfsys@transformshift{1.597966in}{0.636657in}%
\pgfsys@useobject{currentmarker}{}%
\end{pgfscope}%
\begin{pgfscope}%
\pgfsys@transformshift{1.600594in}{0.674325in}%
\pgfsys@useobject{currentmarker}{}%
\end{pgfscope}%
\begin{pgfscope}%
\pgfsys@transformshift{1.603191in}{0.639240in}%
\pgfsys@useobject{currentmarker}{}%
\end{pgfscope}%
\begin{pgfscope}%
\pgfsys@transformshift{1.605759in}{0.661485in}%
\pgfsys@useobject{currentmarker}{}%
\end{pgfscope}%
\begin{pgfscope}%
\pgfsys@transformshift{1.608297in}{0.615765in}%
\pgfsys@useobject{currentmarker}{}%
\end{pgfscope}%
\begin{pgfscope}%
\pgfsys@transformshift{1.610807in}{0.622078in}%
\pgfsys@useobject{currentmarker}{}%
\end{pgfscope}%
\begin{pgfscope}%
\pgfsys@transformshift{1.613290in}{0.691726in}%
\pgfsys@useobject{currentmarker}{}%
\end{pgfscope}%
\begin{pgfscope}%
\pgfsys@transformshift{1.615745in}{0.683584in}%
\pgfsys@useobject{currentmarker}{}%
\end{pgfscope}%
\begin{pgfscope}%
\pgfsys@transformshift{1.618173in}{0.691561in}%
\pgfsys@useobject{currentmarker}{}%
\end{pgfscope}%
\begin{pgfscope}%
\pgfsys@transformshift{1.620576in}{0.680920in}%
\pgfsys@useobject{currentmarker}{}%
\end{pgfscope}%
\begin{pgfscope}%
\pgfsys@transformshift{1.622953in}{0.661756in}%
\pgfsys@useobject{currentmarker}{}%
\end{pgfscope}%
\begin{pgfscope}%
\pgfsys@transformshift{1.625306in}{0.679807in}%
\pgfsys@useobject{currentmarker}{}%
\end{pgfscope}%
\begin{pgfscope}%
\pgfsys@transformshift{1.627634in}{0.665479in}%
\pgfsys@useobject{currentmarker}{}%
\end{pgfscope}%
\begin{pgfscope}%
\pgfsys@transformshift{1.629938in}{0.650788in}%
\pgfsys@useobject{currentmarker}{}%
\end{pgfscope}%
\begin{pgfscope}%
\pgfsys@transformshift{1.632219in}{0.671671in}%
\pgfsys@useobject{currentmarker}{}%
\end{pgfscope}%
\begin{pgfscope}%
\pgfsys@transformshift{1.634477in}{0.654724in}%
\pgfsys@useobject{currentmarker}{}%
\end{pgfscope}%
\begin{pgfscope}%
\pgfsys@transformshift{1.636712in}{0.628973in}%
\pgfsys@useobject{currentmarker}{}%
\end{pgfscope}%
\begin{pgfscope}%
\pgfsys@transformshift{1.638925in}{0.662926in}%
\pgfsys@useobject{currentmarker}{}%
\end{pgfscope}%
\begin{pgfscope}%
\pgfsys@transformshift{1.641117in}{0.671421in}%
\pgfsys@useobject{currentmarker}{}%
\end{pgfscope}%
\begin{pgfscope}%
\pgfsys@transformshift{1.643288in}{0.710784in}%
\pgfsys@useobject{currentmarker}{}%
\end{pgfscope}%
\begin{pgfscope}%
\pgfsys@transformshift{1.645437in}{0.675700in}%
\pgfsys@useobject{currentmarker}{}%
\end{pgfscope}%
\begin{pgfscope}%
\pgfsys@transformshift{1.647567in}{0.679357in}%
\pgfsys@useobject{currentmarker}{}%
\end{pgfscope}%
\begin{pgfscope}%
\pgfsys@transformshift{1.649676in}{0.691826in}%
\pgfsys@useobject{currentmarker}{}%
\end{pgfscope}%
\begin{pgfscope}%
\pgfsys@transformshift{1.651766in}{0.697662in}%
\pgfsys@useobject{currentmarker}{}%
\end{pgfscope}%
\begin{pgfscope}%
\pgfsys@transformshift{1.653837in}{0.714151in}%
\pgfsys@useobject{currentmarker}{}%
\end{pgfscope}%
\begin{pgfscope}%
\pgfsys@transformshift{1.655888in}{0.702506in}%
\pgfsys@useobject{currentmarker}{}%
\end{pgfscope}%
\begin{pgfscope}%
\pgfsys@transformshift{1.657921in}{0.668027in}%
\pgfsys@useobject{currentmarker}{}%
\end{pgfscope}%
\begin{pgfscope}%
\pgfsys@transformshift{1.659936in}{0.695746in}%
\pgfsys@useobject{currentmarker}{}%
\end{pgfscope}%
\begin{pgfscope}%
\pgfsys@transformshift{1.661933in}{0.677445in}%
\pgfsys@useobject{currentmarker}{}%
\end{pgfscope}%
\begin{pgfscope}%
\pgfsys@transformshift{1.663912in}{0.676534in}%
\pgfsys@useobject{currentmarker}{}%
\end{pgfscope}%
\begin{pgfscope}%
\pgfsys@transformshift{1.665874in}{0.675255in}%
\pgfsys@useobject{currentmarker}{}%
\end{pgfscope}%
\begin{pgfscope}%
\pgfsys@transformshift{1.667819in}{0.635969in}%
\pgfsys@useobject{currentmarker}{}%
\end{pgfscope}%
\begin{pgfscope}%
\pgfsys@transformshift{1.669748in}{0.659485in}%
\pgfsys@useobject{currentmarker}{}%
\end{pgfscope}%
\begin{pgfscope}%
\pgfsys@transformshift{1.671660in}{0.705846in}%
\pgfsys@useobject{currentmarker}{}%
\end{pgfscope}%
\begin{pgfscope}%
\pgfsys@transformshift{1.673556in}{0.693948in}%
\pgfsys@useobject{currentmarker}{}%
\end{pgfscope}%
\begin{pgfscope}%
\pgfsys@transformshift{1.675435in}{0.665499in}%
\pgfsys@useobject{currentmarker}{}%
\end{pgfscope}%
\begin{pgfscope}%
\pgfsys@transformshift{1.677300in}{0.667706in}%
\pgfsys@useobject{currentmarker}{}%
\end{pgfscope}%
\begin{pgfscope}%
\pgfsys@transformshift{1.679149in}{0.686616in}%
\pgfsys@useobject{currentmarker}{}%
\end{pgfscope}%
\begin{pgfscope}%
\pgfsys@transformshift{1.680983in}{0.654105in}%
\pgfsys@useobject{currentmarker}{}%
\end{pgfscope}%
\begin{pgfscope}%
\pgfsys@transformshift{1.682802in}{0.685246in}%
\pgfsys@useobject{currentmarker}{}%
\end{pgfscope}%
\begin{pgfscope}%
\pgfsys@transformshift{1.684606in}{0.677587in}%
\pgfsys@useobject{currentmarker}{}%
\end{pgfscope}%
\begin{pgfscope}%
\pgfsys@transformshift{1.686396in}{0.683820in}%
\pgfsys@useobject{currentmarker}{}%
\end{pgfscope}%
\begin{pgfscope}%
\pgfsys@transformshift{1.688172in}{0.667612in}%
\pgfsys@useobject{currentmarker}{}%
\end{pgfscope}%
\begin{pgfscope}%
\pgfsys@transformshift{1.689934in}{0.681745in}%
\pgfsys@useobject{currentmarker}{}%
\end{pgfscope}%
\begin{pgfscope}%
\pgfsys@transformshift{1.691682in}{0.680677in}%
\pgfsys@useobject{currentmarker}{}%
\end{pgfscope}%
\begin{pgfscope}%
\pgfsys@transformshift{1.693417in}{0.654107in}%
\pgfsys@useobject{currentmarker}{}%
\end{pgfscope}%
\begin{pgfscope}%
\pgfsys@transformshift{1.695139in}{0.694216in}%
\pgfsys@useobject{currentmarker}{}%
\end{pgfscope}%
\begin{pgfscope}%
\pgfsys@transformshift{1.696847in}{0.708866in}%
\pgfsys@useobject{currentmarker}{}%
\end{pgfscope}%
\begin{pgfscope}%
\pgfsys@transformshift{1.698543in}{0.681582in}%
\pgfsys@useobject{currentmarker}{}%
\end{pgfscope}%
\begin{pgfscope}%
\pgfsys@transformshift{1.700225in}{0.684852in}%
\pgfsys@useobject{currentmarker}{}%
\end{pgfscope}%
\begin{pgfscope}%
\pgfsys@transformshift{1.701896in}{0.716700in}%
\pgfsys@useobject{currentmarker}{}%
\end{pgfscope}%
\begin{pgfscope}%
\pgfsys@transformshift{1.703554in}{0.706415in}%
\pgfsys@useobject{currentmarker}{}%
\end{pgfscope}%
\begin{pgfscope}%
\pgfsys@transformshift{1.705199in}{0.744503in}%
\pgfsys@useobject{currentmarker}{}%
\end{pgfscope}%
\begin{pgfscope}%
\pgfsys@transformshift{1.706833in}{0.729507in}%
\pgfsys@useobject{currentmarker}{}%
\end{pgfscope}%
\begin{pgfscope}%
\pgfsys@transformshift{1.708455in}{0.698267in}%
\pgfsys@useobject{currentmarker}{}%
\end{pgfscope}%
\begin{pgfscope}%
\pgfsys@transformshift{1.710066in}{0.705614in}%
\pgfsys@useobject{currentmarker}{}%
\end{pgfscope}%
\begin{pgfscope}%
\pgfsys@transformshift{1.711665in}{0.673054in}%
\pgfsys@useobject{currentmarker}{}%
\end{pgfscope}%
\begin{pgfscope}%
\pgfsys@transformshift{1.713252in}{0.743824in}%
\pgfsys@useobject{currentmarker}{}%
\end{pgfscope}%
\begin{pgfscope}%
\pgfsys@transformshift{1.714829in}{0.697156in}%
\pgfsys@useobject{currentmarker}{}%
\end{pgfscope}%
\begin{pgfscope}%
\pgfsys@transformshift{1.716394in}{0.703619in}%
\pgfsys@useobject{currentmarker}{}%
\end{pgfscope}%
\begin{pgfscope}%
\pgfsys@transformshift{1.717949in}{0.635238in}%
\pgfsys@useobject{currentmarker}{}%
\end{pgfscope}%
\begin{pgfscope}%
\pgfsys@transformshift{1.719493in}{0.594479in}%
\pgfsys@useobject{currentmarker}{}%
\end{pgfscope}%
\begin{pgfscope}%
\pgfsys@transformshift{1.721026in}{0.654857in}%
\pgfsys@useobject{currentmarker}{}%
\end{pgfscope}%
\begin{pgfscope}%
\pgfsys@transformshift{1.722550in}{0.689086in}%
\pgfsys@useobject{currentmarker}{}%
\end{pgfscope}%
\begin{pgfscope}%
\pgfsys@transformshift{1.724062in}{0.702642in}%
\pgfsys@useobject{currentmarker}{}%
\end{pgfscope}%
\begin{pgfscope}%
\pgfsys@transformshift{1.725565in}{0.726628in}%
\pgfsys@useobject{currentmarker}{}%
\end{pgfscope}%
\begin{pgfscope}%
\pgfsys@transformshift{1.727058in}{0.690670in}%
\pgfsys@useobject{currentmarker}{}%
\end{pgfscope}%
\begin{pgfscope}%
\pgfsys@transformshift{1.728541in}{0.690783in}%
\pgfsys@useobject{currentmarker}{}%
\end{pgfscope}%
\begin{pgfscope}%
\pgfsys@transformshift{1.730014in}{0.704044in}%
\pgfsys@useobject{currentmarker}{}%
\end{pgfscope}%
\begin{pgfscope}%
\pgfsys@transformshift{1.731477in}{0.655639in}%
\pgfsys@useobject{currentmarker}{}%
\end{pgfscope}%
\begin{pgfscope}%
\pgfsys@transformshift{1.732931in}{0.679834in}%
\pgfsys@useobject{currentmarker}{}%
\end{pgfscope}%
\begin{pgfscope}%
\pgfsys@transformshift{1.734376in}{0.693864in}%
\pgfsys@useobject{currentmarker}{}%
\end{pgfscope}%
\begin{pgfscope}%
\pgfsys@transformshift{1.735812in}{0.725502in}%
\pgfsys@useobject{currentmarker}{}%
\end{pgfscope}%
\begin{pgfscope}%
\pgfsys@transformshift{1.737238in}{0.741002in}%
\pgfsys@useobject{currentmarker}{}%
\end{pgfscope}%
\begin{pgfscope}%
\pgfsys@transformshift{1.738655in}{0.720827in}%
\pgfsys@useobject{currentmarker}{}%
\end{pgfscope}%
\begin{pgfscope}%
\pgfsys@transformshift{1.740064in}{0.657793in}%
\pgfsys@useobject{currentmarker}{}%
\end{pgfscope}%
\begin{pgfscope}%
\pgfsys@transformshift{1.741463in}{0.587835in}%
\pgfsys@useobject{currentmarker}{}%
\end{pgfscope}%
\begin{pgfscope}%
\pgfsys@transformshift{1.742854in}{0.650198in}%
\pgfsys@useobject{currentmarker}{}%
\end{pgfscope}%
\begin{pgfscope}%
\pgfsys@transformshift{1.744237in}{0.682555in}%
\pgfsys@useobject{currentmarker}{}%
\end{pgfscope}%
\begin{pgfscope}%
\pgfsys@transformshift{1.745611in}{0.716846in}%
\pgfsys@useobject{currentmarker}{}%
\end{pgfscope}%
\begin{pgfscope}%
\pgfsys@transformshift{1.746977in}{0.736044in}%
\pgfsys@useobject{currentmarker}{}%
\end{pgfscope}%
\begin{pgfscope}%
\pgfsys@transformshift{1.748334in}{0.699158in}%
\pgfsys@useobject{currentmarker}{}%
\end{pgfscope}%
\begin{pgfscope}%
\pgfsys@transformshift{1.749683in}{0.648137in}%
\pgfsys@useobject{currentmarker}{}%
\end{pgfscope}%
\begin{pgfscope}%
\pgfsys@transformshift{1.751025in}{0.649598in}%
\pgfsys@useobject{currentmarker}{}%
\end{pgfscope}%
\begin{pgfscope}%
\pgfsys@transformshift{1.752358in}{0.613845in}%
\pgfsys@useobject{currentmarker}{}%
\end{pgfscope}%
\begin{pgfscope}%
\pgfsys@transformshift{1.753683in}{0.626339in}%
\pgfsys@useobject{currentmarker}{}%
\end{pgfscope}%
\begin{pgfscope}%
\pgfsys@transformshift{1.755001in}{0.637002in}%
\pgfsys@useobject{currentmarker}{}%
\end{pgfscope}%
\begin{pgfscope}%
\pgfsys@transformshift{1.756311in}{0.664409in}%
\pgfsys@useobject{currentmarker}{}%
\end{pgfscope}%
\begin{pgfscope}%
\pgfsys@transformshift{1.757613in}{0.645225in}%
\pgfsys@useobject{currentmarker}{}%
\end{pgfscope}%
\begin{pgfscope}%
\pgfsys@transformshift{1.758908in}{0.594307in}%
\pgfsys@useobject{currentmarker}{}%
\end{pgfscope}%
\begin{pgfscope}%
\pgfsys@transformshift{1.760195in}{0.577254in}%
\pgfsys@useobject{currentmarker}{}%
\end{pgfscope}%
\begin{pgfscope}%
\pgfsys@transformshift{1.761475in}{0.650974in}%
\pgfsys@useobject{currentmarker}{}%
\end{pgfscope}%
\begin{pgfscope}%
\pgfsys@transformshift{1.762748in}{0.705925in}%
\pgfsys@useobject{currentmarker}{}%
\end{pgfscope}%
\begin{pgfscope}%
\pgfsys@transformshift{1.764014in}{0.684145in}%
\pgfsys@useobject{currentmarker}{}%
\end{pgfscope}%
\begin{pgfscope}%
\pgfsys@transformshift{1.765272in}{0.677040in}%
\pgfsys@useobject{currentmarker}{}%
\end{pgfscope}%
\begin{pgfscope}%
\pgfsys@transformshift{1.766524in}{0.718769in}%
\pgfsys@useobject{currentmarker}{}%
\end{pgfscope}%
\begin{pgfscope}%
\pgfsys@transformshift{1.767769in}{0.725821in}%
\pgfsys@useobject{currentmarker}{}%
\end{pgfscope}%
\begin{pgfscope}%
\pgfsys@transformshift{1.769006in}{0.705053in}%
\pgfsys@useobject{currentmarker}{}%
\end{pgfscope}%
\begin{pgfscope}%
\pgfsys@transformshift{1.770237in}{0.680563in}%
\pgfsys@useobject{currentmarker}{}%
\end{pgfscope}%
\begin{pgfscope}%
\pgfsys@transformshift{1.771462in}{0.687243in}%
\pgfsys@useobject{currentmarker}{}%
\end{pgfscope}%
\begin{pgfscope}%
\pgfsys@transformshift{1.772679in}{0.718924in}%
\pgfsys@useobject{currentmarker}{}%
\end{pgfscope}%
\begin{pgfscope}%
\pgfsys@transformshift{1.773890in}{0.690900in}%
\pgfsys@useobject{currentmarker}{}%
\end{pgfscope}%
\begin{pgfscope}%
\pgfsys@transformshift{1.775095in}{0.685715in}%
\pgfsys@useobject{currentmarker}{}%
\end{pgfscope}%
\begin{pgfscope}%
\pgfsys@transformshift{1.776293in}{0.728935in}%
\pgfsys@useobject{currentmarker}{}%
\end{pgfscope}%
\begin{pgfscope}%
\pgfsys@transformshift{1.777485in}{0.711548in}%
\pgfsys@useobject{currentmarker}{}%
\end{pgfscope}%
\begin{pgfscope}%
\pgfsys@transformshift{1.778670in}{0.676584in}%
\pgfsys@useobject{currentmarker}{}%
\end{pgfscope}%
\begin{pgfscope}%
\pgfsys@transformshift{1.779849in}{0.733642in}%
\pgfsys@useobject{currentmarker}{}%
\end{pgfscope}%
\begin{pgfscope}%
\pgfsys@transformshift{1.781023in}{0.723508in}%
\pgfsys@useobject{currentmarker}{}%
\end{pgfscope}%
\begin{pgfscope}%
\pgfsys@transformshift{1.782190in}{0.678302in}%
\pgfsys@useobject{currentmarker}{}%
\end{pgfscope}%
\begin{pgfscope}%
\pgfsys@transformshift{1.783351in}{0.692014in}%
\pgfsys@useobject{currentmarker}{}%
\end{pgfscope}%
\begin{pgfscope}%
\pgfsys@transformshift{1.784506in}{0.719710in}%
\pgfsys@useobject{currentmarker}{}%
\end{pgfscope}%
\begin{pgfscope}%
\pgfsys@transformshift{1.785655in}{0.640502in}%
\pgfsys@useobject{currentmarker}{}%
\end{pgfscope}%
\begin{pgfscope}%
\pgfsys@transformshift{1.786798in}{0.630401in}%
\pgfsys@useobject{currentmarker}{}%
\end{pgfscope}%
\begin{pgfscope}%
\pgfsys@transformshift{1.787936in}{0.702047in}%
\pgfsys@useobject{currentmarker}{}%
\end{pgfscope}%
\begin{pgfscope}%
\pgfsys@transformshift{1.789067in}{0.730444in}%
\pgfsys@useobject{currentmarker}{}%
\end{pgfscope}%
\begin{pgfscope}%
\pgfsys@transformshift{1.790193in}{0.735588in}%
\pgfsys@useobject{currentmarker}{}%
\end{pgfscope}%
\begin{pgfscope}%
\pgfsys@transformshift{1.791314in}{0.761941in}%
\pgfsys@useobject{currentmarker}{}%
\end{pgfscope}%
\begin{pgfscope}%
\pgfsys@transformshift{1.792429in}{0.762888in}%
\pgfsys@useobject{currentmarker}{}%
\end{pgfscope}%
\begin{pgfscope}%
\pgfsys@transformshift{1.793538in}{0.712199in}%
\pgfsys@useobject{currentmarker}{}%
\end{pgfscope}%
\begin{pgfscope}%
\pgfsys@transformshift{1.794642in}{0.672417in}%
\pgfsys@useobject{currentmarker}{}%
\end{pgfscope}%
\begin{pgfscope}%
\pgfsys@transformshift{1.795741in}{0.631643in}%
\pgfsys@useobject{currentmarker}{}%
\end{pgfscope}%
\begin{pgfscope}%
\pgfsys@transformshift{1.796834in}{0.657503in}%
\pgfsys@useobject{currentmarker}{}%
\end{pgfscope}%
\begin{pgfscope}%
\pgfsys@transformshift{1.797922in}{0.711392in}%
\pgfsys@useobject{currentmarker}{}%
\end{pgfscope}%
\begin{pgfscope}%
\pgfsys@transformshift{1.799004in}{0.690356in}%
\pgfsys@useobject{currentmarker}{}%
\end{pgfscope}%
\begin{pgfscope}%
\pgfsys@transformshift{1.800082in}{0.684105in}%
\pgfsys@useobject{currentmarker}{}%
\end{pgfscope}%
\begin{pgfscope}%
\pgfsys@transformshift{1.801154in}{0.704880in}%
\pgfsys@useobject{currentmarker}{}%
\end{pgfscope}%
\begin{pgfscope}%
\pgfsys@transformshift{1.802221in}{0.653204in}%
\pgfsys@useobject{currentmarker}{}%
\end{pgfscope}%
\begin{pgfscope}%
\pgfsys@transformshift{1.803284in}{0.669124in}%
\pgfsys@useobject{currentmarker}{}%
\end{pgfscope}%
\begin{pgfscope}%
\pgfsys@transformshift{1.804341in}{0.661449in}%
\pgfsys@useobject{currentmarker}{}%
\end{pgfscope}%
\begin{pgfscope}%
\pgfsys@transformshift{1.805393in}{0.666156in}%
\pgfsys@useobject{currentmarker}{}%
\end{pgfscope}%
\begin{pgfscope}%
\pgfsys@transformshift{1.806440in}{0.680729in}%
\pgfsys@useobject{currentmarker}{}%
\end{pgfscope}%
\begin{pgfscope}%
\pgfsys@transformshift{1.807483in}{0.682014in}%
\pgfsys@useobject{currentmarker}{}%
\end{pgfscope}%
\begin{pgfscope}%
\pgfsys@transformshift{1.808520in}{0.694038in}%
\pgfsys@useobject{currentmarker}{}%
\end{pgfscope}%
\begin{pgfscope}%
\pgfsys@transformshift{1.809553in}{0.720240in}%
\pgfsys@useobject{currentmarker}{}%
\end{pgfscope}%
\begin{pgfscope}%
\pgfsys@transformshift{1.810581in}{0.689912in}%
\pgfsys@useobject{currentmarker}{}%
\end{pgfscope}%
\begin{pgfscope}%
\pgfsys@transformshift{1.811605in}{0.688881in}%
\pgfsys@useobject{currentmarker}{}%
\end{pgfscope}%
\begin{pgfscope}%
\pgfsys@transformshift{1.812624in}{0.660018in}%
\pgfsys@useobject{currentmarker}{}%
\end{pgfscope}%
\begin{pgfscope}%
\pgfsys@transformshift{1.813638in}{0.651573in}%
\pgfsys@useobject{currentmarker}{}%
\end{pgfscope}%
\begin{pgfscope}%
\pgfsys@transformshift{1.814648in}{0.677834in}%
\pgfsys@useobject{currentmarker}{}%
\end{pgfscope}%
\begin{pgfscope}%
\pgfsys@transformshift{1.815653in}{0.701839in}%
\pgfsys@useobject{currentmarker}{}%
\end{pgfscope}%
\begin{pgfscope}%
\pgfsys@transformshift{1.816653in}{0.697798in}%
\pgfsys@useobject{currentmarker}{}%
\end{pgfscope}%
\begin{pgfscope}%
\pgfsys@transformshift{1.817650in}{0.682010in}%
\pgfsys@useobject{currentmarker}{}%
\end{pgfscope}%
\begin{pgfscope}%
\pgfsys@transformshift{1.818642in}{0.645968in}%
\pgfsys@useobject{currentmarker}{}%
\end{pgfscope}%
\begin{pgfscope}%
\pgfsys@transformshift{1.819629in}{0.643957in}%
\pgfsys@useobject{currentmarker}{}%
\end{pgfscope}%
\begin{pgfscope}%
\pgfsys@transformshift{1.820612in}{0.685035in}%
\pgfsys@useobject{currentmarker}{}%
\end{pgfscope}%
\begin{pgfscope}%
\pgfsys@transformshift{1.821591in}{0.633501in}%
\pgfsys@useobject{currentmarker}{}%
\end{pgfscope}%
\begin{pgfscope}%
\pgfsys@transformshift{1.822566in}{0.598493in}%
\pgfsys@useobject{currentmarker}{}%
\end{pgfscope}%
\begin{pgfscope}%
\pgfsys@transformshift{1.823536in}{0.660533in}%
\pgfsys@useobject{currentmarker}{}%
\end{pgfscope}%
\begin{pgfscope}%
\pgfsys@transformshift{1.824502in}{0.682253in}%
\pgfsys@useobject{currentmarker}{}%
\end{pgfscope}%
\begin{pgfscope}%
\pgfsys@transformshift{1.825464in}{0.665096in}%
\pgfsys@useobject{currentmarker}{}%
\end{pgfscope}%
\begin{pgfscope}%
\pgfsys@transformshift{1.826423in}{0.720235in}%
\pgfsys@useobject{currentmarker}{}%
\end{pgfscope}%
\begin{pgfscope}%
\pgfsys@transformshift{1.827376in}{0.733376in}%
\pgfsys@useobject{currentmarker}{}%
\end{pgfscope}%
\begin{pgfscope}%
\pgfsys@transformshift{1.828326in}{0.682711in}%
\pgfsys@useobject{currentmarker}{}%
\end{pgfscope}%
\begin{pgfscope}%
\pgfsys@transformshift{1.829272in}{0.689959in}%
\pgfsys@useobject{currentmarker}{}%
\end{pgfscope}%
\begin{pgfscope}%
\pgfsys@transformshift{1.830214in}{0.679436in}%
\pgfsys@useobject{currentmarker}{}%
\end{pgfscope}%
\begin{pgfscope}%
\pgfsys@transformshift{1.831152in}{0.698813in}%
\pgfsys@useobject{currentmarker}{}%
\end{pgfscope}%
\begin{pgfscope}%
\pgfsys@transformshift{1.832086in}{0.709568in}%
\pgfsys@useobject{currentmarker}{}%
\end{pgfscope}%
\begin{pgfscope}%
\pgfsys@transformshift{1.833017in}{0.696279in}%
\pgfsys@useobject{currentmarker}{}%
\end{pgfscope}%
\begin{pgfscope}%
\pgfsys@transformshift{1.833943in}{0.636491in}%
\pgfsys@useobject{currentmarker}{}%
\end{pgfscope}%
\begin{pgfscope}%
\pgfsys@transformshift{1.834866in}{0.649742in}%
\pgfsys@useobject{currentmarker}{}%
\end{pgfscope}%
\begin{pgfscope}%
\pgfsys@transformshift{1.835784in}{0.661139in}%
\pgfsys@useobject{currentmarker}{}%
\end{pgfscope}%
\begin{pgfscope}%
\pgfsys@transformshift{1.836699in}{0.649508in}%
\pgfsys@useobject{currentmarker}{}%
\end{pgfscope}%
\begin{pgfscope}%
\pgfsys@transformshift{1.837611in}{0.665376in}%
\pgfsys@useobject{currentmarker}{}%
\end{pgfscope}%
\begin{pgfscope}%
\pgfsys@transformshift{1.838518in}{0.713712in}%
\pgfsys@useobject{currentmarker}{}%
\end{pgfscope}%
\begin{pgfscope}%
\pgfsys@transformshift{1.839423in}{0.722228in}%
\pgfsys@useobject{currentmarker}{}%
\end{pgfscope}%
\begin{pgfscope}%
\pgfsys@transformshift{1.840323in}{0.656915in}%
\pgfsys@useobject{currentmarker}{}%
\end{pgfscope}%
\begin{pgfscope}%
\pgfsys@transformshift{1.841220in}{0.668250in}%
\pgfsys@useobject{currentmarker}{}%
\end{pgfscope}%
\begin{pgfscope}%
\pgfsys@transformshift{1.842113in}{0.701839in}%
\pgfsys@useobject{currentmarker}{}%
\end{pgfscope}%
\begin{pgfscope}%
\pgfsys@transformshift{1.843003in}{0.715192in}%
\pgfsys@useobject{currentmarker}{}%
\end{pgfscope}%
\begin{pgfscope}%
\pgfsys@transformshift{1.843889in}{0.706440in}%
\pgfsys@useobject{currentmarker}{}%
\end{pgfscope}%
\begin{pgfscope}%
\pgfsys@transformshift{1.844772in}{0.724235in}%
\pgfsys@useobject{currentmarker}{}%
\end{pgfscope}%
\begin{pgfscope}%
\pgfsys@transformshift{1.845651in}{0.732808in}%
\pgfsys@useobject{currentmarker}{}%
\end{pgfscope}%
\begin{pgfscope}%
\pgfsys@transformshift{1.846527in}{0.704651in}%
\pgfsys@useobject{currentmarker}{}%
\end{pgfscope}%
\begin{pgfscope}%
\pgfsys@transformshift{1.847399in}{0.698406in}%
\pgfsys@useobject{currentmarker}{}%
\end{pgfscope}%
\begin{pgfscope}%
\pgfsys@transformshift{1.848268in}{0.716320in}%
\pgfsys@useobject{currentmarker}{}%
\end{pgfscope}%
\begin{pgfscope}%
\pgfsys@transformshift{1.849134in}{0.673464in}%
\pgfsys@useobject{currentmarker}{}%
\end{pgfscope}%
\begin{pgfscope}%
\pgfsys@transformshift{1.849996in}{0.706274in}%
\pgfsys@useobject{currentmarker}{}%
\end{pgfscope}%
\begin{pgfscope}%
\pgfsys@transformshift{1.850855in}{0.734680in}%
\pgfsys@useobject{currentmarker}{}%
\end{pgfscope}%
\begin{pgfscope}%
\pgfsys@transformshift{1.851711in}{0.691395in}%
\pgfsys@useobject{currentmarker}{}%
\end{pgfscope}%
\begin{pgfscope}%
\pgfsys@transformshift{1.852564in}{0.691859in}%
\pgfsys@useobject{currentmarker}{}%
\end{pgfscope}%
\begin{pgfscope}%
\pgfsys@transformshift{1.853413in}{0.689912in}%
\pgfsys@useobject{currentmarker}{}%
\end{pgfscope}%
\begin{pgfscope}%
\pgfsys@transformshift{1.854259in}{0.665906in}%
\pgfsys@useobject{currentmarker}{}%
\end{pgfscope}%
\begin{pgfscope}%
\pgfsys@transformshift{1.855102in}{0.629813in}%
\pgfsys@useobject{currentmarker}{}%
\end{pgfscope}%
\begin{pgfscope}%
\pgfsys@transformshift{1.855942in}{0.607697in}%
\pgfsys@useobject{currentmarker}{}%
\end{pgfscope}%
\begin{pgfscope}%
\pgfsys@transformshift{1.856779in}{0.642579in}%
\pgfsys@useobject{currentmarker}{}%
\end{pgfscope}%
\begin{pgfscope}%
\pgfsys@transformshift{1.857612in}{0.669067in}%
\pgfsys@useobject{currentmarker}{}%
\end{pgfscope}%
\begin{pgfscope}%
\pgfsys@transformshift{1.858443in}{0.661314in}%
\pgfsys@useobject{currentmarker}{}%
\end{pgfscope}%
\begin{pgfscope}%
\pgfsys@transformshift{1.859270in}{0.679590in}%
\pgfsys@useobject{currentmarker}{}%
\end{pgfscope}%
\begin{pgfscope}%
\pgfsys@transformshift{1.860095in}{0.714941in}%
\pgfsys@useobject{currentmarker}{}%
\end{pgfscope}%
\begin{pgfscope}%
\pgfsys@transformshift{1.860916in}{0.650290in}%
\pgfsys@useobject{currentmarker}{}%
\end{pgfscope}%
\begin{pgfscope}%
\pgfsys@transformshift{1.861735in}{0.644860in}%
\pgfsys@useobject{currentmarker}{}%
\end{pgfscope}%
\begin{pgfscope}%
\pgfsys@transformshift{1.862550in}{0.656477in}%
\pgfsys@useobject{currentmarker}{}%
\end{pgfscope}%
\begin{pgfscope}%
\pgfsys@transformshift{1.863362in}{0.697925in}%
\pgfsys@useobject{currentmarker}{}%
\end{pgfscope}%
\begin{pgfscope}%
\pgfsys@transformshift{1.864172in}{0.687373in}%
\pgfsys@useobject{currentmarker}{}%
\end{pgfscope}%
\begin{pgfscope}%
\pgfsys@transformshift{1.864979in}{0.688657in}%
\pgfsys@useobject{currentmarker}{}%
\end{pgfscope}%
\begin{pgfscope}%
\pgfsys@transformshift{1.865782in}{0.683943in}%
\pgfsys@useobject{currentmarker}{}%
\end{pgfscope}%
\begin{pgfscope}%
\pgfsys@transformshift{1.866583in}{0.658410in}%
\pgfsys@useobject{currentmarker}{}%
\end{pgfscope}%
\begin{pgfscope}%
\pgfsys@transformshift{1.867381in}{0.695066in}%
\pgfsys@useobject{currentmarker}{}%
\end{pgfscope}%
\begin{pgfscope}%
\pgfsys@transformshift{1.868177in}{0.689595in}%
\pgfsys@useobject{currentmarker}{}%
\end{pgfscope}%
\begin{pgfscope}%
\pgfsys@transformshift{1.868969in}{0.725096in}%
\pgfsys@useobject{currentmarker}{}%
\end{pgfscope}%
\begin{pgfscope}%
\pgfsys@transformshift{1.869759in}{0.699007in}%
\pgfsys@useobject{currentmarker}{}%
\end{pgfscope}%
\begin{pgfscope}%
\pgfsys@transformshift{1.870546in}{0.609878in}%
\pgfsys@useobject{currentmarker}{}%
\end{pgfscope}%
\begin{pgfscope}%
\pgfsys@transformshift{1.871330in}{0.672790in}%
\pgfsys@useobject{currentmarker}{}%
\end{pgfscope}%
\begin{pgfscope}%
\pgfsys@transformshift{1.872111in}{0.696135in}%
\pgfsys@useobject{currentmarker}{}%
\end{pgfscope}%
\begin{pgfscope}%
\pgfsys@transformshift{1.872890in}{0.656174in}%
\pgfsys@useobject{currentmarker}{}%
\end{pgfscope}%
\begin{pgfscope}%
\pgfsys@transformshift{1.873666in}{0.638168in}%
\pgfsys@useobject{currentmarker}{}%
\end{pgfscope}%
\begin{pgfscope}%
\pgfsys@transformshift{1.874439in}{0.650093in}%
\pgfsys@useobject{currentmarker}{}%
\end{pgfscope}%
\begin{pgfscope}%
\pgfsys@transformshift{1.875210in}{0.704595in}%
\pgfsys@useobject{currentmarker}{}%
\end{pgfscope}%
\begin{pgfscope}%
\pgfsys@transformshift{1.875978in}{0.723683in}%
\pgfsys@useobject{currentmarker}{}%
\end{pgfscope}%
\begin{pgfscope}%
\pgfsys@transformshift{1.876743in}{0.718174in}%
\pgfsys@useobject{currentmarker}{}%
\end{pgfscope}%
\begin{pgfscope}%
\pgfsys@transformshift{1.877506in}{0.655121in}%
\pgfsys@useobject{currentmarker}{}%
\end{pgfscope}%
\begin{pgfscope}%
\pgfsys@transformshift{1.878266in}{0.652059in}%
\pgfsys@useobject{currentmarker}{}%
\end{pgfscope}%
\begin{pgfscope}%
\pgfsys@transformshift{1.879024in}{0.662019in}%
\pgfsys@useobject{currentmarker}{}%
\end{pgfscope}%
\begin{pgfscope}%
\pgfsys@transformshift{1.879779in}{0.696260in}%
\pgfsys@useobject{currentmarker}{}%
\end{pgfscope}%
\begin{pgfscope}%
\pgfsys@transformshift{1.880532in}{0.696172in}%
\pgfsys@useobject{currentmarker}{}%
\end{pgfscope}%
\begin{pgfscope}%
\pgfsys@transformshift{1.881282in}{0.663663in}%
\pgfsys@useobject{currentmarker}{}%
\end{pgfscope}%
\begin{pgfscope}%
\pgfsys@transformshift{1.882029in}{0.700161in}%
\pgfsys@useobject{currentmarker}{}%
\end{pgfscope}%
\begin{pgfscope}%
\pgfsys@transformshift{1.882774in}{0.696647in}%
\pgfsys@useobject{currentmarker}{}%
\end{pgfscope}%
\begin{pgfscope}%
\pgfsys@transformshift{1.883517in}{0.665000in}%
\pgfsys@useobject{currentmarker}{}%
\end{pgfscope}%
\begin{pgfscope}%
\pgfsys@transformshift{1.884257in}{0.716636in}%
\pgfsys@useobject{currentmarker}{}%
\end{pgfscope}%
\begin{pgfscope}%
\pgfsys@transformshift{1.884995in}{0.715462in}%
\pgfsys@useobject{currentmarker}{}%
\end{pgfscope}%
\begin{pgfscope}%
\pgfsys@transformshift{1.885730in}{0.672654in}%
\pgfsys@useobject{currentmarker}{}%
\end{pgfscope}%
\begin{pgfscope}%
\pgfsys@transformshift{1.886463in}{0.656649in}%
\pgfsys@useobject{currentmarker}{}%
\end{pgfscope}%
\begin{pgfscope}%
\pgfsys@transformshift{1.887194in}{0.636036in}%
\pgfsys@useobject{currentmarker}{}%
\end{pgfscope}%
\begin{pgfscope}%
\pgfsys@transformshift{1.887922in}{0.723125in}%
\pgfsys@useobject{currentmarker}{}%
\end{pgfscope}%
\begin{pgfscope}%
\pgfsys@transformshift{1.888648in}{0.735564in}%
\pgfsys@useobject{currentmarker}{}%
\end{pgfscope}%
\begin{pgfscope}%
\pgfsys@transformshift{1.889372in}{0.690903in}%
\pgfsys@useobject{currentmarker}{}%
\end{pgfscope}%
\begin{pgfscope}%
\pgfsys@transformshift{1.890093in}{0.673500in}%
\pgfsys@useobject{currentmarker}{}%
\end{pgfscope}%
\begin{pgfscope}%
\pgfsys@transformshift{1.890812in}{0.672641in}%
\pgfsys@useobject{currentmarker}{}%
\end{pgfscope}%
\begin{pgfscope}%
\pgfsys@transformshift{1.891528in}{0.728284in}%
\pgfsys@useobject{currentmarker}{}%
\end{pgfscope}%
\begin{pgfscope}%
\pgfsys@transformshift{1.892243in}{0.706159in}%
\pgfsys@useobject{currentmarker}{}%
\end{pgfscope}%
\begin{pgfscope}%
\pgfsys@transformshift{1.892955in}{0.669587in}%
\pgfsys@useobject{currentmarker}{}%
\end{pgfscope}%
\begin{pgfscope}%
\pgfsys@transformshift{1.893664in}{0.723837in}%
\pgfsys@useobject{currentmarker}{}%
\end{pgfscope}%
\begin{pgfscope}%
\pgfsys@transformshift{1.894372in}{0.719862in}%
\pgfsys@useobject{currentmarker}{}%
\end{pgfscope}%
\begin{pgfscope}%
\pgfsys@transformshift{1.895077in}{0.707533in}%
\pgfsys@useobject{currentmarker}{}%
\end{pgfscope}%
\begin{pgfscope}%
\pgfsys@transformshift{1.895780in}{0.631641in}%
\pgfsys@useobject{currentmarker}{}%
\end{pgfscope}%
\begin{pgfscope}%
\pgfsys@transformshift{1.896481in}{0.671742in}%
\pgfsys@useobject{currentmarker}{}%
\end{pgfscope}%
\begin{pgfscope}%
\pgfsys@transformshift{1.897180in}{0.681387in}%
\pgfsys@useobject{currentmarker}{}%
\end{pgfscope}%
\begin{pgfscope}%
\pgfsys@transformshift{1.897877in}{0.688453in}%
\pgfsys@useobject{currentmarker}{}%
\end{pgfscope}%
\begin{pgfscope}%
\pgfsys@transformshift{1.898571in}{0.682889in}%
\pgfsys@useobject{currentmarker}{}%
\end{pgfscope}%
\begin{pgfscope}%
\pgfsys@transformshift{1.899264in}{0.682615in}%
\pgfsys@useobject{currentmarker}{}%
\end{pgfscope}%
\begin{pgfscope}%
\pgfsys@transformshift{1.899954in}{0.732756in}%
\pgfsys@useobject{currentmarker}{}%
\end{pgfscope}%
\begin{pgfscope}%
\pgfsys@transformshift{1.900642in}{0.745336in}%
\pgfsys@useobject{currentmarker}{}%
\end{pgfscope}%
\begin{pgfscope}%
\pgfsys@transformshift{1.901328in}{0.695922in}%
\pgfsys@useobject{currentmarker}{}%
\end{pgfscope}%
\begin{pgfscope}%
\pgfsys@transformshift{1.902012in}{0.672189in}%
\pgfsys@useobject{currentmarker}{}%
\end{pgfscope}%
\begin{pgfscope}%
\pgfsys@transformshift{1.902693in}{0.697951in}%
\pgfsys@useobject{currentmarker}{}%
\end{pgfscope}%
\begin{pgfscope}%
\pgfsys@transformshift{1.903373in}{0.777981in}%
\pgfsys@useobject{currentmarker}{}%
\end{pgfscope}%
\begin{pgfscope}%
\pgfsys@transformshift{1.904051in}{0.767909in}%
\pgfsys@useobject{currentmarker}{}%
\end{pgfscope}%
\begin{pgfscope}%
\pgfsys@transformshift{1.904726in}{0.730316in}%
\pgfsys@useobject{currentmarker}{}%
\end{pgfscope}%
\begin{pgfscope}%
\pgfsys@transformshift{1.905400in}{0.732346in}%
\pgfsys@useobject{currentmarker}{}%
\end{pgfscope}%
\begin{pgfscope}%
\pgfsys@transformshift{1.906072in}{0.683454in}%
\pgfsys@useobject{currentmarker}{}%
\end{pgfscope}%
\begin{pgfscope}%
\pgfsys@transformshift{1.906741in}{0.666853in}%
\pgfsys@useobject{currentmarker}{}%
\end{pgfscope}%
\begin{pgfscope}%
\pgfsys@transformshift{1.907409in}{0.669589in}%
\pgfsys@useobject{currentmarker}{}%
\end{pgfscope}%
\begin{pgfscope}%
\pgfsys@transformshift{1.908075in}{0.647607in}%
\pgfsys@useobject{currentmarker}{}%
\end{pgfscope}%
\begin{pgfscope}%
\pgfsys@transformshift{1.908738in}{0.641904in}%
\pgfsys@useobject{currentmarker}{}%
\end{pgfscope}%
\begin{pgfscope}%
\pgfsys@transformshift{1.909400in}{0.654571in}%
\pgfsys@useobject{currentmarker}{}%
\end{pgfscope}%
\begin{pgfscope}%
\pgfsys@transformshift{1.910060in}{0.647251in}%
\pgfsys@useobject{currentmarker}{}%
\end{pgfscope}%
\begin{pgfscope}%
\pgfsys@transformshift{1.910717in}{0.650332in}%
\pgfsys@useobject{currentmarker}{}%
\end{pgfscope}%
\begin{pgfscope}%
\pgfsys@transformshift{1.911373in}{0.704919in}%
\pgfsys@useobject{currentmarker}{}%
\end{pgfscope}%
\begin{pgfscope}%
\pgfsys@transformshift{1.912027in}{0.699830in}%
\pgfsys@useobject{currentmarker}{}%
\end{pgfscope}%
\begin{pgfscope}%
\pgfsys@transformshift{1.912680in}{0.735111in}%
\pgfsys@useobject{currentmarker}{}%
\end{pgfscope}%
\begin{pgfscope}%
\pgfsys@transformshift{1.913330in}{0.745838in}%
\pgfsys@useobject{currentmarker}{}%
\end{pgfscope}%
\begin{pgfscope}%
\pgfsys@transformshift{1.913978in}{0.654917in}%
\pgfsys@useobject{currentmarker}{}%
\end{pgfscope}%
\begin{pgfscope}%
\pgfsys@transformshift{1.914625in}{0.705131in}%
\pgfsys@useobject{currentmarker}{}%
\end{pgfscope}%
\begin{pgfscope}%
\pgfsys@transformshift{1.915269in}{0.693982in}%
\pgfsys@useobject{currentmarker}{}%
\end{pgfscope}%
\begin{pgfscope}%
\pgfsys@transformshift{1.915912in}{0.676394in}%
\pgfsys@useobject{currentmarker}{}%
\end{pgfscope}%
\begin{pgfscope}%
\pgfsys@transformshift{1.916553in}{0.655772in}%
\pgfsys@useobject{currentmarker}{}%
\end{pgfscope}%
\begin{pgfscope}%
\pgfsys@transformshift{1.917192in}{0.584977in}%
\pgfsys@useobject{currentmarker}{}%
\end{pgfscope}%
\begin{pgfscope}%
\pgfsys@transformshift{1.917829in}{0.678108in}%
\pgfsys@useobject{currentmarker}{}%
\end{pgfscope}%
\begin{pgfscope}%
\pgfsys@transformshift{1.918465in}{0.708965in}%
\pgfsys@useobject{currentmarker}{}%
\end{pgfscope}%
\begin{pgfscope}%
\pgfsys@transformshift{1.919099in}{0.681839in}%
\pgfsys@useobject{currentmarker}{}%
\end{pgfscope}%
\begin{pgfscope}%
\pgfsys@transformshift{1.919731in}{0.680839in}%
\pgfsys@useobject{currentmarker}{}%
\end{pgfscope}%
\begin{pgfscope}%
\pgfsys@transformshift{1.920361in}{0.681939in}%
\pgfsys@useobject{currentmarker}{}%
\end{pgfscope}%
\begin{pgfscope}%
\pgfsys@transformshift{1.920989in}{0.672246in}%
\pgfsys@useobject{currentmarker}{}%
\end{pgfscope}%
\begin{pgfscope}%
\pgfsys@transformshift{1.921616in}{0.625546in}%
\pgfsys@useobject{currentmarker}{}%
\end{pgfscope}%
\begin{pgfscope}%
\pgfsys@transformshift{1.922241in}{0.679980in}%
\pgfsys@useobject{currentmarker}{}%
\end{pgfscope}%
\begin{pgfscope}%
\pgfsys@transformshift{1.922864in}{0.628091in}%
\pgfsys@useobject{currentmarker}{}%
\end{pgfscope}%
\begin{pgfscope}%
\pgfsys@transformshift{1.923485in}{0.691263in}%
\pgfsys@useobject{currentmarker}{}%
\end{pgfscope}%
\begin{pgfscope}%
\pgfsys@transformshift{1.924105in}{0.722589in}%
\pgfsys@useobject{currentmarker}{}%
\end{pgfscope}%
\begin{pgfscope}%
\pgfsys@transformshift{1.924723in}{0.693513in}%
\pgfsys@useobject{currentmarker}{}%
\end{pgfscope}%
\begin{pgfscope}%
\pgfsys@transformshift{1.925339in}{0.679451in}%
\pgfsys@useobject{currentmarker}{}%
\end{pgfscope}%
\begin{pgfscope}%
\pgfsys@transformshift{1.925954in}{0.641727in}%
\pgfsys@useobject{currentmarker}{}%
\end{pgfscope}%
\begin{pgfscope}%
\pgfsys@transformshift{1.926567in}{0.677110in}%
\pgfsys@useobject{currentmarker}{}%
\end{pgfscope}%
\begin{pgfscope}%
\pgfsys@transformshift{1.927178in}{0.648725in}%
\pgfsys@useobject{currentmarker}{}%
\end{pgfscope}%
\begin{pgfscope}%
\pgfsys@transformshift{1.927788in}{0.631012in}%
\pgfsys@useobject{currentmarker}{}%
\end{pgfscope}%
\begin{pgfscope}%
\pgfsys@transformshift{1.928396in}{0.613763in}%
\pgfsys@useobject{currentmarker}{}%
\end{pgfscope}%
\begin{pgfscope}%
\pgfsys@transformshift{1.929002in}{0.653333in}%
\pgfsys@useobject{currentmarker}{}%
\end{pgfscope}%
\begin{pgfscope}%
\pgfsys@transformshift{1.929607in}{0.703955in}%
\pgfsys@useobject{currentmarker}{}%
\end{pgfscope}%
\begin{pgfscope}%
\pgfsys@transformshift{1.930210in}{0.682711in}%
\pgfsys@useobject{currentmarker}{}%
\end{pgfscope}%
\begin{pgfscope}%
\pgfsys@transformshift{1.930811in}{0.650598in}%
\pgfsys@useobject{currentmarker}{}%
\end{pgfscope}%
\begin{pgfscope}%
\pgfsys@transformshift{1.931411in}{0.671542in}%
\pgfsys@useobject{currentmarker}{}%
\end{pgfscope}%
\begin{pgfscope}%
\pgfsys@transformshift{1.932010in}{0.668796in}%
\pgfsys@useobject{currentmarker}{}%
\end{pgfscope}%
\begin{pgfscope}%
\pgfsys@transformshift{1.932606in}{0.700689in}%
\pgfsys@useobject{currentmarker}{}%
\end{pgfscope}%
\begin{pgfscope}%
\pgfsys@transformshift{1.933201in}{0.679642in}%
\pgfsys@useobject{currentmarker}{}%
\end{pgfscope}%
\begin{pgfscope}%
\pgfsys@transformshift{1.933795in}{0.667919in}%
\pgfsys@useobject{currentmarker}{}%
\end{pgfscope}%
\begin{pgfscope}%
\pgfsys@transformshift{1.934387in}{0.684634in}%
\pgfsys@useobject{currentmarker}{}%
\end{pgfscope}%
\begin{pgfscope}%
\pgfsys@transformshift{1.934977in}{0.661890in}%
\pgfsys@useobject{currentmarker}{}%
\end{pgfscope}%
\begin{pgfscope}%
\pgfsys@transformshift{1.935566in}{0.590834in}%
\pgfsys@useobject{currentmarker}{}%
\end{pgfscope}%
\begin{pgfscope}%
\pgfsys@transformshift{1.936154in}{0.678117in}%
\pgfsys@useobject{currentmarker}{}%
\end{pgfscope}%
\begin{pgfscope}%
\pgfsys@transformshift{1.936739in}{0.679335in}%
\pgfsys@useobject{currentmarker}{}%
\end{pgfscope}%
\begin{pgfscope}%
\pgfsys@transformshift{1.937324in}{0.674719in}%
\pgfsys@useobject{currentmarker}{}%
\end{pgfscope}%
\begin{pgfscope}%
\pgfsys@transformshift{1.937906in}{0.713566in}%
\pgfsys@useobject{currentmarker}{}%
\end{pgfscope}%
\begin{pgfscope}%
\pgfsys@transformshift{1.938488in}{0.720304in}%
\pgfsys@useobject{currentmarker}{}%
\end{pgfscope}%
\begin{pgfscope}%
\pgfsys@transformshift{1.939067in}{0.728576in}%
\pgfsys@useobject{currentmarker}{}%
\end{pgfscope}%
\begin{pgfscope}%
\pgfsys@transformshift{1.939646in}{0.649249in}%
\pgfsys@useobject{currentmarker}{}%
\end{pgfscope}%
\begin{pgfscope}%
\pgfsys@transformshift{1.940222in}{0.657277in}%
\pgfsys@useobject{currentmarker}{}%
\end{pgfscope}%
\begin{pgfscope}%
\pgfsys@transformshift{1.940798in}{0.644641in}%
\pgfsys@useobject{currentmarker}{}%
\end{pgfscope}%
\begin{pgfscope}%
\pgfsys@transformshift{1.941371in}{0.671918in}%
\pgfsys@useobject{currentmarker}{}%
\end{pgfscope}%
\begin{pgfscope}%
\pgfsys@transformshift{1.941944in}{0.688440in}%
\pgfsys@useobject{currentmarker}{}%
\end{pgfscope}%
\begin{pgfscope}%
\pgfsys@transformshift{1.942515in}{0.693452in}%
\pgfsys@useobject{currentmarker}{}%
\end{pgfscope}%
\begin{pgfscope}%
\pgfsys@transformshift{1.943084in}{0.704868in}%
\pgfsys@useobject{currentmarker}{}%
\end{pgfscope}%
\begin{pgfscope}%
\pgfsys@transformshift{1.943652in}{0.697632in}%
\pgfsys@useobject{currentmarker}{}%
\end{pgfscope}%
\begin{pgfscope}%
\pgfsys@transformshift{1.944219in}{0.694514in}%
\pgfsys@useobject{currentmarker}{}%
\end{pgfscope}%
\begin{pgfscope}%
\pgfsys@transformshift{1.944784in}{0.690763in}%
\pgfsys@useobject{currentmarker}{}%
\end{pgfscope}%
\begin{pgfscope}%
\pgfsys@transformshift{1.945348in}{0.701112in}%
\pgfsys@useobject{currentmarker}{}%
\end{pgfscope}%
\begin{pgfscope}%
\pgfsys@transformshift{1.945910in}{0.705555in}%
\pgfsys@useobject{currentmarker}{}%
\end{pgfscope}%
\begin{pgfscope}%
\pgfsys@transformshift{1.946471in}{0.681147in}%
\pgfsys@useobject{currentmarker}{}%
\end{pgfscope}%
\begin{pgfscope}%
\pgfsys@transformshift{1.947031in}{0.665070in}%
\pgfsys@useobject{currentmarker}{}%
\end{pgfscope}%
\begin{pgfscope}%
\pgfsys@transformshift{1.947589in}{0.662996in}%
\pgfsys@useobject{currentmarker}{}%
\end{pgfscope}%
\begin{pgfscope}%
\pgfsys@transformshift{1.948145in}{0.726122in}%
\pgfsys@useobject{currentmarker}{}%
\end{pgfscope}%
\begin{pgfscope}%
\pgfsys@transformshift{1.948701in}{0.714915in}%
\pgfsys@useobject{currentmarker}{}%
\end{pgfscope}%
\begin{pgfscope}%
\pgfsys@transformshift{1.949255in}{0.699401in}%
\pgfsys@useobject{currentmarker}{}%
\end{pgfscope}%
\begin{pgfscope}%
\pgfsys@transformshift{1.949807in}{0.685883in}%
\pgfsys@useobject{currentmarker}{}%
\end{pgfscope}%
\begin{pgfscope}%
\pgfsys@transformshift{1.950359in}{0.707987in}%
\pgfsys@useobject{currentmarker}{}%
\end{pgfscope}%
\begin{pgfscope}%
\pgfsys@transformshift{1.950909in}{0.684266in}%
\pgfsys@useobject{currentmarker}{}%
\end{pgfscope}%
\begin{pgfscope}%
\pgfsys@transformshift{1.951457in}{0.684038in}%
\pgfsys@useobject{currentmarker}{}%
\end{pgfscope}%
\begin{pgfscope}%
\pgfsys@transformshift{1.952005in}{0.652041in}%
\pgfsys@useobject{currentmarker}{}%
\end{pgfscope}%
\begin{pgfscope}%
\pgfsys@transformshift{1.952550in}{0.682749in}%
\pgfsys@useobject{currentmarker}{}%
\end{pgfscope}%
\begin{pgfscope}%
\pgfsys@transformshift{1.953095in}{0.726677in}%
\pgfsys@useobject{currentmarker}{}%
\end{pgfscope}%
\begin{pgfscope}%
\pgfsys@transformshift{1.953638in}{0.724472in}%
\pgfsys@useobject{currentmarker}{}%
\end{pgfscope}%
\begin{pgfscope}%
\pgfsys@transformshift{1.954180in}{0.706414in}%
\pgfsys@useobject{currentmarker}{}%
\end{pgfscope}%
\begin{pgfscope}%
\pgfsys@transformshift{1.954721in}{0.743811in}%
\pgfsys@useobject{currentmarker}{}%
\end{pgfscope}%
\begin{pgfscope}%
\pgfsys@transformshift{1.955260in}{0.729630in}%
\pgfsys@useobject{currentmarker}{}%
\end{pgfscope}%
\begin{pgfscope}%
\pgfsys@transformshift{1.955799in}{0.682835in}%
\pgfsys@useobject{currentmarker}{}%
\end{pgfscope}%
\begin{pgfscope}%
\pgfsys@transformshift{1.956335in}{0.702734in}%
\pgfsys@useobject{currentmarker}{}%
\end{pgfscope}%
\begin{pgfscope}%
\pgfsys@transformshift{1.956871in}{0.722498in}%
\pgfsys@useobject{currentmarker}{}%
\end{pgfscope}%
\begin{pgfscope}%
\pgfsys@transformshift{1.957405in}{0.713929in}%
\pgfsys@useobject{currentmarker}{}%
\end{pgfscope}%
\begin{pgfscope}%
\pgfsys@transformshift{1.957938in}{0.654413in}%
\pgfsys@useobject{currentmarker}{}%
\end{pgfscope}%
\begin{pgfscope}%
\pgfsys@transformshift{1.958470in}{0.687912in}%
\pgfsys@useobject{currentmarker}{}%
\end{pgfscope}%
\begin{pgfscope}%
\pgfsys@transformshift{1.959000in}{0.731379in}%
\pgfsys@useobject{currentmarker}{}%
\end{pgfscope}%
\begin{pgfscope}%
\pgfsys@transformshift{1.959529in}{0.694951in}%
\pgfsys@useobject{currentmarker}{}%
\end{pgfscope}%
\begin{pgfscope}%
\pgfsys@transformshift{1.960057in}{0.668903in}%
\pgfsys@useobject{currentmarker}{}%
\end{pgfscope}%
\begin{pgfscope}%
\pgfsys@transformshift{1.960584in}{0.626513in}%
\pgfsys@useobject{currentmarker}{}%
\end{pgfscope}%
\begin{pgfscope}%
\pgfsys@transformshift{1.961110in}{0.658328in}%
\pgfsys@useobject{currentmarker}{}%
\end{pgfscope}%
\begin{pgfscope}%
\pgfsys@transformshift{1.961634in}{0.679921in}%
\pgfsys@useobject{currentmarker}{}%
\end{pgfscope}%
\begin{pgfscope}%
\pgfsys@transformshift{1.962157in}{0.669094in}%
\pgfsys@useobject{currentmarker}{}%
\end{pgfscope}%
\begin{pgfscope}%
\pgfsys@transformshift{1.962679in}{0.689480in}%
\pgfsys@useobject{currentmarker}{}%
\end{pgfscope}%
\begin{pgfscope}%
\pgfsys@transformshift{1.963199in}{0.712407in}%
\pgfsys@useobject{currentmarker}{}%
\end{pgfscope}%
\begin{pgfscope}%
\pgfsys@transformshift{1.963719in}{0.685599in}%
\pgfsys@useobject{currentmarker}{}%
\end{pgfscope}%
\begin{pgfscope}%
\pgfsys@transformshift{1.964237in}{0.674118in}%
\pgfsys@useobject{currentmarker}{}%
\end{pgfscope}%
\begin{pgfscope}%
\pgfsys@transformshift{1.964754in}{0.711782in}%
\pgfsys@useobject{currentmarker}{}%
\end{pgfscope}%
\begin{pgfscope}%
\pgfsys@transformshift{1.965270in}{0.715649in}%
\pgfsys@useobject{currentmarker}{}%
\end{pgfscope}%
\begin{pgfscope}%
\pgfsys@transformshift{1.965785in}{0.666679in}%
\pgfsys@useobject{currentmarker}{}%
\end{pgfscope}%
\begin{pgfscope}%
\pgfsys@transformshift{1.966298in}{0.669613in}%
\pgfsys@useobject{currentmarker}{}%
\end{pgfscope}%
\begin{pgfscope}%
\pgfsys@transformshift{1.966810in}{0.719980in}%
\pgfsys@useobject{currentmarker}{}%
\end{pgfscope}%
\begin{pgfscope}%
\pgfsys@transformshift{1.967322in}{0.731727in}%
\pgfsys@useobject{currentmarker}{}%
\end{pgfscope}%
\begin{pgfscope}%
\pgfsys@transformshift{1.967832in}{0.707206in}%
\pgfsys@useobject{currentmarker}{}%
\end{pgfscope}%
\begin{pgfscope}%
\pgfsys@transformshift{1.968340in}{0.658470in}%
\pgfsys@useobject{currentmarker}{}%
\end{pgfscope}%
\begin{pgfscope}%
\pgfsys@transformshift{1.968848in}{0.653897in}%
\pgfsys@useobject{currentmarker}{}%
\end{pgfscope}%
\begin{pgfscope}%
\pgfsys@transformshift{1.969355in}{0.689508in}%
\pgfsys@useobject{currentmarker}{}%
\end{pgfscope}%
\begin{pgfscope}%
\pgfsys@transformshift{1.969860in}{0.662645in}%
\pgfsys@useobject{currentmarker}{}%
\end{pgfscope}%
\begin{pgfscope}%
\pgfsys@transformshift{1.970364in}{0.695372in}%
\pgfsys@useobject{currentmarker}{}%
\end{pgfscope}%
\begin{pgfscope}%
\pgfsys@transformshift{1.970868in}{0.677700in}%
\pgfsys@useobject{currentmarker}{}%
\end{pgfscope}%
\begin{pgfscope}%
\pgfsys@transformshift{1.971370in}{0.656267in}%
\pgfsys@useobject{currentmarker}{}%
\end{pgfscope}%
\begin{pgfscope}%
\pgfsys@transformshift{1.971870in}{0.647691in}%
\pgfsys@useobject{currentmarker}{}%
\end{pgfscope}%
\begin{pgfscope}%
\pgfsys@transformshift{1.972370in}{0.642958in}%
\pgfsys@useobject{currentmarker}{}%
\end{pgfscope}%
\begin{pgfscope}%
\pgfsys@transformshift{1.972869in}{0.712616in}%
\pgfsys@useobject{currentmarker}{}%
\end{pgfscope}%
\begin{pgfscope}%
\pgfsys@transformshift{1.973366in}{0.683745in}%
\pgfsys@useobject{currentmarker}{}%
\end{pgfscope}%
\begin{pgfscope}%
\pgfsys@transformshift{1.973863in}{0.658137in}%
\pgfsys@useobject{currentmarker}{}%
\end{pgfscope}%
\begin{pgfscope}%
\pgfsys@transformshift{1.974358in}{0.683492in}%
\pgfsys@useobject{currentmarker}{}%
\end{pgfscope}%
\begin{pgfscope}%
\pgfsys@transformshift{1.974853in}{0.693674in}%
\pgfsys@useobject{currentmarker}{}%
\end{pgfscope}%
\begin{pgfscope}%
\pgfsys@transformshift{1.975346in}{0.714881in}%
\pgfsys@useobject{currentmarker}{}%
\end{pgfscope}%
\begin{pgfscope}%
\pgfsys@transformshift{1.975838in}{0.707588in}%
\pgfsys@useobject{currentmarker}{}%
\end{pgfscope}%
\begin{pgfscope}%
\pgfsys@transformshift{1.976329in}{0.717928in}%
\pgfsys@useobject{currentmarker}{}%
\end{pgfscope}%
\begin{pgfscope}%
\pgfsys@transformshift{1.976819in}{0.703013in}%
\pgfsys@useobject{currentmarker}{}%
\end{pgfscope}%
\begin{pgfscope}%
\pgfsys@transformshift{1.977308in}{0.692257in}%
\pgfsys@useobject{currentmarker}{}%
\end{pgfscope}%
\begin{pgfscope}%
\pgfsys@transformshift{1.977796in}{0.690604in}%
\pgfsys@useobject{currentmarker}{}%
\end{pgfscope}%
\begin{pgfscope}%
\pgfsys@transformshift{1.978282in}{0.671391in}%
\pgfsys@useobject{currentmarker}{}%
\end{pgfscope}%
\begin{pgfscope}%
\pgfsys@transformshift{1.978768in}{0.661497in}%
\pgfsys@useobject{currentmarker}{}%
\end{pgfscope}%
\begin{pgfscope}%
\pgfsys@transformshift{1.979253in}{0.689898in}%
\pgfsys@useobject{currentmarker}{}%
\end{pgfscope}%
\begin{pgfscope}%
\pgfsys@transformshift{1.979736in}{0.667160in}%
\pgfsys@useobject{currentmarker}{}%
\end{pgfscope}%
\begin{pgfscope}%
\pgfsys@transformshift{1.980219in}{0.676039in}%
\pgfsys@useobject{currentmarker}{}%
\end{pgfscope}%
\begin{pgfscope}%
\pgfsys@transformshift{1.980701in}{0.683001in}%
\pgfsys@useobject{currentmarker}{}%
\end{pgfscope}%
\begin{pgfscope}%
\pgfsys@transformshift{1.981181in}{0.723273in}%
\pgfsys@useobject{currentmarker}{}%
\end{pgfscope}%
\begin{pgfscope}%
\pgfsys@transformshift{1.981661in}{0.715346in}%
\pgfsys@useobject{currentmarker}{}%
\end{pgfscope}%
\begin{pgfscope}%
\pgfsys@transformshift{1.982139in}{0.715285in}%
\pgfsys@useobject{currentmarker}{}%
\end{pgfscope}%
\begin{pgfscope}%
\pgfsys@transformshift{1.982617in}{0.712700in}%
\pgfsys@useobject{currentmarker}{}%
\end{pgfscope}%
\begin{pgfscope}%
\pgfsys@transformshift{1.983093in}{0.691870in}%
\pgfsys@useobject{currentmarker}{}%
\end{pgfscope}%
\begin{pgfscope}%
\pgfsys@transformshift{1.983569in}{0.691911in}%
\pgfsys@useobject{currentmarker}{}%
\end{pgfscope}%
\begin{pgfscope}%
\pgfsys@transformshift{1.984043in}{0.666193in}%
\pgfsys@useobject{currentmarker}{}%
\end{pgfscope}%
\begin{pgfscope}%
\pgfsys@transformshift{1.984517in}{0.692592in}%
\pgfsys@useobject{currentmarker}{}%
\end{pgfscope}%
\begin{pgfscope}%
\pgfsys@transformshift{1.984989in}{0.707882in}%
\pgfsys@useobject{currentmarker}{}%
\end{pgfscope}%
\begin{pgfscope}%
\pgfsys@transformshift{1.985460in}{0.688824in}%
\pgfsys@useobject{currentmarker}{}%
\end{pgfscope}%
\begin{pgfscope}%
\pgfsys@transformshift{1.985931in}{0.659792in}%
\pgfsys@useobject{currentmarker}{}%
\end{pgfscope}%
\begin{pgfscope}%
\pgfsys@transformshift{1.986400in}{0.672267in}%
\pgfsys@useobject{currentmarker}{}%
\end{pgfscope}%
\begin{pgfscope}%
\pgfsys@transformshift{1.986869in}{0.639878in}%
\pgfsys@useobject{currentmarker}{}%
\end{pgfscope}%
\begin{pgfscope}%
\pgfsys@transformshift{1.987336in}{0.668738in}%
\pgfsys@useobject{currentmarker}{}%
\end{pgfscope}%
\begin{pgfscope}%
\pgfsys@transformshift{1.987803in}{0.700702in}%
\pgfsys@useobject{currentmarker}{}%
\end{pgfscope}%
\begin{pgfscope}%
\pgfsys@transformshift{1.988269in}{0.672664in}%
\pgfsys@useobject{currentmarker}{}%
\end{pgfscope}%
\begin{pgfscope}%
\pgfsys@transformshift{1.988733in}{0.633296in}%
\pgfsys@useobject{currentmarker}{}%
\end{pgfscope}%
\begin{pgfscope}%
\pgfsys@transformshift{1.989197in}{0.657388in}%
\pgfsys@useobject{currentmarker}{}%
\end{pgfscope}%
\begin{pgfscope}%
\pgfsys@transformshift{1.989660in}{0.720565in}%
\pgfsys@useobject{currentmarker}{}%
\end{pgfscope}%
\begin{pgfscope}%
\pgfsys@transformshift{1.990121in}{0.733014in}%
\pgfsys@useobject{currentmarker}{}%
\end{pgfscope}%
\begin{pgfscope}%
\pgfsys@transformshift{1.990582in}{0.716427in}%
\pgfsys@useobject{currentmarker}{}%
\end{pgfscope}%
\begin{pgfscope}%
\pgfsys@transformshift{1.991042in}{0.661652in}%
\pgfsys@useobject{currentmarker}{}%
\end{pgfscope}%
\begin{pgfscope}%
\pgfsys@transformshift{1.991501in}{0.646272in}%
\pgfsys@useobject{currentmarker}{}%
\end{pgfscope}%
\begin{pgfscope}%
\pgfsys@transformshift{1.991959in}{0.679121in}%
\pgfsys@useobject{currentmarker}{}%
\end{pgfscope}%
\begin{pgfscope}%
\pgfsys@transformshift{1.992416in}{0.696830in}%
\pgfsys@useobject{currentmarker}{}%
\end{pgfscope}%
\begin{pgfscope}%
\pgfsys@transformshift{1.992872in}{0.689479in}%
\pgfsys@useobject{currentmarker}{}%
\end{pgfscope}%
\begin{pgfscope}%
\pgfsys@transformshift{1.993328in}{0.713746in}%
\pgfsys@useobject{currentmarker}{}%
\end{pgfscope}%
\begin{pgfscope}%
\pgfsys@transformshift{1.993782in}{0.697627in}%
\pgfsys@useobject{currentmarker}{}%
\end{pgfscope}%
\begin{pgfscope}%
\pgfsys@transformshift{1.994235in}{0.678248in}%
\pgfsys@useobject{currentmarker}{}%
\end{pgfscope}%
\begin{pgfscope}%
\pgfsys@transformshift{1.994688in}{0.674137in}%
\pgfsys@useobject{currentmarker}{}%
\end{pgfscope}%
\begin{pgfscope}%
\pgfsys@transformshift{1.995139in}{0.737682in}%
\pgfsys@useobject{currentmarker}{}%
\end{pgfscope}%
\begin{pgfscope}%
\pgfsys@transformshift{1.995590in}{0.735312in}%
\pgfsys@useobject{currentmarker}{}%
\end{pgfscope}%
\begin{pgfscope}%
\pgfsys@transformshift{1.996040in}{0.684619in}%
\pgfsys@useobject{currentmarker}{}%
\end{pgfscope}%
\begin{pgfscope}%
\pgfsys@transformshift{1.996488in}{0.672899in}%
\pgfsys@useobject{currentmarker}{}%
\end{pgfscope}%
\begin{pgfscope}%
\pgfsys@transformshift{1.996936in}{0.691313in}%
\pgfsys@useobject{currentmarker}{}%
\end{pgfscope}%
\begin{pgfscope}%
\pgfsys@transformshift{1.997384in}{0.708423in}%
\pgfsys@useobject{currentmarker}{}%
\end{pgfscope}%
\begin{pgfscope}%
\pgfsys@transformshift{1.997830in}{0.681862in}%
\pgfsys@useobject{currentmarker}{}%
\end{pgfscope}%
\begin{pgfscope}%
\pgfsys@transformshift{1.998275in}{0.689469in}%
\pgfsys@useobject{currentmarker}{}%
\end{pgfscope}%
\begin{pgfscope}%
\pgfsys@transformshift{1.998719in}{0.720516in}%
\pgfsys@useobject{currentmarker}{}%
\end{pgfscope}%
\begin{pgfscope}%
\pgfsys@transformshift{1.999163in}{0.701198in}%
\pgfsys@useobject{currentmarker}{}%
\end{pgfscope}%
\begin{pgfscope}%
\pgfsys@transformshift{1.999606in}{0.643096in}%
\pgfsys@useobject{currentmarker}{}%
\end{pgfscope}%
\begin{pgfscope}%
\pgfsys@transformshift{2.000047in}{0.693557in}%
\pgfsys@useobject{currentmarker}{}%
\end{pgfscope}%
\begin{pgfscope}%
\pgfsys@transformshift{2.000488in}{0.694030in}%
\pgfsys@useobject{currentmarker}{}%
\end{pgfscope}%
\begin{pgfscope}%
\pgfsys@transformshift{2.000928in}{0.713641in}%
\pgfsys@useobject{currentmarker}{}%
\end{pgfscope}%
\begin{pgfscope}%
\pgfsys@transformshift{2.001368in}{0.682006in}%
\pgfsys@useobject{currentmarker}{}%
\end{pgfscope}%
\begin{pgfscope}%
\pgfsys@transformshift{2.001806in}{0.683628in}%
\pgfsys@useobject{currentmarker}{}%
\end{pgfscope}%
\begin{pgfscope}%
\pgfsys@transformshift{2.002243in}{0.727113in}%
\pgfsys@useobject{currentmarker}{}%
\end{pgfscope}%
\begin{pgfscope}%
\pgfsys@transformshift{2.002680in}{0.615398in}%
\pgfsys@useobject{currentmarker}{}%
\end{pgfscope}%
\begin{pgfscope}%
\pgfsys@transformshift{2.003116in}{0.627671in}%
\pgfsys@useobject{currentmarker}{}%
\end{pgfscope}%
\begin{pgfscope}%
\pgfsys@transformshift{2.003551in}{0.665065in}%
\pgfsys@useobject{currentmarker}{}%
\end{pgfscope}%
\begin{pgfscope}%
\pgfsys@transformshift{2.003985in}{0.727121in}%
\pgfsys@useobject{currentmarker}{}%
\end{pgfscope}%
\begin{pgfscope}%
\pgfsys@transformshift{2.004418in}{0.706399in}%
\pgfsys@useobject{currentmarker}{}%
\end{pgfscope}%
\begin{pgfscope}%
\pgfsys@transformshift{2.004851in}{0.613323in}%
\pgfsys@useobject{currentmarker}{}%
\end{pgfscope}%
\begin{pgfscope}%
\pgfsys@transformshift{2.005282in}{0.645379in}%
\pgfsys@useobject{currentmarker}{}%
\end{pgfscope}%
\begin{pgfscope}%
\pgfsys@transformshift{2.005713in}{0.631321in}%
\pgfsys@useobject{currentmarker}{}%
\end{pgfscope}%
\begin{pgfscope}%
\pgfsys@transformshift{2.006143in}{0.722630in}%
\pgfsys@useobject{currentmarker}{}%
\end{pgfscope}%
\begin{pgfscope}%
\pgfsys@transformshift{2.006572in}{0.678407in}%
\pgfsys@useobject{currentmarker}{}%
\end{pgfscope}%
\begin{pgfscope}%
\pgfsys@transformshift{2.007000in}{0.659561in}%
\pgfsys@useobject{currentmarker}{}%
\end{pgfscope}%
\begin{pgfscope}%
\pgfsys@transformshift{2.007428in}{0.698347in}%
\pgfsys@useobject{currentmarker}{}%
\end{pgfscope}%
\begin{pgfscope}%
\pgfsys@transformshift{2.007855in}{0.725342in}%
\pgfsys@useobject{currentmarker}{}%
\end{pgfscope}%
\begin{pgfscope}%
\pgfsys@transformshift{2.008280in}{0.719942in}%
\pgfsys@useobject{currentmarker}{}%
\end{pgfscope}%
\begin{pgfscope}%
\pgfsys@transformshift{2.008706in}{0.678150in}%
\pgfsys@useobject{currentmarker}{}%
\end{pgfscope}%
\begin{pgfscope}%
\pgfsys@transformshift{2.009130in}{0.643497in}%
\pgfsys@useobject{currentmarker}{}%
\end{pgfscope}%
\begin{pgfscope}%
\pgfsys@transformshift{2.009553in}{0.677671in}%
\pgfsys@useobject{currentmarker}{}%
\end{pgfscope}%
\begin{pgfscope}%
\pgfsys@transformshift{2.009976in}{0.720252in}%
\pgfsys@useobject{currentmarker}{}%
\end{pgfscope}%
\begin{pgfscope}%
\pgfsys@transformshift{2.010398in}{0.699281in}%
\pgfsys@useobject{currentmarker}{}%
\end{pgfscope}%
\begin{pgfscope}%
\pgfsys@transformshift{2.010819in}{0.664797in}%
\pgfsys@useobject{currentmarker}{}%
\end{pgfscope}%
\begin{pgfscope}%
\pgfsys@transformshift{2.011239in}{0.683037in}%
\pgfsys@useobject{currentmarker}{}%
\end{pgfscope}%
\begin{pgfscope}%
\pgfsys@transformshift{2.011659in}{0.709691in}%
\pgfsys@useobject{currentmarker}{}%
\end{pgfscope}%
\begin{pgfscope}%
\pgfsys@transformshift{2.012078in}{0.720091in}%
\pgfsys@useobject{currentmarker}{}%
\end{pgfscope}%
\begin{pgfscope}%
\pgfsys@transformshift{2.012495in}{0.676865in}%
\pgfsys@useobject{currentmarker}{}%
\end{pgfscope}%
\begin{pgfscope}%
\pgfsys@transformshift{2.012913in}{0.679947in}%
\pgfsys@useobject{currentmarker}{}%
\end{pgfscope}%
\begin{pgfscope}%
\pgfsys@transformshift{2.013329in}{0.669626in}%
\pgfsys@useobject{currentmarker}{}%
\end{pgfscope}%
\begin{pgfscope}%
\pgfsys@transformshift{2.013745in}{0.695657in}%
\pgfsys@useobject{currentmarker}{}%
\end{pgfscope}%
\begin{pgfscope}%
\pgfsys@transformshift{2.014160in}{0.688982in}%
\pgfsys@useobject{currentmarker}{}%
\end{pgfscope}%
\begin{pgfscope}%
\pgfsys@transformshift{2.014574in}{0.711201in}%
\pgfsys@useobject{currentmarker}{}%
\end{pgfscope}%
\begin{pgfscope}%
\pgfsys@transformshift{2.014987in}{0.716887in}%
\pgfsys@useobject{currentmarker}{}%
\end{pgfscope}%
\begin{pgfscope}%
\pgfsys@transformshift{2.015400in}{0.681504in}%
\pgfsys@useobject{currentmarker}{}%
\end{pgfscope}%
\begin{pgfscope}%
\pgfsys@transformshift{2.015811in}{0.657144in}%
\pgfsys@useobject{currentmarker}{}%
\end{pgfscope}%
\begin{pgfscope}%
\pgfsys@transformshift{2.016223in}{0.696478in}%
\pgfsys@useobject{currentmarker}{}%
\end{pgfscope}%
\begin{pgfscope}%
\pgfsys@transformshift{2.016633in}{0.700675in}%
\pgfsys@useobject{currentmarker}{}%
\end{pgfscope}%
\begin{pgfscope}%
\pgfsys@transformshift{2.017042in}{0.676169in}%
\pgfsys@useobject{currentmarker}{}%
\end{pgfscope}%
\begin{pgfscope}%
\pgfsys@transformshift{2.017451in}{0.660714in}%
\pgfsys@useobject{currentmarker}{}%
\end{pgfscope}%
\begin{pgfscope}%
\pgfsys@transformshift{2.017859in}{0.672194in}%
\pgfsys@useobject{currentmarker}{}%
\end{pgfscope}%
\begin{pgfscope}%
\pgfsys@transformshift{2.018267in}{0.651177in}%
\pgfsys@useobject{currentmarker}{}%
\end{pgfscope}%
\begin{pgfscope}%
\pgfsys@transformshift{2.018673in}{0.694315in}%
\pgfsys@useobject{currentmarker}{}%
\end{pgfscope}%
\begin{pgfscope}%
\pgfsys@transformshift{2.019079in}{0.735697in}%
\pgfsys@useobject{currentmarker}{}%
\end{pgfscope}%
\begin{pgfscope}%
\pgfsys@transformshift{2.019484in}{0.711121in}%
\pgfsys@useobject{currentmarker}{}%
\end{pgfscope}%
\begin{pgfscope}%
\pgfsys@transformshift{2.019889in}{0.701407in}%
\pgfsys@useobject{currentmarker}{}%
\end{pgfscope}%
\begin{pgfscope}%
\pgfsys@transformshift{2.020292in}{0.713142in}%
\pgfsys@useobject{currentmarker}{}%
\end{pgfscope}%
\begin{pgfscope}%
\pgfsys@transformshift{2.020695in}{0.709072in}%
\pgfsys@useobject{currentmarker}{}%
\end{pgfscope}%
\begin{pgfscope}%
\pgfsys@transformshift{2.021098in}{0.695291in}%
\pgfsys@useobject{currentmarker}{}%
\end{pgfscope}%
\begin{pgfscope}%
\pgfsys@transformshift{2.021499in}{0.708589in}%
\pgfsys@useobject{currentmarker}{}%
\end{pgfscope}%
\begin{pgfscope}%
\pgfsys@transformshift{2.021900in}{0.694725in}%
\pgfsys@useobject{currentmarker}{}%
\end{pgfscope}%
\begin{pgfscope}%
\pgfsys@transformshift{2.022300in}{0.678544in}%
\pgfsys@useobject{currentmarker}{}%
\end{pgfscope}%
\begin{pgfscope}%
\pgfsys@transformshift{2.022699in}{0.669755in}%
\pgfsys@useobject{currentmarker}{}%
\end{pgfscope}%
\begin{pgfscope}%
\pgfsys@transformshift{2.023098in}{0.709804in}%
\pgfsys@useobject{currentmarker}{}%
\end{pgfscope}%
\begin{pgfscope}%
\pgfsys@transformshift{2.023496in}{0.725305in}%
\pgfsys@useobject{currentmarker}{}%
\end{pgfscope}%
\begin{pgfscope}%
\pgfsys@transformshift{2.023893in}{0.707500in}%
\pgfsys@useobject{currentmarker}{}%
\end{pgfscope}%
\begin{pgfscope}%
\pgfsys@transformshift{2.024290in}{0.727241in}%
\pgfsys@useobject{currentmarker}{}%
\end{pgfscope}%
\begin{pgfscope}%
\pgfsys@transformshift{2.024686in}{0.679337in}%
\pgfsys@useobject{currentmarker}{}%
\end{pgfscope}%
\begin{pgfscope}%
\pgfsys@transformshift{2.025081in}{0.627676in}%
\pgfsys@useobject{currentmarker}{}%
\end{pgfscope}%
\begin{pgfscope}%
\pgfsys@transformshift{2.025475in}{0.654181in}%
\pgfsys@useobject{currentmarker}{}%
\end{pgfscope}%
\begin{pgfscope}%
\pgfsys@transformshift{2.025869in}{0.716069in}%
\pgfsys@useobject{currentmarker}{}%
\end{pgfscope}%
\begin{pgfscope}%
\pgfsys@transformshift{2.026262in}{0.740961in}%
\pgfsys@useobject{currentmarker}{}%
\end{pgfscope}%
\begin{pgfscope}%
\pgfsys@transformshift{2.026655in}{0.712647in}%
\pgfsys@useobject{currentmarker}{}%
\end{pgfscope}%
\begin{pgfscope}%
\pgfsys@transformshift{2.027046in}{0.667001in}%
\pgfsys@useobject{currentmarker}{}%
\end{pgfscope}%
\begin{pgfscope}%
\pgfsys@transformshift{2.027437in}{0.659820in}%
\pgfsys@useobject{currentmarker}{}%
\end{pgfscope}%
\begin{pgfscope}%
\pgfsys@transformshift{2.027828in}{0.712602in}%
\pgfsys@useobject{currentmarker}{}%
\end{pgfscope}%
\begin{pgfscope}%
\pgfsys@transformshift{2.028217in}{0.691164in}%
\pgfsys@useobject{currentmarker}{}%
\end{pgfscope}%
\begin{pgfscope}%
\pgfsys@transformshift{2.028606in}{0.693104in}%
\pgfsys@useobject{currentmarker}{}%
\end{pgfscope}%
\begin{pgfscope}%
\pgfsys@transformshift{2.028995in}{0.662758in}%
\pgfsys@useobject{currentmarker}{}%
\end{pgfscope}%
\begin{pgfscope}%
\pgfsys@transformshift{2.029382in}{0.654668in}%
\pgfsys@useobject{currentmarker}{}%
\end{pgfscope}%
\begin{pgfscope}%
\pgfsys@transformshift{2.029769in}{0.723919in}%
\pgfsys@useobject{currentmarker}{}%
\end{pgfscope}%
\begin{pgfscope}%
\pgfsys@transformshift{2.030156in}{0.710734in}%
\pgfsys@useobject{currentmarker}{}%
\end{pgfscope}%
\begin{pgfscope}%
\pgfsys@transformshift{2.030541in}{0.651053in}%
\pgfsys@useobject{currentmarker}{}%
\end{pgfscope}%
\begin{pgfscope}%
\pgfsys@transformshift{2.030926in}{0.642579in}%
\pgfsys@useobject{currentmarker}{}%
\end{pgfscope}%
\begin{pgfscope}%
\pgfsys@transformshift{2.031311in}{0.682166in}%
\pgfsys@useobject{currentmarker}{}%
\end{pgfscope}%
\begin{pgfscope}%
\pgfsys@transformshift{2.031694in}{0.713128in}%
\pgfsys@useobject{currentmarker}{}%
\end{pgfscope}%
\begin{pgfscope}%
\pgfsys@transformshift{2.032078in}{0.707779in}%
\pgfsys@useobject{currentmarker}{}%
\end{pgfscope}%
\begin{pgfscope}%
\pgfsys@transformshift{2.032460in}{0.688467in}%
\pgfsys@useobject{currentmarker}{}%
\end{pgfscope}%
\begin{pgfscope}%
\pgfsys@transformshift{2.032842in}{0.662656in}%
\pgfsys@useobject{currentmarker}{}%
\end{pgfscope}%
\begin{pgfscope}%
\pgfsys@transformshift{2.033223in}{0.709500in}%
\pgfsys@useobject{currentmarker}{}%
\end{pgfscope}%
\begin{pgfscope}%
\pgfsys@transformshift{2.033603in}{0.664410in}%
\pgfsys@useobject{currentmarker}{}%
\end{pgfscope}%
\begin{pgfscope}%
\pgfsys@transformshift{2.033983in}{0.693663in}%
\pgfsys@useobject{currentmarker}{}%
\end{pgfscope}%
\begin{pgfscope}%
\pgfsys@transformshift{2.034362in}{0.708342in}%
\pgfsys@useobject{currentmarker}{}%
\end{pgfscope}%
\begin{pgfscope}%
\pgfsys@transformshift{2.034741in}{0.643594in}%
\pgfsys@useobject{currentmarker}{}%
\end{pgfscope}%
\begin{pgfscope}%
\pgfsys@transformshift{2.035119in}{0.696481in}%
\pgfsys@useobject{currentmarker}{}%
\end{pgfscope}%
\begin{pgfscope}%
\pgfsys@transformshift{2.035496in}{0.706928in}%
\pgfsys@useobject{currentmarker}{}%
\end{pgfscope}%
\begin{pgfscope}%
\pgfsys@transformshift{2.035872in}{0.694965in}%
\pgfsys@useobject{currentmarker}{}%
\end{pgfscope}%
\begin{pgfscope}%
\pgfsys@transformshift{2.036248in}{0.737347in}%
\pgfsys@useobject{currentmarker}{}%
\end{pgfscope}%
\begin{pgfscope}%
\pgfsys@transformshift{2.036624in}{0.733288in}%
\pgfsys@useobject{currentmarker}{}%
\end{pgfscope}%
\begin{pgfscope}%
\pgfsys@transformshift{2.036998in}{0.692424in}%
\pgfsys@useobject{currentmarker}{}%
\end{pgfscope}%
\begin{pgfscope}%
\pgfsys@transformshift{2.037373in}{0.685369in}%
\pgfsys@useobject{currentmarker}{}%
\end{pgfscope}%
\begin{pgfscope}%
\pgfsys@transformshift{2.037746in}{0.698289in}%
\pgfsys@useobject{currentmarker}{}%
\end{pgfscope}%
\begin{pgfscope}%
\pgfsys@transformshift{2.038119in}{0.696410in}%
\pgfsys@useobject{currentmarker}{}%
\end{pgfscope}%
\begin{pgfscope}%
\pgfsys@transformshift{2.038491in}{0.699623in}%
\pgfsys@useobject{currentmarker}{}%
\end{pgfscope}%
\begin{pgfscope}%
\pgfsys@transformshift{2.038863in}{0.646670in}%
\pgfsys@useobject{currentmarker}{}%
\end{pgfscope}%
\begin{pgfscope}%
\pgfsys@transformshift{2.039234in}{0.681845in}%
\pgfsys@useobject{currentmarker}{}%
\end{pgfscope}%
\begin{pgfscope}%
\pgfsys@transformshift{2.039604in}{0.679177in}%
\pgfsys@useobject{currentmarker}{}%
\end{pgfscope}%
\begin{pgfscope}%
\pgfsys@transformshift{2.039974in}{0.667761in}%
\pgfsys@useobject{currentmarker}{}%
\end{pgfscope}%
\begin{pgfscope}%
\pgfsys@transformshift{2.040343in}{0.711891in}%
\pgfsys@useobject{currentmarker}{}%
\end{pgfscope}%
\begin{pgfscope}%
\pgfsys@transformshift{2.040712in}{0.679644in}%
\pgfsys@useobject{currentmarker}{}%
\end{pgfscope}%
\begin{pgfscope}%
\pgfsys@transformshift{2.041080in}{0.683760in}%
\pgfsys@useobject{currentmarker}{}%
\end{pgfscope}%
\begin{pgfscope}%
\pgfsys@transformshift{2.041447in}{0.704164in}%
\pgfsys@useobject{currentmarker}{}%
\end{pgfscope}%
\begin{pgfscope}%
\pgfsys@transformshift{2.041814in}{0.729598in}%
\pgfsys@useobject{currentmarker}{}%
\end{pgfscope}%
\begin{pgfscope}%
\pgfsys@transformshift{2.042180in}{0.767870in}%
\pgfsys@useobject{currentmarker}{}%
\end{pgfscope}%
\begin{pgfscope}%
\pgfsys@transformshift{2.042546in}{0.673908in}%
\pgfsys@useobject{currentmarker}{}%
\end{pgfscope}%
\begin{pgfscope}%
\pgfsys@transformshift{2.042911in}{0.682671in}%
\pgfsys@useobject{currentmarker}{}%
\end{pgfscope}%
\begin{pgfscope}%
\pgfsys@transformshift{2.043275in}{0.741426in}%
\pgfsys@useobject{currentmarker}{}%
\end{pgfscope}%
\begin{pgfscope}%
\pgfsys@transformshift{2.043639in}{0.707865in}%
\pgfsys@useobject{currentmarker}{}%
\end{pgfscope}%
\begin{pgfscope}%
\pgfsys@transformshift{2.044002in}{0.670264in}%
\pgfsys@useobject{currentmarker}{}%
\end{pgfscope}%
\begin{pgfscope}%
\pgfsys@transformshift{2.044365in}{0.718361in}%
\pgfsys@useobject{currentmarker}{}%
\end{pgfscope}%
\begin{pgfscope}%
\pgfsys@transformshift{2.044727in}{0.698193in}%
\pgfsys@useobject{currentmarker}{}%
\end{pgfscope}%
\begin{pgfscope}%
\pgfsys@transformshift{2.045088in}{0.663369in}%
\pgfsys@useobject{currentmarker}{}%
\end{pgfscope}%
\begin{pgfscope}%
\pgfsys@transformshift{2.045449in}{0.668999in}%
\pgfsys@useobject{currentmarker}{}%
\end{pgfscope}%
\begin{pgfscope}%
\pgfsys@transformshift{2.045809in}{0.689613in}%
\pgfsys@useobject{currentmarker}{}%
\end{pgfscope}%
\begin{pgfscope}%
\pgfsys@transformshift{2.046169in}{0.698936in}%
\pgfsys@useobject{currentmarker}{}%
\end{pgfscope}%
\begin{pgfscope}%
\pgfsys@transformshift{2.046528in}{0.677767in}%
\pgfsys@useobject{currentmarker}{}%
\end{pgfscope}%
\begin{pgfscope}%
\pgfsys@transformshift{2.046887in}{0.649689in}%
\pgfsys@useobject{currentmarker}{}%
\end{pgfscope}%
\begin{pgfscope}%
\pgfsys@transformshift{2.047245in}{0.695263in}%
\pgfsys@useobject{currentmarker}{}%
\end{pgfscope}%
\begin{pgfscope}%
\pgfsys@transformshift{2.047602in}{0.702306in}%
\pgfsys@useobject{currentmarker}{}%
\end{pgfscope}%
\begin{pgfscope}%
\pgfsys@transformshift{2.047959in}{0.702447in}%
\pgfsys@useobject{currentmarker}{}%
\end{pgfscope}%
\begin{pgfscope}%
\pgfsys@transformshift{2.048316in}{0.703295in}%
\pgfsys@useobject{currentmarker}{}%
\end{pgfscope}%
\begin{pgfscope}%
\pgfsys@transformshift{2.048671in}{0.700192in}%
\pgfsys@useobject{currentmarker}{}%
\end{pgfscope}%
\begin{pgfscope}%
\pgfsys@transformshift{2.049027in}{0.684587in}%
\pgfsys@useobject{currentmarker}{}%
\end{pgfscope}%
\begin{pgfscope}%
\pgfsys@transformshift{2.049381in}{0.692237in}%
\pgfsys@useobject{currentmarker}{}%
\end{pgfscope}%
\begin{pgfscope}%
\pgfsys@transformshift{2.049735in}{0.680570in}%
\pgfsys@useobject{currentmarker}{}%
\end{pgfscope}%
\begin{pgfscope}%
\pgfsys@transformshift{2.050089in}{0.700380in}%
\pgfsys@useobject{currentmarker}{}%
\end{pgfscope}%
\begin{pgfscope}%
\pgfsys@transformshift{2.050442in}{0.716194in}%
\pgfsys@useobject{currentmarker}{}%
\end{pgfscope}%
\begin{pgfscope}%
\pgfsys@transformshift{2.050794in}{0.661851in}%
\pgfsys@useobject{currentmarker}{}%
\end{pgfscope}%
\begin{pgfscope}%
\pgfsys@transformshift{2.051146in}{0.634471in}%
\pgfsys@useobject{currentmarker}{}%
\end{pgfscope}%
\begin{pgfscope}%
\pgfsys@transformshift{2.051497in}{0.721721in}%
\pgfsys@useobject{currentmarker}{}%
\end{pgfscope}%
\begin{pgfscope}%
\pgfsys@transformshift{2.051848in}{0.735382in}%
\pgfsys@useobject{currentmarker}{}%
\end{pgfscope}%
\begin{pgfscope}%
\pgfsys@transformshift{2.052198in}{0.677580in}%
\pgfsys@useobject{currentmarker}{}%
\end{pgfscope}%
\begin{pgfscope}%
\pgfsys@transformshift{2.052548in}{0.737037in}%
\pgfsys@useobject{currentmarker}{}%
\end{pgfscope}%
\begin{pgfscope}%
\pgfsys@transformshift{2.052897in}{0.738914in}%
\pgfsys@useobject{currentmarker}{}%
\end{pgfscope}%
\begin{pgfscope}%
\pgfsys@transformshift{2.053245in}{0.702406in}%
\pgfsys@useobject{currentmarker}{}%
\end{pgfscope}%
\begin{pgfscope}%
\pgfsys@transformshift{2.053593in}{0.639855in}%
\pgfsys@useobject{currentmarker}{}%
\end{pgfscope}%
\begin{pgfscope}%
\pgfsys@transformshift{2.053941in}{0.634808in}%
\pgfsys@useobject{currentmarker}{}%
\end{pgfscope}%
\begin{pgfscope}%
\pgfsys@transformshift{2.054288in}{0.659948in}%
\pgfsys@useobject{currentmarker}{}%
\end{pgfscope}%
\begin{pgfscope}%
\pgfsys@transformshift{2.054634in}{0.673911in}%
\pgfsys@useobject{currentmarker}{}%
\end{pgfscope}%
\begin{pgfscope}%
\pgfsys@transformshift{2.054980in}{0.696914in}%
\pgfsys@useobject{currentmarker}{}%
\end{pgfscope}%
\begin{pgfscope}%
\pgfsys@transformshift{2.055326in}{0.707234in}%
\pgfsys@useobject{currentmarker}{}%
\end{pgfscope}%
\begin{pgfscope}%
\pgfsys@transformshift{2.055670in}{0.717028in}%
\pgfsys@useobject{currentmarker}{}%
\end{pgfscope}%
\begin{pgfscope}%
\pgfsys@transformshift{2.056015in}{0.701907in}%
\pgfsys@useobject{currentmarker}{}%
\end{pgfscope}%
\begin{pgfscope}%
\pgfsys@transformshift{2.056358in}{0.726260in}%
\pgfsys@useobject{currentmarker}{}%
\end{pgfscope}%
\begin{pgfscope}%
\pgfsys@transformshift{2.056702in}{0.701550in}%
\pgfsys@useobject{currentmarker}{}%
\end{pgfscope}%
\begin{pgfscope}%
\pgfsys@transformshift{2.057044in}{0.640781in}%
\pgfsys@useobject{currentmarker}{}%
\end{pgfscope}%
\begin{pgfscope}%
\pgfsys@transformshift{2.057387in}{0.650793in}%
\pgfsys@useobject{currentmarker}{}%
\end{pgfscope}%
\begin{pgfscope}%
\pgfsys@transformshift{2.057728in}{0.695484in}%
\pgfsys@useobject{currentmarker}{}%
\end{pgfscope}%
\begin{pgfscope}%
\pgfsys@transformshift{2.058069in}{0.695348in}%
\pgfsys@useobject{currentmarker}{}%
\end{pgfscope}%
\begin{pgfscope}%
\pgfsys@transformshift{2.058410in}{0.634681in}%
\pgfsys@useobject{currentmarker}{}%
\end{pgfscope}%
\begin{pgfscope}%
\pgfsys@transformshift{2.058750in}{0.679720in}%
\pgfsys@useobject{currentmarker}{}%
\end{pgfscope}%
\begin{pgfscope}%
\pgfsys@transformshift{2.059090in}{0.722419in}%
\pgfsys@useobject{currentmarker}{}%
\end{pgfscope}%
\begin{pgfscope}%
\pgfsys@transformshift{2.059429in}{0.737158in}%
\pgfsys@useobject{currentmarker}{}%
\end{pgfscope}%
\begin{pgfscope}%
\pgfsys@transformshift{2.059767in}{0.678155in}%
\pgfsys@useobject{currentmarker}{}%
\end{pgfscope}%
\begin{pgfscope}%
\pgfsys@transformshift{2.060106in}{0.675984in}%
\pgfsys@useobject{currentmarker}{}%
\end{pgfscope}%
\begin{pgfscope}%
\pgfsys@transformshift{2.060443in}{0.703933in}%
\pgfsys@useobject{currentmarker}{}%
\end{pgfscope}%
\begin{pgfscope}%
\pgfsys@transformshift{2.060780in}{0.688599in}%
\pgfsys@useobject{currentmarker}{}%
\end{pgfscope}%
\begin{pgfscope}%
\pgfsys@transformshift{2.061117in}{0.705560in}%
\pgfsys@useobject{currentmarker}{}%
\end{pgfscope}%
\begin{pgfscope}%
\pgfsys@transformshift{2.061453in}{0.736225in}%
\pgfsys@useobject{currentmarker}{}%
\end{pgfscope}%
\begin{pgfscope}%
\pgfsys@transformshift{2.061788in}{0.674044in}%
\pgfsys@useobject{currentmarker}{}%
\end{pgfscope}%
\begin{pgfscope}%
\pgfsys@transformshift{2.062123in}{0.674447in}%
\pgfsys@useobject{currentmarker}{}%
\end{pgfscope}%
\begin{pgfscope}%
\pgfsys@transformshift{2.062458in}{0.704118in}%
\pgfsys@useobject{currentmarker}{}%
\end{pgfscope}%
\begin{pgfscope}%
\pgfsys@transformshift{2.062792in}{0.638117in}%
\pgfsys@useobject{currentmarker}{}%
\end{pgfscope}%
\begin{pgfscope}%
\pgfsys@transformshift{2.063126in}{0.717221in}%
\pgfsys@useobject{currentmarker}{}%
\end{pgfscope}%
\begin{pgfscope}%
\pgfsys@transformshift{2.063459in}{0.697802in}%
\pgfsys@useobject{currentmarker}{}%
\end{pgfscope}%
\begin{pgfscope}%
\pgfsys@transformshift{2.063791in}{0.673404in}%
\pgfsys@useobject{currentmarker}{}%
\end{pgfscope}%
\begin{pgfscope}%
\pgfsys@transformshift{2.064123in}{0.632849in}%
\pgfsys@useobject{currentmarker}{}%
\end{pgfscope}%
\begin{pgfscope}%
\pgfsys@transformshift{2.064455in}{0.689476in}%
\pgfsys@useobject{currentmarker}{}%
\end{pgfscope}%
\begin{pgfscope}%
\pgfsys@transformshift{2.064786in}{0.705455in}%
\pgfsys@useobject{currentmarker}{}%
\end{pgfscope}%
\begin{pgfscope}%
\pgfsys@transformshift{2.065117in}{0.705583in}%
\pgfsys@useobject{currentmarker}{}%
\end{pgfscope}%
\begin{pgfscope}%
\pgfsys@transformshift{2.065447in}{0.694901in}%
\pgfsys@useobject{currentmarker}{}%
\end{pgfscope}%
\begin{pgfscope}%
\pgfsys@transformshift{2.065776in}{0.682553in}%
\pgfsys@useobject{currentmarker}{}%
\end{pgfscope}%
\begin{pgfscope}%
\pgfsys@transformshift{2.066106in}{0.664829in}%
\pgfsys@useobject{currentmarker}{}%
\end{pgfscope}%
\begin{pgfscope}%
\pgfsys@transformshift{2.066434in}{0.703591in}%
\pgfsys@useobject{currentmarker}{}%
\end{pgfscope}%
\begin{pgfscope}%
\pgfsys@transformshift{2.066762in}{0.704026in}%
\pgfsys@useobject{currentmarker}{}%
\end{pgfscope}%
\begin{pgfscope}%
\pgfsys@transformshift{2.067090in}{0.685056in}%
\pgfsys@useobject{currentmarker}{}%
\end{pgfscope}%
\begin{pgfscope}%
\pgfsys@transformshift{2.067417in}{0.656870in}%
\pgfsys@useobject{currentmarker}{}%
\end{pgfscope}%
\begin{pgfscope}%
\pgfsys@transformshift{2.067744in}{0.648502in}%
\pgfsys@useobject{currentmarker}{}%
\end{pgfscope}%
\begin{pgfscope}%
\pgfsys@transformshift{2.068070in}{0.658419in}%
\pgfsys@useobject{currentmarker}{}%
\end{pgfscope}%
\begin{pgfscope}%
\pgfsys@transformshift{2.068396in}{0.668409in}%
\pgfsys@useobject{currentmarker}{}%
\end{pgfscope}%
\begin{pgfscope}%
\pgfsys@transformshift{2.068722in}{0.676179in}%
\pgfsys@useobject{currentmarker}{}%
\end{pgfscope}%
\begin{pgfscope}%
\pgfsys@transformshift{2.069046in}{0.698195in}%
\pgfsys@useobject{currentmarker}{}%
\end{pgfscope}%
\begin{pgfscope}%
\pgfsys@transformshift{2.069371in}{0.678860in}%
\pgfsys@useobject{currentmarker}{}%
\end{pgfscope}%
\begin{pgfscope}%
\pgfsys@transformshift{2.069695in}{0.665803in}%
\pgfsys@useobject{currentmarker}{}%
\end{pgfscope}%
\begin{pgfscope}%
\pgfsys@transformshift{2.070018in}{0.648618in}%
\pgfsys@useobject{currentmarker}{}%
\end{pgfscope}%
\begin{pgfscope}%
\pgfsys@transformshift{2.070341in}{0.681331in}%
\pgfsys@useobject{currentmarker}{}%
\end{pgfscope}%
\begin{pgfscope}%
\pgfsys@transformshift{2.070664in}{0.679585in}%
\pgfsys@useobject{currentmarker}{}%
\end{pgfscope}%
\begin{pgfscope}%
\pgfsys@transformshift{2.070986in}{0.673417in}%
\pgfsys@useobject{currentmarker}{}%
\end{pgfscope}%
\begin{pgfscope}%
\pgfsys@transformshift{2.071308in}{0.703366in}%
\pgfsys@useobject{currentmarker}{}%
\end{pgfscope}%
\begin{pgfscope}%
\pgfsys@transformshift{2.071629in}{0.722559in}%
\pgfsys@useobject{currentmarker}{}%
\end{pgfscope}%
\begin{pgfscope}%
\pgfsys@transformshift{2.071949in}{0.712303in}%
\pgfsys@useobject{currentmarker}{}%
\end{pgfscope}%
\begin{pgfscope}%
\pgfsys@transformshift{2.072270in}{0.669601in}%
\pgfsys@useobject{currentmarker}{}%
\end{pgfscope}%
\begin{pgfscope}%
\pgfsys@transformshift{2.072589in}{0.708022in}%
\pgfsys@useobject{currentmarker}{}%
\end{pgfscope}%
\begin{pgfscope}%
\pgfsys@transformshift{2.072909in}{0.709569in}%
\pgfsys@useobject{currentmarker}{}%
\end{pgfscope}%
\begin{pgfscope}%
\pgfsys@transformshift{2.073228in}{0.676417in}%
\pgfsys@useobject{currentmarker}{}%
\end{pgfscope}%
\begin{pgfscope}%
\pgfsys@transformshift{2.073546in}{0.688587in}%
\pgfsys@useobject{currentmarker}{}%
\end{pgfscope}%
\begin{pgfscope}%
\pgfsys@transformshift{2.073864in}{0.648466in}%
\pgfsys@useobject{currentmarker}{}%
\end{pgfscope}%
\begin{pgfscope}%
\pgfsys@transformshift{2.074182in}{0.690561in}%
\pgfsys@useobject{currentmarker}{}%
\end{pgfscope}%
\begin{pgfscope}%
\pgfsys@transformshift{2.074499in}{0.680756in}%
\pgfsys@useobject{currentmarker}{}%
\end{pgfscope}%
\begin{pgfscope}%
\pgfsys@transformshift{2.074815in}{0.663431in}%
\pgfsys@useobject{currentmarker}{}%
\end{pgfscope}%
\begin{pgfscope}%
\pgfsys@transformshift{2.075131in}{0.636224in}%
\pgfsys@useobject{currentmarker}{}%
\end{pgfscope}%
\begin{pgfscope}%
\pgfsys@transformshift{2.075447in}{0.695127in}%
\pgfsys@useobject{currentmarker}{}%
\end{pgfscope}%
\begin{pgfscope}%
\pgfsys@transformshift{2.075763in}{0.715822in}%
\pgfsys@useobject{currentmarker}{}%
\end{pgfscope}%
\begin{pgfscope}%
\pgfsys@transformshift{2.076077in}{0.702885in}%
\pgfsys@useobject{currentmarker}{}%
\end{pgfscope}%
\begin{pgfscope}%
\pgfsys@transformshift{2.076392in}{0.655556in}%
\pgfsys@useobject{currentmarker}{}%
\end{pgfscope}%
\begin{pgfscope}%
\pgfsys@transformshift{2.076706in}{0.644361in}%
\pgfsys@useobject{currentmarker}{}%
\end{pgfscope}%
\begin{pgfscope}%
\pgfsys@transformshift{2.077019in}{0.739902in}%
\pgfsys@useobject{currentmarker}{}%
\end{pgfscope}%
\begin{pgfscope}%
\pgfsys@transformshift{2.077332in}{0.735866in}%
\pgfsys@useobject{currentmarker}{}%
\end{pgfscope}%
\begin{pgfscope}%
\pgfsys@transformshift{2.077645in}{0.702597in}%
\pgfsys@useobject{currentmarker}{}%
\end{pgfscope}%
\begin{pgfscope}%
\pgfsys@transformshift{2.077957in}{0.686718in}%
\pgfsys@useobject{currentmarker}{}%
\end{pgfscope}%
\begin{pgfscope}%
\pgfsys@transformshift{2.078269in}{0.692620in}%
\pgfsys@useobject{currentmarker}{}%
\end{pgfscope}%
\begin{pgfscope}%
\pgfsys@transformshift{2.078581in}{0.711564in}%
\pgfsys@useobject{currentmarker}{}%
\end{pgfscope}%
\begin{pgfscope}%
\pgfsys@transformshift{2.078891in}{0.666198in}%
\pgfsys@useobject{currentmarker}{}%
\end{pgfscope}%
\begin{pgfscope}%
\pgfsys@transformshift{2.079202in}{0.727136in}%
\pgfsys@useobject{currentmarker}{}%
\end{pgfscope}%
\begin{pgfscope}%
\pgfsys@transformshift{2.079512in}{0.720165in}%
\pgfsys@useobject{currentmarker}{}%
\end{pgfscope}%
\begin{pgfscope}%
\pgfsys@transformshift{2.079822in}{0.734293in}%
\pgfsys@useobject{currentmarker}{}%
\end{pgfscope}%
\begin{pgfscope}%
\pgfsys@transformshift{2.080131in}{0.728889in}%
\pgfsys@useobject{currentmarker}{}%
\end{pgfscope}%
\begin{pgfscope}%
\pgfsys@transformshift{2.080440in}{0.673558in}%
\pgfsys@useobject{currentmarker}{}%
\end{pgfscope}%
\begin{pgfscope}%
\pgfsys@transformshift{2.080748in}{0.662423in}%
\pgfsys@useobject{currentmarker}{}%
\end{pgfscope}%
\begin{pgfscope}%
\pgfsys@transformshift{2.081056in}{0.679043in}%
\pgfsys@useobject{currentmarker}{}%
\end{pgfscope}%
\begin{pgfscope}%
\pgfsys@transformshift{2.081364in}{0.678654in}%
\pgfsys@useobject{currentmarker}{}%
\end{pgfscope}%
\begin{pgfscope}%
\pgfsys@transformshift{2.081671in}{0.661289in}%
\pgfsys@useobject{currentmarker}{}%
\end{pgfscope}%
\begin{pgfscope}%
\pgfsys@transformshift{2.081977in}{0.704366in}%
\pgfsys@useobject{currentmarker}{}%
\end{pgfscope}%
\begin{pgfscope}%
\pgfsys@transformshift{2.082284in}{0.714085in}%
\pgfsys@useobject{currentmarker}{}%
\end{pgfscope}%
\begin{pgfscope}%
\pgfsys@transformshift{2.082589in}{0.712764in}%
\pgfsys@useobject{currentmarker}{}%
\end{pgfscope}%
\begin{pgfscope}%
\pgfsys@transformshift{2.082895in}{0.709400in}%
\pgfsys@useobject{currentmarker}{}%
\end{pgfscope}%
\begin{pgfscope}%
\pgfsys@transformshift{2.083200in}{0.653037in}%
\pgfsys@useobject{currentmarker}{}%
\end{pgfscope}%
\begin{pgfscope}%
\pgfsys@transformshift{2.083505in}{0.683106in}%
\pgfsys@useobject{currentmarker}{}%
\end{pgfscope}%
\begin{pgfscope}%
\pgfsys@transformshift{2.083809in}{0.696453in}%
\pgfsys@useobject{currentmarker}{}%
\end{pgfscope}%
\begin{pgfscope}%
\pgfsys@transformshift{2.084113in}{0.651662in}%
\pgfsys@useobject{currentmarker}{}%
\end{pgfscope}%
\begin{pgfscope}%
\pgfsys@transformshift{2.084416in}{0.594616in}%
\pgfsys@useobject{currentmarker}{}%
\end{pgfscope}%
\begin{pgfscope}%
\pgfsys@transformshift{2.084719in}{0.703342in}%
\pgfsys@useobject{currentmarker}{}%
\end{pgfscope}%
\begin{pgfscope}%
\pgfsys@transformshift{2.085021in}{0.721562in}%
\pgfsys@useobject{currentmarker}{}%
\end{pgfscope}%
\begin{pgfscope}%
\pgfsys@transformshift{2.085324in}{0.699956in}%
\pgfsys@useobject{currentmarker}{}%
\end{pgfscope}%
\begin{pgfscope}%
\pgfsys@transformshift{2.085625in}{0.686056in}%
\pgfsys@useobject{currentmarker}{}%
\end{pgfscope}%
\begin{pgfscope}%
\pgfsys@transformshift{2.085927in}{0.689240in}%
\pgfsys@useobject{currentmarker}{}%
\end{pgfscope}%
\begin{pgfscope}%
\pgfsys@transformshift{2.086228in}{0.642229in}%
\pgfsys@useobject{currentmarker}{}%
\end{pgfscope}%
\begin{pgfscope}%
\pgfsys@transformshift{2.086528in}{0.674210in}%
\pgfsys@useobject{currentmarker}{}%
\end{pgfscope}%
\begin{pgfscope}%
\pgfsys@transformshift{2.086828in}{0.727240in}%
\pgfsys@useobject{currentmarker}{}%
\end{pgfscope}%
\begin{pgfscope}%
\pgfsys@transformshift{2.087128in}{0.689039in}%
\pgfsys@useobject{currentmarker}{}%
\end{pgfscope}%
\begin{pgfscope}%
\pgfsys@transformshift{2.087427in}{0.697024in}%
\pgfsys@useobject{currentmarker}{}%
\end{pgfscope}%
\begin{pgfscope}%
\pgfsys@transformshift{2.087726in}{0.724743in}%
\pgfsys@useobject{currentmarker}{}%
\end{pgfscope}%
\begin{pgfscope}%
\pgfsys@transformshift{2.088025in}{0.710351in}%
\pgfsys@useobject{currentmarker}{}%
\end{pgfscope}%
\begin{pgfscope}%
\pgfsys@transformshift{2.088323in}{0.707674in}%
\pgfsys@useobject{currentmarker}{}%
\end{pgfscope}%
\begin{pgfscope}%
\pgfsys@transformshift{2.088621in}{0.700164in}%
\pgfsys@useobject{currentmarker}{}%
\end{pgfscope}%
\begin{pgfscope}%
\pgfsys@transformshift{2.088918in}{0.649266in}%
\pgfsys@useobject{currentmarker}{}%
\end{pgfscope}%
\begin{pgfscope}%
\pgfsys@transformshift{2.089215in}{0.684446in}%
\pgfsys@useobject{currentmarker}{}%
\end{pgfscope}%
\begin{pgfscope}%
\pgfsys@transformshift{2.089512in}{0.685066in}%
\pgfsys@useobject{currentmarker}{}%
\end{pgfscope}%
\begin{pgfscope}%
\pgfsys@transformshift{2.089808in}{0.669152in}%
\pgfsys@useobject{currentmarker}{}%
\end{pgfscope}%
\begin{pgfscope}%
\pgfsys@transformshift{2.090104in}{0.718749in}%
\pgfsys@useobject{currentmarker}{}%
\end{pgfscope}%
\begin{pgfscope}%
\pgfsys@transformshift{2.090399in}{0.726863in}%
\pgfsys@useobject{currentmarker}{}%
\end{pgfscope}%
\begin{pgfscope}%
\pgfsys@transformshift{2.090694in}{0.638029in}%
\pgfsys@useobject{currentmarker}{}%
\end{pgfscope}%
\begin{pgfscope}%
\pgfsys@transformshift{2.090989in}{0.684571in}%
\pgfsys@useobject{currentmarker}{}%
\end{pgfscope}%
\begin{pgfscope}%
\pgfsys@transformshift{2.091283in}{0.643534in}%
\pgfsys@useobject{currentmarker}{}%
\end{pgfscope}%
\begin{pgfscope}%
\pgfsys@transformshift{2.091577in}{0.700220in}%
\pgfsys@useobject{currentmarker}{}%
\end{pgfscope}%
\begin{pgfscope}%
\pgfsys@transformshift{2.091870in}{0.697467in}%
\pgfsys@useobject{currentmarker}{}%
\end{pgfscope}%
\begin{pgfscope}%
\pgfsys@transformshift{2.092163in}{0.679173in}%
\pgfsys@useobject{currentmarker}{}%
\end{pgfscope}%
\begin{pgfscope}%
\pgfsys@transformshift{2.092456in}{0.702046in}%
\pgfsys@useobject{currentmarker}{}%
\end{pgfscope}%
\begin{pgfscope}%
\pgfsys@transformshift{2.092748in}{0.698780in}%
\pgfsys@useobject{currentmarker}{}%
\end{pgfscope}%
\begin{pgfscope}%
\pgfsys@transformshift{2.093040in}{0.645798in}%
\pgfsys@useobject{currentmarker}{}%
\end{pgfscope}%
\begin{pgfscope}%
\pgfsys@transformshift{2.093332in}{0.606716in}%
\pgfsys@useobject{currentmarker}{}%
\end{pgfscope}%
\begin{pgfscope}%
\pgfsys@transformshift{2.093623in}{0.682926in}%
\pgfsys@useobject{currentmarker}{}%
\end{pgfscope}%
\begin{pgfscope}%
\pgfsys@transformshift{2.093914in}{0.706566in}%
\pgfsys@useobject{currentmarker}{}%
\end{pgfscope}%
\begin{pgfscope}%
\pgfsys@transformshift{2.094204in}{0.711886in}%
\pgfsys@useobject{currentmarker}{}%
\end{pgfscope}%
\begin{pgfscope}%
\pgfsys@transformshift{2.094494in}{0.713413in}%
\pgfsys@useobject{currentmarker}{}%
\end{pgfscope}%
\begin{pgfscope}%
\pgfsys@transformshift{2.094784in}{0.687541in}%
\pgfsys@useobject{currentmarker}{}%
\end{pgfscope}%
\begin{pgfscope}%
\pgfsys@transformshift{2.095073in}{0.611318in}%
\pgfsys@useobject{currentmarker}{}%
\end{pgfscope}%
\begin{pgfscope}%
\pgfsys@transformshift{2.095362in}{0.664599in}%
\pgfsys@useobject{currentmarker}{}%
\end{pgfscope}%
\begin{pgfscope}%
\pgfsys@transformshift{2.095651in}{0.692065in}%
\pgfsys@useobject{currentmarker}{}%
\end{pgfscope}%
\begin{pgfscope}%
\pgfsys@transformshift{2.095939in}{0.716379in}%
\pgfsys@useobject{currentmarker}{}%
\end{pgfscope}%
\begin{pgfscope}%
\pgfsys@transformshift{2.096227in}{0.715210in}%
\pgfsys@useobject{currentmarker}{}%
\end{pgfscope}%
\begin{pgfscope}%
\pgfsys@transformshift{2.096514in}{0.698681in}%
\pgfsys@useobject{currentmarker}{}%
\end{pgfscope}%
\begin{pgfscope}%
\pgfsys@transformshift{2.096801in}{0.710577in}%
\pgfsys@useobject{currentmarker}{}%
\end{pgfscope}%
\begin{pgfscope}%
\pgfsys@transformshift{2.097088in}{0.723632in}%
\pgfsys@useobject{currentmarker}{}%
\end{pgfscope}%
\begin{pgfscope}%
\pgfsys@transformshift{2.097375in}{0.747204in}%
\pgfsys@useobject{currentmarker}{}%
\end{pgfscope}%
\begin{pgfscope}%
\pgfsys@transformshift{2.097661in}{0.739059in}%
\pgfsys@useobject{currentmarker}{}%
\end{pgfscope}%
\begin{pgfscope}%
\pgfsys@transformshift{2.097946in}{0.738377in}%
\pgfsys@useobject{currentmarker}{}%
\end{pgfscope}%
\begin{pgfscope}%
\pgfsys@transformshift{2.098231in}{0.703604in}%
\pgfsys@useobject{currentmarker}{}%
\end{pgfscope}%
\begin{pgfscope}%
\pgfsys@transformshift{2.098516in}{0.681517in}%
\pgfsys@useobject{currentmarker}{}%
\end{pgfscope}%
\begin{pgfscope}%
\pgfsys@transformshift{2.098801in}{0.700475in}%
\pgfsys@useobject{currentmarker}{}%
\end{pgfscope}%
\begin{pgfscope}%
\pgfsys@transformshift{2.099085in}{0.730318in}%
\pgfsys@useobject{currentmarker}{}%
\end{pgfscope}%
\begin{pgfscope}%
\pgfsys@transformshift{2.099369in}{0.743861in}%
\pgfsys@useobject{currentmarker}{}%
\end{pgfscope}%
\begin{pgfscope}%
\pgfsys@transformshift{2.099652in}{0.697812in}%
\pgfsys@useobject{currentmarker}{}%
\end{pgfscope}%
\begin{pgfscope}%
\pgfsys@transformshift{2.099936in}{0.682909in}%
\pgfsys@useobject{currentmarker}{}%
\end{pgfscope}%
\begin{pgfscope}%
\pgfsys@transformshift{2.100218in}{0.703960in}%
\pgfsys@useobject{currentmarker}{}%
\end{pgfscope}%
\begin{pgfscope}%
\pgfsys@transformshift{2.100501in}{0.711933in}%
\pgfsys@useobject{currentmarker}{}%
\end{pgfscope}%
\begin{pgfscope}%
\pgfsys@transformshift{2.100783in}{0.694501in}%
\pgfsys@useobject{currentmarker}{}%
\end{pgfscope}%
\begin{pgfscope}%
\pgfsys@transformshift{2.101064in}{0.724423in}%
\pgfsys@useobject{currentmarker}{}%
\end{pgfscope}%
\begin{pgfscope}%
\pgfsys@transformshift{2.101346in}{0.702244in}%
\pgfsys@useobject{currentmarker}{}%
\end{pgfscope}%
\begin{pgfscope}%
\pgfsys@transformshift{2.101627in}{0.676372in}%
\pgfsys@useobject{currentmarker}{}%
\end{pgfscope}%
\begin{pgfscope}%
\pgfsys@transformshift{2.101907in}{0.707221in}%
\pgfsys@useobject{currentmarker}{}%
\end{pgfscope}%
\begin{pgfscope}%
\pgfsys@transformshift{2.102188in}{0.699492in}%
\pgfsys@useobject{currentmarker}{}%
\end{pgfscope}%
\begin{pgfscope}%
\pgfsys@transformshift{2.102468in}{0.665012in}%
\pgfsys@useobject{currentmarker}{}%
\end{pgfscope}%
\begin{pgfscope}%
\pgfsys@transformshift{2.102747in}{0.657291in}%
\pgfsys@useobject{currentmarker}{}%
\end{pgfscope}%
\begin{pgfscope}%
\pgfsys@transformshift{2.103026in}{0.746681in}%
\pgfsys@useobject{currentmarker}{}%
\end{pgfscope}%
\begin{pgfscope}%
\pgfsys@transformshift{2.103305in}{0.735866in}%
\pgfsys@useobject{currentmarker}{}%
\end{pgfscope}%
\begin{pgfscope}%
\pgfsys@transformshift{2.103584in}{0.710850in}%
\pgfsys@useobject{currentmarker}{}%
\end{pgfscope}%
\begin{pgfscope}%
\pgfsys@transformshift{2.103862in}{0.712241in}%
\pgfsys@useobject{currentmarker}{}%
\end{pgfscope}%
\begin{pgfscope}%
\pgfsys@transformshift{2.104140in}{0.705714in}%
\pgfsys@useobject{currentmarker}{}%
\end{pgfscope}%
\begin{pgfscope}%
\pgfsys@transformshift{2.104417in}{0.615861in}%
\pgfsys@useobject{currentmarker}{}%
\end{pgfscope}%
\begin{pgfscope}%
\pgfsys@transformshift{2.104695in}{0.679406in}%
\pgfsys@useobject{currentmarker}{}%
\end{pgfscope}%
\begin{pgfscope}%
\pgfsys@transformshift{2.104972in}{0.704359in}%
\pgfsys@useobject{currentmarker}{}%
\end{pgfscope}%
\begin{pgfscope}%
\pgfsys@transformshift{2.105248in}{0.658729in}%
\pgfsys@useobject{currentmarker}{}%
\end{pgfscope}%
\begin{pgfscope}%
\pgfsys@transformshift{2.105524in}{0.649500in}%
\pgfsys@useobject{currentmarker}{}%
\end{pgfscope}%
\begin{pgfscope}%
\pgfsys@transformshift{2.105800in}{0.680622in}%
\pgfsys@useobject{currentmarker}{}%
\end{pgfscope}%
\begin{pgfscope}%
\pgfsys@transformshift{2.106075in}{0.713255in}%
\pgfsys@useobject{currentmarker}{}%
\end{pgfscope}%
\begin{pgfscope}%
\pgfsys@transformshift{2.106351in}{0.692285in}%
\pgfsys@useobject{currentmarker}{}%
\end{pgfscope}%
\begin{pgfscope}%
\pgfsys@transformshift{2.106625in}{0.703750in}%
\pgfsys@useobject{currentmarker}{}%
\end{pgfscope}%
\begin{pgfscope}%
\pgfsys@transformshift{2.106900in}{0.701087in}%
\pgfsys@useobject{currentmarker}{}%
\end{pgfscope}%
\begin{pgfscope}%
\pgfsys@transformshift{2.107174in}{0.689841in}%
\pgfsys@useobject{currentmarker}{}%
\end{pgfscope}%
\begin{pgfscope}%
\pgfsys@transformshift{2.107448in}{0.698794in}%
\pgfsys@useobject{currentmarker}{}%
\end{pgfscope}%
\begin{pgfscope}%
\pgfsys@transformshift{2.107721in}{0.675974in}%
\pgfsys@useobject{currentmarker}{}%
\end{pgfscope}%
\begin{pgfscope}%
\pgfsys@transformshift{2.107994in}{0.676578in}%
\pgfsys@useobject{currentmarker}{}%
\end{pgfscope}%
\begin{pgfscope}%
\pgfsys@transformshift{2.108267in}{0.687730in}%
\pgfsys@useobject{currentmarker}{}%
\end{pgfscope}%
\begin{pgfscope}%
\pgfsys@transformshift{2.108540in}{0.701167in}%
\pgfsys@useobject{currentmarker}{}%
\end{pgfscope}%
\begin{pgfscope}%
\pgfsys@transformshift{2.108812in}{0.665261in}%
\pgfsys@useobject{currentmarker}{}%
\end{pgfscope}%
\begin{pgfscope}%
\pgfsys@transformshift{2.109084in}{0.686478in}%
\pgfsys@useobject{currentmarker}{}%
\end{pgfscope}%
\begin{pgfscope}%
\pgfsys@transformshift{2.109355in}{0.740210in}%
\pgfsys@useobject{currentmarker}{}%
\end{pgfscope}%
\begin{pgfscope}%
\pgfsys@transformshift{2.109626in}{0.735376in}%
\pgfsys@useobject{currentmarker}{}%
\end{pgfscope}%
\begin{pgfscope}%
\pgfsys@transformshift{2.109897in}{0.687587in}%
\pgfsys@useobject{currentmarker}{}%
\end{pgfscope}%
\begin{pgfscope}%
\pgfsys@transformshift{2.110168in}{0.672287in}%
\pgfsys@useobject{currentmarker}{}%
\end{pgfscope}%
\begin{pgfscope}%
\pgfsys@transformshift{2.110438in}{0.694955in}%
\pgfsys@useobject{currentmarker}{}%
\end{pgfscope}%
\begin{pgfscope}%
\pgfsys@transformshift{2.110708in}{0.663092in}%
\pgfsys@useobject{currentmarker}{}%
\end{pgfscope}%
\begin{pgfscope}%
\pgfsys@transformshift{2.110977in}{0.612800in}%
\pgfsys@useobject{currentmarker}{}%
\end{pgfscope}%
\begin{pgfscope}%
\pgfsys@transformshift{2.111246in}{0.669606in}%
\pgfsys@useobject{currentmarker}{}%
\end{pgfscope}%
\begin{pgfscope}%
\pgfsys@transformshift{2.111515in}{0.720428in}%
\pgfsys@useobject{currentmarker}{}%
\end{pgfscope}%
\begin{pgfscope}%
\pgfsys@transformshift{2.111784in}{0.710795in}%
\pgfsys@useobject{currentmarker}{}%
\end{pgfscope}%
\begin{pgfscope}%
\pgfsys@transformshift{2.112052in}{0.712077in}%
\pgfsys@useobject{currentmarker}{}%
\end{pgfscope}%
\begin{pgfscope}%
\pgfsys@transformshift{2.112320in}{0.709327in}%
\pgfsys@useobject{currentmarker}{}%
\end{pgfscope}%
\begin{pgfscope}%
\pgfsys@transformshift{2.112588in}{0.690607in}%
\pgfsys@useobject{currentmarker}{}%
\end{pgfscope}%
\begin{pgfscope}%
\pgfsys@transformshift{2.112855in}{0.716323in}%
\pgfsys@useobject{currentmarker}{}%
\end{pgfscope}%
\begin{pgfscope}%
\pgfsys@transformshift{2.113122in}{0.683422in}%
\pgfsys@useobject{currentmarker}{}%
\end{pgfscope}%
\begin{pgfscope}%
\pgfsys@transformshift{2.113388in}{0.702446in}%
\pgfsys@useobject{currentmarker}{}%
\end{pgfscope}%
\begin{pgfscope}%
\pgfsys@transformshift{2.113655in}{0.691182in}%
\pgfsys@useobject{currentmarker}{}%
\end{pgfscope}%
\begin{pgfscope}%
\pgfsys@transformshift{2.113921in}{0.716278in}%
\pgfsys@useobject{currentmarker}{}%
\end{pgfscope}%
\begin{pgfscope}%
\pgfsys@transformshift{2.114187in}{0.755392in}%
\pgfsys@useobject{currentmarker}{}%
\end{pgfscope}%
\begin{pgfscope}%
\pgfsys@transformshift{2.114452in}{0.730724in}%
\pgfsys@useobject{currentmarker}{}%
\end{pgfscope}%
\begin{pgfscope}%
\pgfsys@transformshift{2.114717in}{0.669069in}%
\pgfsys@useobject{currentmarker}{}%
\end{pgfscope}%
\begin{pgfscope}%
\pgfsys@transformshift{2.114982in}{0.642417in}%
\pgfsys@useobject{currentmarker}{}%
\end{pgfscope}%
\begin{pgfscope}%
\pgfsys@transformshift{2.115246in}{0.670023in}%
\pgfsys@useobject{currentmarker}{}%
\end{pgfscope}%
\begin{pgfscope}%
\pgfsys@transformshift{2.115510in}{0.718623in}%
\pgfsys@useobject{currentmarker}{}%
\end{pgfscope}%
\begin{pgfscope}%
\pgfsys@transformshift{2.115774in}{0.693194in}%
\pgfsys@useobject{currentmarker}{}%
\end{pgfscope}%
\begin{pgfscope}%
\pgfsys@transformshift{2.116038in}{0.623022in}%
\pgfsys@useobject{currentmarker}{}%
\end{pgfscope}%
\begin{pgfscope}%
\pgfsys@transformshift{2.116301in}{0.638122in}%
\pgfsys@useobject{currentmarker}{}%
\end{pgfscope}%
\begin{pgfscope}%
\pgfsys@transformshift{2.116564in}{0.642397in}%
\pgfsys@useobject{currentmarker}{}%
\end{pgfscope}%
\begin{pgfscope}%
\pgfsys@transformshift{2.116826in}{0.683617in}%
\pgfsys@useobject{currentmarker}{}%
\end{pgfscope}%
\begin{pgfscope}%
\pgfsys@transformshift{2.117089in}{0.693917in}%
\pgfsys@useobject{currentmarker}{}%
\end{pgfscope}%
\begin{pgfscope}%
\pgfsys@transformshift{2.117351in}{0.705735in}%
\pgfsys@useobject{currentmarker}{}%
\end{pgfscope}%
\begin{pgfscope}%
\pgfsys@transformshift{2.117612in}{0.724175in}%
\pgfsys@useobject{currentmarker}{}%
\end{pgfscope}%
\begin{pgfscope}%
\pgfsys@transformshift{2.117874in}{0.693062in}%
\pgfsys@useobject{currentmarker}{}%
\end{pgfscope}%
\begin{pgfscope}%
\pgfsys@transformshift{2.118135in}{0.683443in}%
\pgfsys@useobject{currentmarker}{}%
\end{pgfscope}%
\begin{pgfscope}%
\pgfsys@transformshift{2.118396in}{0.674757in}%
\pgfsys@useobject{currentmarker}{}%
\end{pgfscope}%
\begin{pgfscope}%
\pgfsys@transformshift{2.118656in}{0.681696in}%
\pgfsys@useobject{currentmarker}{}%
\end{pgfscope}%
\begin{pgfscope}%
\pgfsys@transformshift{2.118916in}{0.668855in}%
\pgfsys@useobject{currentmarker}{}%
\end{pgfscope}%
\begin{pgfscope}%
\pgfsys@transformshift{2.119176in}{0.655898in}%
\pgfsys@useobject{currentmarker}{}%
\end{pgfscope}%
\begin{pgfscope}%
\pgfsys@transformshift{2.119436in}{0.666543in}%
\pgfsys@useobject{currentmarker}{}%
\end{pgfscope}%
\begin{pgfscope}%
\pgfsys@transformshift{2.119695in}{0.659745in}%
\pgfsys@useobject{currentmarker}{}%
\end{pgfscope}%
\begin{pgfscope}%
\pgfsys@transformshift{2.119954in}{0.650855in}%
\pgfsys@useobject{currentmarker}{}%
\end{pgfscope}%
\begin{pgfscope}%
\pgfsys@transformshift{2.120213in}{0.688619in}%
\pgfsys@useobject{currentmarker}{}%
\end{pgfscope}%
\begin{pgfscope}%
\pgfsys@transformshift{2.120471in}{0.704388in}%
\pgfsys@useobject{currentmarker}{}%
\end{pgfscope}%
\begin{pgfscope}%
\pgfsys@transformshift{2.120729in}{0.704412in}%
\pgfsys@useobject{currentmarker}{}%
\end{pgfscope}%
\begin{pgfscope}%
\pgfsys@transformshift{2.120987in}{0.683473in}%
\pgfsys@useobject{currentmarker}{}%
\end{pgfscope}%
\begin{pgfscope}%
\pgfsys@transformshift{2.121244in}{0.705541in}%
\pgfsys@useobject{currentmarker}{}%
\end{pgfscope}%
\begin{pgfscope}%
\pgfsys@transformshift{2.121501in}{0.742238in}%
\pgfsys@useobject{currentmarker}{}%
\end{pgfscope}%
\begin{pgfscope}%
\pgfsys@transformshift{2.121758in}{0.738039in}%
\pgfsys@useobject{currentmarker}{}%
\end{pgfscope}%
\begin{pgfscope}%
\pgfsys@transformshift{2.122015in}{0.694400in}%
\pgfsys@useobject{currentmarker}{}%
\end{pgfscope}%
\begin{pgfscope}%
\pgfsys@transformshift{2.122271in}{0.689326in}%
\pgfsys@useobject{currentmarker}{}%
\end{pgfscope}%
\begin{pgfscope}%
\pgfsys@transformshift{2.122527in}{0.734128in}%
\pgfsys@useobject{currentmarker}{}%
\end{pgfscope}%
\begin{pgfscope}%
\pgfsys@transformshift{2.122783in}{0.732815in}%
\pgfsys@useobject{currentmarker}{}%
\end{pgfscope}%
\begin{pgfscope}%
\pgfsys@transformshift{2.123038in}{0.722618in}%
\pgfsys@useobject{currentmarker}{}%
\end{pgfscope}%
\begin{pgfscope}%
\pgfsys@transformshift{2.123293in}{0.661240in}%
\pgfsys@useobject{currentmarker}{}%
\end{pgfscope}%
\begin{pgfscope}%
\pgfsys@transformshift{2.123548in}{0.691315in}%
\pgfsys@useobject{currentmarker}{}%
\end{pgfscope}%
\begin{pgfscope}%
\pgfsys@transformshift{2.123803in}{0.709684in}%
\pgfsys@useobject{currentmarker}{}%
\end{pgfscope}%
\begin{pgfscope}%
\pgfsys@transformshift{2.124057in}{0.700585in}%
\pgfsys@useobject{currentmarker}{}%
\end{pgfscope}%
\begin{pgfscope}%
\pgfsys@transformshift{2.124311in}{0.674797in}%
\pgfsys@useobject{currentmarker}{}%
\end{pgfscope}%
\begin{pgfscope}%
\pgfsys@transformshift{2.124565in}{0.683385in}%
\pgfsys@useobject{currentmarker}{}%
\end{pgfscope}%
\begin{pgfscope}%
\pgfsys@transformshift{2.124818in}{0.714241in}%
\pgfsys@useobject{currentmarker}{}%
\end{pgfscope}%
\begin{pgfscope}%
\pgfsys@transformshift{2.125071in}{0.720918in}%
\pgfsys@useobject{currentmarker}{}%
\end{pgfscope}%
\begin{pgfscope}%
\pgfsys@transformshift{2.125324in}{0.710821in}%
\pgfsys@useobject{currentmarker}{}%
\end{pgfscope}%
\begin{pgfscope}%
\pgfsys@transformshift{2.125577in}{0.672604in}%
\pgfsys@useobject{currentmarker}{}%
\end{pgfscope}%
\begin{pgfscope}%
\pgfsys@transformshift{2.125829in}{0.715140in}%
\pgfsys@useobject{currentmarker}{}%
\end{pgfscope}%
\begin{pgfscope}%
\pgfsys@transformshift{2.126081in}{0.688109in}%
\pgfsys@useobject{currentmarker}{}%
\end{pgfscope}%
\begin{pgfscope}%
\pgfsys@transformshift{2.126333in}{0.699245in}%
\pgfsys@useobject{currentmarker}{}%
\end{pgfscope}%
\begin{pgfscope}%
\pgfsys@transformshift{2.126584in}{0.727374in}%
\pgfsys@useobject{currentmarker}{}%
\end{pgfscope}%
\begin{pgfscope}%
\pgfsys@transformshift{2.126835in}{0.702729in}%
\pgfsys@useobject{currentmarker}{}%
\end{pgfscope}%
\begin{pgfscope}%
\pgfsys@transformshift{2.127086in}{0.680082in}%
\pgfsys@useobject{currentmarker}{}%
\end{pgfscope}%
\begin{pgfscope}%
\pgfsys@transformshift{2.127337in}{0.693391in}%
\pgfsys@useobject{currentmarker}{}%
\end{pgfscope}%
\begin{pgfscope}%
\pgfsys@transformshift{2.127587in}{0.696358in}%
\pgfsys@useobject{currentmarker}{}%
\end{pgfscope}%
\begin{pgfscope}%
\pgfsys@transformshift{2.127837in}{0.646178in}%
\pgfsys@useobject{currentmarker}{}%
\end{pgfscope}%
\begin{pgfscope}%
\pgfsys@transformshift{2.128087in}{0.537770in}%
\pgfsys@useobject{currentmarker}{}%
\end{pgfscope}%
\begin{pgfscope}%
\pgfsys@transformshift{2.128336in}{0.641500in}%
\pgfsys@useobject{currentmarker}{}%
\end{pgfscope}%
\begin{pgfscope}%
\pgfsys@transformshift{2.128586in}{0.655168in}%
\pgfsys@useobject{currentmarker}{}%
\end{pgfscope}%
\begin{pgfscope}%
\pgfsys@transformshift{2.128835in}{0.667188in}%
\pgfsys@useobject{currentmarker}{}%
\end{pgfscope}%
\begin{pgfscope}%
\pgfsys@transformshift{2.129083in}{0.688585in}%
\pgfsys@useobject{currentmarker}{}%
\end{pgfscope}%
\begin{pgfscope}%
\pgfsys@transformshift{2.129332in}{0.712099in}%
\pgfsys@useobject{currentmarker}{}%
\end{pgfscope}%
\begin{pgfscope}%
\pgfsys@transformshift{2.129580in}{0.710192in}%
\pgfsys@useobject{currentmarker}{}%
\end{pgfscope}%
\begin{pgfscope}%
\pgfsys@transformshift{2.129827in}{0.695364in}%
\pgfsys@useobject{currentmarker}{}%
\end{pgfscope}%
\begin{pgfscope}%
\pgfsys@transformshift{2.130075in}{0.712881in}%
\pgfsys@useobject{currentmarker}{}%
\end{pgfscope}%
\begin{pgfscope}%
\pgfsys@transformshift{2.130322in}{0.669897in}%
\pgfsys@useobject{currentmarker}{}%
\end{pgfscope}%
\begin{pgfscope}%
\pgfsys@transformshift{2.130569in}{0.648365in}%
\pgfsys@useobject{currentmarker}{}%
\end{pgfscope}%
\begin{pgfscope}%
\pgfsys@transformshift{2.130816in}{0.627377in}%
\pgfsys@useobject{currentmarker}{}%
\end{pgfscope}%
\begin{pgfscope}%
\pgfsys@transformshift{2.131062in}{0.668525in}%
\pgfsys@useobject{currentmarker}{}%
\end{pgfscope}%
\begin{pgfscope}%
\pgfsys@transformshift{2.131309in}{0.678622in}%
\pgfsys@useobject{currentmarker}{}%
\end{pgfscope}%
\begin{pgfscope}%
\pgfsys@transformshift{2.131555in}{0.684508in}%
\pgfsys@useobject{currentmarker}{}%
\end{pgfscope}%
\begin{pgfscope}%
\pgfsys@transformshift{2.131800in}{0.701809in}%
\pgfsys@useobject{currentmarker}{}%
\end{pgfscope}%
\begin{pgfscope}%
\pgfsys@transformshift{2.132046in}{0.674729in}%
\pgfsys@useobject{currentmarker}{}%
\end{pgfscope}%
\begin{pgfscope}%
\pgfsys@transformshift{2.132291in}{0.696212in}%
\pgfsys@useobject{currentmarker}{}%
\end{pgfscope}%
\begin{pgfscope}%
\pgfsys@transformshift{2.132536in}{0.721686in}%
\pgfsys@useobject{currentmarker}{}%
\end{pgfscope}%
\begin{pgfscope}%
\pgfsys@transformshift{2.132780in}{0.698003in}%
\pgfsys@useobject{currentmarker}{}%
\end{pgfscope}%
\begin{pgfscope}%
\pgfsys@transformshift{2.133025in}{0.650063in}%
\pgfsys@useobject{currentmarker}{}%
\end{pgfscope}%
\begin{pgfscope}%
\pgfsys@transformshift{2.133269in}{0.686652in}%
\pgfsys@useobject{currentmarker}{}%
\end{pgfscope}%
\begin{pgfscope}%
\pgfsys@transformshift{2.133512in}{0.656210in}%
\pgfsys@useobject{currentmarker}{}%
\end{pgfscope}%
\begin{pgfscope}%
\pgfsys@transformshift{2.133756in}{0.671656in}%
\pgfsys@useobject{currentmarker}{}%
\end{pgfscope}%
\begin{pgfscope}%
\pgfsys@transformshift{2.133999in}{0.696658in}%
\pgfsys@useobject{currentmarker}{}%
\end{pgfscope}%
\begin{pgfscope}%
\pgfsys@transformshift{2.134242in}{0.745930in}%
\pgfsys@useobject{currentmarker}{}%
\end{pgfscope}%
\begin{pgfscope}%
\pgfsys@transformshift{2.134485in}{0.724559in}%
\pgfsys@useobject{currentmarker}{}%
\end{pgfscope}%
\begin{pgfscope}%
\pgfsys@transformshift{2.134727in}{0.700064in}%
\pgfsys@useobject{currentmarker}{}%
\end{pgfscope}%
\begin{pgfscope}%
\pgfsys@transformshift{2.134970in}{0.730558in}%
\pgfsys@useobject{currentmarker}{}%
\end{pgfscope}%
\begin{pgfscope}%
\pgfsys@transformshift{2.135212in}{0.713248in}%
\pgfsys@useobject{currentmarker}{}%
\end{pgfscope}%
\begin{pgfscope}%
\pgfsys@transformshift{2.135453in}{0.715279in}%
\pgfsys@useobject{currentmarker}{}%
\end{pgfscope}%
\begin{pgfscope}%
\pgfsys@transformshift{2.135695in}{0.705786in}%
\pgfsys@useobject{currentmarker}{}%
\end{pgfscope}%
\begin{pgfscope}%
\pgfsys@transformshift{2.135936in}{0.704108in}%
\pgfsys@useobject{currentmarker}{}%
\end{pgfscope}%
\begin{pgfscope}%
\pgfsys@transformshift{2.136177in}{0.701909in}%
\pgfsys@useobject{currentmarker}{}%
\end{pgfscope}%
\begin{pgfscope}%
\pgfsys@transformshift{2.136417in}{0.743148in}%
\pgfsys@useobject{currentmarker}{}%
\end{pgfscope}%
\begin{pgfscope}%
\pgfsys@transformshift{2.136658in}{0.715475in}%
\pgfsys@useobject{currentmarker}{}%
\end{pgfscope}%
\begin{pgfscope}%
\pgfsys@transformshift{2.136898in}{0.693576in}%
\pgfsys@useobject{currentmarker}{}%
\end{pgfscope}%
\begin{pgfscope}%
\pgfsys@transformshift{2.137138in}{0.700562in}%
\pgfsys@useobject{currentmarker}{}%
\end{pgfscope}%
\begin{pgfscope}%
\pgfsys@transformshift{2.137377in}{0.670131in}%
\pgfsys@useobject{currentmarker}{}%
\end{pgfscope}%
\begin{pgfscope}%
\pgfsys@transformshift{2.137617in}{0.653931in}%
\pgfsys@useobject{currentmarker}{}%
\end{pgfscope}%
\begin{pgfscope}%
\pgfsys@transformshift{2.137856in}{0.637718in}%
\pgfsys@useobject{currentmarker}{}%
\end{pgfscope}%
\begin{pgfscope}%
\pgfsys@transformshift{2.138095in}{0.683129in}%
\pgfsys@useobject{currentmarker}{}%
\end{pgfscope}%
\begin{pgfscope}%
\pgfsys@transformshift{2.138333in}{0.675043in}%
\pgfsys@useobject{currentmarker}{}%
\end{pgfscope}%
\begin{pgfscope}%
\pgfsys@transformshift{2.138572in}{0.687880in}%
\pgfsys@useobject{currentmarker}{}%
\end{pgfscope}%
\begin{pgfscope}%
\pgfsys@transformshift{2.138810in}{0.707329in}%
\pgfsys@useobject{currentmarker}{}%
\end{pgfscope}%
\begin{pgfscope}%
\pgfsys@transformshift{2.139048in}{0.702201in}%
\pgfsys@useobject{currentmarker}{}%
\end{pgfscope}%
\begin{pgfscope}%
\pgfsys@transformshift{2.139285in}{0.706183in}%
\pgfsys@useobject{currentmarker}{}%
\end{pgfscope}%
\begin{pgfscope}%
\pgfsys@transformshift{2.139523in}{0.694124in}%
\pgfsys@useobject{currentmarker}{}%
\end{pgfscope}%
\begin{pgfscope}%
\pgfsys@transformshift{2.139760in}{0.729781in}%
\pgfsys@useobject{currentmarker}{}%
\end{pgfscope}%
\begin{pgfscope}%
\pgfsys@transformshift{2.139997in}{0.708043in}%
\pgfsys@useobject{currentmarker}{}%
\end{pgfscope}%
\begin{pgfscope}%
\pgfsys@transformshift{2.140233in}{0.677962in}%
\pgfsys@useobject{currentmarker}{}%
\end{pgfscope}%
\begin{pgfscope}%
\pgfsys@transformshift{2.140470in}{0.643594in}%
\pgfsys@useobject{currentmarker}{}%
\end{pgfscope}%
\begin{pgfscope}%
\pgfsys@transformshift{2.140706in}{0.639047in}%
\pgfsys@useobject{currentmarker}{}%
\end{pgfscope}%
\begin{pgfscope}%
\pgfsys@transformshift{2.140942in}{0.657941in}%
\pgfsys@useobject{currentmarker}{}%
\end{pgfscope}%
\begin{pgfscope}%
\pgfsys@transformshift{2.141177in}{0.648285in}%
\pgfsys@useobject{currentmarker}{}%
\end{pgfscope}%
\begin{pgfscope}%
\pgfsys@transformshift{2.141412in}{0.716770in}%
\pgfsys@useobject{currentmarker}{}%
\end{pgfscope}%
\begin{pgfscope}%
\pgfsys@transformshift{2.141648in}{0.721059in}%
\pgfsys@useobject{currentmarker}{}%
\end{pgfscope}%
\begin{pgfscope}%
\pgfsys@transformshift{2.141882in}{0.686773in}%
\pgfsys@useobject{currentmarker}{}%
\end{pgfscope}%
\begin{pgfscope}%
\pgfsys@transformshift{2.142117in}{0.685631in}%
\pgfsys@useobject{currentmarker}{}%
\end{pgfscope}%
\begin{pgfscope}%
\pgfsys@transformshift{2.142351in}{0.679741in}%
\pgfsys@useobject{currentmarker}{}%
\end{pgfscope}%
\begin{pgfscope}%
\pgfsys@transformshift{2.142586in}{0.640645in}%
\pgfsys@useobject{currentmarker}{}%
\end{pgfscope}%
\begin{pgfscope}%
\pgfsys@transformshift{2.142820in}{0.650478in}%
\pgfsys@useobject{currentmarker}{}%
\end{pgfscope}%
\begin{pgfscope}%
\pgfsys@transformshift{2.143053in}{0.692080in}%
\pgfsys@useobject{currentmarker}{}%
\end{pgfscope}%
\begin{pgfscope}%
\pgfsys@transformshift{2.143287in}{0.700815in}%
\pgfsys@useobject{currentmarker}{}%
\end{pgfscope}%
\begin{pgfscope}%
\pgfsys@transformshift{2.143520in}{0.714060in}%
\pgfsys@useobject{currentmarker}{}%
\end{pgfscope}%
\begin{pgfscope}%
\pgfsys@transformshift{2.143753in}{0.705327in}%
\pgfsys@useobject{currentmarker}{}%
\end{pgfscope}%
\begin{pgfscope}%
\pgfsys@transformshift{2.143985in}{0.690826in}%
\pgfsys@useobject{currentmarker}{}%
\end{pgfscope}%
\begin{pgfscope}%
\pgfsys@transformshift{2.144218in}{0.745522in}%
\pgfsys@useobject{currentmarker}{}%
\end{pgfscope}%
\begin{pgfscope}%
\pgfsys@transformshift{2.144450in}{0.741101in}%
\pgfsys@useobject{currentmarker}{}%
\end{pgfscope}%
\begin{pgfscope}%
\pgfsys@transformshift{2.144682in}{0.690825in}%
\pgfsys@useobject{currentmarker}{}%
\end{pgfscope}%
\begin{pgfscope}%
\pgfsys@transformshift{2.144914in}{0.687563in}%
\pgfsys@useobject{currentmarker}{}%
\end{pgfscope}%
\begin{pgfscope}%
\pgfsys@transformshift{2.145145in}{0.634590in}%
\pgfsys@useobject{currentmarker}{}%
\end{pgfscope}%
\begin{pgfscope}%
\pgfsys@transformshift{2.145376in}{0.643679in}%
\pgfsys@useobject{currentmarker}{}%
\end{pgfscope}%
\begin{pgfscope}%
\pgfsys@transformshift{2.145607in}{0.682764in}%
\pgfsys@useobject{currentmarker}{}%
\end{pgfscope}%
\begin{pgfscope}%
\pgfsys@transformshift{2.145838in}{0.702387in}%
\pgfsys@useobject{currentmarker}{}%
\end{pgfscope}%
\begin{pgfscope}%
\pgfsys@transformshift{2.146069in}{0.729294in}%
\pgfsys@useobject{currentmarker}{}%
\end{pgfscope}%
\begin{pgfscope}%
\pgfsys@transformshift{2.146299in}{0.707105in}%
\pgfsys@useobject{currentmarker}{}%
\end{pgfscope}%
\begin{pgfscope}%
\pgfsys@transformshift{2.146529in}{0.699953in}%
\pgfsys@useobject{currentmarker}{}%
\end{pgfscope}%
\begin{pgfscope}%
\pgfsys@transformshift{2.146759in}{0.673328in}%
\pgfsys@useobject{currentmarker}{}%
\end{pgfscope}%
\begin{pgfscope}%
\pgfsys@transformshift{2.146988in}{0.622389in}%
\pgfsys@useobject{currentmarker}{}%
\end{pgfscope}%
\begin{pgfscope}%
\pgfsys@transformshift{2.147218in}{0.614056in}%
\pgfsys@useobject{currentmarker}{}%
\end{pgfscope}%
\begin{pgfscope}%
\pgfsys@transformshift{2.147447in}{0.656328in}%
\pgfsys@useobject{currentmarker}{}%
\end{pgfscope}%
\begin{pgfscope}%
\pgfsys@transformshift{2.147676in}{0.663367in}%
\pgfsys@useobject{currentmarker}{}%
\end{pgfscope}%
\begin{pgfscope}%
\pgfsys@transformshift{2.147904in}{0.635906in}%
\pgfsys@useobject{currentmarker}{}%
\end{pgfscope}%
\begin{pgfscope}%
\pgfsys@transformshift{2.148133in}{0.667915in}%
\pgfsys@useobject{currentmarker}{}%
\end{pgfscope}%
\begin{pgfscope}%
\pgfsys@transformshift{2.148361in}{0.684408in}%
\pgfsys@useobject{currentmarker}{}%
\end{pgfscope}%
\begin{pgfscope}%
\pgfsys@transformshift{2.148589in}{0.685674in}%
\pgfsys@useobject{currentmarker}{}%
\end{pgfscope}%
\begin{pgfscope}%
\pgfsys@transformshift{2.148817in}{0.702998in}%
\pgfsys@useobject{currentmarker}{}%
\end{pgfscope}%
\begin{pgfscope}%
\pgfsys@transformshift{2.149044in}{0.652902in}%
\pgfsys@useobject{currentmarker}{}%
\end{pgfscope}%
\begin{pgfscope}%
\pgfsys@transformshift{2.149271in}{0.637779in}%
\pgfsys@useobject{currentmarker}{}%
\end{pgfscope}%
\begin{pgfscope}%
\pgfsys@transformshift{2.149499in}{0.683954in}%
\pgfsys@useobject{currentmarker}{}%
\end{pgfscope}%
\begin{pgfscope}%
\pgfsys@transformshift{2.149725in}{0.701856in}%
\pgfsys@useobject{currentmarker}{}%
\end{pgfscope}%
\begin{pgfscope}%
\pgfsys@transformshift{2.149952in}{0.685950in}%
\pgfsys@useobject{currentmarker}{}%
\end{pgfscope}%
\begin{pgfscope}%
\pgfsys@transformshift{2.150178in}{0.681614in}%
\pgfsys@useobject{currentmarker}{}%
\end{pgfscope}%
\begin{pgfscope}%
\pgfsys@transformshift{2.150404in}{0.698868in}%
\pgfsys@useobject{currentmarker}{}%
\end{pgfscope}%
\begin{pgfscope}%
\pgfsys@transformshift{2.150630in}{0.725756in}%
\pgfsys@useobject{currentmarker}{}%
\end{pgfscope}%
\begin{pgfscope}%
\pgfsys@transformshift{2.150856in}{0.738601in}%
\pgfsys@useobject{currentmarker}{}%
\end{pgfscope}%
\begin{pgfscope}%
\pgfsys@transformshift{2.151081in}{0.705758in}%
\pgfsys@useobject{currentmarker}{}%
\end{pgfscope}%
\begin{pgfscope}%
\pgfsys@transformshift{2.151307in}{0.711785in}%
\pgfsys@useobject{currentmarker}{}%
\end{pgfscope}%
\begin{pgfscope}%
\pgfsys@transformshift{2.151532in}{0.723298in}%
\pgfsys@useobject{currentmarker}{}%
\end{pgfscope}%
\begin{pgfscope}%
\pgfsys@transformshift{2.151756in}{0.753283in}%
\pgfsys@useobject{currentmarker}{}%
\end{pgfscope}%
\begin{pgfscope}%
\pgfsys@transformshift{2.151981in}{0.742932in}%
\pgfsys@useobject{currentmarker}{}%
\end{pgfscope}%
\begin{pgfscope}%
\pgfsys@transformshift{2.152205in}{0.703405in}%
\pgfsys@useobject{currentmarker}{}%
\end{pgfscope}%
\begin{pgfscope}%
\pgfsys@transformshift{2.152429in}{0.694349in}%
\pgfsys@useobject{currentmarker}{}%
\end{pgfscope}%
\begin{pgfscope}%
\pgfsys@transformshift{2.152653in}{0.701611in}%
\pgfsys@useobject{currentmarker}{}%
\end{pgfscope}%
\begin{pgfscope}%
\pgfsys@transformshift{2.152877in}{0.706722in}%
\pgfsys@useobject{currentmarker}{}%
\end{pgfscope}%
\begin{pgfscope}%
\pgfsys@transformshift{2.153100in}{0.712846in}%
\pgfsys@useobject{currentmarker}{}%
\end{pgfscope}%
\begin{pgfscope}%
\pgfsys@transformshift{2.153323in}{0.713600in}%
\pgfsys@useobject{currentmarker}{}%
\end{pgfscope}%
\begin{pgfscope}%
\pgfsys@transformshift{2.153546in}{0.692770in}%
\pgfsys@useobject{currentmarker}{}%
\end{pgfscope}%
\begin{pgfscope}%
\pgfsys@transformshift{2.153769in}{0.683829in}%
\pgfsys@useobject{currentmarker}{}%
\end{pgfscope}%
\begin{pgfscope}%
\pgfsys@transformshift{2.153992in}{0.666580in}%
\pgfsys@useobject{currentmarker}{}%
\end{pgfscope}%
\begin{pgfscope}%
\pgfsys@transformshift{2.154214in}{0.625795in}%
\pgfsys@useobject{currentmarker}{}%
\end{pgfscope}%
\begin{pgfscope}%
\pgfsys@transformshift{2.154436in}{0.633319in}%
\pgfsys@useobject{currentmarker}{}%
\end{pgfscope}%
\begin{pgfscope}%
\pgfsys@transformshift{2.154658in}{0.649150in}%
\pgfsys@useobject{currentmarker}{}%
\end{pgfscope}%
\begin{pgfscope}%
\pgfsys@transformshift{2.154880in}{0.641417in}%
\pgfsys@useobject{currentmarker}{}%
\end{pgfscope}%
\begin{pgfscope}%
\pgfsys@transformshift{2.155101in}{0.616325in}%
\pgfsys@useobject{currentmarker}{}%
\end{pgfscope}%
\begin{pgfscope}%
\pgfsys@transformshift{2.155322in}{0.651545in}%
\pgfsys@useobject{currentmarker}{}%
\end{pgfscope}%
\begin{pgfscope}%
\pgfsys@transformshift{2.155543in}{0.668711in}%
\pgfsys@useobject{currentmarker}{}%
\end{pgfscope}%
\begin{pgfscope}%
\pgfsys@transformshift{2.155764in}{0.697307in}%
\pgfsys@useobject{currentmarker}{}%
\end{pgfscope}%
\begin{pgfscope}%
\pgfsys@transformshift{2.155985in}{0.708155in}%
\pgfsys@useobject{currentmarker}{}%
\end{pgfscope}%
\begin{pgfscope}%
\pgfsys@transformshift{2.156205in}{0.725422in}%
\pgfsys@useobject{currentmarker}{}%
\end{pgfscope}%
\begin{pgfscope}%
\pgfsys@transformshift{2.156425in}{0.697613in}%
\pgfsys@useobject{currentmarker}{}%
\end{pgfscope}%
\begin{pgfscope}%
\pgfsys@transformshift{2.156645in}{0.676023in}%
\pgfsys@useobject{currentmarker}{}%
\end{pgfscope}%
\begin{pgfscope}%
\pgfsys@transformshift{2.156865in}{0.714686in}%
\pgfsys@useobject{currentmarker}{}%
\end{pgfscope}%
\begin{pgfscope}%
\pgfsys@transformshift{2.157084in}{0.705988in}%
\pgfsys@useobject{currentmarker}{}%
\end{pgfscope}%
\begin{pgfscope}%
\pgfsys@transformshift{2.157304in}{0.648566in}%
\pgfsys@useobject{currentmarker}{}%
\end{pgfscope}%
\begin{pgfscope}%
\pgfsys@transformshift{2.157523in}{0.690857in}%
\pgfsys@useobject{currentmarker}{}%
\end{pgfscope}%
\begin{pgfscope}%
\pgfsys@transformshift{2.157741in}{0.715701in}%
\pgfsys@useobject{currentmarker}{}%
\end{pgfscope}%
\begin{pgfscope}%
\pgfsys@transformshift{2.157960in}{0.727885in}%
\pgfsys@useobject{currentmarker}{}%
\end{pgfscope}%
\begin{pgfscope}%
\pgfsys@transformshift{2.158179in}{0.713630in}%
\pgfsys@useobject{currentmarker}{}%
\end{pgfscope}%
\begin{pgfscope}%
\pgfsys@transformshift{2.158397in}{0.678393in}%
\pgfsys@useobject{currentmarker}{}%
\end{pgfscope}%
\begin{pgfscope}%
\pgfsys@transformshift{2.158615in}{0.677546in}%
\pgfsys@useobject{currentmarker}{}%
\end{pgfscope}%
\begin{pgfscope}%
\pgfsys@transformshift{2.158833in}{0.666964in}%
\pgfsys@useobject{currentmarker}{}%
\end{pgfscope}%
\begin{pgfscope}%
\pgfsys@transformshift{2.159050in}{0.683486in}%
\pgfsys@useobject{currentmarker}{}%
\end{pgfscope}%
\begin{pgfscope}%
\pgfsys@transformshift{2.159268in}{0.717500in}%
\pgfsys@useobject{currentmarker}{}%
\end{pgfscope}%
\begin{pgfscope}%
\pgfsys@transformshift{2.159485in}{0.697298in}%
\pgfsys@useobject{currentmarker}{}%
\end{pgfscope}%
\begin{pgfscope}%
\pgfsys@transformshift{2.159702in}{0.698609in}%
\pgfsys@useobject{currentmarker}{}%
\end{pgfscope}%
\begin{pgfscope}%
\pgfsys@transformshift{2.159918in}{0.709981in}%
\pgfsys@useobject{currentmarker}{}%
\end{pgfscope}%
\begin{pgfscope}%
\pgfsys@transformshift{2.160135in}{0.711468in}%
\pgfsys@useobject{currentmarker}{}%
\end{pgfscope}%
\begin{pgfscope}%
\pgfsys@transformshift{2.160351in}{0.695103in}%
\pgfsys@useobject{currentmarker}{}%
\end{pgfscope}%
\begin{pgfscope}%
\pgfsys@transformshift{2.160567in}{0.627772in}%
\pgfsys@useobject{currentmarker}{}%
\end{pgfscope}%
\begin{pgfscope}%
\pgfsys@transformshift{2.160783in}{0.673360in}%
\pgfsys@useobject{currentmarker}{}%
\end{pgfscope}%
\begin{pgfscope}%
\pgfsys@transformshift{2.160999in}{0.635575in}%
\pgfsys@useobject{currentmarker}{}%
\end{pgfscope}%
\begin{pgfscope}%
\pgfsys@transformshift{2.161214in}{0.664083in}%
\pgfsys@useobject{currentmarker}{}%
\end{pgfscope}%
\begin{pgfscope}%
\pgfsys@transformshift{2.161430in}{0.742918in}%
\pgfsys@useobject{currentmarker}{}%
\end{pgfscope}%
\begin{pgfscope}%
\pgfsys@transformshift{2.161645in}{0.731537in}%
\pgfsys@useobject{currentmarker}{}%
\end{pgfscope}%
\begin{pgfscope}%
\pgfsys@transformshift{2.161860in}{0.682687in}%
\pgfsys@useobject{currentmarker}{}%
\end{pgfscope}%
\begin{pgfscope}%
\pgfsys@transformshift{2.162074in}{0.711775in}%
\pgfsys@useobject{currentmarker}{}%
\end{pgfscope}%
\begin{pgfscope}%
\pgfsys@transformshift{2.162289in}{0.680837in}%
\pgfsys@useobject{currentmarker}{}%
\end{pgfscope}%
\begin{pgfscope}%
\pgfsys@transformshift{2.162503in}{0.642610in}%
\pgfsys@useobject{currentmarker}{}%
\end{pgfscope}%
\begin{pgfscope}%
\pgfsys@transformshift{2.162717in}{0.701346in}%
\pgfsys@useobject{currentmarker}{}%
\end{pgfscope}%
\begin{pgfscope}%
\pgfsys@transformshift{2.162931in}{0.728721in}%
\pgfsys@useobject{currentmarker}{}%
\end{pgfscope}%
\begin{pgfscope}%
\pgfsys@transformshift{2.163145in}{0.686114in}%
\pgfsys@useobject{currentmarker}{}%
\end{pgfscope}%
\begin{pgfscope}%
\pgfsys@transformshift{2.163358in}{0.645799in}%
\pgfsys@useobject{currentmarker}{}%
\end{pgfscope}%
\begin{pgfscope}%
\pgfsys@transformshift{2.163571in}{0.681333in}%
\pgfsys@useobject{currentmarker}{}%
\end{pgfscope}%
\begin{pgfscope}%
\pgfsys@transformshift{2.163784in}{0.718259in}%
\pgfsys@useobject{currentmarker}{}%
\end{pgfscope}%
\begin{pgfscope}%
\pgfsys@transformshift{2.163997in}{0.718614in}%
\pgfsys@useobject{currentmarker}{}%
\end{pgfscope}%
\begin{pgfscope}%
\pgfsys@transformshift{2.164210in}{0.684764in}%
\pgfsys@useobject{currentmarker}{}%
\end{pgfscope}%
\begin{pgfscope}%
\pgfsys@transformshift{2.164422in}{0.624946in}%
\pgfsys@useobject{currentmarker}{}%
\end{pgfscope}%
\begin{pgfscope}%
\pgfsys@transformshift{2.164635in}{0.597614in}%
\pgfsys@useobject{currentmarker}{}%
\end{pgfscope}%
\begin{pgfscope}%
\pgfsys@transformshift{2.164847in}{0.608665in}%
\pgfsys@useobject{currentmarker}{}%
\end{pgfscope}%
\begin{pgfscope}%
\pgfsys@transformshift{2.165058in}{0.641295in}%
\pgfsys@useobject{currentmarker}{}%
\end{pgfscope}%
\begin{pgfscope}%
\pgfsys@transformshift{2.165270in}{0.670957in}%
\pgfsys@useobject{currentmarker}{}%
\end{pgfscope}%
\begin{pgfscope}%
\pgfsys@transformshift{2.165481in}{0.672730in}%
\pgfsys@useobject{currentmarker}{}%
\end{pgfscope}%
\begin{pgfscope}%
\pgfsys@transformshift{2.165693in}{0.647972in}%
\pgfsys@useobject{currentmarker}{}%
\end{pgfscope}%
\begin{pgfscope}%
\pgfsys@transformshift{2.165904in}{0.691811in}%
\pgfsys@useobject{currentmarker}{}%
\end{pgfscope}%
\begin{pgfscope}%
\pgfsys@transformshift{2.166115in}{0.708925in}%
\pgfsys@useobject{currentmarker}{}%
\end{pgfscope}%
\begin{pgfscope}%
\pgfsys@transformshift{2.166325in}{0.696077in}%
\pgfsys@useobject{currentmarker}{}%
\end{pgfscope}%
\begin{pgfscope}%
\pgfsys@transformshift{2.166536in}{0.701931in}%
\pgfsys@useobject{currentmarker}{}%
\end{pgfscope}%
\begin{pgfscope}%
\pgfsys@transformshift{2.166746in}{0.701145in}%
\pgfsys@useobject{currentmarker}{}%
\end{pgfscope}%
\begin{pgfscope}%
\pgfsys@transformshift{2.166956in}{0.679993in}%
\pgfsys@useobject{currentmarker}{}%
\end{pgfscope}%
\begin{pgfscope}%
\pgfsys@transformshift{2.167166in}{0.688442in}%
\pgfsys@useobject{currentmarker}{}%
\end{pgfscope}%
\begin{pgfscope}%
\pgfsys@transformshift{2.167375in}{0.696333in}%
\pgfsys@useobject{currentmarker}{}%
\end{pgfscope}%
\begin{pgfscope}%
\pgfsys@transformshift{2.167585in}{0.644643in}%
\pgfsys@useobject{currentmarker}{}%
\end{pgfscope}%
\begin{pgfscope}%
\pgfsys@transformshift{2.167794in}{0.679035in}%
\pgfsys@useobject{currentmarker}{}%
\end{pgfscope}%
\begin{pgfscope}%
\pgfsys@transformshift{2.168003in}{0.696387in}%
\pgfsys@useobject{currentmarker}{}%
\end{pgfscope}%
\begin{pgfscope}%
\pgfsys@transformshift{2.168212in}{0.749094in}%
\pgfsys@useobject{currentmarker}{}%
\end{pgfscope}%
\begin{pgfscope}%
\pgfsys@transformshift{2.168421in}{0.738825in}%
\pgfsys@useobject{currentmarker}{}%
\end{pgfscope}%
\begin{pgfscope}%
\pgfsys@transformshift{2.168629in}{0.695960in}%
\pgfsys@useobject{currentmarker}{}%
\end{pgfscope}%
\begin{pgfscope}%
\pgfsys@transformshift{2.168838in}{0.687099in}%
\pgfsys@useobject{currentmarker}{}%
\end{pgfscope}%
\begin{pgfscope}%
\pgfsys@transformshift{2.169046in}{0.644372in}%
\pgfsys@useobject{currentmarker}{}%
\end{pgfscope}%
\begin{pgfscope}%
\pgfsys@transformshift{2.169254in}{0.675323in}%
\pgfsys@useobject{currentmarker}{}%
\end{pgfscope}%
\begin{pgfscope}%
\pgfsys@transformshift{2.169461in}{0.672887in}%
\pgfsys@useobject{currentmarker}{}%
\end{pgfscope}%
\begin{pgfscope}%
\pgfsys@transformshift{2.169669in}{0.658194in}%
\pgfsys@useobject{currentmarker}{}%
\end{pgfscope}%
\begin{pgfscope}%
\pgfsys@transformshift{2.169876in}{0.665375in}%
\pgfsys@useobject{currentmarker}{}%
\end{pgfscope}%
\begin{pgfscope}%
\pgfsys@transformshift{2.170083in}{0.713265in}%
\pgfsys@useobject{currentmarker}{}%
\end{pgfscope}%
\begin{pgfscope}%
\pgfsys@transformshift{2.170290in}{0.703282in}%
\pgfsys@useobject{currentmarker}{}%
\end{pgfscope}%
\begin{pgfscope}%
\pgfsys@transformshift{2.170497in}{0.685662in}%
\pgfsys@useobject{currentmarker}{}%
\end{pgfscope}%
\begin{pgfscope}%
\pgfsys@transformshift{2.170704in}{0.731755in}%
\pgfsys@useobject{currentmarker}{}%
\end{pgfscope}%
\begin{pgfscope}%
\pgfsys@transformshift{2.170910in}{0.733342in}%
\pgfsys@useobject{currentmarker}{}%
\end{pgfscope}%
\begin{pgfscope}%
\pgfsys@transformshift{2.171116in}{0.646644in}%
\pgfsys@useobject{currentmarker}{}%
\end{pgfscope}%
\begin{pgfscope}%
\pgfsys@transformshift{2.171322in}{0.625717in}%
\pgfsys@useobject{currentmarker}{}%
\end{pgfscope}%
\begin{pgfscope}%
\pgfsys@transformshift{2.171528in}{0.697848in}%
\pgfsys@useobject{currentmarker}{}%
\end{pgfscope}%
\begin{pgfscope}%
\pgfsys@transformshift{2.171734in}{0.713975in}%
\pgfsys@useobject{currentmarker}{}%
\end{pgfscope}%
\begin{pgfscope}%
\pgfsys@transformshift{2.171939in}{0.677776in}%
\pgfsys@useobject{currentmarker}{}%
\end{pgfscope}%
\begin{pgfscope}%
\pgfsys@transformshift{2.172144in}{0.654346in}%
\pgfsys@useobject{currentmarker}{}%
\end{pgfscope}%
\begin{pgfscope}%
\pgfsys@transformshift{2.172350in}{0.700642in}%
\pgfsys@useobject{currentmarker}{}%
\end{pgfscope}%
\begin{pgfscope}%
\pgfsys@transformshift{2.172554in}{0.685776in}%
\pgfsys@useobject{currentmarker}{}%
\end{pgfscope}%
\begin{pgfscope}%
\pgfsys@transformshift{2.172759in}{0.649506in}%
\pgfsys@useobject{currentmarker}{}%
\end{pgfscope}%
\begin{pgfscope}%
\pgfsys@transformshift{2.172964in}{0.655355in}%
\pgfsys@useobject{currentmarker}{}%
\end{pgfscope}%
\begin{pgfscope}%
\pgfsys@transformshift{2.173168in}{0.711693in}%
\pgfsys@useobject{currentmarker}{}%
\end{pgfscope}%
\begin{pgfscope}%
\pgfsys@transformshift{2.173372in}{0.742795in}%
\pgfsys@useobject{currentmarker}{}%
\end{pgfscope}%
\begin{pgfscope}%
\pgfsys@transformshift{2.173576in}{0.684455in}%
\pgfsys@useobject{currentmarker}{}%
\end{pgfscope}%
\begin{pgfscope}%
\pgfsys@transformshift{2.173780in}{0.711509in}%
\pgfsys@useobject{currentmarker}{}%
\end{pgfscope}%
\begin{pgfscope}%
\pgfsys@transformshift{2.173983in}{0.721011in}%
\pgfsys@useobject{currentmarker}{}%
\end{pgfscope}%
\begin{pgfscope}%
\pgfsys@transformshift{2.174187in}{0.689218in}%
\pgfsys@useobject{currentmarker}{}%
\end{pgfscope}%
\begin{pgfscope}%
\pgfsys@transformshift{2.174390in}{0.629086in}%
\pgfsys@useobject{currentmarker}{}%
\end{pgfscope}%
\begin{pgfscope}%
\pgfsys@transformshift{2.174593in}{0.636479in}%
\pgfsys@useobject{currentmarker}{}%
\end{pgfscope}%
\begin{pgfscope}%
\pgfsys@transformshift{2.174796in}{0.670238in}%
\pgfsys@useobject{currentmarker}{}%
\end{pgfscope}%
\begin{pgfscope}%
\pgfsys@transformshift{2.174999in}{0.680191in}%
\pgfsys@useobject{currentmarker}{}%
\end{pgfscope}%
\begin{pgfscope}%
\pgfsys@transformshift{2.175201in}{0.677697in}%
\pgfsys@useobject{currentmarker}{}%
\end{pgfscope}%
\begin{pgfscope}%
\pgfsys@transformshift{2.175403in}{0.694819in}%
\pgfsys@useobject{currentmarker}{}%
\end{pgfscope}%
\begin{pgfscope}%
\pgfsys@transformshift{2.175605in}{0.713414in}%
\pgfsys@useobject{currentmarker}{}%
\end{pgfscope}%
\begin{pgfscope}%
\pgfsys@transformshift{2.175807in}{0.722881in}%
\pgfsys@useobject{currentmarker}{}%
\end{pgfscope}%
\begin{pgfscope}%
\pgfsys@transformshift{2.176009in}{0.719058in}%
\pgfsys@useobject{currentmarker}{}%
\end{pgfscope}%
\begin{pgfscope}%
\pgfsys@transformshift{2.176211in}{0.696879in}%
\pgfsys@useobject{currentmarker}{}%
\end{pgfscope}%
\begin{pgfscope}%
\pgfsys@transformshift{2.176412in}{0.708081in}%
\pgfsys@useobject{currentmarker}{}%
\end{pgfscope}%
\begin{pgfscope}%
\pgfsys@transformshift{2.176613in}{0.674268in}%
\pgfsys@useobject{currentmarker}{}%
\end{pgfscope}%
\begin{pgfscope}%
\pgfsys@transformshift{2.176814in}{0.652944in}%
\pgfsys@useobject{currentmarker}{}%
\end{pgfscope}%
\begin{pgfscope}%
\pgfsys@transformshift{2.177015in}{0.676330in}%
\pgfsys@useobject{currentmarker}{}%
\end{pgfscope}%
\begin{pgfscope}%
\pgfsys@transformshift{2.177216in}{0.685890in}%
\pgfsys@useobject{currentmarker}{}%
\end{pgfscope}%
\begin{pgfscope}%
\pgfsys@transformshift{2.177416in}{0.658205in}%
\pgfsys@useobject{currentmarker}{}%
\end{pgfscope}%
\begin{pgfscope}%
\pgfsys@transformshift{2.177617in}{0.661454in}%
\pgfsys@useobject{currentmarker}{}%
\end{pgfscope}%
\begin{pgfscope}%
\pgfsys@transformshift{2.177817in}{0.742054in}%
\pgfsys@useobject{currentmarker}{}%
\end{pgfscope}%
\begin{pgfscope}%
\pgfsys@transformshift{2.178017in}{0.713410in}%
\pgfsys@useobject{currentmarker}{}%
\end{pgfscope}%
\begin{pgfscope}%
\pgfsys@transformshift{2.178217in}{0.702302in}%
\pgfsys@useobject{currentmarker}{}%
\end{pgfscope}%
\begin{pgfscope}%
\pgfsys@transformshift{2.178416in}{0.666123in}%
\pgfsys@useobject{currentmarker}{}%
\end{pgfscope}%
\begin{pgfscope}%
\pgfsys@transformshift{2.178616in}{0.665098in}%
\pgfsys@useobject{currentmarker}{}%
\end{pgfscope}%
\begin{pgfscope}%
\pgfsys@transformshift{2.178815in}{0.699389in}%
\pgfsys@useobject{currentmarker}{}%
\end{pgfscope}%
\begin{pgfscope}%
\pgfsys@transformshift{2.179014in}{0.666185in}%
\pgfsys@useobject{currentmarker}{}%
\end{pgfscope}%
\begin{pgfscope}%
\pgfsys@transformshift{2.179213in}{0.645196in}%
\pgfsys@useobject{currentmarker}{}%
\end{pgfscope}%
\begin{pgfscope}%
\pgfsys@transformshift{2.179411in}{0.667116in}%
\pgfsys@useobject{currentmarker}{}%
\end{pgfscope}%
\begin{pgfscope}%
\pgfsys@transformshift{2.179610in}{0.719953in}%
\pgfsys@useobject{currentmarker}{}%
\end{pgfscope}%
\begin{pgfscope}%
\pgfsys@transformshift{2.179808in}{0.694452in}%
\pgfsys@useobject{currentmarker}{}%
\end{pgfscope}%
\begin{pgfscope}%
\pgfsys@transformshift{2.180007in}{0.697109in}%
\pgfsys@useobject{currentmarker}{}%
\end{pgfscope}%
\begin{pgfscope}%
\pgfsys@transformshift{2.180205in}{0.709678in}%
\pgfsys@useobject{currentmarker}{}%
\end{pgfscope}%
\begin{pgfscope}%
\pgfsys@transformshift{2.180402in}{0.677380in}%
\pgfsys@useobject{currentmarker}{}%
\end{pgfscope}%
\begin{pgfscope}%
\pgfsys@transformshift{2.180600in}{0.699566in}%
\pgfsys@useobject{currentmarker}{}%
\end{pgfscope}%
\begin{pgfscope}%
\pgfsys@transformshift{2.180798in}{0.715782in}%
\pgfsys@useobject{currentmarker}{}%
\end{pgfscope}%
\begin{pgfscope}%
\pgfsys@transformshift{2.180995in}{0.713088in}%
\pgfsys@useobject{currentmarker}{}%
\end{pgfscope}%
\begin{pgfscope}%
\pgfsys@transformshift{2.181192in}{0.692449in}%
\pgfsys@useobject{currentmarker}{}%
\end{pgfscope}%
\begin{pgfscope}%
\pgfsys@transformshift{2.181389in}{0.674588in}%
\pgfsys@useobject{currentmarker}{}%
\end{pgfscope}%
\begin{pgfscope}%
\pgfsys@transformshift{2.181586in}{0.677246in}%
\pgfsys@useobject{currentmarker}{}%
\end{pgfscope}%
\begin{pgfscope}%
\pgfsys@transformshift{2.181782in}{0.710036in}%
\pgfsys@useobject{currentmarker}{}%
\end{pgfscope}%
\begin{pgfscope}%
\pgfsys@transformshift{2.181979in}{0.654459in}%
\pgfsys@useobject{currentmarker}{}%
\end{pgfscope}%
\begin{pgfscope}%
\pgfsys@transformshift{2.182175in}{0.657518in}%
\pgfsys@useobject{currentmarker}{}%
\end{pgfscope}%
\begin{pgfscope}%
\pgfsys@transformshift{2.182371in}{0.725195in}%
\pgfsys@useobject{currentmarker}{}%
\end{pgfscope}%
\begin{pgfscope}%
\pgfsys@transformshift{2.182567in}{0.737707in}%
\pgfsys@useobject{currentmarker}{}%
\end{pgfscope}%
\begin{pgfscope}%
\pgfsys@transformshift{2.182763in}{0.700662in}%
\pgfsys@useobject{currentmarker}{}%
\end{pgfscope}%
\begin{pgfscope}%
\pgfsys@transformshift{2.182959in}{0.694466in}%
\pgfsys@useobject{currentmarker}{}%
\end{pgfscope}%
\begin{pgfscope}%
\pgfsys@transformshift{2.183154in}{0.698805in}%
\pgfsys@useobject{currentmarker}{}%
\end{pgfscope}%
\begin{pgfscope}%
\pgfsys@transformshift{2.183349in}{0.672956in}%
\pgfsys@useobject{currentmarker}{}%
\end{pgfscope}%
\begin{pgfscope}%
\pgfsys@transformshift{2.183544in}{0.651904in}%
\pgfsys@useobject{currentmarker}{}%
\end{pgfscope}%
\begin{pgfscope}%
\pgfsys@transformshift{2.183739in}{0.644893in}%
\pgfsys@useobject{currentmarker}{}%
\end{pgfscope}%
\begin{pgfscope}%
\pgfsys@transformshift{2.183934in}{0.664196in}%
\pgfsys@useobject{currentmarker}{}%
\end{pgfscope}%
\begin{pgfscope}%
\pgfsys@transformshift{2.184129in}{0.698353in}%
\pgfsys@useobject{currentmarker}{}%
\end{pgfscope}%
\begin{pgfscope}%
\pgfsys@transformshift{2.184323in}{0.699403in}%
\pgfsys@useobject{currentmarker}{}%
\end{pgfscope}%
\begin{pgfscope}%
\pgfsys@transformshift{2.184517in}{0.711595in}%
\pgfsys@useobject{currentmarker}{}%
\end{pgfscope}%
\begin{pgfscope}%
\pgfsys@transformshift{2.184711in}{0.714044in}%
\pgfsys@useobject{currentmarker}{}%
\end{pgfscope}%
\begin{pgfscope}%
\pgfsys@transformshift{2.184905in}{0.723616in}%
\pgfsys@useobject{currentmarker}{}%
\end{pgfscope}%
\begin{pgfscope}%
\pgfsys@transformshift{2.185099in}{0.697467in}%
\pgfsys@useobject{currentmarker}{}%
\end{pgfscope}%
\begin{pgfscope}%
\pgfsys@transformshift{2.185293in}{0.669222in}%
\pgfsys@useobject{currentmarker}{}%
\end{pgfscope}%
\begin{pgfscope}%
\pgfsys@transformshift{2.185486in}{0.665594in}%
\pgfsys@useobject{currentmarker}{}%
\end{pgfscope}%
\begin{pgfscope}%
\pgfsys@transformshift{2.185679in}{0.634014in}%
\pgfsys@useobject{currentmarker}{}%
\end{pgfscope}%
\begin{pgfscope}%
\pgfsys@transformshift{2.185872in}{0.636374in}%
\pgfsys@useobject{currentmarker}{}%
\end{pgfscope}%
\begin{pgfscope}%
\pgfsys@transformshift{2.186065in}{0.671994in}%
\pgfsys@useobject{currentmarker}{}%
\end{pgfscope}%
\begin{pgfscope}%
\pgfsys@transformshift{2.186258in}{0.701422in}%
\pgfsys@useobject{currentmarker}{}%
\end{pgfscope}%
\begin{pgfscope}%
\pgfsys@transformshift{2.186451in}{0.703415in}%
\pgfsys@useobject{currentmarker}{}%
\end{pgfscope}%
\begin{pgfscope}%
\pgfsys@transformshift{2.186643in}{0.716826in}%
\pgfsys@useobject{currentmarker}{}%
\end{pgfscope}%
\begin{pgfscope}%
\pgfsys@transformshift{2.186835in}{0.740229in}%
\pgfsys@useobject{currentmarker}{}%
\end{pgfscope}%
\begin{pgfscope}%
\pgfsys@transformshift{2.187028in}{0.738678in}%
\pgfsys@useobject{currentmarker}{}%
\end{pgfscope}%
\begin{pgfscope}%
\pgfsys@transformshift{2.187219in}{0.756909in}%
\pgfsys@useobject{currentmarker}{}%
\end{pgfscope}%
\begin{pgfscope}%
\pgfsys@transformshift{2.187411in}{0.736522in}%
\pgfsys@useobject{currentmarker}{}%
\end{pgfscope}%
\begin{pgfscope}%
\pgfsys@transformshift{2.187603in}{0.693028in}%
\pgfsys@useobject{currentmarker}{}%
\end{pgfscope}%
\begin{pgfscope}%
\pgfsys@transformshift{2.187794in}{0.721176in}%
\pgfsys@useobject{currentmarker}{}%
\end{pgfscope}%
\begin{pgfscope}%
\pgfsys@transformshift{2.187986in}{0.713530in}%
\pgfsys@useobject{currentmarker}{}%
\end{pgfscope}%
\begin{pgfscope}%
\pgfsys@transformshift{2.188177in}{0.684664in}%
\pgfsys@useobject{currentmarker}{}%
\end{pgfscope}%
\begin{pgfscope}%
\pgfsys@transformshift{2.188368in}{0.651424in}%
\pgfsys@useobject{currentmarker}{}%
\end{pgfscope}%
\begin{pgfscope}%
\pgfsys@transformshift{2.188558in}{0.666047in}%
\pgfsys@useobject{currentmarker}{}%
\end{pgfscope}%
\begin{pgfscope}%
\pgfsys@transformshift{2.188749in}{0.708207in}%
\pgfsys@useobject{currentmarker}{}%
\end{pgfscope}%
\begin{pgfscope}%
\pgfsys@transformshift{2.188939in}{0.690829in}%
\pgfsys@useobject{currentmarker}{}%
\end{pgfscope}%
\begin{pgfscope}%
\pgfsys@transformshift{2.189130in}{0.675269in}%
\pgfsys@useobject{currentmarker}{}%
\end{pgfscope}%
\begin{pgfscope}%
\pgfsys@transformshift{2.189320in}{0.674532in}%
\pgfsys@useobject{currentmarker}{}%
\end{pgfscope}%
\begin{pgfscope}%
\pgfsys@transformshift{2.189510in}{0.696183in}%
\pgfsys@useobject{currentmarker}{}%
\end{pgfscope}%
\begin{pgfscope}%
\pgfsys@transformshift{2.189700in}{0.702756in}%
\pgfsys@useobject{currentmarker}{}%
\end{pgfscope}%
\begin{pgfscope}%
\pgfsys@transformshift{2.189889in}{0.713369in}%
\pgfsys@useobject{currentmarker}{}%
\end{pgfscope}%
\begin{pgfscope}%
\pgfsys@transformshift{2.190079in}{0.690744in}%
\pgfsys@useobject{currentmarker}{}%
\end{pgfscope}%
\begin{pgfscope}%
\pgfsys@transformshift{2.190268in}{0.690628in}%
\pgfsys@useobject{currentmarker}{}%
\end{pgfscope}%
\begin{pgfscope}%
\pgfsys@transformshift{2.190457in}{0.649342in}%
\pgfsys@useobject{currentmarker}{}%
\end{pgfscope}%
\begin{pgfscope}%
\pgfsys@transformshift{2.190646in}{0.578642in}%
\pgfsys@useobject{currentmarker}{}%
\end{pgfscope}%
\begin{pgfscope}%
\pgfsys@transformshift{2.190835in}{0.649178in}%
\pgfsys@useobject{currentmarker}{}%
\end{pgfscope}%
\begin{pgfscope}%
\pgfsys@transformshift{2.191024in}{0.694780in}%
\pgfsys@useobject{currentmarker}{}%
\end{pgfscope}%
\begin{pgfscope}%
\pgfsys@transformshift{2.191213in}{0.689650in}%
\pgfsys@useobject{currentmarker}{}%
\end{pgfscope}%
\begin{pgfscope}%
\pgfsys@transformshift{2.191401in}{0.670119in}%
\pgfsys@useobject{currentmarker}{}%
\end{pgfscope}%
\begin{pgfscope}%
\pgfsys@transformshift{2.191589in}{0.642852in}%
\pgfsys@useobject{currentmarker}{}%
\end{pgfscope}%
\begin{pgfscope}%
\pgfsys@transformshift{2.191777in}{0.665341in}%
\pgfsys@useobject{currentmarker}{}%
\end{pgfscope}%
\begin{pgfscope}%
\pgfsys@transformshift{2.191965in}{0.702653in}%
\pgfsys@useobject{currentmarker}{}%
\end{pgfscope}%
\begin{pgfscope}%
\pgfsys@transformshift{2.192153in}{0.668017in}%
\pgfsys@useobject{currentmarker}{}%
\end{pgfscope}%
\begin{pgfscope}%
\pgfsys@transformshift{2.192340in}{0.711816in}%
\pgfsys@useobject{currentmarker}{}%
\end{pgfscope}%
\begin{pgfscope}%
\pgfsys@transformshift{2.192528in}{0.741393in}%
\pgfsys@useobject{currentmarker}{}%
\end{pgfscope}%
\begin{pgfscope}%
\pgfsys@transformshift{2.192715in}{0.700353in}%
\pgfsys@useobject{currentmarker}{}%
\end{pgfscope}%
\begin{pgfscope}%
\pgfsys@transformshift{2.192902in}{0.688983in}%
\pgfsys@useobject{currentmarker}{}%
\end{pgfscope}%
\begin{pgfscope}%
\pgfsys@transformshift{2.193089in}{0.710088in}%
\pgfsys@useobject{currentmarker}{}%
\end{pgfscope}%
\begin{pgfscope}%
\pgfsys@transformshift{2.193276in}{0.710933in}%
\pgfsys@useobject{currentmarker}{}%
\end{pgfscope}%
\begin{pgfscope}%
\pgfsys@transformshift{2.193463in}{0.683686in}%
\pgfsys@useobject{currentmarker}{}%
\end{pgfscope}%
\begin{pgfscope}%
\pgfsys@transformshift{2.193649in}{0.692132in}%
\pgfsys@useobject{currentmarker}{}%
\end{pgfscope}%
\begin{pgfscope}%
\pgfsys@transformshift{2.193836in}{0.707899in}%
\pgfsys@useobject{currentmarker}{}%
\end{pgfscope}%
\begin{pgfscope}%
\pgfsys@transformshift{2.194022in}{0.689390in}%
\pgfsys@useobject{currentmarker}{}%
\end{pgfscope}%
\begin{pgfscope}%
\pgfsys@transformshift{2.194208in}{0.686547in}%
\pgfsys@useobject{currentmarker}{}%
\end{pgfscope}%
\begin{pgfscope}%
\pgfsys@transformshift{2.194394in}{0.688339in}%
\pgfsys@useobject{currentmarker}{}%
\end{pgfscope}%
\begin{pgfscope}%
\pgfsys@transformshift{2.194580in}{0.706259in}%
\pgfsys@useobject{currentmarker}{}%
\end{pgfscope}%
\begin{pgfscope}%
\pgfsys@transformshift{2.194765in}{0.749049in}%
\pgfsys@useobject{currentmarker}{}%
\end{pgfscope}%
\begin{pgfscope}%
\pgfsys@transformshift{2.194951in}{0.730673in}%
\pgfsys@useobject{currentmarker}{}%
\end{pgfscope}%
\begin{pgfscope}%
\pgfsys@transformshift{2.195136in}{0.700597in}%
\pgfsys@useobject{currentmarker}{}%
\end{pgfscope}%
\begin{pgfscope}%
\pgfsys@transformshift{2.195321in}{0.708377in}%
\pgfsys@useobject{currentmarker}{}%
\end{pgfscope}%
\begin{pgfscope}%
\pgfsys@transformshift{2.195506in}{0.707488in}%
\pgfsys@useobject{currentmarker}{}%
\end{pgfscope}%
\begin{pgfscope}%
\pgfsys@transformshift{2.195691in}{0.620732in}%
\pgfsys@useobject{currentmarker}{}%
\end{pgfscope}%
\begin{pgfscope}%
\pgfsys@transformshift{2.195875in}{0.676553in}%
\pgfsys@useobject{currentmarker}{}%
\end{pgfscope}%
\begin{pgfscope}%
\pgfsys@transformshift{2.196060in}{0.682485in}%
\pgfsys@useobject{currentmarker}{}%
\end{pgfscope}%
\begin{pgfscope}%
\pgfsys@transformshift{2.196244in}{0.711136in}%
\pgfsys@useobject{currentmarker}{}%
\end{pgfscope}%
\begin{pgfscope}%
\pgfsys@transformshift{2.196429in}{0.693215in}%
\pgfsys@useobject{currentmarker}{}%
\end{pgfscope}%
\begin{pgfscope}%
\pgfsys@transformshift{2.196613in}{0.688525in}%
\pgfsys@useobject{currentmarker}{}%
\end{pgfscope}%
\begin{pgfscope}%
\pgfsys@transformshift{2.196797in}{0.700968in}%
\pgfsys@useobject{currentmarker}{}%
\end{pgfscope}%
\begin{pgfscope}%
\pgfsys@transformshift{2.196980in}{0.682053in}%
\pgfsys@useobject{currentmarker}{}%
\end{pgfscope}%
\begin{pgfscope}%
\pgfsys@transformshift{2.197164in}{0.749821in}%
\pgfsys@useobject{currentmarker}{}%
\end{pgfscope}%
\begin{pgfscope}%
\pgfsys@transformshift{2.197347in}{0.735825in}%
\pgfsys@useobject{currentmarker}{}%
\end{pgfscope}%
\begin{pgfscope}%
\pgfsys@transformshift{2.197531in}{0.719072in}%
\pgfsys@useobject{currentmarker}{}%
\end{pgfscope}%
\begin{pgfscope}%
\pgfsys@transformshift{2.197714in}{0.712546in}%
\pgfsys@useobject{currentmarker}{}%
\end{pgfscope}%
\begin{pgfscope}%
\pgfsys@transformshift{2.197897in}{0.691153in}%
\pgfsys@useobject{currentmarker}{}%
\end{pgfscope}%
\begin{pgfscope}%
\pgfsys@transformshift{2.198080in}{0.659651in}%
\pgfsys@useobject{currentmarker}{}%
\end{pgfscope}%
\begin{pgfscope}%
\pgfsys@transformshift{2.198262in}{0.696666in}%
\pgfsys@useobject{currentmarker}{}%
\end{pgfscope}%
\begin{pgfscope}%
\pgfsys@transformshift{2.198445in}{0.705294in}%
\pgfsys@useobject{currentmarker}{}%
\end{pgfscope}%
\begin{pgfscope}%
\pgfsys@transformshift{2.198627in}{0.696474in}%
\pgfsys@useobject{currentmarker}{}%
\end{pgfscope}%
\begin{pgfscope}%
\pgfsys@transformshift{2.198810in}{0.680355in}%
\pgfsys@useobject{currentmarker}{}%
\end{pgfscope}%
\begin{pgfscope}%
\pgfsys@transformshift{2.198992in}{0.681122in}%
\pgfsys@useobject{currentmarker}{}%
\end{pgfscope}%
\begin{pgfscope}%
\pgfsys@transformshift{2.199174in}{0.691101in}%
\pgfsys@useobject{currentmarker}{}%
\end{pgfscope}%
\begin{pgfscope}%
\pgfsys@transformshift{2.199356in}{0.690425in}%
\pgfsys@useobject{currentmarker}{}%
\end{pgfscope}%
\begin{pgfscope}%
\pgfsys@transformshift{2.199537in}{0.699674in}%
\pgfsys@useobject{currentmarker}{}%
\end{pgfscope}%
\begin{pgfscope}%
\pgfsys@transformshift{2.199719in}{0.692043in}%
\pgfsys@useobject{currentmarker}{}%
\end{pgfscope}%
\begin{pgfscope}%
\pgfsys@transformshift{2.199900in}{0.675038in}%
\pgfsys@useobject{currentmarker}{}%
\end{pgfscope}%
\begin{pgfscope}%
\pgfsys@transformshift{2.200081in}{0.677072in}%
\pgfsys@useobject{currentmarker}{}%
\end{pgfscope}%
\begin{pgfscope}%
\pgfsys@transformshift{2.200263in}{0.700637in}%
\pgfsys@useobject{currentmarker}{}%
\end{pgfscope}%
\begin{pgfscope}%
\pgfsys@transformshift{2.200444in}{0.700685in}%
\pgfsys@useobject{currentmarker}{}%
\end{pgfscope}%
\begin{pgfscope}%
\pgfsys@transformshift{2.200624in}{0.689386in}%
\pgfsys@useobject{currentmarker}{}%
\end{pgfscope}%
\begin{pgfscope}%
\pgfsys@transformshift{2.200805in}{0.651971in}%
\pgfsys@useobject{currentmarker}{}%
\end{pgfscope}%
\begin{pgfscope}%
\pgfsys@transformshift{2.200986in}{0.675351in}%
\pgfsys@useobject{currentmarker}{}%
\end{pgfscope}%
\begin{pgfscope}%
\pgfsys@transformshift{2.201166in}{0.702693in}%
\pgfsys@useobject{currentmarker}{}%
\end{pgfscope}%
\begin{pgfscope}%
\pgfsys@transformshift{2.201346in}{0.638826in}%
\pgfsys@useobject{currentmarker}{}%
\end{pgfscope}%
\begin{pgfscope}%
\pgfsys@transformshift{2.201526in}{0.614525in}%
\pgfsys@useobject{currentmarker}{}%
\end{pgfscope}%
\begin{pgfscope}%
\pgfsys@transformshift{2.201706in}{0.627732in}%
\pgfsys@useobject{currentmarker}{}%
\end{pgfscope}%
\begin{pgfscope}%
\pgfsys@transformshift{2.201886in}{0.642268in}%
\pgfsys@useobject{currentmarker}{}%
\end{pgfscope}%
\begin{pgfscope}%
\pgfsys@transformshift{2.202066in}{0.672782in}%
\pgfsys@useobject{currentmarker}{}%
\end{pgfscope}%
\begin{pgfscope}%
\pgfsys@transformshift{2.202245in}{0.637925in}%
\pgfsys@useobject{currentmarker}{}%
\end{pgfscope}%
\begin{pgfscope}%
\pgfsys@transformshift{2.202424in}{0.649807in}%
\pgfsys@useobject{currentmarker}{}%
\end{pgfscope}%
\begin{pgfscope}%
\pgfsys@transformshift{2.202604in}{0.665030in}%
\pgfsys@useobject{currentmarker}{}%
\end{pgfscope}%
\begin{pgfscope}%
\pgfsys@transformshift{2.202783in}{0.682822in}%
\pgfsys@useobject{currentmarker}{}%
\end{pgfscope}%
\begin{pgfscope}%
\pgfsys@transformshift{2.202962in}{0.695607in}%
\pgfsys@useobject{currentmarker}{}%
\end{pgfscope}%
\begin{pgfscope}%
\pgfsys@transformshift{2.203140in}{0.678478in}%
\pgfsys@useobject{currentmarker}{}%
\end{pgfscope}%
\begin{pgfscope}%
\pgfsys@transformshift{2.203319in}{0.650414in}%
\pgfsys@useobject{currentmarker}{}%
\end{pgfscope}%
\begin{pgfscope}%
\pgfsys@transformshift{2.203498in}{0.676657in}%
\pgfsys@useobject{currentmarker}{}%
\end{pgfscope}%
\begin{pgfscope}%
\pgfsys@transformshift{2.203676in}{0.735882in}%
\pgfsys@useobject{currentmarker}{}%
\end{pgfscope}%
\begin{pgfscope}%
\pgfsys@transformshift{2.203854in}{0.730899in}%
\pgfsys@useobject{currentmarker}{}%
\end{pgfscope}%
\begin{pgfscope}%
\pgfsys@transformshift{2.204032in}{0.711051in}%
\pgfsys@useobject{currentmarker}{}%
\end{pgfscope}%
\begin{pgfscope}%
\pgfsys@transformshift{2.204210in}{0.647065in}%
\pgfsys@useobject{currentmarker}{}%
\end{pgfscope}%
\begin{pgfscope}%
\pgfsys@transformshift{2.204388in}{0.657818in}%
\pgfsys@useobject{currentmarker}{}%
\end{pgfscope}%
\begin{pgfscope}%
\pgfsys@transformshift{2.204566in}{0.700146in}%
\pgfsys@useobject{currentmarker}{}%
\end{pgfscope}%
\begin{pgfscope}%
\pgfsys@transformshift{2.204743in}{0.714005in}%
\pgfsys@useobject{currentmarker}{}%
\end{pgfscope}%
\begin{pgfscope}%
\pgfsys@transformshift{2.204921in}{0.622037in}%
\pgfsys@useobject{currentmarker}{}%
\end{pgfscope}%
\begin{pgfscope}%
\pgfsys@transformshift{2.205098in}{0.724855in}%
\pgfsys@useobject{currentmarker}{}%
\end{pgfscope}%
\begin{pgfscope}%
\pgfsys@transformshift{2.205275in}{0.710749in}%
\pgfsys@useobject{currentmarker}{}%
\end{pgfscope}%
\begin{pgfscope}%
\pgfsys@transformshift{2.205452in}{0.691575in}%
\pgfsys@useobject{currentmarker}{}%
\end{pgfscope}%
\begin{pgfscope}%
\pgfsys@transformshift{2.205629in}{0.703546in}%
\pgfsys@useobject{currentmarker}{}%
\end{pgfscope}%
\begin{pgfscope}%
\pgfsys@transformshift{2.205805in}{0.680496in}%
\pgfsys@useobject{currentmarker}{}%
\end{pgfscope}%
\begin{pgfscope}%
\pgfsys@transformshift{2.205982in}{0.658757in}%
\pgfsys@useobject{currentmarker}{}%
\end{pgfscope}%
\begin{pgfscope}%
\pgfsys@transformshift{2.206158in}{0.716500in}%
\pgfsys@useobject{currentmarker}{}%
\end{pgfscope}%
\begin{pgfscope}%
\pgfsys@transformshift{2.206335in}{0.715837in}%
\pgfsys@useobject{currentmarker}{}%
\end{pgfscope}%
\begin{pgfscope}%
\pgfsys@transformshift{2.206511in}{0.676328in}%
\pgfsys@useobject{currentmarker}{}%
\end{pgfscope}%
\begin{pgfscope}%
\pgfsys@transformshift{2.206687in}{0.652973in}%
\pgfsys@useobject{currentmarker}{}%
\end{pgfscope}%
\begin{pgfscope}%
\pgfsys@transformshift{2.206863in}{0.684971in}%
\pgfsys@useobject{currentmarker}{}%
\end{pgfscope}%
\begin{pgfscope}%
\pgfsys@transformshift{2.207038in}{0.684302in}%
\pgfsys@useobject{currentmarker}{}%
\end{pgfscope}%
\begin{pgfscope}%
\pgfsys@transformshift{2.207214in}{0.666102in}%
\pgfsys@useobject{currentmarker}{}%
\end{pgfscope}%
\begin{pgfscope}%
\pgfsys@transformshift{2.207389in}{0.718138in}%
\pgfsys@useobject{currentmarker}{}%
\end{pgfscope}%
\begin{pgfscope}%
\pgfsys@transformshift{2.207565in}{0.721955in}%
\pgfsys@useobject{currentmarker}{}%
\end{pgfscope}%
\begin{pgfscope}%
\pgfsys@transformshift{2.207740in}{0.648573in}%
\pgfsys@useobject{currentmarker}{}%
\end{pgfscope}%
\begin{pgfscope}%
\pgfsys@transformshift{2.207915in}{0.639416in}%
\pgfsys@useobject{currentmarker}{}%
\end{pgfscope}%
\begin{pgfscope}%
\pgfsys@transformshift{2.208090in}{0.640161in}%
\pgfsys@useobject{currentmarker}{}%
\end{pgfscope}%
\begin{pgfscope}%
\pgfsys@transformshift{2.208264in}{0.630922in}%
\pgfsys@useobject{currentmarker}{}%
\end{pgfscope}%
\begin{pgfscope}%
\pgfsys@transformshift{2.208439in}{0.668975in}%
\pgfsys@useobject{currentmarker}{}%
\end{pgfscope}%
\begin{pgfscope}%
\pgfsys@transformshift{2.208614in}{0.609224in}%
\pgfsys@useobject{currentmarker}{}%
\end{pgfscope}%
\begin{pgfscope}%
\pgfsys@transformshift{2.208788in}{0.649926in}%
\pgfsys@useobject{currentmarker}{}%
\end{pgfscope}%
\begin{pgfscope}%
\pgfsys@transformshift{2.208962in}{0.660497in}%
\pgfsys@useobject{currentmarker}{}%
\end{pgfscope}%
\begin{pgfscope}%
\pgfsys@transformshift{2.209136in}{0.685356in}%
\pgfsys@useobject{currentmarker}{}%
\end{pgfscope}%
\begin{pgfscope}%
\pgfsys@transformshift{2.209310in}{0.701833in}%
\pgfsys@useobject{currentmarker}{}%
\end{pgfscope}%
\begin{pgfscope}%
\pgfsys@transformshift{2.209484in}{0.696108in}%
\pgfsys@useobject{currentmarker}{}%
\end{pgfscope}%
\begin{pgfscope}%
\pgfsys@transformshift{2.209658in}{0.694036in}%
\pgfsys@useobject{currentmarker}{}%
\end{pgfscope}%
\begin{pgfscope}%
\pgfsys@transformshift{2.209831in}{0.665283in}%
\pgfsys@useobject{currentmarker}{}%
\end{pgfscope}%
\begin{pgfscope}%
\pgfsys@transformshift{2.210005in}{0.686610in}%
\pgfsys@useobject{currentmarker}{}%
\end{pgfscope}%
\begin{pgfscope}%
\pgfsys@transformshift{2.210178in}{0.659683in}%
\pgfsys@useobject{currentmarker}{}%
\end{pgfscope}%
\begin{pgfscope}%
\pgfsys@transformshift{2.210351in}{0.633098in}%
\pgfsys@useobject{currentmarker}{}%
\end{pgfscope}%
\begin{pgfscope}%
\pgfsys@transformshift{2.210524in}{0.653570in}%
\pgfsys@useobject{currentmarker}{}%
\end{pgfscope}%
\begin{pgfscope}%
\pgfsys@transformshift{2.210697in}{0.657469in}%
\pgfsys@useobject{currentmarker}{}%
\end{pgfscope}%
\begin{pgfscope}%
\pgfsys@transformshift{2.210870in}{0.651103in}%
\pgfsys@useobject{currentmarker}{}%
\end{pgfscope}%
\begin{pgfscope}%
\pgfsys@transformshift{2.211042in}{0.661211in}%
\pgfsys@useobject{currentmarker}{}%
\end{pgfscope}%
\begin{pgfscope}%
\pgfsys@transformshift{2.211215in}{0.699584in}%
\pgfsys@useobject{currentmarker}{}%
\end{pgfscope}%
\begin{pgfscope}%
\pgfsys@transformshift{2.211387in}{0.694151in}%
\pgfsys@useobject{currentmarker}{}%
\end{pgfscope}%
\begin{pgfscope}%
\pgfsys@transformshift{2.211559in}{0.650357in}%
\pgfsys@useobject{currentmarker}{}%
\end{pgfscope}%
\begin{pgfscope}%
\pgfsys@transformshift{2.211731in}{0.647085in}%
\pgfsys@useobject{currentmarker}{}%
\end{pgfscope}%
\begin{pgfscope}%
\pgfsys@transformshift{2.211903in}{0.638594in}%
\pgfsys@useobject{currentmarker}{}%
\end{pgfscope}%
\begin{pgfscope}%
\pgfsys@transformshift{2.212075in}{0.600482in}%
\pgfsys@useobject{currentmarker}{}%
\end{pgfscope}%
\begin{pgfscope}%
\pgfsys@transformshift{2.212247in}{0.672459in}%
\pgfsys@useobject{currentmarker}{}%
\end{pgfscope}%
\begin{pgfscope}%
\pgfsys@transformshift{2.212418in}{0.688343in}%
\pgfsys@useobject{currentmarker}{}%
\end{pgfscope}%
\begin{pgfscope}%
\pgfsys@transformshift{2.212590in}{0.674747in}%
\pgfsys@useobject{currentmarker}{}%
\end{pgfscope}%
\begin{pgfscope}%
\pgfsys@transformshift{2.212761in}{0.660007in}%
\pgfsys@useobject{currentmarker}{}%
\end{pgfscope}%
\begin{pgfscope}%
\pgfsys@transformshift{2.212932in}{0.705731in}%
\pgfsys@useobject{currentmarker}{}%
\end{pgfscope}%
\begin{pgfscope}%
\pgfsys@transformshift{2.213103in}{0.704263in}%
\pgfsys@useobject{currentmarker}{}%
\end{pgfscope}%
\begin{pgfscope}%
\pgfsys@transformshift{2.213274in}{0.681712in}%
\pgfsys@useobject{currentmarker}{}%
\end{pgfscope}%
\begin{pgfscope}%
\pgfsys@transformshift{2.213445in}{0.712115in}%
\pgfsys@useobject{currentmarker}{}%
\end{pgfscope}%
\begin{pgfscope}%
\pgfsys@transformshift{2.213616in}{0.682670in}%
\pgfsys@useobject{currentmarker}{}%
\end{pgfscope}%
\begin{pgfscope}%
\pgfsys@transformshift{2.213786in}{0.687291in}%
\pgfsys@useobject{currentmarker}{}%
\end{pgfscope}%
\begin{pgfscope}%
\pgfsys@transformshift{2.213957in}{0.685866in}%
\pgfsys@useobject{currentmarker}{}%
\end{pgfscope}%
\begin{pgfscope}%
\pgfsys@transformshift{2.214127in}{0.692897in}%
\pgfsys@useobject{currentmarker}{}%
\end{pgfscope}%
\begin{pgfscope}%
\pgfsys@transformshift{2.214297in}{0.648852in}%
\pgfsys@useobject{currentmarker}{}%
\end{pgfscope}%
\begin{pgfscope}%
\pgfsys@transformshift{2.214467in}{0.658796in}%
\pgfsys@useobject{currentmarker}{}%
\end{pgfscope}%
\begin{pgfscope}%
\pgfsys@transformshift{2.214637in}{0.724582in}%
\pgfsys@useobject{currentmarker}{}%
\end{pgfscope}%
\begin{pgfscope}%
\pgfsys@transformshift{2.214807in}{0.700298in}%
\pgfsys@useobject{currentmarker}{}%
\end{pgfscope}%
\begin{pgfscope}%
\pgfsys@transformshift{2.214976in}{0.669909in}%
\pgfsys@useobject{currentmarker}{}%
\end{pgfscope}%
\begin{pgfscope}%
\pgfsys@transformshift{2.215146in}{0.667744in}%
\pgfsys@useobject{currentmarker}{}%
\end{pgfscope}%
\begin{pgfscope}%
\pgfsys@transformshift{2.215315in}{0.686014in}%
\pgfsys@useobject{currentmarker}{}%
\end{pgfscope}%
\begin{pgfscope}%
\pgfsys@transformshift{2.215484in}{0.674212in}%
\pgfsys@useobject{currentmarker}{}%
\end{pgfscope}%
\begin{pgfscope}%
\pgfsys@transformshift{2.215653in}{0.723279in}%
\pgfsys@useobject{currentmarker}{}%
\end{pgfscope}%
\begin{pgfscope}%
\pgfsys@transformshift{2.215822in}{0.727257in}%
\pgfsys@useobject{currentmarker}{}%
\end{pgfscope}%
\begin{pgfscope}%
\pgfsys@transformshift{2.215991in}{0.684906in}%
\pgfsys@useobject{currentmarker}{}%
\end{pgfscope}%
\begin{pgfscope}%
\pgfsys@transformshift{2.216160in}{0.657604in}%
\pgfsys@useobject{currentmarker}{}%
\end{pgfscope}%
\begin{pgfscope}%
\pgfsys@transformshift{2.216328in}{0.691976in}%
\pgfsys@useobject{currentmarker}{}%
\end{pgfscope}%
\begin{pgfscope}%
\pgfsys@transformshift{2.216497in}{0.679774in}%
\pgfsys@useobject{currentmarker}{}%
\end{pgfscope}%
\begin{pgfscope}%
\pgfsys@transformshift{2.216665in}{0.720841in}%
\pgfsys@useobject{currentmarker}{}%
\end{pgfscope}%
\begin{pgfscope}%
\pgfsys@transformshift{2.216833in}{0.711150in}%
\pgfsys@useobject{currentmarker}{}%
\end{pgfscope}%
\begin{pgfscope}%
\pgfsys@transformshift{2.217002in}{0.684779in}%
\pgfsys@useobject{currentmarker}{}%
\end{pgfscope}%
\begin{pgfscope}%
\pgfsys@transformshift{2.217170in}{0.668456in}%
\pgfsys@useobject{currentmarker}{}%
\end{pgfscope}%
\begin{pgfscope}%
\pgfsys@transformshift{2.217337in}{0.689799in}%
\pgfsys@useobject{currentmarker}{}%
\end{pgfscope}%
\begin{pgfscope}%
\pgfsys@transformshift{2.217505in}{0.705746in}%
\pgfsys@useobject{currentmarker}{}%
\end{pgfscope}%
\begin{pgfscope}%
\pgfsys@transformshift{2.217673in}{0.699868in}%
\pgfsys@useobject{currentmarker}{}%
\end{pgfscope}%
\begin{pgfscope}%
\pgfsys@transformshift{2.217840in}{0.740808in}%
\pgfsys@useobject{currentmarker}{}%
\end{pgfscope}%
\begin{pgfscope}%
\pgfsys@transformshift{2.218007in}{0.696147in}%
\pgfsys@useobject{currentmarker}{}%
\end{pgfscope}%
\begin{pgfscope}%
\pgfsys@transformshift{2.218175in}{0.661122in}%
\pgfsys@useobject{currentmarker}{}%
\end{pgfscope}%
\begin{pgfscope}%
\pgfsys@transformshift{2.218342in}{0.646083in}%
\pgfsys@useobject{currentmarker}{}%
\end{pgfscope}%
\begin{pgfscope}%
\pgfsys@transformshift{2.218509in}{0.585375in}%
\pgfsys@useobject{currentmarker}{}%
\end{pgfscope}%
\begin{pgfscope}%
\pgfsys@transformshift{2.218676in}{0.671025in}%
\pgfsys@useobject{currentmarker}{}%
\end{pgfscope}%
\begin{pgfscope}%
\pgfsys@transformshift{2.218842in}{0.657438in}%
\pgfsys@useobject{currentmarker}{}%
\end{pgfscope}%
\begin{pgfscope}%
\pgfsys@transformshift{2.219009in}{0.696004in}%
\pgfsys@useobject{currentmarker}{}%
\end{pgfscope}%
\begin{pgfscope}%
\pgfsys@transformshift{2.219175in}{0.678122in}%
\pgfsys@useobject{currentmarker}{}%
\end{pgfscope}%
\begin{pgfscope}%
\pgfsys@transformshift{2.219342in}{0.717930in}%
\pgfsys@useobject{currentmarker}{}%
\end{pgfscope}%
\begin{pgfscope}%
\pgfsys@transformshift{2.219508in}{0.751045in}%
\pgfsys@useobject{currentmarker}{}%
\end{pgfscope}%
\begin{pgfscope}%
\pgfsys@transformshift{2.219674in}{0.661198in}%
\pgfsys@useobject{currentmarker}{}%
\end{pgfscope}%
\begin{pgfscope}%
\pgfsys@transformshift{2.219840in}{0.678201in}%
\pgfsys@useobject{currentmarker}{}%
\end{pgfscope}%
\begin{pgfscope}%
\pgfsys@transformshift{2.220006in}{0.716313in}%
\pgfsys@useobject{currentmarker}{}%
\end{pgfscope}%
\begin{pgfscope}%
\pgfsys@transformshift{2.220172in}{0.729731in}%
\pgfsys@useobject{currentmarker}{}%
\end{pgfscope}%
\begin{pgfscope}%
\pgfsys@transformshift{2.220337in}{0.698281in}%
\pgfsys@useobject{currentmarker}{}%
\end{pgfscope}%
\begin{pgfscope}%
\pgfsys@transformshift{2.220503in}{0.651955in}%
\pgfsys@useobject{currentmarker}{}%
\end{pgfscope}%
\begin{pgfscope}%
\pgfsys@transformshift{2.220668in}{0.658593in}%
\pgfsys@useobject{currentmarker}{}%
\end{pgfscope}%
\begin{pgfscope}%
\pgfsys@transformshift{2.220833in}{0.662169in}%
\pgfsys@useobject{currentmarker}{}%
\end{pgfscope}%
\begin{pgfscope}%
\pgfsys@transformshift{2.220998in}{0.661269in}%
\pgfsys@useobject{currentmarker}{}%
\end{pgfscope}%
\begin{pgfscope}%
\pgfsys@transformshift{2.221163in}{0.691774in}%
\pgfsys@useobject{currentmarker}{}%
\end{pgfscope}%
\begin{pgfscope}%
\pgfsys@transformshift{2.221328in}{0.684020in}%
\pgfsys@useobject{currentmarker}{}%
\end{pgfscope}%
\begin{pgfscope}%
\pgfsys@transformshift{2.221493in}{0.675992in}%
\pgfsys@useobject{currentmarker}{}%
\end{pgfscope}%
\begin{pgfscope}%
\pgfsys@transformshift{2.221658in}{0.737301in}%
\pgfsys@useobject{currentmarker}{}%
\end{pgfscope}%
\begin{pgfscope}%
\pgfsys@transformshift{2.221822in}{0.695076in}%
\pgfsys@useobject{currentmarker}{}%
\end{pgfscope}%
\begin{pgfscope}%
\pgfsys@transformshift{2.221987in}{0.644947in}%
\pgfsys@useobject{currentmarker}{}%
\end{pgfscope}%
\begin{pgfscope}%
\pgfsys@transformshift{2.222151in}{0.661242in}%
\pgfsys@useobject{currentmarker}{}%
\end{pgfscope}%
\begin{pgfscope}%
\pgfsys@transformshift{2.222315in}{0.657020in}%
\pgfsys@useobject{currentmarker}{}%
\end{pgfscope}%
\begin{pgfscope}%
\pgfsys@transformshift{2.222479in}{0.659632in}%
\pgfsys@useobject{currentmarker}{}%
\end{pgfscope}%
\begin{pgfscope}%
\pgfsys@transformshift{2.222643in}{0.716656in}%
\pgfsys@useobject{currentmarker}{}%
\end{pgfscope}%
\begin{pgfscope}%
\pgfsys@transformshift{2.222807in}{0.726925in}%
\pgfsys@useobject{currentmarker}{}%
\end{pgfscope}%
\begin{pgfscope}%
\pgfsys@transformshift{2.222971in}{0.715557in}%
\pgfsys@useobject{currentmarker}{}%
\end{pgfscope}%
\begin{pgfscope}%
\pgfsys@transformshift{2.223134in}{0.701332in}%
\pgfsys@useobject{currentmarker}{}%
\end{pgfscope}%
\begin{pgfscope}%
\pgfsys@transformshift{2.223298in}{0.724275in}%
\pgfsys@useobject{currentmarker}{}%
\end{pgfscope}%
\begin{pgfscope}%
\pgfsys@transformshift{2.223461in}{0.714163in}%
\pgfsys@useobject{currentmarker}{}%
\end{pgfscope}%
\begin{pgfscope}%
\pgfsys@transformshift{2.223624in}{0.657777in}%
\pgfsys@useobject{currentmarker}{}%
\end{pgfscope}%
\begin{pgfscope}%
\pgfsys@transformshift{2.223787in}{0.686653in}%
\pgfsys@useobject{currentmarker}{}%
\end{pgfscope}%
\begin{pgfscope}%
\pgfsys@transformshift{2.223950in}{0.678309in}%
\pgfsys@useobject{currentmarker}{}%
\end{pgfscope}%
\begin{pgfscope}%
\pgfsys@transformshift{2.224113in}{0.702459in}%
\pgfsys@useobject{currentmarker}{}%
\end{pgfscope}%
\begin{pgfscope}%
\pgfsys@transformshift{2.224276in}{0.738780in}%
\pgfsys@useobject{currentmarker}{}%
\end{pgfscope}%
\begin{pgfscope}%
\pgfsys@transformshift{2.224438in}{0.757536in}%
\pgfsys@useobject{currentmarker}{}%
\end{pgfscope}%
\begin{pgfscope}%
\pgfsys@transformshift{2.224601in}{0.705056in}%
\pgfsys@useobject{currentmarker}{}%
\end{pgfscope}%
\begin{pgfscope}%
\pgfsys@transformshift{2.224763in}{0.682339in}%
\pgfsys@useobject{currentmarker}{}%
\end{pgfscope}%
\begin{pgfscope}%
\pgfsys@transformshift{2.224925in}{0.684487in}%
\pgfsys@useobject{currentmarker}{}%
\end{pgfscope}%
\begin{pgfscope}%
\pgfsys@transformshift{2.225088in}{0.703625in}%
\pgfsys@useobject{currentmarker}{}%
\end{pgfscope}%
\begin{pgfscope}%
\pgfsys@transformshift{2.225250in}{0.702026in}%
\pgfsys@useobject{currentmarker}{}%
\end{pgfscope}%
\begin{pgfscope}%
\pgfsys@transformshift{2.225412in}{0.647286in}%
\pgfsys@useobject{currentmarker}{}%
\end{pgfscope}%
\begin{pgfscope}%
\pgfsys@transformshift{2.225573in}{0.721000in}%
\pgfsys@useobject{currentmarker}{}%
\end{pgfscope}%
\begin{pgfscope}%
\pgfsys@transformshift{2.225735in}{0.721766in}%
\pgfsys@useobject{currentmarker}{}%
\end{pgfscope}%
\begin{pgfscope}%
\pgfsys@transformshift{2.225897in}{0.707266in}%
\pgfsys@useobject{currentmarker}{}%
\end{pgfscope}%
\begin{pgfscope}%
\pgfsys@transformshift{2.226058in}{0.679520in}%
\pgfsys@useobject{currentmarker}{}%
\end{pgfscope}%
\begin{pgfscope}%
\pgfsys@transformshift{2.226219in}{0.659990in}%
\pgfsys@useobject{currentmarker}{}%
\end{pgfscope}%
\begin{pgfscope}%
\pgfsys@transformshift{2.226381in}{0.686797in}%
\pgfsys@useobject{currentmarker}{}%
\end{pgfscope}%
\begin{pgfscope}%
\pgfsys@transformshift{2.226542in}{0.702805in}%
\pgfsys@useobject{currentmarker}{}%
\end{pgfscope}%
\begin{pgfscope}%
\pgfsys@transformshift{2.226703in}{0.622765in}%
\pgfsys@useobject{currentmarker}{}%
\end{pgfscope}%
\begin{pgfscope}%
\pgfsys@transformshift{2.226863in}{0.640439in}%
\pgfsys@useobject{currentmarker}{}%
\end{pgfscope}%
\begin{pgfscope}%
\pgfsys@transformshift{2.227024in}{0.656246in}%
\pgfsys@useobject{currentmarker}{}%
\end{pgfscope}%
\begin{pgfscope}%
\pgfsys@transformshift{2.227185in}{0.673155in}%
\pgfsys@useobject{currentmarker}{}%
\end{pgfscope}%
\begin{pgfscope}%
\pgfsys@transformshift{2.227345in}{0.667212in}%
\pgfsys@useobject{currentmarker}{}%
\end{pgfscope}%
\begin{pgfscope}%
\pgfsys@transformshift{2.227506in}{0.680015in}%
\pgfsys@useobject{currentmarker}{}%
\end{pgfscope}%
\begin{pgfscope}%
\pgfsys@transformshift{2.227666in}{0.686046in}%
\pgfsys@useobject{currentmarker}{}%
\end{pgfscope}%
\begin{pgfscope}%
\pgfsys@transformshift{2.227826in}{0.666267in}%
\pgfsys@useobject{currentmarker}{}%
\end{pgfscope}%
\begin{pgfscope}%
\pgfsys@transformshift{2.227986in}{0.661303in}%
\pgfsys@useobject{currentmarker}{}%
\end{pgfscope}%
\begin{pgfscope}%
\pgfsys@transformshift{2.228146in}{0.652090in}%
\pgfsys@useobject{currentmarker}{}%
\end{pgfscope}%
\begin{pgfscope}%
\pgfsys@transformshift{2.228306in}{0.697238in}%
\pgfsys@useobject{currentmarker}{}%
\end{pgfscope}%
\begin{pgfscope}%
\pgfsys@transformshift{2.228466in}{0.666765in}%
\pgfsys@useobject{currentmarker}{}%
\end{pgfscope}%
\begin{pgfscope}%
\pgfsys@transformshift{2.228625in}{0.695329in}%
\pgfsys@useobject{currentmarker}{}%
\end{pgfscope}%
\begin{pgfscope}%
\pgfsys@transformshift{2.228785in}{0.686566in}%
\pgfsys@useobject{currentmarker}{}%
\end{pgfscope}%
\begin{pgfscope}%
\pgfsys@transformshift{2.228944in}{0.725253in}%
\pgfsys@useobject{currentmarker}{}%
\end{pgfscope}%
\begin{pgfscope}%
\pgfsys@transformshift{2.229104in}{0.740777in}%
\pgfsys@useobject{currentmarker}{}%
\end{pgfscope}%
\begin{pgfscope}%
\pgfsys@transformshift{2.229263in}{0.751138in}%
\pgfsys@useobject{currentmarker}{}%
\end{pgfscope}%
\begin{pgfscope}%
\pgfsys@transformshift{2.229422in}{0.758356in}%
\pgfsys@useobject{currentmarker}{}%
\end{pgfscope}%
\begin{pgfscope}%
\pgfsys@transformshift{2.229581in}{0.748521in}%
\pgfsys@useobject{currentmarker}{}%
\end{pgfscope}%
\begin{pgfscope}%
\pgfsys@transformshift{2.229740in}{0.755899in}%
\pgfsys@useobject{currentmarker}{}%
\end{pgfscope}%
\begin{pgfscope}%
\pgfsys@transformshift{2.229898in}{0.725562in}%
\pgfsys@useobject{currentmarker}{}%
\end{pgfscope}%
\begin{pgfscope}%
\pgfsys@transformshift{2.230057in}{0.679957in}%
\pgfsys@useobject{currentmarker}{}%
\end{pgfscope}%
\begin{pgfscope}%
\pgfsys@transformshift{2.230215in}{0.698274in}%
\pgfsys@useobject{currentmarker}{}%
\end{pgfscope}%
\begin{pgfscope}%
\pgfsys@transformshift{2.230374in}{0.731303in}%
\pgfsys@useobject{currentmarker}{}%
\end{pgfscope}%
\begin{pgfscope}%
\pgfsys@transformshift{2.230532in}{0.663577in}%
\pgfsys@useobject{currentmarker}{}%
\end{pgfscope}%
\begin{pgfscope}%
\pgfsys@transformshift{2.230690in}{0.711493in}%
\pgfsys@useobject{currentmarker}{}%
\end{pgfscope}%
\begin{pgfscope}%
\pgfsys@transformshift{2.230848in}{0.698840in}%
\pgfsys@useobject{currentmarker}{}%
\end{pgfscope}%
\begin{pgfscope}%
\pgfsys@transformshift{2.231006in}{0.653367in}%
\pgfsys@useobject{currentmarker}{}%
\end{pgfscope}%
\begin{pgfscope}%
\pgfsys@transformshift{2.231164in}{0.707880in}%
\pgfsys@useobject{currentmarker}{}%
\end{pgfscope}%
\begin{pgfscope}%
\pgfsys@transformshift{2.231322in}{0.717248in}%
\pgfsys@useobject{currentmarker}{}%
\end{pgfscope}%
\begin{pgfscope}%
\pgfsys@transformshift{2.231479in}{0.676444in}%
\pgfsys@useobject{currentmarker}{}%
\end{pgfscope}%
\begin{pgfscope}%
\pgfsys@transformshift{2.231637in}{0.665145in}%
\pgfsys@useobject{currentmarker}{}%
\end{pgfscope}%
\begin{pgfscope}%
\pgfsys@transformshift{2.231794in}{0.693127in}%
\pgfsys@useobject{currentmarker}{}%
\end{pgfscope}%
\begin{pgfscope}%
\pgfsys@transformshift{2.231951in}{0.693485in}%
\pgfsys@useobject{currentmarker}{}%
\end{pgfscope}%
\begin{pgfscope}%
\pgfsys@transformshift{2.232109in}{0.691696in}%
\pgfsys@useobject{currentmarker}{}%
\end{pgfscope}%
\begin{pgfscope}%
\pgfsys@transformshift{2.232266in}{0.687416in}%
\pgfsys@useobject{currentmarker}{}%
\end{pgfscope}%
\begin{pgfscope}%
\pgfsys@transformshift{2.232423in}{0.665613in}%
\pgfsys@useobject{currentmarker}{}%
\end{pgfscope}%
\begin{pgfscope}%
\pgfsys@transformshift{2.232579in}{0.683715in}%
\pgfsys@useobject{currentmarker}{}%
\end{pgfscope}%
\begin{pgfscope}%
\pgfsys@transformshift{2.232736in}{0.686296in}%
\pgfsys@useobject{currentmarker}{}%
\end{pgfscope}%
\begin{pgfscope}%
\pgfsys@transformshift{2.232893in}{0.715749in}%
\pgfsys@useobject{currentmarker}{}%
\end{pgfscope}%
\begin{pgfscope}%
\pgfsys@transformshift{2.233049in}{0.709108in}%
\pgfsys@useobject{currentmarker}{}%
\end{pgfscope}%
\begin{pgfscope}%
\pgfsys@transformshift{2.233206in}{0.673375in}%
\pgfsys@useobject{currentmarker}{}%
\end{pgfscope}%
\begin{pgfscope}%
\pgfsys@transformshift{2.233362in}{0.704116in}%
\pgfsys@useobject{currentmarker}{}%
\end{pgfscope}%
\begin{pgfscope}%
\pgfsys@transformshift{2.233518in}{0.679770in}%
\pgfsys@useobject{currentmarker}{}%
\end{pgfscope}%
\begin{pgfscope}%
\pgfsys@transformshift{2.233674in}{0.672353in}%
\pgfsys@useobject{currentmarker}{}%
\end{pgfscope}%
\begin{pgfscope}%
\pgfsys@transformshift{2.233830in}{0.662293in}%
\pgfsys@useobject{currentmarker}{}%
\end{pgfscope}%
\begin{pgfscope}%
\pgfsys@transformshift{2.233986in}{0.671686in}%
\pgfsys@useobject{currentmarker}{}%
\end{pgfscope}%
\begin{pgfscope}%
\pgfsys@transformshift{2.234142in}{0.679862in}%
\pgfsys@useobject{currentmarker}{}%
\end{pgfscope}%
\begin{pgfscope}%
\pgfsys@transformshift{2.234297in}{0.722726in}%
\pgfsys@useobject{currentmarker}{}%
\end{pgfscope}%
\begin{pgfscope}%
\pgfsys@transformshift{2.234453in}{0.726783in}%
\pgfsys@useobject{currentmarker}{}%
\end{pgfscope}%
\begin{pgfscope}%
\pgfsys@transformshift{2.234608in}{0.733836in}%
\pgfsys@useobject{currentmarker}{}%
\end{pgfscope}%
\begin{pgfscope}%
\pgfsys@transformshift{2.234763in}{0.711932in}%
\pgfsys@useobject{currentmarker}{}%
\end{pgfscope}%
\begin{pgfscope}%
\pgfsys@transformshift{2.234919in}{0.684367in}%
\pgfsys@useobject{currentmarker}{}%
\end{pgfscope}%
\begin{pgfscope}%
\pgfsys@transformshift{2.235074in}{0.634948in}%
\pgfsys@useobject{currentmarker}{}%
\end{pgfscope}%
\begin{pgfscope}%
\pgfsys@transformshift{2.235229in}{0.737733in}%
\pgfsys@useobject{currentmarker}{}%
\end{pgfscope}%
\begin{pgfscope}%
\pgfsys@transformshift{2.235384in}{0.740202in}%
\pgfsys@useobject{currentmarker}{}%
\end{pgfscope}%
\begin{pgfscope}%
\pgfsys@transformshift{2.235538in}{0.641980in}%
\pgfsys@useobject{currentmarker}{}%
\end{pgfscope}%
\begin{pgfscope}%
\pgfsys@transformshift{2.235693in}{0.648868in}%
\pgfsys@useobject{currentmarker}{}%
\end{pgfscope}%
\begin{pgfscope}%
\pgfsys@transformshift{2.235848in}{0.663565in}%
\pgfsys@useobject{currentmarker}{}%
\end{pgfscope}%
\begin{pgfscope}%
\pgfsys@transformshift{2.236002in}{0.650692in}%
\pgfsys@useobject{currentmarker}{}%
\end{pgfscope}%
\begin{pgfscope}%
\pgfsys@transformshift{2.236156in}{0.711447in}%
\pgfsys@useobject{currentmarker}{}%
\end{pgfscope}%
\begin{pgfscope}%
\pgfsys@transformshift{2.236311in}{0.695723in}%
\pgfsys@useobject{currentmarker}{}%
\end{pgfscope}%
\begin{pgfscope}%
\pgfsys@transformshift{2.236465in}{0.691482in}%
\pgfsys@useobject{currentmarker}{}%
\end{pgfscope}%
\begin{pgfscope}%
\pgfsys@transformshift{2.236619in}{0.644823in}%
\pgfsys@useobject{currentmarker}{}%
\end{pgfscope}%
\begin{pgfscope}%
\pgfsys@transformshift{2.236773in}{0.674541in}%
\pgfsys@useobject{currentmarker}{}%
\end{pgfscope}%
\begin{pgfscope}%
\pgfsys@transformshift{2.236927in}{0.656272in}%
\pgfsys@useobject{currentmarker}{}%
\end{pgfscope}%
\begin{pgfscope}%
\pgfsys@transformshift{2.237080in}{0.666804in}%
\pgfsys@useobject{currentmarker}{}%
\end{pgfscope}%
\begin{pgfscope}%
\pgfsys@transformshift{2.237234in}{0.657752in}%
\pgfsys@useobject{currentmarker}{}%
\end{pgfscope}%
\begin{pgfscope}%
\pgfsys@transformshift{2.237387in}{0.676573in}%
\pgfsys@useobject{currentmarker}{}%
\end{pgfscope}%
\begin{pgfscope}%
\pgfsys@transformshift{2.237541in}{0.711751in}%
\pgfsys@useobject{currentmarker}{}%
\end{pgfscope}%
\begin{pgfscope}%
\pgfsys@transformshift{2.237694in}{0.681665in}%
\pgfsys@useobject{currentmarker}{}%
\end{pgfscope}%
\begin{pgfscope}%
\pgfsys@transformshift{2.237847in}{0.672634in}%
\pgfsys@useobject{currentmarker}{}%
\end{pgfscope}%
\begin{pgfscope}%
\pgfsys@transformshift{2.238000in}{0.703249in}%
\pgfsys@useobject{currentmarker}{}%
\end{pgfscope}%
\begin{pgfscope}%
\pgfsys@transformshift{2.238153in}{0.661057in}%
\pgfsys@useobject{currentmarker}{}%
\end{pgfscope}%
\begin{pgfscope}%
\pgfsys@transformshift{2.238306in}{0.659780in}%
\pgfsys@useobject{currentmarker}{}%
\end{pgfscope}%
\begin{pgfscope}%
\pgfsys@transformshift{2.238459in}{0.707272in}%
\pgfsys@useobject{currentmarker}{}%
\end{pgfscope}%
\begin{pgfscope}%
\pgfsys@transformshift{2.238612in}{0.683002in}%
\pgfsys@useobject{currentmarker}{}%
\end{pgfscope}%
\begin{pgfscope}%
\pgfsys@transformshift{2.238764in}{0.705149in}%
\pgfsys@useobject{currentmarker}{}%
\end{pgfscope}%
\begin{pgfscope}%
\pgfsys@transformshift{2.238917in}{0.722769in}%
\pgfsys@useobject{currentmarker}{}%
\end{pgfscope}%
\begin{pgfscope}%
\pgfsys@transformshift{2.239069in}{0.741962in}%
\pgfsys@useobject{currentmarker}{}%
\end{pgfscope}%
\begin{pgfscope}%
\pgfsys@transformshift{2.239221in}{0.696466in}%
\pgfsys@useobject{currentmarker}{}%
\end{pgfscope}%
\begin{pgfscope}%
\pgfsys@transformshift{2.239373in}{0.704415in}%
\pgfsys@useobject{currentmarker}{}%
\end{pgfscope}%
\begin{pgfscope}%
\pgfsys@transformshift{2.239525in}{0.719826in}%
\pgfsys@useobject{currentmarker}{}%
\end{pgfscope}%
\begin{pgfscope}%
\pgfsys@transformshift{2.239677in}{0.748941in}%
\pgfsys@useobject{currentmarker}{}%
\end{pgfscope}%
\begin{pgfscope}%
\pgfsys@transformshift{2.239829in}{0.731963in}%
\pgfsys@useobject{currentmarker}{}%
\end{pgfscope}%
\begin{pgfscope}%
\pgfsys@transformshift{2.239981in}{0.660837in}%
\pgfsys@useobject{currentmarker}{}%
\end{pgfscope}%
\begin{pgfscope}%
\pgfsys@transformshift{2.240133in}{0.645675in}%
\pgfsys@useobject{currentmarker}{}%
\end{pgfscope}%
\begin{pgfscope}%
\pgfsys@transformshift{2.240284in}{0.650799in}%
\pgfsys@useobject{currentmarker}{}%
\end{pgfscope}%
\begin{pgfscope}%
\pgfsys@transformshift{2.240436in}{0.605374in}%
\pgfsys@useobject{currentmarker}{}%
\end{pgfscope}%
\begin{pgfscope}%
\pgfsys@transformshift{2.240587in}{0.660979in}%
\pgfsys@useobject{currentmarker}{}%
\end{pgfscope}%
\begin{pgfscope}%
\pgfsys@transformshift{2.240738in}{0.648619in}%
\pgfsys@useobject{currentmarker}{}%
\end{pgfscope}%
\begin{pgfscope}%
\pgfsys@transformshift{2.240889in}{0.705784in}%
\pgfsys@useobject{currentmarker}{}%
\end{pgfscope}%
\begin{pgfscope}%
\pgfsys@transformshift{2.241040in}{0.696269in}%
\pgfsys@useobject{currentmarker}{}%
\end{pgfscope}%
\begin{pgfscope}%
\pgfsys@transformshift{2.241191in}{0.724454in}%
\pgfsys@useobject{currentmarker}{}%
\end{pgfscope}%
\begin{pgfscope}%
\pgfsys@transformshift{2.241342in}{0.749613in}%
\pgfsys@useobject{currentmarker}{}%
\end{pgfscope}%
\begin{pgfscope}%
\pgfsys@transformshift{2.241493in}{0.731932in}%
\pgfsys@useobject{currentmarker}{}%
\end{pgfscope}%
\begin{pgfscope}%
\pgfsys@transformshift{2.241643in}{0.709987in}%
\pgfsys@useobject{currentmarker}{}%
\end{pgfscope}%
\begin{pgfscope}%
\pgfsys@transformshift{2.241794in}{0.713527in}%
\pgfsys@useobject{currentmarker}{}%
\end{pgfscope}%
\begin{pgfscope}%
\pgfsys@transformshift{2.241944in}{0.692584in}%
\pgfsys@useobject{currentmarker}{}%
\end{pgfscope}%
\begin{pgfscope}%
\pgfsys@transformshift{2.242095in}{0.652717in}%
\pgfsys@useobject{currentmarker}{}%
\end{pgfscope}%
\begin{pgfscope}%
\pgfsys@transformshift{2.242245in}{0.645505in}%
\pgfsys@useobject{currentmarker}{}%
\end{pgfscope}%
\begin{pgfscope}%
\pgfsys@transformshift{2.242395in}{0.691403in}%
\pgfsys@useobject{currentmarker}{}%
\end{pgfscope}%
\begin{pgfscope}%
\pgfsys@transformshift{2.242545in}{0.675150in}%
\pgfsys@useobject{currentmarker}{}%
\end{pgfscope}%
\begin{pgfscope}%
\pgfsys@transformshift{2.242695in}{0.639311in}%
\pgfsys@useobject{currentmarker}{}%
\end{pgfscope}%
\begin{pgfscope}%
\pgfsys@transformshift{2.242845in}{0.661447in}%
\pgfsys@useobject{currentmarker}{}%
\end{pgfscope}%
\begin{pgfscope}%
\pgfsys@transformshift{2.242994in}{0.647048in}%
\pgfsys@useobject{currentmarker}{}%
\end{pgfscope}%
\begin{pgfscope}%
\pgfsys@transformshift{2.243144in}{0.655932in}%
\pgfsys@useobject{currentmarker}{}%
\end{pgfscope}%
\begin{pgfscope}%
\pgfsys@transformshift{2.243294in}{0.679296in}%
\pgfsys@useobject{currentmarker}{}%
\end{pgfscope}%
\begin{pgfscope}%
\pgfsys@transformshift{2.243443in}{0.723797in}%
\pgfsys@useobject{currentmarker}{}%
\end{pgfscope}%
\begin{pgfscope}%
\pgfsys@transformshift{2.243592in}{0.701801in}%
\pgfsys@useobject{currentmarker}{}%
\end{pgfscope}%
\begin{pgfscope}%
\pgfsys@transformshift{2.243742in}{0.702370in}%
\pgfsys@useobject{currentmarker}{}%
\end{pgfscope}%
\begin{pgfscope}%
\pgfsys@transformshift{2.243891in}{0.712814in}%
\pgfsys@useobject{currentmarker}{}%
\end{pgfscope}%
\begin{pgfscope}%
\pgfsys@transformshift{2.244040in}{0.700526in}%
\pgfsys@useobject{currentmarker}{}%
\end{pgfscope}%
\begin{pgfscope}%
\pgfsys@transformshift{2.244189in}{0.668277in}%
\pgfsys@useobject{currentmarker}{}%
\end{pgfscope}%
\begin{pgfscope}%
\pgfsys@transformshift{2.244337in}{0.654914in}%
\pgfsys@useobject{currentmarker}{}%
\end{pgfscope}%
\begin{pgfscope}%
\pgfsys@transformshift{2.244486in}{0.696492in}%
\pgfsys@useobject{currentmarker}{}%
\end{pgfscope}%
\begin{pgfscope}%
\pgfsys@transformshift{2.244635in}{0.676456in}%
\pgfsys@useobject{currentmarker}{}%
\end{pgfscope}%
\begin{pgfscope}%
\pgfsys@transformshift{2.244783in}{0.667008in}%
\pgfsys@useobject{currentmarker}{}%
\end{pgfscope}%
\begin{pgfscope}%
\pgfsys@transformshift{2.244932in}{0.648812in}%
\pgfsys@useobject{currentmarker}{}%
\end{pgfscope}%
\begin{pgfscope}%
\pgfsys@transformshift{2.245080in}{0.625223in}%
\pgfsys@useobject{currentmarker}{}%
\end{pgfscope}%
\begin{pgfscope}%
\pgfsys@transformshift{2.245228in}{0.668583in}%
\pgfsys@useobject{currentmarker}{}%
\end{pgfscope}%
\begin{pgfscope}%
\pgfsys@transformshift{2.245377in}{0.679072in}%
\pgfsys@useobject{currentmarker}{}%
\end{pgfscope}%
\begin{pgfscope}%
\pgfsys@transformshift{2.245525in}{0.720256in}%
\pgfsys@useobject{currentmarker}{}%
\end{pgfscope}%
\begin{pgfscope}%
\pgfsys@transformshift{2.245672in}{0.743194in}%
\pgfsys@useobject{currentmarker}{}%
\end{pgfscope}%
\begin{pgfscope}%
\pgfsys@transformshift{2.245820in}{0.696647in}%
\pgfsys@useobject{currentmarker}{}%
\end{pgfscope}%
\begin{pgfscope}%
\pgfsys@transformshift{2.245968in}{0.626885in}%
\pgfsys@useobject{currentmarker}{}%
\end{pgfscope}%
\begin{pgfscope}%
\pgfsys@transformshift{2.246116in}{0.641761in}%
\pgfsys@useobject{currentmarker}{}%
\end{pgfscope}%
\begin{pgfscope}%
\pgfsys@transformshift{2.246263in}{0.686672in}%
\pgfsys@useobject{currentmarker}{}%
\end{pgfscope}%
\begin{pgfscope}%
\pgfsys@transformshift{2.246411in}{0.677292in}%
\pgfsys@useobject{currentmarker}{}%
\end{pgfscope}%
\begin{pgfscope}%
\pgfsys@transformshift{2.246558in}{0.707727in}%
\pgfsys@useobject{currentmarker}{}%
\end{pgfscope}%
\begin{pgfscope}%
\pgfsys@transformshift{2.246705in}{0.679275in}%
\pgfsys@useobject{currentmarker}{}%
\end{pgfscope}%
\begin{pgfscope}%
\pgfsys@transformshift{2.246853in}{0.688945in}%
\pgfsys@useobject{currentmarker}{}%
\end{pgfscope}%
\begin{pgfscope}%
\pgfsys@transformshift{2.247000in}{0.733506in}%
\pgfsys@useobject{currentmarker}{}%
\end{pgfscope}%
\begin{pgfscope}%
\pgfsys@transformshift{2.247147in}{0.708092in}%
\pgfsys@useobject{currentmarker}{}%
\end{pgfscope}%
\begin{pgfscope}%
\pgfsys@transformshift{2.247293in}{0.661198in}%
\pgfsys@useobject{currentmarker}{}%
\end{pgfscope}%
\begin{pgfscope}%
\pgfsys@transformshift{2.247440in}{0.655308in}%
\pgfsys@useobject{currentmarker}{}%
\end{pgfscope}%
\begin{pgfscope}%
\pgfsys@transformshift{2.247587in}{0.666091in}%
\pgfsys@useobject{currentmarker}{}%
\end{pgfscope}%
\begin{pgfscope}%
\pgfsys@transformshift{2.247734in}{0.689839in}%
\pgfsys@useobject{currentmarker}{}%
\end{pgfscope}%
\begin{pgfscope}%
\pgfsys@transformshift{2.247880in}{0.697435in}%
\pgfsys@useobject{currentmarker}{}%
\end{pgfscope}%
\begin{pgfscope}%
\pgfsys@transformshift{2.248026in}{0.709277in}%
\pgfsys@useobject{currentmarker}{}%
\end{pgfscope}%
\begin{pgfscope}%
\pgfsys@transformshift{2.248173in}{0.647779in}%
\pgfsys@useobject{currentmarker}{}%
\end{pgfscope}%
\begin{pgfscope}%
\pgfsys@transformshift{2.248319in}{0.666248in}%
\pgfsys@useobject{currentmarker}{}%
\end{pgfscope}%
\begin{pgfscope}%
\pgfsys@transformshift{2.248465in}{0.664228in}%
\pgfsys@useobject{currentmarker}{}%
\end{pgfscope}%
\begin{pgfscope}%
\pgfsys@transformshift{2.248611in}{0.643787in}%
\pgfsys@useobject{currentmarker}{}%
\end{pgfscope}%
\begin{pgfscope}%
\pgfsys@transformshift{2.248757in}{0.661420in}%
\pgfsys@useobject{currentmarker}{}%
\end{pgfscope}%
\begin{pgfscope}%
\pgfsys@transformshift{2.248903in}{0.681459in}%
\pgfsys@useobject{currentmarker}{}%
\end{pgfscope}%
\begin{pgfscope}%
\pgfsys@transformshift{2.249049in}{0.701286in}%
\pgfsys@useobject{currentmarker}{}%
\end{pgfscope}%
\begin{pgfscope}%
\pgfsys@transformshift{2.249194in}{0.699312in}%
\pgfsys@useobject{currentmarker}{}%
\end{pgfscope}%
\begin{pgfscope}%
\pgfsys@transformshift{2.249340in}{0.710291in}%
\pgfsys@useobject{currentmarker}{}%
\end{pgfscope}%
\begin{pgfscope}%
\pgfsys@transformshift{2.249485in}{0.696681in}%
\pgfsys@useobject{currentmarker}{}%
\end{pgfscope}%
\begin{pgfscope}%
\pgfsys@transformshift{2.249631in}{0.662323in}%
\pgfsys@useobject{currentmarker}{}%
\end{pgfscope}%
\begin{pgfscope}%
\pgfsys@transformshift{2.249776in}{0.691154in}%
\pgfsys@useobject{currentmarker}{}%
\end{pgfscope}%
\begin{pgfscope}%
\pgfsys@transformshift{2.249921in}{0.701195in}%
\pgfsys@useobject{currentmarker}{}%
\end{pgfscope}%
\begin{pgfscope}%
\pgfsys@transformshift{2.250066in}{0.607596in}%
\pgfsys@useobject{currentmarker}{}%
\end{pgfscope}%
\begin{pgfscope}%
\pgfsys@transformshift{2.250211in}{0.626246in}%
\pgfsys@useobject{currentmarker}{}%
\end{pgfscope}%
\begin{pgfscope}%
\pgfsys@transformshift{2.250356in}{0.648008in}%
\pgfsys@useobject{currentmarker}{}%
\end{pgfscope}%
\begin{pgfscope}%
\pgfsys@transformshift{2.250501in}{0.720145in}%
\pgfsys@useobject{currentmarker}{}%
\end{pgfscope}%
\begin{pgfscope}%
\pgfsys@transformshift{2.250645in}{0.720709in}%
\pgfsys@useobject{currentmarker}{}%
\end{pgfscope}%
\begin{pgfscope}%
\pgfsys@transformshift{2.250790in}{0.693837in}%
\pgfsys@useobject{currentmarker}{}%
\end{pgfscope}%
\begin{pgfscope}%
\pgfsys@transformshift{2.250935in}{0.664196in}%
\pgfsys@useobject{currentmarker}{}%
\end{pgfscope}%
\begin{pgfscope}%
\pgfsys@transformshift{2.251079in}{0.699913in}%
\pgfsys@useobject{currentmarker}{}%
\end{pgfscope}%
\begin{pgfscope}%
\pgfsys@transformshift{2.251223in}{0.711852in}%
\pgfsys@useobject{currentmarker}{}%
\end{pgfscope}%
\begin{pgfscope}%
\pgfsys@transformshift{2.251368in}{0.690207in}%
\pgfsys@useobject{currentmarker}{}%
\end{pgfscope}%
\begin{pgfscope}%
\pgfsys@transformshift{2.251512in}{0.689711in}%
\pgfsys@useobject{currentmarker}{}%
\end{pgfscope}%
\begin{pgfscope}%
\pgfsys@transformshift{2.251656in}{0.704059in}%
\pgfsys@useobject{currentmarker}{}%
\end{pgfscope}%
\begin{pgfscope}%
\pgfsys@transformshift{2.251800in}{0.718119in}%
\pgfsys@useobject{currentmarker}{}%
\end{pgfscope}%
\begin{pgfscope}%
\pgfsys@transformshift{2.251944in}{0.706381in}%
\pgfsys@useobject{currentmarker}{}%
\end{pgfscope}%
\begin{pgfscope}%
\pgfsys@transformshift{2.252087in}{0.722297in}%
\pgfsys@useobject{currentmarker}{}%
\end{pgfscope}%
\begin{pgfscope}%
\pgfsys@transformshift{2.252231in}{0.721144in}%
\pgfsys@useobject{currentmarker}{}%
\end{pgfscope}%
\begin{pgfscope}%
\pgfsys@transformshift{2.252375in}{0.668920in}%
\pgfsys@useobject{currentmarker}{}%
\end{pgfscope}%
\begin{pgfscope}%
\pgfsys@transformshift{2.252518in}{0.665444in}%
\pgfsys@useobject{currentmarker}{}%
\end{pgfscope}%
\begin{pgfscope}%
\pgfsys@transformshift{2.252662in}{0.649127in}%
\pgfsys@useobject{currentmarker}{}%
\end{pgfscope}%
\begin{pgfscope}%
\pgfsys@transformshift{2.252805in}{0.687528in}%
\pgfsys@useobject{currentmarker}{}%
\end{pgfscope}%
\begin{pgfscope}%
\pgfsys@transformshift{2.252948in}{0.675091in}%
\pgfsys@useobject{currentmarker}{}%
\end{pgfscope}%
\begin{pgfscope}%
\pgfsys@transformshift{2.253091in}{0.687771in}%
\pgfsys@useobject{currentmarker}{}%
\end{pgfscope}%
\begin{pgfscope}%
\pgfsys@transformshift{2.253234in}{0.706607in}%
\pgfsys@useobject{currentmarker}{}%
\end{pgfscope}%
\begin{pgfscope}%
\pgfsys@transformshift{2.253377in}{0.671624in}%
\pgfsys@useobject{currentmarker}{}%
\end{pgfscope}%
\begin{pgfscope}%
\pgfsys@transformshift{2.253520in}{0.704291in}%
\pgfsys@useobject{currentmarker}{}%
\end{pgfscope}%
\begin{pgfscope}%
\pgfsys@transformshift{2.253663in}{0.735978in}%
\pgfsys@useobject{currentmarker}{}%
\end{pgfscope}%
\begin{pgfscope}%
\pgfsys@transformshift{2.253806in}{0.705543in}%
\pgfsys@useobject{currentmarker}{}%
\end{pgfscope}%
\begin{pgfscope}%
\pgfsys@transformshift{2.253948in}{0.673976in}%
\pgfsys@useobject{currentmarker}{}%
\end{pgfscope}%
\begin{pgfscope}%
\pgfsys@transformshift{2.254091in}{0.698799in}%
\pgfsys@useobject{currentmarker}{}%
\end{pgfscope}%
\begin{pgfscope}%
\pgfsys@transformshift{2.254233in}{0.691079in}%
\pgfsys@useobject{currentmarker}{}%
\end{pgfscope}%
\begin{pgfscope}%
\pgfsys@transformshift{2.254375in}{0.666402in}%
\pgfsys@useobject{currentmarker}{}%
\end{pgfscope}%
\begin{pgfscope}%
\pgfsys@transformshift{2.254518in}{0.651965in}%
\pgfsys@useobject{currentmarker}{}%
\end{pgfscope}%
\begin{pgfscope}%
\pgfsys@transformshift{2.254660in}{0.694072in}%
\pgfsys@useobject{currentmarker}{}%
\end{pgfscope}%
\begin{pgfscope}%
\pgfsys@transformshift{2.254802in}{0.726310in}%
\pgfsys@useobject{currentmarker}{}%
\end{pgfscope}%
\begin{pgfscope}%
\pgfsys@transformshift{2.254944in}{0.690760in}%
\pgfsys@useobject{currentmarker}{}%
\end{pgfscope}%
\begin{pgfscope}%
\pgfsys@transformshift{2.255086in}{0.684186in}%
\pgfsys@useobject{currentmarker}{}%
\end{pgfscope}%
\begin{pgfscope}%
\pgfsys@transformshift{2.255227in}{0.698056in}%
\pgfsys@useobject{currentmarker}{}%
\end{pgfscope}%
\begin{pgfscope}%
\pgfsys@transformshift{2.255369in}{0.701274in}%
\pgfsys@useobject{currentmarker}{}%
\end{pgfscope}%
\begin{pgfscope}%
\pgfsys@transformshift{2.255511in}{0.675253in}%
\pgfsys@useobject{currentmarker}{}%
\end{pgfscope}%
\begin{pgfscope}%
\pgfsys@transformshift{2.255652in}{0.712363in}%
\pgfsys@useobject{currentmarker}{}%
\end{pgfscope}%
\begin{pgfscope}%
\pgfsys@transformshift{2.255794in}{0.693217in}%
\pgfsys@useobject{currentmarker}{}%
\end{pgfscope}%
\begin{pgfscope}%
\pgfsys@transformshift{2.255935in}{0.711714in}%
\pgfsys@useobject{currentmarker}{}%
\end{pgfscope}%
\begin{pgfscope}%
\pgfsys@transformshift{2.256076in}{0.666845in}%
\pgfsys@useobject{currentmarker}{}%
\end{pgfscope}%
\begin{pgfscope}%
\pgfsys@transformshift{2.256217in}{0.674149in}%
\pgfsys@useobject{currentmarker}{}%
\end{pgfscope}%
\begin{pgfscope}%
\pgfsys@transformshift{2.256358in}{0.667928in}%
\pgfsys@useobject{currentmarker}{}%
\end{pgfscope}%
\begin{pgfscope}%
\pgfsys@transformshift{2.256499in}{0.643141in}%
\pgfsys@useobject{currentmarker}{}%
\end{pgfscope}%
\begin{pgfscope}%
\pgfsys@transformshift{2.256640in}{0.702011in}%
\pgfsys@useobject{currentmarker}{}%
\end{pgfscope}%
\begin{pgfscope}%
\pgfsys@transformshift{2.256781in}{0.666684in}%
\pgfsys@useobject{currentmarker}{}%
\end{pgfscope}%
\begin{pgfscope}%
\pgfsys@transformshift{2.256922in}{0.666329in}%
\pgfsys@useobject{currentmarker}{}%
\end{pgfscope}%
\begin{pgfscope}%
\pgfsys@transformshift{2.257062in}{0.708523in}%
\pgfsys@useobject{currentmarker}{}%
\end{pgfscope}%
\begin{pgfscope}%
\pgfsys@transformshift{2.257203in}{0.709929in}%
\pgfsys@useobject{currentmarker}{}%
\end{pgfscope}%
\begin{pgfscope}%
\pgfsys@transformshift{2.257343in}{0.684925in}%
\pgfsys@useobject{currentmarker}{}%
\end{pgfscope}%
\begin{pgfscope}%
\pgfsys@transformshift{2.257484in}{0.648857in}%
\pgfsys@useobject{currentmarker}{}%
\end{pgfscope}%
\begin{pgfscope}%
\pgfsys@transformshift{2.257624in}{0.633283in}%
\pgfsys@useobject{currentmarker}{}%
\end{pgfscope}%
\begin{pgfscope}%
\pgfsys@transformshift{2.257764in}{0.658897in}%
\pgfsys@useobject{currentmarker}{}%
\end{pgfscope}%
\begin{pgfscope}%
\pgfsys@transformshift{2.257904in}{0.655753in}%
\pgfsys@useobject{currentmarker}{}%
\end{pgfscope}%
\begin{pgfscope}%
\pgfsys@transformshift{2.258044in}{0.647393in}%
\pgfsys@useobject{currentmarker}{}%
\end{pgfscope}%
\begin{pgfscope}%
\pgfsys@transformshift{2.258184in}{0.687425in}%
\pgfsys@useobject{currentmarker}{}%
\end{pgfscope}%
\begin{pgfscope}%
\pgfsys@transformshift{2.258324in}{0.640621in}%
\pgfsys@useobject{currentmarker}{}%
\end{pgfscope}%
\begin{pgfscope}%
\pgfsys@transformshift{2.258464in}{0.603059in}%
\pgfsys@useobject{currentmarker}{}%
\end{pgfscope}%
\begin{pgfscope}%
\pgfsys@transformshift{2.258604in}{0.703940in}%
\pgfsys@useobject{currentmarker}{}%
\end{pgfscope}%
\begin{pgfscope}%
\pgfsys@transformshift{2.258743in}{0.720699in}%
\pgfsys@useobject{currentmarker}{}%
\end{pgfscope}%
\begin{pgfscope}%
\pgfsys@transformshift{2.258883in}{0.704533in}%
\pgfsys@useobject{currentmarker}{}%
\end{pgfscope}%
\begin{pgfscope}%
\pgfsys@transformshift{2.259022in}{0.740334in}%
\pgfsys@useobject{currentmarker}{}%
\end{pgfscope}%
\begin{pgfscope}%
\pgfsys@transformshift{2.259161in}{0.714317in}%
\pgfsys@useobject{currentmarker}{}%
\end{pgfscope}%
\begin{pgfscope}%
\pgfsys@transformshift{2.259301in}{0.688347in}%
\pgfsys@useobject{currentmarker}{}%
\end{pgfscope}%
\begin{pgfscope}%
\pgfsys@transformshift{2.259440in}{0.640970in}%
\pgfsys@useobject{currentmarker}{}%
\end{pgfscope}%
\begin{pgfscope}%
\pgfsys@transformshift{2.259579in}{0.643069in}%
\pgfsys@useobject{currentmarker}{}%
\end{pgfscope}%
\begin{pgfscope}%
\pgfsys@transformshift{2.259718in}{0.675350in}%
\pgfsys@useobject{currentmarker}{}%
\end{pgfscope}%
\begin{pgfscope}%
\pgfsys@transformshift{2.259857in}{0.681239in}%
\pgfsys@useobject{currentmarker}{}%
\end{pgfscope}%
\begin{pgfscope}%
\pgfsys@transformshift{2.259995in}{0.671943in}%
\pgfsys@useobject{currentmarker}{}%
\end{pgfscope}%
\begin{pgfscope}%
\pgfsys@transformshift{2.260134in}{0.701763in}%
\pgfsys@useobject{currentmarker}{}%
\end{pgfscope}%
\begin{pgfscope}%
\pgfsys@transformshift{2.260273in}{0.745728in}%
\pgfsys@useobject{currentmarker}{}%
\end{pgfscope}%
\begin{pgfscope}%
\pgfsys@transformshift{2.260411in}{0.700523in}%
\pgfsys@useobject{currentmarker}{}%
\end{pgfscope}%
\begin{pgfscope}%
\pgfsys@transformshift{2.260550in}{0.729340in}%
\pgfsys@useobject{currentmarker}{}%
\end{pgfscope}%
\begin{pgfscope}%
\pgfsys@transformshift{2.260688in}{0.731554in}%
\pgfsys@useobject{currentmarker}{}%
\end{pgfscope}%
\begin{pgfscope}%
\pgfsys@transformshift{2.260827in}{0.695235in}%
\pgfsys@useobject{currentmarker}{}%
\end{pgfscope}%
\begin{pgfscope}%
\pgfsys@transformshift{2.260965in}{0.652564in}%
\pgfsys@useobject{currentmarker}{}%
\end{pgfscope}%
\begin{pgfscope}%
\pgfsys@transformshift{2.261103in}{0.705764in}%
\pgfsys@useobject{currentmarker}{}%
\end{pgfscope}%
\begin{pgfscope}%
\pgfsys@transformshift{2.261241in}{0.738737in}%
\pgfsys@useobject{currentmarker}{}%
\end{pgfscope}%
\begin{pgfscope}%
\pgfsys@transformshift{2.261379in}{0.713694in}%
\pgfsys@useobject{currentmarker}{}%
\end{pgfscope}%
\begin{pgfscope}%
\pgfsys@transformshift{2.261517in}{0.692730in}%
\pgfsys@useobject{currentmarker}{}%
\end{pgfscope}%
\begin{pgfscope}%
\pgfsys@transformshift{2.261654in}{0.717327in}%
\pgfsys@useobject{currentmarker}{}%
\end{pgfscope}%
\begin{pgfscope}%
\pgfsys@transformshift{2.261792in}{0.701264in}%
\pgfsys@useobject{currentmarker}{}%
\end{pgfscope}%
\begin{pgfscope}%
\pgfsys@transformshift{2.261930in}{0.648040in}%
\pgfsys@useobject{currentmarker}{}%
\end{pgfscope}%
\begin{pgfscope}%
\pgfsys@transformshift{2.262067in}{0.695017in}%
\pgfsys@useobject{currentmarker}{}%
\end{pgfscope}%
\begin{pgfscope}%
\pgfsys@transformshift{2.262205in}{0.686584in}%
\pgfsys@useobject{currentmarker}{}%
\end{pgfscope}%
\begin{pgfscope}%
\pgfsys@transformshift{2.262342in}{0.673170in}%
\pgfsys@useobject{currentmarker}{}%
\end{pgfscope}%
\begin{pgfscope}%
\pgfsys@transformshift{2.262479in}{0.668230in}%
\pgfsys@useobject{currentmarker}{}%
\end{pgfscope}%
\begin{pgfscope}%
\pgfsys@transformshift{2.262617in}{0.695651in}%
\pgfsys@useobject{currentmarker}{}%
\end{pgfscope}%
\begin{pgfscope}%
\pgfsys@transformshift{2.262754in}{0.677191in}%
\pgfsys@useobject{currentmarker}{}%
\end{pgfscope}%
\begin{pgfscope}%
\pgfsys@transformshift{2.262891in}{0.650184in}%
\pgfsys@useobject{currentmarker}{}%
\end{pgfscope}%
\begin{pgfscope}%
\pgfsys@transformshift{2.263028in}{0.717962in}%
\pgfsys@useobject{currentmarker}{}%
\end{pgfscope}%
\begin{pgfscope}%
\pgfsys@transformshift{2.263165in}{0.706230in}%
\pgfsys@useobject{currentmarker}{}%
\end{pgfscope}%
\begin{pgfscope}%
\pgfsys@transformshift{2.263301in}{0.710286in}%
\pgfsys@useobject{currentmarker}{}%
\end{pgfscope}%
\begin{pgfscope}%
\pgfsys@transformshift{2.263438in}{0.694894in}%
\pgfsys@useobject{currentmarker}{}%
\end{pgfscope}%
\begin{pgfscope}%
\pgfsys@transformshift{2.263575in}{0.670663in}%
\pgfsys@useobject{currentmarker}{}%
\end{pgfscope}%
\begin{pgfscope}%
\pgfsys@transformshift{2.263711in}{0.671463in}%
\pgfsys@useobject{currentmarker}{}%
\end{pgfscope}%
\begin{pgfscope}%
\pgfsys@transformshift{2.263848in}{0.638951in}%
\pgfsys@useobject{currentmarker}{}%
\end{pgfscope}%
\begin{pgfscope}%
\pgfsys@transformshift{2.263984in}{0.608143in}%
\pgfsys@useobject{currentmarker}{}%
\end{pgfscope}%
\begin{pgfscope}%
\pgfsys@transformshift{2.264120in}{0.687217in}%
\pgfsys@useobject{currentmarker}{}%
\end{pgfscope}%
\begin{pgfscope}%
\pgfsys@transformshift{2.264256in}{0.659945in}%
\pgfsys@useobject{currentmarker}{}%
\end{pgfscope}%
\begin{pgfscope}%
\pgfsys@transformshift{2.264392in}{0.667699in}%
\pgfsys@useobject{currentmarker}{}%
\end{pgfscope}%
\begin{pgfscope}%
\pgfsys@transformshift{2.264529in}{0.736234in}%
\pgfsys@useobject{currentmarker}{}%
\end{pgfscope}%
\begin{pgfscope}%
\pgfsys@transformshift{2.264664in}{0.747768in}%
\pgfsys@useobject{currentmarker}{}%
\end{pgfscope}%
\begin{pgfscope}%
\pgfsys@transformshift{2.264800in}{0.737056in}%
\pgfsys@useobject{currentmarker}{}%
\end{pgfscope}%
\begin{pgfscope}%
\pgfsys@transformshift{2.264936in}{0.692515in}%
\pgfsys@useobject{currentmarker}{}%
\end{pgfscope}%
\begin{pgfscope}%
\pgfsys@transformshift{2.265072in}{0.729252in}%
\pgfsys@useobject{currentmarker}{}%
\end{pgfscope}%
\begin{pgfscope}%
\pgfsys@transformshift{2.265207in}{0.724098in}%
\pgfsys@useobject{currentmarker}{}%
\end{pgfscope}%
\begin{pgfscope}%
\pgfsys@transformshift{2.265343in}{0.627153in}%
\pgfsys@useobject{currentmarker}{}%
\end{pgfscope}%
\begin{pgfscope}%
\pgfsys@transformshift{2.265478in}{0.671592in}%
\pgfsys@useobject{currentmarker}{}%
\end{pgfscope}%
\begin{pgfscope}%
\pgfsys@transformshift{2.265614in}{0.720012in}%
\pgfsys@useobject{currentmarker}{}%
\end{pgfscope}%
\begin{pgfscope}%
\pgfsys@transformshift{2.265749in}{0.684455in}%
\pgfsys@useobject{currentmarker}{}%
\end{pgfscope}%
\begin{pgfscope}%
\pgfsys@transformshift{2.265884in}{0.662993in}%
\pgfsys@useobject{currentmarker}{}%
\end{pgfscope}%
\begin{pgfscope}%
\pgfsys@transformshift{2.266019in}{0.673080in}%
\pgfsys@useobject{currentmarker}{}%
\end{pgfscope}%
\begin{pgfscope}%
\pgfsys@transformshift{2.266154in}{0.691437in}%
\pgfsys@useobject{currentmarker}{}%
\end{pgfscope}%
\begin{pgfscope}%
\pgfsys@transformshift{2.266289in}{0.685604in}%
\pgfsys@useobject{currentmarker}{}%
\end{pgfscope}%
\begin{pgfscope}%
\pgfsys@transformshift{2.266424in}{0.691666in}%
\pgfsys@useobject{currentmarker}{}%
\end{pgfscope}%
\begin{pgfscope}%
\pgfsys@transformshift{2.266559in}{0.705084in}%
\pgfsys@useobject{currentmarker}{}%
\end{pgfscope}%
\begin{pgfscope}%
\pgfsys@transformshift{2.266694in}{0.732878in}%
\pgfsys@useobject{currentmarker}{}%
\end{pgfscope}%
\begin{pgfscope}%
\pgfsys@transformshift{2.266828in}{0.717866in}%
\pgfsys@useobject{currentmarker}{}%
\end{pgfscope}%
\begin{pgfscope}%
\pgfsys@transformshift{2.266963in}{0.719829in}%
\pgfsys@useobject{currentmarker}{}%
\end{pgfscope}%
\begin{pgfscope}%
\pgfsys@transformshift{2.267098in}{0.712061in}%
\pgfsys@useobject{currentmarker}{}%
\end{pgfscope}%
\begin{pgfscope}%
\pgfsys@transformshift{2.267232in}{0.719081in}%
\pgfsys@useobject{currentmarker}{}%
\end{pgfscope}%
\begin{pgfscope}%
\pgfsys@transformshift{2.267366in}{0.705006in}%
\pgfsys@useobject{currentmarker}{}%
\end{pgfscope}%
\begin{pgfscope}%
\pgfsys@transformshift{2.267501in}{0.660491in}%
\pgfsys@useobject{currentmarker}{}%
\end{pgfscope}%
\begin{pgfscope}%
\pgfsys@transformshift{2.267635in}{0.636766in}%
\pgfsys@useobject{currentmarker}{}%
\end{pgfscope}%
\begin{pgfscope}%
\pgfsys@transformshift{2.267769in}{0.690835in}%
\pgfsys@useobject{currentmarker}{}%
\end{pgfscope}%
\begin{pgfscope}%
\pgfsys@transformshift{2.267903in}{0.720019in}%
\pgfsys@useobject{currentmarker}{}%
\end{pgfscope}%
\begin{pgfscope}%
\pgfsys@transformshift{2.268037in}{0.623526in}%
\pgfsys@useobject{currentmarker}{}%
\end{pgfscope}%
\begin{pgfscope}%
\pgfsys@transformshift{2.268171in}{0.638314in}%
\pgfsys@useobject{currentmarker}{}%
\end{pgfscope}%
\begin{pgfscope}%
\pgfsys@transformshift{2.268304in}{0.713365in}%
\pgfsys@useobject{currentmarker}{}%
\end{pgfscope}%
\begin{pgfscope}%
\pgfsys@transformshift{2.268438in}{0.684976in}%
\pgfsys@useobject{currentmarker}{}%
\end{pgfscope}%
\begin{pgfscope}%
\pgfsys@transformshift{2.268572in}{0.710740in}%
\pgfsys@useobject{currentmarker}{}%
\end{pgfscope}%
\begin{pgfscope}%
\pgfsys@transformshift{2.268705in}{0.720564in}%
\pgfsys@useobject{currentmarker}{}%
\end{pgfscope}%
\begin{pgfscope}%
\pgfsys@transformshift{2.268839in}{0.699921in}%
\pgfsys@useobject{currentmarker}{}%
\end{pgfscope}%
\begin{pgfscope}%
\pgfsys@transformshift{2.268972in}{0.672164in}%
\pgfsys@useobject{currentmarker}{}%
\end{pgfscope}%
\begin{pgfscope}%
\pgfsys@transformshift{2.269105in}{0.646146in}%
\pgfsys@useobject{currentmarker}{}%
\end{pgfscope}%
\begin{pgfscope}%
\pgfsys@transformshift{2.269238in}{0.733471in}%
\pgfsys@useobject{currentmarker}{}%
\end{pgfscope}%
\begin{pgfscope}%
\pgfsys@transformshift{2.269371in}{0.752755in}%
\pgfsys@useobject{currentmarker}{}%
\end{pgfscope}%
\begin{pgfscope}%
\pgfsys@transformshift{2.269505in}{0.698591in}%
\pgfsys@useobject{currentmarker}{}%
\end{pgfscope}%
\begin{pgfscope}%
\pgfsys@transformshift{2.269638in}{0.681363in}%
\pgfsys@useobject{currentmarker}{}%
\end{pgfscope}%
\begin{pgfscope}%
\pgfsys@transformshift{2.269770in}{0.657057in}%
\pgfsys@useobject{currentmarker}{}%
\end{pgfscope}%
\begin{pgfscope}%
\pgfsys@transformshift{2.269903in}{0.691373in}%
\pgfsys@useobject{currentmarker}{}%
\end{pgfscope}%
\begin{pgfscope}%
\pgfsys@transformshift{2.270036in}{0.719245in}%
\pgfsys@useobject{currentmarker}{}%
\end{pgfscope}%
\begin{pgfscope}%
\pgfsys@transformshift{2.270169in}{0.634740in}%
\pgfsys@useobject{currentmarker}{}%
\end{pgfscope}%
\begin{pgfscope}%
\pgfsys@transformshift{2.270301in}{0.693355in}%
\pgfsys@useobject{currentmarker}{}%
\end{pgfscope}%
\begin{pgfscope}%
\pgfsys@transformshift{2.270434in}{0.700867in}%
\pgfsys@useobject{currentmarker}{}%
\end{pgfscope}%
\begin{pgfscope}%
\pgfsys@transformshift{2.270566in}{0.641875in}%
\pgfsys@useobject{currentmarker}{}%
\end{pgfscope}%
\begin{pgfscope}%
\pgfsys@transformshift{2.270698in}{0.678905in}%
\pgfsys@useobject{currentmarker}{}%
\end{pgfscope}%
\begin{pgfscope}%
\pgfsys@transformshift{2.270831in}{0.696778in}%
\pgfsys@useobject{currentmarker}{}%
\end{pgfscope}%
\begin{pgfscope}%
\pgfsys@transformshift{2.270963in}{0.726094in}%
\pgfsys@useobject{currentmarker}{}%
\end{pgfscope}%
\begin{pgfscope}%
\pgfsys@transformshift{2.271095in}{0.726314in}%
\pgfsys@useobject{currentmarker}{}%
\end{pgfscope}%
\begin{pgfscope}%
\pgfsys@transformshift{2.271227in}{0.655297in}%
\pgfsys@useobject{currentmarker}{}%
\end{pgfscope}%
\begin{pgfscope}%
\pgfsys@transformshift{2.271359in}{0.680052in}%
\pgfsys@useobject{currentmarker}{}%
\end{pgfscope}%
\begin{pgfscope}%
\pgfsys@transformshift{2.271491in}{0.723121in}%
\pgfsys@useobject{currentmarker}{}%
\end{pgfscope}%
\begin{pgfscope}%
\pgfsys@transformshift{2.271623in}{0.702500in}%
\pgfsys@useobject{currentmarker}{}%
\end{pgfscope}%
\begin{pgfscope}%
\pgfsys@transformshift{2.271754in}{0.703693in}%
\pgfsys@useobject{currentmarker}{}%
\end{pgfscope}%
\begin{pgfscope}%
\pgfsys@transformshift{2.271886in}{0.710979in}%
\pgfsys@useobject{currentmarker}{}%
\end{pgfscope}%
\begin{pgfscope}%
\pgfsys@transformshift{2.272018in}{0.705691in}%
\pgfsys@useobject{currentmarker}{}%
\end{pgfscope}%
\begin{pgfscope}%
\pgfsys@transformshift{2.272149in}{0.715745in}%
\pgfsys@useobject{currentmarker}{}%
\end{pgfscope}%
\begin{pgfscope}%
\pgfsys@transformshift{2.272281in}{0.705681in}%
\pgfsys@useobject{currentmarker}{}%
\end{pgfscope}%
\begin{pgfscope}%
\pgfsys@transformshift{2.272412in}{0.648929in}%
\pgfsys@useobject{currentmarker}{}%
\end{pgfscope}%
\begin{pgfscope}%
\pgfsys@transformshift{2.272543in}{0.637630in}%
\pgfsys@useobject{currentmarker}{}%
\end{pgfscope}%
\begin{pgfscope}%
\pgfsys@transformshift{2.272674in}{0.656656in}%
\pgfsys@useobject{currentmarker}{}%
\end{pgfscope}%
\begin{pgfscope}%
\pgfsys@transformshift{2.272805in}{0.680087in}%
\pgfsys@useobject{currentmarker}{}%
\end{pgfscope}%
\begin{pgfscope}%
\pgfsys@transformshift{2.272936in}{0.651217in}%
\pgfsys@useobject{currentmarker}{}%
\end{pgfscope}%
\begin{pgfscope}%
\pgfsys@transformshift{2.273067in}{0.651466in}%
\pgfsys@useobject{currentmarker}{}%
\end{pgfscope}%
\begin{pgfscope}%
\pgfsys@transformshift{2.273198in}{0.677017in}%
\pgfsys@useobject{currentmarker}{}%
\end{pgfscope}%
\begin{pgfscope}%
\pgfsys@transformshift{2.273329in}{0.711601in}%
\pgfsys@useobject{currentmarker}{}%
\end{pgfscope}%
\begin{pgfscope}%
\pgfsys@transformshift{2.273460in}{0.694757in}%
\pgfsys@useobject{currentmarker}{}%
\end{pgfscope}%
\begin{pgfscope}%
\pgfsys@transformshift{2.273590in}{0.657173in}%
\pgfsys@useobject{currentmarker}{}%
\end{pgfscope}%
\begin{pgfscope}%
\pgfsys@transformshift{2.273721in}{0.699199in}%
\pgfsys@useobject{currentmarker}{}%
\end{pgfscope}%
\begin{pgfscope}%
\pgfsys@transformshift{2.273852in}{0.675985in}%
\pgfsys@useobject{currentmarker}{}%
\end{pgfscope}%
\begin{pgfscope}%
\pgfsys@transformshift{2.273982in}{0.701432in}%
\pgfsys@useobject{currentmarker}{}%
\end{pgfscope}%
\begin{pgfscope}%
\pgfsys@transformshift{2.274112in}{0.699791in}%
\pgfsys@useobject{currentmarker}{}%
\end{pgfscope}%
\begin{pgfscope}%
\pgfsys@transformshift{2.274243in}{0.690907in}%
\pgfsys@useobject{currentmarker}{}%
\end{pgfscope}%
\begin{pgfscope}%
\pgfsys@transformshift{2.274373in}{0.655310in}%
\pgfsys@useobject{currentmarker}{}%
\end{pgfscope}%
\begin{pgfscope}%
\pgfsys@transformshift{2.274503in}{0.711355in}%
\pgfsys@useobject{currentmarker}{}%
\end{pgfscope}%
\begin{pgfscope}%
\pgfsys@transformshift{2.274633in}{0.726805in}%
\pgfsys@useobject{currentmarker}{}%
\end{pgfscope}%
\begin{pgfscope}%
\pgfsys@transformshift{2.274763in}{0.669294in}%
\pgfsys@useobject{currentmarker}{}%
\end{pgfscope}%
\begin{pgfscope}%
\pgfsys@transformshift{2.274893in}{0.674684in}%
\pgfsys@useobject{currentmarker}{}%
\end{pgfscope}%
\begin{pgfscope}%
\pgfsys@transformshift{2.275023in}{0.676552in}%
\pgfsys@useobject{currentmarker}{}%
\end{pgfscope}%
\begin{pgfscope}%
\pgfsys@transformshift{2.275152in}{0.645829in}%
\pgfsys@useobject{currentmarker}{}%
\end{pgfscope}%
\begin{pgfscope}%
\pgfsys@transformshift{2.275282in}{0.656811in}%
\pgfsys@useobject{currentmarker}{}%
\end{pgfscope}%
\begin{pgfscope}%
\pgfsys@transformshift{2.275412in}{0.688773in}%
\pgfsys@useobject{currentmarker}{}%
\end{pgfscope}%
\begin{pgfscope}%
\pgfsys@transformshift{2.275541in}{0.756668in}%
\pgfsys@useobject{currentmarker}{}%
\end{pgfscope}%
\begin{pgfscope}%
\pgfsys@transformshift{2.275671in}{0.722126in}%
\pgfsys@useobject{currentmarker}{}%
\end{pgfscope}%
\begin{pgfscope}%
\pgfsys@transformshift{2.275800in}{0.634528in}%
\pgfsys@useobject{currentmarker}{}%
\end{pgfscope}%
\begin{pgfscope}%
\pgfsys@transformshift{2.275929in}{0.638446in}%
\pgfsys@useobject{currentmarker}{}%
\end{pgfscope}%
\begin{pgfscope}%
\pgfsys@transformshift{2.276058in}{0.671414in}%
\pgfsys@useobject{currentmarker}{}%
\end{pgfscope}%
\begin{pgfscope}%
\pgfsys@transformshift{2.276188in}{0.692651in}%
\pgfsys@useobject{currentmarker}{}%
\end{pgfscope}%
\begin{pgfscope}%
\pgfsys@transformshift{2.276317in}{0.680170in}%
\pgfsys@useobject{currentmarker}{}%
\end{pgfscope}%
\begin{pgfscope}%
\pgfsys@transformshift{2.276446in}{0.610918in}%
\pgfsys@useobject{currentmarker}{}%
\end{pgfscope}%
\begin{pgfscope}%
\pgfsys@transformshift{2.276575in}{0.716587in}%
\pgfsys@useobject{currentmarker}{}%
\end{pgfscope}%
\begin{pgfscope}%
\pgfsys@transformshift{2.276703in}{0.774854in}%
\pgfsys@useobject{currentmarker}{}%
\end{pgfscope}%
\begin{pgfscope}%
\pgfsys@transformshift{2.276832in}{0.778969in}%
\pgfsys@useobject{currentmarker}{}%
\end{pgfscope}%
\begin{pgfscope}%
\pgfsys@transformshift{2.276961in}{0.703245in}%
\pgfsys@useobject{currentmarker}{}%
\end{pgfscope}%
\begin{pgfscope}%
\pgfsys@transformshift{2.277090in}{0.629540in}%
\pgfsys@useobject{currentmarker}{}%
\end{pgfscope}%
\begin{pgfscope}%
\pgfsys@transformshift{2.277218in}{0.663104in}%
\pgfsys@useobject{currentmarker}{}%
\end{pgfscope}%
\begin{pgfscope}%
\pgfsys@transformshift{2.277347in}{0.673977in}%
\pgfsys@useobject{currentmarker}{}%
\end{pgfscope}%
\begin{pgfscope}%
\pgfsys@transformshift{2.277475in}{0.672853in}%
\pgfsys@useobject{currentmarker}{}%
\end{pgfscope}%
\begin{pgfscope}%
\pgfsys@transformshift{2.277603in}{0.687483in}%
\pgfsys@useobject{currentmarker}{}%
\end{pgfscope}%
\begin{pgfscope}%
\pgfsys@transformshift{2.277732in}{0.702890in}%
\pgfsys@useobject{currentmarker}{}%
\end{pgfscope}%
\begin{pgfscope}%
\pgfsys@transformshift{2.277860in}{0.721473in}%
\pgfsys@useobject{currentmarker}{}%
\end{pgfscope}%
\begin{pgfscope}%
\pgfsys@transformshift{2.277988in}{0.658342in}%
\pgfsys@useobject{currentmarker}{}%
\end{pgfscope}%
\begin{pgfscope}%
\pgfsys@transformshift{2.278116in}{0.668705in}%
\pgfsys@useobject{currentmarker}{}%
\end{pgfscope}%
\begin{pgfscope}%
\pgfsys@transformshift{2.278244in}{0.693601in}%
\pgfsys@useobject{currentmarker}{}%
\end{pgfscope}%
\begin{pgfscope}%
\pgfsys@transformshift{2.278372in}{0.709506in}%
\pgfsys@useobject{currentmarker}{}%
\end{pgfscope}%
\begin{pgfscope}%
\pgfsys@transformshift{2.278500in}{0.690800in}%
\pgfsys@useobject{currentmarker}{}%
\end{pgfscope}%
\begin{pgfscope}%
\pgfsys@transformshift{2.278627in}{0.671381in}%
\pgfsys@useobject{currentmarker}{}%
\end{pgfscope}%
\begin{pgfscope}%
\pgfsys@transformshift{2.278755in}{0.693400in}%
\pgfsys@useobject{currentmarker}{}%
\end{pgfscope}%
\begin{pgfscope}%
\pgfsys@transformshift{2.278883in}{0.708275in}%
\pgfsys@useobject{currentmarker}{}%
\end{pgfscope}%
\begin{pgfscope}%
\pgfsys@transformshift{2.279010in}{0.711693in}%
\pgfsys@useobject{currentmarker}{}%
\end{pgfscope}%
\begin{pgfscope}%
\pgfsys@transformshift{2.279138in}{0.664223in}%
\pgfsys@useobject{currentmarker}{}%
\end{pgfscope}%
\begin{pgfscope}%
\pgfsys@transformshift{2.279265in}{0.648991in}%
\pgfsys@useobject{currentmarker}{}%
\end{pgfscope}%
\begin{pgfscope}%
\pgfsys@transformshift{2.279392in}{0.683228in}%
\pgfsys@useobject{currentmarker}{}%
\end{pgfscope}%
\begin{pgfscope}%
\pgfsys@transformshift{2.279520in}{0.721054in}%
\pgfsys@useobject{currentmarker}{}%
\end{pgfscope}%
\begin{pgfscope}%
\pgfsys@transformshift{2.279647in}{0.678127in}%
\pgfsys@useobject{currentmarker}{}%
\end{pgfscope}%
\begin{pgfscope}%
\pgfsys@transformshift{2.279774in}{0.635688in}%
\pgfsys@useobject{currentmarker}{}%
\end{pgfscope}%
\begin{pgfscope}%
\pgfsys@transformshift{2.279901in}{0.630343in}%
\pgfsys@useobject{currentmarker}{}%
\end{pgfscope}%
\begin{pgfscope}%
\pgfsys@transformshift{2.280028in}{0.677635in}%
\pgfsys@useobject{currentmarker}{}%
\end{pgfscope}%
\begin{pgfscope}%
\pgfsys@transformshift{2.280155in}{0.665910in}%
\pgfsys@useobject{currentmarker}{}%
\end{pgfscope}%
\begin{pgfscope}%
\pgfsys@transformshift{2.280282in}{0.693542in}%
\pgfsys@useobject{currentmarker}{}%
\end{pgfscope}%
\begin{pgfscope}%
\pgfsys@transformshift{2.280408in}{0.698578in}%
\pgfsys@useobject{currentmarker}{}%
\end{pgfscope}%
\begin{pgfscope}%
\pgfsys@transformshift{2.280535in}{0.711778in}%
\pgfsys@useobject{currentmarker}{}%
\end{pgfscope}%
\begin{pgfscope}%
\pgfsys@transformshift{2.280662in}{0.702844in}%
\pgfsys@useobject{currentmarker}{}%
\end{pgfscope}%
\begin{pgfscope}%
\pgfsys@transformshift{2.280788in}{0.703045in}%
\pgfsys@useobject{currentmarker}{}%
\end{pgfscope}%
\begin{pgfscope}%
\pgfsys@transformshift{2.280915in}{0.731331in}%
\pgfsys@useobject{currentmarker}{}%
\end{pgfscope}%
\begin{pgfscope}%
\pgfsys@transformshift{2.281041in}{0.707720in}%
\pgfsys@useobject{currentmarker}{}%
\end{pgfscope}%
\begin{pgfscope}%
\pgfsys@transformshift{2.281167in}{0.696575in}%
\pgfsys@useobject{currentmarker}{}%
\end{pgfscope}%
\begin{pgfscope}%
\pgfsys@transformshift{2.281294in}{0.703734in}%
\pgfsys@useobject{currentmarker}{}%
\end{pgfscope}%
\begin{pgfscope}%
\pgfsys@transformshift{2.281420in}{0.672502in}%
\pgfsys@useobject{currentmarker}{}%
\end{pgfscope}%
\begin{pgfscope}%
\pgfsys@transformshift{2.281546in}{0.646724in}%
\pgfsys@useobject{currentmarker}{}%
\end{pgfscope}%
\begin{pgfscope}%
\pgfsys@transformshift{2.281672in}{0.719133in}%
\pgfsys@useobject{currentmarker}{}%
\end{pgfscope}%
\begin{pgfscope}%
\pgfsys@transformshift{2.281798in}{0.703440in}%
\pgfsys@useobject{currentmarker}{}%
\end{pgfscope}%
\begin{pgfscope}%
\pgfsys@transformshift{2.281924in}{0.667709in}%
\pgfsys@useobject{currentmarker}{}%
\end{pgfscope}%
\begin{pgfscope}%
\pgfsys@transformshift{2.282050in}{0.660650in}%
\pgfsys@useobject{currentmarker}{}%
\end{pgfscope}%
\begin{pgfscope}%
\pgfsys@transformshift{2.282175in}{0.722714in}%
\pgfsys@useobject{currentmarker}{}%
\end{pgfscope}%
\begin{pgfscope}%
\pgfsys@transformshift{2.282301in}{0.721469in}%
\pgfsys@useobject{currentmarker}{}%
\end{pgfscope}%
\begin{pgfscope}%
\pgfsys@transformshift{2.282427in}{0.723666in}%
\pgfsys@useobject{currentmarker}{}%
\end{pgfscope}%
\begin{pgfscope}%
\pgfsys@transformshift{2.282552in}{0.695727in}%
\pgfsys@useobject{currentmarker}{}%
\end{pgfscope}%
\begin{pgfscope}%
\pgfsys@transformshift{2.282678in}{0.692894in}%
\pgfsys@useobject{currentmarker}{}%
\end{pgfscope}%
\begin{pgfscope}%
\pgfsys@transformshift{2.282803in}{0.662008in}%
\pgfsys@useobject{currentmarker}{}%
\end{pgfscope}%
\begin{pgfscope}%
\pgfsys@transformshift{2.282928in}{0.700781in}%
\pgfsys@useobject{currentmarker}{}%
\end{pgfscope}%
\begin{pgfscope}%
\pgfsys@transformshift{2.283054in}{0.704049in}%
\pgfsys@useobject{currentmarker}{}%
\end{pgfscope}%
\begin{pgfscope}%
\pgfsys@transformshift{2.283179in}{0.678944in}%
\pgfsys@useobject{currentmarker}{}%
\end{pgfscope}%
\begin{pgfscope}%
\pgfsys@transformshift{2.283304in}{0.691229in}%
\pgfsys@useobject{currentmarker}{}%
\end{pgfscope}%
\begin{pgfscope}%
\pgfsys@transformshift{2.283429in}{0.731591in}%
\pgfsys@useobject{currentmarker}{}%
\end{pgfscope}%
\begin{pgfscope}%
\pgfsys@transformshift{2.283554in}{0.701590in}%
\pgfsys@useobject{currentmarker}{}%
\end{pgfscope}%
\begin{pgfscope}%
\pgfsys@transformshift{2.283679in}{0.686443in}%
\pgfsys@useobject{currentmarker}{}%
\end{pgfscope}%
\begin{pgfscope}%
\pgfsys@transformshift{2.283804in}{0.727133in}%
\pgfsys@useobject{currentmarker}{}%
\end{pgfscope}%
\begin{pgfscope}%
\pgfsys@transformshift{2.283928in}{0.693761in}%
\pgfsys@useobject{currentmarker}{}%
\end{pgfscope}%
\begin{pgfscope}%
\pgfsys@transformshift{2.284053in}{0.702641in}%
\pgfsys@useobject{currentmarker}{}%
\end{pgfscope}%
\begin{pgfscope}%
\pgfsys@transformshift{2.284178in}{0.703194in}%
\pgfsys@useobject{currentmarker}{}%
\end{pgfscope}%
\begin{pgfscope}%
\pgfsys@transformshift{2.284302in}{0.714416in}%
\pgfsys@useobject{currentmarker}{}%
\end{pgfscope}%
\begin{pgfscope}%
\pgfsys@transformshift{2.284427in}{0.666375in}%
\pgfsys@useobject{currentmarker}{}%
\end{pgfscope}%
\begin{pgfscope}%
\pgfsys@transformshift{2.284551in}{0.686282in}%
\pgfsys@useobject{currentmarker}{}%
\end{pgfscope}%
\begin{pgfscope}%
\pgfsys@transformshift{2.284676in}{0.654418in}%
\pgfsys@useobject{currentmarker}{}%
\end{pgfscope}%
\begin{pgfscope}%
\pgfsys@transformshift{2.284800in}{0.621316in}%
\pgfsys@useobject{currentmarker}{}%
\end{pgfscope}%
\begin{pgfscope}%
\pgfsys@transformshift{2.284924in}{0.633103in}%
\pgfsys@useobject{currentmarker}{}%
\end{pgfscope}%
\begin{pgfscope}%
\pgfsys@transformshift{2.285048in}{0.639488in}%
\pgfsys@useobject{currentmarker}{}%
\end{pgfscope}%
\begin{pgfscope}%
\pgfsys@transformshift{2.285172in}{0.700319in}%
\pgfsys@useobject{currentmarker}{}%
\end{pgfscope}%
\begin{pgfscope}%
\pgfsys@transformshift{2.285296in}{0.726304in}%
\pgfsys@useobject{currentmarker}{}%
\end{pgfscope}%
\begin{pgfscope}%
\pgfsys@transformshift{2.285420in}{0.710266in}%
\pgfsys@useobject{currentmarker}{}%
\end{pgfscope}%
\begin{pgfscope}%
\pgfsys@transformshift{2.285544in}{0.689100in}%
\pgfsys@useobject{currentmarker}{}%
\end{pgfscope}%
\begin{pgfscope}%
\pgfsys@transformshift{2.285668in}{0.714135in}%
\pgfsys@useobject{currentmarker}{}%
\end{pgfscope}%
\begin{pgfscope}%
\pgfsys@transformshift{2.285792in}{0.709999in}%
\pgfsys@useobject{currentmarker}{}%
\end{pgfscope}%
\begin{pgfscope}%
\pgfsys@transformshift{2.285915in}{0.711591in}%
\pgfsys@useobject{currentmarker}{}%
\end{pgfscope}%
\begin{pgfscope}%
\pgfsys@transformshift{2.286039in}{0.689829in}%
\pgfsys@useobject{currentmarker}{}%
\end{pgfscope}%
\begin{pgfscope}%
\pgfsys@transformshift{2.286163in}{0.719483in}%
\pgfsys@useobject{currentmarker}{}%
\end{pgfscope}%
\begin{pgfscope}%
\pgfsys@transformshift{2.286286in}{0.736449in}%
\pgfsys@useobject{currentmarker}{}%
\end{pgfscope}%
\begin{pgfscope}%
\pgfsys@transformshift{2.286409in}{0.665133in}%
\pgfsys@useobject{currentmarker}{}%
\end{pgfscope}%
\begin{pgfscope}%
\pgfsys@transformshift{2.286533in}{0.687265in}%
\pgfsys@useobject{currentmarker}{}%
\end{pgfscope}%
\begin{pgfscope}%
\pgfsys@transformshift{2.286656in}{0.704019in}%
\pgfsys@useobject{currentmarker}{}%
\end{pgfscope}%
\begin{pgfscope}%
\pgfsys@transformshift{2.286779in}{0.710308in}%
\pgfsys@useobject{currentmarker}{}%
\end{pgfscope}%
\begin{pgfscope}%
\pgfsys@transformshift{2.286902in}{0.728616in}%
\pgfsys@useobject{currentmarker}{}%
\end{pgfscope}%
\begin{pgfscope}%
\pgfsys@transformshift{2.287025in}{0.698451in}%
\pgfsys@useobject{currentmarker}{}%
\end{pgfscope}%
\begin{pgfscope}%
\pgfsys@transformshift{2.287148in}{0.669708in}%
\pgfsys@useobject{currentmarker}{}%
\end{pgfscope}%
\begin{pgfscope}%
\pgfsys@transformshift{2.287271in}{0.647351in}%
\pgfsys@useobject{currentmarker}{}%
\end{pgfscope}%
\begin{pgfscope}%
\pgfsys@transformshift{2.287394in}{0.678809in}%
\pgfsys@useobject{currentmarker}{}%
\end{pgfscope}%
\begin{pgfscope}%
\pgfsys@transformshift{2.287517in}{0.727124in}%
\pgfsys@useobject{currentmarker}{}%
\end{pgfscope}%
\begin{pgfscope}%
\pgfsys@transformshift{2.287640in}{0.694741in}%
\pgfsys@useobject{currentmarker}{}%
\end{pgfscope}%
\begin{pgfscope}%
\pgfsys@transformshift{2.287762in}{0.692225in}%
\pgfsys@useobject{currentmarker}{}%
\end{pgfscope}%
\begin{pgfscope}%
\pgfsys@transformshift{2.287885in}{0.711866in}%
\pgfsys@useobject{currentmarker}{}%
\end{pgfscope}%
\begin{pgfscope}%
\pgfsys@transformshift{2.288007in}{0.698509in}%
\pgfsys@useobject{currentmarker}{}%
\end{pgfscope}%
\begin{pgfscope}%
\pgfsys@transformshift{2.288130in}{0.691019in}%
\pgfsys@useobject{currentmarker}{}%
\end{pgfscope}%
\begin{pgfscope}%
\pgfsys@transformshift{2.288252in}{0.693699in}%
\pgfsys@useobject{currentmarker}{}%
\end{pgfscope}%
\begin{pgfscope}%
\pgfsys@transformshift{2.288375in}{0.661100in}%
\pgfsys@useobject{currentmarker}{}%
\end{pgfscope}%
\begin{pgfscope}%
\pgfsys@transformshift{2.288497in}{0.710186in}%
\pgfsys@useobject{currentmarker}{}%
\end{pgfscope}%
\begin{pgfscope}%
\pgfsys@transformshift{2.288619in}{0.727255in}%
\pgfsys@useobject{currentmarker}{}%
\end{pgfscope}%
\begin{pgfscope}%
\pgfsys@transformshift{2.288741in}{0.695623in}%
\pgfsys@useobject{currentmarker}{}%
\end{pgfscope}%
\begin{pgfscope}%
\pgfsys@transformshift{2.288863in}{0.700188in}%
\pgfsys@useobject{currentmarker}{}%
\end{pgfscope}%
\begin{pgfscope}%
\pgfsys@transformshift{2.288985in}{0.686409in}%
\pgfsys@useobject{currentmarker}{}%
\end{pgfscope}%
\begin{pgfscope}%
\pgfsys@transformshift{2.289107in}{0.668974in}%
\pgfsys@useobject{currentmarker}{}%
\end{pgfscope}%
\begin{pgfscope}%
\pgfsys@transformshift{2.289229in}{0.691811in}%
\pgfsys@useobject{currentmarker}{}%
\end{pgfscope}%
\begin{pgfscope}%
\pgfsys@transformshift{2.289351in}{0.716433in}%
\pgfsys@useobject{currentmarker}{}%
\end{pgfscope}%
\begin{pgfscope}%
\pgfsys@transformshift{2.289473in}{0.706222in}%
\pgfsys@useobject{currentmarker}{}%
\end{pgfscope}%
\begin{pgfscope}%
\pgfsys@transformshift{2.289594in}{0.653794in}%
\pgfsys@useobject{currentmarker}{}%
\end{pgfscope}%
\begin{pgfscope}%
\pgfsys@transformshift{2.289716in}{0.681182in}%
\pgfsys@useobject{currentmarker}{}%
\end{pgfscope}%
\begin{pgfscope}%
\pgfsys@transformshift{2.289837in}{0.755559in}%
\pgfsys@useobject{currentmarker}{}%
\end{pgfscope}%
\begin{pgfscope}%
\pgfsys@transformshift{2.289959in}{0.771019in}%
\pgfsys@useobject{currentmarker}{}%
\end{pgfscope}%
\begin{pgfscope}%
\pgfsys@transformshift{2.290080in}{0.701150in}%
\pgfsys@useobject{currentmarker}{}%
\end{pgfscope}%
\begin{pgfscope}%
\pgfsys@transformshift{2.290202in}{0.714864in}%
\pgfsys@useobject{currentmarker}{}%
\end{pgfscope}%
\begin{pgfscope}%
\pgfsys@transformshift{2.290323in}{0.723854in}%
\pgfsys@useobject{currentmarker}{}%
\end{pgfscope}%
\begin{pgfscope}%
\pgfsys@transformshift{2.290444in}{0.663172in}%
\pgfsys@useobject{currentmarker}{}%
\end{pgfscope}%
\begin{pgfscope}%
\pgfsys@transformshift{2.290565in}{0.723456in}%
\pgfsys@useobject{currentmarker}{}%
\end{pgfscope}%
\begin{pgfscope}%
\pgfsys@transformshift{2.290686in}{0.713712in}%
\pgfsys@useobject{currentmarker}{}%
\end{pgfscope}%
\begin{pgfscope}%
\pgfsys@transformshift{2.290807in}{0.699458in}%
\pgfsys@useobject{currentmarker}{}%
\end{pgfscope}%
\begin{pgfscope}%
\pgfsys@transformshift{2.290928in}{0.701659in}%
\pgfsys@useobject{currentmarker}{}%
\end{pgfscope}%
\begin{pgfscope}%
\pgfsys@transformshift{2.291049in}{0.711828in}%
\pgfsys@useobject{currentmarker}{}%
\end{pgfscope}%
\begin{pgfscope}%
\pgfsys@transformshift{2.291170in}{0.694029in}%
\pgfsys@useobject{currentmarker}{}%
\end{pgfscope}%
\begin{pgfscope}%
\pgfsys@transformshift{2.291291in}{0.731750in}%
\pgfsys@useobject{currentmarker}{}%
\end{pgfscope}%
\begin{pgfscope}%
\pgfsys@transformshift{2.291411in}{0.688668in}%
\pgfsys@useobject{currentmarker}{}%
\end{pgfscope}%
\begin{pgfscope}%
\pgfsys@transformshift{2.291532in}{0.696094in}%
\pgfsys@useobject{currentmarker}{}%
\end{pgfscope}%
\begin{pgfscope}%
\pgfsys@transformshift{2.291653in}{0.718694in}%
\pgfsys@useobject{currentmarker}{}%
\end{pgfscope}%
\begin{pgfscope}%
\pgfsys@transformshift{2.291773in}{0.690980in}%
\pgfsys@useobject{currentmarker}{}%
\end{pgfscope}%
\begin{pgfscope}%
\pgfsys@transformshift{2.291893in}{0.677523in}%
\pgfsys@useobject{currentmarker}{}%
\end{pgfscope}%
\begin{pgfscope}%
\pgfsys@transformshift{2.292014in}{0.674714in}%
\pgfsys@useobject{currentmarker}{}%
\end{pgfscope}%
\begin{pgfscope}%
\pgfsys@transformshift{2.292134in}{0.695218in}%
\pgfsys@useobject{currentmarker}{}%
\end{pgfscope}%
\begin{pgfscope}%
\pgfsys@transformshift{2.292254in}{0.680646in}%
\pgfsys@useobject{currentmarker}{}%
\end{pgfscope}%
\begin{pgfscope}%
\pgfsys@transformshift{2.292374in}{0.670211in}%
\pgfsys@useobject{currentmarker}{}%
\end{pgfscope}%
\begin{pgfscope}%
\pgfsys@transformshift{2.292495in}{0.664592in}%
\pgfsys@useobject{currentmarker}{}%
\end{pgfscope}%
\begin{pgfscope}%
\pgfsys@transformshift{2.292615in}{0.703667in}%
\pgfsys@useobject{currentmarker}{}%
\end{pgfscope}%
\begin{pgfscope}%
\pgfsys@transformshift{2.292735in}{0.702049in}%
\pgfsys@useobject{currentmarker}{}%
\end{pgfscope}%
\begin{pgfscope}%
\pgfsys@transformshift{2.292855in}{0.687211in}%
\pgfsys@useobject{currentmarker}{}%
\end{pgfscope}%
\begin{pgfscope}%
\pgfsys@transformshift{2.292974in}{0.655472in}%
\pgfsys@useobject{currentmarker}{}%
\end{pgfscope}%
\begin{pgfscope}%
\pgfsys@transformshift{2.293094in}{0.647486in}%
\pgfsys@useobject{currentmarker}{}%
\end{pgfscope}%
\begin{pgfscope}%
\pgfsys@transformshift{2.293214in}{0.702027in}%
\pgfsys@useobject{currentmarker}{}%
\end{pgfscope}%
\begin{pgfscope}%
\pgfsys@transformshift{2.293334in}{0.702857in}%
\pgfsys@useobject{currentmarker}{}%
\end{pgfscope}%
\begin{pgfscope}%
\pgfsys@transformshift{2.293453in}{0.668111in}%
\pgfsys@useobject{currentmarker}{}%
\end{pgfscope}%
\begin{pgfscope}%
\pgfsys@transformshift{2.293573in}{0.714585in}%
\pgfsys@useobject{currentmarker}{}%
\end{pgfscope}%
\begin{pgfscope}%
\pgfsys@transformshift{2.293692in}{0.698123in}%
\pgfsys@useobject{currentmarker}{}%
\end{pgfscope}%
\begin{pgfscope}%
\pgfsys@transformshift{2.293812in}{0.668135in}%
\pgfsys@useobject{currentmarker}{}%
\end{pgfscope}%
\begin{pgfscope}%
\pgfsys@transformshift{2.293931in}{0.693478in}%
\pgfsys@useobject{currentmarker}{}%
\end{pgfscope}%
\begin{pgfscope}%
\pgfsys@transformshift{2.294050in}{0.702663in}%
\pgfsys@useobject{currentmarker}{}%
\end{pgfscope}%
\begin{pgfscope}%
\pgfsys@transformshift{2.294169in}{0.712650in}%
\pgfsys@useobject{currentmarker}{}%
\end{pgfscope}%
\begin{pgfscope}%
\pgfsys@transformshift{2.294288in}{0.730968in}%
\pgfsys@useobject{currentmarker}{}%
\end{pgfscope}%
\begin{pgfscope}%
\pgfsys@transformshift{2.294408in}{0.727104in}%
\pgfsys@useobject{currentmarker}{}%
\end{pgfscope}%
\begin{pgfscope}%
\pgfsys@transformshift{2.294527in}{0.692845in}%
\pgfsys@useobject{currentmarker}{}%
\end{pgfscope}%
\begin{pgfscope}%
\pgfsys@transformshift{2.294646in}{0.666759in}%
\pgfsys@useobject{currentmarker}{}%
\end{pgfscope}%
\begin{pgfscope}%
\pgfsys@transformshift{2.294764in}{0.682048in}%
\pgfsys@useobject{currentmarker}{}%
\end{pgfscope}%
\begin{pgfscope}%
\pgfsys@transformshift{2.294883in}{0.668633in}%
\pgfsys@useobject{currentmarker}{}%
\end{pgfscope}%
\begin{pgfscope}%
\pgfsys@transformshift{2.295002in}{0.659203in}%
\pgfsys@useobject{currentmarker}{}%
\end{pgfscope}%
\begin{pgfscope}%
\pgfsys@transformshift{2.295121in}{0.647868in}%
\pgfsys@useobject{currentmarker}{}%
\end{pgfscope}%
\begin{pgfscope}%
\pgfsys@transformshift{2.295239in}{0.701177in}%
\pgfsys@useobject{currentmarker}{}%
\end{pgfscope}%
\begin{pgfscope}%
\pgfsys@transformshift{2.295358in}{0.730335in}%
\pgfsys@useobject{currentmarker}{}%
\end{pgfscope}%
\begin{pgfscope}%
\pgfsys@transformshift{2.295476in}{0.720673in}%
\pgfsys@useobject{currentmarker}{}%
\end{pgfscope}%
\begin{pgfscope}%
\pgfsys@transformshift{2.295595in}{0.676803in}%
\pgfsys@useobject{currentmarker}{}%
\end{pgfscope}%
\begin{pgfscope}%
\pgfsys@transformshift{2.295713in}{0.682513in}%
\pgfsys@useobject{currentmarker}{}%
\end{pgfscope}%
\begin{pgfscope}%
\pgfsys@transformshift{2.295832in}{0.681956in}%
\pgfsys@useobject{currentmarker}{}%
\end{pgfscope}%
\begin{pgfscope}%
\pgfsys@transformshift{2.295950in}{0.700567in}%
\pgfsys@useobject{currentmarker}{}%
\end{pgfscope}%
\begin{pgfscope}%
\pgfsys@transformshift{2.296068in}{0.676486in}%
\pgfsys@useobject{currentmarker}{}%
\end{pgfscope}%
\begin{pgfscope}%
\pgfsys@transformshift{2.296186in}{0.673388in}%
\pgfsys@useobject{currentmarker}{}%
\end{pgfscope}%
\begin{pgfscope}%
\pgfsys@transformshift{2.296304in}{0.662260in}%
\pgfsys@useobject{currentmarker}{}%
\end{pgfscope}%
\begin{pgfscope}%
\pgfsys@transformshift{2.296422in}{0.624237in}%
\pgfsys@useobject{currentmarker}{}%
\end{pgfscope}%
\begin{pgfscope}%
\pgfsys@transformshift{2.296540in}{0.651792in}%
\pgfsys@useobject{currentmarker}{}%
\end{pgfscope}%
\begin{pgfscope}%
\pgfsys@transformshift{2.296658in}{0.712339in}%
\pgfsys@useobject{currentmarker}{}%
\end{pgfscope}%
\begin{pgfscope}%
\pgfsys@transformshift{2.296776in}{0.719236in}%
\pgfsys@useobject{currentmarker}{}%
\end{pgfscope}%
\begin{pgfscope}%
\pgfsys@transformshift{2.296894in}{0.694647in}%
\pgfsys@useobject{currentmarker}{}%
\end{pgfscope}%
\begin{pgfscope}%
\pgfsys@transformshift{2.297012in}{0.686537in}%
\pgfsys@useobject{currentmarker}{}%
\end{pgfscope}%
\begin{pgfscope}%
\pgfsys@transformshift{2.297129in}{0.630181in}%
\pgfsys@useobject{currentmarker}{}%
\end{pgfscope}%
\begin{pgfscope}%
\pgfsys@transformshift{2.297247in}{0.659137in}%
\pgfsys@useobject{currentmarker}{}%
\end{pgfscope}%
\begin{pgfscope}%
\pgfsys@transformshift{2.297364in}{0.711367in}%
\pgfsys@useobject{currentmarker}{}%
\end{pgfscope}%
\begin{pgfscope}%
\pgfsys@transformshift{2.297482in}{0.704890in}%
\pgfsys@useobject{currentmarker}{}%
\end{pgfscope}%
\begin{pgfscope}%
\pgfsys@transformshift{2.297599in}{0.669345in}%
\pgfsys@useobject{currentmarker}{}%
\end{pgfscope}%
\begin{pgfscope}%
\pgfsys@transformshift{2.297717in}{0.660290in}%
\pgfsys@useobject{currentmarker}{}%
\end{pgfscope}%
\begin{pgfscope}%
\pgfsys@transformshift{2.297834in}{0.669023in}%
\pgfsys@useobject{currentmarker}{}%
\end{pgfscope}%
\begin{pgfscope}%
\pgfsys@transformshift{2.297951in}{0.684629in}%
\pgfsys@useobject{currentmarker}{}%
\end{pgfscope}%
\begin{pgfscope}%
\pgfsys@transformshift{2.298068in}{0.674896in}%
\pgfsys@useobject{currentmarker}{}%
\end{pgfscope}%
\begin{pgfscope}%
\pgfsys@transformshift{2.298185in}{0.681418in}%
\pgfsys@useobject{currentmarker}{}%
\end{pgfscope}%
\begin{pgfscope}%
\pgfsys@transformshift{2.298302in}{0.700216in}%
\pgfsys@useobject{currentmarker}{}%
\end{pgfscope}%
\begin{pgfscope}%
\pgfsys@transformshift{2.298419in}{0.644460in}%
\pgfsys@useobject{currentmarker}{}%
\end{pgfscope}%
\begin{pgfscope}%
\pgfsys@transformshift{2.298536in}{0.657686in}%
\pgfsys@useobject{currentmarker}{}%
\end{pgfscope}%
\begin{pgfscope}%
\pgfsys@transformshift{2.298653in}{0.651396in}%
\pgfsys@useobject{currentmarker}{}%
\end{pgfscope}%
\begin{pgfscope}%
\pgfsys@transformshift{2.298770in}{0.647251in}%
\pgfsys@useobject{currentmarker}{}%
\end{pgfscope}%
\begin{pgfscope}%
\pgfsys@transformshift{2.298887in}{0.698020in}%
\pgfsys@useobject{currentmarker}{}%
\end{pgfscope}%
\begin{pgfscope}%
\pgfsys@transformshift{2.299003in}{0.681690in}%
\pgfsys@useobject{currentmarker}{}%
\end{pgfscope}%
\begin{pgfscope}%
\pgfsys@transformshift{2.299120in}{0.679153in}%
\pgfsys@useobject{currentmarker}{}%
\end{pgfscope}%
\begin{pgfscope}%
\pgfsys@transformshift{2.299236in}{0.711757in}%
\pgfsys@useobject{currentmarker}{}%
\end{pgfscope}%
\begin{pgfscope}%
\pgfsys@transformshift{2.299353in}{0.713783in}%
\pgfsys@useobject{currentmarker}{}%
\end{pgfscope}%
\begin{pgfscope}%
\pgfsys@transformshift{2.299469in}{0.713424in}%
\pgfsys@useobject{currentmarker}{}%
\end{pgfscope}%
\begin{pgfscope}%
\pgfsys@transformshift{2.299586in}{0.722065in}%
\pgfsys@useobject{currentmarker}{}%
\end{pgfscope}%
\begin{pgfscope}%
\pgfsys@transformshift{2.299702in}{0.700110in}%
\pgfsys@useobject{currentmarker}{}%
\end{pgfscope}%
\begin{pgfscope}%
\pgfsys@transformshift{2.299818in}{0.695800in}%
\pgfsys@useobject{currentmarker}{}%
\end{pgfscope}%
\begin{pgfscope}%
\pgfsys@transformshift{2.299934in}{0.660623in}%
\pgfsys@useobject{currentmarker}{}%
\end{pgfscope}%
\begin{pgfscope}%
\pgfsys@transformshift{2.300051in}{0.708406in}%
\pgfsys@useobject{currentmarker}{}%
\end{pgfscope}%
\begin{pgfscope}%
\pgfsys@transformshift{2.300167in}{0.704480in}%
\pgfsys@useobject{currentmarker}{}%
\end{pgfscope}%
\begin{pgfscope}%
\pgfsys@transformshift{2.300283in}{0.663695in}%
\pgfsys@useobject{currentmarker}{}%
\end{pgfscope}%
\begin{pgfscope}%
\pgfsys@transformshift{2.300399in}{0.690357in}%
\pgfsys@useobject{currentmarker}{}%
\end{pgfscope}%
\begin{pgfscope}%
\pgfsys@transformshift{2.300515in}{0.689635in}%
\pgfsys@useobject{currentmarker}{}%
\end{pgfscope}%
\begin{pgfscope}%
\pgfsys@transformshift{2.300630in}{0.610372in}%
\pgfsys@useobject{currentmarker}{}%
\end{pgfscope}%
\begin{pgfscope}%
\pgfsys@transformshift{2.300746in}{0.669117in}%
\pgfsys@useobject{currentmarker}{}%
\end{pgfscope}%
\begin{pgfscope}%
\pgfsys@transformshift{2.300862in}{0.672333in}%
\pgfsys@useobject{currentmarker}{}%
\end{pgfscope}%
\begin{pgfscope}%
\pgfsys@transformshift{2.300977in}{0.652687in}%
\pgfsys@useobject{currentmarker}{}%
\end{pgfscope}%
\begin{pgfscope}%
\pgfsys@transformshift{2.301093in}{0.688972in}%
\pgfsys@useobject{currentmarker}{}%
\end{pgfscope}%
\begin{pgfscope}%
\pgfsys@transformshift{2.301209in}{0.690778in}%
\pgfsys@useobject{currentmarker}{}%
\end{pgfscope}%
\begin{pgfscope}%
\pgfsys@transformshift{2.301324in}{0.703628in}%
\pgfsys@useobject{currentmarker}{}%
\end{pgfscope}%
\begin{pgfscope}%
\pgfsys@transformshift{2.301439in}{0.650830in}%
\pgfsys@useobject{currentmarker}{}%
\end{pgfscope}%
\begin{pgfscope}%
\pgfsys@transformshift{2.301555in}{0.711084in}%
\pgfsys@useobject{currentmarker}{}%
\end{pgfscope}%
\begin{pgfscope}%
\pgfsys@transformshift{2.301670in}{0.725936in}%
\pgfsys@useobject{currentmarker}{}%
\end{pgfscope}%
\begin{pgfscope}%
\pgfsys@transformshift{2.301785in}{0.677206in}%
\pgfsys@useobject{currentmarker}{}%
\end{pgfscope}%
\begin{pgfscope}%
\pgfsys@transformshift{2.301901in}{0.656372in}%
\pgfsys@useobject{currentmarker}{}%
\end{pgfscope}%
\begin{pgfscope}%
\pgfsys@transformshift{2.302016in}{0.611925in}%
\pgfsys@useobject{currentmarker}{}%
\end{pgfscope}%
\begin{pgfscope}%
\pgfsys@transformshift{2.302131in}{0.647378in}%
\pgfsys@useobject{currentmarker}{}%
\end{pgfscope}%
\begin{pgfscope}%
\pgfsys@transformshift{2.302246in}{0.702354in}%
\pgfsys@useobject{currentmarker}{}%
\end{pgfscope}%
\begin{pgfscope}%
\pgfsys@transformshift{2.302361in}{0.691732in}%
\pgfsys@useobject{currentmarker}{}%
\end{pgfscope}%
\begin{pgfscope}%
\pgfsys@transformshift{2.302476in}{0.744008in}%
\pgfsys@useobject{currentmarker}{}%
\end{pgfscope}%
\begin{pgfscope}%
\pgfsys@transformshift{2.302590in}{0.724019in}%
\pgfsys@useobject{currentmarker}{}%
\end{pgfscope}%
\begin{pgfscope}%
\pgfsys@transformshift{2.302705in}{0.676087in}%
\pgfsys@useobject{currentmarker}{}%
\end{pgfscope}%
\begin{pgfscope}%
\pgfsys@transformshift{2.302820in}{0.701958in}%
\pgfsys@useobject{currentmarker}{}%
\end{pgfscope}%
\begin{pgfscope}%
\pgfsys@transformshift{2.302934in}{0.691789in}%
\pgfsys@useobject{currentmarker}{}%
\end{pgfscope}%
\begin{pgfscope}%
\pgfsys@transformshift{2.303049in}{0.678858in}%
\pgfsys@useobject{currentmarker}{}%
\end{pgfscope}%
\begin{pgfscope}%
\pgfsys@transformshift{2.303164in}{0.695758in}%
\pgfsys@useobject{currentmarker}{}%
\end{pgfscope}%
\begin{pgfscope}%
\pgfsys@transformshift{2.303278in}{0.674349in}%
\pgfsys@useobject{currentmarker}{}%
\end{pgfscope}%
\begin{pgfscope}%
\pgfsys@transformshift{2.303392in}{0.694564in}%
\pgfsys@useobject{currentmarker}{}%
\end{pgfscope}%
\begin{pgfscope}%
\pgfsys@transformshift{2.303507in}{0.668463in}%
\pgfsys@useobject{currentmarker}{}%
\end{pgfscope}%
\begin{pgfscope}%
\pgfsys@transformshift{2.303621in}{0.693891in}%
\pgfsys@useobject{currentmarker}{}%
\end{pgfscope}%
\begin{pgfscope}%
\pgfsys@transformshift{2.303735in}{0.684243in}%
\pgfsys@useobject{currentmarker}{}%
\end{pgfscope}%
\begin{pgfscope}%
\pgfsys@transformshift{2.303850in}{0.681860in}%
\pgfsys@useobject{currentmarker}{}%
\end{pgfscope}%
\begin{pgfscope}%
\pgfsys@transformshift{2.303964in}{0.682685in}%
\pgfsys@useobject{currentmarker}{}%
\end{pgfscope}%
\begin{pgfscope}%
\pgfsys@transformshift{2.304078in}{0.702691in}%
\pgfsys@useobject{currentmarker}{}%
\end{pgfscope}%
\begin{pgfscope}%
\pgfsys@transformshift{2.304192in}{0.683335in}%
\pgfsys@useobject{currentmarker}{}%
\end{pgfscope}%
\begin{pgfscope}%
\pgfsys@transformshift{2.304306in}{0.685018in}%
\pgfsys@useobject{currentmarker}{}%
\end{pgfscope}%
\begin{pgfscope}%
\pgfsys@transformshift{2.304420in}{0.657241in}%
\pgfsys@useobject{currentmarker}{}%
\end{pgfscope}%
\begin{pgfscope}%
\pgfsys@transformshift{2.304533in}{0.620030in}%
\pgfsys@useobject{currentmarker}{}%
\end{pgfscope}%
\begin{pgfscope}%
\pgfsys@transformshift{2.304647in}{0.647755in}%
\pgfsys@useobject{currentmarker}{}%
\end{pgfscope}%
\begin{pgfscope}%
\pgfsys@transformshift{2.304761in}{0.663109in}%
\pgfsys@useobject{currentmarker}{}%
\end{pgfscope}%
\begin{pgfscope}%
\pgfsys@transformshift{2.304875in}{0.673240in}%
\pgfsys@useobject{currentmarker}{}%
\end{pgfscope}%
\begin{pgfscope}%
\pgfsys@transformshift{2.304988in}{0.685274in}%
\pgfsys@useobject{currentmarker}{}%
\end{pgfscope}%
\begin{pgfscope}%
\pgfsys@transformshift{2.305102in}{0.706652in}%
\pgfsys@useobject{currentmarker}{}%
\end{pgfscope}%
\begin{pgfscope}%
\pgfsys@transformshift{2.305215in}{0.703773in}%
\pgfsys@useobject{currentmarker}{}%
\end{pgfscope}%
\begin{pgfscope}%
\pgfsys@transformshift{2.305329in}{0.655876in}%
\pgfsys@useobject{currentmarker}{}%
\end{pgfscope}%
\begin{pgfscope}%
\pgfsys@transformshift{2.305442in}{0.656922in}%
\pgfsys@useobject{currentmarker}{}%
\end{pgfscope}%
\begin{pgfscope}%
\pgfsys@transformshift{2.305555in}{0.686950in}%
\pgfsys@useobject{currentmarker}{}%
\end{pgfscope}%
\begin{pgfscope}%
\pgfsys@transformshift{2.305669in}{0.722768in}%
\pgfsys@useobject{currentmarker}{}%
\end{pgfscope}%
\begin{pgfscope}%
\pgfsys@transformshift{2.305782in}{0.682267in}%
\pgfsys@useobject{currentmarker}{}%
\end{pgfscope}%
\begin{pgfscope}%
\pgfsys@transformshift{2.305895in}{0.720518in}%
\pgfsys@useobject{currentmarker}{}%
\end{pgfscope}%
\begin{pgfscope}%
\pgfsys@transformshift{2.306008in}{0.670210in}%
\pgfsys@useobject{currentmarker}{}%
\end{pgfscope}%
\begin{pgfscope}%
\pgfsys@transformshift{2.306121in}{0.671884in}%
\pgfsys@useobject{currentmarker}{}%
\end{pgfscope}%
\begin{pgfscope}%
\pgfsys@transformshift{2.306234in}{0.672703in}%
\pgfsys@useobject{currentmarker}{}%
\end{pgfscope}%
\begin{pgfscope}%
\pgfsys@transformshift{2.306347in}{0.635077in}%
\pgfsys@useobject{currentmarker}{}%
\end{pgfscope}%
\begin{pgfscope}%
\pgfsys@transformshift{2.306460in}{0.607463in}%
\pgfsys@useobject{currentmarker}{}%
\end{pgfscope}%
\begin{pgfscope}%
\pgfsys@transformshift{2.306573in}{0.691357in}%
\pgfsys@useobject{currentmarker}{}%
\end{pgfscope}%
\begin{pgfscope}%
\pgfsys@transformshift{2.306685in}{0.694229in}%
\pgfsys@useobject{currentmarker}{}%
\end{pgfscope}%
\begin{pgfscope}%
\pgfsys@transformshift{2.306798in}{0.666563in}%
\pgfsys@useobject{currentmarker}{}%
\end{pgfscope}%
\begin{pgfscope}%
\pgfsys@transformshift{2.306911in}{0.664029in}%
\pgfsys@useobject{currentmarker}{}%
\end{pgfscope}%
\begin{pgfscope}%
\pgfsys@transformshift{2.307023in}{0.689675in}%
\pgfsys@useobject{currentmarker}{}%
\end{pgfscope}%
\begin{pgfscope}%
\pgfsys@transformshift{2.307136in}{0.721194in}%
\pgfsys@useobject{currentmarker}{}%
\end{pgfscope}%
\begin{pgfscope}%
\pgfsys@transformshift{2.307248in}{0.741196in}%
\pgfsys@useobject{currentmarker}{}%
\end{pgfscope}%
\begin{pgfscope}%
\pgfsys@transformshift{2.307361in}{0.711213in}%
\pgfsys@useobject{currentmarker}{}%
\end{pgfscope}%
\begin{pgfscope}%
\pgfsys@transformshift{2.307473in}{0.708915in}%
\pgfsys@useobject{currentmarker}{}%
\end{pgfscope}%
\begin{pgfscope}%
\pgfsys@transformshift{2.307585in}{0.704658in}%
\pgfsys@useobject{currentmarker}{}%
\end{pgfscope}%
\begin{pgfscope}%
\pgfsys@transformshift{2.307698in}{0.683894in}%
\pgfsys@useobject{currentmarker}{}%
\end{pgfscope}%
\begin{pgfscope}%
\pgfsys@transformshift{2.307810in}{0.700205in}%
\pgfsys@useobject{currentmarker}{}%
\end{pgfscope}%
\begin{pgfscope}%
\pgfsys@transformshift{2.307922in}{0.632312in}%
\pgfsys@useobject{currentmarker}{}%
\end{pgfscope}%
\begin{pgfscope}%
\pgfsys@transformshift{2.308034in}{0.654943in}%
\pgfsys@useobject{currentmarker}{}%
\end{pgfscope}%
\begin{pgfscope}%
\pgfsys@transformshift{2.308146in}{0.655636in}%
\pgfsys@useobject{currentmarker}{}%
\end{pgfscope}%
\begin{pgfscope}%
\pgfsys@transformshift{2.308258in}{0.676511in}%
\pgfsys@useobject{currentmarker}{}%
\end{pgfscope}%
\begin{pgfscope}%
\pgfsys@transformshift{2.308370in}{0.688069in}%
\pgfsys@useobject{currentmarker}{}%
\end{pgfscope}%
\begin{pgfscope}%
\pgfsys@transformshift{2.308482in}{0.669588in}%
\pgfsys@useobject{currentmarker}{}%
\end{pgfscope}%
\begin{pgfscope}%
\pgfsys@transformshift{2.308594in}{0.687506in}%
\pgfsys@useobject{currentmarker}{}%
\end{pgfscope}%
\begin{pgfscope}%
\pgfsys@transformshift{2.308705in}{0.695060in}%
\pgfsys@useobject{currentmarker}{}%
\end{pgfscope}%
\begin{pgfscope}%
\pgfsys@transformshift{2.308817in}{0.637399in}%
\pgfsys@useobject{currentmarker}{}%
\end{pgfscope}%
\begin{pgfscope}%
\pgfsys@transformshift{2.308929in}{0.732979in}%
\pgfsys@useobject{currentmarker}{}%
\end{pgfscope}%
\begin{pgfscope}%
\pgfsys@transformshift{2.309040in}{0.713554in}%
\pgfsys@useobject{currentmarker}{}%
\end{pgfscope}%
\begin{pgfscope}%
\pgfsys@transformshift{2.309152in}{0.710393in}%
\pgfsys@useobject{currentmarker}{}%
\end{pgfscope}%
\begin{pgfscope}%
\pgfsys@transformshift{2.309263in}{0.721601in}%
\pgfsys@useobject{currentmarker}{}%
\end{pgfscope}%
\begin{pgfscope}%
\pgfsys@transformshift{2.309375in}{0.670824in}%
\pgfsys@useobject{currentmarker}{}%
\end{pgfscope}%
\begin{pgfscope}%
\pgfsys@transformshift{2.309486in}{0.698575in}%
\pgfsys@useobject{currentmarker}{}%
\end{pgfscope}%
\begin{pgfscope}%
\pgfsys@transformshift{2.309597in}{0.712754in}%
\pgfsys@useobject{currentmarker}{}%
\end{pgfscope}%
\begin{pgfscope}%
\pgfsys@transformshift{2.309708in}{0.659761in}%
\pgfsys@useobject{currentmarker}{}%
\end{pgfscope}%
\begin{pgfscope}%
\pgfsys@transformshift{2.309820in}{0.648335in}%
\pgfsys@useobject{currentmarker}{}%
\end{pgfscope}%
\begin{pgfscope}%
\pgfsys@transformshift{2.309931in}{0.699341in}%
\pgfsys@useobject{currentmarker}{}%
\end{pgfscope}%
\begin{pgfscope}%
\pgfsys@transformshift{2.310042in}{0.654900in}%
\pgfsys@useobject{currentmarker}{}%
\end{pgfscope}%
\begin{pgfscope}%
\pgfsys@transformshift{2.310153in}{0.626986in}%
\pgfsys@useobject{currentmarker}{}%
\end{pgfscope}%
\begin{pgfscope}%
\pgfsys@transformshift{2.310264in}{0.692107in}%
\pgfsys@useobject{currentmarker}{}%
\end{pgfscope}%
\begin{pgfscope}%
\pgfsys@transformshift{2.310375in}{0.692365in}%
\pgfsys@useobject{currentmarker}{}%
\end{pgfscope}%
\begin{pgfscope}%
\pgfsys@transformshift{2.310486in}{0.702024in}%
\pgfsys@useobject{currentmarker}{}%
\end{pgfscope}%
\begin{pgfscope}%
\pgfsys@transformshift{2.310596in}{0.624117in}%
\pgfsys@useobject{currentmarker}{}%
\end{pgfscope}%
\begin{pgfscope}%
\pgfsys@transformshift{2.310707in}{0.653013in}%
\pgfsys@useobject{currentmarker}{}%
\end{pgfscope}%
\begin{pgfscope}%
\pgfsys@transformshift{2.310818in}{0.651099in}%
\pgfsys@useobject{currentmarker}{}%
\end{pgfscope}%
\begin{pgfscope}%
\pgfsys@transformshift{2.310928in}{0.664690in}%
\pgfsys@useobject{currentmarker}{}%
\end{pgfscope}%
\begin{pgfscope}%
\pgfsys@transformshift{2.311039in}{0.663477in}%
\pgfsys@useobject{currentmarker}{}%
\end{pgfscope}%
\begin{pgfscope}%
\pgfsys@transformshift{2.311150in}{0.674024in}%
\pgfsys@useobject{currentmarker}{}%
\end{pgfscope}%
\begin{pgfscope}%
\pgfsys@transformshift{2.311260in}{0.623809in}%
\pgfsys@useobject{currentmarker}{}%
\end{pgfscope}%
\begin{pgfscope}%
\pgfsys@transformshift{2.311370in}{0.676013in}%
\pgfsys@useobject{currentmarker}{}%
\end{pgfscope}%
\begin{pgfscope}%
\pgfsys@transformshift{2.311481in}{0.662475in}%
\pgfsys@useobject{currentmarker}{}%
\end{pgfscope}%
\begin{pgfscope}%
\pgfsys@transformshift{2.311591in}{0.669260in}%
\pgfsys@useobject{currentmarker}{}%
\end{pgfscope}%
\begin{pgfscope}%
\pgfsys@transformshift{2.311701in}{0.668396in}%
\pgfsys@useobject{currentmarker}{}%
\end{pgfscope}%
\begin{pgfscope}%
\pgfsys@transformshift{2.311812in}{0.705033in}%
\pgfsys@useobject{currentmarker}{}%
\end{pgfscope}%
\begin{pgfscope}%
\pgfsys@transformshift{2.311922in}{0.697084in}%
\pgfsys@useobject{currentmarker}{}%
\end{pgfscope}%
\begin{pgfscope}%
\pgfsys@transformshift{2.312032in}{0.698491in}%
\pgfsys@useobject{currentmarker}{}%
\end{pgfscope}%
\begin{pgfscope}%
\pgfsys@transformshift{2.312142in}{0.736602in}%
\pgfsys@useobject{currentmarker}{}%
\end{pgfscope}%
\begin{pgfscope}%
\pgfsys@transformshift{2.312252in}{0.725863in}%
\pgfsys@useobject{currentmarker}{}%
\end{pgfscope}%
\begin{pgfscope}%
\pgfsys@transformshift{2.312362in}{0.740974in}%
\pgfsys@useobject{currentmarker}{}%
\end{pgfscope}%
\begin{pgfscope}%
\pgfsys@transformshift{2.312472in}{0.724632in}%
\pgfsys@useobject{currentmarker}{}%
\end{pgfscope}%
\begin{pgfscope}%
\pgfsys@transformshift{2.312582in}{0.688752in}%
\pgfsys@useobject{currentmarker}{}%
\end{pgfscope}%
\begin{pgfscope}%
\pgfsys@transformshift{2.312691in}{0.657374in}%
\pgfsys@useobject{currentmarker}{}%
\end{pgfscope}%
\end{pgfscope}%
\begin{pgfscope}%
\pgfsetrectcap%
\pgfsetmiterjoin%
\pgfsetlinewidth{0.803000pt}%
\definecolor{currentstroke}{rgb}{0.000000,0.000000,0.000000}%
\pgfsetstrokecolor{currentstroke}%
\pgfsetdash{}{0pt}%
\pgfpathmoveto{\pgfqpoint{0.514278in}{0.417642in}}%
\pgfpathlineto{\pgfqpoint{0.514278in}{1.788330in}}%
\pgfusepath{stroke}%
\end{pgfscope}%
\begin{pgfscope}%
\pgfsetrectcap%
\pgfsetmiterjoin%
\pgfsetlinewidth{0.803000pt}%
\definecolor{currentstroke}{rgb}{0.000000,0.000000,0.000000}%
\pgfsetstrokecolor{currentstroke}%
\pgfsetdash{}{0pt}%
\pgfpathmoveto{\pgfqpoint{2.398330in}{0.417642in}}%
\pgfpathlineto{\pgfqpoint{2.398330in}{1.788330in}}%
\pgfusepath{stroke}%
\end{pgfscope}%
\begin{pgfscope}%
\pgfsetrectcap%
\pgfsetmiterjoin%
\pgfsetlinewidth{0.803000pt}%
\definecolor{currentstroke}{rgb}{0.000000,0.000000,0.000000}%
\pgfsetstrokecolor{currentstroke}%
\pgfsetdash{}{0pt}%
\pgfpathmoveto{\pgfqpoint{0.514278in}{0.417642in}}%
\pgfpathlineto{\pgfqpoint{2.398330in}{0.417642in}}%
\pgfusepath{stroke}%
\end{pgfscope}%
\begin{pgfscope}%
\pgfsetrectcap%
\pgfsetmiterjoin%
\pgfsetlinewidth{0.803000pt}%
\definecolor{currentstroke}{rgb}{0.000000,0.000000,0.000000}%
\pgfsetstrokecolor{currentstroke}%
\pgfsetdash{}{0pt}%
\pgfpathmoveto{\pgfqpoint{0.514278in}{1.788330in}}%
\pgfpathlineto{\pgfqpoint{2.398330in}{1.788330in}}%
\pgfusepath{stroke}%
\end{pgfscope}%
\begin{pgfscope}%
\pgfsetbuttcap%
\pgfsetmiterjoin%
\definecolor{currentfill}{rgb}{1.000000,1.000000,1.000000}%
\pgfsetfillcolor{currentfill}%
\pgfsetfillopacity{0.800000}%
\pgfsetlinewidth{1.003750pt}%
\definecolor{currentstroke}{rgb}{0.800000,0.800000,0.800000}%
\pgfsetstrokecolor{currentstroke}%
\pgfsetstrokeopacity{0.800000}%
\pgfsetdash{}{0pt}%
\pgfpathmoveto{\pgfqpoint{1.712264in}{1.522420in}}%
\pgfpathlineto{\pgfqpoint{2.320552in}{1.522420in}}%
\pgfpathquadraticcurveto{\pgfqpoint{2.342774in}{1.522420in}}{\pgfqpoint{2.342774in}{1.544642in}}%
\pgfpathlineto{\pgfqpoint{2.342774in}{1.710552in}}%
\pgfpathquadraticcurveto{\pgfqpoint{2.342774in}{1.732774in}}{\pgfqpoint{2.320552in}{1.732774in}}%
\pgfpathlineto{\pgfqpoint{1.712264in}{1.732774in}}%
\pgfpathquadraticcurveto{\pgfqpoint{1.690042in}{1.732774in}}{\pgfqpoint{1.690042in}{1.710552in}}%
\pgfpathlineto{\pgfqpoint{1.690042in}{1.544642in}}%
\pgfpathquadraticcurveto{\pgfqpoint{1.690042in}{1.522420in}}{\pgfqpoint{1.712264in}{1.522420in}}%
\pgfpathlineto{\pgfqpoint{1.712264in}{1.522420in}}%
\pgfpathclose%
\pgfusepath{stroke,fill}%
\end{pgfscope}%
\begin{pgfscope}%
\pgfsetbuttcap%
\pgfsetroundjoin%
\pgfsetlinewidth{1.505625pt}%
\definecolor{currentstroke}{rgb}{0.000000,0.447059,0.698039}%
\pgfsetstrokecolor{currentstroke}%
\pgfsetdash{{5.550000pt}{2.400000pt}}{0.000000pt}%
\pgfpathmoveto{\pgfqpoint{1.734486in}{1.627358in}}%
\pgfpathlineto{\pgfqpoint{1.845597in}{1.627358in}}%
\pgfpathlineto{\pgfqpoint{1.956708in}{1.627358in}}%
\pgfusepath{stroke}%
\end{pgfscope}%
\begin{pgfscope}%
\definecolor{textcolor}{rgb}{0.000000,0.000000,0.000000}%
\pgfsetstrokecolor{textcolor}%
\pgfsetfillcolor{textcolor}%
\pgftext[x=2.045597in,y=1.588469in,left,base]{\color{textcolor}\rmfamily\fontsize{8.000000}{9.600000}\selectfont \(\displaystyle h_{0}f^{0}\)}%
\end{pgfscope}%
\end{pgfpicture}%
\makeatother%
\endgroup%

        } % scalebox
        \caption{Power spectral density}
        \label{fig:white_noise_psd}
    \end{subfigure}
    \begin{subfigure}{0.3\linewidth}
        \scalebox{0.75}{%
            %% Creator: Matplotlib, PGF backend
%%
%% To include the figure in your LaTeX document, write
%%   \input{<filename>.pgf}
%%
%% Make sure the required packages are loaded in your preamble
%%   \usepackage{pgf}
%%
%% Also ensure that all the required font packages are loaded; for instance,
%% the lmodern package is sometimes necessary when using math font.
%%   \usepackage{lmodern}
%%
%% Figures using additional raster images can only be included by \input if
%% they are in the same directory as the main LaTeX file. For loading figures
%% from other directories you can use the `import` package
%%   \usepackage{import}
%%
%% and then include the figures with
%%   \import{<path to file>}{<filename>.pgf}
%%
%% Matplotlib used the following preamble
%%   \usepackage{siunitx}
%%   \usepackage{fontspec}
%%   \makeatletter\@ifpackageloaded{underscore}{}{\usepackage[strings]{underscore}}\makeatother
%%
\begingroup%
\makeatletter%
\begin{pgfpicture}%
\pgfpathrectangle{\pgfpointorigin}{\pgfqpoint{2.440000in}{1.830000in}}%
\pgfusepath{use as bounding box, clip}%
\begin{pgfscope}%
\pgfsetbuttcap%
\pgfsetmiterjoin%
\definecolor{currentfill}{rgb}{1.000000,1.000000,1.000000}%
\pgfsetfillcolor{currentfill}%
\pgfsetlinewidth{0.000000pt}%
\definecolor{currentstroke}{rgb}{1.000000,1.000000,1.000000}%
\pgfsetstrokecolor{currentstroke}%
\pgfsetdash{}{0pt}%
\pgfpathmoveto{\pgfqpoint{0.000000in}{0.000000in}}%
\pgfpathlineto{\pgfqpoint{2.440000in}{0.000000in}}%
\pgfpathlineto{\pgfqpoint{2.440000in}{1.830000in}}%
\pgfpathlineto{\pgfqpoint{0.000000in}{1.830000in}}%
\pgfpathlineto{\pgfqpoint{0.000000in}{0.000000in}}%
\pgfpathclose%
\pgfusepath{fill}%
\end{pgfscope}%
\begin{pgfscope}%
\pgfsetbuttcap%
\pgfsetmiterjoin%
\definecolor{currentfill}{rgb}{1.000000,1.000000,1.000000}%
\pgfsetfillcolor{currentfill}%
\pgfsetlinewidth{0.000000pt}%
\definecolor{currentstroke}{rgb}{0.000000,0.000000,0.000000}%
\pgfsetstrokecolor{currentstroke}%
\pgfsetstrokeopacity{0.000000}%
\pgfsetdash{}{0pt}%
\pgfpathmoveto{\pgfqpoint{0.589510in}{0.417642in}}%
\pgfpathlineto{\pgfqpoint{2.398330in}{0.417642in}}%
\pgfpathlineto{\pgfqpoint{2.398330in}{1.788330in}}%
\pgfpathlineto{\pgfqpoint{0.589510in}{1.788330in}}%
\pgfpathlineto{\pgfqpoint{0.589510in}{0.417642in}}%
\pgfpathclose%
\pgfusepath{fill}%
\end{pgfscope}%
\begin{pgfscope}%
\pgfpathrectangle{\pgfqpoint{0.589510in}{0.417642in}}{\pgfqpoint{1.808820in}{1.370688in}}%
\pgfusepath{clip}%
\pgfsetrectcap%
\pgfsetroundjoin%
\pgfsetlinewidth{0.803000pt}%
\definecolor{currentstroke}{rgb}{0.450000,0.450000,0.450000}%
\pgfsetstrokecolor{currentstroke}%
\pgfsetdash{}{0pt}%
\pgfpathmoveto{\pgfqpoint{0.671729in}{0.417642in}}%
\pgfpathlineto{\pgfqpoint{0.671729in}{1.788330in}}%
\pgfusepath{stroke}%
\end{pgfscope}%
\begin{pgfscope}%
\pgfsetbuttcap%
\pgfsetroundjoin%
\definecolor{currentfill}{rgb}{0.000000,0.000000,0.000000}%
\pgfsetfillcolor{currentfill}%
\pgfsetlinewidth{0.803000pt}%
\definecolor{currentstroke}{rgb}{0.000000,0.000000,0.000000}%
\pgfsetstrokecolor{currentstroke}%
\pgfsetdash{}{0pt}%
\pgfsys@defobject{currentmarker}{\pgfqpoint{0.000000in}{-0.048611in}}{\pgfqpoint{0.000000in}{0.000000in}}{%
\pgfpathmoveto{\pgfqpoint{0.000000in}{0.000000in}}%
\pgfpathlineto{\pgfqpoint{0.000000in}{-0.048611in}}%
\pgfusepath{stroke,fill}%
}%
\begin{pgfscope}%
\pgfsys@transformshift{0.671729in}{0.417642in}%
\pgfsys@useobject{currentmarker}{}%
\end{pgfscope}%
\end{pgfscope}%
\begin{pgfscope}%
\definecolor{textcolor}{rgb}{0.000000,0.000000,0.000000}%
\pgfsetstrokecolor{textcolor}%
\pgfsetfillcolor{textcolor}%
\pgftext[x=0.671729in,y=0.320420in,,top]{\color{textcolor}\rmfamily\fontsize{8.000000}{9.600000}\selectfont \(\displaystyle {10^{0}}\)}%
\end{pgfscope}%
\begin{pgfscope}%
\pgfpathrectangle{\pgfqpoint{0.589510in}{0.417642in}}{\pgfqpoint{1.808820in}{1.370688in}}%
\pgfusepath{clip}%
\pgfsetrectcap%
\pgfsetroundjoin%
\pgfsetlinewidth{0.803000pt}%
\definecolor{currentstroke}{rgb}{0.450000,0.450000,0.450000}%
\pgfsetstrokecolor{currentstroke}%
\pgfsetdash{}{0pt}%
\pgfpathmoveto{\pgfqpoint{1.128240in}{0.417642in}}%
\pgfpathlineto{\pgfqpoint{1.128240in}{1.788330in}}%
\pgfusepath{stroke}%
\end{pgfscope}%
\begin{pgfscope}%
\pgfsetbuttcap%
\pgfsetroundjoin%
\definecolor{currentfill}{rgb}{0.000000,0.000000,0.000000}%
\pgfsetfillcolor{currentfill}%
\pgfsetlinewidth{0.803000pt}%
\definecolor{currentstroke}{rgb}{0.000000,0.000000,0.000000}%
\pgfsetstrokecolor{currentstroke}%
\pgfsetdash{}{0pt}%
\pgfsys@defobject{currentmarker}{\pgfqpoint{0.000000in}{-0.048611in}}{\pgfqpoint{0.000000in}{0.000000in}}{%
\pgfpathmoveto{\pgfqpoint{0.000000in}{0.000000in}}%
\pgfpathlineto{\pgfqpoint{0.000000in}{-0.048611in}}%
\pgfusepath{stroke,fill}%
}%
\begin{pgfscope}%
\pgfsys@transformshift{1.128240in}{0.417642in}%
\pgfsys@useobject{currentmarker}{}%
\end{pgfscope}%
\end{pgfscope}%
\begin{pgfscope}%
\definecolor{textcolor}{rgb}{0.000000,0.000000,0.000000}%
\pgfsetstrokecolor{textcolor}%
\pgfsetfillcolor{textcolor}%
\pgftext[x=1.128240in,y=0.320420in,,top]{\color{textcolor}\rmfamily\fontsize{8.000000}{9.600000}\selectfont \(\displaystyle {10^{1}}\)}%
\end{pgfscope}%
\begin{pgfscope}%
\pgfpathrectangle{\pgfqpoint{0.589510in}{0.417642in}}{\pgfqpoint{1.808820in}{1.370688in}}%
\pgfusepath{clip}%
\pgfsetrectcap%
\pgfsetroundjoin%
\pgfsetlinewidth{0.803000pt}%
\definecolor{currentstroke}{rgb}{0.450000,0.450000,0.450000}%
\pgfsetstrokecolor{currentstroke}%
\pgfsetdash{}{0pt}%
\pgfpathmoveto{\pgfqpoint{1.584752in}{0.417642in}}%
\pgfpathlineto{\pgfqpoint{1.584752in}{1.788330in}}%
\pgfusepath{stroke}%
\end{pgfscope}%
\begin{pgfscope}%
\pgfsetbuttcap%
\pgfsetroundjoin%
\definecolor{currentfill}{rgb}{0.000000,0.000000,0.000000}%
\pgfsetfillcolor{currentfill}%
\pgfsetlinewidth{0.803000pt}%
\definecolor{currentstroke}{rgb}{0.000000,0.000000,0.000000}%
\pgfsetstrokecolor{currentstroke}%
\pgfsetdash{}{0pt}%
\pgfsys@defobject{currentmarker}{\pgfqpoint{0.000000in}{-0.048611in}}{\pgfqpoint{0.000000in}{0.000000in}}{%
\pgfpathmoveto{\pgfqpoint{0.000000in}{0.000000in}}%
\pgfpathlineto{\pgfqpoint{0.000000in}{-0.048611in}}%
\pgfusepath{stroke,fill}%
}%
\begin{pgfscope}%
\pgfsys@transformshift{1.584752in}{0.417642in}%
\pgfsys@useobject{currentmarker}{}%
\end{pgfscope}%
\end{pgfscope}%
\begin{pgfscope}%
\definecolor{textcolor}{rgb}{0.000000,0.000000,0.000000}%
\pgfsetstrokecolor{textcolor}%
\pgfsetfillcolor{textcolor}%
\pgftext[x=1.584752in,y=0.320420in,,top]{\color{textcolor}\rmfamily\fontsize{8.000000}{9.600000}\selectfont \(\displaystyle {10^{2}}\)}%
\end{pgfscope}%
\begin{pgfscope}%
\pgfpathrectangle{\pgfqpoint{0.589510in}{0.417642in}}{\pgfqpoint{1.808820in}{1.370688in}}%
\pgfusepath{clip}%
\pgfsetrectcap%
\pgfsetroundjoin%
\pgfsetlinewidth{0.803000pt}%
\definecolor{currentstroke}{rgb}{0.450000,0.450000,0.450000}%
\pgfsetstrokecolor{currentstroke}%
\pgfsetdash{}{0pt}%
\pgfpathmoveto{\pgfqpoint{2.041264in}{0.417642in}}%
\pgfpathlineto{\pgfqpoint{2.041264in}{1.788330in}}%
\pgfusepath{stroke}%
\end{pgfscope}%
\begin{pgfscope}%
\pgfsetbuttcap%
\pgfsetroundjoin%
\definecolor{currentfill}{rgb}{0.000000,0.000000,0.000000}%
\pgfsetfillcolor{currentfill}%
\pgfsetlinewidth{0.803000pt}%
\definecolor{currentstroke}{rgb}{0.000000,0.000000,0.000000}%
\pgfsetstrokecolor{currentstroke}%
\pgfsetdash{}{0pt}%
\pgfsys@defobject{currentmarker}{\pgfqpoint{0.000000in}{-0.048611in}}{\pgfqpoint{0.000000in}{0.000000in}}{%
\pgfpathmoveto{\pgfqpoint{0.000000in}{0.000000in}}%
\pgfpathlineto{\pgfqpoint{0.000000in}{-0.048611in}}%
\pgfusepath{stroke,fill}%
}%
\begin{pgfscope}%
\pgfsys@transformshift{2.041264in}{0.417642in}%
\pgfsys@useobject{currentmarker}{}%
\end{pgfscope}%
\end{pgfscope}%
\begin{pgfscope}%
\definecolor{textcolor}{rgb}{0.000000,0.000000,0.000000}%
\pgfsetstrokecolor{textcolor}%
\pgfsetfillcolor{textcolor}%
\pgftext[x=2.041264in,y=0.320420in,,top]{\color{textcolor}\rmfamily\fontsize{8.000000}{9.600000}\selectfont \(\displaystyle {10^{3}}\)}%
\end{pgfscope}%
\begin{pgfscope}%
\pgfpathrectangle{\pgfqpoint{0.589510in}{0.417642in}}{\pgfqpoint{1.808820in}{1.370688in}}%
\pgfusepath{clip}%
\pgfsetrectcap%
\pgfsetroundjoin%
\pgfsetlinewidth{0.803000pt}%
\definecolor{currentstroke}{rgb}{0.850000,0.850000,0.850000}%
\pgfsetstrokecolor{currentstroke}%
\pgfsetdash{}{0pt}%
\pgfpathmoveto{\pgfqpoint{0.601014in}{0.417642in}}%
\pgfpathlineto{\pgfqpoint{0.601014in}{1.788330in}}%
\pgfusepath{stroke}%
\end{pgfscope}%
\begin{pgfscope}%
\pgfsetbuttcap%
\pgfsetroundjoin%
\definecolor{currentfill}{rgb}{0.000000,0.000000,0.000000}%
\pgfsetfillcolor{currentfill}%
\pgfsetlinewidth{0.602250pt}%
\definecolor{currentstroke}{rgb}{0.000000,0.000000,0.000000}%
\pgfsetstrokecolor{currentstroke}%
\pgfsetdash{}{0pt}%
\pgfsys@defobject{currentmarker}{\pgfqpoint{0.000000in}{-0.027778in}}{\pgfqpoint{0.000000in}{0.000000in}}{%
\pgfpathmoveto{\pgfqpoint{0.000000in}{0.000000in}}%
\pgfpathlineto{\pgfqpoint{0.000000in}{-0.027778in}}%
\pgfusepath{stroke,fill}%
}%
\begin{pgfscope}%
\pgfsys@transformshift{0.601014in}{0.417642in}%
\pgfsys@useobject{currentmarker}{}%
\end{pgfscope}%
\end{pgfscope}%
\begin{pgfscope}%
\pgfpathrectangle{\pgfqpoint{0.589510in}{0.417642in}}{\pgfqpoint{1.808820in}{1.370688in}}%
\pgfusepath{clip}%
\pgfsetrectcap%
\pgfsetroundjoin%
\pgfsetlinewidth{0.803000pt}%
\definecolor{currentstroke}{rgb}{0.850000,0.850000,0.850000}%
\pgfsetstrokecolor{currentstroke}%
\pgfsetdash{}{0pt}%
\pgfpathmoveto{\pgfqpoint{0.627488in}{0.417642in}}%
\pgfpathlineto{\pgfqpoint{0.627488in}{1.788330in}}%
\pgfusepath{stroke}%
\end{pgfscope}%
\begin{pgfscope}%
\pgfsetbuttcap%
\pgfsetroundjoin%
\definecolor{currentfill}{rgb}{0.000000,0.000000,0.000000}%
\pgfsetfillcolor{currentfill}%
\pgfsetlinewidth{0.602250pt}%
\definecolor{currentstroke}{rgb}{0.000000,0.000000,0.000000}%
\pgfsetstrokecolor{currentstroke}%
\pgfsetdash{}{0pt}%
\pgfsys@defobject{currentmarker}{\pgfqpoint{0.000000in}{-0.027778in}}{\pgfqpoint{0.000000in}{0.000000in}}{%
\pgfpathmoveto{\pgfqpoint{0.000000in}{0.000000in}}%
\pgfpathlineto{\pgfqpoint{0.000000in}{-0.027778in}}%
\pgfusepath{stroke,fill}%
}%
\begin{pgfscope}%
\pgfsys@transformshift{0.627488in}{0.417642in}%
\pgfsys@useobject{currentmarker}{}%
\end{pgfscope}%
\end{pgfscope}%
\begin{pgfscope}%
\pgfpathrectangle{\pgfqpoint{0.589510in}{0.417642in}}{\pgfqpoint{1.808820in}{1.370688in}}%
\pgfusepath{clip}%
\pgfsetrectcap%
\pgfsetroundjoin%
\pgfsetlinewidth{0.803000pt}%
\definecolor{currentstroke}{rgb}{0.850000,0.850000,0.850000}%
\pgfsetstrokecolor{currentstroke}%
\pgfsetdash{}{0pt}%
\pgfpathmoveto{\pgfqpoint{0.650840in}{0.417642in}}%
\pgfpathlineto{\pgfqpoint{0.650840in}{1.788330in}}%
\pgfusepath{stroke}%
\end{pgfscope}%
\begin{pgfscope}%
\pgfsetbuttcap%
\pgfsetroundjoin%
\definecolor{currentfill}{rgb}{0.000000,0.000000,0.000000}%
\pgfsetfillcolor{currentfill}%
\pgfsetlinewidth{0.602250pt}%
\definecolor{currentstroke}{rgb}{0.000000,0.000000,0.000000}%
\pgfsetstrokecolor{currentstroke}%
\pgfsetdash{}{0pt}%
\pgfsys@defobject{currentmarker}{\pgfqpoint{0.000000in}{-0.027778in}}{\pgfqpoint{0.000000in}{0.000000in}}{%
\pgfpathmoveto{\pgfqpoint{0.000000in}{0.000000in}}%
\pgfpathlineto{\pgfqpoint{0.000000in}{-0.027778in}}%
\pgfusepath{stroke,fill}%
}%
\begin{pgfscope}%
\pgfsys@transformshift{0.650840in}{0.417642in}%
\pgfsys@useobject{currentmarker}{}%
\end{pgfscope}%
\end{pgfscope}%
\begin{pgfscope}%
\pgfpathrectangle{\pgfqpoint{0.589510in}{0.417642in}}{\pgfqpoint{1.808820in}{1.370688in}}%
\pgfusepath{clip}%
\pgfsetrectcap%
\pgfsetroundjoin%
\pgfsetlinewidth{0.803000pt}%
\definecolor{currentstroke}{rgb}{0.850000,0.850000,0.850000}%
\pgfsetstrokecolor{currentstroke}%
\pgfsetdash{}{0pt}%
\pgfpathmoveto{\pgfqpoint{0.809153in}{0.417642in}}%
\pgfpathlineto{\pgfqpoint{0.809153in}{1.788330in}}%
\pgfusepath{stroke}%
\end{pgfscope}%
\begin{pgfscope}%
\pgfsetbuttcap%
\pgfsetroundjoin%
\definecolor{currentfill}{rgb}{0.000000,0.000000,0.000000}%
\pgfsetfillcolor{currentfill}%
\pgfsetlinewidth{0.602250pt}%
\definecolor{currentstroke}{rgb}{0.000000,0.000000,0.000000}%
\pgfsetstrokecolor{currentstroke}%
\pgfsetdash{}{0pt}%
\pgfsys@defobject{currentmarker}{\pgfqpoint{0.000000in}{-0.027778in}}{\pgfqpoint{0.000000in}{0.000000in}}{%
\pgfpathmoveto{\pgfqpoint{0.000000in}{0.000000in}}%
\pgfpathlineto{\pgfqpoint{0.000000in}{-0.027778in}}%
\pgfusepath{stroke,fill}%
}%
\begin{pgfscope}%
\pgfsys@transformshift{0.809153in}{0.417642in}%
\pgfsys@useobject{currentmarker}{}%
\end{pgfscope}%
\end{pgfscope}%
\begin{pgfscope}%
\pgfpathrectangle{\pgfqpoint{0.589510in}{0.417642in}}{\pgfqpoint{1.808820in}{1.370688in}}%
\pgfusepath{clip}%
\pgfsetrectcap%
\pgfsetroundjoin%
\pgfsetlinewidth{0.803000pt}%
\definecolor{currentstroke}{rgb}{0.850000,0.850000,0.850000}%
\pgfsetstrokecolor{currentstroke}%
\pgfsetdash{}{0pt}%
\pgfpathmoveto{\pgfqpoint{0.889540in}{0.417642in}}%
\pgfpathlineto{\pgfqpoint{0.889540in}{1.788330in}}%
\pgfusepath{stroke}%
\end{pgfscope}%
\begin{pgfscope}%
\pgfsetbuttcap%
\pgfsetroundjoin%
\definecolor{currentfill}{rgb}{0.000000,0.000000,0.000000}%
\pgfsetfillcolor{currentfill}%
\pgfsetlinewidth{0.602250pt}%
\definecolor{currentstroke}{rgb}{0.000000,0.000000,0.000000}%
\pgfsetstrokecolor{currentstroke}%
\pgfsetdash{}{0pt}%
\pgfsys@defobject{currentmarker}{\pgfqpoint{0.000000in}{-0.027778in}}{\pgfqpoint{0.000000in}{0.000000in}}{%
\pgfpathmoveto{\pgfqpoint{0.000000in}{0.000000in}}%
\pgfpathlineto{\pgfqpoint{0.000000in}{-0.027778in}}%
\pgfusepath{stroke,fill}%
}%
\begin{pgfscope}%
\pgfsys@transformshift{0.889540in}{0.417642in}%
\pgfsys@useobject{currentmarker}{}%
\end{pgfscope}%
\end{pgfscope}%
\begin{pgfscope}%
\pgfpathrectangle{\pgfqpoint{0.589510in}{0.417642in}}{\pgfqpoint{1.808820in}{1.370688in}}%
\pgfusepath{clip}%
\pgfsetrectcap%
\pgfsetroundjoin%
\pgfsetlinewidth{0.803000pt}%
\definecolor{currentstroke}{rgb}{0.850000,0.850000,0.850000}%
\pgfsetstrokecolor{currentstroke}%
\pgfsetdash{}{0pt}%
\pgfpathmoveto{\pgfqpoint{0.946576in}{0.417642in}}%
\pgfpathlineto{\pgfqpoint{0.946576in}{1.788330in}}%
\pgfusepath{stroke}%
\end{pgfscope}%
\begin{pgfscope}%
\pgfsetbuttcap%
\pgfsetroundjoin%
\definecolor{currentfill}{rgb}{0.000000,0.000000,0.000000}%
\pgfsetfillcolor{currentfill}%
\pgfsetlinewidth{0.602250pt}%
\definecolor{currentstroke}{rgb}{0.000000,0.000000,0.000000}%
\pgfsetstrokecolor{currentstroke}%
\pgfsetdash{}{0pt}%
\pgfsys@defobject{currentmarker}{\pgfqpoint{0.000000in}{-0.027778in}}{\pgfqpoint{0.000000in}{0.000000in}}{%
\pgfpathmoveto{\pgfqpoint{0.000000in}{0.000000in}}%
\pgfpathlineto{\pgfqpoint{0.000000in}{-0.027778in}}%
\pgfusepath{stroke,fill}%
}%
\begin{pgfscope}%
\pgfsys@transformshift{0.946576in}{0.417642in}%
\pgfsys@useobject{currentmarker}{}%
\end{pgfscope}%
\end{pgfscope}%
\begin{pgfscope}%
\pgfpathrectangle{\pgfqpoint{0.589510in}{0.417642in}}{\pgfqpoint{1.808820in}{1.370688in}}%
\pgfusepath{clip}%
\pgfsetrectcap%
\pgfsetroundjoin%
\pgfsetlinewidth{0.803000pt}%
\definecolor{currentstroke}{rgb}{0.850000,0.850000,0.850000}%
\pgfsetstrokecolor{currentstroke}%
\pgfsetdash{}{0pt}%
\pgfpathmoveto{\pgfqpoint{0.990817in}{0.417642in}}%
\pgfpathlineto{\pgfqpoint{0.990817in}{1.788330in}}%
\pgfusepath{stroke}%
\end{pgfscope}%
\begin{pgfscope}%
\pgfsetbuttcap%
\pgfsetroundjoin%
\definecolor{currentfill}{rgb}{0.000000,0.000000,0.000000}%
\pgfsetfillcolor{currentfill}%
\pgfsetlinewidth{0.602250pt}%
\definecolor{currentstroke}{rgb}{0.000000,0.000000,0.000000}%
\pgfsetstrokecolor{currentstroke}%
\pgfsetdash{}{0pt}%
\pgfsys@defobject{currentmarker}{\pgfqpoint{0.000000in}{-0.027778in}}{\pgfqpoint{0.000000in}{0.000000in}}{%
\pgfpathmoveto{\pgfqpoint{0.000000in}{0.000000in}}%
\pgfpathlineto{\pgfqpoint{0.000000in}{-0.027778in}}%
\pgfusepath{stroke,fill}%
}%
\begin{pgfscope}%
\pgfsys@transformshift{0.990817in}{0.417642in}%
\pgfsys@useobject{currentmarker}{}%
\end{pgfscope}%
\end{pgfscope}%
\begin{pgfscope}%
\pgfpathrectangle{\pgfqpoint{0.589510in}{0.417642in}}{\pgfqpoint{1.808820in}{1.370688in}}%
\pgfusepath{clip}%
\pgfsetrectcap%
\pgfsetroundjoin%
\pgfsetlinewidth{0.803000pt}%
\definecolor{currentstroke}{rgb}{0.850000,0.850000,0.850000}%
\pgfsetstrokecolor{currentstroke}%
\pgfsetdash{}{0pt}%
\pgfpathmoveto{\pgfqpoint{1.026964in}{0.417642in}}%
\pgfpathlineto{\pgfqpoint{1.026964in}{1.788330in}}%
\pgfusepath{stroke}%
\end{pgfscope}%
\begin{pgfscope}%
\pgfsetbuttcap%
\pgfsetroundjoin%
\definecolor{currentfill}{rgb}{0.000000,0.000000,0.000000}%
\pgfsetfillcolor{currentfill}%
\pgfsetlinewidth{0.602250pt}%
\definecolor{currentstroke}{rgb}{0.000000,0.000000,0.000000}%
\pgfsetstrokecolor{currentstroke}%
\pgfsetdash{}{0pt}%
\pgfsys@defobject{currentmarker}{\pgfqpoint{0.000000in}{-0.027778in}}{\pgfqpoint{0.000000in}{0.000000in}}{%
\pgfpathmoveto{\pgfqpoint{0.000000in}{0.000000in}}%
\pgfpathlineto{\pgfqpoint{0.000000in}{-0.027778in}}%
\pgfusepath{stroke,fill}%
}%
\begin{pgfscope}%
\pgfsys@transformshift{1.026964in}{0.417642in}%
\pgfsys@useobject{currentmarker}{}%
\end{pgfscope}%
\end{pgfscope}%
\begin{pgfscope}%
\pgfpathrectangle{\pgfqpoint{0.589510in}{0.417642in}}{\pgfqpoint{1.808820in}{1.370688in}}%
\pgfusepath{clip}%
\pgfsetrectcap%
\pgfsetroundjoin%
\pgfsetlinewidth{0.803000pt}%
\definecolor{currentstroke}{rgb}{0.850000,0.850000,0.850000}%
\pgfsetstrokecolor{currentstroke}%
\pgfsetdash{}{0pt}%
\pgfpathmoveto{\pgfqpoint{1.057526in}{0.417642in}}%
\pgfpathlineto{\pgfqpoint{1.057526in}{1.788330in}}%
\pgfusepath{stroke}%
\end{pgfscope}%
\begin{pgfscope}%
\pgfsetbuttcap%
\pgfsetroundjoin%
\definecolor{currentfill}{rgb}{0.000000,0.000000,0.000000}%
\pgfsetfillcolor{currentfill}%
\pgfsetlinewidth{0.602250pt}%
\definecolor{currentstroke}{rgb}{0.000000,0.000000,0.000000}%
\pgfsetstrokecolor{currentstroke}%
\pgfsetdash{}{0pt}%
\pgfsys@defobject{currentmarker}{\pgfqpoint{0.000000in}{-0.027778in}}{\pgfqpoint{0.000000in}{0.000000in}}{%
\pgfpathmoveto{\pgfqpoint{0.000000in}{0.000000in}}%
\pgfpathlineto{\pgfqpoint{0.000000in}{-0.027778in}}%
\pgfusepath{stroke,fill}%
}%
\begin{pgfscope}%
\pgfsys@transformshift{1.057526in}{0.417642in}%
\pgfsys@useobject{currentmarker}{}%
\end{pgfscope}%
\end{pgfscope}%
\begin{pgfscope}%
\pgfpathrectangle{\pgfqpoint{0.589510in}{0.417642in}}{\pgfqpoint{1.808820in}{1.370688in}}%
\pgfusepath{clip}%
\pgfsetrectcap%
\pgfsetroundjoin%
\pgfsetlinewidth{0.803000pt}%
\definecolor{currentstroke}{rgb}{0.850000,0.850000,0.850000}%
\pgfsetstrokecolor{currentstroke}%
\pgfsetdash{}{0pt}%
\pgfpathmoveto{\pgfqpoint{1.084000in}{0.417642in}}%
\pgfpathlineto{\pgfqpoint{1.084000in}{1.788330in}}%
\pgfusepath{stroke}%
\end{pgfscope}%
\begin{pgfscope}%
\pgfsetbuttcap%
\pgfsetroundjoin%
\definecolor{currentfill}{rgb}{0.000000,0.000000,0.000000}%
\pgfsetfillcolor{currentfill}%
\pgfsetlinewidth{0.602250pt}%
\definecolor{currentstroke}{rgb}{0.000000,0.000000,0.000000}%
\pgfsetstrokecolor{currentstroke}%
\pgfsetdash{}{0pt}%
\pgfsys@defobject{currentmarker}{\pgfqpoint{0.000000in}{-0.027778in}}{\pgfqpoint{0.000000in}{0.000000in}}{%
\pgfpathmoveto{\pgfqpoint{0.000000in}{0.000000in}}%
\pgfpathlineto{\pgfqpoint{0.000000in}{-0.027778in}}%
\pgfusepath{stroke,fill}%
}%
\begin{pgfscope}%
\pgfsys@transformshift{1.084000in}{0.417642in}%
\pgfsys@useobject{currentmarker}{}%
\end{pgfscope}%
\end{pgfscope}%
\begin{pgfscope}%
\pgfpathrectangle{\pgfqpoint{0.589510in}{0.417642in}}{\pgfqpoint{1.808820in}{1.370688in}}%
\pgfusepath{clip}%
\pgfsetrectcap%
\pgfsetroundjoin%
\pgfsetlinewidth{0.803000pt}%
\definecolor{currentstroke}{rgb}{0.850000,0.850000,0.850000}%
\pgfsetstrokecolor{currentstroke}%
\pgfsetdash{}{0pt}%
\pgfpathmoveto{\pgfqpoint{1.107352in}{0.417642in}}%
\pgfpathlineto{\pgfqpoint{1.107352in}{1.788330in}}%
\pgfusepath{stroke}%
\end{pgfscope}%
\begin{pgfscope}%
\pgfsetbuttcap%
\pgfsetroundjoin%
\definecolor{currentfill}{rgb}{0.000000,0.000000,0.000000}%
\pgfsetfillcolor{currentfill}%
\pgfsetlinewidth{0.602250pt}%
\definecolor{currentstroke}{rgb}{0.000000,0.000000,0.000000}%
\pgfsetstrokecolor{currentstroke}%
\pgfsetdash{}{0pt}%
\pgfsys@defobject{currentmarker}{\pgfqpoint{0.000000in}{-0.027778in}}{\pgfqpoint{0.000000in}{0.000000in}}{%
\pgfpathmoveto{\pgfqpoint{0.000000in}{0.000000in}}%
\pgfpathlineto{\pgfqpoint{0.000000in}{-0.027778in}}%
\pgfusepath{stroke,fill}%
}%
\begin{pgfscope}%
\pgfsys@transformshift{1.107352in}{0.417642in}%
\pgfsys@useobject{currentmarker}{}%
\end{pgfscope}%
\end{pgfscope}%
\begin{pgfscope}%
\pgfpathrectangle{\pgfqpoint{0.589510in}{0.417642in}}{\pgfqpoint{1.808820in}{1.370688in}}%
\pgfusepath{clip}%
\pgfsetrectcap%
\pgfsetroundjoin%
\pgfsetlinewidth{0.803000pt}%
\definecolor{currentstroke}{rgb}{0.850000,0.850000,0.850000}%
\pgfsetstrokecolor{currentstroke}%
\pgfsetdash{}{0pt}%
\pgfpathmoveto{\pgfqpoint{1.265664in}{0.417642in}}%
\pgfpathlineto{\pgfqpoint{1.265664in}{1.788330in}}%
\pgfusepath{stroke}%
\end{pgfscope}%
\begin{pgfscope}%
\pgfsetbuttcap%
\pgfsetroundjoin%
\definecolor{currentfill}{rgb}{0.000000,0.000000,0.000000}%
\pgfsetfillcolor{currentfill}%
\pgfsetlinewidth{0.602250pt}%
\definecolor{currentstroke}{rgb}{0.000000,0.000000,0.000000}%
\pgfsetstrokecolor{currentstroke}%
\pgfsetdash{}{0pt}%
\pgfsys@defobject{currentmarker}{\pgfqpoint{0.000000in}{-0.027778in}}{\pgfqpoint{0.000000in}{0.000000in}}{%
\pgfpathmoveto{\pgfqpoint{0.000000in}{0.000000in}}%
\pgfpathlineto{\pgfqpoint{0.000000in}{-0.027778in}}%
\pgfusepath{stroke,fill}%
}%
\begin{pgfscope}%
\pgfsys@transformshift{1.265664in}{0.417642in}%
\pgfsys@useobject{currentmarker}{}%
\end{pgfscope}%
\end{pgfscope}%
\begin{pgfscope}%
\pgfpathrectangle{\pgfqpoint{0.589510in}{0.417642in}}{\pgfqpoint{1.808820in}{1.370688in}}%
\pgfusepath{clip}%
\pgfsetrectcap%
\pgfsetroundjoin%
\pgfsetlinewidth{0.803000pt}%
\definecolor{currentstroke}{rgb}{0.850000,0.850000,0.850000}%
\pgfsetstrokecolor{currentstroke}%
\pgfsetdash{}{0pt}%
\pgfpathmoveto{\pgfqpoint{1.346052in}{0.417642in}}%
\pgfpathlineto{\pgfqpoint{1.346052in}{1.788330in}}%
\pgfusepath{stroke}%
\end{pgfscope}%
\begin{pgfscope}%
\pgfsetbuttcap%
\pgfsetroundjoin%
\definecolor{currentfill}{rgb}{0.000000,0.000000,0.000000}%
\pgfsetfillcolor{currentfill}%
\pgfsetlinewidth{0.602250pt}%
\definecolor{currentstroke}{rgb}{0.000000,0.000000,0.000000}%
\pgfsetstrokecolor{currentstroke}%
\pgfsetdash{}{0pt}%
\pgfsys@defobject{currentmarker}{\pgfqpoint{0.000000in}{-0.027778in}}{\pgfqpoint{0.000000in}{0.000000in}}{%
\pgfpathmoveto{\pgfqpoint{0.000000in}{0.000000in}}%
\pgfpathlineto{\pgfqpoint{0.000000in}{-0.027778in}}%
\pgfusepath{stroke,fill}%
}%
\begin{pgfscope}%
\pgfsys@transformshift{1.346052in}{0.417642in}%
\pgfsys@useobject{currentmarker}{}%
\end{pgfscope}%
\end{pgfscope}%
\begin{pgfscope}%
\pgfpathrectangle{\pgfqpoint{0.589510in}{0.417642in}}{\pgfqpoint{1.808820in}{1.370688in}}%
\pgfusepath{clip}%
\pgfsetrectcap%
\pgfsetroundjoin%
\pgfsetlinewidth{0.803000pt}%
\definecolor{currentstroke}{rgb}{0.850000,0.850000,0.850000}%
\pgfsetstrokecolor{currentstroke}%
\pgfsetdash{}{0pt}%
\pgfpathmoveto{\pgfqpoint{1.403088in}{0.417642in}}%
\pgfpathlineto{\pgfqpoint{1.403088in}{1.788330in}}%
\pgfusepath{stroke}%
\end{pgfscope}%
\begin{pgfscope}%
\pgfsetbuttcap%
\pgfsetroundjoin%
\definecolor{currentfill}{rgb}{0.000000,0.000000,0.000000}%
\pgfsetfillcolor{currentfill}%
\pgfsetlinewidth{0.602250pt}%
\definecolor{currentstroke}{rgb}{0.000000,0.000000,0.000000}%
\pgfsetstrokecolor{currentstroke}%
\pgfsetdash{}{0pt}%
\pgfsys@defobject{currentmarker}{\pgfqpoint{0.000000in}{-0.027778in}}{\pgfqpoint{0.000000in}{0.000000in}}{%
\pgfpathmoveto{\pgfqpoint{0.000000in}{0.000000in}}%
\pgfpathlineto{\pgfqpoint{0.000000in}{-0.027778in}}%
\pgfusepath{stroke,fill}%
}%
\begin{pgfscope}%
\pgfsys@transformshift{1.403088in}{0.417642in}%
\pgfsys@useobject{currentmarker}{}%
\end{pgfscope}%
\end{pgfscope}%
\begin{pgfscope}%
\pgfpathrectangle{\pgfqpoint{0.589510in}{0.417642in}}{\pgfqpoint{1.808820in}{1.370688in}}%
\pgfusepath{clip}%
\pgfsetrectcap%
\pgfsetroundjoin%
\pgfsetlinewidth{0.803000pt}%
\definecolor{currentstroke}{rgb}{0.850000,0.850000,0.850000}%
\pgfsetstrokecolor{currentstroke}%
\pgfsetdash{}{0pt}%
\pgfpathmoveto{\pgfqpoint{1.447328in}{0.417642in}}%
\pgfpathlineto{\pgfqpoint{1.447328in}{1.788330in}}%
\pgfusepath{stroke}%
\end{pgfscope}%
\begin{pgfscope}%
\pgfsetbuttcap%
\pgfsetroundjoin%
\definecolor{currentfill}{rgb}{0.000000,0.000000,0.000000}%
\pgfsetfillcolor{currentfill}%
\pgfsetlinewidth{0.602250pt}%
\definecolor{currentstroke}{rgb}{0.000000,0.000000,0.000000}%
\pgfsetstrokecolor{currentstroke}%
\pgfsetdash{}{0pt}%
\pgfsys@defobject{currentmarker}{\pgfqpoint{0.000000in}{-0.027778in}}{\pgfqpoint{0.000000in}{0.000000in}}{%
\pgfpathmoveto{\pgfqpoint{0.000000in}{0.000000in}}%
\pgfpathlineto{\pgfqpoint{0.000000in}{-0.027778in}}%
\pgfusepath{stroke,fill}%
}%
\begin{pgfscope}%
\pgfsys@transformshift{1.447328in}{0.417642in}%
\pgfsys@useobject{currentmarker}{}%
\end{pgfscope}%
\end{pgfscope}%
\begin{pgfscope}%
\pgfpathrectangle{\pgfqpoint{0.589510in}{0.417642in}}{\pgfqpoint{1.808820in}{1.370688in}}%
\pgfusepath{clip}%
\pgfsetrectcap%
\pgfsetroundjoin%
\pgfsetlinewidth{0.803000pt}%
\definecolor{currentstroke}{rgb}{0.850000,0.850000,0.850000}%
\pgfsetstrokecolor{currentstroke}%
\pgfsetdash{}{0pt}%
\pgfpathmoveto{\pgfqpoint{1.483475in}{0.417642in}}%
\pgfpathlineto{\pgfqpoint{1.483475in}{1.788330in}}%
\pgfusepath{stroke}%
\end{pgfscope}%
\begin{pgfscope}%
\pgfsetbuttcap%
\pgfsetroundjoin%
\definecolor{currentfill}{rgb}{0.000000,0.000000,0.000000}%
\pgfsetfillcolor{currentfill}%
\pgfsetlinewidth{0.602250pt}%
\definecolor{currentstroke}{rgb}{0.000000,0.000000,0.000000}%
\pgfsetstrokecolor{currentstroke}%
\pgfsetdash{}{0pt}%
\pgfsys@defobject{currentmarker}{\pgfqpoint{0.000000in}{-0.027778in}}{\pgfqpoint{0.000000in}{0.000000in}}{%
\pgfpathmoveto{\pgfqpoint{0.000000in}{0.000000in}}%
\pgfpathlineto{\pgfqpoint{0.000000in}{-0.027778in}}%
\pgfusepath{stroke,fill}%
}%
\begin{pgfscope}%
\pgfsys@transformshift{1.483475in}{0.417642in}%
\pgfsys@useobject{currentmarker}{}%
\end{pgfscope}%
\end{pgfscope}%
\begin{pgfscope}%
\pgfpathrectangle{\pgfqpoint{0.589510in}{0.417642in}}{\pgfqpoint{1.808820in}{1.370688in}}%
\pgfusepath{clip}%
\pgfsetrectcap%
\pgfsetroundjoin%
\pgfsetlinewidth{0.803000pt}%
\definecolor{currentstroke}{rgb}{0.850000,0.850000,0.850000}%
\pgfsetstrokecolor{currentstroke}%
\pgfsetdash{}{0pt}%
\pgfpathmoveto{\pgfqpoint{1.514037in}{0.417642in}}%
\pgfpathlineto{\pgfqpoint{1.514037in}{1.788330in}}%
\pgfusepath{stroke}%
\end{pgfscope}%
\begin{pgfscope}%
\pgfsetbuttcap%
\pgfsetroundjoin%
\definecolor{currentfill}{rgb}{0.000000,0.000000,0.000000}%
\pgfsetfillcolor{currentfill}%
\pgfsetlinewidth{0.602250pt}%
\definecolor{currentstroke}{rgb}{0.000000,0.000000,0.000000}%
\pgfsetstrokecolor{currentstroke}%
\pgfsetdash{}{0pt}%
\pgfsys@defobject{currentmarker}{\pgfqpoint{0.000000in}{-0.027778in}}{\pgfqpoint{0.000000in}{0.000000in}}{%
\pgfpathmoveto{\pgfqpoint{0.000000in}{0.000000in}}%
\pgfpathlineto{\pgfqpoint{0.000000in}{-0.027778in}}%
\pgfusepath{stroke,fill}%
}%
\begin{pgfscope}%
\pgfsys@transformshift{1.514037in}{0.417642in}%
\pgfsys@useobject{currentmarker}{}%
\end{pgfscope}%
\end{pgfscope}%
\begin{pgfscope}%
\pgfpathrectangle{\pgfqpoint{0.589510in}{0.417642in}}{\pgfqpoint{1.808820in}{1.370688in}}%
\pgfusepath{clip}%
\pgfsetrectcap%
\pgfsetroundjoin%
\pgfsetlinewidth{0.803000pt}%
\definecolor{currentstroke}{rgb}{0.850000,0.850000,0.850000}%
\pgfsetstrokecolor{currentstroke}%
\pgfsetdash{}{0pt}%
\pgfpathmoveto{\pgfqpoint{1.540511in}{0.417642in}}%
\pgfpathlineto{\pgfqpoint{1.540511in}{1.788330in}}%
\pgfusepath{stroke}%
\end{pgfscope}%
\begin{pgfscope}%
\pgfsetbuttcap%
\pgfsetroundjoin%
\definecolor{currentfill}{rgb}{0.000000,0.000000,0.000000}%
\pgfsetfillcolor{currentfill}%
\pgfsetlinewidth{0.602250pt}%
\definecolor{currentstroke}{rgb}{0.000000,0.000000,0.000000}%
\pgfsetstrokecolor{currentstroke}%
\pgfsetdash{}{0pt}%
\pgfsys@defobject{currentmarker}{\pgfqpoint{0.000000in}{-0.027778in}}{\pgfqpoint{0.000000in}{0.000000in}}{%
\pgfpathmoveto{\pgfqpoint{0.000000in}{0.000000in}}%
\pgfpathlineto{\pgfqpoint{0.000000in}{-0.027778in}}%
\pgfusepath{stroke,fill}%
}%
\begin{pgfscope}%
\pgfsys@transformshift{1.540511in}{0.417642in}%
\pgfsys@useobject{currentmarker}{}%
\end{pgfscope}%
\end{pgfscope}%
\begin{pgfscope}%
\pgfpathrectangle{\pgfqpoint{0.589510in}{0.417642in}}{\pgfqpoint{1.808820in}{1.370688in}}%
\pgfusepath{clip}%
\pgfsetrectcap%
\pgfsetroundjoin%
\pgfsetlinewidth{0.803000pt}%
\definecolor{currentstroke}{rgb}{0.850000,0.850000,0.850000}%
\pgfsetstrokecolor{currentstroke}%
\pgfsetdash{}{0pt}%
\pgfpathmoveto{\pgfqpoint{1.563863in}{0.417642in}}%
\pgfpathlineto{\pgfqpoint{1.563863in}{1.788330in}}%
\pgfusepath{stroke}%
\end{pgfscope}%
\begin{pgfscope}%
\pgfsetbuttcap%
\pgfsetroundjoin%
\definecolor{currentfill}{rgb}{0.000000,0.000000,0.000000}%
\pgfsetfillcolor{currentfill}%
\pgfsetlinewidth{0.602250pt}%
\definecolor{currentstroke}{rgb}{0.000000,0.000000,0.000000}%
\pgfsetstrokecolor{currentstroke}%
\pgfsetdash{}{0pt}%
\pgfsys@defobject{currentmarker}{\pgfqpoint{0.000000in}{-0.027778in}}{\pgfqpoint{0.000000in}{0.000000in}}{%
\pgfpathmoveto{\pgfqpoint{0.000000in}{0.000000in}}%
\pgfpathlineto{\pgfqpoint{0.000000in}{-0.027778in}}%
\pgfusepath{stroke,fill}%
}%
\begin{pgfscope}%
\pgfsys@transformshift{1.563863in}{0.417642in}%
\pgfsys@useobject{currentmarker}{}%
\end{pgfscope}%
\end{pgfscope}%
\begin{pgfscope}%
\pgfpathrectangle{\pgfqpoint{0.589510in}{0.417642in}}{\pgfqpoint{1.808820in}{1.370688in}}%
\pgfusepath{clip}%
\pgfsetrectcap%
\pgfsetroundjoin%
\pgfsetlinewidth{0.803000pt}%
\definecolor{currentstroke}{rgb}{0.850000,0.850000,0.850000}%
\pgfsetstrokecolor{currentstroke}%
\pgfsetdash{}{0pt}%
\pgfpathmoveto{\pgfqpoint{1.722176in}{0.417642in}}%
\pgfpathlineto{\pgfqpoint{1.722176in}{1.788330in}}%
\pgfusepath{stroke}%
\end{pgfscope}%
\begin{pgfscope}%
\pgfsetbuttcap%
\pgfsetroundjoin%
\definecolor{currentfill}{rgb}{0.000000,0.000000,0.000000}%
\pgfsetfillcolor{currentfill}%
\pgfsetlinewidth{0.602250pt}%
\definecolor{currentstroke}{rgb}{0.000000,0.000000,0.000000}%
\pgfsetstrokecolor{currentstroke}%
\pgfsetdash{}{0pt}%
\pgfsys@defobject{currentmarker}{\pgfqpoint{0.000000in}{-0.027778in}}{\pgfqpoint{0.000000in}{0.000000in}}{%
\pgfpathmoveto{\pgfqpoint{0.000000in}{0.000000in}}%
\pgfpathlineto{\pgfqpoint{0.000000in}{-0.027778in}}%
\pgfusepath{stroke,fill}%
}%
\begin{pgfscope}%
\pgfsys@transformshift{1.722176in}{0.417642in}%
\pgfsys@useobject{currentmarker}{}%
\end{pgfscope}%
\end{pgfscope}%
\begin{pgfscope}%
\pgfpathrectangle{\pgfqpoint{0.589510in}{0.417642in}}{\pgfqpoint{1.808820in}{1.370688in}}%
\pgfusepath{clip}%
\pgfsetrectcap%
\pgfsetroundjoin%
\pgfsetlinewidth{0.803000pt}%
\definecolor{currentstroke}{rgb}{0.850000,0.850000,0.850000}%
\pgfsetstrokecolor{currentstroke}%
\pgfsetdash{}{0pt}%
\pgfpathmoveto{\pgfqpoint{1.802563in}{0.417642in}}%
\pgfpathlineto{\pgfqpoint{1.802563in}{1.788330in}}%
\pgfusepath{stroke}%
\end{pgfscope}%
\begin{pgfscope}%
\pgfsetbuttcap%
\pgfsetroundjoin%
\definecolor{currentfill}{rgb}{0.000000,0.000000,0.000000}%
\pgfsetfillcolor{currentfill}%
\pgfsetlinewidth{0.602250pt}%
\definecolor{currentstroke}{rgb}{0.000000,0.000000,0.000000}%
\pgfsetstrokecolor{currentstroke}%
\pgfsetdash{}{0pt}%
\pgfsys@defobject{currentmarker}{\pgfqpoint{0.000000in}{-0.027778in}}{\pgfqpoint{0.000000in}{0.000000in}}{%
\pgfpathmoveto{\pgfqpoint{0.000000in}{0.000000in}}%
\pgfpathlineto{\pgfqpoint{0.000000in}{-0.027778in}}%
\pgfusepath{stroke,fill}%
}%
\begin{pgfscope}%
\pgfsys@transformshift{1.802563in}{0.417642in}%
\pgfsys@useobject{currentmarker}{}%
\end{pgfscope}%
\end{pgfscope}%
\begin{pgfscope}%
\pgfpathrectangle{\pgfqpoint{0.589510in}{0.417642in}}{\pgfqpoint{1.808820in}{1.370688in}}%
\pgfusepath{clip}%
\pgfsetrectcap%
\pgfsetroundjoin%
\pgfsetlinewidth{0.803000pt}%
\definecolor{currentstroke}{rgb}{0.850000,0.850000,0.850000}%
\pgfsetstrokecolor{currentstroke}%
\pgfsetdash{}{0pt}%
\pgfpathmoveto{\pgfqpoint{1.859599in}{0.417642in}}%
\pgfpathlineto{\pgfqpoint{1.859599in}{1.788330in}}%
\pgfusepath{stroke}%
\end{pgfscope}%
\begin{pgfscope}%
\pgfsetbuttcap%
\pgfsetroundjoin%
\definecolor{currentfill}{rgb}{0.000000,0.000000,0.000000}%
\pgfsetfillcolor{currentfill}%
\pgfsetlinewidth{0.602250pt}%
\definecolor{currentstroke}{rgb}{0.000000,0.000000,0.000000}%
\pgfsetstrokecolor{currentstroke}%
\pgfsetdash{}{0pt}%
\pgfsys@defobject{currentmarker}{\pgfqpoint{0.000000in}{-0.027778in}}{\pgfqpoint{0.000000in}{0.000000in}}{%
\pgfpathmoveto{\pgfqpoint{0.000000in}{0.000000in}}%
\pgfpathlineto{\pgfqpoint{0.000000in}{-0.027778in}}%
\pgfusepath{stroke,fill}%
}%
\begin{pgfscope}%
\pgfsys@transformshift{1.859599in}{0.417642in}%
\pgfsys@useobject{currentmarker}{}%
\end{pgfscope}%
\end{pgfscope}%
\begin{pgfscope}%
\pgfpathrectangle{\pgfqpoint{0.589510in}{0.417642in}}{\pgfqpoint{1.808820in}{1.370688in}}%
\pgfusepath{clip}%
\pgfsetrectcap%
\pgfsetroundjoin%
\pgfsetlinewidth{0.803000pt}%
\definecolor{currentstroke}{rgb}{0.850000,0.850000,0.850000}%
\pgfsetstrokecolor{currentstroke}%
\pgfsetdash{}{0pt}%
\pgfpathmoveto{\pgfqpoint{1.903840in}{0.417642in}}%
\pgfpathlineto{\pgfqpoint{1.903840in}{1.788330in}}%
\pgfusepath{stroke}%
\end{pgfscope}%
\begin{pgfscope}%
\pgfsetbuttcap%
\pgfsetroundjoin%
\definecolor{currentfill}{rgb}{0.000000,0.000000,0.000000}%
\pgfsetfillcolor{currentfill}%
\pgfsetlinewidth{0.602250pt}%
\definecolor{currentstroke}{rgb}{0.000000,0.000000,0.000000}%
\pgfsetstrokecolor{currentstroke}%
\pgfsetdash{}{0pt}%
\pgfsys@defobject{currentmarker}{\pgfqpoint{0.000000in}{-0.027778in}}{\pgfqpoint{0.000000in}{0.000000in}}{%
\pgfpathmoveto{\pgfqpoint{0.000000in}{0.000000in}}%
\pgfpathlineto{\pgfqpoint{0.000000in}{-0.027778in}}%
\pgfusepath{stroke,fill}%
}%
\begin{pgfscope}%
\pgfsys@transformshift{1.903840in}{0.417642in}%
\pgfsys@useobject{currentmarker}{}%
\end{pgfscope}%
\end{pgfscope}%
\begin{pgfscope}%
\pgfpathrectangle{\pgfqpoint{0.589510in}{0.417642in}}{\pgfqpoint{1.808820in}{1.370688in}}%
\pgfusepath{clip}%
\pgfsetrectcap%
\pgfsetroundjoin%
\pgfsetlinewidth{0.803000pt}%
\definecolor{currentstroke}{rgb}{0.850000,0.850000,0.850000}%
\pgfsetstrokecolor{currentstroke}%
\pgfsetdash{}{0pt}%
\pgfpathmoveto{\pgfqpoint{1.939987in}{0.417642in}}%
\pgfpathlineto{\pgfqpoint{1.939987in}{1.788330in}}%
\pgfusepath{stroke}%
\end{pgfscope}%
\begin{pgfscope}%
\pgfsetbuttcap%
\pgfsetroundjoin%
\definecolor{currentfill}{rgb}{0.000000,0.000000,0.000000}%
\pgfsetfillcolor{currentfill}%
\pgfsetlinewidth{0.602250pt}%
\definecolor{currentstroke}{rgb}{0.000000,0.000000,0.000000}%
\pgfsetstrokecolor{currentstroke}%
\pgfsetdash{}{0pt}%
\pgfsys@defobject{currentmarker}{\pgfqpoint{0.000000in}{-0.027778in}}{\pgfqpoint{0.000000in}{0.000000in}}{%
\pgfpathmoveto{\pgfqpoint{0.000000in}{0.000000in}}%
\pgfpathlineto{\pgfqpoint{0.000000in}{-0.027778in}}%
\pgfusepath{stroke,fill}%
}%
\begin{pgfscope}%
\pgfsys@transformshift{1.939987in}{0.417642in}%
\pgfsys@useobject{currentmarker}{}%
\end{pgfscope}%
\end{pgfscope}%
\begin{pgfscope}%
\pgfpathrectangle{\pgfqpoint{0.589510in}{0.417642in}}{\pgfqpoint{1.808820in}{1.370688in}}%
\pgfusepath{clip}%
\pgfsetrectcap%
\pgfsetroundjoin%
\pgfsetlinewidth{0.803000pt}%
\definecolor{currentstroke}{rgb}{0.850000,0.850000,0.850000}%
\pgfsetstrokecolor{currentstroke}%
\pgfsetdash{}{0pt}%
\pgfpathmoveto{\pgfqpoint{1.970549in}{0.417642in}}%
\pgfpathlineto{\pgfqpoint{1.970549in}{1.788330in}}%
\pgfusepath{stroke}%
\end{pgfscope}%
\begin{pgfscope}%
\pgfsetbuttcap%
\pgfsetroundjoin%
\definecolor{currentfill}{rgb}{0.000000,0.000000,0.000000}%
\pgfsetfillcolor{currentfill}%
\pgfsetlinewidth{0.602250pt}%
\definecolor{currentstroke}{rgb}{0.000000,0.000000,0.000000}%
\pgfsetstrokecolor{currentstroke}%
\pgfsetdash{}{0pt}%
\pgfsys@defobject{currentmarker}{\pgfqpoint{0.000000in}{-0.027778in}}{\pgfqpoint{0.000000in}{0.000000in}}{%
\pgfpathmoveto{\pgfqpoint{0.000000in}{0.000000in}}%
\pgfpathlineto{\pgfqpoint{0.000000in}{-0.027778in}}%
\pgfusepath{stroke,fill}%
}%
\begin{pgfscope}%
\pgfsys@transformshift{1.970549in}{0.417642in}%
\pgfsys@useobject{currentmarker}{}%
\end{pgfscope}%
\end{pgfscope}%
\begin{pgfscope}%
\pgfpathrectangle{\pgfqpoint{0.589510in}{0.417642in}}{\pgfqpoint{1.808820in}{1.370688in}}%
\pgfusepath{clip}%
\pgfsetrectcap%
\pgfsetroundjoin%
\pgfsetlinewidth{0.803000pt}%
\definecolor{currentstroke}{rgb}{0.850000,0.850000,0.850000}%
\pgfsetstrokecolor{currentstroke}%
\pgfsetdash{}{0pt}%
\pgfpathmoveto{\pgfqpoint{1.997023in}{0.417642in}}%
\pgfpathlineto{\pgfqpoint{1.997023in}{1.788330in}}%
\pgfusepath{stroke}%
\end{pgfscope}%
\begin{pgfscope}%
\pgfsetbuttcap%
\pgfsetroundjoin%
\definecolor{currentfill}{rgb}{0.000000,0.000000,0.000000}%
\pgfsetfillcolor{currentfill}%
\pgfsetlinewidth{0.602250pt}%
\definecolor{currentstroke}{rgb}{0.000000,0.000000,0.000000}%
\pgfsetstrokecolor{currentstroke}%
\pgfsetdash{}{0pt}%
\pgfsys@defobject{currentmarker}{\pgfqpoint{0.000000in}{-0.027778in}}{\pgfqpoint{0.000000in}{0.000000in}}{%
\pgfpathmoveto{\pgfqpoint{0.000000in}{0.000000in}}%
\pgfpathlineto{\pgfqpoint{0.000000in}{-0.027778in}}%
\pgfusepath{stroke,fill}%
}%
\begin{pgfscope}%
\pgfsys@transformshift{1.997023in}{0.417642in}%
\pgfsys@useobject{currentmarker}{}%
\end{pgfscope}%
\end{pgfscope}%
\begin{pgfscope}%
\pgfpathrectangle{\pgfqpoint{0.589510in}{0.417642in}}{\pgfqpoint{1.808820in}{1.370688in}}%
\pgfusepath{clip}%
\pgfsetrectcap%
\pgfsetroundjoin%
\pgfsetlinewidth{0.803000pt}%
\definecolor{currentstroke}{rgb}{0.850000,0.850000,0.850000}%
\pgfsetstrokecolor{currentstroke}%
\pgfsetdash{}{0pt}%
\pgfpathmoveto{\pgfqpoint{2.020375in}{0.417642in}}%
\pgfpathlineto{\pgfqpoint{2.020375in}{1.788330in}}%
\pgfusepath{stroke}%
\end{pgfscope}%
\begin{pgfscope}%
\pgfsetbuttcap%
\pgfsetroundjoin%
\definecolor{currentfill}{rgb}{0.000000,0.000000,0.000000}%
\pgfsetfillcolor{currentfill}%
\pgfsetlinewidth{0.602250pt}%
\definecolor{currentstroke}{rgb}{0.000000,0.000000,0.000000}%
\pgfsetstrokecolor{currentstroke}%
\pgfsetdash{}{0pt}%
\pgfsys@defobject{currentmarker}{\pgfqpoint{0.000000in}{-0.027778in}}{\pgfqpoint{0.000000in}{0.000000in}}{%
\pgfpathmoveto{\pgfqpoint{0.000000in}{0.000000in}}%
\pgfpathlineto{\pgfqpoint{0.000000in}{-0.027778in}}%
\pgfusepath{stroke,fill}%
}%
\begin{pgfscope}%
\pgfsys@transformshift{2.020375in}{0.417642in}%
\pgfsys@useobject{currentmarker}{}%
\end{pgfscope}%
\end{pgfscope}%
\begin{pgfscope}%
\pgfpathrectangle{\pgfqpoint{0.589510in}{0.417642in}}{\pgfqpoint{1.808820in}{1.370688in}}%
\pgfusepath{clip}%
\pgfsetrectcap%
\pgfsetroundjoin%
\pgfsetlinewidth{0.803000pt}%
\definecolor{currentstroke}{rgb}{0.850000,0.850000,0.850000}%
\pgfsetstrokecolor{currentstroke}%
\pgfsetdash{}{0pt}%
\pgfpathmoveto{\pgfqpoint{2.178687in}{0.417642in}}%
\pgfpathlineto{\pgfqpoint{2.178687in}{1.788330in}}%
\pgfusepath{stroke}%
\end{pgfscope}%
\begin{pgfscope}%
\pgfsetbuttcap%
\pgfsetroundjoin%
\definecolor{currentfill}{rgb}{0.000000,0.000000,0.000000}%
\pgfsetfillcolor{currentfill}%
\pgfsetlinewidth{0.602250pt}%
\definecolor{currentstroke}{rgb}{0.000000,0.000000,0.000000}%
\pgfsetstrokecolor{currentstroke}%
\pgfsetdash{}{0pt}%
\pgfsys@defobject{currentmarker}{\pgfqpoint{0.000000in}{-0.027778in}}{\pgfqpoint{0.000000in}{0.000000in}}{%
\pgfpathmoveto{\pgfqpoint{0.000000in}{0.000000in}}%
\pgfpathlineto{\pgfqpoint{0.000000in}{-0.027778in}}%
\pgfusepath{stroke,fill}%
}%
\begin{pgfscope}%
\pgfsys@transformshift{2.178687in}{0.417642in}%
\pgfsys@useobject{currentmarker}{}%
\end{pgfscope}%
\end{pgfscope}%
\begin{pgfscope}%
\pgfpathrectangle{\pgfqpoint{0.589510in}{0.417642in}}{\pgfqpoint{1.808820in}{1.370688in}}%
\pgfusepath{clip}%
\pgfsetrectcap%
\pgfsetroundjoin%
\pgfsetlinewidth{0.803000pt}%
\definecolor{currentstroke}{rgb}{0.850000,0.850000,0.850000}%
\pgfsetstrokecolor{currentstroke}%
\pgfsetdash{}{0pt}%
\pgfpathmoveto{\pgfqpoint{2.259075in}{0.417642in}}%
\pgfpathlineto{\pgfqpoint{2.259075in}{1.788330in}}%
\pgfusepath{stroke}%
\end{pgfscope}%
\begin{pgfscope}%
\pgfsetbuttcap%
\pgfsetroundjoin%
\definecolor{currentfill}{rgb}{0.000000,0.000000,0.000000}%
\pgfsetfillcolor{currentfill}%
\pgfsetlinewidth{0.602250pt}%
\definecolor{currentstroke}{rgb}{0.000000,0.000000,0.000000}%
\pgfsetstrokecolor{currentstroke}%
\pgfsetdash{}{0pt}%
\pgfsys@defobject{currentmarker}{\pgfqpoint{0.000000in}{-0.027778in}}{\pgfqpoint{0.000000in}{0.000000in}}{%
\pgfpathmoveto{\pgfqpoint{0.000000in}{0.000000in}}%
\pgfpathlineto{\pgfqpoint{0.000000in}{-0.027778in}}%
\pgfusepath{stroke,fill}%
}%
\begin{pgfscope}%
\pgfsys@transformshift{2.259075in}{0.417642in}%
\pgfsys@useobject{currentmarker}{}%
\end{pgfscope}%
\end{pgfscope}%
\begin{pgfscope}%
\pgfpathrectangle{\pgfqpoint{0.589510in}{0.417642in}}{\pgfqpoint{1.808820in}{1.370688in}}%
\pgfusepath{clip}%
\pgfsetrectcap%
\pgfsetroundjoin%
\pgfsetlinewidth{0.803000pt}%
\definecolor{currentstroke}{rgb}{0.850000,0.850000,0.850000}%
\pgfsetstrokecolor{currentstroke}%
\pgfsetdash{}{0pt}%
\pgfpathmoveto{\pgfqpoint{2.316111in}{0.417642in}}%
\pgfpathlineto{\pgfqpoint{2.316111in}{1.788330in}}%
\pgfusepath{stroke}%
\end{pgfscope}%
\begin{pgfscope}%
\pgfsetbuttcap%
\pgfsetroundjoin%
\definecolor{currentfill}{rgb}{0.000000,0.000000,0.000000}%
\pgfsetfillcolor{currentfill}%
\pgfsetlinewidth{0.602250pt}%
\definecolor{currentstroke}{rgb}{0.000000,0.000000,0.000000}%
\pgfsetstrokecolor{currentstroke}%
\pgfsetdash{}{0pt}%
\pgfsys@defobject{currentmarker}{\pgfqpoint{0.000000in}{-0.027778in}}{\pgfqpoint{0.000000in}{0.000000in}}{%
\pgfpathmoveto{\pgfqpoint{0.000000in}{0.000000in}}%
\pgfpathlineto{\pgfqpoint{0.000000in}{-0.027778in}}%
\pgfusepath{stroke,fill}%
}%
\begin{pgfscope}%
\pgfsys@transformshift{2.316111in}{0.417642in}%
\pgfsys@useobject{currentmarker}{}%
\end{pgfscope}%
\end{pgfscope}%
\begin{pgfscope}%
\pgfpathrectangle{\pgfqpoint{0.589510in}{0.417642in}}{\pgfqpoint{1.808820in}{1.370688in}}%
\pgfusepath{clip}%
\pgfsetrectcap%
\pgfsetroundjoin%
\pgfsetlinewidth{0.803000pt}%
\definecolor{currentstroke}{rgb}{0.850000,0.850000,0.850000}%
\pgfsetstrokecolor{currentstroke}%
\pgfsetdash{}{0pt}%
\pgfpathmoveto{\pgfqpoint{2.360351in}{0.417642in}}%
\pgfpathlineto{\pgfqpoint{2.360351in}{1.788330in}}%
\pgfusepath{stroke}%
\end{pgfscope}%
\begin{pgfscope}%
\pgfsetbuttcap%
\pgfsetroundjoin%
\definecolor{currentfill}{rgb}{0.000000,0.000000,0.000000}%
\pgfsetfillcolor{currentfill}%
\pgfsetlinewidth{0.602250pt}%
\definecolor{currentstroke}{rgb}{0.000000,0.000000,0.000000}%
\pgfsetstrokecolor{currentstroke}%
\pgfsetdash{}{0pt}%
\pgfsys@defobject{currentmarker}{\pgfqpoint{0.000000in}{-0.027778in}}{\pgfqpoint{0.000000in}{0.000000in}}{%
\pgfpathmoveto{\pgfqpoint{0.000000in}{0.000000in}}%
\pgfpathlineto{\pgfqpoint{0.000000in}{-0.027778in}}%
\pgfusepath{stroke,fill}%
}%
\begin{pgfscope}%
\pgfsys@transformshift{2.360351in}{0.417642in}%
\pgfsys@useobject{currentmarker}{}%
\end{pgfscope}%
\end{pgfscope}%
\begin{pgfscope}%
\pgfpathrectangle{\pgfqpoint{0.589510in}{0.417642in}}{\pgfqpoint{1.808820in}{1.370688in}}%
\pgfusepath{clip}%
\pgfsetrectcap%
\pgfsetroundjoin%
\pgfsetlinewidth{0.803000pt}%
\definecolor{currentstroke}{rgb}{0.850000,0.850000,0.850000}%
\pgfsetstrokecolor{currentstroke}%
\pgfsetdash{}{0pt}%
\pgfpathmoveto{\pgfqpoint{2.396499in}{0.417642in}}%
\pgfpathlineto{\pgfqpoint{2.396499in}{1.788330in}}%
\pgfusepath{stroke}%
\end{pgfscope}%
\begin{pgfscope}%
\pgfsetbuttcap%
\pgfsetroundjoin%
\definecolor{currentfill}{rgb}{0.000000,0.000000,0.000000}%
\pgfsetfillcolor{currentfill}%
\pgfsetlinewidth{0.602250pt}%
\definecolor{currentstroke}{rgb}{0.000000,0.000000,0.000000}%
\pgfsetstrokecolor{currentstroke}%
\pgfsetdash{}{0pt}%
\pgfsys@defobject{currentmarker}{\pgfqpoint{0.000000in}{-0.027778in}}{\pgfqpoint{0.000000in}{0.000000in}}{%
\pgfpathmoveto{\pgfqpoint{0.000000in}{0.000000in}}%
\pgfpathlineto{\pgfqpoint{0.000000in}{-0.027778in}}%
\pgfusepath{stroke,fill}%
}%
\begin{pgfscope}%
\pgfsys@transformshift{2.396499in}{0.417642in}%
\pgfsys@useobject{currentmarker}{}%
\end{pgfscope}%
\end{pgfscope}%
\begin{pgfscope}%
\definecolor{textcolor}{rgb}{0.000000,0.000000,0.000000}%
\pgfsetstrokecolor{textcolor}%
\pgfsetfillcolor{textcolor}%
\pgftext[x=1.493920in,y=0.165003in,,top]{\color{textcolor}\rmfamily\fontsize{10.000000}{12.000000}\selectfont \(\displaystyle \tau\) in \unit{\second}}%
\end{pgfscope}%
\begin{pgfscope}%
\pgfpathrectangle{\pgfqpoint{0.589510in}{0.417642in}}{\pgfqpoint{1.808820in}{1.370688in}}%
\pgfusepath{clip}%
\pgfsetrectcap%
\pgfsetroundjoin%
\pgfsetlinewidth{0.803000pt}%
\definecolor{currentstroke}{rgb}{0.450000,0.450000,0.450000}%
\pgfsetstrokecolor{currentstroke}%
\pgfsetdash{}{0pt}%
\pgfpathmoveto{\pgfqpoint{0.589510in}{0.417642in}}%
\pgfpathlineto{\pgfqpoint{2.398330in}{0.417642in}}%
\pgfusepath{stroke}%
\end{pgfscope}%
\begin{pgfscope}%
\pgfsetbuttcap%
\pgfsetroundjoin%
\definecolor{currentfill}{rgb}{0.000000,0.000000,0.000000}%
\pgfsetfillcolor{currentfill}%
\pgfsetlinewidth{0.803000pt}%
\definecolor{currentstroke}{rgb}{0.000000,0.000000,0.000000}%
\pgfsetstrokecolor{currentstroke}%
\pgfsetdash{}{0pt}%
\pgfsys@defobject{currentmarker}{\pgfqpoint{-0.048611in}{0.000000in}}{\pgfqpoint{-0.000000in}{0.000000in}}{%
\pgfpathmoveto{\pgfqpoint{-0.000000in}{0.000000in}}%
\pgfpathlineto{\pgfqpoint{-0.048611in}{0.000000in}}%
\pgfusepath{stroke,fill}%
}%
\begin{pgfscope}%
\pgfsys@transformshift{0.589510in}{0.417642in}%
\pgfsys@useobject{currentmarker}{}%
\end{pgfscope}%
\end{pgfscope}%
\begin{pgfscope}%
\definecolor{textcolor}{rgb}{0.000000,0.000000,0.000000}%
\pgfsetstrokecolor{textcolor}%
\pgfsetfillcolor{textcolor}%
\pgftext[x=0.236114in, y=0.378489in, left, base]{\color{textcolor}\rmfamily\fontsize{8.000000}{9.600000}\selectfont \(\displaystyle {10^{-2}}\)}%
\end{pgfscope}%
\begin{pgfscope}%
\pgfpathrectangle{\pgfqpoint{0.589510in}{0.417642in}}{\pgfqpoint{1.808820in}{1.370688in}}%
\pgfusepath{clip}%
\pgfsetrectcap%
\pgfsetroundjoin%
\pgfsetlinewidth{0.803000pt}%
\definecolor{currentstroke}{rgb}{0.450000,0.450000,0.450000}%
\pgfsetstrokecolor{currentstroke}%
\pgfsetdash{}{0pt}%
\pgfpathmoveto{\pgfqpoint{0.589510in}{0.826865in}}%
\pgfpathlineto{\pgfqpoint{2.398330in}{0.826865in}}%
\pgfusepath{stroke}%
\end{pgfscope}%
\begin{pgfscope}%
\pgfsetbuttcap%
\pgfsetroundjoin%
\definecolor{currentfill}{rgb}{0.000000,0.000000,0.000000}%
\pgfsetfillcolor{currentfill}%
\pgfsetlinewidth{0.803000pt}%
\definecolor{currentstroke}{rgb}{0.000000,0.000000,0.000000}%
\pgfsetstrokecolor{currentstroke}%
\pgfsetdash{}{0pt}%
\pgfsys@defobject{currentmarker}{\pgfqpoint{-0.048611in}{0.000000in}}{\pgfqpoint{-0.000000in}{0.000000in}}{%
\pgfpathmoveto{\pgfqpoint{-0.000000in}{0.000000in}}%
\pgfpathlineto{\pgfqpoint{-0.048611in}{0.000000in}}%
\pgfusepath{stroke,fill}%
}%
\begin{pgfscope}%
\pgfsys@transformshift{0.589510in}{0.826865in}%
\pgfsys@useobject{currentmarker}{}%
\end{pgfscope}%
\end{pgfscope}%
\begin{pgfscope}%
\definecolor{textcolor}{rgb}{0.000000,0.000000,0.000000}%
\pgfsetstrokecolor{textcolor}%
\pgfsetfillcolor{textcolor}%
\pgftext[x=0.316361in, y=0.787713in, left, base]{\color{textcolor}\rmfamily\fontsize{8.000000}{9.600000}\selectfont \(\displaystyle {10^{0}}\)}%
\end{pgfscope}%
\begin{pgfscope}%
\pgfpathrectangle{\pgfqpoint{0.589510in}{0.417642in}}{\pgfqpoint{1.808820in}{1.370688in}}%
\pgfusepath{clip}%
\pgfsetrectcap%
\pgfsetroundjoin%
\pgfsetlinewidth{0.803000pt}%
\definecolor{currentstroke}{rgb}{0.450000,0.450000,0.450000}%
\pgfsetstrokecolor{currentstroke}%
\pgfsetdash{}{0pt}%
\pgfpathmoveto{\pgfqpoint{0.589510in}{1.236089in}}%
\pgfpathlineto{\pgfqpoint{2.398330in}{1.236089in}}%
\pgfusepath{stroke}%
\end{pgfscope}%
\begin{pgfscope}%
\pgfsetbuttcap%
\pgfsetroundjoin%
\definecolor{currentfill}{rgb}{0.000000,0.000000,0.000000}%
\pgfsetfillcolor{currentfill}%
\pgfsetlinewidth{0.803000pt}%
\definecolor{currentstroke}{rgb}{0.000000,0.000000,0.000000}%
\pgfsetstrokecolor{currentstroke}%
\pgfsetdash{}{0pt}%
\pgfsys@defobject{currentmarker}{\pgfqpoint{-0.048611in}{0.000000in}}{\pgfqpoint{-0.000000in}{0.000000in}}{%
\pgfpathmoveto{\pgfqpoint{-0.000000in}{0.000000in}}%
\pgfpathlineto{\pgfqpoint{-0.048611in}{0.000000in}}%
\pgfusepath{stroke,fill}%
}%
\begin{pgfscope}%
\pgfsys@transformshift{0.589510in}{1.236089in}%
\pgfsys@useobject{currentmarker}{}%
\end{pgfscope}%
\end{pgfscope}%
\begin{pgfscope}%
\definecolor{textcolor}{rgb}{0.000000,0.000000,0.000000}%
\pgfsetstrokecolor{textcolor}%
\pgfsetfillcolor{textcolor}%
\pgftext[x=0.316361in, y=1.196936in, left, base]{\color{textcolor}\rmfamily\fontsize{8.000000}{9.600000}\selectfont \(\displaystyle {10^{2}}\)}%
\end{pgfscope}%
\begin{pgfscope}%
\pgfpathrectangle{\pgfqpoint{0.589510in}{0.417642in}}{\pgfqpoint{1.808820in}{1.370688in}}%
\pgfusepath{clip}%
\pgfsetrectcap%
\pgfsetroundjoin%
\pgfsetlinewidth{0.803000pt}%
\definecolor{currentstroke}{rgb}{0.450000,0.450000,0.450000}%
\pgfsetstrokecolor{currentstroke}%
\pgfsetdash{}{0pt}%
\pgfpathmoveto{\pgfqpoint{0.589510in}{1.645313in}}%
\pgfpathlineto{\pgfqpoint{2.398330in}{1.645313in}}%
\pgfusepath{stroke}%
\end{pgfscope}%
\begin{pgfscope}%
\pgfsetbuttcap%
\pgfsetroundjoin%
\definecolor{currentfill}{rgb}{0.000000,0.000000,0.000000}%
\pgfsetfillcolor{currentfill}%
\pgfsetlinewidth{0.803000pt}%
\definecolor{currentstroke}{rgb}{0.000000,0.000000,0.000000}%
\pgfsetstrokecolor{currentstroke}%
\pgfsetdash{}{0pt}%
\pgfsys@defobject{currentmarker}{\pgfqpoint{-0.048611in}{0.000000in}}{\pgfqpoint{-0.000000in}{0.000000in}}{%
\pgfpathmoveto{\pgfqpoint{-0.000000in}{0.000000in}}%
\pgfpathlineto{\pgfqpoint{-0.048611in}{0.000000in}}%
\pgfusepath{stroke,fill}%
}%
\begin{pgfscope}%
\pgfsys@transformshift{0.589510in}{1.645313in}%
\pgfsys@useobject{currentmarker}{}%
\end{pgfscope}%
\end{pgfscope}%
\begin{pgfscope}%
\definecolor{textcolor}{rgb}{0.000000,0.000000,0.000000}%
\pgfsetstrokecolor{textcolor}%
\pgfsetfillcolor{textcolor}%
\pgftext[x=0.316361in, y=1.606160in, left, base]{\color{textcolor}\rmfamily\fontsize{8.000000}{9.600000}\selectfont \(\displaystyle {10^{4}}\)}%
\end{pgfscope}%
\begin{pgfscope}%
\definecolor{textcolor}{rgb}{0.000000,0.000000,0.000000}%
\pgfsetstrokecolor{textcolor}%
\pgfsetfillcolor{textcolor}%
\pgftext[x=0.180559in,y=1.102986in,,bottom,rotate=90.000000]{\color{textcolor}\rmfamily\fontsize{10.000000}{12.000000}\selectfont ADEV \(\displaystyle \sigma_A(\tau)\)}%
\end{pgfscope}%
\begin{pgfscope}%
\pgfpathrectangle{\pgfqpoint{0.589510in}{0.417642in}}{\pgfqpoint{1.808820in}{1.370688in}}%
\pgfusepath{clip}%
\pgfsetbuttcap%
\pgfsetroundjoin%
\pgfsetlinewidth{1.505625pt}%
\definecolor{currentstroke}{rgb}{0.003922,0.450980,0.698039}%
\pgfsetstrokecolor{currentstroke}%
\pgfsetdash{{5.550000pt}{2.400000pt}}{0.000000pt}%
\pgfpathmoveto{\pgfqpoint{0.671729in}{0.826865in}}%
\pgfpathlineto{\pgfqpoint{0.809153in}{0.796068in}}%
\pgfpathlineto{\pgfqpoint{0.946576in}{0.765271in}}%
\pgfpathlineto{\pgfqpoint{1.128240in}{0.724560in}}%
\pgfpathlineto{\pgfqpoint{1.265664in}{0.693762in}}%
\pgfpathlineto{\pgfqpoint{1.403088in}{0.662965in}}%
\pgfpathlineto{\pgfqpoint{1.584752in}{0.622254in}}%
\pgfpathlineto{\pgfqpoint{1.722176in}{0.591457in}}%
\pgfpathlineto{\pgfqpoint{1.859599in}{0.560659in}}%
\pgfpathlineto{\pgfqpoint{2.041264in}{0.519948in}}%
\pgfpathlineto{\pgfqpoint{2.178687in}{0.489151in}}%
\pgfpathlineto{\pgfqpoint{2.316111in}{0.458354in}}%
\pgfusepath{stroke}%
\end{pgfscope}%
\begin{pgfscope}%
\pgfpathrectangle{\pgfqpoint{0.589510in}{0.417642in}}{\pgfqpoint{1.808820in}{1.370688in}}%
\pgfusepath{clip}%
\pgfsetbuttcap%
\pgfsetroundjoin%
\definecolor{currentfill}{rgb}{0.003922,0.450980,0.698039}%
\pgfsetfillcolor{currentfill}%
\pgfsetlinewidth{1.003750pt}%
\definecolor{currentstroke}{rgb}{0.003922,0.450980,0.698039}%
\pgfsetstrokecolor{currentstroke}%
\pgfsetdash{}{0pt}%
\pgfsys@defobject{currentmarker}{\pgfqpoint{-0.020833in}{-0.020833in}}{\pgfqpoint{0.020833in}{0.020833in}}{%
\pgfpathmoveto{\pgfqpoint{0.000000in}{-0.020833in}}%
\pgfpathcurveto{\pgfqpoint{0.005525in}{-0.020833in}}{\pgfqpoint{0.010825in}{-0.018638in}}{\pgfqpoint{0.014731in}{-0.014731in}}%
\pgfpathcurveto{\pgfqpoint{0.018638in}{-0.010825in}}{\pgfqpoint{0.020833in}{-0.005525in}}{\pgfqpoint{0.020833in}{0.000000in}}%
\pgfpathcurveto{\pgfqpoint{0.020833in}{0.005525in}}{\pgfqpoint{0.018638in}{0.010825in}}{\pgfqpoint{0.014731in}{0.014731in}}%
\pgfpathcurveto{\pgfqpoint{0.010825in}{0.018638in}}{\pgfqpoint{0.005525in}{0.020833in}}{\pgfqpoint{0.000000in}{0.020833in}}%
\pgfpathcurveto{\pgfqpoint{-0.005525in}{0.020833in}}{\pgfqpoint{-0.010825in}{0.018638in}}{\pgfqpoint{-0.014731in}{0.014731in}}%
\pgfpathcurveto{\pgfqpoint{-0.018638in}{0.010825in}}{\pgfqpoint{-0.020833in}{0.005525in}}{\pgfqpoint{-0.020833in}{0.000000in}}%
\pgfpathcurveto{\pgfqpoint{-0.020833in}{-0.005525in}}{\pgfqpoint{-0.018638in}{-0.010825in}}{\pgfqpoint{-0.014731in}{-0.014731in}}%
\pgfpathcurveto{\pgfqpoint{-0.010825in}{-0.018638in}}{\pgfqpoint{-0.005525in}{-0.020833in}}{\pgfqpoint{0.000000in}{-0.020833in}}%
\pgfpathlineto{\pgfqpoint{0.000000in}{-0.020833in}}%
\pgfpathclose%
\pgfusepath{stroke,fill}%
}%
\begin{pgfscope}%
\pgfsys@transformshift{0.671729in}{0.827492in}%
\pgfsys@useobject{currentmarker}{}%
\end{pgfscope}%
\begin{pgfscope}%
\pgfsys@transformshift{0.809153in}{0.796721in}%
\pgfsys@useobject{currentmarker}{}%
\end{pgfscope}%
\begin{pgfscope}%
\pgfsys@transformshift{0.946576in}{0.765394in}%
\pgfsys@useobject{currentmarker}{}%
\end{pgfscope}%
\begin{pgfscope}%
\pgfsys@transformshift{1.128240in}{0.722837in}%
\pgfsys@useobject{currentmarker}{}%
\end{pgfscope}%
\begin{pgfscope}%
\pgfsys@transformshift{1.265664in}{0.689068in}%
\pgfsys@useobject{currentmarker}{}%
\end{pgfscope}%
\begin{pgfscope}%
\pgfsys@transformshift{1.403088in}{0.662183in}%
\pgfsys@useobject{currentmarker}{}%
\end{pgfscope}%
\begin{pgfscope}%
\pgfsys@transformshift{1.584752in}{0.624431in}%
\pgfsys@useobject{currentmarker}{}%
\end{pgfscope}%
\begin{pgfscope}%
\pgfsys@transformshift{1.722176in}{0.589132in}%
\pgfsys@useobject{currentmarker}{}%
\end{pgfscope}%
\begin{pgfscope}%
\pgfsys@transformshift{1.859599in}{0.544283in}%
\pgfsys@useobject{currentmarker}{}%
\end{pgfscope}%
\begin{pgfscope}%
\pgfsys@transformshift{2.041264in}{0.497009in}%
\pgfsys@useobject{currentmarker}{}%
\end{pgfscope}%
\begin{pgfscope}%
\pgfsys@transformshift{2.178687in}{0.507181in}%
\pgfsys@useobject{currentmarker}{}%
\end{pgfscope}%
\begin{pgfscope}%
\pgfsys@transformshift{2.316111in}{0.455299in}%
\pgfsys@useobject{currentmarker}{}%
\end{pgfscope}%
\end{pgfscope}%
\begin{pgfscope}%
\pgfsetrectcap%
\pgfsetmiterjoin%
\pgfsetlinewidth{0.803000pt}%
\definecolor{currentstroke}{rgb}{0.000000,0.000000,0.000000}%
\pgfsetstrokecolor{currentstroke}%
\pgfsetdash{}{0pt}%
\pgfpathmoveto{\pgfqpoint{0.589510in}{0.417642in}}%
\pgfpathlineto{\pgfqpoint{0.589510in}{1.788330in}}%
\pgfusepath{stroke}%
\end{pgfscope}%
\begin{pgfscope}%
\pgfsetrectcap%
\pgfsetmiterjoin%
\pgfsetlinewidth{0.803000pt}%
\definecolor{currentstroke}{rgb}{0.000000,0.000000,0.000000}%
\pgfsetstrokecolor{currentstroke}%
\pgfsetdash{}{0pt}%
\pgfpathmoveto{\pgfqpoint{2.398330in}{0.417642in}}%
\pgfpathlineto{\pgfqpoint{2.398330in}{1.788330in}}%
\pgfusepath{stroke}%
\end{pgfscope}%
\begin{pgfscope}%
\pgfsetrectcap%
\pgfsetmiterjoin%
\pgfsetlinewidth{0.803000pt}%
\definecolor{currentstroke}{rgb}{0.000000,0.000000,0.000000}%
\pgfsetstrokecolor{currentstroke}%
\pgfsetdash{}{0pt}%
\pgfpathmoveto{\pgfqpoint{0.589510in}{0.417642in}}%
\pgfpathlineto{\pgfqpoint{2.398330in}{0.417642in}}%
\pgfusepath{stroke}%
\end{pgfscope}%
\begin{pgfscope}%
\pgfsetrectcap%
\pgfsetmiterjoin%
\pgfsetlinewidth{0.803000pt}%
\definecolor{currentstroke}{rgb}{0.000000,0.000000,0.000000}%
\pgfsetstrokecolor{currentstroke}%
\pgfsetdash{}{0pt}%
\pgfpathmoveto{\pgfqpoint{0.589510in}{1.788330in}}%
\pgfpathlineto{\pgfqpoint{2.398330in}{1.788330in}}%
\pgfusepath{stroke}%
\end{pgfscope}%
\begin{pgfscope}%
\pgfsetbuttcap%
\pgfsetmiterjoin%
\definecolor{currentfill}{rgb}{1.000000,1.000000,1.000000}%
\pgfsetfillcolor{currentfill}%
\pgfsetfillopacity{0.800000}%
\pgfsetlinewidth{1.003750pt}%
\definecolor{currentstroke}{rgb}{0.800000,0.800000,0.800000}%
\pgfsetstrokecolor{currentstroke}%
\pgfsetstrokeopacity{0.800000}%
\pgfsetdash{}{0pt}%
\pgfpathmoveto{\pgfqpoint{1.289694in}{1.471662in}}%
\pgfpathlineto{\pgfqpoint{2.320552in}{1.471662in}}%
\pgfpathquadraticcurveto{\pgfqpoint{2.342774in}{1.471662in}}{\pgfqpoint{2.342774in}{1.493884in}}%
\pgfpathlineto{\pgfqpoint{2.342774in}{1.710552in}}%
\pgfpathquadraticcurveto{\pgfqpoint{2.342774in}{1.732774in}}{\pgfqpoint{2.320552in}{1.732774in}}%
\pgfpathlineto{\pgfqpoint{1.289694in}{1.732774in}}%
\pgfpathquadraticcurveto{\pgfqpoint{1.267472in}{1.732774in}}{\pgfqpoint{1.267472in}{1.710552in}}%
\pgfpathlineto{\pgfqpoint{1.267472in}{1.493884in}}%
\pgfpathquadraticcurveto{\pgfqpoint{1.267472in}{1.471662in}}{\pgfqpoint{1.289694in}{1.471662in}}%
\pgfpathlineto{\pgfqpoint{1.289694in}{1.471662in}}%
\pgfpathclose%
\pgfusepath{stroke,fill}%
\end{pgfscope}%
\begin{pgfscope}%
\pgfsetbuttcap%
\pgfsetroundjoin%
\pgfsetlinewidth{1.505625pt}%
\definecolor{currentstroke}{rgb}{0.003922,0.450980,0.698039}%
\pgfsetstrokecolor{currentstroke}%
\pgfsetdash{{5.550000pt}{2.400000pt}}{0.000000pt}%
\pgfpathmoveto{\pgfqpoint{1.311916in}{1.595930in}}%
\pgfpathlineto{\pgfqpoint{1.423028in}{1.595930in}}%
\pgfpathlineto{\pgfqpoint{1.534139in}{1.595930in}}%
\pgfusepath{stroke}%
\end{pgfscope}%
\begin{pgfscope}%
\definecolor{textcolor}{rgb}{0.000000,0.000000,0.000000}%
\pgfsetstrokecolor{textcolor}%
\pgfsetfillcolor{textcolor}%
\pgftext[x=1.623028in,y=1.557041in,left,base]{\color{textcolor}\rmfamily\fontsize{8.000000}{9.600000}\selectfont \(\displaystyle \propto\sqrt{h_{0}}\tau^{-0.5}\)}%
\end{pgfscope}%
\end{pgfpicture}%
\makeatother%
\endgroup%

        } % scalebox
        \caption{Allan deviation}
        \label{fig:white_noise_adev}
    \end{subfigure}
    \caption{Different representations of white noise.}
    \label{fig:white_noise_simulated}
\end{figure}

From this simulation, several features can be observed. First of all, the power spectral density is flat and constant with $h_0 = 2$, which is in accordance with table \ref{tab:adev_alpha_mu} and the normalization mentioned earlier. Figure \ref{fig:white_noise_adev} shows the typical $\tau^{-\frac1 2}$ dependence of white noise in the Allan deviation plot. This immediatly explains, why filtering white noise scales with $\frac{1}{\sqrt{n}}$ with $n$ being the number of samples averaged.

\minisec{Burst Noise}
Burst noise, popcorn noise, or sometimes referred to as random telegraph signal is a random bi-stable change in a signal and is caused by a generation recombination processes. This, for example, happens in semiconductors if there is a site, that can trap an electrons for a prologned period of time and then randomly release it. Imporities causing lattice defects are discussed in this context \cite{kay2012operational,burst_noise_psd,popcorn_noise_orgin,technote_ti_popcorn_noise}. Such latttice defects can also be introduced by ion implantation during doping.

The discussion is split into two parts. First the power spectral density is derived and then the Allan variance. The spectral density of burst noise caused by a single trap site was derived in \cite{burst_noise_wiener_khinchin} by \citeauthor{burst_noise_wiener_khinchin}. The author used the autocorrelation function of the burst noise signal and applied the Wiener-Khinchin (Wiener-Хи́нчин) theorem, which connects the autocorrelation function with the power spectral density. A nicely detailed derivation can be found in \cite{fundamentals_of_noise_processes}, which also discusses the preconditions, that must be met, like stationarity of the process. The burst noise signal consists of two energie levels, called $0$ and $1$, split by $\Delta y$. Multiple burst noise signals can be superimposed in a real device. This would then result in mutiple levels, but they can be treated separately. The measurement interval over an even number of transitions, so that one ends in the same state as the measurement has started, is the time $T$. The mean lifetime of the levels is called $\bar \tau_0$ and $\bar \tau_1$:
\begin{equation}
    \bar \tau_{0} \approx \frac 1 N \sum_{i}^N \tau_{0,i} \qquad \bar \tau_{1} \approx \frac 1 N \sum_{i}^N \tau_{1,i}
\end{equation}

Figure \ref{fig:burst_noise} shows a burst noise signal along with the definitions above.

\begin{figure}[h]
    \centering
    \scalebox{1}{%
        \import{figures/}{burst_noise.tex}
    } % scalebox
    \caption{A random burst noise signal.}
    \label{fig:burst_noise}
\end{figure}

Using these definitions, one can then derive \cite{burst_noise_wiener_khinchin}:
\begin{align}
    R_{xx} (T) &= (\Delta y)^2 \cdot \frac{\bar \tau_1 \bar \tau_0 e^{-\left(\frac{1}{\bar \tau_1}+\frac{1}{\bar \tau_0}\right)T}}{\left(\bar \tau_1 + \bar \tau_0\right)^2} \, \text{and}\\
    S(\omega) &= 4 R_{xx}(0) \frac{\frac{1}{\tau_1} + \frac{1}{\tau_0}}{1 + \omega^2 \left(\frac{1}{\bar \tau_1} + \frac{1}{\bar \tau_0}\right)^2} \qquad \omega > 0 . \label{eqn:burst_noise}
\end{align}
Note, that the power spectral density is the one-sided version, hence an additional factor of $2$ is included. The d.c. term was ommitted here and can usually be neglected, because it is not relevant for calculating the power spectral density of real measurement data as it only contributes a single peak at $\omega=0$. Using $\frac{1}{\bar \tau} = \frac{1}{\bar \tau_1} + \frac{1}{\bar \tau_0}$ equation \ref{eqn:burst_noise} can be rewritten to give a more intuitive form:
\begin{equation}
    S(\omega) = 4 R_{xx}(0) \frac{\bar \tau}{\omega^2 + \bar \tau^2}
\end{equation}
This is a Lorentzian function and it can be seen, that a single trap site has a power spectral density that is proportional to $\frac{1}{f^2}$ at higher frequencies and it is also immediately evident, that for low frequencies, the power spectral density must be flat.

With the spectral density in hand, it is now possible to calculate the Allan variance as it was done by

%\begin{figure}[ht]
%    \centering
%    %% Creator: Matplotlib, PGF backend
%%
%% To include the figure in your LaTeX document, write
%%   \input{<filename>.pgf}
%%
%% Make sure the required packages are loaded in your preamble
%%   \usepackage{pgf}
%%
%% Also ensure that all the required font packages are loaded; for instance,
%% the lmodern package is sometimes necessary when using math font.
%%   \usepackage{lmodern}
%%
%% Figures using additional raster images can only be included by \input if
%% they are in the same directory as the main LaTeX file. For loading figures
%% from other directories you can use the `import` package
%%   \usepackage{import}
%%
%% and then include the figures with
%%   \import{<path to file>}{<filename>.pgf}
%%
%% Matplotlib used the following preamble
%%   \usepackage{siunitx}
%%   \usepackage{fontspec}
%%
\begingroup%
\makeatletter%
\begin{pgfpicture}%
\pgfpathrectangle{\pgfpointorigin}{\pgfqpoint{5.490000in}{3.390000in}}%
\pgfusepath{use as bounding box, clip}%
\begin{pgfscope}%
\pgfsetbuttcap%
\pgfsetmiterjoin%
\definecolor{currentfill}{rgb}{1.000000,1.000000,1.000000}%
\pgfsetfillcolor{currentfill}%
\pgfsetlinewidth{0.000000pt}%
\definecolor{currentstroke}{rgb}{1.000000,1.000000,1.000000}%
\pgfsetstrokecolor{currentstroke}%
\pgfsetdash{}{0pt}%
\pgfpathmoveto{\pgfqpoint{0.000000in}{0.000000in}}%
\pgfpathlineto{\pgfqpoint{5.490000in}{0.000000in}}%
\pgfpathlineto{\pgfqpoint{5.490000in}{3.390000in}}%
\pgfpathlineto{\pgfqpoint{0.000000in}{3.390000in}}%
\pgfpathlineto{\pgfqpoint{0.000000in}{0.000000in}}%
\pgfpathclose%
\pgfusepath{fill}%
\end{pgfscope}%
\begin{pgfscope}%
\pgfsetbuttcap%
\pgfsetmiterjoin%
\definecolor{currentfill}{rgb}{1.000000,1.000000,1.000000}%
\pgfsetfillcolor{currentfill}%
\pgfsetlinewidth{0.000000pt}%
\definecolor{currentstroke}{rgb}{0.000000,0.000000,0.000000}%
\pgfsetstrokecolor{currentstroke}%
\pgfsetstrokeopacity{0.000000}%
\pgfsetdash{}{0pt}%
\pgfpathmoveto{\pgfqpoint{0.605343in}{0.417642in}}%
\pgfpathlineto{\pgfqpoint{5.448330in}{0.417642in}}%
\pgfpathlineto{\pgfqpoint{5.448330in}{3.348330in}}%
\pgfpathlineto{\pgfqpoint{0.605343in}{3.348330in}}%
\pgfpathlineto{\pgfqpoint{0.605343in}{0.417642in}}%
\pgfpathclose%
\pgfusepath{fill}%
\end{pgfscope}%
\begin{pgfscope}%
\pgfpathrectangle{\pgfqpoint{0.605343in}{0.417642in}}{\pgfqpoint{4.842987in}{2.930688in}}%
\pgfusepath{clip}%
\pgfsetrectcap%
\pgfsetroundjoin%
\pgfsetlinewidth{0.803000pt}%
\definecolor{currentstroke}{rgb}{0.450000,0.450000,0.450000}%
\pgfsetstrokecolor{currentstroke}%
\pgfsetdash{}{0pt}%
\pgfpathmoveto{\pgfqpoint{0.825479in}{0.417642in}}%
\pgfpathlineto{\pgfqpoint{0.825479in}{3.348330in}}%
\pgfusepath{stroke}%
\end{pgfscope}%
\begin{pgfscope}%
\pgfsetbuttcap%
\pgfsetroundjoin%
\definecolor{currentfill}{rgb}{0.000000,0.000000,0.000000}%
\pgfsetfillcolor{currentfill}%
\pgfsetlinewidth{0.803000pt}%
\definecolor{currentstroke}{rgb}{0.000000,0.000000,0.000000}%
\pgfsetstrokecolor{currentstroke}%
\pgfsetdash{}{0pt}%
\pgfsys@defobject{currentmarker}{\pgfqpoint{0.000000in}{-0.048611in}}{\pgfqpoint{0.000000in}{0.000000in}}{%
\pgfpathmoveto{\pgfqpoint{0.000000in}{0.000000in}}%
\pgfpathlineto{\pgfqpoint{0.000000in}{-0.048611in}}%
\pgfusepath{stroke,fill}%
}%
\begin{pgfscope}%
\pgfsys@transformshift{0.825479in}{0.417642in}%
\pgfsys@useobject{currentmarker}{}%
\end{pgfscope}%
\end{pgfscope}%
\begin{pgfscope}%
\definecolor{textcolor}{rgb}{0.000000,0.000000,0.000000}%
\pgfsetstrokecolor{textcolor}%
\pgfsetfillcolor{textcolor}%
\pgftext[x=0.825479in,y=0.320420in,,top]{\color{textcolor}\rmfamily\fontsize{8.000000}{9.600000}\selectfont \(\displaystyle {10^{-3}}\)}%
\end{pgfscope}%
\begin{pgfscope}%
\pgfpathrectangle{\pgfqpoint{0.605343in}{0.417642in}}{\pgfqpoint{4.842987in}{2.930688in}}%
\pgfusepath{clip}%
\pgfsetrectcap%
\pgfsetroundjoin%
\pgfsetlinewidth{0.803000pt}%
\definecolor{currentstroke}{rgb}{0.450000,0.450000,0.450000}%
\pgfsetstrokecolor{currentstroke}%
\pgfsetdash{}{0pt}%
\pgfpathmoveto{\pgfqpoint{1.559265in}{0.417642in}}%
\pgfpathlineto{\pgfqpoint{1.559265in}{3.348330in}}%
\pgfusepath{stroke}%
\end{pgfscope}%
\begin{pgfscope}%
\pgfsetbuttcap%
\pgfsetroundjoin%
\definecolor{currentfill}{rgb}{0.000000,0.000000,0.000000}%
\pgfsetfillcolor{currentfill}%
\pgfsetlinewidth{0.803000pt}%
\definecolor{currentstroke}{rgb}{0.000000,0.000000,0.000000}%
\pgfsetstrokecolor{currentstroke}%
\pgfsetdash{}{0pt}%
\pgfsys@defobject{currentmarker}{\pgfqpoint{0.000000in}{-0.048611in}}{\pgfqpoint{0.000000in}{0.000000in}}{%
\pgfpathmoveto{\pgfqpoint{0.000000in}{0.000000in}}%
\pgfpathlineto{\pgfqpoint{0.000000in}{-0.048611in}}%
\pgfusepath{stroke,fill}%
}%
\begin{pgfscope}%
\pgfsys@transformshift{1.559265in}{0.417642in}%
\pgfsys@useobject{currentmarker}{}%
\end{pgfscope}%
\end{pgfscope}%
\begin{pgfscope}%
\definecolor{textcolor}{rgb}{0.000000,0.000000,0.000000}%
\pgfsetstrokecolor{textcolor}%
\pgfsetfillcolor{textcolor}%
\pgftext[x=1.559265in,y=0.320420in,,top]{\color{textcolor}\rmfamily\fontsize{8.000000}{9.600000}\selectfont \(\displaystyle {10^{-2}}\)}%
\end{pgfscope}%
\begin{pgfscope}%
\pgfpathrectangle{\pgfqpoint{0.605343in}{0.417642in}}{\pgfqpoint{4.842987in}{2.930688in}}%
\pgfusepath{clip}%
\pgfsetrectcap%
\pgfsetroundjoin%
\pgfsetlinewidth{0.803000pt}%
\definecolor{currentstroke}{rgb}{0.450000,0.450000,0.450000}%
\pgfsetstrokecolor{currentstroke}%
\pgfsetdash{}{0pt}%
\pgfpathmoveto{\pgfqpoint{2.293051in}{0.417642in}}%
\pgfpathlineto{\pgfqpoint{2.293051in}{3.348330in}}%
\pgfusepath{stroke}%
\end{pgfscope}%
\begin{pgfscope}%
\pgfsetbuttcap%
\pgfsetroundjoin%
\definecolor{currentfill}{rgb}{0.000000,0.000000,0.000000}%
\pgfsetfillcolor{currentfill}%
\pgfsetlinewidth{0.803000pt}%
\definecolor{currentstroke}{rgb}{0.000000,0.000000,0.000000}%
\pgfsetstrokecolor{currentstroke}%
\pgfsetdash{}{0pt}%
\pgfsys@defobject{currentmarker}{\pgfqpoint{0.000000in}{-0.048611in}}{\pgfqpoint{0.000000in}{0.000000in}}{%
\pgfpathmoveto{\pgfqpoint{0.000000in}{0.000000in}}%
\pgfpathlineto{\pgfqpoint{0.000000in}{-0.048611in}}%
\pgfusepath{stroke,fill}%
}%
\begin{pgfscope}%
\pgfsys@transformshift{2.293051in}{0.417642in}%
\pgfsys@useobject{currentmarker}{}%
\end{pgfscope}%
\end{pgfscope}%
\begin{pgfscope}%
\definecolor{textcolor}{rgb}{0.000000,0.000000,0.000000}%
\pgfsetstrokecolor{textcolor}%
\pgfsetfillcolor{textcolor}%
\pgftext[x=2.293051in,y=0.320420in,,top]{\color{textcolor}\rmfamily\fontsize{8.000000}{9.600000}\selectfont \(\displaystyle {10^{-1}}\)}%
\end{pgfscope}%
\begin{pgfscope}%
\pgfpathrectangle{\pgfqpoint{0.605343in}{0.417642in}}{\pgfqpoint{4.842987in}{2.930688in}}%
\pgfusepath{clip}%
\pgfsetrectcap%
\pgfsetroundjoin%
\pgfsetlinewidth{0.803000pt}%
\definecolor{currentstroke}{rgb}{0.450000,0.450000,0.450000}%
\pgfsetstrokecolor{currentstroke}%
\pgfsetdash{}{0pt}%
\pgfpathmoveto{\pgfqpoint{3.026837in}{0.417642in}}%
\pgfpathlineto{\pgfqpoint{3.026837in}{3.348330in}}%
\pgfusepath{stroke}%
\end{pgfscope}%
\begin{pgfscope}%
\pgfsetbuttcap%
\pgfsetroundjoin%
\definecolor{currentfill}{rgb}{0.000000,0.000000,0.000000}%
\pgfsetfillcolor{currentfill}%
\pgfsetlinewidth{0.803000pt}%
\definecolor{currentstroke}{rgb}{0.000000,0.000000,0.000000}%
\pgfsetstrokecolor{currentstroke}%
\pgfsetdash{}{0pt}%
\pgfsys@defobject{currentmarker}{\pgfqpoint{0.000000in}{-0.048611in}}{\pgfqpoint{0.000000in}{0.000000in}}{%
\pgfpathmoveto{\pgfqpoint{0.000000in}{0.000000in}}%
\pgfpathlineto{\pgfqpoint{0.000000in}{-0.048611in}}%
\pgfusepath{stroke,fill}%
}%
\begin{pgfscope}%
\pgfsys@transformshift{3.026837in}{0.417642in}%
\pgfsys@useobject{currentmarker}{}%
\end{pgfscope}%
\end{pgfscope}%
\begin{pgfscope}%
\definecolor{textcolor}{rgb}{0.000000,0.000000,0.000000}%
\pgfsetstrokecolor{textcolor}%
\pgfsetfillcolor{textcolor}%
\pgftext[x=3.026837in,y=0.320420in,,top]{\color{textcolor}\rmfamily\fontsize{8.000000}{9.600000}\selectfont \(\displaystyle {10^{0}}\)}%
\end{pgfscope}%
\begin{pgfscope}%
\pgfpathrectangle{\pgfqpoint{0.605343in}{0.417642in}}{\pgfqpoint{4.842987in}{2.930688in}}%
\pgfusepath{clip}%
\pgfsetrectcap%
\pgfsetroundjoin%
\pgfsetlinewidth{0.803000pt}%
\definecolor{currentstroke}{rgb}{0.450000,0.450000,0.450000}%
\pgfsetstrokecolor{currentstroke}%
\pgfsetdash{}{0pt}%
\pgfpathmoveto{\pgfqpoint{3.760623in}{0.417642in}}%
\pgfpathlineto{\pgfqpoint{3.760623in}{3.348330in}}%
\pgfusepath{stroke}%
\end{pgfscope}%
\begin{pgfscope}%
\pgfsetbuttcap%
\pgfsetroundjoin%
\definecolor{currentfill}{rgb}{0.000000,0.000000,0.000000}%
\pgfsetfillcolor{currentfill}%
\pgfsetlinewidth{0.803000pt}%
\definecolor{currentstroke}{rgb}{0.000000,0.000000,0.000000}%
\pgfsetstrokecolor{currentstroke}%
\pgfsetdash{}{0pt}%
\pgfsys@defobject{currentmarker}{\pgfqpoint{0.000000in}{-0.048611in}}{\pgfqpoint{0.000000in}{0.000000in}}{%
\pgfpathmoveto{\pgfqpoint{0.000000in}{0.000000in}}%
\pgfpathlineto{\pgfqpoint{0.000000in}{-0.048611in}}%
\pgfusepath{stroke,fill}%
}%
\begin{pgfscope}%
\pgfsys@transformshift{3.760623in}{0.417642in}%
\pgfsys@useobject{currentmarker}{}%
\end{pgfscope}%
\end{pgfscope}%
\begin{pgfscope}%
\definecolor{textcolor}{rgb}{0.000000,0.000000,0.000000}%
\pgfsetstrokecolor{textcolor}%
\pgfsetfillcolor{textcolor}%
\pgftext[x=3.760623in,y=0.320420in,,top]{\color{textcolor}\rmfamily\fontsize{8.000000}{9.600000}\selectfont \(\displaystyle {10^{1}}\)}%
\end{pgfscope}%
\begin{pgfscope}%
\pgfpathrectangle{\pgfqpoint{0.605343in}{0.417642in}}{\pgfqpoint{4.842987in}{2.930688in}}%
\pgfusepath{clip}%
\pgfsetrectcap%
\pgfsetroundjoin%
\pgfsetlinewidth{0.803000pt}%
\definecolor{currentstroke}{rgb}{0.450000,0.450000,0.450000}%
\pgfsetstrokecolor{currentstroke}%
\pgfsetdash{}{0pt}%
\pgfpathmoveto{\pgfqpoint{4.494408in}{0.417642in}}%
\pgfpathlineto{\pgfqpoint{4.494408in}{3.348330in}}%
\pgfusepath{stroke}%
\end{pgfscope}%
\begin{pgfscope}%
\pgfsetbuttcap%
\pgfsetroundjoin%
\definecolor{currentfill}{rgb}{0.000000,0.000000,0.000000}%
\pgfsetfillcolor{currentfill}%
\pgfsetlinewidth{0.803000pt}%
\definecolor{currentstroke}{rgb}{0.000000,0.000000,0.000000}%
\pgfsetstrokecolor{currentstroke}%
\pgfsetdash{}{0pt}%
\pgfsys@defobject{currentmarker}{\pgfqpoint{0.000000in}{-0.048611in}}{\pgfqpoint{0.000000in}{0.000000in}}{%
\pgfpathmoveto{\pgfqpoint{0.000000in}{0.000000in}}%
\pgfpathlineto{\pgfqpoint{0.000000in}{-0.048611in}}%
\pgfusepath{stroke,fill}%
}%
\begin{pgfscope}%
\pgfsys@transformshift{4.494408in}{0.417642in}%
\pgfsys@useobject{currentmarker}{}%
\end{pgfscope}%
\end{pgfscope}%
\begin{pgfscope}%
\definecolor{textcolor}{rgb}{0.000000,0.000000,0.000000}%
\pgfsetstrokecolor{textcolor}%
\pgfsetfillcolor{textcolor}%
\pgftext[x=4.494408in,y=0.320420in,,top]{\color{textcolor}\rmfamily\fontsize{8.000000}{9.600000}\selectfont \(\displaystyle {10^{2}}\)}%
\end{pgfscope}%
\begin{pgfscope}%
\pgfpathrectangle{\pgfqpoint{0.605343in}{0.417642in}}{\pgfqpoint{4.842987in}{2.930688in}}%
\pgfusepath{clip}%
\pgfsetrectcap%
\pgfsetroundjoin%
\pgfsetlinewidth{0.803000pt}%
\definecolor{currentstroke}{rgb}{0.450000,0.450000,0.450000}%
\pgfsetstrokecolor{currentstroke}%
\pgfsetdash{}{0pt}%
\pgfpathmoveto{\pgfqpoint{5.228194in}{0.417642in}}%
\pgfpathlineto{\pgfqpoint{5.228194in}{3.348330in}}%
\pgfusepath{stroke}%
\end{pgfscope}%
\begin{pgfscope}%
\pgfsetbuttcap%
\pgfsetroundjoin%
\definecolor{currentfill}{rgb}{0.000000,0.000000,0.000000}%
\pgfsetfillcolor{currentfill}%
\pgfsetlinewidth{0.803000pt}%
\definecolor{currentstroke}{rgb}{0.000000,0.000000,0.000000}%
\pgfsetstrokecolor{currentstroke}%
\pgfsetdash{}{0pt}%
\pgfsys@defobject{currentmarker}{\pgfqpoint{0.000000in}{-0.048611in}}{\pgfqpoint{0.000000in}{0.000000in}}{%
\pgfpathmoveto{\pgfqpoint{0.000000in}{0.000000in}}%
\pgfpathlineto{\pgfqpoint{0.000000in}{-0.048611in}}%
\pgfusepath{stroke,fill}%
}%
\begin{pgfscope}%
\pgfsys@transformshift{5.228194in}{0.417642in}%
\pgfsys@useobject{currentmarker}{}%
\end{pgfscope}%
\end{pgfscope}%
\begin{pgfscope}%
\definecolor{textcolor}{rgb}{0.000000,0.000000,0.000000}%
\pgfsetstrokecolor{textcolor}%
\pgfsetfillcolor{textcolor}%
\pgftext[x=5.228194in,y=0.320420in,,top]{\color{textcolor}\rmfamily\fontsize{8.000000}{9.600000}\selectfont \(\displaystyle {10^{3}}\)}%
\end{pgfscope}%
\begin{pgfscope}%
\pgfpathrectangle{\pgfqpoint{0.605343in}{0.417642in}}{\pgfqpoint{4.842987in}{2.930688in}}%
\pgfusepath{clip}%
\pgfsetrectcap%
\pgfsetroundjoin%
\pgfsetlinewidth{0.803000pt}%
\definecolor{currentstroke}{rgb}{0.850000,0.850000,0.850000}%
\pgfsetstrokecolor{currentstroke}%
\pgfsetdash{}{0pt}%
\pgfpathmoveto{\pgfqpoint{0.662690in}{0.417642in}}%
\pgfpathlineto{\pgfqpoint{0.662690in}{3.348330in}}%
\pgfusepath{stroke}%
\end{pgfscope}%
\begin{pgfscope}%
\pgfsetbuttcap%
\pgfsetroundjoin%
\definecolor{currentfill}{rgb}{0.000000,0.000000,0.000000}%
\pgfsetfillcolor{currentfill}%
\pgfsetlinewidth{0.602250pt}%
\definecolor{currentstroke}{rgb}{0.000000,0.000000,0.000000}%
\pgfsetstrokecolor{currentstroke}%
\pgfsetdash{}{0pt}%
\pgfsys@defobject{currentmarker}{\pgfqpoint{0.000000in}{-0.027778in}}{\pgfqpoint{0.000000in}{0.000000in}}{%
\pgfpathmoveto{\pgfqpoint{0.000000in}{0.000000in}}%
\pgfpathlineto{\pgfqpoint{0.000000in}{-0.027778in}}%
\pgfusepath{stroke,fill}%
}%
\begin{pgfscope}%
\pgfsys@transformshift{0.662690in}{0.417642in}%
\pgfsys@useobject{currentmarker}{}%
\end{pgfscope}%
\end{pgfscope}%
\begin{pgfscope}%
\pgfpathrectangle{\pgfqpoint{0.605343in}{0.417642in}}{\pgfqpoint{4.842987in}{2.930688in}}%
\pgfusepath{clip}%
\pgfsetrectcap%
\pgfsetroundjoin%
\pgfsetlinewidth{0.803000pt}%
\definecolor{currentstroke}{rgb}{0.850000,0.850000,0.850000}%
\pgfsetstrokecolor{currentstroke}%
\pgfsetdash{}{0pt}%
\pgfpathmoveto{\pgfqpoint{0.711814in}{0.417642in}}%
\pgfpathlineto{\pgfqpoint{0.711814in}{3.348330in}}%
\pgfusepath{stroke}%
\end{pgfscope}%
\begin{pgfscope}%
\pgfsetbuttcap%
\pgfsetroundjoin%
\definecolor{currentfill}{rgb}{0.000000,0.000000,0.000000}%
\pgfsetfillcolor{currentfill}%
\pgfsetlinewidth{0.602250pt}%
\definecolor{currentstroke}{rgb}{0.000000,0.000000,0.000000}%
\pgfsetstrokecolor{currentstroke}%
\pgfsetdash{}{0pt}%
\pgfsys@defobject{currentmarker}{\pgfqpoint{0.000000in}{-0.027778in}}{\pgfqpoint{0.000000in}{0.000000in}}{%
\pgfpathmoveto{\pgfqpoint{0.000000in}{0.000000in}}%
\pgfpathlineto{\pgfqpoint{0.000000in}{-0.027778in}}%
\pgfusepath{stroke,fill}%
}%
\begin{pgfscope}%
\pgfsys@transformshift{0.711814in}{0.417642in}%
\pgfsys@useobject{currentmarker}{}%
\end{pgfscope}%
\end{pgfscope}%
\begin{pgfscope}%
\pgfpathrectangle{\pgfqpoint{0.605343in}{0.417642in}}{\pgfqpoint{4.842987in}{2.930688in}}%
\pgfusepath{clip}%
\pgfsetrectcap%
\pgfsetroundjoin%
\pgfsetlinewidth{0.803000pt}%
\definecolor{currentstroke}{rgb}{0.850000,0.850000,0.850000}%
\pgfsetstrokecolor{currentstroke}%
\pgfsetdash{}{0pt}%
\pgfpathmoveto{\pgfqpoint{0.754368in}{0.417642in}}%
\pgfpathlineto{\pgfqpoint{0.754368in}{3.348330in}}%
\pgfusepath{stroke}%
\end{pgfscope}%
\begin{pgfscope}%
\pgfsetbuttcap%
\pgfsetroundjoin%
\definecolor{currentfill}{rgb}{0.000000,0.000000,0.000000}%
\pgfsetfillcolor{currentfill}%
\pgfsetlinewidth{0.602250pt}%
\definecolor{currentstroke}{rgb}{0.000000,0.000000,0.000000}%
\pgfsetstrokecolor{currentstroke}%
\pgfsetdash{}{0pt}%
\pgfsys@defobject{currentmarker}{\pgfqpoint{0.000000in}{-0.027778in}}{\pgfqpoint{0.000000in}{0.000000in}}{%
\pgfpathmoveto{\pgfqpoint{0.000000in}{0.000000in}}%
\pgfpathlineto{\pgfqpoint{0.000000in}{-0.027778in}}%
\pgfusepath{stroke,fill}%
}%
\begin{pgfscope}%
\pgfsys@transformshift{0.754368in}{0.417642in}%
\pgfsys@useobject{currentmarker}{}%
\end{pgfscope}%
\end{pgfscope}%
\begin{pgfscope}%
\pgfpathrectangle{\pgfqpoint{0.605343in}{0.417642in}}{\pgfqpoint{4.842987in}{2.930688in}}%
\pgfusepath{clip}%
\pgfsetrectcap%
\pgfsetroundjoin%
\pgfsetlinewidth{0.803000pt}%
\definecolor{currentstroke}{rgb}{0.850000,0.850000,0.850000}%
\pgfsetstrokecolor{currentstroke}%
\pgfsetdash{}{0pt}%
\pgfpathmoveto{\pgfqpoint{0.791903in}{0.417642in}}%
\pgfpathlineto{\pgfqpoint{0.791903in}{3.348330in}}%
\pgfusepath{stroke}%
\end{pgfscope}%
\begin{pgfscope}%
\pgfsetbuttcap%
\pgfsetroundjoin%
\definecolor{currentfill}{rgb}{0.000000,0.000000,0.000000}%
\pgfsetfillcolor{currentfill}%
\pgfsetlinewidth{0.602250pt}%
\definecolor{currentstroke}{rgb}{0.000000,0.000000,0.000000}%
\pgfsetstrokecolor{currentstroke}%
\pgfsetdash{}{0pt}%
\pgfsys@defobject{currentmarker}{\pgfqpoint{0.000000in}{-0.027778in}}{\pgfqpoint{0.000000in}{0.000000in}}{%
\pgfpathmoveto{\pgfqpoint{0.000000in}{0.000000in}}%
\pgfpathlineto{\pgfqpoint{0.000000in}{-0.027778in}}%
\pgfusepath{stroke,fill}%
}%
\begin{pgfscope}%
\pgfsys@transformshift{0.791903in}{0.417642in}%
\pgfsys@useobject{currentmarker}{}%
\end{pgfscope}%
\end{pgfscope}%
\begin{pgfscope}%
\pgfpathrectangle{\pgfqpoint{0.605343in}{0.417642in}}{\pgfqpoint{4.842987in}{2.930688in}}%
\pgfusepath{clip}%
\pgfsetrectcap%
\pgfsetroundjoin%
\pgfsetlinewidth{0.803000pt}%
\definecolor{currentstroke}{rgb}{0.850000,0.850000,0.850000}%
\pgfsetstrokecolor{currentstroke}%
\pgfsetdash{}{0pt}%
\pgfpathmoveto{\pgfqpoint{1.046371in}{0.417642in}}%
\pgfpathlineto{\pgfqpoint{1.046371in}{3.348330in}}%
\pgfusepath{stroke}%
\end{pgfscope}%
\begin{pgfscope}%
\pgfsetbuttcap%
\pgfsetroundjoin%
\definecolor{currentfill}{rgb}{0.000000,0.000000,0.000000}%
\pgfsetfillcolor{currentfill}%
\pgfsetlinewidth{0.602250pt}%
\definecolor{currentstroke}{rgb}{0.000000,0.000000,0.000000}%
\pgfsetstrokecolor{currentstroke}%
\pgfsetdash{}{0pt}%
\pgfsys@defobject{currentmarker}{\pgfqpoint{0.000000in}{-0.027778in}}{\pgfqpoint{0.000000in}{0.000000in}}{%
\pgfpathmoveto{\pgfqpoint{0.000000in}{0.000000in}}%
\pgfpathlineto{\pgfqpoint{0.000000in}{-0.027778in}}%
\pgfusepath{stroke,fill}%
}%
\begin{pgfscope}%
\pgfsys@transformshift{1.046371in}{0.417642in}%
\pgfsys@useobject{currentmarker}{}%
\end{pgfscope}%
\end{pgfscope}%
\begin{pgfscope}%
\pgfpathrectangle{\pgfqpoint{0.605343in}{0.417642in}}{\pgfqpoint{4.842987in}{2.930688in}}%
\pgfusepath{clip}%
\pgfsetrectcap%
\pgfsetroundjoin%
\pgfsetlinewidth{0.803000pt}%
\definecolor{currentstroke}{rgb}{0.850000,0.850000,0.850000}%
\pgfsetstrokecolor{currentstroke}%
\pgfsetdash{}{0pt}%
\pgfpathmoveto{\pgfqpoint{1.175584in}{0.417642in}}%
\pgfpathlineto{\pgfqpoint{1.175584in}{3.348330in}}%
\pgfusepath{stroke}%
\end{pgfscope}%
\begin{pgfscope}%
\pgfsetbuttcap%
\pgfsetroundjoin%
\definecolor{currentfill}{rgb}{0.000000,0.000000,0.000000}%
\pgfsetfillcolor{currentfill}%
\pgfsetlinewidth{0.602250pt}%
\definecolor{currentstroke}{rgb}{0.000000,0.000000,0.000000}%
\pgfsetstrokecolor{currentstroke}%
\pgfsetdash{}{0pt}%
\pgfsys@defobject{currentmarker}{\pgfqpoint{0.000000in}{-0.027778in}}{\pgfqpoint{0.000000in}{0.000000in}}{%
\pgfpathmoveto{\pgfqpoint{0.000000in}{0.000000in}}%
\pgfpathlineto{\pgfqpoint{0.000000in}{-0.027778in}}%
\pgfusepath{stroke,fill}%
}%
\begin{pgfscope}%
\pgfsys@transformshift{1.175584in}{0.417642in}%
\pgfsys@useobject{currentmarker}{}%
\end{pgfscope}%
\end{pgfscope}%
\begin{pgfscope}%
\pgfpathrectangle{\pgfqpoint{0.605343in}{0.417642in}}{\pgfqpoint{4.842987in}{2.930688in}}%
\pgfusepath{clip}%
\pgfsetrectcap%
\pgfsetroundjoin%
\pgfsetlinewidth{0.803000pt}%
\definecolor{currentstroke}{rgb}{0.850000,0.850000,0.850000}%
\pgfsetstrokecolor{currentstroke}%
\pgfsetdash{}{0pt}%
\pgfpathmoveto{\pgfqpoint{1.267262in}{0.417642in}}%
\pgfpathlineto{\pgfqpoint{1.267262in}{3.348330in}}%
\pgfusepath{stroke}%
\end{pgfscope}%
\begin{pgfscope}%
\pgfsetbuttcap%
\pgfsetroundjoin%
\definecolor{currentfill}{rgb}{0.000000,0.000000,0.000000}%
\pgfsetfillcolor{currentfill}%
\pgfsetlinewidth{0.602250pt}%
\definecolor{currentstroke}{rgb}{0.000000,0.000000,0.000000}%
\pgfsetstrokecolor{currentstroke}%
\pgfsetdash{}{0pt}%
\pgfsys@defobject{currentmarker}{\pgfqpoint{0.000000in}{-0.027778in}}{\pgfqpoint{0.000000in}{0.000000in}}{%
\pgfpathmoveto{\pgfqpoint{0.000000in}{0.000000in}}%
\pgfpathlineto{\pgfqpoint{0.000000in}{-0.027778in}}%
\pgfusepath{stroke,fill}%
}%
\begin{pgfscope}%
\pgfsys@transformshift{1.267262in}{0.417642in}%
\pgfsys@useobject{currentmarker}{}%
\end{pgfscope}%
\end{pgfscope}%
\begin{pgfscope}%
\pgfpathrectangle{\pgfqpoint{0.605343in}{0.417642in}}{\pgfqpoint{4.842987in}{2.930688in}}%
\pgfusepath{clip}%
\pgfsetrectcap%
\pgfsetroundjoin%
\pgfsetlinewidth{0.803000pt}%
\definecolor{currentstroke}{rgb}{0.850000,0.850000,0.850000}%
\pgfsetstrokecolor{currentstroke}%
\pgfsetdash{}{0pt}%
\pgfpathmoveto{\pgfqpoint{1.338374in}{0.417642in}}%
\pgfpathlineto{\pgfqpoint{1.338374in}{3.348330in}}%
\pgfusepath{stroke}%
\end{pgfscope}%
\begin{pgfscope}%
\pgfsetbuttcap%
\pgfsetroundjoin%
\definecolor{currentfill}{rgb}{0.000000,0.000000,0.000000}%
\pgfsetfillcolor{currentfill}%
\pgfsetlinewidth{0.602250pt}%
\definecolor{currentstroke}{rgb}{0.000000,0.000000,0.000000}%
\pgfsetstrokecolor{currentstroke}%
\pgfsetdash{}{0pt}%
\pgfsys@defobject{currentmarker}{\pgfqpoint{0.000000in}{-0.027778in}}{\pgfqpoint{0.000000in}{0.000000in}}{%
\pgfpathmoveto{\pgfqpoint{0.000000in}{0.000000in}}%
\pgfpathlineto{\pgfqpoint{0.000000in}{-0.027778in}}%
\pgfusepath{stroke,fill}%
}%
\begin{pgfscope}%
\pgfsys@transformshift{1.338374in}{0.417642in}%
\pgfsys@useobject{currentmarker}{}%
\end{pgfscope}%
\end{pgfscope}%
\begin{pgfscope}%
\pgfpathrectangle{\pgfqpoint{0.605343in}{0.417642in}}{\pgfqpoint{4.842987in}{2.930688in}}%
\pgfusepath{clip}%
\pgfsetrectcap%
\pgfsetroundjoin%
\pgfsetlinewidth{0.803000pt}%
\definecolor{currentstroke}{rgb}{0.850000,0.850000,0.850000}%
\pgfsetstrokecolor{currentstroke}%
\pgfsetdash{}{0pt}%
\pgfpathmoveto{\pgfqpoint{1.396476in}{0.417642in}}%
\pgfpathlineto{\pgfqpoint{1.396476in}{3.348330in}}%
\pgfusepath{stroke}%
\end{pgfscope}%
\begin{pgfscope}%
\pgfsetbuttcap%
\pgfsetroundjoin%
\definecolor{currentfill}{rgb}{0.000000,0.000000,0.000000}%
\pgfsetfillcolor{currentfill}%
\pgfsetlinewidth{0.602250pt}%
\definecolor{currentstroke}{rgb}{0.000000,0.000000,0.000000}%
\pgfsetstrokecolor{currentstroke}%
\pgfsetdash{}{0pt}%
\pgfsys@defobject{currentmarker}{\pgfqpoint{0.000000in}{-0.027778in}}{\pgfqpoint{0.000000in}{0.000000in}}{%
\pgfpathmoveto{\pgfqpoint{0.000000in}{0.000000in}}%
\pgfpathlineto{\pgfqpoint{0.000000in}{-0.027778in}}%
\pgfusepath{stroke,fill}%
}%
\begin{pgfscope}%
\pgfsys@transformshift{1.396476in}{0.417642in}%
\pgfsys@useobject{currentmarker}{}%
\end{pgfscope}%
\end{pgfscope}%
\begin{pgfscope}%
\pgfpathrectangle{\pgfqpoint{0.605343in}{0.417642in}}{\pgfqpoint{4.842987in}{2.930688in}}%
\pgfusepath{clip}%
\pgfsetrectcap%
\pgfsetroundjoin%
\pgfsetlinewidth{0.803000pt}%
\definecolor{currentstroke}{rgb}{0.850000,0.850000,0.850000}%
\pgfsetstrokecolor{currentstroke}%
\pgfsetdash{}{0pt}%
\pgfpathmoveto{\pgfqpoint{1.445600in}{0.417642in}}%
\pgfpathlineto{\pgfqpoint{1.445600in}{3.348330in}}%
\pgfusepath{stroke}%
\end{pgfscope}%
\begin{pgfscope}%
\pgfsetbuttcap%
\pgfsetroundjoin%
\definecolor{currentfill}{rgb}{0.000000,0.000000,0.000000}%
\pgfsetfillcolor{currentfill}%
\pgfsetlinewidth{0.602250pt}%
\definecolor{currentstroke}{rgb}{0.000000,0.000000,0.000000}%
\pgfsetstrokecolor{currentstroke}%
\pgfsetdash{}{0pt}%
\pgfsys@defobject{currentmarker}{\pgfqpoint{0.000000in}{-0.027778in}}{\pgfqpoint{0.000000in}{0.000000in}}{%
\pgfpathmoveto{\pgfqpoint{0.000000in}{0.000000in}}%
\pgfpathlineto{\pgfqpoint{0.000000in}{-0.027778in}}%
\pgfusepath{stroke,fill}%
}%
\begin{pgfscope}%
\pgfsys@transformshift{1.445600in}{0.417642in}%
\pgfsys@useobject{currentmarker}{}%
\end{pgfscope}%
\end{pgfscope}%
\begin{pgfscope}%
\pgfpathrectangle{\pgfqpoint{0.605343in}{0.417642in}}{\pgfqpoint{4.842987in}{2.930688in}}%
\pgfusepath{clip}%
\pgfsetrectcap%
\pgfsetroundjoin%
\pgfsetlinewidth{0.803000pt}%
\definecolor{currentstroke}{rgb}{0.850000,0.850000,0.850000}%
\pgfsetstrokecolor{currentstroke}%
\pgfsetdash{}{0pt}%
\pgfpathmoveto{\pgfqpoint{1.488154in}{0.417642in}}%
\pgfpathlineto{\pgfqpoint{1.488154in}{3.348330in}}%
\pgfusepath{stroke}%
\end{pgfscope}%
\begin{pgfscope}%
\pgfsetbuttcap%
\pgfsetroundjoin%
\definecolor{currentfill}{rgb}{0.000000,0.000000,0.000000}%
\pgfsetfillcolor{currentfill}%
\pgfsetlinewidth{0.602250pt}%
\definecolor{currentstroke}{rgb}{0.000000,0.000000,0.000000}%
\pgfsetstrokecolor{currentstroke}%
\pgfsetdash{}{0pt}%
\pgfsys@defobject{currentmarker}{\pgfqpoint{0.000000in}{-0.027778in}}{\pgfqpoint{0.000000in}{0.000000in}}{%
\pgfpathmoveto{\pgfqpoint{0.000000in}{0.000000in}}%
\pgfpathlineto{\pgfqpoint{0.000000in}{-0.027778in}}%
\pgfusepath{stroke,fill}%
}%
\begin{pgfscope}%
\pgfsys@transformshift{1.488154in}{0.417642in}%
\pgfsys@useobject{currentmarker}{}%
\end{pgfscope}%
\end{pgfscope}%
\begin{pgfscope}%
\pgfpathrectangle{\pgfqpoint{0.605343in}{0.417642in}}{\pgfqpoint{4.842987in}{2.930688in}}%
\pgfusepath{clip}%
\pgfsetrectcap%
\pgfsetroundjoin%
\pgfsetlinewidth{0.803000pt}%
\definecolor{currentstroke}{rgb}{0.850000,0.850000,0.850000}%
\pgfsetstrokecolor{currentstroke}%
\pgfsetdash{}{0pt}%
\pgfpathmoveto{\pgfqpoint{1.525689in}{0.417642in}}%
\pgfpathlineto{\pgfqpoint{1.525689in}{3.348330in}}%
\pgfusepath{stroke}%
\end{pgfscope}%
\begin{pgfscope}%
\pgfsetbuttcap%
\pgfsetroundjoin%
\definecolor{currentfill}{rgb}{0.000000,0.000000,0.000000}%
\pgfsetfillcolor{currentfill}%
\pgfsetlinewidth{0.602250pt}%
\definecolor{currentstroke}{rgb}{0.000000,0.000000,0.000000}%
\pgfsetstrokecolor{currentstroke}%
\pgfsetdash{}{0pt}%
\pgfsys@defobject{currentmarker}{\pgfqpoint{0.000000in}{-0.027778in}}{\pgfqpoint{0.000000in}{0.000000in}}{%
\pgfpathmoveto{\pgfqpoint{0.000000in}{0.000000in}}%
\pgfpathlineto{\pgfqpoint{0.000000in}{-0.027778in}}%
\pgfusepath{stroke,fill}%
}%
\begin{pgfscope}%
\pgfsys@transformshift{1.525689in}{0.417642in}%
\pgfsys@useobject{currentmarker}{}%
\end{pgfscope}%
\end{pgfscope}%
\begin{pgfscope}%
\pgfpathrectangle{\pgfqpoint{0.605343in}{0.417642in}}{\pgfqpoint{4.842987in}{2.930688in}}%
\pgfusepath{clip}%
\pgfsetrectcap%
\pgfsetroundjoin%
\pgfsetlinewidth{0.803000pt}%
\definecolor{currentstroke}{rgb}{0.850000,0.850000,0.850000}%
\pgfsetstrokecolor{currentstroke}%
\pgfsetdash{}{0pt}%
\pgfpathmoveto{\pgfqpoint{1.780157in}{0.417642in}}%
\pgfpathlineto{\pgfqpoint{1.780157in}{3.348330in}}%
\pgfusepath{stroke}%
\end{pgfscope}%
\begin{pgfscope}%
\pgfsetbuttcap%
\pgfsetroundjoin%
\definecolor{currentfill}{rgb}{0.000000,0.000000,0.000000}%
\pgfsetfillcolor{currentfill}%
\pgfsetlinewidth{0.602250pt}%
\definecolor{currentstroke}{rgb}{0.000000,0.000000,0.000000}%
\pgfsetstrokecolor{currentstroke}%
\pgfsetdash{}{0pt}%
\pgfsys@defobject{currentmarker}{\pgfqpoint{0.000000in}{-0.027778in}}{\pgfqpoint{0.000000in}{0.000000in}}{%
\pgfpathmoveto{\pgfqpoint{0.000000in}{0.000000in}}%
\pgfpathlineto{\pgfqpoint{0.000000in}{-0.027778in}}%
\pgfusepath{stroke,fill}%
}%
\begin{pgfscope}%
\pgfsys@transformshift{1.780157in}{0.417642in}%
\pgfsys@useobject{currentmarker}{}%
\end{pgfscope}%
\end{pgfscope}%
\begin{pgfscope}%
\pgfpathrectangle{\pgfqpoint{0.605343in}{0.417642in}}{\pgfqpoint{4.842987in}{2.930688in}}%
\pgfusepath{clip}%
\pgfsetrectcap%
\pgfsetroundjoin%
\pgfsetlinewidth{0.803000pt}%
\definecolor{currentstroke}{rgb}{0.850000,0.850000,0.850000}%
\pgfsetstrokecolor{currentstroke}%
\pgfsetdash{}{0pt}%
\pgfpathmoveto{\pgfqpoint{1.909370in}{0.417642in}}%
\pgfpathlineto{\pgfqpoint{1.909370in}{3.348330in}}%
\pgfusepath{stroke}%
\end{pgfscope}%
\begin{pgfscope}%
\pgfsetbuttcap%
\pgfsetroundjoin%
\definecolor{currentfill}{rgb}{0.000000,0.000000,0.000000}%
\pgfsetfillcolor{currentfill}%
\pgfsetlinewidth{0.602250pt}%
\definecolor{currentstroke}{rgb}{0.000000,0.000000,0.000000}%
\pgfsetstrokecolor{currentstroke}%
\pgfsetdash{}{0pt}%
\pgfsys@defobject{currentmarker}{\pgfqpoint{0.000000in}{-0.027778in}}{\pgfqpoint{0.000000in}{0.000000in}}{%
\pgfpathmoveto{\pgfqpoint{0.000000in}{0.000000in}}%
\pgfpathlineto{\pgfqpoint{0.000000in}{-0.027778in}}%
\pgfusepath{stroke,fill}%
}%
\begin{pgfscope}%
\pgfsys@transformshift{1.909370in}{0.417642in}%
\pgfsys@useobject{currentmarker}{}%
\end{pgfscope}%
\end{pgfscope}%
\begin{pgfscope}%
\pgfpathrectangle{\pgfqpoint{0.605343in}{0.417642in}}{\pgfqpoint{4.842987in}{2.930688in}}%
\pgfusepath{clip}%
\pgfsetrectcap%
\pgfsetroundjoin%
\pgfsetlinewidth{0.803000pt}%
\definecolor{currentstroke}{rgb}{0.850000,0.850000,0.850000}%
\pgfsetstrokecolor{currentstroke}%
\pgfsetdash{}{0pt}%
\pgfpathmoveto{\pgfqpoint{2.001048in}{0.417642in}}%
\pgfpathlineto{\pgfqpoint{2.001048in}{3.348330in}}%
\pgfusepath{stroke}%
\end{pgfscope}%
\begin{pgfscope}%
\pgfsetbuttcap%
\pgfsetroundjoin%
\definecolor{currentfill}{rgb}{0.000000,0.000000,0.000000}%
\pgfsetfillcolor{currentfill}%
\pgfsetlinewidth{0.602250pt}%
\definecolor{currentstroke}{rgb}{0.000000,0.000000,0.000000}%
\pgfsetstrokecolor{currentstroke}%
\pgfsetdash{}{0pt}%
\pgfsys@defobject{currentmarker}{\pgfqpoint{0.000000in}{-0.027778in}}{\pgfqpoint{0.000000in}{0.000000in}}{%
\pgfpathmoveto{\pgfqpoint{0.000000in}{0.000000in}}%
\pgfpathlineto{\pgfqpoint{0.000000in}{-0.027778in}}%
\pgfusepath{stroke,fill}%
}%
\begin{pgfscope}%
\pgfsys@transformshift{2.001048in}{0.417642in}%
\pgfsys@useobject{currentmarker}{}%
\end{pgfscope}%
\end{pgfscope}%
\begin{pgfscope}%
\pgfpathrectangle{\pgfqpoint{0.605343in}{0.417642in}}{\pgfqpoint{4.842987in}{2.930688in}}%
\pgfusepath{clip}%
\pgfsetrectcap%
\pgfsetroundjoin%
\pgfsetlinewidth{0.803000pt}%
\definecolor{currentstroke}{rgb}{0.850000,0.850000,0.850000}%
\pgfsetstrokecolor{currentstroke}%
\pgfsetdash{}{0pt}%
\pgfpathmoveto{\pgfqpoint{2.072159in}{0.417642in}}%
\pgfpathlineto{\pgfqpoint{2.072159in}{3.348330in}}%
\pgfusepath{stroke}%
\end{pgfscope}%
\begin{pgfscope}%
\pgfsetbuttcap%
\pgfsetroundjoin%
\definecolor{currentfill}{rgb}{0.000000,0.000000,0.000000}%
\pgfsetfillcolor{currentfill}%
\pgfsetlinewidth{0.602250pt}%
\definecolor{currentstroke}{rgb}{0.000000,0.000000,0.000000}%
\pgfsetstrokecolor{currentstroke}%
\pgfsetdash{}{0pt}%
\pgfsys@defobject{currentmarker}{\pgfqpoint{0.000000in}{-0.027778in}}{\pgfqpoint{0.000000in}{0.000000in}}{%
\pgfpathmoveto{\pgfqpoint{0.000000in}{0.000000in}}%
\pgfpathlineto{\pgfqpoint{0.000000in}{-0.027778in}}%
\pgfusepath{stroke,fill}%
}%
\begin{pgfscope}%
\pgfsys@transformshift{2.072159in}{0.417642in}%
\pgfsys@useobject{currentmarker}{}%
\end{pgfscope}%
\end{pgfscope}%
\begin{pgfscope}%
\pgfpathrectangle{\pgfqpoint{0.605343in}{0.417642in}}{\pgfqpoint{4.842987in}{2.930688in}}%
\pgfusepath{clip}%
\pgfsetrectcap%
\pgfsetroundjoin%
\pgfsetlinewidth{0.803000pt}%
\definecolor{currentstroke}{rgb}{0.850000,0.850000,0.850000}%
\pgfsetstrokecolor{currentstroke}%
\pgfsetdash{}{0pt}%
\pgfpathmoveto{\pgfqpoint{2.130261in}{0.417642in}}%
\pgfpathlineto{\pgfqpoint{2.130261in}{3.348330in}}%
\pgfusepath{stroke}%
\end{pgfscope}%
\begin{pgfscope}%
\pgfsetbuttcap%
\pgfsetroundjoin%
\definecolor{currentfill}{rgb}{0.000000,0.000000,0.000000}%
\pgfsetfillcolor{currentfill}%
\pgfsetlinewidth{0.602250pt}%
\definecolor{currentstroke}{rgb}{0.000000,0.000000,0.000000}%
\pgfsetstrokecolor{currentstroke}%
\pgfsetdash{}{0pt}%
\pgfsys@defobject{currentmarker}{\pgfqpoint{0.000000in}{-0.027778in}}{\pgfqpoint{0.000000in}{0.000000in}}{%
\pgfpathmoveto{\pgfqpoint{0.000000in}{0.000000in}}%
\pgfpathlineto{\pgfqpoint{0.000000in}{-0.027778in}}%
\pgfusepath{stroke,fill}%
}%
\begin{pgfscope}%
\pgfsys@transformshift{2.130261in}{0.417642in}%
\pgfsys@useobject{currentmarker}{}%
\end{pgfscope}%
\end{pgfscope}%
\begin{pgfscope}%
\pgfpathrectangle{\pgfqpoint{0.605343in}{0.417642in}}{\pgfqpoint{4.842987in}{2.930688in}}%
\pgfusepath{clip}%
\pgfsetrectcap%
\pgfsetroundjoin%
\pgfsetlinewidth{0.803000pt}%
\definecolor{currentstroke}{rgb}{0.850000,0.850000,0.850000}%
\pgfsetstrokecolor{currentstroke}%
\pgfsetdash{}{0pt}%
\pgfpathmoveto{\pgfqpoint{2.179386in}{0.417642in}}%
\pgfpathlineto{\pgfqpoint{2.179386in}{3.348330in}}%
\pgfusepath{stroke}%
\end{pgfscope}%
\begin{pgfscope}%
\pgfsetbuttcap%
\pgfsetroundjoin%
\definecolor{currentfill}{rgb}{0.000000,0.000000,0.000000}%
\pgfsetfillcolor{currentfill}%
\pgfsetlinewidth{0.602250pt}%
\definecolor{currentstroke}{rgb}{0.000000,0.000000,0.000000}%
\pgfsetstrokecolor{currentstroke}%
\pgfsetdash{}{0pt}%
\pgfsys@defobject{currentmarker}{\pgfqpoint{0.000000in}{-0.027778in}}{\pgfqpoint{0.000000in}{0.000000in}}{%
\pgfpathmoveto{\pgfqpoint{0.000000in}{0.000000in}}%
\pgfpathlineto{\pgfqpoint{0.000000in}{-0.027778in}}%
\pgfusepath{stroke,fill}%
}%
\begin{pgfscope}%
\pgfsys@transformshift{2.179386in}{0.417642in}%
\pgfsys@useobject{currentmarker}{}%
\end{pgfscope}%
\end{pgfscope}%
\begin{pgfscope}%
\pgfpathrectangle{\pgfqpoint{0.605343in}{0.417642in}}{\pgfqpoint{4.842987in}{2.930688in}}%
\pgfusepath{clip}%
\pgfsetrectcap%
\pgfsetroundjoin%
\pgfsetlinewidth{0.803000pt}%
\definecolor{currentstroke}{rgb}{0.850000,0.850000,0.850000}%
\pgfsetstrokecolor{currentstroke}%
\pgfsetdash{}{0pt}%
\pgfpathmoveto{\pgfqpoint{2.221940in}{0.417642in}}%
\pgfpathlineto{\pgfqpoint{2.221940in}{3.348330in}}%
\pgfusepath{stroke}%
\end{pgfscope}%
\begin{pgfscope}%
\pgfsetbuttcap%
\pgfsetroundjoin%
\definecolor{currentfill}{rgb}{0.000000,0.000000,0.000000}%
\pgfsetfillcolor{currentfill}%
\pgfsetlinewidth{0.602250pt}%
\definecolor{currentstroke}{rgb}{0.000000,0.000000,0.000000}%
\pgfsetstrokecolor{currentstroke}%
\pgfsetdash{}{0pt}%
\pgfsys@defobject{currentmarker}{\pgfqpoint{0.000000in}{-0.027778in}}{\pgfqpoint{0.000000in}{0.000000in}}{%
\pgfpathmoveto{\pgfqpoint{0.000000in}{0.000000in}}%
\pgfpathlineto{\pgfqpoint{0.000000in}{-0.027778in}}%
\pgfusepath{stroke,fill}%
}%
\begin{pgfscope}%
\pgfsys@transformshift{2.221940in}{0.417642in}%
\pgfsys@useobject{currentmarker}{}%
\end{pgfscope}%
\end{pgfscope}%
\begin{pgfscope}%
\pgfpathrectangle{\pgfqpoint{0.605343in}{0.417642in}}{\pgfqpoint{4.842987in}{2.930688in}}%
\pgfusepath{clip}%
\pgfsetrectcap%
\pgfsetroundjoin%
\pgfsetlinewidth{0.803000pt}%
\definecolor{currentstroke}{rgb}{0.850000,0.850000,0.850000}%
\pgfsetstrokecolor{currentstroke}%
\pgfsetdash{}{0pt}%
\pgfpathmoveto{\pgfqpoint{2.259475in}{0.417642in}}%
\pgfpathlineto{\pgfqpoint{2.259475in}{3.348330in}}%
\pgfusepath{stroke}%
\end{pgfscope}%
\begin{pgfscope}%
\pgfsetbuttcap%
\pgfsetroundjoin%
\definecolor{currentfill}{rgb}{0.000000,0.000000,0.000000}%
\pgfsetfillcolor{currentfill}%
\pgfsetlinewidth{0.602250pt}%
\definecolor{currentstroke}{rgb}{0.000000,0.000000,0.000000}%
\pgfsetstrokecolor{currentstroke}%
\pgfsetdash{}{0pt}%
\pgfsys@defobject{currentmarker}{\pgfqpoint{0.000000in}{-0.027778in}}{\pgfqpoint{0.000000in}{0.000000in}}{%
\pgfpathmoveto{\pgfqpoint{0.000000in}{0.000000in}}%
\pgfpathlineto{\pgfqpoint{0.000000in}{-0.027778in}}%
\pgfusepath{stroke,fill}%
}%
\begin{pgfscope}%
\pgfsys@transformshift{2.259475in}{0.417642in}%
\pgfsys@useobject{currentmarker}{}%
\end{pgfscope}%
\end{pgfscope}%
\begin{pgfscope}%
\pgfpathrectangle{\pgfqpoint{0.605343in}{0.417642in}}{\pgfqpoint{4.842987in}{2.930688in}}%
\pgfusepath{clip}%
\pgfsetrectcap%
\pgfsetroundjoin%
\pgfsetlinewidth{0.803000pt}%
\definecolor{currentstroke}{rgb}{0.850000,0.850000,0.850000}%
\pgfsetstrokecolor{currentstroke}%
\pgfsetdash{}{0pt}%
\pgfpathmoveto{\pgfqpoint{2.513942in}{0.417642in}}%
\pgfpathlineto{\pgfqpoint{2.513942in}{3.348330in}}%
\pgfusepath{stroke}%
\end{pgfscope}%
\begin{pgfscope}%
\pgfsetbuttcap%
\pgfsetroundjoin%
\definecolor{currentfill}{rgb}{0.000000,0.000000,0.000000}%
\pgfsetfillcolor{currentfill}%
\pgfsetlinewidth{0.602250pt}%
\definecolor{currentstroke}{rgb}{0.000000,0.000000,0.000000}%
\pgfsetstrokecolor{currentstroke}%
\pgfsetdash{}{0pt}%
\pgfsys@defobject{currentmarker}{\pgfqpoint{0.000000in}{-0.027778in}}{\pgfqpoint{0.000000in}{0.000000in}}{%
\pgfpathmoveto{\pgfqpoint{0.000000in}{0.000000in}}%
\pgfpathlineto{\pgfqpoint{0.000000in}{-0.027778in}}%
\pgfusepath{stroke,fill}%
}%
\begin{pgfscope}%
\pgfsys@transformshift{2.513942in}{0.417642in}%
\pgfsys@useobject{currentmarker}{}%
\end{pgfscope}%
\end{pgfscope}%
\begin{pgfscope}%
\pgfpathrectangle{\pgfqpoint{0.605343in}{0.417642in}}{\pgfqpoint{4.842987in}{2.930688in}}%
\pgfusepath{clip}%
\pgfsetrectcap%
\pgfsetroundjoin%
\pgfsetlinewidth{0.803000pt}%
\definecolor{currentstroke}{rgb}{0.850000,0.850000,0.850000}%
\pgfsetstrokecolor{currentstroke}%
\pgfsetdash{}{0pt}%
\pgfpathmoveto{\pgfqpoint{2.643156in}{0.417642in}}%
\pgfpathlineto{\pgfqpoint{2.643156in}{3.348330in}}%
\pgfusepath{stroke}%
\end{pgfscope}%
\begin{pgfscope}%
\pgfsetbuttcap%
\pgfsetroundjoin%
\definecolor{currentfill}{rgb}{0.000000,0.000000,0.000000}%
\pgfsetfillcolor{currentfill}%
\pgfsetlinewidth{0.602250pt}%
\definecolor{currentstroke}{rgb}{0.000000,0.000000,0.000000}%
\pgfsetstrokecolor{currentstroke}%
\pgfsetdash{}{0pt}%
\pgfsys@defobject{currentmarker}{\pgfqpoint{0.000000in}{-0.027778in}}{\pgfqpoint{0.000000in}{0.000000in}}{%
\pgfpathmoveto{\pgfqpoint{0.000000in}{0.000000in}}%
\pgfpathlineto{\pgfqpoint{0.000000in}{-0.027778in}}%
\pgfusepath{stroke,fill}%
}%
\begin{pgfscope}%
\pgfsys@transformshift{2.643156in}{0.417642in}%
\pgfsys@useobject{currentmarker}{}%
\end{pgfscope}%
\end{pgfscope}%
\begin{pgfscope}%
\pgfpathrectangle{\pgfqpoint{0.605343in}{0.417642in}}{\pgfqpoint{4.842987in}{2.930688in}}%
\pgfusepath{clip}%
\pgfsetrectcap%
\pgfsetroundjoin%
\pgfsetlinewidth{0.803000pt}%
\definecolor{currentstroke}{rgb}{0.850000,0.850000,0.850000}%
\pgfsetstrokecolor{currentstroke}%
\pgfsetdash{}{0pt}%
\pgfpathmoveto{\pgfqpoint{2.734834in}{0.417642in}}%
\pgfpathlineto{\pgfqpoint{2.734834in}{3.348330in}}%
\pgfusepath{stroke}%
\end{pgfscope}%
\begin{pgfscope}%
\pgfsetbuttcap%
\pgfsetroundjoin%
\definecolor{currentfill}{rgb}{0.000000,0.000000,0.000000}%
\pgfsetfillcolor{currentfill}%
\pgfsetlinewidth{0.602250pt}%
\definecolor{currentstroke}{rgb}{0.000000,0.000000,0.000000}%
\pgfsetstrokecolor{currentstroke}%
\pgfsetdash{}{0pt}%
\pgfsys@defobject{currentmarker}{\pgfqpoint{0.000000in}{-0.027778in}}{\pgfqpoint{0.000000in}{0.000000in}}{%
\pgfpathmoveto{\pgfqpoint{0.000000in}{0.000000in}}%
\pgfpathlineto{\pgfqpoint{0.000000in}{-0.027778in}}%
\pgfusepath{stroke,fill}%
}%
\begin{pgfscope}%
\pgfsys@transformshift{2.734834in}{0.417642in}%
\pgfsys@useobject{currentmarker}{}%
\end{pgfscope}%
\end{pgfscope}%
\begin{pgfscope}%
\pgfpathrectangle{\pgfqpoint{0.605343in}{0.417642in}}{\pgfqpoint{4.842987in}{2.930688in}}%
\pgfusepath{clip}%
\pgfsetrectcap%
\pgfsetroundjoin%
\pgfsetlinewidth{0.803000pt}%
\definecolor{currentstroke}{rgb}{0.850000,0.850000,0.850000}%
\pgfsetstrokecolor{currentstroke}%
\pgfsetdash{}{0pt}%
\pgfpathmoveto{\pgfqpoint{2.805945in}{0.417642in}}%
\pgfpathlineto{\pgfqpoint{2.805945in}{3.348330in}}%
\pgfusepath{stroke}%
\end{pgfscope}%
\begin{pgfscope}%
\pgfsetbuttcap%
\pgfsetroundjoin%
\definecolor{currentfill}{rgb}{0.000000,0.000000,0.000000}%
\pgfsetfillcolor{currentfill}%
\pgfsetlinewidth{0.602250pt}%
\definecolor{currentstroke}{rgb}{0.000000,0.000000,0.000000}%
\pgfsetstrokecolor{currentstroke}%
\pgfsetdash{}{0pt}%
\pgfsys@defobject{currentmarker}{\pgfqpoint{0.000000in}{-0.027778in}}{\pgfqpoint{0.000000in}{0.000000in}}{%
\pgfpathmoveto{\pgfqpoint{0.000000in}{0.000000in}}%
\pgfpathlineto{\pgfqpoint{0.000000in}{-0.027778in}}%
\pgfusepath{stroke,fill}%
}%
\begin{pgfscope}%
\pgfsys@transformshift{2.805945in}{0.417642in}%
\pgfsys@useobject{currentmarker}{}%
\end{pgfscope}%
\end{pgfscope}%
\begin{pgfscope}%
\pgfpathrectangle{\pgfqpoint{0.605343in}{0.417642in}}{\pgfqpoint{4.842987in}{2.930688in}}%
\pgfusepath{clip}%
\pgfsetrectcap%
\pgfsetroundjoin%
\pgfsetlinewidth{0.803000pt}%
\definecolor{currentstroke}{rgb}{0.850000,0.850000,0.850000}%
\pgfsetstrokecolor{currentstroke}%
\pgfsetdash{}{0pt}%
\pgfpathmoveto{\pgfqpoint{2.864047in}{0.417642in}}%
\pgfpathlineto{\pgfqpoint{2.864047in}{3.348330in}}%
\pgfusepath{stroke}%
\end{pgfscope}%
\begin{pgfscope}%
\pgfsetbuttcap%
\pgfsetroundjoin%
\definecolor{currentfill}{rgb}{0.000000,0.000000,0.000000}%
\pgfsetfillcolor{currentfill}%
\pgfsetlinewidth{0.602250pt}%
\definecolor{currentstroke}{rgb}{0.000000,0.000000,0.000000}%
\pgfsetstrokecolor{currentstroke}%
\pgfsetdash{}{0pt}%
\pgfsys@defobject{currentmarker}{\pgfqpoint{0.000000in}{-0.027778in}}{\pgfqpoint{0.000000in}{0.000000in}}{%
\pgfpathmoveto{\pgfqpoint{0.000000in}{0.000000in}}%
\pgfpathlineto{\pgfqpoint{0.000000in}{-0.027778in}}%
\pgfusepath{stroke,fill}%
}%
\begin{pgfscope}%
\pgfsys@transformshift{2.864047in}{0.417642in}%
\pgfsys@useobject{currentmarker}{}%
\end{pgfscope}%
\end{pgfscope}%
\begin{pgfscope}%
\pgfpathrectangle{\pgfqpoint{0.605343in}{0.417642in}}{\pgfqpoint{4.842987in}{2.930688in}}%
\pgfusepath{clip}%
\pgfsetrectcap%
\pgfsetroundjoin%
\pgfsetlinewidth{0.803000pt}%
\definecolor{currentstroke}{rgb}{0.850000,0.850000,0.850000}%
\pgfsetstrokecolor{currentstroke}%
\pgfsetdash{}{0pt}%
\pgfpathmoveto{\pgfqpoint{2.913172in}{0.417642in}}%
\pgfpathlineto{\pgfqpoint{2.913172in}{3.348330in}}%
\pgfusepath{stroke}%
\end{pgfscope}%
\begin{pgfscope}%
\pgfsetbuttcap%
\pgfsetroundjoin%
\definecolor{currentfill}{rgb}{0.000000,0.000000,0.000000}%
\pgfsetfillcolor{currentfill}%
\pgfsetlinewidth{0.602250pt}%
\definecolor{currentstroke}{rgb}{0.000000,0.000000,0.000000}%
\pgfsetstrokecolor{currentstroke}%
\pgfsetdash{}{0pt}%
\pgfsys@defobject{currentmarker}{\pgfqpoint{0.000000in}{-0.027778in}}{\pgfqpoint{0.000000in}{0.000000in}}{%
\pgfpathmoveto{\pgfqpoint{0.000000in}{0.000000in}}%
\pgfpathlineto{\pgfqpoint{0.000000in}{-0.027778in}}%
\pgfusepath{stroke,fill}%
}%
\begin{pgfscope}%
\pgfsys@transformshift{2.913172in}{0.417642in}%
\pgfsys@useobject{currentmarker}{}%
\end{pgfscope}%
\end{pgfscope}%
\begin{pgfscope}%
\pgfpathrectangle{\pgfqpoint{0.605343in}{0.417642in}}{\pgfqpoint{4.842987in}{2.930688in}}%
\pgfusepath{clip}%
\pgfsetrectcap%
\pgfsetroundjoin%
\pgfsetlinewidth{0.803000pt}%
\definecolor{currentstroke}{rgb}{0.850000,0.850000,0.850000}%
\pgfsetstrokecolor{currentstroke}%
\pgfsetdash{}{0pt}%
\pgfpathmoveto{\pgfqpoint{2.955726in}{0.417642in}}%
\pgfpathlineto{\pgfqpoint{2.955726in}{3.348330in}}%
\pgfusepath{stroke}%
\end{pgfscope}%
\begin{pgfscope}%
\pgfsetbuttcap%
\pgfsetroundjoin%
\definecolor{currentfill}{rgb}{0.000000,0.000000,0.000000}%
\pgfsetfillcolor{currentfill}%
\pgfsetlinewidth{0.602250pt}%
\definecolor{currentstroke}{rgb}{0.000000,0.000000,0.000000}%
\pgfsetstrokecolor{currentstroke}%
\pgfsetdash{}{0pt}%
\pgfsys@defobject{currentmarker}{\pgfqpoint{0.000000in}{-0.027778in}}{\pgfqpoint{0.000000in}{0.000000in}}{%
\pgfpathmoveto{\pgfqpoint{0.000000in}{0.000000in}}%
\pgfpathlineto{\pgfqpoint{0.000000in}{-0.027778in}}%
\pgfusepath{stroke,fill}%
}%
\begin{pgfscope}%
\pgfsys@transformshift{2.955726in}{0.417642in}%
\pgfsys@useobject{currentmarker}{}%
\end{pgfscope}%
\end{pgfscope}%
\begin{pgfscope}%
\pgfpathrectangle{\pgfqpoint{0.605343in}{0.417642in}}{\pgfqpoint{4.842987in}{2.930688in}}%
\pgfusepath{clip}%
\pgfsetrectcap%
\pgfsetroundjoin%
\pgfsetlinewidth{0.803000pt}%
\definecolor{currentstroke}{rgb}{0.850000,0.850000,0.850000}%
\pgfsetstrokecolor{currentstroke}%
\pgfsetdash{}{0pt}%
\pgfpathmoveto{\pgfqpoint{2.993261in}{0.417642in}}%
\pgfpathlineto{\pgfqpoint{2.993261in}{3.348330in}}%
\pgfusepath{stroke}%
\end{pgfscope}%
\begin{pgfscope}%
\pgfsetbuttcap%
\pgfsetroundjoin%
\definecolor{currentfill}{rgb}{0.000000,0.000000,0.000000}%
\pgfsetfillcolor{currentfill}%
\pgfsetlinewidth{0.602250pt}%
\definecolor{currentstroke}{rgb}{0.000000,0.000000,0.000000}%
\pgfsetstrokecolor{currentstroke}%
\pgfsetdash{}{0pt}%
\pgfsys@defobject{currentmarker}{\pgfqpoint{0.000000in}{-0.027778in}}{\pgfqpoint{0.000000in}{0.000000in}}{%
\pgfpathmoveto{\pgfqpoint{0.000000in}{0.000000in}}%
\pgfpathlineto{\pgfqpoint{0.000000in}{-0.027778in}}%
\pgfusepath{stroke,fill}%
}%
\begin{pgfscope}%
\pgfsys@transformshift{2.993261in}{0.417642in}%
\pgfsys@useobject{currentmarker}{}%
\end{pgfscope}%
\end{pgfscope}%
\begin{pgfscope}%
\pgfpathrectangle{\pgfqpoint{0.605343in}{0.417642in}}{\pgfqpoint{4.842987in}{2.930688in}}%
\pgfusepath{clip}%
\pgfsetrectcap%
\pgfsetroundjoin%
\pgfsetlinewidth{0.803000pt}%
\definecolor{currentstroke}{rgb}{0.850000,0.850000,0.850000}%
\pgfsetstrokecolor{currentstroke}%
\pgfsetdash{}{0pt}%
\pgfpathmoveto{\pgfqpoint{3.247728in}{0.417642in}}%
\pgfpathlineto{\pgfqpoint{3.247728in}{3.348330in}}%
\pgfusepath{stroke}%
\end{pgfscope}%
\begin{pgfscope}%
\pgfsetbuttcap%
\pgfsetroundjoin%
\definecolor{currentfill}{rgb}{0.000000,0.000000,0.000000}%
\pgfsetfillcolor{currentfill}%
\pgfsetlinewidth{0.602250pt}%
\definecolor{currentstroke}{rgb}{0.000000,0.000000,0.000000}%
\pgfsetstrokecolor{currentstroke}%
\pgfsetdash{}{0pt}%
\pgfsys@defobject{currentmarker}{\pgfqpoint{0.000000in}{-0.027778in}}{\pgfqpoint{0.000000in}{0.000000in}}{%
\pgfpathmoveto{\pgfqpoint{0.000000in}{0.000000in}}%
\pgfpathlineto{\pgfqpoint{0.000000in}{-0.027778in}}%
\pgfusepath{stroke,fill}%
}%
\begin{pgfscope}%
\pgfsys@transformshift{3.247728in}{0.417642in}%
\pgfsys@useobject{currentmarker}{}%
\end{pgfscope}%
\end{pgfscope}%
\begin{pgfscope}%
\pgfpathrectangle{\pgfqpoint{0.605343in}{0.417642in}}{\pgfqpoint{4.842987in}{2.930688in}}%
\pgfusepath{clip}%
\pgfsetrectcap%
\pgfsetroundjoin%
\pgfsetlinewidth{0.803000pt}%
\definecolor{currentstroke}{rgb}{0.850000,0.850000,0.850000}%
\pgfsetstrokecolor{currentstroke}%
\pgfsetdash{}{0pt}%
\pgfpathmoveto{\pgfqpoint{3.376942in}{0.417642in}}%
\pgfpathlineto{\pgfqpoint{3.376942in}{3.348330in}}%
\pgfusepath{stroke}%
\end{pgfscope}%
\begin{pgfscope}%
\pgfsetbuttcap%
\pgfsetroundjoin%
\definecolor{currentfill}{rgb}{0.000000,0.000000,0.000000}%
\pgfsetfillcolor{currentfill}%
\pgfsetlinewidth{0.602250pt}%
\definecolor{currentstroke}{rgb}{0.000000,0.000000,0.000000}%
\pgfsetstrokecolor{currentstroke}%
\pgfsetdash{}{0pt}%
\pgfsys@defobject{currentmarker}{\pgfqpoint{0.000000in}{-0.027778in}}{\pgfqpoint{0.000000in}{0.000000in}}{%
\pgfpathmoveto{\pgfqpoint{0.000000in}{0.000000in}}%
\pgfpathlineto{\pgfqpoint{0.000000in}{-0.027778in}}%
\pgfusepath{stroke,fill}%
}%
\begin{pgfscope}%
\pgfsys@transformshift{3.376942in}{0.417642in}%
\pgfsys@useobject{currentmarker}{}%
\end{pgfscope}%
\end{pgfscope}%
\begin{pgfscope}%
\pgfpathrectangle{\pgfqpoint{0.605343in}{0.417642in}}{\pgfqpoint{4.842987in}{2.930688in}}%
\pgfusepath{clip}%
\pgfsetrectcap%
\pgfsetroundjoin%
\pgfsetlinewidth{0.803000pt}%
\definecolor{currentstroke}{rgb}{0.850000,0.850000,0.850000}%
\pgfsetstrokecolor{currentstroke}%
\pgfsetdash{}{0pt}%
\pgfpathmoveto{\pgfqpoint{3.468620in}{0.417642in}}%
\pgfpathlineto{\pgfqpoint{3.468620in}{3.348330in}}%
\pgfusepath{stroke}%
\end{pgfscope}%
\begin{pgfscope}%
\pgfsetbuttcap%
\pgfsetroundjoin%
\definecolor{currentfill}{rgb}{0.000000,0.000000,0.000000}%
\pgfsetfillcolor{currentfill}%
\pgfsetlinewidth{0.602250pt}%
\definecolor{currentstroke}{rgb}{0.000000,0.000000,0.000000}%
\pgfsetstrokecolor{currentstroke}%
\pgfsetdash{}{0pt}%
\pgfsys@defobject{currentmarker}{\pgfqpoint{0.000000in}{-0.027778in}}{\pgfqpoint{0.000000in}{0.000000in}}{%
\pgfpathmoveto{\pgfqpoint{0.000000in}{0.000000in}}%
\pgfpathlineto{\pgfqpoint{0.000000in}{-0.027778in}}%
\pgfusepath{stroke,fill}%
}%
\begin{pgfscope}%
\pgfsys@transformshift{3.468620in}{0.417642in}%
\pgfsys@useobject{currentmarker}{}%
\end{pgfscope}%
\end{pgfscope}%
\begin{pgfscope}%
\pgfpathrectangle{\pgfqpoint{0.605343in}{0.417642in}}{\pgfqpoint{4.842987in}{2.930688in}}%
\pgfusepath{clip}%
\pgfsetrectcap%
\pgfsetroundjoin%
\pgfsetlinewidth{0.803000pt}%
\definecolor{currentstroke}{rgb}{0.850000,0.850000,0.850000}%
\pgfsetstrokecolor{currentstroke}%
\pgfsetdash{}{0pt}%
\pgfpathmoveto{\pgfqpoint{3.539731in}{0.417642in}}%
\pgfpathlineto{\pgfqpoint{3.539731in}{3.348330in}}%
\pgfusepath{stroke}%
\end{pgfscope}%
\begin{pgfscope}%
\pgfsetbuttcap%
\pgfsetroundjoin%
\definecolor{currentfill}{rgb}{0.000000,0.000000,0.000000}%
\pgfsetfillcolor{currentfill}%
\pgfsetlinewidth{0.602250pt}%
\definecolor{currentstroke}{rgb}{0.000000,0.000000,0.000000}%
\pgfsetstrokecolor{currentstroke}%
\pgfsetdash{}{0pt}%
\pgfsys@defobject{currentmarker}{\pgfqpoint{0.000000in}{-0.027778in}}{\pgfqpoint{0.000000in}{0.000000in}}{%
\pgfpathmoveto{\pgfqpoint{0.000000in}{0.000000in}}%
\pgfpathlineto{\pgfqpoint{0.000000in}{-0.027778in}}%
\pgfusepath{stroke,fill}%
}%
\begin{pgfscope}%
\pgfsys@transformshift{3.539731in}{0.417642in}%
\pgfsys@useobject{currentmarker}{}%
\end{pgfscope}%
\end{pgfscope}%
\begin{pgfscope}%
\pgfpathrectangle{\pgfqpoint{0.605343in}{0.417642in}}{\pgfqpoint{4.842987in}{2.930688in}}%
\pgfusepath{clip}%
\pgfsetrectcap%
\pgfsetroundjoin%
\pgfsetlinewidth{0.803000pt}%
\definecolor{currentstroke}{rgb}{0.850000,0.850000,0.850000}%
\pgfsetstrokecolor{currentstroke}%
\pgfsetdash{}{0pt}%
\pgfpathmoveto{\pgfqpoint{3.597833in}{0.417642in}}%
\pgfpathlineto{\pgfqpoint{3.597833in}{3.348330in}}%
\pgfusepath{stroke}%
\end{pgfscope}%
\begin{pgfscope}%
\pgfsetbuttcap%
\pgfsetroundjoin%
\definecolor{currentfill}{rgb}{0.000000,0.000000,0.000000}%
\pgfsetfillcolor{currentfill}%
\pgfsetlinewidth{0.602250pt}%
\definecolor{currentstroke}{rgb}{0.000000,0.000000,0.000000}%
\pgfsetstrokecolor{currentstroke}%
\pgfsetdash{}{0pt}%
\pgfsys@defobject{currentmarker}{\pgfqpoint{0.000000in}{-0.027778in}}{\pgfqpoint{0.000000in}{0.000000in}}{%
\pgfpathmoveto{\pgfqpoint{0.000000in}{0.000000in}}%
\pgfpathlineto{\pgfqpoint{0.000000in}{-0.027778in}}%
\pgfusepath{stroke,fill}%
}%
\begin{pgfscope}%
\pgfsys@transformshift{3.597833in}{0.417642in}%
\pgfsys@useobject{currentmarker}{}%
\end{pgfscope}%
\end{pgfscope}%
\begin{pgfscope}%
\pgfpathrectangle{\pgfqpoint{0.605343in}{0.417642in}}{\pgfqpoint{4.842987in}{2.930688in}}%
\pgfusepath{clip}%
\pgfsetrectcap%
\pgfsetroundjoin%
\pgfsetlinewidth{0.803000pt}%
\definecolor{currentstroke}{rgb}{0.850000,0.850000,0.850000}%
\pgfsetstrokecolor{currentstroke}%
\pgfsetdash{}{0pt}%
\pgfpathmoveto{\pgfqpoint{3.646958in}{0.417642in}}%
\pgfpathlineto{\pgfqpoint{3.646958in}{3.348330in}}%
\pgfusepath{stroke}%
\end{pgfscope}%
\begin{pgfscope}%
\pgfsetbuttcap%
\pgfsetroundjoin%
\definecolor{currentfill}{rgb}{0.000000,0.000000,0.000000}%
\pgfsetfillcolor{currentfill}%
\pgfsetlinewidth{0.602250pt}%
\definecolor{currentstroke}{rgb}{0.000000,0.000000,0.000000}%
\pgfsetstrokecolor{currentstroke}%
\pgfsetdash{}{0pt}%
\pgfsys@defobject{currentmarker}{\pgfqpoint{0.000000in}{-0.027778in}}{\pgfqpoint{0.000000in}{0.000000in}}{%
\pgfpathmoveto{\pgfqpoint{0.000000in}{0.000000in}}%
\pgfpathlineto{\pgfqpoint{0.000000in}{-0.027778in}}%
\pgfusepath{stroke,fill}%
}%
\begin{pgfscope}%
\pgfsys@transformshift{3.646958in}{0.417642in}%
\pgfsys@useobject{currentmarker}{}%
\end{pgfscope}%
\end{pgfscope}%
\begin{pgfscope}%
\pgfpathrectangle{\pgfqpoint{0.605343in}{0.417642in}}{\pgfqpoint{4.842987in}{2.930688in}}%
\pgfusepath{clip}%
\pgfsetrectcap%
\pgfsetroundjoin%
\pgfsetlinewidth{0.803000pt}%
\definecolor{currentstroke}{rgb}{0.850000,0.850000,0.850000}%
\pgfsetstrokecolor{currentstroke}%
\pgfsetdash{}{0pt}%
\pgfpathmoveto{\pgfqpoint{3.689511in}{0.417642in}}%
\pgfpathlineto{\pgfqpoint{3.689511in}{3.348330in}}%
\pgfusepath{stroke}%
\end{pgfscope}%
\begin{pgfscope}%
\pgfsetbuttcap%
\pgfsetroundjoin%
\definecolor{currentfill}{rgb}{0.000000,0.000000,0.000000}%
\pgfsetfillcolor{currentfill}%
\pgfsetlinewidth{0.602250pt}%
\definecolor{currentstroke}{rgb}{0.000000,0.000000,0.000000}%
\pgfsetstrokecolor{currentstroke}%
\pgfsetdash{}{0pt}%
\pgfsys@defobject{currentmarker}{\pgfqpoint{0.000000in}{-0.027778in}}{\pgfqpoint{0.000000in}{0.000000in}}{%
\pgfpathmoveto{\pgfqpoint{0.000000in}{0.000000in}}%
\pgfpathlineto{\pgfqpoint{0.000000in}{-0.027778in}}%
\pgfusepath{stroke,fill}%
}%
\begin{pgfscope}%
\pgfsys@transformshift{3.689511in}{0.417642in}%
\pgfsys@useobject{currentmarker}{}%
\end{pgfscope}%
\end{pgfscope}%
\begin{pgfscope}%
\pgfpathrectangle{\pgfqpoint{0.605343in}{0.417642in}}{\pgfqpoint{4.842987in}{2.930688in}}%
\pgfusepath{clip}%
\pgfsetrectcap%
\pgfsetroundjoin%
\pgfsetlinewidth{0.803000pt}%
\definecolor{currentstroke}{rgb}{0.850000,0.850000,0.850000}%
\pgfsetstrokecolor{currentstroke}%
\pgfsetdash{}{0pt}%
\pgfpathmoveto{\pgfqpoint{3.727046in}{0.417642in}}%
\pgfpathlineto{\pgfqpoint{3.727046in}{3.348330in}}%
\pgfusepath{stroke}%
\end{pgfscope}%
\begin{pgfscope}%
\pgfsetbuttcap%
\pgfsetroundjoin%
\definecolor{currentfill}{rgb}{0.000000,0.000000,0.000000}%
\pgfsetfillcolor{currentfill}%
\pgfsetlinewidth{0.602250pt}%
\definecolor{currentstroke}{rgb}{0.000000,0.000000,0.000000}%
\pgfsetstrokecolor{currentstroke}%
\pgfsetdash{}{0pt}%
\pgfsys@defobject{currentmarker}{\pgfqpoint{0.000000in}{-0.027778in}}{\pgfqpoint{0.000000in}{0.000000in}}{%
\pgfpathmoveto{\pgfqpoint{0.000000in}{0.000000in}}%
\pgfpathlineto{\pgfqpoint{0.000000in}{-0.027778in}}%
\pgfusepath{stroke,fill}%
}%
\begin{pgfscope}%
\pgfsys@transformshift{3.727046in}{0.417642in}%
\pgfsys@useobject{currentmarker}{}%
\end{pgfscope}%
\end{pgfscope}%
\begin{pgfscope}%
\pgfpathrectangle{\pgfqpoint{0.605343in}{0.417642in}}{\pgfqpoint{4.842987in}{2.930688in}}%
\pgfusepath{clip}%
\pgfsetrectcap%
\pgfsetroundjoin%
\pgfsetlinewidth{0.803000pt}%
\definecolor{currentstroke}{rgb}{0.850000,0.850000,0.850000}%
\pgfsetstrokecolor{currentstroke}%
\pgfsetdash{}{0pt}%
\pgfpathmoveto{\pgfqpoint{3.981514in}{0.417642in}}%
\pgfpathlineto{\pgfqpoint{3.981514in}{3.348330in}}%
\pgfusepath{stroke}%
\end{pgfscope}%
\begin{pgfscope}%
\pgfsetbuttcap%
\pgfsetroundjoin%
\definecolor{currentfill}{rgb}{0.000000,0.000000,0.000000}%
\pgfsetfillcolor{currentfill}%
\pgfsetlinewidth{0.602250pt}%
\definecolor{currentstroke}{rgb}{0.000000,0.000000,0.000000}%
\pgfsetstrokecolor{currentstroke}%
\pgfsetdash{}{0pt}%
\pgfsys@defobject{currentmarker}{\pgfqpoint{0.000000in}{-0.027778in}}{\pgfqpoint{0.000000in}{0.000000in}}{%
\pgfpathmoveto{\pgfqpoint{0.000000in}{0.000000in}}%
\pgfpathlineto{\pgfqpoint{0.000000in}{-0.027778in}}%
\pgfusepath{stroke,fill}%
}%
\begin{pgfscope}%
\pgfsys@transformshift{3.981514in}{0.417642in}%
\pgfsys@useobject{currentmarker}{}%
\end{pgfscope}%
\end{pgfscope}%
\begin{pgfscope}%
\pgfpathrectangle{\pgfqpoint{0.605343in}{0.417642in}}{\pgfqpoint{4.842987in}{2.930688in}}%
\pgfusepath{clip}%
\pgfsetrectcap%
\pgfsetroundjoin%
\pgfsetlinewidth{0.803000pt}%
\definecolor{currentstroke}{rgb}{0.850000,0.850000,0.850000}%
\pgfsetstrokecolor{currentstroke}%
\pgfsetdash{}{0pt}%
\pgfpathmoveto{\pgfqpoint{4.110727in}{0.417642in}}%
\pgfpathlineto{\pgfqpoint{4.110727in}{3.348330in}}%
\pgfusepath{stroke}%
\end{pgfscope}%
\begin{pgfscope}%
\pgfsetbuttcap%
\pgfsetroundjoin%
\definecolor{currentfill}{rgb}{0.000000,0.000000,0.000000}%
\pgfsetfillcolor{currentfill}%
\pgfsetlinewidth{0.602250pt}%
\definecolor{currentstroke}{rgb}{0.000000,0.000000,0.000000}%
\pgfsetstrokecolor{currentstroke}%
\pgfsetdash{}{0pt}%
\pgfsys@defobject{currentmarker}{\pgfqpoint{0.000000in}{-0.027778in}}{\pgfqpoint{0.000000in}{0.000000in}}{%
\pgfpathmoveto{\pgfqpoint{0.000000in}{0.000000in}}%
\pgfpathlineto{\pgfqpoint{0.000000in}{-0.027778in}}%
\pgfusepath{stroke,fill}%
}%
\begin{pgfscope}%
\pgfsys@transformshift{4.110727in}{0.417642in}%
\pgfsys@useobject{currentmarker}{}%
\end{pgfscope}%
\end{pgfscope}%
\begin{pgfscope}%
\pgfpathrectangle{\pgfqpoint{0.605343in}{0.417642in}}{\pgfqpoint{4.842987in}{2.930688in}}%
\pgfusepath{clip}%
\pgfsetrectcap%
\pgfsetroundjoin%
\pgfsetlinewidth{0.803000pt}%
\definecolor{currentstroke}{rgb}{0.850000,0.850000,0.850000}%
\pgfsetstrokecolor{currentstroke}%
\pgfsetdash{}{0pt}%
\pgfpathmoveto{\pgfqpoint{4.202406in}{0.417642in}}%
\pgfpathlineto{\pgfqpoint{4.202406in}{3.348330in}}%
\pgfusepath{stroke}%
\end{pgfscope}%
\begin{pgfscope}%
\pgfsetbuttcap%
\pgfsetroundjoin%
\definecolor{currentfill}{rgb}{0.000000,0.000000,0.000000}%
\pgfsetfillcolor{currentfill}%
\pgfsetlinewidth{0.602250pt}%
\definecolor{currentstroke}{rgb}{0.000000,0.000000,0.000000}%
\pgfsetstrokecolor{currentstroke}%
\pgfsetdash{}{0pt}%
\pgfsys@defobject{currentmarker}{\pgfqpoint{0.000000in}{-0.027778in}}{\pgfqpoint{0.000000in}{0.000000in}}{%
\pgfpathmoveto{\pgfqpoint{0.000000in}{0.000000in}}%
\pgfpathlineto{\pgfqpoint{0.000000in}{-0.027778in}}%
\pgfusepath{stroke,fill}%
}%
\begin{pgfscope}%
\pgfsys@transformshift{4.202406in}{0.417642in}%
\pgfsys@useobject{currentmarker}{}%
\end{pgfscope}%
\end{pgfscope}%
\begin{pgfscope}%
\pgfpathrectangle{\pgfqpoint{0.605343in}{0.417642in}}{\pgfqpoint{4.842987in}{2.930688in}}%
\pgfusepath{clip}%
\pgfsetrectcap%
\pgfsetroundjoin%
\pgfsetlinewidth{0.803000pt}%
\definecolor{currentstroke}{rgb}{0.850000,0.850000,0.850000}%
\pgfsetstrokecolor{currentstroke}%
\pgfsetdash{}{0pt}%
\pgfpathmoveto{\pgfqpoint{4.273517in}{0.417642in}}%
\pgfpathlineto{\pgfqpoint{4.273517in}{3.348330in}}%
\pgfusepath{stroke}%
\end{pgfscope}%
\begin{pgfscope}%
\pgfsetbuttcap%
\pgfsetroundjoin%
\definecolor{currentfill}{rgb}{0.000000,0.000000,0.000000}%
\pgfsetfillcolor{currentfill}%
\pgfsetlinewidth{0.602250pt}%
\definecolor{currentstroke}{rgb}{0.000000,0.000000,0.000000}%
\pgfsetstrokecolor{currentstroke}%
\pgfsetdash{}{0pt}%
\pgfsys@defobject{currentmarker}{\pgfqpoint{0.000000in}{-0.027778in}}{\pgfqpoint{0.000000in}{0.000000in}}{%
\pgfpathmoveto{\pgfqpoint{0.000000in}{0.000000in}}%
\pgfpathlineto{\pgfqpoint{0.000000in}{-0.027778in}}%
\pgfusepath{stroke,fill}%
}%
\begin{pgfscope}%
\pgfsys@transformshift{4.273517in}{0.417642in}%
\pgfsys@useobject{currentmarker}{}%
\end{pgfscope}%
\end{pgfscope}%
\begin{pgfscope}%
\pgfpathrectangle{\pgfqpoint{0.605343in}{0.417642in}}{\pgfqpoint{4.842987in}{2.930688in}}%
\pgfusepath{clip}%
\pgfsetrectcap%
\pgfsetroundjoin%
\pgfsetlinewidth{0.803000pt}%
\definecolor{currentstroke}{rgb}{0.850000,0.850000,0.850000}%
\pgfsetstrokecolor{currentstroke}%
\pgfsetdash{}{0pt}%
\pgfpathmoveto{\pgfqpoint{4.331619in}{0.417642in}}%
\pgfpathlineto{\pgfqpoint{4.331619in}{3.348330in}}%
\pgfusepath{stroke}%
\end{pgfscope}%
\begin{pgfscope}%
\pgfsetbuttcap%
\pgfsetroundjoin%
\definecolor{currentfill}{rgb}{0.000000,0.000000,0.000000}%
\pgfsetfillcolor{currentfill}%
\pgfsetlinewidth{0.602250pt}%
\definecolor{currentstroke}{rgb}{0.000000,0.000000,0.000000}%
\pgfsetstrokecolor{currentstroke}%
\pgfsetdash{}{0pt}%
\pgfsys@defobject{currentmarker}{\pgfqpoint{0.000000in}{-0.027778in}}{\pgfqpoint{0.000000in}{0.000000in}}{%
\pgfpathmoveto{\pgfqpoint{0.000000in}{0.000000in}}%
\pgfpathlineto{\pgfqpoint{0.000000in}{-0.027778in}}%
\pgfusepath{stroke,fill}%
}%
\begin{pgfscope}%
\pgfsys@transformshift{4.331619in}{0.417642in}%
\pgfsys@useobject{currentmarker}{}%
\end{pgfscope}%
\end{pgfscope}%
\begin{pgfscope}%
\pgfpathrectangle{\pgfqpoint{0.605343in}{0.417642in}}{\pgfqpoint{4.842987in}{2.930688in}}%
\pgfusepath{clip}%
\pgfsetrectcap%
\pgfsetroundjoin%
\pgfsetlinewidth{0.803000pt}%
\definecolor{currentstroke}{rgb}{0.850000,0.850000,0.850000}%
\pgfsetstrokecolor{currentstroke}%
\pgfsetdash{}{0pt}%
\pgfpathmoveto{\pgfqpoint{4.380744in}{0.417642in}}%
\pgfpathlineto{\pgfqpoint{4.380744in}{3.348330in}}%
\pgfusepath{stroke}%
\end{pgfscope}%
\begin{pgfscope}%
\pgfsetbuttcap%
\pgfsetroundjoin%
\definecolor{currentfill}{rgb}{0.000000,0.000000,0.000000}%
\pgfsetfillcolor{currentfill}%
\pgfsetlinewidth{0.602250pt}%
\definecolor{currentstroke}{rgb}{0.000000,0.000000,0.000000}%
\pgfsetstrokecolor{currentstroke}%
\pgfsetdash{}{0pt}%
\pgfsys@defobject{currentmarker}{\pgfqpoint{0.000000in}{-0.027778in}}{\pgfqpoint{0.000000in}{0.000000in}}{%
\pgfpathmoveto{\pgfqpoint{0.000000in}{0.000000in}}%
\pgfpathlineto{\pgfqpoint{0.000000in}{-0.027778in}}%
\pgfusepath{stroke,fill}%
}%
\begin{pgfscope}%
\pgfsys@transformshift{4.380744in}{0.417642in}%
\pgfsys@useobject{currentmarker}{}%
\end{pgfscope}%
\end{pgfscope}%
\begin{pgfscope}%
\pgfpathrectangle{\pgfqpoint{0.605343in}{0.417642in}}{\pgfqpoint{4.842987in}{2.930688in}}%
\pgfusepath{clip}%
\pgfsetrectcap%
\pgfsetroundjoin%
\pgfsetlinewidth{0.803000pt}%
\definecolor{currentstroke}{rgb}{0.850000,0.850000,0.850000}%
\pgfsetstrokecolor{currentstroke}%
\pgfsetdash{}{0pt}%
\pgfpathmoveto{\pgfqpoint{4.423297in}{0.417642in}}%
\pgfpathlineto{\pgfqpoint{4.423297in}{3.348330in}}%
\pgfusepath{stroke}%
\end{pgfscope}%
\begin{pgfscope}%
\pgfsetbuttcap%
\pgfsetroundjoin%
\definecolor{currentfill}{rgb}{0.000000,0.000000,0.000000}%
\pgfsetfillcolor{currentfill}%
\pgfsetlinewidth{0.602250pt}%
\definecolor{currentstroke}{rgb}{0.000000,0.000000,0.000000}%
\pgfsetstrokecolor{currentstroke}%
\pgfsetdash{}{0pt}%
\pgfsys@defobject{currentmarker}{\pgfqpoint{0.000000in}{-0.027778in}}{\pgfqpoint{0.000000in}{0.000000in}}{%
\pgfpathmoveto{\pgfqpoint{0.000000in}{0.000000in}}%
\pgfpathlineto{\pgfqpoint{0.000000in}{-0.027778in}}%
\pgfusepath{stroke,fill}%
}%
\begin{pgfscope}%
\pgfsys@transformshift{4.423297in}{0.417642in}%
\pgfsys@useobject{currentmarker}{}%
\end{pgfscope}%
\end{pgfscope}%
\begin{pgfscope}%
\pgfpathrectangle{\pgfqpoint{0.605343in}{0.417642in}}{\pgfqpoint{4.842987in}{2.930688in}}%
\pgfusepath{clip}%
\pgfsetrectcap%
\pgfsetroundjoin%
\pgfsetlinewidth{0.803000pt}%
\definecolor{currentstroke}{rgb}{0.850000,0.850000,0.850000}%
\pgfsetstrokecolor{currentstroke}%
\pgfsetdash{}{0pt}%
\pgfpathmoveto{\pgfqpoint{4.460832in}{0.417642in}}%
\pgfpathlineto{\pgfqpoint{4.460832in}{3.348330in}}%
\pgfusepath{stroke}%
\end{pgfscope}%
\begin{pgfscope}%
\pgfsetbuttcap%
\pgfsetroundjoin%
\definecolor{currentfill}{rgb}{0.000000,0.000000,0.000000}%
\pgfsetfillcolor{currentfill}%
\pgfsetlinewidth{0.602250pt}%
\definecolor{currentstroke}{rgb}{0.000000,0.000000,0.000000}%
\pgfsetstrokecolor{currentstroke}%
\pgfsetdash{}{0pt}%
\pgfsys@defobject{currentmarker}{\pgfqpoint{0.000000in}{-0.027778in}}{\pgfqpoint{0.000000in}{0.000000in}}{%
\pgfpathmoveto{\pgfqpoint{0.000000in}{0.000000in}}%
\pgfpathlineto{\pgfqpoint{0.000000in}{-0.027778in}}%
\pgfusepath{stroke,fill}%
}%
\begin{pgfscope}%
\pgfsys@transformshift{4.460832in}{0.417642in}%
\pgfsys@useobject{currentmarker}{}%
\end{pgfscope}%
\end{pgfscope}%
\begin{pgfscope}%
\pgfpathrectangle{\pgfqpoint{0.605343in}{0.417642in}}{\pgfqpoint{4.842987in}{2.930688in}}%
\pgfusepath{clip}%
\pgfsetrectcap%
\pgfsetroundjoin%
\pgfsetlinewidth{0.803000pt}%
\definecolor{currentstroke}{rgb}{0.850000,0.850000,0.850000}%
\pgfsetstrokecolor{currentstroke}%
\pgfsetdash{}{0pt}%
\pgfpathmoveto{\pgfqpoint{4.715300in}{0.417642in}}%
\pgfpathlineto{\pgfqpoint{4.715300in}{3.348330in}}%
\pgfusepath{stroke}%
\end{pgfscope}%
\begin{pgfscope}%
\pgfsetbuttcap%
\pgfsetroundjoin%
\definecolor{currentfill}{rgb}{0.000000,0.000000,0.000000}%
\pgfsetfillcolor{currentfill}%
\pgfsetlinewidth{0.602250pt}%
\definecolor{currentstroke}{rgb}{0.000000,0.000000,0.000000}%
\pgfsetstrokecolor{currentstroke}%
\pgfsetdash{}{0pt}%
\pgfsys@defobject{currentmarker}{\pgfqpoint{0.000000in}{-0.027778in}}{\pgfqpoint{0.000000in}{0.000000in}}{%
\pgfpathmoveto{\pgfqpoint{0.000000in}{0.000000in}}%
\pgfpathlineto{\pgfqpoint{0.000000in}{-0.027778in}}%
\pgfusepath{stroke,fill}%
}%
\begin{pgfscope}%
\pgfsys@transformshift{4.715300in}{0.417642in}%
\pgfsys@useobject{currentmarker}{}%
\end{pgfscope}%
\end{pgfscope}%
\begin{pgfscope}%
\pgfpathrectangle{\pgfqpoint{0.605343in}{0.417642in}}{\pgfqpoint{4.842987in}{2.930688in}}%
\pgfusepath{clip}%
\pgfsetrectcap%
\pgfsetroundjoin%
\pgfsetlinewidth{0.803000pt}%
\definecolor{currentstroke}{rgb}{0.850000,0.850000,0.850000}%
\pgfsetstrokecolor{currentstroke}%
\pgfsetdash{}{0pt}%
\pgfpathmoveto{\pgfqpoint{4.844513in}{0.417642in}}%
\pgfpathlineto{\pgfqpoint{4.844513in}{3.348330in}}%
\pgfusepath{stroke}%
\end{pgfscope}%
\begin{pgfscope}%
\pgfsetbuttcap%
\pgfsetroundjoin%
\definecolor{currentfill}{rgb}{0.000000,0.000000,0.000000}%
\pgfsetfillcolor{currentfill}%
\pgfsetlinewidth{0.602250pt}%
\definecolor{currentstroke}{rgb}{0.000000,0.000000,0.000000}%
\pgfsetstrokecolor{currentstroke}%
\pgfsetdash{}{0pt}%
\pgfsys@defobject{currentmarker}{\pgfqpoint{0.000000in}{-0.027778in}}{\pgfqpoint{0.000000in}{0.000000in}}{%
\pgfpathmoveto{\pgfqpoint{0.000000in}{0.000000in}}%
\pgfpathlineto{\pgfqpoint{0.000000in}{-0.027778in}}%
\pgfusepath{stroke,fill}%
}%
\begin{pgfscope}%
\pgfsys@transformshift{4.844513in}{0.417642in}%
\pgfsys@useobject{currentmarker}{}%
\end{pgfscope}%
\end{pgfscope}%
\begin{pgfscope}%
\pgfpathrectangle{\pgfqpoint{0.605343in}{0.417642in}}{\pgfqpoint{4.842987in}{2.930688in}}%
\pgfusepath{clip}%
\pgfsetrectcap%
\pgfsetroundjoin%
\pgfsetlinewidth{0.803000pt}%
\definecolor{currentstroke}{rgb}{0.850000,0.850000,0.850000}%
\pgfsetstrokecolor{currentstroke}%
\pgfsetdash{}{0pt}%
\pgfpathmoveto{\pgfqpoint{4.936192in}{0.417642in}}%
\pgfpathlineto{\pgfqpoint{4.936192in}{3.348330in}}%
\pgfusepath{stroke}%
\end{pgfscope}%
\begin{pgfscope}%
\pgfsetbuttcap%
\pgfsetroundjoin%
\definecolor{currentfill}{rgb}{0.000000,0.000000,0.000000}%
\pgfsetfillcolor{currentfill}%
\pgfsetlinewidth{0.602250pt}%
\definecolor{currentstroke}{rgb}{0.000000,0.000000,0.000000}%
\pgfsetstrokecolor{currentstroke}%
\pgfsetdash{}{0pt}%
\pgfsys@defobject{currentmarker}{\pgfqpoint{0.000000in}{-0.027778in}}{\pgfqpoint{0.000000in}{0.000000in}}{%
\pgfpathmoveto{\pgfqpoint{0.000000in}{0.000000in}}%
\pgfpathlineto{\pgfqpoint{0.000000in}{-0.027778in}}%
\pgfusepath{stroke,fill}%
}%
\begin{pgfscope}%
\pgfsys@transformshift{4.936192in}{0.417642in}%
\pgfsys@useobject{currentmarker}{}%
\end{pgfscope}%
\end{pgfscope}%
\begin{pgfscope}%
\pgfpathrectangle{\pgfqpoint{0.605343in}{0.417642in}}{\pgfqpoint{4.842987in}{2.930688in}}%
\pgfusepath{clip}%
\pgfsetrectcap%
\pgfsetroundjoin%
\pgfsetlinewidth{0.803000pt}%
\definecolor{currentstroke}{rgb}{0.850000,0.850000,0.850000}%
\pgfsetstrokecolor{currentstroke}%
\pgfsetdash{}{0pt}%
\pgfpathmoveto{\pgfqpoint{5.007303in}{0.417642in}}%
\pgfpathlineto{\pgfqpoint{5.007303in}{3.348330in}}%
\pgfusepath{stroke}%
\end{pgfscope}%
\begin{pgfscope}%
\pgfsetbuttcap%
\pgfsetroundjoin%
\definecolor{currentfill}{rgb}{0.000000,0.000000,0.000000}%
\pgfsetfillcolor{currentfill}%
\pgfsetlinewidth{0.602250pt}%
\definecolor{currentstroke}{rgb}{0.000000,0.000000,0.000000}%
\pgfsetstrokecolor{currentstroke}%
\pgfsetdash{}{0pt}%
\pgfsys@defobject{currentmarker}{\pgfqpoint{0.000000in}{-0.027778in}}{\pgfqpoint{0.000000in}{0.000000in}}{%
\pgfpathmoveto{\pgfqpoint{0.000000in}{0.000000in}}%
\pgfpathlineto{\pgfqpoint{0.000000in}{-0.027778in}}%
\pgfusepath{stroke,fill}%
}%
\begin{pgfscope}%
\pgfsys@transformshift{5.007303in}{0.417642in}%
\pgfsys@useobject{currentmarker}{}%
\end{pgfscope}%
\end{pgfscope}%
\begin{pgfscope}%
\pgfpathrectangle{\pgfqpoint{0.605343in}{0.417642in}}{\pgfqpoint{4.842987in}{2.930688in}}%
\pgfusepath{clip}%
\pgfsetrectcap%
\pgfsetroundjoin%
\pgfsetlinewidth{0.803000pt}%
\definecolor{currentstroke}{rgb}{0.850000,0.850000,0.850000}%
\pgfsetstrokecolor{currentstroke}%
\pgfsetdash{}{0pt}%
\pgfpathmoveto{\pgfqpoint{5.065405in}{0.417642in}}%
\pgfpathlineto{\pgfqpoint{5.065405in}{3.348330in}}%
\pgfusepath{stroke}%
\end{pgfscope}%
\begin{pgfscope}%
\pgfsetbuttcap%
\pgfsetroundjoin%
\definecolor{currentfill}{rgb}{0.000000,0.000000,0.000000}%
\pgfsetfillcolor{currentfill}%
\pgfsetlinewidth{0.602250pt}%
\definecolor{currentstroke}{rgb}{0.000000,0.000000,0.000000}%
\pgfsetstrokecolor{currentstroke}%
\pgfsetdash{}{0pt}%
\pgfsys@defobject{currentmarker}{\pgfqpoint{0.000000in}{-0.027778in}}{\pgfqpoint{0.000000in}{0.000000in}}{%
\pgfpathmoveto{\pgfqpoint{0.000000in}{0.000000in}}%
\pgfpathlineto{\pgfqpoint{0.000000in}{-0.027778in}}%
\pgfusepath{stroke,fill}%
}%
\begin{pgfscope}%
\pgfsys@transformshift{5.065405in}{0.417642in}%
\pgfsys@useobject{currentmarker}{}%
\end{pgfscope}%
\end{pgfscope}%
\begin{pgfscope}%
\pgfpathrectangle{\pgfqpoint{0.605343in}{0.417642in}}{\pgfqpoint{4.842987in}{2.930688in}}%
\pgfusepath{clip}%
\pgfsetrectcap%
\pgfsetroundjoin%
\pgfsetlinewidth{0.803000pt}%
\definecolor{currentstroke}{rgb}{0.850000,0.850000,0.850000}%
\pgfsetstrokecolor{currentstroke}%
\pgfsetdash{}{0pt}%
\pgfpathmoveto{\pgfqpoint{5.114529in}{0.417642in}}%
\pgfpathlineto{\pgfqpoint{5.114529in}{3.348330in}}%
\pgfusepath{stroke}%
\end{pgfscope}%
\begin{pgfscope}%
\pgfsetbuttcap%
\pgfsetroundjoin%
\definecolor{currentfill}{rgb}{0.000000,0.000000,0.000000}%
\pgfsetfillcolor{currentfill}%
\pgfsetlinewidth{0.602250pt}%
\definecolor{currentstroke}{rgb}{0.000000,0.000000,0.000000}%
\pgfsetstrokecolor{currentstroke}%
\pgfsetdash{}{0pt}%
\pgfsys@defobject{currentmarker}{\pgfqpoint{0.000000in}{-0.027778in}}{\pgfqpoint{0.000000in}{0.000000in}}{%
\pgfpathmoveto{\pgfqpoint{0.000000in}{0.000000in}}%
\pgfpathlineto{\pgfqpoint{0.000000in}{-0.027778in}}%
\pgfusepath{stroke,fill}%
}%
\begin{pgfscope}%
\pgfsys@transformshift{5.114529in}{0.417642in}%
\pgfsys@useobject{currentmarker}{}%
\end{pgfscope}%
\end{pgfscope}%
\begin{pgfscope}%
\pgfpathrectangle{\pgfqpoint{0.605343in}{0.417642in}}{\pgfqpoint{4.842987in}{2.930688in}}%
\pgfusepath{clip}%
\pgfsetrectcap%
\pgfsetroundjoin%
\pgfsetlinewidth{0.803000pt}%
\definecolor{currentstroke}{rgb}{0.850000,0.850000,0.850000}%
\pgfsetstrokecolor{currentstroke}%
\pgfsetdash{}{0pt}%
\pgfpathmoveto{\pgfqpoint{5.157083in}{0.417642in}}%
\pgfpathlineto{\pgfqpoint{5.157083in}{3.348330in}}%
\pgfusepath{stroke}%
\end{pgfscope}%
\begin{pgfscope}%
\pgfsetbuttcap%
\pgfsetroundjoin%
\definecolor{currentfill}{rgb}{0.000000,0.000000,0.000000}%
\pgfsetfillcolor{currentfill}%
\pgfsetlinewidth{0.602250pt}%
\definecolor{currentstroke}{rgb}{0.000000,0.000000,0.000000}%
\pgfsetstrokecolor{currentstroke}%
\pgfsetdash{}{0pt}%
\pgfsys@defobject{currentmarker}{\pgfqpoint{0.000000in}{-0.027778in}}{\pgfqpoint{0.000000in}{0.000000in}}{%
\pgfpathmoveto{\pgfqpoint{0.000000in}{0.000000in}}%
\pgfpathlineto{\pgfqpoint{0.000000in}{-0.027778in}}%
\pgfusepath{stroke,fill}%
}%
\begin{pgfscope}%
\pgfsys@transformshift{5.157083in}{0.417642in}%
\pgfsys@useobject{currentmarker}{}%
\end{pgfscope}%
\end{pgfscope}%
\begin{pgfscope}%
\pgfpathrectangle{\pgfqpoint{0.605343in}{0.417642in}}{\pgfqpoint{4.842987in}{2.930688in}}%
\pgfusepath{clip}%
\pgfsetrectcap%
\pgfsetroundjoin%
\pgfsetlinewidth{0.803000pt}%
\definecolor{currentstroke}{rgb}{0.850000,0.850000,0.850000}%
\pgfsetstrokecolor{currentstroke}%
\pgfsetdash{}{0pt}%
\pgfpathmoveto{\pgfqpoint{5.194618in}{0.417642in}}%
\pgfpathlineto{\pgfqpoint{5.194618in}{3.348330in}}%
\pgfusepath{stroke}%
\end{pgfscope}%
\begin{pgfscope}%
\pgfsetbuttcap%
\pgfsetroundjoin%
\definecolor{currentfill}{rgb}{0.000000,0.000000,0.000000}%
\pgfsetfillcolor{currentfill}%
\pgfsetlinewidth{0.602250pt}%
\definecolor{currentstroke}{rgb}{0.000000,0.000000,0.000000}%
\pgfsetstrokecolor{currentstroke}%
\pgfsetdash{}{0pt}%
\pgfsys@defobject{currentmarker}{\pgfqpoint{0.000000in}{-0.027778in}}{\pgfqpoint{0.000000in}{0.000000in}}{%
\pgfpathmoveto{\pgfqpoint{0.000000in}{0.000000in}}%
\pgfpathlineto{\pgfqpoint{0.000000in}{-0.027778in}}%
\pgfusepath{stroke,fill}%
}%
\begin{pgfscope}%
\pgfsys@transformshift{5.194618in}{0.417642in}%
\pgfsys@useobject{currentmarker}{}%
\end{pgfscope}%
\end{pgfscope}%
\begin{pgfscope}%
\definecolor{textcolor}{rgb}{0.000000,0.000000,0.000000}%
\pgfsetstrokecolor{textcolor}%
\pgfsetfillcolor{textcolor}%
\pgftext[x=3.026837in,y=0.165003in,,top]{\color{textcolor}\rmfamily\fontsize{10.000000}{12.000000}\selectfont Frequency in \unit{\Hz}}%
\end{pgfscope}%
\begin{pgfscope}%
\pgfpathrectangle{\pgfqpoint{0.605343in}{0.417642in}}{\pgfqpoint{4.842987in}{2.930688in}}%
\pgfusepath{clip}%
\pgfsetrectcap%
\pgfsetroundjoin%
\pgfsetlinewidth{0.803000pt}%
\definecolor{currentstroke}{rgb}{0.450000,0.450000,0.450000}%
\pgfsetstrokecolor{currentstroke}%
\pgfsetdash{}{0pt}%
\pgfpathmoveto{\pgfqpoint{0.605343in}{0.856755in}}%
\pgfpathlineto{\pgfqpoint{5.448330in}{0.856755in}}%
\pgfusepath{stroke}%
\end{pgfscope}%
\begin{pgfscope}%
\pgfsetbuttcap%
\pgfsetroundjoin%
\definecolor{currentfill}{rgb}{0.000000,0.000000,0.000000}%
\pgfsetfillcolor{currentfill}%
\pgfsetlinewidth{0.803000pt}%
\definecolor{currentstroke}{rgb}{0.000000,0.000000,0.000000}%
\pgfsetstrokecolor{currentstroke}%
\pgfsetdash{}{0pt}%
\pgfsys@defobject{currentmarker}{\pgfqpoint{-0.048611in}{0.000000in}}{\pgfqpoint{-0.000000in}{0.000000in}}{%
\pgfpathmoveto{\pgfqpoint{-0.000000in}{0.000000in}}%
\pgfpathlineto{\pgfqpoint{-0.048611in}{0.000000in}}%
\pgfusepath{stroke,fill}%
}%
\begin{pgfscope}%
\pgfsys@transformshift{0.605343in}{0.856755in}%
\pgfsys@useobject{currentmarker}{}%
\end{pgfscope}%
\end{pgfscope}%
\begin{pgfscope}%
\definecolor{textcolor}{rgb}{0.000000,0.000000,0.000000}%
\pgfsetstrokecolor{textcolor}%
\pgfsetfillcolor{textcolor}%
\pgftext[x=0.251948in, y=0.817602in, left, base]{\color{textcolor}\rmfamily\fontsize{8.000000}{9.600000}\selectfont \(\displaystyle {10^{-8}}\)}%
\end{pgfscope}%
\begin{pgfscope}%
\pgfpathrectangle{\pgfqpoint{0.605343in}{0.417642in}}{\pgfqpoint{4.842987in}{2.930688in}}%
\pgfusepath{clip}%
\pgfsetrectcap%
\pgfsetroundjoin%
\pgfsetlinewidth{0.803000pt}%
\definecolor{currentstroke}{rgb}{0.450000,0.450000,0.450000}%
\pgfsetstrokecolor{currentstroke}%
\pgfsetdash{}{0pt}%
\pgfpathmoveto{\pgfqpoint{0.605343in}{1.469399in}}%
\pgfpathlineto{\pgfqpoint{5.448330in}{1.469399in}}%
\pgfusepath{stroke}%
\end{pgfscope}%
\begin{pgfscope}%
\pgfsetbuttcap%
\pgfsetroundjoin%
\definecolor{currentfill}{rgb}{0.000000,0.000000,0.000000}%
\pgfsetfillcolor{currentfill}%
\pgfsetlinewidth{0.803000pt}%
\definecolor{currentstroke}{rgb}{0.000000,0.000000,0.000000}%
\pgfsetstrokecolor{currentstroke}%
\pgfsetdash{}{0pt}%
\pgfsys@defobject{currentmarker}{\pgfqpoint{-0.048611in}{0.000000in}}{\pgfqpoint{-0.000000in}{0.000000in}}{%
\pgfpathmoveto{\pgfqpoint{-0.000000in}{0.000000in}}%
\pgfpathlineto{\pgfqpoint{-0.048611in}{0.000000in}}%
\pgfusepath{stroke,fill}%
}%
\begin{pgfscope}%
\pgfsys@transformshift{0.605343in}{1.469399in}%
\pgfsys@useobject{currentmarker}{}%
\end{pgfscope}%
\end{pgfscope}%
\begin{pgfscope}%
\definecolor{textcolor}{rgb}{0.000000,0.000000,0.000000}%
\pgfsetstrokecolor{textcolor}%
\pgfsetfillcolor{textcolor}%
\pgftext[x=0.251948in, y=1.430246in, left, base]{\color{textcolor}\rmfamily\fontsize{8.000000}{9.600000}\selectfont \(\displaystyle {10^{-6}}\)}%
\end{pgfscope}%
\begin{pgfscope}%
\pgfpathrectangle{\pgfqpoint{0.605343in}{0.417642in}}{\pgfqpoint{4.842987in}{2.930688in}}%
\pgfusepath{clip}%
\pgfsetrectcap%
\pgfsetroundjoin%
\pgfsetlinewidth{0.803000pt}%
\definecolor{currentstroke}{rgb}{0.450000,0.450000,0.450000}%
\pgfsetstrokecolor{currentstroke}%
\pgfsetdash{}{0pt}%
\pgfpathmoveto{\pgfqpoint{0.605343in}{2.082042in}}%
\pgfpathlineto{\pgfqpoint{5.448330in}{2.082042in}}%
\pgfusepath{stroke}%
\end{pgfscope}%
\begin{pgfscope}%
\pgfsetbuttcap%
\pgfsetroundjoin%
\definecolor{currentfill}{rgb}{0.000000,0.000000,0.000000}%
\pgfsetfillcolor{currentfill}%
\pgfsetlinewidth{0.803000pt}%
\definecolor{currentstroke}{rgb}{0.000000,0.000000,0.000000}%
\pgfsetstrokecolor{currentstroke}%
\pgfsetdash{}{0pt}%
\pgfsys@defobject{currentmarker}{\pgfqpoint{-0.048611in}{0.000000in}}{\pgfqpoint{-0.000000in}{0.000000in}}{%
\pgfpathmoveto{\pgfqpoint{-0.000000in}{0.000000in}}%
\pgfpathlineto{\pgfqpoint{-0.048611in}{0.000000in}}%
\pgfusepath{stroke,fill}%
}%
\begin{pgfscope}%
\pgfsys@transformshift{0.605343in}{2.082042in}%
\pgfsys@useobject{currentmarker}{}%
\end{pgfscope}%
\end{pgfscope}%
\begin{pgfscope}%
\definecolor{textcolor}{rgb}{0.000000,0.000000,0.000000}%
\pgfsetstrokecolor{textcolor}%
\pgfsetfillcolor{textcolor}%
\pgftext[x=0.251948in, y=2.042890in, left, base]{\color{textcolor}\rmfamily\fontsize{8.000000}{9.600000}\selectfont \(\displaystyle {10^{-4}}\)}%
\end{pgfscope}%
\begin{pgfscope}%
\pgfpathrectangle{\pgfqpoint{0.605343in}{0.417642in}}{\pgfqpoint{4.842987in}{2.930688in}}%
\pgfusepath{clip}%
\pgfsetrectcap%
\pgfsetroundjoin%
\pgfsetlinewidth{0.803000pt}%
\definecolor{currentstroke}{rgb}{0.450000,0.450000,0.450000}%
\pgfsetstrokecolor{currentstroke}%
\pgfsetdash{}{0pt}%
\pgfpathmoveto{\pgfqpoint{0.605343in}{2.694686in}}%
\pgfpathlineto{\pgfqpoint{5.448330in}{2.694686in}}%
\pgfusepath{stroke}%
\end{pgfscope}%
\begin{pgfscope}%
\pgfsetbuttcap%
\pgfsetroundjoin%
\definecolor{currentfill}{rgb}{0.000000,0.000000,0.000000}%
\pgfsetfillcolor{currentfill}%
\pgfsetlinewidth{0.803000pt}%
\definecolor{currentstroke}{rgb}{0.000000,0.000000,0.000000}%
\pgfsetstrokecolor{currentstroke}%
\pgfsetdash{}{0pt}%
\pgfsys@defobject{currentmarker}{\pgfqpoint{-0.048611in}{0.000000in}}{\pgfqpoint{-0.000000in}{0.000000in}}{%
\pgfpathmoveto{\pgfqpoint{-0.000000in}{0.000000in}}%
\pgfpathlineto{\pgfqpoint{-0.048611in}{0.000000in}}%
\pgfusepath{stroke,fill}%
}%
\begin{pgfscope}%
\pgfsys@transformshift{0.605343in}{2.694686in}%
\pgfsys@useobject{currentmarker}{}%
\end{pgfscope}%
\end{pgfscope}%
\begin{pgfscope}%
\definecolor{textcolor}{rgb}{0.000000,0.000000,0.000000}%
\pgfsetstrokecolor{textcolor}%
\pgfsetfillcolor{textcolor}%
\pgftext[x=0.251948in, y=2.655534in, left, base]{\color{textcolor}\rmfamily\fontsize{8.000000}{9.600000}\selectfont \(\displaystyle {10^{-2}}\)}%
\end{pgfscope}%
\begin{pgfscope}%
\pgfpathrectangle{\pgfqpoint{0.605343in}{0.417642in}}{\pgfqpoint{4.842987in}{2.930688in}}%
\pgfusepath{clip}%
\pgfsetrectcap%
\pgfsetroundjoin%
\pgfsetlinewidth{0.803000pt}%
\definecolor{currentstroke}{rgb}{0.450000,0.450000,0.450000}%
\pgfsetstrokecolor{currentstroke}%
\pgfsetdash{}{0pt}%
\pgfpathmoveto{\pgfqpoint{0.605343in}{3.307330in}}%
\pgfpathlineto{\pgfqpoint{5.448330in}{3.307330in}}%
\pgfusepath{stroke}%
\end{pgfscope}%
\begin{pgfscope}%
\pgfsetbuttcap%
\pgfsetroundjoin%
\definecolor{currentfill}{rgb}{0.000000,0.000000,0.000000}%
\pgfsetfillcolor{currentfill}%
\pgfsetlinewidth{0.803000pt}%
\definecolor{currentstroke}{rgb}{0.000000,0.000000,0.000000}%
\pgfsetstrokecolor{currentstroke}%
\pgfsetdash{}{0pt}%
\pgfsys@defobject{currentmarker}{\pgfqpoint{-0.048611in}{0.000000in}}{\pgfqpoint{-0.000000in}{0.000000in}}{%
\pgfpathmoveto{\pgfqpoint{-0.000000in}{0.000000in}}%
\pgfpathlineto{\pgfqpoint{-0.048611in}{0.000000in}}%
\pgfusepath{stroke,fill}%
}%
\begin{pgfscope}%
\pgfsys@transformshift{0.605343in}{3.307330in}%
\pgfsys@useobject{currentmarker}{}%
\end{pgfscope}%
\end{pgfscope}%
\begin{pgfscope}%
\definecolor{textcolor}{rgb}{0.000000,0.000000,0.000000}%
\pgfsetstrokecolor{textcolor}%
\pgfsetfillcolor{textcolor}%
\pgftext[x=0.332194in, y=3.268178in, left, base]{\color{textcolor}\rmfamily\fontsize{8.000000}{9.600000}\selectfont \(\displaystyle {10^{0}}\)}%
\end{pgfscope}%
\begin{pgfscope}%
\definecolor{textcolor}{rgb}{0.000000,0.000000,0.000000}%
\pgfsetstrokecolor{textcolor}%
\pgfsetfillcolor{textcolor}%
\pgftext[x=0.196393in,y=1.882986in,,bottom,rotate=90.000000]{\color{textcolor}\rmfamily\fontsize{10.000000}{12.000000}\selectfont Power spectral density in \(\displaystyle \unit{\V^2 \per \Hz}\)}%
\end{pgfscope}%
\begin{pgfscope}%
\pgfpathrectangle{\pgfqpoint{0.605343in}{0.417642in}}{\pgfqpoint{4.842987in}{2.930688in}}%
\pgfusepath{clip}%
\pgfsetrectcap%
\pgfsetroundjoin%
\pgfsetlinewidth{1.003750pt}%
\definecolor{currentstroke}{rgb}{0.000000,0.447059,0.698039}%
\pgfsetstrokecolor{currentstroke}%
\pgfsetstrokeopacity{0.700000}%
\pgfsetdash{}{0pt}%
\pgfpathmoveto{\pgfqpoint{0.825479in}{3.215117in}}%
\pgfpathlineto{\pgfqpoint{0.915331in}{3.215116in}}%
\pgfpathlineto{\pgfqpoint{1.005182in}{3.215114in}}%
\pgfpathlineto{\pgfqpoint{1.095033in}{3.215111in}}%
\pgfpathlineto{\pgfqpoint{1.184885in}{3.215106in}}%
\pgfpathlineto{\pgfqpoint{1.274736in}{3.215096in}}%
\pgfpathlineto{\pgfqpoint{1.364587in}{3.215080in}}%
\pgfpathlineto{\pgfqpoint{1.454439in}{3.215050in}}%
\pgfpathlineto{\pgfqpoint{1.544290in}{3.214999in}}%
\pgfpathlineto{\pgfqpoint{1.634141in}{3.214908in}}%
\pgfpathlineto{\pgfqpoint{1.723992in}{3.214750in}}%
\pgfpathlineto{\pgfqpoint{1.813844in}{3.214471in}}%
\pgfpathlineto{\pgfqpoint{1.903695in}{3.213983in}}%
\pgfpathlineto{\pgfqpoint{1.993546in}{3.213129in}}%
\pgfpathlineto{\pgfqpoint{2.083398in}{3.211642in}}%
\pgfpathlineto{\pgfqpoint{2.173249in}{3.209067in}}%
\pgfpathlineto{\pgfqpoint{2.263100in}{3.204660in}}%
\pgfpathlineto{\pgfqpoint{2.352952in}{3.197252in}}%
\pgfpathlineto{\pgfqpoint{2.442803in}{3.185154in}}%
\pgfpathlineto{\pgfqpoint{2.532654in}{3.166241in}}%
\pgfpathlineto{\pgfqpoint{2.622506in}{3.138387in}}%
\pgfpathlineto{\pgfqpoint{2.712357in}{3.100247in}}%
\pgfpathlineto{\pgfqpoint{2.802208in}{3.051916in}}%
\pgfpathlineto{\pgfqpoint{2.892060in}{2.994874in}}%
\pgfpathlineto{\pgfqpoint{2.981911in}{2.931284in}}%
\pgfpathlineto{\pgfqpoint{3.071762in}{2.863236in}}%
\pgfpathlineto{\pgfqpoint{3.161614in}{2.792353in}}%
\pgfpathlineto{\pgfqpoint{3.251465in}{2.719747in}}%
\pgfpathlineto{\pgfqpoint{3.341316in}{2.646121in}}%
\pgfpathlineto{\pgfqpoint{3.431168in}{2.571902in}}%
\pgfpathlineto{\pgfqpoint{3.521019in}{2.497340in}}%
\pgfpathlineto{\pgfqpoint{3.610870in}{2.422583in}}%
\pgfpathlineto{\pgfqpoint{3.700722in}{2.347714in}}%
\pgfpathlineto{\pgfqpoint{3.790573in}{2.272781in}}%
\pgfpathlineto{\pgfqpoint{3.880424in}{2.197811in}}%
\pgfpathlineto{\pgfqpoint{3.970276in}{2.122821in}}%
\pgfpathlineto{\pgfqpoint{4.060127in}{2.047819in}}%
\pgfpathlineto{\pgfqpoint{4.149978in}{1.972810in}}%
\pgfpathlineto{\pgfqpoint{4.239830in}{1.897798in}}%
\pgfpathlineto{\pgfqpoint{4.329681in}{1.822783in}}%
\pgfpathlineto{\pgfqpoint{4.419532in}{1.747767in}}%
\pgfpathlineto{\pgfqpoint{4.509384in}{1.672750in}}%
\pgfpathlineto{\pgfqpoint{4.599235in}{1.597733in}}%
\pgfpathlineto{\pgfqpoint{4.689086in}{1.522716in}}%
\pgfpathlineto{\pgfqpoint{4.778938in}{1.447698in}}%
\pgfpathlineto{\pgfqpoint{4.868789in}{1.372681in}}%
\pgfpathlineto{\pgfqpoint{4.958640in}{1.297663in}}%
\pgfpathlineto{\pgfqpoint{5.048492in}{1.222646in}}%
\pgfpathlineto{\pgfqpoint{5.138343in}{1.147628in}}%
\pgfpathlineto{\pgfqpoint{5.228194in}{1.072611in}}%
\pgfusepath{stroke}%
\end{pgfscope}%
\begin{pgfscope}%
\pgfpathrectangle{\pgfqpoint{0.605343in}{0.417642in}}{\pgfqpoint{4.842987in}{2.930688in}}%
\pgfusepath{clip}%
\pgfsetrectcap%
\pgfsetroundjoin%
\pgfsetlinewidth{1.003750pt}%
\definecolor{currentstroke}{rgb}{0.000000,0.619608,0.450980}%
\pgfsetstrokecolor{currentstroke}%
\pgfsetstrokeopacity{0.700000}%
\pgfsetdash{}{0pt}%
\pgfpathmoveto{\pgfqpoint{0.825479in}{3.147390in}}%
\pgfpathlineto{\pgfqpoint{0.915331in}{3.147386in}}%
\pgfpathlineto{\pgfqpoint{1.005182in}{3.147381in}}%
\pgfpathlineto{\pgfqpoint{1.095033in}{3.147371in}}%
\pgfpathlineto{\pgfqpoint{1.184885in}{3.147353in}}%
\pgfpathlineto{\pgfqpoint{1.274736in}{3.147321in}}%
\pgfpathlineto{\pgfqpoint{1.364587in}{3.147266in}}%
\pgfpathlineto{\pgfqpoint{1.454439in}{3.147169in}}%
\pgfpathlineto{\pgfqpoint{1.544290in}{3.147000in}}%
\pgfpathlineto{\pgfqpoint{1.634141in}{3.146701in}}%
\pgfpathlineto{\pgfqpoint{1.723992in}{3.146179in}}%
\pgfpathlineto{\pgfqpoint{1.813844in}{3.145266in}}%
\pgfpathlineto{\pgfqpoint{1.903695in}{3.143677in}}%
\pgfpathlineto{\pgfqpoint{1.993546in}{3.140928in}}%
\pgfpathlineto{\pgfqpoint{2.083398in}{3.136232in}}%
\pgfpathlineto{\pgfqpoint{2.173249in}{3.128359in}}%
\pgfpathlineto{\pgfqpoint{2.263100in}{3.115559in}}%
\pgfpathlineto{\pgfqpoint{2.352952in}{3.095675in}}%
\pgfpathlineto{\pgfqpoint{2.442803in}{3.066629in}}%
\pgfpathlineto{\pgfqpoint{2.532654in}{3.027227in}}%
\pgfpathlineto{\pgfqpoint{2.622506in}{2.977747in}}%
\pgfpathlineto{\pgfqpoint{2.712357in}{2.919797in}}%
\pgfpathlineto{\pgfqpoint{2.802208in}{2.855568in}}%
\pgfpathlineto{\pgfqpoint{2.892060in}{2.787103in}}%
\pgfpathlineto{\pgfqpoint{2.981911in}{2.715963in}}%
\pgfpathlineto{\pgfqpoint{3.071762in}{2.643204in}}%
\pgfpathlineto{\pgfqpoint{3.161614in}{2.569488in}}%
\pgfpathlineto{\pgfqpoint{3.251465in}{2.495217in}}%
\pgfpathlineto{\pgfqpoint{3.341316in}{2.420626in}}%
\pgfpathlineto{\pgfqpoint{3.431168in}{2.345852in}}%
\pgfpathlineto{\pgfqpoint{3.521019in}{2.270973in}}%
\pgfpathlineto{\pgfqpoint{3.610870in}{2.196034in}}%
\pgfpathlineto{\pgfqpoint{3.700722in}{2.121061in}}%
\pgfpathlineto{\pgfqpoint{3.790573in}{2.046069in}}%
\pgfpathlineto{\pgfqpoint{3.880424in}{1.971066in}}%
\pgfpathlineto{\pgfqpoint{3.970276in}{1.896057in}}%
\pgfpathlineto{\pgfqpoint{4.060127in}{1.821044in}}%
\pgfpathlineto{\pgfqpoint{4.149978in}{1.746029in}}%
\pgfpathlineto{\pgfqpoint{4.239830in}{1.671013in}}%
\pgfpathlineto{\pgfqpoint{4.329681in}{1.595996in}}%
\pgfpathlineto{\pgfqpoint{4.419532in}{1.520979in}}%
\pgfpathlineto{\pgfqpoint{4.509384in}{1.445962in}}%
\pgfpathlineto{\pgfqpoint{4.599235in}{1.370944in}}%
\pgfpathlineto{\pgfqpoint{4.689086in}{1.295927in}}%
\pgfpathlineto{\pgfqpoint{4.778938in}{1.220909in}}%
\pgfpathlineto{\pgfqpoint{4.868789in}{1.145892in}}%
\pgfpathlineto{\pgfqpoint{4.958640in}{1.070874in}}%
\pgfpathlineto{\pgfqpoint{5.048492in}{0.995856in}}%
\pgfpathlineto{\pgfqpoint{5.138343in}{0.920839in}}%
\pgfpathlineto{\pgfqpoint{5.228194in}{0.845821in}}%
\pgfusepath{stroke}%
\end{pgfscope}%
\begin{pgfscope}%
\pgfpathrectangle{\pgfqpoint{0.605343in}{0.417642in}}{\pgfqpoint{4.842987in}{2.930688in}}%
\pgfusepath{clip}%
\pgfsetrectcap%
\pgfsetroundjoin%
\pgfsetlinewidth{1.003750pt}%
\definecolor{currentstroke}{rgb}{0.835294,0.368627,0.000000}%
\pgfsetstrokecolor{currentstroke}%
\pgfsetstrokeopacity{0.700000}%
\pgfsetdash{}{0pt}%
\pgfpathmoveto{\pgfqpoint{0.825479in}{2.875134in}}%
\pgfpathlineto{\pgfqpoint{0.915331in}{2.875130in}}%
\pgfpathlineto{\pgfqpoint{1.005182in}{2.875123in}}%
\pgfpathlineto{\pgfqpoint{1.095033in}{2.875111in}}%
\pgfpathlineto{\pgfqpoint{1.184885in}{2.875090in}}%
\pgfpathlineto{\pgfqpoint{1.274736in}{2.875053in}}%
\pgfpathlineto{\pgfqpoint{1.364587in}{2.874988in}}%
\pgfpathlineto{\pgfqpoint{1.454439in}{2.874873in}}%
\pgfpathlineto{\pgfqpoint{1.544290in}{2.874672in}}%
\pgfpathlineto{\pgfqpoint{1.634141in}{2.874318in}}%
\pgfpathlineto{\pgfqpoint{1.723992in}{2.873700in}}%
\pgfpathlineto{\pgfqpoint{1.813844in}{2.872619in}}%
\pgfpathlineto{\pgfqpoint{1.903695in}{2.870741in}}%
\pgfpathlineto{\pgfqpoint{1.993546in}{2.867504in}}%
\pgfpathlineto{\pgfqpoint{2.083398in}{2.861999in}}%
\pgfpathlineto{\pgfqpoint{2.173249in}{2.852842in}}%
\pgfpathlineto{\pgfqpoint{2.263100in}{2.838136in}}%
\pgfpathlineto{\pgfqpoint{2.352952in}{2.815673in}}%
\pgfpathlineto{\pgfqpoint{2.442803in}{2.783557in}}%
\pgfpathlineto{\pgfqpoint{2.532654in}{2.741002in}}%
\pgfpathlineto{\pgfqpoint{2.622506in}{2.688745in}}%
\pgfpathlineto{\pgfqpoint{2.712357in}{2.628659in}}%
\pgfpathlineto{\pgfqpoint{2.802208in}{2.562951in}}%
\pgfpathlineto{\pgfqpoint{2.892060in}{2.493536in}}%
\pgfpathlineto{\pgfqpoint{2.981911in}{2.421815in}}%
\pgfpathlineto{\pgfqpoint{3.071762in}{2.348710in}}%
\pgfpathlineto{\pgfqpoint{3.161614in}{2.274793in}}%
\pgfpathlineto{\pgfqpoint{3.251465in}{2.200406in}}%
\pgfpathlineto{\pgfqpoint{3.341316in}{2.125748in}}%
\pgfpathlineto{\pgfqpoint{3.431168in}{2.050936in}}%
\pgfpathlineto{\pgfqpoint{3.521019in}{1.976035in}}%
\pgfpathlineto{\pgfqpoint{3.610870in}{1.901084in}}%
\pgfpathlineto{\pgfqpoint{3.700722in}{1.826105in}}%
\pgfpathlineto{\pgfqpoint{3.790573in}{1.751109in}}%
\pgfpathlineto{\pgfqpoint{3.880424in}{1.676103in}}%
\pgfpathlineto{\pgfqpoint{3.970276in}{1.601093in}}%
\pgfpathlineto{\pgfqpoint{4.060127in}{1.526079in}}%
\pgfpathlineto{\pgfqpoint{4.149978in}{1.451064in}}%
\pgfpathlineto{\pgfqpoint{4.239830in}{1.376047in}}%
\pgfpathlineto{\pgfqpoint{4.329681in}{1.301030in}}%
\pgfpathlineto{\pgfqpoint{4.419532in}{1.226013in}}%
\pgfpathlineto{\pgfqpoint{4.509384in}{1.150996in}}%
\pgfpathlineto{\pgfqpoint{4.599235in}{1.075978in}}%
\pgfpathlineto{\pgfqpoint{4.689086in}{1.000961in}}%
\pgfpathlineto{\pgfqpoint{4.778938in}{0.925943in}}%
\pgfpathlineto{\pgfqpoint{4.868789in}{0.850926in}}%
\pgfpathlineto{\pgfqpoint{4.958640in}{0.775908in}}%
\pgfpathlineto{\pgfqpoint{5.048492in}{0.700890in}}%
\pgfpathlineto{\pgfqpoint{5.138343in}{0.625873in}}%
\pgfpathlineto{\pgfqpoint{5.228194in}{0.550855in}}%
\pgfusepath{stroke}%
\end{pgfscope}%
\begin{pgfscope}%
\pgfsetrectcap%
\pgfsetmiterjoin%
\pgfsetlinewidth{0.803000pt}%
\definecolor{currentstroke}{rgb}{0.000000,0.000000,0.000000}%
\pgfsetstrokecolor{currentstroke}%
\pgfsetdash{}{0pt}%
\pgfpathmoveto{\pgfqpoint{0.605343in}{0.417642in}}%
\pgfpathlineto{\pgfqpoint{0.605343in}{3.348330in}}%
\pgfusepath{stroke}%
\end{pgfscope}%
\begin{pgfscope}%
\pgfsetrectcap%
\pgfsetmiterjoin%
\pgfsetlinewidth{0.803000pt}%
\definecolor{currentstroke}{rgb}{0.000000,0.000000,0.000000}%
\pgfsetstrokecolor{currentstroke}%
\pgfsetdash{}{0pt}%
\pgfpathmoveto{\pgfqpoint{5.448330in}{0.417642in}}%
\pgfpathlineto{\pgfqpoint{5.448330in}{3.348330in}}%
\pgfusepath{stroke}%
\end{pgfscope}%
\begin{pgfscope}%
\pgfsetrectcap%
\pgfsetmiterjoin%
\pgfsetlinewidth{0.803000pt}%
\definecolor{currentstroke}{rgb}{0.000000,0.000000,0.000000}%
\pgfsetstrokecolor{currentstroke}%
\pgfsetdash{}{0pt}%
\pgfpathmoveto{\pgfqpoint{0.605343in}{0.417642in}}%
\pgfpathlineto{\pgfqpoint{5.448330in}{0.417642in}}%
\pgfusepath{stroke}%
\end{pgfscope}%
\begin{pgfscope}%
\pgfsetrectcap%
\pgfsetmiterjoin%
\pgfsetlinewidth{0.803000pt}%
\definecolor{currentstroke}{rgb}{0.000000,0.000000,0.000000}%
\pgfsetstrokecolor{currentstroke}%
\pgfsetdash{}{0pt}%
\pgfpathmoveto{\pgfqpoint{0.605343in}{3.348330in}}%
\pgfpathlineto{\pgfqpoint{5.448330in}{3.348330in}}%
\pgfusepath{stroke}%
\end{pgfscope}%
\begin{pgfscope}%
\pgfsetbuttcap%
\pgfsetmiterjoin%
\definecolor{currentfill}{rgb}{1.000000,1.000000,1.000000}%
\pgfsetfillcolor{currentfill}%
\pgfsetfillopacity{0.800000}%
\pgfsetlinewidth{1.003750pt}%
\definecolor{currentstroke}{rgb}{0.800000,0.800000,0.800000}%
\pgfsetstrokecolor{currentstroke}%
\pgfsetstrokeopacity{0.800000}%
\pgfsetdash{}{0pt}%
\pgfpathmoveto{\pgfqpoint{4.570239in}{2.794775in}}%
\pgfpathlineto{\pgfqpoint{5.370552in}{2.794775in}}%
\pgfpathquadraticcurveto{\pgfqpoint{5.392774in}{2.794775in}}{\pgfqpoint{5.392774in}{2.816997in}}%
\pgfpathlineto{\pgfqpoint{5.392774in}{3.270552in}}%
\pgfpathquadraticcurveto{\pgfqpoint{5.392774in}{3.292774in}}{\pgfqpoint{5.370552in}{3.292774in}}%
\pgfpathlineto{\pgfqpoint{4.570239in}{3.292774in}}%
\pgfpathquadraticcurveto{\pgfqpoint{4.548017in}{3.292774in}}{\pgfqpoint{4.548017in}{3.270552in}}%
\pgfpathlineto{\pgfqpoint{4.548017in}{2.816997in}}%
\pgfpathquadraticcurveto{\pgfqpoint{4.548017in}{2.794775in}}{\pgfqpoint{4.570239in}{2.794775in}}%
\pgfpathlineto{\pgfqpoint{4.570239in}{2.794775in}}%
\pgfpathclose%
\pgfusepath{stroke,fill}%
\end{pgfscope}%
\begin{pgfscope}%
\pgfsetrectcap%
\pgfsetroundjoin%
\pgfsetlinewidth{1.003750pt}%
\definecolor{currentstroke}{rgb}{0.000000,0.447059,0.698039}%
\pgfsetstrokecolor{currentstroke}%
\pgfsetstrokeopacity{0.700000}%
\pgfsetdash{}{0pt}%
\pgfpathmoveto{\pgfqpoint{4.592461in}{3.209441in}}%
\pgfpathlineto{\pgfqpoint{4.703572in}{3.209441in}}%
\pgfpathlineto{\pgfqpoint{4.814683in}{3.209441in}}%
\pgfusepath{stroke}%
\end{pgfscope}%
\begin{pgfscope}%
\definecolor{textcolor}{rgb}{0.000000,0.000000,0.000000}%
\pgfsetstrokecolor{textcolor}%
\pgfsetfillcolor{textcolor}%
\pgftext[x=4.903572in,y=3.170552in,left,base]{\color{textcolor}\rmfamily\fontsize{8.000000}{9.600000}\selectfont \(\displaystyle \tau_1=1\)}%
\end{pgfscope}%
\begin{pgfscope}%
\pgfsetrectcap%
\pgfsetroundjoin%
\pgfsetlinewidth{1.003750pt}%
\definecolor{currentstroke}{rgb}{0.000000,0.619608,0.450980}%
\pgfsetstrokecolor{currentstroke}%
\pgfsetstrokeopacity{0.700000}%
\pgfsetdash{}{0pt}%
\pgfpathmoveto{\pgfqpoint{4.592461in}{3.054552in}}%
\pgfpathlineto{\pgfqpoint{4.703572in}{3.054552in}}%
\pgfpathlineto{\pgfqpoint{4.814683in}{3.054552in}}%
\pgfusepath{stroke}%
\end{pgfscope}%
\begin{pgfscope}%
\definecolor{textcolor}{rgb}{0.000000,0.000000,0.000000}%
\pgfsetstrokecolor{textcolor}%
\pgfsetfillcolor{textcolor}%
\pgftext[x=4.903572in,y=3.015663in,left,base]{\color{textcolor}\rmfamily\fontsize{8.000000}{9.600000}\selectfont \(\displaystyle \tau_1=10\)}%
\end{pgfscope}%
\begin{pgfscope}%
\pgfsetrectcap%
\pgfsetroundjoin%
\pgfsetlinewidth{1.003750pt}%
\definecolor{currentstroke}{rgb}{0.835294,0.368627,0.000000}%
\pgfsetstrokecolor{currentstroke}%
\pgfsetstrokeopacity{0.700000}%
\pgfsetdash{}{0pt}%
\pgfpathmoveto{\pgfqpoint{4.592461in}{2.899664in}}%
\pgfpathlineto{\pgfqpoint{4.703572in}{2.899664in}}%
\pgfpathlineto{\pgfqpoint{4.814683in}{2.899664in}}%
\pgfusepath{stroke}%
\end{pgfscope}%
\begin{pgfscope}%
\definecolor{textcolor}{rgb}{0.000000,0.000000,0.000000}%
\pgfsetstrokecolor{textcolor}%
\pgfsetfillcolor{textcolor}%
\pgftext[x=4.903572in,y=2.860775in,left,base]{\color{textcolor}\rmfamily\fontsize{8.000000}{9.600000}\selectfont \(\displaystyle \tau_1=100\)}%
\end{pgfscope}%
\end{pgfpicture}%
\makeatother%
\endgroup%

%    \caption{Burst noise for $\tau_0=\qty{1}{\second}$ and different $\tau_1$}
%    \label{fig:burst_noise_psd}
%\end{figure}




This type of noise has become less prevalent in modern manufacturing processes, because the quality of the semiconductors has improved. But if, for example a trap site is located very close to an important structure, for example a high precision Zener diode, its effect might be so strong, that it can be clearly seen.


The small wiggles at longer $\tau$ are typical end-of-data errors caused by spectral leakage, because there are insufficient samples to average over \cite{adev_long_tau}. As it was discussed above, the Allan deviation can only be estimated given a limited number of samples using equation \ref{eqn:adev_estimator}. This leads to the fact, that at $\frac{\tau}{2}$ there are only $2$ samples left, so there no averaging possible to improve the estimate of the Allan deviation. This leads to oscillations at low frequencies or large $\tau$.


% TODO: Add adev simulations

The coefficients given here are using the assumption, that the Allan deviation is the appropriate measure for the sample data. This might not always be the case, because the Allan deviation assumes a dead time of $\theta = 0$. This problem was extensively discussed by \citeauthor{psd_to_adev} \cite{psd_to_adev} and even special models were developed to account for the algorithms of modern frequency counters \cite{adev_frequency_counter}. It is therefore important to discuss typical measurement settings for voltmeter to estimate errors that arise from those settings. Typical settings, that affect the dead time of a voltmeter are auto-zeroing and line synchronization. Auto-zeroing is typically done by adding additional measurements to the normal input integratiion cycle. These measurements are a zero measurement to correct for offset drift and a measurement of the reference voltage to correct for gain errors. The implementation details and type of measurements are manufacturer dependant and must be determinded for every multimeter used.

% check \cite{psd_to_adev} Appendix II for details on dead time
% Compare PSD in Generation-Recombination Noise, Allan Variance, and Low-Frequency Gain Instabilities in Microwave Amplifiers to our controller. The hump look similar. Due to popcorn noise

\section{Temperature Controller}

\subsection{Tuning of a PID controller}
The number of emperical algorithms to determine a set of PID parameters ($\mathrm{K_p, K_i, K_d}$) are numerous. In this work only the most common algorithms and a few notable exceptions will be presented.
\subsection{Design}



% FIXME: Change variable names according to Noise Sources in Bulk CMOS
\chapter{Preparation}%
\label{sec:preparation}
\begin{chapquote}{Lewis Carroll, \textit{Alice in Wonderland}}
``Begin at the beginning,'' the King said gravely, ``and go on till you come to the end: then stop.''
\end{chapquote}
%\section{Grounding and Shielding}
%Add parts from "references\\Grounding and Shielding.pdf"

\section{Laser System}%
\label{sec:prep_laser_system}
%\subsection{Requirements Laser System}
The ARTEMIS experiment is currently in the process of commissioning. Due to the relative abundance of Argon, the ability to create highly charged ion species using an electron beam along with a scientifically relevant transition at $\lambda = \qty{441.25575(17)}{\nm}$ \cite{ar13+_wavelength} with a lifetime of \qty{9.573(6)}{\ms} \cite{ar13+_lifetime} makes \ce{Ar^{13+}} an ideal candidate for this purpose. Figure \ref{fig:bohr_ar13} shows the simplified electronic configuration of the boron-like \ce{Ar^{13+}} investigated.
\begin{figure}[ht]
    \centering
    \begin{subfigure}{0.28\linewidth}
        \centering
        \import{figures/}{fig_bohr_Ar13.tex}
        \caption{Shell model of boron-like \ce{Ar^{13+}}.}
        \label{fig:bohr_ar13}
    \end{subfigure}
    \begin{subfigure}{0.6\linewidth}
        \centering
        \scalebox{0.75}{%
            \import{figures/}{fig_level_transitions.tex}
        }% scalebox
        \caption{Level diagram of the \ce{Ar^{13+}} transitions accessible at \qty{441}{\nm}. Frequencies with numbers denote optical transitions, letters denote microwave transitions.}
        \label{fig:level_transitions}
    \end{subfigure}
    \caption{Electronic configuration and optically accessible transitions of \ce{Ar^{13+}}.}
    \label{fig:ar13}
\end{figure}

The optical transitions of highly charged \ce{Ar^{13+}} around \qty{441}{\nm} shown in figure \ref{fig:level_transitions} can be used for laser spectroscopy. The transition of interest for commissioning is the transition from the ground state $[(1s)^2 (2s)^2 2p]^2 \, P_{1/2}$ to the first exited state $[(1s)^2 (2s)^2 2p]^2 \, P_{3/2}$ \cite{optical_transitions_ar13}. The Zeeman splitting introduced by the \qty{7}{\tesla} magnet of the ARTEMIS Penning trap results in a frequency splitting of the $^2P_{1/2}$ ground state by $\nu_d = \qty{65}{\GHz}$ and the exited $^2 P_{3/2}$ state of $\nu_a = \qty{130}{\GHz}$. The natural linewidth of these transitions is $\Gamma \approx 2 \pi \times \qty{16.63(1)}{\Hz}$ which is fairly small, but there is substantial Doppler broadening of
\begin{equation}
    \Delta \nu (\lambda = \qty{441}{\nm}, T=\qty{4}{\K}, m=\qty{39.948}{u}) = \frac{2}{\lambda}\sqrt{2 \ln 2 \frac{k_B T}{m}} \approx 2 \pi \times \qty{150}{\MHz} \, , \label{eqn:doppler_broadening}
\end{equation}
seen in the trap which is kept at a temperature of \qty{4}{\K}.

The simplified laser setup shown in figure \ref{fig:laser_system_old} as a schematic was characterised by \citeauthor{thesis_alex} \cite{thesis_alex} and its transfer accuracy of the the wavelength was calculated to be \qty{1.2}{\MHz}. Considering the additional long-term drift of the system led to an upper limit of the absolute wavelength uncertainty of \qty{2.2}{\MHz}. This can be considered more than adequate given the Doppler broadening of \qty{150}{\MHz} calculated in equation \ref{eqn:doppler_broadening}. While being sufficiently accurate, the system has the significant drawback of being fairly complicated to manage if not maintained on a daily basis. Its performance and uncertainty relies on the exact knowledge of the tellurium spectrum surrounding both wavelengths of the lasers.
\begin{figure}[ht]
    \centering
    %\scalebox{1} % scalebox
    \caption{Simplified setup of the laser system in use at ARTEMIS prior to this work. Blue lines are laser beams, black lines are electronic signals delivering feedback.}
    \label{fig:laser_system_old}
\end{figure}

To operate the system, the master laser must first be locked to a tellurium reference transition which has to be searched manually using the tellurium spectrum charted by \cite{thesis_alex}. While not complicated, it requires a trained user nonetheless. The transfer cavity is then locked to the master laser, a straight forward process. The spectroscopy laser now has to be locked to the correct fringe of the transfer cavity. This is done by observing the tellurium background again to adjust the diode laser mode to a frequency close to the desired cavity fringe. This is by no means an automated process and requires a competent operator. Although this locking procedure requires training it does work reliably, but there are more issues that are not readily apparent.

One potential problem lies with the master laser. The whole calibration is geared towards a reference transition at \qty{452.756}{\nm} catalogued by \citeauthor{te_reference_lines} \cite{te_reference_lines}. Replacing the master laser in case of a failure is challenging. Blue laser diodes are less flexible when tuning in comparison to their NIR cousins as discussed in \cite{thesis_baus,thesis_alex}. The diode used in the current laser was handpicked for this wavelength. This limits the availability of replacement parts and increases their replacement value. Using another diode and wavelength along with a different tellurium transition would require creating a new tellurium map which is a laborious process.

Another challenge is created by the locking scheme which becomes more complicated when introducing a third laser to complete the closed transition for the laser-microwave double-resonance spectroscopy. The setup in its present condition was already prepared for the second spectroscopy laser which must also be locked to the transfer cavity. Currently both lasers are sent into the transfer cavity with perpendicular polarisations to separate the beams using a polarizing beam splitter. Since the reference laser is running at \qty{453}{\nm} while the spectroscopy laser is using \qty{441}{\nm}, it is also possible to separate those two via a dichroic mirror. While possible, this scheme requires the close overlap of three beams along with their reflection as required for saturation. Such a setup strongly couples the three beam paths and further complicates future adjustments.

The final issue regarding the transfer cavity concerns the required high voltage for the piezoelectric actuator to adjust its length. This piezo requires up to \qty{1.25}{\kV} to reach the necessary translation required for scanning the tellurium spectrum. Not only does this pose a risk to an untrained operator, but it was also discovered during the commissioning of the current system that high frequency noise from the voltage supply is radiated and can enter the experiment. These issues were minimized by testing several power supplies, choosing the lowest noise device and keeping the supply well separated from the rest of the experiment increasing space required for the whole system.

During the commissioning and testing of the laser system it was also found that the laser drivers had issues with blue laser diodes. Those drivers used a modified in-house current source normally used for NIR diodes based on the design of \citeauthor{libbrecht_hall} \cite{libbrecht_hall}. With these drivers an increasing instability was observed when adjusting the current up to the operating point. The origin of this problem lies in the larger operating voltage of the blue laser diode of up to \qty{7}{\V} compared to the more moderate \qty{2.5}{\V} of NIR diodes. The details are discussed in section \ref{sec:compliance_voltage}. Other commercial drivers tested either had the same problem or were far noisier and therefore harder to work with, due to the modification made by the manufacturers to increase the compliance voltage.

To conclude, the commissioning of the laser system has brought up two issues with the current system that impact its availability and performance. The transfer cavity can be considered the Achilles' heel of the system and replacing it with a more flexible alternative like a high performance wavemeter can greatly improve the usability and flexibility of the whole laser system as it breaks up the tight dependency on the tellurium reference laser. Additionally this opens up alternative wavelengths for the spectroscopy of other highly charges ion species. The second issue was found with the laser driver and the apparent lack of commercial solutions sparked the development of a novel laser driver for the next generation of laser diodes to gain direct access to more wavelengths. This is discussed in the next section.

\clearpage
\section{Laser Current Driver}%
\label{sec:laser_current_driver}
% Include Emission wavelength dependence of characteristic temperature of InGaN laser diodes
% Check Diode Laser Characteristics
% I-lamda in Wavelength Dependence of InGaN Laser Diode Characteristics
% also Determination of piezoelectric fields in strained GaInN quantum wells using the quantum-confined Stark effect
Laser diodes are current driven devices, because
\begin{equation*}
    P_{out} \propto I
\end{equation*}
and the diode current $I$ approximately follows the Shockley equation \cite{shockley_diode}
\begin{equation}
    I = I_0 \left( e^{\frac{qV_d}{k_B T}} - 1\right) \, . \label{eqn:shockley}
\end{equation}

where $k_B$ is Boltzmann constant, $T$ the temperature, $q$ the electron charge and $V_d$ the diode voltage. The exponential dependence of the current on the supply voltage calls for a current source to drive a laser diode safely without risking thermal damage due to excessive injection currents.

The primary function of a laser driver is therefore to provide a stable, but user adjustable, current. This current can typically be modulated at frequencies up to several \unit{\MHz} to shape the frequency and amplitude of the laser output light. Additional features, like current and voltage limits, aid in protecting the expensive laser diodes and it is not uncommon to have additional safeguards inside the laser head that are under control of the laser driver like a shorting relay to ensure the laser diode is shorted when the driver is disconnected or disabled.

The focus of this work lies on two types of laser diodes, indium gallium nitride (InGaN) and aluminium gallium arsenide (AlGaAs), but is not limited to those two types. The former material is, for example, used for blue laser diodes at around \qty{450}{\nm}, discussed in the previous section, and laser diodes up to green wavelengths, the latter is used for near-infrared (NIR) laser diodes such as \qty{780}{\nm} laser diodes. Both wavelengths are used for experiments in the Atoms-Photons-Quanta group (in future referred to as \textit{the group}). The former type is used in the ARTEMIS experiment for the spectroscopy of highly charged ions, the latter is extensively used to manipulate and control the rubidium atoms used by the quantum computing experiments. This section deals with the design challenges of such a device used for high precision laser spectroscopy. First the design requirements are established and, from those conditions, technical specifications are developed.

The design requirements are split into three parts which need to be discussed: The environment includes effects like temperature, humidity and time. That section mostly focuses on the ambient temperature though because its effects are the most pronounced. The current source electrical requirements, like drift, noise, output impedance, and modulation bandwidth are discussed as these have a profound impact on the intended application in experiments. Finally, the user interface including the external communication interfaces are defined.

\subsection{Design Goals: Ambient Environment}%
\label{sec:design_goal_environment}
The lasers and its accompanying driver is to be mostly used in a clean laboratory environment. In this particular use case the air is filtered using H14 HEPA filters, but less rigorously controlled environments must be considered as well, because not all fields of application are in optical labs. A mostly dust free industrial environment is considered acceptable as well. Typical lab temperatures are in the range of \qtyrange[range-phrase=\textup{~to~}]{20}{30}{\celsius}. This temperature range was also encountered in the labs discussed in this work before improvements were implemented as part of this work. The upper end of the range must be considered when operating the devices inside a rack where the temperatures are even higher and the device should therefore be tested for its upper limit. A temperature of \qty{35}{\celsius} is a typical value measured inside the racks used in the lab. Humidity is only controlled with dehumidifiers, limiting only the upper bound, resulting in a range of \qtyrange[range-phrase=\textup{~to~}]{15}{60}{\percent rH}.

Figure \ref{fig:lab_temperature_start_of_project} shows a typical one day span of the lab temperature as it was found at the start of this project, plotted using Matplotlib \cite{matplotlib}.
\begin{figure}[ht]
    \centering
    %% Creator: Matplotlib, PGF backend
%%
%% To include the figure in your LaTeX document, write
%%   \input{<filename>.pgf}
%%
%% Make sure the required packages are loaded in your preamble
%%   \usepackage{pgf}
%%
%% Also ensure that all the required font packages are loaded; for instance,
%% the lmodern package is sometimes necessary when using math font.
%%   \usepackage{lmodern}
%%
%% Figures using additional raster images can only be included by \input if
%% they are in the same directory as the main LaTeX file. For loading figures
%% from other directories you can use the `import` package
%%   \usepackage{import}
%%
%% and then include the figures with
%%   \import{<path to file>}{<filename>.pgf}
%%
%% Matplotlib used the following preamble
%%   \usepackage{siunitx}
%%   \usepackage{fontspec}
%%
\begingroup%
\makeatletter%
\begin{pgfpicture}%
\pgfpathrectangle{\pgfpointorigin}{\pgfqpoint{5.208662in}{3.219130in}}%
\pgfusepath{use as bounding box, clip}%
\begin{pgfscope}%
\pgfsetbuttcap%
\pgfsetmiterjoin%
\definecolor{currentfill}{rgb}{1.000000,1.000000,1.000000}%
\pgfsetfillcolor{currentfill}%
\pgfsetlinewidth{0.000000pt}%
\definecolor{currentstroke}{rgb}{1.000000,1.000000,1.000000}%
\pgfsetstrokecolor{currentstroke}%
\pgfsetdash{}{0pt}%
\pgfpathmoveto{\pgfqpoint{0.000000in}{0.000000in}}%
\pgfpathlineto{\pgfqpoint{5.208662in}{0.000000in}}%
\pgfpathlineto{\pgfqpoint{5.208662in}{3.219130in}}%
\pgfpathlineto{\pgfqpoint{0.000000in}{3.219130in}}%
\pgfpathlineto{\pgfqpoint{0.000000in}{0.000000in}}%
\pgfpathclose%
\pgfusepath{fill}%
\end{pgfscope}%
\begin{pgfscope}%
\pgfsetbuttcap%
\pgfsetmiterjoin%
\definecolor{currentfill}{rgb}{1.000000,1.000000,1.000000}%
\pgfsetfillcolor{currentfill}%
\pgfsetlinewidth{0.000000pt}%
\definecolor{currentstroke}{rgb}{0.000000,0.000000,0.000000}%
\pgfsetstrokecolor{currentstroke}%
\pgfsetstrokeopacity{0.000000}%
\pgfsetdash{}{0pt}%
\pgfpathmoveto{\pgfqpoint{0.693677in}{0.539544in}}%
\pgfpathlineto{\pgfqpoint{5.058662in}{0.539544in}}%
\pgfpathlineto{\pgfqpoint{5.058662in}{3.053228in}}%
\pgfpathlineto{\pgfqpoint{0.693677in}{3.053228in}}%
\pgfpathlineto{\pgfqpoint{0.693677in}{0.539544in}}%
\pgfpathclose%
\pgfusepath{fill}%
\end{pgfscope}%
\begin{pgfscope}%
\pgfsetbuttcap%
\pgfsetroundjoin%
\definecolor{currentfill}{rgb}{0.000000,0.000000,0.000000}%
\pgfsetfillcolor{currentfill}%
\pgfsetlinewidth{0.803000pt}%
\definecolor{currentstroke}{rgb}{0.000000,0.000000,0.000000}%
\pgfsetstrokecolor{currentstroke}%
\pgfsetdash{}{0pt}%
\pgfsys@defobject{currentmarker}{\pgfqpoint{0.000000in}{-0.048611in}}{\pgfqpoint{0.000000in}{0.000000in}}{%
\pgfpathmoveto{\pgfqpoint{0.000000in}{0.000000in}}%
\pgfpathlineto{\pgfqpoint{0.000000in}{-0.048611in}}%
\pgfusepath{stroke,fill}%
}%
\begin{pgfscope}%
\pgfsys@transformshift{0.892085in}{0.539544in}%
\pgfsys@useobject{currentmarker}{}%
\end{pgfscope}%
\end{pgfscope}%
\begin{pgfscope}%
\definecolor{textcolor}{rgb}{0.000000,0.000000,0.000000}%
\pgfsetstrokecolor{textcolor}%
\pgfsetfillcolor{textcolor}%
\pgftext[x=0.892085in,y=0.442322in,,top]{\color{textcolor}\rmfamily\fontsize{8.000000}{9.600000}\selectfont \(\displaystyle {00{:}00}\)}%
\end{pgfscope}%
\begin{pgfscope}%
\pgfsetbuttcap%
\pgfsetroundjoin%
\definecolor{currentfill}{rgb}{0.000000,0.000000,0.000000}%
\pgfsetfillcolor{currentfill}%
\pgfsetlinewidth{0.803000pt}%
\definecolor{currentstroke}{rgb}{0.000000,0.000000,0.000000}%
\pgfsetstrokecolor{currentstroke}%
\pgfsetdash{}{0pt}%
\pgfsys@defobject{currentmarker}{\pgfqpoint{0.000000in}{-0.048611in}}{\pgfqpoint{0.000000in}{0.000000in}}{%
\pgfpathmoveto{\pgfqpoint{0.000000in}{0.000000in}}%
\pgfpathlineto{\pgfqpoint{0.000000in}{-0.048611in}}%
\pgfusepath{stroke,fill}%
}%
\begin{pgfscope}%
\pgfsys@transformshift{1.388451in}{0.539544in}%
\pgfsys@useobject{currentmarker}{}%
\end{pgfscope}%
\end{pgfscope}%
\begin{pgfscope}%
\definecolor{textcolor}{rgb}{0.000000,0.000000,0.000000}%
\pgfsetstrokecolor{textcolor}%
\pgfsetfillcolor{textcolor}%
\pgftext[x=1.388451in,y=0.442322in,,top]{\color{textcolor}\rmfamily\fontsize{8.000000}{9.600000}\selectfont \(\displaystyle {03{:}00}\)}%
\end{pgfscope}%
\begin{pgfscope}%
\pgfsetbuttcap%
\pgfsetroundjoin%
\definecolor{currentfill}{rgb}{0.000000,0.000000,0.000000}%
\pgfsetfillcolor{currentfill}%
\pgfsetlinewidth{0.803000pt}%
\definecolor{currentstroke}{rgb}{0.000000,0.000000,0.000000}%
\pgfsetstrokecolor{currentstroke}%
\pgfsetdash{}{0pt}%
\pgfsys@defobject{currentmarker}{\pgfqpoint{0.000000in}{-0.048611in}}{\pgfqpoint{0.000000in}{0.000000in}}{%
\pgfpathmoveto{\pgfqpoint{0.000000in}{0.000000in}}%
\pgfpathlineto{\pgfqpoint{0.000000in}{-0.048611in}}%
\pgfusepath{stroke,fill}%
}%
\begin{pgfscope}%
\pgfsys@transformshift{1.884817in}{0.539544in}%
\pgfsys@useobject{currentmarker}{}%
\end{pgfscope}%
\end{pgfscope}%
\begin{pgfscope}%
\definecolor{textcolor}{rgb}{0.000000,0.000000,0.000000}%
\pgfsetstrokecolor{textcolor}%
\pgfsetfillcolor{textcolor}%
\pgftext[x=1.884817in,y=0.442322in,,top]{\color{textcolor}\rmfamily\fontsize{8.000000}{9.600000}\selectfont \(\displaystyle {06{:}00}\)}%
\end{pgfscope}%
\begin{pgfscope}%
\pgfsetbuttcap%
\pgfsetroundjoin%
\definecolor{currentfill}{rgb}{0.000000,0.000000,0.000000}%
\pgfsetfillcolor{currentfill}%
\pgfsetlinewidth{0.803000pt}%
\definecolor{currentstroke}{rgb}{0.000000,0.000000,0.000000}%
\pgfsetstrokecolor{currentstroke}%
\pgfsetdash{}{0pt}%
\pgfsys@defobject{currentmarker}{\pgfqpoint{0.000000in}{-0.048611in}}{\pgfqpoint{0.000000in}{0.000000in}}{%
\pgfpathmoveto{\pgfqpoint{0.000000in}{0.000000in}}%
\pgfpathlineto{\pgfqpoint{0.000000in}{-0.048611in}}%
\pgfusepath{stroke,fill}%
}%
\begin{pgfscope}%
\pgfsys@transformshift{2.381182in}{0.539544in}%
\pgfsys@useobject{currentmarker}{}%
\end{pgfscope}%
\end{pgfscope}%
\begin{pgfscope}%
\definecolor{textcolor}{rgb}{0.000000,0.000000,0.000000}%
\pgfsetstrokecolor{textcolor}%
\pgfsetfillcolor{textcolor}%
\pgftext[x=2.381182in,y=0.442322in,,top]{\color{textcolor}\rmfamily\fontsize{8.000000}{9.600000}\selectfont \(\displaystyle {09{:}00}\)}%
\end{pgfscope}%
\begin{pgfscope}%
\pgfsetbuttcap%
\pgfsetroundjoin%
\definecolor{currentfill}{rgb}{0.000000,0.000000,0.000000}%
\pgfsetfillcolor{currentfill}%
\pgfsetlinewidth{0.803000pt}%
\definecolor{currentstroke}{rgb}{0.000000,0.000000,0.000000}%
\pgfsetstrokecolor{currentstroke}%
\pgfsetdash{}{0pt}%
\pgfsys@defobject{currentmarker}{\pgfqpoint{0.000000in}{-0.048611in}}{\pgfqpoint{0.000000in}{0.000000in}}{%
\pgfpathmoveto{\pgfqpoint{0.000000in}{0.000000in}}%
\pgfpathlineto{\pgfqpoint{0.000000in}{-0.048611in}}%
\pgfusepath{stroke,fill}%
}%
\begin{pgfscope}%
\pgfsys@transformshift{2.877548in}{0.539544in}%
\pgfsys@useobject{currentmarker}{}%
\end{pgfscope}%
\end{pgfscope}%
\begin{pgfscope}%
\definecolor{textcolor}{rgb}{0.000000,0.000000,0.000000}%
\pgfsetstrokecolor{textcolor}%
\pgfsetfillcolor{textcolor}%
\pgftext[x=2.877548in,y=0.442322in,,top]{\color{textcolor}\rmfamily\fontsize{8.000000}{9.600000}\selectfont \(\displaystyle {12{:}00}\)}%
\end{pgfscope}%
\begin{pgfscope}%
\pgfsetbuttcap%
\pgfsetroundjoin%
\definecolor{currentfill}{rgb}{0.000000,0.000000,0.000000}%
\pgfsetfillcolor{currentfill}%
\pgfsetlinewidth{0.803000pt}%
\definecolor{currentstroke}{rgb}{0.000000,0.000000,0.000000}%
\pgfsetstrokecolor{currentstroke}%
\pgfsetdash{}{0pt}%
\pgfsys@defobject{currentmarker}{\pgfqpoint{0.000000in}{-0.048611in}}{\pgfqpoint{0.000000in}{0.000000in}}{%
\pgfpathmoveto{\pgfqpoint{0.000000in}{0.000000in}}%
\pgfpathlineto{\pgfqpoint{0.000000in}{-0.048611in}}%
\pgfusepath{stroke,fill}%
}%
\begin{pgfscope}%
\pgfsys@transformshift{3.373914in}{0.539544in}%
\pgfsys@useobject{currentmarker}{}%
\end{pgfscope}%
\end{pgfscope}%
\begin{pgfscope}%
\definecolor{textcolor}{rgb}{0.000000,0.000000,0.000000}%
\pgfsetstrokecolor{textcolor}%
\pgfsetfillcolor{textcolor}%
\pgftext[x=3.373914in,y=0.442322in,,top]{\color{textcolor}\rmfamily\fontsize{8.000000}{9.600000}\selectfont \(\displaystyle {15{:}00}\)}%
\end{pgfscope}%
\begin{pgfscope}%
\pgfsetbuttcap%
\pgfsetroundjoin%
\definecolor{currentfill}{rgb}{0.000000,0.000000,0.000000}%
\pgfsetfillcolor{currentfill}%
\pgfsetlinewidth{0.803000pt}%
\definecolor{currentstroke}{rgb}{0.000000,0.000000,0.000000}%
\pgfsetstrokecolor{currentstroke}%
\pgfsetdash{}{0pt}%
\pgfsys@defobject{currentmarker}{\pgfqpoint{0.000000in}{-0.048611in}}{\pgfqpoint{0.000000in}{0.000000in}}{%
\pgfpathmoveto{\pgfqpoint{0.000000in}{0.000000in}}%
\pgfpathlineto{\pgfqpoint{0.000000in}{-0.048611in}}%
\pgfusepath{stroke,fill}%
}%
\begin{pgfscope}%
\pgfsys@transformshift{3.870280in}{0.539544in}%
\pgfsys@useobject{currentmarker}{}%
\end{pgfscope}%
\end{pgfscope}%
\begin{pgfscope}%
\definecolor{textcolor}{rgb}{0.000000,0.000000,0.000000}%
\pgfsetstrokecolor{textcolor}%
\pgfsetfillcolor{textcolor}%
\pgftext[x=3.870280in,y=0.442322in,,top]{\color{textcolor}\rmfamily\fontsize{8.000000}{9.600000}\selectfont \(\displaystyle {18{:}00}\)}%
\end{pgfscope}%
\begin{pgfscope}%
\pgfsetbuttcap%
\pgfsetroundjoin%
\definecolor{currentfill}{rgb}{0.000000,0.000000,0.000000}%
\pgfsetfillcolor{currentfill}%
\pgfsetlinewidth{0.803000pt}%
\definecolor{currentstroke}{rgb}{0.000000,0.000000,0.000000}%
\pgfsetstrokecolor{currentstroke}%
\pgfsetdash{}{0pt}%
\pgfsys@defobject{currentmarker}{\pgfqpoint{0.000000in}{-0.048611in}}{\pgfqpoint{0.000000in}{0.000000in}}{%
\pgfpathmoveto{\pgfqpoint{0.000000in}{0.000000in}}%
\pgfpathlineto{\pgfqpoint{0.000000in}{-0.048611in}}%
\pgfusepath{stroke,fill}%
}%
\begin{pgfscope}%
\pgfsys@transformshift{4.366645in}{0.539544in}%
\pgfsys@useobject{currentmarker}{}%
\end{pgfscope}%
\end{pgfscope}%
\begin{pgfscope}%
\definecolor{textcolor}{rgb}{0.000000,0.000000,0.000000}%
\pgfsetstrokecolor{textcolor}%
\pgfsetfillcolor{textcolor}%
\pgftext[x=4.366645in,y=0.442322in,,top]{\color{textcolor}\rmfamily\fontsize{8.000000}{9.600000}\selectfont \(\displaystyle {21{:}00}\)}%
\end{pgfscope}%
\begin{pgfscope}%
\pgfsetbuttcap%
\pgfsetroundjoin%
\definecolor{currentfill}{rgb}{0.000000,0.000000,0.000000}%
\pgfsetfillcolor{currentfill}%
\pgfsetlinewidth{0.803000pt}%
\definecolor{currentstroke}{rgb}{0.000000,0.000000,0.000000}%
\pgfsetstrokecolor{currentstroke}%
\pgfsetdash{}{0pt}%
\pgfsys@defobject{currentmarker}{\pgfqpoint{0.000000in}{-0.048611in}}{\pgfqpoint{0.000000in}{0.000000in}}{%
\pgfpathmoveto{\pgfqpoint{0.000000in}{0.000000in}}%
\pgfpathlineto{\pgfqpoint{0.000000in}{-0.048611in}}%
\pgfusepath{stroke,fill}%
}%
\begin{pgfscope}%
\pgfsys@transformshift{4.863011in}{0.539544in}%
\pgfsys@useobject{currentmarker}{}%
\end{pgfscope}%
\end{pgfscope}%
\begin{pgfscope}%
\definecolor{textcolor}{rgb}{0.000000,0.000000,0.000000}%
\pgfsetstrokecolor{textcolor}%
\pgfsetfillcolor{textcolor}%
\pgftext[x=4.863011in,y=0.442322in,,top]{\color{textcolor}\rmfamily\fontsize{8.000000}{9.600000}\selectfont \(\displaystyle {00{:}00}\)}%
\end{pgfscope}%
\begin{pgfscope}%
\definecolor{textcolor}{rgb}{0.000000,0.000000,0.000000}%
\pgfsetstrokecolor{textcolor}%
\pgfsetfillcolor{textcolor}%
\pgftext[x=2.876169in,y=0.288100in,,top]{\color{textcolor}\rmfamily\fontsize{10.000000}{12.000000}\selectfont Time (UTC)}%
\end{pgfscope}%
\begin{pgfscope}%
\pgfsetbuttcap%
\pgfsetroundjoin%
\definecolor{currentfill}{rgb}{0.000000,0.000000,0.000000}%
\pgfsetfillcolor{currentfill}%
\pgfsetlinewidth{0.803000pt}%
\definecolor{currentstroke}{rgb}{0.000000,0.000000,0.000000}%
\pgfsetstrokecolor{currentstroke}%
\pgfsetdash{}{0pt}%
\pgfsys@defobject{currentmarker}{\pgfqpoint{-0.048611in}{0.000000in}}{\pgfqpoint{-0.000000in}{0.000000in}}{%
\pgfpathmoveto{\pgfqpoint{-0.000000in}{0.000000in}}%
\pgfpathlineto{\pgfqpoint{-0.048611in}{0.000000in}}%
\pgfusepath{stroke,fill}%
}%
\begin{pgfscope}%
\pgfsys@transformshift{0.693677in}{0.745209in}%
\pgfsys@useobject{currentmarker}{}%
\end{pgfscope}%
\end{pgfscope}%
\begin{pgfscope}%
\definecolor{textcolor}{rgb}{0.000000,0.000000,0.000000}%
\pgfsetstrokecolor{textcolor}%
\pgfsetfillcolor{textcolor}%
\pgftext[x=0.327546in, y=0.706654in, left, base]{\color{textcolor}\rmfamily\fontsize{8.000000}{9.600000}\selectfont \(\displaystyle {20.00}\)}%
\end{pgfscope}%
\begin{pgfscope}%
\pgfsetbuttcap%
\pgfsetroundjoin%
\definecolor{currentfill}{rgb}{0.000000,0.000000,0.000000}%
\pgfsetfillcolor{currentfill}%
\pgfsetlinewidth{0.803000pt}%
\definecolor{currentstroke}{rgb}{0.000000,0.000000,0.000000}%
\pgfsetstrokecolor{currentstroke}%
\pgfsetdash{}{0pt}%
\pgfsys@defobject{currentmarker}{\pgfqpoint{-0.048611in}{0.000000in}}{\pgfqpoint{-0.000000in}{0.000000in}}{%
\pgfpathmoveto{\pgfqpoint{-0.000000in}{0.000000in}}%
\pgfpathlineto{\pgfqpoint{-0.048611in}{0.000000in}}%
\pgfusepath{stroke,fill}%
}%
\begin{pgfscope}%
\pgfsys@transformshift{0.693677in}{1.071662in}%
\pgfsys@useobject{currentmarker}{}%
\end{pgfscope}%
\end{pgfscope}%
\begin{pgfscope}%
\definecolor{textcolor}{rgb}{0.000000,0.000000,0.000000}%
\pgfsetstrokecolor{textcolor}%
\pgfsetfillcolor{textcolor}%
\pgftext[x=0.327546in, y=1.033106in, left, base]{\color{textcolor}\rmfamily\fontsize{8.000000}{9.600000}\selectfont \(\displaystyle {20.25}\)}%
\end{pgfscope}%
\begin{pgfscope}%
\pgfsetbuttcap%
\pgfsetroundjoin%
\definecolor{currentfill}{rgb}{0.000000,0.000000,0.000000}%
\pgfsetfillcolor{currentfill}%
\pgfsetlinewidth{0.803000pt}%
\definecolor{currentstroke}{rgb}{0.000000,0.000000,0.000000}%
\pgfsetstrokecolor{currentstroke}%
\pgfsetdash{}{0pt}%
\pgfsys@defobject{currentmarker}{\pgfqpoint{-0.048611in}{0.000000in}}{\pgfqpoint{-0.000000in}{0.000000in}}{%
\pgfpathmoveto{\pgfqpoint{-0.000000in}{0.000000in}}%
\pgfpathlineto{\pgfqpoint{-0.048611in}{0.000000in}}%
\pgfusepath{stroke,fill}%
}%
\begin{pgfscope}%
\pgfsys@transformshift{0.693677in}{1.398114in}%
\pgfsys@useobject{currentmarker}{}%
\end{pgfscope}%
\end{pgfscope}%
\begin{pgfscope}%
\definecolor{textcolor}{rgb}{0.000000,0.000000,0.000000}%
\pgfsetstrokecolor{textcolor}%
\pgfsetfillcolor{textcolor}%
\pgftext[x=0.327546in, y=1.359559in, left, base]{\color{textcolor}\rmfamily\fontsize{8.000000}{9.600000}\selectfont \(\displaystyle {20.50}\)}%
\end{pgfscope}%
\begin{pgfscope}%
\pgfsetbuttcap%
\pgfsetroundjoin%
\definecolor{currentfill}{rgb}{0.000000,0.000000,0.000000}%
\pgfsetfillcolor{currentfill}%
\pgfsetlinewidth{0.803000pt}%
\definecolor{currentstroke}{rgb}{0.000000,0.000000,0.000000}%
\pgfsetstrokecolor{currentstroke}%
\pgfsetdash{}{0pt}%
\pgfsys@defobject{currentmarker}{\pgfqpoint{-0.048611in}{0.000000in}}{\pgfqpoint{-0.000000in}{0.000000in}}{%
\pgfpathmoveto{\pgfqpoint{-0.000000in}{0.000000in}}%
\pgfpathlineto{\pgfqpoint{-0.048611in}{0.000000in}}%
\pgfusepath{stroke,fill}%
}%
\begin{pgfscope}%
\pgfsys@transformshift{0.693677in}{1.724567in}%
\pgfsys@useobject{currentmarker}{}%
\end{pgfscope}%
\end{pgfscope}%
\begin{pgfscope}%
\definecolor{textcolor}{rgb}{0.000000,0.000000,0.000000}%
\pgfsetstrokecolor{textcolor}%
\pgfsetfillcolor{textcolor}%
\pgftext[x=0.327546in, y=1.686011in, left, base]{\color{textcolor}\rmfamily\fontsize{8.000000}{9.600000}\selectfont \(\displaystyle {20.75}\)}%
\end{pgfscope}%
\begin{pgfscope}%
\pgfsetbuttcap%
\pgfsetroundjoin%
\definecolor{currentfill}{rgb}{0.000000,0.000000,0.000000}%
\pgfsetfillcolor{currentfill}%
\pgfsetlinewidth{0.803000pt}%
\definecolor{currentstroke}{rgb}{0.000000,0.000000,0.000000}%
\pgfsetstrokecolor{currentstroke}%
\pgfsetdash{}{0pt}%
\pgfsys@defobject{currentmarker}{\pgfqpoint{-0.048611in}{0.000000in}}{\pgfqpoint{-0.000000in}{0.000000in}}{%
\pgfpathmoveto{\pgfqpoint{-0.000000in}{0.000000in}}%
\pgfpathlineto{\pgfqpoint{-0.048611in}{0.000000in}}%
\pgfusepath{stroke,fill}%
}%
\begin{pgfscope}%
\pgfsys@transformshift{0.693677in}{2.051019in}%
\pgfsys@useobject{currentmarker}{}%
\end{pgfscope}%
\end{pgfscope}%
\begin{pgfscope}%
\definecolor{textcolor}{rgb}{0.000000,0.000000,0.000000}%
\pgfsetstrokecolor{textcolor}%
\pgfsetfillcolor{textcolor}%
\pgftext[x=0.327546in, y=2.012463in, left, base]{\color{textcolor}\rmfamily\fontsize{8.000000}{9.600000}\selectfont \(\displaystyle {21.00}\)}%
\end{pgfscope}%
\begin{pgfscope}%
\pgfsetbuttcap%
\pgfsetroundjoin%
\definecolor{currentfill}{rgb}{0.000000,0.000000,0.000000}%
\pgfsetfillcolor{currentfill}%
\pgfsetlinewidth{0.803000pt}%
\definecolor{currentstroke}{rgb}{0.000000,0.000000,0.000000}%
\pgfsetstrokecolor{currentstroke}%
\pgfsetdash{}{0pt}%
\pgfsys@defobject{currentmarker}{\pgfqpoint{-0.048611in}{0.000000in}}{\pgfqpoint{-0.000000in}{0.000000in}}{%
\pgfpathmoveto{\pgfqpoint{-0.000000in}{0.000000in}}%
\pgfpathlineto{\pgfqpoint{-0.048611in}{0.000000in}}%
\pgfusepath{stroke,fill}%
}%
\begin{pgfscope}%
\pgfsys@transformshift{0.693677in}{2.377471in}%
\pgfsys@useobject{currentmarker}{}%
\end{pgfscope}%
\end{pgfscope}%
\begin{pgfscope}%
\definecolor{textcolor}{rgb}{0.000000,0.000000,0.000000}%
\pgfsetstrokecolor{textcolor}%
\pgfsetfillcolor{textcolor}%
\pgftext[x=0.327546in, y=2.338916in, left, base]{\color{textcolor}\rmfamily\fontsize{8.000000}{9.600000}\selectfont \(\displaystyle {21.25}\)}%
\end{pgfscope}%
\begin{pgfscope}%
\pgfsetbuttcap%
\pgfsetroundjoin%
\definecolor{currentfill}{rgb}{0.000000,0.000000,0.000000}%
\pgfsetfillcolor{currentfill}%
\pgfsetlinewidth{0.803000pt}%
\definecolor{currentstroke}{rgb}{0.000000,0.000000,0.000000}%
\pgfsetstrokecolor{currentstroke}%
\pgfsetdash{}{0pt}%
\pgfsys@defobject{currentmarker}{\pgfqpoint{-0.048611in}{0.000000in}}{\pgfqpoint{-0.000000in}{0.000000in}}{%
\pgfpathmoveto{\pgfqpoint{-0.000000in}{0.000000in}}%
\pgfpathlineto{\pgfqpoint{-0.048611in}{0.000000in}}%
\pgfusepath{stroke,fill}%
}%
\begin{pgfscope}%
\pgfsys@transformshift{0.693677in}{2.703924in}%
\pgfsys@useobject{currentmarker}{}%
\end{pgfscope}%
\end{pgfscope}%
\begin{pgfscope}%
\definecolor{textcolor}{rgb}{0.000000,0.000000,0.000000}%
\pgfsetstrokecolor{textcolor}%
\pgfsetfillcolor{textcolor}%
\pgftext[x=0.327546in, y=2.665368in, left, base]{\color{textcolor}\rmfamily\fontsize{8.000000}{9.600000}\selectfont \(\displaystyle {21.50}\)}%
\end{pgfscope}%
\begin{pgfscope}%
\pgfsetbuttcap%
\pgfsetroundjoin%
\definecolor{currentfill}{rgb}{0.000000,0.000000,0.000000}%
\pgfsetfillcolor{currentfill}%
\pgfsetlinewidth{0.803000pt}%
\definecolor{currentstroke}{rgb}{0.000000,0.000000,0.000000}%
\pgfsetstrokecolor{currentstroke}%
\pgfsetdash{}{0pt}%
\pgfsys@defobject{currentmarker}{\pgfqpoint{-0.048611in}{0.000000in}}{\pgfqpoint{-0.000000in}{0.000000in}}{%
\pgfpathmoveto{\pgfqpoint{-0.000000in}{0.000000in}}%
\pgfpathlineto{\pgfqpoint{-0.048611in}{0.000000in}}%
\pgfusepath{stroke,fill}%
}%
\begin{pgfscope}%
\pgfsys@transformshift{0.693677in}{3.030376in}%
\pgfsys@useobject{currentmarker}{}%
\end{pgfscope}%
\end{pgfscope}%
\begin{pgfscope}%
\definecolor{textcolor}{rgb}{0.000000,0.000000,0.000000}%
\pgfsetstrokecolor{textcolor}%
\pgfsetfillcolor{textcolor}%
\pgftext[x=0.327546in, y=2.991821in, left, base]{\color{textcolor}\rmfamily\fontsize{8.000000}{9.600000}\selectfont \(\displaystyle {21.75}\)}%
\end{pgfscope}%
\begin{pgfscope}%
\definecolor{textcolor}{rgb}{0.000000,0.000000,0.000000}%
\pgfsetstrokecolor{textcolor}%
\pgfsetfillcolor{textcolor}%
\pgftext[x=0.271991in,y=1.796386in,,bottom,rotate=90.000000]{\color{textcolor}\rmfamily\fontsize{10.000000}{12.000000}\selectfont Temperature in \unit{\celsius}}%
\end{pgfscope}%
\begin{pgfscope}%
\pgfpathrectangle{\pgfqpoint{0.693677in}{0.539544in}}{\pgfqpoint{4.364985in}{2.513684in}}%
\pgfusepath{clip}%
\pgfsetrectcap%
\pgfsetroundjoin%
\pgfsetlinewidth{0.501875pt}%
\definecolor{currentstroke}{rgb}{0.121569,0.466667,0.705882}%
\pgfsetstrokecolor{currentstroke}%
\pgfsetstrokeopacity{0.700000}%
\pgfsetdash{}{0pt}%
\pgfpathmoveto{\pgfqpoint{0.892085in}{2.612517in}}%
\pgfpathlineto{\pgfqpoint{0.894843in}{2.703924in}}%
\pgfpathlineto{\pgfqpoint{0.930691in}{2.782272in}}%
\pgfpathlineto{\pgfqpoint{0.933449in}{2.703924in}}%
\pgfpathlineto{\pgfqpoint{0.936207in}{2.782272in}}%
\pgfpathlineto{\pgfqpoint{0.938964in}{2.703924in}}%
\pgfpathlineto{\pgfqpoint{0.941722in}{2.782272in}}%
\pgfpathlineto{\pgfqpoint{0.944479in}{2.703924in}}%
\pgfpathlineto{\pgfqpoint{0.947237in}{2.782272in}}%
\pgfpathlineto{\pgfqpoint{0.949995in}{2.703924in}}%
\pgfpathlineto{\pgfqpoint{0.952752in}{2.782272in}}%
\pgfpathlineto{\pgfqpoint{0.955510in}{2.703924in}}%
\pgfpathlineto{\pgfqpoint{0.958267in}{2.782272in}}%
\pgfpathlineto{\pgfqpoint{0.961025in}{2.703924in}}%
\pgfpathlineto{\pgfqpoint{0.963783in}{2.782272in}}%
\pgfpathlineto{\pgfqpoint{0.966540in}{2.703924in}}%
\pgfpathlineto{\pgfqpoint{0.969298in}{2.782272in}}%
\pgfpathlineto{\pgfqpoint{0.972055in}{2.703924in}}%
\pgfpathlineto{\pgfqpoint{0.974813in}{2.782272in}}%
\pgfpathlineto{\pgfqpoint{0.977570in}{2.703924in}}%
\pgfpathlineto{\pgfqpoint{0.980328in}{2.782272in}}%
\pgfpathlineto{\pgfqpoint{0.985843in}{2.703924in}}%
\pgfpathlineto{\pgfqpoint{0.988601in}{2.782272in}}%
\pgfpathlineto{\pgfqpoint{1.032722in}{2.860621in}}%
\pgfpathlineto{\pgfqpoint{1.035480in}{2.782272in}}%
\pgfpathlineto{\pgfqpoint{1.040995in}{2.860621in}}%
\pgfpathlineto{\pgfqpoint{1.043753in}{2.782272in}}%
\pgfpathlineto{\pgfqpoint{1.046510in}{2.860621in}}%
\pgfpathlineto{\pgfqpoint{1.049268in}{2.782272in}}%
\pgfpathlineto{\pgfqpoint{1.054783in}{2.860621in}}%
\pgfpathlineto{\pgfqpoint{1.057540in}{2.782272in}}%
\pgfpathlineto{\pgfqpoint{1.060298in}{2.860621in}}%
\pgfpathlineto{\pgfqpoint{1.063056in}{2.782272in}}%
\pgfpathlineto{\pgfqpoint{1.065813in}{2.860621in}}%
\pgfpathlineto{\pgfqpoint{1.068571in}{2.782272in}}%
\pgfpathlineto{\pgfqpoint{1.071328in}{2.860621in}}%
\pgfpathlineto{\pgfqpoint{1.074086in}{2.782272in}}%
\pgfpathlineto{\pgfqpoint{1.076844in}{2.860621in}}%
\pgfpathlineto{\pgfqpoint{1.079601in}{2.782272in}}%
\pgfpathlineto{\pgfqpoint{1.082359in}{2.860621in}}%
\pgfpathlineto{\pgfqpoint{1.085116in}{2.782272in}}%
\pgfpathlineto{\pgfqpoint{1.087874in}{2.860621in}}%
\pgfpathlineto{\pgfqpoint{1.090632in}{2.782272in}}%
\pgfpathlineto{\pgfqpoint{1.093389in}{2.860621in}}%
\pgfpathlineto{\pgfqpoint{1.096147in}{2.782272in}}%
\pgfpathlineto{\pgfqpoint{1.098904in}{2.860621in}}%
\pgfpathlineto{\pgfqpoint{1.109935in}{2.782272in}}%
\pgfpathlineto{\pgfqpoint{1.112692in}{2.860621in}}%
\pgfpathlineto{\pgfqpoint{1.123723in}{2.782272in}}%
\pgfpathlineto{\pgfqpoint{1.126480in}{2.860621in}}%
\pgfpathlineto{\pgfqpoint{1.178874in}{2.938970in}}%
\pgfpathlineto{\pgfqpoint{1.181632in}{2.860621in}}%
\pgfpathlineto{\pgfqpoint{1.184390in}{2.938970in}}%
\pgfpathlineto{\pgfqpoint{1.187147in}{2.860621in}}%
\pgfpathlineto{\pgfqpoint{1.189905in}{2.938970in}}%
\pgfpathlineto{\pgfqpoint{1.220238in}{2.051019in}}%
\pgfpathlineto{\pgfqpoint{1.225753in}{1.881264in}}%
\pgfpathlineto{\pgfqpoint{1.231268in}{1.802915in}}%
\pgfpathlineto{\pgfqpoint{1.234026in}{1.724567in}}%
\pgfpathlineto{\pgfqpoint{1.239541in}{1.633160in}}%
\pgfpathlineto{\pgfqpoint{1.242299in}{1.554811in}}%
\pgfpathlineto{\pgfqpoint{1.250572in}{1.476463in}}%
\pgfpathlineto{\pgfqpoint{1.256087in}{1.398114in}}%
\pgfpathlineto{\pgfqpoint{1.258844in}{1.476463in}}%
\pgfpathlineto{\pgfqpoint{1.264360in}{1.306707in}}%
\pgfpathlineto{\pgfqpoint{1.272632in}{1.228359in}}%
\pgfpathlineto{\pgfqpoint{1.275390in}{1.306707in}}%
\pgfpathlineto{\pgfqpoint{1.280905in}{1.150010in}}%
\pgfpathlineto{\pgfqpoint{1.283663in}{1.228359in}}%
\pgfpathlineto{\pgfqpoint{1.286420in}{1.150010in}}%
\pgfpathlineto{\pgfqpoint{1.291935in}{1.071662in}}%
\pgfpathlineto{\pgfqpoint{1.294693in}{1.150010in}}%
\pgfpathlineto{\pgfqpoint{1.297451in}{1.071662in}}%
\pgfpathlineto{\pgfqpoint{1.300208in}{1.150010in}}%
\pgfpathlineto{\pgfqpoint{1.302966in}{1.071662in}}%
\pgfpathlineto{\pgfqpoint{1.313996in}{0.980255in}}%
\pgfpathlineto{\pgfqpoint{1.316754in}{1.071662in}}%
\pgfpathlineto{\pgfqpoint{1.319511in}{0.980255in}}%
\pgfpathlineto{\pgfqpoint{1.322269in}{1.071662in}}%
\pgfpathlineto{\pgfqpoint{1.325026in}{0.980255in}}%
\pgfpathlineto{\pgfqpoint{1.330542in}{0.901906in}}%
\pgfpathlineto{\pgfqpoint{1.333299in}{0.980255in}}%
\pgfpathlineto{\pgfqpoint{1.336057in}{0.901906in}}%
\pgfpathlineto{\pgfqpoint{1.347087in}{0.823558in}}%
\pgfpathlineto{\pgfqpoint{1.349845in}{0.901906in}}%
\pgfpathlineto{\pgfqpoint{1.352602in}{0.823558in}}%
\pgfpathlineto{\pgfqpoint{1.355360in}{0.901906in}}%
\pgfpathlineto{\pgfqpoint{1.358118in}{0.823558in}}%
\pgfpathlineto{\pgfqpoint{1.369148in}{0.745209in}}%
\pgfpathlineto{\pgfqpoint{1.371905in}{0.823558in}}%
\pgfpathlineto{\pgfqpoint{1.374663in}{0.745209in}}%
\pgfpathlineto{\pgfqpoint{1.377421in}{0.823558in}}%
\pgfpathlineto{\pgfqpoint{1.382936in}{0.901906in}}%
\pgfpathlineto{\pgfqpoint{1.388451in}{1.071662in}}%
\pgfpathlineto{\pgfqpoint{1.393966in}{1.150010in}}%
\pgfpathlineto{\pgfqpoint{1.396724in}{1.228359in}}%
\pgfpathlineto{\pgfqpoint{1.402239in}{1.306707in}}%
\pgfpathlineto{\pgfqpoint{1.404996in}{1.398114in}}%
\pgfpathlineto{\pgfqpoint{1.432572in}{1.802915in}}%
\pgfpathlineto{\pgfqpoint{1.438088in}{1.724567in}}%
\pgfpathlineto{\pgfqpoint{1.443603in}{1.881264in}}%
\pgfpathlineto{\pgfqpoint{1.449118in}{1.959612in}}%
\pgfpathlineto{\pgfqpoint{1.451875in}{1.881264in}}%
\pgfpathlineto{\pgfqpoint{1.454633in}{1.959612in}}%
\pgfpathlineto{\pgfqpoint{1.460148in}{2.051019in}}%
\pgfpathlineto{\pgfqpoint{1.462906in}{1.959612in}}%
\pgfpathlineto{\pgfqpoint{1.468421in}{2.129368in}}%
\pgfpathlineto{\pgfqpoint{1.471179in}{2.051019in}}%
\pgfpathlineto{\pgfqpoint{1.473936in}{2.129368in}}%
\pgfpathlineto{\pgfqpoint{1.482209in}{2.207716in}}%
\pgfpathlineto{\pgfqpoint{1.484967in}{2.129368in}}%
\pgfpathlineto{\pgfqpoint{1.487724in}{2.207716in}}%
\pgfpathlineto{\pgfqpoint{1.490482in}{2.129368in}}%
\pgfpathlineto{\pgfqpoint{1.493239in}{2.207716in}}%
\pgfpathlineto{\pgfqpoint{1.501512in}{2.286065in}}%
\pgfpathlineto{\pgfqpoint{1.504270in}{2.207716in}}%
\pgfpathlineto{\pgfqpoint{1.507027in}{2.286065in}}%
\pgfpathlineto{\pgfqpoint{1.509785in}{2.207716in}}%
\pgfpathlineto{\pgfqpoint{1.512542in}{2.286065in}}%
\pgfpathlineto{\pgfqpoint{1.520815in}{2.377471in}}%
\pgfpathlineto{\pgfqpoint{1.523573in}{2.286065in}}%
\pgfpathlineto{\pgfqpoint{1.526330in}{2.377471in}}%
\pgfpathlineto{\pgfqpoint{1.529088in}{2.286065in}}%
\pgfpathlineto{\pgfqpoint{1.531846in}{2.377471in}}%
\pgfpathlineto{\pgfqpoint{1.545633in}{2.455820in}}%
\pgfpathlineto{\pgfqpoint{1.548391in}{2.377471in}}%
\pgfpathlineto{\pgfqpoint{1.551149in}{2.455820in}}%
\pgfpathlineto{\pgfqpoint{1.553906in}{2.377471in}}%
\pgfpathlineto{\pgfqpoint{1.556664in}{2.455820in}}%
\pgfpathlineto{\pgfqpoint{1.562179in}{2.377471in}}%
\pgfpathlineto{\pgfqpoint{1.564937in}{2.455820in}}%
\pgfpathlineto{\pgfqpoint{1.581482in}{2.534169in}}%
\pgfpathlineto{\pgfqpoint{1.584240in}{2.455820in}}%
\pgfpathlineto{\pgfqpoint{1.586997in}{2.534169in}}%
\pgfpathlineto{\pgfqpoint{1.589755in}{2.455820in}}%
\pgfpathlineto{\pgfqpoint{1.592512in}{2.534169in}}%
\pgfpathlineto{\pgfqpoint{1.595270in}{2.455820in}}%
\pgfpathlineto{\pgfqpoint{1.598028in}{2.534169in}}%
\pgfpathlineto{\pgfqpoint{1.620088in}{2.612517in}}%
\pgfpathlineto{\pgfqpoint{1.622846in}{2.534169in}}%
\pgfpathlineto{\pgfqpoint{1.628361in}{2.612517in}}%
\pgfpathlineto{\pgfqpoint{1.631119in}{2.534169in}}%
\pgfpathlineto{\pgfqpoint{1.633876in}{2.612517in}}%
\pgfpathlineto{\pgfqpoint{1.636634in}{2.534169in}}%
\pgfpathlineto{\pgfqpoint{1.639391in}{2.612517in}}%
\pgfpathlineto{\pgfqpoint{1.642149in}{2.534169in}}%
\pgfpathlineto{\pgfqpoint{1.644907in}{2.612517in}}%
\pgfpathlineto{\pgfqpoint{1.647664in}{2.534169in}}%
\pgfpathlineto{\pgfqpoint{1.650422in}{2.612517in}}%
\pgfpathlineto{\pgfqpoint{1.686270in}{2.703924in}}%
\pgfpathlineto{\pgfqpoint{1.689028in}{2.612517in}}%
\pgfpathlineto{\pgfqpoint{1.691786in}{2.703924in}}%
\pgfpathlineto{\pgfqpoint{1.694543in}{2.612517in}}%
\pgfpathlineto{\pgfqpoint{1.697301in}{2.703924in}}%
\pgfpathlineto{\pgfqpoint{1.700058in}{2.612517in}}%
\pgfpathlineto{\pgfqpoint{1.702816in}{2.703924in}}%
\pgfpathlineto{\pgfqpoint{1.705574in}{2.612517in}}%
\pgfpathlineto{\pgfqpoint{1.708331in}{2.703924in}}%
\pgfpathlineto{\pgfqpoint{1.711089in}{2.612517in}}%
\pgfpathlineto{\pgfqpoint{1.713846in}{2.703924in}}%
\pgfpathlineto{\pgfqpoint{1.716604in}{2.612517in}}%
\pgfpathlineto{\pgfqpoint{1.719361in}{2.703924in}}%
\pgfpathlineto{\pgfqpoint{1.724877in}{2.612517in}}%
\pgfpathlineto{\pgfqpoint{1.727634in}{2.703924in}}%
\pgfpathlineto{\pgfqpoint{1.733149in}{2.612517in}}%
\pgfpathlineto{\pgfqpoint{1.735907in}{2.703924in}}%
\pgfpathlineto{\pgfqpoint{1.755210in}{2.782272in}}%
\pgfpathlineto{\pgfqpoint{1.757968in}{2.703924in}}%
\pgfpathlineto{\pgfqpoint{1.766240in}{2.782272in}}%
\pgfpathlineto{\pgfqpoint{1.768998in}{2.703924in}}%
\pgfpathlineto{\pgfqpoint{1.771756in}{2.782272in}}%
\pgfpathlineto{\pgfqpoint{1.774513in}{2.703924in}}%
\pgfpathlineto{\pgfqpoint{1.777271in}{2.782272in}}%
\pgfpathlineto{\pgfqpoint{1.780028in}{2.703924in}}%
\pgfpathlineto{\pgfqpoint{1.782786in}{2.782272in}}%
\pgfpathlineto{\pgfqpoint{1.785544in}{2.703924in}}%
\pgfpathlineto{\pgfqpoint{1.788301in}{2.782272in}}%
\pgfpathlineto{\pgfqpoint{1.791059in}{2.703924in}}%
\pgfpathlineto{\pgfqpoint{1.793816in}{2.782272in}}%
\pgfpathlineto{\pgfqpoint{1.799331in}{2.703924in}}%
\pgfpathlineto{\pgfqpoint{1.802089in}{2.782272in}}%
\pgfpathlineto{\pgfqpoint{1.804847in}{2.703924in}}%
\pgfpathlineto{\pgfqpoint{1.807604in}{2.782272in}}%
\pgfpathlineto{\pgfqpoint{1.810362in}{2.703924in}}%
\pgfpathlineto{\pgfqpoint{1.813119in}{2.782272in}}%
\pgfpathlineto{\pgfqpoint{1.884817in}{2.860621in}}%
\pgfpathlineto{\pgfqpoint{1.887574in}{2.782272in}}%
\pgfpathlineto{\pgfqpoint{1.890332in}{2.860621in}}%
\pgfpathlineto{\pgfqpoint{1.893089in}{2.782272in}}%
\pgfpathlineto{\pgfqpoint{1.895847in}{2.860621in}}%
\pgfpathlineto{\pgfqpoint{1.898605in}{2.782272in}}%
\pgfpathlineto{\pgfqpoint{1.901362in}{2.860621in}}%
\pgfpathlineto{\pgfqpoint{1.904120in}{2.782272in}}%
\pgfpathlineto{\pgfqpoint{1.906877in}{2.860621in}}%
\pgfpathlineto{\pgfqpoint{1.909635in}{2.782272in}}%
\pgfpathlineto{\pgfqpoint{1.912393in}{2.860621in}}%
\pgfpathlineto{\pgfqpoint{1.915150in}{2.782272in}}%
\pgfpathlineto{\pgfqpoint{1.917908in}{2.860621in}}%
\pgfpathlineto{\pgfqpoint{1.920665in}{2.782272in}}%
\pgfpathlineto{\pgfqpoint{1.923423in}{2.860621in}}%
\pgfpathlineto{\pgfqpoint{1.926181in}{2.782272in}}%
\pgfpathlineto{\pgfqpoint{1.928938in}{2.860621in}}%
\pgfpathlineto{\pgfqpoint{1.939968in}{2.782272in}}%
\pgfpathlineto{\pgfqpoint{1.942726in}{2.860621in}}%
\pgfpathlineto{\pgfqpoint{1.992363in}{2.782272in}}%
\pgfpathlineto{\pgfqpoint{2.017181in}{2.051019in}}%
\pgfpathlineto{\pgfqpoint{2.022696in}{1.881264in}}%
\pgfpathlineto{\pgfqpoint{2.033726in}{1.724567in}}%
\pgfpathlineto{\pgfqpoint{2.036484in}{1.633160in}}%
\pgfpathlineto{\pgfqpoint{2.041999in}{1.554811in}}%
\pgfpathlineto{\pgfqpoint{2.050272in}{1.476463in}}%
\pgfpathlineto{\pgfqpoint{2.066817in}{1.228359in}}%
\pgfpathlineto{\pgfqpoint{2.069575in}{1.306707in}}%
\pgfpathlineto{\pgfqpoint{2.072333in}{1.228359in}}%
\pgfpathlineto{\pgfqpoint{2.077848in}{1.150010in}}%
\pgfpathlineto{\pgfqpoint{2.080605in}{1.228359in}}%
\pgfpathlineto{\pgfqpoint{2.083363in}{1.150010in}}%
\pgfpathlineto{\pgfqpoint{2.088878in}{1.071662in}}%
\pgfpathlineto{\pgfqpoint{2.091636in}{1.150010in}}%
\pgfpathlineto{\pgfqpoint{2.094393in}{1.071662in}}%
\pgfpathlineto{\pgfqpoint{2.102666in}{0.980255in}}%
\pgfpathlineto{\pgfqpoint{2.105424in}{1.071662in}}%
\pgfpathlineto{\pgfqpoint{2.108181in}{0.980255in}}%
\pgfpathlineto{\pgfqpoint{2.110939in}{1.071662in}}%
\pgfpathlineto{\pgfqpoint{2.116454in}{0.901906in}}%
\pgfpathlineto{\pgfqpoint{2.119212in}{0.980255in}}%
\pgfpathlineto{\pgfqpoint{2.121969in}{0.901906in}}%
\pgfpathlineto{\pgfqpoint{2.124727in}{0.980255in}}%
\pgfpathlineto{\pgfqpoint{2.127484in}{0.901906in}}%
\pgfpathlineto{\pgfqpoint{2.133000in}{0.823558in}}%
\pgfpathlineto{\pgfqpoint{2.135757in}{0.901906in}}%
\pgfpathlineto{\pgfqpoint{2.138515in}{0.823558in}}%
\pgfpathlineto{\pgfqpoint{2.152303in}{0.745209in}}%
\pgfpathlineto{\pgfqpoint{2.155060in}{0.823558in}}%
\pgfpathlineto{\pgfqpoint{2.157818in}{0.745209in}}%
\pgfpathlineto{\pgfqpoint{2.160575in}{0.823558in}}%
\pgfpathlineto{\pgfqpoint{2.163333in}{0.745209in}}%
\pgfpathlineto{\pgfqpoint{2.182636in}{0.823558in}}%
\pgfpathlineto{\pgfqpoint{2.188151in}{0.980255in}}%
\pgfpathlineto{\pgfqpoint{2.193666in}{1.071662in}}%
\pgfpathlineto{\pgfqpoint{2.196424in}{1.150010in}}%
\pgfpathlineto{\pgfqpoint{2.201939in}{1.228359in}}%
\pgfpathlineto{\pgfqpoint{2.204697in}{1.306707in}}%
\pgfpathlineto{\pgfqpoint{2.210212in}{1.398114in}}%
\pgfpathlineto{\pgfqpoint{2.212970in}{1.476463in}}%
\pgfpathlineto{\pgfqpoint{2.218485in}{1.554811in}}%
\pgfpathlineto{\pgfqpoint{2.221242in}{1.476463in}}%
\pgfpathlineto{\pgfqpoint{2.226758in}{1.633160in}}%
\pgfpathlineto{\pgfqpoint{2.237788in}{1.802915in}}%
\pgfpathlineto{\pgfqpoint{2.246061in}{1.881264in}}%
\pgfpathlineto{\pgfqpoint{2.248818in}{1.802915in}}%
\pgfpathlineto{\pgfqpoint{2.254333in}{1.959612in}}%
\pgfpathlineto{\pgfqpoint{2.257091in}{1.881264in}}%
\pgfpathlineto{\pgfqpoint{2.259849in}{1.959612in}}%
\pgfpathlineto{\pgfqpoint{2.265364in}{2.051019in}}%
\pgfpathlineto{\pgfqpoint{2.268121in}{1.959612in}}%
\pgfpathlineto{\pgfqpoint{2.270879in}{2.051019in}}%
\pgfpathlineto{\pgfqpoint{2.276394in}{2.129368in}}%
\pgfpathlineto{\pgfqpoint{2.279152in}{2.051019in}}%
\pgfpathlineto{\pgfqpoint{2.281909in}{2.129368in}}%
\pgfpathlineto{\pgfqpoint{2.292940in}{2.207716in}}%
\pgfpathlineto{\pgfqpoint{2.295697in}{2.129368in}}%
\pgfpathlineto{\pgfqpoint{2.298455in}{2.207716in}}%
\pgfpathlineto{\pgfqpoint{2.309485in}{2.286065in}}%
\pgfpathlineto{\pgfqpoint{2.312243in}{2.207716in}}%
\pgfpathlineto{\pgfqpoint{2.315000in}{2.286065in}}%
\pgfpathlineto{\pgfqpoint{2.317758in}{2.207716in}}%
\pgfpathlineto{\pgfqpoint{2.320516in}{2.286065in}}%
\pgfpathlineto{\pgfqpoint{2.334303in}{2.377471in}}%
\pgfpathlineto{\pgfqpoint{2.337061in}{2.286065in}}%
\pgfpathlineto{\pgfqpoint{2.339819in}{2.377471in}}%
\pgfpathlineto{\pgfqpoint{2.342576in}{2.286065in}}%
\pgfpathlineto{\pgfqpoint{2.345334in}{2.377471in}}%
\pgfpathlineto{\pgfqpoint{2.348091in}{2.286065in}}%
\pgfpathlineto{\pgfqpoint{2.350849in}{2.377471in}}%
\pgfpathlineto{\pgfqpoint{2.367394in}{2.455820in}}%
\pgfpathlineto{\pgfqpoint{2.370152in}{2.377471in}}%
\pgfpathlineto{\pgfqpoint{2.372910in}{2.455820in}}%
\pgfpathlineto{\pgfqpoint{2.375667in}{2.377471in}}%
\pgfpathlineto{\pgfqpoint{2.378425in}{2.455820in}}%
\pgfpathlineto{\pgfqpoint{2.381182in}{2.377471in}}%
\pgfpathlineto{\pgfqpoint{2.383940in}{2.455820in}}%
\pgfpathlineto{\pgfqpoint{2.397728in}{2.534169in}}%
\pgfpathlineto{\pgfqpoint{2.400486in}{2.455820in}}%
\pgfpathlineto{\pgfqpoint{2.403243in}{2.534169in}}%
\pgfpathlineto{\pgfqpoint{2.406001in}{2.455820in}}%
\pgfpathlineto{\pgfqpoint{2.408758in}{2.534169in}}%
\pgfpathlineto{\pgfqpoint{2.411516in}{2.455820in}}%
\pgfpathlineto{\pgfqpoint{2.414273in}{2.534169in}}%
\pgfpathlineto{\pgfqpoint{2.417031in}{2.455820in}}%
\pgfpathlineto{\pgfqpoint{2.419789in}{2.534169in}}%
\pgfpathlineto{\pgfqpoint{2.422546in}{2.455820in}}%
\pgfpathlineto{\pgfqpoint{2.425304in}{2.534169in}}%
\pgfpathlineto{\pgfqpoint{2.447365in}{2.612517in}}%
\pgfpathlineto{\pgfqpoint{2.450122in}{2.534169in}}%
\pgfpathlineto{\pgfqpoint{2.452880in}{2.612517in}}%
\pgfpathlineto{\pgfqpoint{2.455637in}{2.534169in}}%
\pgfpathlineto{\pgfqpoint{2.458395in}{2.612517in}}%
\pgfpathlineto{\pgfqpoint{2.461152in}{2.534169in}}%
\pgfpathlineto{\pgfqpoint{2.463910in}{2.612517in}}%
\pgfpathlineto{\pgfqpoint{2.466668in}{2.534169in}}%
\pgfpathlineto{\pgfqpoint{2.469425in}{2.612517in}}%
\pgfpathlineto{\pgfqpoint{2.472183in}{2.534169in}}%
\pgfpathlineto{\pgfqpoint{2.474940in}{2.612517in}}%
\pgfpathlineto{\pgfqpoint{2.477698in}{2.534169in}}%
\pgfpathlineto{\pgfqpoint{2.480456in}{2.612517in}}%
\pgfpathlineto{\pgfqpoint{2.516304in}{2.703924in}}%
\pgfpathlineto{\pgfqpoint{2.519062in}{2.612517in}}%
\pgfpathlineto{\pgfqpoint{2.521819in}{2.703924in}}%
\pgfpathlineto{\pgfqpoint{2.524577in}{2.612517in}}%
\pgfpathlineto{\pgfqpoint{2.527335in}{2.703924in}}%
\pgfpathlineto{\pgfqpoint{2.530092in}{2.612517in}}%
\pgfpathlineto{\pgfqpoint{2.532850in}{2.703924in}}%
\pgfpathlineto{\pgfqpoint{2.535607in}{2.612517in}}%
\pgfpathlineto{\pgfqpoint{2.538365in}{2.703924in}}%
\pgfpathlineto{\pgfqpoint{2.541122in}{2.612517in}}%
\pgfpathlineto{\pgfqpoint{2.543880in}{2.703924in}}%
\pgfpathlineto{\pgfqpoint{2.546638in}{2.612517in}}%
\pgfpathlineto{\pgfqpoint{2.549395in}{2.703924in}}%
\pgfpathlineto{\pgfqpoint{2.552153in}{2.612517in}}%
\pgfpathlineto{\pgfqpoint{2.554910in}{2.703924in}}%
\pgfpathlineto{\pgfqpoint{2.571456in}{2.782272in}}%
\pgfpathlineto{\pgfqpoint{2.574214in}{2.703924in}}%
\pgfpathlineto{\pgfqpoint{2.579729in}{2.782272in}}%
\pgfpathlineto{\pgfqpoint{2.582486in}{2.703924in}}%
\pgfpathlineto{\pgfqpoint{2.585244in}{2.782272in}}%
\pgfpathlineto{\pgfqpoint{2.588001in}{2.703924in}}%
\pgfpathlineto{\pgfqpoint{2.590759in}{2.782272in}}%
\pgfpathlineto{\pgfqpoint{2.593517in}{2.703924in}}%
\pgfpathlineto{\pgfqpoint{2.596274in}{2.782272in}}%
\pgfpathlineto{\pgfqpoint{2.599032in}{2.703924in}}%
\pgfpathlineto{\pgfqpoint{2.601789in}{2.782272in}}%
\pgfpathlineto{\pgfqpoint{2.604547in}{2.703924in}}%
\pgfpathlineto{\pgfqpoint{2.607305in}{2.782272in}}%
\pgfpathlineto{\pgfqpoint{2.610062in}{2.703924in}}%
\pgfpathlineto{\pgfqpoint{2.612820in}{2.782272in}}%
\pgfpathlineto{\pgfqpoint{2.615577in}{2.703924in}}%
\pgfpathlineto{\pgfqpoint{2.618335in}{2.782272in}}%
\pgfpathlineto{\pgfqpoint{2.621093in}{2.703924in}}%
\pgfpathlineto{\pgfqpoint{2.623850in}{2.782272in}}%
\pgfpathlineto{\pgfqpoint{2.626608in}{2.703924in}}%
\pgfpathlineto{\pgfqpoint{2.629365in}{2.782272in}}%
\pgfpathlineto{\pgfqpoint{2.634880in}{2.703924in}}%
\pgfpathlineto{\pgfqpoint{2.637638in}{2.782272in}}%
\pgfpathlineto{\pgfqpoint{2.640396in}{2.703924in}}%
\pgfpathlineto{\pgfqpoint{2.643153in}{2.782272in}}%
\pgfpathlineto{\pgfqpoint{2.645911in}{2.703924in}}%
\pgfpathlineto{\pgfqpoint{2.648668in}{2.782272in}}%
\pgfpathlineto{\pgfqpoint{2.703820in}{2.860621in}}%
\pgfpathlineto{\pgfqpoint{2.706578in}{2.782272in}}%
\pgfpathlineto{\pgfqpoint{2.709335in}{2.860621in}}%
\pgfpathlineto{\pgfqpoint{2.712093in}{2.782272in}}%
\pgfpathlineto{\pgfqpoint{2.714851in}{2.860621in}}%
\pgfpathlineto{\pgfqpoint{2.717608in}{2.782272in}}%
\pgfpathlineto{\pgfqpoint{2.720366in}{2.860621in}}%
\pgfpathlineto{\pgfqpoint{2.723123in}{2.782272in}}%
\pgfpathlineto{\pgfqpoint{2.725881in}{2.860621in}}%
\pgfpathlineto{\pgfqpoint{2.728638in}{2.782272in}}%
\pgfpathlineto{\pgfqpoint{2.731396in}{2.860621in}}%
\pgfpathlineto{\pgfqpoint{2.734154in}{2.782272in}}%
\pgfpathlineto{\pgfqpoint{2.736911in}{2.860621in}}%
\pgfpathlineto{\pgfqpoint{2.739669in}{2.782272in}}%
\pgfpathlineto{\pgfqpoint{2.742426in}{2.860621in}}%
\pgfpathlineto{\pgfqpoint{2.745184in}{2.782272in}}%
\pgfpathlineto{\pgfqpoint{2.747942in}{2.860621in}}%
\pgfpathlineto{\pgfqpoint{2.750699in}{2.782272in}}%
\pgfpathlineto{\pgfqpoint{2.753457in}{2.860621in}}%
\pgfpathlineto{\pgfqpoint{2.758972in}{2.782272in}}%
\pgfpathlineto{\pgfqpoint{2.761729in}{2.860621in}}%
\pgfpathlineto{\pgfqpoint{2.764487in}{2.782272in}}%
\pgfpathlineto{\pgfqpoint{2.767245in}{2.860621in}}%
\pgfpathlineto{\pgfqpoint{2.770002in}{2.782272in}}%
\pgfpathlineto{\pgfqpoint{2.772760in}{2.860621in}}%
\pgfpathlineto{\pgfqpoint{2.827912in}{2.782272in}}%
\pgfpathlineto{\pgfqpoint{2.852730in}{2.051019in}}%
\pgfpathlineto{\pgfqpoint{2.861003in}{1.802915in}}%
\pgfpathlineto{\pgfqpoint{2.863760in}{1.724567in}}%
\pgfpathlineto{\pgfqpoint{2.869275in}{1.633160in}}%
\pgfpathlineto{\pgfqpoint{2.872033in}{1.554811in}}%
\pgfpathlineto{\pgfqpoint{2.874791in}{1.633160in}}%
\pgfpathlineto{\pgfqpoint{2.880306in}{1.476463in}}%
\pgfpathlineto{\pgfqpoint{2.888579in}{1.398114in}}%
\pgfpathlineto{\pgfqpoint{2.894094in}{1.306707in}}%
\pgfpathlineto{\pgfqpoint{2.910639in}{1.150010in}}%
\pgfpathlineto{\pgfqpoint{2.913397in}{1.228359in}}%
\pgfpathlineto{\pgfqpoint{2.916154in}{1.150010in}}%
\pgfpathlineto{\pgfqpoint{2.921670in}{1.071662in}}%
\pgfpathlineto{\pgfqpoint{2.924427in}{1.150010in}}%
\pgfpathlineto{\pgfqpoint{2.927185in}{1.071662in}}%
\pgfpathlineto{\pgfqpoint{2.932700in}{0.980255in}}%
\pgfpathlineto{\pgfqpoint{2.935457in}{1.071662in}}%
\pgfpathlineto{\pgfqpoint{2.938215in}{0.980255in}}%
\pgfpathlineto{\pgfqpoint{2.949245in}{0.901906in}}%
\pgfpathlineto{\pgfqpoint{2.952003in}{0.980255in}}%
\pgfpathlineto{\pgfqpoint{2.954761in}{0.901906in}}%
\pgfpathlineto{\pgfqpoint{2.968549in}{0.823558in}}%
\pgfpathlineto{\pgfqpoint{2.971306in}{0.901906in}}%
\pgfpathlineto{\pgfqpoint{2.974064in}{0.823558in}}%
\pgfpathlineto{\pgfqpoint{2.976821in}{0.901906in}}%
\pgfpathlineto{\pgfqpoint{2.979579in}{0.823558in}}%
\pgfpathlineto{\pgfqpoint{2.996124in}{0.745209in}}%
\pgfpathlineto{\pgfqpoint{2.998882in}{0.823558in}}%
\pgfpathlineto{\pgfqpoint{3.001640in}{0.745209in}}%
\pgfpathlineto{\pgfqpoint{3.004397in}{0.823558in}}%
\pgfpathlineto{\pgfqpoint{3.007155in}{0.745209in}}%
\pgfpathlineto{\pgfqpoint{3.012670in}{0.901906in}}%
\pgfpathlineto{\pgfqpoint{3.018185in}{0.980255in}}%
\pgfpathlineto{\pgfqpoint{3.023700in}{1.150010in}}%
\pgfpathlineto{\pgfqpoint{3.029215in}{1.228359in}}%
\pgfpathlineto{\pgfqpoint{3.031973in}{1.306707in}}%
\pgfpathlineto{\pgfqpoint{3.043003in}{1.476463in}}%
\pgfpathlineto{\pgfqpoint{3.048519in}{1.554811in}}%
\pgfpathlineto{\pgfqpoint{3.051276in}{1.476463in}}%
\pgfpathlineto{\pgfqpoint{3.059549in}{1.724567in}}%
\pgfpathlineto{\pgfqpoint{3.062307in}{1.633160in}}%
\pgfpathlineto{\pgfqpoint{3.067822in}{1.802915in}}%
\pgfpathlineto{\pgfqpoint{3.073337in}{1.881264in}}%
\pgfpathlineto{\pgfqpoint{3.078852in}{1.802915in}}%
\pgfpathlineto{\pgfqpoint{3.084367in}{1.959612in}}%
\pgfpathlineto{\pgfqpoint{3.087125in}{1.881264in}}%
\pgfpathlineto{\pgfqpoint{3.092640in}{2.051019in}}%
\pgfpathlineto{\pgfqpoint{3.095398in}{1.959612in}}%
\pgfpathlineto{\pgfqpoint{3.098155in}{2.051019in}}%
\pgfpathlineto{\pgfqpoint{3.103670in}{2.129368in}}%
\pgfpathlineto{\pgfqpoint{3.106428in}{2.051019in}}%
\pgfpathlineto{\pgfqpoint{3.109185in}{2.129368in}}%
\pgfpathlineto{\pgfqpoint{3.120216in}{2.207716in}}%
\pgfpathlineto{\pgfqpoint{3.122973in}{2.129368in}}%
\pgfpathlineto{\pgfqpoint{3.125731in}{2.207716in}}%
\pgfpathlineto{\pgfqpoint{3.128489in}{2.129368in}}%
\pgfpathlineto{\pgfqpoint{3.131246in}{2.207716in}}%
\pgfpathlineto{\pgfqpoint{3.136761in}{2.286065in}}%
\pgfpathlineto{\pgfqpoint{3.139519in}{2.207716in}}%
\pgfpathlineto{\pgfqpoint{3.142277in}{2.286065in}}%
\pgfpathlineto{\pgfqpoint{3.147792in}{2.207716in}}%
\pgfpathlineto{\pgfqpoint{3.150549in}{2.286065in}}%
\pgfpathlineto{\pgfqpoint{3.158822in}{2.377471in}}%
\pgfpathlineto{\pgfqpoint{3.161580in}{2.286065in}}%
\pgfpathlineto{\pgfqpoint{3.164337in}{2.377471in}}%
\pgfpathlineto{\pgfqpoint{3.167095in}{2.286065in}}%
\pgfpathlineto{\pgfqpoint{3.169852in}{2.377471in}}%
\pgfpathlineto{\pgfqpoint{3.186398in}{2.455820in}}%
\pgfpathlineto{\pgfqpoint{3.189156in}{2.377471in}}%
\pgfpathlineto{\pgfqpoint{3.191913in}{2.455820in}}%
\pgfpathlineto{\pgfqpoint{3.194671in}{2.377471in}}%
\pgfpathlineto{\pgfqpoint{3.197428in}{2.455820in}}%
\pgfpathlineto{\pgfqpoint{3.200186in}{2.377471in}}%
\pgfpathlineto{\pgfqpoint{3.202943in}{2.455820in}}%
\pgfpathlineto{\pgfqpoint{3.216731in}{2.534169in}}%
\pgfpathlineto{\pgfqpoint{3.219489in}{2.455820in}}%
\pgfpathlineto{\pgfqpoint{3.222247in}{2.534169in}}%
\pgfpathlineto{\pgfqpoint{3.225004in}{2.455820in}}%
\pgfpathlineto{\pgfqpoint{3.227762in}{2.534169in}}%
\pgfpathlineto{\pgfqpoint{3.230519in}{2.455820in}}%
\pgfpathlineto{\pgfqpoint{3.233277in}{2.534169in}}%
\pgfpathlineto{\pgfqpoint{3.236035in}{2.455820in}}%
\pgfpathlineto{\pgfqpoint{3.238792in}{2.534169in}}%
\pgfpathlineto{\pgfqpoint{3.244307in}{2.455820in}}%
\pgfpathlineto{\pgfqpoint{3.247065in}{2.534169in}}%
\pgfpathlineto{\pgfqpoint{3.266368in}{2.612517in}}%
\pgfpathlineto{\pgfqpoint{3.269126in}{2.534169in}}%
\pgfpathlineto{\pgfqpoint{3.274641in}{2.612517in}}%
\pgfpathlineto{\pgfqpoint{3.277398in}{2.534169in}}%
\pgfpathlineto{\pgfqpoint{3.280156in}{2.612517in}}%
\pgfpathlineto{\pgfqpoint{3.282914in}{2.534169in}}%
\pgfpathlineto{\pgfqpoint{3.285671in}{2.612517in}}%
\pgfpathlineto{\pgfqpoint{3.288429in}{2.534169in}}%
\pgfpathlineto{\pgfqpoint{3.291186in}{2.612517in}}%
\pgfpathlineto{\pgfqpoint{3.293944in}{2.534169in}}%
\pgfpathlineto{\pgfqpoint{3.296701in}{2.612517in}}%
\pgfpathlineto{\pgfqpoint{3.299459in}{2.534169in}}%
\pgfpathlineto{\pgfqpoint{3.302217in}{2.612517in}}%
\pgfpathlineto{\pgfqpoint{3.332550in}{2.703924in}}%
\pgfpathlineto{\pgfqpoint{3.335308in}{2.612517in}}%
\pgfpathlineto{\pgfqpoint{3.338065in}{2.703924in}}%
\pgfpathlineto{\pgfqpoint{3.340823in}{2.612517in}}%
\pgfpathlineto{\pgfqpoint{3.343580in}{2.703924in}}%
\pgfpathlineto{\pgfqpoint{3.346338in}{2.612517in}}%
\pgfpathlineto{\pgfqpoint{3.349096in}{2.703924in}}%
\pgfpathlineto{\pgfqpoint{3.351853in}{2.612517in}}%
\pgfpathlineto{\pgfqpoint{3.354611in}{2.703924in}}%
\pgfpathlineto{\pgfqpoint{3.357368in}{2.612517in}}%
\pgfpathlineto{\pgfqpoint{3.360126in}{2.703924in}}%
\pgfpathlineto{\pgfqpoint{3.362884in}{2.612517in}}%
\pgfpathlineto{\pgfqpoint{3.365641in}{2.703924in}}%
\pgfpathlineto{\pgfqpoint{3.368399in}{2.612517in}}%
\pgfpathlineto{\pgfqpoint{3.371156in}{2.703924in}}%
\pgfpathlineto{\pgfqpoint{3.373914in}{2.612517in}}%
\pgfpathlineto{\pgfqpoint{3.376671in}{2.703924in}}%
\pgfpathlineto{\pgfqpoint{3.382187in}{2.612517in}}%
\pgfpathlineto{\pgfqpoint{3.384944in}{2.703924in}}%
\pgfpathlineto{\pgfqpoint{3.415278in}{2.782272in}}%
\pgfpathlineto{\pgfqpoint{3.418035in}{2.703924in}}%
\pgfpathlineto{\pgfqpoint{3.420793in}{2.782272in}}%
\pgfpathlineto{\pgfqpoint{3.423550in}{2.703924in}}%
\pgfpathlineto{\pgfqpoint{3.426308in}{2.782272in}}%
\pgfpathlineto{\pgfqpoint{3.429066in}{2.703924in}}%
\pgfpathlineto{\pgfqpoint{3.431823in}{2.782272in}}%
\pgfpathlineto{\pgfqpoint{3.434581in}{2.703924in}}%
\pgfpathlineto{\pgfqpoint{3.437338in}{2.782272in}}%
\pgfpathlineto{\pgfqpoint{3.440096in}{2.703924in}}%
\pgfpathlineto{\pgfqpoint{3.442854in}{2.782272in}}%
\pgfpathlineto{\pgfqpoint{3.445611in}{2.703924in}}%
\pgfpathlineto{\pgfqpoint{3.448369in}{2.782272in}}%
\pgfpathlineto{\pgfqpoint{3.451126in}{2.703924in}}%
\pgfpathlineto{\pgfqpoint{3.453884in}{2.782272in}}%
\pgfpathlineto{\pgfqpoint{3.456642in}{2.703924in}}%
\pgfpathlineto{\pgfqpoint{3.459399in}{2.782272in}}%
\pgfpathlineto{\pgfqpoint{3.462157in}{2.703924in}}%
\pgfpathlineto{\pgfqpoint{3.464914in}{2.782272in}}%
\pgfpathlineto{\pgfqpoint{3.467672in}{2.703924in}}%
\pgfpathlineto{\pgfqpoint{3.470429in}{2.782272in}}%
\pgfpathlineto{\pgfqpoint{3.536612in}{2.860621in}}%
\pgfpathlineto{\pgfqpoint{3.539369in}{2.782272in}}%
\pgfpathlineto{\pgfqpoint{3.544884in}{2.860621in}}%
\pgfpathlineto{\pgfqpoint{3.547642in}{2.782272in}}%
\pgfpathlineto{\pgfqpoint{3.550399in}{2.860621in}}%
\pgfpathlineto{\pgfqpoint{3.553157in}{2.782272in}}%
\pgfpathlineto{\pgfqpoint{3.555915in}{2.860621in}}%
\pgfpathlineto{\pgfqpoint{3.558672in}{2.782272in}}%
\pgfpathlineto{\pgfqpoint{3.561430in}{2.860621in}}%
\pgfpathlineto{\pgfqpoint{3.564187in}{2.782272in}}%
\pgfpathlineto{\pgfqpoint{3.566945in}{2.860621in}}%
\pgfpathlineto{\pgfqpoint{3.569703in}{2.782272in}}%
\pgfpathlineto{\pgfqpoint{3.572460in}{2.860621in}}%
\pgfpathlineto{\pgfqpoint{3.575218in}{2.782272in}}%
\pgfpathlineto{\pgfqpoint{3.577975in}{2.860621in}}%
\pgfpathlineto{\pgfqpoint{3.580733in}{2.782272in}}%
\pgfpathlineto{\pgfqpoint{3.583491in}{2.860621in}}%
\pgfpathlineto{\pgfqpoint{3.586248in}{2.782272in}}%
\pgfpathlineto{\pgfqpoint{3.589006in}{2.860621in}}%
\pgfpathlineto{\pgfqpoint{3.622097in}{2.782272in}}%
\pgfpathlineto{\pgfqpoint{3.627612in}{2.703924in}}%
\pgfpathlineto{\pgfqpoint{3.635885in}{2.455820in}}%
\pgfpathlineto{\pgfqpoint{3.660703in}{1.724567in}}%
\pgfpathlineto{\pgfqpoint{3.663461in}{1.633160in}}%
\pgfpathlineto{\pgfqpoint{3.674491in}{1.476463in}}%
\pgfpathlineto{\pgfqpoint{3.677248in}{1.554811in}}%
\pgfpathlineto{\pgfqpoint{3.682764in}{1.398114in}}%
\pgfpathlineto{\pgfqpoint{3.685521in}{1.476463in}}%
\pgfpathlineto{\pgfqpoint{3.691036in}{1.306707in}}%
\pgfpathlineto{\pgfqpoint{3.696552in}{1.228359in}}%
\pgfpathlineto{\pgfqpoint{3.699309in}{1.306707in}}%
\pgfpathlineto{\pgfqpoint{3.704824in}{1.150010in}}%
\pgfpathlineto{\pgfqpoint{3.707582in}{1.228359in}}%
\pgfpathlineto{\pgfqpoint{3.710340in}{1.150010in}}%
\pgfpathlineto{\pgfqpoint{3.715855in}{1.071662in}}%
\pgfpathlineto{\pgfqpoint{3.718612in}{1.150010in}}%
\pgfpathlineto{\pgfqpoint{3.721370in}{1.071662in}}%
\pgfpathlineto{\pgfqpoint{3.726885in}{0.980255in}}%
\pgfpathlineto{\pgfqpoint{3.729643in}{1.071662in}}%
\pgfpathlineto{\pgfqpoint{3.732400in}{0.980255in}}%
\pgfpathlineto{\pgfqpoint{3.740673in}{0.901906in}}%
\pgfpathlineto{\pgfqpoint{3.743431in}{0.980255in}}%
\pgfpathlineto{\pgfqpoint{3.746188in}{0.901906in}}%
\pgfpathlineto{\pgfqpoint{3.751703in}{0.823558in}}%
\pgfpathlineto{\pgfqpoint{3.754461in}{0.901906in}}%
\pgfpathlineto{\pgfqpoint{3.757219in}{0.823558in}}%
\pgfpathlineto{\pgfqpoint{3.759976in}{0.901906in}}%
\pgfpathlineto{\pgfqpoint{3.762734in}{0.823558in}}%
\pgfpathlineto{\pgfqpoint{3.773764in}{0.745209in}}%
\pgfpathlineto{\pgfqpoint{3.776522in}{0.823558in}}%
\pgfpathlineto{\pgfqpoint{3.779279in}{0.745209in}}%
\pgfpathlineto{\pgfqpoint{3.795825in}{0.653803in}}%
\pgfpathlineto{\pgfqpoint{3.798582in}{0.745209in}}%
\pgfpathlineto{\pgfqpoint{3.804098in}{0.823558in}}%
\pgfpathlineto{\pgfqpoint{3.806855in}{0.901906in}}%
\pgfpathlineto{\pgfqpoint{3.812370in}{0.980255in}}%
\pgfpathlineto{\pgfqpoint{3.815128in}{1.071662in}}%
\pgfpathlineto{\pgfqpoint{3.820643in}{1.150010in}}%
\pgfpathlineto{\pgfqpoint{3.826158in}{1.306707in}}%
\pgfpathlineto{\pgfqpoint{3.831673in}{1.398114in}}%
\pgfpathlineto{\pgfqpoint{3.834431in}{1.476463in}}%
\pgfpathlineto{\pgfqpoint{3.839946in}{1.554811in}}%
\pgfpathlineto{\pgfqpoint{3.848219in}{1.633160in}}%
\pgfpathlineto{\pgfqpoint{3.853734in}{1.724567in}}%
\pgfpathlineto{\pgfqpoint{3.856492in}{1.633160in}}%
\pgfpathlineto{\pgfqpoint{3.862007in}{1.802915in}}%
\pgfpathlineto{\pgfqpoint{3.867522in}{1.881264in}}%
\pgfpathlineto{\pgfqpoint{3.870280in}{1.802915in}}%
\pgfpathlineto{\pgfqpoint{3.875795in}{1.959612in}}%
\pgfpathlineto{\pgfqpoint{3.878552in}{1.881264in}}%
\pgfpathlineto{\pgfqpoint{3.881310in}{1.959612in}}%
\pgfpathlineto{\pgfqpoint{3.886825in}{2.051019in}}%
\pgfpathlineto{\pgfqpoint{3.889583in}{1.959612in}}%
\pgfpathlineto{\pgfqpoint{3.892340in}{2.051019in}}%
\pgfpathlineto{\pgfqpoint{3.897855in}{2.129368in}}%
\pgfpathlineto{\pgfqpoint{3.900613in}{2.051019in}}%
\pgfpathlineto{\pgfqpoint{3.903371in}{2.129368in}}%
\pgfpathlineto{\pgfqpoint{3.911643in}{2.207716in}}%
\pgfpathlineto{\pgfqpoint{3.914401in}{2.129368in}}%
\pgfpathlineto{\pgfqpoint{3.917159in}{2.207716in}}%
\pgfpathlineto{\pgfqpoint{3.930947in}{2.286065in}}%
\pgfpathlineto{\pgfqpoint{3.933704in}{2.207716in}}%
\pgfpathlineto{\pgfqpoint{3.936462in}{2.286065in}}%
\pgfpathlineto{\pgfqpoint{3.939219in}{2.207716in}}%
\pgfpathlineto{\pgfqpoint{3.941977in}{2.286065in}}%
\pgfpathlineto{\pgfqpoint{3.953007in}{2.377471in}}%
\pgfpathlineto{\pgfqpoint{3.955765in}{2.286065in}}%
\pgfpathlineto{\pgfqpoint{3.958522in}{2.377471in}}%
\pgfpathlineto{\pgfqpoint{3.961280in}{2.286065in}}%
\pgfpathlineto{\pgfqpoint{3.964038in}{2.377471in}}%
\pgfpathlineto{\pgfqpoint{3.980583in}{2.455820in}}%
\pgfpathlineto{\pgfqpoint{3.983341in}{2.377471in}}%
\pgfpathlineto{\pgfqpoint{3.986098in}{2.455820in}}%
\pgfpathlineto{\pgfqpoint{3.988856in}{2.377471in}}%
\pgfpathlineto{\pgfqpoint{3.991613in}{2.455820in}}%
\pgfpathlineto{\pgfqpoint{3.997129in}{2.377471in}}%
\pgfpathlineto{\pgfqpoint{3.999886in}{2.455820in}}%
\pgfpathlineto{\pgfqpoint{4.013674in}{2.534169in}}%
\pgfpathlineto{\pgfqpoint{4.016432in}{2.455820in}}%
\pgfpathlineto{\pgfqpoint{4.019189in}{2.534169in}}%
\pgfpathlineto{\pgfqpoint{4.021947in}{2.455820in}}%
\pgfpathlineto{\pgfqpoint{4.024705in}{2.534169in}}%
\pgfpathlineto{\pgfqpoint{4.027462in}{2.455820in}}%
\pgfpathlineto{\pgfqpoint{4.030220in}{2.534169in}}%
\pgfpathlineto{\pgfqpoint{4.032977in}{2.455820in}}%
\pgfpathlineto{\pgfqpoint{4.035735in}{2.534169in}}%
\pgfpathlineto{\pgfqpoint{4.063311in}{2.612517in}}%
\pgfpathlineto{\pgfqpoint{4.066068in}{2.534169in}}%
\pgfpathlineto{\pgfqpoint{4.068826in}{2.612517in}}%
\pgfpathlineto{\pgfqpoint{4.071583in}{2.534169in}}%
\pgfpathlineto{\pgfqpoint{4.074341in}{2.612517in}}%
\pgfpathlineto{\pgfqpoint{4.077099in}{2.534169in}}%
\pgfpathlineto{\pgfqpoint{4.079856in}{2.612517in}}%
\pgfpathlineto{\pgfqpoint{4.082614in}{2.534169in}}%
\pgfpathlineto{\pgfqpoint{4.085371in}{2.612517in}}%
\pgfpathlineto{\pgfqpoint{4.088129in}{2.534169in}}%
\pgfpathlineto{\pgfqpoint{4.090887in}{2.612517in}}%
\pgfpathlineto{\pgfqpoint{4.129493in}{2.703924in}}%
\pgfpathlineto{\pgfqpoint{4.132250in}{2.612517in}}%
\pgfpathlineto{\pgfqpoint{4.135008in}{2.703924in}}%
\pgfpathlineto{\pgfqpoint{4.137766in}{2.612517in}}%
\pgfpathlineto{\pgfqpoint{4.140523in}{2.703924in}}%
\pgfpathlineto{\pgfqpoint{4.143281in}{2.612517in}}%
\pgfpathlineto{\pgfqpoint{4.146038in}{2.703924in}}%
\pgfpathlineto{\pgfqpoint{4.148796in}{2.612517in}}%
\pgfpathlineto{\pgfqpoint{4.151554in}{2.703924in}}%
\pgfpathlineto{\pgfqpoint{4.154311in}{2.612517in}}%
\pgfpathlineto{\pgfqpoint{4.157069in}{2.703924in}}%
\pgfpathlineto{\pgfqpoint{4.159826in}{2.612517in}}%
\pgfpathlineto{\pgfqpoint{4.162584in}{2.703924in}}%
\pgfpathlineto{\pgfqpoint{4.165341in}{2.612517in}}%
\pgfpathlineto{\pgfqpoint{4.168099in}{2.703924in}}%
\pgfpathlineto{\pgfqpoint{4.170857in}{2.612517in}}%
\pgfpathlineto{\pgfqpoint{4.173614in}{2.703924in}}%
\pgfpathlineto{\pgfqpoint{4.176372in}{2.612517in}}%
\pgfpathlineto{\pgfqpoint{4.179129in}{2.703924in}}%
\pgfpathlineto{\pgfqpoint{4.206705in}{2.782272in}}%
\pgfpathlineto{\pgfqpoint{4.209463in}{2.703924in}}%
\pgfpathlineto{\pgfqpoint{4.212220in}{2.782272in}}%
\pgfpathlineto{\pgfqpoint{4.214978in}{2.703924in}}%
\pgfpathlineto{\pgfqpoint{4.217736in}{2.782272in}}%
\pgfpathlineto{\pgfqpoint{4.220493in}{2.703924in}}%
\pgfpathlineto{\pgfqpoint{4.223251in}{2.782272in}}%
\pgfpathlineto{\pgfqpoint{4.226008in}{2.703924in}}%
\pgfpathlineto{\pgfqpoint{4.228766in}{2.782272in}}%
\pgfpathlineto{\pgfqpoint{4.231524in}{2.703924in}}%
\pgfpathlineto{\pgfqpoint{4.234281in}{2.782272in}}%
\pgfpathlineto{\pgfqpoint{4.237039in}{2.703924in}}%
\pgfpathlineto{\pgfqpoint{4.239796in}{2.782272in}}%
\pgfpathlineto{\pgfqpoint{4.242554in}{2.703924in}}%
\pgfpathlineto{\pgfqpoint{4.245311in}{2.782272in}}%
\pgfpathlineto{\pgfqpoint{4.248069in}{2.703924in}}%
\pgfpathlineto{\pgfqpoint{4.250827in}{2.782272in}}%
\pgfpathlineto{\pgfqpoint{4.256342in}{2.703924in}}%
\pgfpathlineto{\pgfqpoint{4.259099in}{2.782272in}}%
\pgfpathlineto{\pgfqpoint{4.264615in}{2.703924in}}%
\pgfpathlineto{\pgfqpoint{4.267372in}{2.782272in}}%
\pgfpathlineto{\pgfqpoint{4.319766in}{2.860621in}}%
\pgfpathlineto{\pgfqpoint{4.322524in}{2.782272in}}%
\pgfpathlineto{\pgfqpoint{4.328039in}{2.860621in}}%
\pgfpathlineto{\pgfqpoint{4.330797in}{2.782272in}}%
\pgfpathlineto{\pgfqpoint{4.333554in}{2.860621in}}%
\pgfpathlineto{\pgfqpoint{4.336312in}{2.782272in}}%
\pgfpathlineto{\pgfqpoint{4.339069in}{2.860621in}}%
\pgfpathlineto{\pgfqpoint{4.341827in}{2.782272in}}%
\pgfpathlineto{\pgfqpoint{4.344585in}{2.860621in}}%
\pgfpathlineto{\pgfqpoint{4.347342in}{2.782272in}}%
\pgfpathlineto{\pgfqpoint{4.350100in}{2.860621in}}%
\pgfpathlineto{\pgfqpoint{4.352857in}{2.782272in}}%
\pgfpathlineto{\pgfqpoint{4.355615in}{2.860621in}}%
\pgfpathlineto{\pgfqpoint{4.358373in}{2.782272in}}%
\pgfpathlineto{\pgfqpoint{4.361130in}{2.860621in}}%
\pgfpathlineto{\pgfqpoint{4.363888in}{2.782272in}}%
\pgfpathlineto{\pgfqpoint{4.366645in}{2.860621in}}%
\pgfpathlineto{\pgfqpoint{4.369403in}{2.782272in}}%
\pgfpathlineto{\pgfqpoint{4.372161in}{2.860621in}}%
\pgfpathlineto{\pgfqpoint{4.374918in}{2.782272in}}%
\pgfpathlineto{\pgfqpoint{4.377676in}{2.860621in}}%
\pgfpathlineto{\pgfqpoint{4.380433in}{2.782272in}}%
\pgfpathlineto{\pgfqpoint{4.383191in}{2.860621in}}%
\pgfpathlineto{\pgfqpoint{4.385948in}{2.782272in}}%
\pgfpathlineto{\pgfqpoint{4.388706in}{2.860621in}}%
\pgfpathlineto{\pgfqpoint{4.391464in}{2.782272in}}%
\pgfpathlineto{\pgfqpoint{4.394221in}{2.860621in}}%
\pgfpathlineto{\pgfqpoint{4.424555in}{2.782272in}}%
\pgfpathlineto{\pgfqpoint{4.449373in}{2.051019in}}%
\pgfpathlineto{\pgfqpoint{4.457646in}{1.802915in}}%
\pgfpathlineto{\pgfqpoint{4.463161in}{1.633160in}}%
\pgfpathlineto{\pgfqpoint{4.479706in}{1.398114in}}%
\pgfpathlineto{\pgfqpoint{4.487979in}{1.306707in}}%
\pgfpathlineto{\pgfqpoint{4.490737in}{1.398114in}}%
\pgfpathlineto{\pgfqpoint{4.493494in}{1.306707in}}%
\pgfpathlineto{\pgfqpoint{4.504525in}{1.150010in}}%
\pgfpathlineto{\pgfqpoint{4.507282in}{1.228359in}}%
\pgfpathlineto{\pgfqpoint{4.510040in}{1.150010in}}%
\pgfpathlineto{\pgfqpoint{4.515555in}{1.071662in}}%
\pgfpathlineto{\pgfqpoint{4.518313in}{1.150010in}}%
\pgfpathlineto{\pgfqpoint{4.521070in}{1.071662in}}%
\pgfpathlineto{\pgfqpoint{4.526585in}{0.980255in}}%
\pgfpathlineto{\pgfqpoint{4.529343in}{1.071662in}}%
\pgfpathlineto{\pgfqpoint{4.532101in}{0.980255in}}%
\pgfpathlineto{\pgfqpoint{4.540373in}{0.901906in}}%
\pgfpathlineto{\pgfqpoint{4.543131in}{0.980255in}}%
\pgfpathlineto{\pgfqpoint{4.545889in}{0.901906in}}%
\pgfpathlineto{\pgfqpoint{4.556919in}{0.823558in}}%
\pgfpathlineto{\pgfqpoint{4.559676in}{0.901906in}}%
\pgfpathlineto{\pgfqpoint{4.562434in}{0.823558in}}%
\pgfpathlineto{\pgfqpoint{4.565192in}{0.901906in}}%
\pgfpathlineto{\pgfqpoint{4.567949in}{0.823558in}}%
\pgfpathlineto{\pgfqpoint{4.576222in}{0.745209in}}%
\pgfpathlineto{\pgfqpoint{4.578980in}{0.823558in}}%
\pgfpathlineto{\pgfqpoint{4.581737in}{0.745209in}}%
\pgfpathlineto{\pgfqpoint{4.595525in}{0.653803in}}%
\pgfpathlineto{\pgfqpoint{4.603798in}{0.901906in}}%
\pgfpathlineto{\pgfqpoint{4.609313in}{0.980255in}}%
\pgfpathlineto{\pgfqpoint{4.614828in}{1.150010in}}%
\pgfpathlineto{\pgfqpoint{4.620343in}{1.228359in}}%
\pgfpathlineto{\pgfqpoint{4.623101in}{1.306707in}}%
\pgfpathlineto{\pgfqpoint{4.634131in}{1.476463in}}%
\pgfpathlineto{\pgfqpoint{4.650677in}{1.724567in}}%
\pgfpathlineto{\pgfqpoint{4.658950in}{1.802915in}}%
\pgfpathlineto{\pgfqpoint{4.661707in}{1.724567in}}%
\pgfpathlineto{\pgfqpoint{4.667222in}{1.881264in}}%
\pgfpathlineto{\pgfqpoint{4.672738in}{1.959612in}}%
\pgfpathlineto{\pgfqpoint{4.675495in}{1.881264in}}%
\pgfpathlineto{\pgfqpoint{4.678253in}{1.959612in}}%
\pgfpathlineto{\pgfqpoint{4.683768in}{2.051019in}}%
\pgfpathlineto{\pgfqpoint{4.689283in}{1.959612in}}%
\pgfpathlineto{\pgfqpoint{4.694798in}{2.129368in}}%
\pgfpathlineto{\pgfqpoint{4.697556in}{2.051019in}}%
\pgfpathlineto{\pgfqpoint{4.700313in}{2.129368in}}%
\pgfpathlineto{\pgfqpoint{4.711344in}{2.207716in}}%
\pgfpathlineto{\pgfqpoint{4.714101in}{2.129368in}}%
\pgfpathlineto{\pgfqpoint{4.716859in}{2.207716in}}%
\pgfpathlineto{\pgfqpoint{4.727889in}{2.286065in}}%
\pgfpathlineto{\pgfqpoint{4.730647in}{2.207716in}}%
\pgfpathlineto{\pgfqpoint{4.733404in}{2.286065in}}%
\pgfpathlineto{\pgfqpoint{4.738920in}{2.207716in}}%
\pgfpathlineto{\pgfqpoint{4.741677in}{2.286065in}}%
\pgfpathlineto{\pgfqpoint{4.747192in}{2.377471in}}%
\pgfpathlineto{\pgfqpoint{4.749950in}{2.286065in}}%
\pgfpathlineto{\pgfqpoint{4.752708in}{2.377471in}}%
\pgfpathlineto{\pgfqpoint{4.755465in}{2.286065in}}%
\pgfpathlineto{\pgfqpoint{4.758223in}{2.377471in}}%
\pgfpathlineto{\pgfqpoint{4.760980in}{2.286065in}}%
\pgfpathlineto{\pgfqpoint{4.763738in}{2.377471in}}%
\pgfpathlineto{\pgfqpoint{4.783041in}{2.455820in}}%
\pgfpathlineto{\pgfqpoint{4.785799in}{2.377471in}}%
\pgfpathlineto{\pgfqpoint{4.788556in}{2.455820in}}%
\pgfpathlineto{\pgfqpoint{4.791314in}{2.377471in}}%
\pgfpathlineto{\pgfqpoint{4.794071in}{2.455820in}}%
\pgfpathlineto{\pgfqpoint{4.796829in}{2.377471in}}%
\pgfpathlineto{\pgfqpoint{4.799587in}{2.455820in}}%
\pgfpathlineto{\pgfqpoint{4.802344in}{2.377471in}}%
\pgfpathlineto{\pgfqpoint{4.805102in}{2.455820in}}%
\pgfpathlineto{\pgfqpoint{4.821647in}{2.534169in}}%
\pgfpathlineto{\pgfqpoint{4.824405in}{2.455820in}}%
\pgfpathlineto{\pgfqpoint{4.827162in}{2.534169in}}%
\pgfpathlineto{\pgfqpoint{4.829920in}{2.455820in}}%
\pgfpathlineto{\pgfqpoint{4.832678in}{2.534169in}}%
\pgfpathlineto{\pgfqpoint{4.835435in}{2.455820in}}%
\pgfpathlineto{\pgfqpoint{4.838193in}{2.534169in}}%
\pgfpathlineto{\pgfqpoint{4.840950in}{2.455820in}}%
\pgfpathlineto{\pgfqpoint{4.843708in}{2.534169in}}%
\pgfpathlineto{\pgfqpoint{4.860253in}{2.612517in}}%
\pgfpathlineto{\pgfqpoint{4.860253in}{2.612517in}}%
\pgfusepath{stroke}%
\end{pgfscope}%
\begin{pgfscope}%
\pgfsetrectcap%
\pgfsetmiterjoin%
\pgfsetlinewidth{0.803000pt}%
\definecolor{currentstroke}{rgb}{0.000000,0.000000,0.000000}%
\pgfsetstrokecolor{currentstroke}%
\pgfsetdash{}{0pt}%
\pgfpathmoveto{\pgfqpoint{0.693677in}{0.539544in}}%
\pgfpathlineto{\pgfqpoint{0.693677in}{3.053228in}}%
\pgfusepath{stroke}%
\end{pgfscope}%
\begin{pgfscope}%
\pgfsetrectcap%
\pgfsetmiterjoin%
\pgfsetlinewidth{0.803000pt}%
\definecolor{currentstroke}{rgb}{0.000000,0.000000,0.000000}%
\pgfsetstrokecolor{currentstroke}%
\pgfsetdash{}{0pt}%
\pgfpathmoveto{\pgfqpoint{5.058662in}{0.539544in}}%
\pgfpathlineto{\pgfqpoint{5.058662in}{3.053228in}}%
\pgfusepath{stroke}%
\end{pgfscope}%
\begin{pgfscope}%
\pgfsetrectcap%
\pgfsetmiterjoin%
\pgfsetlinewidth{0.803000pt}%
\definecolor{currentstroke}{rgb}{0.000000,0.000000,0.000000}%
\pgfsetstrokecolor{currentstroke}%
\pgfsetdash{}{0pt}%
\pgfpathmoveto{\pgfqpoint{0.693677in}{0.539544in}}%
\pgfpathlineto{\pgfqpoint{5.058662in}{0.539544in}}%
\pgfusepath{stroke}%
\end{pgfscope}%
\begin{pgfscope}%
\pgfsetrectcap%
\pgfsetmiterjoin%
\pgfsetlinewidth{0.803000pt}%
\definecolor{currentstroke}{rgb}{0.000000,0.000000,0.000000}%
\pgfsetstrokecolor{currentstroke}%
\pgfsetdash{}{0pt}%
\pgfpathmoveto{\pgfqpoint{0.693677in}{3.053228in}}%
\pgfpathlineto{\pgfqpoint{5.058662in}{3.053228in}}%
\pgfusepath{stroke}%
\end{pgfscope}%
\begin{pgfscope}%
\pgfsetbuttcap%
\pgfsetmiterjoin%
\definecolor{currentfill}{rgb}{1.000000,1.000000,1.000000}%
\pgfsetfillcolor{currentfill}%
\pgfsetfillopacity{0.800000}%
\pgfsetlinewidth{1.003750pt}%
\definecolor{currentstroke}{rgb}{0.800000,0.800000,0.800000}%
\pgfsetstrokecolor{currentstroke}%
\pgfsetstrokeopacity{0.800000}%
\pgfsetdash{}{0pt}%
\pgfpathmoveto{\pgfqpoint{0.771455in}{2.809450in}}%
\pgfpathlineto{\pgfqpoint{2.105788in}{2.809450in}}%
\pgfpathquadraticcurveto{\pgfqpoint{2.128010in}{2.809450in}}{\pgfqpoint{2.128010in}{2.831672in}}%
\pgfpathlineto{\pgfqpoint{2.128010in}{2.975450in}}%
\pgfpathquadraticcurveto{\pgfqpoint{2.128010in}{2.997672in}}{\pgfqpoint{2.105788in}{2.997672in}}%
\pgfpathlineto{\pgfqpoint{0.771455in}{2.997672in}}%
\pgfpathquadraticcurveto{\pgfqpoint{0.749232in}{2.997672in}}{\pgfqpoint{0.749232in}{2.975450in}}%
\pgfpathlineto{\pgfqpoint{0.749232in}{2.831672in}}%
\pgfpathquadraticcurveto{\pgfqpoint{0.749232in}{2.809450in}}{\pgfqpoint{0.771455in}{2.809450in}}%
\pgfpathlineto{\pgfqpoint{0.771455in}{2.809450in}}%
\pgfpathclose%
\pgfusepath{stroke,fill}%
\end{pgfscope}%
\begin{pgfscope}%
\pgfsetrectcap%
\pgfsetroundjoin%
\pgfsetlinewidth{0.501875pt}%
\definecolor{currentstroke}{rgb}{0.121569,0.466667,0.705882}%
\pgfsetstrokecolor{currentstroke}%
\pgfsetstrokeopacity{0.700000}%
\pgfsetdash{}{0pt}%
\pgfpathmoveto{\pgfqpoint{0.793677in}{2.914339in}}%
\pgfpathlineto{\pgfqpoint{0.904788in}{2.914339in}}%
\pgfpathlineto{\pgfqpoint{1.015899in}{2.914339in}}%
\pgfusepath{stroke}%
\end{pgfscope}%
\begin{pgfscope}%
\definecolor{textcolor}{rgb}{0.000000,0.000000,0.000000}%
\pgfsetstrokecolor{textcolor}%
\pgfsetfillcolor{textcolor}%
\pgftext[x=1.104788in,y=2.875450in,left,base]{\color{textcolor}\rmfamily\fontsize{8.000000}{9.600000}\selectfont Room temperature}%
\end{pgfscope}%
\end{pgfpicture}%
\makeatother%
\endgroup%
% data/plot_generic.py
    \caption{Temperature in lab 011 of the APQ group on 2016-11-26. Recorded by the LabKraken monitor. See section \ref{sec:res_labkraken} for details.}
    \label{fig:lab_temperature_start_of_project}
\end{figure}

As it can be seen there are strong oscillations of the temperature around the setpoint of \qty{21}{\celsius} as a result of the on–off air conditioning temperature controller. The commercial controller initially installed was using an IMI Heimeier \device{EMO T} thermoelectric actuator \cite{datasheet_heimeier_emo_t}, which is a two-step valve. Although this solution was later replaced by a custom design described in section \ref{sec:lab_temp_control}, these type of controllers are found in many other labs and temperature swings of \qty{2}{\kelvin} must therefore be expected.

These environmental parameters can now be used to estimate the design requirements for the laser driver. In comparison to the other laser system used in this group, the \qty{450}{\nm} system \cite{thesis_baus} required for the spectroscopy of highly charged ions \cite{thesis_alex} at GSI is the more demanding system. This system was found to be more susceptible to changes of the drive current since the wavelength selective filter element was far broader in comparison to a \qty{780}{\nm} system \cite{two_filter_paper}. This laser is stable over regions of tens of \unit{\uA} and requires a maximum drive current of \qty{145}{\mA} \cite{datasheet_osram_pl450b}.

From these considerations, the requirements for the driver can be inferred. It should be able to supply at least \qty{150}{\mA} and stay well within \qty{10}{\uA} over the whole environmental range. Given a worst-case scenario a tolerance of $3\sigma$ (\qty{99.7}{\percent}) must be met \cite{worst_case_design}.

The environmental parameters that mostly affect current sources are temperature and humidity. Air pressure is typically a matter of concern for high voltage systems \cite{IPC-2221B} and secondary in consideration for this design as it is a low voltage system (\qty{<= 48}{\V}). Air pressure effects are also the most expensive to test for, as a pressure chamber is required. Humidity affects electronics both directly though corrosion and also indirectly because the epoxy resin used in the FR-4 PCBs and component moulding is hygroscopic and the absorbed humidity leads to swelling and mechanical stress. This effect is very slow at ambient temperature and can easily take days to show \cite{epoxy_humidity}. This parameter is therefore handled via the long-term stability and not specified separately.

Given environmental conditions, the relative coefficients can be calculated. This estimation assumes a minimum setpoint resolution of 2 steps within the mode-hop-free region of the laser and calculates the \qty{99.7}{\percent} confidence interval. The steps are given in table \ref{tab:dgdrive_tempco}:
\begin{table}[hb]
    \centering
    \begin{tabular}{llr}
        \toprule
        Property& Value& Result \\
        \midrule
        Stable range & \qty{10}{\uA}& \qty{10}{\uA}\\
        2 steps of resolution  & $\div 2$& \qty{5}{\uA} \\
        $1 \sigma$  & $\div 2$& \qty{2.5}{\uA} \\
        Maximum output& \qty{150}{\mA}& \qty{17}{\uA \per \A}\\
        Temperature range& \qty{5}{\K}& \qty{3}{\uA \per \A \per \K}\\
        Worst case ($3 \sigma$)& $\div 3$& \qty{1}{\uA \per \A \per \K}\\
        \bottomrule
    \end{tabular}
    \caption{Estimated requirement for the temperature coefficient of the laser driver.}
    \label{tab:dgdrive_tempco}
\end{table}

While the requirements look moderate at first sight, tuning a quick estimation shown in table \ref{tab:dgdrive_tempco} leads to a temperature coefficient of \qty[per-mode = symbol]{1}{\uA \per \A \per \K} or even tighter when using a higher output driver -- a rather formidable specification for a current source.

Regarding the long-term stability, a \qty{30}{\day} number can be estimated. One may be inclined to call for a drift which is smaller than the stable range, but this would be short-sighted, as there are other factors to consider. The laser including the external resonator has its own figure of merit regarding the spectral drift rate. \citeauthor{ecdl_stability} \cite{ecdl_stability} reported a drift of \qty{2.9}{\MHz \per \hour}, which was attributed either to the external resonator itself, the piezo or the collimation lens. It is most likely that this drift was caused by mechanical changes of the external resonator as it defines the output mode of the laser. The mechanical drift limits the required stability of the current source considerably, as a typical frequency change of the internal resonator with the current of \qty[per-mode=symbol]{3}{\MHz \per \micro \A} \cite{diodelaser_modulation} can be assumed. The (linear) ageing drift of the external resonator over \qty{30}{\day} is equivalent to a \qty{720}{\uA} drift over the same period. For the electronics, the drift is assumed to follow an Arrhenius-like equation resulting from stress, caused during manufacturing. This may eventually change to a slow linear drift after several months of relaxation. The coefficient can either be a positive or negative and leads to
\begin{table}[hb]
    \centering
    \begin{tabular}{llr}
        \toprule
        Property& Value& Result \\
        \midrule
        Ageing drift limit & \qty{720}{\uA}& \qty{720}{\uA}\\
        $1 \sigma$  & $\div 2$& \qty{360}{\uA} \\
        Maximum output& \qty{500}{\mA}& \qty{720}{\uA \per \A}\\
        Worst case ($3 \sigma$)& $\div 3$& \qty{240}{\uA \per \A}\\
        \bottomrule
    \end{tabular}
    \caption{Estimated requirement for the long-term stability over \qty{30}{\day} of the laser driver.}
    \label{tab:dgdrive_stability}
\end{table}

Based on these numbers, it is straightforward to see that the long-term stability of a laser driver is less important than the short-term temperature coefficient since the limiting factor is the mechanical construction of the laser. This necessitates an atomic reference for long-term stability and to compensate for acoustic resonances of the external resonator. Regarding the choice of suitable devices, the tight specification of the temperature coefficient most likely leads to a choice of components that will pass these long-term criteria as well, alleviating a bit the burden of proof as long-term drift specifications are hard to come by since they need a lot of time to validate and cannot be extrapolated from high temperature burn-in tests \cite{voltage_reference_drift}.

%While \qty{800}{\uA \per \A} over a \qty{30}{\day} period may seem large at first, it is actually very hard to accurately produce a current. To put this number in perspective, a commercial high-end current source like the Keithley \device{2600B} is specified for the \qty{100}{\mA} range at about \qty{170}{\uA \per \A} for a \qty{30}{\day} period when calculated from the 1-year specification \cite{datasheet_keithley2600}, again assuming an Arrhenius-like equation as the basis.

All of this leads to the following design specifications regarding the stability of the current driver:
\begin{center}
    \begin{specifications}[label={lst:dgDrive_specs_environment}]{Current source, environmental}
    \begin{itemize}
        \item Temperature range \qtyrange[text-series-to-math, reset-text-series = false, reset-math-version = false, range-phrase=\textup{~to~}]{20}{35}{\celsius}
        \item \textbf{Temperature coefficient \qty[text-series-to-math, reset-text-series = false, reset-math-version = false]{<= 1}{\uA \per \A \per \K}}
        \item Humidity (non-condensing) \qty{<= 75}{\percent rH}
        \item Humidity coefficient not specified, but included in the long-term drift
        \item Maximum altitude not specified
        \item Long-term drift over \qty{30}{\day} \qty{<= 240}{\uA \per \A}
    \end{itemize}
    \end{specifications}
\end{center}

%A basic laser current driver design that has some of the  can be found in the work of \citeauthor{libbrecht_hall} \cite{libbrecht_hall}. While this design contains all the basic features, like a current source, a modulation input and a voltage limit, there are several shortcomings that have emerged over the years with new generations of laser didoes. The laser driver used by legacy applications in this group is based on the aforementioned paper and has been successfully employed in several projects over the years, but several limitations have come up in recent years. In order to derive the design requirements of a new generation of laser drivers the important design elements need to be identified first. The essential design elements are the bulk current source, the modulation current source, the reference element and output programming. The next 4 sections will deal with each element and outline the design goals that were identified while employing the legacy generation of diode drivers in several experiments.

\subsection{Design Goals: Current Source}
% https://www.laserdiodecontrol.com/laser-diode-parameter-overview
% Diode Lasers and Photonic Integrated Circuits (Characteristic temperature)
% Near Threshold Operation of Semiconductor Lasers and Resonant-Type Laser Amplifiers
The change in output current caused by load impedance should be an order of magnitude less than the drift specification to ensure a negligible effect compared to the drift over time. The load resistance presented by the laser diodes most commonly used in our experiments ranges from \qty{50}{\ohm} \cite{datasheet_osram_pl450b} to \qty{30}{\ohm} \cite{datasheet_adl_785} and \qtyrange{10}{15}{\ohm} for \qty{780}{\nm} laser diode \cite{datasheet_sharp_780nm,datasheet_thorlabs_780nm}. The output impedance requirement can therefore be estimated as
\begin{align}
    \frac{R_{load}}{R_{out}} &= \frac{I_{set}}{I_{out}} - 1 \leq \qty[per-mode = symbol]{6.7}{\uA \per \A} \nonumber\\
    R_{out} &\geq \frac{\qty{50}{\ohm}}{\qty[per-mode = symbol]{6.7}{\uA \per \A}} = \qty{7.5}{\mega \ohm}
\end{align}

An output impedance of more than \qty{7.5}{\mega \ohm} for slowly changing loads is a tough requirement, depending on the type of current source, which requires carefully selected components. A high output impedance is, for example, of importance to suppress radiated noise coming from external sources. Especially low frequency components from the mains supply can magnetically couple into the cables, because they are long enough. This noise can be substantial and a high output impedance at low frequencies is therefore important. Other applications will be discussed throughout this work. While a subpar output impedance is more of a limiting factor, the compliance voltage discussed next is a key requirement.

The compliance voltage is the maximum voltage the current source can apply to the load and is another non-ideal component of a real current source. The required voltage strongly depends on the type of laser diode used. The near-infrared laser diodes discussed above have an operating voltage of \qtyrange{1.5}{3}{\V}, while the Osram \device{PL 450B} blue laser diode is specified for \qtyrange{5.5}{7}{\V}. The \qty{7}{\V} required by the Osram laser diode is fairly high for a Fabry–Pérot laser diode and has proven difficult in the past \cite{thesis_baus} as most laser current drivers available are designed for the much lower forward voltage of the near infrared laser diodes. Even higher voltages of around \qtyrange{12}{15}{\V} are required for quantum cascade lasers, but these are currently neither used nor is their use planned in any experiment in the group.

The maximum output current of the laser driver currently required for laser diodes used in the group is \qty{250}{\mA} for the Thorlabs \device{L785H1} \cite{datasheet_thorlabs_780nm}. Therefore a maximum output current of \qty{300}{\mA} is considered sufficient.

The current noise of the laser driver can be estimated from the laser linewidth sought after as the laser frequency is sensitive to the injection current. At low frequencies, about \qty[per-mode=symbol]{-3}{\MHz \per \micro \A} can be attributed to the thermal expansion of the internal resonator of the diode due to resistive heating \cite{diodelaser_modulation}. Above \qty{1}{\MHz} this effect starts declining and exposes the change of the refractive index due to the presence of charge carriers. This high frequency effect is an order of magnitude weaker. Since the frequency sensitivity to current variations of the laser diode drops with higher frequencies, the most important range is from DC to \qty{100}{\kHz}.

To estimate the linewidth requirement, it is important to look at the experimental setup. While the spectroscopy of \ce{Ar^13+} at \qty{4}{\K} is limited to around \qty{150}{\MHz}  as shown on page \pageref{eqn:doppler_broadening}, the quantum computing experiments in this group have more stringent needs. It was shown in \cite{ecdl_stability, ecdl_silicone_housing,ecdl_linewidth_scholten} that with reasonable expense a passive linewidth of less than \qty{100}{\kHz} can be achieved. Using the relationship of the frequency sensitivity to a current modulation of laser diodes, \qty{100}{\kHz} translates to a current noise of \qty{30}{\nA_{rms}} from \qty{1}{\Hz} to \qty{100}{\kHz}. The lower \qty{1}{\Hz} limit is chosen fairly arbitrary, but the presence of $\frac 1 f$-noise inhibits a definition down to DC. There should be negligible amounts noise below \qty{1}{\Hz} compared to the upper \qty{100}{\kHz} though.

The final aspect of the current source that needs to be specified, is the bandwidth of the current steering input. The bandwidth in these terms defines a reasonably flat (\qty{\leq 3}{\dB}) response. As it was discussed above, beyond a frequency of \qty{1}{\MHz}, the frequency sensitivity of the laser diode to current modulation drops by an order of magnitude, altering the transfer function and introducing new challenges for control loops. Therefore a minimum bandwidth of \qty{1}{\MHz} is considered sufficient.

Above \qty{1}{\MHz} it is recommended to either use more dedicated solutions like the direct modulation at the laser head presented in \cite{current_mod_paper} or switch to acousto-optic modulators (AOMs) or electro-optic modulators (EOMs).

This leads to the following requirements regarding the current source of the laser driver:
\begin{center}
    \begin{specifications}[label={lst:dgDrive_specs_electrical}]{Current source, electrical}
    \begin{itemize}
        \item Maximum output current \qty{300}{\mA}, optionally \qty{500}{\mA}
        \item \textbf{Compliance voltage \qty[text-series-to-math, reset-text-series = false, reset-math-version = false]{\geq 8}{\V}}
        \item Output impedance \qty{\geq 7.5}{\mega\ohm} at low frequencies (close to DC)
        \item \textbf{Current noise \qty[text-series-to-math, reset-text-series = false, reset-math-version = false]{\leq 30}{\nA_{rms}} from DC to \qty[text-series-to-math, reset-text-series = false, reset-math-version = false]{100}{\kHz}}
        \item \qty{3}{\dB}-bandwidth of the modulation source \qty{\geq 1}{\MHz}
    \end{itemize}
    \end{specifications}
\end{center}

\subsection{Design Goals: User Interface and Form Factor}
The user interface must allow repeatability and reproducibility of the outputs. The reason is that the laser system is intended to be portable to be moved from the university where it is performance tested to the GSI facility. Within the labs, systems are usually moved from test stands to the actual experiment as well. Requiring as little setup efforts as possible is a big advantage.

The interface must both be accessible both locally and remotely to allow simple adjustment of the parameters while on the bench and also from within the experimental control software. The local controls must be directly accessible to humans without tools to give a better user experience.

The remote user interface is strictly required because the Penning trap and the laser system are spatially separated with the laser system being located in a special laser lab for environmental as well as safety reasons. This separation is about \qty{30}{\meter}. Ideally this remote interface is computer controlled to give full access to all features of the laser system. USB or Ethernet is preferred as this does not require extra hardware in the lab.

Regarding the application programming interface (API), support for both Python and optionally LabVIEW is favoured, as most of the group has switched from LabVIEW to labscript suite \cite{labscript_2013} on Python to run the experiments.

The form factor should allow integration into standard 19-inch racks to allow simple transportation from the experiment location at GSI to the university for testing and calibration.

\begin{center}
    \begin{specifications}[label={lst:dgDrive_specs_api}]{Current source, user interface}
    \begin{itemize}
        \item Local control via the front end without tools
        \item Remote access via a digital interface
        \item Software API supporting \textbf{Python} and optionally LabVIEW
    \end{itemize}
    \end{specifications}
\end{center}

\clearpage
\section{Laser Temperature Controller}%
\label{sec:laser_temperatrure_controller}
The external cavity diode laser (ECDL) design employed at GSI and in this group, based on \cite{ecdl_paris}, consists of two parts: The laser diode, mounted in an aluminium frame containing a collimator, which is mounted in an external resonator also made of aluminium. The aluminum used for the external resonator is AlZn4.5Mg1, also called alloy 7200 \cite{datasheet_laser_alu}. It has a moderate thermal coefficient of expansion of \qty{23.1}{\micro\meter \per \m \per \K}, which is one order of magnitude larger than that of Invar, but is significantly easier to machine.

In order to derive the required stability criteria the laser diode and the external resonator must both be considered. The influence of external parameters on the laser wavelength were discussed in the work of \citeauthor{thesis_tilman} \cite{thesis_tilman}. The temperature sensitivity of a typical near-infrared laser diode at \qty{780}{\nm} along with the external resonator used in this group were calculated to be
\begin{align*}
    K_{T,diode} &\approx \qty{-3}{\GHz \per \K}\\
    K_{T,resonator} &\approx \qty{-9}{\GHz \per \K}\,.
\end{align*}

From these number it is clear that the resonator marks the lower bound. Going to a blue \qty{441}{\nm} laser this criterion is even more critical, because $K_{T,resonator}$ is proportional to the laser frequency and the frequency almost doubles, this leads to a sensitivity of the resonator on the order of
\begin{equation}
    K_{T,resonator} \approx \qty{-16}{\GHz \per \K}\,.
\end{equation}

This implies that in order to match the stability of the laser current driver, the temperate stability should be far better than \qty{1}{\milli \K}. A temperature stability of better than \qty{100}{\micro \K} has been demonstrated before \cite{tempcontroller_10uK,tempcontroller_10uK_jw,tempcontroller_15uK,tempcontroller_30uK,tempcontroller_40uK,tempcontroller_50uK,tempcontroller_65uK}, but all of these solution have in common that they use either multiple layers of shielding and control or elaborate baths into which the subject is submerged. The controller itself is then typically placed inside the controlled environment to shield it from external effects. This type of setup is not feasible in this situation as it would require a considerable redesign of the laser resonator. The laser resonators in use in this group \cite{thesis_tilman} have been set up over the course of several years and there are dozens of them in use. This existing design must therefore be taken into consideration as well. The resonator in its current state does does not have an airtight seal. The sensitivity of the laser frequency to the barometric pressure can be estimated using the formula developed by \citeauthor{ciddor} \cite{ciddor} to be
\begin{equation}
    K_{baro} =  \qty{-75}{\MHz \per \hecto \Pa}\,.
\end{equation}

This leads to a frequency drift of several \qty{100}{\MHz} due to a pressure drift of around \qty{\pm 10}{\hecto \Pa} observed in the lab over a typical day. A long-term drift of \qty{55}{\hecto \Pa} over the year 2022 was also recorded by the monitoring software LabKraken. On shorter time scales the air pressure varies on the order of tens of \unit{\Pa}. This must be matched by the temperature controller. It is therefore sufficient to call for a stability of \qty{<1}{\milli \K} when using an unsealed resonator. To guarantee such a stability, the resolution of the driver should be at least \qty{200}{\micro \K}, preferably \qty{100}{\micro \K}.

The type of temperature transducer used in the laser design is a \qty{10}{\kilo \ohm} thermistor, so the design must work with this type of sensor, while the support of other sensors like a PT100 is optional.

Finally, a problem often encountered with analog proportional–integral–derivative (PID) controllers when temperature controlling large, well isolated bodies is the long time scales involved. One example found in the lab are high finesse cavities mounted in vacuum enclosures. These extremely stable cavities are extensively used to reduce the linewidth of lasers to a few \unit{\Hz}. The time scales involved necessitate very long integration times $T_i$ or a rather small gain of the integral term $k_i$ of the PID controller. See section \ref{sec:pid_controller_basics} for details on PID controllers and the terminology. An illustration of the problem encountered with an analog controller can be seen in figure \ref{fig:stability_cavity}. It shows the temperature of a Stable Laser Systems \device{VH 6020} cavity housing used for a high finesse Fabry-Pérot cavity which has a time constant of \qtyrange[range-units = single, range-phrase={~to~}]{4}{7}{\hour} \cite{datasheet_vh6020}.
\begin{figure}[ht]
    \centering
    %% Creator: Matplotlib, PGF backend
%%
%% To include the figure in your LaTeX document, write
%%   \input{<filename>.pgf}
%%
%% Make sure the required packages are loaded in your preamble
%%   \usepackage{pgf}
%%
%% Also ensure that all the required font packages are loaded; for instance,
%% the lmodern package is sometimes necessary when using math font.
%%   \usepackage{lmodern}
%%
%% Figures using additional raster images can only be included by \input if
%% they are in the same directory as the main LaTeX file. For loading figures
%% from other directories you can use the `import` package
%%   \usepackage{import}
%%
%% and then include the figures with
%%   \import{<path to file>}{<filename>.pgf}
%%
%% Matplotlib used the following preamble
%%   \usepackage{siunitx}
%%   \usepackage{fontspec}
%%
\begingroup%
\makeatletter%
\begin{pgfpicture}%
\pgfpathrectangle{\pgfpointorigin}{\pgfqpoint{5.431103in}{3.356606in}}%
\pgfusepath{use as bounding box, clip}%
\begin{pgfscope}%
\pgfsetbuttcap%
\pgfsetmiterjoin%
\definecolor{currentfill}{rgb}{1.000000,1.000000,1.000000}%
\pgfsetfillcolor{currentfill}%
\pgfsetlinewidth{0.000000pt}%
\definecolor{currentstroke}{rgb}{1.000000,1.000000,1.000000}%
\pgfsetstrokecolor{currentstroke}%
\pgfsetdash{}{0pt}%
\pgfpathmoveto{\pgfqpoint{0.000000in}{0.000000in}}%
\pgfpathlineto{\pgfqpoint{5.431103in}{0.000000in}}%
\pgfpathlineto{\pgfqpoint{5.431103in}{3.356606in}}%
\pgfpathlineto{\pgfqpoint{0.000000in}{3.356606in}}%
\pgfpathlineto{\pgfqpoint{0.000000in}{0.000000in}}%
\pgfpathclose%
\pgfusepath{fill}%
\end{pgfscope}%
\begin{pgfscope}%
\pgfsetbuttcap%
\pgfsetmiterjoin%
\definecolor{currentfill}{rgb}{1.000000,1.000000,1.000000}%
\pgfsetfillcolor{currentfill}%
\pgfsetlinewidth{0.000000pt}%
\definecolor{currentstroke}{rgb}{0.000000,0.000000,0.000000}%
\pgfsetstrokecolor{currentstroke}%
\pgfsetstrokeopacity{0.000000}%
\pgfsetdash{}{0pt}%
\pgfpathmoveto{\pgfqpoint{0.752485in}{0.539544in}}%
\pgfpathlineto{\pgfqpoint{5.255215in}{0.539544in}}%
\pgfpathlineto{\pgfqpoint{5.255215in}{3.206606in}}%
\pgfpathlineto{\pgfqpoint{0.752485in}{3.206606in}}%
\pgfpathlineto{\pgfqpoint{0.752485in}{0.539544in}}%
\pgfpathclose%
\pgfusepath{fill}%
\end{pgfscope}%
\begin{pgfscope}%
\pgfpathrectangle{\pgfqpoint{0.752485in}{0.539544in}}{\pgfqpoint{4.502730in}{2.667062in}}%
\pgfusepath{clip}%
\pgfsetrectcap%
\pgfsetroundjoin%
\pgfsetlinewidth{0.803000pt}%
\definecolor{currentstroke}{rgb}{0.450000,0.450000,0.450000}%
\pgfsetstrokecolor{currentstroke}%
\pgfsetdash{}{0pt}%
\pgfpathmoveto{\pgfqpoint{0.956951in}{0.539544in}}%
\pgfpathlineto{\pgfqpoint{0.956951in}{3.206606in}}%
\pgfusepath{stroke}%
\end{pgfscope}%
\begin{pgfscope}%
\pgfsetbuttcap%
\pgfsetroundjoin%
\definecolor{currentfill}{rgb}{0.000000,0.000000,0.000000}%
\pgfsetfillcolor{currentfill}%
\pgfsetlinewidth{0.803000pt}%
\definecolor{currentstroke}{rgb}{0.000000,0.000000,0.000000}%
\pgfsetstrokecolor{currentstroke}%
\pgfsetdash{}{0pt}%
\pgfsys@defobject{currentmarker}{\pgfqpoint{0.000000in}{-0.048611in}}{\pgfqpoint{0.000000in}{0.000000in}}{%
\pgfpathmoveto{\pgfqpoint{0.000000in}{0.000000in}}%
\pgfpathlineto{\pgfqpoint{0.000000in}{-0.048611in}}%
\pgfusepath{stroke,fill}%
}%
\begin{pgfscope}%
\pgfsys@transformshift{0.956951in}{0.539544in}%
\pgfsys@useobject{currentmarker}{}%
\end{pgfscope}%
\end{pgfscope}%
\begin{pgfscope}%
\definecolor{textcolor}{rgb}{0.000000,0.000000,0.000000}%
\pgfsetstrokecolor{textcolor}%
\pgfsetfillcolor{textcolor}%
\pgftext[x=0.956951in,y=0.442322in,,top]{\color{textcolor}\rmfamily\fontsize{8.000000}{9.600000}\selectfont May\(\displaystyle {{-}04}\)}%
\end{pgfscope}%
\begin{pgfscope}%
\pgfpathrectangle{\pgfqpoint{0.752485in}{0.539544in}}{\pgfqpoint{4.502730in}{2.667062in}}%
\pgfusepath{clip}%
\pgfsetrectcap%
\pgfsetroundjoin%
\pgfsetlinewidth{0.803000pt}%
\definecolor{currentstroke}{rgb}{0.450000,0.450000,0.450000}%
\pgfsetstrokecolor{currentstroke}%
\pgfsetdash{}{0pt}%
\pgfpathmoveto{\pgfqpoint{1.468651in}{0.539544in}}%
\pgfpathlineto{\pgfqpoint{1.468651in}{3.206606in}}%
\pgfusepath{stroke}%
\end{pgfscope}%
\begin{pgfscope}%
\pgfsetbuttcap%
\pgfsetroundjoin%
\definecolor{currentfill}{rgb}{0.000000,0.000000,0.000000}%
\pgfsetfillcolor{currentfill}%
\pgfsetlinewidth{0.803000pt}%
\definecolor{currentstroke}{rgb}{0.000000,0.000000,0.000000}%
\pgfsetstrokecolor{currentstroke}%
\pgfsetdash{}{0pt}%
\pgfsys@defobject{currentmarker}{\pgfqpoint{0.000000in}{-0.048611in}}{\pgfqpoint{0.000000in}{0.000000in}}{%
\pgfpathmoveto{\pgfqpoint{0.000000in}{0.000000in}}%
\pgfpathlineto{\pgfqpoint{0.000000in}{-0.048611in}}%
\pgfusepath{stroke,fill}%
}%
\begin{pgfscope}%
\pgfsys@transformshift{1.468651in}{0.539544in}%
\pgfsys@useobject{currentmarker}{}%
\end{pgfscope}%
\end{pgfscope}%
\begin{pgfscope}%
\definecolor{textcolor}{rgb}{0.000000,0.000000,0.000000}%
\pgfsetstrokecolor{textcolor}%
\pgfsetfillcolor{textcolor}%
\pgftext[x=1.468651in,y=0.442322in,,top]{\color{textcolor}\rmfamily\fontsize{8.000000}{9.600000}\selectfont \(\displaystyle {03{:}00}\)}%
\end{pgfscope}%
\begin{pgfscope}%
\pgfpathrectangle{\pgfqpoint{0.752485in}{0.539544in}}{\pgfqpoint{4.502730in}{2.667062in}}%
\pgfusepath{clip}%
\pgfsetrectcap%
\pgfsetroundjoin%
\pgfsetlinewidth{0.803000pt}%
\definecolor{currentstroke}{rgb}{0.450000,0.450000,0.450000}%
\pgfsetstrokecolor{currentstroke}%
\pgfsetdash{}{0pt}%
\pgfpathmoveto{\pgfqpoint{1.980351in}{0.539544in}}%
\pgfpathlineto{\pgfqpoint{1.980351in}{3.206606in}}%
\pgfusepath{stroke}%
\end{pgfscope}%
\begin{pgfscope}%
\pgfsetbuttcap%
\pgfsetroundjoin%
\definecolor{currentfill}{rgb}{0.000000,0.000000,0.000000}%
\pgfsetfillcolor{currentfill}%
\pgfsetlinewidth{0.803000pt}%
\definecolor{currentstroke}{rgb}{0.000000,0.000000,0.000000}%
\pgfsetstrokecolor{currentstroke}%
\pgfsetdash{}{0pt}%
\pgfsys@defobject{currentmarker}{\pgfqpoint{0.000000in}{-0.048611in}}{\pgfqpoint{0.000000in}{0.000000in}}{%
\pgfpathmoveto{\pgfqpoint{0.000000in}{0.000000in}}%
\pgfpathlineto{\pgfqpoint{0.000000in}{-0.048611in}}%
\pgfusepath{stroke,fill}%
}%
\begin{pgfscope}%
\pgfsys@transformshift{1.980351in}{0.539544in}%
\pgfsys@useobject{currentmarker}{}%
\end{pgfscope}%
\end{pgfscope}%
\begin{pgfscope}%
\definecolor{textcolor}{rgb}{0.000000,0.000000,0.000000}%
\pgfsetstrokecolor{textcolor}%
\pgfsetfillcolor{textcolor}%
\pgftext[x=1.980351in,y=0.442322in,,top]{\color{textcolor}\rmfamily\fontsize{8.000000}{9.600000}\selectfont \(\displaystyle {06{:}00}\)}%
\end{pgfscope}%
\begin{pgfscope}%
\pgfpathrectangle{\pgfqpoint{0.752485in}{0.539544in}}{\pgfqpoint{4.502730in}{2.667062in}}%
\pgfusepath{clip}%
\pgfsetrectcap%
\pgfsetroundjoin%
\pgfsetlinewidth{0.803000pt}%
\definecolor{currentstroke}{rgb}{0.450000,0.450000,0.450000}%
\pgfsetstrokecolor{currentstroke}%
\pgfsetdash{}{0pt}%
\pgfpathmoveto{\pgfqpoint{2.492051in}{0.539544in}}%
\pgfpathlineto{\pgfqpoint{2.492051in}{3.206606in}}%
\pgfusepath{stroke}%
\end{pgfscope}%
\begin{pgfscope}%
\pgfsetbuttcap%
\pgfsetroundjoin%
\definecolor{currentfill}{rgb}{0.000000,0.000000,0.000000}%
\pgfsetfillcolor{currentfill}%
\pgfsetlinewidth{0.803000pt}%
\definecolor{currentstroke}{rgb}{0.000000,0.000000,0.000000}%
\pgfsetstrokecolor{currentstroke}%
\pgfsetdash{}{0pt}%
\pgfsys@defobject{currentmarker}{\pgfqpoint{0.000000in}{-0.048611in}}{\pgfqpoint{0.000000in}{0.000000in}}{%
\pgfpathmoveto{\pgfqpoint{0.000000in}{0.000000in}}%
\pgfpathlineto{\pgfqpoint{0.000000in}{-0.048611in}}%
\pgfusepath{stroke,fill}%
}%
\begin{pgfscope}%
\pgfsys@transformshift{2.492051in}{0.539544in}%
\pgfsys@useobject{currentmarker}{}%
\end{pgfscope}%
\end{pgfscope}%
\begin{pgfscope}%
\definecolor{textcolor}{rgb}{0.000000,0.000000,0.000000}%
\pgfsetstrokecolor{textcolor}%
\pgfsetfillcolor{textcolor}%
\pgftext[x=2.492051in,y=0.442322in,,top]{\color{textcolor}\rmfamily\fontsize{8.000000}{9.600000}\selectfont \(\displaystyle {09{:}00}\)}%
\end{pgfscope}%
\begin{pgfscope}%
\pgfpathrectangle{\pgfqpoint{0.752485in}{0.539544in}}{\pgfqpoint{4.502730in}{2.667062in}}%
\pgfusepath{clip}%
\pgfsetrectcap%
\pgfsetroundjoin%
\pgfsetlinewidth{0.803000pt}%
\definecolor{currentstroke}{rgb}{0.450000,0.450000,0.450000}%
\pgfsetstrokecolor{currentstroke}%
\pgfsetdash{}{0pt}%
\pgfpathmoveto{\pgfqpoint{3.003751in}{0.539544in}}%
\pgfpathlineto{\pgfqpoint{3.003751in}{3.206606in}}%
\pgfusepath{stroke}%
\end{pgfscope}%
\begin{pgfscope}%
\pgfsetbuttcap%
\pgfsetroundjoin%
\definecolor{currentfill}{rgb}{0.000000,0.000000,0.000000}%
\pgfsetfillcolor{currentfill}%
\pgfsetlinewidth{0.803000pt}%
\definecolor{currentstroke}{rgb}{0.000000,0.000000,0.000000}%
\pgfsetstrokecolor{currentstroke}%
\pgfsetdash{}{0pt}%
\pgfsys@defobject{currentmarker}{\pgfqpoint{0.000000in}{-0.048611in}}{\pgfqpoint{0.000000in}{0.000000in}}{%
\pgfpathmoveto{\pgfqpoint{0.000000in}{0.000000in}}%
\pgfpathlineto{\pgfqpoint{0.000000in}{-0.048611in}}%
\pgfusepath{stroke,fill}%
}%
\begin{pgfscope}%
\pgfsys@transformshift{3.003751in}{0.539544in}%
\pgfsys@useobject{currentmarker}{}%
\end{pgfscope}%
\end{pgfscope}%
\begin{pgfscope}%
\definecolor{textcolor}{rgb}{0.000000,0.000000,0.000000}%
\pgfsetstrokecolor{textcolor}%
\pgfsetfillcolor{textcolor}%
\pgftext[x=3.003751in,y=0.442322in,,top]{\color{textcolor}\rmfamily\fontsize{8.000000}{9.600000}\selectfont \(\displaystyle {12{:}00}\)}%
\end{pgfscope}%
\begin{pgfscope}%
\pgfpathrectangle{\pgfqpoint{0.752485in}{0.539544in}}{\pgfqpoint{4.502730in}{2.667062in}}%
\pgfusepath{clip}%
\pgfsetrectcap%
\pgfsetroundjoin%
\pgfsetlinewidth{0.803000pt}%
\definecolor{currentstroke}{rgb}{0.450000,0.450000,0.450000}%
\pgfsetstrokecolor{currentstroke}%
\pgfsetdash{}{0pt}%
\pgfpathmoveto{\pgfqpoint{3.515451in}{0.539544in}}%
\pgfpathlineto{\pgfqpoint{3.515451in}{3.206606in}}%
\pgfusepath{stroke}%
\end{pgfscope}%
\begin{pgfscope}%
\pgfsetbuttcap%
\pgfsetroundjoin%
\definecolor{currentfill}{rgb}{0.000000,0.000000,0.000000}%
\pgfsetfillcolor{currentfill}%
\pgfsetlinewidth{0.803000pt}%
\definecolor{currentstroke}{rgb}{0.000000,0.000000,0.000000}%
\pgfsetstrokecolor{currentstroke}%
\pgfsetdash{}{0pt}%
\pgfsys@defobject{currentmarker}{\pgfqpoint{0.000000in}{-0.048611in}}{\pgfqpoint{0.000000in}{0.000000in}}{%
\pgfpathmoveto{\pgfqpoint{0.000000in}{0.000000in}}%
\pgfpathlineto{\pgfqpoint{0.000000in}{-0.048611in}}%
\pgfusepath{stroke,fill}%
}%
\begin{pgfscope}%
\pgfsys@transformshift{3.515451in}{0.539544in}%
\pgfsys@useobject{currentmarker}{}%
\end{pgfscope}%
\end{pgfscope}%
\begin{pgfscope}%
\definecolor{textcolor}{rgb}{0.000000,0.000000,0.000000}%
\pgfsetstrokecolor{textcolor}%
\pgfsetfillcolor{textcolor}%
\pgftext[x=3.515451in,y=0.442322in,,top]{\color{textcolor}\rmfamily\fontsize{8.000000}{9.600000}\selectfont \(\displaystyle {15{:}00}\)}%
\end{pgfscope}%
\begin{pgfscope}%
\pgfpathrectangle{\pgfqpoint{0.752485in}{0.539544in}}{\pgfqpoint{4.502730in}{2.667062in}}%
\pgfusepath{clip}%
\pgfsetrectcap%
\pgfsetroundjoin%
\pgfsetlinewidth{0.803000pt}%
\definecolor{currentstroke}{rgb}{0.450000,0.450000,0.450000}%
\pgfsetstrokecolor{currentstroke}%
\pgfsetdash{}{0pt}%
\pgfpathmoveto{\pgfqpoint{4.027151in}{0.539544in}}%
\pgfpathlineto{\pgfqpoint{4.027151in}{3.206606in}}%
\pgfusepath{stroke}%
\end{pgfscope}%
\begin{pgfscope}%
\pgfsetbuttcap%
\pgfsetroundjoin%
\definecolor{currentfill}{rgb}{0.000000,0.000000,0.000000}%
\pgfsetfillcolor{currentfill}%
\pgfsetlinewidth{0.803000pt}%
\definecolor{currentstroke}{rgb}{0.000000,0.000000,0.000000}%
\pgfsetstrokecolor{currentstroke}%
\pgfsetdash{}{0pt}%
\pgfsys@defobject{currentmarker}{\pgfqpoint{0.000000in}{-0.048611in}}{\pgfqpoint{0.000000in}{0.000000in}}{%
\pgfpathmoveto{\pgfqpoint{0.000000in}{0.000000in}}%
\pgfpathlineto{\pgfqpoint{0.000000in}{-0.048611in}}%
\pgfusepath{stroke,fill}%
}%
\begin{pgfscope}%
\pgfsys@transformshift{4.027151in}{0.539544in}%
\pgfsys@useobject{currentmarker}{}%
\end{pgfscope}%
\end{pgfscope}%
\begin{pgfscope}%
\definecolor{textcolor}{rgb}{0.000000,0.000000,0.000000}%
\pgfsetstrokecolor{textcolor}%
\pgfsetfillcolor{textcolor}%
\pgftext[x=4.027151in,y=0.442322in,,top]{\color{textcolor}\rmfamily\fontsize{8.000000}{9.600000}\selectfont \(\displaystyle {18{:}00}\)}%
\end{pgfscope}%
\begin{pgfscope}%
\pgfpathrectangle{\pgfqpoint{0.752485in}{0.539544in}}{\pgfqpoint{4.502730in}{2.667062in}}%
\pgfusepath{clip}%
\pgfsetrectcap%
\pgfsetroundjoin%
\pgfsetlinewidth{0.803000pt}%
\definecolor{currentstroke}{rgb}{0.450000,0.450000,0.450000}%
\pgfsetstrokecolor{currentstroke}%
\pgfsetdash{}{0pt}%
\pgfpathmoveto{\pgfqpoint{4.538851in}{0.539544in}}%
\pgfpathlineto{\pgfqpoint{4.538851in}{3.206606in}}%
\pgfusepath{stroke}%
\end{pgfscope}%
\begin{pgfscope}%
\pgfsetbuttcap%
\pgfsetroundjoin%
\definecolor{currentfill}{rgb}{0.000000,0.000000,0.000000}%
\pgfsetfillcolor{currentfill}%
\pgfsetlinewidth{0.803000pt}%
\definecolor{currentstroke}{rgb}{0.000000,0.000000,0.000000}%
\pgfsetstrokecolor{currentstroke}%
\pgfsetdash{}{0pt}%
\pgfsys@defobject{currentmarker}{\pgfqpoint{0.000000in}{-0.048611in}}{\pgfqpoint{0.000000in}{0.000000in}}{%
\pgfpathmoveto{\pgfqpoint{0.000000in}{0.000000in}}%
\pgfpathlineto{\pgfqpoint{0.000000in}{-0.048611in}}%
\pgfusepath{stroke,fill}%
}%
\begin{pgfscope}%
\pgfsys@transformshift{4.538851in}{0.539544in}%
\pgfsys@useobject{currentmarker}{}%
\end{pgfscope}%
\end{pgfscope}%
\begin{pgfscope}%
\definecolor{textcolor}{rgb}{0.000000,0.000000,0.000000}%
\pgfsetstrokecolor{textcolor}%
\pgfsetfillcolor{textcolor}%
\pgftext[x=4.538851in,y=0.442322in,,top]{\color{textcolor}\rmfamily\fontsize{8.000000}{9.600000}\selectfont \(\displaystyle {21{:}00}\)}%
\end{pgfscope}%
\begin{pgfscope}%
\pgfpathrectangle{\pgfqpoint{0.752485in}{0.539544in}}{\pgfqpoint{4.502730in}{2.667062in}}%
\pgfusepath{clip}%
\pgfsetrectcap%
\pgfsetroundjoin%
\pgfsetlinewidth{0.803000pt}%
\definecolor{currentstroke}{rgb}{0.450000,0.450000,0.450000}%
\pgfsetstrokecolor{currentstroke}%
\pgfsetdash{}{0pt}%
\pgfpathmoveto{\pgfqpoint{5.050551in}{0.539544in}}%
\pgfpathlineto{\pgfqpoint{5.050551in}{3.206606in}}%
\pgfusepath{stroke}%
\end{pgfscope}%
\begin{pgfscope}%
\pgfsetbuttcap%
\pgfsetroundjoin%
\definecolor{currentfill}{rgb}{0.000000,0.000000,0.000000}%
\pgfsetfillcolor{currentfill}%
\pgfsetlinewidth{0.803000pt}%
\definecolor{currentstroke}{rgb}{0.000000,0.000000,0.000000}%
\pgfsetstrokecolor{currentstroke}%
\pgfsetdash{}{0pt}%
\pgfsys@defobject{currentmarker}{\pgfqpoint{0.000000in}{-0.048611in}}{\pgfqpoint{0.000000in}{0.000000in}}{%
\pgfpathmoveto{\pgfqpoint{0.000000in}{0.000000in}}%
\pgfpathlineto{\pgfqpoint{0.000000in}{-0.048611in}}%
\pgfusepath{stroke,fill}%
}%
\begin{pgfscope}%
\pgfsys@transformshift{5.050551in}{0.539544in}%
\pgfsys@useobject{currentmarker}{}%
\end{pgfscope}%
\end{pgfscope}%
\begin{pgfscope}%
\definecolor{textcolor}{rgb}{0.000000,0.000000,0.000000}%
\pgfsetstrokecolor{textcolor}%
\pgfsetfillcolor{textcolor}%
\pgftext[x=5.050551in,y=0.442322in,,top]{\color{textcolor}\rmfamily\fontsize{8.000000}{9.600000}\selectfont May\(\displaystyle {{-}05}\)}%
\end{pgfscope}%
\begin{pgfscope}%
\definecolor{textcolor}{rgb}{0.000000,0.000000,0.000000}%
\pgfsetstrokecolor{textcolor}%
\pgfsetfillcolor{textcolor}%
\pgftext[x=3.003850in,y=0.288100in,,top]{\color{textcolor}\rmfamily\fontsize{10.000000}{12.000000}\selectfont Time (UTC)}%
\end{pgfscope}%
\begin{pgfscope}%
\definecolor{textcolor}{rgb}{0.000000,0.000000,0.000000}%
\pgfsetstrokecolor{textcolor}%
\pgfsetfillcolor{textcolor}%
\pgftext[x=5.255215in,y=0.301989in,right,top]{\color{textcolor}\rmfamily\fontsize{8.000000}{9.600000}\selectfont \(\displaystyle {2023{-}}\)May\(\displaystyle {{-}05}\)}%
\end{pgfscope}%
\begin{pgfscope}%
\pgfpathrectangle{\pgfqpoint{0.752485in}{0.539544in}}{\pgfqpoint{4.502730in}{2.667062in}}%
\pgfusepath{clip}%
\pgfsetrectcap%
\pgfsetroundjoin%
\pgfsetlinewidth{0.803000pt}%
\definecolor{currentstroke}{rgb}{0.450000,0.450000,0.450000}%
\pgfsetstrokecolor{currentstroke}%
\pgfsetdash{}{0pt}%
\pgfpathmoveto{\pgfqpoint{0.752485in}{0.668416in}}%
\pgfpathlineto{\pgfqpoint{5.255215in}{0.668416in}}%
\pgfusepath{stroke}%
\end{pgfscope}%
\begin{pgfscope}%
\pgfsetbuttcap%
\pgfsetroundjoin%
\definecolor{currentfill}{rgb}{0.000000,0.000000,0.000000}%
\pgfsetfillcolor{currentfill}%
\pgfsetlinewidth{0.803000pt}%
\definecolor{currentstroke}{rgb}{0.000000,0.000000,0.000000}%
\pgfsetstrokecolor{currentstroke}%
\pgfsetdash{}{0pt}%
\pgfsys@defobject{currentmarker}{\pgfqpoint{-0.048611in}{0.000000in}}{\pgfqpoint{-0.000000in}{0.000000in}}{%
\pgfpathmoveto{\pgfqpoint{-0.000000in}{0.000000in}}%
\pgfpathlineto{\pgfqpoint{-0.048611in}{0.000000in}}%
\pgfusepath{stroke,fill}%
}%
\begin{pgfscope}%
\pgfsys@transformshift{0.752485in}{0.668416in}%
\pgfsys@useobject{currentmarker}{}%
\end{pgfscope}%
\end{pgfscope}%
\begin{pgfscope}%
\definecolor{textcolor}{rgb}{0.000000,0.000000,0.000000}%
\pgfsetstrokecolor{textcolor}%
\pgfsetfillcolor{textcolor}%
\pgftext[x=0.327326in, y=0.629861in, left, base]{\color{textcolor}\rmfamily\fontsize{8.000000}{9.600000}\selectfont \(\displaystyle {31.006}\)}%
\end{pgfscope}%
\begin{pgfscope}%
\pgfpathrectangle{\pgfqpoint{0.752485in}{0.539544in}}{\pgfqpoint{4.502730in}{2.667062in}}%
\pgfusepath{clip}%
\pgfsetrectcap%
\pgfsetroundjoin%
\pgfsetlinewidth{0.803000pt}%
\definecolor{currentstroke}{rgb}{0.450000,0.450000,0.450000}%
\pgfsetstrokecolor{currentstroke}%
\pgfsetdash{}{0pt}%
\pgfpathmoveto{\pgfqpoint{0.752485in}{1.045748in}}%
\pgfpathlineto{\pgfqpoint{5.255215in}{1.045748in}}%
\pgfusepath{stroke}%
\end{pgfscope}%
\begin{pgfscope}%
\pgfsetbuttcap%
\pgfsetroundjoin%
\definecolor{currentfill}{rgb}{0.000000,0.000000,0.000000}%
\pgfsetfillcolor{currentfill}%
\pgfsetlinewidth{0.803000pt}%
\definecolor{currentstroke}{rgb}{0.000000,0.000000,0.000000}%
\pgfsetstrokecolor{currentstroke}%
\pgfsetdash{}{0pt}%
\pgfsys@defobject{currentmarker}{\pgfqpoint{-0.048611in}{0.000000in}}{\pgfqpoint{-0.000000in}{0.000000in}}{%
\pgfpathmoveto{\pgfqpoint{-0.000000in}{0.000000in}}%
\pgfpathlineto{\pgfqpoint{-0.048611in}{0.000000in}}%
\pgfusepath{stroke,fill}%
}%
\begin{pgfscope}%
\pgfsys@transformshift{0.752485in}{1.045748in}%
\pgfsys@useobject{currentmarker}{}%
\end{pgfscope}%
\end{pgfscope}%
\begin{pgfscope}%
\definecolor{textcolor}{rgb}{0.000000,0.000000,0.000000}%
\pgfsetstrokecolor{textcolor}%
\pgfsetfillcolor{textcolor}%
\pgftext[x=0.327326in, y=1.007193in, left, base]{\color{textcolor}\rmfamily\fontsize{8.000000}{9.600000}\selectfont \(\displaystyle {31.008}\)}%
\end{pgfscope}%
\begin{pgfscope}%
\pgfpathrectangle{\pgfqpoint{0.752485in}{0.539544in}}{\pgfqpoint{4.502730in}{2.667062in}}%
\pgfusepath{clip}%
\pgfsetrectcap%
\pgfsetroundjoin%
\pgfsetlinewidth{0.803000pt}%
\definecolor{currentstroke}{rgb}{0.450000,0.450000,0.450000}%
\pgfsetstrokecolor{currentstroke}%
\pgfsetdash{}{0pt}%
\pgfpathmoveto{\pgfqpoint{0.752485in}{1.423080in}}%
\pgfpathlineto{\pgfqpoint{5.255215in}{1.423080in}}%
\pgfusepath{stroke}%
\end{pgfscope}%
\begin{pgfscope}%
\pgfsetbuttcap%
\pgfsetroundjoin%
\definecolor{currentfill}{rgb}{0.000000,0.000000,0.000000}%
\pgfsetfillcolor{currentfill}%
\pgfsetlinewidth{0.803000pt}%
\definecolor{currentstroke}{rgb}{0.000000,0.000000,0.000000}%
\pgfsetstrokecolor{currentstroke}%
\pgfsetdash{}{0pt}%
\pgfsys@defobject{currentmarker}{\pgfqpoint{-0.048611in}{0.000000in}}{\pgfqpoint{-0.000000in}{0.000000in}}{%
\pgfpathmoveto{\pgfqpoint{-0.000000in}{0.000000in}}%
\pgfpathlineto{\pgfqpoint{-0.048611in}{0.000000in}}%
\pgfusepath{stroke,fill}%
}%
\begin{pgfscope}%
\pgfsys@transformshift{0.752485in}{1.423080in}%
\pgfsys@useobject{currentmarker}{}%
\end{pgfscope}%
\end{pgfscope}%
\begin{pgfscope}%
\definecolor{textcolor}{rgb}{0.000000,0.000000,0.000000}%
\pgfsetstrokecolor{textcolor}%
\pgfsetfillcolor{textcolor}%
\pgftext[x=0.327326in, y=1.384525in, left, base]{\color{textcolor}\rmfamily\fontsize{8.000000}{9.600000}\selectfont \(\displaystyle {31.010}\)}%
\end{pgfscope}%
\begin{pgfscope}%
\pgfpathrectangle{\pgfqpoint{0.752485in}{0.539544in}}{\pgfqpoint{4.502730in}{2.667062in}}%
\pgfusepath{clip}%
\pgfsetrectcap%
\pgfsetroundjoin%
\pgfsetlinewidth{0.803000pt}%
\definecolor{currentstroke}{rgb}{0.450000,0.450000,0.450000}%
\pgfsetstrokecolor{currentstroke}%
\pgfsetdash{}{0pt}%
\pgfpathmoveto{\pgfqpoint{0.752485in}{1.800412in}}%
\pgfpathlineto{\pgfqpoint{5.255215in}{1.800412in}}%
\pgfusepath{stroke}%
\end{pgfscope}%
\begin{pgfscope}%
\pgfsetbuttcap%
\pgfsetroundjoin%
\definecolor{currentfill}{rgb}{0.000000,0.000000,0.000000}%
\pgfsetfillcolor{currentfill}%
\pgfsetlinewidth{0.803000pt}%
\definecolor{currentstroke}{rgb}{0.000000,0.000000,0.000000}%
\pgfsetstrokecolor{currentstroke}%
\pgfsetdash{}{0pt}%
\pgfsys@defobject{currentmarker}{\pgfqpoint{-0.048611in}{0.000000in}}{\pgfqpoint{-0.000000in}{0.000000in}}{%
\pgfpathmoveto{\pgfqpoint{-0.000000in}{0.000000in}}%
\pgfpathlineto{\pgfqpoint{-0.048611in}{0.000000in}}%
\pgfusepath{stroke,fill}%
}%
\begin{pgfscope}%
\pgfsys@transformshift{0.752485in}{1.800412in}%
\pgfsys@useobject{currentmarker}{}%
\end{pgfscope}%
\end{pgfscope}%
\begin{pgfscope}%
\definecolor{textcolor}{rgb}{0.000000,0.000000,0.000000}%
\pgfsetstrokecolor{textcolor}%
\pgfsetfillcolor{textcolor}%
\pgftext[x=0.327326in, y=1.761857in, left, base]{\color{textcolor}\rmfamily\fontsize{8.000000}{9.600000}\selectfont \(\displaystyle {31.012}\)}%
\end{pgfscope}%
\begin{pgfscope}%
\pgfpathrectangle{\pgfqpoint{0.752485in}{0.539544in}}{\pgfqpoint{4.502730in}{2.667062in}}%
\pgfusepath{clip}%
\pgfsetrectcap%
\pgfsetroundjoin%
\pgfsetlinewidth{0.803000pt}%
\definecolor{currentstroke}{rgb}{0.450000,0.450000,0.450000}%
\pgfsetstrokecolor{currentstroke}%
\pgfsetdash{}{0pt}%
\pgfpathmoveto{\pgfqpoint{0.752485in}{2.177744in}}%
\pgfpathlineto{\pgfqpoint{5.255215in}{2.177744in}}%
\pgfusepath{stroke}%
\end{pgfscope}%
\begin{pgfscope}%
\pgfsetbuttcap%
\pgfsetroundjoin%
\definecolor{currentfill}{rgb}{0.000000,0.000000,0.000000}%
\pgfsetfillcolor{currentfill}%
\pgfsetlinewidth{0.803000pt}%
\definecolor{currentstroke}{rgb}{0.000000,0.000000,0.000000}%
\pgfsetstrokecolor{currentstroke}%
\pgfsetdash{}{0pt}%
\pgfsys@defobject{currentmarker}{\pgfqpoint{-0.048611in}{0.000000in}}{\pgfqpoint{-0.000000in}{0.000000in}}{%
\pgfpathmoveto{\pgfqpoint{-0.000000in}{0.000000in}}%
\pgfpathlineto{\pgfqpoint{-0.048611in}{0.000000in}}%
\pgfusepath{stroke,fill}%
}%
\begin{pgfscope}%
\pgfsys@transformshift{0.752485in}{2.177744in}%
\pgfsys@useobject{currentmarker}{}%
\end{pgfscope}%
\end{pgfscope}%
\begin{pgfscope}%
\definecolor{textcolor}{rgb}{0.000000,0.000000,0.000000}%
\pgfsetstrokecolor{textcolor}%
\pgfsetfillcolor{textcolor}%
\pgftext[x=0.327326in, y=2.139189in, left, base]{\color{textcolor}\rmfamily\fontsize{8.000000}{9.600000}\selectfont \(\displaystyle {31.014}\)}%
\end{pgfscope}%
\begin{pgfscope}%
\pgfpathrectangle{\pgfqpoint{0.752485in}{0.539544in}}{\pgfqpoint{4.502730in}{2.667062in}}%
\pgfusepath{clip}%
\pgfsetrectcap%
\pgfsetroundjoin%
\pgfsetlinewidth{0.803000pt}%
\definecolor{currentstroke}{rgb}{0.450000,0.450000,0.450000}%
\pgfsetstrokecolor{currentstroke}%
\pgfsetdash{}{0pt}%
\pgfpathmoveto{\pgfqpoint{0.752485in}{2.555076in}}%
\pgfpathlineto{\pgfqpoint{5.255215in}{2.555076in}}%
\pgfusepath{stroke}%
\end{pgfscope}%
\begin{pgfscope}%
\pgfsetbuttcap%
\pgfsetroundjoin%
\definecolor{currentfill}{rgb}{0.000000,0.000000,0.000000}%
\pgfsetfillcolor{currentfill}%
\pgfsetlinewidth{0.803000pt}%
\definecolor{currentstroke}{rgb}{0.000000,0.000000,0.000000}%
\pgfsetstrokecolor{currentstroke}%
\pgfsetdash{}{0pt}%
\pgfsys@defobject{currentmarker}{\pgfqpoint{-0.048611in}{0.000000in}}{\pgfqpoint{-0.000000in}{0.000000in}}{%
\pgfpathmoveto{\pgfqpoint{-0.000000in}{0.000000in}}%
\pgfpathlineto{\pgfqpoint{-0.048611in}{0.000000in}}%
\pgfusepath{stroke,fill}%
}%
\begin{pgfscope}%
\pgfsys@transformshift{0.752485in}{2.555076in}%
\pgfsys@useobject{currentmarker}{}%
\end{pgfscope}%
\end{pgfscope}%
\begin{pgfscope}%
\definecolor{textcolor}{rgb}{0.000000,0.000000,0.000000}%
\pgfsetstrokecolor{textcolor}%
\pgfsetfillcolor{textcolor}%
\pgftext[x=0.327326in, y=2.516521in, left, base]{\color{textcolor}\rmfamily\fontsize{8.000000}{9.600000}\selectfont \(\displaystyle {31.016}\)}%
\end{pgfscope}%
\begin{pgfscope}%
\pgfpathrectangle{\pgfqpoint{0.752485in}{0.539544in}}{\pgfqpoint{4.502730in}{2.667062in}}%
\pgfusepath{clip}%
\pgfsetrectcap%
\pgfsetroundjoin%
\pgfsetlinewidth{0.803000pt}%
\definecolor{currentstroke}{rgb}{0.450000,0.450000,0.450000}%
\pgfsetstrokecolor{currentstroke}%
\pgfsetdash{}{0pt}%
\pgfpathmoveto{\pgfqpoint{0.752485in}{2.932408in}}%
\pgfpathlineto{\pgfqpoint{5.255215in}{2.932408in}}%
\pgfusepath{stroke}%
\end{pgfscope}%
\begin{pgfscope}%
\pgfsetbuttcap%
\pgfsetroundjoin%
\definecolor{currentfill}{rgb}{0.000000,0.000000,0.000000}%
\pgfsetfillcolor{currentfill}%
\pgfsetlinewidth{0.803000pt}%
\definecolor{currentstroke}{rgb}{0.000000,0.000000,0.000000}%
\pgfsetstrokecolor{currentstroke}%
\pgfsetdash{}{0pt}%
\pgfsys@defobject{currentmarker}{\pgfqpoint{-0.048611in}{0.000000in}}{\pgfqpoint{-0.000000in}{0.000000in}}{%
\pgfpathmoveto{\pgfqpoint{-0.000000in}{0.000000in}}%
\pgfpathlineto{\pgfqpoint{-0.048611in}{0.000000in}}%
\pgfusepath{stroke,fill}%
}%
\begin{pgfscope}%
\pgfsys@transformshift{0.752485in}{2.932408in}%
\pgfsys@useobject{currentmarker}{}%
\end{pgfscope}%
\end{pgfscope}%
\begin{pgfscope}%
\definecolor{textcolor}{rgb}{0.000000,0.000000,0.000000}%
\pgfsetstrokecolor{textcolor}%
\pgfsetfillcolor{textcolor}%
\pgftext[x=0.327326in, y=2.893853in, left, base]{\color{textcolor}\rmfamily\fontsize{8.000000}{9.600000}\selectfont \(\displaystyle {31.018}\)}%
\end{pgfscope}%
\begin{pgfscope}%
\definecolor{textcolor}{rgb}{0.000000,0.000000,0.000000}%
\pgfsetstrokecolor{textcolor}%
\pgfsetfillcolor{textcolor}%
\pgftext[x=0.271770in,y=1.873075in,,bottom,rotate=90.000000]{\color{textcolor}\rmfamily\fontsize{10.000000}{12.000000}\selectfont Temperature in \unit{\celsius}}%
\end{pgfscope}%
\begin{pgfscope}%
\pgfpathrectangle{\pgfqpoint{0.752485in}{0.539544in}}{\pgfqpoint{4.502730in}{2.667062in}}%
\pgfusepath{clip}%
\pgfsetrectcap%
\pgfsetroundjoin%
\pgfsetlinewidth{1.505625pt}%
\definecolor{currentstroke}{rgb}{0.003922,0.450980,0.698039}%
\pgfsetstrokecolor{currentstroke}%
\pgfsetstrokeopacity{0.700000}%
\pgfsetdash{}{0pt}%
\pgfpathmoveto{\pgfqpoint{0.957154in}{1.872887in}}%
\pgfpathlineto{\pgfqpoint{0.957367in}{1.930616in}}%
\pgfpathlineto{\pgfqpoint{0.960868in}{1.872887in}}%
\pgfpathlineto{\pgfqpoint{0.964784in}{1.930616in}}%
\pgfpathlineto{\pgfqpoint{0.973090in}{1.930616in}}%
\pgfpathlineto{\pgfqpoint{0.973287in}{1.988346in}}%
\pgfpathlineto{\pgfqpoint{0.977696in}{1.930616in}}%
\pgfpathlineto{\pgfqpoint{0.980903in}{1.988346in}}%
\pgfpathlineto{\pgfqpoint{0.986922in}{2.046077in}}%
\pgfpathlineto{\pgfqpoint{0.988285in}{1.988346in}}%
\pgfpathlineto{\pgfqpoint{0.994514in}{1.988346in}}%
\pgfpathlineto{\pgfqpoint{0.995715in}{2.046077in}}%
\pgfpathlineto{\pgfqpoint{1.011602in}{2.046077in}}%
\pgfpathlineto{\pgfqpoint{1.013127in}{2.103808in}}%
\pgfpathlineto{\pgfqpoint{1.015305in}{2.046077in}}%
\pgfpathlineto{\pgfqpoint{1.022633in}{2.046077in}}%
\pgfpathlineto{\pgfqpoint{1.022878in}{2.103808in}}%
\pgfpathlineto{\pgfqpoint{1.025932in}{2.046077in}}%
\pgfpathlineto{\pgfqpoint{1.028965in}{2.103808in}}%
\pgfpathlineto{\pgfqpoint{1.034934in}{2.103808in}}%
\pgfpathlineto{\pgfqpoint{1.035056in}{2.046077in}}%
\pgfpathlineto{\pgfqpoint{1.042312in}{2.046077in}}%
\pgfpathlineto{\pgfqpoint{1.042741in}{1.988346in}}%
\pgfpathlineto{\pgfqpoint{1.050556in}{1.988346in}}%
\pgfpathlineto{\pgfqpoint{1.050728in}{1.930616in}}%
\pgfpathlineto{\pgfqpoint{1.058566in}{1.930616in}}%
\pgfpathlineto{\pgfqpoint{1.058710in}{1.872887in}}%
\pgfpathlineto{\pgfqpoint{1.066339in}{1.872887in}}%
\pgfpathlineto{\pgfqpoint{1.066425in}{1.815159in}}%
\pgfpathlineto{\pgfqpoint{1.072053in}{1.815159in}}%
\pgfpathlineto{\pgfqpoint{1.077151in}{1.699705in}}%
\pgfpathlineto{\pgfqpoint{1.078502in}{1.757432in}}%
\pgfpathlineto{\pgfqpoint{1.081402in}{1.699705in}}%
\pgfpathlineto{\pgfqpoint{1.089189in}{1.699705in}}%
\pgfpathlineto{\pgfqpoint{1.089226in}{1.641980in}}%
\pgfpathlineto{\pgfqpoint{1.101742in}{1.641980in}}%
\pgfpathlineto{\pgfqpoint{1.104017in}{1.584255in}}%
\pgfpathlineto{\pgfqpoint{1.106029in}{1.641980in}}%
\pgfpathlineto{\pgfqpoint{1.110552in}{1.584255in}}%
\pgfpathlineto{\pgfqpoint{1.117259in}{1.584255in}}%
\pgfpathlineto{\pgfqpoint{1.119276in}{1.641980in}}%
\pgfpathlineto{\pgfqpoint{1.126470in}{1.641980in}}%
\pgfpathlineto{\pgfqpoint{1.130540in}{1.757432in}}%
\pgfpathlineto{\pgfqpoint{1.132762in}{1.699705in}}%
\pgfpathlineto{\pgfqpoint{1.136571in}{1.815159in}}%
\pgfpathlineto{\pgfqpoint{1.142690in}{1.815159in}}%
\pgfpathlineto{\pgfqpoint{1.143265in}{1.872887in}}%
\pgfpathlineto{\pgfqpoint{1.149711in}{1.872887in}}%
\pgfpathlineto{\pgfqpoint{1.150876in}{1.930616in}}%
\pgfpathlineto{\pgfqpoint{1.155786in}{1.930616in}}%
\pgfpathlineto{\pgfqpoint{1.157038in}{1.988346in}}%
\pgfpathlineto{\pgfqpoint{1.162936in}{1.988346in}}%
\pgfpathlineto{\pgfqpoint{1.163267in}{2.046077in}}%
\pgfpathlineto{\pgfqpoint{1.170955in}{2.046077in}}%
\pgfpathlineto{\pgfqpoint{1.171004in}{2.103808in}}%
\pgfpathlineto{\pgfqpoint{1.178891in}{2.103808in}}%
\pgfpathlineto{\pgfqpoint{1.181314in}{2.161541in}}%
\pgfpathlineto{\pgfqpoint{1.182676in}{2.103808in}}%
\pgfpathlineto{\pgfqpoint{1.189815in}{2.161541in}}%
\pgfpathlineto{\pgfqpoint{1.190294in}{2.103808in}}%
\pgfpathlineto{\pgfqpoint{1.201937in}{2.103808in}}%
\pgfpathlineto{\pgfqpoint{1.202038in}{2.046077in}}%
\pgfpathlineto{\pgfqpoint{1.209728in}{2.046077in}}%
\pgfpathlineto{\pgfqpoint{1.209752in}{1.988346in}}%
\pgfpathlineto{\pgfqpoint{1.216079in}{1.988346in}}%
\pgfpathlineto{\pgfqpoint{1.220343in}{1.872887in}}%
\pgfpathlineto{\pgfqpoint{1.225847in}{1.872887in}}%
\pgfpathlineto{\pgfqpoint{1.225920in}{1.815159in}}%
\pgfpathlineto{\pgfqpoint{1.233583in}{1.757432in}}%
\pgfpathlineto{\pgfqpoint{1.233681in}{1.699705in}}%
\pgfpathlineto{\pgfqpoint{1.241975in}{1.641980in}}%
\pgfpathlineto{\pgfqpoint{1.242048in}{1.584255in}}%
\pgfpathlineto{\pgfqpoint{1.247000in}{1.584255in}}%
\pgfpathlineto{\pgfqpoint{1.251007in}{1.468808in}}%
\pgfpathlineto{\pgfqpoint{1.257738in}{1.468808in}}%
\pgfpathlineto{\pgfqpoint{1.257811in}{1.411086in}}%
\pgfpathlineto{\pgfqpoint{1.269911in}{1.411086in}}%
\pgfpathlineto{\pgfqpoint{1.270009in}{1.353364in}}%
\pgfpathlineto{\pgfqpoint{1.273076in}{1.411086in}}%
\pgfpathlineto{\pgfqpoint{1.276071in}{1.353364in}}%
\pgfpathlineto{\pgfqpoint{1.281957in}{1.353364in}}%
\pgfpathlineto{\pgfqpoint{1.282207in}{1.411086in}}%
\pgfpathlineto{\pgfqpoint{1.290249in}{1.411086in}}%
\pgfpathlineto{\pgfqpoint{1.290437in}{1.468808in}}%
\pgfpathlineto{\pgfqpoint{1.297821in}{1.468808in}}%
\pgfpathlineto{\pgfqpoint{1.298090in}{1.526531in}}%
\pgfpathlineto{\pgfqpoint{1.306260in}{1.526531in}}%
\pgfpathlineto{\pgfqpoint{1.306284in}{1.584255in}}%
\pgfpathlineto{\pgfqpoint{1.313801in}{1.584255in}}%
\pgfpathlineto{\pgfqpoint{1.313923in}{1.641980in}}%
\pgfpathlineto{\pgfqpoint{1.320346in}{1.641980in}}%
\pgfpathlineto{\pgfqpoint{1.323220in}{1.757432in}}%
\pgfpathlineto{\pgfqpoint{1.330054in}{1.757432in}}%
\pgfpathlineto{\pgfqpoint{1.330079in}{1.815159in}}%
\pgfpathlineto{\pgfqpoint{1.336308in}{1.815159in}}%
\pgfpathlineto{\pgfqpoint{1.339011in}{1.930616in}}%
\pgfpathlineto{\pgfqpoint{1.343762in}{1.930616in}}%
\pgfpathlineto{\pgfqpoint{1.346675in}{2.046077in}}%
\pgfpathlineto{\pgfqpoint{1.351784in}{2.046077in}}%
\pgfpathlineto{\pgfqpoint{1.356022in}{2.161541in}}%
\pgfpathlineto{\pgfqpoint{1.361968in}{2.161541in}}%
\pgfpathlineto{\pgfqpoint{1.362115in}{2.219274in}}%
\pgfpathlineto{\pgfqpoint{1.368051in}{2.219274in}}%
\pgfpathlineto{\pgfqpoint{1.369903in}{2.277008in}}%
\pgfpathlineto{\pgfqpoint{1.378045in}{2.277008in}}%
\pgfpathlineto{\pgfqpoint{1.378168in}{2.334743in}}%
\pgfpathlineto{\pgfqpoint{1.403201in}{2.334743in}}%
\pgfpathlineto{\pgfqpoint{1.405503in}{2.277008in}}%
\pgfpathlineto{\pgfqpoint{1.406633in}{2.334743in}}%
\pgfpathlineto{\pgfqpoint{1.410442in}{2.277008in}}%
\pgfpathlineto{\pgfqpoint{1.417651in}{2.219274in}}%
\pgfpathlineto{\pgfqpoint{1.417896in}{2.277008in}}%
\pgfpathlineto{\pgfqpoint{1.426182in}{2.277008in}}%
\pgfpathlineto{\pgfqpoint{1.426279in}{2.219274in}}%
\pgfpathlineto{\pgfqpoint{1.450281in}{2.219274in}}%
\pgfpathlineto{\pgfqpoint{1.453180in}{2.277008in}}%
\pgfpathlineto{\pgfqpoint{1.454099in}{2.219274in}}%
\pgfpathlineto{\pgfqpoint{1.458069in}{2.277008in}}%
\pgfpathlineto{\pgfqpoint{1.462000in}{2.219274in}}%
\pgfpathlineto{\pgfqpoint{1.466251in}{2.277008in}}%
\pgfpathlineto{\pgfqpoint{1.470033in}{2.219274in}}%
\pgfpathlineto{\pgfqpoint{1.474446in}{2.277008in}}%
\pgfpathlineto{\pgfqpoint{1.481776in}{2.277008in}}%
\pgfpathlineto{\pgfqpoint{1.481888in}{2.219274in}}%
\pgfpathlineto{\pgfqpoint{1.487613in}{2.277008in}}%
\pgfpathlineto{\pgfqpoint{1.489654in}{2.219274in}}%
\pgfpathlineto{\pgfqpoint{1.493399in}{2.277008in}}%
\pgfpathlineto{\pgfqpoint{1.497821in}{2.219274in}}%
\pgfpathlineto{\pgfqpoint{1.502048in}{2.277008in}}%
\pgfpathlineto{\pgfqpoint{1.506141in}{2.219274in}}%
\pgfpathlineto{\pgfqpoint{1.509898in}{2.277008in}}%
\pgfpathlineto{\pgfqpoint{1.513766in}{2.219274in}}%
\pgfpathlineto{\pgfqpoint{1.519208in}{2.277008in}}%
\pgfpathlineto{\pgfqpoint{1.532831in}{2.277008in}}%
\pgfpathlineto{\pgfqpoint{1.535539in}{2.334743in}}%
\pgfpathlineto{\pgfqpoint{1.538083in}{2.277008in}}%
\pgfpathlineto{\pgfqpoint{1.542567in}{2.334743in}}%
\pgfpathlineto{\pgfqpoint{1.555502in}{2.334743in}}%
\pgfpathlineto{\pgfqpoint{1.555697in}{2.392479in}}%
\pgfpathlineto{\pgfqpoint{1.560552in}{2.334743in}}%
\pgfpathlineto{\pgfqpoint{1.563952in}{2.392479in}}%
\pgfpathlineto{\pgfqpoint{1.575129in}{2.392479in}}%
\pgfpathlineto{\pgfqpoint{1.577180in}{2.450216in}}%
\pgfpathlineto{\pgfqpoint{1.579824in}{2.392479in}}%
\pgfpathlineto{\pgfqpoint{1.583592in}{2.450216in}}%
\pgfpathlineto{\pgfqpoint{1.590386in}{2.450216in}}%
\pgfpathlineto{\pgfqpoint{1.590840in}{2.392479in}}%
\pgfpathlineto{\pgfqpoint{1.602267in}{2.392479in}}%
\pgfpathlineto{\pgfqpoint{1.604376in}{2.334743in}}%
\pgfpathlineto{\pgfqpoint{1.606467in}{2.392479in}}%
\pgfpathlineto{\pgfqpoint{1.614008in}{2.334743in}}%
\pgfpathlineto{\pgfqpoint{1.614130in}{2.277008in}}%
\pgfpathlineto{\pgfqpoint{1.621700in}{2.277008in}}%
\pgfpathlineto{\pgfqpoint{1.621773in}{2.219274in}}%
\pgfpathlineto{\pgfqpoint{1.629682in}{2.219274in}}%
\pgfpathlineto{\pgfqpoint{1.629916in}{2.161541in}}%
\pgfpathlineto{\pgfqpoint{1.635646in}{2.161541in}}%
\pgfpathlineto{\pgfqpoint{1.637729in}{2.103808in}}%
\pgfpathlineto{\pgfqpoint{1.643702in}{2.103808in}}%
\pgfpathlineto{\pgfqpoint{1.643813in}{2.046077in}}%
\pgfpathlineto{\pgfqpoint{1.649877in}{2.046077in}}%
\pgfpathlineto{\pgfqpoint{1.649926in}{1.988346in}}%
\pgfpathlineto{\pgfqpoint{1.657999in}{1.988346in}}%
\pgfpathlineto{\pgfqpoint{1.658073in}{1.930616in}}%
\pgfpathlineto{\pgfqpoint{1.668278in}{1.930616in}}%
\pgfpathlineto{\pgfqpoint{1.668713in}{1.872887in}}%
\pgfpathlineto{\pgfqpoint{1.692713in}{1.872887in}}%
\pgfpathlineto{\pgfqpoint{1.692786in}{1.930616in}}%
\pgfpathlineto{\pgfqpoint{1.702080in}{1.930616in}}%
\pgfpathlineto{\pgfqpoint{1.702704in}{1.988346in}}%
\pgfpathlineto{\pgfqpoint{1.709317in}{1.988346in}}%
\pgfpathlineto{\pgfqpoint{1.713998in}{2.103808in}}%
\pgfpathlineto{\pgfqpoint{1.719241in}{2.103808in}}%
\pgfpathlineto{\pgfqpoint{1.722449in}{2.219274in}}%
\pgfpathlineto{\pgfqpoint{1.727668in}{2.219274in}}%
\pgfpathlineto{\pgfqpoint{1.731147in}{2.334743in}}%
\pgfpathlineto{\pgfqpoint{1.735407in}{2.334743in}}%
\pgfpathlineto{\pgfqpoint{1.739119in}{2.450216in}}%
\pgfpathlineto{\pgfqpoint{1.745348in}{2.450216in}}%
\pgfpathlineto{\pgfqpoint{1.747084in}{2.507953in}}%
\pgfpathlineto{\pgfqpoint{1.759255in}{2.507953in}}%
\pgfpathlineto{\pgfqpoint{1.759747in}{2.450216in}}%
\pgfpathlineto{\pgfqpoint{1.762745in}{2.507953in}}%
\pgfpathlineto{\pgfqpoint{1.767997in}{2.392479in}}%
\pgfpathlineto{\pgfqpoint{1.769053in}{2.450216in}}%
\pgfpathlineto{\pgfqpoint{1.773505in}{2.334743in}}%
\pgfpathlineto{\pgfqpoint{1.777154in}{2.334743in}}%
\pgfpathlineto{\pgfqpoint{1.780595in}{2.219274in}}%
\pgfpathlineto{\pgfqpoint{1.784069in}{2.219274in}}%
\pgfpathlineto{\pgfqpoint{1.786563in}{2.103808in}}%
\pgfpathlineto{\pgfqpoint{1.793341in}{2.046077in}}%
\pgfpathlineto{\pgfqpoint{1.795763in}{1.930616in}}%
\pgfpathlineto{\pgfqpoint{1.799043in}{1.930616in}}%
\pgfpathlineto{\pgfqpoint{1.804720in}{1.757432in}}%
\pgfpathlineto{\pgfqpoint{1.807789in}{1.757432in}}%
\pgfpathlineto{\pgfqpoint{1.813972in}{1.584255in}}%
\pgfpathlineto{\pgfqpoint{1.818934in}{1.584255in}}%
\pgfpathlineto{\pgfqpoint{1.821833in}{1.468808in}}%
\pgfpathlineto{\pgfqpoint{1.827653in}{1.468808in}}%
\pgfpathlineto{\pgfqpoint{1.827826in}{1.411086in}}%
\pgfpathlineto{\pgfqpoint{1.835897in}{1.411086in}}%
\pgfpathlineto{\pgfqpoint{1.836473in}{1.353364in}}%
\pgfpathlineto{\pgfqpoint{1.848021in}{1.353364in}}%
\pgfpathlineto{\pgfqpoint{1.848046in}{1.411086in}}%
\pgfpathlineto{\pgfqpoint{1.855662in}{1.411086in}}%
\pgfpathlineto{\pgfqpoint{1.858723in}{1.526531in}}%
\pgfpathlineto{\pgfqpoint{1.865147in}{1.584255in}}%
\pgfpathlineto{\pgfqpoint{1.868002in}{1.699705in}}%
\pgfpathlineto{\pgfqpoint{1.873674in}{1.757432in}}%
\pgfpathlineto{\pgfqpoint{1.876140in}{1.872887in}}%
\pgfpathlineto{\pgfqpoint{1.885250in}{2.046077in}}%
\pgfpathlineto{\pgfqpoint{1.889080in}{2.219274in}}%
\pgfpathlineto{\pgfqpoint{1.891434in}{2.219274in}}%
\pgfpathlineto{\pgfqpoint{1.893536in}{2.334743in}}%
\pgfpathlineto{\pgfqpoint{1.898622in}{2.392479in}}%
\pgfpathlineto{\pgfqpoint{1.903526in}{2.565692in}}%
\pgfpathlineto{\pgfqpoint{1.908803in}{2.623431in}}%
\pgfpathlineto{\pgfqpoint{1.911865in}{2.738912in}}%
\pgfpathlineto{\pgfqpoint{1.916154in}{2.738912in}}%
\pgfpathlineto{\pgfqpoint{1.919373in}{2.854397in}}%
\pgfpathlineto{\pgfqpoint{1.925528in}{2.854397in}}%
\pgfpathlineto{\pgfqpoint{1.927542in}{2.912140in}}%
\pgfpathlineto{\pgfqpoint{1.938560in}{2.912140in}}%
\pgfpathlineto{\pgfqpoint{1.941040in}{2.854397in}}%
\pgfpathlineto{\pgfqpoint{1.947541in}{2.854397in}}%
\pgfpathlineto{\pgfqpoint{1.947566in}{2.796654in}}%
\pgfpathlineto{\pgfqpoint{1.952331in}{2.796654in}}%
\pgfpathlineto{\pgfqpoint{1.955427in}{2.681171in}}%
\pgfpathlineto{\pgfqpoint{1.962185in}{2.623431in}}%
\pgfpathlineto{\pgfqpoint{1.964612in}{2.507953in}}%
\pgfpathlineto{\pgfqpoint{1.967364in}{2.507953in}}%
\pgfpathlineto{\pgfqpoint{1.972206in}{2.334743in}}%
\pgfpathlineto{\pgfqpoint{1.975511in}{2.334743in}}%
\pgfpathlineto{\pgfqpoint{1.980926in}{2.161541in}}%
\pgfpathlineto{\pgfqpoint{1.984393in}{2.161541in}}%
\pgfpathlineto{\pgfqpoint{1.987463in}{2.046077in}}%
\pgfpathlineto{\pgfqpoint{1.995199in}{1.872887in}}%
\pgfpathlineto{\pgfqpoint{2.002202in}{1.815159in}}%
\pgfpathlineto{\pgfqpoint{2.005101in}{1.699705in}}%
\pgfpathlineto{\pgfqpoint{2.011943in}{1.641980in}}%
\pgfpathlineto{\pgfqpoint{2.016320in}{1.526531in}}%
\pgfpathlineto{\pgfqpoint{2.025841in}{1.526531in}}%
\pgfpathlineto{\pgfqpoint{2.026396in}{1.468808in}}%
\pgfpathlineto{\pgfqpoint{2.035740in}{1.468808in}}%
\pgfpathlineto{\pgfqpoint{2.035789in}{1.526531in}}%
\pgfpathlineto{\pgfqpoint{2.043666in}{1.526531in}}%
\pgfpathlineto{\pgfqpoint{2.043984in}{1.584255in}}%
\pgfpathlineto{\pgfqpoint{2.050187in}{1.584255in}}%
\pgfpathlineto{\pgfqpoint{2.053346in}{1.699705in}}%
\pgfpathlineto{\pgfqpoint{2.057682in}{1.699705in}}%
\pgfpathlineto{\pgfqpoint{2.060711in}{1.815159in}}%
\pgfpathlineto{\pgfqpoint{2.064155in}{1.815159in}}%
\pgfpathlineto{\pgfqpoint{2.069321in}{1.988346in}}%
\pgfpathlineto{\pgfqpoint{2.071946in}{1.988346in}}%
\pgfpathlineto{\pgfqpoint{2.076637in}{2.161541in}}%
\pgfpathlineto{\pgfqpoint{2.079511in}{2.161541in}}%
\pgfpathlineto{\pgfqpoint{2.084907in}{2.334743in}}%
\pgfpathlineto{\pgfqpoint{2.088570in}{2.334743in}}%
\pgfpathlineto{\pgfqpoint{2.091590in}{2.450216in}}%
\pgfpathlineto{\pgfqpoint{2.095830in}{2.450216in}}%
\pgfpathlineto{\pgfqpoint{2.099906in}{2.565692in}}%
\pgfpathlineto{\pgfqpoint{2.104170in}{2.623431in}}%
\pgfpathlineto{\pgfqpoint{2.123458in}{2.623431in}}%
\pgfpathlineto{\pgfqpoint{2.125657in}{2.565692in}}%
\pgfpathlineto{\pgfqpoint{2.131120in}{2.565692in}}%
\pgfpathlineto{\pgfqpoint{2.131488in}{2.507953in}}%
\pgfpathlineto{\pgfqpoint{2.137207in}{2.507953in}}%
\pgfpathlineto{\pgfqpoint{2.137329in}{2.450216in}}%
\pgfpathlineto{\pgfqpoint{2.143558in}{2.450216in}}%
\pgfpathlineto{\pgfqpoint{2.143754in}{2.392479in}}%
\pgfpathlineto{\pgfqpoint{2.148972in}{2.392479in}}%
\pgfpathlineto{\pgfqpoint{2.152905in}{2.277008in}}%
\pgfpathlineto{\pgfqpoint{2.161604in}{2.219274in}}%
\pgfpathlineto{\pgfqpoint{2.162181in}{2.161541in}}%
\pgfpathlineto{\pgfqpoint{2.169697in}{2.103808in}}%
\pgfpathlineto{\pgfqpoint{2.173222in}{1.988346in}}%
\pgfpathlineto{\pgfqpoint{2.177461in}{1.988346in}}%
\pgfpathlineto{\pgfqpoint{2.180770in}{1.872887in}}%
\pgfpathlineto{\pgfqpoint{2.184459in}{1.872887in}}%
\pgfpathlineto{\pgfqpoint{2.187905in}{1.757432in}}%
\pgfpathlineto{\pgfqpoint{2.193199in}{1.757432in}}%
\pgfpathlineto{\pgfqpoint{2.196353in}{1.641980in}}%
\pgfpathlineto{\pgfqpoint{2.203261in}{1.641980in}}%
\pgfpathlineto{\pgfqpoint{2.203964in}{1.584255in}}%
\pgfpathlineto{\pgfqpoint{2.211823in}{1.584255in}}%
\pgfpathlineto{\pgfqpoint{2.212564in}{1.526531in}}%
\pgfpathlineto{\pgfqpoint{2.230361in}{1.526531in}}%
\pgfpathlineto{\pgfqpoint{2.231487in}{1.584255in}}%
\pgfpathlineto{\pgfqpoint{2.238939in}{1.584255in}}%
\pgfpathlineto{\pgfqpoint{2.239110in}{1.641980in}}%
\pgfpathlineto{\pgfqpoint{2.244600in}{1.641980in}}%
\pgfpathlineto{\pgfqpoint{2.247977in}{1.757432in}}%
\pgfpathlineto{\pgfqpoint{2.252472in}{1.757432in}}%
\pgfpathlineto{\pgfqpoint{2.255430in}{1.872887in}}%
\pgfpathlineto{\pgfqpoint{2.259791in}{1.872887in}}%
\pgfpathlineto{\pgfqpoint{2.259950in}{1.930616in}}%
\pgfpathlineto{\pgfqpoint{2.263625in}{1.930616in}}%
\pgfpathlineto{\pgfqpoint{2.266496in}{2.046077in}}%
\pgfpathlineto{\pgfqpoint{2.271335in}{2.046077in}}%
\pgfpathlineto{\pgfqpoint{2.274663in}{2.161541in}}%
\pgfpathlineto{\pgfqpoint{2.280174in}{2.161541in}}%
\pgfpathlineto{\pgfqpoint{2.284011in}{2.277008in}}%
\pgfpathlineto{\pgfqpoint{2.291457in}{2.277008in}}%
\pgfpathlineto{\pgfqpoint{2.291506in}{2.334743in}}%
\pgfpathlineto{\pgfqpoint{2.299441in}{2.334743in}}%
\pgfpathlineto{\pgfqpoint{2.299601in}{2.392479in}}%
\pgfpathlineto{\pgfqpoint{2.303753in}{2.334743in}}%
\pgfpathlineto{\pgfqpoint{2.307178in}{2.392479in}}%
\pgfpathlineto{\pgfqpoint{2.311139in}{2.334743in}}%
\pgfpathlineto{\pgfqpoint{2.318806in}{2.334743in}}%
\pgfpathlineto{\pgfqpoint{2.319085in}{2.277008in}}%
\pgfpathlineto{\pgfqpoint{2.325629in}{2.277008in}}%
\pgfpathlineto{\pgfqpoint{2.329342in}{2.161541in}}%
\pgfpathlineto{\pgfqpoint{2.333779in}{2.161541in}}%
\pgfpathlineto{\pgfqpoint{2.337168in}{2.046077in}}%
\pgfpathlineto{\pgfqpoint{2.341467in}{2.046077in}}%
\pgfpathlineto{\pgfqpoint{2.344243in}{1.930616in}}%
\pgfpathlineto{\pgfqpoint{2.349207in}{1.930616in}}%
\pgfpathlineto{\pgfqpoint{2.352266in}{1.815159in}}%
\pgfpathlineto{\pgfqpoint{2.356587in}{1.815159in}}%
\pgfpathlineto{\pgfqpoint{2.363199in}{1.641980in}}%
\pgfpathlineto{\pgfqpoint{2.368174in}{1.641980in}}%
\pgfpathlineto{\pgfqpoint{2.371281in}{1.526531in}}%
\pgfpathlineto{\pgfqpoint{2.376708in}{1.526531in}}%
\pgfpathlineto{\pgfqpoint{2.377383in}{1.468808in}}%
\pgfpathlineto{\pgfqpoint{2.382791in}{1.468808in}}%
\pgfpathlineto{\pgfqpoint{2.384952in}{1.411086in}}%
\pgfpathlineto{\pgfqpoint{2.385862in}{1.468808in}}%
\pgfpathlineto{\pgfqpoint{2.389341in}{1.411086in}}%
\pgfpathlineto{\pgfqpoint{2.395899in}{1.353364in}}%
\pgfpathlineto{\pgfqpoint{2.397388in}{1.411086in}}%
\pgfpathlineto{\pgfqpoint{2.401624in}{1.353364in}}%
\pgfpathlineto{\pgfqpoint{2.408892in}{1.353364in}}%
\pgfpathlineto{\pgfqpoint{2.408954in}{1.411086in}}%
\pgfpathlineto{\pgfqpoint{2.416219in}{1.411086in}}%
\pgfpathlineto{\pgfqpoint{2.417507in}{1.468808in}}%
\pgfpathlineto{\pgfqpoint{2.423675in}{1.468808in}}%
\pgfpathlineto{\pgfqpoint{2.423699in}{1.526531in}}%
\pgfpathlineto{\pgfqpoint{2.428935in}{1.526531in}}%
\pgfpathlineto{\pgfqpoint{2.431937in}{1.641980in}}%
\pgfpathlineto{\pgfqpoint{2.436291in}{1.641980in}}%
\pgfpathlineto{\pgfqpoint{2.439818in}{1.757432in}}%
\pgfpathlineto{\pgfqpoint{2.447153in}{1.815159in}}%
\pgfpathlineto{\pgfqpoint{2.450099in}{1.930616in}}%
\pgfpathlineto{\pgfqpoint{2.454499in}{1.930616in}}%
\pgfpathlineto{\pgfqpoint{2.457497in}{2.046077in}}%
\pgfpathlineto{\pgfqpoint{2.463684in}{2.046077in}}%
\pgfpathlineto{\pgfqpoint{2.468860in}{2.161541in}}%
\pgfpathlineto{\pgfqpoint{2.482176in}{2.161541in}}%
\pgfpathlineto{\pgfqpoint{2.484929in}{2.219274in}}%
\pgfpathlineto{\pgfqpoint{2.486612in}{2.161541in}}%
\pgfpathlineto{\pgfqpoint{2.494759in}{2.161541in}}%
\pgfpathlineto{\pgfqpoint{2.495567in}{2.103808in}}%
\pgfpathlineto{\pgfqpoint{2.501634in}{2.103808in}}%
\pgfpathlineto{\pgfqpoint{2.502899in}{2.046077in}}%
\pgfpathlineto{\pgfqpoint{2.504825in}{2.103808in}}%
\pgfpathlineto{\pgfqpoint{2.508852in}{1.988346in}}%
\pgfpathlineto{\pgfqpoint{2.515029in}{1.988346in}}%
\pgfpathlineto{\pgfqpoint{2.518555in}{1.872887in}}%
\pgfpathlineto{\pgfqpoint{2.523293in}{1.872887in}}%
\pgfpathlineto{\pgfqpoint{2.523318in}{1.815159in}}%
\pgfpathlineto{\pgfqpoint{2.528159in}{1.815159in}}%
\pgfpathlineto{\pgfqpoint{2.531758in}{1.699705in}}%
\pgfpathlineto{\pgfqpoint{2.537912in}{1.699705in}}%
\pgfpathlineto{\pgfqpoint{2.542520in}{1.584255in}}%
\pgfpathlineto{\pgfqpoint{2.542582in}{1.641980in}}%
\pgfpathlineto{\pgfqpoint{2.546414in}{1.584255in}}%
\pgfpathlineto{\pgfqpoint{2.558139in}{1.584255in}}%
\pgfpathlineto{\pgfqpoint{2.562212in}{1.526531in}}%
\pgfpathlineto{\pgfqpoint{2.562812in}{1.584255in}}%
\pgfpathlineto{\pgfqpoint{2.571145in}{1.584255in}}%
\pgfpathlineto{\pgfqpoint{2.574128in}{1.641980in}}%
\pgfpathlineto{\pgfqpoint{2.577789in}{1.584255in}}%
\pgfpathlineto{\pgfqpoint{2.582862in}{1.584255in}}%
\pgfpathlineto{\pgfqpoint{2.584542in}{1.699705in}}%
\pgfpathlineto{\pgfqpoint{2.590955in}{1.699705in}}%
\pgfpathlineto{\pgfqpoint{2.595725in}{1.815159in}}%
\pgfpathlineto{\pgfqpoint{2.596951in}{1.757432in}}%
\pgfpathlineto{\pgfqpoint{2.600473in}{1.872887in}}%
\pgfpathlineto{\pgfqpoint{2.606914in}{1.872887in}}%
\pgfpathlineto{\pgfqpoint{2.610169in}{1.988346in}}%
\pgfpathlineto{\pgfqpoint{2.614480in}{2.046077in}}%
\pgfpathlineto{\pgfqpoint{2.619063in}{1.988346in}}%
\pgfpathlineto{\pgfqpoint{2.620191in}{2.046077in}}%
\pgfpathlineto{\pgfqpoint{2.629290in}{2.046077in}}%
\pgfpathlineto{\pgfqpoint{2.631210in}{2.103808in}}%
\pgfpathlineto{\pgfqpoint{2.632363in}{2.046077in}}%
\pgfpathlineto{\pgfqpoint{2.636015in}{2.103808in}}%
\pgfpathlineto{\pgfqpoint{2.643777in}{2.103808in}}%
\pgfpathlineto{\pgfqpoint{2.644256in}{2.046077in}}%
\pgfpathlineto{\pgfqpoint{2.648030in}{2.103808in}}%
\pgfpathlineto{\pgfqpoint{2.651556in}{2.046077in}}%
\pgfpathlineto{\pgfqpoint{2.656321in}{1.988346in}}%
\pgfpathlineto{\pgfqpoint{2.659294in}{2.046077in}}%
\pgfpathlineto{\pgfqpoint{2.663128in}{1.988346in}}%
\pgfpathlineto{\pgfqpoint{2.671296in}{1.988346in}}%
\pgfpathlineto{\pgfqpoint{2.671395in}{1.930616in}}%
\pgfpathlineto{\pgfqpoint{2.677355in}{1.930616in}}%
\pgfpathlineto{\pgfqpoint{2.682670in}{1.815159in}}%
\pgfpathlineto{\pgfqpoint{2.684712in}{1.872887in}}%
\pgfpathlineto{\pgfqpoint{2.689119in}{1.757432in}}%
\pgfpathlineto{\pgfqpoint{2.690965in}{1.815159in}}%
\pgfpathlineto{\pgfqpoint{2.698396in}{1.815159in}}%
\pgfpathlineto{\pgfqpoint{2.698483in}{1.757432in}}%
\pgfpathlineto{\pgfqpoint{2.709821in}{1.757432in}}%
\pgfpathlineto{\pgfqpoint{2.712301in}{1.815159in}}%
\pgfpathlineto{\pgfqpoint{2.713637in}{1.757432in}}%
\pgfpathlineto{\pgfqpoint{2.717460in}{1.815159in}}%
\pgfpathlineto{\pgfqpoint{2.725469in}{1.815159in}}%
\pgfpathlineto{\pgfqpoint{2.725973in}{1.872887in}}%
\pgfpathlineto{\pgfqpoint{2.733947in}{1.872887in}}%
\pgfpathlineto{\pgfqpoint{2.734143in}{1.930616in}}%
\pgfpathlineto{\pgfqpoint{2.739671in}{1.930616in}}%
\pgfpathlineto{\pgfqpoint{2.743821in}{2.046077in}}%
\pgfpathlineto{\pgfqpoint{2.751725in}{2.046077in}}%
\pgfpathlineto{\pgfqpoint{2.751798in}{2.103808in}}%
\pgfpathlineto{\pgfqpoint{2.755971in}{2.103808in}}%
\pgfpathlineto{\pgfqpoint{2.760276in}{2.219274in}}%
\pgfpathlineto{\pgfqpoint{2.764855in}{2.219274in}}%
\pgfpathlineto{\pgfqpoint{2.768659in}{2.334743in}}%
\pgfpathlineto{\pgfqpoint{2.774699in}{2.334743in}}%
\pgfpathlineto{\pgfqpoint{2.774773in}{2.392479in}}%
\pgfpathlineto{\pgfqpoint{2.780642in}{2.392479in}}%
\pgfpathlineto{\pgfqpoint{2.782107in}{2.450216in}}%
\pgfpathlineto{\pgfqpoint{2.788293in}{2.450216in}}%
\pgfpathlineto{\pgfqpoint{2.789563in}{2.507953in}}%
\pgfpathlineto{\pgfqpoint{2.795864in}{2.507953in}}%
\pgfpathlineto{\pgfqpoint{2.796981in}{2.565692in}}%
\pgfpathlineto{\pgfqpoint{2.804275in}{2.565692in}}%
\pgfpathlineto{\pgfqpoint{2.805397in}{2.623431in}}%
\pgfpathlineto{\pgfqpoint{2.819056in}{2.623431in}}%
\pgfpathlineto{\pgfqpoint{2.819606in}{2.681171in}}%
\pgfpathlineto{\pgfqpoint{2.828013in}{2.623431in}}%
\pgfpathlineto{\pgfqpoint{2.828259in}{2.681171in}}%
\pgfpathlineto{\pgfqpoint{2.832499in}{2.623431in}}%
\pgfpathlineto{\pgfqpoint{2.836169in}{2.681171in}}%
\pgfpathlineto{\pgfqpoint{2.840568in}{2.623431in}}%
\pgfpathlineto{\pgfqpoint{2.852596in}{2.623431in}}%
\pgfpathlineto{\pgfqpoint{2.852767in}{2.565692in}}%
\pgfpathlineto{\pgfqpoint{2.868515in}{2.565692in}}%
\pgfpathlineto{\pgfqpoint{2.869608in}{2.623431in}}%
\pgfpathlineto{\pgfqpoint{2.876534in}{2.623431in}}%
\pgfpathlineto{\pgfqpoint{2.876657in}{2.681171in}}%
\pgfpathlineto{\pgfqpoint{2.884609in}{2.738912in}}%
\pgfpathlineto{\pgfqpoint{2.894669in}{2.969885in}}%
\pgfpathlineto{\pgfqpoint{2.901831in}{2.969885in}}%
\pgfpathlineto{\pgfqpoint{2.903562in}{3.027630in}}%
\pgfpathlineto{\pgfqpoint{2.909667in}{3.085376in}}%
\pgfpathlineto{\pgfqpoint{2.911412in}{3.027630in}}%
\pgfpathlineto{\pgfqpoint{2.916450in}{3.085376in}}%
\pgfpathlineto{\pgfqpoint{2.920717in}{3.085376in}}%
\pgfpathlineto{\pgfqpoint{2.925676in}{2.969885in}}%
\pgfpathlineto{\pgfqpoint{2.930443in}{2.969885in}}%
\pgfpathlineto{\pgfqpoint{2.933880in}{2.854397in}}%
\pgfpathlineto{\pgfqpoint{2.939016in}{2.796654in}}%
\pgfpathlineto{\pgfqpoint{2.941104in}{2.681171in}}%
\pgfpathlineto{\pgfqpoint{2.945583in}{2.623431in}}%
\pgfpathlineto{\pgfqpoint{2.949608in}{2.450216in}}%
\pgfpathlineto{\pgfqpoint{2.955690in}{2.334743in}}%
\pgfpathlineto{\pgfqpoint{2.956940in}{2.219274in}}%
\pgfpathlineto{\pgfqpoint{2.962190in}{2.103808in}}%
\pgfpathlineto{\pgfqpoint{2.965102in}{1.930616in}}%
\pgfpathlineto{\pgfqpoint{2.968604in}{1.872887in}}%
\pgfpathlineto{\pgfqpoint{2.973562in}{1.641980in}}%
\pgfpathlineto{\pgfqpoint{2.977139in}{1.584255in}}%
\pgfpathlineto{\pgfqpoint{2.980198in}{1.411086in}}%
\pgfpathlineto{\pgfqpoint{2.983872in}{1.353364in}}%
\pgfpathlineto{\pgfqpoint{2.987900in}{1.180205in}}%
\pgfpathlineto{\pgfqpoint{2.990404in}{1.180205in}}%
\pgfpathlineto{\pgfqpoint{2.995351in}{1.007054in}}%
\pgfpathlineto{\pgfqpoint{2.999108in}{1.007054in}}%
\pgfpathlineto{\pgfqpoint{2.999513in}{0.949339in}}%
\pgfpathlineto{\pgfqpoint{3.007400in}{0.891624in}}%
\pgfpathlineto{\pgfqpoint{3.007424in}{0.833910in}}%
\pgfpathlineto{\pgfqpoint{3.014586in}{0.833910in}}%
\pgfpathlineto{\pgfqpoint{3.015017in}{0.776197in}}%
\pgfpathlineto{\pgfqpoint{3.022300in}{0.776197in}}%
\pgfpathlineto{\pgfqpoint{3.022373in}{0.833910in}}%
\pgfpathlineto{\pgfqpoint{3.028529in}{0.833910in}}%
\pgfpathlineto{\pgfqpoint{3.035742in}{0.949339in}}%
\pgfpathlineto{\pgfqpoint{3.035791in}{1.007054in}}%
\pgfpathlineto{\pgfqpoint{3.041565in}{1.064770in}}%
\pgfpathlineto{\pgfqpoint{3.044204in}{1.180205in}}%
\pgfpathlineto{\pgfqpoint{3.049428in}{1.237924in}}%
\pgfpathlineto{\pgfqpoint{3.056255in}{1.468808in}}%
\pgfpathlineto{\pgfqpoint{3.056475in}{1.411086in}}%
\pgfpathlineto{\pgfqpoint{3.060825in}{1.584255in}}%
\pgfpathlineto{\pgfqpoint{3.065522in}{1.641980in}}%
\pgfpathlineto{\pgfqpoint{3.073308in}{1.872887in}}%
\pgfpathlineto{\pgfqpoint{3.079636in}{1.930616in}}%
\pgfpathlineto{\pgfqpoint{3.082801in}{2.046077in}}%
\pgfpathlineto{\pgfqpoint{3.088884in}{2.046077in}}%
\pgfpathlineto{\pgfqpoint{3.089461in}{2.103808in}}%
\pgfpathlineto{\pgfqpoint{3.109349in}{2.103808in}}%
\pgfpathlineto{\pgfqpoint{3.109374in}{2.046077in}}%
\pgfpathlineto{\pgfqpoint{3.115477in}{2.046077in}}%
\pgfpathlineto{\pgfqpoint{3.115600in}{1.988346in}}%
\pgfpathlineto{\pgfqpoint{3.121353in}{1.930616in}}%
\pgfpathlineto{\pgfqpoint{3.123698in}{1.815159in}}%
\pgfpathlineto{\pgfqpoint{3.126656in}{1.815159in}}%
\pgfpathlineto{\pgfqpoint{3.129277in}{1.699705in}}%
\pgfpathlineto{\pgfqpoint{3.134455in}{1.641980in}}%
\pgfpathlineto{\pgfqpoint{3.138843in}{1.468808in}}%
\pgfpathlineto{\pgfqpoint{3.143942in}{1.411086in}}%
\pgfpathlineto{\pgfqpoint{3.146166in}{1.295644in}}%
\pgfpathlineto{\pgfqpoint{3.155393in}{1.122487in}}%
\pgfpathlineto{\pgfqpoint{3.157309in}{1.007054in}}%
\pgfpathlineto{\pgfqpoint{3.161242in}{0.891624in}}%
\pgfpathlineto{\pgfqpoint{3.164743in}{0.891624in}}%
\pgfpathlineto{\pgfqpoint{3.170652in}{0.718485in}}%
\pgfpathlineto{\pgfqpoint{3.176550in}{0.718485in}}%
\pgfpathlineto{\pgfqpoint{3.176648in}{0.660774in}}%
\pgfpathlineto{\pgfqpoint{3.184607in}{0.660774in}}%
\pgfpathlineto{\pgfqpoint{3.184839in}{0.718485in}}%
\pgfpathlineto{\pgfqpoint{3.193184in}{0.718485in}}%
\pgfpathlineto{\pgfqpoint{3.193281in}{0.776197in}}%
\pgfpathlineto{\pgfqpoint{3.198047in}{0.776197in}}%
\pgfpathlineto{\pgfqpoint{3.200652in}{0.891624in}}%
\pgfpathlineto{\pgfqpoint{3.206402in}{0.949339in}}%
\pgfpathlineto{\pgfqpoint{3.208437in}{1.064770in}}%
\pgfpathlineto{\pgfqpoint{3.212775in}{1.122487in}}%
\pgfpathlineto{\pgfqpoint{3.217875in}{1.353364in}}%
\pgfpathlineto{\pgfqpoint{3.219704in}{1.353364in}}%
\pgfpathlineto{\pgfqpoint{3.225048in}{1.584255in}}%
\pgfpathlineto{\pgfqpoint{3.228756in}{1.641980in}}%
\pgfpathlineto{\pgfqpoint{3.233623in}{1.872887in}}%
\pgfpathlineto{\pgfqpoint{3.236019in}{1.872887in}}%
\pgfpathlineto{\pgfqpoint{3.239235in}{2.046077in}}%
\pgfpathlineto{\pgfqpoint{3.241198in}{2.046077in}}%
\pgfpathlineto{\pgfqpoint{3.245053in}{2.219274in}}%
\pgfpathlineto{\pgfqpoint{3.247469in}{2.219274in}}%
\pgfpathlineto{\pgfqpoint{3.251931in}{2.392479in}}%
\pgfpathlineto{\pgfqpoint{3.258306in}{2.507953in}}%
\pgfpathlineto{\pgfqpoint{3.263671in}{2.507953in}}%
\pgfpathlineto{\pgfqpoint{3.264412in}{2.565692in}}%
\pgfpathlineto{\pgfqpoint{3.280824in}{2.565692in}}%
\pgfpathlineto{\pgfqpoint{3.281204in}{2.507953in}}%
\pgfpathlineto{\pgfqpoint{3.286573in}{2.507953in}}%
\pgfpathlineto{\pgfqpoint{3.290077in}{2.392479in}}%
\pgfpathlineto{\pgfqpoint{3.293809in}{2.392479in}}%
\pgfpathlineto{\pgfqpoint{3.297278in}{2.277008in}}%
\pgfpathlineto{\pgfqpoint{3.303099in}{2.219274in}}%
\pgfpathlineto{\pgfqpoint{3.305429in}{2.103808in}}%
\pgfpathlineto{\pgfqpoint{3.310099in}{2.046077in}}%
\pgfpathlineto{\pgfqpoint{3.315993in}{1.815159in}}%
\pgfpathlineto{\pgfqpoint{3.318696in}{1.815159in}}%
\pgfpathlineto{\pgfqpoint{3.324335in}{1.584255in}}%
\pgfpathlineto{\pgfqpoint{3.329032in}{1.526531in}}%
\pgfpathlineto{\pgfqpoint{3.331633in}{1.411086in}}%
\pgfpathlineto{\pgfqpoint{3.335540in}{1.411086in}}%
\pgfpathlineto{\pgfqpoint{3.338808in}{1.295644in}}%
\pgfpathlineto{\pgfqpoint{3.344188in}{1.295644in}}%
\pgfpathlineto{\pgfqpoint{3.347813in}{1.180205in}}%
\pgfpathlineto{\pgfqpoint{3.358451in}{1.180205in}}%
\pgfpathlineto{\pgfqpoint{3.359164in}{1.237924in}}%
\pgfpathlineto{\pgfqpoint{3.366436in}{1.237924in}}%
\pgfpathlineto{\pgfqpoint{3.367369in}{1.295644in}}%
\pgfpathlineto{\pgfqpoint{3.371410in}{1.295644in}}%
\pgfpathlineto{\pgfqpoint{3.374690in}{1.411086in}}%
\pgfpathlineto{\pgfqpoint{3.380781in}{1.468808in}}%
\pgfpathlineto{\pgfqpoint{3.383301in}{1.584255in}}%
\pgfpathlineto{\pgfqpoint{3.387609in}{1.641980in}}%
\pgfpathlineto{\pgfqpoint{3.391519in}{1.815159in}}%
\pgfpathlineto{\pgfqpoint{3.395635in}{1.872887in}}%
\pgfpathlineto{\pgfqpoint{3.399136in}{2.046077in}}%
\pgfpathlineto{\pgfqpoint{3.406127in}{2.219274in}}%
\pgfpathlineto{\pgfqpoint{3.407597in}{2.334743in}}%
\pgfpathlineto{\pgfqpoint{3.413900in}{2.392479in}}%
\pgfpathlineto{\pgfqpoint{3.415378in}{2.565692in}}%
\pgfpathlineto{\pgfqpoint{3.421176in}{2.738912in}}%
\pgfpathlineto{\pgfqpoint{3.424196in}{2.738912in}}%
\pgfpathlineto{\pgfqpoint{3.432366in}{2.969885in}}%
\pgfpathlineto{\pgfqpoint{3.433064in}{2.912140in}}%
\pgfpathlineto{\pgfqpoint{3.436826in}{3.027630in}}%
\pgfpathlineto{\pgfqpoint{3.441568in}{2.969885in}}%
\pgfpathlineto{\pgfqpoint{3.444823in}{3.027630in}}%
\pgfpathlineto{\pgfqpoint{3.453596in}{3.027630in}}%
\pgfpathlineto{\pgfqpoint{3.456752in}{2.912140in}}%
\pgfpathlineto{\pgfqpoint{3.461308in}{2.912140in}}%
\pgfpathlineto{\pgfqpoint{3.464638in}{2.796654in}}%
\pgfpathlineto{\pgfqpoint{3.467972in}{2.796654in}}%
\pgfpathlineto{\pgfqpoint{3.472045in}{2.623431in}}%
\pgfpathlineto{\pgfqpoint{3.476543in}{2.565692in}}%
\pgfpathlineto{\pgfqpoint{3.486758in}{2.161541in}}%
\pgfpathlineto{\pgfqpoint{3.490324in}{2.103808in}}%
\pgfpathlineto{\pgfqpoint{3.493676in}{1.930616in}}%
\pgfpathlineto{\pgfqpoint{3.499362in}{1.815159in}}%
\pgfpathlineto{\pgfqpoint{3.502951in}{1.641980in}}%
\pgfpathlineto{\pgfqpoint{3.506599in}{1.584255in}}%
\pgfpathlineto{\pgfqpoint{3.510567in}{1.411086in}}%
\pgfpathlineto{\pgfqpoint{3.515059in}{1.353364in}}%
\pgfpathlineto{\pgfqpoint{3.519560in}{1.180205in}}%
\pgfpathlineto{\pgfqpoint{3.524596in}{1.180205in}}%
\pgfpathlineto{\pgfqpoint{3.527470in}{1.064770in}}%
\pgfpathlineto{\pgfqpoint{3.532213in}{1.064770in}}%
\pgfpathlineto{\pgfqpoint{3.536142in}{0.949339in}}%
\pgfpathlineto{\pgfqpoint{3.542463in}{0.949339in}}%
\pgfpathlineto{\pgfqpoint{3.543710in}{0.891624in}}%
\pgfpathlineto{\pgfqpoint{3.550778in}{0.891624in}}%
\pgfpathlineto{\pgfqpoint{3.550974in}{0.949339in}}%
\pgfpathlineto{\pgfqpoint{3.558969in}{0.949339in}}%
\pgfpathlineto{\pgfqpoint{3.559595in}{1.007054in}}%
\pgfpathlineto{\pgfqpoint{3.567191in}{1.064770in}}%
\pgfpathlineto{\pgfqpoint{3.569727in}{1.180205in}}%
\pgfpathlineto{\pgfqpoint{3.576183in}{1.237924in}}%
\pgfpathlineto{\pgfqpoint{3.578234in}{1.353364in}}%
\pgfpathlineto{\pgfqpoint{3.584463in}{1.411086in}}%
\pgfpathlineto{\pgfqpoint{3.586839in}{1.526531in}}%
\pgfpathlineto{\pgfqpoint{3.592448in}{1.584255in}}%
\pgfpathlineto{\pgfqpoint{3.595226in}{1.699705in}}%
\pgfpathlineto{\pgfqpoint{3.599181in}{1.699705in}}%
\pgfpathlineto{\pgfqpoint{3.603062in}{1.815159in}}%
\pgfpathlineto{\pgfqpoint{3.609527in}{1.815159in}}%
\pgfpathlineto{\pgfqpoint{3.609991in}{1.872887in}}%
\pgfpathlineto{\pgfqpoint{3.617347in}{1.872887in}}%
\pgfpathlineto{\pgfqpoint{3.618566in}{1.930616in}}%
\pgfpathlineto{\pgfqpoint{3.625604in}{1.930616in}}%
\pgfpathlineto{\pgfqpoint{3.625776in}{1.872887in}}%
\pgfpathlineto{\pgfqpoint{3.633637in}{1.872887in}}%
\pgfpathlineto{\pgfqpoint{3.633942in}{1.815159in}}%
\pgfpathlineto{\pgfqpoint{3.640750in}{1.815159in}}%
\pgfpathlineto{\pgfqpoint{3.645224in}{1.699705in}}%
\pgfpathlineto{\pgfqpoint{3.646348in}{1.757432in}}%
\pgfpathlineto{\pgfqpoint{3.650134in}{1.641980in}}%
\pgfpathlineto{\pgfqpoint{3.654740in}{1.641980in}}%
\pgfpathlineto{\pgfqpoint{3.658501in}{1.526531in}}%
\pgfpathlineto{\pgfqpoint{3.662063in}{1.526531in}}%
\pgfpathlineto{\pgfqpoint{3.667960in}{1.353364in}}%
\pgfpathlineto{\pgfqpoint{3.672445in}{1.353364in}}%
\pgfpathlineto{\pgfqpoint{3.675344in}{1.237924in}}%
\pgfpathlineto{\pgfqpoint{3.680088in}{1.237924in}}%
\pgfpathlineto{\pgfqpoint{3.683171in}{1.122487in}}%
\pgfpathlineto{\pgfqpoint{3.688759in}{1.122487in}}%
\pgfpathlineto{\pgfqpoint{3.688783in}{1.064770in}}%
\pgfpathlineto{\pgfqpoint{3.696827in}{1.064770in}}%
\pgfpathlineto{\pgfqpoint{3.696994in}{1.007054in}}%
\pgfpathlineto{\pgfqpoint{3.708673in}{1.007054in}}%
\pgfpathlineto{\pgfqpoint{3.710753in}{0.949339in}}%
\pgfpathlineto{\pgfqpoint{3.712716in}{1.007054in}}%
\pgfpathlineto{\pgfqpoint{3.720406in}{1.007054in}}%
\pgfpathlineto{\pgfqpoint{3.720834in}{1.064770in}}%
\pgfpathlineto{\pgfqpoint{3.727235in}{1.064770in}}%
\pgfpathlineto{\pgfqpoint{3.727928in}{1.122487in}}%
\pgfpathlineto{\pgfqpoint{3.732580in}{1.122487in}}%
\pgfpathlineto{\pgfqpoint{3.736252in}{1.237924in}}%
\pgfpathlineto{\pgfqpoint{3.737184in}{1.180205in}}%
\pgfpathlineto{\pgfqpoint{3.741426in}{1.295644in}}%
\pgfpathlineto{\pgfqpoint{3.746166in}{1.295644in}}%
\pgfpathlineto{\pgfqpoint{3.749374in}{1.411086in}}%
\pgfpathlineto{\pgfqpoint{3.754175in}{1.411086in}}%
\pgfpathlineto{\pgfqpoint{3.757718in}{1.526531in}}%
\pgfpathlineto{\pgfqpoint{3.762859in}{1.526531in}}%
\pgfpathlineto{\pgfqpoint{3.766240in}{1.641980in}}%
\pgfpathlineto{\pgfqpoint{3.772370in}{1.641980in}}%
\pgfpathlineto{\pgfqpoint{3.772566in}{1.699705in}}%
\pgfpathlineto{\pgfqpoint{3.780995in}{1.699705in}}%
\pgfpathlineto{\pgfqpoint{3.781215in}{1.757432in}}%
\pgfpathlineto{\pgfqpoint{3.801549in}{1.757432in}}%
\pgfpathlineto{\pgfqpoint{3.802827in}{1.815159in}}%
\pgfpathlineto{\pgfqpoint{3.806684in}{1.757432in}}%
\pgfpathlineto{\pgfqpoint{3.812244in}{1.815159in}}%
\pgfpathlineto{\pgfqpoint{3.815833in}{1.757432in}}%
\pgfpathlineto{\pgfqpoint{3.824201in}{1.757432in}}%
\pgfpathlineto{\pgfqpoint{3.824849in}{1.699705in}}%
\pgfpathlineto{\pgfqpoint{3.831891in}{1.699705in}}%
\pgfpathlineto{\pgfqpoint{3.833009in}{1.641980in}}%
\pgfpathlineto{\pgfqpoint{3.839530in}{1.641980in}}%
\pgfpathlineto{\pgfqpoint{3.841717in}{1.584255in}}%
\pgfpathlineto{\pgfqpoint{3.843349in}{1.641980in}}%
\pgfpathlineto{\pgfqpoint{3.850361in}{1.584255in}}%
\pgfpathlineto{\pgfqpoint{3.850606in}{1.526531in}}%
\pgfpathlineto{\pgfqpoint{3.859419in}{1.526531in}}%
\pgfpathlineto{\pgfqpoint{3.862627in}{1.468808in}}%
\pgfpathlineto{\pgfqpoint{3.863020in}{1.526531in}}%
\pgfpathlineto{\pgfqpoint{3.866707in}{1.468808in}}%
\pgfpathlineto{\pgfqpoint{3.882351in}{1.468808in}}%
\pgfpathlineto{\pgfqpoint{3.883432in}{1.526531in}}%
\pgfpathlineto{\pgfqpoint{3.894933in}{1.526531in}}%
\pgfpathlineto{\pgfqpoint{3.898741in}{1.641980in}}%
\pgfpathlineto{\pgfqpoint{3.900324in}{1.584255in}}%
\pgfpathlineto{\pgfqpoint{3.904148in}{1.699705in}}%
\pgfpathlineto{\pgfqpoint{3.910449in}{1.699705in}}%
\pgfpathlineto{\pgfqpoint{3.913239in}{1.815159in}}%
\pgfpathlineto{\pgfqpoint{3.917622in}{1.815159in}}%
\pgfpathlineto{\pgfqpoint{3.922309in}{1.930616in}}%
\pgfpathlineto{\pgfqpoint{3.927253in}{1.930616in}}%
\pgfpathlineto{\pgfqpoint{3.931465in}{2.046077in}}%
\pgfpathlineto{\pgfqpoint{3.937721in}{2.046077in}}%
\pgfpathlineto{\pgfqpoint{3.942745in}{2.161541in}}%
\pgfpathlineto{\pgfqpoint{3.942941in}{2.103808in}}%
\pgfpathlineto{\pgfqpoint{3.951417in}{2.161541in}}%
\pgfpathlineto{\pgfqpoint{3.951441in}{2.219274in}}%
\pgfpathlineto{\pgfqpoint{3.972031in}{2.219274in}}%
\pgfpathlineto{\pgfqpoint{3.974711in}{2.161541in}}%
\pgfpathlineto{\pgfqpoint{3.977032in}{2.219274in}}%
\pgfpathlineto{\pgfqpoint{3.983173in}{2.103808in}}%
\pgfpathlineto{\pgfqpoint{3.983909in}{2.161541in}}%
\pgfpathlineto{\pgfqpoint{3.989463in}{2.046077in}}%
\pgfpathlineto{\pgfqpoint{3.995517in}{2.046077in}}%
\pgfpathlineto{\pgfqpoint{3.997189in}{1.988346in}}%
\pgfpathlineto{\pgfqpoint{4.003418in}{1.988346in}}%
\pgfpathlineto{\pgfqpoint{4.003675in}{1.930616in}}%
\pgfpathlineto{\pgfqpoint{4.010265in}{1.930616in}}%
\pgfpathlineto{\pgfqpoint{4.011418in}{1.872887in}}%
\pgfpathlineto{\pgfqpoint{4.016808in}{1.872887in}}%
\pgfpathlineto{\pgfqpoint{4.021406in}{1.757432in}}%
\pgfpathlineto{\pgfqpoint{4.027163in}{1.757432in}}%
\pgfpathlineto{\pgfqpoint{4.027728in}{1.699705in}}%
\pgfpathlineto{\pgfqpoint{4.035397in}{1.699705in}}%
\pgfpathlineto{\pgfqpoint{4.036388in}{1.641980in}}%
\pgfpathlineto{\pgfqpoint{4.042139in}{1.641980in}}%
\pgfpathlineto{\pgfqpoint{4.045272in}{1.584255in}}%
\pgfpathlineto{\pgfqpoint{4.064130in}{1.584255in}}%
\pgfpathlineto{\pgfqpoint{4.068243in}{1.641980in}}%
\pgfpathlineto{\pgfqpoint{4.071070in}{1.584255in}}%
\pgfpathlineto{\pgfqpoint{4.075295in}{1.699705in}}%
\pgfpathlineto{\pgfqpoint{4.077971in}{1.641980in}}%
\pgfpathlineto{\pgfqpoint{4.084519in}{1.757432in}}%
\pgfpathlineto{\pgfqpoint{4.086003in}{1.699705in}}%
\pgfpathlineto{\pgfqpoint{4.089066in}{1.757432in}}%
\pgfpathlineto{\pgfqpoint{4.094480in}{1.757432in}}%
\pgfpathlineto{\pgfqpoint{4.100635in}{1.872887in}}%
\pgfpathlineto{\pgfqpoint{4.106049in}{1.872887in}}%
\pgfpathlineto{\pgfqpoint{4.106914in}{1.930616in}}%
\pgfpathlineto{\pgfqpoint{4.113262in}{1.930616in}}%
\pgfpathlineto{\pgfqpoint{4.117268in}{2.046077in}}%
\pgfpathlineto{\pgfqpoint{4.118867in}{1.988346in}}%
\pgfpathlineto{\pgfqpoint{4.124076in}{2.103808in}}%
\pgfpathlineto{\pgfqpoint{4.130733in}{2.103808in}}%
\pgfpathlineto{\pgfqpoint{4.130757in}{2.161541in}}%
\pgfpathlineto{\pgfqpoint{4.156800in}{2.161541in}}%
\pgfpathlineto{\pgfqpoint{4.159941in}{2.103808in}}%
\pgfpathlineto{\pgfqpoint{4.164440in}{2.161541in}}%
\pgfpathlineto{\pgfqpoint{4.165595in}{2.103808in}}%
\pgfpathlineto{\pgfqpoint{4.174095in}{2.103808in}}%
\pgfpathlineto{\pgfqpoint{4.174266in}{2.046077in}}%
\pgfpathlineto{\pgfqpoint{4.182940in}{2.046077in}}%
\pgfpathlineto{\pgfqpoint{4.188802in}{1.930616in}}%
\pgfpathlineto{\pgfqpoint{4.193641in}{1.988346in}}%
\pgfpathlineto{\pgfqpoint{4.194699in}{1.930616in}}%
\pgfpathlineto{\pgfqpoint{4.203594in}{1.930616in}}%
\pgfpathlineto{\pgfqpoint{4.203618in}{1.872887in}}%
\pgfpathlineto{\pgfqpoint{4.217457in}{1.872887in}}%
\pgfpathlineto{\pgfqpoint{4.217554in}{1.815159in}}%
\pgfpathlineto{\pgfqpoint{4.253492in}{1.815159in}}%
\pgfpathlineto{\pgfqpoint{4.253638in}{1.872887in}}%
\pgfpathlineto{\pgfqpoint{4.266402in}{1.872887in}}%
\pgfpathlineto{\pgfqpoint{4.268159in}{1.930616in}}%
\pgfpathlineto{\pgfqpoint{4.274672in}{1.930616in}}%
\pgfpathlineto{\pgfqpoint{4.275198in}{1.988346in}}%
\pgfpathlineto{\pgfqpoint{4.283614in}{1.988346in}}%
\pgfpathlineto{\pgfqpoint{4.296097in}{2.103808in}}%
\pgfpathlineto{\pgfqpoint{4.299943in}{2.046077in}}%
\pgfpathlineto{\pgfqpoint{4.304144in}{2.103808in}}%
\pgfpathlineto{\pgfqpoint{4.322365in}{2.103808in}}%
\pgfpathlineto{\pgfqpoint{4.324811in}{2.046077in}}%
\pgfpathlineto{\pgfqpoint{4.331780in}{2.046077in}}%
\pgfpathlineto{\pgfqpoint{4.333941in}{1.988346in}}%
\pgfpathlineto{\pgfqpoint{4.339690in}{1.988346in}}%
\pgfpathlineto{\pgfqpoint{4.339837in}{1.930616in}}%
\pgfpathlineto{\pgfqpoint{4.345759in}{1.930616in}}%
\pgfpathlineto{\pgfqpoint{4.346875in}{1.872887in}}%
\pgfpathlineto{\pgfqpoint{4.351892in}{1.872887in}}%
\pgfpathlineto{\pgfqpoint{4.352817in}{1.815159in}}%
\pgfpathlineto{\pgfqpoint{4.358356in}{1.815159in}}%
\pgfpathlineto{\pgfqpoint{4.361805in}{1.699705in}}%
\pgfpathlineto{\pgfqpoint{4.366451in}{1.699705in}}%
\pgfpathlineto{\pgfqpoint{4.369907in}{1.584255in}}%
\pgfpathlineto{\pgfqpoint{4.375849in}{1.584255in}}%
\pgfpathlineto{\pgfqpoint{4.380233in}{1.468808in}}%
\pgfpathlineto{\pgfqpoint{4.386176in}{1.468808in}}%
\pgfpathlineto{\pgfqpoint{4.389122in}{1.411086in}}%
\pgfpathlineto{\pgfqpoint{4.390011in}{1.468808in}}%
\pgfpathlineto{\pgfqpoint{4.397037in}{1.468808in}}%
\pgfpathlineto{\pgfqpoint{4.398585in}{1.411086in}}%
\pgfpathlineto{\pgfqpoint{4.406324in}{1.411086in}}%
\pgfpathlineto{\pgfqpoint{4.406446in}{1.468808in}}%
\pgfpathlineto{\pgfqpoint{4.414630in}{1.468808in}}%
\pgfpathlineto{\pgfqpoint{4.414750in}{1.526531in}}%
\pgfpathlineto{\pgfqpoint{4.420891in}{1.526531in}}%
\pgfpathlineto{\pgfqpoint{4.424390in}{1.641980in}}%
\pgfpathlineto{\pgfqpoint{4.429040in}{1.641980in}}%
\pgfpathlineto{\pgfqpoint{4.433113in}{1.757432in}}%
\pgfpathlineto{\pgfqpoint{4.438050in}{1.757432in}}%
\pgfpathlineto{\pgfqpoint{4.445239in}{1.930616in}}%
\pgfpathlineto{\pgfqpoint{4.445916in}{1.872887in}}%
\pgfpathlineto{\pgfqpoint{4.453741in}{2.046077in}}%
\pgfpathlineto{\pgfqpoint{4.460853in}{2.046077in}}%
\pgfpathlineto{\pgfqpoint{4.460902in}{2.103808in}}%
\pgfpathlineto{\pgfqpoint{4.470658in}{2.103808in}}%
\pgfpathlineto{\pgfqpoint{4.470682in}{2.161541in}}%
\pgfpathlineto{\pgfqpoint{4.496369in}{2.161541in}}%
\pgfpathlineto{\pgfqpoint{4.496467in}{2.103808in}}%
\pgfpathlineto{\pgfqpoint{4.504776in}{2.103808in}}%
\pgfpathlineto{\pgfqpoint{4.504919in}{2.046077in}}%
\pgfpathlineto{\pgfqpoint{4.510768in}{2.046077in}}%
\pgfpathlineto{\pgfqpoint{4.514168in}{1.930616in}}%
\pgfpathlineto{\pgfqpoint{4.519673in}{1.930616in}}%
\pgfpathlineto{\pgfqpoint{4.523368in}{1.815159in}}%
\pgfpathlineto{\pgfqpoint{4.529400in}{1.815159in}}%
\pgfpathlineto{\pgfqpoint{4.529844in}{1.757432in}}%
\pgfpathlineto{\pgfqpoint{4.534795in}{1.757432in}}%
\pgfpathlineto{\pgfqpoint{4.538700in}{1.641980in}}%
\pgfpathlineto{\pgfqpoint{4.545937in}{1.641980in}}%
\pgfpathlineto{\pgfqpoint{4.545986in}{1.584255in}}%
\pgfpathlineto{\pgfqpoint{4.553172in}{1.584255in}}%
\pgfpathlineto{\pgfqpoint{4.554328in}{1.526531in}}%
\pgfpathlineto{\pgfqpoint{4.562967in}{1.468808in}}%
\pgfpathlineto{\pgfqpoint{4.564120in}{1.526531in}}%
\pgfpathlineto{\pgfqpoint{4.568738in}{1.468808in}}%
\pgfpathlineto{\pgfqpoint{4.574500in}{1.468808in}}%
\pgfpathlineto{\pgfqpoint{4.583402in}{1.526531in}}%
\pgfpathlineto{\pgfqpoint{4.583764in}{1.584255in}}%
\pgfpathlineto{\pgfqpoint{4.588390in}{1.584255in}}%
\pgfpathlineto{\pgfqpoint{4.591509in}{1.699705in}}%
\pgfpathlineto{\pgfqpoint{4.595747in}{1.699705in}}%
\pgfpathlineto{\pgfqpoint{4.598221in}{1.815159in}}%
\pgfpathlineto{\pgfqpoint{4.601983in}{1.815159in}}%
\pgfpathlineto{\pgfqpoint{4.606798in}{1.988346in}}%
\pgfpathlineto{\pgfqpoint{4.613358in}{2.046077in}}%
\pgfpathlineto{\pgfqpoint{4.615533in}{2.161541in}}%
\pgfpathlineto{\pgfqpoint{4.618515in}{2.161541in}}%
\pgfpathlineto{\pgfqpoint{4.624206in}{2.334743in}}%
\pgfpathlineto{\pgfqpoint{4.628284in}{2.334743in}}%
\pgfpathlineto{\pgfqpoint{4.631759in}{2.450216in}}%
\pgfpathlineto{\pgfqpoint{4.637817in}{2.450216in}}%
\pgfpathlineto{\pgfqpoint{4.642919in}{2.565692in}}%
\pgfpathlineto{\pgfqpoint{4.651476in}{2.565692in}}%
\pgfpathlineto{\pgfqpoint{4.651599in}{2.623431in}}%
\pgfpathlineto{\pgfqpoint{4.662730in}{2.623431in}}%
\pgfpathlineto{\pgfqpoint{4.663194in}{2.565692in}}%
\pgfpathlineto{\pgfqpoint{4.671362in}{2.565692in}}%
\pgfpathlineto{\pgfqpoint{4.677494in}{2.507953in}}%
\pgfpathlineto{\pgfqpoint{4.680773in}{2.392479in}}%
\pgfpathlineto{\pgfqpoint{4.684605in}{2.392479in}}%
\pgfpathlineto{\pgfqpoint{4.687549in}{2.277008in}}%
\pgfpathlineto{\pgfqpoint{4.693082in}{2.219274in}}%
\pgfpathlineto{\pgfqpoint{4.697158in}{2.046077in}}%
\pgfpathlineto{\pgfqpoint{4.702020in}{1.988346in}}%
\pgfpathlineto{\pgfqpoint{4.704014in}{1.872887in}}%
\pgfpathlineto{\pgfqpoint{4.709285in}{1.815159in}}%
\pgfpathlineto{\pgfqpoint{4.711371in}{1.699705in}}%
\pgfpathlineto{\pgfqpoint{4.714145in}{1.699705in}}%
\pgfpathlineto{\pgfqpoint{4.716504in}{1.584255in}}%
\pgfpathlineto{\pgfqpoint{4.719831in}{1.584255in}}%
\pgfpathlineto{\pgfqpoint{4.728208in}{1.353364in}}%
\pgfpathlineto{\pgfqpoint{4.728722in}{1.411086in}}%
\pgfpathlineto{\pgfqpoint{4.732332in}{1.295644in}}%
\pgfpathlineto{\pgfqpoint{4.738352in}{1.237924in}}%
\pgfpathlineto{\pgfqpoint{4.756365in}{1.237924in}}%
\pgfpathlineto{\pgfqpoint{4.758039in}{1.295644in}}%
\pgfpathlineto{\pgfqpoint{4.763956in}{1.295644in}}%
\pgfpathlineto{\pgfqpoint{4.767558in}{1.411086in}}%
\pgfpathlineto{\pgfqpoint{4.772061in}{1.411086in}}%
\pgfpathlineto{\pgfqpoint{4.774937in}{1.526531in}}%
\pgfpathlineto{\pgfqpoint{4.778121in}{1.526531in}}%
\pgfpathlineto{\pgfqpoint{4.783132in}{1.699705in}}%
\pgfpathlineto{\pgfqpoint{4.785835in}{1.699705in}}%
\pgfpathlineto{\pgfqpoint{4.790198in}{1.872887in}}%
\pgfpathlineto{\pgfqpoint{4.795321in}{1.930616in}}%
\pgfpathlineto{\pgfqpoint{4.799876in}{2.103808in}}%
\pgfpathlineto{\pgfqpoint{4.802946in}{2.103808in}}%
\pgfpathlineto{\pgfqpoint{4.805775in}{2.219274in}}%
\pgfpathlineto{\pgfqpoint{4.812790in}{2.277008in}}%
\pgfpathlineto{\pgfqpoint{4.816423in}{2.392479in}}%
\pgfpathlineto{\pgfqpoint{4.821941in}{2.392479in}}%
\pgfpathlineto{\pgfqpoint{4.822185in}{2.450216in}}%
\pgfpathlineto{\pgfqpoint{4.830329in}{2.450216in}}%
\pgfpathlineto{\pgfqpoint{4.832205in}{2.507953in}}%
\pgfpathlineto{\pgfqpoint{4.833421in}{2.450216in}}%
\pgfpathlineto{\pgfqpoint{4.836392in}{2.507953in}}%
\pgfpathlineto{\pgfqpoint{4.839412in}{2.450216in}}%
\pgfpathlineto{\pgfqpoint{4.846057in}{2.450216in}}%
\pgfpathlineto{\pgfqpoint{4.846094in}{2.392479in}}%
\pgfpathlineto{\pgfqpoint{4.851990in}{2.392479in}}%
\pgfpathlineto{\pgfqpoint{4.855588in}{2.277008in}}%
\pgfpathlineto{\pgfqpoint{4.859754in}{2.277008in}}%
\pgfpathlineto{\pgfqpoint{4.862850in}{2.161541in}}%
\pgfpathlineto{\pgfqpoint{4.866202in}{2.161541in}}%
\pgfpathlineto{\pgfqpoint{4.872480in}{1.988346in}}%
\pgfpathlineto{\pgfqpoint{4.876151in}{1.988346in}}%
\pgfpathlineto{\pgfqpoint{4.878587in}{1.872887in}}%
\pgfpathlineto{\pgfqpoint{4.883624in}{1.872887in}}%
\pgfpathlineto{\pgfqpoint{4.886951in}{1.757432in}}%
\pgfpathlineto{\pgfqpoint{4.892411in}{1.757432in}}%
\pgfpathlineto{\pgfqpoint{4.897612in}{1.641980in}}%
\pgfpathlineto{\pgfqpoint{4.904250in}{1.641980in}}%
\pgfpathlineto{\pgfqpoint{4.905021in}{1.584255in}}%
\pgfpathlineto{\pgfqpoint{4.908257in}{1.641980in}}%
\pgfpathlineto{\pgfqpoint{4.912086in}{1.584255in}}%
\pgfpathlineto{\pgfqpoint{4.920516in}{1.584255in}}%
\pgfpathlineto{\pgfqpoint{4.921240in}{1.641980in}}%
\pgfpathlineto{\pgfqpoint{4.928686in}{1.641980in}}%
\pgfpathlineto{\pgfqpoint{4.928932in}{1.699705in}}%
\pgfpathlineto{\pgfqpoint{4.933598in}{1.699705in}}%
\pgfpathlineto{\pgfqpoint{4.937654in}{1.815159in}}%
\pgfpathlineto{\pgfqpoint{4.942395in}{1.815159in}}%
\pgfpathlineto{\pgfqpoint{4.946729in}{1.930616in}}%
\pgfpathlineto{\pgfqpoint{4.951950in}{1.930616in}}%
\pgfpathlineto{\pgfqpoint{4.955720in}{2.046077in}}%
\pgfpathlineto{\pgfqpoint{4.959451in}{2.046077in}}%
\pgfpathlineto{\pgfqpoint{4.959673in}{2.103808in}}%
\pgfpathlineto{\pgfqpoint{4.965494in}{2.103808in}}%
\pgfpathlineto{\pgfqpoint{4.965518in}{2.161541in}}%
\pgfpathlineto{\pgfqpoint{4.969691in}{2.219274in}}%
\pgfpathlineto{\pgfqpoint{4.977259in}{2.219274in}}%
\pgfpathlineto{\pgfqpoint{4.977382in}{2.277008in}}%
\pgfpathlineto{\pgfqpoint{4.993621in}{2.277008in}}%
\pgfpathlineto{\pgfqpoint{4.994025in}{2.219274in}}%
\pgfpathlineto{\pgfqpoint{4.998126in}{2.277008in}}%
\pgfpathlineto{\pgfqpoint{5.002747in}{2.161541in}}%
\pgfpathlineto{\pgfqpoint{5.010511in}{2.161541in}}%
\pgfpathlineto{\pgfqpoint{5.011371in}{2.103808in}}%
\pgfpathlineto{\pgfqpoint{5.016663in}{2.103808in}}%
\pgfpathlineto{\pgfqpoint{5.019637in}{1.988346in}}%
\pgfpathlineto{\pgfqpoint{5.023950in}{1.988346in}}%
\pgfpathlineto{\pgfqpoint{5.026628in}{1.872887in}}%
\pgfpathlineto{\pgfqpoint{5.030609in}{1.872887in}}%
\pgfpathlineto{\pgfqpoint{5.037340in}{1.699705in}}%
\pgfpathlineto{\pgfqpoint{5.039258in}{1.757432in}}%
\pgfpathlineto{\pgfqpoint{5.042526in}{1.641980in}}%
\pgfpathlineto{\pgfqpoint{5.047858in}{1.641980in}}%
\pgfpathlineto{\pgfqpoint{5.050546in}{1.584255in}}%
\pgfpathlineto{\pgfqpoint{5.050546in}{1.584255in}}%
\pgfusepath{stroke}%
\end{pgfscope}%
\begin{pgfscope}%
\pgfsetrectcap%
\pgfsetmiterjoin%
\pgfsetlinewidth{0.803000pt}%
\definecolor{currentstroke}{rgb}{0.000000,0.000000,0.000000}%
\pgfsetstrokecolor{currentstroke}%
\pgfsetdash{}{0pt}%
\pgfpathmoveto{\pgfqpoint{0.752485in}{0.539544in}}%
\pgfpathlineto{\pgfqpoint{0.752485in}{3.206606in}}%
\pgfusepath{stroke}%
\end{pgfscope}%
\begin{pgfscope}%
\pgfsetrectcap%
\pgfsetmiterjoin%
\pgfsetlinewidth{0.803000pt}%
\definecolor{currentstroke}{rgb}{0.000000,0.000000,0.000000}%
\pgfsetstrokecolor{currentstroke}%
\pgfsetdash{}{0pt}%
\pgfpathmoveto{\pgfqpoint{5.255215in}{0.539544in}}%
\pgfpathlineto{\pgfqpoint{5.255215in}{3.206606in}}%
\pgfusepath{stroke}%
\end{pgfscope}%
\begin{pgfscope}%
\pgfsetrectcap%
\pgfsetmiterjoin%
\pgfsetlinewidth{0.803000pt}%
\definecolor{currentstroke}{rgb}{0.000000,0.000000,0.000000}%
\pgfsetstrokecolor{currentstroke}%
\pgfsetdash{}{0pt}%
\pgfpathmoveto{\pgfqpoint{0.752485in}{0.539544in}}%
\pgfpathlineto{\pgfqpoint{5.255215in}{0.539544in}}%
\pgfusepath{stroke}%
\end{pgfscope}%
\begin{pgfscope}%
\pgfsetrectcap%
\pgfsetmiterjoin%
\pgfsetlinewidth{0.803000pt}%
\definecolor{currentstroke}{rgb}{0.000000,0.000000,0.000000}%
\pgfsetstrokecolor{currentstroke}%
\pgfsetdash{}{0pt}%
\pgfpathmoveto{\pgfqpoint{0.752485in}{3.206606in}}%
\pgfpathlineto{\pgfqpoint{5.255215in}{3.206606in}}%
\pgfusepath{stroke}%
\end{pgfscope}%
\begin{pgfscope}%
\pgfsetbuttcap%
\pgfsetmiterjoin%
\definecolor{currentfill}{rgb}{1.000000,1.000000,1.000000}%
\pgfsetfillcolor{currentfill}%
\pgfsetfillopacity{0.800000}%
\pgfsetlinewidth{1.003750pt}%
\definecolor{currentstroke}{rgb}{0.800000,0.800000,0.800000}%
\pgfsetstrokecolor{currentstroke}%
\pgfsetstrokeopacity{0.800000}%
\pgfsetdash{}{0pt}%
\pgfpathmoveto{\pgfqpoint{0.830263in}{2.962828in}}%
\pgfpathlineto{\pgfqpoint{1.848152in}{2.962828in}}%
\pgfpathquadraticcurveto{\pgfqpoint{1.870374in}{2.962828in}}{\pgfqpoint{1.870374in}{2.985051in}}%
\pgfpathlineto{\pgfqpoint{1.870374in}{3.128828in}}%
\pgfpathquadraticcurveto{\pgfqpoint{1.870374in}{3.151050in}}{\pgfqpoint{1.848152in}{3.151050in}}%
\pgfpathlineto{\pgfqpoint{0.830263in}{3.151050in}}%
\pgfpathquadraticcurveto{\pgfqpoint{0.808040in}{3.151050in}}{\pgfqpoint{0.808040in}{3.128828in}}%
\pgfpathlineto{\pgfqpoint{0.808040in}{2.985051in}}%
\pgfpathquadraticcurveto{\pgfqpoint{0.808040in}{2.962828in}}{\pgfqpoint{0.830263in}{2.962828in}}%
\pgfpathlineto{\pgfqpoint{0.830263in}{2.962828in}}%
\pgfpathclose%
\pgfusepath{stroke,fill}%
\end{pgfscope}%
\begin{pgfscope}%
\pgfsetrectcap%
\pgfsetroundjoin%
\pgfsetlinewidth{1.505625pt}%
\definecolor{currentstroke}{rgb}{0.003922,0.450980,0.698039}%
\pgfsetstrokecolor{currentstroke}%
\pgfsetstrokeopacity{0.700000}%
\pgfsetdash{}{0pt}%
\pgfpathmoveto{\pgfqpoint{0.852485in}{3.067717in}}%
\pgfpathlineto{\pgfqpoint{0.963596in}{3.067717in}}%
\pgfpathlineto{\pgfqpoint{1.074707in}{3.067717in}}%
\pgfusepath{stroke}%
\end{pgfscope}%
\begin{pgfscope}%
\definecolor{textcolor}{rgb}{0.000000,0.000000,0.000000}%
\pgfsetstrokecolor{textcolor}%
\pgfsetfillcolor{textcolor}%
\pgftext[x=1.163596in,y=3.028828in,left,base]{\color{textcolor}\rmfamily\fontsize{8.000000}{9.600000}\selectfont Temperature}%
\end{pgfscope}%
\end{pgfpicture}%
\makeatother%
\endgroup%
% data/plot_generic.py
    \caption{Temperature of a Stable Laser Systems \device{VH 6020} controlled by a Team Wavelength \device{HTC1500} temperature controller.}
    \label{fig:stability_cavity}
\end{figure}

The Team Wavelength \device{HTC1500} used in this example is an analog PID controller configured for the maximum specified integration time of \qty{10}{\s} using a \qty{10}{\uF} capacitor. As can be seen by the oscillatory behaviour, this time constant is far too short. Longer time scales can easily be reached by digital controllers which allow very long integration times limited only by the numerical resolution. A digital system can extend the scope of application from lasers to many other systems like those cavities to improve their stability. In addition, a digital system gives more control over the PID tuning parameters, also ensuring repeatability which greatly simplifies setting up new laser systems because a common set of PID parameters can be used a starting point before tuning the controller. Another benefit is the possibility to implement a modified algorithm and additional filters as detailed in section \ref{sec:pid_controller_basics}. This versatility to quickly adapt the programming again increases the number of applications. Integrating autotuning algorithms to help the user find a usable set of parameters reduces setup times of new systems. For all of those reasons, the new controller should be based on a digital design.

The final aspect to be considered is the output driver of the controller. The laser design used in this group uses two Peltier elements to cool both the resonator and the laser diode independently. The driver must therefore integrate two channels. While the biggest TECs currently in use are Laird \device{CP14,127,06,L1,W4.5} which can draw up to \qty{6}{\A} at \qty{15.4}{\V} \cite{datasheet_tec} their optimal coefficient of power is between \qtyrange[range-units = single]{1}{2}{\A} and \qtyrange[range-units = single]{5}{8}{\V}. Having a driver that can output \qty{4}{\A} at \qty{12}{\V} is considered more than sufficient, even for larger TECs and future projects.

Commercial temperature controllers specifying a stability of better than \qty{1}{\milli \kelvin} are hard to come by, especially with multiple channels. Two units were tested for this laser setup. The Vescent \device{SLICE-QTC} and an ILX Lightwave \device{LDT-5948}. The latter is specified for a stability of \qty{5}{\milli \kelvin}, but their application note claims a better performance \cite{appnote_ilx_millikelvin}.

The requirements for the temperature controller can be summarised as:
\begin{center}
    \begin{specifications}[label={lst:dgTemp_requirements}]{Temperature controller}
    \begin{itemize}
        \item Stability: \qty{<1}{\milli \K}
        \item Resolution: \qty{<200}{\micro \K}, \qty{<100}{\micro \K} preferred
        \item Temperature sensor: \qty{10}{\kilo \ohm} thermistor
        \item Two or more channels
        \item Output power: \qty{4}{\A} at \qty{12}{\V}
        \item Digital interface to work with long time scales and reproducible PID parameters
    \end{itemize}
    \end{specifications}
\end{center}

\clearpage
\section{LabKraken}%
\label{sec:prep_labkraken}
\subsection{Design Goals}
LabKraken is designed to be an asynchronous, resilient data acquisition suite that scales to thousands of sensors and across different networks to serve the need for monitoring and automation required for large scale experiments spanning multiple sites. It is written in Python and supports many sensors and instruments found in a scientific environment. Such sensors include Standard Commands for Programmable Instruments (SCPI) capable devices accessible via Ethernet or GPIB or sensors using a serial protocol. Many other Ethernet capable devices are also supported via a simple driver interface.

\subsection{Software Architecture}
To meet the increasing demand for high quality data, LabKraken needs to scale to thousands of sensors which must to be served concurrently. This problem is commonly referred to as the C10K problem as dubbed by \citeauthor{10kProblem} back in 1999 \cite{10kProblem} and refers to handling \num{10000} concurrent connections via network sockets. While today millions of concurrent connections can be handled by servers, handling \num{10000} can still be challenging, especially if the data sources are heterogeneous as is typical for sensor networks of diverse sensors from different manufacturers.

In order to meet the design goals, an asynchronous architecture was chosen and several different approaches were implemented over time. All in all, four complete rewrites of the software were made to arrive at the architecture introduced here. The reason for the rewrites is mostly historical and can be explained by the development of the Python programming language which was used to write the code. The first version was written using Python 2.6 and exclusively supported sensors made by Tinkerforge. With the release of Python 3.5 which supported a new syntax for asynchronous coroutines, the software was rewritten from scratch to support this new syntax, because it made the code a lot more verbose and easier to follow. When Python 3.7 was released asynchronous generator expressions where mature enough to be used in productions and the program was again rewritten to use the new syntax. In 2021 a new approach was taken and the program was once more rewritten with a functional programming style. Some of those approaches will be discussed in the next sections to highlight limits of the programming style used and the improvements made to overcome them. This development underlines important steps in the progress of asynchronous programming made by the Python language in recent years that can be applied to many other problems, for example process control. Specifically so since Python is a very popular language among scientists and used in many experiments. Each of the following sections discusses the same program, but written in different programming styles to show the differences. Especially the last example presenting a function programming style is interesting for experimental control as it gives a clean representation of the data flow from the producer to the consumer \cite{concurrent_programming}.

The example program that will be discussed does the following job. It opens a network connection to a remote (Tinkerforge) sensor platform, then queries the other side for its sensors. When the sensors are returned it looks for a specific sensor, then starts reading data from that sensor to finally print it. The example itself is designed around the Tinkerforge sensors to present a working example instead of the typically used pseudocode. It does represent a common program flow in a sensor application though and the concept is not limited to the Tinkerforge programming API.

\subsubsection{Threaded Design}
The first version of LabKraken used a threaded design approach, because the original libraries of the Tinkerforge sensors are built around threads. Most threaded programs make extensive use of callbacks. These are functions that are passed from the main thread to the worker, typically on creation, and are then called by the worker to inform the main thread of its activity. The downside is that main thread has no knowledge about the caller, the callback might have even been passed on by the worker to another thread.

\lstinputlisting[firstline=1, firstnumber=1, frame=single, breakindent=.5\textwidth, frame=single, breaklines=true, numbers=left, xleftmargin=2em, numberstyle=\tiny, style=mypython]{source/lab_kraken_threads.py}

The program starts at line \num{24} by making a connection to the host, then the first callback functions are registered with the connection object. These callbacks allow the connection thread to signal the program when the connection has been established (\textit{cb\_connected}) and when new sensors are found (\textit{cb\_enumerate}). The main program is now finished and it waits until terminated by the user. All the work is done inside the thread and the program flow unfortunately looses itself in the callbacks which get called by the connection object and their order can only be guessed from the documentation as the main program has no control over it. As the program continues it first enters the \textit{cb\_connected} callback where it will query the host for its sensors in line \num{9}. The answer will be returned through the \textit{cb\_enumerate} callback. This function filters the sensor id for a known sensor and then attaches another callback for the sensor to return data. It then configures the sensor. This program flow is typical for a callback driven design and the reader may imagine how more complex tasks are implemented. As the program grows, more and more layers of callbacks will be added and in the end, the code will be impossible to read without intimate knowledge. The effort of maintaining the callback driven code resulted in the decision to redesign the program when moving to Python 3.

To untangle this problem, Python 3.7 introduced so-called generators. This is a type of expression that will produce values from an iterator. An iterator is an (infinite) ordered series of values or events which can be processed by requesting the next value until the series is exhausted, if it is finite. The main advantage is that the logic of the program stays within the main part and only the data gathering is done outside of this scope. A generator based program is shown next.

\subsubsection{Generator Design}
In addition to a different coding style the code base is moved from a multithreaded to a asynchronous approach using Python asyncio. The difference is that asyncio uses a single thread as opposed to multithreaded code. Multiple threads can run concurrently on multiple processor cores, so different cores can process data at the same time. Asynchronous programs must pause the execution of code paths because they run within a single thread on a single core. This type of programming works best when the tasks are not computationally intensive but input/output bound by external peripherals like a network. While the processor is waiting for the slower network it can work on other tasks. The advantage is that access to shared resources is greatly simplified as these resources will never be accessed at the same time. The code is shown below.
\lstinputlisting[firstline=4, firstnumber=1, frame=single, breakindent=.5\textwidth, frame=single, breaklines=true, numbers=left, xleftmargin=2em, numberstyle=\tiny, style=mypython]{source/lab_kraken_async.py}

The first impression that can be gather from the new design is that the code has become more concise. To understand it, a few Python language keywords used must be introduced. In order to yield control to the next task, the keyword \textit{await} is used which is put in front of a function call. This will pause the current execution and wait until the function has returned with a result. Another important language feature used is a so-called context. A context is created using the \textit{async with} command and it makes sure that after leaving the context certain commands are executed. This can be used to clean up after the the creation and use of certain objects like the Ethernet connection. The connection context will make sure that the Ethernet connection will be properly terminated no matter whether enclosed content was shut down gracefully or not. The iterator uses the \textit{async for} keyword and works asynchronously as well. It pauses the code until a new event can be produced.

The code starts at line \num{26} and runs the \textit{main()} function. This function first connects to the host by creating a context in line \num{15} where the \textit{connection} variable is usable. Using this connection the sensor platform is queried in line \num{16}. In comparison to the precious example it is now far easier to follow the program flow because the context and generator reveal what is happening next. Unfortunately, reading the sensor still requires passing it to a new task because the generator will keep generating more sensors in the meantime. Although the code is split into multiple tasks, the nesting of callbacks as in the previous example is resolved and the readability of the code has improved tremendously.

The only problem is the error handling, because these worker tasks do not communicate with the original task that created them. This can be solved using the so-called observer pattern where an observer tasks watches the workers and handles such event. Directly implementing this pattern creates a myriad of events and event handler registrations. Missing one such event can break the whole program and leads to bugs that are hard diagnose and fix. To simplify this pattern a stream based approach can be applied. The observable is treated as a stream of events which is being processed by the observer using a chain of operators and actions executed in a certain order. This is much like an assembly line where different tasks are executed as the product passes each station.

\subsubsection{Stream Design}
The previously mentioned observer pattern is often implemented using data streams representing the subjects observed while the consumers are the observers. Using functional programming style these data streams can be written in a very concise form as shown in the following version of the example program.
\lstinputlisting[firstline=5, firstnumber=1, frame=single, breakindent=.5\textwidth, frame=single, breaklines=true, numbers=left, xleftmargin=2em, numberstyle=\tiny, style=mypython]{source/lab_kraken_stream.py}

The program starts in line \num{14} and enters the \textit{main()} function. Here, a context is used again to open the connection to the sensor platform. The sensors are queried next and a stream is created to read the reply, then filter for the specified sensor which is then read and the result is printed.

Using this programming style the intent of the program is revealed immediately, even before starting the stream. The syntax uses was borrowed from the Python library \textit{aiostreams} \cite{aiostreams}, which is similar to ReactiveX \cite{reactivex}, a library developed by Microsoft to operate on data streams. The interesting aspect of this code is the use of the pipe operator which inject the result of one function into the next function as its parameters. This way a chain of function calls is created. The \textit{lambda} keyword denotes a small anonymous function, but regular functions can also be used. In combination with an operator like \textit{filter}, \textit{switchmap}, or \textit{print}, they dictate the program flow, hence the name functional programming. These operators need some introduction though. The \textit{filter} operator is simple to understand as it will only pass on the input when the function, to which the input is passed as well, returns true. The \textit{switchmap} operator is more interesting. It is a combination of a \text{map} operator and a \text{switch} operator. The former applies a function to the input and then passes on the output of the function, in this case \textit{read\_temperature} which creates an iterator. The latter operator will take its most recent input and iterate, producing temperature values. This operator will terminate the iteration when a new input is passed and then iterate the new input. This is handy as it automatically makes sure that there is only one reader per sensor and for example new sensor configurations can be injected into the data stream above the \textit{switchmap} which automatically replace the old sensor reader.

This style of programming was found to be ideal for real-time data processing, as it allows to continuously update configurations or add and remove sensors, or even hosts, without having to worry about what happens along the pipeline.

\subsubsection{Device Identifiers}
Every sensor network needs device identifiers. Preferably those identifiers should be unique. Typically a device has some kind of internal identifier. Here are a few examples of the sensors used in the authors network:
\begin{table}[ht]
    \centering
    \begin{tabularx}{0.95\textwidth}{|l|p{3cm}|X|}
        \hline
        Device Type& Identifiers& Example\\
        \hline
        GPIB (SCPI)& \textit{*IDN?}& \small{Keysight Technologies,34470A,MYXXXXXXXX,A.03.03-02.40-03.03-00.52-02-01} or\newline\small{Agilent Technologies,34410A,MYXXXXXXXX,A.03.03-02.40-03.03-00.52-02-01}\\
        \hline
        Tinkerforge& A base58 encoded integer device id& QE9 (163684)\\
        \hline
        LabNode& UUID & cc2f2159-e2fb-4ed9-8021-7771890b37ad\\
        \hline
    \end{tabularx}
    \caption{Device identifiers used by common devices found in the lab. The serial number of the Keysight \device{34470A} DMM was obscured on purpose.}
    \label{tab:common_device_ids}
\end{table}

As it can be seen in table \ref{tab:common_device_ids}, most of these identifiers do not guarantee to uniquely identify a device within a network. The Tinkerforge id is the weakest, as it is a \qty{32}{\bit} integer (\num{4294967295} options), which might easily collide with another id from a different manufacturer. For better readability the id is typically presented as a base58 encoded string. An encoder/decoder example can be found in the TinkerforgeAsync library \cite{TinkerforgeAsync}.

The id string returned by a SCPI device is slightly more useful, but again does not guarantee uniqueness. As per the SCPI specification it returns a string containing \textit{\$manufacturer,\$name,\$serial,\$revision}. Even when ignoring the software revision part which might change on update, the same device might return a different id depending on its settings. The id string shown in table \ref{tab:common_device_ids} relate to the same device, but the latter uses a compatibility flag in the settings.

The only reasonably unique id is used by the LabNodes. The universal unique identifier (UUID) or globally unique identifier (GUID), as dubbed by Microsoft, can be used for networks with participant numbers going into the millions. There are several versions defined in RFC 4122 \cite{rfc_uuid} and the LabNodes use version 4, which is a random \qty{128}{\bit} identifier with \qty{122}{\bit} of entropy. Of the remaining \qty{6}{\bit}, \qty{4}{\bit} are reserved for the UUID version and \qty{2}{\bit} for the variant. This allows to prove the usefulness as a unique id as below.

Calculating the chance of a collision between two random UUIDs is called the birthday problem \cite{BirthdayProblem} in probability theory. The probability of at least one collision in $n$ devices out of $M = 2^{122}$ possibilities can be calculated as follows:
\begin{align}
    p(n) &= 1 - 1 \cdot \left(1 - \frac{1}{M}\right) \cdot \left(1 - \frac{2}{M}\right) \dots \left(1 - \frac{n-1}{M}\right) \nonumber\\
    &= 1 - \prod_{k=1}^{n-1} \left(1 - \frac{k}{M} \right)
\end{align}
Using the Taylor series $e^x = 1+x \dots$, assuming $n \ll M$ and approximating we can simplify this to:
\begin{align}
    p(n) &\approx 1 - \left(e^\frac{-1}{M} \cdot e^\frac{-2}{M} \dots e^\frac{-(n-1)}{M} \right) \nonumber\\
    &\approx 1 - \left(e^\frac{-n(n-1)/2}{M} \right) \nonumber\\
    &\approx 1 - \left(1 - \frac{n^2}{2 M} \right) = \frac{n^2}{2 M}
\end{align}
For one million devices using random UUIDs, this gives a probability of about \num{2e-25}, which is negligible.

In the LabKraken implementation, all devices, except for the LabNodes which already have a UUID, will be mapped to UUIDs using the underlying configuration database. It is up to the user to ensure the uniqueness of the non-UUID ids reported by the devices to ensure proper mapping. These UUIDs can then be used to address and configure each device on the sensor network.

%\subsubsection{Ethernet Bus and Synchronous Buses}
%There are inherent challenges involved with the Ethernet bus for instrumentation. The Ethernet bus is intrinsically asynchronous and multiple controllers can talk to the device at the same time. Not only that, but different processes within the same controller can talk to the same device. This makes deterministic statements about the device state challenging. A device that is not designed to work asynchronously in the first place may have trouble with multiple requests coming in from different clients. This must be kept in mind when using serial adapters like USB or GPIB to Ethernet.

%While it is impossible to rule out the possibility of multiple controllers on a network, care was taken to synchronize the workers within Kraken.
%\subsection{Databases}
%\subsubsection{Cardinality}
%\begin{itemize}
% \item TimescaleDB vs Influx
% \item Example Sensors vs. Experiment
%\end{itemize}

\clearpage
\section{Short Introduction to Control Theory}
This section will give a very brief introduction to some basic concepts of control theory. Many systems require control over one or more process variables. For example, temperature control of a room or a device, or even creating a programmable current from a voltage is one such problem. All of this requires control over a process and is established through feedback, which allows a controller to be aware of the state of the system.

The focus of this section is narrowed down to the concept of feedback and control with regard to developing and understanding PID controllers for temperature control. Simpler feedback loops like those typically used around op-amps will not be primarily considered in this section and are discussed in the relevant part of the documentation. In the following sections, first general properties of the Laplace transform and useful relationships are introduced, then, a model for the system and its controller will be developed, finally, using the model, tuning of the control parameters using different tuning algorithms will be discussed.

\subsection{Introduction to the Transfer Function and the Laplace Domain}%
\label{sec:transfer_function}
There are two types of control systems: open- and closed-loop systems. A system is called open loop, if the output of a system does not feed back to its input as in figure \ref{fig:open_loop}. On the other hand, if the output influences the input of the system via feedback, it is called a closed-loop system, as shown in figure \ref{fig:closed_loop}. Although feedback can be treated in static systems, it is more useful to treat it in dynamic systems, either in the time-domain or the frequency-domain. To discuss these systems, the terminology used in the following section needs to be defined. $G(s)$ is called the transfer function of the system, while $U(s)$ is the input, $Y(s)$ is the output, $s$ is a complex frequency domain variable, $\beta$ is the feedback parameter, also called feedback fraction, as shown in figure \ref{fig:closed_loop}. In this section, upper case letters are used to denote functions in the Laplace domain, while lower case letters are referring to functions in the time domain. Normally, the transfer function is denoted $H(s)$ but to prevent confusion with the Heaviside function $H(t)$, the letter $G$ is used here for the transfer function. In later chapters the common form $H(s)$ is used.
\begin{figure}[ht]
    \centering
    \begin{subfigure}{0.4\linewidth}
        \centering
        \import{figures/}{open_loop.tex}
        \caption{Open-loop system.}
        \label{fig:open_loop}
    \end{subfigure}
    \begin{subfigure}{0.4\linewidth}
        \centering
        \import{figures/}{closed_loop.tex}
        \caption{Closed-loop system.}
        \label{fig:closed_loop}
    \end{subfigure}
    \caption{Block diagram of a closed- and an open-loop control system.}
    \label{fig:feedback_systems}
\end{figure}

It is convenient to express the transfer function in its Laplace transform for several reasons that will be explained below. The systems to be discussed are physical system and hence are causal. That means the output only depends on past and present inputs, but not future inputs. For this reason, only the one-sided or unilateral Laplace transform needs to be considered. It is defined as:
\begin{equation}
    \mathscr{L}\left( f(t) \right) = F(s) = \int_0^\infty f(t) e^{-st}\,dt.
\end{equation}

with $f: \mathbb{R}^+ \to \mathbb{R}$ that is integrable and grows no faster than $c \cdot e^{s_0t}$ for $s_0, c \in \mathbb{R}$. The latter attribute is important for deriving the rules of differentiation and integration.

To understand the benefits of using the Laplace representation of the transfer function, a few useful properties should be discussed with regard to the PID controller. First of all, the Laplace transform is linear:
\begin{align}
    \mathscr{L}\left(a \cdot f(t) + b \cdot g(t) \right) &= \int_0^\infty (a \cdot f(t) + b \cdot g(t)) e^{-st}\,dt \nonumber\\
    &= a \int_0^\infty f(t) e^{-st}\,dt + b \int_0^\infty g(t) e^{-st}\,dt \nonumber\\
    &= a \mathscr{L}\left(f(t)\right) + b \mathscr{L}\left(g(t)\right)
\end{align}

Another interesting property is the derivative and integral of a function $f$. The function $f$ must, of course, be differentiable and grow no faster than the exponential function as defined above:
\begin{align}
    \mathscr{L}\left(\frac{df}{dt}\right) &= \int_0^\infty \underbracket{f'(t)}_{v'(t)} \underbracket{\vphantom{f'(t)}e^{-st}}_{u(t)}\,dt \nonumber\\
    &= \left[e^{-st} f(t) \right]_0^\infty - \int_0^\infty (-s)f'(t)\,dt \nonumber\\
    &= -f(0) + s \int_0^\infty f'(t)\,dt \nonumber\\
    &= s F(s) - f(0)
\end{align}

\begin{align}
    \mathscr{L} \left( \int_0^t f(\tau)\,d\tau \right) &= \int_0^\infty \left(\int_0^t f(\tau)\,d\tau e^{-st} \right)\,dt \nonumber\\
    &= \int_0^\infty \underbracket{e^{-st}\vphantom{\int_0^t}}_{v'(t)} \underbracket{\int_0^t f(t)\,d\tau}_{u(t)}\,dt \nonumber\\
    &= \left[\frac{-1}{s} e^{-st} \int_0^t f(t)\,d\tau \right]_0^\infty - \int_0^\infty \frac{-1}{s} e^{-s\tau} f(\tau)\,d\tau \nonumber\\
    &= 0 + \frac{1}{s} \int_0^\infty e^{-s\tau} f(\tau)\,d\tau \nonumber\\
    &= \frac{1}{s} F(s) \label{eqn:lapace_integration}
\end{align}

If the initial state $f(0)$ can be chosen to be $0$, the differentiation becomes a simple multiplication by $s$, while the integration becomes a division by $s$. These three properties greatly simplify the calculations required for studying a proportional–integral–derivative controller in section \ref{sec:pid_controller_basics}.

Finally, the most important aspect is the possibility to give a simple relation between the input $u(t)$ and the output $y(t)$ of a system. This relationship between input and output of a system as shown in figure \ref{fig:open_loop} is given by the convolution, see e.g. \cite{pid_basics}. Assuming the system has an initial state of $0$ for $t<0$, hence $u(t<0) = 0$ and $g(t<0) = 0$, one can calculate:
\begin{equation}
    y(t) = (u \ast g)(t) = \int_0^\infty u(\tau) g(t-\tau)\,d\tau
    \label{eqn:convolution}
\end{equation}

Applying the Laplace transform, greatly simplifies this:
\begin{align}
    Y(s) &= \int_0^\infty e^{-st} y(t)\,dt \nonumber\\
    \overset{\ref{eqn:convolution}}&{=} \int_0^\infty \underbrace{e^{-st}}_{e^{-s(t-\tau)}e^{-s\tau}} \int_0^\infty u(\tau) g(t-\tau)\,d\tau\,dt \nonumber\\
    &= \int_0^\infty \int_0^t e^{-s(t-\tau)} e^{-s\tau} g(t-\tau) u(\tau)\,d\tau\,dt \nonumber\\
    &= \int_0^\infty e^{-s\tau} u(\tau)\,d\tau \int_0^\infty e^{-st} g(t)\,dt \nonumber\\
    &= U(s) \cdot G(s)
\end{align}

This formula is a lot simpler than the convolution of $u(t)$ and $g(t)$, therefore the use of the Laplace transform has become very popular in control theory.

Having derived some of the most useful properties, it is interesting to look at a few functions, which are heavily used in control theory, like a function delayed by the time interval $\theta$. To demonstrate its properties, let $f(t-\theta)$ be
\begin{equation}
    g(t) \coloneqq \begin{cases} f(t-\theta), & t \geq \theta \\ 0, & t < \theta \end{cases} \,. \label{eqn:delayed_f}
\end{equation}

The reason for this definition is that it is mandatory for the system to be causal. This means, it is impossible to get information from the future ($t<\theta$). To satisfy this requirement, any constant other than \num{0} may be chosen, as is done later in section \ref{sec:pid_tuning_rules}, when determining tuning parameters and fitting experimental data to a model. An example of such a time delayed function $g(t)$ is shown in figure \ref{fig:heaviside_delayed}.
\begin{figure}[ht]
    \centering
    \begin{subfigure}{0.4\linewidth}
        \centering
        \scalebox{0.75}{%
            \import{figures/}{laplace_no_delay.tex}
        } % scalebox
        \caption{Original signal $f(t)$.}
        \label{fig:heaviside}
    \end{subfigure}
    \begin{subfigure}{0.4\linewidth}
        \centering
        \scalebox{0.75}{%
            \import{figures/}{laplace_time_delay.tex}
        } % scalebox
        \caption{Delayed signal $f(t-2)$.}
        \label{fig:heaviside_delayed}
    \end{subfigure}
\end{figure}

The Laplace transform of a delayed signal $g(t)$ can be calculated as follows:
\begin{align}
    \mathscr{L}\left( g(t) \right) &= \int_0^\infty g(t) e^{-st}\,dt \nonumber\\
    \overset{\ref{eqn:delayed_f}}&{=} \int_\theta^\infty f(t-\theta) e^{-st}\,dt \nonumber\\
    \overset{\tau \coloneqq t-\theta}&{=} \int_0^\infty f(\tau) e^{-s(\tau+\theta)}\,d\tau \nonumber\\
    &= e^{-s\theta} \int_0^\infty f(\tau) e^{-s\tau} \,d\tau \nonumber\\
    &= e^{-s\theta} F(s) \label{eqn:laplace_delayed}
\end{align}

To satisfy the causality requirement in the time domain, the Heaviside function $H(t)$ can be used to give a more concise representation of $g(t)$:
\begin{align}
    \mathscr{L}\left( f(t-\theta) H(t-\theta) \right) = e^{-s\theta} F(s) \label{eqn:laplace_causality}
\end{align}

Lastly, the Laplace transform of $e^{at}$ is given, which is commonly used in differential equations:
\begin{align}
    \mathscr{L}\left(e^{at} \right) &= \int_0^\infty e^{(a-s)t}\,dt = \frac{1}{a-s} \left[e^{(a-s)t} \right]_0^\infty = \frac{1}{s-a} \label{eqn:laplace_exponential}
\end{align}

Using these tools, it is possible calculate the transfer function of a closed-loop temperature controller, which will be done in the next section.

\subsection{A Model for Temperature Control}%
\label{sec:temperature_control_model}
\begin{figure}[ht]
    \centering
    %\scalebox{1} % scalebox
    \caption{Simple temperature model of a generic system.}
    \label{fig:first-order_model_room}
\end{figure}

In order to describe a closed-loop system using a transfer function $G(s)$, one has to first create a model for the process and the controller involved. This section will derive the simple, but very useful first order model with dead-time. This model can be derived from the idea that the system at temperature $T_{system}$ has a thermal capacitance $C_{system}$, an influx of heat $\dot Q_{load}$ from a thermal load and a controller removing heat from the system through a heat exchanger with a resistance of $R_{force}$. Additionally, there is some leakage through the walls of the system to the ambient environment via $R_{leakage}$. This analogy of thermodynamics with electrodynamics allows to create the model shown in figure \ref{fig:first-order_model_room}. Since this model is to be used for a temperature controller, more simplifications can be made and a so-called small-signal model can be developed as opposed to the large-signal model shown above. The small-signal model is an approximation around a working point that is valid for small deviations around it, similar to a Taylor approximation. The small-signal model can be used to calculate the system response to small changes of the controller output in order to estimate the control parameters at that working point.

Using the small-signal approach, the system response can be split into a constant and a dynamic part: the 0\textsuperscript{th} and 1\textsuperscript{st} order of the Taylor approximation. In order to simplify the system shown in figure \ref{fig:first-order_model_room}, the assumption can be made that the system load $\dot Q_{load}$ and the flux through $R_{leakage}$ is \textit{reasonably stable}. \textit{Reasonably stable} means that it can be treated as small deviations and additionally any changes are within the bandwidth of the controller and well suppressed. This allows to treat them as (almost) constant effects. Constant effect can be neglected in the small-signal model because they only manifest as a constant offset applied to the output of the controller and have no dynamics. This leaves only the room with its heat capacity and the heat exchanger as dominant factors in the small-signal model shown in figure \ref{fig:first-order_model}. Here $T_{force}$ and $T_{system}$ were replaced by $T_{in}$ and $T_{out}$, $R_{force}$ and $C_{system}$ were replaced by $R$ and $C$ for better readability.
\begin{figure}[htb]
    \centering
    \scalebox{1}{%
        \import{figures/}{first-order_model_kirchhoff.tex}
    } % scalebox
    \caption{Simplifications of the temperature model of a room lead to this first order model.}
    \label{fig:first-order_model}
\end{figure}

This is the classic $RC$ circuit. To calculate the transfer function, a relationship between $T_{in}$ and $T_{out}$ is required and exploiting the analogy of thermodynamics and electrodynamics again, using Kirchhoff's second law, following the arrow in figure \ref{fig:first-order_model} one finds:
\begin{alignat}{1}
    \sum T_i &= 0 \nonumber\\
    T_{in}(t) - \dot{Q}(t) R - \frac 1 C \int \dot{Q}(t)\,dt &= 0 \label{eqn:first-order_model_kirchhoff}
\end{alignat}

Taking the Laplace transform, applying equation \ref{eqn:lapace_integration}, solving for $ \dot Q(s)$ and using $T_{out} = \frac{1}{sC} \dot Q(s)$ to replace $\dot Q$, equation \ref{eqn:first-order_model_kirchhoff} can be written as:
\begin{align*}
    T_{in}(s) - \dot{Q}(s) R - \frac{1}{sC} \dot{Q}(s) &= 0\\
    \dot{Q}(s) = \frac{T_{in}(s)}{R-\frac{1}{sC}} &= \frac{T_{out}}{\frac{1}{sC}}
\end{align*}

This allows to calculate the transfer function of the process $P$ as
\begin{align}
    P(s) &= \frac{T_{out}}{T_{in}} = \frac{\frac{1}{sC}}{R-\frac{1}{sC}} \nonumber\\
    &= \frac{1}{sRC + 1} \nonumber\\
    &= \frac{1}{1 + s\tau} = \frac{K}{1 + s\tau}\,. \label{eqn:first-order_model}
\end{align}
with the system gain $K$ and the time constant $\tau$. In case of the $RC$ circuit, the gain is $1$, but other systems may have a gain factor of $K \neq 1$. This is generally the case when using any type of sensor that converts the measurand into the input signal. $K$ is therefore included here for the sake of generality.

Equation \ref{eqn:first-order_model} is called the transfer function of a first order model, because its origin is a differential equation of first order. This model describes homogeneous systems like a room very well, as can be seen in section \ref{sec:pid_controller_tuning}, but in order to derive the transfer function including the controller and the sensor some more work is required to derive the sensor transfer function.

Expanding on figure \ref{fig:open_loop} and equation \ref{eqn:convolution} the open-loop transfer function of the process and its sensor becomes:
\begin{equation}
    G(s) = P(s) \cdot S(s)
\end{equation}
and the block diagram changes to
\begin{figure}[htb]
    \centering
    %\scalebox{1}% scalebox
    \caption{Open-loop system with sensor.}
\end{figure}

The transfer function of the sensor, given an ideal linear transducer, can be modeled as a delay line with delay $\theta$ and $f(t-\theta) = H(t-\theta)$. A sensor gain of $1$ is assumed here, because any system gain is assumed be included in the parameter $K$ of the process transfer funciton. Using equation \ref{eqn:laplace_delayed}, $S(s)$ can be written as
\begin{equation}
    S(s) = e^{-\theta s} .
\end{equation}

The full system model including the time delay can now be written as:
\begin{equation}
    G(s) = \frac{K}{1 + s\tau} e^{-\theta s} \label{eqn:first-order_plus_dead_time_model}
\end{equation}

This is called a first order plus dead-time model (FOPDT) or first order plus time-delay model (FOPTD). While the Laplace representation is useful to mathematically explore the mode, in order to fit experimental data to this model it is more convenient to transform the transfer function \ref{eqn:first-order_plus_dead_time_model} into the time domain. To have a meaningful result, an input $U(s)$ is required, because $G(s)$ is only a transformation. In principal, any function can serve this purpose, but a step function is typically used, for example by \citeauthor{ziegler_nichols} \cite{ziegler_nichols} and many others \cite{tuning_rules,pessen_integral,simc,simc_paper,pid_controllers_for_time_delay_systems,pi_stabilization_of_fopdt_systems, pid_basics}. The step function is both simple to calculate and to apply to a real system in form of a controller output change. This technique is also explored in more detail in section \ref{sec:pid_controller_tuning}. Using equations \ref{eqn:laplace_delayed} and \ref{eqn:laplace_exponential}, the Heaviside $H(t)$ step function transforms into
\begin{equation}
    \mathscr{L} \left(u(t) \right) = U(s) = \mathscr{L} \left( \Delta u H(t) \right) = \frac{\Delta u}{s}
\end{equation}
with the step size $\Delta u$. The output response $Y(s)$ of the system to the step can then be calculated analytically.
\begin{align}
    Y(s) &= U(s) \cdot G(s)\nonumber\\
    &= \frac{\Delta u}{s} \frac{K}{1 + s\tau} e^{-\theta s} \nonumber\\
    &=  K \Delta u \frac{1}{s (1 + s\tau)} e^{-\theta s} \nonumber\\
    &= K \Delta u \left(\frac{1}{s} - \frac{\tau}{s\tau+1} \right) e^{-\theta s} \nonumber\\
    &= K \Delta u \left(\frac{1}{s} - \frac{1}{s+\frac{1}{\tau}} \right) e^{-\theta s}
\end{align}

To derive $y(t)$, the inverse Laplace transform of $Y(s)$ is required. Unfortunately, this is not as simple as the Laplace transform. Fortunately though the required equations were already derived in equations \ref{eqn:lapace_integration} and \ref{eqn:laplace_exponential}. Making sure causality is guaranteed as shown in equation \ref{eqn:laplace_causality}, the simple first order model can be transformed back into the time domain.
\begin{align}
     y(t) &= \mathscr{L}^{-1} \left(Y(s)\right) \nonumber\\
     &= K \Delta u \mathscr{L}^{-1} \left(\frac{1}{s} e^{-\theta s} \right)  - K \mathscr{L}^{-1} \left( \frac{1}{s+\frac{1}{\tau}} e^{-\theta s} \right) \nonumber\\
    \overset{\ref{eqn:laplace_exponential}}&{=} K \Delta u \cdot 1 \cdot H(t-\theta) - \left(e^{-\frac{t-\theta}{\tau}} \right) H(t-\theta) \nonumber\\
    &= K \Delta u \left(1-e^{-\frac{t-\theta}{\tau}} \right) H(t-\theta) \label{eqn:first-order_plus_dead_time_model_time-domain}
\end{align}

The time domain solution of the FOPDT model can now be used to extract the parameters $\tau$, $\theta$ and $K$ from a real physical system.

The procedure can be summarised from the above as follows. The controller must be set to a constant output and the room must be given time to reach equilibrium. Once the temperature has settled, an output step of $\Delta u$ is applied. The system will respond after a time delay and then follow an exponential function. A simulation of the step response applied to a first order model with time delay is shown in figure \ref{fig:fopdt}. The gain is $K=1$. The solid black line shows the response of the transfer function, including the system and the sensor. The dashed lines show the individual components, the Heaviside function governing the delay and the exponential term of the system. The controller output step $\Delta u = 1$ is applied at $t=0$ and not shown explicitly. From figure \ref{fig:fopdt} it can be clearly seen that the sensor does not register a change until the time delay $\theta$ has passed and the Heaviside function changes from $0$ to $1$. Then the system responds with an exponential decay towards \num{1}.
\begin{figure}[ht]
    \centering
    %% Creator: Matplotlib, PGF backend
%%
%% To include the figure in your LaTeX document, write
%%   \input{<filename>.pgf}
%%
%% Make sure the required packages are loaded in your preamble
%%   \usepackage{pgf}
%%
%% Also ensure that all the required font packages are loaded; for instance,
%% the lmodern package is sometimes necessary when using math font.
%%   \usepackage{lmodern}
%%
%% Figures using additional raster images can only be included by \input if
%% they are in the same directory as the main LaTeX file. For loading figures
%% from other directories you can use the `import` package
%%   \usepackage{import}
%%
%% and then include the figures with
%%   \import{<path to file>}{<filename>.pgf}
%%
%% Matplotlib used the following preamble
%%   \usepackage{siunitx}
%%   \sisetup{per-mode = symbol}%
%%   \usepackage{fontspec}
%%   \makeatletter\@ifpackageloaded{underscore}{}{\usepackage[strings]{underscore}}\makeatother
%%
\begingroup%
\makeatletter%
\begin{pgfpicture}%
\pgfpathrectangle{\pgfpointorigin}{\pgfqpoint{5.431103in}{3.356606in}}%
\pgfusepath{use as bounding box, clip}%
\begin{pgfscope}%
\pgfsetbuttcap%
\pgfsetmiterjoin%
\definecolor{currentfill}{rgb}{1.000000,1.000000,1.000000}%
\pgfsetfillcolor{currentfill}%
\pgfsetlinewidth{0.000000pt}%
\definecolor{currentstroke}{rgb}{1.000000,1.000000,1.000000}%
\pgfsetstrokecolor{currentstroke}%
\pgfsetdash{}{0pt}%
\pgfpathmoveto{\pgfqpoint{0.000000in}{0.000000in}}%
\pgfpathlineto{\pgfqpoint{5.431103in}{0.000000in}}%
\pgfpathlineto{\pgfqpoint{5.431103in}{3.356606in}}%
\pgfpathlineto{\pgfqpoint{0.000000in}{3.356606in}}%
\pgfpathlineto{\pgfqpoint{0.000000in}{0.000000in}}%
\pgfpathclose%
\pgfusepath{fill}%
\end{pgfscope}%
\begin{pgfscope}%
\pgfsetbuttcap%
\pgfsetmiterjoin%
\definecolor{currentfill}{rgb}{1.000000,1.000000,1.000000}%
\pgfsetfillcolor{currentfill}%
\pgfsetlinewidth{0.000000pt}%
\definecolor{currentstroke}{rgb}{0.000000,0.000000,0.000000}%
\pgfsetstrokecolor{currentstroke}%
\pgfsetstrokeopacity{0.000000}%
\pgfsetdash{}{0pt}%
\pgfpathmoveto{\pgfqpoint{0.667540in}{0.524170in}}%
\pgfpathlineto{\pgfqpoint{5.222294in}{0.524170in}}%
\pgfpathlineto{\pgfqpoint{5.222294in}{3.168170in}}%
\pgfpathlineto{\pgfqpoint{0.667540in}{3.168170in}}%
\pgfpathlineto{\pgfqpoint{0.667540in}{0.524170in}}%
\pgfpathclose%
\pgfusepath{fill}%
\end{pgfscope}%
\begin{pgfscope}%
\pgfsetbuttcap%
\pgfsetroundjoin%
\definecolor{currentfill}{rgb}{0.000000,0.000000,0.000000}%
\pgfsetfillcolor{currentfill}%
\pgfsetlinewidth{0.803000pt}%
\definecolor{currentstroke}{rgb}{0.000000,0.000000,0.000000}%
\pgfsetstrokecolor{currentstroke}%
\pgfsetdash{}{0pt}%
\pgfsys@defobject{currentmarker}{\pgfqpoint{0.000000in}{-0.048611in}}{\pgfqpoint{0.000000in}{0.000000in}}{%
\pgfpathmoveto{\pgfqpoint{0.000000in}{0.000000in}}%
\pgfpathlineto{\pgfqpoint{0.000000in}{-0.048611in}}%
\pgfusepath{stroke,fill}%
}%
\begin{pgfscope}%
\pgfsys@transformshift{0.667540in}{0.524170in}%
\pgfsys@useobject{currentmarker}{}%
\end{pgfscope}%
\end{pgfscope}%
\begin{pgfscope}%
\definecolor{textcolor}{rgb}{0.000000,0.000000,0.000000}%
\pgfsetstrokecolor{textcolor}%
\pgfsetfillcolor{textcolor}%
\pgftext[x=0.667540in,y=0.426948in,,top]{\color{textcolor}\rmfamily\fontsize{8.000000}{9.600000}\selectfont \(\displaystyle {0}\)}%
\end{pgfscope}%
\begin{pgfscope}%
\pgfsetbuttcap%
\pgfsetroundjoin%
\definecolor{currentfill}{rgb}{0.000000,0.000000,0.000000}%
\pgfsetfillcolor{currentfill}%
\pgfsetlinewidth{0.803000pt}%
\definecolor{currentstroke}{rgb}{0.000000,0.000000,0.000000}%
\pgfsetstrokecolor{currentstroke}%
\pgfsetdash{}{0pt}%
\pgfsys@defobject{currentmarker}{\pgfqpoint{0.000000in}{-0.048611in}}{\pgfqpoint{0.000000in}{0.000000in}}{%
\pgfpathmoveto{\pgfqpoint{0.000000in}{0.000000in}}%
\pgfpathlineto{\pgfqpoint{0.000000in}{-0.048611in}}%
\pgfusepath{stroke,fill}%
}%
\begin{pgfscope}%
\pgfsys@transformshift{1.578491in}{0.524170in}%
\pgfsys@useobject{currentmarker}{}%
\end{pgfscope}%
\end{pgfscope}%
\begin{pgfscope}%
\definecolor{textcolor}{rgb}{0.000000,0.000000,0.000000}%
\pgfsetstrokecolor{textcolor}%
\pgfsetfillcolor{textcolor}%
\pgftext[x=1.578491in,y=0.426948in,,top]{\color{textcolor}\rmfamily\fontsize{8.000000}{9.600000}\selectfont \(\displaystyle {2}\)}%
\end{pgfscope}%
\begin{pgfscope}%
\pgfsetbuttcap%
\pgfsetroundjoin%
\definecolor{currentfill}{rgb}{0.000000,0.000000,0.000000}%
\pgfsetfillcolor{currentfill}%
\pgfsetlinewidth{0.803000pt}%
\definecolor{currentstroke}{rgb}{0.000000,0.000000,0.000000}%
\pgfsetstrokecolor{currentstroke}%
\pgfsetdash{}{0pt}%
\pgfsys@defobject{currentmarker}{\pgfqpoint{0.000000in}{-0.048611in}}{\pgfqpoint{0.000000in}{0.000000in}}{%
\pgfpathmoveto{\pgfqpoint{0.000000in}{0.000000in}}%
\pgfpathlineto{\pgfqpoint{0.000000in}{-0.048611in}}%
\pgfusepath{stroke,fill}%
}%
\begin{pgfscope}%
\pgfsys@transformshift{2.489442in}{0.524170in}%
\pgfsys@useobject{currentmarker}{}%
\end{pgfscope}%
\end{pgfscope}%
\begin{pgfscope}%
\definecolor{textcolor}{rgb}{0.000000,0.000000,0.000000}%
\pgfsetstrokecolor{textcolor}%
\pgfsetfillcolor{textcolor}%
\pgftext[x=2.489442in,y=0.426948in,,top]{\color{textcolor}\rmfamily\fontsize{8.000000}{9.600000}\selectfont \(\displaystyle {4}\)}%
\end{pgfscope}%
\begin{pgfscope}%
\pgfsetbuttcap%
\pgfsetroundjoin%
\definecolor{currentfill}{rgb}{0.000000,0.000000,0.000000}%
\pgfsetfillcolor{currentfill}%
\pgfsetlinewidth{0.803000pt}%
\definecolor{currentstroke}{rgb}{0.000000,0.000000,0.000000}%
\pgfsetstrokecolor{currentstroke}%
\pgfsetdash{}{0pt}%
\pgfsys@defobject{currentmarker}{\pgfqpoint{0.000000in}{-0.048611in}}{\pgfqpoint{0.000000in}{0.000000in}}{%
\pgfpathmoveto{\pgfqpoint{0.000000in}{0.000000in}}%
\pgfpathlineto{\pgfqpoint{0.000000in}{-0.048611in}}%
\pgfusepath{stroke,fill}%
}%
\begin{pgfscope}%
\pgfsys@transformshift{3.400393in}{0.524170in}%
\pgfsys@useobject{currentmarker}{}%
\end{pgfscope}%
\end{pgfscope}%
\begin{pgfscope}%
\definecolor{textcolor}{rgb}{0.000000,0.000000,0.000000}%
\pgfsetstrokecolor{textcolor}%
\pgfsetfillcolor{textcolor}%
\pgftext[x=3.400393in,y=0.426948in,,top]{\color{textcolor}\rmfamily\fontsize{8.000000}{9.600000}\selectfont \(\displaystyle {6}\)}%
\end{pgfscope}%
\begin{pgfscope}%
\pgfsetbuttcap%
\pgfsetroundjoin%
\definecolor{currentfill}{rgb}{0.000000,0.000000,0.000000}%
\pgfsetfillcolor{currentfill}%
\pgfsetlinewidth{0.803000pt}%
\definecolor{currentstroke}{rgb}{0.000000,0.000000,0.000000}%
\pgfsetstrokecolor{currentstroke}%
\pgfsetdash{}{0pt}%
\pgfsys@defobject{currentmarker}{\pgfqpoint{0.000000in}{-0.048611in}}{\pgfqpoint{0.000000in}{0.000000in}}{%
\pgfpathmoveto{\pgfqpoint{0.000000in}{0.000000in}}%
\pgfpathlineto{\pgfqpoint{0.000000in}{-0.048611in}}%
\pgfusepath{stroke,fill}%
}%
\begin{pgfscope}%
\pgfsys@transformshift{4.311344in}{0.524170in}%
\pgfsys@useobject{currentmarker}{}%
\end{pgfscope}%
\end{pgfscope}%
\begin{pgfscope}%
\definecolor{textcolor}{rgb}{0.000000,0.000000,0.000000}%
\pgfsetstrokecolor{textcolor}%
\pgfsetfillcolor{textcolor}%
\pgftext[x=4.311344in,y=0.426948in,,top]{\color{textcolor}\rmfamily\fontsize{8.000000}{9.600000}\selectfont \(\displaystyle {8}\)}%
\end{pgfscope}%
\begin{pgfscope}%
\pgfsetbuttcap%
\pgfsetroundjoin%
\definecolor{currentfill}{rgb}{0.000000,0.000000,0.000000}%
\pgfsetfillcolor{currentfill}%
\pgfsetlinewidth{0.803000pt}%
\definecolor{currentstroke}{rgb}{0.000000,0.000000,0.000000}%
\pgfsetstrokecolor{currentstroke}%
\pgfsetdash{}{0pt}%
\pgfsys@defobject{currentmarker}{\pgfqpoint{0.000000in}{-0.048611in}}{\pgfqpoint{0.000000in}{0.000000in}}{%
\pgfpathmoveto{\pgfqpoint{0.000000in}{0.000000in}}%
\pgfpathlineto{\pgfqpoint{0.000000in}{-0.048611in}}%
\pgfusepath{stroke,fill}%
}%
\begin{pgfscope}%
\pgfsys@transformshift{5.222294in}{0.524170in}%
\pgfsys@useobject{currentmarker}{}%
\end{pgfscope}%
\end{pgfscope}%
\begin{pgfscope}%
\definecolor{textcolor}{rgb}{0.000000,0.000000,0.000000}%
\pgfsetstrokecolor{textcolor}%
\pgfsetfillcolor{textcolor}%
\pgftext[x=5.222294in,y=0.426948in,,top]{\color{textcolor}\rmfamily\fontsize{8.000000}{9.600000}\selectfont \(\displaystyle {10}\)}%
\end{pgfscope}%
\begin{pgfscope}%
\definecolor{textcolor}{rgb}{0.000000,0.000000,0.000000}%
\pgfsetstrokecolor{textcolor}%
\pgfsetfillcolor{textcolor}%
\pgftext[x=2.944917in,y=0.272725in,,top]{\color{textcolor}\rmfamily\fontsize{10.000000}{12.000000}\selectfont Time}%
\end{pgfscope}%
\begin{pgfscope}%
\pgfsetbuttcap%
\pgfsetroundjoin%
\definecolor{currentfill}{rgb}{0.000000,0.000000,0.000000}%
\pgfsetfillcolor{currentfill}%
\pgfsetlinewidth{0.803000pt}%
\definecolor{currentstroke}{rgb}{0.000000,0.000000,0.000000}%
\pgfsetstrokecolor{currentstroke}%
\pgfsetdash{}{0pt}%
\pgfsys@defobject{currentmarker}{\pgfqpoint{-0.048611in}{0.000000in}}{\pgfqpoint{-0.000000in}{0.000000in}}{%
\pgfpathmoveto{\pgfqpoint{-0.000000in}{0.000000in}}%
\pgfpathlineto{\pgfqpoint{-0.048611in}{0.000000in}}%
\pgfusepath{stroke,fill}%
}%
\begin{pgfscope}%
\pgfsys@transformshift{0.667540in}{0.524170in}%
\pgfsys@useobject{currentmarker}{}%
\end{pgfscope}%
\end{pgfscope}%
\begin{pgfscope}%
\definecolor{textcolor}{rgb}{0.000000,0.000000,0.000000}%
\pgfsetstrokecolor{textcolor}%
\pgfsetfillcolor{textcolor}%
\pgftext[x=0.327644in, y=0.485614in, left, base]{\color{textcolor}\rmfamily\fontsize{8.000000}{9.600000}\selectfont \(\displaystyle {\ensuremath{-}1.0}\)}%
\end{pgfscope}%
\begin{pgfscope}%
\pgfsetbuttcap%
\pgfsetroundjoin%
\definecolor{currentfill}{rgb}{0.000000,0.000000,0.000000}%
\pgfsetfillcolor{currentfill}%
\pgfsetlinewidth{0.803000pt}%
\definecolor{currentstroke}{rgb}{0.000000,0.000000,0.000000}%
\pgfsetstrokecolor{currentstroke}%
\pgfsetdash{}{0pt}%
\pgfsys@defobject{currentmarker}{\pgfqpoint{-0.048611in}{0.000000in}}{\pgfqpoint{-0.000000in}{0.000000in}}{%
\pgfpathmoveto{\pgfqpoint{-0.000000in}{0.000000in}}%
\pgfpathlineto{\pgfqpoint{-0.048611in}{0.000000in}}%
\pgfusepath{stroke,fill}%
}%
\begin{pgfscope}%
\pgfsys@transformshift{0.667540in}{1.052970in}%
\pgfsys@useobject{currentmarker}{}%
\end{pgfscope}%
\end{pgfscope}%
\begin{pgfscope}%
\definecolor{textcolor}{rgb}{0.000000,0.000000,0.000000}%
\pgfsetstrokecolor{textcolor}%
\pgfsetfillcolor{textcolor}%
\pgftext[x=0.327644in, y=1.014414in, left, base]{\color{textcolor}\rmfamily\fontsize{8.000000}{9.600000}\selectfont \(\displaystyle {\ensuremath{-}0.5}\)}%
\end{pgfscope}%
\begin{pgfscope}%
\pgfsetbuttcap%
\pgfsetroundjoin%
\definecolor{currentfill}{rgb}{0.000000,0.000000,0.000000}%
\pgfsetfillcolor{currentfill}%
\pgfsetlinewidth{0.803000pt}%
\definecolor{currentstroke}{rgb}{0.000000,0.000000,0.000000}%
\pgfsetstrokecolor{currentstroke}%
\pgfsetdash{}{0pt}%
\pgfsys@defobject{currentmarker}{\pgfqpoint{-0.048611in}{0.000000in}}{\pgfqpoint{-0.000000in}{0.000000in}}{%
\pgfpathmoveto{\pgfqpoint{-0.000000in}{0.000000in}}%
\pgfpathlineto{\pgfqpoint{-0.048611in}{0.000000in}}%
\pgfusepath{stroke,fill}%
}%
\begin{pgfscope}%
\pgfsys@transformshift{0.667540in}{1.581770in}%
\pgfsys@useobject{currentmarker}{}%
\end{pgfscope}%
\end{pgfscope}%
\begin{pgfscope}%
\definecolor{textcolor}{rgb}{0.000000,0.000000,0.000000}%
\pgfsetstrokecolor{textcolor}%
\pgfsetfillcolor{textcolor}%
\pgftext[x=0.419467in, y=1.543214in, left, base]{\color{textcolor}\rmfamily\fontsize{8.000000}{9.600000}\selectfont \(\displaystyle {0.0}\)}%
\end{pgfscope}%
\begin{pgfscope}%
\pgfsetbuttcap%
\pgfsetroundjoin%
\definecolor{currentfill}{rgb}{0.000000,0.000000,0.000000}%
\pgfsetfillcolor{currentfill}%
\pgfsetlinewidth{0.803000pt}%
\definecolor{currentstroke}{rgb}{0.000000,0.000000,0.000000}%
\pgfsetstrokecolor{currentstroke}%
\pgfsetdash{}{0pt}%
\pgfsys@defobject{currentmarker}{\pgfqpoint{-0.048611in}{0.000000in}}{\pgfqpoint{-0.000000in}{0.000000in}}{%
\pgfpathmoveto{\pgfqpoint{-0.000000in}{0.000000in}}%
\pgfpathlineto{\pgfqpoint{-0.048611in}{0.000000in}}%
\pgfusepath{stroke,fill}%
}%
\begin{pgfscope}%
\pgfsys@transformshift{0.667540in}{2.110570in}%
\pgfsys@useobject{currentmarker}{}%
\end{pgfscope}%
\end{pgfscope}%
\begin{pgfscope}%
\definecolor{textcolor}{rgb}{0.000000,0.000000,0.000000}%
\pgfsetstrokecolor{textcolor}%
\pgfsetfillcolor{textcolor}%
\pgftext[x=0.419467in, y=2.072014in, left, base]{\color{textcolor}\rmfamily\fontsize{8.000000}{9.600000}\selectfont \(\displaystyle {0.5}\)}%
\end{pgfscope}%
\begin{pgfscope}%
\pgfsetbuttcap%
\pgfsetroundjoin%
\definecolor{currentfill}{rgb}{0.000000,0.000000,0.000000}%
\pgfsetfillcolor{currentfill}%
\pgfsetlinewidth{0.803000pt}%
\definecolor{currentstroke}{rgb}{0.000000,0.000000,0.000000}%
\pgfsetstrokecolor{currentstroke}%
\pgfsetdash{}{0pt}%
\pgfsys@defobject{currentmarker}{\pgfqpoint{-0.048611in}{0.000000in}}{\pgfqpoint{-0.000000in}{0.000000in}}{%
\pgfpathmoveto{\pgfqpoint{-0.000000in}{0.000000in}}%
\pgfpathlineto{\pgfqpoint{-0.048611in}{0.000000in}}%
\pgfusepath{stroke,fill}%
}%
\begin{pgfscope}%
\pgfsys@transformshift{0.667540in}{2.639370in}%
\pgfsys@useobject{currentmarker}{}%
\end{pgfscope}%
\end{pgfscope}%
\begin{pgfscope}%
\definecolor{textcolor}{rgb}{0.000000,0.000000,0.000000}%
\pgfsetstrokecolor{textcolor}%
\pgfsetfillcolor{textcolor}%
\pgftext[x=0.419467in, y=2.600814in, left, base]{\color{textcolor}\rmfamily\fontsize{8.000000}{9.600000}\selectfont \(\displaystyle {1.0}\)}%
\end{pgfscope}%
\begin{pgfscope}%
\pgfsetbuttcap%
\pgfsetroundjoin%
\definecolor{currentfill}{rgb}{0.000000,0.000000,0.000000}%
\pgfsetfillcolor{currentfill}%
\pgfsetlinewidth{0.803000pt}%
\definecolor{currentstroke}{rgb}{0.000000,0.000000,0.000000}%
\pgfsetstrokecolor{currentstroke}%
\pgfsetdash{}{0pt}%
\pgfsys@defobject{currentmarker}{\pgfqpoint{-0.048611in}{0.000000in}}{\pgfqpoint{-0.000000in}{0.000000in}}{%
\pgfpathmoveto{\pgfqpoint{-0.000000in}{0.000000in}}%
\pgfpathlineto{\pgfqpoint{-0.048611in}{0.000000in}}%
\pgfusepath{stroke,fill}%
}%
\begin{pgfscope}%
\pgfsys@transformshift{0.667540in}{3.168170in}%
\pgfsys@useobject{currentmarker}{}%
\end{pgfscope}%
\end{pgfscope}%
\begin{pgfscope}%
\definecolor{textcolor}{rgb}{0.000000,0.000000,0.000000}%
\pgfsetstrokecolor{textcolor}%
\pgfsetfillcolor{textcolor}%
\pgftext[x=0.419467in, y=3.129614in, left, base]{\color{textcolor}\rmfamily\fontsize{8.000000}{9.600000}\selectfont \(\displaystyle {1.5}\)}%
\end{pgfscope}%
\begin{pgfscope}%
\definecolor{textcolor}{rgb}{0.000000,0.000000,0.000000}%
\pgfsetstrokecolor{textcolor}%
\pgfsetfillcolor{textcolor}%
\pgftext[x=0.272089in,y=1.846170in,,bottom,rotate=90.000000]{\color{textcolor}\rmfamily\fontsize{10.000000}{12.000000}\selectfont Process Output}%
\end{pgfscope}%
\begin{pgfscope}%
\pgfpathrectangle{\pgfqpoint{0.667540in}{0.524170in}}{\pgfqpoint{4.554755in}{2.644000in}}%
\pgfusepath{clip}%
\pgfsetbuttcap%
\pgfsetroundjoin%
\pgfsetlinewidth{1.505625pt}%
\definecolor{currentstroke}{rgb}{0.003922,0.450980,0.698039}%
\pgfsetstrokecolor{currentstroke}%
\pgfsetstrokeopacity{0.700000}%
\pgfsetdash{{5.550000pt}{2.400000pt}}{0.000000pt}%
\pgfpathmoveto{\pgfqpoint{1.853769in}{0.514170in}}%
\pgfpathlineto{\pgfqpoint{1.897324in}{0.613494in}}%
\pgfpathlineto{\pgfqpoint{1.942871in}{0.712297in}}%
\pgfpathlineto{\pgfqpoint{1.988419in}{0.806281in}}%
\pgfpathlineto{\pgfqpoint{2.033966in}{0.895682in}}%
\pgfpathlineto{\pgfqpoint{2.079514in}{0.980723in}}%
\pgfpathlineto{\pgfqpoint{2.125061in}{1.061616in}}%
\pgfpathlineto{\pgfqpoint{2.170609in}{1.138564in}}%
\pgfpathlineto{\pgfqpoint{2.216156in}{1.211759in}}%
\pgfpathlineto{\pgfqpoint{2.261704in}{1.281385in}}%
\pgfpathlineto{\pgfqpoint{2.307252in}{1.347614in}}%
\pgfpathlineto{\pgfqpoint{2.352799in}{1.410614in}}%
\pgfpathlineto{\pgfqpoint{2.398347in}{1.470541in}}%
\pgfpathlineto{\pgfqpoint{2.443894in}{1.527545in}}%
\pgfpathlineto{\pgfqpoint{2.489442in}{1.581770in}}%
\pgfpathlineto{\pgfqpoint{2.534989in}{1.633350in}}%
\pgfpathlineto{\pgfqpoint{2.580537in}{1.682414in}}%
\pgfpathlineto{\pgfqpoint{2.626084in}{1.729085in}}%
\pgfpathlineto{\pgfqpoint{2.671632in}{1.773480in}}%
\pgfpathlineto{\pgfqpoint{2.717179in}{1.815710in}}%
\pgfpathlineto{\pgfqpoint{2.762727in}{1.855880in}}%
\pgfpathlineto{\pgfqpoint{2.808275in}{1.894092in}}%
\pgfpathlineto{\pgfqpoint{2.853822in}{1.930439in}}%
\pgfpathlineto{\pgfqpoint{2.899370in}{1.965014in}}%
\pgfpathlineto{\pgfqpoint{2.944917in}{1.997903in}}%
\pgfpathlineto{\pgfqpoint{2.990465in}{2.029188in}}%
\pgfpathlineto{\pgfqpoint{3.036012in}{2.058947in}}%
\pgfpathlineto{\pgfqpoint{3.081560in}{2.087254in}}%
\pgfpathlineto{\pgfqpoint{3.127107in}{2.114181in}}%
\pgfpathlineto{\pgfqpoint{3.172655in}{2.139795in}}%
\pgfpathlineto{\pgfqpoint{3.218202in}{2.164160in}}%
\pgfpathlineto{\pgfqpoint{3.263750in}{2.187336in}}%
\pgfpathlineto{\pgfqpoint{3.309298in}{2.209382in}}%
\pgfpathlineto{\pgfqpoint{3.354845in}{2.230353in}}%
\pgfpathlineto{\pgfqpoint{3.400393in}{2.250301in}}%
\pgfpathlineto{\pgfqpoint{3.445940in}{2.269276in}}%
\pgfpathlineto{\pgfqpoint{3.491488in}{2.287325in}}%
\pgfpathlineto{\pgfqpoint{3.537035in}{2.304495in}}%
\pgfpathlineto{\pgfqpoint{3.582583in}{2.320827in}}%
\pgfpathlineto{\pgfqpoint{3.628130in}{2.336362in}}%
\pgfpathlineto{\pgfqpoint{3.673678in}{2.351140in}}%
\pgfpathlineto{\pgfqpoint{3.719225in}{2.365197in}}%
\pgfpathlineto{\pgfqpoint{3.764773in}{2.378569in}}%
\pgfpathlineto{\pgfqpoint{3.810321in}{2.391288in}}%
\pgfpathlineto{\pgfqpoint{3.855868in}{2.403387in}}%
\pgfpathlineto{\pgfqpoint{3.901416in}{2.414896in}}%
\pgfpathlineto{\pgfqpoint{3.946963in}{2.425844in}}%
\pgfpathlineto{\pgfqpoint{3.992511in}{2.436258in}}%
\pgfpathlineto{\pgfqpoint{4.038058in}{2.446164in}}%
\pgfpathlineto{\pgfqpoint{4.083606in}{2.455586in}}%
\pgfpathlineto{\pgfqpoint{4.129153in}{2.464550in}}%
\pgfpathlineto{\pgfqpoint{4.174701in}{2.473076in}}%
\pgfpathlineto{\pgfqpoint{4.220248in}{2.481186in}}%
\pgfpathlineto{\pgfqpoint{4.265796in}{2.488901in}}%
\pgfpathlineto{\pgfqpoint{4.311344in}{2.496239in}}%
\pgfpathlineto{\pgfqpoint{4.356891in}{2.503220in}}%
\pgfpathlineto{\pgfqpoint{4.402439in}{2.509860in}}%
\pgfpathlineto{\pgfqpoint{4.447986in}{2.516176in}}%
\pgfpathlineto{\pgfqpoint{4.493534in}{2.522184in}}%
\pgfpathlineto{\pgfqpoint{4.539081in}{2.527900in}}%
\pgfpathlineto{\pgfqpoint{4.584629in}{2.533336in}}%
\pgfpathlineto{\pgfqpoint{4.630176in}{2.538507in}}%
\pgfpathlineto{\pgfqpoint{4.675724in}{2.543427in}}%
\pgfpathlineto{\pgfqpoint{4.721271in}{2.548106in}}%
\pgfpathlineto{\pgfqpoint{4.766819in}{2.552557in}}%
\pgfpathlineto{\pgfqpoint{4.812367in}{2.556791in}}%
\pgfpathlineto{\pgfqpoint{4.857914in}{2.560818in}}%
\pgfpathlineto{\pgfqpoint{4.903462in}{2.564649in}}%
\pgfpathlineto{\pgfqpoint{4.949009in}{2.568293in}}%
\pgfpathlineto{\pgfqpoint{4.994557in}{2.571760in}}%
\pgfpathlineto{\pgfqpoint{5.040104in}{2.575057in}}%
\pgfpathlineto{\pgfqpoint{5.085652in}{2.578194in}}%
\pgfpathlineto{\pgfqpoint{5.131199in}{2.581177in}}%
\pgfpathlineto{\pgfqpoint{5.176747in}{2.584015in}}%
\pgfpathlineto{\pgfqpoint{5.222294in}{2.586715in}}%
\pgfusepath{stroke}%
\end{pgfscope}%
\begin{pgfscope}%
\pgfpathrectangle{\pgfqpoint{0.667540in}{0.524170in}}{\pgfqpoint{4.554755in}{2.644000in}}%
\pgfusepath{clip}%
\pgfsetbuttcap%
\pgfsetroundjoin%
\pgfsetlinewidth{1.505625pt}%
\definecolor{currentstroke}{rgb}{0.007843,0.619608,0.450980}%
\pgfsetstrokecolor{currentstroke}%
\pgfsetstrokeopacity{0.700000}%
\pgfsetdash{{1.500000pt}{2.475000pt}}{0.000000pt}%
\pgfpathmoveto{\pgfqpoint{0.667540in}{1.581770in}}%
\pgfpathlineto{\pgfqpoint{2.489442in}{1.581770in}}%
\pgfpathlineto{\pgfqpoint{2.489487in}{2.639370in}}%
\pgfpathlineto{\pgfqpoint{5.222294in}{2.639370in}}%
\pgfusepath{stroke}%
\end{pgfscope}%
\begin{pgfscope}%
\pgfpathrectangle{\pgfqpoint{0.667540in}{0.524170in}}{\pgfqpoint{4.554755in}{2.644000in}}%
\pgfusepath{clip}%
\pgfsetrectcap%
\pgfsetroundjoin%
\pgfsetlinewidth{1.505625pt}%
\definecolor{currentstroke}{rgb}{0.835294,0.368627,0.000000}%
\pgfsetstrokecolor{currentstroke}%
\pgfsetdash{}{0pt}%
\pgfpathmoveto{\pgfqpoint{0.667540in}{1.581770in}}%
\pgfpathlineto{\pgfqpoint{0.713087in}{1.581770in}}%
\pgfpathlineto{\pgfqpoint{0.758635in}{1.581770in}}%
\pgfpathlineto{\pgfqpoint{0.804183in}{1.581770in}}%
\pgfpathlineto{\pgfqpoint{0.849730in}{1.581770in}}%
\pgfpathlineto{\pgfqpoint{0.895278in}{1.581770in}}%
\pgfpathlineto{\pgfqpoint{0.940825in}{1.581770in}}%
\pgfpathlineto{\pgfqpoint{0.986373in}{1.581770in}}%
\pgfpathlineto{\pgfqpoint{1.031920in}{1.581770in}}%
\pgfpathlineto{\pgfqpoint{1.077468in}{1.581770in}}%
\pgfpathlineto{\pgfqpoint{1.123015in}{1.581770in}}%
\pgfpathlineto{\pgfqpoint{1.168563in}{1.581770in}}%
\pgfpathlineto{\pgfqpoint{1.214110in}{1.581770in}}%
\pgfpathlineto{\pgfqpoint{1.259658in}{1.581770in}}%
\pgfpathlineto{\pgfqpoint{1.305206in}{1.581770in}}%
\pgfpathlineto{\pgfqpoint{1.350753in}{1.581770in}}%
\pgfpathlineto{\pgfqpoint{1.396301in}{1.581770in}}%
\pgfpathlineto{\pgfqpoint{1.441848in}{1.581770in}}%
\pgfpathlineto{\pgfqpoint{1.487396in}{1.581770in}}%
\pgfpathlineto{\pgfqpoint{1.532943in}{1.581770in}}%
\pgfpathlineto{\pgfqpoint{1.578491in}{1.581770in}}%
\pgfpathlineto{\pgfqpoint{1.624038in}{1.581770in}}%
\pgfpathlineto{\pgfqpoint{1.669586in}{1.581770in}}%
\pgfpathlineto{\pgfqpoint{1.715133in}{1.581770in}}%
\pgfpathlineto{\pgfqpoint{1.760681in}{1.581770in}}%
\pgfpathlineto{\pgfqpoint{1.806229in}{1.581770in}}%
\pgfpathlineto{\pgfqpoint{1.851776in}{1.581770in}}%
\pgfpathlineto{\pgfqpoint{1.897324in}{1.581770in}}%
\pgfpathlineto{\pgfqpoint{1.942871in}{1.581770in}}%
\pgfpathlineto{\pgfqpoint{1.988419in}{1.581770in}}%
\pgfpathlineto{\pgfqpoint{2.033966in}{1.581770in}}%
\pgfpathlineto{\pgfqpoint{2.079514in}{1.581770in}}%
\pgfpathlineto{\pgfqpoint{2.125061in}{1.581770in}}%
\pgfpathlineto{\pgfqpoint{2.170609in}{1.581770in}}%
\pgfpathlineto{\pgfqpoint{2.216156in}{1.581770in}}%
\pgfpathlineto{\pgfqpoint{2.261704in}{1.581770in}}%
\pgfpathlineto{\pgfqpoint{2.307252in}{1.581770in}}%
\pgfpathlineto{\pgfqpoint{2.352799in}{1.581770in}}%
\pgfpathlineto{\pgfqpoint{2.398347in}{1.581770in}}%
\pgfpathlineto{\pgfqpoint{2.443894in}{1.581770in}}%
\pgfpathlineto{\pgfqpoint{2.489442in}{1.581770in}}%
\pgfpathlineto{\pgfqpoint{2.534989in}{1.633350in}}%
\pgfpathlineto{\pgfqpoint{2.580537in}{1.682414in}}%
\pgfpathlineto{\pgfqpoint{2.626084in}{1.729085in}}%
\pgfpathlineto{\pgfqpoint{2.671632in}{1.773480in}}%
\pgfpathlineto{\pgfqpoint{2.717179in}{1.815710in}}%
\pgfpathlineto{\pgfqpoint{2.762727in}{1.855880in}}%
\pgfpathlineto{\pgfqpoint{2.808275in}{1.894092in}}%
\pgfpathlineto{\pgfqpoint{2.853822in}{1.930439in}}%
\pgfpathlineto{\pgfqpoint{2.899370in}{1.965014in}}%
\pgfpathlineto{\pgfqpoint{2.944917in}{1.997903in}}%
\pgfpathlineto{\pgfqpoint{2.990465in}{2.029188in}}%
\pgfpathlineto{\pgfqpoint{3.036012in}{2.058946in}}%
\pgfpathlineto{\pgfqpoint{3.081560in}{2.087254in}}%
\pgfpathlineto{\pgfqpoint{3.127107in}{2.114181in}}%
\pgfpathlineto{\pgfqpoint{3.172655in}{2.139795in}}%
\pgfpathlineto{\pgfqpoint{3.218202in}{2.164159in}}%
\pgfpathlineto{\pgfqpoint{3.263750in}{2.187336in}}%
\pgfpathlineto{\pgfqpoint{3.309298in}{2.209382in}}%
\pgfpathlineto{\pgfqpoint{3.354845in}{2.230352in}}%
\pgfpathlineto{\pgfqpoint{3.400393in}{2.250300in}}%
\pgfpathlineto{\pgfqpoint{3.445940in}{2.269275in}}%
\pgfpathlineto{\pgfqpoint{3.491488in}{2.287325in}}%
\pgfpathlineto{\pgfqpoint{3.537035in}{2.304495in}}%
\pgfpathlineto{\pgfqpoint{3.582583in}{2.320827in}}%
\pgfpathlineto{\pgfqpoint{3.628130in}{2.336362in}}%
\pgfpathlineto{\pgfqpoint{3.673678in}{2.351140in}}%
\pgfpathlineto{\pgfqpoint{3.719225in}{2.365197in}}%
\pgfpathlineto{\pgfqpoint{3.764773in}{2.378569in}}%
\pgfpathlineto{\pgfqpoint{3.810321in}{2.391288in}}%
\pgfpathlineto{\pgfqpoint{3.855868in}{2.403387in}}%
\pgfpathlineto{\pgfqpoint{3.901416in}{2.414896in}}%
\pgfpathlineto{\pgfqpoint{3.946963in}{2.425844in}}%
\pgfpathlineto{\pgfqpoint{3.992511in}{2.436258in}}%
\pgfpathlineto{\pgfqpoint{4.038058in}{2.446163in}}%
\pgfpathlineto{\pgfqpoint{4.083606in}{2.455586in}}%
\pgfpathlineto{\pgfqpoint{4.129153in}{2.464549in}}%
\pgfpathlineto{\pgfqpoint{4.174701in}{2.473075in}}%
\pgfpathlineto{\pgfqpoint{4.220248in}{2.481186in}}%
\pgfpathlineto{\pgfqpoint{4.265796in}{2.488900in}}%
\pgfpathlineto{\pgfqpoint{4.311344in}{2.496239in}}%
\pgfpathlineto{\pgfqpoint{4.356891in}{2.503219in}}%
\pgfpathlineto{\pgfqpoint{4.402439in}{2.509860in}}%
\pgfpathlineto{\pgfqpoint{4.447986in}{2.516176in}}%
\pgfpathlineto{\pgfqpoint{4.493534in}{2.522184in}}%
\pgfpathlineto{\pgfqpoint{4.539081in}{2.527899in}}%
\pgfpathlineto{\pgfqpoint{4.584629in}{2.533336in}}%
\pgfpathlineto{\pgfqpoint{4.630176in}{2.538507in}}%
\pgfpathlineto{\pgfqpoint{4.675724in}{2.543426in}}%
\pgfpathlineto{\pgfqpoint{4.721271in}{2.548105in}}%
\pgfpathlineto{\pgfqpoint{4.766819in}{2.552557in}}%
\pgfpathlineto{\pgfqpoint{4.812367in}{2.556790in}}%
\pgfpathlineto{\pgfqpoint{4.857914in}{2.560818in}}%
\pgfpathlineto{\pgfqpoint{4.903462in}{2.564649in}}%
\pgfpathlineto{\pgfqpoint{4.949009in}{2.568293in}}%
\pgfpathlineto{\pgfqpoint{4.994557in}{2.571760in}}%
\pgfpathlineto{\pgfqpoint{5.040104in}{2.575057in}}%
\pgfpathlineto{\pgfqpoint{5.085652in}{2.578194in}}%
\pgfpathlineto{\pgfqpoint{5.131199in}{2.581177in}}%
\pgfpathlineto{\pgfqpoint{5.176747in}{2.584015in}}%
\pgfpathlineto{\pgfqpoint{5.222294in}{2.586715in}}%
\pgfusepath{stroke}%
\end{pgfscope}%
\begin{pgfscope}%
\pgfsetrectcap%
\pgfsetmiterjoin%
\pgfsetlinewidth{0.803000pt}%
\definecolor{currentstroke}{rgb}{0.000000,0.000000,0.000000}%
\pgfsetstrokecolor{currentstroke}%
\pgfsetdash{}{0pt}%
\pgfpathmoveto{\pgfqpoint{0.667540in}{0.524170in}}%
\pgfpathlineto{\pgfqpoint{0.667540in}{3.168170in}}%
\pgfusepath{stroke}%
\end{pgfscope}%
\begin{pgfscope}%
\pgfsetrectcap%
\pgfsetmiterjoin%
\pgfsetlinewidth{0.803000pt}%
\definecolor{currentstroke}{rgb}{0.000000,0.000000,0.000000}%
\pgfsetstrokecolor{currentstroke}%
\pgfsetdash{}{0pt}%
\pgfpathmoveto{\pgfqpoint{5.222294in}{0.524170in}}%
\pgfpathlineto{\pgfqpoint{5.222294in}{3.168170in}}%
\pgfusepath{stroke}%
\end{pgfscope}%
\begin{pgfscope}%
\pgfsetrectcap%
\pgfsetmiterjoin%
\pgfsetlinewidth{0.803000pt}%
\definecolor{currentstroke}{rgb}{0.000000,0.000000,0.000000}%
\pgfsetstrokecolor{currentstroke}%
\pgfsetdash{}{0pt}%
\pgfpathmoveto{\pgfqpoint{0.667540in}{0.524170in}}%
\pgfpathlineto{\pgfqpoint{5.222294in}{0.524170in}}%
\pgfusepath{stroke}%
\end{pgfscope}%
\begin{pgfscope}%
\pgfsetrectcap%
\pgfsetmiterjoin%
\pgfsetlinewidth{0.803000pt}%
\definecolor{currentstroke}{rgb}{0.000000,0.000000,0.000000}%
\pgfsetstrokecolor{currentstroke}%
\pgfsetdash{}{0pt}%
\pgfpathmoveto{\pgfqpoint{0.667540in}{3.168170in}}%
\pgfpathlineto{\pgfqpoint{5.222294in}{3.168170in}}%
\pgfusepath{stroke}%
\end{pgfscope}%
\begin{pgfscope}%
\pgfsetbuttcap%
\pgfsetmiterjoin%
\definecolor{currentfill}{rgb}{1.000000,1.000000,1.000000}%
\pgfsetfillcolor{currentfill}%
\pgfsetfillopacity{0.800000}%
\pgfsetlinewidth{1.003750pt}%
\definecolor{currentstroke}{rgb}{0.800000,0.800000,0.800000}%
\pgfsetstrokecolor{currentstroke}%
\pgfsetstrokeopacity{0.800000}%
\pgfsetdash{}{0pt}%
\pgfpathmoveto{\pgfqpoint{0.745318in}{2.542112in}}%
\pgfpathlineto{\pgfqpoint{1.643290in}{2.542112in}}%
\pgfpathquadraticcurveto{\pgfqpoint{1.665513in}{2.542112in}}{\pgfqpoint{1.665513in}{2.564334in}}%
\pgfpathlineto{\pgfqpoint{1.665513in}{3.090392in}}%
\pgfpathquadraticcurveto{\pgfqpoint{1.665513in}{3.112614in}}{\pgfqpoint{1.643290in}{3.112614in}}%
\pgfpathlineto{\pgfqpoint{0.745318in}{3.112614in}}%
\pgfpathquadraticcurveto{\pgfqpoint{0.723095in}{3.112614in}}{\pgfqpoint{0.723095in}{3.090392in}}%
\pgfpathlineto{\pgfqpoint{0.723095in}{2.564334in}}%
\pgfpathquadraticcurveto{\pgfqpoint{0.723095in}{2.542112in}}{\pgfqpoint{0.745318in}{2.542112in}}%
\pgfpathlineto{\pgfqpoint{0.745318in}{2.542112in}}%
\pgfpathclose%
\pgfusepath{stroke,fill}%
\end{pgfscope}%
\begin{pgfscope}%
\pgfsetbuttcap%
\pgfsetroundjoin%
\pgfsetlinewidth{1.505625pt}%
\definecolor{currentstroke}{rgb}{0.003922,0.450980,0.698039}%
\pgfsetstrokecolor{currentstroke}%
\pgfsetstrokeopacity{0.700000}%
\pgfsetdash{{5.550000pt}{2.400000pt}}{0.000000pt}%
\pgfpathmoveto{\pgfqpoint{0.767540in}{2.980334in}}%
\pgfpathlineto{\pgfqpoint{0.878651in}{2.980334in}}%
\pgfpathlineto{\pgfqpoint{0.989762in}{2.980334in}}%
\pgfusepath{stroke}%
\end{pgfscope}%
\begin{pgfscope}%
\definecolor{textcolor}{rgb}{0.000000,0.000000,0.000000}%
\pgfsetstrokecolor{textcolor}%
\pgfsetfillcolor{textcolor}%
\pgftext[x=1.078651in,y=2.941445in,left,base]{\color{textcolor}\rmfamily\fontsize{8.000000}{9.600000}\selectfont \(\displaystyle 1-e^{-\frac{t-\theta}{\tau} }\)}%
\end{pgfscope}%
\begin{pgfscope}%
\pgfsetbuttcap%
\pgfsetroundjoin%
\pgfsetlinewidth{1.505625pt}%
\definecolor{currentstroke}{rgb}{0.007843,0.619608,0.450980}%
\pgfsetstrokecolor{currentstroke}%
\pgfsetstrokeopacity{0.700000}%
\pgfsetdash{{1.500000pt}{2.475000pt}}{0.000000pt}%
\pgfpathmoveto{\pgfqpoint{0.767540in}{2.819889in}}%
\pgfpathlineto{\pgfqpoint{0.878651in}{2.819889in}}%
\pgfpathlineto{\pgfqpoint{0.989762in}{2.819889in}}%
\pgfusepath{stroke}%
\end{pgfscope}%
\begin{pgfscope}%
\definecolor{textcolor}{rgb}{0.000000,0.000000,0.000000}%
\pgfsetstrokecolor{textcolor}%
\pgfsetfillcolor{textcolor}%
\pgftext[x=1.078651in,y=2.781001in,left,base]{\color{textcolor}\rmfamily\fontsize{8.000000}{9.600000}\selectfont \(\displaystyle H(t- \theta)\)}%
\end{pgfscope}%
\begin{pgfscope}%
\pgfsetrectcap%
\pgfsetroundjoin%
\pgfsetlinewidth{1.505625pt}%
\definecolor{currentstroke}{rgb}{0.835294,0.368627,0.000000}%
\pgfsetstrokecolor{currentstroke}%
\pgfsetdash{}{0pt}%
\pgfpathmoveto{\pgfqpoint{0.767540in}{2.653223in}}%
\pgfpathlineto{\pgfqpoint{0.878651in}{2.653223in}}%
\pgfpathlineto{\pgfqpoint{0.989762in}{2.653223in}}%
\pgfusepath{stroke}%
\end{pgfscope}%
\begin{pgfscope}%
\definecolor{textcolor}{rgb}{0.000000,0.000000,0.000000}%
\pgfsetstrokecolor{textcolor}%
\pgfsetfillcolor{textcolor}%
\pgftext[x=1.078651in,y=2.614334in,left,base]{\color{textcolor}\rmfamily\fontsize{8.000000}{9.600000}\selectfont \(\displaystyle y(t)\)}%
\end{pgfscope}%
\end{pgfpicture}%
\makeatother%
\endgroup%
% data/simulations/sim_laplace_fopdt.py
    \caption{Time domain plot of a first order plus dead time model showing individual components of the model and the composite function $y(t)$. Model parameters used: $K= \Delta u = 1$, $\tau=2$, $\theta=4$.}
    \label{fig:fopdt}
\end{figure}

So far, only open-loop systems were discussed. Using the FOPDT model, the system parameters can now be extracted from an existing system using a fit to the time domain reaction of such a system to a step input. Having extracted the system parameters, the next step is to design a controller around the system and close the loop to realize a controlled system. This is shown in the next section.

\subsection{PID Controller Basics}%
\label{sec:pid_controller_basics}
While there are many different types of controllers, like the bang–bang controller utilized in many temperature controller, which turns on at a certain threshold and turns off at another one, producing the saw-tooth shaped room temperature curve shown in figure \ref{fig:lab_temperature_start_of_project}, a continuous control system is desired to keep fluctuations to a minimum. The most commonly used controller type for non-integrating systems is the proportional–integral–derivative (PID) controller \cite{pid_in_industry}. A non-integrating system is a system without memory whose steady state does not depend on previous inputs. The advantage of applying a PID controller is that the controller does not need any special knowledge about the system model. A universal PID is simple to implement and can be tuned to control a wide range of different systems. While there are many variations of the PID algorithm \cite{pid_controller}, this section only introduces the basic, parallel, PID controller commonly used in digital implementation and deals with some of the shortcomings in practical applications.

In order to extend the FOPDT system derived in the previous section \ref{sec:temperature_control_model}, with the PID controller one has to move to a closed-loop system. Adding to figure \ref{fig:closed_loop} and inserting a new control block into the transfer function yields figure \ref{fig:closed_loop_pid}.
\begin{figure}[ht]
    \centering
    \scalebox{1}{%
        \import{figures/}{closed_loop_pid.tex}
    }% scalebox
    \caption{Closed-loop system with a PID controller.}
    \label{fig:closed_loop_pid}
\end{figure}

The error signal $E(s)$ used by the PID controller is the difference between the setpoint and the control parameter, in this case the room temperature. The transfer function of the PID controller can be split into three parts. A proportional part that is proportional to the error representing the present state, an integral part that is proportional to the accumulated error, representing the past state, and a derivative part that is proportional to the change in the error, extrapolating into the future. Analytically, it can be written as
\begin{align}
    c(t) &= k_p e(t) + k_i \int_0^t e(\tau) \,d\tau + k_d \frac{\mathrm{d}e(t)}{\mathrm{d}t} \label{eqn:pid_controller}\\
    C(s) &= k_p + k_i \frac{1}{s} + k_d s \,. \label{eqn:pid_controller_laplace}
\end{align}

The following discussion will mostly focus on equation \ref{eqn:pid_controller}, because, the time-domain equation is the one that can be implemented in software. As hinted above, there are a few shortcomings with the classic PID equation, when used in a real system which requires dynamic changes of the the setpoint or $k_i$.

The first problem that needs addressing is occurring when changing the PID parameter $k_i$, because equation \ref{eqn:pid_controller} is given for a time-independent $k_i$. Assuming a settled system without external disturbances, the output is fully determined by the integrator value because the error is zero. Now, when $k_i$ is changed, the output immediately changes, due to the change of the integral term. This is unintended. To fix it, the integral term must be changed to
\begin{equation}
    k_i \int_0^t e(\tau) \,d\tau \Rightarrow \int_0^t k_i(\tau) e(\tau) \,d\tau \,.
\end{equation}

This way, when adjusting $k_i$, its new value is applied to future error values only and there is no sudden kick.

The next issue is called \text{derivative kick}. When looking at the derivative part of equation \ref{eqn:pid_controller}, it can be seen that when instantly changing the setpoint, as in a step function, $\frac{\mathrm{d}e(t)}{\mathrm{d}t} \to \infty$. This behaviour is not intended and to fix this, the derivative part can be modified as follows.
\begin{align}
    \frac{\mathrm{d}e(t)}{\mathrm{d}t} &= \frac{\mathrm{d}\left(u(t) - y(t)\right)}{\mathrm{d}t} = \underbrace{\cancel{\frac{\mathrm{d}u(t)}{\mathrm{d}t}}}_{\to \infty} - \frac{\mathrm{d}y(t)}{\mathrm{d}t} \nonumber\\
    &=- \frac{\mathrm{d}y(t)}{\mathrm{d}t}
\end{align}

The new derivative term is equal to the unmodified one, except in the case of setpoint changes. Removing the setpoint from the equation, the controller behaves as intended. This solution is sometimes called \textit{derivative on measurement} as opposed to \textit{derivative on error}.
\begin{figure}[hb]
    \centering
    %% Creator: Matplotlib, PGF backend
%%
%% To include the figure in your LaTeX document, write
%%   \input{<filename>.pgf}
%%
%% Make sure the required packages are loaded in your preamble
%%   \usepackage{pgf}
%%
%% Also ensure that all the required font packages are loaded; for instance,
%% the lmodern package is sometimes necessary when using math font.
%%   \usepackage{lmodern}
%%
%% Figures using additional raster images can only be included by \input if
%% they are in the same directory as the main LaTeX file. For loading figures
%% from other directories you can use the `import` package
%%   \usepackage{import}
%%
%% and then include the figures with
%%   \import{<path to file>}{<filename>.pgf}
%%
%% Matplotlib used the following preamble
%%   \usepackage{siunitx}
%%   \usepackage{fontspec}
%%   \makeatletter\@ifpackageloaded{underscore}{}{\usepackage[strings]{underscore}}\makeatother
%%
\begingroup%
\makeatletter%
\begin{pgfpicture}%
\pgfpathrectangle{\pgfpointorigin}{\pgfqpoint{5.431103in}{3.356606in}}%
\pgfusepath{use as bounding box, clip}%
\begin{pgfscope}%
\pgfsetbuttcap%
\pgfsetmiterjoin%
\definecolor{currentfill}{rgb}{1.000000,1.000000,1.000000}%
\pgfsetfillcolor{currentfill}%
\pgfsetlinewidth{0.000000pt}%
\definecolor{currentstroke}{rgb}{1.000000,1.000000,1.000000}%
\pgfsetstrokecolor{currentstroke}%
\pgfsetdash{}{0pt}%
\pgfpathmoveto{\pgfqpoint{0.000000in}{0.000000in}}%
\pgfpathlineto{\pgfqpoint{5.431103in}{0.000000in}}%
\pgfpathlineto{\pgfqpoint{5.431103in}{3.356606in}}%
\pgfpathlineto{\pgfqpoint{0.000000in}{3.356606in}}%
\pgfpathlineto{\pgfqpoint{0.000000in}{0.000000in}}%
\pgfpathclose%
\pgfusepath{fill}%
\end{pgfscope}%
\begin{pgfscope}%
\pgfsetbuttcap%
\pgfsetmiterjoin%
\definecolor{currentfill}{rgb}{1.000000,1.000000,1.000000}%
\pgfsetfillcolor{currentfill}%
\pgfsetlinewidth{0.000000pt}%
\definecolor{currentstroke}{rgb}{0.000000,0.000000,0.000000}%
\pgfsetstrokecolor{currentstroke}%
\pgfsetstrokeopacity{0.000000}%
\pgfsetdash{}{0pt}%
\pgfpathmoveto{\pgfqpoint{0.594124in}{0.540713in}}%
\pgfpathlineto{\pgfqpoint{5.281103in}{0.540713in}}%
\pgfpathlineto{\pgfqpoint{5.281103in}{3.206606in}}%
\pgfpathlineto{\pgfqpoint{0.594124in}{3.206606in}}%
\pgfpathlineto{\pgfqpoint{0.594124in}{0.540713in}}%
\pgfpathclose%
\pgfusepath{fill}%
\end{pgfscope}%
\begin{pgfscope}%
\pgfsetbuttcap%
\pgfsetroundjoin%
\definecolor{currentfill}{rgb}{0.000000,0.000000,0.000000}%
\pgfsetfillcolor{currentfill}%
\pgfsetlinewidth{0.803000pt}%
\definecolor{currentstroke}{rgb}{0.000000,0.000000,0.000000}%
\pgfsetstrokecolor{currentstroke}%
\pgfsetdash{}{0pt}%
\pgfsys@defobject{currentmarker}{\pgfqpoint{0.000000in}{-0.048611in}}{\pgfqpoint{0.000000in}{0.000000in}}{%
\pgfpathmoveto{\pgfqpoint{0.000000in}{0.000000in}}%
\pgfpathlineto{\pgfqpoint{0.000000in}{-0.048611in}}%
\pgfusepath{stroke,fill}%
}%
\begin{pgfscope}%
\pgfsys@transformshift{0.807169in}{0.540713in}%
\pgfsys@useobject{currentmarker}{}%
\end{pgfscope}%
\end{pgfscope}%
\begin{pgfscope}%
\definecolor{textcolor}{rgb}{0.000000,0.000000,0.000000}%
\pgfsetstrokecolor{textcolor}%
\pgfsetfillcolor{textcolor}%
\pgftext[x=0.807169in,y=0.443491in,,top]{\color{textcolor}\rmfamily\fontsize{8.000000}{9.600000}\selectfont \(\displaystyle {10^{-2}}\)}%
\end{pgfscope}%
\begin{pgfscope}%
\pgfsetbuttcap%
\pgfsetroundjoin%
\definecolor{currentfill}{rgb}{0.000000,0.000000,0.000000}%
\pgfsetfillcolor{currentfill}%
\pgfsetlinewidth{0.803000pt}%
\definecolor{currentstroke}{rgb}{0.000000,0.000000,0.000000}%
\pgfsetstrokecolor{currentstroke}%
\pgfsetdash{}{0pt}%
\pgfsys@defobject{currentmarker}{\pgfqpoint{0.000000in}{-0.048611in}}{\pgfqpoint{0.000000in}{0.000000in}}{%
\pgfpathmoveto{\pgfqpoint{0.000000in}{0.000000in}}%
\pgfpathlineto{\pgfqpoint{0.000000in}{-0.048611in}}%
\pgfusepath{stroke,fill}%
}%
\begin{pgfscope}%
\pgfsys@transformshift{1.517317in}{0.540713in}%
\pgfsys@useobject{currentmarker}{}%
\end{pgfscope}%
\end{pgfscope}%
\begin{pgfscope}%
\definecolor{textcolor}{rgb}{0.000000,0.000000,0.000000}%
\pgfsetstrokecolor{textcolor}%
\pgfsetfillcolor{textcolor}%
\pgftext[x=1.517317in,y=0.443491in,,top]{\color{textcolor}\rmfamily\fontsize{8.000000}{9.600000}\selectfont \(\displaystyle {10^{-1}}\)}%
\end{pgfscope}%
\begin{pgfscope}%
\pgfsetbuttcap%
\pgfsetroundjoin%
\definecolor{currentfill}{rgb}{0.000000,0.000000,0.000000}%
\pgfsetfillcolor{currentfill}%
\pgfsetlinewidth{0.803000pt}%
\definecolor{currentstroke}{rgb}{0.000000,0.000000,0.000000}%
\pgfsetstrokecolor{currentstroke}%
\pgfsetdash{}{0pt}%
\pgfsys@defobject{currentmarker}{\pgfqpoint{0.000000in}{-0.048611in}}{\pgfqpoint{0.000000in}{0.000000in}}{%
\pgfpathmoveto{\pgfqpoint{0.000000in}{0.000000in}}%
\pgfpathlineto{\pgfqpoint{0.000000in}{-0.048611in}}%
\pgfusepath{stroke,fill}%
}%
\begin{pgfscope}%
\pgfsys@transformshift{2.227465in}{0.540713in}%
\pgfsys@useobject{currentmarker}{}%
\end{pgfscope}%
\end{pgfscope}%
\begin{pgfscope}%
\definecolor{textcolor}{rgb}{0.000000,0.000000,0.000000}%
\pgfsetstrokecolor{textcolor}%
\pgfsetfillcolor{textcolor}%
\pgftext[x=2.227465in,y=0.443491in,,top]{\color{textcolor}\rmfamily\fontsize{8.000000}{9.600000}\selectfont \(\displaystyle {10^{0}}\)}%
\end{pgfscope}%
\begin{pgfscope}%
\pgfsetbuttcap%
\pgfsetroundjoin%
\definecolor{currentfill}{rgb}{0.000000,0.000000,0.000000}%
\pgfsetfillcolor{currentfill}%
\pgfsetlinewidth{0.803000pt}%
\definecolor{currentstroke}{rgb}{0.000000,0.000000,0.000000}%
\pgfsetstrokecolor{currentstroke}%
\pgfsetdash{}{0pt}%
\pgfsys@defobject{currentmarker}{\pgfqpoint{0.000000in}{-0.048611in}}{\pgfqpoint{0.000000in}{0.000000in}}{%
\pgfpathmoveto{\pgfqpoint{0.000000in}{0.000000in}}%
\pgfpathlineto{\pgfqpoint{0.000000in}{-0.048611in}}%
\pgfusepath{stroke,fill}%
}%
\begin{pgfscope}%
\pgfsys@transformshift{2.937613in}{0.540713in}%
\pgfsys@useobject{currentmarker}{}%
\end{pgfscope}%
\end{pgfscope}%
\begin{pgfscope}%
\definecolor{textcolor}{rgb}{0.000000,0.000000,0.000000}%
\pgfsetstrokecolor{textcolor}%
\pgfsetfillcolor{textcolor}%
\pgftext[x=2.937613in,y=0.443491in,,top]{\color{textcolor}\rmfamily\fontsize{8.000000}{9.600000}\selectfont \(\displaystyle {10^{1}}\)}%
\end{pgfscope}%
\begin{pgfscope}%
\pgfsetbuttcap%
\pgfsetroundjoin%
\definecolor{currentfill}{rgb}{0.000000,0.000000,0.000000}%
\pgfsetfillcolor{currentfill}%
\pgfsetlinewidth{0.803000pt}%
\definecolor{currentstroke}{rgb}{0.000000,0.000000,0.000000}%
\pgfsetstrokecolor{currentstroke}%
\pgfsetdash{}{0pt}%
\pgfsys@defobject{currentmarker}{\pgfqpoint{0.000000in}{-0.048611in}}{\pgfqpoint{0.000000in}{0.000000in}}{%
\pgfpathmoveto{\pgfqpoint{0.000000in}{0.000000in}}%
\pgfpathlineto{\pgfqpoint{0.000000in}{-0.048611in}}%
\pgfusepath{stroke,fill}%
}%
\begin{pgfscope}%
\pgfsys@transformshift{3.647762in}{0.540713in}%
\pgfsys@useobject{currentmarker}{}%
\end{pgfscope}%
\end{pgfscope}%
\begin{pgfscope}%
\definecolor{textcolor}{rgb}{0.000000,0.000000,0.000000}%
\pgfsetstrokecolor{textcolor}%
\pgfsetfillcolor{textcolor}%
\pgftext[x=3.647762in,y=0.443491in,,top]{\color{textcolor}\rmfamily\fontsize{8.000000}{9.600000}\selectfont \(\displaystyle {10^{2}}\)}%
\end{pgfscope}%
\begin{pgfscope}%
\pgfsetbuttcap%
\pgfsetroundjoin%
\definecolor{currentfill}{rgb}{0.000000,0.000000,0.000000}%
\pgfsetfillcolor{currentfill}%
\pgfsetlinewidth{0.803000pt}%
\definecolor{currentstroke}{rgb}{0.000000,0.000000,0.000000}%
\pgfsetstrokecolor{currentstroke}%
\pgfsetdash{}{0pt}%
\pgfsys@defobject{currentmarker}{\pgfqpoint{0.000000in}{-0.048611in}}{\pgfqpoint{0.000000in}{0.000000in}}{%
\pgfpathmoveto{\pgfqpoint{0.000000in}{0.000000in}}%
\pgfpathlineto{\pgfqpoint{0.000000in}{-0.048611in}}%
\pgfusepath{stroke,fill}%
}%
\begin{pgfscope}%
\pgfsys@transformshift{4.357910in}{0.540713in}%
\pgfsys@useobject{currentmarker}{}%
\end{pgfscope}%
\end{pgfscope}%
\begin{pgfscope}%
\definecolor{textcolor}{rgb}{0.000000,0.000000,0.000000}%
\pgfsetstrokecolor{textcolor}%
\pgfsetfillcolor{textcolor}%
\pgftext[x=4.357910in,y=0.443491in,,top]{\color{textcolor}\rmfamily\fontsize{8.000000}{9.600000}\selectfont \(\displaystyle {10^{3}}\)}%
\end{pgfscope}%
\begin{pgfscope}%
\pgfsetbuttcap%
\pgfsetroundjoin%
\definecolor{currentfill}{rgb}{0.000000,0.000000,0.000000}%
\pgfsetfillcolor{currentfill}%
\pgfsetlinewidth{0.803000pt}%
\definecolor{currentstroke}{rgb}{0.000000,0.000000,0.000000}%
\pgfsetstrokecolor{currentstroke}%
\pgfsetdash{}{0pt}%
\pgfsys@defobject{currentmarker}{\pgfqpoint{0.000000in}{-0.048611in}}{\pgfqpoint{0.000000in}{0.000000in}}{%
\pgfpathmoveto{\pgfqpoint{0.000000in}{0.000000in}}%
\pgfpathlineto{\pgfqpoint{0.000000in}{-0.048611in}}%
\pgfusepath{stroke,fill}%
}%
\begin{pgfscope}%
\pgfsys@transformshift{5.068058in}{0.540713in}%
\pgfsys@useobject{currentmarker}{}%
\end{pgfscope}%
\end{pgfscope}%
\begin{pgfscope}%
\definecolor{textcolor}{rgb}{0.000000,0.000000,0.000000}%
\pgfsetstrokecolor{textcolor}%
\pgfsetfillcolor{textcolor}%
\pgftext[x=5.068058in,y=0.443491in,,top]{\color{textcolor}\rmfamily\fontsize{8.000000}{9.600000}\selectfont \(\displaystyle {10^{4}}\)}%
\end{pgfscope}%
\begin{pgfscope}%
\pgfsetbuttcap%
\pgfsetroundjoin%
\definecolor{currentfill}{rgb}{0.000000,0.000000,0.000000}%
\pgfsetfillcolor{currentfill}%
\pgfsetlinewidth{0.602250pt}%
\definecolor{currentstroke}{rgb}{0.000000,0.000000,0.000000}%
\pgfsetstrokecolor{currentstroke}%
\pgfsetdash{}{0pt}%
\pgfsys@defobject{currentmarker}{\pgfqpoint{0.000000in}{-0.027778in}}{\pgfqpoint{0.000000in}{0.000000in}}{%
\pgfpathmoveto{\pgfqpoint{0.000000in}{0.000000in}}%
\pgfpathlineto{\pgfqpoint{0.000000in}{-0.027778in}}%
\pgfusepath{stroke,fill}%
}%
\begin{pgfscope}%
\pgfsys@transformshift{0.649623in}{0.540713in}%
\pgfsys@useobject{currentmarker}{}%
\end{pgfscope}%
\end{pgfscope}%
\begin{pgfscope}%
\pgfsetbuttcap%
\pgfsetroundjoin%
\definecolor{currentfill}{rgb}{0.000000,0.000000,0.000000}%
\pgfsetfillcolor{currentfill}%
\pgfsetlinewidth{0.602250pt}%
\definecolor{currentstroke}{rgb}{0.000000,0.000000,0.000000}%
\pgfsetstrokecolor{currentstroke}%
\pgfsetdash{}{0pt}%
\pgfsys@defobject{currentmarker}{\pgfqpoint{0.000000in}{-0.027778in}}{\pgfqpoint{0.000000in}{0.000000in}}{%
\pgfpathmoveto{\pgfqpoint{0.000000in}{0.000000in}}%
\pgfpathlineto{\pgfqpoint{0.000000in}{-0.027778in}}%
\pgfusepath{stroke,fill}%
}%
\begin{pgfscope}%
\pgfsys@transformshift{0.697165in}{0.540713in}%
\pgfsys@useobject{currentmarker}{}%
\end{pgfscope}%
\end{pgfscope}%
\begin{pgfscope}%
\pgfsetbuttcap%
\pgfsetroundjoin%
\definecolor{currentfill}{rgb}{0.000000,0.000000,0.000000}%
\pgfsetfillcolor{currentfill}%
\pgfsetlinewidth{0.602250pt}%
\definecolor{currentstroke}{rgb}{0.000000,0.000000,0.000000}%
\pgfsetstrokecolor{currentstroke}%
\pgfsetdash{}{0pt}%
\pgfsys@defobject{currentmarker}{\pgfqpoint{0.000000in}{-0.027778in}}{\pgfqpoint{0.000000in}{0.000000in}}{%
\pgfpathmoveto{\pgfqpoint{0.000000in}{0.000000in}}%
\pgfpathlineto{\pgfqpoint{0.000000in}{-0.027778in}}%
\pgfusepath{stroke,fill}%
}%
\begin{pgfscope}%
\pgfsys@transformshift{0.738348in}{0.540713in}%
\pgfsys@useobject{currentmarker}{}%
\end{pgfscope}%
\end{pgfscope}%
\begin{pgfscope}%
\pgfsetbuttcap%
\pgfsetroundjoin%
\definecolor{currentfill}{rgb}{0.000000,0.000000,0.000000}%
\pgfsetfillcolor{currentfill}%
\pgfsetlinewidth{0.602250pt}%
\definecolor{currentstroke}{rgb}{0.000000,0.000000,0.000000}%
\pgfsetstrokecolor{currentstroke}%
\pgfsetdash{}{0pt}%
\pgfsys@defobject{currentmarker}{\pgfqpoint{0.000000in}{-0.027778in}}{\pgfqpoint{0.000000in}{0.000000in}}{%
\pgfpathmoveto{\pgfqpoint{0.000000in}{0.000000in}}%
\pgfpathlineto{\pgfqpoint{0.000000in}{-0.027778in}}%
\pgfusepath{stroke,fill}%
}%
\begin{pgfscope}%
\pgfsys@transformshift{0.774674in}{0.540713in}%
\pgfsys@useobject{currentmarker}{}%
\end{pgfscope}%
\end{pgfscope}%
\begin{pgfscope}%
\pgfsetbuttcap%
\pgfsetroundjoin%
\definecolor{currentfill}{rgb}{0.000000,0.000000,0.000000}%
\pgfsetfillcolor{currentfill}%
\pgfsetlinewidth{0.602250pt}%
\definecolor{currentstroke}{rgb}{0.000000,0.000000,0.000000}%
\pgfsetstrokecolor{currentstroke}%
\pgfsetdash{}{0pt}%
\pgfsys@defobject{currentmarker}{\pgfqpoint{0.000000in}{-0.027778in}}{\pgfqpoint{0.000000in}{0.000000in}}{%
\pgfpathmoveto{\pgfqpoint{0.000000in}{0.000000in}}%
\pgfpathlineto{\pgfqpoint{0.000000in}{-0.027778in}}%
\pgfusepath{stroke,fill}%
}%
\begin{pgfscope}%
\pgfsys@transformshift{1.020945in}{0.540713in}%
\pgfsys@useobject{currentmarker}{}%
\end{pgfscope}%
\end{pgfscope}%
\begin{pgfscope}%
\pgfsetbuttcap%
\pgfsetroundjoin%
\definecolor{currentfill}{rgb}{0.000000,0.000000,0.000000}%
\pgfsetfillcolor{currentfill}%
\pgfsetlinewidth{0.602250pt}%
\definecolor{currentstroke}{rgb}{0.000000,0.000000,0.000000}%
\pgfsetstrokecolor{currentstroke}%
\pgfsetdash{}{0pt}%
\pgfsys@defobject{currentmarker}{\pgfqpoint{0.000000in}{-0.027778in}}{\pgfqpoint{0.000000in}{0.000000in}}{%
\pgfpathmoveto{\pgfqpoint{0.000000in}{0.000000in}}%
\pgfpathlineto{\pgfqpoint{0.000000in}{-0.027778in}}%
\pgfusepath{stroke,fill}%
}%
\begin{pgfscope}%
\pgfsys@transformshift{1.145996in}{0.540713in}%
\pgfsys@useobject{currentmarker}{}%
\end{pgfscope}%
\end{pgfscope}%
\begin{pgfscope}%
\pgfsetbuttcap%
\pgfsetroundjoin%
\definecolor{currentfill}{rgb}{0.000000,0.000000,0.000000}%
\pgfsetfillcolor{currentfill}%
\pgfsetlinewidth{0.602250pt}%
\definecolor{currentstroke}{rgb}{0.000000,0.000000,0.000000}%
\pgfsetstrokecolor{currentstroke}%
\pgfsetdash{}{0pt}%
\pgfsys@defobject{currentmarker}{\pgfqpoint{0.000000in}{-0.027778in}}{\pgfqpoint{0.000000in}{0.000000in}}{%
\pgfpathmoveto{\pgfqpoint{0.000000in}{0.000000in}}%
\pgfpathlineto{\pgfqpoint{0.000000in}{-0.027778in}}%
\pgfusepath{stroke,fill}%
}%
\begin{pgfscope}%
\pgfsys@transformshift{1.234721in}{0.540713in}%
\pgfsys@useobject{currentmarker}{}%
\end{pgfscope}%
\end{pgfscope}%
\begin{pgfscope}%
\pgfsetbuttcap%
\pgfsetroundjoin%
\definecolor{currentfill}{rgb}{0.000000,0.000000,0.000000}%
\pgfsetfillcolor{currentfill}%
\pgfsetlinewidth{0.602250pt}%
\definecolor{currentstroke}{rgb}{0.000000,0.000000,0.000000}%
\pgfsetstrokecolor{currentstroke}%
\pgfsetdash{}{0pt}%
\pgfsys@defobject{currentmarker}{\pgfqpoint{0.000000in}{-0.027778in}}{\pgfqpoint{0.000000in}{0.000000in}}{%
\pgfpathmoveto{\pgfqpoint{0.000000in}{0.000000in}}%
\pgfpathlineto{\pgfqpoint{0.000000in}{-0.027778in}}%
\pgfusepath{stroke,fill}%
}%
\begin{pgfscope}%
\pgfsys@transformshift{1.303541in}{0.540713in}%
\pgfsys@useobject{currentmarker}{}%
\end{pgfscope}%
\end{pgfscope}%
\begin{pgfscope}%
\pgfsetbuttcap%
\pgfsetroundjoin%
\definecolor{currentfill}{rgb}{0.000000,0.000000,0.000000}%
\pgfsetfillcolor{currentfill}%
\pgfsetlinewidth{0.602250pt}%
\definecolor{currentstroke}{rgb}{0.000000,0.000000,0.000000}%
\pgfsetstrokecolor{currentstroke}%
\pgfsetdash{}{0pt}%
\pgfsys@defobject{currentmarker}{\pgfqpoint{0.000000in}{-0.027778in}}{\pgfqpoint{0.000000in}{0.000000in}}{%
\pgfpathmoveto{\pgfqpoint{0.000000in}{0.000000in}}%
\pgfpathlineto{\pgfqpoint{0.000000in}{-0.027778in}}%
\pgfusepath{stroke,fill}%
}%
\begin{pgfscope}%
\pgfsys@transformshift{1.359772in}{0.540713in}%
\pgfsys@useobject{currentmarker}{}%
\end{pgfscope}%
\end{pgfscope}%
\begin{pgfscope}%
\pgfsetbuttcap%
\pgfsetroundjoin%
\definecolor{currentfill}{rgb}{0.000000,0.000000,0.000000}%
\pgfsetfillcolor{currentfill}%
\pgfsetlinewidth{0.602250pt}%
\definecolor{currentstroke}{rgb}{0.000000,0.000000,0.000000}%
\pgfsetstrokecolor{currentstroke}%
\pgfsetdash{}{0pt}%
\pgfsys@defobject{currentmarker}{\pgfqpoint{0.000000in}{-0.027778in}}{\pgfqpoint{0.000000in}{0.000000in}}{%
\pgfpathmoveto{\pgfqpoint{0.000000in}{0.000000in}}%
\pgfpathlineto{\pgfqpoint{0.000000in}{-0.027778in}}%
\pgfusepath{stroke,fill}%
}%
\begin{pgfscope}%
\pgfsys@transformshift{1.407314in}{0.540713in}%
\pgfsys@useobject{currentmarker}{}%
\end{pgfscope}%
\end{pgfscope}%
\begin{pgfscope}%
\pgfsetbuttcap%
\pgfsetroundjoin%
\definecolor{currentfill}{rgb}{0.000000,0.000000,0.000000}%
\pgfsetfillcolor{currentfill}%
\pgfsetlinewidth{0.602250pt}%
\definecolor{currentstroke}{rgb}{0.000000,0.000000,0.000000}%
\pgfsetstrokecolor{currentstroke}%
\pgfsetdash{}{0pt}%
\pgfsys@defobject{currentmarker}{\pgfqpoint{0.000000in}{-0.027778in}}{\pgfqpoint{0.000000in}{0.000000in}}{%
\pgfpathmoveto{\pgfqpoint{0.000000in}{0.000000in}}%
\pgfpathlineto{\pgfqpoint{0.000000in}{-0.027778in}}%
\pgfusepath{stroke,fill}%
}%
\begin{pgfscope}%
\pgfsys@transformshift{1.448497in}{0.540713in}%
\pgfsys@useobject{currentmarker}{}%
\end{pgfscope}%
\end{pgfscope}%
\begin{pgfscope}%
\pgfsetbuttcap%
\pgfsetroundjoin%
\definecolor{currentfill}{rgb}{0.000000,0.000000,0.000000}%
\pgfsetfillcolor{currentfill}%
\pgfsetlinewidth{0.602250pt}%
\definecolor{currentstroke}{rgb}{0.000000,0.000000,0.000000}%
\pgfsetstrokecolor{currentstroke}%
\pgfsetdash{}{0pt}%
\pgfsys@defobject{currentmarker}{\pgfqpoint{0.000000in}{-0.027778in}}{\pgfqpoint{0.000000in}{0.000000in}}{%
\pgfpathmoveto{\pgfqpoint{0.000000in}{0.000000in}}%
\pgfpathlineto{\pgfqpoint{0.000000in}{-0.027778in}}%
\pgfusepath{stroke,fill}%
}%
\begin{pgfscope}%
\pgfsys@transformshift{1.484822in}{0.540713in}%
\pgfsys@useobject{currentmarker}{}%
\end{pgfscope}%
\end{pgfscope}%
\begin{pgfscope}%
\pgfsetbuttcap%
\pgfsetroundjoin%
\definecolor{currentfill}{rgb}{0.000000,0.000000,0.000000}%
\pgfsetfillcolor{currentfill}%
\pgfsetlinewidth{0.602250pt}%
\definecolor{currentstroke}{rgb}{0.000000,0.000000,0.000000}%
\pgfsetstrokecolor{currentstroke}%
\pgfsetdash{}{0pt}%
\pgfsys@defobject{currentmarker}{\pgfqpoint{0.000000in}{-0.027778in}}{\pgfqpoint{0.000000in}{0.000000in}}{%
\pgfpathmoveto{\pgfqpoint{0.000000in}{0.000000in}}%
\pgfpathlineto{\pgfqpoint{0.000000in}{-0.027778in}}%
\pgfusepath{stroke,fill}%
}%
\begin{pgfscope}%
\pgfsys@transformshift{1.731093in}{0.540713in}%
\pgfsys@useobject{currentmarker}{}%
\end{pgfscope}%
\end{pgfscope}%
\begin{pgfscope}%
\pgfsetbuttcap%
\pgfsetroundjoin%
\definecolor{currentfill}{rgb}{0.000000,0.000000,0.000000}%
\pgfsetfillcolor{currentfill}%
\pgfsetlinewidth{0.602250pt}%
\definecolor{currentstroke}{rgb}{0.000000,0.000000,0.000000}%
\pgfsetstrokecolor{currentstroke}%
\pgfsetdash{}{0pt}%
\pgfsys@defobject{currentmarker}{\pgfqpoint{0.000000in}{-0.027778in}}{\pgfqpoint{0.000000in}{0.000000in}}{%
\pgfpathmoveto{\pgfqpoint{0.000000in}{0.000000in}}%
\pgfpathlineto{\pgfqpoint{0.000000in}{-0.027778in}}%
\pgfusepath{stroke,fill}%
}%
\begin{pgfscope}%
\pgfsys@transformshift{1.856144in}{0.540713in}%
\pgfsys@useobject{currentmarker}{}%
\end{pgfscope}%
\end{pgfscope}%
\begin{pgfscope}%
\pgfsetbuttcap%
\pgfsetroundjoin%
\definecolor{currentfill}{rgb}{0.000000,0.000000,0.000000}%
\pgfsetfillcolor{currentfill}%
\pgfsetlinewidth{0.602250pt}%
\definecolor{currentstroke}{rgb}{0.000000,0.000000,0.000000}%
\pgfsetstrokecolor{currentstroke}%
\pgfsetdash{}{0pt}%
\pgfsys@defobject{currentmarker}{\pgfqpoint{0.000000in}{-0.027778in}}{\pgfqpoint{0.000000in}{0.000000in}}{%
\pgfpathmoveto{\pgfqpoint{0.000000in}{0.000000in}}%
\pgfpathlineto{\pgfqpoint{0.000000in}{-0.027778in}}%
\pgfusepath{stroke,fill}%
}%
\begin{pgfscope}%
\pgfsys@transformshift{1.944869in}{0.540713in}%
\pgfsys@useobject{currentmarker}{}%
\end{pgfscope}%
\end{pgfscope}%
\begin{pgfscope}%
\pgfsetbuttcap%
\pgfsetroundjoin%
\definecolor{currentfill}{rgb}{0.000000,0.000000,0.000000}%
\pgfsetfillcolor{currentfill}%
\pgfsetlinewidth{0.602250pt}%
\definecolor{currentstroke}{rgb}{0.000000,0.000000,0.000000}%
\pgfsetstrokecolor{currentstroke}%
\pgfsetdash{}{0pt}%
\pgfsys@defobject{currentmarker}{\pgfqpoint{0.000000in}{-0.027778in}}{\pgfqpoint{0.000000in}{0.000000in}}{%
\pgfpathmoveto{\pgfqpoint{0.000000in}{0.000000in}}%
\pgfpathlineto{\pgfqpoint{0.000000in}{-0.027778in}}%
\pgfusepath{stroke,fill}%
}%
\begin{pgfscope}%
\pgfsys@transformshift{2.013689in}{0.540713in}%
\pgfsys@useobject{currentmarker}{}%
\end{pgfscope}%
\end{pgfscope}%
\begin{pgfscope}%
\pgfsetbuttcap%
\pgfsetroundjoin%
\definecolor{currentfill}{rgb}{0.000000,0.000000,0.000000}%
\pgfsetfillcolor{currentfill}%
\pgfsetlinewidth{0.602250pt}%
\definecolor{currentstroke}{rgb}{0.000000,0.000000,0.000000}%
\pgfsetstrokecolor{currentstroke}%
\pgfsetdash{}{0pt}%
\pgfsys@defobject{currentmarker}{\pgfqpoint{0.000000in}{-0.027778in}}{\pgfqpoint{0.000000in}{0.000000in}}{%
\pgfpathmoveto{\pgfqpoint{0.000000in}{0.000000in}}%
\pgfpathlineto{\pgfqpoint{0.000000in}{-0.027778in}}%
\pgfusepath{stroke,fill}%
}%
\begin{pgfscope}%
\pgfsys@transformshift{2.069920in}{0.540713in}%
\pgfsys@useobject{currentmarker}{}%
\end{pgfscope}%
\end{pgfscope}%
\begin{pgfscope}%
\pgfsetbuttcap%
\pgfsetroundjoin%
\definecolor{currentfill}{rgb}{0.000000,0.000000,0.000000}%
\pgfsetfillcolor{currentfill}%
\pgfsetlinewidth{0.602250pt}%
\definecolor{currentstroke}{rgb}{0.000000,0.000000,0.000000}%
\pgfsetstrokecolor{currentstroke}%
\pgfsetdash{}{0pt}%
\pgfsys@defobject{currentmarker}{\pgfqpoint{0.000000in}{-0.027778in}}{\pgfqpoint{0.000000in}{0.000000in}}{%
\pgfpathmoveto{\pgfqpoint{0.000000in}{0.000000in}}%
\pgfpathlineto{\pgfqpoint{0.000000in}{-0.027778in}}%
\pgfusepath{stroke,fill}%
}%
\begin{pgfscope}%
\pgfsys@transformshift{2.117462in}{0.540713in}%
\pgfsys@useobject{currentmarker}{}%
\end{pgfscope}%
\end{pgfscope}%
\begin{pgfscope}%
\pgfsetbuttcap%
\pgfsetroundjoin%
\definecolor{currentfill}{rgb}{0.000000,0.000000,0.000000}%
\pgfsetfillcolor{currentfill}%
\pgfsetlinewidth{0.602250pt}%
\definecolor{currentstroke}{rgb}{0.000000,0.000000,0.000000}%
\pgfsetstrokecolor{currentstroke}%
\pgfsetdash{}{0pt}%
\pgfsys@defobject{currentmarker}{\pgfqpoint{0.000000in}{-0.027778in}}{\pgfqpoint{0.000000in}{0.000000in}}{%
\pgfpathmoveto{\pgfqpoint{0.000000in}{0.000000in}}%
\pgfpathlineto{\pgfqpoint{0.000000in}{-0.027778in}}%
\pgfusepath{stroke,fill}%
}%
\begin{pgfscope}%
\pgfsys@transformshift{2.158645in}{0.540713in}%
\pgfsys@useobject{currentmarker}{}%
\end{pgfscope}%
\end{pgfscope}%
\begin{pgfscope}%
\pgfsetbuttcap%
\pgfsetroundjoin%
\definecolor{currentfill}{rgb}{0.000000,0.000000,0.000000}%
\pgfsetfillcolor{currentfill}%
\pgfsetlinewidth{0.602250pt}%
\definecolor{currentstroke}{rgb}{0.000000,0.000000,0.000000}%
\pgfsetstrokecolor{currentstroke}%
\pgfsetdash{}{0pt}%
\pgfsys@defobject{currentmarker}{\pgfqpoint{0.000000in}{-0.027778in}}{\pgfqpoint{0.000000in}{0.000000in}}{%
\pgfpathmoveto{\pgfqpoint{0.000000in}{0.000000in}}%
\pgfpathlineto{\pgfqpoint{0.000000in}{-0.027778in}}%
\pgfusepath{stroke,fill}%
}%
\begin{pgfscope}%
\pgfsys@transformshift{2.194971in}{0.540713in}%
\pgfsys@useobject{currentmarker}{}%
\end{pgfscope}%
\end{pgfscope}%
\begin{pgfscope}%
\pgfsetbuttcap%
\pgfsetroundjoin%
\definecolor{currentfill}{rgb}{0.000000,0.000000,0.000000}%
\pgfsetfillcolor{currentfill}%
\pgfsetlinewidth{0.602250pt}%
\definecolor{currentstroke}{rgb}{0.000000,0.000000,0.000000}%
\pgfsetstrokecolor{currentstroke}%
\pgfsetdash{}{0pt}%
\pgfsys@defobject{currentmarker}{\pgfqpoint{0.000000in}{-0.027778in}}{\pgfqpoint{0.000000in}{0.000000in}}{%
\pgfpathmoveto{\pgfqpoint{0.000000in}{0.000000in}}%
\pgfpathlineto{\pgfqpoint{0.000000in}{-0.027778in}}%
\pgfusepath{stroke,fill}%
}%
\begin{pgfscope}%
\pgfsys@transformshift{2.441241in}{0.540713in}%
\pgfsys@useobject{currentmarker}{}%
\end{pgfscope}%
\end{pgfscope}%
\begin{pgfscope}%
\pgfsetbuttcap%
\pgfsetroundjoin%
\definecolor{currentfill}{rgb}{0.000000,0.000000,0.000000}%
\pgfsetfillcolor{currentfill}%
\pgfsetlinewidth{0.602250pt}%
\definecolor{currentstroke}{rgb}{0.000000,0.000000,0.000000}%
\pgfsetstrokecolor{currentstroke}%
\pgfsetdash{}{0pt}%
\pgfsys@defobject{currentmarker}{\pgfqpoint{0.000000in}{-0.027778in}}{\pgfqpoint{0.000000in}{0.000000in}}{%
\pgfpathmoveto{\pgfqpoint{0.000000in}{0.000000in}}%
\pgfpathlineto{\pgfqpoint{0.000000in}{-0.027778in}}%
\pgfusepath{stroke,fill}%
}%
\begin{pgfscope}%
\pgfsys@transformshift{2.566292in}{0.540713in}%
\pgfsys@useobject{currentmarker}{}%
\end{pgfscope}%
\end{pgfscope}%
\begin{pgfscope}%
\pgfsetbuttcap%
\pgfsetroundjoin%
\definecolor{currentfill}{rgb}{0.000000,0.000000,0.000000}%
\pgfsetfillcolor{currentfill}%
\pgfsetlinewidth{0.602250pt}%
\definecolor{currentstroke}{rgb}{0.000000,0.000000,0.000000}%
\pgfsetstrokecolor{currentstroke}%
\pgfsetdash{}{0pt}%
\pgfsys@defobject{currentmarker}{\pgfqpoint{0.000000in}{-0.027778in}}{\pgfqpoint{0.000000in}{0.000000in}}{%
\pgfpathmoveto{\pgfqpoint{0.000000in}{0.000000in}}%
\pgfpathlineto{\pgfqpoint{0.000000in}{-0.027778in}}%
\pgfusepath{stroke,fill}%
}%
\begin{pgfscope}%
\pgfsys@transformshift{2.655017in}{0.540713in}%
\pgfsys@useobject{currentmarker}{}%
\end{pgfscope}%
\end{pgfscope}%
\begin{pgfscope}%
\pgfsetbuttcap%
\pgfsetroundjoin%
\definecolor{currentfill}{rgb}{0.000000,0.000000,0.000000}%
\pgfsetfillcolor{currentfill}%
\pgfsetlinewidth{0.602250pt}%
\definecolor{currentstroke}{rgb}{0.000000,0.000000,0.000000}%
\pgfsetstrokecolor{currentstroke}%
\pgfsetdash{}{0pt}%
\pgfsys@defobject{currentmarker}{\pgfqpoint{0.000000in}{-0.027778in}}{\pgfqpoint{0.000000in}{0.000000in}}{%
\pgfpathmoveto{\pgfqpoint{0.000000in}{0.000000in}}%
\pgfpathlineto{\pgfqpoint{0.000000in}{-0.027778in}}%
\pgfusepath{stroke,fill}%
}%
\begin{pgfscope}%
\pgfsys@transformshift{2.723838in}{0.540713in}%
\pgfsys@useobject{currentmarker}{}%
\end{pgfscope}%
\end{pgfscope}%
\begin{pgfscope}%
\pgfsetbuttcap%
\pgfsetroundjoin%
\definecolor{currentfill}{rgb}{0.000000,0.000000,0.000000}%
\pgfsetfillcolor{currentfill}%
\pgfsetlinewidth{0.602250pt}%
\definecolor{currentstroke}{rgb}{0.000000,0.000000,0.000000}%
\pgfsetstrokecolor{currentstroke}%
\pgfsetdash{}{0pt}%
\pgfsys@defobject{currentmarker}{\pgfqpoint{0.000000in}{-0.027778in}}{\pgfqpoint{0.000000in}{0.000000in}}{%
\pgfpathmoveto{\pgfqpoint{0.000000in}{0.000000in}}%
\pgfpathlineto{\pgfqpoint{0.000000in}{-0.027778in}}%
\pgfusepath{stroke,fill}%
}%
\begin{pgfscope}%
\pgfsys@transformshift{2.780068in}{0.540713in}%
\pgfsys@useobject{currentmarker}{}%
\end{pgfscope}%
\end{pgfscope}%
\begin{pgfscope}%
\pgfsetbuttcap%
\pgfsetroundjoin%
\definecolor{currentfill}{rgb}{0.000000,0.000000,0.000000}%
\pgfsetfillcolor{currentfill}%
\pgfsetlinewidth{0.602250pt}%
\definecolor{currentstroke}{rgb}{0.000000,0.000000,0.000000}%
\pgfsetstrokecolor{currentstroke}%
\pgfsetdash{}{0pt}%
\pgfsys@defobject{currentmarker}{\pgfqpoint{0.000000in}{-0.027778in}}{\pgfqpoint{0.000000in}{0.000000in}}{%
\pgfpathmoveto{\pgfqpoint{0.000000in}{0.000000in}}%
\pgfpathlineto{\pgfqpoint{0.000000in}{-0.027778in}}%
\pgfusepath{stroke,fill}%
}%
\begin{pgfscope}%
\pgfsys@transformshift{2.827610in}{0.540713in}%
\pgfsys@useobject{currentmarker}{}%
\end{pgfscope}%
\end{pgfscope}%
\begin{pgfscope}%
\pgfsetbuttcap%
\pgfsetroundjoin%
\definecolor{currentfill}{rgb}{0.000000,0.000000,0.000000}%
\pgfsetfillcolor{currentfill}%
\pgfsetlinewidth{0.602250pt}%
\definecolor{currentstroke}{rgb}{0.000000,0.000000,0.000000}%
\pgfsetstrokecolor{currentstroke}%
\pgfsetdash{}{0pt}%
\pgfsys@defobject{currentmarker}{\pgfqpoint{0.000000in}{-0.027778in}}{\pgfqpoint{0.000000in}{0.000000in}}{%
\pgfpathmoveto{\pgfqpoint{0.000000in}{0.000000in}}%
\pgfpathlineto{\pgfqpoint{0.000000in}{-0.027778in}}%
\pgfusepath{stroke,fill}%
}%
\begin{pgfscope}%
\pgfsys@transformshift{2.868793in}{0.540713in}%
\pgfsys@useobject{currentmarker}{}%
\end{pgfscope}%
\end{pgfscope}%
\begin{pgfscope}%
\pgfsetbuttcap%
\pgfsetroundjoin%
\definecolor{currentfill}{rgb}{0.000000,0.000000,0.000000}%
\pgfsetfillcolor{currentfill}%
\pgfsetlinewidth{0.602250pt}%
\definecolor{currentstroke}{rgb}{0.000000,0.000000,0.000000}%
\pgfsetstrokecolor{currentstroke}%
\pgfsetdash{}{0pt}%
\pgfsys@defobject{currentmarker}{\pgfqpoint{0.000000in}{-0.027778in}}{\pgfqpoint{0.000000in}{0.000000in}}{%
\pgfpathmoveto{\pgfqpoint{0.000000in}{0.000000in}}%
\pgfpathlineto{\pgfqpoint{0.000000in}{-0.027778in}}%
\pgfusepath{stroke,fill}%
}%
\begin{pgfscope}%
\pgfsys@transformshift{2.905119in}{0.540713in}%
\pgfsys@useobject{currentmarker}{}%
\end{pgfscope}%
\end{pgfscope}%
\begin{pgfscope}%
\pgfsetbuttcap%
\pgfsetroundjoin%
\definecolor{currentfill}{rgb}{0.000000,0.000000,0.000000}%
\pgfsetfillcolor{currentfill}%
\pgfsetlinewidth{0.602250pt}%
\definecolor{currentstroke}{rgb}{0.000000,0.000000,0.000000}%
\pgfsetstrokecolor{currentstroke}%
\pgfsetdash{}{0pt}%
\pgfsys@defobject{currentmarker}{\pgfqpoint{0.000000in}{-0.027778in}}{\pgfqpoint{0.000000in}{0.000000in}}{%
\pgfpathmoveto{\pgfqpoint{0.000000in}{0.000000in}}%
\pgfpathlineto{\pgfqpoint{0.000000in}{-0.027778in}}%
\pgfusepath{stroke,fill}%
}%
\begin{pgfscope}%
\pgfsys@transformshift{3.151389in}{0.540713in}%
\pgfsys@useobject{currentmarker}{}%
\end{pgfscope}%
\end{pgfscope}%
\begin{pgfscope}%
\pgfsetbuttcap%
\pgfsetroundjoin%
\definecolor{currentfill}{rgb}{0.000000,0.000000,0.000000}%
\pgfsetfillcolor{currentfill}%
\pgfsetlinewidth{0.602250pt}%
\definecolor{currentstroke}{rgb}{0.000000,0.000000,0.000000}%
\pgfsetstrokecolor{currentstroke}%
\pgfsetdash{}{0pt}%
\pgfsys@defobject{currentmarker}{\pgfqpoint{0.000000in}{-0.027778in}}{\pgfqpoint{0.000000in}{0.000000in}}{%
\pgfpathmoveto{\pgfqpoint{0.000000in}{0.000000in}}%
\pgfpathlineto{\pgfqpoint{0.000000in}{-0.027778in}}%
\pgfusepath{stroke,fill}%
}%
\begin{pgfscope}%
\pgfsys@transformshift{3.276440in}{0.540713in}%
\pgfsys@useobject{currentmarker}{}%
\end{pgfscope}%
\end{pgfscope}%
\begin{pgfscope}%
\pgfsetbuttcap%
\pgfsetroundjoin%
\definecolor{currentfill}{rgb}{0.000000,0.000000,0.000000}%
\pgfsetfillcolor{currentfill}%
\pgfsetlinewidth{0.602250pt}%
\definecolor{currentstroke}{rgb}{0.000000,0.000000,0.000000}%
\pgfsetstrokecolor{currentstroke}%
\pgfsetdash{}{0pt}%
\pgfsys@defobject{currentmarker}{\pgfqpoint{0.000000in}{-0.027778in}}{\pgfqpoint{0.000000in}{0.000000in}}{%
\pgfpathmoveto{\pgfqpoint{0.000000in}{0.000000in}}%
\pgfpathlineto{\pgfqpoint{0.000000in}{-0.027778in}}%
\pgfusepath{stroke,fill}%
}%
\begin{pgfscope}%
\pgfsys@transformshift{3.365165in}{0.540713in}%
\pgfsys@useobject{currentmarker}{}%
\end{pgfscope}%
\end{pgfscope}%
\begin{pgfscope}%
\pgfsetbuttcap%
\pgfsetroundjoin%
\definecolor{currentfill}{rgb}{0.000000,0.000000,0.000000}%
\pgfsetfillcolor{currentfill}%
\pgfsetlinewidth{0.602250pt}%
\definecolor{currentstroke}{rgb}{0.000000,0.000000,0.000000}%
\pgfsetstrokecolor{currentstroke}%
\pgfsetdash{}{0pt}%
\pgfsys@defobject{currentmarker}{\pgfqpoint{0.000000in}{-0.027778in}}{\pgfqpoint{0.000000in}{0.000000in}}{%
\pgfpathmoveto{\pgfqpoint{0.000000in}{0.000000in}}%
\pgfpathlineto{\pgfqpoint{0.000000in}{-0.027778in}}%
\pgfusepath{stroke,fill}%
}%
\begin{pgfscope}%
\pgfsys@transformshift{3.433986in}{0.540713in}%
\pgfsys@useobject{currentmarker}{}%
\end{pgfscope}%
\end{pgfscope}%
\begin{pgfscope}%
\pgfsetbuttcap%
\pgfsetroundjoin%
\definecolor{currentfill}{rgb}{0.000000,0.000000,0.000000}%
\pgfsetfillcolor{currentfill}%
\pgfsetlinewidth{0.602250pt}%
\definecolor{currentstroke}{rgb}{0.000000,0.000000,0.000000}%
\pgfsetstrokecolor{currentstroke}%
\pgfsetdash{}{0pt}%
\pgfsys@defobject{currentmarker}{\pgfqpoint{0.000000in}{-0.027778in}}{\pgfqpoint{0.000000in}{0.000000in}}{%
\pgfpathmoveto{\pgfqpoint{0.000000in}{0.000000in}}%
\pgfpathlineto{\pgfqpoint{0.000000in}{-0.027778in}}%
\pgfusepath{stroke,fill}%
}%
\begin{pgfscope}%
\pgfsys@transformshift{3.490216in}{0.540713in}%
\pgfsys@useobject{currentmarker}{}%
\end{pgfscope}%
\end{pgfscope}%
\begin{pgfscope}%
\pgfsetbuttcap%
\pgfsetroundjoin%
\definecolor{currentfill}{rgb}{0.000000,0.000000,0.000000}%
\pgfsetfillcolor{currentfill}%
\pgfsetlinewidth{0.602250pt}%
\definecolor{currentstroke}{rgb}{0.000000,0.000000,0.000000}%
\pgfsetstrokecolor{currentstroke}%
\pgfsetdash{}{0pt}%
\pgfsys@defobject{currentmarker}{\pgfqpoint{0.000000in}{-0.027778in}}{\pgfqpoint{0.000000in}{0.000000in}}{%
\pgfpathmoveto{\pgfqpoint{0.000000in}{0.000000in}}%
\pgfpathlineto{\pgfqpoint{0.000000in}{-0.027778in}}%
\pgfusepath{stroke,fill}%
}%
\begin{pgfscope}%
\pgfsys@transformshift{3.537758in}{0.540713in}%
\pgfsys@useobject{currentmarker}{}%
\end{pgfscope}%
\end{pgfscope}%
\begin{pgfscope}%
\pgfsetbuttcap%
\pgfsetroundjoin%
\definecolor{currentfill}{rgb}{0.000000,0.000000,0.000000}%
\pgfsetfillcolor{currentfill}%
\pgfsetlinewidth{0.602250pt}%
\definecolor{currentstroke}{rgb}{0.000000,0.000000,0.000000}%
\pgfsetstrokecolor{currentstroke}%
\pgfsetdash{}{0pt}%
\pgfsys@defobject{currentmarker}{\pgfqpoint{0.000000in}{-0.027778in}}{\pgfqpoint{0.000000in}{0.000000in}}{%
\pgfpathmoveto{\pgfqpoint{0.000000in}{0.000000in}}%
\pgfpathlineto{\pgfqpoint{0.000000in}{-0.027778in}}%
\pgfusepath{stroke,fill}%
}%
\begin{pgfscope}%
\pgfsys@transformshift{3.578941in}{0.540713in}%
\pgfsys@useobject{currentmarker}{}%
\end{pgfscope}%
\end{pgfscope}%
\begin{pgfscope}%
\pgfsetbuttcap%
\pgfsetroundjoin%
\definecolor{currentfill}{rgb}{0.000000,0.000000,0.000000}%
\pgfsetfillcolor{currentfill}%
\pgfsetlinewidth{0.602250pt}%
\definecolor{currentstroke}{rgb}{0.000000,0.000000,0.000000}%
\pgfsetstrokecolor{currentstroke}%
\pgfsetdash{}{0pt}%
\pgfsys@defobject{currentmarker}{\pgfqpoint{0.000000in}{-0.027778in}}{\pgfqpoint{0.000000in}{0.000000in}}{%
\pgfpathmoveto{\pgfqpoint{0.000000in}{0.000000in}}%
\pgfpathlineto{\pgfqpoint{0.000000in}{-0.027778in}}%
\pgfusepath{stroke,fill}%
}%
\begin{pgfscope}%
\pgfsys@transformshift{3.615267in}{0.540713in}%
\pgfsys@useobject{currentmarker}{}%
\end{pgfscope}%
\end{pgfscope}%
\begin{pgfscope}%
\pgfsetbuttcap%
\pgfsetroundjoin%
\definecolor{currentfill}{rgb}{0.000000,0.000000,0.000000}%
\pgfsetfillcolor{currentfill}%
\pgfsetlinewidth{0.602250pt}%
\definecolor{currentstroke}{rgb}{0.000000,0.000000,0.000000}%
\pgfsetstrokecolor{currentstroke}%
\pgfsetdash{}{0pt}%
\pgfsys@defobject{currentmarker}{\pgfqpoint{0.000000in}{-0.027778in}}{\pgfqpoint{0.000000in}{0.000000in}}{%
\pgfpathmoveto{\pgfqpoint{0.000000in}{0.000000in}}%
\pgfpathlineto{\pgfqpoint{0.000000in}{-0.027778in}}%
\pgfusepath{stroke,fill}%
}%
\begin{pgfscope}%
\pgfsys@transformshift{3.861538in}{0.540713in}%
\pgfsys@useobject{currentmarker}{}%
\end{pgfscope}%
\end{pgfscope}%
\begin{pgfscope}%
\pgfsetbuttcap%
\pgfsetroundjoin%
\definecolor{currentfill}{rgb}{0.000000,0.000000,0.000000}%
\pgfsetfillcolor{currentfill}%
\pgfsetlinewidth{0.602250pt}%
\definecolor{currentstroke}{rgb}{0.000000,0.000000,0.000000}%
\pgfsetstrokecolor{currentstroke}%
\pgfsetdash{}{0pt}%
\pgfsys@defobject{currentmarker}{\pgfqpoint{0.000000in}{-0.027778in}}{\pgfqpoint{0.000000in}{0.000000in}}{%
\pgfpathmoveto{\pgfqpoint{0.000000in}{0.000000in}}%
\pgfpathlineto{\pgfqpoint{0.000000in}{-0.027778in}}%
\pgfusepath{stroke,fill}%
}%
\begin{pgfscope}%
\pgfsys@transformshift{3.986588in}{0.540713in}%
\pgfsys@useobject{currentmarker}{}%
\end{pgfscope}%
\end{pgfscope}%
\begin{pgfscope}%
\pgfsetbuttcap%
\pgfsetroundjoin%
\definecolor{currentfill}{rgb}{0.000000,0.000000,0.000000}%
\pgfsetfillcolor{currentfill}%
\pgfsetlinewidth{0.602250pt}%
\definecolor{currentstroke}{rgb}{0.000000,0.000000,0.000000}%
\pgfsetstrokecolor{currentstroke}%
\pgfsetdash{}{0pt}%
\pgfsys@defobject{currentmarker}{\pgfqpoint{0.000000in}{-0.027778in}}{\pgfqpoint{0.000000in}{0.000000in}}{%
\pgfpathmoveto{\pgfqpoint{0.000000in}{0.000000in}}%
\pgfpathlineto{\pgfqpoint{0.000000in}{-0.027778in}}%
\pgfusepath{stroke,fill}%
}%
\begin{pgfscope}%
\pgfsys@transformshift{4.075313in}{0.540713in}%
\pgfsys@useobject{currentmarker}{}%
\end{pgfscope}%
\end{pgfscope}%
\begin{pgfscope}%
\pgfsetbuttcap%
\pgfsetroundjoin%
\definecolor{currentfill}{rgb}{0.000000,0.000000,0.000000}%
\pgfsetfillcolor{currentfill}%
\pgfsetlinewidth{0.602250pt}%
\definecolor{currentstroke}{rgb}{0.000000,0.000000,0.000000}%
\pgfsetstrokecolor{currentstroke}%
\pgfsetdash{}{0pt}%
\pgfsys@defobject{currentmarker}{\pgfqpoint{0.000000in}{-0.027778in}}{\pgfqpoint{0.000000in}{0.000000in}}{%
\pgfpathmoveto{\pgfqpoint{0.000000in}{0.000000in}}%
\pgfpathlineto{\pgfqpoint{0.000000in}{-0.027778in}}%
\pgfusepath{stroke,fill}%
}%
\begin{pgfscope}%
\pgfsys@transformshift{4.144134in}{0.540713in}%
\pgfsys@useobject{currentmarker}{}%
\end{pgfscope}%
\end{pgfscope}%
\begin{pgfscope}%
\pgfsetbuttcap%
\pgfsetroundjoin%
\definecolor{currentfill}{rgb}{0.000000,0.000000,0.000000}%
\pgfsetfillcolor{currentfill}%
\pgfsetlinewidth{0.602250pt}%
\definecolor{currentstroke}{rgb}{0.000000,0.000000,0.000000}%
\pgfsetstrokecolor{currentstroke}%
\pgfsetdash{}{0pt}%
\pgfsys@defobject{currentmarker}{\pgfqpoint{0.000000in}{-0.027778in}}{\pgfqpoint{0.000000in}{0.000000in}}{%
\pgfpathmoveto{\pgfqpoint{0.000000in}{0.000000in}}%
\pgfpathlineto{\pgfqpoint{0.000000in}{-0.027778in}}%
\pgfusepath{stroke,fill}%
}%
\begin{pgfscope}%
\pgfsys@transformshift{4.200364in}{0.540713in}%
\pgfsys@useobject{currentmarker}{}%
\end{pgfscope}%
\end{pgfscope}%
\begin{pgfscope}%
\pgfsetbuttcap%
\pgfsetroundjoin%
\definecolor{currentfill}{rgb}{0.000000,0.000000,0.000000}%
\pgfsetfillcolor{currentfill}%
\pgfsetlinewidth{0.602250pt}%
\definecolor{currentstroke}{rgb}{0.000000,0.000000,0.000000}%
\pgfsetstrokecolor{currentstroke}%
\pgfsetdash{}{0pt}%
\pgfsys@defobject{currentmarker}{\pgfqpoint{0.000000in}{-0.027778in}}{\pgfqpoint{0.000000in}{0.000000in}}{%
\pgfpathmoveto{\pgfqpoint{0.000000in}{0.000000in}}%
\pgfpathlineto{\pgfqpoint{0.000000in}{-0.027778in}}%
\pgfusepath{stroke,fill}%
}%
\begin{pgfscope}%
\pgfsys@transformshift{4.247907in}{0.540713in}%
\pgfsys@useobject{currentmarker}{}%
\end{pgfscope}%
\end{pgfscope}%
\begin{pgfscope}%
\pgfsetbuttcap%
\pgfsetroundjoin%
\definecolor{currentfill}{rgb}{0.000000,0.000000,0.000000}%
\pgfsetfillcolor{currentfill}%
\pgfsetlinewidth{0.602250pt}%
\definecolor{currentstroke}{rgb}{0.000000,0.000000,0.000000}%
\pgfsetstrokecolor{currentstroke}%
\pgfsetdash{}{0pt}%
\pgfsys@defobject{currentmarker}{\pgfqpoint{0.000000in}{-0.027778in}}{\pgfqpoint{0.000000in}{0.000000in}}{%
\pgfpathmoveto{\pgfqpoint{0.000000in}{0.000000in}}%
\pgfpathlineto{\pgfqpoint{0.000000in}{-0.027778in}}%
\pgfusepath{stroke,fill}%
}%
\begin{pgfscope}%
\pgfsys@transformshift{4.289089in}{0.540713in}%
\pgfsys@useobject{currentmarker}{}%
\end{pgfscope}%
\end{pgfscope}%
\begin{pgfscope}%
\pgfsetbuttcap%
\pgfsetroundjoin%
\definecolor{currentfill}{rgb}{0.000000,0.000000,0.000000}%
\pgfsetfillcolor{currentfill}%
\pgfsetlinewidth{0.602250pt}%
\definecolor{currentstroke}{rgb}{0.000000,0.000000,0.000000}%
\pgfsetstrokecolor{currentstroke}%
\pgfsetdash{}{0pt}%
\pgfsys@defobject{currentmarker}{\pgfqpoint{0.000000in}{-0.027778in}}{\pgfqpoint{0.000000in}{0.000000in}}{%
\pgfpathmoveto{\pgfqpoint{0.000000in}{0.000000in}}%
\pgfpathlineto{\pgfqpoint{0.000000in}{-0.027778in}}%
\pgfusepath{stroke,fill}%
}%
\begin{pgfscope}%
\pgfsys@transformshift{4.325415in}{0.540713in}%
\pgfsys@useobject{currentmarker}{}%
\end{pgfscope}%
\end{pgfscope}%
\begin{pgfscope}%
\pgfsetbuttcap%
\pgfsetroundjoin%
\definecolor{currentfill}{rgb}{0.000000,0.000000,0.000000}%
\pgfsetfillcolor{currentfill}%
\pgfsetlinewidth{0.602250pt}%
\definecolor{currentstroke}{rgb}{0.000000,0.000000,0.000000}%
\pgfsetstrokecolor{currentstroke}%
\pgfsetdash{}{0pt}%
\pgfsys@defobject{currentmarker}{\pgfqpoint{0.000000in}{-0.027778in}}{\pgfqpoint{0.000000in}{0.000000in}}{%
\pgfpathmoveto{\pgfqpoint{0.000000in}{0.000000in}}%
\pgfpathlineto{\pgfqpoint{0.000000in}{-0.027778in}}%
\pgfusepath{stroke,fill}%
}%
\begin{pgfscope}%
\pgfsys@transformshift{4.571686in}{0.540713in}%
\pgfsys@useobject{currentmarker}{}%
\end{pgfscope}%
\end{pgfscope}%
\begin{pgfscope}%
\pgfsetbuttcap%
\pgfsetroundjoin%
\definecolor{currentfill}{rgb}{0.000000,0.000000,0.000000}%
\pgfsetfillcolor{currentfill}%
\pgfsetlinewidth{0.602250pt}%
\definecolor{currentstroke}{rgb}{0.000000,0.000000,0.000000}%
\pgfsetstrokecolor{currentstroke}%
\pgfsetdash{}{0pt}%
\pgfsys@defobject{currentmarker}{\pgfqpoint{0.000000in}{-0.027778in}}{\pgfqpoint{0.000000in}{0.000000in}}{%
\pgfpathmoveto{\pgfqpoint{0.000000in}{0.000000in}}%
\pgfpathlineto{\pgfqpoint{0.000000in}{-0.027778in}}%
\pgfusepath{stroke,fill}%
}%
\begin{pgfscope}%
\pgfsys@transformshift{4.696737in}{0.540713in}%
\pgfsys@useobject{currentmarker}{}%
\end{pgfscope}%
\end{pgfscope}%
\begin{pgfscope}%
\pgfsetbuttcap%
\pgfsetroundjoin%
\definecolor{currentfill}{rgb}{0.000000,0.000000,0.000000}%
\pgfsetfillcolor{currentfill}%
\pgfsetlinewidth{0.602250pt}%
\definecolor{currentstroke}{rgb}{0.000000,0.000000,0.000000}%
\pgfsetstrokecolor{currentstroke}%
\pgfsetdash{}{0pt}%
\pgfsys@defobject{currentmarker}{\pgfqpoint{0.000000in}{-0.027778in}}{\pgfqpoint{0.000000in}{0.000000in}}{%
\pgfpathmoveto{\pgfqpoint{0.000000in}{0.000000in}}%
\pgfpathlineto{\pgfqpoint{0.000000in}{-0.027778in}}%
\pgfusepath{stroke,fill}%
}%
\begin{pgfscope}%
\pgfsys@transformshift{4.785462in}{0.540713in}%
\pgfsys@useobject{currentmarker}{}%
\end{pgfscope}%
\end{pgfscope}%
\begin{pgfscope}%
\pgfsetbuttcap%
\pgfsetroundjoin%
\definecolor{currentfill}{rgb}{0.000000,0.000000,0.000000}%
\pgfsetfillcolor{currentfill}%
\pgfsetlinewidth{0.602250pt}%
\definecolor{currentstroke}{rgb}{0.000000,0.000000,0.000000}%
\pgfsetstrokecolor{currentstroke}%
\pgfsetdash{}{0pt}%
\pgfsys@defobject{currentmarker}{\pgfqpoint{0.000000in}{-0.027778in}}{\pgfqpoint{0.000000in}{0.000000in}}{%
\pgfpathmoveto{\pgfqpoint{0.000000in}{0.000000in}}%
\pgfpathlineto{\pgfqpoint{0.000000in}{-0.027778in}}%
\pgfusepath{stroke,fill}%
}%
\begin{pgfscope}%
\pgfsys@transformshift{4.854282in}{0.540713in}%
\pgfsys@useobject{currentmarker}{}%
\end{pgfscope}%
\end{pgfscope}%
\begin{pgfscope}%
\pgfsetbuttcap%
\pgfsetroundjoin%
\definecolor{currentfill}{rgb}{0.000000,0.000000,0.000000}%
\pgfsetfillcolor{currentfill}%
\pgfsetlinewidth{0.602250pt}%
\definecolor{currentstroke}{rgb}{0.000000,0.000000,0.000000}%
\pgfsetstrokecolor{currentstroke}%
\pgfsetdash{}{0pt}%
\pgfsys@defobject{currentmarker}{\pgfqpoint{0.000000in}{-0.027778in}}{\pgfqpoint{0.000000in}{0.000000in}}{%
\pgfpathmoveto{\pgfqpoint{0.000000in}{0.000000in}}%
\pgfpathlineto{\pgfqpoint{0.000000in}{-0.027778in}}%
\pgfusepath{stroke,fill}%
}%
\begin{pgfscope}%
\pgfsys@transformshift{4.910513in}{0.540713in}%
\pgfsys@useobject{currentmarker}{}%
\end{pgfscope}%
\end{pgfscope}%
\begin{pgfscope}%
\pgfsetbuttcap%
\pgfsetroundjoin%
\definecolor{currentfill}{rgb}{0.000000,0.000000,0.000000}%
\pgfsetfillcolor{currentfill}%
\pgfsetlinewidth{0.602250pt}%
\definecolor{currentstroke}{rgb}{0.000000,0.000000,0.000000}%
\pgfsetstrokecolor{currentstroke}%
\pgfsetdash{}{0pt}%
\pgfsys@defobject{currentmarker}{\pgfqpoint{0.000000in}{-0.027778in}}{\pgfqpoint{0.000000in}{0.000000in}}{%
\pgfpathmoveto{\pgfqpoint{0.000000in}{0.000000in}}%
\pgfpathlineto{\pgfqpoint{0.000000in}{-0.027778in}}%
\pgfusepath{stroke,fill}%
}%
\begin{pgfscope}%
\pgfsys@transformshift{4.958055in}{0.540713in}%
\pgfsys@useobject{currentmarker}{}%
\end{pgfscope}%
\end{pgfscope}%
\begin{pgfscope}%
\pgfsetbuttcap%
\pgfsetroundjoin%
\definecolor{currentfill}{rgb}{0.000000,0.000000,0.000000}%
\pgfsetfillcolor{currentfill}%
\pgfsetlinewidth{0.602250pt}%
\definecolor{currentstroke}{rgb}{0.000000,0.000000,0.000000}%
\pgfsetstrokecolor{currentstroke}%
\pgfsetdash{}{0pt}%
\pgfsys@defobject{currentmarker}{\pgfqpoint{0.000000in}{-0.027778in}}{\pgfqpoint{0.000000in}{0.000000in}}{%
\pgfpathmoveto{\pgfqpoint{0.000000in}{0.000000in}}%
\pgfpathlineto{\pgfqpoint{0.000000in}{-0.027778in}}%
\pgfusepath{stroke,fill}%
}%
\begin{pgfscope}%
\pgfsys@transformshift{4.999238in}{0.540713in}%
\pgfsys@useobject{currentmarker}{}%
\end{pgfscope}%
\end{pgfscope}%
\begin{pgfscope}%
\pgfsetbuttcap%
\pgfsetroundjoin%
\definecolor{currentfill}{rgb}{0.000000,0.000000,0.000000}%
\pgfsetfillcolor{currentfill}%
\pgfsetlinewidth{0.602250pt}%
\definecolor{currentstroke}{rgb}{0.000000,0.000000,0.000000}%
\pgfsetstrokecolor{currentstroke}%
\pgfsetdash{}{0pt}%
\pgfsys@defobject{currentmarker}{\pgfqpoint{0.000000in}{-0.027778in}}{\pgfqpoint{0.000000in}{0.000000in}}{%
\pgfpathmoveto{\pgfqpoint{0.000000in}{0.000000in}}%
\pgfpathlineto{\pgfqpoint{0.000000in}{-0.027778in}}%
\pgfusepath{stroke,fill}%
}%
\begin{pgfscope}%
\pgfsys@transformshift{5.035563in}{0.540713in}%
\pgfsys@useobject{currentmarker}{}%
\end{pgfscope}%
\end{pgfscope}%
\begin{pgfscope}%
\definecolor{textcolor}{rgb}{0.000000,0.000000,0.000000}%
\pgfsetstrokecolor{textcolor}%
\pgfsetfillcolor{textcolor}%
\pgftext[x=2.937613in,y=0.288074in,,top]{\color{textcolor}\rmfamily\fontsize{10.000000}{12.000000}\selectfont Frequency in \unit{\radian \per \s}}%
\end{pgfscope}%
\begin{pgfscope}%
\pgfsetbuttcap%
\pgfsetroundjoin%
\definecolor{currentfill}{rgb}{0.000000,0.000000,0.000000}%
\pgfsetfillcolor{currentfill}%
\pgfsetlinewidth{0.803000pt}%
\definecolor{currentstroke}{rgb}{0.000000,0.000000,0.000000}%
\pgfsetstrokecolor{currentstroke}%
\pgfsetdash{}{0pt}%
\pgfsys@defobject{currentmarker}{\pgfqpoint{-0.048611in}{0.000000in}}{\pgfqpoint{-0.000000in}{0.000000in}}{%
\pgfpathmoveto{\pgfqpoint{-0.000000in}{0.000000in}}%
\pgfpathlineto{\pgfqpoint{-0.048611in}{0.000000in}}%
\pgfusepath{stroke,fill}%
}%
\begin{pgfscope}%
\pgfsys@transformshift{0.594124in}{0.661875in}%
\pgfsys@useobject{currentmarker}{}%
\end{pgfscope}%
\end{pgfscope}%
\begin{pgfscope}%
\definecolor{textcolor}{rgb}{0.000000,0.000000,0.000000}%
\pgfsetstrokecolor{textcolor}%
\pgfsetfillcolor{textcolor}%
\pgftext[x=0.320975in, y=0.622723in, left, base]{\color{textcolor}\rmfamily\fontsize{8.000000}{9.600000}\selectfont \(\displaystyle {10^{0}}\)}%
\end{pgfscope}%
\begin{pgfscope}%
\pgfsetbuttcap%
\pgfsetroundjoin%
\definecolor{currentfill}{rgb}{0.000000,0.000000,0.000000}%
\pgfsetfillcolor{currentfill}%
\pgfsetlinewidth{0.803000pt}%
\definecolor{currentstroke}{rgb}{0.000000,0.000000,0.000000}%
\pgfsetstrokecolor{currentstroke}%
\pgfsetdash{}{0pt}%
\pgfsys@defobject{currentmarker}{\pgfqpoint{-0.048611in}{0.000000in}}{\pgfqpoint{-0.000000in}{0.000000in}}{%
\pgfpathmoveto{\pgfqpoint{-0.000000in}{0.000000in}}%
\pgfpathlineto{\pgfqpoint{-0.048611in}{0.000000in}}%
\pgfusepath{stroke,fill}%
}%
\begin{pgfscope}%
\pgfsys@transformshift{0.594124in}{1.873639in}%
\pgfsys@useobject{currentmarker}{}%
\end{pgfscope}%
\end{pgfscope}%
\begin{pgfscope}%
\definecolor{textcolor}{rgb}{0.000000,0.000000,0.000000}%
\pgfsetstrokecolor{textcolor}%
\pgfsetfillcolor{textcolor}%
\pgftext[x=0.320975in, y=1.834486in, left, base]{\color{textcolor}\rmfamily\fontsize{8.000000}{9.600000}\selectfont \(\displaystyle {10^{1}}\)}%
\end{pgfscope}%
\begin{pgfscope}%
\pgfsetbuttcap%
\pgfsetroundjoin%
\definecolor{currentfill}{rgb}{0.000000,0.000000,0.000000}%
\pgfsetfillcolor{currentfill}%
\pgfsetlinewidth{0.803000pt}%
\definecolor{currentstroke}{rgb}{0.000000,0.000000,0.000000}%
\pgfsetstrokecolor{currentstroke}%
\pgfsetdash{}{0pt}%
\pgfsys@defobject{currentmarker}{\pgfqpoint{-0.048611in}{0.000000in}}{\pgfqpoint{-0.000000in}{0.000000in}}{%
\pgfpathmoveto{\pgfqpoint{-0.000000in}{0.000000in}}%
\pgfpathlineto{\pgfqpoint{-0.048611in}{0.000000in}}%
\pgfusepath{stroke,fill}%
}%
\begin{pgfscope}%
\pgfsys@transformshift{0.594124in}{3.085403in}%
\pgfsys@useobject{currentmarker}{}%
\end{pgfscope}%
\end{pgfscope}%
\begin{pgfscope}%
\definecolor{textcolor}{rgb}{0.000000,0.000000,0.000000}%
\pgfsetstrokecolor{textcolor}%
\pgfsetfillcolor{textcolor}%
\pgftext[x=0.320975in, y=3.046250in, left, base]{\color{textcolor}\rmfamily\fontsize{8.000000}{9.600000}\selectfont \(\displaystyle {10^{2}}\)}%
\end{pgfscope}%
\begin{pgfscope}%
\pgfsetbuttcap%
\pgfsetroundjoin%
\definecolor{currentfill}{rgb}{0.000000,0.000000,0.000000}%
\pgfsetfillcolor{currentfill}%
\pgfsetlinewidth{0.602250pt}%
\definecolor{currentstroke}{rgb}{0.000000,0.000000,0.000000}%
\pgfsetstrokecolor{currentstroke}%
\pgfsetdash{}{0pt}%
\pgfsys@defobject{currentmarker}{\pgfqpoint{-0.027778in}{0.000000in}}{\pgfqpoint{-0.000000in}{0.000000in}}{%
\pgfpathmoveto{\pgfqpoint{-0.000000in}{0.000000in}}%
\pgfpathlineto{\pgfqpoint{-0.027778in}{0.000000in}}%
\pgfusepath{stroke,fill}%
}%
\begin{pgfscope}%
\pgfsys@transformshift{0.594124in}{0.544443in}%
\pgfsys@useobject{currentmarker}{}%
\end{pgfscope}%
\end{pgfscope}%
\begin{pgfscope}%
\pgfsetbuttcap%
\pgfsetroundjoin%
\definecolor{currentfill}{rgb}{0.000000,0.000000,0.000000}%
\pgfsetfillcolor{currentfill}%
\pgfsetlinewidth{0.602250pt}%
\definecolor{currentstroke}{rgb}{0.000000,0.000000,0.000000}%
\pgfsetstrokecolor{currentstroke}%
\pgfsetdash{}{0pt}%
\pgfsys@defobject{currentmarker}{\pgfqpoint{-0.027778in}{0.000000in}}{\pgfqpoint{-0.000000in}{0.000000in}}{%
\pgfpathmoveto{\pgfqpoint{-0.000000in}{0.000000in}}%
\pgfpathlineto{\pgfqpoint{-0.027778in}{0.000000in}}%
\pgfusepath{stroke,fill}%
}%
\begin{pgfscope}%
\pgfsys@transformshift{0.594124in}{0.606428in}%
\pgfsys@useobject{currentmarker}{}%
\end{pgfscope}%
\end{pgfscope}%
\begin{pgfscope}%
\pgfsetbuttcap%
\pgfsetroundjoin%
\definecolor{currentfill}{rgb}{0.000000,0.000000,0.000000}%
\pgfsetfillcolor{currentfill}%
\pgfsetlinewidth{0.602250pt}%
\definecolor{currentstroke}{rgb}{0.000000,0.000000,0.000000}%
\pgfsetstrokecolor{currentstroke}%
\pgfsetdash{}{0pt}%
\pgfsys@defobject{currentmarker}{\pgfqpoint{-0.027778in}{0.000000in}}{\pgfqpoint{-0.000000in}{0.000000in}}{%
\pgfpathmoveto{\pgfqpoint{-0.000000in}{0.000000in}}%
\pgfpathlineto{\pgfqpoint{-0.027778in}{0.000000in}}%
\pgfusepath{stroke,fill}%
}%
\begin{pgfscope}%
\pgfsys@transformshift{0.594124in}{1.026653in}%
\pgfsys@useobject{currentmarker}{}%
\end{pgfscope}%
\end{pgfscope}%
\begin{pgfscope}%
\pgfsetbuttcap%
\pgfsetroundjoin%
\definecolor{currentfill}{rgb}{0.000000,0.000000,0.000000}%
\pgfsetfillcolor{currentfill}%
\pgfsetlinewidth{0.602250pt}%
\definecolor{currentstroke}{rgb}{0.000000,0.000000,0.000000}%
\pgfsetstrokecolor{currentstroke}%
\pgfsetdash{}{0pt}%
\pgfsys@defobject{currentmarker}{\pgfqpoint{-0.027778in}{0.000000in}}{\pgfqpoint{-0.000000in}{0.000000in}}{%
\pgfpathmoveto{\pgfqpoint{-0.000000in}{0.000000in}}%
\pgfpathlineto{\pgfqpoint{-0.027778in}{0.000000in}}%
\pgfusepath{stroke,fill}%
}%
\begin{pgfscope}%
\pgfsys@transformshift{0.594124in}{1.240034in}%
\pgfsys@useobject{currentmarker}{}%
\end{pgfscope}%
\end{pgfscope}%
\begin{pgfscope}%
\pgfsetbuttcap%
\pgfsetroundjoin%
\definecolor{currentfill}{rgb}{0.000000,0.000000,0.000000}%
\pgfsetfillcolor{currentfill}%
\pgfsetlinewidth{0.602250pt}%
\definecolor{currentstroke}{rgb}{0.000000,0.000000,0.000000}%
\pgfsetstrokecolor{currentstroke}%
\pgfsetdash{}{0pt}%
\pgfsys@defobject{currentmarker}{\pgfqpoint{-0.027778in}{0.000000in}}{\pgfqpoint{-0.000000in}{0.000000in}}{%
\pgfpathmoveto{\pgfqpoint{-0.000000in}{0.000000in}}%
\pgfpathlineto{\pgfqpoint{-0.027778in}{0.000000in}}%
\pgfusepath{stroke,fill}%
}%
\begin{pgfscope}%
\pgfsys@transformshift{0.594124in}{1.391430in}%
\pgfsys@useobject{currentmarker}{}%
\end{pgfscope}%
\end{pgfscope}%
\begin{pgfscope}%
\pgfsetbuttcap%
\pgfsetroundjoin%
\definecolor{currentfill}{rgb}{0.000000,0.000000,0.000000}%
\pgfsetfillcolor{currentfill}%
\pgfsetlinewidth{0.602250pt}%
\definecolor{currentstroke}{rgb}{0.000000,0.000000,0.000000}%
\pgfsetstrokecolor{currentstroke}%
\pgfsetdash{}{0pt}%
\pgfsys@defobject{currentmarker}{\pgfqpoint{-0.027778in}{0.000000in}}{\pgfqpoint{-0.000000in}{0.000000in}}{%
\pgfpathmoveto{\pgfqpoint{-0.000000in}{0.000000in}}%
\pgfpathlineto{\pgfqpoint{-0.027778in}{0.000000in}}%
\pgfusepath{stroke,fill}%
}%
\begin{pgfscope}%
\pgfsys@transformshift{0.594124in}{1.508862in}%
\pgfsys@useobject{currentmarker}{}%
\end{pgfscope}%
\end{pgfscope}%
\begin{pgfscope}%
\pgfsetbuttcap%
\pgfsetroundjoin%
\definecolor{currentfill}{rgb}{0.000000,0.000000,0.000000}%
\pgfsetfillcolor{currentfill}%
\pgfsetlinewidth{0.602250pt}%
\definecolor{currentstroke}{rgb}{0.000000,0.000000,0.000000}%
\pgfsetstrokecolor{currentstroke}%
\pgfsetdash{}{0pt}%
\pgfsys@defobject{currentmarker}{\pgfqpoint{-0.027778in}{0.000000in}}{\pgfqpoint{-0.000000in}{0.000000in}}{%
\pgfpathmoveto{\pgfqpoint{-0.000000in}{0.000000in}}%
\pgfpathlineto{\pgfqpoint{-0.027778in}{0.000000in}}%
\pgfusepath{stroke,fill}%
}%
\begin{pgfscope}%
\pgfsys@transformshift{0.594124in}{1.604811in}%
\pgfsys@useobject{currentmarker}{}%
\end{pgfscope}%
\end{pgfscope}%
\begin{pgfscope}%
\pgfsetbuttcap%
\pgfsetroundjoin%
\definecolor{currentfill}{rgb}{0.000000,0.000000,0.000000}%
\pgfsetfillcolor{currentfill}%
\pgfsetlinewidth{0.602250pt}%
\definecolor{currentstroke}{rgb}{0.000000,0.000000,0.000000}%
\pgfsetstrokecolor{currentstroke}%
\pgfsetdash{}{0pt}%
\pgfsys@defobject{currentmarker}{\pgfqpoint{-0.027778in}{0.000000in}}{\pgfqpoint{-0.000000in}{0.000000in}}{%
\pgfpathmoveto{\pgfqpoint{-0.000000in}{0.000000in}}%
\pgfpathlineto{\pgfqpoint{-0.027778in}{0.000000in}}%
\pgfusepath{stroke,fill}%
}%
\begin{pgfscope}%
\pgfsys@transformshift{0.594124in}{1.685935in}%
\pgfsys@useobject{currentmarker}{}%
\end{pgfscope}%
\end{pgfscope}%
\begin{pgfscope}%
\pgfsetbuttcap%
\pgfsetroundjoin%
\definecolor{currentfill}{rgb}{0.000000,0.000000,0.000000}%
\pgfsetfillcolor{currentfill}%
\pgfsetlinewidth{0.602250pt}%
\definecolor{currentstroke}{rgb}{0.000000,0.000000,0.000000}%
\pgfsetstrokecolor{currentstroke}%
\pgfsetdash{}{0pt}%
\pgfsys@defobject{currentmarker}{\pgfqpoint{-0.027778in}{0.000000in}}{\pgfqpoint{-0.000000in}{0.000000in}}{%
\pgfpathmoveto{\pgfqpoint{-0.000000in}{0.000000in}}%
\pgfpathlineto{\pgfqpoint{-0.027778in}{0.000000in}}%
\pgfusepath{stroke,fill}%
}%
\begin{pgfscope}%
\pgfsys@transformshift{0.594124in}{1.756207in}%
\pgfsys@useobject{currentmarker}{}%
\end{pgfscope}%
\end{pgfscope}%
\begin{pgfscope}%
\pgfsetbuttcap%
\pgfsetroundjoin%
\definecolor{currentfill}{rgb}{0.000000,0.000000,0.000000}%
\pgfsetfillcolor{currentfill}%
\pgfsetlinewidth{0.602250pt}%
\definecolor{currentstroke}{rgb}{0.000000,0.000000,0.000000}%
\pgfsetstrokecolor{currentstroke}%
\pgfsetdash{}{0pt}%
\pgfsys@defobject{currentmarker}{\pgfqpoint{-0.027778in}{0.000000in}}{\pgfqpoint{-0.000000in}{0.000000in}}{%
\pgfpathmoveto{\pgfqpoint{-0.000000in}{0.000000in}}%
\pgfpathlineto{\pgfqpoint{-0.027778in}{0.000000in}}%
\pgfusepath{stroke,fill}%
}%
\begin{pgfscope}%
\pgfsys@transformshift{0.594124in}{1.818192in}%
\pgfsys@useobject{currentmarker}{}%
\end{pgfscope}%
\end{pgfscope}%
\begin{pgfscope}%
\pgfsetbuttcap%
\pgfsetroundjoin%
\definecolor{currentfill}{rgb}{0.000000,0.000000,0.000000}%
\pgfsetfillcolor{currentfill}%
\pgfsetlinewidth{0.602250pt}%
\definecolor{currentstroke}{rgb}{0.000000,0.000000,0.000000}%
\pgfsetstrokecolor{currentstroke}%
\pgfsetdash{}{0pt}%
\pgfsys@defobject{currentmarker}{\pgfqpoint{-0.027778in}{0.000000in}}{\pgfqpoint{-0.000000in}{0.000000in}}{%
\pgfpathmoveto{\pgfqpoint{-0.000000in}{0.000000in}}%
\pgfpathlineto{\pgfqpoint{-0.027778in}{0.000000in}}%
\pgfusepath{stroke,fill}%
}%
\begin{pgfscope}%
\pgfsys@transformshift{0.594124in}{2.238416in}%
\pgfsys@useobject{currentmarker}{}%
\end{pgfscope}%
\end{pgfscope}%
\begin{pgfscope}%
\pgfsetbuttcap%
\pgfsetroundjoin%
\definecolor{currentfill}{rgb}{0.000000,0.000000,0.000000}%
\pgfsetfillcolor{currentfill}%
\pgfsetlinewidth{0.602250pt}%
\definecolor{currentstroke}{rgb}{0.000000,0.000000,0.000000}%
\pgfsetstrokecolor{currentstroke}%
\pgfsetdash{}{0pt}%
\pgfsys@defobject{currentmarker}{\pgfqpoint{-0.027778in}{0.000000in}}{\pgfqpoint{-0.000000in}{0.000000in}}{%
\pgfpathmoveto{\pgfqpoint{-0.000000in}{0.000000in}}%
\pgfpathlineto{\pgfqpoint{-0.027778in}{0.000000in}}%
\pgfusepath{stroke,fill}%
}%
\begin{pgfscope}%
\pgfsys@transformshift{0.594124in}{2.451797in}%
\pgfsys@useobject{currentmarker}{}%
\end{pgfscope}%
\end{pgfscope}%
\begin{pgfscope}%
\pgfsetbuttcap%
\pgfsetroundjoin%
\definecolor{currentfill}{rgb}{0.000000,0.000000,0.000000}%
\pgfsetfillcolor{currentfill}%
\pgfsetlinewidth{0.602250pt}%
\definecolor{currentstroke}{rgb}{0.000000,0.000000,0.000000}%
\pgfsetstrokecolor{currentstroke}%
\pgfsetdash{}{0pt}%
\pgfsys@defobject{currentmarker}{\pgfqpoint{-0.027778in}{0.000000in}}{\pgfqpoint{-0.000000in}{0.000000in}}{%
\pgfpathmoveto{\pgfqpoint{-0.000000in}{0.000000in}}%
\pgfpathlineto{\pgfqpoint{-0.027778in}{0.000000in}}%
\pgfusepath{stroke,fill}%
}%
\begin{pgfscope}%
\pgfsys@transformshift{0.594124in}{2.603194in}%
\pgfsys@useobject{currentmarker}{}%
\end{pgfscope}%
\end{pgfscope}%
\begin{pgfscope}%
\pgfsetbuttcap%
\pgfsetroundjoin%
\definecolor{currentfill}{rgb}{0.000000,0.000000,0.000000}%
\pgfsetfillcolor{currentfill}%
\pgfsetlinewidth{0.602250pt}%
\definecolor{currentstroke}{rgb}{0.000000,0.000000,0.000000}%
\pgfsetstrokecolor{currentstroke}%
\pgfsetdash{}{0pt}%
\pgfsys@defobject{currentmarker}{\pgfqpoint{-0.027778in}{0.000000in}}{\pgfqpoint{-0.000000in}{0.000000in}}{%
\pgfpathmoveto{\pgfqpoint{-0.000000in}{0.000000in}}%
\pgfpathlineto{\pgfqpoint{-0.027778in}{0.000000in}}%
\pgfusepath{stroke,fill}%
}%
\begin{pgfscope}%
\pgfsys@transformshift{0.594124in}{2.720626in}%
\pgfsys@useobject{currentmarker}{}%
\end{pgfscope}%
\end{pgfscope}%
\begin{pgfscope}%
\pgfsetbuttcap%
\pgfsetroundjoin%
\definecolor{currentfill}{rgb}{0.000000,0.000000,0.000000}%
\pgfsetfillcolor{currentfill}%
\pgfsetlinewidth{0.602250pt}%
\definecolor{currentstroke}{rgb}{0.000000,0.000000,0.000000}%
\pgfsetstrokecolor{currentstroke}%
\pgfsetdash{}{0pt}%
\pgfsys@defobject{currentmarker}{\pgfqpoint{-0.027778in}{0.000000in}}{\pgfqpoint{-0.000000in}{0.000000in}}{%
\pgfpathmoveto{\pgfqpoint{-0.000000in}{0.000000in}}%
\pgfpathlineto{\pgfqpoint{-0.027778in}{0.000000in}}%
\pgfusepath{stroke,fill}%
}%
\begin{pgfscope}%
\pgfsys@transformshift{0.594124in}{2.816575in}%
\pgfsys@useobject{currentmarker}{}%
\end{pgfscope}%
\end{pgfscope}%
\begin{pgfscope}%
\pgfsetbuttcap%
\pgfsetroundjoin%
\definecolor{currentfill}{rgb}{0.000000,0.000000,0.000000}%
\pgfsetfillcolor{currentfill}%
\pgfsetlinewidth{0.602250pt}%
\definecolor{currentstroke}{rgb}{0.000000,0.000000,0.000000}%
\pgfsetstrokecolor{currentstroke}%
\pgfsetdash{}{0pt}%
\pgfsys@defobject{currentmarker}{\pgfqpoint{-0.027778in}{0.000000in}}{\pgfqpoint{-0.000000in}{0.000000in}}{%
\pgfpathmoveto{\pgfqpoint{-0.000000in}{0.000000in}}%
\pgfpathlineto{\pgfqpoint{-0.027778in}{0.000000in}}%
\pgfusepath{stroke,fill}%
}%
\begin{pgfscope}%
\pgfsys@transformshift{0.594124in}{2.897698in}%
\pgfsys@useobject{currentmarker}{}%
\end{pgfscope}%
\end{pgfscope}%
\begin{pgfscope}%
\pgfsetbuttcap%
\pgfsetroundjoin%
\definecolor{currentfill}{rgb}{0.000000,0.000000,0.000000}%
\pgfsetfillcolor{currentfill}%
\pgfsetlinewidth{0.602250pt}%
\definecolor{currentstroke}{rgb}{0.000000,0.000000,0.000000}%
\pgfsetstrokecolor{currentstroke}%
\pgfsetdash{}{0pt}%
\pgfsys@defobject{currentmarker}{\pgfqpoint{-0.027778in}{0.000000in}}{\pgfqpoint{-0.000000in}{0.000000in}}{%
\pgfpathmoveto{\pgfqpoint{-0.000000in}{0.000000in}}%
\pgfpathlineto{\pgfqpoint{-0.027778in}{0.000000in}}%
\pgfusepath{stroke,fill}%
}%
\begin{pgfscope}%
\pgfsys@transformshift{0.594124in}{2.967971in}%
\pgfsys@useobject{currentmarker}{}%
\end{pgfscope}%
\end{pgfscope}%
\begin{pgfscope}%
\pgfsetbuttcap%
\pgfsetroundjoin%
\definecolor{currentfill}{rgb}{0.000000,0.000000,0.000000}%
\pgfsetfillcolor{currentfill}%
\pgfsetlinewidth{0.602250pt}%
\definecolor{currentstroke}{rgb}{0.000000,0.000000,0.000000}%
\pgfsetstrokecolor{currentstroke}%
\pgfsetdash{}{0pt}%
\pgfsys@defobject{currentmarker}{\pgfqpoint{-0.027778in}{0.000000in}}{\pgfqpoint{-0.000000in}{0.000000in}}{%
\pgfpathmoveto{\pgfqpoint{-0.000000in}{0.000000in}}%
\pgfpathlineto{\pgfqpoint{-0.027778in}{0.000000in}}%
\pgfusepath{stroke,fill}%
}%
\begin{pgfscope}%
\pgfsys@transformshift{0.594124in}{3.029955in}%
\pgfsys@useobject{currentmarker}{}%
\end{pgfscope}%
\end{pgfscope}%
\begin{pgfscope}%
\definecolor{textcolor}{rgb}{0.000000,0.000000,0.000000}%
\pgfsetstrokecolor{textcolor}%
\pgfsetfillcolor{textcolor}%
\pgftext[x=0.265420in,y=1.873660in,,bottom,rotate=90.000000]{\color{textcolor}\rmfamily\fontsize{10.000000}{12.000000}\selectfont Controller Output}%
\end{pgfscope}%
\begin{pgfscope}%
\pgfpathrectangle{\pgfqpoint{0.594124in}{0.540713in}}{\pgfqpoint{4.686978in}{2.665893in}}%
\pgfusepath{clip}%
\pgfsetrectcap%
\pgfsetroundjoin%
\pgfsetlinewidth{1.003750pt}%
\definecolor{currentstroke}{rgb}{0.003922,0.450980,0.698039}%
\pgfsetstrokecolor{currentstroke}%
\pgfsetstrokeopacity{0.700000}%
\pgfsetdash{}{0pt}%
\pgfpathmoveto{\pgfqpoint{0.807169in}{1.876252in}}%
\pgfpathlineto{\pgfqpoint{0.894126in}{1.729835in}}%
\pgfpathlineto{\pgfqpoint{0.981083in}{1.584870in}}%
\pgfpathlineto{\pgfqpoint{1.068040in}{1.442385in}}%
\pgfpathlineto{\pgfqpoint{1.154997in}{1.304055in}}%
\pgfpathlineto{\pgfqpoint{1.241953in}{1.172458in}}%
\pgfpathlineto{\pgfqpoint{1.328910in}{1.051206in}}%
\pgfpathlineto{\pgfqpoint{1.415867in}{0.944625in}}%
\pgfpathlineto{\pgfqpoint{1.502824in}{0.856669in}}%
\pgfpathlineto{\pgfqpoint{1.589781in}{0.789313in}}%
\pgfpathlineto{\pgfqpoint{1.676738in}{0.741553in}}%
\pgfpathlineto{\pgfqpoint{1.763695in}{0.709930in}}%
\pgfpathlineto{\pgfqpoint{1.850652in}{0.690081in}}%
\pgfpathlineto{\pgfqpoint{1.937609in}{0.678082in}}%
\pgfpathlineto{\pgfqpoint{2.024566in}{0.671005in}}%
\pgfpathlineto{\pgfqpoint{2.111523in}{0.666894in}}%
\pgfpathlineto{\pgfqpoint{2.198480in}{0.664533in}}%
\pgfpathlineto{\pgfqpoint{2.285437in}{0.663191in}}%
\pgfpathlineto{\pgfqpoint{2.372393in}{0.662444in}}%
\pgfpathlineto{\pgfqpoint{2.459350in}{0.662052in}}%
\pgfpathlineto{\pgfqpoint{2.546307in}{0.661890in}}%
\pgfpathlineto{\pgfqpoint{2.633264in}{0.661904in}}%
\pgfpathlineto{\pgfqpoint{2.720221in}{0.662099in}}%
\pgfpathlineto{\pgfqpoint{2.807178in}{0.662539in}}%
\pgfpathlineto{\pgfqpoint{2.894135in}{0.663365in}}%
\pgfpathlineto{\pgfqpoint{2.981092in}{0.664841in}}%
\pgfpathlineto{\pgfqpoint{3.068049in}{0.667432in}}%
\pgfpathlineto{\pgfqpoint{3.155006in}{0.671936in}}%
\pgfpathlineto{\pgfqpoint{3.241963in}{0.679674in}}%
\pgfpathlineto{\pgfqpoint{3.328920in}{0.692749in}}%
\pgfpathlineto{\pgfqpoint{3.415877in}{0.714265in}}%
\pgfpathlineto{\pgfqpoint{3.502833in}{0.748279in}}%
\pgfpathlineto{\pgfqpoint{3.589790in}{0.799114in}}%
\pgfpathlineto{\pgfqpoint{3.676747in}{0.869916in}}%
\pgfpathlineto{\pgfqpoint{3.763704in}{0.961186in}}%
\pgfpathlineto{\pgfqpoint{3.850661in}{1.070516in}}%
\pgfpathlineto{\pgfqpoint{3.937618in}{1.193783in}}%
\pgfpathlineto{\pgfqpoint{4.024575in}{1.326725in}}%
\pgfpathlineto{\pgfqpoint{4.111532in}{1.465899in}}%
\pgfpathlineto{\pgfqpoint{4.198489in}{1.608893in}}%
\pgfpathlineto{\pgfqpoint{4.285446in}{1.754158in}}%
\pgfpathlineto{\pgfqpoint{4.372403in}{1.900749in}}%
\pgfpathlineto{\pgfqpoint{4.459360in}{2.048105in}}%
\pgfpathlineto{\pgfqpoint{4.546317in}{2.195900in}}%
\pgfpathlineto{\pgfqpoint{4.633273in}{2.343947in}}%
\pgfpathlineto{\pgfqpoint{4.720230in}{2.492136in}}%
\pgfpathlineto{\pgfqpoint{4.807187in}{2.640408in}}%
\pgfpathlineto{\pgfqpoint{4.894144in}{2.788725in}}%
\pgfpathlineto{\pgfqpoint{4.981101in}{2.937070in}}%
\pgfpathlineto{\pgfqpoint{5.068058in}{3.085429in}}%
\pgfusepath{stroke}%
\end{pgfscope}%
\begin{pgfscope}%
\pgfpathrectangle{\pgfqpoint{0.594124in}{0.540713in}}{\pgfqpoint{4.686978in}{2.665893in}}%
\pgfusepath{clip}%
\pgfsetrectcap%
\pgfsetroundjoin%
\pgfsetlinewidth{1.003750pt}%
\definecolor{currentstroke}{rgb}{0.870588,0.560784,0.019608}%
\pgfsetstrokecolor{currentstroke}%
\pgfsetstrokeopacity{0.700000}%
\pgfsetdash{}{0pt}%
\pgfpathmoveto{\pgfqpoint{0.807169in}{1.876252in}}%
\pgfpathlineto{\pgfqpoint{0.894126in}{1.729835in}}%
\pgfpathlineto{\pgfqpoint{0.981083in}{1.584870in}}%
\pgfpathlineto{\pgfqpoint{1.068040in}{1.442385in}}%
\pgfpathlineto{\pgfqpoint{1.154997in}{1.304055in}}%
\pgfpathlineto{\pgfqpoint{1.241953in}{1.172458in}}%
\pgfpathlineto{\pgfqpoint{1.328910in}{1.051206in}}%
\pgfpathlineto{\pgfqpoint{1.415867in}{0.944625in}}%
\pgfpathlineto{\pgfqpoint{1.502824in}{0.856669in}}%
\pgfpathlineto{\pgfqpoint{1.589781in}{0.789313in}}%
\pgfpathlineto{\pgfqpoint{1.676738in}{0.741553in}}%
\pgfpathlineto{\pgfqpoint{1.763695in}{0.709930in}}%
\pgfpathlineto{\pgfqpoint{1.850652in}{0.690082in}}%
\pgfpathlineto{\pgfqpoint{1.937609in}{0.678083in}}%
\pgfpathlineto{\pgfqpoint{2.024566in}{0.671006in}}%
\pgfpathlineto{\pgfqpoint{2.111523in}{0.666897in}}%
\pgfpathlineto{\pgfqpoint{2.198480in}{0.664537in}}%
\pgfpathlineto{\pgfqpoint{2.285437in}{0.663199in}}%
\pgfpathlineto{\pgfqpoint{2.372393in}{0.662457in}}%
\pgfpathlineto{\pgfqpoint{2.459350in}{0.662076in}}%
\pgfpathlineto{\pgfqpoint{2.546307in}{0.661932in}}%
\pgfpathlineto{\pgfqpoint{2.633264in}{0.661977in}}%
\pgfpathlineto{\pgfqpoint{2.720221in}{0.662228in}}%
\pgfpathlineto{\pgfqpoint{2.807178in}{0.662764in}}%
\pgfpathlineto{\pgfqpoint{2.894135in}{0.663759in}}%
\pgfpathlineto{\pgfqpoint{2.981092in}{0.665529in}}%
\pgfpathlineto{\pgfqpoint{3.068049in}{0.668629in}}%
\pgfpathlineto{\pgfqpoint{3.155006in}{0.673997in}}%
\pgfpathlineto{\pgfqpoint{3.241963in}{0.683176in}}%
\pgfpathlineto{\pgfqpoint{3.328920in}{0.698560in}}%
\pgfpathlineto{\pgfqpoint{3.415877in}{0.723559in}}%
\pgfpathlineto{\pgfqpoint{3.502833in}{0.762357in}}%
\pgfpathlineto{\pgfqpoint{3.589790in}{0.818923in}}%
\pgfpathlineto{\pgfqpoint{3.676747in}{0.895389in}}%
\pgfpathlineto{\pgfqpoint{3.763704in}{0.990788in}}%
\pgfpathlineto{\pgfqpoint{3.850661in}{1.101219in}}%
\pgfpathlineto{\pgfqpoint{3.937618in}{1.221136in}}%
\pgfpathlineto{\pgfqpoint{4.024575in}{1.344539in}}%
\pgfpathlineto{\pgfqpoint{4.111532in}{1.465410in}}%
\pgfpathlineto{\pgfqpoint{4.198489in}{1.577736in}}%
\pgfpathlineto{\pgfqpoint{4.285446in}{1.675836in}}%
\pgfpathlineto{\pgfqpoint{4.372403in}{1.755451in}}%
\pgfpathlineto{\pgfqpoint{4.459360in}{1.815099in}}%
\pgfpathlineto{\pgfqpoint{4.546317in}{1.856474in}}%
\pgfpathlineto{\pgfqpoint{4.633273in}{1.883370in}}%
\pgfpathlineto{\pgfqpoint{4.720230in}{1.900026in}}%
\pgfpathlineto{\pgfqpoint{4.807187in}{1.910005in}}%
\pgfpathlineto{\pgfqpoint{4.894144in}{1.915860in}}%
\pgfpathlineto{\pgfqpoint{4.981101in}{1.919251in}}%
\pgfpathlineto{\pgfqpoint{5.068058in}{1.921201in}}%
\pgfusepath{stroke}%
\end{pgfscope}%
\begin{pgfscope}%
\pgfsetrectcap%
\pgfsetmiterjoin%
\pgfsetlinewidth{0.803000pt}%
\definecolor{currentstroke}{rgb}{0.000000,0.000000,0.000000}%
\pgfsetstrokecolor{currentstroke}%
\pgfsetdash{}{0pt}%
\pgfpathmoveto{\pgfqpoint{0.594124in}{0.540713in}}%
\pgfpathlineto{\pgfqpoint{0.594124in}{3.206606in}}%
\pgfusepath{stroke}%
\end{pgfscope}%
\begin{pgfscope}%
\pgfsetrectcap%
\pgfsetmiterjoin%
\pgfsetlinewidth{0.803000pt}%
\definecolor{currentstroke}{rgb}{0.000000,0.000000,0.000000}%
\pgfsetstrokecolor{currentstroke}%
\pgfsetdash{}{0pt}%
\pgfpathmoveto{\pgfqpoint{5.281103in}{0.540713in}}%
\pgfpathlineto{\pgfqpoint{5.281103in}{3.206606in}}%
\pgfusepath{stroke}%
\end{pgfscope}%
\begin{pgfscope}%
\pgfsetrectcap%
\pgfsetmiterjoin%
\pgfsetlinewidth{0.803000pt}%
\definecolor{currentstroke}{rgb}{0.000000,0.000000,0.000000}%
\pgfsetstrokecolor{currentstroke}%
\pgfsetdash{}{0pt}%
\pgfpathmoveto{\pgfqpoint{0.594124in}{0.540713in}}%
\pgfpathlineto{\pgfqpoint{5.281103in}{0.540713in}}%
\pgfusepath{stroke}%
\end{pgfscope}%
\begin{pgfscope}%
\pgfsetrectcap%
\pgfsetmiterjoin%
\pgfsetlinewidth{0.803000pt}%
\definecolor{currentstroke}{rgb}{0.000000,0.000000,0.000000}%
\pgfsetstrokecolor{currentstroke}%
\pgfsetdash{}{0pt}%
\pgfpathmoveto{\pgfqpoint{0.594124in}{3.206606in}}%
\pgfpathlineto{\pgfqpoint{5.281103in}{3.206606in}}%
\pgfusepath{stroke}%
\end{pgfscope}%
\begin{pgfscope}%
\pgfsetbuttcap%
\pgfsetmiterjoin%
\definecolor{currentfill}{rgb}{1.000000,1.000000,1.000000}%
\pgfsetfillcolor{currentfill}%
\pgfsetfillopacity{0.800000}%
\pgfsetlinewidth{1.003750pt}%
\definecolor{currentstroke}{rgb}{0.800000,0.800000,0.800000}%
\pgfsetstrokecolor{currentstroke}%
\pgfsetstrokeopacity{0.800000}%
\pgfsetdash{}{0pt}%
\pgfpathmoveto{\pgfqpoint{0.671902in}{2.746699in}}%
\pgfpathlineto{\pgfqpoint{3.834384in}{2.746699in}}%
\pgfpathquadraticcurveto{\pgfqpoint{3.856607in}{2.746699in}}{\pgfqpoint{3.856607in}{2.768921in}}%
\pgfpathlineto{\pgfqpoint{3.856607in}{3.128828in}}%
\pgfpathquadraticcurveto{\pgfqpoint{3.856607in}{3.151050in}}{\pgfqpoint{3.834384in}{3.151050in}}%
\pgfpathlineto{\pgfqpoint{0.671902in}{3.151050in}}%
\pgfpathquadraticcurveto{\pgfqpoint{0.649680in}{3.151050in}}{\pgfqpoint{0.649680in}{3.128828in}}%
\pgfpathlineto{\pgfqpoint{0.649680in}{2.768921in}}%
\pgfpathquadraticcurveto{\pgfqpoint{0.649680in}{2.746699in}}{\pgfqpoint{0.671902in}{2.746699in}}%
\pgfpathlineto{\pgfqpoint{0.671902in}{2.746699in}}%
\pgfpathclose%
\pgfusepath{stroke,fill}%
\end{pgfscope}%
\begin{pgfscope}%
\pgfsetrectcap%
\pgfsetroundjoin%
\pgfsetlinewidth{1.003750pt}%
\definecolor{currentstroke}{rgb}{0.003922,0.450980,0.698039}%
\pgfsetstrokecolor{currentstroke}%
\pgfsetstrokeopacity{0.700000}%
\pgfsetdash{}{0pt}%
\pgfpathmoveto{\pgfqpoint{0.694124in}{3.045634in}}%
\pgfpathlineto{\pgfqpoint{0.805235in}{3.045634in}}%
\pgfpathlineto{\pgfqpoint{0.916347in}{3.045634in}}%
\pgfusepath{stroke}%
\end{pgfscope}%
\begin{pgfscope}%
\definecolor{textcolor}{rgb}{0.000000,0.000000,0.000000}%
\pgfsetstrokecolor{textcolor}%
\pgfsetfillcolor{textcolor}%
\pgftext[x=1.005235in,y=3.006745in,left,base]{\color{textcolor}\rmfamily\fontsize{8.000000}{9.600000}\selectfont PID, \(\displaystyle K_p=1\), \(\displaystyle K_i=\qty{0.1}{\per \s}\), \(\displaystyle K_d=\qty{0.01}{\s}\)}%
\end{pgfscope}%
\begin{pgfscope}%
\pgfsetrectcap%
\pgfsetroundjoin%
\pgfsetlinewidth{1.003750pt}%
\definecolor{currentstroke}{rgb}{0.870588,0.560784,0.019608}%
\pgfsetstrokecolor{currentstroke}%
\pgfsetstrokeopacity{0.700000}%
\pgfsetdash{}{0pt}%
\pgfpathmoveto{\pgfqpoint{0.694124in}{2.860125in}}%
\pgfpathlineto{\pgfqpoint{0.805235in}{2.860125in}}%
\pgfpathlineto{\pgfqpoint{0.916347in}{2.860125in}}%
\pgfusepath{stroke}%
\end{pgfscope}%
\begin{pgfscope}%
\definecolor{textcolor}{rgb}{0.000000,0.000000,0.000000}%
\pgfsetstrokecolor{textcolor}%
\pgfsetfillcolor{textcolor}%
\pgftext[x=1.005235in,y=2.821236in,left,base]{\color{textcolor}\rmfamily\fontsize{8.000000}{9.600000}\selectfont PID+filter,\(\displaystyle K_p=1\), \(\displaystyle K_i=\qty{0.1}{\per \s}\), \(\displaystyle K_d=\qty{0.01}{\s}\), \(\displaystyle \alpha=\num0.1\)}%
\end{pgfscope}%
\end{pgfpicture}%
\makeatother%
\endgroup%
% data/simulations/sim_pid_controller_bode.py
    \caption{Magnitude plot over frequency of the PID controller transfer function. Both the ideal PID controller and the PID controller with a filtered derivative are shown.}
    \label{fig:sim_pid_controller}
\end{figure}

Another issue is caused by the derivative term with noisy inputs. Assuming there is a very short input spike due to noise, the differential of the derivative term will again be sent to very high values, pushing the output away from the correct value and forcing the controller to slowly rebalance.

To further discuss the problem and its solution it is best to visit the frequency domain and visualize the transfer function of the PID controller as shown in figure \ref{fig:sim_pid_controller}. The ideal PID controller without filtering of the derivative displays a very strong response to low frequency inputs. This is due to the integral action, which removes any (constant) offset. It needs to have infinite gain at DC to push the offset to zero. In reality this is limited by input noise. Then follows a plateau with a magnitude of $k_p$ for the proportional term and finally the differential gain starts growing in magnitude and keeps steadily growing with rising frequency, just as expected.

With some knowledge about the process or the sensor it is possible to define an upper frequency, above which inputs become unrealistic and must therefore be unwanted noise. By filtering the derivative term with a first order filter causes it to roll off and its gain becomes constant as shown in figure \ref{fig:sim_pid_controller}. By adding the filter the PID controller transfer function changes to

\begin{equation}
    C(s) = k_p + k_i \frac{1}{s} + \frac{k_d s}{1 + s \alpha k_d} \,. \label{eqn:pid_controller_filtered}
\end{equation}

Typically $\alpha$ is in the range of \numrange{0.05}{0.2} \citep[p. 129]{pid_controller}.

An alternative is to filter the whole input. Depending on the filter cutoff, there is not much difference to equation \ref{eqn:pid_controller_filtered}, because the filter will not touch the proportional and integral part of the transfer function if both are well within its passband.

From figure \ref{fig:sim_pid_controller} it can also be seen, why in some publications, the gain $k_p$ is applied to all three terms and $k_i$ and $k_d$ are replaced with $T_i$ and $T_d$ to accommodate for that.
\begin{equation}
    C(s) = k_p \left(1 + \frac{1}{T_i s} + \frac{T_d s}{1 + s \alpha T_d} \right) \label{eqn:pid_controller_series}
\end{equation}

Using this form allows to shift the overall gain up and down keeping its shape instead of just the $k_p$ part, thus changing the corner frequencies. The alternative form is only given here for the sake of completeness. The author uses the ideal form shown in equation \ref{eqn:pid_controller_laplace} with the parameters $k_p$, $k_i$, and $k_d$ wherever possible.

This concludes the discussion of the PID controller and the introduction of the basic terms. It now begs the question how the controller interacts with the system and how to derive the optimal PID parameters from a given system or model. Thus, the next section discusses controller tuning rules and their effect on the system performance.

\subsection{PID Tuning Rules}%
\label{sec:pid_tuning_rules}
While many PID tuning rules can be found in the literature, their application depends on the underlying system and the desired system response. This section will discuss several proposed solutions and compare them to the authors use case. The section aims to give a simple method to determine decent PI/PID parameters for the applications found in the lab. Among the methods discussed are the most classic set of tuning rules developed by \citeauthor{ziegler_nichols} \cite{ziegler_nichols}, and an improved version called Skogestad Internal Model Control (SIMC) presented by \citeauthor{simc_paper} \cite{simc_paper} which promises better performance for non-integrating systems. These rules all include simple instructions to extract the necessary parameters using pen and paper. Using a computer and fitting algorithms, the bar for \textit{simple} has been raised considerably, so more complex approaches can be undertaken which extract more parameters from the system. Using these additional parameters, more precise control is promised by \citeauthor{pid_basics} \cite{pid_basics, advanced_pid_control} with a method called AMIGO. Finally, it is possible to shape the control loop to result in a desired transfer function. This technique is mostly used in motor control \cite{pid_controller,advanced_pid_control} and also requires the model parameters.

All of these rules will be compared against a demo model of a room to explain the details. It is the first order model with delay which was derived in equation \ref{eqn:first-order_plus_dead_time_model}. The discussion is limited to the FOPDT model, because the systems treated in this work could be modelled very well using this equation. Higher-order models are discussed in more details for example in \cite{advanced_pid_control,pid_controller,simc_paper}, in case the reader encounters such a system and feels the need to extract the model parameters.
\begin{equation}
    G(s) = \frac{K e^{-\theta s}}{1 + s \tau} \label{eqn:demo_process_model}
\end{equation}

The following parameters were extracted from lab 011 of the APQ group, using the techniques shown in section \ref{sec:temperature_control_model} using equation \ref{eqn:first-order_plus_dead_time_model_time-domain}. The details are discussed in section \ref{sec:pid_controller_tuning}. The system gain $K$ was scaled to the full scale output (\qty{4095}{\bit}) of the controller, hence the somewhat strange unit \unit[per-mode=power]{\K \bit\per\bit}.
\begin{table}[hb]
    \centering
    \begin{tabular}{ccc}
        \toprule
        Gain K& Lag $\tau$& Delay $\theta$ \\
        \midrule
        \qty[per-mode=power]{13.07}{\K \bit\per\bit}& \qty{395}{\s}& \qty{187}{\s}\\
        \bottomrule
    \end{tabular}
    \caption{Example paramters extracted from lab 011 using the techniques shown here and as applied in section \ref{sec:pid_controller_tuning}.}.
    \label{tab:pid_example_model}
\end{table}

Before detailing the tuning parameters, the loop shaping method will be explained first, because it cannot only be used to derive custom rules but was also used to create the SIMC rules proposed by \citeauthor{simc_paper} \cite{simc_paper}. The aim of this method is to derive a controller that shapes the model in such a way that a desired system response to setpoint changes is achieved. A general closed-loop system with a controller $C$ and a system $G$ is shown in figure \ref{fig:closed_loop_controller}. This will be used as a basis to find the required controller for a desired transfer function $\frac{Y(s)}{U(s)}$.
\begin{figure}[ht]
    \centering
    \scalebox{1}{%
        \import{figures/}{closed_loop_controller.tex}
    }% scalebox
    \caption{Closed-loop system $G$ with a controller $C$.}
    \label{fig:closed_loop_controller}
\end{figure}

Starting with the transfer function of the controlled system, made up of the controller and the system, most experimenters would, at least in a feverish dream, prefer a transfer function of the following divine form
\begin{equation*}
    \frac{Y(s)}{U(s)} = 1 \,,
\end{equation*}
but unfortunately life is more profane and there is no controller that will always (and with warp speed) force a system to a certain setpoint. One may therefore settle for the second-best choice, a first order low pass with a slow roll-off and a small delay, which must be added to ensure causality. One therefore arrives at
\begin{equation}
    \frac{Y(s)}{U(s)} = \frac{e^{-\theta s}}{1 + s \tau_c}\,, \label{eqn:desired_transfer_function}
\end{equation}
where $\tau_c$ is the closed-loop time constant and a measure for the aggressiveness of the controller. A small $\tau_c$ results in a more aggressive controller with faster response.

For the system shown figure \ref{fig:closed_loop_controller} the closed-loop transfer function is found to be
\begin{align*}
    \frac{Y(s)}{U(s)} &= \frac{C(s) G(s)}{C(s) G(s) + 1} \\
    \Rightarrow C(s) &= \frac{1}{G(s)} \frac{1}{\frac{Y(s)}{U(s)} -1}
\end{align*}

This loop now needs to be shaped into the desired transfer function given in equation \ref{eqn:desired_transfer_function}, so substituting $\frac{Y(s)}{U(s)}$ yields
\begin{align}
    C(s) &= \frac{1}{G(s)} \frac{e^{-\theta s}}{s \tau_c +1 - \underbrace{e^{-\theta s}}_{\approx 1 - \theta s}}\\
    &\approx \frac{1}{G(s)} \frac{e^{-\theta s}}{s (\tau_c + \theta)} \,.
\end{align}

$e^{-\theta s}$ was approximated using a first order Taylor expansion. The desired controller response now only depends on the system (including the sensor) to be controlled. So, substituting the system equation \ref{eqn:demo_process_model} results in
\begin{align}
    C(s) &= \frac{1}{K} \frac{s \tau + 1}{(\tau_c + \theta) s} \nonumber\\
    &= \underbrace{\frac{1}{K} \frac{\tau}{\tau_c + \theta}}_{k_p} + \underbrace{\frac{1}{K} \frac{1}{\tau_c + \theta}}_{k_i} \frac{1}{s}\,.
\end{align}

This is a PI controller with $k_p = \frac{1}{K} \frac{\tau}{\tau_c + \theta}$ and $k_i = \frac{1}{K} \frac{1}{\tau_c + \theta}$. From these calculations, it can be seen that a first order model can be fully treated using a PI controller. Second order (and higher order) models typically necessitate a PID or more sophisticated controller for optimal control. The problems discussed in this work mainly focus temperature control of (mostly) homogeneous objects, so the focus lies on the PI controller for most of the remaining section but the ideas and simulations can similarly be applied to the PID controller as well. Any caveats to be expected when treating a PID instead of a PI controller will be discussed.

Using the loop shaping technique, it is fairly easy to derive custom rules in case the model parameters can be extracted. As mentioned above, one such loop-shaped tuning rule is the SIMC rule set and the authors of those rules give advice for an ample variety of different models and also investigate the parameter choice regarding stability, load, and setpoint disturbances. Before attempting a custom approach, it is therefore recommended to check \cite{simc_paper} for an appropriate set of rules for more complex models in order to save time and effort.
\begin{table}
    \centering
    \begin{tabular}{lcccc}
        \toprule
        Tuning Rule& $k_p$& $T_i$ & $T_d$ & Source \\
        \midrule
        Z-N PI & $\frac{0.9 \tau}{K \theta}$ & $\frac{\theta}{0.3}$ & -- & \cite{ziegler_nichols}\\
        Z-N PID & $\frac{1.2 \tau}{K \theta}$ & $2 \theta$ & $\frac{\theta}{2}$ & \cite{ziegler_nichols}\\
        SIMC PI & $\frac{\tau}{K (\tau_c + \theta)}$ & $\min\left(\tau, 4 (\tau_c+\theta)\right)$ & -- & \cite{simc_paper}\\
        SIMC PID & $\frac{\tau_1}{K (\tau_c + \theta)}$ & $\min\left(\tau_1, 4 (\tau_c+\theta)\right)$ & $\tau_2$ & \cite{simc_paper}\\
        AMIGO PI & $\frac{0.15}{K} + \left(0.35 - \frac{\tau \theta}{\left(\tau + \theta\right)^2}\right) \frac{\tau}{K \theta}$ & $0.35 \theta + \frac{13 \tau^2 \theta}{\tau^2 + 12 \tau \theta + 7 \theta^2}$ & -- & \citep[p. 228]{advanced_pid_control}\\
        AMIGO PID & $\frac{1}{K} \left(0.2 + 0.45 \frac{\tau}{\theta}\right)$ & $\frac{0.4 \theta + 0.8 \tau}{\theta + 0.1 \tau} \theta$ & $\frac{0.5 \tau \theta}{0.3 \theta + \tau}$ & \citep[p. 233]{advanced_pid_control}\\
        \bottomrule
    \end{tabular}
    \caption{PI/PID parameters for different tuning rules. The PI controllers assume a first order model, the PID rules are required when dealing with a second order model.}
    \label{tab:pid_tuning_parameters}
\end{table}

For reasons of brevity, in table \ref{tab:pid_tuning_parameters}, the PID parameters are given as $k_p$, $T_i$ and $T_d$ as introduced in equation \ref{eqn:pid_controller_series}. $k_i$ and $k_d$ can be calculated from
\begin{align*}
    k_i &= \frac{k_p}{T_i}\\
    k_d &= k_p T_d\,.
\end{align*}

Regarding the SIMC PI/PID algorithm, \citeauthor{simc_paper} \cite{simc_paper} and \citep[ch. 5]{simc} suggests using $\tau_c = \theta$ for “\textit{tightest possible subject to maintaining smooth control}“. Following this recommendation, the minimum can be calculated from the parameters given in table \ref{tab:pid_example_model} on page \pageref{tab:pid_example_model} as $\min\left(\tau, 4 (\tau_c+\theta)\right) = \min\left(\tau, 8 \theta\right) = \tau$.

Using the rules above, the full system can be simulated now. This was done using Python. The simulation source code can be found in \external{data/simulations/sim\_pid\_controller.py} as part of the online supplemental material \cite{supplemental_material}. The simulation can be used to model arbitrary PI(D) controller and arbitrary models can be used as well. It allows to compare different settings before applying them to a real system. It also considerably shortens deployment times because especially for systems with long time scales, it becomes difficult to test several parameter sets on the fly, thus a simulation can reduce deployment time to a few minutes instead of hours.

The simulation emulates the PID controller developed for the lab temperature controller. By default it has a sampling rate of \qty{1}{\Hz}. The simulation  will apply a setpoint change of \qty{+1}{\K} \qty{10}{\s} into the simulation. After the simulation, it will plot the time domain response of the controlled system. The setpoint change in this scenario is very similar to the load disturbances that are expected. Typically a noise source is used here instead, but in contrast to the statistical noise, which could be used to test for disturbance rejection, the situation in labs are different and cannot be modelled with stationary noise. While there is some noise coming from the sensor and the lab, the major disturbances are usually caused by the experimenters instead of the lab itself. These are events like a device being switched on or off for an extended period of time, longer than the controller needs to settle. This is equivalent to a setpoint change in terms of the error term in equation \ref{eqn:pid_controller}, since there is no difference in the error term between a setpoint and a process variable change. Do note, that this is not true for the PID controller, whose derivative term directly works on the measurement (or process variable) as this was explicitly implemented above. For PID controllers, there is a difference between the setpoint change behaviour and system noise rejection. This must be kept in mind and tested accordingly.

Simulating the model above and using the PI parameters derived from table \ref{tab:pid_tuning_parameters}, gives the plot shown in figure \ref{fig:pid_controller_comparison}.
\begin{figure}[ht]
    \centering
    %% Creator: Matplotlib, PGF backend
%%
%% To include the figure in your LaTeX document, write
%%   \input{<filename>.pgf}
%%
%% Make sure the required packages are loaded in your preamble
%%   \usepackage{pgf}
%%
%% Also ensure that all the required font packages are loaded; for instance,
%% the lmodern package is sometimes necessary when using math font.
%%   \usepackage{lmodern}
%%
%% Figures using additional raster images can only be included by \input if
%% they are in the same directory as the main LaTeX file. For loading figures
%% from other directories you can use the `import` package
%%   \usepackage{import}
%%
%% and then include the figures with
%%   \import{<path to file>}{<filename>.pgf}
%%
%% Matplotlib used the following preamble
%%   \def\mathdefault#1{#1}
%%   \everymath=\expandafter{\the\everymath\displaystyle}
%%   \usepackage{siunitx}
%%   \sisetup{per-mode = symbol}%
%%   \ifdefined\pdftexversion\else  % non-pdftex case.
%%     \usepackage{fontspec}
%%   \fi
%%   \makeatletter\@ifpackageloaded{underscore}{}{\usepackage[strings]{underscore}}\makeatother
%%
\begingroup%
\makeatletter%
\begin{pgfpicture}%
\pgfpathrectangle{\pgfpointorigin}{\pgfqpoint{5.492126in}{3.394321in}}%
\pgfusepath{use as bounding box, clip}%
\begin{pgfscope}%
\pgfsetbuttcap%
\pgfsetmiterjoin%
\definecolor{currentfill}{rgb}{1.000000,1.000000,1.000000}%
\pgfsetfillcolor{currentfill}%
\pgfsetlinewidth{0.000000pt}%
\definecolor{currentstroke}{rgb}{1.000000,1.000000,1.000000}%
\pgfsetstrokecolor{currentstroke}%
\pgfsetdash{}{0pt}%
\pgfpathmoveto{\pgfqpoint{0.000000in}{0.000000in}}%
\pgfpathlineto{\pgfqpoint{5.492126in}{0.000000in}}%
\pgfpathlineto{\pgfqpoint{5.492126in}{3.394321in}}%
\pgfpathlineto{\pgfqpoint{0.000000in}{3.394321in}}%
\pgfpathlineto{\pgfqpoint{0.000000in}{0.000000in}}%
\pgfpathclose%
\pgfusepath{fill}%
\end{pgfscope}%
\begin{pgfscope}%
\pgfsetbuttcap%
\pgfsetmiterjoin%
\definecolor{currentfill}{rgb}{1.000000,1.000000,1.000000}%
\pgfsetfillcolor{currentfill}%
\pgfsetlinewidth{0.000000pt}%
\definecolor{currentstroke}{rgb}{0.000000,0.000000,0.000000}%
\pgfsetstrokecolor{currentstroke}%
\pgfsetstrokeopacity{0.000000}%
\pgfsetdash{}{0pt}%
\pgfpathmoveto{\pgfqpoint{0.576061in}{0.524170in}}%
\pgfpathlineto{\pgfqpoint{5.342126in}{0.524170in}}%
\pgfpathlineto{\pgfqpoint{5.342126in}{3.244321in}}%
\pgfpathlineto{\pgfqpoint{0.576061in}{3.244321in}}%
\pgfpathlineto{\pgfqpoint{0.576061in}{0.524170in}}%
\pgfpathclose%
\pgfusepath{fill}%
\end{pgfscope}%
\begin{pgfscope}%
\pgfsetbuttcap%
\pgfsetroundjoin%
\definecolor{currentfill}{rgb}{0.000000,0.000000,0.000000}%
\pgfsetfillcolor{currentfill}%
\pgfsetlinewidth{0.803000pt}%
\definecolor{currentstroke}{rgb}{0.000000,0.000000,0.000000}%
\pgfsetstrokecolor{currentstroke}%
\pgfsetdash{}{0pt}%
\pgfsys@defobject{currentmarker}{\pgfqpoint{0.000000in}{-0.048611in}}{\pgfqpoint{0.000000in}{0.000000in}}{%
\pgfpathmoveto{\pgfqpoint{0.000000in}{0.000000in}}%
\pgfpathlineto{\pgfqpoint{0.000000in}{-0.048611in}}%
\pgfusepath{stroke,fill}%
}%
\begin{pgfscope}%
\pgfsys@transformshift{0.792700in}{0.524170in}%
\pgfsys@useobject{currentmarker}{}%
\end{pgfscope}%
\end{pgfscope}%
\begin{pgfscope}%
\definecolor{textcolor}{rgb}{0.000000,0.000000,0.000000}%
\pgfsetstrokecolor{textcolor}%
\pgfsetfillcolor{textcolor}%
\pgftext[x=0.792700in,y=0.426948in,,top]{\color{textcolor}{\rmfamily\fontsize{8.000000}{9.600000}\selectfont\catcode`\^=\active\def^{\ifmmode\sp\else\^{}\fi}\catcode`\%=\active\def%{\%}$\mathdefault{0}$}}%
\end{pgfscope}%
\begin{pgfscope}%
\pgfsetbuttcap%
\pgfsetroundjoin%
\definecolor{currentfill}{rgb}{0.000000,0.000000,0.000000}%
\pgfsetfillcolor{currentfill}%
\pgfsetlinewidth{0.803000pt}%
\definecolor{currentstroke}{rgb}{0.000000,0.000000,0.000000}%
\pgfsetstrokecolor{currentstroke}%
\pgfsetdash{}{0pt}%
\pgfsys@defobject{currentmarker}{\pgfqpoint{0.000000in}{-0.048611in}}{\pgfqpoint{0.000000in}{0.000000in}}{%
\pgfpathmoveto{\pgfqpoint{0.000000in}{0.000000in}}%
\pgfpathlineto{\pgfqpoint{0.000000in}{-0.048611in}}%
\pgfusepath{stroke,fill}%
}%
\begin{pgfscope}%
\pgfsys@transformshift{1.515032in}{0.524170in}%
\pgfsys@useobject{currentmarker}{}%
\end{pgfscope}%
\end{pgfscope}%
\begin{pgfscope}%
\definecolor{textcolor}{rgb}{0.000000,0.000000,0.000000}%
\pgfsetstrokecolor{textcolor}%
\pgfsetfillcolor{textcolor}%
\pgftext[x=1.515032in,y=0.426948in,,top]{\color{textcolor}{\rmfamily\fontsize{8.000000}{9.600000}\selectfont\catcode`\^=\active\def^{\ifmmode\sp\else\^{}\fi}\catcode`\%=\active\def%{\%}$\mathdefault{10}$}}%
\end{pgfscope}%
\begin{pgfscope}%
\pgfsetbuttcap%
\pgfsetroundjoin%
\definecolor{currentfill}{rgb}{0.000000,0.000000,0.000000}%
\pgfsetfillcolor{currentfill}%
\pgfsetlinewidth{0.803000pt}%
\definecolor{currentstroke}{rgb}{0.000000,0.000000,0.000000}%
\pgfsetstrokecolor{currentstroke}%
\pgfsetdash{}{0pt}%
\pgfsys@defobject{currentmarker}{\pgfqpoint{0.000000in}{-0.048611in}}{\pgfqpoint{0.000000in}{0.000000in}}{%
\pgfpathmoveto{\pgfqpoint{0.000000in}{0.000000in}}%
\pgfpathlineto{\pgfqpoint{0.000000in}{-0.048611in}}%
\pgfusepath{stroke,fill}%
}%
\begin{pgfscope}%
\pgfsys@transformshift{2.237364in}{0.524170in}%
\pgfsys@useobject{currentmarker}{}%
\end{pgfscope}%
\end{pgfscope}%
\begin{pgfscope}%
\definecolor{textcolor}{rgb}{0.000000,0.000000,0.000000}%
\pgfsetstrokecolor{textcolor}%
\pgfsetfillcolor{textcolor}%
\pgftext[x=2.237364in,y=0.426948in,,top]{\color{textcolor}{\rmfamily\fontsize{8.000000}{9.600000}\selectfont\catcode`\^=\active\def^{\ifmmode\sp\else\^{}\fi}\catcode`\%=\active\def%{\%}$\mathdefault{20}$}}%
\end{pgfscope}%
\begin{pgfscope}%
\pgfsetbuttcap%
\pgfsetroundjoin%
\definecolor{currentfill}{rgb}{0.000000,0.000000,0.000000}%
\pgfsetfillcolor{currentfill}%
\pgfsetlinewidth{0.803000pt}%
\definecolor{currentstroke}{rgb}{0.000000,0.000000,0.000000}%
\pgfsetstrokecolor{currentstroke}%
\pgfsetdash{}{0pt}%
\pgfsys@defobject{currentmarker}{\pgfqpoint{0.000000in}{-0.048611in}}{\pgfqpoint{0.000000in}{0.000000in}}{%
\pgfpathmoveto{\pgfqpoint{0.000000in}{0.000000in}}%
\pgfpathlineto{\pgfqpoint{0.000000in}{-0.048611in}}%
\pgfusepath{stroke,fill}%
}%
\begin{pgfscope}%
\pgfsys@transformshift{2.959695in}{0.524170in}%
\pgfsys@useobject{currentmarker}{}%
\end{pgfscope}%
\end{pgfscope}%
\begin{pgfscope}%
\definecolor{textcolor}{rgb}{0.000000,0.000000,0.000000}%
\pgfsetstrokecolor{textcolor}%
\pgfsetfillcolor{textcolor}%
\pgftext[x=2.959695in,y=0.426948in,,top]{\color{textcolor}{\rmfamily\fontsize{8.000000}{9.600000}\selectfont\catcode`\^=\active\def^{\ifmmode\sp\else\^{}\fi}\catcode`\%=\active\def%{\%}$\mathdefault{30}$}}%
\end{pgfscope}%
\begin{pgfscope}%
\pgfsetbuttcap%
\pgfsetroundjoin%
\definecolor{currentfill}{rgb}{0.000000,0.000000,0.000000}%
\pgfsetfillcolor{currentfill}%
\pgfsetlinewidth{0.803000pt}%
\definecolor{currentstroke}{rgb}{0.000000,0.000000,0.000000}%
\pgfsetstrokecolor{currentstroke}%
\pgfsetdash{}{0pt}%
\pgfsys@defobject{currentmarker}{\pgfqpoint{0.000000in}{-0.048611in}}{\pgfqpoint{0.000000in}{0.000000in}}{%
\pgfpathmoveto{\pgfqpoint{0.000000in}{0.000000in}}%
\pgfpathlineto{\pgfqpoint{0.000000in}{-0.048611in}}%
\pgfusepath{stroke,fill}%
}%
\begin{pgfscope}%
\pgfsys@transformshift{3.682027in}{0.524170in}%
\pgfsys@useobject{currentmarker}{}%
\end{pgfscope}%
\end{pgfscope}%
\begin{pgfscope}%
\definecolor{textcolor}{rgb}{0.000000,0.000000,0.000000}%
\pgfsetstrokecolor{textcolor}%
\pgfsetfillcolor{textcolor}%
\pgftext[x=3.682027in,y=0.426948in,,top]{\color{textcolor}{\rmfamily\fontsize{8.000000}{9.600000}\selectfont\catcode`\^=\active\def^{\ifmmode\sp\else\^{}\fi}\catcode`\%=\active\def%{\%}$\mathdefault{40}$}}%
\end{pgfscope}%
\begin{pgfscope}%
\pgfsetbuttcap%
\pgfsetroundjoin%
\definecolor{currentfill}{rgb}{0.000000,0.000000,0.000000}%
\pgfsetfillcolor{currentfill}%
\pgfsetlinewidth{0.803000pt}%
\definecolor{currentstroke}{rgb}{0.000000,0.000000,0.000000}%
\pgfsetstrokecolor{currentstroke}%
\pgfsetdash{}{0pt}%
\pgfsys@defobject{currentmarker}{\pgfqpoint{0.000000in}{-0.048611in}}{\pgfqpoint{0.000000in}{0.000000in}}{%
\pgfpathmoveto{\pgfqpoint{0.000000in}{0.000000in}}%
\pgfpathlineto{\pgfqpoint{0.000000in}{-0.048611in}}%
\pgfusepath{stroke,fill}%
}%
\begin{pgfscope}%
\pgfsys@transformshift{4.404359in}{0.524170in}%
\pgfsys@useobject{currentmarker}{}%
\end{pgfscope}%
\end{pgfscope}%
\begin{pgfscope}%
\definecolor{textcolor}{rgb}{0.000000,0.000000,0.000000}%
\pgfsetstrokecolor{textcolor}%
\pgfsetfillcolor{textcolor}%
\pgftext[x=4.404359in,y=0.426948in,,top]{\color{textcolor}{\rmfamily\fontsize{8.000000}{9.600000}\selectfont\catcode`\^=\active\def^{\ifmmode\sp\else\^{}\fi}\catcode`\%=\active\def%{\%}$\mathdefault{50}$}}%
\end{pgfscope}%
\begin{pgfscope}%
\pgfsetbuttcap%
\pgfsetroundjoin%
\definecolor{currentfill}{rgb}{0.000000,0.000000,0.000000}%
\pgfsetfillcolor{currentfill}%
\pgfsetlinewidth{0.803000pt}%
\definecolor{currentstroke}{rgb}{0.000000,0.000000,0.000000}%
\pgfsetstrokecolor{currentstroke}%
\pgfsetdash{}{0pt}%
\pgfsys@defobject{currentmarker}{\pgfqpoint{0.000000in}{-0.048611in}}{\pgfqpoint{0.000000in}{0.000000in}}{%
\pgfpathmoveto{\pgfqpoint{0.000000in}{0.000000in}}%
\pgfpathlineto{\pgfqpoint{0.000000in}{-0.048611in}}%
\pgfusepath{stroke,fill}%
}%
\begin{pgfscope}%
\pgfsys@transformshift{5.126691in}{0.524170in}%
\pgfsys@useobject{currentmarker}{}%
\end{pgfscope}%
\end{pgfscope}%
\begin{pgfscope}%
\definecolor{textcolor}{rgb}{0.000000,0.000000,0.000000}%
\pgfsetstrokecolor{textcolor}%
\pgfsetfillcolor{textcolor}%
\pgftext[x=5.126691in,y=0.426948in,,top]{\color{textcolor}{\rmfamily\fontsize{8.000000}{9.600000}\selectfont\catcode`\^=\active\def^{\ifmmode\sp\else\^{}\fi}\catcode`\%=\active\def%{\%}$\mathdefault{60}$}}%
\end{pgfscope}%
\begin{pgfscope}%
\definecolor{textcolor}{rgb}{0.000000,0.000000,0.000000}%
\pgfsetstrokecolor{textcolor}%
\pgfsetfillcolor{textcolor}%
\pgftext[x=2.959093in,y=0.272725in,,top]{\color{textcolor}{\rmfamily\fontsize{10.000000}{12.000000}\selectfont\catcode`\^=\active\def^{\ifmmode\sp\else\^{}\fi}\catcode`\%=\active\def%{\%}Time in \unit{\minute}}}%
\end{pgfscope}%
\begin{pgfscope}%
\pgfsetbuttcap%
\pgfsetroundjoin%
\definecolor{currentfill}{rgb}{0.000000,0.000000,0.000000}%
\pgfsetfillcolor{currentfill}%
\pgfsetlinewidth{0.803000pt}%
\definecolor{currentstroke}{rgb}{0.000000,0.000000,0.000000}%
\pgfsetstrokecolor{currentstroke}%
\pgfsetdash{}{0pt}%
\pgfsys@defobject{currentmarker}{\pgfqpoint{-0.048611in}{0.000000in}}{\pgfqpoint{-0.000000in}{0.000000in}}{%
\pgfpathmoveto{\pgfqpoint{-0.000000in}{0.000000in}}%
\pgfpathlineto{\pgfqpoint{-0.048611in}{0.000000in}}%
\pgfusepath{stroke,fill}%
}%
\begin{pgfscope}%
\pgfsys@transformshift{0.576061in}{0.647813in}%
\pgfsys@useobject{currentmarker}{}%
\end{pgfscope}%
\end{pgfscope}%
\begin{pgfscope}%
\definecolor{textcolor}{rgb}{0.000000,0.000000,0.000000}%
\pgfsetstrokecolor{textcolor}%
\pgfsetfillcolor{textcolor}%
\pgftext[x=0.327987in, y=0.609257in, left, base]{\color{textcolor}{\rmfamily\fontsize{8.000000}{9.600000}\selectfont\catcode`\^=\active\def^{\ifmmode\sp\else\^{}\fi}\catcode`\%=\active\def%{\%}$\mathdefault{0.0}$}}%
\end{pgfscope}%
\begin{pgfscope}%
\pgfsetbuttcap%
\pgfsetroundjoin%
\definecolor{currentfill}{rgb}{0.000000,0.000000,0.000000}%
\pgfsetfillcolor{currentfill}%
\pgfsetlinewidth{0.803000pt}%
\definecolor{currentstroke}{rgb}{0.000000,0.000000,0.000000}%
\pgfsetstrokecolor{currentstroke}%
\pgfsetdash{}{0pt}%
\pgfsys@defobject{currentmarker}{\pgfqpoint{-0.048611in}{0.000000in}}{\pgfqpoint{-0.000000in}{0.000000in}}{%
\pgfpathmoveto{\pgfqpoint{-0.000000in}{0.000000in}}%
\pgfpathlineto{\pgfqpoint{-0.048611in}{0.000000in}}%
\pgfusepath{stroke,fill}%
}%
\begin{pgfscope}%
\pgfsys@transformshift{0.576061in}{1.045131in}%
\pgfsys@useobject{currentmarker}{}%
\end{pgfscope}%
\end{pgfscope}%
\begin{pgfscope}%
\definecolor{textcolor}{rgb}{0.000000,0.000000,0.000000}%
\pgfsetstrokecolor{textcolor}%
\pgfsetfillcolor{textcolor}%
\pgftext[x=0.327987in, y=1.006576in, left, base]{\color{textcolor}{\rmfamily\fontsize{8.000000}{9.600000}\selectfont\catcode`\^=\active\def^{\ifmmode\sp\else\^{}\fi}\catcode`\%=\active\def%{\%}$\mathdefault{0.2}$}}%
\end{pgfscope}%
\begin{pgfscope}%
\pgfsetbuttcap%
\pgfsetroundjoin%
\definecolor{currentfill}{rgb}{0.000000,0.000000,0.000000}%
\pgfsetfillcolor{currentfill}%
\pgfsetlinewidth{0.803000pt}%
\definecolor{currentstroke}{rgb}{0.000000,0.000000,0.000000}%
\pgfsetstrokecolor{currentstroke}%
\pgfsetdash{}{0pt}%
\pgfsys@defobject{currentmarker}{\pgfqpoint{-0.048611in}{0.000000in}}{\pgfqpoint{-0.000000in}{0.000000in}}{%
\pgfpathmoveto{\pgfqpoint{-0.000000in}{0.000000in}}%
\pgfpathlineto{\pgfqpoint{-0.048611in}{0.000000in}}%
\pgfusepath{stroke,fill}%
}%
\begin{pgfscope}%
\pgfsys@transformshift{0.576061in}{1.442450in}%
\pgfsys@useobject{currentmarker}{}%
\end{pgfscope}%
\end{pgfscope}%
\begin{pgfscope}%
\definecolor{textcolor}{rgb}{0.000000,0.000000,0.000000}%
\pgfsetstrokecolor{textcolor}%
\pgfsetfillcolor{textcolor}%
\pgftext[x=0.327987in, y=1.403894in, left, base]{\color{textcolor}{\rmfamily\fontsize{8.000000}{9.600000}\selectfont\catcode`\^=\active\def^{\ifmmode\sp\else\^{}\fi}\catcode`\%=\active\def%{\%}$\mathdefault{0.4}$}}%
\end{pgfscope}%
\begin{pgfscope}%
\pgfsetbuttcap%
\pgfsetroundjoin%
\definecolor{currentfill}{rgb}{0.000000,0.000000,0.000000}%
\pgfsetfillcolor{currentfill}%
\pgfsetlinewidth{0.803000pt}%
\definecolor{currentstroke}{rgb}{0.000000,0.000000,0.000000}%
\pgfsetstrokecolor{currentstroke}%
\pgfsetdash{}{0pt}%
\pgfsys@defobject{currentmarker}{\pgfqpoint{-0.048611in}{0.000000in}}{\pgfqpoint{-0.000000in}{0.000000in}}{%
\pgfpathmoveto{\pgfqpoint{-0.000000in}{0.000000in}}%
\pgfpathlineto{\pgfqpoint{-0.048611in}{0.000000in}}%
\pgfusepath{stroke,fill}%
}%
\begin{pgfscope}%
\pgfsys@transformshift{0.576061in}{1.839768in}%
\pgfsys@useobject{currentmarker}{}%
\end{pgfscope}%
\end{pgfscope}%
\begin{pgfscope}%
\definecolor{textcolor}{rgb}{0.000000,0.000000,0.000000}%
\pgfsetstrokecolor{textcolor}%
\pgfsetfillcolor{textcolor}%
\pgftext[x=0.327987in, y=1.801213in, left, base]{\color{textcolor}{\rmfamily\fontsize{8.000000}{9.600000}\selectfont\catcode`\^=\active\def^{\ifmmode\sp\else\^{}\fi}\catcode`\%=\active\def%{\%}$\mathdefault{0.6}$}}%
\end{pgfscope}%
\begin{pgfscope}%
\pgfsetbuttcap%
\pgfsetroundjoin%
\definecolor{currentfill}{rgb}{0.000000,0.000000,0.000000}%
\pgfsetfillcolor{currentfill}%
\pgfsetlinewidth{0.803000pt}%
\definecolor{currentstroke}{rgb}{0.000000,0.000000,0.000000}%
\pgfsetstrokecolor{currentstroke}%
\pgfsetdash{}{0pt}%
\pgfsys@defobject{currentmarker}{\pgfqpoint{-0.048611in}{0.000000in}}{\pgfqpoint{-0.000000in}{0.000000in}}{%
\pgfpathmoveto{\pgfqpoint{-0.000000in}{0.000000in}}%
\pgfpathlineto{\pgfqpoint{-0.048611in}{0.000000in}}%
\pgfusepath{stroke,fill}%
}%
\begin{pgfscope}%
\pgfsys@transformshift{0.576061in}{2.237087in}%
\pgfsys@useobject{currentmarker}{}%
\end{pgfscope}%
\end{pgfscope}%
\begin{pgfscope}%
\definecolor{textcolor}{rgb}{0.000000,0.000000,0.000000}%
\pgfsetstrokecolor{textcolor}%
\pgfsetfillcolor{textcolor}%
\pgftext[x=0.327987in, y=2.198531in, left, base]{\color{textcolor}{\rmfamily\fontsize{8.000000}{9.600000}\selectfont\catcode`\^=\active\def^{\ifmmode\sp\else\^{}\fi}\catcode`\%=\active\def%{\%}$\mathdefault{0.8}$}}%
\end{pgfscope}%
\begin{pgfscope}%
\pgfsetbuttcap%
\pgfsetroundjoin%
\definecolor{currentfill}{rgb}{0.000000,0.000000,0.000000}%
\pgfsetfillcolor{currentfill}%
\pgfsetlinewidth{0.803000pt}%
\definecolor{currentstroke}{rgb}{0.000000,0.000000,0.000000}%
\pgfsetstrokecolor{currentstroke}%
\pgfsetdash{}{0pt}%
\pgfsys@defobject{currentmarker}{\pgfqpoint{-0.048611in}{0.000000in}}{\pgfqpoint{-0.000000in}{0.000000in}}{%
\pgfpathmoveto{\pgfqpoint{-0.000000in}{0.000000in}}%
\pgfpathlineto{\pgfqpoint{-0.048611in}{0.000000in}}%
\pgfusepath{stroke,fill}%
}%
\begin{pgfscope}%
\pgfsys@transformshift{0.576061in}{2.634405in}%
\pgfsys@useobject{currentmarker}{}%
\end{pgfscope}%
\end{pgfscope}%
\begin{pgfscope}%
\definecolor{textcolor}{rgb}{0.000000,0.000000,0.000000}%
\pgfsetstrokecolor{textcolor}%
\pgfsetfillcolor{textcolor}%
\pgftext[x=0.327987in, y=2.595849in, left, base]{\color{textcolor}{\rmfamily\fontsize{8.000000}{9.600000}\selectfont\catcode`\^=\active\def^{\ifmmode\sp\else\^{}\fi}\catcode`\%=\active\def%{\%}$\mathdefault{1.0}$}}%
\end{pgfscope}%
\begin{pgfscope}%
\pgfsetbuttcap%
\pgfsetroundjoin%
\definecolor{currentfill}{rgb}{0.000000,0.000000,0.000000}%
\pgfsetfillcolor{currentfill}%
\pgfsetlinewidth{0.803000pt}%
\definecolor{currentstroke}{rgb}{0.000000,0.000000,0.000000}%
\pgfsetstrokecolor{currentstroke}%
\pgfsetdash{}{0pt}%
\pgfsys@defobject{currentmarker}{\pgfqpoint{-0.048611in}{0.000000in}}{\pgfqpoint{-0.000000in}{0.000000in}}{%
\pgfpathmoveto{\pgfqpoint{-0.000000in}{0.000000in}}%
\pgfpathlineto{\pgfqpoint{-0.048611in}{0.000000in}}%
\pgfusepath{stroke,fill}%
}%
\begin{pgfscope}%
\pgfsys@transformshift{0.576061in}{3.031723in}%
\pgfsys@useobject{currentmarker}{}%
\end{pgfscope}%
\end{pgfscope}%
\begin{pgfscope}%
\definecolor{textcolor}{rgb}{0.000000,0.000000,0.000000}%
\pgfsetstrokecolor{textcolor}%
\pgfsetfillcolor{textcolor}%
\pgftext[x=0.327987in, y=2.993168in, left, base]{\color{textcolor}{\rmfamily\fontsize{8.000000}{9.600000}\selectfont\catcode`\^=\active\def^{\ifmmode\sp\else\^{}\fi}\catcode`\%=\active\def%{\%}$\mathdefault{1.2}$}}%
\end{pgfscope}%
\begin{pgfscope}%
\definecolor{textcolor}{rgb}{0.000000,0.000000,0.000000}%
\pgfsetstrokecolor{textcolor}%
\pgfsetfillcolor{textcolor}%
\pgftext[x=0.272432in,y=1.884245in,,bottom,rotate=90.000000]{\color{textcolor}{\rmfamily\fontsize{10.000000}{12.000000}\selectfont\catcode`\^=\active\def^{\ifmmode\sp\else\^{}\fi}\catcode`\%=\active\def%{\%}Temperature deviation in \unit{\K}}}%
\end{pgfscope}%
\begin{pgfscope}%
\pgfpathrectangle{\pgfqpoint{0.576061in}{0.524170in}}{\pgfqpoint{4.766066in}{2.720151in}}%
\pgfusepath{clip}%
\pgfsetrectcap%
\pgfsetroundjoin%
\pgfsetlinewidth{1.003750pt}%
\definecolor{currentstroke}{rgb}{0.003922,0.450980,0.698039}%
\pgfsetstrokecolor{currentstroke}%
\pgfsetstrokeopacity{0.700000}%
\pgfsetdash{}{0pt}%
\pgfpathmoveto{\pgfqpoint{0.792700in}{0.647813in}}%
\pgfpathlineto{\pgfqpoint{1.029866in}{0.647813in}}%
\pgfpathlineto{\pgfqpoint{1.091264in}{1.128686in}}%
\pgfpathlineto{\pgfqpoint{1.155070in}{1.610834in}}%
\pgfpathlineto{\pgfqpoint{1.222487in}{2.103262in}}%
\pgfpathlineto{\pgfqpoint{1.271847in}{2.449827in}}%
\pgfpathlineto{\pgfqpoint{1.292313in}{2.575992in}}%
\pgfpathlineto{\pgfqpoint{1.311575in}{2.681901in}}%
\pgfpathlineto{\pgfqpoint{1.330837in}{2.775672in}}%
\pgfpathlineto{\pgfqpoint{1.348895in}{2.852812in}}%
\pgfpathlineto{\pgfqpoint{1.365750in}{2.915610in}}%
\pgfpathlineto{\pgfqpoint{1.381400in}{2.966127in}}%
\pgfpathlineto{\pgfqpoint{1.397051in}{3.009276in}}%
\pgfpathlineto{\pgfqpoint{1.411498in}{3.042680in}}%
\pgfpathlineto{\pgfqpoint{1.424740in}{3.067968in}}%
\pgfpathlineto{\pgfqpoint{1.436779in}{3.086595in}}%
\pgfpathlineto{\pgfqpoint{1.448818in}{3.101124in}}%
\pgfpathlineto{\pgfqpoint{1.459653in}{3.110738in}}%
\pgfpathlineto{\pgfqpoint{1.469284in}{3.116564in}}%
\pgfpathlineto{\pgfqpoint{1.478915in}{3.119854in}}%
\pgfpathlineto{\pgfqpoint{1.488546in}{3.120642in}}%
\pgfpathlineto{\pgfqpoint{1.498177in}{3.119030in}}%
\pgfpathlineto{\pgfqpoint{1.507808in}{3.115154in}}%
\pgfpathlineto{\pgfqpoint{1.518643in}{3.108254in}}%
\pgfpathlineto{\pgfqpoint{1.530682in}{3.097652in}}%
\pgfpathlineto{\pgfqpoint{1.543925in}{3.082709in}}%
\pgfpathlineto{\pgfqpoint{1.558372in}{3.062859in}}%
\pgfpathlineto{\pgfqpoint{1.575226in}{3.035554in}}%
\pgfpathlineto{\pgfqpoint{1.594488in}{2.999657in}}%
\pgfpathlineto{\pgfqpoint{1.616158in}{2.954382in}}%
\pgfpathlineto{\pgfqpoint{1.643848in}{2.890922in}}%
\pgfpathlineto{\pgfqpoint{1.683576in}{2.793545in}}%
\pgfpathlineto{\pgfqpoint{1.746178in}{2.640211in}}%
\pgfpathlineto{\pgfqpoint{1.776275in}{2.572503in}}%
\pgfpathlineto{\pgfqpoint{1.801557in}{2.520711in}}%
\pgfpathlineto{\pgfqpoint{1.824431in}{2.478631in}}%
\pgfpathlineto{\pgfqpoint{1.846101in}{2.443410in}}%
\pgfpathlineto{\pgfqpoint{1.865363in}{2.416143in}}%
\pgfpathlineto{\pgfqpoint{1.883421in}{2.394161in}}%
\pgfpathlineto{\pgfqpoint{1.901479in}{2.375706in}}%
\pgfpathlineto{\pgfqpoint{1.918334in}{2.361676in}}%
\pgfpathlineto{\pgfqpoint{1.933984in}{2.351388in}}%
\pgfpathlineto{\pgfqpoint{1.949635in}{2.343692in}}%
\pgfpathlineto{\pgfqpoint{1.965285in}{2.338518in}}%
\pgfpathlineto{\pgfqpoint{1.980936in}{2.335777in}}%
\pgfpathlineto{\pgfqpoint{1.996586in}{2.335361in}}%
\pgfpathlineto{\pgfqpoint{2.012237in}{2.337145in}}%
\pgfpathlineto{\pgfqpoint{2.029091in}{2.341369in}}%
\pgfpathlineto{\pgfqpoint{2.047150in}{2.348336in}}%
\pgfpathlineto{\pgfqpoint{2.066412in}{2.358283in}}%
\pgfpathlineto{\pgfqpoint{2.086878in}{2.371355in}}%
\pgfpathlineto{\pgfqpoint{2.110955in}{2.389521in}}%
\pgfpathlineto{\pgfqpoint{2.138645in}{2.413350in}}%
\pgfpathlineto{\pgfqpoint{2.174761in}{2.447631in}}%
\pgfpathlineto{\pgfqpoint{2.296354in}{2.565626in}}%
\pgfpathlineto{\pgfqpoint{2.328859in}{2.592627in}}%
\pgfpathlineto{\pgfqpoint{2.357752in}{2.613761in}}%
\pgfpathlineto{\pgfqpoint{2.384238in}{2.630474in}}%
\pgfpathlineto{\pgfqpoint{2.409519in}{2.643912in}}%
\pgfpathlineto{\pgfqpoint{2.434801in}{2.654832in}}%
\pgfpathlineto{\pgfqpoint{2.460083in}{2.663243in}}%
\pgfpathlineto{\pgfqpoint{2.485364in}{2.669213in}}%
\pgfpathlineto{\pgfqpoint{2.510646in}{2.672861in}}%
\pgfpathlineto{\pgfqpoint{2.537131in}{2.674369in}}%
\pgfpathlineto{\pgfqpoint{2.564821in}{2.673663in}}%
\pgfpathlineto{\pgfqpoint{2.594918in}{2.670594in}}%
\pgfpathlineto{\pgfqpoint{2.628627in}{2.664817in}}%
\pgfpathlineto{\pgfqpoint{2.668355in}{2.655646in}}%
\pgfpathlineto{\pgfqpoint{2.724938in}{2.640051in}}%
\pgfpathlineto{\pgfqpoint{2.823656in}{2.612716in}}%
\pgfpathlineto{\pgfqpoint{2.871812in}{2.601925in}}%
\pgfpathlineto{\pgfqpoint{2.915152in}{2.594509in}}%
\pgfpathlineto{\pgfqpoint{2.956084in}{2.589739in}}%
\pgfpathlineto{\pgfqpoint{2.997016in}{2.587176in}}%
\pgfpathlineto{\pgfqpoint{3.040356in}{2.586742in}}%
\pgfpathlineto{\pgfqpoint{3.086103in}{2.588529in}}%
\pgfpathlineto{\pgfqpoint{3.139074in}{2.592903in}}%
\pgfpathlineto{\pgfqpoint{3.207696in}{2.600979in}}%
\pgfpathlineto{\pgfqpoint{3.395502in}{2.624287in}}%
\pgfpathlineto{\pgfqpoint{3.464124in}{2.629834in}}%
\pgfpathlineto{\pgfqpoint{3.531541in}{2.633028in}}%
\pgfpathlineto{\pgfqpoint{3.603775in}{2.634160in}}%
\pgfpathlineto{\pgfqpoint{3.690454in}{2.633166in}}%
\pgfpathlineto{\pgfqpoint{3.863814in}{2.628240in}}%
\pgfpathlineto{\pgfqpoint{3.995038in}{2.625842in}}%
\pgfpathlineto{\pgfqpoint{4.116630in}{2.625933in}}%
\pgfpathlineto{\pgfqpoint{4.282766in}{2.628553in}}%
\pgfpathlineto{\pgfqpoint{4.544010in}{2.632512in}}%
\pgfpathlineto{\pgfqpoint{4.749874in}{2.633025in}}%
\pgfpathlineto{\pgfqpoint{5.125487in}{2.632587in}}%
\pgfpathlineto{\pgfqpoint{5.125487in}{2.632587in}}%
\pgfusepath{stroke}%
\end{pgfscope}%
\begin{pgfscope}%
\pgfpathrectangle{\pgfqpoint{0.576061in}{0.524170in}}{\pgfqpoint{4.766066in}{2.720151in}}%
\pgfusepath{clip}%
\pgfsetrectcap%
\pgfsetroundjoin%
\pgfsetlinewidth{1.003750pt}%
\definecolor{currentstroke}{rgb}{0.870588,0.560784,0.019608}%
\pgfsetstrokecolor{currentstroke}%
\pgfsetstrokeopacity{0.700000}%
\pgfsetdash{}{0pt}%
\pgfpathmoveto{\pgfqpoint{0.792700in}{0.647813in}}%
\pgfpathlineto{\pgfqpoint{1.029866in}{0.647813in}}%
\pgfpathlineto{\pgfqpoint{1.287497in}{1.785341in}}%
\pgfpathlineto{\pgfqpoint{1.318798in}{1.909427in}}%
\pgfpathlineto{\pgfqpoint{1.348895in}{2.019595in}}%
\pgfpathlineto{\pgfqpoint{1.377789in}{2.116920in}}%
\pgfpathlineto{\pgfqpoint{1.405478in}{2.202435in}}%
\pgfpathlineto{\pgfqpoint{1.431964in}{2.277129in}}%
\pgfpathlineto{\pgfqpoint{1.457245in}{2.341950in}}%
\pgfpathlineto{\pgfqpoint{1.481323in}{2.397801in}}%
\pgfpathlineto{\pgfqpoint{1.505401in}{2.447965in}}%
\pgfpathlineto{\pgfqpoint{1.528275in}{2.490584in}}%
\pgfpathlineto{\pgfqpoint{1.551148in}{2.528566in}}%
\pgfpathlineto{\pgfqpoint{1.574022in}{2.562176in}}%
\pgfpathlineto{\pgfqpoint{1.595692in}{2.590222in}}%
\pgfpathlineto{\pgfqpoint{1.617362in}{2.614807in}}%
\pgfpathlineto{\pgfqpoint{1.639032in}{2.636153in}}%
\pgfpathlineto{\pgfqpoint{1.660702in}{2.654487in}}%
\pgfpathlineto{\pgfqpoint{1.682372in}{2.670033in}}%
\pgfpathlineto{\pgfqpoint{1.705246in}{2.683667in}}%
\pgfpathlineto{\pgfqpoint{1.728120in}{2.694709in}}%
\pgfpathlineto{\pgfqpoint{1.752197in}{2.703816in}}%
\pgfpathlineto{\pgfqpoint{1.777479in}{2.710907in}}%
\pgfpathlineto{\pgfqpoint{1.803964in}{2.715956in}}%
\pgfpathlineto{\pgfqpoint{1.831654in}{2.718984in}}%
\pgfpathlineto{\pgfqpoint{1.861751in}{2.720074in}}%
\pgfpathlineto{\pgfqpoint{1.895460in}{2.719067in}}%
\pgfpathlineto{\pgfqpoint{1.933984in}{2.715679in}}%
\pgfpathlineto{\pgfqpoint{1.980936in}{2.709252in}}%
\pgfpathlineto{\pgfqpoint{2.045946in}{2.697926in}}%
\pgfpathlineto{\pgfqpoint{2.237364in}{2.663351in}}%
\pgfpathlineto{\pgfqpoint{2.312005in}{2.652877in}}%
\pgfpathlineto{\pgfqpoint{2.384238in}{2.644977in}}%
\pgfpathlineto{\pgfqpoint{2.460083in}{2.638937in}}%
\pgfpathlineto{\pgfqpoint{2.543151in}{2.634609in}}%
\pgfpathlineto{\pgfqpoint{2.639462in}{2.631912in}}%
\pgfpathlineto{\pgfqpoint{2.761054in}{2.630877in}}%
\pgfpathlineto{\pgfqpoint{2.954880in}{2.631773in}}%
\pgfpathlineto{\pgfqpoint{3.436434in}{2.634374in}}%
\pgfpathlineto{\pgfqpoint{4.085329in}{2.634449in}}%
\pgfpathlineto{\pgfqpoint{5.125487in}{2.634404in}}%
\pgfpathlineto{\pgfqpoint{5.125487in}{2.634404in}}%
\pgfusepath{stroke}%
\end{pgfscope}%
\begin{pgfscope}%
\pgfpathrectangle{\pgfqpoint{0.576061in}{0.524170in}}{\pgfqpoint{4.766066in}{2.720151in}}%
\pgfusepath{clip}%
\pgfsetrectcap%
\pgfsetroundjoin%
\pgfsetlinewidth{1.003750pt}%
\definecolor{currentstroke}{rgb}{0.800000,0.470588,0.737255}%
\pgfsetstrokecolor{currentstroke}%
\pgfsetstrokeopacity{0.700000}%
\pgfsetdash{}{0pt}%
\pgfpathmoveto{\pgfqpoint{0.792700in}{0.647813in}}%
\pgfpathlineto{\pgfqpoint{1.029866in}{0.647813in}}%
\pgfpathlineto{\pgfqpoint{1.153866in}{0.873323in}}%
\pgfpathlineto{\pgfqpoint{1.341672in}{1.215797in}}%
\pgfpathlineto{\pgfqpoint{1.395847in}{1.307368in}}%
\pgfpathlineto{\pgfqpoint{1.447614in}{1.390613in}}%
\pgfpathlineto{\pgfqpoint{1.498177in}{1.467810in}}%
\pgfpathlineto{\pgfqpoint{1.548741in}{1.540932in}}%
\pgfpathlineto{\pgfqpoint{1.599304in}{1.610089in}}%
\pgfpathlineto{\pgfqpoint{1.648663in}{1.673902in}}%
\pgfpathlineto{\pgfqpoint{1.699226in}{1.735623in}}%
\pgfpathlineto{\pgfqpoint{1.749790in}{1.793802in}}%
\pgfpathlineto{\pgfqpoint{1.800353in}{1.848598in}}%
\pgfpathlineto{\pgfqpoint{1.850916in}{1.900166in}}%
\pgfpathlineto{\pgfqpoint{1.902683in}{1.949782in}}%
\pgfpathlineto{\pgfqpoint{1.954450in}{1.996342in}}%
\pgfpathlineto{\pgfqpoint{2.007421in}{2.040986in}}%
\pgfpathlineto{\pgfqpoint{2.060392in}{2.082760in}}%
\pgfpathlineto{\pgfqpoint{2.114567in}{2.122683in}}%
\pgfpathlineto{\pgfqpoint{2.169946in}{2.160734in}}%
\pgfpathlineto{\pgfqpoint{2.226529in}{2.196903in}}%
\pgfpathlineto{\pgfqpoint{2.284315in}{2.231193in}}%
\pgfpathlineto{\pgfqpoint{2.343306in}{2.263613in}}%
\pgfpathlineto{\pgfqpoint{2.403500in}{2.294186in}}%
\pgfpathlineto{\pgfqpoint{2.466102in}{2.323478in}}%
\pgfpathlineto{\pgfqpoint{2.529908in}{2.350897in}}%
\pgfpathlineto{\pgfqpoint{2.596122in}{2.376947in}}%
\pgfpathlineto{\pgfqpoint{2.664743in}{2.401565in}}%
\pgfpathlineto{\pgfqpoint{2.735772in}{2.424706in}}%
\pgfpathlineto{\pgfqpoint{2.810413in}{2.446679in}}%
\pgfpathlineto{\pgfqpoint{2.888666in}{2.467374in}}%
\pgfpathlineto{\pgfqpoint{2.970530in}{2.486706in}}%
\pgfpathlineto{\pgfqpoint{3.057210in}{2.504858in}}%
\pgfpathlineto{\pgfqpoint{3.148705in}{2.521713in}}%
\pgfpathlineto{\pgfqpoint{3.246220in}{2.537377in}}%
\pgfpathlineto{\pgfqpoint{3.349755in}{2.551732in}}%
\pgfpathlineto{\pgfqpoint{3.461716in}{2.564972in}}%
\pgfpathlineto{\pgfqpoint{3.582105in}{2.576945in}}%
\pgfpathlineto{\pgfqpoint{3.713328in}{2.587744in}}%
\pgfpathlineto{\pgfqpoint{3.858998in}{2.597458in}}%
\pgfpathlineto{\pgfqpoint{4.021523in}{2.606009in}}%
\pgfpathlineto{\pgfqpoint{4.205718in}{2.613411in}}%
\pgfpathlineto{\pgfqpoint{4.417602in}{2.619645in}}%
\pgfpathlineto{\pgfqpoint{4.669214in}{2.624759in}}%
\pgfpathlineto{\pgfqpoint{4.979817in}{2.628762in}}%
\pgfpathlineto{\pgfqpoint{5.125487in}{2.630036in}}%
\pgfpathlineto{\pgfqpoint{5.125487in}{2.630036in}}%
\pgfusepath{stroke}%
\end{pgfscope}%
\begin{pgfscope}%
\pgfpathrectangle{\pgfqpoint{0.576061in}{0.524170in}}{\pgfqpoint{4.766066in}{2.720151in}}%
\pgfusepath{clip}%
\pgfsetbuttcap%
\pgfsetroundjoin%
\pgfsetlinewidth{1.003750pt}%
\definecolor{currentstroke}{rgb}{0.007843,0.619608,0.450980}%
\pgfsetstrokecolor{currentstroke}%
\pgfsetstrokeopacity{0.700000}%
\pgfsetdash{{3.700000pt}{1.600000pt}}{0.000000pt}%
\pgfpathmoveto{\pgfqpoint{0.792700in}{0.647813in}}%
\pgfpathlineto{\pgfqpoint{0.803535in}{0.647813in}}%
\pgfpathlineto{\pgfqpoint{0.804739in}{2.634405in}}%
\pgfpathlineto{\pgfqpoint{5.125487in}{2.634405in}}%
\pgfpathlineto{\pgfqpoint{5.125487in}{2.634405in}}%
\pgfusepath{stroke}%
\end{pgfscope}%
\begin{pgfscope}%
\pgfsetrectcap%
\pgfsetmiterjoin%
\pgfsetlinewidth{0.803000pt}%
\definecolor{currentstroke}{rgb}{0.000000,0.000000,0.000000}%
\pgfsetstrokecolor{currentstroke}%
\pgfsetdash{}{0pt}%
\pgfpathmoveto{\pgfqpoint{0.576061in}{0.524170in}}%
\pgfpathlineto{\pgfqpoint{0.576061in}{3.244321in}}%
\pgfusepath{stroke}%
\end{pgfscope}%
\begin{pgfscope}%
\pgfsetrectcap%
\pgfsetmiterjoin%
\pgfsetlinewidth{0.803000pt}%
\definecolor{currentstroke}{rgb}{0.000000,0.000000,0.000000}%
\pgfsetstrokecolor{currentstroke}%
\pgfsetdash{}{0pt}%
\pgfpathmoveto{\pgfqpoint{5.342126in}{0.524170in}}%
\pgfpathlineto{\pgfqpoint{5.342126in}{3.244321in}}%
\pgfusepath{stroke}%
\end{pgfscope}%
\begin{pgfscope}%
\pgfsetrectcap%
\pgfsetmiterjoin%
\pgfsetlinewidth{0.803000pt}%
\definecolor{currentstroke}{rgb}{0.000000,0.000000,0.000000}%
\pgfsetstrokecolor{currentstroke}%
\pgfsetdash{}{0pt}%
\pgfpathmoveto{\pgfqpoint{0.576061in}{0.524170in}}%
\pgfpathlineto{\pgfqpoint{5.342126in}{0.524170in}}%
\pgfusepath{stroke}%
\end{pgfscope}%
\begin{pgfscope}%
\pgfsetrectcap%
\pgfsetmiterjoin%
\pgfsetlinewidth{0.803000pt}%
\definecolor{currentstroke}{rgb}{0.000000,0.000000,0.000000}%
\pgfsetstrokecolor{currentstroke}%
\pgfsetdash{}{0pt}%
\pgfpathmoveto{\pgfqpoint{0.576061in}{3.244321in}}%
\pgfpathlineto{\pgfqpoint{5.342126in}{3.244321in}}%
\pgfusepath{stroke}%
\end{pgfscope}%
\begin{pgfscope}%
\pgfsetbuttcap%
\pgfsetmiterjoin%
\definecolor{currentfill}{rgb}{1.000000,1.000000,1.000000}%
\pgfsetfillcolor{currentfill}%
\pgfsetfillopacity{0.800000}%
\pgfsetlinewidth{1.003750pt}%
\definecolor{currentstroke}{rgb}{0.800000,0.800000,0.800000}%
\pgfsetstrokecolor{currentstroke}%
\pgfsetstrokeopacity{0.800000}%
\pgfsetdash{}{0pt}%
\pgfpathmoveto{\pgfqpoint{4.147682in}{0.579725in}}%
\pgfpathlineto{\pgfqpoint{5.264348in}{0.579725in}}%
\pgfpathquadraticcurveto{\pgfqpoint{5.286571in}{0.579725in}}{\pgfqpoint{5.286571in}{0.601948in}}%
\pgfpathlineto{\pgfqpoint{5.286571in}{1.212058in}}%
\pgfpathquadraticcurveto{\pgfqpoint{5.286571in}{1.234280in}}{\pgfqpoint{5.264348in}{1.234280in}}%
\pgfpathlineto{\pgfqpoint{4.147682in}{1.234280in}}%
\pgfpathquadraticcurveto{\pgfqpoint{4.125459in}{1.234280in}}{\pgfqpoint{4.125459in}{1.212058in}}%
\pgfpathlineto{\pgfqpoint{4.125459in}{0.601948in}}%
\pgfpathquadraticcurveto{\pgfqpoint{4.125459in}{0.579725in}}{\pgfqpoint{4.147682in}{0.579725in}}%
\pgfpathlineto{\pgfqpoint{4.147682in}{0.579725in}}%
\pgfpathclose%
\pgfusepath{stroke,fill}%
\end{pgfscope}%
\begin{pgfscope}%
\pgfsetrectcap%
\pgfsetroundjoin%
\pgfsetlinewidth{1.003750pt}%
\definecolor{currentstroke}{rgb}{0.003922,0.450980,0.698039}%
\pgfsetstrokecolor{currentstroke}%
\pgfsetstrokeopacity{0.700000}%
\pgfsetdash{}{0pt}%
\pgfpathmoveto{\pgfqpoint{4.169904in}{1.150947in}}%
\pgfpathlineto{\pgfqpoint{4.281015in}{1.150947in}}%
\pgfpathlineto{\pgfqpoint{4.392126in}{1.150947in}}%
\pgfusepath{stroke}%
\end{pgfscope}%
\begin{pgfscope}%
\definecolor{textcolor}{rgb}{0.000000,0.000000,0.000000}%
\pgfsetstrokecolor{textcolor}%
\pgfsetfillcolor{textcolor}%
\pgftext[x=4.481015in,y=1.112058in,left,base]{\color{textcolor}{\rmfamily\fontsize{8.000000}{9.600000}\selectfont\catcode`\^=\active\def^{\ifmmode\sp\else\^{}\fi}\catcode`\%=\active\def%{\%}Ziegler-Nichols}}%
\end{pgfscope}%
\begin{pgfscope}%
\pgfsetrectcap%
\pgfsetroundjoin%
\pgfsetlinewidth{1.003750pt}%
\definecolor{currentstroke}{rgb}{0.870588,0.560784,0.019608}%
\pgfsetstrokecolor{currentstroke}%
\pgfsetstrokeopacity{0.700000}%
\pgfsetdash{}{0pt}%
\pgfpathmoveto{\pgfqpoint{4.169904in}{0.994836in}}%
\pgfpathlineto{\pgfqpoint{4.281015in}{0.994836in}}%
\pgfpathlineto{\pgfqpoint{4.392126in}{0.994836in}}%
\pgfusepath{stroke}%
\end{pgfscope}%
\begin{pgfscope}%
\definecolor{textcolor}{rgb}{0.000000,0.000000,0.000000}%
\pgfsetstrokecolor{textcolor}%
\pgfsetfillcolor{textcolor}%
\pgftext[x=4.481015in,y=0.955947in,left,base]{\color{textcolor}{\rmfamily\fontsize{8.000000}{9.600000}\selectfont\catcode`\^=\active\def^{\ifmmode\sp\else\^{}\fi}\catcode`\%=\active\def%{\%}SIMC}}%
\end{pgfscope}%
\begin{pgfscope}%
\pgfsetrectcap%
\pgfsetroundjoin%
\pgfsetlinewidth{1.003750pt}%
\definecolor{currentstroke}{rgb}{0.800000,0.470588,0.737255}%
\pgfsetstrokecolor{currentstroke}%
\pgfsetstrokeopacity{0.700000}%
\pgfsetdash{}{0pt}%
\pgfpathmoveto{\pgfqpoint{4.169904in}{0.839947in}}%
\pgfpathlineto{\pgfqpoint{4.281015in}{0.839947in}}%
\pgfpathlineto{\pgfqpoint{4.392126in}{0.839947in}}%
\pgfusepath{stroke}%
\end{pgfscope}%
\begin{pgfscope}%
\definecolor{textcolor}{rgb}{0.000000,0.000000,0.000000}%
\pgfsetstrokecolor{textcolor}%
\pgfsetfillcolor{textcolor}%
\pgftext[x=4.481015in,y=0.801058in,left,base]{\color{textcolor}{\rmfamily\fontsize{8.000000}{9.600000}\selectfont\catcode`\^=\active\def^{\ifmmode\sp\else\^{}\fi}\catcode`\%=\active\def%{\%}AMIGO}}%
\end{pgfscope}%
\begin{pgfscope}%
\pgfsetbuttcap%
\pgfsetroundjoin%
\pgfsetlinewidth{1.003750pt}%
\definecolor{currentstroke}{rgb}{0.007843,0.619608,0.450980}%
\pgfsetstrokecolor{currentstroke}%
\pgfsetstrokeopacity{0.700000}%
\pgfsetdash{{3.700000pt}{1.600000pt}}{0.000000pt}%
\pgfpathmoveto{\pgfqpoint{4.169904in}{0.684614in}}%
\pgfpathlineto{\pgfqpoint{4.281015in}{0.684614in}}%
\pgfpathlineto{\pgfqpoint{4.392126in}{0.684614in}}%
\pgfusepath{stroke}%
\end{pgfscope}%
\begin{pgfscope}%
\definecolor{textcolor}{rgb}{0.000000,0.000000,0.000000}%
\pgfsetstrokecolor{textcolor}%
\pgfsetfillcolor{textcolor}%
\pgftext[x=4.481015in,y=0.645725in,left,base]{\color{textcolor}{\rmfamily\fontsize{8.000000}{9.600000}\selectfont\catcode`\^=\active\def^{\ifmmode\sp\else\^{}\fi}\catcode`\%=\active\def%{\%}Setpoint}}%
\end{pgfscope}%
\end{pgfpicture}%
\makeatother%
\endgroup%
% data/simulations/sim_pid_controller.py
    \caption{Different PI Controllers tuned with parameter derived using the following methods: Ziegler-Nichols, SIMC and AMIGO. The system model is the FOPTD model for room 011.}
    \label{fig:pid_controller_comparison}
\end{figure}

As it can be seen in figure \ref{fig:pid_controller_comparison}, the Ziegler-Nichols tuning rule produces a very aggressive PI controller that shows quite a bit ringing, which is undesired for this application. The AMIGO rules are rather conservative, but do not produce any overshoot. The SIMC rules have proven the most useful for this application so far. This experience is in line with the results from \citeauthor{thesis_liebmann} \cite{thesis_liebmann}, who tested different PID tuning algorithms for their viability for temperature control in the labs discussed here.

To conclude, several PID tuning rules were presented and using a Python simulation tool it is possible to test a set of PID parameters before implementation. Using an example based on parameters extracted from a real environment, the different tuning rules were applied to a model for a real lab and the SIMC tuning rules were found to give the best results for this application. The reader should now be able to extract the model parameters from physical systems and have the tools to choose an optimal set of tuning parameters for the PID controller. Further reading recommendations are for a broad overview \cite{pid_controller}, and for more details \cite{advanced_pid_control}.

\clearpage
\section{Noise and Allan Deviation}%
\label{sec:allan_deviation}
The Allan variance \cite{adev} $\sigma_A^2(\tau)$ is a two-sample variance and used as a measure of stability. The Allan deviation $\sigma_A(\tau)$ is the square root of the variance. Originally, the Allan variance was used to quantify the performance of oscillators, namely the frequency stability, but it can be used to evaluate any quantity. In order to define the Allan variance, a few terms need to be defined first. A single measurement value of a time series $y(t)$ can be written as
\begin{equation}
    \bar y_k(t) = \frac{1}{\tau} \int_{t_{k}}^{t_{k}+\tau} y(t)\,dt . \label{eqn:allan_variance_measurement}
\end{equation}
This is the $k$-th measurement with a measurement time or integration time $\tau$. The latter term is frequently used for digital multimeters (DMM). $t_k$ is the start of the $k$-th sampling interval including the dead time $\theta$
\begin{equation}
    t_{k+1} = t_k + T
\end{equation}
with
\begin{equation}
    T \coloneqq \tau + \theta .
\end{equation}

\begin{figure}[hb]
    \centering
    \scalebox{1}{%
        \import{figures/}{allan_variance_definitions.tex}
    }% scalebox
    \caption{Measurement interval according to equation \ref{eqn:allan_variance_measurement}. The shaded region is the signal acquisition period.}
    \label{fig:allan_variance_definitions}
\end{figure}

Using this, the deviation over $N$ samples is defined as \cite{adev,psd_to_adev}
\begin{equation}
    \sigma_y^2(N,T,\tau) = \left\langle \frac{1}{N-1} \left(\sum _{k=0}^{N-1}\bar y_k^2(t)-\frac{1}{N}\left(\sum _{k=0}^{N-1} \bar y_k(t)\right)^2\right)\right\rangle
\end{equation}
The $\langle \; \rangle$ denotes the (infinite time) average over all measurands $y_k$ or, simply put, the expected value.

The Allan variance is a special case of this definition with zero dead-time ($\theta=0$) and only 2 samples:
\begin{align}
    \sigma_A^2(\tau) &= \sigma_A^2(N=2,T=\tau,\tau) \label{eqn:allan_coefficients}\\
    &= \left\langle \frac{\left(\bar y_{k+1} - \bar y_k \right)^2}{2} \right\rangle
\end{align}
It can be shown \cite{psd_to_adev} that \ref{eqn:adev_estimator} is indeed more useful than $\sigma_A^2(N\to\infty,T=\tau,\tau)$, because $\sigma_A^2(N=2,T=\tau,\tau)$ converges for processes that do not have a convergent $\sigma_A^2(N\to\infty,T=\tau,\tau)$.

In practice, no experiment can take an infinite number of samples, so typically the Allan variance must be estimated using a number of samples $m$:
\begin{equation}
    \sigma_A^2(\tau) \approx \frac1 m \sum_{k=1}^m \frac{\left(\bar y_{k+1} - \bar y_{k} \right)^2}{2} \label{eqn:adev_estimator}
\end{equation}
This estimation can lead to artifacts in the results as discussed later. In order to derive the Allan variance from a set of data points, the different values of $\tau$ are usually obtained by averaging over a number of samples as there is no dead time (by definition of the Allan variance).

Additionally, the Allan variance is mathematically related to the two-sided power spectral density $S_y(f)$ \cite{psd_to_adev}:
\begin{equation}
    \sigma_A^2(\tau) = 2 \int_0^\infty S_y(f) \frac{\sin^4\left( \pi f \tau \right)}{(\pi f \tau)^2}\,df \label{eqn:psd_to_adev}
\end{equation}

and therefore all processes that can be observed in the power spectral density plot can also be seen in the Allan deviation. The inverse transform however, is not always possible as shown by \citeauthor{inverse_adev} \cite{inverse_adev}.

Distinguishing different noise processes using the Allan deviation will be elaborated in the next section.

%TODO: Add Shot noise
\subsection{Identifying Noise in Allan Deviation Plots}
It was already mentioned by \citeauthor{adev} in \cite{adev} that types of noise, whose spectral density follows a power law
\begin{equation}
    S(f) = h_{\alpha} \cdot f^\alpha \label{eqn:power_law}
\end{equation}
can be easily identified in the Allan deviation plot. The constant $h_\alpha$ is called the power (intensity) coefficient. The most common types of noise encountered in experimental data and their representations can be found in table \ref{tab:adev_alpha}, which serves as a summary of this section. Since those types of noise are present in any measurement or electronic device, it warrants a further discussion to understand their root causes and ideas to minimize them. While not a type of noise, linear drift can also be easily identified in the Allan deviation plot. It is therefore included in table \ref{tab:adev_alpha} as well.
\begin{table}[ht]
    \centering
    \begin{tabular}{lcc}
        \toprule
        Amplitude noise type& Power-law coefficient $\alpha$& Allan variance $\sigma_A^2$\\
        \midrule
            White noise & $0$& $\frac 1 2 h_0 \tau^{-1}$ \cite{adev_noise_types}\\
            Flicker noise& $-1$& $2 \ln 2 \, h_{-1} \tau^0$ \cite{adev_noise_types}\\
            Random walk noise& $-2$& $\frac 3 2 \pi^2 h_{-2} \tau^{1}$ \cite{adev_noise_types}\\
            Burst noise& $0 \textrm{ and } -\!2$& $y_{rms}^2\frac{\bar \tau^2}{\tau^2} \left(4 e^{-\frac{\tau}{\bar \tau}} - e^{-\frac{2 \tau}{\bar \tau}} + 2 \frac{\tau}{\bar \tau} - 3 \right)$\\
            Drift & --& $\frac 1 2 D^2 \tau^2$ \cite{adev_drift}\\
        \bottomrule
    \end{tabular}
    \caption{Power law representations of different noise types using the Allan variance.}
    \label{tab:adev_alpha}
\end{table}

In order to arrive at a good understanding of the features seen in an Allan deviation plot, this section will provide the reader with examples of each type of noise and the corresponding time domain, power spectral density and Allan deviation plot. Since a complete overview is not available in current literature, all required mathematical descriptions and simulation tools will be discussed here. The simulations were done using Python and the source code is linked to in the discussions. The files are found in the online supplemental material found at \cite{supplemental_material}. Using these scripts, all the graphs shown can be recreated and explored further.

\subsubsection{White Noise}%
\label{sec:white_noise}
White noise is probably the most common type of noise found in measurement data. Johnson noise found in resistors, caused by the random fluctuation of the charge carriers, is one example of mostly white noise up to a bandwidth of \qty{100}{\MHz}, from where on quantum corrections are required \cite{nist_johnson_noise}. Amplifiers also tend to have a white noise spectrum at higher frequencies.

For the latter reason, white noise typically makes up for a considerable amount of noise in measurements, unless one works at very low frequencies. White noise is a series of uncorrelated random events and therefore characterised by a uniform power spectral density, which means there is the same power in a given bandwidth at all frequencies up to infinity. White noise therefore has infinite power (variance). In reality a measurement is always limited in bandwidth and hence the above property of a constant power spectral density only holds within that bandwidth. Those bandlimited samples of white noise thus have a finite variance.
Since white noise is so common, a few of its properties should be mentioned. One such property is that the variance $\sigma_{x+y}^2$ of two uncorrelated variables $x$ and $y$ adds as:
\begin{equation}
    \sigma_{x+y}^2  = \sigma_x^2 + \sigma_y^2 + \underbrace{2\,\mathrm{Cov}(x,y)}_{\text{uncorrelated}\, =\, 0}\ = \sigma_x^2 + \sigma_y^2 \label{eqn:adding_white_noise}
\end{equation}

This results in simple addition rules for variances from different sources, but it must be stressed here that this property is only valid for uncorrelated sources like white noise, although it is usually incorrectly applied to all measurements which unfortunately obscures rather than clarifies the uncertainties involved.

In order to demonstrate the effect of white noise in Allan deviation plots, it was simulated using the \textit{AllanTools} library written by \citeauthor{allantools} \cite{allantools}. The noise generator is based on the work of \citeauthor{noise_generation} \cite{noise_generation}. The full Python program code is published online \cite{supplemental_material} and found in \external{data/simulations/sim\_allan\_variance.py}. To allow better comparison, all noise densities are normalised to give an Allan deviation of $\sigma_A(\tau_0)=1$, with $\tau_0$ being the smallest time interval.
\begin{figure}[ht]
    \centering
    \begin{subfigure}{0.32\linewidth}
        \centering
        \scalebox{0.75}{%
            %% Creator: Matplotlib, PGF backend
%%
%% To include the figure in your LaTeX document, write
%%   \input{<filename>.pgf}
%%
%% Make sure the required packages are loaded in your preamble
%%   \usepackage{pgf}
%%
%% Also ensure that all the required font packages are loaded; for instance,
%% the lmodern package is sometimes necessary when using math font.
%%   \usepackage{lmodern}
%%
%% Figures using additional raster images can only be included by \input if
%% they are in the same directory as the main LaTeX file. For loading figures
%% from other directories you can use the `import` package
%%   \usepackage{import}
%%
%% and then include the figures with
%%   \import{<path to file>}{<filename>.pgf}
%%
%% Matplotlib used the following preamble
%%   \usepackage{siunitx}
%%   \usepackage{fontspec}
%%
\begingroup%
\makeatletter%
\begin{pgfpicture}%
\pgfpathrectangle{\pgfpointorigin}{\pgfqpoint{2.440945in}{1.830709in}}%
\pgfusepath{use as bounding box, clip}%
\begin{pgfscope}%
\pgfsetbuttcap%
\pgfsetmiterjoin%
\definecolor{currentfill}{rgb}{1.000000,1.000000,1.000000}%
\pgfsetfillcolor{currentfill}%
\pgfsetlinewidth{0.000000pt}%
\definecolor{currentstroke}{rgb}{1.000000,1.000000,1.000000}%
\pgfsetstrokecolor{currentstroke}%
\pgfsetdash{}{0pt}%
\pgfpathmoveto{\pgfqpoint{0.000000in}{0.000000in}}%
\pgfpathlineto{\pgfqpoint{2.440945in}{0.000000in}}%
\pgfpathlineto{\pgfqpoint{2.440945in}{1.830709in}}%
\pgfpathlineto{\pgfqpoint{0.000000in}{1.830709in}}%
\pgfpathlineto{\pgfqpoint{0.000000in}{0.000000in}}%
\pgfpathclose%
\pgfusepath{fill}%
\end{pgfscope}%
\begin{pgfscope}%
\pgfsetbuttcap%
\pgfsetmiterjoin%
\definecolor{currentfill}{rgb}{1.000000,1.000000,1.000000}%
\pgfsetfillcolor{currentfill}%
\pgfsetlinewidth{0.000000pt}%
\definecolor{currentstroke}{rgb}{0.000000,0.000000,0.000000}%
\pgfsetstrokecolor{currentstroke}%
\pgfsetstrokeopacity{0.000000}%
\pgfsetdash{}{0pt}%
\pgfpathmoveto{\pgfqpoint{0.563510in}{0.416447in}}%
\pgfpathlineto{\pgfqpoint{2.399275in}{0.416447in}}%
\pgfpathlineto{\pgfqpoint{2.399275in}{1.789039in}}%
\pgfpathlineto{\pgfqpoint{0.563510in}{1.789039in}}%
\pgfpathlineto{\pgfqpoint{0.563510in}{0.416447in}}%
\pgfpathclose%
\pgfusepath{fill}%
\end{pgfscope}%
\begin{pgfscope}%
\pgfpathrectangle{\pgfqpoint{0.563510in}{0.416447in}}{\pgfqpoint{1.835765in}{1.372591in}}%
\pgfusepath{clip}%
\pgfsetrectcap%
\pgfsetroundjoin%
\pgfsetlinewidth{0.803000pt}%
\definecolor{currentstroke}{rgb}{0.450000,0.450000,0.450000}%
\pgfsetstrokecolor{currentstroke}%
\pgfsetdash{}{0pt}%
\pgfpathmoveto{\pgfqpoint{0.646954in}{0.416447in}}%
\pgfpathlineto{\pgfqpoint{0.646954in}{1.789039in}}%
\pgfusepath{stroke}%
\end{pgfscope}%
\begin{pgfscope}%
\pgfsetbuttcap%
\pgfsetroundjoin%
\definecolor{currentfill}{rgb}{0.000000,0.000000,0.000000}%
\pgfsetfillcolor{currentfill}%
\pgfsetlinewidth{0.803000pt}%
\definecolor{currentstroke}{rgb}{0.000000,0.000000,0.000000}%
\pgfsetstrokecolor{currentstroke}%
\pgfsetdash{}{0pt}%
\pgfsys@defobject{currentmarker}{\pgfqpoint{0.000000in}{-0.048611in}}{\pgfqpoint{0.000000in}{0.000000in}}{%
\pgfpathmoveto{\pgfqpoint{0.000000in}{0.000000in}}%
\pgfpathlineto{\pgfqpoint{0.000000in}{-0.048611in}}%
\pgfusepath{stroke,fill}%
}%
\begin{pgfscope}%
\pgfsys@transformshift{0.646954in}{0.416447in}%
\pgfsys@useobject{currentmarker}{}%
\end{pgfscope}%
\end{pgfscope}%
\begin{pgfscope}%
\definecolor{textcolor}{rgb}{0.000000,0.000000,0.000000}%
\pgfsetstrokecolor{textcolor}%
\pgfsetfillcolor{textcolor}%
\pgftext[x=0.646954in,y=0.319225in,,top]{\color{textcolor}\rmfamily\fontsize{8.000000}{9.600000}\selectfont \(\displaystyle {0}\)}%
\end{pgfscope}%
\begin{pgfscope}%
\pgfpathrectangle{\pgfqpoint{0.563510in}{0.416447in}}{\pgfqpoint{1.835765in}{1.372591in}}%
\pgfusepath{clip}%
\pgfsetrectcap%
\pgfsetroundjoin%
\pgfsetlinewidth{0.803000pt}%
\definecolor{currentstroke}{rgb}{0.450000,0.450000,0.450000}%
\pgfsetstrokecolor{currentstroke}%
\pgfsetdash{}{0pt}%
\pgfpathmoveto{\pgfqpoint{1.156317in}{0.416447in}}%
\pgfpathlineto{\pgfqpoint{1.156317in}{1.789039in}}%
\pgfusepath{stroke}%
\end{pgfscope}%
\begin{pgfscope}%
\pgfsetbuttcap%
\pgfsetroundjoin%
\definecolor{currentfill}{rgb}{0.000000,0.000000,0.000000}%
\pgfsetfillcolor{currentfill}%
\pgfsetlinewidth{0.803000pt}%
\definecolor{currentstroke}{rgb}{0.000000,0.000000,0.000000}%
\pgfsetstrokecolor{currentstroke}%
\pgfsetdash{}{0pt}%
\pgfsys@defobject{currentmarker}{\pgfqpoint{0.000000in}{-0.048611in}}{\pgfqpoint{0.000000in}{0.000000in}}{%
\pgfpathmoveto{\pgfqpoint{0.000000in}{0.000000in}}%
\pgfpathlineto{\pgfqpoint{0.000000in}{-0.048611in}}%
\pgfusepath{stroke,fill}%
}%
\begin{pgfscope}%
\pgfsys@transformshift{1.156317in}{0.416447in}%
\pgfsys@useobject{currentmarker}{}%
\end{pgfscope}%
\end{pgfscope}%
\begin{pgfscope}%
\definecolor{textcolor}{rgb}{0.000000,0.000000,0.000000}%
\pgfsetstrokecolor{textcolor}%
\pgfsetfillcolor{textcolor}%
\pgftext[x=1.156317in,y=0.319225in,,top]{\color{textcolor}\rmfamily\fontsize{8.000000}{9.600000}\selectfont \(\displaystyle {5000}\)}%
\end{pgfscope}%
\begin{pgfscope}%
\pgfpathrectangle{\pgfqpoint{0.563510in}{0.416447in}}{\pgfqpoint{1.835765in}{1.372591in}}%
\pgfusepath{clip}%
\pgfsetrectcap%
\pgfsetroundjoin%
\pgfsetlinewidth{0.803000pt}%
\definecolor{currentstroke}{rgb}{0.450000,0.450000,0.450000}%
\pgfsetstrokecolor{currentstroke}%
\pgfsetdash{}{0pt}%
\pgfpathmoveto{\pgfqpoint{1.665680in}{0.416447in}}%
\pgfpathlineto{\pgfqpoint{1.665680in}{1.789039in}}%
\pgfusepath{stroke}%
\end{pgfscope}%
\begin{pgfscope}%
\pgfsetbuttcap%
\pgfsetroundjoin%
\definecolor{currentfill}{rgb}{0.000000,0.000000,0.000000}%
\pgfsetfillcolor{currentfill}%
\pgfsetlinewidth{0.803000pt}%
\definecolor{currentstroke}{rgb}{0.000000,0.000000,0.000000}%
\pgfsetstrokecolor{currentstroke}%
\pgfsetdash{}{0pt}%
\pgfsys@defobject{currentmarker}{\pgfqpoint{0.000000in}{-0.048611in}}{\pgfqpoint{0.000000in}{0.000000in}}{%
\pgfpathmoveto{\pgfqpoint{0.000000in}{0.000000in}}%
\pgfpathlineto{\pgfqpoint{0.000000in}{-0.048611in}}%
\pgfusepath{stroke,fill}%
}%
\begin{pgfscope}%
\pgfsys@transformshift{1.665680in}{0.416447in}%
\pgfsys@useobject{currentmarker}{}%
\end{pgfscope}%
\end{pgfscope}%
\begin{pgfscope}%
\definecolor{textcolor}{rgb}{0.000000,0.000000,0.000000}%
\pgfsetstrokecolor{textcolor}%
\pgfsetfillcolor{textcolor}%
\pgftext[x=1.665680in,y=0.319225in,,top]{\color{textcolor}\rmfamily\fontsize{8.000000}{9.600000}\selectfont \(\displaystyle {10000}\)}%
\end{pgfscope}%
\begin{pgfscope}%
\pgfpathrectangle{\pgfqpoint{0.563510in}{0.416447in}}{\pgfqpoint{1.835765in}{1.372591in}}%
\pgfusepath{clip}%
\pgfsetrectcap%
\pgfsetroundjoin%
\pgfsetlinewidth{0.803000pt}%
\definecolor{currentstroke}{rgb}{0.450000,0.450000,0.450000}%
\pgfsetstrokecolor{currentstroke}%
\pgfsetdash{}{0pt}%
\pgfpathmoveto{\pgfqpoint{2.175043in}{0.416447in}}%
\pgfpathlineto{\pgfqpoint{2.175043in}{1.789039in}}%
\pgfusepath{stroke}%
\end{pgfscope}%
\begin{pgfscope}%
\pgfsetbuttcap%
\pgfsetroundjoin%
\definecolor{currentfill}{rgb}{0.000000,0.000000,0.000000}%
\pgfsetfillcolor{currentfill}%
\pgfsetlinewidth{0.803000pt}%
\definecolor{currentstroke}{rgb}{0.000000,0.000000,0.000000}%
\pgfsetstrokecolor{currentstroke}%
\pgfsetdash{}{0pt}%
\pgfsys@defobject{currentmarker}{\pgfqpoint{0.000000in}{-0.048611in}}{\pgfqpoint{0.000000in}{0.000000in}}{%
\pgfpathmoveto{\pgfqpoint{0.000000in}{0.000000in}}%
\pgfpathlineto{\pgfqpoint{0.000000in}{-0.048611in}}%
\pgfusepath{stroke,fill}%
}%
\begin{pgfscope}%
\pgfsys@transformshift{2.175043in}{0.416447in}%
\pgfsys@useobject{currentmarker}{}%
\end{pgfscope}%
\end{pgfscope}%
\begin{pgfscope}%
\definecolor{textcolor}{rgb}{0.000000,0.000000,0.000000}%
\pgfsetstrokecolor{textcolor}%
\pgfsetfillcolor{textcolor}%
\pgftext[x=2.175043in,y=0.319225in,,top]{\color{textcolor}\rmfamily\fontsize{8.000000}{9.600000}\selectfont \(\displaystyle {15000}\)}%
\end{pgfscope}%
\begin{pgfscope}%
\definecolor{textcolor}{rgb}{0.000000,0.000000,0.000000}%
\pgfsetstrokecolor{textcolor}%
\pgfsetfillcolor{textcolor}%
\pgftext[x=1.481392in,y=0.165003in,,top]{\color{textcolor}\rmfamily\fontsize{10.000000}{12.000000}\selectfont Time in \unit{\second}}%
\end{pgfscope}%
\begin{pgfscope}%
\pgfpathrectangle{\pgfqpoint{0.563510in}{0.416447in}}{\pgfqpoint{1.835765in}{1.372591in}}%
\pgfusepath{clip}%
\pgfsetrectcap%
\pgfsetroundjoin%
\pgfsetlinewidth{0.803000pt}%
\definecolor{currentstroke}{rgb}{0.450000,0.450000,0.450000}%
\pgfsetstrokecolor{currentstroke}%
\pgfsetdash{}{0pt}%
\pgfpathmoveto{\pgfqpoint{0.563510in}{0.574823in}}%
\pgfpathlineto{\pgfqpoint{2.399275in}{0.574823in}}%
\pgfusepath{stroke}%
\end{pgfscope}%
\begin{pgfscope}%
\pgfsetbuttcap%
\pgfsetroundjoin%
\definecolor{currentfill}{rgb}{0.000000,0.000000,0.000000}%
\pgfsetfillcolor{currentfill}%
\pgfsetlinewidth{0.803000pt}%
\definecolor{currentstroke}{rgb}{0.000000,0.000000,0.000000}%
\pgfsetstrokecolor{currentstroke}%
\pgfsetdash{}{0pt}%
\pgfsys@defobject{currentmarker}{\pgfqpoint{-0.048611in}{0.000000in}}{\pgfqpoint{-0.000000in}{0.000000in}}{%
\pgfpathmoveto{\pgfqpoint{-0.000000in}{0.000000in}}%
\pgfpathlineto{\pgfqpoint{-0.048611in}{0.000000in}}%
\pgfusepath{stroke,fill}%
}%
\begin{pgfscope}%
\pgfsys@transformshift{0.563510in}{0.574823in}%
\pgfsys@useobject{currentmarker}{}%
\end{pgfscope}%
\end{pgfscope}%
\begin{pgfscope}%
\definecolor{textcolor}{rgb}{0.000000,0.000000,0.000000}%
\pgfsetstrokecolor{textcolor}%
\pgfsetfillcolor{textcolor}%
\pgftext[x=0.223614in, y=0.536268in, left, base]{\color{textcolor}\rmfamily\fontsize{8.000000}{9.600000}\selectfont \(\displaystyle {\ensuremath{-}5.0}\)}%
\end{pgfscope}%
\begin{pgfscope}%
\pgfpathrectangle{\pgfqpoint{0.563510in}{0.416447in}}{\pgfqpoint{1.835765in}{1.372591in}}%
\pgfusepath{clip}%
\pgfsetrectcap%
\pgfsetroundjoin%
\pgfsetlinewidth{0.803000pt}%
\definecolor{currentstroke}{rgb}{0.450000,0.450000,0.450000}%
\pgfsetstrokecolor{currentstroke}%
\pgfsetdash{}{0pt}%
\pgfpathmoveto{\pgfqpoint{0.563510in}{0.838783in}}%
\pgfpathlineto{\pgfqpoint{2.399275in}{0.838783in}}%
\pgfusepath{stroke}%
\end{pgfscope}%
\begin{pgfscope}%
\pgfsetbuttcap%
\pgfsetroundjoin%
\definecolor{currentfill}{rgb}{0.000000,0.000000,0.000000}%
\pgfsetfillcolor{currentfill}%
\pgfsetlinewidth{0.803000pt}%
\definecolor{currentstroke}{rgb}{0.000000,0.000000,0.000000}%
\pgfsetstrokecolor{currentstroke}%
\pgfsetdash{}{0pt}%
\pgfsys@defobject{currentmarker}{\pgfqpoint{-0.048611in}{0.000000in}}{\pgfqpoint{-0.000000in}{0.000000in}}{%
\pgfpathmoveto{\pgfqpoint{-0.000000in}{0.000000in}}%
\pgfpathlineto{\pgfqpoint{-0.048611in}{0.000000in}}%
\pgfusepath{stroke,fill}%
}%
\begin{pgfscope}%
\pgfsys@transformshift{0.563510in}{0.838783in}%
\pgfsys@useobject{currentmarker}{}%
\end{pgfscope}%
\end{pgfscope}%
\begin{pgfscope}%
\definecolor{textcolor}{rgb}{0.000000,0.000000,0.000000}%
\pgfsetstrokecolor{textcolor}%
\pgfsetfillcolor{textcolor}%
\pgftext[x=0.223614in, y=0.800228in, left, base]{\color{textcolor}\rmfamily\fontsize{8.000000}{9.600000}\selectfont \(\displaystyle {\ensuremath{-}2.5}\)}%
\end{pgfscope}%
\begin{pgfscope}%
\pgfpathrectangle{\pgfqpoint{0.563510in}{0.416447in}}{\pgfqpoint{1.835765in}{1.372591in}}%
\pgfusepath{clip}%
\pgfsetrectcap%
\pgfsetroundjoin%
\pgfsetlinewidth{0.803000pt}%
\definecolor{currentstroke}{rgb}{0.450000,0.450000,0.450000}%
\pgfsetstrokecolor{currentstroke}%
\pgfsetdash{}{0pt}%
\pgfpathmoveto{\pgfqpoint{0.563510in}{1.102743in}}%
\pgfpathlineto{\pgfqpoint{2.399275in}{1.102743in}}%
\pgfusepath{stroke}%
\end{pgfscope}%
\begin{pgfscope}%
\pgfsetbuttcap%
\pgfsetroundjoin%
\definecolor{currentfill}{rgb}{0.000000,0.000000,0.000000}%
\pgfsetfillcolor{currentfill}%
\pgfsetlinewidth{0.803000pt}%
\definecolor{currentstroke}{rgb}{0.000000,0.000000,0.000000}%
\pgfsetstrokecolor{currentstroke}%
\pgfsetdash{}{0pt}%
\pgfsys@defobject{currentmarker}{\pgfqpoint{-0.048611in}{0.000000in}}{\pgfqpoint{-0.000000in}{0.000000in}}{%
\pgfpathmoveto{\pgfqpoint{-0.000000in}{0.000000in}}%
\pgfpathlineto{\pgfqpoint{-0.048611in}{0.000000in}}%
\pgfusepath{stroke,fill}%
}%
\begin{pgfscope}%
\pgfsys@transformshift{0.563510in}{1.102743in}%
\pgfsys@useobject{currentmarker}{}%
\end{pgfscope}%
\end{pgfscope}%
\begin{pgfscope}%
\definecolor{textcolor}{rgb}{0.000000,0.000000,0.000000}%
\pgfsetstrokecolor{textcolor}%
\pgfsetfillcolor{textcolor}%
\pgftext[x=0.315437in, y=1.064188in, left, base]{\color{textcolor}\rmfamily\fontsize{8.000000}{9.600000}\selectfont \(\displaystyle {0.0}\)}%
\end{pgfscope}%
\begin{pgfscope}%
\pgfpathrectangle{\pgfqpoint{0.563510in}{0.416447in}}{\pgfqpoint{1.835765in}{1.372591in}}%
\pgfusepath{clip}%
\pgfsetrectcap%
\pgfsetroundjoin%
\pgfsetlinewidth{0.803000pt}%
\definecolor{currentstroke}{rgb}{0.450000,0.450000,0.450000}%
\pgfsetstrokecolor{currentstroke}%
\pgfsetdash{}{0pt}%
\pgfpathmoveto{\pgfqpoint{0.563510in}{1.366703in}}%
\pgfpathlineto{\pgfqpoint{2.399275in}{1.366703in}}%
\pgfusepath{stroke}%
\end{pgfscope}%
\begin{pgfscope}%
\pgfsetbuttcap%
\pgfsetroundjoin%
\definecolor{currentfill}{rgb}{0.000000,0.000000,0.000000}%
\pgfsetfillcolor{currentfill}%
\pgfsetlinewidth{0.803000pt}%
\definecolor{currentstroke}{rgb}{0.000000,0.000000,0.000000}%
\pgfsetstrokecolor{currentstroke}%
\pgfsetdash{}{0pt}%
\pgfsys@defobject{currentmarker}{\pgfqpoint{-0.048611in}{0.000000in}}{\pgfqpoint{-0.000000in}{0.000000in}}{%
\pgfpathmoveto{\pgfqpoint{-0.000000in}{0.000000in}}%
\pgfpathlineto{\pgfqpoint{-0.048611in}{0.000000in}}%
\pgfusepath{stroke,fill}%
}%
\begin{pgfscope}%
\pgfsys@transformshift{0.563510in}{1.366703in}%
\pgfsys@useobject{currentmarker}{}%
\end{pgfscope}%
\end{pgfscope}%
\begin{pgfscope}%
\definecolor{textcolor}{rgb}{0.000000,0.000000,0.000000}%
\pgfsetstrokecolor{textcolor}%
\pgfsetfillcolor{textcolor}%
\pgftext[x=0.315437in, y=1.328147in, left, base]{\color{textcolor}\rmfamily\fontsize{8.000000}{9.600000}\selectfont \(\displaystyle {2.5}\)}%
\end{pgfscope}%
\begin{pgfscope}%
\pgfpathrectangle{\pgfqpoint{0.563510in}{0.416447in}}{\pgfqpoint{1.835765in}{1.372591in}}%
\pgfusepath{clip}%
\pgfsetrectcap%
\pgfsetroundjoin%
\pgfsetlinewidth{0.803000pt}%
\definecolor{currentstroke}{rgb}{0.450000,0.450000,0.450000}%
\pgfsetstrokecolor{currentstroke}%
\pgfsetdash{}{0pt}%
\pgfpathmoveto{\pgfqpoint{0.563510in}{1.630663in}}%
\pgfpathlineto{\pgfqpoint{2.399275in}{1.630663in}}%
\pgfusepath{stroke}%
\end{pgfscope}%
\begin{pgfscope}%
\pgfsetbuttcap%
\pgfsetroundjoin%
\definecolor{currentfill}{rgb}{0.000000,0.000000,0.000000}%
\pgfsetfillcolor{currentfill}%
\pgfsetlinewidth{0.803000pt}%
\definecolor{currentstroke}{rgb}{0.000000,0.000000,0.000000}%
\pgfsetstrokecolor{currentstroke}%
\pgfsetdash{}{0pt}%
\pgfsys@defobject{currentmarker}{\pgfqpoint{-0.048611in}{0.000000in}}{\pgfqpoint{-0.000000in}{0.000000in}}{%
\pgfpathmoveto{\pgfqpoint{-0.000000in}{0.000000in}}%
\pgfpathlineto{\pgfqpoint{-0.048611in}{0.000000in}}%
\pgfusepath{stroke,fill}%
}%
\begin{pgfscope}%
\pgfsys@transformshift{0.563510in}{1.630663in}%
\pgfsys@useobject{currentmarker}{}%
\end{pgfscope}%
\end{pgfscope}%
\begin{pgfscope}%
\definecolor{textcolor}{rgb}{0.000000,0.000000,0.000000}%
\pgfsetstrokecolor{textcolor}%
\pgfsetfillcolor{textcolor}%
\pgftext[x=0.315437in, y=1.592107in, left, base]{\color{textcolor}\rmfamily\fontsize{8.000000}{9.600000}\selectfont \(\displaystyle {5.0}\)}%
\end{pgfscope}%
\begin{pgfscope}%
\definecolor{textcolor}{rgb}{0.000000,0.000000,0.000000}%
\pgfsetstrokecolor{textcolor}%
\pgfsetfillcolor{textcolor}%
\pgftext[x=0.168059in,y=1.102743in,,bottom,rotate=90.000000]{\color{textcolor}\rmfamily\fontsize{10.000000}{12.000000}\selectfont Ampl. in arb. unit}%
\end{pgfscope}%
\begin{pgfscope}%
\pgfpathrectangle{\pgfqpoint{0.563510in}{0.416447in}}{\pgfqpoint{1.835765in}{1.372591in}}%
\pgfusepath{clip}%
\pgfsetrectcap%
\pgfsetroundjoin%
\pgfsetlinewidth{1.505625pt}%
\definecolor{currentstroke}{rgb}{0.000000,0.447059,0.698039}%
\pgfsetstrokecolor{currentstroke}%
\pgfsetdash{}{0pt}%
\pgfpathmoveto{\pgfqpoint{0.646954in}{1.088145in}}%
\pgfpathlineto{\pgfqpoint{0.647463in}{1.269483in}}%
\pgfpathlineto{\pgfqpoint{0.648787in}{0.953627in}}%
\pgfpathlineto{\pgfqpoint{0.651843in}{0.916594in}}%
\pgfpathlineto{\pgfqpoint{0.652353in}{1.211600in}}%
\pgfpathlineto{\pgfqpoint{0.654390in}{0.826140in}}%
\pgfpathlineto{\pgfqpoint{0.656122in}{1.205016in}}%
\pgfpathlineto{\pgfqpoint{0.658058in}{0.900152in}}%
\pgfpathlineto{\pgfqpoint{0.659586in}{1.334020in}}%
\pgfpathlineto{\pgfqpoint{0.661318in}{0.933019in}}%
\pgfpathlineto{\pgfqpoint{0.662744in}{1.299739in}}%
\pgfpathlineto{\pgfqpoint{0.665087in}{1.389949in}}%
\pgfpathlineto{\pgfqpoint{0.666106in}{0.942800in}}%
\pgfpathlineto{\pgfqpoint{0.668143in}{1.509530in}}%
\pgfpathlineto{\pgfqpoint{0.669569in}{0.932466in}}%
\pgfpathlineto{\pgfqpoint{0.672524in}{1.326809in}}%
\pgfpathlineto{\pgfqpoint{0.673542in}{0.760517in}}%
\pgfpathlineto{\pgfqpoint{0.675784in}{1.327957in}}%
\pgfpathlineto{\pgfqpoint{0.676599in}{1.040516in}}%
\pgfpathlineto{\pgfqpoint{0.679757in}{1.323666in}}%
\pgfpathlineto{\pgfqpoint{0.679960in}{0.974544in}}%
\pgfpathlineto{\pgfqpoint{0.682609in}{1.258517in}}%
\pgfpathlineto{\pgfqpoint{0.683424in}{1.015401in}}%
\pgfpathlineto{\pgfqpoint{0.685360in}{1.333951in}}%
\pgfpathlineto{\pgfqpoint{0.687193in}{0.921864in}}%
\pgfpathlineto{\pgfqpoint{0.689638in}{1.320325in}}%
\pgfpathlineto{\pgfqpoint{0.690759in}{0.887433in}}%
\pgfpathlineto{\pgfqpoint{0.693713in}{1.310187in}}%
\pgfpathlineto{\pgfqpoint{0.694834in}{0.859697in}}%
\pgfpathlineto{\pgfqpoint{0.695547in}{1.427823in}}%
\pgfpathlineto{\pgfqpoint{0.697686in}{0.956742in}}%
\pgfpathlineto{\pgfqpoint{0.699316in}{1.182500in}}%
\pgfpathlineto{\pgfqpoint{0.702270in}{0.841777in}}%
\pgfpathlineto{\pgfqpoint{0.704002in}{1.321872in}}%
\pgfpathlineto{\pgfqpoint{0.704512in}{0.887550in}}%
\pgfpathlineto{\pgfqpoint{0.706243in}{1.342492in}}%
\pgfpathlineto{\pgfqpoint{0.708790in}{0.914939in}}%
\pgfpathlineto{\pgfqpoint{0.709402in}{1.360975in}}%
\pgfpathlineto{\pgfqpoint{0.712661in}{0.817995in}}%
\pgfpathlineto{\pgfqpoint{0.713476in}{1.374449in}}%
\pgfpathlineto{\pgfqpoint{0.714903in}{0.822843in}}%
\pgfpathlineto{\pgfqpoint{0.717246in}{1.225150in}}%
\pgfpathlineto{\pgfqpoint{0.718978in}{0.882925in}}%
\pgfpathlineto{\pgfqpoint{0.720506in}{1.181220in}}%
\pgfpathlineto{\pgfqpoint{0.722237in}{0.908376in}}%
\pgfpathlineto{\pgfqpoint{0.723766in}{1.380680in}}%
\pgfpathlineto{\pgfqpoint{0.725803in}{0.936953in}}%
\pgfpathlineto{\pgfqpoint{0.726822in}{1.216856in}}%
\pgfpathlineto{\pgfqpoint{0.729878in}{0.945892in}}%
\pgfpathlineto{\pgfqpoint{0.730591in}{1.345470in}}%
\pgfpathlineto{\pgfqpoint{0.732119in}{0.997333in}}%
\pgfpathlineto{\pgfqpoint{0.734768in}{1.292359in}}%
\pgfpathlineto{\pgfqpoint{0.735685in}{0.957109in}}%
\pgfpathlineto{\pgfqpoint{0.737926in}{1.248677in}}%
\pgfpathlineto{\pgfqpoint{0.739148in}{0.959164in}}%
\pgfpathlineto{\pgfqpoint{0.741695in}{1.301363in}}%
\pgfpathlineto{\pgfqpoint{0.742612in}{0.967564in}}%
\pgfpathlineto{\pgfqpoint{0.745363in}{1.292893in}}%
\pgfpathlineto{\pgfqpoint{0.746381in}{0.846820in}}%
\pgfpathlineto{\pgfqpoint{0.748317in}{1.292550in}}%
\pgfpathlineto{\pgfqpoint{0.749641in}{1.046231in}}%
\pgfpathlineto{\pgfqpoint{0.751067in}{1.328932in}}%
\pgfpathlineto{\pgfqpoint{0.753512in}{0.936301in}}%
\pgfpathlineto{\pgfqpoint{0.754531in}{1.226564in}}%
\pgfpathlineto{\pgfqpoint{0.756467in}{0.938393in}}%
\pgfpathlineto{\pgfqpoint{0.759013in}{0.796945in}}%
\pgfpathlineto{\pgfqpoint{0.759828in}{1.304834in}}%
\pgfpathlineto{\pgfqpoint{0.761458in}{0.981707in}}%
\pgfpathlineto{\pgfqpoint{0.764718in}{1.340564in}}%
\pgfpathlineto{\pgfqpoint{0.765024in}{0.838846in}}%
\pgfpathlineto{\pgfqpoint{0.767163in}{1.249716in}}%
\pgfpathlineto{\pgfqpoint{0.768488in}{0.934549in}}%
\pgfpathlineto{\pgfqpoint{0.770016in}{1.287470in}}%
\pgfpathlineto{\pgfqpoint{0.772257in}{0.958114in}}%
\pgfpathlineto{\pgfqpoint{0.774804in}{1.279340in}}%
\pgfpathlineto{\pgfqpoint{0.775721in}{0.953478in}}%
\pgfpathlineto{\pgfqpoint{0.778369in}{1.365964in}}%
\pgfpathlineto{\pgfqpoint{0.779286in}{1.032203in}}%
\pgfpathlineto{\pgfqpoint{0.781833in}{0.894166in}}%
\pgfpathlineto{\pgfqpoint{0.782342in}{1.264366in}}%
\pgfpathlineto{\pgfqpoint{0.783972in}{0.833090in}}%
\pgfpathlineto{\pgfqpoint{0.785806in}{1.229082in}}%
\pgfpathlineto{\pgfqpoint{0.788658in}{1.315324in}}%
\pgfpathlineto{\pgfqpoint{0.789066in}{0.960508in}}%
\pgfpathlineto{\pgfqpoint{0.790798in}{1.331333in}}%
\pgfpathlineto{\pgfqpoint{0.793854in}{0.965818in}}%
\pgfpathlineto{\pgfqpoint{0.794873in}{1.377439in}}%
\pgfpathlineto{\pgfqpoint{0.796401in}{0.976477in}}%
\pgfpathlineto{\pgfqpoint{0.798744in}{1.372848in}}%
\pgfpathlineto{\pgfqpoint{0.800985in}{0.905667in}}%
\pgfpathlineto{\pgfqpoint{0.801596in}{1.298506in}}%
\pgfpathlineto{\pgfqpoint{0.803634in}{0.799478in}}%
\pgfpathlineto{\pgfqpoint{0.804652in}{1.232288in}}%
\pgfpathlineto{\pgfqpoint{0.806588in}{0.937763in}}%
\pgfpathlineto{\pgfqpoint{0.809542in}{1.319982in}}%
\pgfpathlineto{\pgfqpoint{0.810969in}{0.937695in}}%
\pgfpathlineto{\pgfqpoint{0.812293in}{1.300533in}}%
\pgfpathlineto{\pgfqpoint{0.813515in}{1.001898in}}%
\pgfpathlineto{\pgfqpoint{0.816164in}{0.878709in}}%
\pgfpathlineto{\pgfqpoint{0.817387in}{1.273301in}}%
\pgfpathlineto{\pgfqpoint{0.819322in}{0.960629in}}%
\pgfpathlineto{\pgfqpoint{0.820748in}{1.303340in}}%
\pgfpathlineto{\pgfqpoint{0.823601in}{0.932087in}}%
\pgfpathlineto{\pgfqpoint{0.824008in}{1.283356in}}%
\pgfpathlineto{\pgfqpoint{0.825434in}{0.909458in}}%
\pgfpathlineto{\pgfqpoint{0.827268in}{1.281713in}}%
\pgfpathlineto{\pgfqpoint{0.829204in}{0.904755in}}%
\pgfpathlineto{\pgfqpoint{0.831445in}{1.272968in}}%
\pgfpathlineto{\pgfqpoint{0.832871in}{0.987614in}}%
\pgfpathlineto{\pgfqpoint{0.835418in}{1.290187in}}%
\pgfpathlineto{\pgfqpoint{0.836640in}{0.918126in}}%
\pgfpathlineto{\pgfqpoint{0.837863in}{1.250105in}}%
\pgfpathlineto{\pgfqpoint{0.840104in}{1.271535in}}%
\pgfpathlineto{\pgfqpoint{0.841021in}{0.936884in}}%
\pgfpathlineto{\pgfqpoint{0.843873in}{1.381943in}}%
\pgfpathlineto{\pgfqpoint{0.845503in}{0.915539in}}%
\pgfpathlineto{\pgfqpoint{0.846216in}{1.434039in}}%
\pgfpathlineto{\pgfqpoint{0.848661in}{0.957078in}}%
\pgfpathlineto{\pgfqpoint{0.849782in}{1.228273in}}%
\pgfpathlineto{\pgfqpoint{0.853042in}{0.783931in}}%
\pgfpathlineto{\pgfqpoint{0.853246in}{1.292848in}}%
\pgfpathlineto{\pgfqpoint{0.856098in}{0.960641in}}%
\pgfpathlineto{\pgfqpoint{0.857524in}{1.252962in}}%
\pgfpathlineto{\pgfqpoint{0.859052in}{0.860303in}}%
\pgfpathlineto{\pgfqpoint{0.860173in}{1.212457in}}%
\pgfpathlineto{\pgfqpoint{0.862618in}{0.832202in}}%
\pgfpathlineto{\pgfqpoint{0.863535in}{1.220398in}}%
\pgfpathlineto{\pgfqpoint{0.865470in}{0.933713in}}%
\pgfpathlineto{\pgfqpoint{0.868425in}{1.323365in}}%
\pgfpathlineto{\pgfqpoint{0.869342in}{0.869206in}}%
\pgfpathlineto{\pgfqpoint{0.870972in}{1.293253in}}%
\pgfpathlineto{\pgfqpoint{0.873824in}{0.905666in}}%
\pgfpathlineto{\pgfqpoint{0.875250in}{1.311423in}}%
\pgfpathlineto{\pgfqpoint{0.876880in}{0.827780in}}%
\pgfpathlineto{\pgfqpoint{0.877593in}{1.228653in}}%
\pgfpathlineto{\pgfqpoint{0.879121in}{0.920526in}}%
\pgfpathlineto{\pgfqpoint{0.881668in}{1.431101in}}%
\pgfpathlineto{\pgfqpoint{0.883094in}{0.986584in}}%
\pgfpathlineto{\pgfqpoint{0.885641in}{0.875385in}}%
\pgfpathlineto{\pgfqpoint{0.886252in}{1.214655in}}%
\pgfpathlineto{\pgfqpoint{0.887984in}{0.922474in}}%
\pgfpathlineto{\pgfqpoint{0.890225in}{1.212049in}}%
\pgfpathlineto{\pgfqpoint{0.891957in}{0.965528in}}%
\pgfpathlineto{\pgfqpoint{0.893485in}{1.226233in}}%
\pgfpathlineto{\pgfqpoint{0.895319in}{0.933114in}}%
\pgfpathlineto{\pgfqpoint{0.896949in}{1.259772in}}%
\pgfpathlineto{\pgfqpoint{0.898783in}{0.956927in}}%
\pgfpathlineto{\pgfqpoint{0.900820in}{1.255167in}}%
\pgfpathlineto{\pgfqpoint{0.903265in}{0.868421in}}%
\pgfpathlineto{\pgfqpoint{0.903673in}{1.417939in}}%
\pgfpathlineto{\pgfqpoint{0.905404in}{0.971215in}}%
\pgfpathlineto{\pgfqpoint{0.906831in}{1.300287in}}%
\pgfpathlineto{\pgfqpoint{0.908766in}{0.949795in}}%
\pgfpathlineto{\pgfqpoint{0.910702in}{1.233397in}}%
\pgfpathlineto{\pgfqpoint{0.912434in}{0.869871in}}%
\pgfpathlineto{\pgfqpoint{0.914064in}{1.251665in}}%
\pgfpathlineto{\pgfqpoint{0.916509in}{0.954338in}}%
\pgfpathlineto{\pgfqpoint{0.917324in}{1.192967in}}%
\pgfpathlineto{\pgfqpoint{0.919361in}{0.895521in}}%
\pgfpathlineto{\pgfqpoint{0.922112in}{1.315207in}}%
\pgfpathlineto{\pgfqpoint{0.923436in}{0.881014in}}%
\pgfpathlineto{\pgfqpoint{0.924353in}{1.309961in}}%
\pgfpathlineto{\pgfqpoint{0.926390in}{0.835662in}}%
\pgfpathlineto{\pgfqpoint{0.928224in}{1.197278in}}%
\pgfpathlineto{\pgfqpoint{0.929345in}{0.939234in}}%
\pgfpathlineto{\pgfqpoint{0.932197in}{1.387084in}}%
\pgfpathlineto{\pgfqpoint{0.934031in}{0.899212in}}%
\pgfpathlineto{\pgfqpoint{0.934642in}{1.282669in}}%
\pgfpathlineto{\pgfqpoint{0.936272in}{0.942610in}}%
\pgfpathlineto{\pgfqpoint{0.938004in}{1.227303in}}%
\pgfpathlineto{\pgfqpoint{0.940652in}{0.812651in}}%
\pgfpathlineto{\pgfqpoint{0.941773in}{1.517291in}}%
\pgfpathlineto{\pgfqpoint{0.943505in}{1.025566in}}%
\pgfpathlineto{\pgfqpoint{0.945135in}{1.269367in}}%
\pgfpathlineto{\pgfqpoint{0.947070in}{0.786927in}}%
\pgfpathlineto{\pgfqpoint{0.949719in}{1.312568in}}%
\pgfpathlineto{\pgfqpoint{0.950840in}{0.928014in}}%
\pgfpathlineto{\pgfqpoint{0.952775in}{1.302053in}}%
\pgfpathlineto{\pgfqpoint{0.954201in}{0.803713in}}%
\pgfpathlineto{\pgfqpoint{0.955730in}{1.150626in}}%
\pgfpathlineto{\pgfqpoint{0.957054in}{0.886302in}}%
\pgfpathlineto{\pgfqpoint{0.959193in}{1.288036in}}%
\pgfpathlineto{\pgfqpoint{0.960823in}{0.895052in}}%
\pgfpathlineto{\pgfqpoint{0.962861in}{1.276902in}}%
\pgfpathlineto{\pgfqpoint{0.965306in}{0.901146in}}%
\pgfpathlineto{\pgfqpoint{0.966019in}{1.260585in}}%
\pgfpathlineto{\pgfqpoint{0.967954in}{0.795812in}}%
\pgfpathlineto{\pgfqpoint{0.970094in}{1.184679in}}%
\pgfpathlineto{\pgfqpoint{0.972335in}{0.904244in}}%
\pgfpathlineto{\pgfqpoint{0.973150in}{1.237771in}}%
\pgfpathlineto{\pgfqpoint{0.974474in}{0.927830in}}%
\pgfpathlineto{\pgfqpoint{0.977021in}{1.405600in}}%
\pgfpathlineto{\pgfqpoint{0.977836in}{0.906579in}}%
\pgfpathlineto{\pgfqpoint{0.980485in}{1.245856in}}%
\pgfpathlineto{\pgfqpoint{0.982115in}{0.840070in}}%
\pgfpathlineto{\pgfqpoint{0.984254in}{1.322996in}}%
\pgfpathlineto{\pgfqpoint{0.985273in}{0.939198in}}%
\pgfpathlineto{\pgfqpoint{0.987004in}{1.292410in}}%
\pgfpathlineto{\pgfqpoint{0.988838in}{0.877983in}}%
\pgfpathlineto{\pgfqpoint{0.990468in}{1.293554in}}%
\pgfpathlineto{\pgfqpoint{0.992200in}{0.936238in}}%
\pgfpathlineto{\pgfqpoint{0.994136in}{1.318859in}}%
\pgfpathlineto{\pgfqpoint{0.995664in}{0.915225in}}%
\pgfpathlineto{\pgfqpoint{0.997599in}{1.270477in}}%
\pgfpathlineto{\pgfqpoint{0.998720in}{0.793440in}}%
\pgfpathlineto{\pgfqpoint{1.001267in}{1.249119in}}%
\pgfpathlineto{\pgfqpoint{1.002285in}{0.934062in}}%
\pgfpathlineto{\pgfqpoint{1.004628in}{1.283507in}}%
\pgfpathlineto{\pgfqpoint{1.005545in}{0.839494in}}%
\pgfpathlineto{\pgfqpoint{1.008805in}{1.367923in}}%
\pgfpathlineto{\pgfqpoint{1.009722in}{0.871523in}}%
\pgfpathlineto{\pgfqpoint{1.011556in}{1.327085in}}%
\pgfpathlineto{\pgfqpoint{1.012880in}{0.946538in}}%
\pgfpathlineto{\pgfqpoint{1.014306in}{1.216983in}}%
\pgfpathlineto{\pgfqpoint{1.016446in}{0.939848in}}%
\pgfpathlineto{\pgfqpoint{1.018483in}{1.185402in}}%
\pgfpathlineto{\pgfqpoint{1.019400in}{0.882445in}}%
\pgfpathlineto{\pgfqpoint{1.022354in}{1.365139in}}%
\pgfpathlineto{\pgfqpoint{1.023373in}{1.042376in}}%
\pgfpathlineto{\pgfqpoint{1.025410in}{1.445162in}}%
\pgfpathlineto{\pgfqpoint{1.026735in}{0.858196in}}%
\pgfpathlineto{\pgfqpoint{1.029587in}{1.318354in}}%
\pgfpathlineto{\pgfqpoint{1.030097in}{1.016902in}}%
\pgfpathlineto{\pgfqpoint{1.032847in}{1.240348in}}%
\pgfpathlineto{\pgfqpoint{1.034681in}{0.894901in}}%
\pgfpathlineto{\pgfqpoint{1.035088in}{1.147109in}}%
\pgfpathlineto{\pgfqpoint{1.037330in}{0.952500in}}%
\pgfpathlineto{\pgfqpoint{1.038450in}{1.272605in}}%
\pgfpathlineto{\pgfqpoint{1.041099in}{0.955958in}}%
\pgfpathlineto{\pgfqpoint{1.043442in}{0.901289in}}%
\pgfpathlineto{\pgfqpoint{1.043748in}{1.249870in}}%
\pgfpathlineto{\pgfqpoint{1.045581in}{0.944471in}}%
\pgfpathlineto{\pgfqpoint{1.047313in}{1.264079in}}%
\pgfpathlineto{\pgfqpoint{1.049147in}{0.967698in}}%
\pgfpathlineto{\pgfqpoint{1.051082in}{1.315524in}}%
\pgfpathlineto{\pgfqpoint{1.052814in}{0.882907in}}%
\pgfpathlineto{\pgfqpoint{1.054139in}{1.276257in}}%
\pgfpathlineto{\pgfqpoint{1.056583in}{0.907218in}}%
\pgfpathlineto{\pgfqpoint{1.059130in}{1.399925in}}%
\pgfpathlineto{\pgfqpoint{1.059436in}{1.024816in}}%
\pgfpathlineto{\pgfqpoint{1.062186in}{0.890788in}}%
\pgfpathlineto{\pgfqpoint{1.062696in}{1.279984in}}%
\pgfpathlineto{\pgfqpoint{1.064530in}{0.944344in}}%
\pgfpathlineto{\pgfqpoint{1.067280in}{1.292660in}}%
\pgfpathlineto{\pgfqpoint{1.069114in}{0.824980in}}%
\pgfpathlineto{\pgfqpoint{1.069725in}{1.221819in}}%
\pgfpathlineto{\pgfqpoint{1.071661in}{0.939848in}}%
\pgfpathlineto{\pgfqpoint{1.073291in}{1.350768in}}%
\pgfpathlineto{\pgfqpoint{1.075430in}{0.801340in}}%
\pgfpathlineto{\pgfqpoint{1.077365in}{1.300060in}}%
\pgfpathlineto{\pgfqpoint{1.078792in}{0.908268in}}%
\pgfpathlineto{\pgfqpoint{1.080524in}{1.253147in}}%
\pgfpathlineto{\pgfqpoint{1.082255in}{0.889274in}}%
\pgfpathlineto{\pgfqpoint{1.083682in}{1.262083in}}%
\pgfpathlineto{\pgfqpoint{1.085413in}{0.934033in}}%
\pgfpathlineto{\pgfqpoint{1.087756in}{1.282176in}}%
\pgfpathlineto{\pgfqpoint{1.088775in}{0.884881in}}%
\pgfpathlineto{\pgfqpoint{1.091526in}{1.324252in}}%
\pgfpathlineto{\pgfqpoint{1.092646in}{0.852334in}}%
\pgfpathlineto{\pgfqpoint{1.094174in}{1.189314in}}%
\pgfpathlineto{\pgfqpoint{1.096416in}{0.838988in}}%
\pgfpathlineto{\pgfqpoint{1.097333in}{1.235294in}}%
\pgfpathlineto{\pgfqpoint{1.099574in}{0.898253in}}%
\pgfpathlineto{\pgfqpoint{1.101306in}{1.298006in}}%
\pgfpathlineto{\pgfqpoint{1.103037in}{0.955507in}}%
\pgfpathlineto{\pgfqpoint{1.104769in}{1.279847in}}%
\pgfpathlineto{\pgfqpoint{1.106297in}{0.977416in}}%
\pgfpathlineto{\pgfqpoint{1.107724in}{1.227346in}}%
\pgfpathlineto{\pgfqpoint{1.109557in}{0.910466in}}%
\pgfpathlineto{\pgfqpoint{1.111493in}{1.284246in}}%
\pgfpathlineto{\pgfqpoint{1.113021in}{0.978475in}}%
\pgfpathlineto{\pgfqpoint{1.115670in}{1.275380in}}%
\pgfpathlineto{\pgfqpoint{1.117198in}{0.803148in}}%
\pgfpathlineto{\pgfqpoint{1.118318in}{1.218470in}}%
\pgfpathlineto{\pgfqpoint{1.120356in}{0.932044in}}%
\pgfpathlineto{\pgfqpoint{1.121578in}{1.259214in}}%
\pgfpathlineto{\pgfqpoint{1.124431in}{0.950322in}}%
\pgfpathlineto{\pgfqpoint{1.125347in}{1.237106in}}%
\pgfpathlineto{\pgfqpoint{1.128200in}{0.767997in}}%
\pgfpathlineto{\pgfqpoint{1.128506in}{1.270903in}}%
\pgfpathlineto{\pgfqpoint{1.131358in}{0.944657in}}%
\pgfpathlineto{\pgfqpoint{1.132580in}{1.211433in}}%
\pgfpathlineto{\pgfqpoint{1.133701in}{0.767334in}}%
\pgfpathlineto{\pgfqpoint{1.136655in}{1.253444in}}%
\pgfpathlineto{\pgfqpoint{1.137165in}{1.017859in}}%
\pgfpathlineto{\pgfqpoint{1.140527in}{0.872759in}}%
\pgfpathlineto{\pgfqpoint{1.141036in}{1.371878in}}%
\pgfpathlineto{\pgfqpoint{1.142462in}{0.924869in}}%
\pgfpathlineto{\pgfqpoint{1.144703in}{1.272683in}}%
\pgfpathlineto{\pgfqpoint{1.146130in}{0.995284in}}%
\pgfpathlineto{\pgfqpoint{1.149186in}{0.871382in}}%
\pgfpathlineto{\pgfqpoint{1.149288in}{1.264760in}}%
\pgfpathlineto{\pgfqpoint{1.151631in}{0.956055in}}%
\pgfpathlineto{\pgfqpoint{1.153159in}{1.264154in}}%
\pgfpathlineto{\pgfqpoint{1.155909in}{1.431416in}}%
\pgfpathlineto{\pgfqpoint{1.156419in}{0.913152in}}%
\pgfpathlineto{\pgfqpoint{1.157947in}{1.318364in}}%
\pgfpathlineto{\pgfqpoint{1.160901in}{0.883711in}}%
\pgfpathlineto{\pgfqpoint{1.163040in}{1.272478in}}%
\pgfpathlineto{\pgfqpoint{1.163652in}{0.938521in}}%
\pgfpathlineto{\pgfqpoint{1.164874in}{1.348191in}}%
\pgfpathlineto{\pgfqpoint{1.167013in}{0.993995in}}%
\pgfpathlineto{\pgfqpoint{1.169866in}{0.935686in}}%
\pgfpathlineto{\pgfqpoint{1.171496in}{1.343034in}}%
\pgfpathlineto{\pgfqpoint{1.172616in}{0.883694in}}%
\pgfpathlineto{\pgfqpoint{1.174043in}{1.292945in}}%
\pgfpathlineto{\pgfqpoint{1.175978in}{0.907775in}}%
\pgfpathlineto{\pgfqpoint{1.176997in}{1.354214in}}%
\pgfpathlineto{\pgfqpoint{1.179136in}{0.989417in}}%
\pgfpathlineto{\pgfqpoint{1.180562in}{1.191205in}}%
\pgfpathlineto{\pgfqpoint{1.183822in}{1.375985in}}%
\pgfpathlineto{\pgfqpoint{1.183924in}{0.976071in}}%
\pgfpathlineto{\pgfqpoint{1.185758in}{1.268904in}}%
\pgfpathlineto{\pgfqpoint{1.188407in}{0.943307in}}%
\pgfpathlineto{\pgfqpoint{1.190138in}{1.228738in}}%
\pgfpathlineto{\pgfqpoint{1.191565in}{0.870327in}}%
\pgfpathlineto{\pgfqpoint{1.193806in}{1.270510in}}%
\pgfpathlineto{\pgfqpoint{1.194621in}{0.865239in}}%
\pgfpathlineto{\pgfqpoint{1.196455in}{1.187543in}}%
\pgfpathlineto{\pgfqpoint{1.198696in}{0.894113in}}%
\pgfpathlineto{\pgfqpoint{1.199613in}{1.204516in}}%
\pgfpathlineto{\pgfqpoint{1.201650in}{0.960229in}}%
\pgfpathlineto{\pgfqpoint{1.204095in}{1.262750in}}%
\pgfpathlineto{\pgfqpoint{1.205521in}{0.850999in}}%
\pgfpathlineto{\pgfqpoint{1.206438in}{1.322639in}}%
\pgfpathlineto{\pgfqpoint{1.208679in}{0.942364in}}%
\pgfpathlineto{\pgfqpoint{1.210717in}{1.366798in}}%
\pgfpathlineto{\pgfqpoint{1.211735in}{0.926294in}}%
\pgfpathlineto{\pgfqpoint{1.213875in}{1.217588in}}%
\pgfpathlineto{\pgfqpoint{1.216014in}{0.813534in}}%
\pgfpathlineto{\pgfqpoint{1.218459in}{1.289258in}}%
\pgfpathlineto{\pgfqpoint{1.218867in}{0.833129in}}%
\pgfpathlineto{\pgfqpoint{1.221006in}{1.322855in}}%
\pgfpathlineto{\pgfqpoint{1.222025in}{1.038594in}}%
\pgfpathlineto{\pgfqpoint{1.224775in}{1.400947in}}%
\pgfpathlineto{\pgfqpoint{1.225998in}{0.944997in}}%
\pgfpathlineto{\pgfqpoint{1.228443in}{1.303219in}}%
\pgfpathlineto{\pgfqpoint{1.229461in}{0.901612in}}%
\pgfpathlineto{\pgfqpoint{1.231295in}{1.251830in}}%
\pgfpathlineto{\pgfqpoint{1.232925in}{0.959574in}}%
\pgfpathlineto{\pgfqpoint{1.234657in}{1.302132in}}%
\pgfpathlineto{\pgfqpoint{1.236796in}{0.838089in}}%
\pgfpathlineto{\pgfqpoint{1.238528in}{1.388139in}}%
\pgfpathlineto{\pgfqpoint{1.239547in}{0.970782in}}%
\pgfpathlineto{\pgfqpoint{1.242399in}{1.411816in}}%
\pgfpathlineto{\pgfqpoint{1.243418in}{0.867969in}}%
\pgfpathlineto{\pgfqpoint{1.245252in}{1.236947in}}%
\pgfpathlineto{\pgfqpoint{1.246678in}{0.863484in}}%
\pgfpathlineto{\pgfqpoint{1.248715in}{1.279211in}}%
\pgfpathlineto{\pgfqpoint{1.251262in}{0.858417in}}%
\pgfpathlineto{\pgfqpoint{1.251873in}{1.338354in}}%
\pgfpathlineto{\pgfqpoint{1.254318in}{0.957511in}}%
\pgfpathlineto{\pgfqpoint{1.256254in}{1.313000in}}%
\pgfpathlineto{\pgfqpoint{1.256661in}{1.018581in}}%
\pgfpathlineto{\pgfqpoint{1.259717in}{1.332764in}}%
\pgfpathlineto{\pgfqpoint{1.260940in}{0.895212in}}%
\pgfpathlineto{\pgfqpoint{1.262366in}{1.335240in}}%
\pgfpathlineto{\pgfqpoint{1.264811in}{1.014508in}}%
\pgfpathlineto{\pgfqpoint{1.265626in}{1.283010in}}%
\pgfpathlineto{\pgfqpoint{1.267867in}{0.982025in}}%
\pgfpathlineto{\pgfqpoint{1.268886in}{1.349580in}}%
\pgfpathlineto{\pgfqpoint{1.271229in}{0.949113in}}%
\pgfpathlineto{\pgfqpoint{1.273368in}{1.302528in}}%
\pgfpathlineto{\pgfqpoint{1.274285in}{0.937544in}}%
\pgfpathlineto{\pgfqpoint{1.276730in}{1.308940in}}%
\pgfpathlineto{\pgfqpoint{1.277851in}{0.816335in}}%
\pgfpathlineto{\pgfqpoint{1.280805in}{1.301228in}}%
\pgfpathlineto{\pgfqpoint{1.281824in}{0.840069in}}%
\pgfpathlineto{\pgfqpoint{1.283759in}{1.381261in}}%
\pgfpathlineto{\pgfqpoint{1.284371in}{0.996796in}}%
\pgfpathlineto{\pgfqpoint{1.287427in}{1.255715in}}%
\pgfpathlineto{\pgfqpoint{1.289057in}{0.897135in}}%
\pgfpathlineto{\pgfqpoint{1.290279in}{1.348194in}}%
\pgfpathlineto{\pgfqpoint{1.291807in}{0.959342in}}%
\pgfpathlineto{\pgfqpoint{1.293743in}{1.396263in}}%
\pgfpathlineto{\pgfqpoint{1.295475in}{0.859745in}}%
\pgfpathlineto{\pgfqpoint{1.296799in}{1.320015in}}%
\pgfpathlineto{\pgfqpoint{1.298735in}{0.951356in}}%
\pgfpathlineto{\pgfqpoint{1.300568in}{1.258572in}}%
\pgfpathlineto{\pgfqpoint{1.301995in}{0.826310in}}%
\pgfpathlineto{\pgfqpoint{1.304032in}{1.205178in}}%
\pgfpathlineto{\pgfqpoint{1.306273in}{0.851375in}}%
\pgfpathlineto{\pgfqpoint{1.307496in}{1.366537in}}%
\pgfpathlineto{\pgfqpoint{1.309329in}{0.958090in}}%
\pgfpathlineto{\pgfqpoint{1.310654in}{1.281428in}}%
\pgfpathlineto{\pgfqpoint{1.313201in}{0.820352in}}%
\pgfpathlineto{\pgfqpoint{1.314219in}{1.224886in}}%
\pgfpathlineto{\pgfqpoint{1.317072in}{0.913423in}}%
\pgfpathlineto{\pgfqpoint{1.317581in}{1.271063in}}%
\pgfpathlineto{\pgfqpoint{1.319007in}{0.916347in}}%
\pgfpathlineto{\pgfqpoint{1.320943in}{1.352492in}}%
\pgfpathlineto{\pgfqpoint{1.322980in}{0.964200in}}%
\pgfpathlineto{\pgfqpoint{1.324814in}{1.294563in}}%
\pgfpathlineto{\pgfqpoint{1.325935in}{0.970393in}}%
\pgfpathlineto{\pgfqpoint{1.329093in}{1.221773in}}%
\pgfpathlineto{\pgfqpoint{1.329602in}{0.868370in}}%
\pgfpathlineto{\pgfqpoint{1.331232in}{1.286943in}}%
\pgfpathlineto{\pgfqpoint{1.333473in}{0.922131in}}%
\pgfpathlineto{\pgfqpoint{1.335307in}{1.320692in}}%
\pgfpathlineto{\pgfqpoint{1.336733in}{0.868601in}}%
\pgfpathlineto{\pgfqpoint{1.339178in}{1.265252in}}%
\pgfpathlineto{\pgfqpoint{1.339789in}{0.854573in}}%
\pgfpathlineto{\pgfqpoint{1.341623in}{1.313070in}}%
\pgfpathlineto{\pgfqpoint{1.343355in}{0.915809in}}%
\pgfpathlineto{\pgfqpoint{1.345902in}{1.306875in}}%
\pgfpathlineto{\pgfqpoint{1.346920in}{0.960213in}}%
\pgfpathlineto{\pgfqpoint{1.348856in}{1.475355in}}%
\pgfpathlineto{\pgfqpoint{1.350792in}{0.879978in}}%
\pgfpathlineto{\pgfqpoint{1.353440in}{1.255477in}}%
\pgfpathlineto{\pgfqpoint{1.354255in}{0.913305in}}%
\pgfpathlineto{\pgfqpoint{1.355682in}{1.384325in}}%
\pgfpathlineto{\pgfqpoint{1.357821in}{0.896685in}}%
\pgfpathlineto{\pgfqpoint{1.359043in}{1.204351in}}%
\pgfpathlineto{\pgfqpoint{1.361998in}{1.319054in}}%
\pgfpathlineto{\pgfqpoint{1.363831in}{0.861402in}}%
\pgfpathlineto{\pgfqpoint{1.364035in}{1.183661in}}%
\pgfpathlineto{\pgfqpoint{1.366989in}{0.883050in}}%
\pgfpathlineto{\pgfqpoint{1.367804in}{1.320230in}}%
\pgfpathlineto{\pgfqpoint{1.368823in}{0.903598in}}%
\pgfpathlineto{\pgfqpoint{1.371777in}{1.287852in}}%
\pgfpathlineto{\pgfqpoint{1.372694in}{0.898535in}}%
\pgfpathlineto{\pgfqpoint{1.374120in}{1.245605in}}%
\pgfpathlineto{\pgfqpoint{1.375750in}{0.884039in}}%
\pgfpathlineto{\pgfqpoint{1.377075in}{1.217325in}}%
\pgfpathlineto{\pgfqpoint{1.378908in}{0.884680in}}%
\pgfpathlineto{\pgfqpoint{1.381353in}{1.352704in}}%
\pgfpathlineto{\pgfqpoint{1.382474in}{0.862580in}}%
\pgfpathlineto{\pgfqpoint{1.383493in}{1.280921in}}%
\pgfpathlineto{\pgfqpoint{1.385225in}{0.900302in}}%
\pgfpathlineto{\pgfqpoint{1.386753in}{1.246263in}}%
\pgfpathlineto{\pgfqpoint{1.389503in}{0.790822in}}%
\pgfpathlineto{\pgfqpoint{1.390420in}{1.221155in}}%
\pgfpathlineto{\pgfqpoint{1.391846in}{0.897221in}}%
\pgfpathlineto{\pgfqpoint{1.393578in}{1.413482in}}%
\pgfpathlineto{\pgfqpoint{1.395310in}{0.966793in}}%
\pgfpathlineto{\pgfqpoint{1.396532in}{1.307949in}}%
\pgfpathlineto{\pgfqpoint{1.398977in}{1.018699in}}%
\pgfpathlineto{\pgfqpoint{1.401219in}{1.278604in}}%
\pgfpathlineto{\pgfqpoint{1.401524in}{0.904772in}}%
\pgfpathlineto{\pgfqpoint{1.404886in}{1.397556in}}%
\pgfpathlineto{\pgfqpoint{1.406720in}{0.960416in}}%
\pgfpathlineto{\pgfqpoint{1.408655in}{1.305317in}}%
\pgfpathlineto{\pgfqpoint{1.410693in}{0.809873in}}%
\pgfpathlineto{\pgfqpoint{1.412017in}{1.356655in}}%
\pgfpathlineto{\pgfqpoint{1.413036in}{0.935303in}}%
\pgfpathlineto{\pgfqpoint{1.415888in}{1.280097in}}%
\pgfpathlineto{\pgfqpoint{1.416194in}{0.941667in}}%
\pgfpathlineto{\pgfqpoint{1.418333in}{1.259403in}}%
\pgfpathlineto{\pgfqpoint{1.419658in}{0.993018in}}%
\pgfpathlineto{\pgfqpoint{1.422204in}{0.713311in}}%
\pgfpathlineto{\pgfqpoint{1.423529in}{1.390133in}}%
\pgfpathlineto{\pgfqpoint{1.425159in}{0.884356in}}%
\pgfpathlineto{\pgfqpoint{1.425872in}{1.318608in}}%
\pgfpathlineto{\pgfqpoint{1.428928in}{0.722526in}}%
\pgfpathlineto{\pgfqpoint{1.429539in}{1.265583in}}%
\pgfpathlineto{\pgfqpoint{1.430965in}{0.953984in}}%
\pgfpathlineto{\pgfqpoint{1.432697in}{1.239912in}}%
\pgfpathlineto{\pgfqpoint{1.434022in}{0.932136in}}%
\pgfpathlineto{\pgfqpoint{1.437078in}{1.347448in}}%
\pgfpathlineto{\pgfqpoint{1.437281in}{0.965684in}}%
\pgfpathlineto{\pgfqpoint{1.439726in}{1.377986in}}%
\pgfpathlineto{\pgfqpoint{1.441356in}{0.909133in}}%
\pgfpathlineto{\pgfqpoint{1.442171in}{1.284672in}}%
\pgfpathlineto{\pgfqpoint{1.444413in}{0.981352in}}%
\pgfpathlineto{\pgfqpoint{1.446552in}{1.314167in}}%
\pgfpathlineto{\pgfqpoint{1.447367in}{0.801826in}}%
\pgfpathlineto{\pgfqpoint{1.448793in}{1.405263in}}%
\pgfpathlineto{\pgfqpoint{1.450321in}{0.927978in}}%
\pgfpathlineto{\pgfqpoint{1.452053in}{1.220219in}}%
\pgfpathlineto{\pgfqpoint{1.454600in}{0.945993in}}%
\pgfpathlineto{\pgfqpoint{1.455415in}{1.358127in}}%
\pgfpathlineto{\pgfqpoint{1.457045in}{0.956005in}}%
\pgfpathlineto{\pgfqpoint{1.458777in}{1.271403in}}%
\pgfpathlineto{\pgfqpoint{1.460305in}{1.013971in}}%
\pgfpathlineto{\pgfqpoint{1.461731in}{1.322123in}}%
\pgfpathlineto{\pgfqpoint{1.464278in}{0.771366in}}%
\pgfpathlineto{\pgfqpoint{1.465398in}{1.312367in}}%
\pgfpathlineto{\pgfqpoint{1.466723in}{0.966210in}}%
\pgfpathlineto{\pgfqpoint{1.468556in}{1.225686in}}%
\pgfpathlineto{\pgfqpoint{1.470594in}{0.893192in}}%
\pgfpathlineto{\pgfqpoint{1.471511in}{1.381133in}}%
\pgfpathlineto{\pgfqpoint{1.473141in}{0.898748in}}%
\pgfpathlineto{\pgfqpoint{1.475382in}{1.296232in}}%
\pgfpathlineto{\pgfqpoint{1.477419in}{0.942267in}}%
\pgfpathlineto{\pgfqpoint{1.478642in}{1.418736in}}%
\pgfpathlineto{\pgfqpoint{1.480170in}{0.760491in}}%
\pgfpathlineto{\pgfqpoint{1.482004in}{1.335711in}}%
\pgfpathlineto{\pgfqpoint{1.482920in}{0.938092in}}%
\pgfpathlineto{\pgfqpoint{1.484958in}{1.333521in}}%
\pgfpathlineto{\pgfqpoint{1.487505in}{0.904215in}}%
\pgfpathlineto{\pgfqpoint{1.487810in}{1.253446in}}%
\pgfpathlineto{\pgfqpoint{1.490663in}{0.791995in}}%
\pgfpathlineto{\pgfqpoint{1.491478in}{1.336156in}}%
\pgfpathlineto{\pgfqpoint{1.493006in}{0.952374in}}%
\pgfpathlineto{\pgfqpoint{1.494636in}{1.293600in}}%
\pgfpathlineto{\pgfqpoint{1.496877in}{0.892637in}}%
\pgfpathlineto{\pgfqpoint{1.497896in}{1.334336in}}%
\pgfpathlineto{\pgfqpoint{1.499831in}{0.845044in}}%
\pgfpathlineto{\pgfqpoint{1.501156in}{1.274859in}}%
\pgfpathlineto{\pgfqpoint{1.503091in}{0.853052in}}%
\pgfpathlineto{\pgfqpoint{1.504721in}{1.202384in}}%
\pgfpathlineto{\pgfqpoint{1.506962in}{0.964265in}}%
\pgfpathlineto{\pgfqpoint{1.507370in}{1.336042in}}%
\pgfpathlineto{\pgfqpoint{1.509000in}{0.986097in}}%
\pgfpathlineto{\pgfqpoint{1.511139in}{1.312190in}}%
\pgfpathlineto{\pgfqpoint{1.512565in}{0.914817in}}%
\pgfpathlineto{\pgfqpoint{1.514501in}{1.238781in}}%
\pgfpathlineto{\pgfqpoint{1.515927in}{0.950416in}}%
\pgfpathlineto{\pgfqpoint{1.517353in}{1.306128in}}%
\pgfpathlineto{\pgfqpoint{1.519289in}{0.854148in}}%
\pgfpathlineto{\pgfqpoint{1.521530in}{1.299557in}}%
\pgfpathlineto{\pgfqpoint{1.522040in}{0.929965in}}%
\pgfpathlineto{\pgfqpoint{1.523669in}{1.287757in}}%
\pgfpathlineto{\pgfqpoint{1.526013in}{0.962549in}}%
\pgfpathlineto{\pgfqpoint{1.527031in}{1.273477in}}%
\pgfpathlineto{\pgfqpoint{1.528559in}{0.978766in}}%
\pgfpathlineto{\pgfqpoint{1.530189in}{1.268471in}}%
\pgfpathlineto{\pgfqpoint{1.532838in}{0.954918in}}%
\pgfpathlineto{\pgfqpoint{1.533551in}{1.286914in}}%
\pgfpathlineto{\pgfqpoint{1.535792in}{1.436153in}}%
\pgfpathlineto{\pgfqpoint{1.537117in}{0.946226in}}%
\pgfpathlineto{\pgfqpoint{1.538747in}{1.236756in}}%
\pgfpathlineto{\pgfqpoint{1.540784in}{0.957323in}}%
\pgfpathlineto{\pgfqpoint{1.543127in}{1.235377in}}%
\pgfpathlineto{\pgfqpoint{1.543840in}{0.890090in}}%
\pgfpathlineto{\pgfqpoint{1.545470in}{1.244513in}}%
\pgfpathlineto{\pgfqpoint{1.547100in}{0.960242in}}%
\pgfpathlineto{\pgfqpoint{1.548425in}{1.300087in}}%
\pgfpathlineto{\pgfqpoint{1.549749in}{0.927775in}}%
\pgfpathlineto{\pgfqpoint{1.551888in}{1.286246in}}%
\pgfpathlineto{\pgfqpoint{1.554231in}{0.940386in}}%
\pgfpathlineto{\pgfqpoint{1.555148in}{1.288885in}}%
\pgfpathlineto{\pgfqpoint{1.556371in}{0.996701in}}%
\pgfpathlineto{\pgfqpoint{1.559121in}{1.213688in}}%
\pgfpathlineto{\pgfqpoint{1.560140in}{0.939610in}}%
\pgfpathlineto{\pgfqpoint{1.562177in}{1.278716in}}%
\pgfpathlineto{\pgfqpoint{1.562992in}{0.911288in}}%
\pgfpathlineto{\pgfqpoint{1.564622in}{1.223963in}}%
\pgfpathlineto{\pgfqpoint{1.567067in}{0.978178in}}%
\pgfpathlineto{\pgfqpoint{1.568392in}{1.237709in}}%
\pgfpathlineto{\pgfqpoint{1.569614in}{0.946243in}}%
\pgfpathlineto{\pgfqpoint{1.571244in}{1.319667in}}%
\pgfpathlineto{\pgfqpoint{1.573078in}{0.980629in}}%
\pgfpathlineto{\pgfqpoint{1.574911in}{1.297874in}}%
\pgfpathlineto{\pgfqpoint{1.575930in}{0.871305in}}%
\pgfpathlineto{\pgfqpoint{1.577662in}{1.213640in}}%
\pgfpathlineto{\pgfqpoint{1.580616in}{0.970338in}}%
\pgfpathlineto{\pgfqpoint{1.581635in}{1.235527in}}%
\pgfpathlineto{\pgfqpoint{1.583876in}{0.840108in}}%
\pgfpathlineto{\pgfqpoint{1.584080in}{1.278455in}}%
\pgfpathlineto{\pgfqpoint{1.586423in}{0.941224in}}%
\pgfpathlineto{\pgfqpoint{1.587544in}{1.227015in}}%
\pgfpathlineto{\pgfqpoint{1.589989in}{1.395039in}}%
\pgfpathlineto{\pgfqpoint{1.590905in}{0.899391in}}%
\pgfpathlineto{\pgfqpoint{1.592230in}{1.223485in}}%
\pgfpathlineto{\pgfqpoint{1.595286in}{1.305835in}}%
\pgfpathlineto{\pgfqpoint{1.595795in}{0.940510in}}%
\pgfpathlineto{\pgfqpoint{1.597323in}{1.251690in}}%
\pgfpathlineto{\pgfqpoint{1.599768in}{0.874544in}}%
\pgfpathlineto{\pgfqpoint{1.600380in}{1.253090in}}%
\pgfpathlineto{\pgfqpoint{1.602010in}{0.884726in}}%
\pgfpathlineto{\pgfqpoint{1.603538in}{1.262627in}}%
\pgfpathlineto{\pgfqpoint{1.605881in}{0.973467in}}%
\pgfpathlineto{\pgfqpoint{1.607409in}{1.316742in}}%
\pgfpathlineto{\pgfqpoint{1.608529in}{0.917104in}}%
\pgfpathlineto{\pgfqpoint{1.611178in}{1.254112in}}%
\pgfpathlineto{\pgfqpoint{1.611993in}{0.878184in}}%
\pgfpathlineto{\pgfqpoint{1.613317in}{1.228935in}}%
\pgfpathlineto{\pgfqpoint{1.615660in}{0.850284in}}%
\pgfpathlineto{\pgfqpoint{1.616985in}{1.332051in}}%
\pgfpathlineto{\pgfqpoint{1.619634in}{0.885928in}}%
\pgfpathlineto{\pgfqpoint{1.620245in}{1.314806in}}%
\pgfpathlineto{\pgfqpoint{1.622078in}{0.812884in}}%
\pgfpathlineto{\pgfqpoint{1.623199in}{1.224111in}}%
\pgfpathlineto{\pgfqpoint{1.625440in}{0.806906in}}%
\pgfpathlineto{\pgfqpoint{1.627580in}{1.356312in}}%
\pgfpathlineto{\pgfqpoint{1.628395in}{0.954216in}}%
\pgfpathlineto{\pgfqpoint{1.630228in}{1.263064in}}%
\pgfpathlineto{\pgfqpoint{1.632266in}{0.933684in}}%
\pgfpathlineto{\pgfqpoint{1.634201in}{1.369910in}}%
\pgfpathlineto{\pgfqpoint{1.634914in}{0.945499in}}%
\pgfpathlineto{\pgfqpoint{1.636137in}{1.174326in}}%
\pgfpathlineto{\pgfqpoint{1.637767in}{0.957556in}}%
\pgfpathlineto{\pgfqpoint{1.640008in}{1.224413in}}%
\pgfpathlineto{\pgfqpoint{1.642147in}{0.688601in}}%
\pgfpathlineto{\pgfqpoint{1.642657in}{1.270347in}}%
\pgfpathlineto{\pgfqpoint{1.645407in}{1.346181in}}%
\pgfpathlineto{\pgfqpoint{1.646019in}{0.883959in}}%
\pgfpathlineto{\pgfqpoint{1.648667in}{1.331387in}}%
\pgfpathlineto{\pgfqpoint{1.649177in}{0.973681in}}%
\pgfpathlineto{\pgfqpoint{1.651214in}{1.319622in}}%
\pgfpathlineto{\pgfqpoint{1.652436in}{0.973104in}}%
\pgfpathlineto{\pgfqpoint{1.655187in}{1.273036in}}%
\pgfpathlineto{\pgfqpoint{1.655900in}{0.969891in}}%
\pgfpathlineto{\pgfqpoint{1.657734in}{1.213285in}}%
\pgfpathlineto{\pgfqpoint{1.660077in}{0.855356in}}%
\pgfpathlineto{\pgfqpoint{1.661911in}{1.270532in}}%
\pgfpathlineto{\pgfqpoint{1.663439in}{0.916920in}}%
\pgfpathlineto{\pgfqpoint{1.664559in}{1.254111in}}%
\pgfpathlineto{\pgfqpoint{1.666801in}{0.963816in}}%
\pgfpathlineto{\pgfqpoint{1.667514in}{1.292216in}}%
\pgfpathlineto{\pgfqpoint{1.669042in}{0.894022in}}%
\pgfpathlineto{\pgfqpoint{1.670774in}{1.169986in}}%
\pgfpathlineto{\pgfqpoint{1.672098in}{0.958655in}}%
\pgfpathlineto{\pgfqpoint{1.673728in}{1.297932in}}%
\pgfpathlineto{\pgfqpoint{1.675969in}{0.880742in}}%
\pgfpathlineto{\pgfqpoint{1.677803in}{1.367038in}}%
\pgfpathlineto{\pgfqpoint{1.678516in}{0.815296in}}%
\pgfpathlineto{\pgfqpoint{1.680350in}{1.220779in}}%
\pgfpathlineto{\pgfqpoint{1.682285in}{0.841234in}}%
\pgfpathlineto{\pgfqpoint{1.684424in}{1.410426in}}%
\pgfpathlineto{\pgfqpoint{1.685036in}{1.074676in}}%
\pgfpathlineto{\pgfqpoint{1.687684in}{0.813398in}}%
\pgfpathlineto{\pgfqpoint{1.688398in}{1.237326in}}%
\pgfpathlineto{\pgfqpoint{1.690639in}{0.962058in}}%
\pgfpathlineto{\pgfqpoint{1.693287in}{1.310612in}}%
\pgfpathlineto{\pgfqpoint{1.695732in}{0.854019in}}%
\pgfpathlineto{\pgfqpoint{1.697057in}{1.178117in}}%
\pgfpathlineto{\pgfqpoint{1.699400in}{1.302878in}}%
\pgfpathlineto{\pgfqpoint{1.700317in}{0.891395in}}%
\pgfpathlineto{\pgfqpoint{1.701743in}{1.374384in}}%
\pgfpathlineto{\pgfqpoint{1.703067in}{0.958150in}}%
\pgfpathlineto{\pgfqpoint{1.705105in}{1.294014in}}%
\pgfpathlineto{\pgfqpoint{1.706225in}{0.905329in}}%
\pgfpathlineto{\pgfqpoint{1.707957in}{1.237091in}}%
\pgfpathlineto{\pgfqpoint{1.709995in}{0.942608in}}%
\pgfpathlineto{\pgfqpoint{1.712134in}{1.380681in}}%
\pgfpathlineto{\pgfqpoint{1.712745in}{1.013452in}}%
\pgfpathlineto{\pgfqpoint{1.714477in}{1.219567in}}%
\pgfpathlineto{\pgfqpoint{1.717227in}{0.903979in}}%
\pgfpathlineto{\pgfqpoint{1.718450in}{1.326299in}}%
\pgfpathlineto{\pgfqpoint{1.719978in}{0.931383in}}%
\pgfpathlineto{\pgfqpoint{1.721302in}{1.198140in}}%
\pgfpathlineto{\pgfqpoint{1.723645in}{0.907612in}}%
\pgfpathlineto{\pgfqpoint{1.724766in}{1.398523in}}%
\pgfpathlineto{\pgfqpoint{1.727211in}{0.907019in}}%
\pgfpathlineto{\pgfqpoint{1.727517in}{1.222543in}}%
\pgfpathlineto{\pgfqpoint{1.730063in}{0.941348in}}%
\pgfpathlineto{\pgfqpoint{1.731388in}{1.310063in}}%
\pgfpathlineto{\pgfqpoint{1.733120in}{0.921657in}}%
\pgfpathlineto{\pgfqpoint{1.734444in}{1.219170in}}%
\pgfpathlineto{\pgfqpoint{1.736787in}{0.882671in}}%
\pgfpathlineto{\pgfqpoint{1.738010in}{1.326160in}}%
\pgfpathlineto{\pgfqpoint{1.739028in}{0.893857in}}%
\pgfpathlineto{\pgfqpoint{1.740556in}{1.261996in}}%
\pgfpathlineto{\pgfqpoint{1.743612in}{0.784032in}}%
\pgfpathlineto{\pgfqpoint{1.743918in}{1.314588in}}%
\pgfpathlineto{\pgfqpoint{1.745854in}{0.918807in}}%
\pgfpathlineto{\pgfqpoint{1.746974in}{1.231852in}}%
\pgfpathlineto{\pgfqpoint{1.749317in}{0.936593in}}%
\pgfpathlineto{\pgfqpoint{1.750540in}{1.276466in}}%
\pgfpathlineto{\pgfqpoint{1.752883in}{0.963854in}}%
\pgfpathlineto{\pgfqpoint{1.753698in}{1.258132in}}%
\pgfpathlineto{\pgfqpoint{1.755124in}{1.009231in}}%
\pgfpathlineto{\pgfqpoint{1.757773in}{0.922701in}}%
\pgfpathlineto{\pgfqpoint{1.759301in}{1.274352in}}%
\pgfpathlineto{\pgfqpoint{1.761135in}{0.839587in}}%
\pgfpathlineto{\pgfqpoint{1.762255in}{1.338524in}}%
\pgfpathlineto{\pgfqpoint{1.763376in}{0.788810in}}%
\pgfpathlineto{\pgfqpoint{1.765209in}{1.217753in}}%
\pgfpathlineto{\pgfqpoint{1.766839in}{0.964165in}}%
\pgfpathlineto{\pgfqpoint{1.769386in}{1.240843in}}%
\pgfpathlineto{\pgfqpoint{1.770711in}{0.926664in}}%
\pgfpathlineto{\pgfqpoint{1.772442in}{1.331986in}}%
\pgfpathlineto{\pgfqpoint{1.773767in}{0.913852in}}%
\pgfpathlineto{\pgfqpoint{1.775702in}{1.285458in}}%
\pgfpathlineto{\pgfqpoint{1.776415in}{0.791597in}}%
\pgfpathlineto{\pgfqpoint{1.778045in}{1.174387in}}%
\pgfpathlineto{\pgfqpoint{1.780694in}{0.936310in}}%
\pgfpathlineto{\pgfqpoint{1.781815in}{1.258021in}}%
\pgfpathlineto{\pgfqpoint{1.783037in}{0.841158in}}%
\pgfpathlineto{\pgfqpoint{1.785788in}{0.746336in}}%
\pgfpathlineto{\pgfqpoint{1.786399in}{1.208682in}}%
\pgfpathlineto{\pgfqpoint{1.787927in}{0.949453in}}%
\pgfpathlineto{\pgfqpoint{1.790474in}{1.389384in}}%
\pgfpathlineto{\pgfqpoint{1.791493in}{0.964056in}}%
\pgfpathlineto{\pgfqpoint{1.793530in}{0.814562in}}%
\pgfpathlineto{\pgfqpoint{1.794956in}{1.300200in}}%
\pgfpathlineto{\pgfqpoint{1.795873in}{0.999230in}}%
\pgfpathlineto{\pgfqpoint{1.798318in}{0.873040in}}%
\pgfpathlineto{\pgfqpoint{1.800356in}{1.272068in}}%
\pgfpathlineto{\pgfqpoint{1.801476in}{0.952289in}}%
\pgfpathlineto{\pgfqpoint{1.803310in}{1.266850in}}%
\pgfpathlineto{\pgfqpoint{1.805245in}{0.913116in}}%
\pgfpathlineto{\pgfqpoint{1.805857in}{1.205776in}}%
\pgfpathlineto{\pgfqpoint{1.807487in}{0.850553in}}%
\pgfpathlineto{\pgfqpoint{1.808913in}{1.263298in}}%
\pgfpathlineto{\pgfqpoint{1.810848in}{0.752087in}}%
\pgfpathlineto{\pgfqpoint{1.813701in}{1.360501in}}%
\pgfpathlineto{\pgfqpoint{1.814618in}{0.943312in}}%
\pgfpathlineto{\pgfqpoint{1.815840in}{1.253084in}}%
\pgfpathlineto{\pgfqpoint{1.817063in}{0.986461in}}%
\pgfpathlineto{\pgfqpoint{1.818795in}{1.299432in}}%
\pgfpathlineto{\pgfqpoint{1.820526in}{0.795635in}}%
\pgfpathlineto{\pgfqpoint{1.822054in}{1.267154in}}%
\pgfpathlineto{\pgfqpoint{1.824194in}{0.890116in}}%
\pgfpathlineto{\pgfqpoint{1.825518in}{1.207666in}}%
\pgfpathlineto{\pgfqpoint{1.828269in}{1.369171in}}%
\pgfpathlineto{\pgfqpoint{1.828574in}{0.906786in}}%
\pgfpathlineto{\pgfqpoint{1.831427in}{1.367580in}}%
\pgfpathlineto{\pgfqpoint{1.832955in}{0.789502in}}%
\pgfpathlineto{\pgfqpoint{1.833668in}{1.260186in}}%
\pgfpathlineto{\pgfqpoint{1.836113in}{0.885079in}}%
\pgfpathlineto{\pgfqpoint{1.836928in}{1.308835in}}%
\pgfpathlineto{\pgfqpoint{1.838965in}{0.927347in}}%
\pgfpathlineto{\pgfqpoint{1.840697in}{1.286892in}}%
\pgfpathlineto{\pgfqpoint{1.843550in}{0.782403in}}%
\pgfpathlineto{\pgfqpoint{1.844976in}{1.189110in}}%
\pgfpathlineto{\pgfqpoint{1.847115in}{0.908890in}}%
\pgfpathlineto{\pgfqpoint{1.848134in}{1.269114in}}%
\pgfpathlineto{\pgfqpoint{1.850069in}{0.970634in}}%
\pgfpathlineto{\pgfqpoint{1.851597in}{1.280205in}}%
\pgfpathlineto{\pgfqpoint{1.853839in}{0.861670in}}%
\pgfpathlineto{\pgfqpoint{1.855367in}{1.275597in}}%
\pgfpathlineto{\pgfqpoint{1.856589in}{0.941708in}}%
\pgfpathlineto{\pgfqpoint{1.858219in}{1.317201in}}%
\pgfpathlineto{\pgfqpoint{1.859849in}{0.911891in}}%
\pgfpathlineto{\pgfqpoint{1.861785in}{1.360355in}}%
\pgfpathlineto{\pgfqpoint{1.864535in}{0.786656in}}%
\pgfpathlineto{\pgfqpoint{1.865962in}{1.265774in}}%
\pgfpathlineto{\pgfqpoint{1.868406in}{0.949530in}}%
\pgfpathlineto{\pgfqpoint{1.869323in}{1.226785in}}%
\pgfpathlineto{\pgfqpoint{1.872074in}{1.361275in}}%
\pgfpathlineto{\pgfqpoint{1.873093in}{0.856092in}}%
\pgfpathlineto{\pgfqpoint{1.874213in}{1.210178in}}%
\pgfpathlineto{\pgfqpoint{1.876760in}{0.944034in}}%
\pgfpathlineto{\pgfqpoint{1.877983in}{1.267274in}}%
\pgfpathlineto{\pgfqpoint{1.879511in}{0.930083in}}%
\pgfpathlineto{\pgfqpoint{1.880631in}{1.183345in}}%
\pgfpathlineto{\pgfqpoint{1.882567in}{1.309654in}}%
\pgfpathlineto{\pgfqpoint{1.884604in}{0.869662in}}%
\pgfpathlineto{\pgfqpoint{1.886642in}{1.369291in}}%
\pgfpathlineto{\pgfqpoint{1.887253in}{0.892915in}}%
\pgfpathlineto{\pgfqpoint{1.889800in}{1.279996in}}%
\pgfpathlineto{\pgfqpoint{1.890920in}{0.874559in}}%
\pgfpathlineto{\pgfqpoint{1.893365in}{1.345763in}}%
\pgfpathlineto{\pgfqpoint{1.893773in}{0.988012in}}%
\pgfpathlineto{\pgfqpoint{1.896829in}{0.889358in}}%
\pgfpathlineto{\pgfqpoint{1.897848in}{1.319886in}}%
\pgfpathlineto{\pgfqpoint{1.898561in}{0.953063in}}%
\pgfpathlineto{\pgfqpoint{1.900802in}{1.303024in}}%
\pgfpathlineto{\pgfqpoint{1.902636in}{0.846161in}}%
\pgfpathlineto{\pgfqpoint{1.903654in}{1.163155in}}%
\pgfpathlineto{\pgfqpoint{1.905182in}{0.852612in}}%
\pgfpathlineto{\pgfqpoint{1.908137in}{1.353449in}}%
\pgfpathlineto{\pgfqpoint{1.908748in}{0.954207in}}%
\pgfpathlineto{\pgfqpoint{1.911091in}{1.270611in}}%
\pgfpathlineto{\pgfqpoint{1.912212in}{0.940064in}}%
\pgfpathlineto{\pgfqpoint{1.913230in}{1.265133in}}%
\pgfpathlineto{\pgfqpoint{1.915981in}{0.913415in}}%
\pgfpathlineto{\pgfqpoint{1.917000in}{1.263887in}}%
\pgfpathlineto{\pgfqpoint{1.918120in}{0.914993in}}%
\pgfpathlineto{\pgfqpoint{1.919852in}{1.199802in}}%
\pgfpathlineto{\pgfqpoint{1.921991in}{1.325202in}}%
\pgfpathlineto{\pgfqpoint{1.923825in}{0.940842in}}%
\pgfpathlineto{\pgfqpoint{1.925251in}{1.225080in}}%
\pgfpathlineto{\pgfqpoint{1.926270in}{0.949518in}}%
\pgfpathlineto{\pgfqpoint{1.929734in}{1.416276in}}%
\pgfpathlineto{\pgfqpoint{1.931262in}{0.886185in}}%
\pgfpathlineto{\pgfqpoint{1.933503in}{1.306443in}}%
\pgfpathlineto{\pgfqpoint{1.934522in}{0.949629in}}%
\pgfpathlineto{\pgfqpoint{1.936050in}{1.273238in}}%
\pgfpathlineto{\pgfqpoint{1.939208in}{0.842961in}}%
\pgfpathlineto{\pgfqpoint{1.939412in}{1.226517in}}%
\pgfpathlineto{\pgfqpoint{1.941449in}{1.449493in}}%
\pgfpathlineto{\pgfqpoint{1.942672in}{0.959146in}}%
\pgfpathlineto{\pgfqpoint{1.944505in}{1.342851in}}%
\pgfpathlineto{\pgfqpoint{1.946135in}{0.918953in}}%
\pgfpathlineto{\pgfqpoint{1.947765in}{1.239785in}}%
\pgfpathlineto{\pgfqpoint{1.949497in}{0.887948in}}%
\pgfpathlineto{\pgfqpoint{1.950720in}{1.272419in}}%
\pgfpathlineto{\pgfqpoint{1.952655in}{0.983607in}}%
\pgfpathlineto{\pgfqpoint{1.955304in}{1.330510in}}%
\pgfpathlineto{\pgfqpoint{1.955711in}{0.936637in}}%
\pgfpathlineto{\pgfqpoint{1.958564in}{0.868952in}}%
\pgfpathlineto{\pgfqpoint{1.958869in}{1.191808in}}%
\pgfpathlineto{\pgfqpoint{1.961518in}{0.915911in}}%
\pgfpathlineto{\pgfqpoint{1.962333in}{1.402875in}}%
\pgfpathlineto{\pgfqpoint{1.963963in}{0.867325in}}%
\pgfpathlineto{\pgfqpoint{1.965899in}{1.312156in}}%
\pgfpathlineto{\pgfqpoint{1.967019in}{0.968516in}}%
\pgfpathlineto{\pgfqpoint{1.968649in}{1.186227in}}%
\pgfpathlineto{\pgfqpoint{1.970483in}{0.859387in}}%
\pgfpathlineto{\pgfqpoint{1.972520in}{1.306134in}}%
\pgfpathlineto{\pgfqpoint{1.974048in}{1.040768in}}%
\pgfpathlineto{\pgfqpoint{1.976595in}{1.331539in}}%
\pgfpathlineto{\pgfqpoint{1.977003in}{0.988170in}}%
\pgfpathlineto{\pgfqpoint{1.979040in}{1.298600in}}%
\pgfpathlineto{\pgfqpoint{1.980670in}{0.982727in}}%
\pgfpathlineto{\pgfqpoint{1.982809in}{1.385669in}}%
\pgfpathlineto{\pgfqpoint{1.983930in}{0.841881in}}%
\pgfpathlineto{\pgfqpoint{1.985560in}{1.217764in}}%
\pgfpathlineto{\pgfqpoint{1.986884in}{0.923414in}}%
\pgfpathlineto{\pgfqpoint{1.988412in}{1.423791in}}%
\pgfpathlineto{\pgfqpoint{1.990857in}{0.914198in}}%
\pgfpathlineto{\pgfqpoint{1.992182in}{1.429693in}}%
\pgfpathlineto{\pgfqpoint{1.993506in}{0.956268in}}%
\pgfpathlineto{\pgfqpoint{1.996257in}{1.379933in}}%
\pgfpathlineto{\pgfqpoint{1.996562in}{0.937071in}}%
\pgfpathlineto{\pgfqpoint{1.998090in}{1.224883in}}%
\pgfpathlineto{\pgfqpoint{2.000739in}{1.283033in}}%
\pgfpathlineto{\pgfqpoint{2.002165in}{0.906055in}}%
\pgfpathlineto{\pgfqpoint{2.003388in}{1.349074in}}%
\pgfpathlineto{\pgfqpoint{2.005833in}{0.886002in}}%
\pgfpathlineto{\pgfqpoint{2.007055in}{1.198749in}}%
\pgfpathlineto{\pgfqpoint{2.008583in}{0.864912in}}%
\pgfpathlineto{\pgfqpoint{2.009704in}{1.248462in}}%
\pgfpathlineto{\pgfqpoint{2.011232in}{0.995170in}}%
\pgfpathlineto{\pgfqpoint{2.012760in}{1.319623in}}%
\pgfpathlineto{\pgfqpoint{2.015612in}{0.804979in}}%
\pgfpathlineto{\pgfqpoint{2.016427in}{1.439194in}}%
\pgfpathlineto{\pgfqpoint{2.017650in}{1.012038in}}%
\pgfpathlineto{\pgfqpoint{2.020095in}{1.325561in}}%
\pgfpathlineto{\pgfqpoint{2.021725in}{0.951829in}}%
\pgfpathlineto{\pgfqpoint{2.023966in}{1.386797in}}%
\pgfpathlineto{\pgfqpoint{2.024272in}{1.010381in}}%
\pgfpathlineto{\pgfqpoint{2.027022in}{1.325088in}}%
\pgfpathlineto{\pgfqpoint{2.027837in}{0.915748in}}%
\pgfpathlineto{\pgfqpoint{2.030486in}{0.789331in}}%
\pgfpathlineto{\pgfqpoint{2.030588in}{1.245234in}}%
\pgfpathlineto{\pgfqpoint{2.033746in}{1.365152in}}%
\pgfpathlineto{\pgfqpoint{2.034153in}{0.908072in}}%
\pgfpathlineto{\pgfqpoint{2.036191in}{1.279909in}}%
\pgfpathlineto{\pgfqpoint{2.037108in}{0.886895in}}%
\pgfpathlineto{\pgfqpoint{2.039654in}{1.363740in}}%
\pgfpathlineto{\pgfqpoint{2.040469in}{0.923776in}}%
\pgfpathlineto{\pgfqpoint{2.042201in}{1.199009in}}%
\pgfpathlineto{\pgfqpoint{2.044035in}{0.868348in}}%
\pgfpathlineto{\pgfqpoint{2.045767in}{1.262310in}}%
\pgfpathlineto{\pgfqpoint{2.047193in}{0.851710in}}%
\pgfpathlineto{\pgfqpoint{2.048619in}{1.303440in}}%
\pgfpathlineto{\pgfqpoint{2.051268in}{0.852416in}}%
\pgfpathlineto{\pgfqpoint{2.052898in}{1.360557in}}%
\pgfpathlineto{\pgfqpoint{2.053407in}{1.005389in}}%
\pgfpathlineto{\pgfqpoint{2.055139in}{1.319350in}}%
\pgfpathlineto{\pgfqpoint{2.057788in}{1.006527in}}%
\pgfpathlineto{\pgfqpoint{2.058908in}{1.265496in}}%
\pgfpathlineto{\pgfqpoint{2.060436in}{0.858437in}}%
\pgfpathlineto{\pgfqpoint{2.061964in}{1.204025in}}%
\pgfpathlineto{\pgfqpoint{2.064511in}{0.795303in}}%
\pgfpathlineto{\pgfqpoint{2.064817in}{1.315938in}}%
\pgfpathlineto{\pgfqpoint{2.067160in}{0.790867in}}%
\pgfpathlineto{\pgfqpoint{2.068179in}{1.290750in}}%
\pgfpathlineto{\pgfqpoint{2.069809in}{0.998943in}}%
\pgfpathlineto{\pgfqpoint{2.071744in}{1.219711in}}%
\pgfpathlineto{\pgfqpoint{2.073782in}{0.962092in}}%
\pgfpathlineto{\pgfqpoint{2.075615in}{1.273510in}}%
\pgfpathlineto{\pgfqpoint{2.076430in}{0.921509in}}%
\pgfpathlineto{\pgfqpoint{2.078977in}{1.329781in}}%
\pgfpathlineto{\pgfqpoint{2.079487in}{1.024109in}}%
\pgfpathlineto{\pgfqpoint{2.081320in}{1.296186in}}%
\pgfpathlineto{\pgfqpoint{2.083256in}{0.874570in}}%
\pgfpathlineto{\pgfqpoint{2.085701in}{1.329056in}}%
\pgfpathlineto{\pgfqpoint{2.086210in}{0.893759in}}%
\pgfpathlineto{\pgfqpoint{2.088044in}{1.355407in}}%
\pgfpathlineto{\pgfqpoint{2.089776in}{0.992643in}}%
\pgfpathlineto{\pgfqpoint{2.090998in}{1.260268in}}%
\pgfpathlineto{\pgfqpoint{2.093443in}{0.954995in}}%
\pgfpathlineto{\pgfqpoint{2.094564in}{1.283267in}}%
\pgfpathlineto{\pgfqpoint{2.095888in}{0.982453in}}%
\pgfpathlineto{\pgfqpoint{2.098537in}{1.350111in}}%
\pgfpathlineto{\pgfqpoint{2.099046in}{1.068687in}}%
\pgfpathlineto{\pgfqpoint{2.102102in}{0.889968in}}%
\pgfpathlineto{\pgfqpoint{2.103529in}{1.357374in}}%
\pgfpathlineto{\pgfqpoint{2.103936in}{0.949989in}}%
\pgfpathlineto{\pgfqpoint{2.105872in}{1.288919in}}%
\pgfpathlineto{\pgfqpoint{2.108418in}{0.909998in}}%
\pgfpathlineto{\pgfqpoint{2.109233in}{1.298902in}}%
\pgfpathlineto{\pgfqpoint{2.110660in}{0.984262in}}%
\pgfpathlineto{\pgfqpoint{2.112188in}{1.255985in}}%
\pgfpathlineto{\pgfqpoint{2.113920in}{0.965810in}}%
\pgfpathlineto{\pgfqpoint{2.116364in}{1.314222in}}%
\pgfpathlineto{\pgfqpoint{2.116976in}{1.022222in}}%
\pgfpathlineto{\pgfqpoint{2.119421in}{0.870834in}}%
\pgfpathlineto{\pgfqpoint{2.120337in}{1.295166in}}%
\pgfpathlineto{\pgfqpoint{2.122884in}{0.896670in}}%
\pgfpathlineto{\pgfqpoint{2.123699in}{1.347436in}}%
\pgfpathlineto{\pgfqpoint{2.125227in}{1.003103in}}%
\pgfpathlineto{\pgfqpoint{2.126857in}{1.244600in}}%
\pgfpathlineto{\pgfqpoint{2.128589in}{0.939673in}}%
\pgfpathlineto{\pgfqpoint{2.130321in}{1.260302in}}%
\pgfpathlineto{\pgfqpoint{2.132257in}{0.751201in}}%
\pgfpathlineto{\pgfqpoint{2.133377in}{1.231809in}}%
\pgfpathlineto{\pgfqpoint{2.136332in}{1.338443in}}%
\pgfpathlineto{\pgfqpoint{2.136637in}{0.905906in}}%
\pgfpathlineto{\pgfqpoint{2.139693in}{0.759560in}}%
\pgfpathlineto{\pgfqpoint{2.139795in}{1.325746in}}%
\pgfpathlineto{\pgfqpoint{2.141629in}{1.014791in}}%
\pgfpathlineto{\pgfqpoint{2.143157in}{1.229783in}}%
\pgfpathlineto{\pgfqpoint{2.145908in}{0.805175in}}%
\pgfpathlineto{\pgfqpoint{2.146926in}{1.346018in}}%
\pgfpathlineto{\pgfqpoint{2.147945in}{0.928898in}}%
\pgfpathlineto{\pgfqpoint{2.150899in}{1.292451in}}%
\pgfpathlineto{\pgfqpoint{2.151307in}{0.856104in}}%
\pgfpathlineto{\pgfqpoint{2.153039in}{1.196391in}}%
\pgfpathlineto{\pgfqpoint{2.154770in}{0.884155in}}%
\pgfpathlineto{\pgfqpoint{2.156095in}{1.204182in}}%
\pgfpathlineto{\pgfqpoint{2.158030in}{0.956671in}}%
\pgfpathlineto{\pgfqpoint{2.159457in}{1.238285in}}%
\pgfpathlineto{\pgfqpoint{2.160985in}{1.078808in}}%
\pgfpathlineto{\pgfqpoint{2.163837in}{0.913093in}}%
\pgfpathlineto{\pgfqpoint{2.164346in}{1.178966in}}%
\pgfpathlineto{\pgfqpoint{2.165976in}{1.434321in}}%
\pgfpathlineto{\pgfqpoint{2.167606in}{0.927484in}}%
\pgfpathlineto{\pgfqpoint{2.169338in}{1.283832in}}%
\pgfpathlineto{\pgfqpoint{2.171579in}{1.387613in}}%
\pgfpathlineto{\pgfqpoint{2.173311in}{0.928112in}}%
\pgfpathlineto{\pgfqpoint{2.174636in}{1.241291in}}%
\pgfpathlineto{\pgfqpoint{2.175756in}{0.969387in}}%
\pgfpathlineto{\pgfqpoint{2.178303in}{1.286883in}}%
\pgfpathlineto{\pgfqpoint{2.179220in}{0.923880in}}%
\pgfpathlineto{\pgfqpoint{2.180850in}{1.304638in}}%
\pgfpathlineto{\pgfqpoint{2.183499in}{0.847720in}}%
\pgfpathlineto{\pgfqpoint{2.184314in}{1.228462in}}%
\pgfpathlineto{\pgfqpoint{2.185434in}{0.942565in}}%
\pgfpathlineto{\pgfqpoint{2.187064in}{1.324347in}}%
\pgfpathlineto{\pgfqpoint{2.191445in}{0.718924in}}%
\pgfpathlineto{\pgfqpoint{2.192667in}{1.335377in}}%
\pgfpathlineto{\pgfqpoint{2.194399in}{0.872446in}}%
\pgfpathlineto{\pgfqpoint{2.196131in}{1.387911in}}%
\pgfpathlineto{\pgfqpoint{2.196946in}{1.009378in}}%
\pgfpathlineto{\pgfqpoint{2.199289in}{1.384724in}}%
\pgfpathlineto{\pgfqpoint{2.200104in}{0.962859in}}%
\pgfpathlineto{\pgfqpoint{2.201937in}{1.296430in}}%
\pgfpathlineto{\pgfqpoint{2.204892in}{0.876375in}}%
\pgfpathlineto{\pgfqpoint{2.205605in}{1.217612in}}%
\pgfpathlineto{\pgfqpoint{2.207744in}{1.382317in}}%
\pgfpathlineto{\pgfqpoint{2.208254in}{0.944173in}}%
\pgfpathlineto{\pgfqpoint{2.209884in}{1.262315in}}%
\pgfpathlineto{\pgfqpoint{2.211615in}{0.932434in}}%
\pgfpathlineto{\pgfqpoint{2.213245in}{1.224948in}}%
\pgfpathlineto{\pgfqpoint{2.215894in}{0.961701in}}%
\pgfpathlineto{\pgfqpoint{2.216607in}{1.210206in}}%
\pgfpathlineto{\pgfqpoint{2.219154in}{0.976510in}}%
\pgfpathlineto{\pgfqpoint{2.219663in}{1.277450in}}%
\pgfpathlineto{\pgfqpoint{2.221803in}{0.939272in}}%
\pgfpathlineto{\pgfqpoint{2.223636in}{1.272664in}}%
\pgfpathlineto{\pgfqpoint{2.225164in}{0.903345in}}%
\pgfpathlineto{\pgfqpoint{2.226183in}{1.240118in}}%
\pgfpathlineto{\pgfqpoint{2.228119in}{0.861836in}}%
\pgfpathlineto{\pgfqpoint{2.229443in}{1.308347in}}%
\pgfpathlineto{\pgfqpoint{2.231786in}{0.753482in}}%
\pgfpathlineto{\pgfqpoint{2.232703in}{1.257464in}}%
\pgfpathlineto{\pgfqpoint{2.235759in}{0.848016in}}%
\pgfpathlineto{\pgfqpoint{2.236269in}{1.328979in}}%
\pgfpathlineto{\pgfqpoint{2.238917in}{0.898957in}}%
\pgfpathlineto{\pgfqpoint{2.239732in}{1.315062in}}%
\pgfpathlineto{\pgfqpoint{2.240955in}{0.846200in}}%
\pgfpathlineto{\pgfqpoint{2.243094in}{1.264002in}}%
\pgfpathlineto{\pgfqpoint{2.245335in}{0.955126in}}%
\pgfpathlineto{\pgfqpoint{2.246048in}{1.226688in}}%
\pgfpathlineto{\pgfqpoint{2.248901in}{1.247495in}}%
\pgfpathlineto{\pgfqpoint{2.249308in}{0.802468in}}%
\pgfpathlineto{\pgfqpoint{2.250938in}{1.259265in}}%
\pgfpathlineto{\pgfqpoint{2.253383in}{0.969887in}}%
\pgfpathlineto{\pgfqpoint{2.253994in}{1.283535in}}%
\pgfpathlineto{\pgfqpoint{2.255624in}{0.858164in}}%
\pgfpathlineto{\pgfqpoint{2.257560in}{1.287204in}}%
\pgfpathlineto{\pgfqpoint{2.258986in}{0.938163in}}%
\pgfpathlineto{\pgfqpoint{2.260820in}{1.575662in}}%
\pgfpathlineto{\pgfqpoint{2.262857in}{0.890602in}}%
\pgfpathlineto{\pgfqpoint{2.265200in}{1.311346in}}%
\pgfpathlineto{\pgfqpoint{2.265913in}{0.966817in}}%
\pgfpathlineto{\pgfqpoint{2.267238in}{1.246900in}}%
\pgfpathlineto{\pgfqpoint{2.269683in}{0.978604in}}%
\pgfpathlineto{\pgfqpoint{2.271618in}{1.299003in}}%
\pgfpathlineto{\pgfqpoint{2.272943in}{0.936623in}}%
\pgfpathlineto{\pgfqpoint{2.274063in}{1.392970in}}%
\pgfpathlineto{\pgfqpoint{2.275490in}{0.973696in}}%
\pgfpathlineto{\pgfqpoint{2.277323in}{1.294744in}}%
\pgfpathlineto{\pgfqpoint{2.278444in}{0.981444in}}%
\pgfpathlineto{\pgfqpoint{2.280176in}{1.235322in}}%
\pgfpathlineto{\pgfqpoint{2.281806in}{0.942472in}}%
\pgfpathlineto{\pgfqpoint{2.283334in}{1.217938in}}%
\pgfpathlineto{\pgfqpoint{2.285881in}{0.911925in}}%
\pgfpathlineto{\pgfqpoint{2.286696in}{1.364267in}}%
\pgfpathlineto{\pgfqpoint{2.289140in}{0.974800in}}%
\pgfpathlineto{\pgfqpoint{2.289854in}{1.254263in}}%
\pgfpathlineto{\pgfqpoint{2.293623in}{0.808827in}}%
\pgfpathlineto{\pgfqpoint{2.294845in}{1.284981in}}%
\pgfpathlineto{\pgfqpoint{2.296272in}{0.943505in}}%
\pgfpathlineto{\pgfqpoint{2.298513in}{1.303539in}}%
\pgfpathlineto{\pgfqpoint{2.300754in}{0.945444in}}%
\pgfpathlineto{\pgfqpoint{2.301263in}{1.314852in}}%
\pgfpathlineto{\pgfqpoint{2.302995in}{0.831377in}}%
\pgfpathlineto{\pgfqpoint{2.305134in}{1.263294in}}%
\pgfpathlineto{\pgfqpoint{2.306764in}{0.949976in}}%
\pgfpathlineto{\pgfqpoint{2.308191in}{1.390829in}}%
\pgfpathlineto{\pgfqpoint{2.309515in}{0.971151in}}%
\pgfpathlineto{\pgfqpoint{2.311552in}{1.249715in}}%
\pgfpathlineto{\pgfqpoint{2.312775in}{0.886978in}}%
\pgfpathlineto{\pgfqpoint{2.315118in}{1.275331in}}%
\pgfpathlineto{\pgfqpoint{2.315831in}{1.131360in}}%
\pgfpathlineto{\pgfqpoint{2.315831in}{1.131360in}}%
\pgfusepath{stroke}%
\end{pgfscope}%
\begin{pgfscope}%
\pgfsetrectcap%
\pgfsetmiterjoin%
\pgfsetlinewidth{0.803000pt}%
\definecolor{currentstroke}{rgb}{0.000000,0.000000,0.000000}%
\pgfsetstrokecolor{currentstroke}%
\pgfsetdash{}{0pt}%
\pgfpathmoveto{\pgfqpoint{0.563510in}{0.416447in}}%
\pgfpathlineto{\pgfqpoint{0.563510in}{1.789039in}}%
\pgfusepath{stroke}%
\end{pgfscope}%
\begin{pgfscope}%
\pgfsetrectcap%
\pgfsetmiterjoin%
\pgfsetlinewidth{0.803000pt}%
\definecolor{currentstroke}{rgb}{0.000000,0.000000,0.000000}%
\pgfsetstrokecolor{currentstroke}%
\pgfsetdash{}{0pt}%
\pgfpathmoveto{\pgfqpoint{2.399275in}{0.416447in}}%
\pgfpathlineto{\pgfqpoint{2.399275in}{1.789039in}}%
\pgfusepath{stroke}%
\end{pgfscope}%
\begin{pgfscope}%
\pgfsetrectcap%
\pgfsetmiterjoin%
\pgfsetlinewidth{0.803000pt}%
\definecolor{currentstroke}{rgb}{0.000000,0.000000,0.000000}%
\pgfsetstrokecolor{currentstroke}%
\pgfsetdash{}{0pt}%
\pgfpathmoveto{\pgfqpoint{0.563510in}{0.416447in}}%
\pgfpathlineto{\pgfqpoint{2.399275in}{0.416447in}}%
\pgfusepath{stroke}%
\end{pgfscope}%
\begin{pgfscope}%
\pgfsetrectcap%
\pgfsetmiterjoin%
\pgfsetlinewidth{0.803000pt}%
\definecolor{currentstroke}{rgb}{0.000000,0.000000,0.000000}%
\pgfsetstrokecolor{currentstroke}%
\pgfsetdash{}{0pt}%
\pgfpathmoveto{\pgfqpoint{0.563510in}{1.789039in}}%
\pgfpathlineto{\pgfqpoint{2.399275in}{1.789039in}}%
\pgfusepath{stroke}%
\end{pgfscope}%
\begin{pgfscope}%
\pgfsetbuttcap%
\pgfsetmiterjoin%
\definecolor{currentfill}{rgb}{1.000000,1.000000,1.000000}%
\pgfsetfillcolor{currentfill}%
\pgfsetfillopacity{0.800000}%
\pgfsetlinewidth{1.003750pt}%
\definecolor{currentstroke}{rgb}{0.800000,0.800000,0.800000}%
\pgfsetstrokecolor{currentstroke}%
\pgfsetstrokeopacity{0.800000}%
\pgfsetdash{}{0pt}%
\pgfpathmoveto{\pgfqpoint{0.641288in}{1.545261in}}%
\pgfpathlineto{\pgfqpoint{1.610399in}{1.545261in}}%
\pgfpathquadraticcurveto{\pgfqpoint{1.632621in}{1.545261in}}{\pgfqpoint{1.632621in}{1.567483in}}%
\pgfpathlineto{\pgfqpoint{1.632621in}{1.711261in}}%
\pgfpathquadraticcurveto{\pgfqpoint{1.632621in}{1.733483in}}{\pgfqpoint{1.610399in}{1.733483in}}%
\pgfpathlineto{\pgfqpoint{0.641288in}{1.733483in}}%
\pgfpathquadraticcurveto{\pgfqpoint{0.619065in}{1.733483in}}{\pgfqpoint{0.619065in}{1.711261in}}%
\pgfpathlineto{\pgfqpoint{0.619065in}{1.567483in}}%
\pgfpathquadraticcurveto{\pgfqpoint{0.619065in}{1.545261in}}{\pgfqpoint{0.641288in}{1.545261in}}%
\pgfpathlineto{\pgfqpoint{0.641288in}{1.545261in}}%
\pgfpathclose%
\pgfusepath{stroke,fill}%
\end{pgfscope}%
\begin{pgfscope}%
\pgfsetrectcap%
\pgfsetroundjoin%
\pgfsetlinewidth{1.505625pt}%
\definecolor{currentstroke}{rgb}{0.000000,0.447059,0.698039}%
\pgfsetstrokecolor{currentstroke}%
\pgfsetdash{}{0pt}%
\pgfpathmoveto{\pgfqpoint{0.663510in}{1.650150in}}%
\pgfpathlineto{\pgfqpoint{0.774621in}{1.650150in}}%
\pgfpathlineto{\pgfqpoint{0.885732in}{1.650150in}}%
\pgfusepath{stroke}%
\end{pgfscope}%
\begin{pgfscope}%
\definecolor{textcolor}{rgb}{0.000000,0.000000,0.000000}%
\pgfsetstrokecolor{textcolor}%
\pgfsetfillcolor{textcolor}%
\pgftext[x=0.974621in,y=1.611261in,left,base]{\color{textcolor}\rmfamily\fontsize{8.000000}{9.600000}\selectfont White noise}%
\end{pgfscope}%
\end{pgfpicture}%
\makeatother%
\endgroup%
% data/simulations/sim_allan_variance.py
        } % scalebox
        \caption{Time domain}
        \label{fig:white_noise_time}
    \end{subfigure}
    \hfill
    \begin{subfigure}{0.32\linewidth}
        \centering
        \scalebox{0.75}{%
            %% Creator: Matplotlib, PGF backend
%%
%% To include the figure in your LaTeX document, write
%%   \input{<filename>.pgf}
%%
%% Make sure the required packages are loaded in your preamble
%%   \usepackage{pgf}
%%
%% Also ensure that all the required font packages are loaded; for instance,
%% the lmodern package is sometimes necessary when using math font.
%%   \usepackage{lmodern}
%%
%% Figures using additional raster images can only be included by \input if
%% they are in the same directory as the main LaTeX file. For loading figures
%% from other directories you can use the `import` package
%%   \usepackage{import}
%%
%% and then include the figures with
%%   \import{<path to file>}{<filename>.pgf}
%%
%% Matplotlib used the following preamble
%%   \def\mathdefault#1{#1}
%%   \everymath=\expandafter{\the\everymath\displaystyle}
%%   \usepackage{siunitx}
%%   \sisetup{per-mode = symbol}%
%%   \ifdefined\pdftexversion\else  % non-pdftex case.
%%     \usepackage{fontspec}
%%   \fi
%%   \makeatletter\@ifpackageloaded{underscore}{}{\usepackage[strings]{underscore}}\makeatother
%%
\begingroup%
\makeatletter%
\begin{pgfpicture}%
\pgfpathrectangle{\pgfpointorigin}{\pgfqpoint{2.440945in}{1.830709in}}%
\pgfusepath{use as bounding box, clip}%
\begin{pgfscope}%
\pgfsetbuttcap%
\pgfsetmiterjoin%
\definecolor{currentfill}{rgb}{1.000000,1.000000,1.000000}%
\pgfsetfillcolor{currentfill}%
\pgfsetlinewidth{0.000000pt}%
\definecolor{currentstroke}{rgb}{1.000000,1.000000,1.000000}%
\pgfsetstrokecolor{currentstroke}%
\pgfsetdash{}{0pt}%
\pgfpathmoveto{\pgfqpoint{0.000000in}{0.000000in}}%
\pgfpathlineto{\pgfqpoint{2.440945in}{0.000000in}}%
\pgfpathlineto{\pgfqpoint{2.440945in}{1.830709in}}%
\pgfpathlineto{\pgfqpoint{0.000000in}{1.830709in}}%
\pgfpathlineto{\pgfqpoint{0.000000in}{0.000000in}}%
\pgfpathclose%
\pgfusepath{fill}%
\end{pgfscope}%
\begin{pgfscope}%
\pgfsetbuttcap%
\pgfsetmiterjoin%
\definecolor{currentfill}{rgb}{1.000000,1.000000,1.000000}%
\pgfsetfillcolor{currentfill}%
\pgfsetlinewidth{0.000000pt}%
\definecolor{currentstroke}{rgb}{0.000000,0.000000,0.000000}%
\pgfsetstrokecolor{currentstroke}%
\pgfsetstrokeopacity{0.000000}%
\pgfsetdash{}{0pt}%
\pgfpathmoveto{\pgfqpoint{0.514278in}{0.417642in}}%
\pgfpathlineto{\pgfqpoint{2.399275in}{0.417642in}}%
\pgfpathlineto{\pgfqpoint{2.399275in}{1.789039in}}%
\pgfpathlineto{\pgfqpoint{0.514278in}{1.789039in}}%
\pgfpathlineto{\pgfqpoint{0.514278in}{0.417642in}}%
\pgfpathclose%
\pgfusepath{fill}%
\end{pgfscope}%
\begin{pgfscope}%
\pgfpathrectangle{\pgfqpoint{0.514278in}{0.417642in}}{\pgfqpoint{1.884996in}{1.371397in}}%
\pgfusepath{clip}%
\pgfsetrectcap%
\pgfsetroundjoin%
\pgfsetlinewidth{0.803000pt}%
\definecolor{currentstroke}{rgb}{0.450000,0.450000,0.450000}%
\pgfsetstrokecolor{currentstroke}%
\pgfsetdash{}{0pt}%
\pgfpathmoveto{\pgfqpoint{0.916836in}{0.417642in}}%
\pgfpathlineto{\pgfqpoint{0.916836in}{1.789039in}}%
\pgfusepath{stroke}%
\end{pgfscope}%
\begin{pgfscope}%
\pgfsetbuttcap%
\pgfsetroundjoin%
\definecolor{currentfill}{rgb}{0.000000,0.000000,0.000000}%
\pgfsetfillcolor{currentfill}%
\pgfsetlinewidth{0.803000pt}%
\definecolor{currentstroke}{rgb}{0.000000,0.000000,0.000000}%
\pgfsetstrokecolor{currentstroke}%
\pgfsetdash{}{0pt}%
\pgfsys@defobject{currentmarker}{\pgfqpoint{0.000000in}{-0.048611in}}{\pgfqpoint{0.000000in}{0.000000in}}{%
\pgfpathmoveto{\pgfqpoint{0.000000in}{0.000000in}}%
\pgfpathlineto{\pgfqpoint{0.000000in}{-0.048611in}}%
\pgfusepath{stroke,fill}%
}%
\begin{pgfscope}%
\pgfsys@transformshift{0.916836in}{0.417642in}%
\pgfsys@useobject{currentmarker}{}%
\end{pgfscope}%
\end{pgfscope}%
\begin{pgfscope}%
\definecolor{textcolor}{rgb}{0.000000,0.000000,0.000000}%
\pgfsetstrokecolor{textcolor}%
\pgfsetfillcolor{textcolor}%
\pgftext[x=0.916836in,y=0.320420in,,top]{\color{textcolor}{\rmfamily\fontsize{8.000000}{9.600000}\selectfont\catcode`\^=\active\def^{\ifmmode\sp\else\^{}\fi}\catcode`\%=\active\def%{\%}$\mathdefault{10^{-3}}$}}%
\end{pgfscope}%
\begin{pgfscope}%
\pgfpathrectangle{\pgfqpoint{0.514278in}{0.417642in}}{\pgfqpoint{1.884996in}{1.371397in}}%
\pgfusepath{clip}%
\pgfsetrectcap%
\pgfsetroundjoin%
\pgfsetlinewidth{0.803000pt}%
\definecolor{currentstroke}{rgb}{0.450000,0.450000,0.450000}%
\pgfsetstrokecolor{currentstroke}%
\pgfsetdash{}{0pt}%
\pgfpathmoveto{\pgfqpoint{1.434391in}{0.417642in}}%
\pgfpathlineto{\pgfqpoint{1.434391in}{1.789039in}}%
\pgfusepath{stroke}%
\end{pgfscope}%
\begin{pgfscope}%
\pgfsetbuttcap%
\pgfsetroundjoin%
\definecolor{currentfill}{rgb}{0.000000,0.000000,0.000000}%
\pgfsetfillcolor{currentfill}%
\pgfsetlinewidth{0.803000pt}%
\definecolor{currentstroke}{rgb}{0.000000,0.000000,0.000000}%
\pgfsetstrokecolor{currentstroke}%
\pgfsetdash{}{0pt}%
\pgfsys@defobject{currentmarker}{\pgfqpoint{0.000000in}{-0.048611in}}{\pgfqpoint{0.000000in}{0.000000in}}{%
\pgfpathmoveto{\pgfqpoint{0.000000in}{0.000000in}}%
\pgfpathlineto{\pgfqpoint{0.000000in}{-0.048611in}}%
\pgfusepath{stroke,fill}%
}%
\begin{pgfscope}%
\pgfsys@transformshift{1.434391in}{0.417642in}%
\pgfsys@useobject{currentmarker}{}%
\end{pgfscope}%
\end{pgfscope}%
\begin{pgfscope}%
\definecolor{textcolor}{rgb}{0.000000,0.000000,0.000000}%
\pgfsetstrokecolor{textcolor}%
\pgfsetfillcolor{textcolor}%
\pgftext[x=1.434391in,y=0.320420in,,top]{\color{textcolor}{\rmfamily\fontsize{8.000000}{9.600000}\selectfont\catcode`\^=\active\def^{\ifmmode\sp\else\^{}\fi}\catcode`\%=\active\def%{\%}$\mathdefault{10^{-2}}$}}%
\end{pgfscope}%
\begin{pgfscope}%
\pgfpathrectangle{\pgfqpoint{0.514278in}{0.417642in}}{\pgfqpoint{1.884996in}{1.371397in}}%
\pgfusepath{clip}%
\pgfsetrectcap%
\pgfsetroundjoin%
\pgfsetlinewidth{0.803000pt}%
\definecolor{currentstroke}{rgb}{0.450000,0.450000,0.450000}%
\pgfsetstrokecolor{currentstroke}%
\pgfsetdash{}{0pt}%
\pgfpathmoveto{\pgfqpoint{1.951947in}{0.417642in}}%
\pgfpathlineto{\pgfqpoint{1.951947in}{1.789039in}}%
\pgfusepath{stroke}%
\end{pgfscope}%
\begin{pgfscope}%
\pgfsetbuttcap%
\pgfsetroundjoin%
\definecolor{currentfill}{rgb}{0.000000,0.000000,0.000000}%
\pgfsetfillcolor{currentfill}%
\pgfsetlinewidth{0.803000pt}%
\definecolor{currentstroke}{rgb}{0.000000,0.000000,0.000000}%
\pgfsetstrokecolor{currentstroke}%
\pgfsetdash{}{0pt}%
\pgfsys@defobject{currentmarker}{\pgfqpoint{0.000000in}{-0.048611in}}{\pgfqpoint{0.000000in}{0.000000in}}{%
\pgfpathmoveto{\pgfqpoint{0.000000in}{0.000000in}}%
\pgfpathlineto{\pgfqpoint{0.000000in}{-0.048611in}}%
\pgfusepath{stroke,fill}%
}%
\begin{pgfscope}%
\pgfsys@transformshift{1.951947in}{0.417642in}%
\pgfsys@useobject{currentmarker}{}%
\end{pgfscope}%
\end{pgfscope}%
\begin{pgfscope}%
\definecolor{textcolor}{rgb}{0.000000,0.000000,0.000000}%
\pgfsetstrokecolor{textcolor}%
\pgfsetfillcolor{textcolor}%
\pgftext[x=1.951947in,y=0.320420in,,top]{\color{textcolor}{\rmfamily\fontsize{8.000000}{9.600000}\selectfont\catcode`\^=\active\def^{\ifmmode\sp\else\^{}\fi}\catcode`\%=\active\def%{\%}$\mathdefault{10^{-1}}$}}%
\end{pgfscope}%
\begin{pgfscope}%
\pgfpathrectangle{\pgfqpoint{0.514278in}{0.417642in}}{\pgfqpoint{1.884996in}{1.371397in}}%
\pgfusepath{clip}%
\pgfsetrectcap%
\pgfsetroundjoin%
\pgfsetlinewidth{0.803000pt}%
\definecolor{currentstroke}{rgb}{0.850000,0.850000,0.850000}%
\pgfsetstrokecolor{currentstroke}%
\pgfsetdash{}{0pt}%
\pgfpathmoveto{\pgfqpoint{0.555080in}{0.417642in}}%
\pgfpathlineto{\pgfqpoint{0.555080in}{1.789039in}}%
\pgfusepath{stroke}%
\end{pgfscope}%
\begin{pgfscope}%
\pgfsetbuttcap%
\pgfsetroundjoin%
\definecolor{currentfill}{rgb}{0.000000,0.000000,0.000000}%
\pgfsetfillcolor{currentfill}%
\pgfsetlinewidth{0.602250pt}%
\definecolor{currentstroke}{rgb}{0.000000,0.000000,0.000000}%
\pgfsetstrokecolor{currentstroke}%
\pgfsetdash{}{0pt}%
\pgfsys@defobject{currentmarker}{\pgfqpoint{0.000000in}{-0.027778in}}{\pgfqpoint{0.000000in}{0.000000in}}{%
\pgfpathmoveto{\pgfqpoint{0.000000in}{0.000000in}}%
\pgfpathlineto{\pgfqpoint{0.000000in}{-0.027778in}}%
\pgfusepath{stroke,fill}%
}%
\begin{pgfscope}%
\pgfsys@transformshift{0.555080in}{0.417642in}%
\pgfsys@useobject{currentmarker}{}%
\end{pgfscope}%
\end{pgfscope}%
\begin{pgfscope}%
\pgfpathrectangle{\pgfqpoint{0.514278in}{0.417642in}}{\pgfqpoint{1.884996in}{1.371397in}}%
\pgfusepath{clip}%
\pgfsetrectcap%
\pgfsetroundjoin%
\pgfsetlinewidth{0.803000pt}%
\definecolor{currentstroke}{rgb}{0.850000,0.850000,0.850000}%
\pgfsetstrokecolor{currentstroke}%
\pgfsetdash{}{0pt}%
\pgfpathmoveto{\pgfqpoint{0.646217in}{0.417642in}}%
\pgfpathlineto{\pgfqpoint{0.646217in}{1.789039in}}%
\pgfusepath{stroke}%
\end{pgfscope}%
\begin{pgfscope}%
\pgfsetbuttcap%
\pgfsetroundjoin%
\definecolor{currentfill}{rgb}{0.000000,0.000000,0.000000}%
\pgfsetfillcolor{currentfill}%
\pgfsetlinewidth{0.602250pt}%
\definecolor{currentstroke}{rgb}{0.000000,0.000000,0.000000}%
\pgfsetstrokecolor{currentstroke}%
\pgfsetdash{}{0pt}%
\pgfsys@defobject{currentmarker}{\pgfqpoint{0.000000in}{-0.027778in}}{\pgfqpoint{0.000000in}{0.000000in}}{%
\pgfpathmoveto{\pgfqpoint{0.000000in}{0.000000in}}%
\pgfpathlineto{\pgfqpoint{0.000000in}{-0.027778in}}%
\pgfusepath{stroke,fill}%
}%
\begin{pgfscope}%
\pgfsys@transformshift{0.646217in}{0.417642in}%
\pgfsys@useobject{currentmarker}{}%
\end{pgfscope}%
\end{pgfscope}%
\begin{pgfscope}%
\pgfpathrectangle{\pgfqpoint{0.514278in}{0.417642in}}{\pgfqpoint{1.884996in}{1.371397in}}%
\pgfusepath{clip}%
\pgfsetrectcap%
\pgfsetroundjoin%
\pgfsetlinewidth{0.803000pt}%
\definecolor{currentstroke}{rgb}{0.850000,0.850000,0.850000}%
\pgfsetstrokecolor{currentstroke}%
\pgfsetdash{}{0pt}%
\pgfpathmoveto{\pgfqpoint{0.710880in}{0.417642in}}%
\pgfpathlineto{\pgfqpoint{0.710880in}{1.789039in}}%
\pgfusepath{stroke}%
\end{pgfscope}%
\begin{pgfscope}%
\pgfsetbuttcap%
\pgfsetroundjoin%
\definecolor{currentfill}{rgb}{0.000000,0.000000,0.000000}%
\pgfsetfillcolor{currentfill}%
\pgfsetlinewidth{0.602250pt}%
\definecolor{currentstroke}{rgb}{0.000000,0.000000,0.000000}%
\pgfsetstrokecolor{currentstroke}%
\pgfsetdash{}{0pt}%
\pgfsys@defobject{currentmarker}{\pgfqpoint{0.000000in}{-0.027778in}}{\pgfqpoint{0.000000in}{0.000000in}}{%
\pgfpathmoveto{\pgfqpoint{0.000000in}{0.000000in}}%
\pgfpathlineto{\pgfqpoint{0.000000in}{-0.027778in}}%
\pgfusepath{stroke,fill}%
}%
\begin{pgfscope}%
\pgfsys@transformshift{0.710880in}{0.417642in}%
\pgfsys@useobject{currentmarker}{}%
\end{pgfscope}%
\end{pgfscope}%
\begin{pgfscope}%
\pgfpathrectangle{\pgfqpoint{0.514278in}{0.417642in}}{\pgfqpoint{1.884996in}{1.371397in}}%
\pgfusepath{clip}%
\pgfsetrectcap%
\pgfsetroundjoin%
\pgfsetlinewidth{0.803000pt}%
\definecolor{currentstroke}{rgb}{0.850000,0.850000,0.850000}%
\pgfsetstrokecolor{currentstroke}%
\pgfsetdash{}{0pt}%
\pgfpathmoveto{\pgfqpoint{0.761036in}{0.417642in}}%
\pgfpathlineto{\pgfqpoint{0.761036in}{1.789039in}}%
\pgfusepath{stroke}%
\end{pgfscope}%
\begin{pgfscope}%
\pgfsetbuttcap%
\pgfsetroundjoin%
\definecolor{currentfill}{rgb}{0.000000,0.000000,0.000000}%
\pgfsetfillcolor{currentfill}%
\pgfsetlinewidth{0.602250pt}%
\definecolor{currentstroke}{rgb}{0.000000,0.000000,0.000000}%
\pgfsetstrokecolor{currentstroke}%
\pgfsetdash{}{0pt}%
\pgfsys@defobject{currentmarker}{\pgfqpoint{0.000000in}{-0.027778in}}{\pgfqpoint{0.000000in}{0.000000in}}{%
\pgfpathmoveto{\pgfqpoint{0.000000in}{0.000000in}}%
\pgfpathlineto{\pgfqpoint{0.000000in}{-0.027778in}}%
\pgfusepath{stroke,fill}%
}%
\begin{pgfscope}%
\pgfsys@transformshift{0.761036in}{0.417642in}%
\pgfsys@useobject{currentmarker}{}%
\end{pgfscope}%
\end{pgfscope}%
\begin{pgfscope}%
\pgfpathrectangle{\pgfqpoint{0.514278in}{0.417642in}}{\pgfqpoint{1.884996in}{1.371397in}}%
\pgfusepath{clip}%
\pgfsetrectcap%
\pgfsetroundjoin%
\pgfsetlinewidth{0.803000pt}%
\definecolor{currentstroke}{rgb}{0.850000,0.850000,0.850000}%
\pgfsetstrokecolor{currentstroke}%
\pgfsetdash{}{0pt}%
\pgfpathmoveto{\pgfqpoint{0.802017in}{0.417642in}}%
\pgfpathlineto{\pgfqpoint{0.802017in}{1.789039in}}%
\pgfusepath{stroke}%
\end{pgfscope}%
\begin{pgfscope}%
\pgfsetbuttcap%
\pgfsetroundjoin%
\definecolor{currentfill}{rgb}{0.000000,0.000000,0.000000}%
\pgfsetfillcolor{currentfill}%
\pgfsetlinewidth{0.602250pt}%
\definecolor{currentstroke}{rgb}{0.000000,0.000000,0.000000}%
\pgfsetstrokecolor{currentstroke}%
\pgfsetdash{}{0pt}%
\pgfsys@defobject{currentmarker}{\pgfqpoint{0.000000in}{-0.027778in}}{\pgfqpoint{0.000000in}{0.000000in}}{%
\pgfpathmoveto{\pgfqpoint{0.000000in}{0.000000in}}%
\pgfpathlineto{\pgfqpoint{0.000000in}{-0.027778in}}%
\pgfusepath{stroke,fill}%
}%
\begin{pgfscope}%
\pgfsys@transformshift{0.802017in}{0.417642in}%
\pgfsys@useobject{currentmarker}{}%
\end{pgfscope}%
\end{pgfscope}%
\begin{pgfscope}%
\pgfpathrectangle{\pgfqpoint{0.514278in}{0.417642in}}{\pgfqpoint{1.884996in}{1.371397in}}%
\pgfusepath{clip}%
\pgfsetrectcap%
\pgfsetroundjoin%
\pgfsetlinewidth{0.803000pt}%
\definecolor{currentstroke}{rgb}{0.850000,0.850000,0.850000}%
\pgfsetstrokecolor{currentstroke}%
\pgfsetdash{}{0pt}%
\pgfpathmoveto{\pgfqpoint{0.836665in}{0.417642in}}%
\pgfpathlineto{\pgfqpoint{0.836665in}{1.789039in}}%
\pgfusepath{stroke}%
\end{pgfscope}%
\begin{pgfscope}%
\pgfsetbuttcap%
\pgfsetroundjoin%
\definecolor{currentfill}{rgb}{0.000000,0.000000,0.000000}%
\pgfsetfillcolor{currentfill}%
\pgfsetlinewidth{0.602250pt}%
\definecolor{currentstroke}{rgb}{0.000000,0.000000,0.000000}%
\pgfsetstrokecolor{currentstroke}%
\pgfsetdash{}{0pt}%
\pgfsys@defobject{currentmarker}{\pgfqpoint{0.000000in}{-0.027778in}}{\pgfqpoint{0.000000in}{0.000000in}}{%
\pgfpathmoveto{\pgfqpoint{0.000000in}{0.000000in}}%
\pgfpathlineto{\pgfqpoint{0.000000in}{-0.027778in}}%
\pgfusepath{stroke,fill}%
}%
\begin{pgfscope}%
\pgfsys@transformshift{0.836665in}{0.417642in}%
\pgfsys@useobject{currentmarker}{}%
\end{pgfscope}%
\end{pgfscope}%
\begin{pgfscope}%
\pgfpathrectangle{\pgfqpoint{0.514278in}{0.417642in}}{\pgfqpoint{1.884996in}{1.371397in}}%
\pgfusepath{clip}%
\pgfsetrectcap%
\pgfsetroundjoin%
\pgfsetlinewidth{0.803000pt}%
\definecolor{currentstroke}{rgb}{0.850000,0.850000,0.850000}%
\pgfsetstrokecolor{currentstroke}%
\pgfsetdash{}{0pt}%
\pgfpathmoveto{\pgfqpoint{0.866679in}{0.417642in}}%
\pgfpathlineto{\pgfqpoint{0.866679in}{1.789039in}}%
\pgfusepath{stroke}%
\end{pgfscope}%
\begin{pgfscope}%
\pgfsetbuttcap%
\pgfsetroundjoin%
\definecolor{currentfill}{rgb}{0.000000,0.000000,0.000000}%
\pgfsetfillcolor{currentfill}%
\pgfsetlinewidth{0.602250pt}%
\definecolor{currentstroke}{rgb}{0.000000,0.000000,0.000000}%
\pgfsetstrokecolor{currentstroke}%
\pgfsetdash{}{0pt}%
\pgfsys@defobject{currentmarker}{\pgfqpoint{0.000000in}{-0.027778in}}{\pgfqpoint{0.000000in}{0.000000in}}{%
\pgfpathmoveto{\pgfqpoint{0.000000in}{0.000000in}}%
\pgfpathlineto{\pgfqpoint{0.000000in}{-0.027778in}}%
\pgfusepath{stroke,fill}%
}%
\begin{pgfscope}%
\pgfsys@transformshift{0.866679in}{0.417642in}%
\pgfsys@useobject{currentmarker}{}%
\end{pgfscope}%
\end{pgfscope}%
\begin{pgfscope}%
\pgfpathrectangle{\pgfqpoint{0.514278in}{0.417642in}}{\pgfqpoint{1.884996in}{1.371397in}}%
\pgfusepath{clip}%
\pgfsetrectcap%
\pgfsetroundjoin%
\pgfsetlinewidth{0.803000pt}%
\definecolor{currentstroke}{rgb}{0.850000,0.850000,0.850000}%
\pgfsetstrokecolor{currentstroke}%
\pgfsetdash{}{0pt}%
\pgfpathmoveto{\pgfqpoint{0.893154in}{0.417642in}}%
\pgfpathlineto{\pgfqpoint{0.893154in}{1.789039in}}%
\pgfusepath{stroke}%
\end{pgfscope}%
\begin{pgfscope}%
\pgfsetbuttcap%
\pgfsetroundjoin%
\definecolor{currentfill}{rgb}{0.000000,0.000000,0.000000}%
\pgfsetfillcolor{currentfill}%
\pgfsetlinewidth{0.602250pt}%
\definecolor{currentstroke}{rgb}{0.000000,0.000000,0.000000}%
\pgfsetstrokecolor{currentstroke}%
\pgfsetdash{}{0pt}%
\pgfsys@defobject{currentmarker}{\pgfqpoint{0.000000in}{-0.027778in}}{\pgfqpoint{0.000000in}{0.000000in}}{%
\pgfpathmoveto{\pgfqpoint{0.000000in}{0.000000in}}%
\pgfpathlineto{\pgfqpoint{0.000000in}{-0.027778in}}%
\pgfusepath{stroke,fill}%
}%
\begin{pgfscope}%
\pgfsys@transformshift{0.893154in}{0.417642in}%
\pgfsys@useobject{currentmarker}{}%
\end{pgfscope}%
\end{pgfscope}%
\begin{pgfscope}%
\pgfpathrectangle{\pgfqpoint{0.514278in}{0.417642in}}{\pgfqpoint{1.884996in}{1.371397in}}%
\pgfusepath{clip}%
\pgfsetrectcap%
\pgfsetroundjoin%
\pgfsetlinewidth{0.803000pt}%
\definecolor{currentstroke}{rgb}{0.850000,0.850000,0.850000}%
\pgfsetstrokecolor{currentstroke}%
\pgfsetdash{}{0pt}%
\pgfpathmoveto{\pgfqpoint{1.072635in}{0.417642in}}%
\pgfpathlineto{\pgfqpoint{1.072635in}{1.789039in}}%
\pgfusepath{stroke}%
\end{pgfscope}%
\begin{pgfscope}%
\pgfsetbuttcap%
\pgfsetroundjoin%
\definecolor{currentfill}{rgb}{0.000000,0.000000,0.000000}%
\pgfsetfillcolor{currentfill}%
\pgfsetlinewidth{0.602250pt}%
\definecolor{currentstroke}{rgb}{0.000000,0.000000,0.000000}%
\pgfsetstrokecolor{currentstroke}%
\pgfsetdash{}{0pt}%
\pgfsys@defobject{currentmarker}{\pgfqpoint{0.000000in}{-0.027778in}}{\pgfqpoint{0.000000in}{0.000000in}}{%
\pgfpathmoveto{\pgfqpoint{0.000000in}{0.000000in}}%
\pgfpathlineto{\pgfqpoint{0.000000in}{-0.027778in}}%
\pgfusepath{stroke,fill}%
}%
\begin{pgfscope}%
\pgfsys@transformshift{1.072635in}{0.417642in}%
\pgfsys@useobject{currentmarker}{}%
\end{pgfscope}%
\end{pgfscope}%
\begin{pgfscope}%
\pgfpathrectangle{\pgfqpoint{0.514278in}{0.417642in}}{\pgfqpoint{1.884996in}{1.371397in}}%
\pgfusepath{clip}%
\pgfsetrectcap%
\pgfsetroundjoin%
\pgfsetlinewidth{0.803000pt}%
\definecolor{currentstroke}{rgb}{0.850000,0.850000,0.850000}%
\pgfsetstrokecolor{currentstroke}%
\pgfsetdash{}{0pt}%
\pgfpathmoveto{\pgfqpoint{1.163773in}{0.417642in}}%
\pgfpathlineto{\pgfqpoint{1.163773in}{1.789039in}}%
\pgfusepath{stroke}%
\end{pgfscope}%
\begin{pgfscope}%
\pgfsetbuttcap%
\pgfsetroundjoin%
\definecolor{currentfill}{rgb}{0.000000,0.000000,0.000000}%
\pgfsetfillcolor{currentfill}%
\pgfsetlinewidth{0.602250pt}%
\definecolor{currentstroke}{rgb}{0.000000,0.000000,0.000000}%
\pgfsetstrokecolor{currentstroke}%
\pgfsetdash{}{0pt}%
\pgfsys@defobject{currentmarker}{\pgfqpoint{0.000000in}{-0.027778in}}{\pgfqpoint{0.000000in}{0.000000in}}{%
\pgfpathmoveto{\pgfqpoint{0.000000in}{0.000000in}}%
\pgfpathlineto{\pgfqpoint{0.000000in}{-0.027778in}}%
\pgfusepath{stroke,fill}%
}%
\begin{pgfscope}%
\pgfsys@transformshift{1.163773in}{0.417642in}%
\pgfsys@useobject{currentmarker}{}%
\end{pgfscope}%
\end{pgfscope}%
\begin{pgfscope}%
\pgfpathrectangle{\pgfqpoint{0.514278in}{0.417642in}}{\pgfqpoint{1.884996in}{1.371397in}}%
\pgfusepath{clip}%
\pgfsetrectcap%
\pgfsetroundjoin%
\pgfsetlinewidth{0.803000pt}%
\definecolor{currentstroke}{rgb}{0.850000,0.850000,0.850000}%
\pgfsetstrokecolor{currentstroke}%
\pgfsetdash{}{0pt}%
\pgfpathmoveto{\pgfqpoint{1.228435in}{0.417642in}}%
\pgfpathlineto{\pgfqpoint{1.228435in}{1.789039in}}%
\pgfusepath{stroke}%
\end{pgfscope}%
\begin{pgfscope}%
\pgfsetbuttcap%
\pgfsetroundjoin%
\definecolor{currentfill}{rgb}{0.000000,0.000000,0.000000}%
\pgfsetfillcolor{currentfill}%
\pgfsetlinewidth{0.602250pt}%
\definecolor{currentstroke}{rgb}{0.000000,0.000000,0.000000}%
\pgfsetstrokecolor{currentstroke}%
\pgfsetdash{}{0pt}%
\pgfsys@defobject{currentmarker}{\pgfqpoint{0.000000in}{-0.027778in}}{\pgfqpoint{0.000000in}{0.000000in}}{%
\pgfpathmoveto{\pgfqpoint{0.000000in}{0.000000in}}%
\pgfpathlineto{\pgfqpoint{0.000000in}{-0.027778in}}%
\pgfusepath{stroke,fill}%
}%
\begin{pgfscope}%
\pgfsys@transformshift{1.228435in}{0.417642in}%
\pgfsys@useobject{currentmarker}{}%
\end{pgfscope}%
\end{pgfscope}%
\begin{pgfscope}%
\pgfpathrectangle{\pgfqpoint{0.514278in}{0.417642in}}{\pgfqpoint{1.884996in}{1.371397in}}%
\pgfusepath{clip}%
\pgfsetrectcap%
\pgfsetroundjoin%
\pgfsetlinewidth{0.803000pt}%
\definecolor{currentstroke}{rgb}{0.850000,0.850000,0.850000}%
\pgfsetstrokecolor{currentstroke}%
\pgfsetdash{}{0pt}%
\pgfpathmoveto{\pgfqpoint{1.278592in}{0.417642in}}%
\pgfpathlineto{\pgfqpoint{1.278592in}{1.789039in}}%
\pgfusepath{stroke}%
\end{pgfscope}%
\begin{pgfscope}%
\pgfsetbuttcap%
\pgfsetroundjoin%
\definecolor{currentfill}{rgb}{0.000000,0.000000,0.000000}%
\pgfsetfillcolor{currentfill}%
\pgfsetlinewidth{0.602250pt}%
\definecolor{currentstroke}{rgb}{0.000000,0.000000,0.000000}%
\pgfsetstrokecolor{currentstroke}%
\pgfsetdash{}{0pt}%
\pgfsys@defobject{currentmarker}{\pgfqpoint{0.000000in}{-0.027778in}}{\pgfqpoint{0.000000in}{0.000000in}}{%
\pgfpathmoveto{\pgfqpoint{0.000000in}{0.000000in}}%
\pgfpathlineto{\pgfqpoint{0.000000in}{-0.027778in}}%
\pgfusepath{stroke,fill}%
}%
\begin{pgfscope}%
\pgfsys@transformshift{1.278592in}{0.417642in}%
\pgfsys@useobject{currentmarker}{}%
\end{pgfscope}%
\end{pgfscope}%
\begin{pgfscope}%
\pgfpathrectangle{\pgfqpoint{0.514278in}{0.417642in}}{\pgfqpoint{1.884996in}{1.371397in}}%
\pgfusepath{clip}%
\pgfsetrectcap%
\pgfsetroundjoin%
\pgfsetlinewidth{0.803000pt}%
\definecolor{currentstroke}{rgb}{0.850000,0.850000,0.850000}%
\pgfsetstrokecolor{currentstroke}%
\pgfsetdash{}{0pt}%
\pgfpathmoveto{\pgfqpoint{1.319572in}{0.417642in}}%
\pgfpathlineto{\pgfqpoint{1.319572in}{1.789039in}}%
\pgfusepath{stroke}%
\end{pgfscope}%
\begin{pgfscope}%
\pgfsetbuttcap%
\pgfsetroundjoin%
\definecolor{currentfill}{rgb}{0.000000,0.000000,0.000000}%
\pgfsetfillcolor{currentfill}%
\pgfsetlinewidth{0.602250pt}%
\definecolor{currentstroke}{rgb}{0.000000,0.000000,0.000000}%
\pgfsetstrokecolor{currentstroke}%
\pgfsetdash{}{0pt}%
\pgfsys@defobject{currentmarker}{\pgfqpoint{0.000000in}{-0.027778in}}{\pgfqpoint{0.000000in}{0.000000in}}{%
\pgfpathmoveto{\pgfqpoint{0.000000in}{0.000000in}}%
\pgfpathlineto{\pgfqpoint{0.000000in}{-0.027778in}}%
\pgfusepath{stroke,fill}%
}%
\begin{pgfscope}%
\pgfsys@transformshift{1.319572in}{0.417642in}%
\pgfsys@useobject{currentmarker}{}%
\end{pgfscope}%
\end{pgfscope}%
\begin{pgfscope}%
\pgfpathrectangle{\pgfqpoint{0.514278in}{0.417642in}}{\pgfqpoint{1.884996in}{1.371397in}}%
\pgfusepath{clip}%
\pgfsetrectcap%
\pgfsetroundjoin%
\pgfsetlinewidth{0.803000pt}%
\definecolor{currentstroke}{rgb}{0.850000,0.850000,0.850000}%
\pgfsetstrokecolor{currentstroke}%
\pgfsetdash{}{0pt}%
\pgfpathmoveto{\pgfqpoint{1.354221in}{0.417642in}}%
\pgfpathlineto{\pgfqpoint{1.354221in}{1.789039in}}%
\pgfusepath{stroke}%
\end{pgfscope}%
\begin{pgfscope}%
\pgfsetbuttcap%
\pgfsetroundjoin%
\definecolor{currentfill}{rgb}{0.000000,0.000000,0.000000}%
\pgfsetfillcolor{currentfill}%
\pgfsetlinewidth{0.602250pt}%
\definecolor{currentstroke}{rgb}{0.000000,0.000000,0.000000}%
\pgfsetstrokecolor{currentstroke}%
\pgfsetdash{}{0pt}%
\pgfsys@defobject{currentmarker}{\pgfqpoint{0.000000in}{-0.027778in}}{\pgfqpoint{0.000000in}{0.000000in}}{%
\pgfpathmoveto{\pgfqpoint{0.000000in}{0.000000in}}%
\pgfpathlineto{\pgfqpoint{0.000000in}{-0.027778in}}%
\pgfusepath{stroke,fill}%
}%
\begin{pgfscope}%
\pgfsys@transformshift{1.354221in}{0.417642in}%
\pgfsys@useobject{currentmarker}{}%
\end{pgfscope}%
\end{pgfscope}%
\begin{pgfscope}%
\pgfpathrectangle{\pgfqpoint{0.514278in}{0.417642in}}{\pgfqpoint{1.884996in}{1.371397in}}%
\pgfusepath{clip}%
\pgfsetrectcap%
\pgfsetroundjoin%
\pgfsetlinewidth{0.803000pt}%
\definecolor{currentstroke}{rgb}{0.850000,0.850000,0.850000}%
\pgfsetstrokecolor{currentstroke}%
\pgfsetdash{}{0pt}%
\pgfpathmoveto{\pgfqpoint{1.384235in}{0.417642in}}%
\pgfpathlineto{\pgfqpoint{1.384235in}{1.789039in}}%
\pgfusepath{stroke}%
\end{pgfscope}%
\begin{pgfscope}%
\pgfsetbuttcap%
\pgfsetroundjoin%
\definecolor{currentfill}{rgb}{0.000000,0.000000,0.000000}%
\pgfsetfillcolor{currentfill}%
\pgfsetlinewidth{0.602250pt}%
\definecolor{currentstroke}{rgb}{0.000000,0.000000,0.000000}%
\pgfsetstrokecolor{currentstroke}%
\pgfsetdash{}{0pt}%
\pgfsys@defobject{currentmarker}{\pgfqpoint{0.000000in}{-0.027778in}}{\pgfqpoint{0.000000in}{0.000000in}}{%
\pgfpathmoveto{\pgfqpoint{0.000000in}{0.000000in}}%
\pgfpathlineto{\pgfqpoint{0.000000in}{-0.027778in}}%
\pgfusepath{stroke,fill}%
}%
\begin{pgfscope}%
\pgfsys@transformshift{1.384235in}{0.417642in}%
\pgfsys@useobject{currentmarker}{}%
\end{pgfscope}%
\end{pgfscope}%
\begin{pgfscope}%
\pgfpathrectangle{\pgfqpoint{0.514278in}{0.417642in}}{\pgfqpoint{1.884996in}{1.371397in}}%
\pgfusepath{clip}%
\pgfsetrectcap%
\pgfsetroundjoin%
\pgfsetlinewidth{0.803000pt}%
\definecolor{currentstroke}{rgb}{0.850000,0.850000,0.850000}%
\pgfsetstrokecolor{currentstroke}%
\pgfsetdash{}{0pt}%
\pgfpathmoveto{\pgfqpoint{1.410709in}{0.417642in}}%
\pgfpathlineto{\pgfqpoint{1.410709in}{1.789039in}}%
\pgfusepath{stroke}%
\end{pgfscope}%
\begin{pgfscope}%
\pgfsetbuttcap%
\pgfsetroundjoin%
\definecolor{currentfill}{rgb}{0.000000,0.000000,0.000000}%
\pgfsetfillcolor{currentfill}%
\pgfsetlinewidth{0.602250pt}%
\definecolor{currentstroke}{rgb}{0.000000,0.000000,0.000000}%
\pgfsetstrokecolor{currentstroke}%
\pgfsetdash{}{0pt}%
\pgfsys@defobject{currentmarker}{\pgfqpoint{0.000000in}{-0.027778in}}{\pgfqpoint{0.000000in}{0.000000in}}{%
\pgfpathmoveto{\pgfqpoint{0.000000in}{0.000000in}}%
\pgfpathlineto{\pgfqpoint{0.000000in}{-0.027778in}}%
\pgfusepath{stroke,fill}%
}%
\begin{pgfscope}%
\pgfsys@transformshift{1.410709in}{0.417642in}%
\pgfsys@useobject{currentmarker}{}%
\end{pgfscope}%
\end{pgfscope}%
\begin{pgfscope}%
\pgfpathrectangle{\pgfqpoint{0.514278in}{0.417642in}}{\pgfqpoint{1.884996in}{1.371397in}}%
\pgfusepath{clip}%
\pgfsetrectcap%
\pgfsetroundjoin%
\pgfsetlinewidth{0.803000pt}%
\definecolor{currentstroke}{rgb}{0.850000,0.850000,0.850000}%
\pgfsetstrokecolor{currentstroke}%
\pgfsetdash{}{0pt}%
\pgfpathmoveto{\pgfqpoint{1.590191in}{0.417642in}}%
\pgfpathlineto{\pgfqpoint{1.590191in}{1.789039in}}%
\pgfusepath{stroke}%
\end{pgfscope}%
\begin{pgfscope}%
\pgfsetbuttcap%
\pgfsetroundjoin%
\definecolor{currentfill}{rgb}{0.000000,0.000000,0.000000}%
\pgfsetfillcolor{currentfill}%
\pgfsetlinewidth{0.602250pt}%
\definecolor{currentstroke}{rgb}{0.000000,0.000000,0.000000}%
\pgfsetstrokecolor{currentstroke}%
\pgfsetdash{}{0pt}%
\pgfsys@defobject{currentmarker}{\pgfqpoint{0.000000in}{-0.027778in}}{\pgfqpoint{0.000000in}{0.000000in}}{%
\pgfpathmoveto{\pgfqpoint{0.000000in}{0.000000in}}%
\pgfpathlineto{\pgfqpoint{0.000000in}{-0.027778in}}%
\pgfusepath{stroke,fill}%
}%
\begin{pgfscope}%
\pgfsys@transformshift{1.590191in}{0.417642in}%
\pgfsys@useobject{currentmarker}{}%
\end{pgfscope}%
\end{pgfscope}%
\begin{pgfscope}%
\pgfpathrectangle{\pgfqpoint{0.514278in}{0.417642in}}{\pgfqpoint{1.884996in}{1.371397in}}%
\pgfusepath{clip}%
\pgfsetrectcap%
\pgfsetroundjoin%
\pgfsetlinewidth{0.803000pt}%
\definecolor{currentstroke}{rgb}{0.850000,0.850000,0.850000}%
\pgfsetstrokecolor{currentstroke}%
\pgfsetdash{}{0pt}%
\pgfpathmoveto{\pgfqpoint{1.681328in}{0.417642in}}%
\pgfpathlineto{\pgfqpoint{1.681328in}{1.789039in}}%
\pgfusepath{stroke}%
\end{pgfscope}%
\begin{pgfscope}%
\pgfsetbuttcap%
\pgfsetroundjoin%
\definecolor{currentfill}{rgb}{0.000000,0.000000,0.000000}%
\pgfsetfillcolor{currentfill}%
\pgfsetlinewidth{0.602250pt}%
\definecolor{currentstroke}{rgb}{0.000000,0.000000,0.000000}%
\pgfsetstrokecolor{currentstroke}%
\pgfsetdash{}{0pt}%
\pgfsys@defobject{currentmarker}{\pgfqpoint{0.000000in}{-0.027778in}}{\pgfqpoint{0.000000in}{0.000000in}}{%
\pgfpathmoveto{\pgfqpoint{0.000000in}{0.000000in}}%
\pgfpathlineto{\pgfqpoint{0.000000in}{-0.027778in}}%
\pgfusepath{stroke,fill}%
}%
\begin{pgfscope}%
\pgfsys@transformshift{1.681328in}{0.417642in}%
\pgfsys@useobject{currentmarker}{}%
\end{pgfscope}%
\end{pgfscope}%
\begin{pgfscope}%
\pgfpathrectangle{\pgfqpoint{0.514278in}{0.417642in}}{\pgfqpoint{1.884996in}{1.371397in}}%
\pgfusepath{clip}%
\pgfsetrectcap%
\pgfsetroundjoin%
\pgfsetlinewidth{0.803000pt}%
\definecolor{currentstroke}{rgb}{0.850000,0.850000,0.850000}%
\pgfsetstrokecolor{currentstroke}%
\pgfsetdash{}{0pt}%
\pgfpathmoveto{\pgfqpoint{1.745991in}{0.417642in}}%
\pgfpathlineto{\pgfqpoint{1.745991in}{1.789039in}}%
\pgfusepath{stroke}%
\end{pgfscope}%
\begin{pgfscope}%
\pgfsetbuttcap%
\pgfsetroundjoin%
\definecolor{currentfill}{rgb}{0.000000,0.000000,0.000000}%
\pgfsetfillcolor{currentfill}%
\pgfsetlinewidth{0.602250pt}%
\definecolor{currentstroke}{rgb}{0.000000,0.000000,0.000000}%
\pgfsetstrokecolor{currentstroke}%
\pgfsetdash{}{0pt}%
\pgfsys@defobject{currentmarker}{\pgfqpoint{0.000000in}{-0.027778in}}{\pgfqpoint{0.000000in}{0.000000in}}{%
\pgfpathmoveto{\pgfqpoint{0.000000in}{0.000000in}}%
\pgfpathlineto{\pgfqpoint{0.000000in}{-0.027778in}}%
\pgfusepath{stroke,fill}%
}%
\begin{pgfscope}%
\pgfsys@transformshift{1.745991in}{0.417642in}%
\pgfsys@useobject{currentmarker}{}%
\end{pgfscope}%
\end{pgfscope}%
\begin{pgfscope}%
\pgfpathrectangle{\pgfqpoint{0.514278in}{0.417642in}}{\pgfqpoint{1.884996in}{1.371397in}}%
\pgfusepath{clip}%
\pgfsetrectcap%
\pgfsetroundjoin%
\pgfsetlinewidth{0.803000pt}%
\definecolor{currentstroke}{rgb}{0.850000,0.850000,0.850000}%
\pgfsetstrokecolor{currentstroke}%
\pgfsetdash{}{0pt}%
\pgfpathmoveto{\pgfqpoint{1.796147in}{0.417642in}}%
\pgfpathlineto{\pgfqpoint{1.796147in}{1.789039in}}%
\pgfusepath{stroke}%
\end{pgfscope}%
\begin{pgfscope}%
\pgfsetbuttcap%
\pgfsetroundjoin%
\definecolor{currentfill}{rgb}{0.000000,0.000000,0.000000}%
\pgfsetfillcolor{currentfill}%
\pgfsetlinewidth{0.602250pt}%
\definecolor{currentstroke}{rgb}{0.000000,0.000000,0.000000}%
\pgfsetstrokecolor{currentstroke}%
\pgfsetdash{}{0pt}%
\pgfsys@defobject{currentmarker}{\pgfqpoint{0.000000in}{-0.027778in}}{\pgfqpoint{0.000000in}{0.000000in}}{%
\pgfpathmoveto{\pgfqpoint{0.000000in}{0.000000in}}%
\pgfpathlineto{\pgfqpoint{0.000000in}{-0.027778in}}%
\pgfusepath{stroke,fill}%
}%
\begin{pgfscope}%
\pgfsys@transformshift{1.796147in}{0.417642in}%
\pgfsys@useobject{currentmarker}{}%
\end{pgfscope}%
\end{pgfscope}%
\begin{pgfscope}%
\pgfpathrectangle{\pgfqpoint{0.514278in}{0.417642in}}{\pgfqpoint{1.884996in}{1.371397in}}%
\pgfusepath{clip}%
\pgfsetrectcap%
\pgfsetroundjoin%
\pgfsetlinewidth{0.803000pt}%
\definecolor{currentstroke}{rgb}{0.850000,0.850000,0.850000}%
\pgfsetstrokecolor{currentstroke}%
\pgfsetdash{}{0pt}%
\pgfpathmoveto{\pgfqpoint{1.837128in}{0.417642in}}%
\pgfpathlineto{\pgfqpoint{1.837128in}{1.789039in}}%
\pgfusepath{stroke}%
\end{pgfscope}%
\begin{pgfscope}%
\pgfsetbuttcap%
\pgfsetroundjoin%
\definecolor{currentfill}{rgb}{0.000000,0.000000,0.000000}%
\pgfsetfillcolor{currentfill}%
\pgfsetlinewidth{0.602250pt}%
\definecolor{currentstroke}{rgb}{0.000000,0.000000,0.000000}%
\pgfsetstrokecolor{currentstroke}%
\pgfsetdash{}{0pt}%
\pgfsys@defobject{currentmarker}{\pgfqpoint{0.000000in}{-0.027778in}}{\pgfqpoint{0.000000in}{0.000000in}}{%
\pgfpathmoveto{\pgfqpoint{0.000000in}{0.000000in}}%
\pgfpathlineto{\pgfqpoint{0.000000in}{-0.027778in}}%
\pgfusepath{stroke,fill}%
}%
\begin{pgfscope}%
\pgfsys@transformshift{1.837128in}{0.417642in}%
\pgfsys@useobject{currentmarker}{}%
\end{pgfscope}%
\end{pgfscope}%
\begin{pgfscope}%
\pgfpathrectangle{\pgfqpoint{0.514278in}{0.417642in}}{\pgfqpoint{1.884996in}{1.371397in}}%
\pgfusepath{clip}%
\pgfsetrectcap%
\pgfsetroundjoin%
\pgfsetlinewidth{0.803000pt}%
\definecolor{currentstroke}{rgb}{0.850000,0.850000,0.850000}%
\pgfsetstrokecolor{currentstroke}%
\pgfsetdash{}{0pt}%
\pgfpathmoveto{\pgfqpoint{1.871777in}{0.417642in}}%
\pgfpathlineto{\pgfqpoint{1.871777in}{1.789039in}}%
\pgfusepath{stroke}%
\end{pgfscope}%
\begin{pgfscope}%
\pgfsetbuttcap%
\pgfsetroundjoin%
\definecolor{currentfill}{rgb}{0.000000,0.000000,0.000000}%
\pgfsetfillcolor{currentfill}%
\pgfsetlinewidth{0.602250pt}%
\definecolor{currentstroke}{rgb}{0.000000,0.000000,0.000000}%
\pgfsetstrokecolor{currentstroke}%
\pgfsetdash{}{0pt}%
\pgfsys@defobject{currentmarker}{\pgfqpoint{0.000000in}{-0.027778in}}{\pgfqpoint{0.000000in}{0.000000in}}{%
\pgfpathmoveto{\pgfqpoint{0.000000in}{0.000000in}}%
\pgfpathlineto{\pgfqpoint{0.000000in}{-0.027778in}}%
\pgfusepath{stroke,fill}%
}%
\begin{pgfscope}%
\pgfsys@transformshift{1.871777in}{0.417642in}%
\pgfsys@useobject{currentmarker}{}%
\end{pgfscope}%
\end{pgfscope}%
\begin{pgfscope}%
\pgfpathrectangle{\pgfqpoint{0.514278in}{0.417642in}}{\pgfqpoint{1.884996in}{1.371397in}}%
\pgfusepath{clip}%
\pgfsetrectcap%
\pgfsetroundjoin%
\pgfsetlinewidth{0.803000pt}%
\definecolor{currentstroke}{rgb}{0.850000,0.850000,0.850000}%
\pgfsetstrokecolor{currentstroke}%
\pgfsetdash{}{0pt}%
\pgfpathmoveto{\pgfqpoint{1.901791in}{0.417642in}}%
\pgfpathlineto{\pgfqpoint{1.901791in}{1.789039in}}%
\pgfusepath{stroke}%
\end{pgfscope}%
\begin{pgfscope}%
\pgfsetbuttcap%
\pgfsetroundjoin%
\definecolor{currentfill}{rgb}{0.000000,0.000000,0.000000}%
\pgfsetfillcolor{currentfill}%
\pgfsetlinewidth{0.602250pt}%
\definecolor{currentstroke}{rgb}{0.000000,0.000000,0.000000}%
\pgfsetstrokecolor{currentstroke}%
\pgfsetdash{}{0pt}%
\pgfsys@defobject{currentmarker}{\pgfqpoint{0.000000in}{-0.027778in}}{\pgfqpoint{0.000000in}{0.000000in}}{%
\pgfpathmoveto{\pgfqpoint{0.000000in}{0.000000in}}%
\pgfpathlineto{\pgfqpoint{0.000000in}{-0.027778in}}%
\pgfusepath{stroke,fill}%
}%
\begin{pgfscope}%
\pgfsys@transformshift{1.901791in}{0.417642in}%
\pgfsys@useobject{currentmarker}{}%
\end{pgfscope}%
\end{pgfscope}%
\begin{pgfscope}%
\pgfpathrectangle{\pgfqpoint{0.514278in}{0.417642in}}{\pgfqpoint{1.884996in}{1.371397in}}%
\pgfusepath{clip}%
\pgfsetrectcap%
\pgfsetroundjoin%
\pgfsetlinewidth{0.803000pt}%
\definecolor{currentstroke}{rgb}{0.850000,0.850000,0.850000}%
\pgfsetstrokecolor{currentstroke}%
\pgfsetdash{}{0pt}%
\pgfpathmoveto{\pgfqpoint{1.928265in}{0.417642in}}%
\pgfpathlineto{\pgfqpoint{1.928265in}{1.789039in}}%
\pgfusepath{stroke}%
\end{pgfscope}%
\begin{pgfscope}%
\pgfsetbuttcap%
\pgfsetroundjoin%
\definecolor{currentfill}{rgb}{0.000000,0.000000,0.000000}%
\pgfsetfillcolor{currentfill}%
\pgfsetlinewidth{0.602250pt}%
\definecolor{currentstroke}{rgb}{0.000000,0.000000,0.000000}%
\pgfsetstrokecolor{currentstroke}%
\pgfsetdash{}{0pt}%
\pgfsys@defobject{currentmarker}{\pgfqpoint{0.000000in}{-0.027778in}}{\pgfqpoint{0.000000in}{0.000000in}}{%
\pgfpathmoveto{\pgfqpoint{0.000000in}{0.000000in}}%
\pgfpathlineto{\pgfqpoint{0.000000in}{-0.027778in}}%
\pgfusepath{stroke,fill}%
}%
\begin{pgfscope}%
\pgfsys@transformshift{1.928265in}{0.417642in}%
\pgfsys@useobject{currentmarker}{}%
\end{pgfscope}%
\end{pgfscope}%
\begin{pgfscope}%
\pgfpathrectangle{\pgfqpoint{0.514278in}{0.417642in}}{\pgfqpoint{1.884996in}{1.371397in}}%
\pgfusepath{clip}%
\pgfsetrectcap%
\pgfsetroundjoin%
\pgfsetlinewidth{0.803000pt}%
\definecolor{currentstroke}{rgb}{0.850000,0.850000,0.850000}%
\pgfsetstrokecolor{currentstroke}%
\pgfsetdash{}{0pt}%
\pgfpathmoveto{\pgfqpoint{2.107747in}{0.417642in}}%
\pgfpathlineto{\pgfqpoint{2.107747in}{1.789039in}}%
\pgfusepath{stroke}%
\end{pgfscope}%
\begin{pgfscope}%
\pgfsetbuttcap%
\pgfsetroundjoin%
\definecolor{currentfill}{rgb}{0.000000,0.000000,0.000000}%
\pgfsetfillcolor{currentfill}%
\pgfsetlinewidth{0.602250pt}%
\definecolor{currentstroke}{rgb}{0.000000,0.000000,0.000000}%
\pgfsetstrokecolor{currentstroke}%
\pgfsetdash{}{0pt}%
\pgfsys@defobject{currentmarker}{\pgfqpoint{0.000000in}{-0.027778in}}{\pgfqpoint{0.000000in}{0.000000in}}{%
\pgfpathmoveto{\pgfqpoint{0.000000in}{0.000000in}}%
\pgfpathlineto{\pgfqpoint{0.000000in}{-0.027778in}}%
\pgfusepath{stroke,fill}%
}%
\begin{pgfscope}%
\pgfsys@transformshift{2.107747in}{0.417642in}%
\pgfsys@useobject{currentmarker}{}%
\end{pgfscope}%
\end{pgfscope}%
\begin{pgfscope}%
\pgfpathrectangle{\pgfqpoint{0.514278in}{0.417642in}}{\pgfqpoint{1.884996in}{1.371397in}}%
\pgfusepath{clip}%
\pgfsetrectcap%
\pgfsetroundjoin%
\pgfsetlinewidth{0.803000pt}%
\definecolor{currentstroke}{rgb}{0.850000,0.850000,0.850000}%
\pgfsetstrokecolor{currentstroke}%
\pgfsetdash{}{0pt}%
\pgfpathmoveto{\pgfqpoint{2.198884in}{0.417642in}}%
\pgfpathlineto{\pgfqpoint{2.198884in}{1.789039in}}%
\pgfusepath{stroke}%
\end{pgfscope}%
\begin{pgfscope}%
\pgfsetbuttcap%
\pgfsetroundjoin%
\definecolor{currentfill}{rgb}{0.000000,0.000000,0.000000}%
\pgfsetfillcolor{currentfill}%
\pgfsetlinewidth{0.602250pt}%
\definecolor{currentstroke}{rgb}{0.000000,0.000000,0.000000}%
\pgfsetstrokecolor{currentstroke}%
\pgfsetdash{}{0pt}%
\pgfsys@defobject{currentmarker}{\pgfqpoint{0.000000in}{-0.027778in}}{\pgfqpoint{0.000000in}{0.000000in}}{%
\pgfpathmoveto{\pgfqpoint{0.000000in}{0.000000in}}%
\pgfpathlineto{\pgfqpoint{0.000000in}{-0.027778in}}%
\pgfusepath{stroke,fill}%
}%
\begin{pgfscope}%
\pgfsys@transformshift{2.198884in}{0.417642in}%
\pgfsys@useobject{currentmarker}{}%
\end{pgfscope}%
\end{pgfscope}%
\begin{pgfscope}%
\pgfpathrectangle{\pgfqpoint{0.514278in}{0.417642in}}{\pgfqpoint{1.884996in}{1.371397in}}%
\pgfusepath{clip}%
\pgfsetrectcap%
\pgfsetroundjoin%
\pgfsetlinewidth{0.803000pt}%
\definecolor{currentstroke}{rgb}{0.850000,0.850000,0.850000}%
\pgfsetstrokecolor{currentstroke}%
\pgfsetdash{}{0pt}%
\pgfpathmoveto{\pgfqpoint{2.263547in}{0.417642in}}%
\pgfpathlineto{\pgfqpoint{2.263547in}{1.789039in}}%
\pgfusepath{stroke}%
\end{pgfscope}%
\begin{pgfscope}%
\pgfsetbuttcap%
\pgfsetroundjoin%
\definecolor{currentfill}{rgb}{0.000000,0.000000,0.000000}%
\pgfsetfillcolor{currentfill}%
\pgfsetlinewidth{0.602250pt}%
\definecolor{currentstroke}{rgb}{0.000000,0.000000,0.000000}%
\pgfsetstrokecolor{currentstroke}%
\pgfsetdash{}{0pt}%
\pgfsys@defobject{currentmarker}{\pgfqpoint{0.000000in}{-0.027778in}}{\pgfqpoint{0.000000in}{0.000000in}}{%
\pgfpathmoveto{\pgfqpoint{0.000000in}{0.000000in}}%
\pgfpathlineto{\pgfqpoint{0.000000in}{-0.027778in}}%
\pgfusepath{stroke,fill}%
}%
\begin{pgfscope}%
\pgfsys@transformshift{2.263547in}{0.417642in}%
\pgfsys@useobject{currentmarker}{}%
\end{pgfscope}%
\end{pgfscope}%
\begin{pgfscope}%
\pgfpathrectangle{\pgfqpoint{0.514278in}{0.417642in}}{\pgfqpoint{1.884996in}{1.371397in}}%
\pgfusepath{clip}%
\pgfsetrectcap%
\pgfsetroundjoin%
\pgfsetlinewidth{0.803000pt}%
\definecolor{currentstroke}{rgb}{0.850000,0.850000,0.850000}%
\pgfsetstrokecolor{currentstroke}%
\pgfsetdash{}{0pt}%
\pgfpathmoveto{\pgfqpoint{2.313703in}{0.417642in}}%
\pgfpathlineto{\pgfqpoint{2.313703in}{1.789039in}}%
\pgfusepath{stroke}%
\end{pgfscope}%
\begin{pgfscope}%
\pgfsetbuttcap%
\pgfsetroundjoin%
\definecolor{currentfill}{rgb}{0.000000,0.000000,0.000000}%
\pgfsetfillcolor{currentfill}%
\pgfsetlinewidth{0.602250pt}%
\definecolor{currentstroke}{rgb}{0.000000,0.000000,0.000000}%
\pgfsetstrokecolor{currentstroke}%
\pgfsetdash{}{0pt}%
\pgfsys@defobject{currentmarker}{\pgfqpoint{0.000000in}{-0.027778in}}{\pgfqpoint{0.000000in}{0.000000in}}{%
\pgfpathmoveto{\pgfqpoint{0.000000in}{0.000000in}}%
\pgfpathlineto{\pgfqpoint{0.000000in}{-0.027778in}}%
\pgfusepath{stroke,fill}%
}%
\begin{pgfscope}%
\pgfsys@transformshift{2.313703in}{0.417642in}%
\pgfsys@useobject{currentmarker}{}%
\end{pgfscope}%
\end{pgfscope}%
\begin{pgfscope}%
\pgfpathrectangle{\pgfqpoint{0.514278in}{0.417642in}}{\pgfqpoint{1.884996in}{1.371397in}}%
\pgfusepath{clip}%
\pgfsetrectcap%
\pgfsetroundjoin%
\pgfsetlinewidth{0.803000pt}%
\definecolor{currentstroke}{rgb}{0.850000,0.850000,0.850000}%
\pgfsetstrokecolor{currentstroke}%
\pgfsetdash{}{0pt}%
\pgfpathmoveto{\pgfqpoint{2.354684in}{0.417642in}}%
\pgfpathlineto{\pgfqpoint{2.354684in}{1.789039in}}%
\pgfusepath{stroke}%
\end{pgfscope}%
\begin{pgfscope}%
\pgfsetbuttcap%
\pgfsetroundjoin%
\definecolor{currentfill}{rgb}{0.000000,0.000000,0.000000}%
\pgfsetfillcolor{currentfill}%
\pgfsetlinewidth{0.602250pt}%
\definecolor{currentstroke}{rgb}{0.000000,0.000000,0.000000}%
\pgfsetstrokecolor{currentstroke}%
\pgfsetdash{}{0pt}%
\pgfsys@defobject{currentmarker}{\pgfqpoint{0.000000in}{-0.027778in}}{\pgfqpoint{0.000000in}{0.000000in}}{%
\pgfpathmoveto{\pgfqpoint{0.000000in}{0.000000in}}%
\pgfpathlineto{\pgfqpoint{0.000000in}{-0.027778in}}%
\pgfusepath{stroke,fill}%
}%
\begin{pgfscope}%
\pgfsys@transformshift{2.354684in}{0.417642in}%
\pgfsys@useobject{currentmarker}{}%
\end{pgfscope}%
\end{pgfscope}%
\begin{pgfscope}%
\pgfpathrectangle{\pgfqpoint{0.514278in}{0.417642in}}{\pgfqpoint{1.884996in}{1.371397in}}%
\pgfusepath{clip}%
\pgfsetrectcap%
\pgfsetroundjoin%
\pgfsetlinewidth{0.803000pt}%
\definecolor{currentstroke}{rgb}{0.850000,0.850000,0.850000}%
\pgfsetstrokecolor{currentstroke}%
\pgfsetdash{}{0pt}%
\pgfpathmoveto{\pgfqpoint{2.389333in}{0.417642in}}%
\pgfpathlineto{\pgfqpoint{2.389333in}{1.789039in}}%
\pgfusepath{stroke}%
\end{pgfscope}%
\begin{pgfscope}%
\pgfsetbuttcap%
\pgfsetroundjoin%
\definecolor{currentfill}{rgb}{0.000000,0.000000,0.000000}%
\pgfsetfillcolor{currentfill}%
\pgfsetlinewidth{0.602250pt}%
\definecolor{currentstroke}{rgb}{0.000000,0.000000,0.000000}%
\pgfsetstrokecolor{currentstroke}%
\pgfsetdash{}{0pt}%
\pgfsys@defobject{currentmarker}{\pgfqpoint{0.000000in}{-0.027778in}}{\pgfqpoint{0.000000in}{0.000000in}}{%
\pgfpathmoveto{\pgfqpoint{0.000000in}{0.000000in}}%
\pgfpathlineto{\pgfqpoint{0.000000in}{-0.027778in}}%
\pgfusepath{stroke,fill}%
}%
\begin{pgfscope}%
\pgfsys@transformshift{2.389333in}{0.417642in}%
\pgfsys@useobject{currentmarker}{}%
\end{pgfscope}%
\end{pgfscope}%
\begin{pgfscope}%
\definecolor{textcolor}{rgb}{0.000000,0.000000,0.000000}%
\pgfsetstrokecolor{textcolor}%
\pgfsetfillcolor{textcolor}%
\pgftext[x=1.456777in,y=0.165003in,,top]{\color{textcolor}{\rmfamily\fontsize{10.000000}{12.000000}\selectfont\catcode`\^=\active\def^{\ifmmode\sp\else\^{}\fi}\catcode`\%=\active\def%{\%}Frequency in $\unit{\Hz}$}}%
\end{pgfscope}%
\begin{pgfscope}%
\pgfpathrectangle{\pgfqpoint{0.514278in}{0.417642in}}{\pgfqpoint{1.884996in}{1.371397in}}%
\pgfusepath{clip}%
\pgfsetrectcap%
\pgfsetroundjoin%
\pgfsetlinewidth{0.803000pt}%
\definecolor{currentstroke}{rgb}{0.450000,0.450000,0.450000}%
\pgfsetstrokecolor{currentstroke}%
\pgfsetdash{}{0pt}%
\pgfpathmoveto{\pgfqpoint{0.514278in}{0.640670in}}%
\pgfpathlineto{\pgfqpoint{2.399275in}{0.640670in}}%
\pgfusepath{stroke}%
\end{pgfscope}%
\begin{pgfscope}%
\pgfsetbuttcap%
\pgfsetroundjoin%
\definecolor{currentfill}{rgb}{0.000000,0.000000,0.000000}%
\pgfsetfillcolor{currentfill}%
\pgfsetlinewidth{0.803000pt}%
\definecolor{currentstroke}{rgb}{0.000000,0.000000,0.000000}%
\pgfsetstrokecolor{currentstroke}%
\pgfsetdash{}{0pt}%
\pgfsys@defobject{currentmarker}{\pgfqpoint{-0.048611in}{0.000000in}}{\pgfqpoint{-0.000000in}{0.000000in}}{%
\pgfpathmoveto{\pgfqpoint{-0.000000in}{0.000000in}}%
\pgfpathlineto{\pgfqpoint{-0.048611in}{0.000000in}}%
\pgfusepath{stroke,fill}%
}%
\begin{pgfscope}%
\pgfsys@transformshift{0.514278in}{0.640670in}%
\pgfsys@useobject{currentmarker}{}%
\end{pgfscope}%
\end{pgfscope}%
\begin{pgfscope}%
\definecolor{textcolor}{rgb}{0.000000,0.000000,0.000000}%
\pgfsetstrokecolor{textcolor}%
\pgfsetfillcolor{textcolor}%
\pgftext[x=0.241129in, y=0.601518in, left, base]{\color{textcolor}{\rmfamily\fontsize{8.000000}{9.600000}\selectfont\catcode`\^=\active\def^{\ifmmode\sp\else\^{}\fi}\catcode`\%=\active\def%{\%}$\mathdefault{10^{0}}$}}%
\end{pgfscope}%
\begin{pgfscope}%
\pgfpathrectangle{\pgfqpoint{0.514278in}{0.417642in}}{\pgfqpoint{1.884996in}{1.371397in}}%
\pgfusepath{clip}%
\pgfsetrectcap%
\pgfsetroundjoin%
\pgfsetlinewidth{0.803000pt}%
\definecolor{currentstroke}{rgb}{0.450000,0.450000,0.450000}%
\pgfsetstrokecolor{currentstroke}%
\pgfsetdash{}{0pt}%
\pgfpathmoveto{\pgfqpoint{0.514278in}{0.983520in}}%
\pgfpathlineto{\pgfqpoint{2.399275in}{0.983520in}}%
\pgfusepath{stroke}%
\end{pgfscope}%
\begin{pgfscope}%
\pgfsetbuttcap%
\pgfsetroundjoin%
\definecolor{currentfill}{rgb}{0.000000,0.000000,0.000000}%
\pgfsetfillcolor{currentfill}%
\pgfsetlinewidth{0.803000pt}%
\definecolor{currentstroke}{rgb}{0.000000,0.000000,0.000000}%
\pgfsetstrokecolor{currentstroke}%
\pgfsetdash{}{0pt}%
\pgfsys@defobject{currentmarker}{\pgfqpoint{-0.048611in}{0.000000in}}{\pgfqpoint{-0.000000in}{0.000000in}}{%
\pgfpathmoveto{\pgfqpoint{-0.000000in}{0.000000in}}%
\pgfpathlineto{\pgfqpoint{-0.048611in}{0.000000in}}%
\pgfusepath{stroke,fill}%
}%
\begin{pgfscope}%
\pgfsys@transformshift{0.514278in}{0.983520in}%
\pgfsys@useobject{currentmarker}{}%
\end{pgfscope}%
\end{pgfscope}%
\begin{pgfscope}%
\definecolor{textcolor}{rgb}{0.000000,0.000000,0.000000}%
\pgfsetstrokecolor{textcolor}%
\pgfsetfillcolor{textcolor}%
\pgftext[x=0.241129in, y=0.944367in, left, base]{\color{textcolor}{\rmfamily\fontsize{8.000000}{9.600000}\selectfont\catcode`\^=\active\def^{\ifmmode\sp\else\^{}\fi}\catcode`\%=\active\def%{\%}$\mathdefault{10^{2}}$}}%
\end{pgfscope}%
\begin{pgfscope}%
\pgfpathrectangle{\pgfqpoint{0.514278in}{0.417642in}}{\pgfqpoint{1.884996in}{1.371397in}}%
\pgfusepath{clip}%
\pgfsetrectcap%
\pgfsetroundjoin%
\pgfsetlinewidth{0.803000pt}%
\definecolor{currentstroke}{rgb}{0.450000,0.450000,0.450000}%
\pgfsetstrokecolor{currentstroke}%
\pgfsetdash{}{0pt}%
\pgfpathmoveto{\pgfqpoint{0.514278in}{1.326369in}}%
\pgfpathlineto{\pgfqpoint{2.399275in}{1.326369in}}%
\pgfusepath{stroke}%
\end{pgfscope}%
\begin{pgfscope}%
\pgfsetbuttcap%
\pgfsetroundjoin%
\definecolor{currentfill}{rgb}{0.000000,0.000000,0.000000}%
\pgfsetfillcolor{currentfill}%
\pgfsetlinewidth{0.803000pt}%
\definecolor{currentstroke}{rgb}{0.000000,0.000000,0.000000}%
\pgfsetstrokecolor{currentstroke}%
\pgfsetdash{}{0pt}%
\pgfsys@defobject{currentmarker}{\pgfqpoint{-0.048611in}{0.000000in}}{\pgfqpoint{-0.000000in}{0.000000in}}{%
\pgfpathmoveto{\pgfqpoint{-0.000000in}{0.000000in}}%
\pgfpathlineto{\pgfqpoint{-0.048611in}{0.000000in}}%
\pgfusepath{stroke,fill}%
}%
\begin{pgfscope}%
\pgfsys@transformshift{0.514278in}{1.326369in}%
\pgfsys@useobject{currentmarker}{}%
\end{pgfscope}%
\end{pgfscope}%
\begin{pgfscope}%
\definecolor{textcolor}{rgb}{0.000000,0.000000,0.000000}%
\pgfsetstrokecolor{textcolor}%
\pgfsetfillcolor{textcolor}%
\pgftext[x=0.241129in, y=1.287216in, left, base]{\color{textcolor}{\rmfamily\fontsize{8.000000}{9.600000}\selectfont\catcode`\^=\active\def^{\ifmmode\sp\else\^{}\fi}\catcode`\%=\active\def%{\%}$\mathdefault{10^{4}}$}}%
\end{pgfscope}%
\begin{pgfscope}%
\pgfpathrectangle{\pgfqpoint{0.514278in}{0.417642in}}{\pgfqpoint{1.884996in}{1.371397in}}%
\pgfusepath{clip}%
\pgfsetrectcap%
\pgfsetroundjoin%
\pgfsetlinewidth{0.803000pt}%
\definecolor{currentstroke}{rgb}{0.450000,0.450000,0.450000}%
\pgfsetstrokecolor{currentstroke}%
\pgfsetdash{}{0pt}%
\pgfpathmoveto{\pgfqpoint{0.514278in}{1.669218in}}%
\pgfpathlineto{\pgfqpoint{2.399275in}{1.669218in}}%
\pgfusepath{stroke}%
\end{pgfscope}%
\begin{pgfscope}%
\pgfsetbuttcap%
\pgfsetroundjoin%
\definecolor{currentfill}{rgb}{0.000000,0.000000,0.000000}%
\pgfsetfillcolor{currentfill}%
\pgfsetlinewidth{0.803000pt}%
\definecolor{currentstroke}{rgb}{0.000000,0.000000,0.000000}%
\pgfsetstrokecolor{currentstroke}%
\pgfsetdash{}{0pt}%
\pgfsys@defobject{currentmarker}{\pgfqpoint{-0.048611in}{0.000000in}}{\pgfqpoint{-0.000000in}{0.000000in}}{%
\pgfpathmoveto{\pgfqpoint{-0.000000in}{0.000000in}}%
\pgfpathlineto{\pgfqpoint{-0.048611in}{0.000000in}}%
\pgfusepath{stroke,fill}%
}%
\begin{pgfscope}%
\pgfsys@transformshift{0.514278in}{1.669218in}%
\pgfsys@useobject{currentmarker}{}%
\end{pgfscope}%
\end{pgfscope}%
\begin{pgfscope}%
\definecolor{textcolor}{rgb}{0.000000,0.000000,0.000000}%
\pgfsetstrokecolor{textcolor}%
\pgfsetfillcolor{textcolor}%
\pgftext[x=0.241129in, y=1.630065in, left, base]{\color{textcolor}{\rmfamily\fontsize{8.000000}{9.600000}\selectfont\catcode`\^=\active\def^{\ifmmode\sp\else\^{}\fi}\catcode`\%=\active\def%{\%}$\mathdefault{10^{6}}$}}%
\end{pgfscope}%
\begin{pgfscope}%
\definecolor{textcolor}{rgb}{0.000000,0.000000,0.000000}%
\pgfsetstrokecolor{textcolor}%
\pgfsetfillcolor{textcolor}%
\pgftext[x=0.185574in,y=1.103340in,,bottom,rotate=90.000000]{\color{textcolor}{\rmfamily\fontsize{10.000000}{12.000000}\selectfont\catcode`\^=\active\def^{\ifmmode\sp\else\^{}\fi}\catcode`\%=\active\def%{\%}$S_y(f)$ in $\unit{1 \per \Hz}$}}%
\end{pgfscope}%
\begin{pgfscope}%
\pgfpathrectangle{\pgfqpoint{0.514278in}{0.417642in}}{\pgfqpoint{1.884996in}{1.371397in}}%
\pgfusepath{clip}%
\pgfsetbuttcap%
\pgfsetroundjoin%
\pgfsetlinewidth{1.505625pt}%
\definecolor{currentstroke}{rgb}{0.003922,0.450980,0.698039}%
\pgfsetstrokecolor{currentstroke}%
\pgfsetdash{{5.550000pt}{2.400000pt}}{0.000000pt}%
\pgfpathmoveto{\pgfqpoint{0.599960in}{0.692274in}}%
\pgfpathlineto{\pgfqpoint{0.755760in}{0.692274in}}%
\pgfpathlineto{\pgfqpoint{0.846897in}{0.692274in}}%
\pgfpathlineto{\pgfqpoint{0.911560in}{0.692274in}}%
\pgfpathlineto{\pgfqpoint{0.961716in}{0.692274in}}%
\pgfpathlineto{\pgfqpoint{1.002697in}{0.692274in}}%
\pgfpathlineto{\pgfqpoint{1.037345in}{0.692274in}}%
\pgfpathlineto{\pgfqpoint{1.067360in}{0.692274in}}%
\pgfpathlineto{\pgfqpoint{1.093834in}{0.692274in}}%
\pgfpathlineto{\pgfqpoint{1.117516in}{0.692274in}}%
\pgfpathlineto{\pgfqpoint{1.138939in}{0.692274in}}%
\pgfpathlineto{\pgfqpoint{1.158497in}{0.692274in}}%
\pgfpathlineto{\pgfqpoint{1.176488in}{0.692274in}}%
\pgfpathlineto{\pgfqpoint{1.193145in}{0.692274in}}%
\pgfpathlineto{\pgfqpoint{1.208653in}{0.692274in}}%
\pgfpathlineto{\pgfqpoint{1.223159in}{0.692274in}}%
\pgfpathlineto{\pgfqpoint{1.236786in}{0.692274in}}%
\pgfpathlineto{\pgfqpoint{1.249634in}{0.692274in}}%
\pgfpathlineto{\pgfqpoint{1.261786in}{0.692274in}}%
\pgfpathlineto{\pgfqpoint{1.273316in}{0.692274in}}%
\pgfpathlineto{\pgfqpoint{1.284282in}{0.692274in}}%
\pgfpathlineto{\pgfqpoint{1.294739in}{0.692274in}}%
\pgfpathlineto{\pgfqpoint{1.309564in}{0.692274in}}%
\pgfpathlineto{\pgfqpoint{1.323472in}{0.692274in}}%
\pgfpathlineto{\pgfqpoint{1.332288in}{0.692274in}}%
\pgfpathlineto{\pgfqpoint{1.340771in}{0.692274in}}%
\pgfpathlineto{\pgfqpoint{1.348945in}{0.692274in}}%
\pgfpathlineto{\pgfqpoint{1.360675in}{0.692274in}}%
\pgfpathlineto{\pgfqpoint{1.371823in}{0.692274in}}%
\pgfpathlineto{\pgfqpoint{1.382444in}{0.692274in}}%
\pgfpathlineto{\pgfqpoint{1.392586in}{0.692274in}}%
\pgfpathlineto{\pgfqpoint{1.402290in}{0.692274in}}%
\pgfpathlineto{\pgfqpoint{1.414609in}{0.692274in}}%
\pgfpathlineto{\pgfqpoint{1.426288in}{0.692274in}}%
\pgfpathlineto{\pgfqpoint{1.434666in}{0.692274in}}%
\pgfpathlineto{\pgfqpoint{1.442742in}{0.692274in}}%
\pgfpathlineto{\pgfqpoint{1.453078in}{0.692274in}}%
\pgfpathlineto{\pgfqpoint{1.465364in}{0.692274in}}%
\pgfpathlineto{\pgfqpoint{1.477013in}{0.692274in}}%
\pgfpathlineto{\pgfqpoint{1.485916in}{0.692274in}}%
\pgfpathlineto{\pgfqpoint{1.496570in}{0.692274in}}%
\pgfpathlineto{\pgfqpoint{1.506743in}{0.692274in}}%
\pgfpathlineto{\pgfqpoint{1.516475in}{0.692274in}}%
\pgfpathlineto{\pgfqpoint{1.527623in}{0.692274in}}%
\pgfpathlineto{\pgfqpoint{1.538244in}{0.692274in}}%
\pgfpathlineto{\pgfqpoint{1.548386in}{0.692274in}}%
\pgfpathlineto{\pgfqpoint{1.559667in}{0.692274in}}%
\pgfpathlineto{\pgfqpoint{1.570409in}{0.692274in}}%
\pgfpathlineto{\pgfqpoint{1.580661in}{0.692274in}}%
\pgfpathlineto{\pgfqpoint{1.590465in}{0.692274in}}%
\pgfpathlineto{\pgfqpoint{1.599860in}{0.692274in}}%
\pgfpathlineto{\pgfqpoint{1.611390in}{0.692274in}}%
\pgfpathlineto{\pgfqpoint{1.622356in}{0.692274in}}%
\pgfpathlineto{\pgfqpoint{1.631675in}{0.692274in}}%
\pgfpathlineto{\pgfqpoint{1.641716in}{0.692274in}}%
\pgfpathlineto{\pgfqpoint{1.652370in}{0.692274in}}%
\pgfpathlineto{\pgfqpoint{1.663535in}{0.692274in}}%
\pgfpathlineto{\pgfqpoint{1.674171in}{0.692274in}}%
\pgfpathlineto{\pgfqpoint{1.684327in}{0.692274in}}%
\pgfpathlineto{\pgfqpoint{1.694907in}{0.692274in}}%
\pgfpathlineto{\pgfqpoint{1.705010in}{0.692274in}}%
\pgfpathlineto{\pgfqpoint{1.715467in}{0.692274in}}%
\pgfpathlineto{\pgfqpoint{1.726209in}{0.692274in}}%
\pgfpathlineto{\pgfqpoint{1.736461in}{0.692274in}}%
\pgfpathlineto{\pgfqpoint{1.746950in}{0.692274in}}%
\pgfpathlineto{\pgfqpoint{1.756971in}{0.692274in}}%
\pgfpathlineto{\pgfqpoint{1.767189in}{0.692274in}}%
\pgfpathlineto{\pgfqpoint{1.778156in}{0.692274in}}%
\pgfpathlineto{\pgfqpoint{1.788612in}{0.692274in}}%
\pgfpathlineto{\pgfqpoint{1.798604in}{0.692274in}}%
\pgfpathlineto{\pgfqpoint{1.808690in}{0.692274in}}%
\pgfpathlineto{\pgfqpoint{1.819335in}{0.692274in}}%
\pgfpathlineto{\pgfqpoint{1.829971in}{0.692274in}}%
\pgfpathlineto{\pgfqpoint{1.840127in}{0.692274in}}%
\pgfpathlineto{\pgfqpoint{1.850275in}{0.692274in}}%
\pgfpathlineto{\pgfqpoint{1.860810in}{0.692274in}}%
\pgfpathlineto{\pgfqpoint{1.871266in}{0.692274in}}%
\pgfpathlineto{\pgfqpoint{1.881634in}{0.692274in}}%
\pgfpathlineto{\pgfqpoint{1.892260in}{0.692274in}}%
\pgfpathlineto{\pgfqpoint{1.902749in}{0.692274in}}%
\pgfpathlineto{\pgfqpoint{1.913097in}{0.692274in}}%
\pgfpathlineto{\pgfqpoint{1.923613in}{0.692274in}}%
\pgfpathlineto{\pgfqpoint{1.933956in}{0.692274in}}%
\pgfpathlineto{\pgfqpoint{1.944128in}{0.692274in}}%
\pgfpathlineto{\pgfqpoint{1.954404in}{0.692274in}}%
\pgfpathlineto{\pgfqpoint{1.965008in}{0.692274in}}%
\pgfpathlineto{\pgfqpoint{1.975629in}{0.692274in}}%
\pgfpathlineto{\pgfqpoint{1.986007in}{0.692274in}}%
\pgfpathlineto{\pgfqpoint{1.996378in}{0.692274in}}%
\pgfpathlineto{\pgfqpoint{2.006721in}{0.692274in}}%
\pgfpathlineto{\pgfqpoint{2.017021in}{0.692274in}}%
\pgfpathlineto{\pgfqpoint{2.027459in}{0.692274in}}%
\pgfpathlineto{\pgfqpoint{2.037808in}{0.692274in}}%
\pgfpathlineto{\pgfqpoint{2.048239in}{0.692274in}}%
\pgfpathlineto{\pgfqpoint{2.058720in}{0.692274in}}%
\pgfpathlineto{\pgfqpoint{2.069060in}{0.692274in}}%
\pgfpathlineto{\pgfqpoint{2.079412in}{0.692274in}}%
\pgfpathlineto{\pgfqpoint{2.089756in}{0.692274in}}%
\pgfpathlineto{\pgfqpoint{2.100212in}{0.692274in}}%
\pgfpathlineto{\pgfqpoint{2.110610in}{0.692274in}}%
\pgfpathlineto{\pgfqpoint{2.121067in}{0.692274in}}%
\pgfpathlineto{\pgfqpoint{2.131552in}{0.692274in}}%
\pgfpathlineto{\pgfqpoint{2.141925in}{0.692274in}}%
\pgfpathlineto{\pgfqpoint{2.152290in}{0.692274in}}%
\pgfpathlineto{\pgfqpoint{2.162629in}{0.692274in}}%
\pgfpathlineto{\pgfqpoint{2.173026in}{0.692274in}}%
\pgfpathlineto{\pgfqpoint{2.183455in}{0.692274in}}%
\pgfpathlineto{\pgfqpoint{2.193889in}{0.692274in}}%
\pgfpathlineto{\pgfqpoint{2.204307in}{0.692274in}}%
\pgfpathlineto{\pgfqpoint{2.214690in}{0.692274in}}%
\pgfpathlineto{\pgfqpoint{2.225104in}{0.692274in}}%
\pgfpathlineto{\pgfqpoint{2.235523in}{0.692274in}}%
\pgfpathlineto{\pgfqpoint{2.245927in}{0.692274in}}%
\pgfpathlineto{\pgfqpoint{2.256295in}{0.692274in}}%
\pgfpathlineto{\pgfqpoint{2.266681in}{0.692274in}}%
\pgfpathlineto{\pgfqpoint{2.277125in}{0.692274in}}%
\pgfpathlineto{\pgfqpoint{2.287537in}{0.692274in}}%
\pgfpathlineto{\pgfqpoint{2.297901in}{0.692274in}}%
\pgfpathlineto{\pgfqpoint{2.308315in}{0.692274in}}%
\pgfpathlineto{\pgfqpoint{2.313593in}{0.692274in}}%
\pgfusepath{stroke}%
\end{pgfscope}%
\begin{pgfscope}%
\pgfpathrectangle{\pgfqpoint{0.514278in}{0.417642in}}{\pgfqpoint{1.884996in}{1.371397in}}%
\pgfusepath{clip}%
\pgfsetbuttcap%
\pgfsetroundjoin%
\definecolor{currentfill}{rgb}{0.003922,0.450980,0.698039}%
\pgfsetfillcolor{currentfill}%
\pgfsetlinewidth{1.003750pt}%
\definecolor{currentstroke}{rgb}{0.003922,0.450980,0.698039}%
\pgfsetstrokecolor{currentstroke}%
\pgfsetdash{}{0pt}%
\pgfsys@defobject{currentmarker}{\pgfqpoint{-0.006944in}{-0.006944in}}{\pgfqpoint{0.006944in}{0.006944in}}{%
\pgfpathmoveto{\pgfqpoint{0.000000in}{-0.006944in}}%
\pgfpathcurveto{\pgfqpoint{0.001842in}{-0.006944in}}{\pgfqpoint{0.003608in}{-0.006213in}}{\pgfqpoint{0.004910in}{-0.004910in}}%
\pgfpathcurveto{\pgfqpoint{0.006213in}{-0.003608in}}{\pgfqpoint{0.006944in}{-0.001842in}}{\pgfqpoint{0.006944in}{0.000000in}}%
\pgfpathcurveto{\pgfqpoint{0.006944in}{0.001842in}}{\pgfqpoint{0.006213in}{0.003608in}}{\pgfqpoint{0.004910in}{0.004910in}}%
\pgfpathcurveto{\pgfqpoint{0.003608in}{0.006213in}}{\pgfqpoint{0.001842in}{0.006944in}}{\pgfqpoint{0.000000in}{0.006944in}}%
\pgfpathcurveto{\pgfqpoint{-0.001842in}{0.006944in}}{\pgfqpoint{-0.003608in}{0.006213in}}{\pgfqpoint{-0.004910in}{0.004910in}}%
\pgfpathcurveto{\pgfqpoint{-0.006213in}{0.003608in}}{\pgfqpoint{-0.006944in}{0.001842in}}{\pgfqpoint{-0.006944in}{0.000000in}}%
\pgfpathcurveto{\pgfqpoint{-0.006944in}{-0.001842in}}{\pgfqpoint{-0.006213in}{-0.003608in}}{\pgfqpoint{-0.004910in}{-0.004910in}}%
\pgfpathcurveto{\pgfqpoint{-0.003608in}{-0.006213in}}{\pgfqpoint{-0.001842in}{-0.006944in}}{\pgfqpoint{0.000000in}{-0.006944in}}%
\pgfpathlineto{\pgfqpoint{0.000000in}{-0.006944in}}%
\pgfpathclose%
\pgfusepath{stroke,fill}%
}%
\begin{pgfscope}%
\pgfsys@transformshift{0.599960in}{0.663805in}%
\pgfsys@useobject{currentmarker}{}%
\end{pgfscope}%
\begin{pgfscope}%
\pgfsys@transformshift{0.755760in}{0.618622in}%
\pgfsys@useobject{currentmarker}{}%
\end{pgfscope}%
\begin{pgfscope}%
\pgfsys@transformshift{0.846897in}{0.651992in}%
\pgfsys@useobject{currentmarker}{}%
\end{pgfscope}%
\begin{pgfscope}%
\pgfsys@transformshift{0.911560in}{0.646224in}%
\pgfsys@useobject{currentmarker}{}%
\end{pgfscope}%
\begin{pgfscope}%
\pgfsys@transformshift{0.961716in}{0.650738in}%
\pgfsys@useobject{currentmarker}{}%
\end{pgfscope}%
\begin{pgfscope}%
\pgfsys@transformshift{1.002697in}{0.619601in}%
\pgfsys@useobject{currentmarker}{}%
\end{pgfscope}%
\begin{pgfscope}%
\pgfsys@transformshift{1.037345in}{0.703596in}%
\pgfsys@useobject{currentmarker}{}%
\end{pgfscope}%
\begin{pgfscope}%
\pgfsys@transformshift{1.067360in}{0.721542in}%
\pgfsys@useobject{currentmarker}{}%
\end{pgfscope}%
\begin{pgfscope}%
\pgfsys@transformshift{1.093834in}{0.711710in}%
\pgfsys@useobject{currentmarker}{}%
\end{pgfscope}%
\begin{pgfscope}%
\pgfsys@transformshift{1.117516in}{0.701460in}%
\pgfsys@useobject{currentmarker}{}%
\end{pgfscope}%
\begin{pgfscope}%
\pgfsys@transformshift{1.138939in}{0.674348in}%
\pgfsys@useobject{currentmarker}{}%
\end{pgfscope}%
\begin{pgfscope}%
\pgfsys@transformshift{1.158497in}{0.714512in}%
\pgfsys@useobject{currentmarker}{}%
\end{pgfscope}%
\begin{pgfscope}%
\pgfsys@transformshift{1.176488in}{0.691000in}%
\pgfsys@useobject{currentmarker}{}%
\end{pgfscope}%
\begin{pgfscope}%
\pgfsys@transformshift{1.193145in}{0.630301in}%
\pgfsys@useobject{currentmarker}{}%
\end{pgfscope}%
\begin{pgfscope}%
\pgfsys@transformshift{1.208653in}{0.701372in}%
\pgfsys@useobject{currentmarker}{}%
\end{pgfscope}%
\begin{pgfscope}%
\pgfsys@transformshift{1.223159in}{0.729470in}%
\pgfsys@useobject{currentmarker}{}%
\end{pgfscope}%
\begin{pgfscope}%
\pgfsys@transformshift{1.236786in}{0.708209in}%
\pgfsys@useobject{currentmarker}{}%
\end{pgfscope}%
\begin{pgfscope}%
\pgfsys@transformshift{1.249634in}{0.605316in}%
\pgfsys@useobject{currentmarker}{}%
\end{pgfscope}%
\begin{pgfscope}%
\pgfsys@transformshift{1.261786in}{0.661660in}%
\pgfsys@useobject{currentmarker}{}%
\end{pgfscope}%
\begin{pgfscope}%
\pgfsys@transformshift{1.273316in}{0.731416in}%
\pgfsys@useobject{currentmarker}{}%
\end{pgfscope}%
\begin{pgfscope}%
\pgfsys@transformshift{1.284282in}{0.749145in}%
\pgfsys@useobject{currentmarker}{}%
\end{pgfscope}%
\begin{pgfscope}%
\pgfsys@transformshift{1.294739in}{0.714727in}%
\pgfsys@useobject{currentmarker}{}%
\end{pgfscope}%
\begin{pgfscope}%
\pgfsys@transformshift{1.309564in}{0.676465in}%
\pgfsys@useobject{currentmarker}{}%
\end{pgfscope}%
\begin{pgfscope}%
\pgfsys@transformshift{1.323472in}{0.677410in}%
\pgfsys@useobject{currentmarker}{}%
\end{pgfscope}%
\begin{pgfscope}%
\pgfsys@transformshift{1.332288in}{0.659617in}%
\pgfsys@useobject{currentmarker}{}%
\end{pgfscope}%
\begin{pgfscope}%
\pgfsys@transformshift{1.340771in}{0.657152in}%
\pgfsys@useobject{currentmarker}{}%
\end{pgfscope}%
\begin{pgfscope}%
\pgfsys@transformshift{1.348945in}{0.706322in}%
\pgfsys@useobject{currentmarker}{}%
\end{pgfscope}%
\begin{pgfscope}%
\pgfsys@transformshift{1.360675in}{0.696295in}%
\pgfsys@useobject{currentmarker}{}%
\end{pgfscope}%
\begin{pgfscope}%
\pgfsys@transformshift{1.371823in}{0.699681in}%
\pgfsys@useobject{currentmarker}{}%
\end{pgfscope}%
\begin{pgfscope}%
\pgfsys@transformshift{1.382444in}{0.679848in}%
\pgfsys@useobject{currentmarker}{}%
\end{pgfscope}%
\begin{pgfscope}%
\pgfsys@transformshift{1.392586in}{0.711199in}%
\pgfsys@useobject{currentmarker}{}%
\end{pgfscope}%
\begin{pgfscope}%
\pgfsys@transformshift{1.402290in}{0.694159in}%
\pgfsys@useobject{currentmarker}{}%
\end{pgfscope}%
\begin{pgfscope}%
\pgfsys@transformshift{1.414609in}{0.686601in}%
\pgfsys@useobject{currentmarker}{}%
\end{pgfscope}%
\begin{pgfscope}%
\pgfsys@transformshift{1.426288in}{0.659788in}%
\pgfsys@useobject{currentmarker}{}%
\end{pgfscope}%
\begin{pgfscope}%
\pgfsys@transformshift{1.434666in}{0.708572in}%
\pgfsys@useobject{currentmarker}{}%
\end{pgfscope}%
\begin{pgfscope}%
\pgfsys@transformshift{1.442742in}{0.732958in}%
\pgfsys@useobject{currentmarker}{}%
\end{pgfscope}%
\begin{pgfscope}%
\pgfsys@transformshift{1.453078in}{0.663074in}%
\pgfsys@useobject{currentmarker}{}%
\end{pgfscope}%
\begin{pgfscope}%
\pgfsys@transformshift{1.465364in}{0.674540in}%
\pgfsys@useobject{currentmarker}{}%
\end{pgfscope}%
\begin{pgfscope}%
\pgfsys@transformshift{1.477013in}{0.667392in}%
\pgfsys@useobject{currentmarker}{}%
\end{pgfscope}%
\begin{pgfscope}%
\pgfsys@transformshift{1.485916in}{0.608273in}%
\pgfsys@useobject{currentmarker}{}%
\end{pgfscope}%
\begin{pgfscope}%
\pgfsys@transformshift{1.496570in}{0.700949in}%
\pgfsys@useobject{currentmarker}{}%
\end{pgfscope}%
\begin{pgfscope}%
\pgfsys@transformshift{1.506743in}{0.723191in}%
\pgfsys@useobject{currentmarker}{}%
\end{pgfscope}%
\begin{pgfscope}%
\pgfsys@transformshift{1.516475in}{0.704537in}%
\pgfsys@useobject{currentmarker}{}%
\end{pgfscope}%
\begin{pgfscope}%
\pgfsys@transformshift{1.527623in}{0.648969in}%
\pgfsys@useobject{currentmarker}{}%
\end{pgfscope}%
\begin{pgfscope}%
\pgfsys@transformshift{1.538244in}{0.653022in}%
\pgfsys@useobject{currentmarker}{}%
\end{pgfscope}%
\begin{pgfscope}%
\pgfsys@transformshift{1.548386in}{0.696554in}%
\pgfsys@useobject{currentmarker}{}%
\end{pgfscope}%
\begin{pgfscope}%
\pgfsys@transformshift{1.559667in}{0.685699in}%
\pgfsys@useobject{currentmarker}{}%
\end{pgfscope}%
\begin{pgfscope}%
\pgfsys@transformshift{1.570409in}{0.687434in}%
\pgfsys@useobject{currentmarker}{}%
\end{pgfscope}%
\begin{pgfscope}%
\pgfsys@transformshift{1.580661in}{0.667780in}%
\pgfsys@useobject{currentmarker}{}%
\end{pgfscope}%
\begin{pgfscope}%
\pgfsys@transformshift{1.590465in}{0.672550in}%
\pgfsys@useobject{currentmarker}{}%
\end{pgfscope}%
\begin{pgfscope}%
\pgfsys@transformshift{1.599860in}{0.648800in}%
\pgfsys@useobject{currentmarker}{}%
\end{pgfscope}%
\begin{pgfscope}%
\pgfsys@transformshift{1.611390in}{0.661268in}%
\pgfsys@useobject{currentmarker}{}%
\end{pgfscope}%
\begin{pgfscope}%
\pgfsys@transformshift{1.622356in}{0.679397in}%
\pgfsys@useobject{currentmarker}{}%
\end{pgfscope}%
\begin{pgfscope}%
\pgfsys@transformshift{1.631675in}{0.661258in}%
\pgfsys@useobject{currentmarker}{}%
\end{pgfscope}%
\begin{pgfscope}%
\pgfsys@transformshift{1.641716in}{0.674669in}%
\pgfsys@useobject{currentmarker}{}%
\end{pgfscope}%
\begin{pgfscope}%
\pgfsys@transformshift{1.652370in}{0.698131in}%
\pgfsys@useobject{currentmarker}{}%
\end{pgfscope}%
\begin{pgfscope}%
\pgfsys@transformshift{1.663535in}{0.673649in}%
\pgfsys@useobject{currentmarker}{}%
\end{pgfscope}%
\begin{pgfscope}%
\pgfsys@transformshift{1.674171in}{0.680894in}%
\pgfsys@useobject{currentmarker}{}%
\end{pgfscope}%
\begin{pgfscope}%
\pgfsys@transformshift{1.684327in}{0.676831in}%
\pgfsys@useobject{currentmarker}{}%
\end{pgfscope}%
\begin{pgfscope}%
\pgfsys@transformshift{1.694907in}{0.685461in}%
\pgfsys@useobject{currentmarker}{}%
\end{pgfscope}%
\begin{pgfscope}%
\pgfsys@transformshift{1.705010in}{0.716158in}%
\pgfsys@useobject{currentmarker}{}%
\end{pgfscope}%
\begin{pgfscope}%
\pgfsys@transformshift{1.715467in}{0.691709in}%
\pgfsys@useobject{currentmarker}{}%
\end{pgfscope}%
\begin{pgfscope}%
\pgfsys@transformshift{1.726209in}{0.696848in}%
\pgfsys@useobject{currentmarker}{}%
\end{pgfscope}%
\begin{pgfscope}%
\pgfsys@transformshift{1.736461in}{0.702946in}%
\pgfsys@useobject{currentmarker}{}%
\end{pgfscope}%
\begin{pgfscope}%
\pgfsys@transformshift{1.746950in}{0.683494in}%
\pgfsys@useobject{currentmarker}{}%
\end{pgfscope}%
\begin{pgfscope}%
\pgfsys@transformshift{1.756971in}{0.627751in}%
\pgfsys@useobject{currentmarker}{}%
\end{pgfscope}%
\begin{pgfscope}%
\pgfsys@transformshift{1.767189in}{0.696131in}%
\pgfsys@useobject{currentmarker}{}%
\end{pgfscope}%
\begin{pgfscope}%
\pgfsys@transformshift{1.778156in}{0.708459in}%
\pgfsys@useobject{currentmarker}{}%
\end{pgfscope}%
\begin{pgfscope}%
\pgfsys@transformshift{1.788612in}{0.720422in}%
\pgfsys@useobject{currentmarker}{}%
\end{pgfscope}%
\begin{pgfscope}%
\pgfsys@transformshift{1.798604in}{0.684406in}%
\pgfsys@useobject{currentmarker}{}%
\end{pgfscope}%
\begin{pgfscope}%
\pgfsys@transformshift{1.808690in}{0.683552in}%
\pgfsys@useobject{currentmarker}{}%
\end{pgfscope}%
\begin{pgfscope}%
\pgfsys@transformshift{1.819335in}{0.667244in}%
\pgfsys@useobject{currentmarker}{}%
\end{pgfscope}%
\begin{pgfscope}%
\pgfsys@transformshift{1.829971in}{0.694637in}%
\pgfsys@useobject{currentmarker}{}%
\end{pgfscope}%
\begin{pgfscope}%
\pgfsys@transformshift{1.840127in}{0.687990in}%
\pgfsys@useobject{currentmarker}{}%
\end{pgfscope}%
\begin{pgfscope}%
\pgfsys@transformshift{1.850275in}{0.705543in}%
\pgfsys@useobject{currentmarker}{}%
\end{pgfscope}%
\begin{pgfscope}%
\pgfsys@transformshift{1.860810in}{0.669379in}%
\pgfsys@useobject{currentmarker}{}%
\end{pgfscope}%
\begin{pgfscope}%
\pgfsys@transformshift{1.871266in}{0.681188in}%
\pgfsys@useobject{currentmarker}{}%
\end{pgfscope}%
\begin{pgfscope}%
\pgfsys@transformshift{1.881634in}{0.692273in}%
\pgfsys@useobject{currentmarker}{}%
\end{pgfscope}%
\begin{pgfscope}%
\pgfsys@transformshift{1.892260in}{0.696708in}%
\pgfsys@useobject{currentmarker}{}%
\end{pgfscope}%
\begin{pgfscope}%
\pgfsys@transformshift{1.902749in}{0.717949in}%
\pgfsys@useobject{currentmarker}{}%
\end{pgfscope}%
\begin{pgfscope}%
\pgfsys@transformshift{1.913097in}{0.683058in}%
\pgfsys@useobject{currentmarker}{}%
\end{pgfscope}%
\begin{pgfscope}%
\pgfsys@transformshift{1.923613in}{0.676800in}%
\pgfsys@useobject{currentmarker}{}%
\end{pgfscope}%
\begin{pgfscope}%
\pgfsys@transformshift{1.933956in}{0.674384in}%
\pgfsys@useobject{currentmarker}{}%
\end{pgfscope}%
\begin{pgfscope}%
\pgfsys@transformshift{1.944128in}{0.692094in}%
\pgfsys@useobject{currentmarker}{}%
\end{pgfscope}%
\begin{pgfscope}%
\pgfsys@transformshift{1.954404in}{0.704235in}%
\pgfsys@useobject{currentmarker}{}%
\end{pgfscope}%
\begin{pgfscope}%
\pgfsys@transformshift{1.965008in}{0.691487in}%
\pgfsys@useobject{currentmarker}{}%
\end{pgfscope}%
\begin{pgfscope}%
\pgfsys@transformshift{1.975629in}{0.685705in}%
\pgfsys@useobject{currentmarker}{}%
\end{pgfscope}%
\begin{pgfscope}%
\pgfsys@transformshift{1.986007in}{0.692158in}%
\pgfsys@useobject{currentmarker}{}%
\end{pgfscope}%
\begin{pgfscope}%
\pgfsys@transformshift{1.996378in}{0.695510in}%
\pgfsys@useobject{currentmarker}{}%
\end{pgfscope}%
\begin{pgfscope}%
\pgfsys@transformshift{2.006721in}{0.688511in}%
\pgfsys@useobject{currentmarker}{}%
\end{pgfscope}%
\begin{pgfscope}%
\pgfsys@transformshift{2.017021in}{0.695353in}%
\pgfsys@useobject{currentmarker}{}%
\end{pgfscope}%
\begin{pgfscope}%
\pgfsys@transformshift{2.027459in}{0.695166in}%
\pgfsys@useobject{currentmarker}{}%
\end{pgfscope}%
\begin{pgfscope}%
\pgfsys@transformshift{2.037808in}{0.696314in}%
\pgfsys@useobject{currentmarker}{}%
\end{pgfscope}%
\begin{pgfscope}%
\pgfsys@transformshift{2.048239in}{0.700824in}%
\pgfsys@useobject{currentmarker}{}%
\end{pgfscope}%
\begin{pgfscope}%
\pgfsys@transformshift{2.058720in}{0.694145in}%
\pgfsys@useobject{currentmarker}{}%
\end{pgfscope}%
\begin{pgfscope}%
\pgfsys@transformshift{2.069060in}{0.687403in}%
\pgfsys@useobject{currentmarker}{}%
\end{pgfscope}%
\begin{pgfscope}%
\pgfsys@transformshift{2.079412in}{0.696275in}%
\pgfsys@useobject{currentmarker}{}%
\end{pgfscope}%
\begin{pgfscope}%
\pgfsys@transformshift{2.089756in}{0.689918in}%
\pgfsys@useobject{currentmarker}{}%
\end{pgfscope}%
\begin{pgfscope}%
\pgfsys@transformshift{2.100212in}{0.707787in}%
\pgfsys@useobject{currentmarker}{}%
\end{pgfscope}%
\begin{pgfscope}%
\pgfsys@transformshift{2.110610in}{0.698158in}%
\pgfsys@useobject{currentmarker}{}%
\end{pgfscope}%
\begin{pgfscope}%
\pgfsys@transformshift{2.121067in}{0.695132in}%
\pgfsys@useobject{currentmarker}{}%
\end{pgfscope}%
\begin{pgfscope}%
\pgfsys@transformshift{2.131552in}{0.692670in}%
\pgfsys@useobject{currentmarker}{}%
\end{pgfscope}%
\begin{pgfscope}%
\pgfsys@transformshift{2.141925in}{0.694971in}%
\pgfsys@useobject{currentmarker}{}%
\end{pgfscope}%
\begin{pgfscope}%
\pgfsys@transformshift{2.152290in}{0.691283in}%
\pgfsys@useobject{currentmarker}{}%
\end{pgfscope}%
\begin{pgfscope}%
\pgfsys@transformshift{2.162629in}{0.691112in}%
\pgfsys@useobject{currentmarker}{}%
\end{pgfscope}%
\begin{pgfscope}%
\pgfsys@transformshift{2.173026in}{0.693066in}%
\pgfsys@useobject{currentmarker}{}%
\end{pgfscope}%
\begin{pgfscope}%
\pgfsys@transformshift{2.183455in}{0.697970in}%
\pgfsys@useobject{currentmarker}{}%
\end{pgfscope}%
\begin{pgfscope}%
\pgfsys@transformshift{2.193889in}{0.696976in}%
\pgfsys@useobject{currentmarker}{}%
\end{pgfscope}%
\begin{pgfscope}%
\pgfsys@transformshift{2.204307in}{0.684822in}%
\pgfsys@useobject{currentmarker}{}%
\end{pgfscope}%
\begin{pgfscope}%
\pgfsys@transformshift{2.214690in}{0.682311in}%
\pgfsys@useobject{currentmarker}{}%
\end{pgfscope}%
\begin{pgfscope}%
\pgfsys@transformshift{2.225104in}{0.696410in}%
\pgfsys@useobject{currentmarker}{}%
\end{pgfscope}%
\begin{pgfscope}%
\pgfsys@transformshift{2.235523in}{0.699589in}%
\pgfsys@useobject{currentmarker}{}%
\end{pgfscope}%
\begin{pgfscope}%
\pgfsys@transformshift{2.245927in}{0.686990in}%
\pgfsys@useobject{currentmarker}{}%
\end{pgfscope}%
\begin{pgfscope}%
\pgfsys@transformshift{2.256295in}{0.691147in}%
\pgfsys@useobject{currentmarker}{}%
\end{pgfscope}%
\begin{pgfscope}%
\pgfsys@transformshift{2.266681in}{0.697284in}%
\pgfsys@useobject{currentmarker}{}%
\end{pgfscope}%
\begin{pgfscope}%
\pgfsys@transformshift{2.277125in}{0.694587in}%
\pgfsys@useobject{currentmarker}{}%
\end{pgfscope}%
\begin{pgfscope}%
\pgfsys@transformshift{2.287537in}{0.701604in}%
\pgfsys@useobject{currentmarker}{}%
\end{pgfscope}%
\begin{pgfscope}%
\pgfsys@transformshift{2.297901in}{0.687224in}%
\pgfsys@useobject{currentmarker}{}%
\end{pgfscope}%
\begin{pgfscope}%
\pgfsys@transformshift{2.308315in}{0.688592in}%
\pgfsys@useobject{currentmarker}{}%
\end{pgfscope}%
\begin{pgfscope}%
\pgfsys@transformshift{2.313593in}{0.674838in}%
\pgfsys@useobject{currentmarker}{}%
\end{pgfscope}%
\end{pgfscope}%
\begin{pgfscope}%
\pgfsetrectcap%
\pgfsetmiterjoin%
\pgfsetlinewidth{0.803000pt}%
\definecolor{currentstroke}{rgb}{0.000000,0.000000,0.000000}%
\pgfsetstrokecolor{currentstroke}%
\pgfsetdash{}{0pt}%
\pgfpathmoveto{\pgfqpoint{0.514278in}{0.417642in}}%
\pgfpathlineto{\pgfqpoint{0.514278in}{1.789039in}}%
\pgfusepath{stroke}%
\end{pgfscope}%
\begin{pgfscope}%
\pgfsetrectcap%
\pgfsetmiterjoin%
\pgfsetlinewidth{0.803000pt}%
\definecolor{currentstroke}{rgb}{0.000000,0.000000,0.000000}%
\pgfsetstrokecolor{currentstroke}%
\pgfsetdash{}{0pt}%
\pgfpathmoveto{\pgfqpoint{2.399275in}{0.417642in}}%
\pgfpathlineto{\pgfqpoint{2.399275in}{1.789039in}}%
\pgfusepath{stroke}%
\end{pgfscope}%
\begin{pgfscope}%
\pgfsetrectcap%
\pgfsetmiterjoin%
\pgfsetlinewidth{0.803000pt}%
\definecolor{currentstroke}{rgb}{0.000000,0.000000,0.000000}%
\pgfsetstrokecolor{currentstroke}%
\pgfsetdash{}{0pt}%
\pgfpathmoveto{\pgfqpoint{0.514278in}{0.417642in}}%
\pgfpathlineto{\pgfqpoint{2.399275in}{0.417642in}}%
\pgfusepath{stroke}%
\end{pgfscope}%
\begin{pgfscope}%
\pgfsetrectcap%
\pgfsetmiterjoin%
\pgfsetlinewidth{0.803000pt}%
\definecolor{currentstroke}{rgb}{0.000000,0.000000,0.000000}%
\pgfsetstrokecolor{currentstroke}%
\pgfsetdash{}{0pt}%
\pgfpathmoveto{\pgfqpoint{0.514278in}{1.789039in}}%
\pgfpathlineto{\pgfqpoint{2.399275in}{1.789039in}}%
\pgfusepath{stroke}%
\end{pgfscope}%
\begin{pgfscope}%
\pgfsetbuttcap%
\pgfsetmiterjoin%
\definecolor{currentfill}{rgb}{1.000000,1.000000,1.000000}%
\pgfsetfillcolor{currentfill}%
\pgfsetfillopacity{0.800000}%
\pgfsetlinewidth{1.003750pt}%
\definecolor{currentstroke}{rgb}{0.800000,0.800000,0.800000}%
\pgfsetstrokecolor{currentstroke}%
\pgfsetstrokeopacity{0.800000}%
\pgfsetdash{}{0pt}%
\pgfpathmoveto{\pgfqpoint{1.713209in}{1.523128in}}%
\pgfpathlineto{\pgfqpoint{2.321497in}{1.523128in}}%
\pgfpathquadraticcurveto{\pgfqpoint{2.343719in}{1.523128in}}{\pgfqpoint{2.343719in}{1.545351in}}%
\pgfpathlineto{\pgfqpoint{2.343719in}{1.711261in}}%
\pgfpathquadraticcurveto{\pgfqpoint{2.343719in}{1.733483in}}{\pgfqpoint{2.321497in}{1.733483in}}%
\pgfpathlineto{\pgfqpoint{1.713209in}{1.733483in}}%
\pgfpathquadraticcurveto{\pgfqpoint{1.690987in}{1.733483in}}{\pgfqpoint{1.690987in}{1.711261in}}%
\pgfpathlineto{\pgfqpoint{1.690987in}{1.545351in}}%
\pgfpathquadraticcurveto{\pgfqpoint{1.690987in}{1.523128in}}{\pgfqpoint{1.713209in}{1.523128in}}%
\pgfpathlineto{\pgfqpoint{1.713209in}{1.523128in}}%
\pgfpathclose%
\pgfusepath{stroke,fill}%
\end{pgfscope}%
\begin{pgfscope}%
\pgfsetbuttcap%
\pgfsetroundjoin%
\pgfsetlinewidth{1.505625pt}%
\definecolor{currentstroke}{rgb}{0.003922,0.450980,0.698039}%
\pgfsetstrokecolor{currentstroke}%
\pgfsetdash{{5.550000pt}{2.400000pt}}{0.000000pt}%
\pgfpathmoveto{\pgfqpoint{1.735431in}{1.628067in}}%
\pgfpathlineto{\pgfqpoint{1.846542in}{1.628067in}}%
\pgfpathlineto{\pgfqpoint{1.957653in}{1.628067in}}%
\pgfusepath{stroke}%
\end{pgfscope}%
\begin{pgfscope}%
\definecolor{textcolor}{rgb}{0.000000,0.000000,0.000000}%
\pgfsetstrokecolor{textcolor}%
\pgfsetfillcolor{textcolor}%
\pgftext[x=2.046542in,y=1.589178in,left,base]{\color{textcolor}{\rmfamily\fontsize{8.000000}{9.600000}\selectfont\catcode`\^=\active\def^{\ifmmode\sp\else\^{}\fi}\catcode`\%=\active\def%{\%}$\displaystyle h_{0}f^{0}$}}%
\end{pgfscope}%
\end{pgfpicture}%
\makeatother%
\endgroup%
% data/simulations/sim_allan_variance.py
        } % scalebox
        \caption{Power spectral density}
        \label{fig:white_noise_psd}
    \end{subfigure}
    \hfill
    \begin{subfigure}{0.32\linewidth}
        \centering
        \scalebox{0.75}{%
            %% Creator: Matplotlib, PGF backend
%%
%% To include the figure in your LaTeX document, write
%%   \input{<filename>.pgf}
%%
%% Make sure the required packages are loaded in your preamble
%%   \usepackage{pgf}
%%
%% Also ensure that all the required font packages are loaded; for instance,
%% the lmodern package is sometimes necessary when using math font.
%%   \usepackage{lmodern}
%%
%% Figures using additional raster images can only be included by \input if
%% they are in the same directory as the main LaTeX file. For loading figures
%% from other directories you can use the `import` package
%%   \usepackage{import}
%%
%% and then include the figures with
%%   \import{<path to file>}{<filename>.pgf}
%%
%% Matplotlib used the following preamble
%%   \def\mathdefault#1{#1}
%%   \everymath=\expandafter{\the\everymath\displaystyle}
%%   \usepackage{siunitx}
%%   \sisetup{per-mode = symbol}%
%%   \ifdefined\pdftexversion\else  % non-pdftex case.
%%     \usepackage{fontspec}
%%   \fi
%%   \makeatletter\@ifpackageloaded{underscore}{}{\usepackage[strings]{underscore}}\makeatother
%%
\begingroup%
\makeatletter%
\begin{pgfpicture}%
\pgfpathrectangle{\pgfpointorigin}{\pgfqpoint{2.440945in}{1.830709in}}%
\pgfusepath{use as bounding box, clip}%
\begin{pgfscope}%
\pgfsetbuttcap%
\pgfsetmiterjoin%
\definecolor{currentfill}{rgb}{1.000000,1.000000,1.000000}%
\pgfsetfillcolor{currentfill}%
\pgfsetlinewidth{0.000000pt}%
\definecolor{currentstroke}{rgb}{1.000000,1.000000,1.000000}%
\pgfsetstrokecolor{currentstroke}%
\pgfsetdash{}{0pt}%
\pgfpathmoveto{\pgfqpoint{0.000000in}{0.000000in}}%
\pgfpathlineto{\pgfqpoint{2.440945in}{0.000000in}}%
\pgfpathlineto{\pgfqpoint{2.440945in}{1.830709in}}%
\pgfpathlineto{\pgfqpoint{0.000000in}{1.830709in}}%
\pgfpathlineto{\pgfqpoint{0.000000in}{0.000000in}}%
\pgfpathclose%
\pgfusepath{fill}%
\end{pgfscope}%
\begin{pgfscope}%
\pgfsetbuttcap%
\pgfsetmiterjoin%
\definecolor{currentfill}{rgb}{1.000000,1.000000,1.000000}%
\pgfsetfillcolor{currentfill}%
\pgfsetlinewidth{0.000000pt}%
\definecolor{currentstroke}{rgb}{0.000000,0.000000,0.000000}%
\pgfsetstrokecolor{currentstroke}%
\pgfsetstrokeopacity{0.000000}%
\pgfsetdash{}{0pt}%
\pgfpathmoveto{\pgfqpoint{0.589510in}{0.417642in}}%
\pgfpathlineto{\pgfqpoint{2.399275in}{0.417642in}}%
\pgfpathlineto{\pgfqpoint{2.399275in}{1.789039in}}%
\pgfpathlineto{\pgfqpoint{0.589510in}{1.789039in}}%
\pgfpathlineto{\pgfqpoint{0.589510in}{0.417642in}}%
\pgfpathclose%
\pgfusepath{fill}%
\end{pgfscope}%
\begin{pgfscope}%
\pgfpathrectangle{\pgfqpoint{0.589510in}{0.417642in}}{\pgfqpoint{1.809765in}{1.371397in}}%
\pgfusepath{clip}%
\pgfsetrectcap%
\pgfsetroundjoin%
\pgfsetlinewidth{0.803000pt}%
\definecolor{currentstroke}{rgb}{0.450000,0.450000,0.450000}%
\pgfsetstrokecolor{currentstroke}%
\pgfsetdash{}{0pt}%
\pgfpathmoveto{\pgfqpoint{0.671772in}{0.417642in}}%
\pgfpathlineto{\pgfqpoint{0.671772in}{1.789039in}}%
\pgfusepath{stroke}%
\end{pgfscope}%
\begin{pgfscope}%
\pgfsetbuttcap%
\pgfsetroundjoin%
\definecolor{currentfill}{rgb}{0.000000,0.000000,0.000000}%
\pgfsetfillcolor{currentfill}%
\pgfsetlinewidth{0.803000pt}%
\definecolor{currentstroke}{rgb}{0.000000,0.000000,0.000000}%
\pgfsetstrokecolor{currentstroke}%
\pgfsetdash{}{0pt}%
\pgfsys@defobject{currentmarker}{\pgfqpoint{0.000000in}{-0.048611in}}{\pgfqpoint{0.000000in}{0.000000in}}{%
\pgfpathmoveto{\pgfqpoint{0.000000in}{0.000000in}}%
\pgfpathlineto{\pgfqpoint{0.000000in}{-0.048611in}}%
\pgfusepath{stroke,fill}%
}%
\begin{pgfscope}%
\pgfsys@transformshift{0.671772in}{0.417642in}%
\pgfsys@useobject{currentmarker}{}%
\end{pgfscope}%
\end{pgfscope}%
\begin{pgfscope}%
\definecolor{textcolor}{rgb}{0.000000,0.000000,0.000000}%
\pgfsetstrokecolor{textcolor}%
\pgfsetfillcolor{textcolor}%
\pgftext[x=0.671772in,y=0.320420in,,top]{\color{textcolor}{\rmfamily\fontsize{8.000000}{9.600000}\selectfont\catcode`\^=\active\def^{\ifmmode\sp\else\^{}\fi}\catcode`\%=\active\def%{\%}$\mathdefault{10^{0}}$}}%
\end{pgfscope}%
\begin{pgfscope}%
\pgfpathrectangle{\pgfqpoint{0.589510in}{0.417642in}}{\pgfqpoint{1.809765in}{1.371397in}}%
\pgfusepath{clip}%
\pgfsetrectcap%
\pgfsetroundjoin%
\pgfsetlinewidth{0.803000pt}%
\definecolor{currentstroke}{rgb}{0.450000,0.450000,0.450000}%
\pgfsetstrokecolor{currentstroke}%
\pgfsetdash{}{0pt}%
\pgfpathmoveto{\pgfqpoint{1.128522in}{0.417642in}}%
\pgfpathlineto{\pgfqpoint{1.128522in}{1.789039in}}%
\pgfusepath{stroke}%
\end{pgfscope}%
\begin{pgfscope}%
\pgfsetbuttcap%
\pgfsetroundjoin%
\definecolor{currentfill}{rgb}{0.000000,0.000000,0.000000}%
\pgfsetfillcolor{currentfill}%
\pgfsetlinewidth{0.803000pt}%
\definecolor{currentstroke}{rgb}{0.000000,0.000000,0.000000}%
\pgfsetstrokecolor{currentstroke}%
\pgfsetdash{}{0pt}%
\pgfsys@defobject{currentmarker}{\pgfqpoint{0.000000in}{-0.048611in}}{\pgfqpoint{0.000000in}{0.000000in}}{%
\pgfpathmoveto{\pgfqpoint{0.000000in}{0.000000in}}%
\pgfpathlineto{\pgfqpoint{0.000000in}{-0.048611in}}%
\pgfusepath{stroke,fill}%
}%
\begin{pgfscope}%
\pgfsys@transformshift{1.128522in}{0.417642in}%
\pgfsys@useobject{currentmarker}{}%
\end{pgfscope}%
\end{pgfscope}%
\begin{pgfscope}%
\definecolor{textcolor}{rgb}{0.000000,0.000000,0.000000}%
\pgfsetstrokecolor{textcolor}%
\pgfsetfillcolor{textcolor}%
\pgftext[x=1.128522in,y=0.320420in,,top]{\color{textcolor}{\rmfamily\fontsize{8.000000}{9.600000}\selectfont\catcode`\^=\active\def^{\ifmmode\sp\else\^{}\fi}\catcode`\%=\active\def%{\%}$\mathdefault{10^{1}}$}}%
\end{pgfscope}%
\begin{pgfscope}%
\pgfpathrectangle{\pgfqpoint{0.589510in}{0.417642in}}{\pgfqpoint{1.809765in}{1.371397in}}%
\pgfusepath{clip}%
\pgfsetrectcap%
\pgfsetroundjoin%
\pgfsetlinewidth{0.803000pt}%
\definecolor{currentstroke}{rgb}{0.450000,0.450000,0.450000}%
\pgfsetstrokecolor{currentstroke}%
\pgfsetdash{}{0pt}%
\pgfpathmoveto{\pgfqpoint{1.585272in}{0.417642in}}%
\pgfpathlineto{\pgfqpoint{1.585272in}{1.789039in}}%
\pgfusepath{stroke}%
\end{pgfscope}%
\begin{pgfscope}%
\pgfsetbuttcap%
\pgfsetroundjoin%
\definecolor{currentfill}{rgb}{0.000000,0.000000,0.000000}%
\pgfsetfillcolor{currentfill}%
\pgfsetlinewidth{0.803000pt}%
\definecolor{currentstroke}{rgb}{0.000000,0.000000,0.000000}%
\pgfsetstrokecolor{currentstroke}%
\pgfsetdash{}{0pt}%
\pgfsys@defobject{currentmarker}{\pgfqpoint{0.000000in}{-0.048611in}}{\pgfqpoint{0.000000in}{0.000000in}}{%
\pgfpathmoveto{\pgfqpoint{0.000000in}{0.000000in}}%
\pgfpathlineto{\pgfqpoint{0.000000in}{-0.048611in}}%
\pgfusepath{stroke,fill}%
}%
\begin{pgfscope}%
\pgfsys@transformshift{1.585272in}{0.417642in}%
\pgfsys@useobject{currentmarker}{}%
\end{pgfscope}%
\end{pgfscope}%
\begin{pgfscope}%
\definecolor{textcolor}{rgb}{0.000000,0.000000,0.000000}%
\pgfsetstrokecolor{textcolor}%
\pgfsetfillcolor{textcolor}%
\pgftext[x=1.585272in,y=0.320420in,,top]{\color{textcolor}{\rmfamily\fontsize{8.000000}{9.600000}\selectfont\catcode`\^=\active\def^{\ifmmode\sp\else\^{}\fi}\catcode`\%=\active\def%{\%}$\mathdefault{10^{2}}$}}%
\end{pgfscope}%
\begin{pgfscope}%
\pgfpathrectangle{\pgfqpoint{0.589510in}{0.417642in}}{\pgfqpoint{1.809765in}{1.371397in}}%
\pgfusepath{clip}%
\pgfsetrectcap%
\pgfsetroundjoin%
\pgfsetlinewidth{0.803000pt}%
\definecolor{currentstroke}{rgb}{0.450000,0.450000,0.450000}%
\pgfsetstrokecolor{currentstroke}%
\pgfsetdash{}{0pt}%
\pgfpathmoveto{\pgfqpoint{2.042022in}{0.417642in}}%
\pgfpathlineto{\pgfqpoint{2.042022in}{1.789039in}}%
\pgfusepath{stroke}%
\end{pgfscope}%
\begin{pgfscope}%
\pgfsetbuttcap%
\pgfsetroundjoin%
\definecolor{currentfill}{rgb}{0.000000,0.000000,0.000000}%
\pgfsetfillcolor{currentfill}%
\pgfsetlinewidth{0.803000pt}%
\definecolor{currentstroke}{rgb}{0.000000,0.000000,0.000000}%
\pgfsetstrokecolor{currentstroke}%
\pgfsetdash{}{0pt}%
\pgfsys@defobject{currentmarker}{\pgfqpoint{0.000000in}{-0.048611in}}{\pgfqpoint{0.000000in}{0.000000in}}{%
\pgfpathmoveto{\pgfqpoint{0.000000in}{0.000000in}}%
\pgfpathlineto{\pgfqpoint{0.000000in}{-0.048611in}}%
\pgfusepath{stroke,fill}%
}%
\begin{pgfscope}%
\pgfsys@transformshift{2.042022in}{0.417642in}%
\pgfsys@useobject{currentmarker}{}%
\end{pgfscope}%
\end{pgfscope}%
\begin{pgfscope}%
\definecolor{textcolor}{rgb}{0.000000,0.000000,0.000000}%
\pgfsetstrokecolor{textcolor}%
\pgfsetfillcolor{textcolor}%
\pgftext[x=2.042022in,y=0.320420in,,top]{\color{textcolor}{\rmfamily\fontsize{8.000000}{9.600000}\selectfont\catcode`\^=\active\def^{\ifmmode\sp\else\^{}\fi}\catcode`\%=\active\def%{\%}$\mathdefault{10^{3}}$}}%
\end{pgfscope}%
\begin{pgfscope}%
\pgfpathrectangle{\pgfqpoint{0.589510in}{0.417642in}}{\pgfqpoint{1.809765in}{1.371397in}}%
\pgfusepath{clip}%
\pgfsetrectcap%
\pgfsetroundjoin%
\pgfsetlinewidth{0.803000pt}%
\definecolor{currentstroke}{rgb}{0.850000,0.850000,0.850000}%
\pgfsetstrokecolor{currentstroke}%
\pgfsetdash{}{0pt}%
\pgfpathmoveto{\pgfqpoint{0.601020in}{0.417642in}}%
\pgfpathlineto{\pgfqpoint{0.601020in}{1.789039in}}%
\pgfusepath{stroke}%
\end{pgfscope}%
\begin{pgfscope}%
\pgfsetbuttcap%
\pgfsetroundjoin%
\definecolor{currentfill}{rgb}{0.000000,0.000000,0.000000}%
\pgfsetfillcolor{currentfill}%
\pgfsetlinewidth{0.602250pt}%
\definecolor{currentstroke}{rgb}{0.000000,0.000000,0.000000}%
\pgfsetstrokecolor{currentstroke}%
\pgfsetdash{}{0pt}%
\pgfsys@defobject{currentmarker}{\pgfqpoint{0.000000in}{-0.027778in}}{\pgfqpoint{0.000000in}{0.000000in}}{%
\pgfpathmoveto{\pgfqpoint{0.000000in}{0.000000in}}%
\pgfpathlineto{\pgfqpoint{0.000000in}{-0.027778in}}%
\pgfusepath{stroke,fill}%
}%
\begin{pgfscope}%
\pgfsys@transformshift{0.601020in}{0.417642in}%
\pgfsys@useobject{currentmarker}{}%
\end{pgfscope}%
\end{pgfscope}%
\begin{pgfscope}%
\pgfpathrectangle{\pgfqpoint{0.589510in}{0.417642in}}{\pgfqpoint{1.809765in}{1.371397in}}%
\pgfusepath{clip}%
\pgfsetrectcap%
\pgfsetroundjoin%
\pgfsetlinewidth{0.803000pt}%
\definecolor{currentstroke}{rgb}{0.850000,0.850000,0.850000}%
\pgfsetstrokecolor{currentstroke}%
\pgfsetdash{}{0pt}%
\pgfpathmoveto{\pgfqpoint{0.627508in}{0.417642in}}%
\pgfpathlineto{\pgfqpoint{0.627508in}{1.789039in}}%
\pgfusepath{stroke}%
\end{pgfscope}%
\begin{pgfscope}%
\pgfsetbuttcap%
\pgfsetroundjoin%
\definecolor{currentfill}{rgb}{0.000000,0.000000,0.000000}%
\pgfsetfillcolor{currentfill}%
\pgfsetlinewidth{0.602250pt}%
\definecolor{currentstroke}{rgb}{0.000000,0.000000,0.000000}%
\pgfsetstrokecolor{currentstroke}%
\pgfsetdash{}{0pt}%
\pgfsys@defobject{currentmarker}{\pgfqpoint{0.000000in}{-0.027778in}}{\pgfqpoint{0.000000in}{0.000000in}}{%
\pgfpathmoveto{\pgfqpoint{0.000000in}{0.000000in}}%
\pgfpathlineto{\pgfqpoint{0.000000in}{-0.027778in}}%
\pgfusepath{stroke,fill}%
}%
\begin{pgfscope}%
\pgfsys@transformshift{0.627508in}{0.417642in}%
\pgfsys@useobject{currentmarker}{}%
\end{pgfscope}%
\end{pgfscope}%
\begin{pgfscope}%
\pgfpathrectangle{\pgfqpoint{0.589510in}{0.417642in}}{\pgfqpoint{1.809765in}{1.371397in}}%
\pgfusepath{clip}%
\pgfsetrectcap%
\pgfsetroundjoin%
\pgfsetlinewidth{0.803000pt}%
\definecolor{currentstroke}{rgb}{0.850000,0.850000,0.850000}%
\pgfsetstrokecolor{currentstroke}%
\pgfsetdash{}{0pt}%
\pgfpathmoveto{\pgfqpoint{0.650872in}{0.417642in}}%
\pgfpathlineto{\pgfqpoint{0.650872in}{1.789039in}}%
\pgfusepath{stroke}%
\end{pgfscope}%
\begin{pgfscope}%
\pgfsetbuttcap%
\pgfsetroundjoin%
\definecolor{currentfill}{rgb}{0.000000,0.000000,0.000000}%
\pgfsetfillcolor{currentfill}%
\pgfsetlinewidth{0.602250pt}%
\definecolor{currentstroke}{rgb}{0.000000,0.000000,0.000000}%
\pgfsetstrokecolor{currentstroke}%
\pgfsetdash{}{0pt}%
\pgfsys@defobject{currentmarker}{\pgfqpoint{0.000000in}{-0.027778in}}{\pgfqpoint{0.000000in}{0.000000in}}{%
\pgfpathmoveto{\pgfqpoint{0.000000in}{0.000000in}}%
\pgfpathlineto{\pgfqpoint{0.000000in}{-0.027778in}}%
\pgfusepath{stroke,fill}%
}%
\begin{pgfscope}%
\pgfsys@transformshift{0.650872in}{0.417642in}%
\pgfsys@useobject{currentmarker}{}%
\end{pgfscope}%
\end{pgfscope}%
\begin{pgfscope}%
\pgfpathrectangle{\pgfqpoint{0.589510in}{0.417642in}}{\pgfqpoint{1.809765in}{1.371397in}}%
\pgfusepath{clip}%
\pgfsetrectcap%
\pgfsetroundjoin%
\pgfsetlinewidth{0.803000pt}%
\definecolor{currentstroke}{rgb}{0.850000,0.850000,0.850000}%
\pgfsetstrokecolor{currentstroke}%
\pgfsetdash{}{0pt}%
\pgfpathmoveto{\pgfqpoint{0.809267in}{0.417642in}}%
\pgfpathlineto{\pgfqpoint{0.809267in}{1.789039in}}%
\pgfusepath{stroke}%
\end{pgfscope}%
\begin{pgfscope}%
\pgfsetbuttcap%
\pgfsetroundjoin%
\definecolor{currentfill}{rgb}{0.000000,0.000000,0.000000}%
\pgfsetfillcolor{currentfill}%
\pgfsetlinewidth{0.602250pt}%
\definecolor{currentstroke}{rgb}{0.000000,0.000000,0.000000}%
\pgfsetstrokecolor{currentstroke}%
\pgfsetdash{}{0pt}%
\pgfsys@defobject{currentmarker}{\pgfqpoint{0.000000in}{-0.027778in}}{\pgfqpoint{0.000000in}{0.000000in}}{%
\pgfpathmoveto{\pgfqpoint{0.000000in}{0.000000in}}%
\pgfpathlineto{\pgfqpoint{0.000000in}{-0.027778in}}%
\pgfusepath{stroke,fill}%
}%
\begin{pgfscope}%
\pgfsys@transformshift{0.809267in}{0.417642in}%
\pgfsys@useobject{currentmarker}{}%
\end{pgfscope}%
\end{pgfscope}%
\begin{pgfscope}%
\pgfpathrectangle{\pgfqpoint{0.589510in}{0.417642in}}{\pgfqpoint{1.809765in}{1.371397in}}%
\pgfusepath{clip}%
\pgfsetrectcap%
\pgfsetroundjoin%
\pgfsetlinewidth{0.803000pt}%
\definecolor{currentstroke}{rgb}{0.850000,0.850000,0.850000}%
\pgfsetstrokecolor{currentstroke}%
\pgfsetdash{}{0pt}%
\pgfpathmoveto{\pgfqpoint{0.889697in}{0.417642in}}%
\pgfpathlineto{\pgfqpoint{0.889697in}{1.789039in}}%
\pgfusepath{stroke}%
\end{pgfscope}%
\begin{pgfscope}%
\pgfsetbuttcap%
\pgfsetroundjoin%
\definecolor{currentfill}{rgb}{0.000000,0.000000,0.000000}%
\pgfsetfillcolor{currentfill}%
\pgfsetlinewidth{0.602250pt}%
\definecolor{currentstroke}{rgb}{0.000000,0.000000,0.000000}%
\pgfsetstrokecolor{currentstroke}%
\pgfsetdash{}{0pt}%
\pgfsys@defobject{currentmarker}{\pgfqpoint{0.000000in}{-0.027778in}}{\pgfqpoint{0.000000in}{0.000000in}}{%
\pgfpathmoveto{\pgfqpoint{0.000000in}{0.000000in}}%
\pgfpathlineto{\pgfqpoint{0.000000in}{-0.027778in}}%
\pgfusepath{stroke,fill}%
}%
\begin{pgfscope}%
\pgfsys@transformshift{0.889697in}{0.417642in}%
\pgfsys@useobject{currentmarker}{}%
\end{pgfscope}%
\end{pgfscope}%
\begin{pgfscope}%
\pgfpathrectangle{\pgfqpoint{0.589510in}{0.417642in}}{\pgfqpoint{1.809765in}{1.371397in}}%
\pgfusepath{clip}%
\pgfsetrectcap%
\pgfsetroundjoin%
\pgfsetlinewidth{0.803000pt}%
\definecolor{currentstroke}{rgb}{0.850000,0.850000,0.850000}%
\pgfsetstrokecolor{currentstroke}%
\pgfsetdash{}{0pt}%
\pgfpathmoveto{\pgfqpoint{0.946763in}{0.417642in}}%
\pgfpathlineto{\pgfqpoint{0.946763in}{1.789039in}}%
\pgfusepath{stroke}%
\end{pgfscope}%
\begin{pgfscope}%
\pgfsetbuttcap%
\pgfsetroundjoin%
\definecolor{currentfill}{rgb}{0.000000,0.000000,0.000000}%
\pgfsetfillcolor{currentfill}%
\pgfsetlinewidth{0.602250pt}%
\definecolor{currentstroke}{rgb}{0.000000,0.000000,0.000000}%
\pgfsetstrokecolor{currentstroke}%
\pgfsetdash{}{0pt}%
\pgfsys@defobject{currentmarker}{\pgfqpoint{0.000000in}{-0.027778in}}{\pgfqpoint{0.000000in}{0.000000in}}{%
\pgfpathmoveto{\pgfqpoint{0.000000in}{0.000000in}}%
\pgfpathlineto{\pgfqpoint{0.000000in}{-0.027778in}}%
\pgfusepath{stroke,fill}%
}%
\begin{pgfscope}%
\pgfsys@transformshift{0.946763in}{0.417642in}%
\pgfsys@useobject{currentmarker}{}%
\end{pgfscope}%
\end{pgfscope}%
\begin{pgfscope}%
\pgfpathrectangle{\pgfqpoint{0.589510in}{0.417642in}}{\pgfqpoint{1.809765in}{1.371397in}}%
\pgfusepath{clip}%
\pgfsetrectcap%
\pgfsetroundjoin%
\pgfsetlinewidth{0.803000pt}%
\definecolor{currentstroke}{rgb}{0.850000,0.850000,0.850000}%
\pgfsetstrokecolor{currentstroke}%
\pgfsetdash{}{0pt}%
\pgfpathmoveto{\pgfqpoint{0.991026in}{0.417642in}}%
\pgfpathlineto{\pgfqpoint{0.991026in}{1.789039in}}%
\pgfusepath{stroke}%
\end{pgfscope}%
\begin{pgfscope}%
\pgfsetbuttcap%
\pgfsetroundjoin%
\definecolor{currentfill}{rgb}{0.000000,0.000000,0.000000}%
\pgfsetfillcolor{currentfill}%
\pgfsetlinewidth{0.602250pt}%
\definecolor{currentstroke}{rgb}{0.000000,0.000000,0.000000}%
\pgfsetstrokecolor{currentstroke}%
\pgfsetdash{}{0pt}%
\pgfsys@defobject{currentmarker}{\pgfqpoint{0.000000in}{-0.027778in}}{\pgfqpoint{0.000000in}{0.000000in}}{%
\pgfpathmoveto{\pgfqpoint{0.000000in}{0.000000in}}%
\pgfpathlineto{\pgfqpoint{0.000000in}{-0.027778in}}%
\pgfusepath{stroke,fill}%
}%
\begin{pgfscope}%
\pgfsys@transformshift{0.991026in}{0.417642in}%
\pgfsys@useobject{currentmarker}{}%
\end{pgfscope}%
\end{pgfscope}%
\begin{pgfscope}%
\pgfpathrectangle{\pgfqpoint{0.589510in}{0.417642in}}{\pgfqpoint{1.809765in}{1.371397in}}%
\pgfusepath{clip}%
\pgfsetrectcap%
\pgfsetroundjoin%
\pgfsetlinewidth{0.803000pt}%
\definecolor{currentstroke}{rgb}{0.850000,0.850000,0.850000}%
\pgfsetstrokecolor{currentstroke}%
\pgfsetdash{}{0pt}%
\pgfpathmoveto{\pgfqpoint{1.027192in}{0.417642in}}%
\pgfpathlineto{\pgfqpoint{1.027192in}{1.789039in}}%
\pgfusepath{stroke}%
\end{pgfscope}%
\begin{pgfscope}%
\pgfsetbuttcap%
\pgfsetroundjoin%
\definecolor{currentfill}{rgb}{0.000000,0.000000,0.000000}%
\pgfsetfillcolor{currentfill}%
\pgfsetlinewidth{0.602250pt}%
\definecolor{currentstroke}{rgb}{0.000000,0.000000,0.000000}%
\pgfsetstrokecolor{currentstroke}%
\pgfsetdash{}{0pt}%
\pgfsys@defobject{currentmarker}{\pgfqpoint{0.000000in}{-0.027778in}}{\pgfqpoint{0.000000in}{0.000000in}}{%
\pgfpathmoveto{\pgfqpoint{0.000000in}{0.000000in}}%
\pgfpathlineto{\pgfqpoint{0.000000in}{-0.027778in}}%
\pgfusepath{stroke,fill}%
}%
\begin{pgfscope}%
\pgfsys@transformshift{1.027192in}{0.417642in}%
\pgfsys@useobject{currentmarker}{}%
\end{pgfscope}%
\end{pgfscope}%
\begin{pgfscope}%
\pgfpathrectangle{\pgfqpoint{0.589510in}{0.417642in}}{\pgfqpoint{1.809765in}{1.371397in}}%
\pgfusepath{clip}%
\pgfsetrectcap%
\pgfsetroundjoin%
\pgfsetlinewidth{0.803000pt}%
\definecolor{currentstroke}{rgb}{0.850000,0.850000,0.850000}%
\pgfsetstrokecolor{currentstroke}%
\pgfsetdash{}{0pt}%
\pgfpathmoveto{\pgfqpoint{1.057770in}{0.417642in}}%
\pgfpathlineto{\pgfqpoint{1.057770in}{1.789039in}}%
\pgfusepath{stroke}%
\end{pgfscope}%
\begin{pgfscope}%
\pgfsetbuttcap%
\pgfsetroundjoin%
\definecolor{currentfill}{rgb}{0.000000,0.000000,0.000000}%
\pgfsetfillcolor{currentfill}%
\pgfsetlinewidth{0.602250pt}%
\definecolor{currentstroke}{rgb}{0.000000,0.000000,0.000000}%
\pgfsetstrokecolor{currentstroke}%
\pgfsetdash{}{0pt}%
\pgfsys@defobject{currentmarker}{\pgfqpoint{0.000000in}{-0.027778in}}{\pgfqpoint{0.000000in}{0.000000in}}{%
\pgfpathmoveto{\pgfqpoint{0.000000in}{0.000000in}}%
\pgfpathlineto{\pgfqpoint{0.000000in}{-0.027778in}}%
\pgfusepath{stroke,fill}%
}%
\begin{pgfscope}%
\pgfsys@transformshift{1.057770in}{0.417642in}%
\pgfsys@useobject{currentmarker}{}%
\end{pgfscope}%
\end{pgfscope}%
\begin{pgfscope}%
\pgfpathrectangle{\pgfqpoint{0.589510in}{0.417642in}}{\pgfqpoint{1.809765in}{1.371397in}}%
\pgfusepath{clip}%
\pgfsetrectcap%
\pgfsetroundjoin%
\pgfsetlinewidth{0.803000pt}%
\definecolor{currentstroke}{rgb}{0.850000,0.850000,0.850000}%
\pgfsetstrokecolor{currentstroke}%
\pgfsetdash{}{0pt}%
\pgfpathmoveto{\pgfqpoint{1.084258in}{0.417642in}}%
\pgfpathlineto{\pgfqpoint{1.084258in}{1.789039in}}%
\pgfusepath{stroke}%
\end{pgfscope}%
\begin{pgfscope}%
\pgfsetbuttcap%
\pgfsetroundjoin%
\definecolor{currentfill}{rgb}{0.000000,0.000000,0.000000}%
\pgfsetfillcolor{currentfill}%
\pgfsetlinewidth{0.602250pt}%
\definecolor{currentstroke}{rgb}{0.000000,0.000000,0.000000}%
\pgfsetstrokecolor{currentstroke}%
\pgfsetdash{}{0pt}%
\pgfsys@defobject{currentmarker}{\pgfqpoint{0.000000in}{-0.027778in}}{\pgfqpoint{0.000000in}{0.000000in}}{%
\pgfpathmoveto{\pgfqpoint{0.000000in}{0.000000in}}%
\pgfpathlineto{\pgfqpoint{0.000000in}{-0.027778in}}%
\pgfusepath{stroke,fill}%
}%
\begin{pgfscope}%
\pgfsys@transformshift{1.084258in}{0.417642in}%
\pgfsys@useobject{currentmarker}{}%
\end{pgfscope}%
\end{pgfscope}%
\begin{pgfscope}%
\pgfpathrectangle{\pgfqpoint{0.589510in}{0.417642in}}{\pgfqpoint{1.809765in}{1.371397in}}%
\pgfusepath{clip}%
\pgfsetrectcap%
\pgfsetroundjoin%
\pgfsetlinewidth{0.803000pt}%
\definecolor{currentstroke}{rgb}{0.850000,0.850000,0.850000}%
\pgfsetstrokecolor{currentstroke}%
\pgfsetdash{}{0pt}%
\pgfpathmoveto{\pgfqpoint{1.107622in}{0.417642in}}%
\pgfpathlineto{\pgfqpoint{1.107622in}{1.789039in}}%
\pgfusepath{stroke}%
\end{pgfscope}%
\begin{pgfscope}%
\pgfsetbuttcap%
\pgfsetroundjoin%
\definecolor{currentfill}{rgb}{0.000000,0.000000,0.000000}%
\pgfsetfillcolor{currentfill}%
\pgfsetlinewidth{0.602250pt}%
\definecolor{currentstroke}{rgb}{0.000000,0.000000,0.000000}%
\pgfsetstrokecolor{currentstroke}%
\pgfsetdash{}{0pt}%
\pgfsys@defobject{currentmarker}{\pgfqpoint{0.000000in}{-0.027778in}}{\pgfqpoint{0.000000in}{0.000000in}}{%
\pgfpathmoveto{\pgfqpoint{0.000000in}{0.000000in}}%
\pgfpathlineto{\pgfqpoint{0.000000in}{-0.027778in}}%
\pgfusepath{stroke,fill}%
}%
\begin{pgfscope}%
\pgfsys@transformshift{1.107622in}{0.417642in}%
\pgfsys@useobject{currentmarker}{}%
\end{pgfscope}%
\end{pgfscope}%
\begin{pgfscope}%
\pgfpathrectangle{\pgfqpoint{0.589510in}{0.417642in}}{\pgfqpoint{1.809765in}{1.371397in}}%
\pgfusepath{clip}%
\pgfsetrectcap%
\pgfsetroundjoin%
\pgfsetlinewidth{0.803000pt}%
\definecolor{currentstroke}{rgb}{0.850000,0.850000,0.850000}%
\pgfsetstrokecolor{currentstroke}%
\pgfsetdash{}{0pt}%
\pgfpathmoveto{\pgfqpoint{1.266017in}{0.417642in}}%
\pgfpathlineto{\pgfqpoint{1.266017in}{1.789039in}}%
\pgfusepath{stroke}%
\end{pgfscope}%
\begin{pgfscope}%
\pgfsetbuttcap%
\pgfsetroundjoin%
\definecolor{currentfill}{rgb}{0.000000,0.000000,0.000000}%
\pgfsetfillcolor{currentfill}%
\pgfsetlinewidth{0.602250pt}%
\definecolor{currentstroke}{rgb}{0.000000,0.000000,0.000000}%
\pgfsetstrokecolor{currentstroke}%
\pgfsetdash{}{0pt}%
\pgfsys@defobject{currentmarker}{\pgfqpoint{0.000000in}{-0.027778in}}{\pgfqpoint{0.000000in}{0.000000in}}{%
\pgfpathmoveto{\pgfqpoint{0.000000in}{0.000000in}}%
\pgfpathlineto{\pgfqpoint{0.000000in}{-0.027778in}}%
\pgfusepath{stroke,fill}%
}%
\begin{pgfscope}%
\pgfsys@transformshift{1.266017in}{0.417642in}%
\pgfsys@useobject{currentmarker}{}%
\end{pgfscope}%
\end{pgfscope}%
\begin{pgfscope}%
\pgfpathrectangle{\pgfqpoint{0.589510in}{0.417642in}}{\pgfqpoint{1.809765in}{1.371397in}}%
\pgfusepath{clip}%
\pgfsetrectcap%
\pgfsetroundjoin%
\pgfsetlinewidth{0.803000pt}%
\definecolor{currentstroke}{rgb}{0.850000,0.850000,0.850000}%
\pgfsetstrokecolor{currentstroke}%
\pgfsetdash{}{0pt}%
\pgfpathmoveto{\pgfqpoint{1.346447in}{0.417642in}}%
\pgfpathlineto{\pgfqpoint{1.346447in}{1.789039in}}%
\pgfusepath{stroke}%
\end{pgfscope}%
\begin{pgfscope}%
\pgfsetbuttcap%
\pgfsetroundjoin%
\definecolor{currentfill}{rgb}{0.000000,0.000000,0.000000}%
\pgfsetfillcolor{currentfill}%
\pgfsetlinewidth{0.602250pt}%
\definecolor{currentstroke}{rgb}{0.000000,0.000000,0.000000}%
\pgfsetstrokecolor{currentstroke}%
\pgfsetdash{}{0pt}%
\pgfsys@defobject{currentmarker}{\pgfqpoint{0.000000in}{-0.027778in}}{\pgfqpoint{0.000000in}{0.000000in}}{%
\pgfpathmoveto{\pgfqpoint{0.000000in}{0.000000in}}%
\pgfpathlineto{\pgfqpoint{0.000000in}{-0.027778in}}%
\pgfusepath{stroke,fill}%
}%
\begin{pgfscope}%
\pgfsys@transformshift{1.346447in}{0.417642in}%
\pgfsys@useobject{currentmarker}{}%
\end{pgfscope}%
\end{pgfscope}%
\begin{pgfscope}%
\pgfpathrectangle{\pgfqpoint{0.589510in}{0.417642in}}{\pgfqpoint{1.809765in}{1.371397in}}%
\pgfusepath{clip}%
\pgfsetrectcap%
\pgfsetroundjoin%
\pgfsetlinewidth{0.803000pt}%
\definecolor{currentstroke}{rgb}{0.850000,0.850000,0.850000}%
\pgfsetstrokecolor{currentstroke}%
\pgfsetdash{}{0pt}%
\pgfpathmoveto{\pgfqpoint{1.403513in}{0.417642in}}%
\pgfpathlineto{\pgfqpoint{1.403513in}{1.789039in}}%
\pgfusepath{stroke}%
\end{pgfscope}%
\begin{pgfscope}%
\pgfsetbuttcap%
\pgfsetroundjoin%
\definecolor{currentfill}{rgb}{0.000000,0.000000,0.000000}%
\pgfsetfillcolor{currentfill}%
\pgfsetlinewidth{0.602250pt}%
\definecolor{currentstroke}{rgb}{0.000000,0.000000,0.000000}%
\pgfsetstrokecolor{currentstroke}%
\pgfsetdash{}{0pt}%
\pgfsys@defobject{currentmarker}{\pgfqpoint{0.000000in}{-0.027778in}}{\pgfqpoint{0.000000in}{0.000000in}}{%
\pgfpathmoveto{\pgfqpoint{0.000000in}{0.000000in}}%
\pgfpathlineto{\pgfqpoint{0.000000in}{-0.027778in}}%
\pgfusepath{stroke,fill}%
}%
\begin{pgfscope}%
\pgfsys@transformshift{1.403513in}{0.417642in}%
\pgfsys@useobject{currentmarker}{}%
\end{pgfscope}%
\end{pgfscope}%
\begin{pgfscope}%
\pgfpathrectangle{\pgfqpoint{0.589510in}{0.417642in}}{\pgfqpoint{1.809765in}{1.371397in}}%
\pgfusepath{clip}%
\pgfsetrectcap%
\pgfsetroundjoin%
\pgfsetlinewidth{0.803000pt}%
\definecolor{currentstroke}{rgb}{0.850000,0.850000,0.850000}%
\pgfsetstrokecolor{currentstroke}%
\pgfsetdash{}{0pt}%
\pgfpathmoveto{\pgfqpoint{1.447776in}{0.417642in}}%
\pgfpathlineto{\pgfqpoint{1.447776in}{1.789039in}}%
\pgfusepath{stroke}%
\end{pgfscope}%
\begin{pgfscope}%
\pgfsetbuttcap%
\pgfsetroundjoin%
\definecolor{currentfill}{rgb}{0.000000,0.000000,0.000000}%
\pgfsetfillcolor{currentfill}%
\pgfsetlinewidth{0.602250pt}%
\definecolor{currentstroke}{rgb}{0.000000,0.000000,0.000000}%
\pgfsetstrokecolor{currentstroke}%
\pgfsetdash{}{0pt}%
\pgfsys@defobject{currentmarker}{\pgfqpoint{0.000000in}{-0.027778in}}{\pgfqpoint{0.000000in}{0.000000in}}{%
\pgfpathmoveto{\pgfqpoint{0.000000in}{0.000000in}}%
\pgfpathlineto{\pgfqpoint{0.000000in}{-0.027778in}}%
\pgfusepath{stroke,fill}%
}%
\begin{pgfscope}%
\pgfsys@transformshift{1.447776in}{0.417642in}%
\pgfsys@useobject{currentmarker}{}%
\end{pgfscope}%
\end{pgfscope}%
\begin{pgfscope}%
\pgfpathrectangle{\pgfqpoint{0.589510in}{0.417642in}}{\pgfqpoint{1.809765in}{1.371397in}}%
\pgfusepath{clip}%
\pgfsetrectcap%
\pgfsetroundjoin%
\pgfsetlinewidth{0.803000pt}%
\definecolor{currentstroke}{rgb}{0.850000,0.850000,0.850000}%
\pgfsetstrokecolor{currentstroke}%
\pgfsetdash{}{0pt}%
\pgfpathmoveto{\pgfqpoint{1.483942in}{0.417642in}}%
\pgfpathlineto{\pgfqpoint{1.483942in}{1.789039in}}%
\pgfusepath{stroke}%
\end{pgfscope}%
\begin{pgfscope}%
\pgfsetbuttcap%
\pgfsetroundjoin%
\definecolor{currentfill}{rgb}{0.000000,0.000000,0.000000}%
\pgfsetfillcolor{currentfill}%
\pgfsetlinewidth{0.602250pt}%
\definecolor{currentstroke}{rgb}{0.000000,0.000000,0.000000}%
\pgfsetstrokecolor{currentstroke}%
\pgfsetdash{}{0pt}%
\pgfsys@defobject{currentmarker}{\pgfqpoint{0.000000in}{-0.027778in}}{\pgfqpoint{0.000000in}{0.000000in}}{%
\pgfpathmoveto{\pgfqpoint{0.000000in}{0.000000in}}%
\pgfpathlineto{\pgfqpoint{0.000000in}{-0.027778in}}%
\pgfusepath{stroke,fill}%
}%
\begin{pgfscope}%
\pgfsys@transformshift{1.483942in}{0.417642in}%
\pgfsys@useobject{currentmarker}{}%
\end{pgfscope}%
\end{pgfscope}%
\begin{pgfscope}%
\pgfpathrectangle{\pgfqpoint{0.589510in}{0.417642in}}{\pgfqpoint{1.809765in}{1.371397in}}%
\pgfusepath{clip}%
\pgfsetrectcap%
\pgfsetroundjoin%
\pgfsetlinewidth{0.803000pt}%
\definecolor{currentstroke}{rgb}{0.850000,0.850000,0.850000}%
\pgfsetstrokecolor{currentstroke}%
\pgfsetdash{}{0pt}%
\pgfpathmoveto{\pgfqpoint{1.514520in}{0.417642in}}%
\pgfpathlineto{\pgfqpoint{1.514520in}{1.789039in}}%
\pgfusepath{stroke}%
\end{pgfscope}%
\begin{pgfscope}%
\pgfsetbuttcap%
\pgfsetroundjoin%
\definecolor{currentfill}{rgb}{0.000000,0.000000,0.000000}%
\pgfsetfillcolor{currentfill}%
\pgfsetlinewidth{0.602250pt}%
\definecolor{currentstroke}{rgb}{0.000000,0.000000,0.000000}%
\pgfsetstrokecolor{currentstroke}%
\pgfsetdash{}{0pt}%
\pgfsys@defobject{currentmarker}{\pgfqpoint{0.000000in}{-0.027778in}}{\pgfqpoint{0.000000in}{0.000000in}}{%
\pgfpathmoveto{\pgfqpoint{0.000000in}{0.000000in}}%
\pgfpathlineto{\pgfqpoint{0.000000in}{-0.027778in}}%
\pgfusepath{stroke,fill}%
}%
\begin{pgfscope}%
\pgfsys@transformshift{1.514520in}{0.417642in}%
\pgfsys@useobject{currentmarker}{}%
\end{pgfscope}%
\end{pgfscope}%
\begin{pgfscope}%
\pgfpathrectangle{\pgfqpoint{0.589510in}{0.417642in}}{\pgfqpoint{1.809765in}{1.371397in}}%
\pgfusepath{clip}%
\pgfsetrectcap%
\pgfsetroundjoin%
\pgfsetlinewidth{0.803000pt}%
\definecolor{currentstroke}{rgb}{0.850000,0.850000,0.850000}%
\pgfsetstrokecolor{currentstroke}%
\pgfsetdash{}{0pt}%
\pgfpathmoveto{\pgfqpoint{1.541008in}{0.417642in}}%
\pgfpathlineto{\pgfqpoint{1.541008in}{1.789039in}}%
\pgfusepath{stroke}%
\end{pgfscope}%
\begin{pgfscope}%
\pgfsetbuttcap%
\pgfsetroundjoin%
\definecolor{currentfill}{rgb}{0.000000,0.000000,0.000000}%
\pgfsetfillcolor{currentfill}%
\pgfsetlinewidth{0.602250pt}%
\definecolor{currentstroke}{rgb}{0.000000,0.000000,0.000000}%
\pgfsetstrokecolor{currentstroke}%
\pgfsetdash{}{0pt}%
\pgfsys@defobject{currentmarker}{\pgfqpoint{0.000000in}{-0.027778in}}{\pgfqpoint{0.000000in}{0.000000in}}{%
\pgfpathmoveto{\pgfqpoint{0.000000in}{0.000000in}}%
\pgfpathlineto{\pgfqpoint{0.000000in}{-0.027778in}}%
\pgfusepath{stroke,fill}%
}%
\begin{pgfscope}%
\pgfsys@transformshift{1.541008in}{0.417642in}%
\pgfsys@useobject{currentmarker}{}%
\end{pgfscope}%
\end{pgfscope}%
\begin{pgfscope}%
\pgfpathrectangle{\pgfqpoint{0.589510in}{0.417642in}}{\pgfqpoint{1.809765in}{1.371397in}}%
\pgfusepath{clip}%
\pgfsetrectcap%
\pgfsetroundjoin%
\pgfsetlinewidth{0.803000pt}%
\definecolor{currentstroke}{rgb}{0.850000,0.850000,0.850000}%
\pgfsetstrokecolor{currentstroke}%
\pgfsetdash{}{0pt}%
\pgfpathmoveto{\pgfqpoint{1.564372in}{0.417642in}}%
\pgfpathlineto{\pgfqpoint{1.564372in}{1.789039in}}%
\pgfusepath{stroke}%
\end{pgfscope}%
\begin{pgfscope}%
\pgfsetbuttcap%
\pgfsetroundjoin%
\definecolor{currentfill}{rgb}{0.000000,0.000000,0.000000}%
\pgfsetfillcolor{currentfill}%
\pgfsetlinewidth{0.602250pt}%
\definecolor{currentstroke}{rgb}{0.000000,0.000000,0.000000}%
\pgfsetstrokecolor{currentstroke}%
\pgfsetdash{}{0pt}%
\pgfsys@defobject{currentmarker}{\pgfqpoint{0.000000in}{-0.027778in}}{\pgfqpoint{0.000000in}{0.000000in}}{%
\pgfpathmoveto{\pgfqpoint{0.000000in}{0.000000in}}%
\pgfpathlineto{\pgfqpoint{0.000000in}{-0.027778in}}%
\pgfusepath{stroke,fill}%
}%
\begin{pgfscope}%
\pgfsys@transformshift{1.564372in}{0.417642in}%
\pgfsys@useobject{currentmarker}{}%
\end{pgfscope}%
\end{pgfscope}%
\begin{pgfscope}%
\pgfpathrectangle{\pgfqpoint{0.589510in}{0.417642in}}{\pgfqpoint{1.809765in}{1.371397in}}%
\pgfusepath{clip}%
\pgfsetrectcap%
\pgfsetroundjoin%
\pgfsetlinewidth{0.803000pt}%
\definecolor{currentstroke}{rgb}{0.850000,0.850000,0.850000}%
\pgfsetstrokecolor{currentstroke}%
\pgfsetdash{}{0pt}%
\pgfpathmoveto{\pgfqpoint{1.722767in}{0.417642in}}%
\pgfpathlineto{\pgfqpoint{1.722767in}{1.789039in}}%
\pgfusepath{stroke}%
\end{pgfscope}%
\begin{pgfscope}%
\pgfsetbuttcap%
\pgfsetroundjoin%
\definecolor{currentfill}{rgb}{0.000000,0.000000,0.000000}%
\pgfsetfillcolor{currentfill}%
\pgfsetlinewidth{0.602250pt}%
\definecolor{currentstroke}{rgb}{0.000000,0.000000,0.000000}%
\pgfsetstrokecolor{currentstroke}%
\pgfsetdash{}{0pt}%
\pgfsys@defobject{currentmarker}{\pgfqpoint{0.000000in}{-0.027778in}}{\pgfqpoint{0.000000in}{0.000000in}}{%
\pgfpathmoveto{\pgfqpoint{0.000000in}{0.000000in}}%
\pgfpathlineto{\pgfqpoint{0.000000in}{-0.027778in}}%
\pgfusepath{stroke,fill}%
}%
\begin{pgfscope}%
\pgfsys@transformshift{1.722767in}{0.417642in}%
\pgfsys@useobject{currentmarker}{}%
\end{pgfscope}%
\end{pgfscope}%
\begin{pgfscope}%
\pgfpathrectangle{\pgfqpoint{0.589510in}{0.417642in}}{\pgfqpoint{1.809765in}{1.371397in}}%
\pgfusepath{clip}%
\pgfsetrectcap%
\pgfsetroundjoin%
\pgfsetlinewidth{0.803000pt}%
\definecolor{currentstroke}{rgb}{0.850000,0.850000,0.850000}%
\pgfsetstrokecolor{currentstroke}%
\pgfsetdash{}{0pt}%
\pgfpathmoveto{\pgfqpoint{1.803197in}{0.417642in}}%
\pgfpathlineto{\pgfqpoint{1.803197in}{1.789039in}}%
\pgfusepath{stroke}%
\end{pgfscope}%
\begin{pgfscope}%
\pgfsetbuttcap%
\pgfsetroundjoin%
\definecolor{currentfill}{rgb}{0.000000,0.000000,0.000000}%
\pgfsetfillcolor{currentfill}%
\pgfsetlinewidth{0.602250pt}%
\definecolor{currentstroke}{rgb}{0.000000,0.000000,0.000000}%
\pgfsetstrokecolor{currentstroke}%
\pgfsetdash{}{0pt}%
\pgfsys@defobject{currentmarker}{\pgfqpoint{0.000000in}{-0.027778in}}{\pgfqpoint{0.000000in}{0.000000in}}{%
\pgfpathmoveto{\pgfqpoint{0.000000in}{0.000000in}}%
\pgfpathlineto{\pgfqpoint{0.000000in}{-0.027778in}}%
\pgfusepath{stroke,fill}%
}%
\begin{pgfscope}%
\pgfsys@transformshift{1.803197in}{0.417642in}%
\pgfsys@useobject{currentmarker}{}%
\end{pgfscope}%
\end{pgfscope}%
\begin{pgfscope}%
\pgfpathrectangle{\pgfqpoint{0.589510in}{0.417642in}}{\pgfqpoint{1.809765in}{1.371397in}}%
\pgfusepath{clip}%
\pgfsetrectcap%
\pgfsetroundjoin%
\pgfsetlinewidth{0.803000pt}%
\definecolor{currentstroke}{rgb}{0.850000,0.850000,0.850000}%
\pgfsetstrokecolor{currentstroke}%
\pgfsetdash{}{0pt}%
\pgfpathmoveto{\pgfqpoint{1.860263in}{0.417642in}}%
\pgfpathlineto{\pgfqpoint{1.860263in}{1.789039in}}%
\pgfusepath{stroke}%
\end{pgfscope}%
\begin{pgfscope}%
\pgfsetbuttcap%
\pgfsetroundjoin%
\definecolor{currentfill}{rgb}{0.000000,0.000000,0.000000}%
\pgfsetfillcolor{currentfill}%
\pgfsetlinewidth{0.602250pt}%
\definecolor{currentstroke}{rgb}{0.000000,0.000000,0.000000}%
\pgfsetstrokecolor{currentstroke}%
\pgfsetdash{}{0pt}%
\pgfsys@defobject{currentmarker}{\pgfqpoint{0.000000in}{-0.027778in}}{\pgfqpoint{0.000000in}{0.000000in}}{%
\pgfpathmoveto{\pgfqpoint{0.000000in}{0.000000in}}%
\pgfpathlineto{\pgfqpoint{0.000000in}{-0.027778in}}%
\pgfusepath{stroke,fill}%
}%
\begin{pgfscope}%
\pgfsys@transformshift{1.860263in}{0.417642in}%
\pgfsys@useobject{currentmarker}{}%
\end{pgfscope}%
\end{pgfscope}%
\begin{pgfscope}%
\pgfpathrectangle{\pgfqpoint{0.589510in}{0.417642in}}{\pgfqpoint{1.809765in}{1.371397in}}%
\pgfusepath{clip}%
\pgfsetrectcap%
\pgfsetroundjoin%
\pgfsetlinewidth{0.803000pt}%
\definecolor{currentstroke}{rgb}{0.850000,0.850000,0.850000}%
\pgfsetstrokecolor{currentstroke}%
\pgfsetdash{}{0pt}%
\pgfpathmoveto{\pgfqpoint{1.904526in}{0.417642in}}%
\pgfpathlineto{\pgfqpoint{1.904526in}{1.789039in}}%
\pgfusepath{stroke}%
\end{pgfscope}%
\begin{pgfscope}%
\pgfsetbuttcap%
\pgfsetroundjoin%
\definecolor{currentfill}{rgb}{0.000000,0.000000,0.000000}%
\pgfsetfillcolor{currentfill}%
\pgfsetlinewidth{0.602250pt}%
\definecolor{currentstroke}{rgb}{0.000000,0.000000,0.000000}%
\pgfsetstrokecolor{currentstroke}%
\pgfsetdash{}{0pt}%
\pgfsys@defobject{currentmarker}{\pgfqpoint{0.000000in}{-0.027778in}}{\pgfqpoint{0.000000in}{0.000000in}}{%
\pgfpathmoveto{\pgfqpoint{0.000000in}{0.000000in}}%
\pgfpathlineto{\pgfqpoint{0.000000in}{-0.027778in}}%
\pgfusepath{stroke,fill}%
}%
\begin{pgfscope}%
\pgfsys@transformshift{1.904526in}{0.417642in}%
\pgfsys@useobject{currentmarker}{}%
\end{pgfscope}%
\end{pgfscope}%
\begin{pgfscope}%
\pgfpathrectangle{\pgfqpoint{0.589510in}{0.417642in}}{\pgfqpoint{1.809765in}{1.371397in}}%
\pgfusepath{clip}%
\pgfsetrectcap%
\pgfsetroundjoin%
\pgfsetlinewidth{0.803000pt}%
\definecolor{currentstroke}{rgb}{0.850000,0.850000,0.850000}%
\pgfsetstrokecolor{currentstroke}%
\pgfsetdash{}{0pt}%
\pgfpathmoveto{\pgfqpoint{1.940693in}{0.417642in}}%
\pgfpathlineto{\pgfqpoint{1.940693in}{1.789039in}}%
\pgfusepath{stroke}%
\end{pgfscope}%
\begin{pgfscope}%
\pgfsetbuttcap%
\pgfsetroundjoin%
\definecolor{currentfill}{rgb}{0.000000,0.000000,0.000000}%
\pgfsetfillcolor{currentfill}%
\pgfsetlinewidth{0.602250pt}%
\definecolor{currentstroke}{rgb}{0.000000,0.000000,0.000000}%
\pgfsetstrokecolor{currentstroke}%
\pgfsetdash{}{0pt}%
\pgfsys@defobject{currentmarker}{\pgfqpoint{0.000000in}{-0.027778in}}{\pgfqpoint{0.000000in}{0.000000in}}{%
\pgfpathmoveto{\pgfqpoint{0.000000in}{0.000000in}}%
\pgfpathlineto{\pgfqpoint{0.000000in}{-0.027778in}}%
\pgfusepath{stroke,fill}%
}%
\begin{pgfscope}%
\pgfsys@transformshift{1.940693in}{0.417642in}%
\pgfsys@useobject{currentmarker}{}%
\end{pgfscope}%
\end{pgfscope}%
\begin{pgfscope}%
\pgfpathrectangle{\pgfqpoint{0.589510in}{0.417642in}}{\pgfqpoint{1.809765in}{1.371397in}}%
\pgfusepath{clip}%
\pgfsetrectcap%
\pgfsetroundjoin%
\pgfsetlinewidth{0.803000pt}%
\definecolor{currentstroke}{rgb}{0.850000,0.850000,0.850000}%
\pgfsetstrokecolor{currentstroke}%
\pgfsetdash{}{0pt}%
\pgfpathmoveto{\pgfqpoint{1.971270in}{0.417642in}}%
\pgfpathlineto{\pgfqpoint{1.971270in}{1.789039in}}%
\pgfusepath{stroke}%
\end{pgfscope}%
\begin{pgfscope}%
\pgfsetbuttcap%
\pgfsetroundjoin%
\definecolor{currentfill}{rgb}{0.000000,0.000000,0.000000}%
\pgfsetfillcolor{currentfill}%
\pgfsetlinewidth{0.602250pt}%
\definecolor{currentstroke}{rgb}{0.000000,0.000000,0.000000}%
\pgfsetstrokecolor{currentstroke}%
\pgfsetdash{}{0pt}%
\pgfsys@defobject{currentmarker}{\pgfqpoint{0.000000in}{-0.027778in}}{\pgfqpoint{0.000000in}{0.000000in}}{%
\pgfpathmoveto{\pgfqpoint{0.000000in}{0.000000in}}%
\pgfpathlineto{\pgfqpoint{0.000000in}{-0.027778in}}%
\pgfusepath{stroke,fill}%
}%
\begin{pgfscope}%
\pgfsys@transformshift{1.971270in}{0.417642in}%
\pgfsys@useobject{currentmarker}{}%
\end{pgfscope}%
\end{pgfscope}%
\begin{pgfscope}%
\pgfpathrectangle{\pgfqpoint{0.589510in}{0.417642in}}{\pgfqpoint{1.809765in}{1.371397in}}%
\pgfusepath{clip}%
\pgfsetrectcap%
\pgfsetroundjoin%
\pgfsetlinewidth{0.803000pt}%
\definecolor{currentstroke}{rgb}{0.850000,0.850000,0.850000}%
\pgfsetstrokecolor{currentstroke}%
\pgfsetdash{}{0pt}%
\pgfpathmoveto{\pgfqpoint{1.997758in}{0.417642in}}%
\pgfpathlineto{\pgfqpoint{1.997758in}{1.789039in}}%
\pgfusepath{stroke}%
\end{pgfscope}%
\begin{pgfscope}%
\pgfsetbuttcap%
\pgfsetroundjoin%
\definecolor{currentfill}{rgb}{0.000000,0.000000,0.000000}%
\pgfsetfillcolor{currentfill}%
\pgfsetlinewidth{0.602250pt}%
\definecolor{currentstroke}{rgb}{0.000000,0.000000,0.000000}%
\pgfsetstrokecolor{currentstroke}%
\pgfsetdash{}{0pt}%
\pgfsys@defobject{currentmarker}{\pgfqpoint{0.000000in}{-0.027778in}}{\pgfqpoint{0.000000in}{0.000000in}}{%
\pgfpathmoveto{\pgfqpoint{0.000000in}{0.000000in}}%
\pgfpathlineto{\pgfqpoint{0.000000in}{-0.027778in}}%
\pgfusepath{stroke,fill}%
}%
\begin{pgfscope}%
\pgfsys@transformshift{1.997758in}{0.417642in}%
\pgfsys@useobject{currentmarker}{}%
\end{pgfscope}%
\end{pgfscope}%
\begin{pgfscope}%
\pgfpathrectangle{\pgfqpoint{0.589510in}{0.417642in}}{\pgfqpoint{1.809765in}{1.371397in}}%
\pgfusepath{clip}%
\pgfsetrectcap%
\pgfsetroundjoin%
\pgfsetlinewidth{0.803000pt}%
\definecolor{currentstroke}{rgb}{0.850000,0.850000,0.850000}%
\pgfsetstrokecolor{currentstroke}%
\pgfsetdash{}{0pt}%
\pgfpathmoveto{\pgfqpoint{2.021122in}{0.417642in}}%
\pgfpathlineto{\pgfqpoint{2.021122in}{1.789039in}}%
\pgfusepath{stroke}%
\end{pgfscope}%
\begin{pgfscope}%
\pgfsetbuttcap%
\pgfsetroundjoin%
\definecolor{currentfill}{rgb}{0.000000,0.000000,0.000000}%
\pgfsetfillcolor{currentfill}%
\pgfsetlinewidth{0.602250pt}%
\definecolor{currentstroke}{rgb}{0.000000,0.000000,0.000000}%
\pgfsetstrokecolor{currentstroke}%
\pgfsetdash{}{0pt}%
\pgfsys@defobject{currentmarker}{\pgfqpoint{0.000000in}{-0.027778in}}{\pgfqpoint{0.000000in}{0.000000in}}{%
\pgfpathmoveto{\pgfqpoint{0.000000in}{0.000000in}}%
\pgfpathlineto{\pgfqpoint{0.000000in}{-0.027778in}}%
\pgfusepath{stroke,fill}%
}%
\begin{pgfscope}%
\pgfsys@transformshift{2.021122in}{0.417642in}%
\pgfsys@useobject{currentmarker}{}%
\end{pgfscope}%
\end{pgfscope}%
\begin{pgfscope}%
\pgfpathrectangle{\pgfqpoint{0.589510in}{0.417642in}}{\pgfqpoint{1.809765in}{1.371397in}}%
\pgfusepath{clip}%
\pgfsetrectcap%
\pgfsetroundjoin%
\pgfsetlinewidth{0.803000pt}%
\definecolor{currentstroke}{rgb}{0.850000,0.850000,0.850000}%
\pgfsetstrokecolor{currentstroke}%
\pgfsetdash{}{0pt}%
\pgfpathmoveto{\pgfqpoint{2.179517in}{0.417642in}}%
\pgfpathlineto{\pgfqpoint{2.179517in}{1.789039in}}%
\pgfusepath{stroke}%
\end{pgfscope}%
\begin{pgfscope}%
\pgfsetbuttcap%
\pgfsetroundjoin%
\definecolor{currentfill}{rgb}{0.000000,0.000000,0.000000}%
\pgfsetfillcolor{currentfill}%
\pgfsetlinewidth{0.602250pt}%
\definecolor{currentstroke}{rgb}{0.000000,0.000000,0.000000}%
\pgfsetstrokecolor{currentstroke}%
\pgfsetdash{}{0pt}%
\pgfsys@defobject{currentmarker}{\pgfqpoint{0.000000in}{-0.027778in}}{\pgfqpoint{0.000000in}{0.000000in}}{%
\pgfpathmoveto{\pgfqpoint{0.000000in}{0.000000in}}%
\pgfpathlineto{\pgfqpoint{0.000000in}{-0.027778in}}%
\pgfusepath{stroke,fill}%
}%
\begin{pgfscope}%
\pgfsys@transformshift{2.179517in}{0.417642in}%
\pgfsys@useobject{currentmarker}{}%
\end{pgfscope}%
\end{pgfscope}%
\begin{pgfscope}%
\pgfpathrectangle{\pgfqpoint{0.589510in}{0.417642in}}{\pgfqpoint{1.809765in}{1.371397in}}%
\pgfusepath{clip}%
\pgfsetrectcap%
\pgfsetroundjoin%
\pgfsetlinewidth{0.803000pt}%
\definecolor{currentstroke}{rgb}{0.850000,0.850000,0.850000}%
\pgfsetstrokecolor{currentstroke}%
\pgfsetdash{}{0pt}%
\pgfpathmoveto{\pgfqpoint{2.259947in}{0.417642in}}%
\pgfpathlineto{\pgfqpoint{2.259947in}{1.789039in}}%
\pgfusepath{stroke}%
\end{pgfscope}%
\begin{pgfscope}%
\pgfsetbuttcap%
\pgfsetroundjoin%
\definecolor{currentfill}{rgb}{0.000000,0.000000,0.000000}%
\pgfsetfillcolor{currentfill}%
\pgfsetlinewidth{0.602250pt}%
\definecolor{currentstroke}{rgb}{0.000000,0.000000,0.000000}%
\pgfsetstrokecolor{currentstroke}%
\pgfsetdash{}{0pt}%
\pgfsys@defobject{currentmarker}{\pgfqpoint{0.000000in}{-0.027778in}}{\pgfqpoint{0.000000in}{0.000000in}}{%
\pgfpathmoveto{\pgfqpoint{0.000000in}{0.000000in}}%
\pgfpathlineto{\pgfqpoint{0.000000in}{-0.027778in}}%
\pgfusepath{stroke,fill}%
}%
\begin{pgfscope}%
\pgfsys@transformshift{2.259947in}{0.417642in}%
\pgfsys@useobject{currentmarker}{}%
\end{pgfscope}%
\end{pgfscope}%
\begin{pgfscope}%
\pgfpathrectangle{\pgfqpoint{0.589510in}{0.417642in}}{\pgfqpoint{1.809765in}{1.371397in}}%
\pgfusepath{clip}%
\pgfsetrectcap%
\pgfsetroundjoin%
\pgfsetlinewidth{0.803000pt}%
\definecolor{currentstroke}{rgb}{0.850000,0.850000,0.850000}%
\pgfsetstrokecolor{currentstroke}%
\pgfsetdash{}{0pt}%
\pgfpathmoveto{\pgfqpoint{2.317013in}{0.417642in}}%
\pgfpathlineto{\pgfqpoint{2.317013in}{1.789039in}}%
\pgfusepath{stroke}%
\end{pgfscope}%
\begin{pgfscope}%
\pgfsetbuttcap%
\pgfsetroundjoin%
\definecolor{currentfill}{rgb}{0.000000,0.000000,0.000000}%
\pgfsetfillcolor{currentfill}%
\pgfsetlinewidth{0.602250pt}%
\definecolor{currentstroke}{rgb}{0.000000,0.000000,0.000000}%
\pgfsetstrokecolor{currentstroke}%
\pgfsetdash{}{0pt}%
\pgfsys@defobject{currentmarker}{\pgfqpoint{0.000000in}{-0.027778in}}{\pgfqpoint{0.000000in}{0.000000in}}{%
\pgfpathmoveto{\pgfqpoint{0.000000in}{0.000000in}}%
\pgfpathlineto{\pgfqpoint{0.000000in}{-0.027778in}}%
\pgfusepath{stroke,fill}%
}%
\begin{pgfscope}%
\pgfsys@transformshift{2.317013in}{0.417642in}%
\pgfsys@useobject{currentmarker}{}%
\end{pgfscope}%
\end{pgfscope}%
\begin{pgfscope}%
\pgfpathrectangle{\pgfqpoint{0.589510in}{0.417642in}}{\pgfqpoint{1.809765in}{1.371397in}}%
\pgfusepath{clip}%
\pgfsetrectcap%
\pgfsetroundjoin%
\pgfsetlinewidth{0.803000pt}%
\definecolor{currentstroke}{rgb}{0.850000,0.850000,0.850000}%
\pgfsetstrokecolor{currentstroke}%
\pgfsetdash{}{0pt}%
\pgfpathmoveto{\pgfqpoint{2.361277in}{0.417642in}}%
\pgfpathlineto{\pgfqpoint{2.361277in}{1.789039in}}%
\pgfusepath{stroke}%
\end{pgfscope}%
\begin{pgfscope}%
\pgfsetbuttcap%
\pgfsetroundjoin%
\definecolor{currentfill}{rgb}{0.000000,0.000000,0.000000}%
\pgfsetfillcolor{currentfill}%
\pgfsetlinewidth{0.602250pt}%
\definecolor{currentstroke}{rgb}{0.000000,0.000000,0.000000}%
\pgfsetstrokecolor{currentstroke}%
\pgfsetdash{}{0pt}%
\pgfsys@defobject{currentmarker}{\pgfqpoint{0.000000in}{-0.027778in}}{\pgfqpoint{0.000000in}{0.000000in}}{%
\pgfpathmoveto{\pgfqpoint{0.000000in}{0.000000in}}%
\pgfpathlineto{\pgfqpoint{0.000000in}{-0.027778in}}%
\pgfusepath{stroke,fill}%
}%
\begin{pgfscope}%
\pgfsys@transformshift{2.361277in}{0.417642in}%
\pgfsys@useobject{currentmarker}{}%
\end{pgfscope}%
\end{pgfscope}%
\begin{pgfscope}%
\pgfpathrectangle{\pgfqpoint{0.589510in}{0.417642in}}{\pgfqpoint{1.809765in}{1.371397in}}%
\pgfusepath{clip}%
\pgfsetrectcap%
\pgfsetroundjoin%
\pgfsetlinewidth{0.803000pt}%
\definecolor{currentstroke}{rgb}{0.850000,0.850000,0.850000}%
\pgfsetstrokecolor{currentstroke}%
\pgfsetdash{}{0pt}%
\pgfpathmoveto{\pgfqpoint{2.397443in}{0.417642in}}%
\pgfpathlineto{\pgfqpoint{2.397443in}{1.789039in}}%
\pgfusepath{stroke}%
\end{pgfscope}%
\begin{pgfscope}%
\pgfsetbuttcap%
\pgfsetroundjoin%
\definecolor{currentfill}{rgb}{0.000000,0.000000,0.000000}%
\pgfsetfillcolor{currentfill}%
\pgfsetlinewidth{0.602250pt}%
\definecolor{currentstroke}{rgb}{0.000000,0.000000,0.000000}%
\pgfsetstrokecolor{currentstroke}%
\pgfsetdash{}{0pt}%
\pgfsys@defobject{currentmarker}{\pgfqpoint{0.000000in}{-0.027778in}}{\pgfqpoint{0.000000in}{0.000000in}}{%
\pgfpathmoveto{\pgfqpoint{0.000000in}{0.000000in}}%
\pgfpathlineto{\pgfqpoint{0.000000in}{-0.027778in}}%
\pgfusepath{stroke,fill}%
}%
\begin{pgfscope}%
\pgfsys@transformshift{2.397443in}{0.417642in}%
\pgfsys@useobject{currentmarker}{}%
\end{pgfscope}%
\end{pgfscope}%
\begin{pgfscope}%
\definecolor{textcolor}{rgb}{0.000000,0.000000,0.000000}%
\pgfsetstrokecolor{textcolor}%
\pgfsetfillcolor{textcolor}%
\pgftext[x=1.494392in,y=0.165003in,,top]{\color{textcolor}{\rmfamily\fontsize{10.000000}{12.000000}\selectfont\catcode`\^=\active\def^{\ifmmode\sp\else\^{}\fi}\catcode`\%=\active\def%{\%}$\tau$ in \unit{\second}}}%
\end{pgfscope}%
\begin{pgfscope}%
\pgfpathrectangle{\pgfqpoint{0.589510in}{0.417642in}}{\pgfqpoint{1.809765in}{1.371397in}}%
\pgfusepath{clip}%
\pgfsetrectcap%
\pgfsetroundjoin%
\pgfsetlinewidth{0.803000pt}%
\definecolor{currentstroke}{rgb}{0.450000,0.450000,0.450000}%
\pgfsetstrokecolor{currentstroke}%
\pgfsetdash{}{0pt}%
\pgfpathmoveto{\pgfqpoint{0.589510in}{0.417642in}}%
\pgfpathlineto{\pgfqpoint{2.399275in}{0.417642in}}%
\pgfusepath{stroke}%
\end{pgfscope}%
\begin{pgfscope}%
\pgfsetbuttcap%
\pgfsetroundjoin%
\definecolor{currentfill}{rgb}{0.000000,0.000000,0.000000}%
\pgfsetfillcolor{currentfill}%
\pgfsetlinewidth{0.803000pt}%
\definecolor{currentstroke}{rgb}{0.000000,0.000000,0.000000}%
\pgfsetstrokecolor{currentstroke}%
\pgfsetdash{}{0pt}%
\pgfsys@defobject{currentmarker}{\pgfqpoint{-0.048611in}{0.000000in}}{\pgfqpoint{-0.000000in}{0.000000in}}{%
\pgfpathmoveto{\pgfqpoint{-0.000000in}{0.000000in}}%
\pgfpathlineto{\pgfqpoint{-0.048611in}{0.000000in}}%
\pgfusepath{stroke,fill}%
}%
\begin{pgfscope}%
\pgfsys@transformshift{0.589510in}{0.417642in}%
\pgfsys@useobject{currentmarker}{}%
\end{pgfscope}%
\end{pgfscope}%
\begin{pgfscope}%
\definecolor{textcolor}{rgb}{0.000000,0.000000,0.000000}%
\pgfsetstrokecolor{textcolor}%
\pgfsetfillcolor{textcolor}%
\pgftext[x=0.236114in, y=0.378489in, left, base]{\color{textcolor}{\rmfamily\fontsize{8.000000}{9.600000}\selectfont\catcode`\^=\active\def^{\ifmmode\sp\else\^{}\fi}\catcode`\%=\active\def%{\%}$\mathdefault{10^{-2}}$}}%
\end{pgfscope}%
\begin{pgfscope}%
\pgfpathrectangle{\pgfqpoint{0.589510in}{0.417642in}}{\pgfqpoint{1.809765in}{1.371397in}}%
\pgfusepath{clip}%
\pgfsetrectcap%
\pgfsetroundjoin%
\pgfsetlinewidth{0.803000pt}%
\definecolor{currentstroke}{rgb}{0.450000,0.450000,0.450000}%
\pgfsetstrokecolor{currentstroke}%
\pgfsetdash{}{0pt}%
\pgfpathmoveto{\pgfqpoint{0.589510in}{0.622360in}}%
\pgfpathlineto{\pgfqpoint{2.399275in}{0.622360in}}%
\pgfusepath{stroke}%
\end{pgfscope}%
\begin{pgfscope}%
\pgfsetbuttcap%
\pgfsetroundjoin%
\definecolor{currentfill}{rgb}{0.000000,0.000000,0.000000}%
\pgfsetfillcolor{currentfill}%
\pgfsetlinewidth{0.803000pt}%
\definecolor{currentstroke}{rgb}{0.000000,0.000000,0.000000}%
\pgfsetstrokecolor{currentstroke}%
\pgfsetdash{}{0pt}%
\pgfsys@defobject{currentmarker}{\pgfqpoint{-0.048611in}{0.000000in}}{\pgfqpoint{-0.000000in}{0.000000in}}{%
\pgfpathmoveto{\pgfqpoint{-0.000000in}{0.000000in}}%
\pgfpathlineto{\pgfqpoint{-0.048611in}{0.000000in}}%
\pgfusepath{stroke,fill}%
}%
\begin{pgfscope}%
\pgfsys@transformshift{0.589510in}{0.622360in}%
\pgfsys@useobject{currentmarker}{}%
\end{pgfscope}%
\end{pgfscope}%
\begin{pgfscope}%
\definecolor{textcolor}{rgb}{0.000000,0.000000,0.000000}%
\pgfsetstrokecolor{textcolor}%
\pgfsetfillcolor{textcolor}%
\pgftext[x=0.236114in, y=0.583207in, left, base]{\color{textcolor}{\rmfamily\fontsize{8.000000}{9.600000}\selectfont\catcode`\^=\active\def^{\ifmmode\sp\else\^{}\fi}\catcode`\%=\active\def%{\%}$\mathdefault{10^{-1}}$}}%
\end{pgfscope}%
\begin{pgfscope}%
\pgfpathrectangle{\pgfqpoint{0.589510in}{0.417642in}}{\pgfqpoint{1.809765in}{1.371397in}}%
\pgfusepath{clip}%
\pgfsetrectcap%
\pgfsetroundjoin%
\pgfsetlinewidth{0.803000pt}%
\definecolor{currentstroke}{rgb}{0.450000,0.450000,0.450000}%
\pgfsetstrokecolor{currentstroke}%
\pgfsetdash{}{0pt}%
\pgfpathmoveto{\pgfqpoint{0.589510in}{0.827077in}}%
\pgfpathlineto{\pgfqpoint{2.399275in}{0.827077in}}%
\pgfusepath{stroke}%
\end{pgfscope}%
\begin{pgfscope}%
\pgfsetbuttcap%
\pgfsetroundjoin%
\definecolor{currentfill}{rgb}{0.000000,0.000000,0.000000}%
\pgfsetfillcolor{currentfill}%
\pgfsetlinewidth{0.803000pt}%
\definecolor{currentstroke}{rgb}{0.000000,0.000000,0.000000}%
\pgfsetstrokecolor{currentstroke}%
\pgfsetdash{}{0pt}%
\pgfsys@defobject{currentmarker}{\pgfqpoint{-0.048611in}{0.000000in}}{\pgfqpoint{-0.000000in}{0.000000in}}{%
\pgfpathmoveto{\pgfqpoint{-0.000000in}{0.000000in}}%
\pgfpathlineto{\pgfqpoint{-0.048611in}{0.000000in}}%
\pgfusepath{stroke,fill}%
}%
\begin{pgfscope}%
\pgfsys@transformshift{0.589510in}{0.827077in}%
\pgfsys@useobject{currentmarker}{}%
\end{pgfscope}%
\end{pgfscope}%
\begin{pgfscope}%
\definecolor{textcolor}{rgb}{0.000000,0.000000,0.000000}%
\pgfsetstrokecolor{textcolor}%
\pgfsetfillcolor{textcolor}%
\pgftext[x=0.316361in, y=0.787924in, left, base]{\color{textcolor}{\rmfamily\fontsize{8.000000}{9.600000}\selectfont\catcode`\^=\active\def^{\ifmmode\sp\else\^{}\fi}\catcode`\%=\active\def%{\%}$\mathdefault{10^{0}}$}}%
\end{pgfscope}%
\begin{pgfscope}%
\pgfpathrectangle{\pgfqpoint{0.589510in}{0.417642in}}{\pgfqpoint{1.809765in}{1.371397in}}%
\pgfusepath{clip}%
\pgfsetrectcap%
\pgfsetroundjoin%
\pgfsetlinewidth{0.803000pt}%
\definecolor{currentstroke}{rgb}{0.450000,0.450000,0.450000}%
\pgfsetstrokecolor{currentstroke}%
\pgfsetdash{}{0pt}%
\pgfpathmoveto{\pgfqpoint{0.589510in}{1.031795in}}%
\pgfpathlineto{\pgfqpoint{2.399275in}{1.031795in}}%
\pgfusepath{stroke}%
\end{pgfscope}%
\begin{pgfscope}%
\pgfsetbuttcap%
\pgfsetroundjoin%
\definecolor{currentfill}{rgb}{0.000000,0.000000,0.000000}%
\pgfsetfillcolor{currentfill}%
\pgfsetlinewidth{0.803000pt}%
\definecolor{currentstroke}{rgb}{0.000000,0.000000,0.000000}%
\pgfsetstrokecolor{currentstroke}%
\pgfsetdash{}{0pt}%
\pgfsys@defobject{currentmarker}{\pgfqpoint{-0.048611in}{0.000000in}}{\pgfqpoint{-0.000000in}{0.000000in}}{%
\pgfpathmoveto{\pgfqpoint{-0.000000in}{0.000000in}}%
\pgfpathlineto{\pgfqpoint{-0.048611in}{0.000000in}}%
\pgfusepath{stroke,fill}%
}%
\begin{pgfscope}%
\pgfsys@transformshift{0.589510in}{1.031795in}%
\pgfsys@useobject{currentmarker}{}%
\end{pgfscope}%
\end{pgfscope}%
\begin{pgfscope}%
\definecolor{textcolor}{rgb}{0.000000,0.000000,0.000000}%
\pgfsetstrokecolor{textcolor}%
\pgfsetfillcolor{textcolor}%
\pgftext[x=0.316361in, y=0.992642in, left, base]{\color{textcolor}{\rmfamily\fontsize{8.000000}{9.600000}\selectfont\catcode`\^=\active\def^{\ifmmode\sp\else\^{}\fi}\catcode`\%=\active\def%{\%}$\mathdefault{10^{1}}$}}%
\end{pgfscope}%
\begin{pgfscope}%
\pgfpathrectangle{\pgfqpoint{0.589510in}{0.417642in}}{\pgfqpoint{1.809765in}{1.371397in}}%
\pgfusepath{clip}%
\pgfsetrectcap%
\pgfsetroundjoin%
\pgfsetlinewidth{0.803000pt}%
\definecolor{currentstroke}{rgb}{0.450000,0.450000,0.450000}%
\pgfsetstrokecolor{currentstroke}%
\pgfsetdash{}{0pt}%
\pgfpathmoveto{\pgfqpoint{0.589510in}{1.236512in}}%
\pgfpathlineto{\pgfqpoint{2.399275in}{1.236512in}}%
\pgfusepath{stroke}%
\end{pgfscope}%
\begin{pgfscope}%
\pgfsetbuttcap%
\pgfsetroundjoin%
\definecolor{currentfill}{rgb}{0.000000,0.000000,0.000000}%
\pgfsetfillcolor{currentfill}%
\pgfsetlinewidth{0.803000pt}%
\definecolor{currentstroke}{rgb}{0.000000,0.000000,0.000000}%
\pgfsetstrokecolor{currentstroke}%
\pgfsetdash{}{0pt}%
\pgfsys@defobject{currentmarker}{\pgfqpoint{-0.048611in}{0.000000in}}{\pgfqpoint{-0.000000in}{0.000000in}}{%
\pgfpathmoveto{\pgfqpoint{-0.000000in}{0.000000in}}%
\pgfpathlineto{\pgfqpoint{-0.048611in}{0.000000in}}%
\pgfusepath{stroke,fill}%
}%
\begin{pgfscope}%
\pgfsys@transformshift{0.589510in}{1.236512in}%
\pgfsys@useobject{currentmarker}{}%
\end{pgfscope}%
\end{pgfscope}%
\begin{pgfscope}%
\definecolor{textcolor}{rgb}{0.000000,0.000000,0.000000}%
\pgfsetstrokecolor{textcolor}%
\pgfsetfillcolor{textcolor}%
\pgftext[x=0.316361in, y=1.197359in, left, base]{\color{textcolor}{\rmfamily\fontsize{8.000000}{9.600000}\selectfont\catcode`\^=\active\def^{\ifmmode\sp\else\^{}\fi}\catcode`\%=\active\def%{\%}$\mathdefault{10^{2}}$}}%
\end{pgfscope}%
\begin{pgfscope}%
\pgfpathrectangle{\pgfqpoint{0.589510in}{0.417642in}}{\pgfqpoint{1.809765in}{1.371397in}}%
\pgfusepath{clip}%
\pgfsetrectcap%
\pgfsetroundjoin%
\pgfsetlinewidth{0.803000pt}%
\definecolor{currentstroke}{rgb}{0.450000,0.450000,0.450000}%
\pgfsetstrokecolor{currentstroke}%
\pgfsetdash{}{0pt}%
\pgfpathmoveto{\pgfqpoint{0.589510in}{1.441230in}}%
\pgfpathlineto{\pgfqpoint{2.399275in}{1.441230in}}%
\pgfusepath{stroke}%
\end{pgfscope}%
\begin{pgfscope}%
\pgfsetbuttcap%
\pgfsetroundjoin%
\definecolor{currentfill}{rgb}{0.000000,0.000000,0.000000}%
\pgfsetfillcolor{currentfill}%
\pgfsetlinewidth{0.803000pt}%
\definecolor{currentstroke}{rgb}{0.000000,0.000000,0.000000}%
\pgfsetstrokecolor{currentstroke}%
\pgfsetdash{}{0pt}%
\pgfsys@defobject{currentmarker}{\pgfqpoint{-0.048611in}{0.000000in}}{\pgfqpoint{-0.000000in}{0.000000in}}{%
\pgfpathmoveto{\pgfqpoint{-0.000000in}{0.000000in}}%
\pgfpathlineto{\pgfqpoint{-0.048611in}{0.000000in}}%
\pgfusepath{stroke,fill}%
}%
\begin{pgfscope}%
\pgfsys@transformshift{0.589510in}{1.441230in}%
\pgfsys@useobject{currentmarker}{}%
\end{pgfscope}%
\end{pgfscope}%
\begin{pgfscope}%
\definecolor{textcolor}{rgb}{0.000000,0.000000,0.000000}%
\pgfsetstrokecolor{textcolor}%
\pgfsetfillcolor{textcolor}%
\pgftext[x=0.316361in, y=1.402077in, left, base]{\color{textcolor}{\rmfamily\fontsize{8.000000}{9.600000}\selectfont\catcode`\^=\active\def^{\ifmmode\sp\else\^{}\fi}\catcode`\%=\active\def%{\%}$\mathdefault{10^{3}}$}}%
\end{pgfscope}%
\begin{pgfscope}%
\pgfpathrectangle{\pgfqpoint{0.589510in}{0.417642in}}{\pgfqpoint{1.809765in}{1.371397in}}%
\pgfusepath{clip}%
\pgfsetrectcap%
\pgfsetroundjoin%
\pgfsetlinewidth{0.803000pt}%
\definecolor{currentstroke}{rgb}{0.450000,0.450000,0.450000}%
\pgfsetstrokecolor{currentstroke}%
\pgfsetdash{}{0pt}%
\pgfpathmoveto{\pgfqpoint{0.589510in}{1.645947in}}%
\pgfpathlineto{\pgfqpoint{2.399275in}{1.645947in}}%
\pgfusepath{stroke}%
\end{pgfscope}%
\begin{pgfscope}%
\pgfsetbuttcap%
\pgfsetroundjoin%
\definecolor{currentfill}{rgb}{0.000000,0.000000,0.000000}%
\pgfsetfillcolor{currentfill}%
\pgfsetlinewidth{0.803000pt}%
\definecolor{currentstroke}{rgb}{0.000000,0.000000,0.000000}%
\pgfsetstrokecolor{currentstroke}%
\pgfsetdash{}{0pt}%
\pgfsys@defobject{currentmarker}{\pgfqpoint{-0.048611in}{0.000000in}}{\pgfqpoint{-0.000000in}{0.000000in}}{%
\pgfpathmoveto{\pgfqpoint{-0.000000in}{0.000000in}}%
\pgfpathlineto{\pgfqpoint{-0.048611in}{0.000000in}}%
\pgfusepath{stroke,fill}%
}%
\begin{pgfscope}%
\pgfsys@transformshift{0.589510in}{1.645947in}%
\pgfsys@useobject{currentmarker}{}%
\end{pgfscope}%
\end{pgfscope}%
\begin{pgfscope}%
\definecolor{textcolor}{rgb}{0.000000,0.000000,0.000000}%
\pgfsetstrokecolor{textcolor}%
\pgfsetfillcolor{textcolor}%
\pgftext[x=0.316361in, y=1.606795in, left, base]{\color{textcolor}{\rmfamily\fontsize{8.000000}{9.600000}\selectfont\catcode`\^=\active\def^{\ifmmode\sp\else\^{}\fi}\catcode`\%=\active\def%{\%}$\mathdefault{10^{4}}$}}%
\end{pgfscope}%
\begin{pgfscope}%
\pgfsetbuttcap%
\pgfsetroundjoin%
\definecolor{currentfill}{rgb}{0.000000,0.000000,0.000000}%
\pgfsetfillcolor{currentfill}%
\pgfsetlinewidth{0.602250pt}%
\definecolor{currentstroke}{rgb}{0.000000,0.000000,0.000000}%
\pgfsetstrokecolor{currentstroke}%
\pgfsetdash{}{0pt}%
\pgfsys@defobject{currentmarker}{\pgfqpoint{-0.027778in}{0.000000in}}{\pgfqpoint{-0.000000in}{0.000000in}}{%
\pgfpathmoveto{\pgfqpoint{-0.000000in}{0.000000in}}%
\pgfpathlineto{\pgfqpoint{-0.027778in}{0.000000in}}%
\pgfusepath{stroke,fill}%
}%
\begin{pgfscope}%
\pgfsys@transformshift{0.589510in}{0.479268in}%
\pgfsys@useobject{currentmarker}{}%
\end{pgfscope}%
\end{pgfscope}%
\begin{pgfscope}%
\pgfsetbuttcap%
\pgfsetroundjoin%
\definecolor{currentfill}{rgb}{0.000000,0.000000,0.000000}%
\pgfsetfillcolor{currentfill}%
\pgfsetlinewidth{0.602250pt}%
\definecolor{currentstroke}{rgb}{0.000000,0.000000,0.000000}%
\pgfsetstrokecolor{currentstroke}%
\pgfsetdash{}{0pt}%
\pgfsys@defobject{currentmarker}{\pgfqpoint{-0.027778in}{0.000000in}}{\pgfqpoint{-0.000000in}{0.000000in}}{%
\pgfpathmoveto{\pgfqpoint{-0.000000in}{0.000000in}}%
\pgfpathlineto{\pgfqpoint{-0.027778in}{0.000000in}}%
\pgfusepath{stroke,fill}%
}%
\begin{pgfscope}%
\pgfsys@transformshift{0.589510in}{0.515317in}%
\pgfsys@useobject{currentmarker}{}%
\end{pgfscope}%
\end{pgfscope}%
\begin{pgfscope}%
\pgfsetbuttcap%
\pgfsetroundjoin%
\definecolor{currentfill}{rgb}{0.000000,0.000000,0.000000}%
\pgfsetfillcolor{currentfill}%
\pgfsetlinewidth{0.602250pt}%
\definecolor{currentstroke}{rgb}{0.000000,0.000000,0.000000}%
\pgfsetstrokecolor{currentstroke}%
\pgfsetdash{}{0pt}%
\pgfsys@defobject{currentmarker}{\pgfqpoint{-0.027778in}{0.000000in}}{\pgfqpoint{-0.000000in}{0.000000in}}{%
\pgfpathmoveto{\pgfqpoint{-0.000000in}{0.000000in}}%
\pgfpathlineto{\pgfqpoint{-0.027778in}{0.000000in}}%
\pgfusepath{stroke,fill}%
}%
\begin{pgfscope}%
\pgfsys@transformshift{0.589510in}{0.540894in}%
\pgfsys@useobject{currentmarker}{}%
\end{pgfscope}%
\end{pgfscope}%
\begin{pgfscope}%
\pgfsetbuttcap%
\pgfsetroundjoin%
\definecolor{currentfill}{rgb}{0.000000,0.000000,0.000000}%
\pgfsetfillcolor{currentfill}%
\pgfsetlinewidth{0.602250pt}%
\definecolor{currentstroke}{rgb}{0.000000,0.000000,0.000000}%
\pgfsetstrokecolor{currentstroke}%
\pgfsetdash{}{0pt}%
\pgfsys@defobject{currentmarker}{\pgfqpoint{-0.027778in}{0.000000in}}{\pgfqpoint{-0.000000in}{0.000000in}}{%
\pgfpathmoveto{\pgfqpoint{-0.000000in}{0.000000in}}%
\pgfpathlineto{\pgfqpoint{-0.027778in}{0.000000in}}%
\pgfusepath{stroke,fill}%
}%
\begin{pgfscope}%
\pgfsys@transformshift{0.589510in}{0.560733in}%
\pgfsys@useobject{currentmarker}{}%
\end{pgfscope}%
\end{pgfscope}%
\begin{pgfscope}%
\pgfsetbuttcap%
\pgfsetroundjoin%
\definecolor{currentfill}{rgb}{0.000000,0.000000,0.000000}%
\pgfsetfillcolor{currentfill}%
\pgfsetlinewidth{0.602250pt}%
\definecolor{currentstroke}{rgb}{0.000000,0.000000,0.000000}%
\pgfsetstrokecolor{currentstroke}%
\pgfsetdash{}{0pt}%
\pgfsys@defobject{currentmarker}{\pgfqpoint{-0.027778in}{0.000000in}}{\pgfqpoint{-0.000000in}{0.000000in}}{%
\pgfpathmoveto{\pgfqpoint{-0.000000in}{0.000000in}}%
\pgfpathlineto{\pgfqpoint{-0.027778in}{0.000000in}}%
\pgfusepath{stroke,fill}%
}%
\begin{pgfscope}%
\pgfsys@transformshift{0.589510in}{0.576943in}%
\pgfsys@useobject{currentmarker}{}%
\end{pgfscope}%
\end{pgfscope}%
\begin{pgfscope}%
\pgfsetbuttcap%
\pgfsetroundjoin%
\definecolor{currentfill}{rgb}{0.000000,0.000000,0.000000}%
\pgfsetfillcolor{currentfill}%
\pgfsetlinewidth{0.602250pt}%
\definecolor{currentstroke}{rgb}{0.000000,0.000000,0.000000}%
\pgfsetstrokecolor{currentstroke}%
\pgfsetdash{}{0pt}%
\pgfsys@defobject{currentmarker}{\pgfqpoint{-0.027778in}{0.000000in}}{\pgfqpoint{-0.000000in}{0.000000in}}{%
\pgfpathmoveto{\pgfqpoint{-0.000000in}{0.000000in}}%
\pgfpathlineto{\pgfqpoint{-0.027778in}{0.000000in}}%
\pgfusepath{stroke,fill}%
}%
\begin{pgfscope}%
\pgfsys@transformshift{0.589510in}{0.590648in}%
\pgfsys@useobject{currentmarker}{}%
\end{pgfscope}%
\end{pgfscope}%
\begin{pgfscope}%
\pgfsetbuttcap%
\pgfsetroundjoin%
\definecolor{currentfill}{rgb}{0.000000,0.000000,0.000000}%
\pgfsetfillcolor{currentfill}%
\pgfsetlinewidth{0.602250pt}%
\definecolor{currentstroke}{rgb}{0.000000,0.000000,0.000000}%
\pgfsetstrokecolor{currentstroke}%
\pgfsetdash{}{0pt}%
\pgfsys@defobject{currentmarker}{\pgfqpoint{-0.027778in}{0.000000in}}{\pgfqpoint{-0.000000in}{0.000000in}}{%
\pgfpathmoveto{\pgfqpoint{-0.000000in}{0.000000in}}%
\pgfpathlineto{\pgfqpoint{-0.027778in}{0.000000in}}%
\pgfusepath{stroke,fill}%
}%
\begin{pgfscope}%
\pgfsys@transformshift{0.589510in}{0.602520in}%
\pgfsys@useobject{currentmarker}{}%
\end{pgfscope}%
\end{pgfscope}%
\begin{pgfscope}%
\pgfsetbuttcap%
\pgfsetroundjoin%
\definecolor{currentfill}{rgb}{0.000000,0.000000,0.000000}%
\pgfsetfillcolor{currentfill}%
\pgfsetlinewidth{0.602250pt}%
\definecolor{currentstroke}{rgb}{0.000000,0.000000,0.000000}%
\pgfsetstrokecolor{currentstroke}%
\pgfsetdash{}{0pt}%
\pgfsys@defobject{currentmarker}{\pgfqpoint{-0.027778in}{0.000000in}}{\pgfqpoint{-0.000000in}{0.000000in}}{%
\pgfpathmoveto{\pgfqpoint{-0.000000in}{0.000000in}}%
\pgfpathlineto{\pgfqpoint{-0.027778in}{0.000000in}}%
\pgfusepath{stroke,fill}%
}%
\begin{pgfscope}%
\pgfsys@transformshift{0.589510in}{0.612992in}%
\pgfsys@useobject{currentmarker}{}%
\end{pgfscope}%
\end{pgfscope}%
\begin{pgfscope}%
\pgfsetbuttcap%
\pgfsetroundjoin%
\definecolor{currentfill}{rgb}{0.000000,0.000000,0.000000}%
\pgfsetfillcolor{currentfill}%
\pgfsetlinewidth{0.602250pt}%
\definecolor{currentstroke}{rgb}{0.000000,0.000000,0.000000}%
\pgfsetstrokecolor{currentstroke}%
\pgfsetdash{}{0pt}%
\pgfsys@defobject{currentmarker}{\pgfqpoint{-0.027778in}{0.000000in}}{\pgfqpoint{-0.000000in}{0.000000in}}{%
\pgfpathmoveto{\pgfqpoint{-0.000000in}{0.000000in}}%
\pgfpathlineto{\pgfqpoint{-0.027778in}{0.000000in}}%
\pgfusepath{stroke,fill}%
}%
\begin{pgfscope}%
\pgfsys@transformshift{0.589510in}{0.683986in}%
\pgfsys@useobject{currentmarker}{}%
\end{pgfscope}%
\end{pgfscope}%
\begin{pgfscope}%
\pgfsetbuttcap%
\pgfsetroundjoin%
\definecolor{currentfill}{rgb}{0.000000,0.000000,0.000000}%
\pgfsetfillcolor{currentfill}%
\pgfsetlinewidth{0.602250pt}%
\definecolor{currentstroke}{rgb}{0.000000,0.000000,0.000000}%
\pgfsetstrokecolor{currentstroke}%
\pgfsetdash{}{0pt}%
\pgfsys@defobject{currentmarker}{\pgfqpoint{-0.027778in}{0.000000in}}{\pgfqpoint{-0.000000in}{0.000000in}}{%
\pgfpathmoveto{\pgfqpoint{-0.000000in}{0.000000in}}%
\pgfpathlineto{\pgfqpoint{-0.027778in}{0.000000in}}%
\pgfusepath{stroke,fill}%
}%
\begin{pgfscope}%
\pgfsys@transformshift{0.589510in}{0.720035in}%
\pgfsys@useobject{currentmarker}{}%
\end{pgfscope}%
\end{pgfscope}%
\begin{pgfscope}%
\pgfsetbuttcap%
\pgfsetroundjoin%
\definecolor{currentfill}{rgb}{0.000000,0.000000,0.000000}%
\pgfsetfillcolor{currentfill}%
\pgfsetlinewidth{0.602250pt}%
\definecolor{currentstroke}{rgb}{0.000000,0.000000,0.000000}%
\pgfsetstrokecolor{currentstroke}%
\pgfsetdash{}{0pt}%
\pgfsys@defobject{currentmarker}{\pgfqpoint{-0.027778in}{0.000000in}}{\pgfqpoint{-0.000000in}{0.000000in}}{%
\pgfpathmoveto{\pgfqpoint{-0.000000in}{0.000000in}}%
\pgfpathlineto{\pgfqpoint{-0.027778in}{0.000000in}}%
\pgfusepath{stroke,fill}%
}%
\begin{pgfscope}%
\pgfsys@transformshift{0.589510in}{0.745612in}%
\pgfsys@useobject{currentmarker}{}%
\end{pgfscope}%
\end{pgfscope}%
\begin{pgfscope}%
\pgfsetbuttcap%
\pgfsetroundjoin%
\definecolor{currentfill}{rgb}{0.000000,0.000000,0.000000}%
\pgfsetfillcolor{currentfill}%
\pgfsetlinewidth{0.602250pt}%
\definecolor{currentstroke}{rgb}{0.000000,0.000000,0.000000}%
\pgfsetstrokecolor{currentstroke}%
\pgfsetdash{}{0pt}%
\pgfsys@defobject{currentmarker}{\pgfqpoint{-0.027778in}{0.000000in}}{\pgfqpoint{-0.000000in}{0.000000in}}{%
\pgfpathmoveto{\pgfqpoint{-0.000000in}{0.000000in}}%
\pgfpathlineto{\pgfqpoint{-0.027778in}{0.000000in}}%
\pgfusepath{stroke,fill}%
}%
\begin{pgfscope}%
\pgfsys@transformshift{0.589510in}{0.765451in}%
\pgfsys@useobject{currentmarker}{}%
\end{pgfscope}%
\end{pgfscope}%
\begin{pgfscope}%
\pgfsetbuttcap%
\pgfsetroundjoin%
\definecolor{currentfill}{rgb}{0.000000,0.000000,0.000000}%
\pgfsetfillcolor{currentfill}%
\pgfsetlinewidth{0.602250pt}%
\definecolor{currentstroke}{rgb}{0.000000,0.000000,0.000000}%
\pgfsetstrokecolor{currentstroke}%
\pgfsetdash{}{0pt}%
\pgfsys@defobject{currentmarker}{\pgfqpoint{-0.027778in}{0.000000in}}{\pgfqpoint{-0.000000in}{0.000000in}}{%
\pgfpathmoveto{\pgfqpoint{-0.000000in}{0.000000in}}%
\pgfpathlineto{\pgfqpoint{-0.027778in}{0.000000in}}%
\pgfusepath{stroke,fill}%
}%
\begin{pgfscope}%
\pgfsys@transformshift{0.589510in}{0.781661in}%
\pgfsys@useobject{currentmarker}{}%
\end{pgfscope}%
\end{pgfscope}%
\begin{pgfscope}%
\pgfsetbuttcap%
\pgfsetroundjoin%
\definecolor{currentfill}{rgb}{0.000000,0.000000,0.000000}%
\pgfsetfillcolor{currentfill}%
\pgfsetlinewidth{0.602250pt}%
\definecolor{currentstroke}{rgb}{0.000000,0.000000,0.000000}%
\pgfsetstrokecolor{currentstroke}%
\pgfsetdash{}{0pt}%
\pgfsys@defobject{currentmarker}{\pgfqpoint{-0.027778in}{0.000000in}}{\pgfqpoint{-0.000000in}{0.000000in}}{%
\pgfpathmoveto{\pgfqpoint{-0.000000in}{0.000000in}}%
\pgfpathlineto{\pgfqpoint{-0.027778in}{0.000000in}}%
\pgfusepath{stroke,fill}%
}%
\begin{pgfscope}%
\pgfsys@transformshift{0.589510in}{0.795366in}%
\pgfsys@useobject{currentmarker}{}%
\end{pgfscope}%
\end{pgfscope}%
\begin{pgfscope}%
\pgfsetbuttcap%
\pgfsetroundjoin%
\definecolor{currentfill}{rgb}{0.000000,0.000000,0.000000}%
\pgfsetfillcolor{currentfill}%
\pgfsetlinewidth{0.602250pt}%
\definecolor{currentstroke}{rgb}{0.000000,0.000000,0.000000}%
\pgfsetstrokecolor{currentstroke}%
\pgfsetdash{}{0pt}%
\pgfsys@defobject{currentmarker}{\pgfqpoint{-0.027778in}{0.000000in}}{\pgfqpoint{-0.000000in}{0.000000in}}{%
\pgfpathmoveto{\pgfqpoint{-0.000000in}{0.000000in}}%
\pgfpathlineto{\pgfqpoint{-0.027778in}{0.000000in}}%
\pgfusepath{stroke,fill}%
}%
\begin{pgfscope}%
\pgfsys@transformshift{0.589510in}{0.807238in}%
\pgfsys@useobject{currentmarker}{}%
\end{pgfscope}%
\end{pgfscope}%
\begin{pgfscope}%
\pgfsetbuttcap%
\pgfsetroundjoin%
\definecolor{currentfill}{rgb}{0.000000,0.000000,0.000000}%
\pgfsetfillcolor{currentfill}%
\pgfsetlinewidth{0.602250pt}%
\definecolor{currentstroke}{rgb}{0.000000,0.000000,0.000000}%
\pgfsetstrokecolor{currentstroke}%
\pgfsetdash{}{0pt}%
\pgfsys@defobject{currentmarker}{\pgfqpoint{-0.027778in}{0.000000in}}{\pgfqpoint{-0.000000in}{0.000000in}}{%
\pgfpathmoveto{\pgfqpoint{-0.000000in}{0.000000in}}%
\pgfpathlineto{\pgfqpoint{-0.027778in}{0.000000in}}%
\pgfusepath{stroke,fill}%
}%
\begin{pgfscope}%
\pgfsys@transformshift{0.589510in}{0.817710in}%
\pgfsys@useobject{currentmarker}{}%
\end{pgfscope}%
\end{pgfscope}%
\begin{pgfscope}%
\pgfsetbuttcap%
\pgfsetroundjoin%
\definecolor{currentfill}{rgb}{0.000000,0.000000,0.000000}%
\pgfsetfillcolor{currentfill}%
\pgfsetlinewidth{0.602250pt}%
\definecolor{currentstroke}{rgb}{0.000000,0.000000,0.000000}%
\pgfsetstrokecolor{currentstroke}%
\pgfsetdash{}{0pt}%
\pgfsys@defobject{currentmarker}{\pgfqpoint{-0.027778in}{0.000000in}}{\pgfqpoint{-0.000000in}{0.000000in}}{%
\pgfpathmoveto{\pgfqpoint{-0.000000in}{0.000000in}}%
\pgfpathlineto{\pgfqpoint{-0.027778in}{0.000000in}}%
\pgfusepath{stroke,fill}%
}%
\begin{pgfscope}%
\pgfsys@transformshift{0.589510in}{0.888703in}%
\pgfsys@useobject{currentmarker}{}%
\end{pgfscope}%
\end{pgfscope}%
\begin{pgfscope}%
\pgfsetbuttcap%
\pgfsetroundjoin%
\definecolor{currentfill}{rgb}{0.000000,0.000000,0.000000}%
\pgfsetfillcolor{currentfill}%
\pgfsetlinewidth{0.602250pt}%
\definecolor{currentstroke}{rgb}{0.000000,0.000000,0.000000}%
\pgfsetstrokecolor{currentstroke}%
\pgfsetdash{}{0pt}%
\pgfsys@defobject{currentmarker}{\pgfqpoint{-0.027778in}{0.000000in}}{\pgfqpoint{-0.000000in}{0.000000in}}{%
\pgfpathmoveto{\pgfqpoint{-0.000000in}{0.000000in}}%
\pgfpathlineto{\pgfqpoint{-0.027778in}{0.000000in}}%
\pgfusepath{stroke,fill}%
}%
\begin{pgfscope}%
\pgfsys@transformshift{0.589510in}{0.924752in}%
\pgfsys@useobject{currentmarker}{}%
\end{pgfscope}%
\end{pgfscope}%
\begin{pgfscope}%
\pgfsetbuttcap%
\pgfsetroundjoin%
\definecolor{currentfill}{rgb}{0.000000,0.000000,0.000000}%
\pgfsetfillcolor{currentfill}%
\pgfsetlinewidth{0.602250pt}%
\definecolor{currentstroke}{rgb}{0.000000,0.000000,0.000000}%
\pgfsetstrokecolor{currentstroke}%
\pgfsetdash{}{0pt}%
\pgfsys@defobject{currentmarker}{\pgfqpoint{-0.027778in}{0.000000in}}{\pgfqpoint{-0.000000in}{0.000000in}}{%
\pgfpathmoveto{\pgfqpoint{-0.000000in}{0.000000in}}%
\pgfpathlineto{\pgfqpoint{-0.027778in}{0.000000in}}%
\pgfusepath{stroke,fill}%
}%
\begin{pgfscope}%
\pgfsys@transformshift{0.589510in}{0.950329in}%
\pgfsys@useobject{currentmarker}{}%
\end{pgfscope}%
\end{pgfscope}%
\begin{pgfscope}%
\pgfsetbuttcap%
\pgfsetroundjoin%
\definecolor{currentfill}{rgb}{0.000000,0.000000,0.000000}%
\pgfsetfillcolor{currentfill}%
\pgfsetlinewidth{0.602250pt}%
\definecolor{currentstroke}{rgb}{0.000000,0.000000,0.000000}%
\pgfsetstrokecolor{currentstroke}%
\pgfsetdash{}{0pt}%
\pgfsys@defobject{currentmarker}{\pgfqpoint{-0.027778in}{0.000000in}}{\pgfqpoint{-0.000000in}{0.000000in}}{%
\pgfpathmoveto{\pgfqpoint{-0.000000in}{0.000000in}}%
\pgfpathlineto{\pgfqpoint{-0.027778in}{0.000000in}}%
\pgfusepath{stroke,fill}%
}%
\begin{pgfscope}%
\pgfsys@transformshift{0.589510in}{0.970168in}%
\pgfsys@useobject{currentmarker}{}%
\end{pgfscope}%
\end{pgfscope}%
\begin{pgfscope}%
\pgfsetbuttcap%
\pgfsetroundjoin%
\definecolor{currentfill}{rgb}{0.000000,0.000000,0.000000}%
\pgfsetfillcolor{currentfill}%
\pgfsetlinewidth{0.602250pt}%
\definecolor{currentstroke}{rgb}{0.000000,0.000000,0.000000}%
\pgfsetstrokecolor{currentstroke}%
\pgfsetdash{}{0pt}%
\pgfsys@defobject{currentmarker}{\pgfqpoint{-0.027778in}{0.000000in}}{\pgfqpoint{-0.000000in}{0.000000in}}{%
\pgfpathmoveto{\pgfqpoint{-0.000000in}{0.000000in}}%
\pgfpathlineto{\pgfqpoint{-0.027778in}{0.000000in}}%
\pgfusepath{stroke,fill}%
}%
\begin{pgfscope}%
\pgfsys@transformshift{0.589510in}{0.986378in}%
\pgfsys@useobject{currentmarker}{}%
\end{pgfscope}%
\end{pgfscope}%
\begin{pgfscope}%
\pgfsetbuttcap%
\pgfsetroundjoin%
\definecolor{currentfill}{rgb}{0.000000,0.000000,0.000000}%
\pgfsetfillcolor{currentfill}%
\pgfsetlinewidth{0.602250pt}%
\definecolor{currentstroke}{rgb}{0.000000,0.000000,0.000000}%
\pgfsetstrokecolor{currentstroke}%
\pgfsetdash{}{0pt}%
\pgfsys@defobject{currentmarker}{\pgfqpoint{-0.027778in}{0.000000in}}{\pgfqpoint{-0.000000in}{0.000000in}}{%
\pgfpathmoveto{\pgfqpoint{-0.000000in}{0.000000in}}%
\pgfpathlineto{\pgfqpoint{-0.027778in}{0.000000in}}%
\pgfusepath{stroke,fill}%
}%
\begin{pgfscope}%
\pgfsys@transformshift{0.589510in}{1.000083in}%
\pgfsys@useobject{currentmarker}{}%
\end{pgfscope}%
\end{pgfscope}%
\begin{pgfscope}%
\pgfsetbuttcap%
\pgfsetroundjoin%
\definecolor{currentfill}{rgb}{0.000000,0.000000,0.000000}%
\pgfsetfillcolor{currentfill}%
\pgfsetlinewidth{0.602250pt}%
\definecolor{currentstroke}{rgb}{0.000000,0.000000,0.000000}%
\pgfsetstrokecolor{currentstroke}%
\pgfsetdash{}{0pt}%
\pgfsys@defobject{currentmarker}{\pgfqpoint{-0.027778in}{0.000000in}}{\pgfqpoint{-0.000000in}{0.000000in}}{%
\pgfpathmoveto{\pgfqpoint{-0.000000in}{0.000000in}}%
\pgfpathlineto{\pgfqpoint{-0.027778in}{0.000000in}}%
\pgfusepath{stroke,fill}%
}%
\begin{pgfscope}%
\pgfsys@transformshift{0.589510in}{1.011955in}%
\pgfsys@useobject{currentmarker}{}%
\end{pgfscope}%
\end{pgfscope}%
\begin{pgfscope}%
\pgfsetbuttcap%
\pgfsetroundjoin%
\definecolor{currentfill}{rgb}{0.000000,0.000000,0.000000}%
\pgfsetfillcolor{currentfill}%
\pgfsetlinewidth{0.602250pt}%
\definecolor{currentstroke}{rgb}{0.000000,0.000000,0.000000}%
\pgfsetstrokecolor{currentstroke}%
\pgfsetdash{}{0pt}%
\pgfsys@defobject{currentmarker}{\pgfqpoint{-0.027778in}{0.000000in}}{\pgfqpoint{-0.000000in}{0.000000in}}{%
\pgfpathmoveto{\pgfqpoint{-0.000000in}{0.000000in}}%
\pgfpathlineto{\pgfqpoint{-0.027778in}{0.000000in}}%
\pgfusepath{stroke,fill}%
}%
\begin{pgfscope}%
\pgfsys@transformshift{0.589510in}{1.022427in}%
\pgfsys@useobject{currentmarker}{}%
\end{pgfscope}%
\end{pgfscope}%
\begin{pgfscope}%
\pgfsetbuttcap%
\pgfsetroundjoin%
\definecolor{currentfill}{rgb}{0.000000,0.000000,0.000000}%
\pgfsetfillcolor{currentfill}%
\pgfsetlinewidth{0.602250pt}%
\definecolor{currentstroke}{rgb}{0.000000,0.000000,0.000000}%
\pgfsetstrokecolor{currentstroke}%
\pgfsetdash{}{0pt}%
\pgfsys@defobject{currentmarker}{\pgfqpoint{-0.027778in}{0.000000in}}{\pgfqpoint{-0.000000in}{0.000000in}}{%
\pgfpathmoveto{\pgfqpoint{-0.000000in}{0.000000in}}%
\pgfpathlineto{\pgfqpoint{-0.027778in}{0.000000in}}%
\pgfusepath{stroke,fill}%
}%
\begin{pgfscope}%
\pgfsys@transformshift{0.589510in}{1.093421in}%
\pgfsys@useobject{currentmarker}{}%
\end{pgfscope}%
\end{pgfscope}%
\begin{pgfscope}%
\pgfsetbuttcap%
\pgfsetroundjoin%
\definecolor{currentfill}{rgb}{0.000000,0.000000,0.000000}%
\pgfsetfillcolor{currentfill}%
\pgfsetlinewidth{0.602250pt}%
\definecolor{currentstroke}{rgb}{0.000000,0.000000,0.000000}%
\pgfsetstrokecolor{currentstroke}%
\pgfsetdash{}{0pt}%
\pgfsys@defobject{currentmarker}{\pgfqpoint{-0.027778in}{0.000000in}}{\pgfqpoint{-0.000000in}{0.000000in}}{%
\pgfpathmoveto{\pgfqpoint{-0.000000in}{0.000000in}}%
\pgfpathlineto{\pgfqpoint{-0.027778in}{0.000000in}}%
\pgfusepath{stroke,fill}%
}%
\begin{pgfscope}%
\pgfsys@transformshift{0.589510in}{1.129470in}%
\pgfsys@useobject{currentmarker}{}%
\end{pgfscope}%
\end{pgfscope}%
\begin{pgfscope}%
\pgfsetbuttcap%
\pgfsetroundjoin%
\definecolor{currentfill}{rgb}{0.000000,0.000000,0.000000}%
\pgfsetfillcolor{currentfill}%
\pgfsetlinewidth{0.602250pt}%
\definecolor{currentstroke}{rgb}{0.000000,0.000000,0.000000}%
\pgfsetstrokecolor{currentstroke}%
\pgfsetdash{}{0pt}%
\pgfsys@defobject{currentmarker}{\pgfqpoint{-0.027778in}{0.000000in}}{\pgfqpoint{-0.000000in}{0.000000in}}{%
\pgfpathmoveto{\pgfqpoint{-0.000000in}{0.000000in}}%
\pgfpathlineto{\pgfqpoint{-0.027778in}{0.000000in}}%
\pgfusepath{stroke,fill}%
}%
\begin{pgfscope}%
\pgfsys@transformshift{0.589510in}{1.155047in}%
\pgfsys@useobject{currentmarker}{}%
\end{pgfscope}%
\end{pgfscope}%
\begin{pgfscope}%
\pgfsetbuttcap%
\pgfsetroundjoin%
\definecolor{currentfill}{rgb}{0.000000,0.000000,0.000000}%
\pgfsetfillcolor{currentfill}%
\pgfsetlinewidth{0.602250pt}%
\definecolor{currentstroke}{rgb}{0.000000,0.000000,0.000000}%
\pgfsetstrokecolor{currentstroke}%
\pgfsetdash{}{0pt}%
\pgfsys@defobject{currentmarker}{\pgfqpoint{-0.027778in}{0.000000in}}{\pgfqpoint{-0.000000in}{0.000000in}}{%
\pgfpathmoveto{\pgfqpoint{-0.000000in}{0.000000in}}%
\pgfpathlineto{\pgfqpoint{-0.027778in}{0.000000in}}%
\pgfusepath{stroke,fill}%
}%
\begin{pgfscope}%
\pgfsys@transformshift{0.589510in}{1.174886in}%
\pgfsys@useobject{currentmarker}{}%
\end{pgfscope}%
\end{pgfscope}%
\begin{pgfscope}%
\pgfsetbuttcap%
\pgfsetroundjoin%
\definecolor{currentfill}{rgb}{0.000000,0.000000,0.000000}%
\pgfsetfillcolor{currentfill}%
\pgfsetlinewidth{0.602250pt}%
\definecolor{currentstroke}{rgb}{0.000000,0.000000,0.000000}%
\pgfsetstrokecolor{currentstroke}%
\pgfsetdash{}{0pt}%
\pgfsys@defobject{currentmarker}{\pgfqpoint{-0.027778in}{0.000000in}}{\pgfqpoint{-0.000000in}{0.000000in}}{%
\pgfpathmoveto{\pgfqpoint{-0.000000in}{0.000000in}}%
\pgfpathlineto{\pgfqpoint{-0.027778in}{0.000000in}}%
\pgfusepath{stroke,fill}%
}%
\begin{pgfscope}%
\pgfsys@transformshift{0.589510in}{1.191096in}%
\pgfsys@useobject{currentmarker}{}%
\end{pgfscope}%
\end{pgfscope}%
\begin{pgfscope}%
\pgfsetbuttcap%
\pgfsetroundjoin%
\definecolor{currentfill}{rgb}{0.000000,0.000000,0.000000}%
\pgfsetfillcolor{currentfill}%
\pgfsetlinewidth{0.602250pt}%
\definecolor{currentstroke}{rgb}{0.000000,0.000000,0.000000}%
\pgfsetstrokecolor{currentstroke}%
\pgfsetdash{}{0pt}%
\pgfsys@defobject{currentmarker}{\pgfqpoint{-0.027778in}{0.000000in}}{\pgfqpoint{-0.000000in}{0.000000in}}{%
\pgfpathmoveto{\pgfqpoint{-0.000000in}{0.000000in}}%
\pgfpathlineto{\pgfqpoint{-0.027778in}{0.000000in}}%
\pgfusepath{stroke,fill}%
}%
\begin{pgfscope}%
\pgfsys@transformshift{0.589510in}{1.204801in}%
\pgfsys@useobject{currentmarker}{}%
\end{pgfscope}%
\end{pgfscope}%
\begin{pgfscope}%
\pgfsetbuttcap%
\pgfsetroundjoin%
\definecolor{currentfill}{rgb}{0.000000,0.000000,0.000000}%
\pgfsetfillcolor{currentfill}%
\pgfsetlinewidth{0.602250pt}%
\definecolor{currentstroke}{rgb}{0.000000,0.000000,0.000000}%
\pgfsetstrokecolor{currentstroke}%
\pgfsetdash{}{0pt}%
\pgfsys@defobject{currentmarker}{\pgfqpoint{-0.027778in}{0.000000in}}{\pgfqpoint{-0.000000in}{0.000000in}}{%
\pgfpathmoveto{\pgfqpoint{-0.000000in}{0.000000in}}%
\pgfpathlineto{\pgfqpoint{-0.027778in}{0.000000in}}%
\pgfusepath{stroke,fill}%
}%
\begin{pgfscope}%
\pgfsys@transformshift{0.589510in}{1.216673in}%
\pgfsys@useobject{currentmarker}{}%
\end{pgfscope}%
\end{pgfscope}%
\begin{pgfscope}%
\pgfsetbuttcap%
\pgfsetroundjoin%
\definecolor{currentfill}{rgb}{0.000000,0.000000,0.000000}%
\pgfsetfillcolor{currentfill}%
\pgfsetlinewidth{0.602250pt}%
\definecolor{currentstroke}{rgb}{0.000000,0.000000,0.000000}%
\pgfsetstrokecolor{currentstroke}%
\pgfsetdash{}{0pt}%
\pgfsys@defobject{currentmarker}{\pgfqpoint{-0.027778in}{0.000000in}}{\pgfqpoint{-0.000000in}{0.000000in}}{%
\pgfpathmoveto{\pgfqpoint{-0.000000in}{0.000000in}}%
\pgfpathlineto{\pgfqpoint{-0.027778in}{0.000000in}}%
\pgfusepath{stroke,fill}%
}%
\begin{pgfscope}%
\pgfsys@transformshift{0.589510in}{1.227145in}%
\pgfsys@useobject{currentmarker}{}%
\end{pgfscope}%
\end{pgfscope}%
\begin{pgfscope}%
\pgfsetbuttcap%
\pgfsetroundjoin%
\definecolor{currentfill}{rgb}{0.000000,0.000000,0.000000}%
\pgfsetfillcolor{currentfill}%
\pgfsetlinewidth{0.602250pt}%
\definecolor{currentstroke}{rgb}{0.000000,0.000000,0.000000}%
\pgfsetstrokecolor{currentstroke}%
\pgfsetdash{}{0pt}%
\pgfsys@defobject{currentmarker}{\pgfqpoint{-0.027778in}{0.000000in}}{\pgfqpoint{-0.000000in}{0.000000in}}{%
\pgfpathmoveto{\pgfqpoint{-0.000000in}{0.000000in}}%
\pgfpathlineto{\pgfqpoint{-0.027778in}{0.000000in}}%
\pgfusepath{stroke,fill}%
}%
\begin{pgfscope}%
\pgfsys@transformshift{0.589510in}{1.298138in}%
\pgfsys@useobject{currentmarker}{}%
\end{pgfscope}%
\end{pgfscope}%
\begin{pgfscope}%
\pgfsetbuttcap%
\pgfsetroundjoin%
\definecolor{currentfill}{rgb}{0.000000,0.000000,0.000000}%
\pgfsetfillcolor{currentfill}%
\pgfsetlinewidth{0.602250pt}%
\definecolor{currentstroke}{rgb}{0.000000,0.000000,0.000000}%
\pgfsetstrokecolor{currentstroke}%
\pgfsetdash{}{0pt}%
\pgfsys@defobject{currentmarker}{\pgfqpoint{-0.027778in}{0.000000in}}{\pgfqpoint{-0.000000in}{0.000000in}}{%
\pgfpathmoveto{\pgfqpoint{-0.000000in}{0.000000in}}%
\pgfpathlineto{\pgfqpoint{-0.027778in}{0.000000in}}%
\pgfusepath{stroke,fill}%
}%
\begin{pgfscope}%
\pgfsys@transformshift{0.589510in}{1.334187in}%
\pgfsys@useobject{currentmarker}{}%
\end{pgfscope}%
\end{pgfscope}%
\begin{pgfscope}%
\pgfsetbuttcap%
\pgfsetroundjoin%
\definecolor{currentfill}{rgb}{0.000000,0.000000,0.000000}%
\pgfsetfillcolor{currentfill}%
\pgfsetlinewidth{0.602250pt}%
\definecolor{currentstroke}{rgb}{0.000000,0.000000,0.000000}%
\pgfsetstrokecolor{currentstroke}%
\pgfsetdash{}{0pt}%
\pgfsys@defobject{currentmarker}{\pgfqpoint{-0.027778in}{0.000000in}}{\pgfqpoint{-0.000000in}{0.000000in}}{%
\pgfpathmoveto{\pgfqpoint{-0.000000in}{0.000000in}}%
\pgfpathlineto{\pgfqpoint{-0.027778in}{0.000000in}}%
\pgfusepath{stroke,fill}%
}%
\begin{pgfscope}%
\pgfsys@transformshift{0.589510in}{1.359764in}%
\pgfsys@useobject{currentmarker}{}%
\end{pgfscope}%
\end{pgfscope}%
\begin{pgfscope}%
\pgfsetbuttcap%
\pgfsetroundjoin%
\definecolor{currentfill}{rgb}{0.000000,0.000000,0.000000}%
\pgfsetfillcolor{currentfill}%
\pgfsetlinewidth{0.602250pt}%
\definecolor{currentstroke}{rgb}{0.000000,0.000000,0.000000}%
\pgfsetstrokecolor{currentstroke}%
\pgfsetdash{}{0pt}%
\pgfsys@defobject{currentmarker}{\pgfqpoint{-0.027778in}{0.000000in}}{\pgfqpoint{-0.000000in}{0.000000in}}{%
\pgfpathmoveto{\pgfqpoint{-0.000000in}{0.000000in}}%
\pgfpathlineto{\pgfqpoint{-0.027778in}{0.000000in}}%
\pgfusepath{stroke,fill}%
}%
\begin{pgfscope}%
\pgfsys@transformshift{0.589510in}{1.379604in}%
\pgfsys@useobject{currentmarker}{}%
\end{pgfscope}%
\end{pgfscope}%
\begin{pgfscope}%
\pgfsetbuttcap%
\pgfsetroundjoin%
\definecolor{currentfill}{rgb}{0.000000,0.000000,0.000000}%
\pgfsetfillcolor{currentfill}%
\pgfsetlinewidth{0.602250pt}%
\definecolor{currentstroke}{rgb}{0.000000,0.000000,0.000000}%
\pgfsetstrokecolor{currentstroke}%
\pgfsetdash{}{0pt}%
\pgfsys@defobject{currentmarker}{\pgfqpoint{-0.027778in}{0.000000in}}{\pgfqpoint{-0.000000in}{0.000000in}}{%
\pgfpathmoveto{\pgfqpoint{-0.000000in}{0.000000in}}%
\pgfpathlineto{\pgfqpoint{-0.027778in}{0.000000in}}%
\pgfusepath{stroke,fill}%
}%
\begin{pgfscope}%
\pgfsys@transformshift{0.589510in}{1.395813in}%
\pgfsys@useobject{currentmarker}{}%
\end{pgfscope}%
\end{pgfscope}%
\begin{pgfscope}%
\pgfsetbuttcap%
\pgfsetroundjoin%
\definecolor{currentfill}{rgb}{0.000000,0.000000,0.000000}%
\pgfsetfillcolor{currentfill}%
\pgfsetlinewidth{0.602250pt}%
\definecolor{currentstroke}{rgb}{0.000000,0.000000,0.000000}%
\pgfsetstrokecolor{currentstroke}%
\pgfsetdash{}{0pt}%
\pgfsys@defobject{currentmarker}{\pgfqpoint{-0.027778in}{0.000000in}}{\pgfqpoint{-0.000000in}{0.000000in}}{%
\pgfpathmoveto{\pgfqpoint{-0.000000in}{0.000000in}}%
\pgfpathlineto{\pgfqpoint{-0.027778in}{0.000000in}}%
\pgfusepath{stroke,fill}%
}%
\begin{pgfscope}%
\pgfsys@transformshift{0.589510in}{1.409519in}%
\pgfsys@useobject{currentmarker}{}%
\end{pgfscope}%
\end{pgfscope}%
\begin{pgfscope}%
\pgfsetbuttcap%
\pgfsetroundjoin%
\definecolor{currentfill}{rgb}{0.000000,0.000000,0.000000}%
\pgfsetfillcolor{currentfill}%
\pgfsetlinewidth{0.602250pt}%
\definecolor{currentstroke}{rgb}{0.000000,0.000000,0.000000}%
\pgfsetstrokecolor{currentstroke}%
\pgfsetdash{}{0pt}%
\pgfsys@defobject{currentmarker}{\pgfqpoint{-0.027778in}{0.000000in}}{\pgfqpoint{-0.000000in}{0.000000in}}{%
\pgfpathmoveto{\pgfqpoint{-0.000000in}{0.000000in}}%
\pgfpathlineto{\pgfqpoint{-0.027778in}{0.000000in}}%
\pgfusepath{stroke,fill}%
}%
\begin{pgfscope}%
\pgfsys@transformshift{0.589510in}{1.421391in}%
\pgfsys@useobject{currentmarker}{}%
\end{pgfscope}%
\end{pgfscope}%
\begin{pgfscope}%
\pgfsetbuttcap%
\pgfsetroundjoin%
\definecolor{currentfill}{rgb}{0.000000,0.000000,0.000000}%
\pgfsetfillcolor{currentfill}%
\pgfsetlinewidth{0.602250pt}%
\definecolor{currentstroke}{rgb}{0.000000,0.000000,0.000000}%
\pgfsetstrokecolor{currentstroke}%
\pgfsetdash{}{0pt}%
\pgfsys@defobject{currentmarker}{\pgfqpoint{-0.027778in}{0.000000in}}{\pgfqpoint{-0.000000in}{0.000000in}}{%
\pgfpathmoveto{\pgfqpoint{-0.000000in}{0.000000in}}%
\pgfpathlineto{\pgfqpoint{-0.027778in}{0.000000in}}%
\pgfusepath{stroke,fill}%
}%
\begin{pgfscope}%
\pgfsys@transformshift{0.589510in}{1.431862in}%
\pgfsys@useobject{currentmarker}{}%
\end{pgfscope}%
\end{pgfscope}%
\begin{pgfscope}%
\pgfsetbuttcap%
\pgfsetroundjoin%
\definecolor{currentfill}{rgb}{0.000000,0.000000,0.000000}%
\pgfsetfillcolor{currentfill}%
\pgfsetlinewidth{0.602250pt}%
\definecolor{currentstroke}{rgb}{0.000000,0.000000,0.000000}%
\pgfsetstrokecolor{currentstroke}%
\pgfsetdash{}{0pt}%
\pgfsys@defobject{currentmarker}{\pgfqpoint{-0.027778in}{0.000000in}}{\pgfqpoint{-0.000000in}{0.000000in}}{%
\pgfpathmoveto{\pgfqpoint{-0.000000in}{0.000000in}}%
\pgfpathlineto{\pgfqpoint{-0.027778in}{0.000000in}}%
\pgfusepath{stroke,fill}%
}%
\begin{pgfscope}%
\pgfsys@transformshift{0.589510in}{1.502856in}%
\pgfsys@useobject{currentmarker}{}%
\end{pgfscope}%
\end{pgfscope}%
\begin{pgfscope}%
\pgfsetbuttcap%
\pgfsetroundjoin%
\definecolor{currentfill}{rgb}{0.000000,0.000000,0.000000}%
\pgfsetfillcolor{currentfill}%
\pgfsetlinewidth{0.602250pt}%
\definecolor{currentstroke}{rgb}{0.000000,0.000000,0.000000}%
\pgfsetstrokecolor{currentstroke}%
\pgfsetdash{}{0pt}%
\pgfsys@defobject{currentmarker}{\pgfqpoint{-0.027778in}{0.000000in}}{\pgfqpoint{-0.000000in}{0.000000in}}{%
\pgfpathmoveto{\pgfqpoint{-0.000000in}{0.000000in}}%
\pgfpathlineto{\pgfqpoint{-0.027778in}{0.000000in}}%
\pgfusepath{stroke,fill}%
}%
\begin{pgfscope}%
\pgfsys@transformshift{0.589510in}{1.538905in}%
\pgfsys@useobject{currentmarker}{}%
\end{pgfscope}%
\end{pgfscope}%
\begin{pgfscope}%
\pgfsetbuttcap%
\pgfsetroundjoin%
\definecolor{currentfill}{rgb}{0.000000,0.000000,0.000000}%
\pgfsetfillcolor{currentfill}%
\pgfsetlinewidth{0.602250pt}%
\definecolor{currentstroke}{rgb}{0.000000,0.000000,0.000000}%
\pgfsetstrokecolor{currentstroke}%
\pgfsetdash{}{0pt}%
\pgfsys@defobject{currentmarker}{\pgfqpoint{-0.027778in}{0.000000in}}{\pgfqpoint{-0.000000in}{0.000000in}}{%
\pgfpathmoveto{\pgfqpoint{-0.000000in}{0.000000in}}%
\pgfpathlineto{\pgfqpoint{-0.027778in}{0.000000in}}%
\pgfusepath{stroke,fill}%
}%
\begin{pgfscope}%
\pgfsys@transformshift{0.589510in}{1.564482in}%
\pgfsys@useobject{currentmarker}{}%
\end{pgfscope}%
\end{pgfscope}%
\begin{pgfscope}%
\pgfsetbuttcap%
\pgfsetroundjoin%
\definecolor{currentfill}{rgb}{0.000000,0.000000,0.000000}%
\pgfsetfillcolor{currentfill}%
\pgfsetlinewidth{0.602250pt}%
\definecolor{currentstroke}{rgb}{0.000000,0.000000,0.000000}%
\pgfsetstrokecolor{currentstroke}%
\pgfsetdash{}{0pt}%
\pgfsys@defobject{currentmarker}{\pgfqpoint{-0.027778in}{0.000000in}}{\pgfqpoint{-0.000000in}{0.000000in}}{%
\pgfpathmoveto{\pgfqpoint{-0.000000in}{0.000000in}}%
\pgfpathlineto{\pgfqpoint{-0.027778in}{0.000000in}}%
\pgfusepath{stroke,fill}%
}%
\begin{pgfscope}%
\pgfsys@transformshift{0.589510in}{1.584321in}%
\pgfsys@useobject{currentmarker}{}%
\end{pgfscope}%
\end{pgfscope}%
\begin{pgfscope}%
\pgfsetbuttcap%
\pgfsetroundjoin%
\definecolor{currentfill}{rgb}{0.000000,0.000000,0.000000}%
\pgfsetfillcolor{currentfill}%
\pgfsetlinewidth{0.602250pt}%
\definecolor{currentstroke}{rgb}{0.000000,0.000000,0.000000}%
\pgfsetstrokecolor{currentstroke}%
\pgfsetdash{}{0pt}%
\pgfsys@defobject{currentmarker}{\pgfqpoint{-0.027778in}{0.000000in}}{\pgfqpoint{-0.000000in}{0.000000in}}{%
\pgfpathmoveto{\pgfqpoint{-0.000000in}{0.000000in}}%
\pgfpathlineto{\pgfqpoint{-0.027778in}{0.000000in}}%
\pgfusepath{stroke,fill}%
}%
\begin{pgfscope}%
\pgfsys@transformshift{0.589510in}{1.600531in}%
\pgfsys@useobject{currentmarker}{}%
\end{pgfscope}%
\end{pgfscope}%
\begin{pgfscope}%
\pgfsetbuttcap%
\pgfsetroundjoin%
\definecolor{currentfill}{rgb}{0.000000,0.000000,0.000000}%
\pgfsetfillcolor{currentfill}%
\pgfsetlinewidth{0.602250pt}%
\definecolor{currentstroke}{rgb}{0.000000,0.000000,0.000000}%
\pgfsetstrokecolor{currentstroke}%
\pgfsetdash{}{0pt}%
\pgfsys@defobject{currentmarker}{\pgfqpoint{-0.027778in}{0.000000in}}{\pgfqpoint{-0.000000in}{0.000000in}}{%
\pgfpathmoveto{\pgfqpoint{-0.000000in}{0.000000in}}%
\pgfpathlineto{\pgfqpoint{-0.027778in}{0.000000in}}%
\pgfusepath{stroke,fill}%
}%
\begin{pgfscope}%
\pgfsys@transformshift{0.589510in}{1.614236in}%
\pgfsys@useobject{currentmarker}{}%
\end{pgfscope}%
\end{pgfscope}%
\begin{pgfscope}%
\pgfsetbuttcap%
\pgfsetroundjoin%
\definecolor{currentfill}{rgb}{0.000000,0.000000,0.000000}%
\pgfsetfillcolor{currentfill}%
\pgfsetlinewidth{0.602250pt}%
\definecolor{currentstroke}{rgb}{0.000000,0.000000,0.000000}%
\pgfsetstrokecolor{currentstroke}%
\pgfsetdash{}{0pt}%
\pgfsys@defobject{currentmarker}{\pgfqpoint{-0.027778in}{0.000000in}}{\pgfqpoint{-0.000000in}{0.000000in}}{%
\pgfpathmoveto{\pgfqpoint{-0.000000in}{0.000000in}}%
\pgfpathlineto{\pgfqpoint{-0.027778in}{0.000000in}}%
\pgfusepath{stroke,fill}%
}%
\begin{pgfscope}%
\pgfsys@transformshift{0.589510in}{1.626108in}%
\pgfsys@useobject{currentmarker}{}%
\end{pgfscope}%
\end{pgfscope}%
\begin{pgfscope}%
\pgfsetbuttcap%
\pgfsetroundjoin%
\definecolor{currentfill}{rgb}{0.000000,0.000000,0.000000}%
\pgfsetfillcolor{currentfill}%
\pgfsetlinewidth{0.602250pt}%
\definecolor{currentstroke}{rgb}{0.000000,0.000000,0.000000}%
\pgfsetstrokecolor{currentstroke}%
\pgfsetdash{}{0pt}%
\pgfsys@defobject{currentmarker}{\pgfqpoint{-0.027778in}{0.000000in}}{\pgfqpoint{-0.000000in}{0.000000in}}{%
\pgfpathmoveto{\pgfqpoint{-0.000000in}{0.000000in}}%
\pgfpathlineto{\pgfqpoint{-0.027778in}{0.000000in}}%
\pgfusepath{stroke,fill}%
}%
\begin{pgfscope}%
\pgfsys@transformshift{0.589510in}{1.636580in}%
\pgfsys@useobject{currentmarker}{}%
\end{pgfscope}%
\end{pgfscope}%
\begin{pgfscope}%
\pgfsetbuttcap%
\pgfsetroundjoin%
\definecolor{currentfill}{rgb}{0.000000,0.000000,0.000000}%
\pgfsetfillcolor{currentfill}%
\pgfsetlinewidth{0.602250pt}%
\definecolor{currentstroke}{rgb}{0.000000,0.000000,0.000000}%
\pgfsetstrokecolor{currentstroke}%
\pgfsetdash{}{0pt}%
\pgfsys@defobject{currentmarker}{\pgfqpoint{-0.027778in}{0.000000in}}{\pgfqpoint{-0.000000in}{0.000000in}}{%
\pgfpathmoveto{\pgfqpoint{-0.000000in}{0.000000in}}%
\pgfpathlineto{\pgfqpoint{-0.027778in}{0.000000in}}%
\pgfusepath{stroke,fill}%
}%
\begin{pgfscope}%
\pgfsys@transformshift{0.589510in}{1.707573in}%
\pgfsys@useobject{currentmarker}{}%
\end{pgfscope}%
\end{pgfscope}%
\begin{pgfscope}%
\pgfsetbuttcap%
\pgfsetroundjoin%
\definecolor{currentfill}{rgb}{0.000000,0.000000,0.000000}%
\pgfsetfillcolor{currentfill}%
\pgfsetlinewidth{0.602250pt}%
\definecolor{currentstroke}{rgb}{0.000000,0.000000,0.000000}%
\pgfsetstrokecolor{currentstroke}%
\pgfsetdash{}{0pt}%
\pgfsys@defobject{currentmarker}{\pgfqpoint{-0.027778in}{0.000000in}}{\pgfqpoint{-0.000000in}{0.000000in}}{%
\pgfpathmoveto{\pgfqpoint{-0.000000in}{0.000000in}}%
\pgfpathlineto{\pgfqpoint{-0.027778in}{0.000000in}}%
\pgfusepath{stroke,fill}%
}%
\begin{pgfscope}%
\pgfsys@transformshift{0.589510in}{1.743622in}%
\pgfsys@useobject{currentmarker}{}%
\end{pgfscope}%
\end{pgfscope}%
\begin{pgfscope}%
\pgfsetbuttcap%
\pgfsetroundjoin%
\definecolor{currentfill}{rgb}{0.000000,0.000000,0.000000}%
\pgfsetfillcolor{currentfill}%
\pgfsetlinewidth{0.602250pt}%
\definecolor{currentstroke}{rgb}{0.000000,0.000000,0.000000}%
\pgfsetstrokecolor{currentstroke}%
\pgfsetdash{}{0pt}%
\pgfsys@defobject{currentmarker}{\pgfqpoint{-0.027778in}{0.000000in}}{\pgfqpoint{-0.000000in}{0.000000in}}{%
\pgfpathmoveto{\pgfqpoint{-0.000000in}{0.000000in}}%
\pgfpathlineto{\pgfqpoint{-0.027778in}{0.000000in}}%
\pgfusepath{stroke,fill}%
}%
\begin{pgfscope}%
\pgfsys@transformshift{0.589510in}{1.769200in}%
\pgfsys@useobject{currentmarker}{}%
\end{pgfscope}%
\end{pgfscope}%
\begin{pgfscope}%
\pgfsetbuttcap%
\pgfsetroundjoin%
\definecolor{currentfill}{rgb}{0.000000,0.000000,0.000000}%
\pgfsetfillcolor{currentfill}%
\pgfsetlinewidth{0.602250pt}%
\definecolor{currentstroke}{rgb}{0.000000,0.000000,0.000000}%
\pgfsetstrokecolor{currentstroke}%
\pgfsetdash{}{0pt}%
\pgfsys@defobject{currentmarker}{\pgfqpoint{-0.027778in}{0.000000in}}{\pgfqpoint{-0.000000in}{0.000000in}}{%
\pgfpathmoveto{\pgfqpoint{-0.000000in}{0.000000in}}%
\pgfpathlineto{\pgfqpoint{-0.027778in}{0.000000in}}%
\pgfusepath{stroke,fill}%
}%
\begin{pgfscope}%
\pgfsys@transformshift{0.589510in}{1.789039in}%
\pgfsys@useobject{currentmarker}{}%
\end{pgfscope}%
\end{pgfscope}%
\begin{pgfscope}%
\definecolor{textcolor}{rgb}{0.000000,0.000000,0.000000}%
\pgfsetstrokecolor{textcolor}%
\pgfsetfillcolor{textcolor}%
\pgftext[x=0.180559in,y=1.103340in,,bottom,rotate=90.000000]{\color{textcolor}{\rmfamily\fontsize{10.000000}{12.000000}\selectfont\catcode`\^=\active\def^{\ifmmode\sp\else\^{}\fi}\catcode`\%=\active\def%{\%}ADEV $\sigma_A(\tau)$}}%
\end{pgfscope}%
\begin{pgfscope}%
\pgfpathrectangle{\pgfqpoint{0.589510in}{0.417642in}}{\pgfqpoint{1.809765in}{1.371397in}}%
\pgfusepath{clip}%
\pgfsetbuttcap%
\pgfsetroundjoin%
\pgfsetlinewidth{1.505625pt}%
\definecolor{currentstroke}{rgb}{0.003922,0.450980,0.698039}%
\pgfsetstrokecolor{currentstroke}%
\pgfsetdash{{5.550000pt}{2.400000pt}}{0.000000pt}%
\pgfpathmoveto{\pgfqpoint{0.671772in}{0.827077in}}%
\pgfpathlineto{\pgfqpoint{0.809267in}{0.796264in}}%
\pgfpathlineto{\pgfqpoint{0.946763in}{0.765451in}}%
\pgfpathlineto{\pgfqpoint{1.128522in}{0.724718in}}%
\pgfpathlineto{\pgfqpoint{1.266017in}{0.693905in}}%
\pgfpathlineto{\pgfqpoint{1.403513in}{0.663092in}}%
\pgfpathlineto{\pgfqpoint{1.585272in}{0.622360in}}%
\pgfpathlineto{\pgfqpoint{1.722767in}{0.591546in}}%
\pgfpathlineto{\pgfqpoint{1.860263in}{0.560733in}}%
\pgfpathlineto{\pgfqpoint{2.042022in}{0.520001in}}%
\pgfpathlineto{\pgfqpoint{2.179517in}{0.489188in}}%
\pgfpathlineto{\pgfqpoint{2.317013in}{0.458375in}}%
\pgfusepath{stroke}%
\end{pgfscope}%
\begin{pgfscope}%
\pgfpathrectangle{\pgfqpoint{0.589510in}{0.417642in}}{\pgfqpoint{1.809765in}{1.371397in}}%
\pgfusepath{clip}%
\pgfsetbuttcap%
\pgfsetroundjoin%
\definecolor{currentfill}{rgb}{0.003922,0.450980,0.698039}%
\pgfsetfillcolor{currentfill}%
\pgfsetlinewidth{1.003750pt}%
\definecolor{currentstroke}{rgb}{0.003922,0.450980,0.698039}%
\pgfsetstrokecolor{currentstroke}%
\pgfsetdash{}{0pt}%
\pgfsys@defobject{currentmarker}{\pgfqpoint{-0.020833in}{-0.020833in}}{\pgfqpoint{0.020833in}{0.020833in}}{%
\pgfpathmoveto{\pgfqpoint{0.000000in}{-0.020833in}}%
\pgfpathcurveto{\pgfqpoint{0.005525in}{-0.020833in}}{\pgfqpoint{0.010825in}{-0.018638in}}{\pgfqpoint{0.014731in}{-0.014731in}}%
\pgfpathcurveto{\pgfqpoint{0.018638in}{-0.010825in}}{\pgfqpoint{0.020833in}{-0.005525in}}{\pgfqpoint{0.020833in}{0.000000in}}%
\pgfpathcurveto{\pgfqpoint{0.020833in}{0.005525in}}{\pgfqpoint{0.018638in}{0.010825in}}{\pgfqpoint{0.014731in}{0.014731in}}%
\pgfpathcurveto{\pgfqpoint{0.010825in}{0.018638in}}{\pgfqpoint{0.005525in}{0.020833in}}{\pgfqpoint{0.000000in}{0.020833in}}%
\pgfpathcurveto{\pgfqpoint{-0.005525in}{0.020833in}}{\pgfqpoint{-0.010825in}{0.018638in}}{\pgfqpoint{-0.014731in}{0.014731in}}%
\pgfpathcurveto{\pgfqpoint{-0.018638in}{0.010825in}}{\pgfqpoint{-0.020833in}{0.005525in}}{\pgfqpoint{-0.020833in}{0.000000in}}%
\pgfpathcurveto{\pgfqpoint{-0.020833in}{-0.005525in}}{\pgfqpoint{-0.018638in}{-0.010825in}}{\pgfqpoint{-0.014731in}{-0.014731in}}%
\pgfpathcurveto{\pgfqpoint{-0.010825in}{-0.018638in}}{\pgfqpoint{-0.005525in}{-0.020833in}}{\pgfqpoint{0.000000in}{-0.020833in}}%
\pgfpathlineto{\pgfqpoint{0.000000in}{-0.020833in}}%
\pgfpathclose%
\pgfusepath{stroke,fill}%
}%
\begin{pgfscope}%
\pgfsys@transformshift{0.671772in}{0.827704in}%
\pgfsys@useobject{currentmarker}{}%
\end{pgfscope}%
\begin{pgfscope}%
\pgfsys@transformshift{0.809267in}{0.796917in}%
\pgfsys@useobject{currentmarker}{}%
\end{pgfscope}%
\begin{pgfscope}%
\pgfsys@transformshift{0.946763in}{0.765573in}%
\pgfsys@useobject{currentmarker}{}%
\end{pgfscope}%
\begin{pgfscope}%
\pgfsys@transformshift{1.128522in}{0.722995in}%
\pgfsys@useobject{currentmarker}{}%
\end{pgfscope}%
\begin{pgfscope}%
\pgfsys@transformshift{1.266017in}{0.689209in}%
\pgfsys@useobject{currentmarker}{}%
\end{pgfscope}%
\begin{pgfscope}%
\pgfsys@transformshift{1.403513in}{0.662309in}%
\pgfsys@useobject{currentmarker}{}%
\end{pgfscope}%
\begin{pgfscope}%
\pgfsys@transformshift{1.585272in}{0.624538in}%
\pgfsys@useobject{currentmarker}{}%
\end{pgfscope}%
\begin{pgfscope}%
\pgfsys@transformshift{1.722767in}{0.589221in}%
\pgfsys@useobject{currentmarker}{}%
\end{pgfscope}%
\begin{pgfscope}%
\pgfsys@transformshift{1.860263in}{0.544348in}%
\pgfsys@useobject{currentmarker}{}%
\end{pgfscope}%
\begin{pgfscope}%
\pgfsys@transformshift{2.042022in}{0.497050in}%
\pgfsys@useobject{currentmarker}{}%
\end{pgfscope}%
\begin{pgfscope}%
\pgfsys@transformshift{2.179517in}{0.507227in}%
\pgfsys@useobject{currentmarker}{}%
\end{pgfscope}%
\begin{pgfscope}%
\pgfsys@transformshift{2.317013in}{0.455319in}%
\pgfsys@useobject{currentmarker}{}%
\end{pgfscope}%
\end{pgfscope}%
\begin{pgfscope}%
\pgfsetrectcap%
\pgfsetmiterjoin%
\pgfsetlinewidth{0.803000pt}%
\definecolor{currentstroke}{rgb}{0.000000,0.000000,0.000000}%
\pgfsetstrokecolor{currentstroke}%
\pgfsetdash{}{0pt}%
\pgfpathmoveto{\pgfqpoint{0.589510in}{0.417642in}}%
\pgfpathlineto{\pgfqpoint{0.589510in}{1.789039in}}%
\pgfusepath{stroke}%
\end{pgfscope}%
\begin{pgfscope}%
\pgfsetrectcap%
\pgfsetmiterjoin%
\pgfsetlinewidth{0.803000pt}%
\definecolor{currentstroke}{rgb}{0.000000,0.000000,0.000000}%
\pgfsetstrokecolor{currentstroke}%
\pgfsetdash{}{0pt}%
\pgfpathmoveto{\pgfqpoint{2.399275in}{0.417642in}}%
\pgfpathlineto{\pgfqpoint{2.399275in}{1.789039in}}%
\pgfusepath{stroke}%
\end{pgfscope}%
\begin{pgfscope}%
\pgfsetrectcap%
\pgfsetmiterjoin%
\pgfsetlinewidth{0.803000pt}%
\definecolor{currentstroke}{rgb}{0.000000,0.000000,0.000000}%
\pgfsetstrokecolor{currentstroke}%
\pgfsetdash{}{0pt}%
\pgfpathmoveto{\pgfqpoint{0.589510in}{0.417642in}}%
\pgfpathlineto{\pgfqpoint{2.399275in}{0.417642in}}%
\pgfusepath{stroke}%
\end{pgfscope}%
\begin{pgfscope}%
\pgfsetrectcap%
\pgfsetmiterjoin%
\pgfsetlinewidth{0.803000pt}%
\definecolor{currentstroke}{rgb}{0.000000,0.000000,0.000000}%
\pgfsetstrokecolor{currentstroke}%
\pgfsetdash{}{0pt}%
\pgfpathmoveto{\pgfqpoint{0.589510in}{1.789039in}}%
\pgfpathlineto{\pgfqpoint{2.399275in}{1.789039in}}%
\pgfusepath{stroke}%
\end{pgfscope}%
\begin{pgfscope}%
\pgfsetbuttcap%
\pgfsetmiterjoin%
\definecolor{currentfill}{rgb}{1.000000,1.000000,1.000000}%
\pgfsetfillcolor{currentfill}%
\pgfsetfillopacity{0.800000}%
\pgfsetlinewidth{1.003750pt}%
\definecolor{currentstroke}{rgb}{0.800000,0.800000,0.800000}%
\pgfsetstrokecolor{currentstroke}%
\pgfsetstrokeopacity{0.800000}%
\pgfsetdash{}{0pt}%
\pgfpathmoveto{\pgfqpoint{1.290639in}{1.472371in}}%
\pgfpathlineto{\pgfqpoint{2.321497in}{1.472371in}}%
\pgfpathquadraticcurveto{\pgfqpoint{2.343719in}{1.472371in}}{\pgfqpoint{2.343719in}{1.494593in}}%
\pgfpathlineto{\pgfqpoint{2.343719in}{1.711261in}}%
\pgfpathquadraticcurveto{\pgfqpoint{2.343719in}{1.733483in}}{\pgfqpoint{2.321497in}{1.733483in}}%
\pgfpathlineto{\pgfqpoint{1.290639in}{1.733483in}}%
\pgfpathquadraticcurveto{\pgfqpoint{1.268417in}{1.733483in}}{\pgfqpoint{1.268417in}{1.711261in}}%
\pgfpathlineto{\pgfqpoint{1.268417in}{1.494593in}}%
\pgfpathquadraticcurveto{\pgfqpoint{1.268417in}{1.472371in}}{\pgfqpoint{1.290639in}{1.472371in}}%
\pgfpathlineto{\pgfqpoint{1.290639in}{1.472371in}}%
\pgfpathclose%
\pgfusepath{stroke,fill}%
\end{pgfscope}%
\begin{pgfscope}%
\pgfsetbuttcap%
\pgfsetroundjoin%
\pgfsetlinewidth{1.505625pt}%
\definecolor{currentstroke}{rgb}{0.003922,0.450980,0.698039}%
\pgfsetstrokecolor{currentstroke}%
\pgfsetdash{{5.550000pt}{2.400000pt}}{0.000000pt}%
\pgfpathmoveto{\pgfqpoint{1.312861in}{1.596639in}}%
\pgfpathlineto{\pgfqpoint{1.423972in}{1.596639in}}%
\pgfpathlineto{\pgfqpoint{1.535084in}{1.596639in}}%
\pgfusepath{stroke}%
\end{pgfscope}%
\begin{pgfscope}%
\definecolor{textcolor}{rgb}{0.000000,0.000000,0.000000}%
\pgfsetstrokecolor{textcolor}%
\pgfsetfillcolor{textcolor}%
\pgftext[x=1.623972in,y=1.557750in,left,base]{\color{textcolor}{\rmfamily\fontsize{8.000000}{9.600000}\selectfont\catcode`\^=\active\def^{\ifmmode\sp\else\^{}\fi}\catcode`\%=\active\def%{\%}$\displaystyle \propto\sqrt{h_{0}}\tau^{-0.5}$}}%
\end{pgfscope}%
\end{pgfpicture}%
\makeatother%
\endgroup%
% data/simulations/sim_allan_variance.py
        } % scalebox
        \caption{Allan deviation}
        \label{fig:white_noise_adev}
    \end{subfigure}
    \caption{Different representations of white noise.}
    \label{fig:white_noise_simulated}
\end{figure}

Figure \ref{fig:white_noise_simulated} shows a sample of white noise in its three different forms. Figure \ref{fig:white_noise_time} is the time series representation from which the power spectral density was calculated and is shown in figure \ref{fig:white_noise_psd}. The dashed line shows the expectation value of the power spectral density and the Allan deviation.

From this simulation, several features can be observed. First of all, the power spectral density is flat and constant with $h_0 = 2$, which is in accordance with table \ref{tab:adev_alpha} and the normalisation mentioned earlier. Figure \ref{fig:white_noise_adev} shows the typical $\tau^{-\frac 1 2}$ dependence of white noise in the Allan deviation plot. This immediately explains, why filtering white noise scales with $\frac{1}{\sqrt{n}}$ with $n$ being the number of samples averaged.

\subsubsection{Burst Noise}%
\label{sec:theory_burst_noise}
Burst noise, popcorn noise, or sometimes referred to as random telegraph signal is a random bi-stable change in a signal and is caused by generation-recombination processes. This happens, for example, in semiconductors if there is a site that can trap an electron for a prolonged period of time and then randomly release it. Impurities causing lattice defects are discussed in this context \cite{kay2012operational,burst_noise_psd,popcorn_noise_orgin,technote_ti_popcorn_noise}. Such lattice defects can also be introduced by ion implantation during doping. Fortunately, this type of noise has become less prevalent in modern manufacturing processes, because the quality of the semiconductors has improved. But if a trap site is located very close to an important structure, for example a high precision Zener diode, its effect might be so strong that it can be clearly seen.

The discussion is split into two parts. First the power spectral density is calculated and then the Allan variance is calculated using that result.

The spectral density of burst noise caused by a single trap site was derived in \cite{burst_noise_wiener_khinchin} by \citeauthor{burst_noise_wiener_khinchin}. \citeauthor{burst_noise_wiener_khinchin} used the autocorrelation function of the burst noise signal and applied the Wiener-Khinchin (Wiener-Хи́нчин) theorem, which connects the autocorrelation function with the power spectral density. A more detailed derivation can be found in \cite{fundamentals_of_noise_processes}, in this paper the preconditions like stationarity of the process, are also discussed. The burst noise signal consists of two energy levels, called $0$ and $1$, split by $\Delta y$. Multiple burst noise signals can be superimposed in a real device. This would then result in multiple levels, but they can be treated separately. The measurement interval over an even number of transitions, so that one ends in the same state as the measurement has started, is the time $T$. The mean lifetime of the levels is called $\bar \tau_0$ and $\bar \tau_1$:
\begin{equation}
    \bar \tau_{0} \approx \frac 1 N \sum_{i}^N \tau_{0,i} \qquad \bar \tau_{1} \approx \frac 1 N \sum_{i}^N \tau_{1,i}
\end{equation}

Figure \ref{fig:burst_noise} shows a burst noise signal along with the definitions above.
\begin{figure}[hb]
    \centering
    %\scalebox{1} % scalebox
    \caption{A random burst noise signal.}
    \label{fig:burst_noise}
\end{figure}

Using these definitions, one can then derive \cite{burst_noise_wiener_khinchin}:
\begin{align}
    R_{xx}(T) &= \Delta y^2 \cdot \frac{\bar \tau_1 \bar \tau_0 e^{-\left(\frac{1}{\bar \tau_1}+\frac{1}{\bar \tau_0}\right)T}}{\left(\bar \tau_1 + \bar \tau_0\right)^2} \quad \text{and} \label{eqn:burst_noise_correlation}\\
    S(\omega) &= 4 R_{xx}(0) \frac{\frac{1}{\bar \tau_1} + \frac{1}{\bar \tau_0}}{\left(\frac{1}{\bar \tau_1} + \frac{1}{\bar \tau_0}\right)^2 + \omega^2} \qquad \omega > 0 . \label{eqn:burst_noise_psd}
\end{align}
Note, that the power spectral density is the one-sided version, hence an additional factor of $2$ is included. The constant term was omitted here and can usually be neglected, because it is not relevant for calculating the power spectral density as it only contributes a single peak at $\omega=0$. Using the following definitions of the average time constant and the duty cycle
\begin{align}
    \frac{1}{\bar \tau} &= \frac{1}{\bar \tau_1} + \frac{1}{\bar \tau_0} \quad \mathrm{and} \label{eqn:definition_bar_tau}\\
    D_i &= \frac{\bar \tau_i}{\bar \tau_1 + \bar \tau_0} \quad i \in \{0 ; 1\}
\end{align}

equations \ref{eqn:burst_noise_correlation} and \ref{eqn:burst_noise_psd} can be rewritten to give a more intuitive form.
\begin{align}
    R_{xx}(T) &= \Delta y^2 D_1 D_0 \, e^{-\left(\frac{1}{\bar \tau_1}+\frac{1}{\bar \tau_0}\right)T}\\
    S(\omega) &= 4 R_{xx}(0) \frac{\bar \tau}{1 + \omega^2 \bar \tau^2} \label{eqn:burst_noise_lorentzian}
\end{align}

The special case $\bar \tau_0 = \bar \tau_1$ with $D_i=\frac 1 2$ is the previously mentioned case of random telegraph noise.

$R_{xx}(0)$ can be identified as the mean squared value of $y$:
\begin{equation}
    y_{rms} = \sqrt{R_{xx}(0)} \,.
\end{equation}

Equation \ref{eqn:burst_noise_lorentzian} is a Lorentzian function and from this, it can be easily seen that a single trap site has a power spectral density that is proportional to $\frac{1}{f^2}$ at high frequencies and is flat at low frequencies.

With the spectral density in hand, it is now possible to calculate the Allan variance as it was done by \citeauthor{allen_dev_flicker} in \cite{allen_dev_flicker} for the classic example of random telegraph noise where $\bar \tau_1 = \bar \tau_0$. Do note that table I given by \citeauthor{allen_dev_flicker} shows the total number of events instead of the instantaneous number of events typically given. Hence their notation must be multiplied by $\frac{1}{\tau^2}$ (or $\frac{1}{T^2}$ in their notation). For the generic case with $\bar \tau_1$, $\bar \tau_0$ and the definition of $\bar \tau$ given in equation \ref{eqn:definition_bar_tau} one finds for the Allan variance of burst noise:
\begin{equation}
    \sigma^2_A(\tau) = R_{xx}(0) \frac{\bar \tau^2}{\tau^2} \left(4 e^{-\frac{\tau}{\bar \tau}} - e^{-\frac{2 \tau}{\bar \tau}} + 2 \frac{\tau}{\bar \tau} - 3 \right) \label{eqn:burst_noise_avar}
\end{equation}

Having arrived at equations \ref{eqn:burst_noise_lorentzian} and \ref{eqn:burst_noise_avar} of the power spectral density and Allan variance, it it now possible to model it. For this purpose, parts of the Python library \textit{qtt} \cite{qtt} was used. This algorithm written by \citeauthor{qtt} implements continuous-time Markov chains to simulate the burst noise signal. The result can be see in figure \ref{fig:burst_noise_simulated}. For these simulations one of the time constants, namely the lifetime of the lower state $\bar \tau_0$ was held constant, while the lifetime of the upper state was varied to show the effect of different $\bar \tau$. By looking at the time domain in figure \ref{fig:burst_noise_time} it can be seen that the maximum average number of state changes can be observed, when $\bar \tau_1 = \bar \tau_0$. If $\bar \tau_1 > \bar \tau_0$ the system will favour the upper, while if $\bar \tau_1 < \bar \tau_0$ it will favour the lower state instead. This explains why the noise is strongest for random telegraph noise when $\bar \tau_1 = \bar \tau_0$, which can also be seen in the power spectral density plot in figure \ref{fig:burst_noise_psd}. Looking at the Allan deviation in figure \ref{fig:burst_noise_adev} confirms this, but also shows another interesting implication as it shows an obvious maximum. If the application allows a choice over the sampling interval $\tau$, the effect of the burst noise can be mitigated by staying well clear of the maximum.

The small deviation from the analytical solution in figure \ref{fig:burst_noise_adev} suggesting an upwards trend at large $\tau$ is a typical so-called end-of-data error. As it was discussed above, the Allan deviation can only be estimated given a limited number of samples using equation \ref{eqn:adev_estimator} and going to longer $\tau$ means there are fewer samples to average over.
\begin{figure}[ht]
    \centering
    \begin{subfigure}{0.8\linewidth}
        \centering
        \scalebox{1}{%
            %% Creator: Matplotlib, PGF backend
%%
%% To include the figure in your LaTeX document, write
%%   \input{<filename>.pgf}
%%
%% Make sure the required packages are loaded in your preamble
%%   \usepackage{pgf}
%%
%% Also ensure that all the required font packages are loaded; for instance,
%% the lmodern package is sometimes necessary when using math font.
%%   \usepackage{lmodern}
%%
%% Figures using additional raster images can only be included by \input if
%% they are in the same directory as the main LaTeX file. For loading figures
%% from other directories you can use the `import` package
%%   \usepackage{import}
%%
%% and then include the figures with
%%   \import{<path to file>}{<filename>.pgf}
%%
%% Matplotlib used the following preamble
%%   \usepackage{siunitx}
%%   \usepackage{fontspec}
%%   \makeatletter\@ifpackageloaded{underscore}{}{\usepackage[strings]{underscore}}\makeatother
%%
\begingroup%
\makeatletter%
\begin{pgfpicture}%
\pgfpathrectangle{\pgfpointorigin}{\pgfqpoint{4.068242in}{2.514312in}}%
\pgfusepath{use as bounding box, clip}%
\begin{pgfscope}%
\pgfsetbuttcap%
\pgfsetmiterjoin%
\definecolor{currentfill}{rgb}{1.000000,1.000000,1.000000}%
\pgfsetfillcolor{currentfill}%
\pgfsetlinewidth{0.000000pt}%
\definecolor{currentstroke}{rgb}{1.000000,1.000000,1.000000}%
\pgfsetstrokecolor{currentstroke}%
\pgfsetdash{}{0pt}%
\pgfpathmoveto{\pgfqpoint{0.000000in}{0.000000in}}%
\pgfpathlineto{\pgfqpoint{4.068242in}{0.000000in}}%
\pgfpathlineto{\pgfqpoint{4.068242in}{2.514312in}}%
\pgfpathlineto{\pgfqpoint{0.000000in}{2.514312in}}%
\pgfpathlineto{\pgfqpoint{0.000000in}{0.000000in}}%
\pgfpathclose%
\pgfusepath{fill}%
\end{pgfscope}%
\begin{pgfscope}%
\pgfsetbuttcap%
\pgfsetmiterjoin%
\definecolor{currentfill}{rgb}{1.000000,1.000000,1.000000}%
\pgfsetfillcolor{currentfill}%
\pgfsetlinewidth{0.000000pt}%
\definecolor{currentstroke}{rgb}{0.000000,0.000000,0.000000}%
\pgfsetstrokecolor{currentstroke}%
\pgfsetstrokeopacity{0.000000}%
\pgfsetdash{}{0pt}%
\pgfpathmoveto{\pgfqpoint{0.471687in}{0.416447in}}%
\pgfpathlineto{\pgfqpoint{4.026572in}{0.416447in}}%
\pgfpathlineto{\pgfqpoint{4.026572in}{2.472642in}}%
\pgfpathlineto{\pgfqpoint{0.471687in}{2.472642in}}%
\pgfpathlineto{\pgfqpoint{0.471687in}{0.416447in}}%
\pgfpathclose%
\pgfusepath{fill}%
\end{pgfscope}%
\begin{pgfscope}%
\pgfpathrectangle{\pgfqpoint{0.471687in}{0.416447in}}{\pgfqpoint{3.554884in}{2.056194in}}%
\pgfusepath{clip}%
\pgfsetrectcap%
\pgfsetroundjoin%
\pgfsetlinewidth{0.803000pt}%
\definecolor{currentstroke}{rgb}{0.450000,0.450000,0.450000}%
\pgfsetstrokecolor{currentstroke}%
\pgfsetdash{}{0pt}%
\pgfpathmoveto{\pgfqpoint{0.633273in}{0.416447in}}%
\pgfpathlineto{\pgfqpoint{0.633273in}{2.472642in}}%
\pgfusepath{stroke}%
\end{pgfscope}%
\begin{pgfscope}%
\pgfsetbuttcap%
\pgfsetroundjoin%
\definecolor{currentfill}{rgb}{0.000000,0.000000,0.000000}%
\pgfsetfillcolor{currentfill}%
\pgfsetlinewidth{0.803000pt}%
\definecolor{currentstroke}{rgb}{0.000000,0.000000,0.000000}%
\pgfsetstrokecolor{currentstroke}%
\pgfsetdash{}{0pt}%
\pgfsys@defobject{currentmarker}{\pgfqpoint{0.000000in}{-0.048611in}}{\pgfqpoint{0.000000in}{0.000000in}}{%
\pgfpathmoveto{\pgfqpoint{0.000000in}{0.000000in}}%
\pgfpathlineto{\pgfqpoint{0.000000in}{-0.048611in}}%
\pgfusepath{stroke,fill}%
}%
\begin{pgfscope}%
\pgfsys@transformshift{0.633273in}{0.416447in}%
\pgfsys@useobject{currentmarker}{}%
\end{pgfscope}%
\end{pgfscope}%
\begin{pgfscope}%
\definecolor{textcolor}{rgb}{0.000000,0.000000,0.000000}%
\pgfsetstrokecolor{textcolor}%
\pgfsetfillcolor{textcolor}%
\pgftext[x=0.633273in,y=0.319225in,,top]{\color{textcolor}\rmfamily\fontsize{8.000000}{9.600000}\selectfont \(\displaystyle {0}\)}%
\end{pgfscope}%
\begin{pgfscope}%
\pgfpathrectangle{\pgfqpoint{0.471687in}{0.416447in}}{\pgfqpoint{3.554884in}{2.056194in}}%
\pgfusepath{clip}%
\pgfsetrectcap%
\pgfsetroundjoin%
\pgfsetlinewidth{0.803000pt}%
\definecolor{currentstroke}{rgb}{0.450000,0.450000,0.450000}%
\pgfsetstrokecolor{currentstroke}%
\pgfsetdash{}{0pt}%
\pgfpathmoveto{\pgfqpoint{1.279939in}{0.416447in}}%
\pgfpathlineto{\pgfqpoint{1.279939in}{2.472642in}}%
\pgfusepath{stroke}%
\end{pgfscope}%
\begin{pgfscope}%
\pgfsetbuttcap%
\pgfsetroundjoin%
\definecolor{currentfill}{rgb}{0.000000,0.000000,0.000000}%
\pgfsetfillcolor{currentfill}%
\pgfsetlinewidth{0.803000pt}%
\definecolor{currentstroke}{rgb}{0.000000,0.000000,0.000000}%
\pgfsetstrokecolor{currentstroke}%
\pgfsetdash{}{0pt}%
\pgfsys@defobject{currentmarker}{\pgfqpoint{0.000000in}{-0.048611in}}{\pgfqpoint{0.000000in}{0.000000in}}{%
\pgfpathmoveto{\pgfqpoint{0.000000in}{0.000000in}}%
\pgfpathlineto{\pgfqpoint{0.000000in}{-0.048611in}}%
\pgfusepath{stroke,fill}%
}%
\begin{pgfscope}%
\pgfsys@transformshift{1.279939in}{0.416447in}%
\pgfsys@useobject{currentmarker}{}%
\end{pgfscope}%
\end{pgfscope}%
\begin{pgfscope}%
\definecolor{textcolor}{rgb}{0.000000,0.000000,0.000000}%
\pgfsetstrokecolor{textcolor}%
\pgfsetfillcolor{textcolor}%
\pgftext[x=1.279939in,y=0.319225in,,top]{\color{textcolor}\rmfamily\fontsize{8.000000}{9.600000}\selectfont \(\displaystyle {2}\)}%
\end{pgfscope}%
\begin{pgfscope}%
\pgfpathrectangle{\pgfqpoint{0.471687in}{0.416447in}}{\pgfqpoint{3.554884in}{2.056194in}}%
\pgfusepath{clip}%
\pgfsetrectcap%
\pgfsetroundjoin%
\pgfsetlinewidth{0.803000pt}%
\definecolor{currentstroke}{rgb}{0.450000,0.450000,0.450000}%
\pgfsetstrokecolor{currentstroke}%
\pgfsetdash{}{0pt}%
\pgfpathmoveto{\pgfqpoint{1.926605in}{0.416447in}}%
\pgfpathlineto{\pgfqpoint{1.926605in}{2.472642in}}%
\pgfusepath{stroke}%
\end{pgfscope}%
\begin{pgfscope}%
\pgfsetbuttcap%
\pgfsetroundjoin%
\definecolor{currentfill}{rgb}{0.000000,0.000000,0.000000}%
\pgfsetfillcolor{currentfill}%
\pgfsetlinewidth{0.803000pt}%
\definecolor{currentstroke}{rgb}{0.000000,0.000000,0.000000}%
\pgfsetstrokecolor{currentstroke}%
\pgfsetdash{}{0pt}%
\pgfsys@defobject{currentmarker}{\pgfqpoint{0.000000in}{-0.048611in}}{\pgfqpoint{0.000000in}{0.000000in}}{%
\pgfpathmoveto{\pgfqpoint{0.000000in}{0.000000in}}%
\pgfpathlineto{\pgfqpoint{0.000000in}{-0.048611in}}%
\pgfusepath{stroke,fill}%
}%
\begin{pgfscope}%
\pgfsys@transformshift{1.926605in}{0.416447in}%
\pgfsys@useobject{currentmarker}{}%
\end{pgfscope}%
\end{pgfscope}%
\begin{pgfscope}%
\definecolor{textcolor}{rgb}{0.000000,0.000000,0.000000}%
\pgfsetstrokecolor{textcolor}%
\pgfsetfillcolor{textcolor}%
\pgftext[x=1.926605in,y=0.319225in,,top]{\color{textcolor}\rmfamily\fontsize{8.000000}{9.600000}\selectfont \(\displaystyle {4}\)}%
\end{pgfscope}%
\begin{pgfscope}%
\pgfpathrectangle{\pgfqpoint{0.471687in}{0.416447in}}{\pgfqpoint{3.554884in}{2.056194in}}%
\pgfusepath{clip}%
\pgfsetrectcap%
\pgfsetroundjoin%
\pgfsetlinewidth{0.803000pt}%
\definecolor{currentstroke}{rgb}{0.450000,0.450000,0.450000}%
\pgfsetstrokecolor{currentstroke}%
\pgfsetdash{}{0pt}%
\pgfpathmoveto{\pgfqpoint{2.573271in}{0.416447in}}%
\pgfpathlineto{\pgfqpoint{2.573271in}{2.472642in}}%
\pgfusepath{stroke}%
\end{pgfscope}%
\begin{pgfscope}%
\pgfsetbuttcap%
\pgfsetroundjoin%
\definecolor{currentfill}{rgb}{0.000000,0.000000,0.000000}%
\pgfsetfillcolor{currentfill}%
\pgfsetlinewidth{0.803000pt}%
\definecolor{currentstroke}{rgb}{0.000000,0.000000,0.000000}%
\pgfsetstrokecolor{currentstroke}%
\pgfsetdash{}{0pt}%
\pgfsys@defobject{currentmarker}{\pgfqpoint{0.000000in}{-0.048611in}}{\pgfqpoint{0.000000in}{0.000000in}}{%
\pgfpathmoveto{\pgfqpoint{0.000000in}{0.000000in}}%
\pgfpathlineto{\pgfqpoint{0.000000in}{-0.048611in}}%
\pgfusepath{stroke,fill}%
}%
\begin{pgfscope}%
\pgfsys@transformshift{2.573271in}{0.416447in}%
\pgfsys@useobject{currentmarker}{}%
\end{pgfscope}%
\end{pgfscope}%
\begin{pgfscope}%
\definecolor{textcolor}{rgb}{0.000000,0.000000,0.000000}%
\pgfsetstrokecolor{textcolor}%
\pgfsetfillcolor{textcolor}%
\pgftext[x=2.573271in,y=0.319225in,,top]{\color{textcolor}\rmfamily\fontsize{8.000000}{9.600000}\selectfont \(\displaystyle {6}\)}%
\end{pgfscope}%
\begin{pgfscope}%
\pgfpathrectangle{\pgfqpoint{0.471687in}{0.416447in}}{\pgfqpoint{3.554884in}{2.056194in}}%
\pgfusepath{clip}%
\pgfsetrectcap%
\pgfsetroundjoin%
\pgfsetlinewidth{0.803000pt}%
\definecolor{currentstroke}{rgb}{0.450000,0.450000,0.450000}%
\pgfsetstrokecolor{currentstroke}%
\pgfsetdash{}{0pt}%
\pgfpathmoveto{\pgfqpoint{3.219937in}{0.416447in}}%
\pgfpathlineto{\pgfqpoint{3.219937in}{2.472642in}}%
\pgfusepath{stroke}%
\end{pgfscope}%
\begin{pgfscope}%
\pgfsetbuttcap%
\pgfsetroundjoin%
\definecolor{currentfill}{rgb}{0.000000,0.000000,0.000000}%
\pgfsetfillcolor{currentfill}%
\pgfsetlinewidth{0.803000pt}%
\definecolor{currentstroke}{rgb}{0.000000,0.000000,0.000000}%
\pgfsetstrokecolor{currentstroke}%
\pgfsetdash{}{0pt}%
\pgfsys@defobject{currentmarker}{\pgfqpoint{0.000000in}{-0.048611in}}{\pgfqpoint{0.000000in}{0.000000in}}{%
\pgfpathmoveto{\pgfqpoint{0.000000in}{0.000000in}}%
\pgfpathlineto{\pgfqpoint{0.000000in}{-0.048611in}}%
\pgfusepath{stroke,fill}%
}%
\begin{pgfscope}%
\pgfsys@transformshift{3.219937in}{0.416447in}%
\pgfsys@useobject{currentmarker}{}%
\end{pgfscope}%
\end{pgfscope}%
\begin{pgfscope}%
\definecolor{textcolor}{rgb}{0.000000,0.000000,0.000000}%
\pgfsetstrokecolor{textcolor}%
\pgfsetfillcolor{textcolor}%
\pgftext[x=3.219937in,y=0.319225in,,top]{\color{textcolor}\rmfamily\fontsize{8.000000}{9.600000}\selectfont \(\displaystyle {8}\)}%
\end{pgfscope}%
\begin{pgfscope}%
\pgfpathrectangle{\pgfqpoint{0.471687in}{0.416447in}}{\pgfqpoint{3.554884in}{2.056194in}}%
\pgfusepath{clip}%
\pgfsetrectcap%
\pgfsetroundjoin%
\pgfsetlinewidth{0.803000pt}%
\definecolor{currentstroke}{rgb}{0.450000,0.450000,0.450000}%
\pgfsetstrokecolor{currentstroke}%
\pgfsetdash{}{0pt}%
\pgfpathmoveto{\pgfqpoint{3.866603in}{0.416447in}}%
\pgfpathlineto{\pgfqpoint{3.866603in}{2.472642in}}%
\pgfusepath{stroke}%
\end{pgfscope}%
\begin{pgfscope}%
\pgfsetbuttcap%
\pgfsetroundjoin%
\definecolor{currentfill}{rgb}{0.000000,0.000000,0.000000}%
\pgfsetfillcolor{currentfill}%
\pgfsetlinewidth{0.803000pt}%
\definecolor{currentstroke}{rgb}{0.000000,0.000000,0.000000}%
\pgfsetstrokecolor{currentstroke}%
\pgfsetdash{}{0pt}%
\pgfsys@defobject{currentmarker}{\pgfqpoint{0.000000in}{-0.048611in}}{\pgfqpoint{0.000000in}{0.000000in}}{%
\pgfpathmoveto{\pgfqpoint{0.000000in}{0.000000in}}%
\pgfpathlineto{\pgfqpoint{0.000000in}{-0.048611in}}%
\pgfusepath{stroke,fill}%
}%
\begin{pgfscope}%
\pgfsys@transformshift{3.866603in}{0.416447in}%
\pgfsys@useobject{currentmarker}{}%
\end{pgfscope}%
\end{pgfscope}%
\begin{pgfscope}%
\definecolor{textcolor}{rgb}{0.000000,0.000000,0.000000}%
\pgfsetstrokecolor{textcolor}%
\pgfsetfillcolor{textcolor}%
\pgftext[x=3.866603in,y=0.319225in,,top]{\color{textcolor}\rmfamily\fontsize{8.000000}{9.600000}\selectfont \(\displaystyle {10}\)}%
\end{pgfscope}%
\begin{pgfscope}%
\definecolor{textcolor}{rgb}{0.000000,0.000000,0.000000}%
\pgfsetstrokecolor{textcolor}%
\pgfsetfillcolor{textcolor}%
\pgftext[x=2.249130in,y=0.165003in,,top]{\color{textcolor}\rmfamily\fontsize{10.000000}{12.000000}\selectfont Time in \unit{\second}}%
\end{pgfscope}%
\begin{pgfscope}%
\pgfpathrectangle{\pgfqpoint{0.471687in}{0.416447in}}{\pgfqpoint{3.554884in}{2.056194in}}%
\pgfusepath{clip}%
\pgfsetrectcap%
\pgfsetroundjoin%
\pgfsetlinewidth{0.803000pt}%
\definecolor{currentstroke}{rgb}{0.450000,0.450000,0.450000}%
\pgfsetstrokecolor{currentstroke}%
\pgfsetdash{}{0pt}%
\pgfpathmoveto{\pgfqpoint{0.471687in}{0.509911in}}%
\pgfpathlineto{\pgfqpoint{4.026572in}{0.509911in}}%
\pgfusepath{stroke}%
\end{pgfscope}%
\begin{pgfscope}%
\pgfsetbuttcap%
\pgfsetroundjoin%
\definecolor{currentfill}{rgb}{0.000000,0.000000,0.000000}%
\pgfsetfillcolor{currentfill}%
\pgfsetlinewidth{0.803000pt}%
\definecolor{currentstroke}{rgb}{0.000000,0.000000,0.000000}%
\pgfsetstrokecolor{currentstroke}%
\pgfsetdash{}{0pt}%
\pgfsys@defobject{currentmarker}{\pgfqpoint{-0.048611in}{0.000000in}}{\pgfqpoint{-0.000000in}{0.000000in}}{%
\pgfpathmoveto{\pgfqpoint{-0.000000in}{0.000000in}}%
\pgfpathlineto{\pgfqpoint{-0.048611in}{0.000000in}}%
\pgfusepath{stroke,fill}%
}%
\begin{pgfscope}%
\pgfsys@transformshift{0.471687in}{0.509911in}%
\pgfsys@useobject{currentmarker}{}%
\end{pgfscope}%
\end{pgfscope}%
\begin{pgfscope}%
\definecolor{textcolor}{rgb}{0.000000,0.000000,0.000000}%
\pgfsetstrokecolor{textcolor}%
\pgfsetfillcolor{textcolor}%
\pgftext[x=0.223614in, y=0.471355in, left, base]{\color{textcolor}\rmfamily\fontsize{8.000000}{9.600000}\selectfont \(\displaystyle {0.0}\)}%
\end{pgfscope}%
\begin{pgfscope}%
\pgfpathrectangle{\pgfqpoint{0.471687in}{0.416447in}}{\pgfqpoint{3.554884in}{2.056194in}}%
\pgfusepath{clip}%
\pgfsetrectcap%
\pgfsetroundjoin%
\pgfsetlinewidth{0.803000pt}%
\definecolor{currentstroke}{rgb}{0.450000,0.450000,0.450000}%
\pgfsetstrokecolor{currentstroke}%
\pgfsetdash{}{0pt}%
\pgfpathmoveto{\pgfqpoint{0.471687in}{0.821455in}}%
\pgfpathlineto{\pgfqpoint{4.026572in}{0.821455in}}%
\pgfusepath{stroke}%
\end{pgfscope}%
\begin{pgfscope}%
\pgfsetbuttcap%
\pgfsetroundjoin%
\definecolor{currentfill}{rgb}{0.000000,0.000000,0.000000}%
\pgfsetfillcolor{currentfill}%
\pgfsetlinewidth{0.803000pt}%
\definecolor{currentstroke}{rgb}{0.000000,0.000000,0.000000}%
\pgfsetstrokecolor{currentstroke}%
\pgfsetdash{}{0pt}%
\pgfsys@defobject{currentmarker}{\pgfqpoint{-0.048611in}{0.000000in}}{\pgfqpoint{-0.000000in}{0.000000in}}{%
\pgfpathmoveto{\pgfqpoint{-0.000000in}{0.000000in}}%
\pgfpathlineto{\pgfqpoint{-0.048611in}{0.000000in}}%
\pgfusepath{stroke,fill}%
}%
\begin{pgfscope}%
\pgfsys@transformshift{0.471687in}{0.821455in}%
\pgfsys@useobject{currentmarker}{}%
\end{pgfscope}%
\end{pgfscope}%
\begin{pgfscope}%
\definecolor{textcolor}{rgb}{0.000000,0.000000,0.000000}%
\pgfsetstrokecolor{textcolor}%
\pgfsetfillcolor{textcolor}%
\pgftext[x=0.223614in, y=0.782900in, left, base]{\color{textcolor}\rmfamily\fontsize{8.000000}{9.600000}\selectfont \(\displaystyle {0.5}\)}%
\end{pgfscope}%
\begin{pgfscope}%
\pgfpathrectangle{\pgfqpoint{0.471687in}{0.416447in}}{\pgfqpoint{3.554884in}{2.056194in}}%
\pgfusepath{clip}%
\pgfsetrectcap%
\pgfsetroundjoin%
\pgfsetlinewidth{0.803000pt}%
\definecolor{currentstroke}{rgb}{0.450000,0.450000,0.450000}%
\pgfsetstrokecolor{currentstroke}%
\pgfsetdash{}{0pt}%
\pgfpathmoveto{\pgfqpoint{0.471687in}{1.133000in}}%
\pgfpathlineto{\pgfqpoint{4.026572in}{1.133000in}}%
\pgfusepath{stroke}%
\end{pgfscope}%
\begin{pgfscope}%
\pgfsetbuttcap%
\pgfsetroundjoin%
\definecolor{currentfill}{rgb}{0.000000,0.000000,0.000000}%
\pgfsetfillcolor{currentfill}%
\pgfsetlinewidth{0.803000pt}%
\definecolor{currentstroke}{rgb}{0.000000,0.000000,0.000000}%
\pgfsetstrokecolor{currentstroke}%
\pgfsetdash{}{0pt}%
\pgfsys@defobject{currentmarker}{\pgfqpoint{-0.048611in}{0.000000in}}{\pgfqpoint{-0.000000in}{0.000000in}}{%
\pgfpathmoveto{\pgfqpoint{-0.000000in}{0.000000in}}%
\pgfpathlineto{\pgfqpoint{-0.048611in}{0.000000in}}%
\pgfusepath{stroke,fill}%
}%
\begin{pgfscope}%
\pgfsys@transformshift{0.471687in}{1.133000in}%
\pgfsys@useobject{currentmarker}{}%
\end{pgfscope}%
\end{pgfscope}%
\begin{pgfscope}%
\definecolor{textcolor}{rgb}{0.000000,0.000000,0.000000}%
\pgfsetstrokecolor{textcolor}%
\pgfsetfillcolor{textcolor}%
\pgftext[x=0.223614in, y=1.094444in, left, base]{\color{textcolor}\rmfamily\fontsize{8.000000}{9.600000}\selectfont \(\displaystyle {1.0}\)}%
\end{pgfscope}%
\begin{pgfscope}%
\pgfpathrectangle{\pgfqpoint{0.471687in}{0.416447in}}{\pgfqpoint{3.554884in}{2.056194in}}%
\pgfusepath{clip}%
\pgfsetrectcap%
\pgfsetroundjoin%
\pgfsetlinewidth{0.803000pt}%
\definecolor{currentstroke}{rgb}{0.450000,0.450000,0.450000}%
\pgfsetstrokecolor{currentstroke}%
\pgfsetdash{}{0pt}%
\pgfpathmoveto{\pgfqpoint{0.471687in}{1.444545in}}%
\pgfpathlineto{\pgfqpoint{4.026572in}{1.444545in}}%
\pgfusepath{stroke}%
\end{pgfscope}%
\begin{pgfscope}%
\pgfsetbuttcap%
\pgfsetroundjoin%
\definecolor{currentfill}{rgb}{0.000000,0.000000,0.000000}%
\pgfsetfillcolor{currentfill}%
\pgfsetlinewidth{0.803000pt}%
\definecolor{currentstroke}{rgb}{0.000000,0.000000,0.000000}%
\pgfsetstrokecolor{currentstroke}%
\pgfsetdash{}{0pt}%
\pgfsys@defobject{currentmarker}{\pgfqpoint{-0.048611in}{0.000000in}}{\pgfqpoint{-0.000000in}{0.000000in}}{%
\pgfpathmoveto{\pgfqpoint{-0.000000in}{0.000000in}}%
\pgfpathlineto{\pgfqpoint{-0.048611in}{0.000000in}}%
\pgfusepath{stroke,fill}%
}%
\begin{pgfscope}%
\pgfsys@transformshift{0.471687in}{1.444545in}%
\pgfsys@useobject{currentmarker}{}%
\end{pgfscope}%
\end{pgfscope}%
\begin{pgfscope}%
\definecolor{textcolor}{rgb}{0.000000,0.000000,0.000000}%
\pgfsetstrokecolor{textcolor}%
\pgfsetfillcolor{textcolor}%
\pgftext[x=0.223614in, y=1.405989in, left, base]{\color{textcolor}\rmfamily\fontsize{8.000000}{9.600000}\selectfont \(\displaystyle {1.5}\)}%
\end{pgfscope}%
\begin{pgfscope}%
\pgfpathrectangle{\pgfqpoint{0.471687in}{0.416447in}}{\pgfqpoint{3.554884in}{2.056194in}}%
\pgfusepath{clip}%
\pgfsetrectcap%
\pgfsetroundjoin%
\pgfsetlinewidth{0.803000pt}%
\definecolor{currentstroke}{rgb}{0.450000,0.450000,0.450000}%
\pgfsetstrokecolor{currentstroke}%
\pgfsetdash{}{0pt}%
\pgfpathmoveto{\pgfqpoint{0.471687in}{1.756089in}}%
\pgfpathlineto{\pgfqpoint{4.026572in}{1.756089in}}%
\pgfusepath{stroke}%
\end{pgfscope}%
\begin{pgfscope}%
\pgfsetbuttcap%
\pgfsetroundjoin%
\definecolor{currentfill}{rgb}{0.000000,0.000000,0.000000}%
\pgfsetfillcolor{currentfill}%
\pgfsetlinewidth{0.803000pt}%
\definecolor{currentstroke}{rgb}{0.000000,0.000000,0.000000}%
\pgfsetstrokecolor{currentstroke}%
\pgfsetdash{}{0pt}%
\pgfsys@defobject{currentmarker}{\pgfqpoint{-0.048611in}{0.000000in}}{\pgfqpoint{-0.000000in}{0.000000in}}{%
\pgfpathmoveto{\pgfqpoint{-0.000000in}{0.000000in}}%
\pgfpathlineto{\pgfqpoint{-0.048611in}{0.000000in}}%
\pgfusepath{stroke,fill}%
}%
\begin{pgfscope}%
\pgfsys@transformshift{0.471687in}{1.756089in}%
\pgfsys@useobject{currentmarker}{}%
\end{pgfscope}%
\end{pgfscope}%
\begin{pgfscope}%
\definecolor{textcolor}{rgb}{0.000000,0.000000,0.000000}%
\pgfsetstrokecolor{textcolor}%
\pgfsetfillcolor{textcolor}%
\pgftext[x=0.223614in, y=1.717534in, left, base]{\color{textcolor}\rmfamily\fontsize{8.000000}{9.600000}\selectfont \(\displaystyle {2.0}\)}%
\end{pgfscope}%
\begin{pgfscope}%
\pgfpathrectangle{\pgfqpoint{0.471687in}{0.416447in}}{\pgfqpoint{3.554884in}{2.056194in}}%
\pgfusepath{clip}%
\pgfsetrectcap%
\pgfsetroundjoin%
\pgfsetlinewidth{0.803000pt}%
\definecolor{currentstroke}{rgb}{0.450000,0.450000,0.450000}%
\pgfsetstrokecolor{currentstroke}%
\pgfsetdash{}{0pt}%
\pgfpathmoveto{\pgfqpoint{0.471687in}{2.067634in}}%
\pgfpathlineto{\pgfqpoint{4.026572in}{2.067634in}}%
\pgfusepath{stroke}%
\end{pgfscope}%
\begin{pgfscope}%
\pgfsetbuttcap%
\pgfsetroundjoin%
\definecolor{currentfill}{rgb}{0.000000,0.000000,0.000000}%
\pgfsetfillcolor{currentfill}%
\pgfsetlinewidth{0.803000pt}%
\definecolor{currentstroke}{rgb}{0.000000,0.000000,0.000000}%
\pgfsetstrokecolor{currentstroke}%
\pgfsetdash{}{0pt}%
\pgfsys@defobject{currentmarker}{\pgfqpoint{-0.048611in}{0.000000in}}{\pgfqpoint{-0.000000in}{0.000000in}}{%
\pgfpathmoveto{\pgfqpoint{-0.000000in}{0.000000in}}%
\pgfpathlineto{\pgfqpoint{-0.048611in}{0.000000in}}%
\pgfusepath{stroke,fill}%
}%
\begin{pgfscope}%
\pgfsys@transformshift{0.471687in}{2.067634in}%
\pgfsys@useobject{currentmarker}{}%
\end{pgfscope}%
\end{pgfscope}%
\begin{pgfscope}%
\definecolor{textcolor}{rgb}{0.000000,0.000000,0.000000}%
\pgfsetstrokecolor{textcolor}%
\pgfsetfillcolor{textcolor}%
\pgftext[x=0.223614in, y=2.029078in, left, base]{\color{textcolor}\rmfamily\fontsize{8.000000}{9.600000}\selectfont \(\displaystyle {2.5}\)}%
\end{pgfscope}%
\begin{pgfscope}%
\pgfpathrectangle{\pgfqpoint{0.471687in}{0.416447in}}{\pgfqpoint{3.554884in}{2.056194in}}%
\pgfusepath{clip}%
\pgfsetrectcap%
\pgfsetroundjoin%
\pgfsetlinewidth{0.803000pt}%
\definecolor{currentstroke}{rgb}{0.450000,0.450000,0.450000}%
\pgfsetstrokecolor{currentstroke}%
\pgfsetdash{}{0pt}%
\pgfpathmoveto{\pgfqpoint{0.471687in}{2.379178in}}%
\pgfpathlineto{\pgfqpoint{4.026572in}{2.379178in}}%
\pgfusepath{stroke}%
\end{pgfscope}%
\begin{pgfscope}%
\pgfsetbuttcap%
\pgfsetroundjoin%
\definecolor{currentfill}{rgb}{0.000000,0.000000,0.000000}%
\pgfsetfillcolor{currentfill}%
\pgfsetlinewidth{0.803000pt}%
\definecolor{currentstroke}{rgb}{0.000000,0.000000,0.000000}%
\pgfsetstrokecolor{currentstroke}%
\pgfsetdash{}{0pt}%
\pgfsys@defobject{currentmarker}{\pgfqpoint{-0.048611in}{0.000000in}}{\pgfqpoint{-0.000000in}{0.000000in}}{%
\pgfpathmoveto{\pgfqpoint{-0.000000in}{0.000000in}}%
\pgfpathlineto{\pgfqpoint{-0.048611in}{0.000000in}}%
\pgfusepath{stroke,fill}%
}%
\begin{pgfscope}%
\pgfsys@transformshift{0.471687in}{2.379178in}%
\pgfsys@useobject{currentmarker}{}%
\end{pgfscope}%
\end{pgfscope}%
\begin{pgfscope}%
\definecolor{textcolor}{rgb}{0.000000,0.000000,0.000000}%
\pgfsetstrokecolor{textcolor}%
\pgfsetfillcolor{textcolor}%
\pgftext[x=0.223614in, y=2.340623in, left, base]{\color{textcolor}\rmfamily\fontsize{8.000000}{9.600000}\selectfont \(\displaystyle {3.0}\)}%
\end{pgfscope}%
\begin{pgfscope}%
\definecolor{textcolor}{rgb}{0.000000,0.000000,0.000000}%
\pgfsetstrokecolor{textcolor}%
\pgfsetfillcolor{textcolor}%
\pgftext[x=0.168059in,y=1.444545in,,bottom,rotate=90.000000]{\color{textcolor}\rmfamily\fontsize{10.000000}{12.000000}\selectfont Amplitude in arb. unit}%
\end{pgfscope}%
\begin{pgfscope}%
\pgfpathrectangle{\pgfqpoint{0.471687in}{0.416447in}}{\pgfqpoint{3.554884in}{2.056194in}}%
\pgfusepath{clip}%
\pgfsetrectcap%
\pgfsetroundjoin%
\pgfsetlinewidth{1.505625pt}%
\definecolor{currentstroke}{rgb}{0.003922,0.450980,0.698039}%
\pgfsetstrokecolor{currentstroke}%
\pgfsetdash{}{0pt}%
\pgfpathmoveto{\pgfqpoint{0.633273in}{0.509911in}}%
\pgfpathlineto{\pgfqpoint{0.736740in}{0.509911in}}%
\pgfpathlineto{\pgfqpoint{0.736740in}{1.133000in}}%
\pgfpathlineto{\pgfqpoint{0.748056in}{1.133000in}}%
\pgfpathlineto{\pgfqpoint{0.748056in}{0.509911in}}%
\pgfpathlineto{\pgfqpoint{0.974389in}{0.509911in}}%
\pgfpathlineto{\pgfqpoint{0.974389in}{1.133000in}}%
\pgfpathlineto{\pgfqpoint{0.987323in}{1.133000in}}%
\pgfpathlineto{\pgfqpoint{0.987323in}{0.509911in}}%
\pgfpathlineto{\pgfqpoint{1.073006in}{0.509911in}}%
\pgfpathlineto{\pgfqpoint{1.073006in}{1.133000in}}%
\pgfpathlineto{\pgfqpoint{1.106956in}{1.133000in}}%
\pgfpathlineto{\pgfqpoint{1.106956in}{0.509911in}}%
\pgfpathlineto{\pgfqpoint{1.905588in}{0.509911in}}%
\pgfpathlineto{\pgfqpoint{1.905588in}{1.133000in}}%
\pgfpathlineto{\pgfqpoint{1.989655in}{1.133000in}}%
\pgfpathlineto{\pgfqpoint{1.989655in}{0.509911in}}%
\pgfpathlineto{\pgfqpoint{2.249938in}{0.509911in}}%
\pgfpathlineto{\pgfqpoint{2.249938in}{1.133000in}}%
\pgfpathlineto{\pgfqpoint{2.264488in}{1.133000in}}%
\pgfpathlineto{\pgfqpoint{2.264488in}{0.509911in}}%
\pgfpathlineto{\pgfqpoint{2.321071in}{0.509911in}}%
\pgfpathlineto{\pgfqpoint{2.321071in}{1.133000in}}%
\pgfpathlineto{\pgfqpoint{2.342088in}{1.133000in}}%
\pgfpathlineto{\pgfqpoint{2.342088in}{0.509911in}}%
\pgfpathlineto{\pgfqpoint{2.670271in}{0.509911in}}%
\pgfpathlineto{\pgfqpoint{2.670271in}{1.133000in}}%
\pgfpathlineto{\pgfqpoint{2.673504in}{1.133000in}}%
\pgfpathlineto{\pgfqpoint{2.673504in}{0.509911in}}%
\pgfpathlineto{\pgfqpoint{2.983904in}{0.509911in}}%
\pgfpathlineto{\pgfqpoint{2.983904in}{1.133000in}}%
\pgfpathlineto{\pgfqpoint{2.998454in}{1.133000in}}%
\pgfpathlineto{\pgfqpoint{2.998454in}{0.509911in}}%
\pgfpathlineto{\pgfqpoint{3.541653in}{0.509911in}}%
\pgfpathlineto{\pgfqpoint{3.541653in}{1.133000in}}%
\pgfpathlineto{\pgfqpoint{3.585303in}{1.133000in}}%
\pgfpathlineto{\pgfqpoint{3.585303in}{0.509911in}}%
\pgfpathlineto{\pgfqpoint{3.864986in}{0.509911in}}%
\pgfpathlineto{\pgfqpoint{3.864986in}{0.509911in}}%
\pgfusepath{stroke}%
\end{pgfscope}%
\begin{pgfscope}%
\pgfpathrectangle{\pgfqpoint{0.471687in}{0.416447in}}{\pgfqpoint{3.554884in}{2.056194in}}%
\pgfusepath{clip}%
\pgfsetrectcap%
\pgfsetroundjoin%
\pgfsetlinewidth{1.505625pt}%
\definecolor{currentstroke}{rgb}{0.007843,0.619608,0.450980}%
\pgfsetstrokecolor{currentstroke}%
\pgfsetdash{}{0pt}%
\pgfpathmoveto{\pgfqpoint{0.633273in}{1.133000in}}%
\pgfpathlineto{\pgfqpoint{0.736740in}{1.133000in}}%
\pgfpathlineto{\pgfqpoint{0.736740in}{1.756089in}}%
\pgfpathlineto{\pgfqpoint{0.917806in}{1.756089in}}%
\pgfpathlineto{\pgfqpoint{0.917806in}{1.133000in}}%
\pgfpathlineto{\pgfqpoint{1.144139in}{1.133000in}}%
\pgfpathlineto{\pgfqpoint{1.144139in}{1.756089in}}%
\pgfpathlineto{\pgfqpoint{1.619439in}{1.756089in}}%
\pgfpathlineto{\pgfqpoint{1.619439in}{1.133000in}}%
\pgfpathlineto{\pgfqpoint{1.705122in}{1.133000in}}%
\pgfpathlineto{\pgfqpoint{1.705122in}{1.756089in}}%
\pgfpathlineto{\pgfqpoint{2.097971in}{1.756089in}}%
\pgfpathlineto{\pgfqpoint{2.097971in}{1.133000in}}%
\pgfpathlineto{\pgfqpoint{2.896604in}{1.133000in}}%
\pgfpathlineto{\pgfqpoint{2.896604in}{1.756089in}}%
\pgfpathlineto{\pgfqpoint{3.864986in}{1.756089in}}%
\pgfpathlineto{\pgfqpoint{3.864986in}{1.756089in}}%
\pgfusepath{stroke}%
\end{pgfscope}%
\begin{pgfscope}%
\pgfpathrectangle{\pgfqpoint{0.471687in}{0.416447in}}{\pgfqpoint{3.554884in}{2.056194in}}%
\pgfusepath{clip}%
\pgfsetrectcap%
\pgfsetroundjoin%
\pgfsetlinewidth{1.505625pt}%
\definecolor{currentstroke}{rgb}{0.835294,0.368627,0.000000}%
\pgfsetstrokecolor{currentstroke}%
\pgfsetdash{}{0pt}%
\pgfpathmoveto{\pgfqpoint{0.633273in}{2.379178in}}%
\pgfpathlineto{\pgfqpoint{0.812723in}{2.379178in}}%
\pgfpathlineto{\pgfqpoint{0.812723in}{1.756089in}}%
\pgfpathlineto{\pgfqpoint{0.917806in}{1.756089in}}%
\pgfpathlineto{\pgfqpoint{0.917806in}{2.379178in}}%
\pgfpathlineto{\pgfqpoint{3.864986in}{2.379178in}}%
\pgfpathlineto{\pgfqpoint{3.864986in}{2.379178in}}%
\pgfusepath{stroke}%
\end{pgfscope}%
\begin{pgfscope}%
\pgfsetrectcap%
\pgfsetmiterjoin%
\pgfsetlinewidth{0.803000pt}%
\definecolor{currentstroke}{rgb}{0.000000,0.000000,0.000000}%
\pgfsetstrokecolor{currentstroke}%
\pgfsetdash{}{0pt}%
\pgfpathmoveto{\pgfqpoint{0.471687in}{0.416447in}}%
\pgfpathlineto{\pgfqpoint{0.471687in}{2.472642in}}%
\pgfusepath{stroke}%
\end{pgfscope}%
\begin{pgfscope}%
\pgfsetrectcap%
\pgfsetmiterjoin%
\pgfsetlinewidth{0.803000pt}%
\definecolor{currentstroke}{rgb}{0.000000,0.000000,0.000000}%
\pgfsetstrokecolor{currentstroke}%
\pgfsetdash{}{0pt}%
\pgfpathmoveto{\pgfqpoint{4.026572in}{0.416447in}}%
\pgfpathlineto{\pgfqpoint{4.026572in}{2.472642in}}%
\pgfusepath{stroke}%
\end{pgfscope}%
\begin{pgfscope}%
\pgfsetrectcap%
\pgfsetmiterjoin%
\pgfsetlinewidth{0.803000pt}%
\definecolor{currentstroke}{rgb}{0.000000,0.000000,0.000000}%
\pgfsetstrokecolor{currentstroke}%
\pgfsetdash{}{0pt}%
\pgfpathmoveto{\pgfqpoint{0.471687in}{0.416447in}}%
\pgfpathlineto{\pgfqpoint{4.026572in}{0.416447in}}%
\pgfusepath{stroke}%
\end{pgfscope}%
\begin{pgfscope}%
\pgfsetrectcap%
\pgfsetmiterjoin%
\pgfsetlinewidth{0.803000pt}%
\definecolor{currentstroke}{rgb}{0.000000,0.000000,0.000000}%
\pgfsetstrokecolor{currentstroke}%
\pgfsetdash{}{0pt}%
\pgfpathmoveto{\pgfqpoint{0.471687in}{2.472642in}}%
\pgfpathlineto{\pgfqpoint{4.026572in}{2.472642in}}%
\pgfusepath{stroke}%
\end{pgfscope}%
\begin{pgfscope}%
\pgfsetbuttcap%
\pgfsetmiterjoin%
\definecolor{currentfill}{rgb}{1.000000,1.000000,1.000000}%
\pgfsetfillcolor{currentfill}%
\pgfsetfillopacity{0.800000}%
\pgfsetlinewidth{1.003750pt}%
\definecolor{currentstroke}{rgb}{0.800000,0.800000,0.800000}%
\pgfsetstrokecolor{currentstroke}%
\pgfsetstrokeopacity{0.800000}%
\pgfsetdash{}{0pt}%
\pgfpathmoveto{\pgfqpoint{3.108484in}{1.919086in}}%
\pgfpathlineto{\pgfqpoint{3.948794in}{1.919086in}}%
\pgfpathquadraticcurveto{\pgfqpoint{3.971016in}{1.919086in}}{\pgfqpoint{3.971016in}{1.941309in}}%
\pgfpathlineto{\pgfqpoint{3.971016in}{2.394864in}}%
\pgfpathquadraticcurveto{\pgfqpoint{3.971016in}{2.417086in}}{\pgfqpoint{3.948794in}{2.417086in}}%
\pgfpathlineto{\pgfqpoint{3.108484in}{2.417086in}}%
\pgfpathquadraticcurveto{\pgfqpoint{3.086261in}{2.417086in}}{\pgfqpoint{3.086261in}{2.394864in}}%
\pgfpathlineto{\pgfqpoint{3.086261in}{1.941309in}}%
\pgfpathquadraticcurveto{\pgfqpoint{3.086261in}{1.919086in}}{\pgfqpoint{3.108484in}{1.919086in}}%
\pgfpathlineto{\pgfqpoint{3.108484in}{1.919086in}}%
\pgfpathclose%
\pgfusepath{stroke,fill}%
\end{pgfscope}%
\begin{pgfscope}%
\pgfsetrectcap%
\pgfsetroundjoin%
\pgfsetlinewidth{1.505625pt}%
\definecolor{currentstroke}{rgb}{0.003922,0.450980,0.698039}%
\pgfsetstrokecolor{currentstroke}%
\pgfsetdash{}{0pt}%
\pgfpathmoveto{\pgfqpoint{3.130706in}{2.333753in}}%
\pgfpathlineto{\pgfqpoint{3.130706in}{2.333753in}}%
\pgfpathlineto{\pgfqpoint{3.241817in}{2.333753in}}%
\pgfpathlineto{\pgfqpoint{3.241817in}{2.333753in}}%
\pgfpathlineto{\pgfqpoint{3.352928in}{2.333753in}}%
\pgfusepath{stroke}%
\end{pgfscope}%
\begin{pgfscope}%
\definecolor{textcolor}{rgb}{0.000000,0.000000,0.000000}%
\pgfsetstrokecolor{textcolor}%
\pgfsetfillcolor{textcolor}%
\pgftext[x=3.441817in,y=2.294864in,left,base]{\color{textcolor}\rmfamily\fontsize{8.000000}{9.600000}\selectfont \(\displaystyle \bar\tau_1=\qty{0.1}{\s}\)}%
\end{pgfscope}%
\begin{pgfscope}%
\pgfsetrectcap%
\pgfsetroundjoin%
\pgfsetlinewidth{1.505625pt}%
\definecolor{currentstroke}{rgb}{0.007843,0.619608,0.450980}%
\pgfsetstrokecolor{currentstroke}%
\pgfsetdash{}{0pt}%
\pgfpathmoveto{\pgfqpoint{3.130706in}{2.178864in}}%
\pgfpathlineto{\pgfqpoint{3.130706in}{2.178864in}}%
\pgfpathlineto{\pgfqpoint{3.241817in}{2.178864in}}%
\pgfpathlineto{\pgfqpoint{3.241817in}{2.178864in}}%
\pgfpathlineto{\pgfqpoint{3.352928in}{2.178864in}}%
\pgfusepath{stroke}%
\end{pgfscope}%
\begin{pgfscope}%
\definecolor{textcolor}{rgb}{0.000000,0.000000,0.000000}%
\pgfsetstrokecolor{textcolor}%
\pgfsetfillcolor{textcolor}%
\pgftext[x=3.441817in,y=2.139975in,left,base]{\color{textcolor}\rmfamily\fontsize{8.000000}{9.600000}\selectfont \(\displaystyle \bar\tau_1=\qty{1}{\s}\)}%
\end{pgfscope}%
\begin{pgfscope}%
\pgfsetrectcap%
\pgfsetroundjoin%
\pgfsetlinewidth{1.505625pt}%
\definecolor{currentstroke}{rgb}{0.835294,0.368627,0.000000}%
\pgfsetstrokecolor{currentstroke}%
\pgfsetdash{}{0pt}%
\pgfpathmoveto{\pgfqpoint{3.130706in}{2.023975in}}%
\pgfpathlineto{\pgfqpoint{3.130706in}{2.023975in}}%
\pgfpathlineto{\pgfqpoint{3.241817in}{2.023975in}}%
\pgfpathlineto{\pgfqpoint{3.241817in}{2.023975in}}%
\pgfpathlineto{\pgfqpoint{3.352928in}{2.023975in}}%
\pgfusepath{stroke}%
\end{pgfscope}%
\begin{pgfscope}%
\definecolor{textcolor}{rgb}{0.000000,0.000000,0.000000}%
\pgfsetstrokecolor{textcolor}%
\pgfsetfillcolor{textcolor}%
\pgftext[x=3.441817in,y=1.985086in,left,base]{\color{textcolor}\rmfamily\fontsize{8.000000}{9.600000}\selectfont \(\displaystyle \bar\tau_1=\qty{10}{\s}\)}%
\end{pgfscope}%
\end{pgfpicture}%
\makeatother%
\endgroup%
% data/simulations/sim_burst_noise.py
        } % scalebox
        \caption{Time domain}
        \label{fig:burst_noise_time}
    \end{subfigure}
    \begin{subfigure}{0.8\linewidth}
        \centering
        \scalebox{1}{%
            %% Creator: Matplotlib, PGF backend
%%
%% To include the figure in your LaTeX document, write
%%   \input{<filename>.pgf}
%%
%% Make sure the required packages are loaded in your preamble
%%   \usepackage{pgf}
%%
%% Also ensure that all the required font packages are loaded; for instance,
%% the lmodern package is sometimes necessary when using math font.
%%   \usepackage{lmodern}
%%
%% Figures using additional raster images can only be included by \input if
%% they are in the same directory as the main LaTeX file. For loading figures
%% from other directories you can use the `import` package
%%   \usepackage{import}
%%
%% and then include the figures with
%%   \import{<path to file>}{<filename>.pgf}
%%
%% Matplotlib used the following preamble
%%   \usepackage{siunitx}
%%   \sisetup{per-mode = symbol}%
%%   \usepackage{fontspec}
%%   \makeatletter\@ifpackageloaded{underscore}{}{\usepackage[strings]{underscore}}\makeatother
%%
\begingroup%
\makeatletter%
\begin{pgfpicture}%
\pgfpathrectangle{\pgfpointorigin}{\pgfqpoint{4.068242in}{2.514312in}}%
\pgfusepath{use as bounding box, clip}%
\begin{pgfscope}%
\pgfsetbuttcap%
\pgfsetmiterjoin%
\definecolor{currentfill}{rgb}{1.000000,1.000000,1.000000}%
\pgfsetfillcolor{currentfill}%
\pgfsetlinewidth{0.000000pt}%
\definecolor{currentstroke}{rgb}{1.000000,1.000000,1.000000}%
\pgfsetstrokecolor{currentstroke}%
\pgfsetdash{}{0pt}%
\pgfpathmoveto{\pgfqpoint{0.000000in}{0.000000in}}%
\pgfpathlineto{\pgfqpoint{4.068242in}{0.000000in}}%
\pgfpathlineto{\pgfqpoint{4.068242in}{2.514312in}}%
\pgfpathlineto{\pgfqpoint{0.000000in}{2.514312in}}%
\pgfpathlineto{\pgfqpoint{0.000000in}{0.000000in}}%
\pgfpathclose%
\pgfusepath{fill}%
\end{pgfscope}%
\begin{pgfscope}%
\pgfsetbuttcap%
\pgfsetmiterjoin%
\definecolor{currentfill}{rgb}{1.000000,1.000000,1.000000}%
\pgfsetfillcolor{currentfill}%
\pgfsetlinewidth{0.000000pt}%
\definecolor{currentstroke}{rgb}{0.000000,0.000000,0.000000}%
\pgfsetstrokecolor{currentstroke}%
\pgfsetstrokeopacity{0.000000}%
\pgfsetdash{}{0pt}%
\pgfpathmoveto{\pgfqpoint{0.594525in}{0.417642in}}%
\pgfpathlineto{\pgfqpoint{4.026572in}{0.417642in}}%
\pgfpathlineto{\pgfqpoint{4.026572in}{2.433919in}}%
\pgfpathlineto{\pgfqpoint{0.594525in}{2.433919in}}%
\pgfpathlineto{\pgfqpoint{0.594525in}{0.417642in}}%
\pgfpathclose%
\pgfusepath{fill}%
\end{pgfscope}%
\begin{pgfscope}%
\pgfpathrectangle{\pgfqpoint{0.594525in}{0.417642in}}{\pgfqpoint{3.432047in}{2.016277in}}%
\pgfusepath{clip}%
\pgfsetrectcap%
\pgfsetroundjoin%
\pgfsetlinewidth{0.803000pt}%
\definecolor{currentstroke}{rgb}{0.450000,0.450000,0.450000}%
\pgfsetstrokecolor{currentstroke}%
\pgfsetdash{}{0pt}%
\pgfpathmoveto{\pgfqpoint{0.750527in}{0.417642in}}%
\pgfpathlineto{\pgfqpoint{0.750527in}{2.433919in}}%
\pgfusepath{stroke}%
\end{pgfscope}%
\begin{pgfscope}%
\pgfsetbuttcap%
\pgfsetroundjoin%
\definecolor{currentfill}{rgb}{0.000000,0.000000,0.000000}%
\pgfsetfillcolor{currentfill}%
\pgfsetlinewidth{0.803000pt}%
\definecolor{currentstroke}{rgb}{0.000000,0.000000,0.000000}%
\pgfsetstrokecolor{currentstroke}%
\pgfsetdash{}{0pt}%
\pgfsys@defobject{currentmarker}{\pgfqpoint{0.000000in}{-0.048611in}}{\pgfqpoint{0.000000in}{0.000000in}}{%
\pgfpathmoveto{\pgfqpoint{0.000000in}{0.000000in}}%
\pgfpathlineto{\pgfqpoint{0.000000in}{-0.048611in}}%
\pgfusepath{stroke,fill}%
}%
\begin{pgfscope}%
\pgfsys@transformshift{0.750527in}{0.417642in}%
\pgfsys@useobject{currentmarker}{}%
\end{pgfscope}%
\end{pgfscope}%
\begin{pgfscope}%
\definecolor{textcolor}{rgb}{0.000000,0.000000,0.000000}%
\pgfsetstrokecolor{textcolor}%
\pgfsetfillcolor{textcolor}%
\pgftext[x=0.750527in,y=0.320420in,,top]{\color{textcolor}\rmfamily\fontsize{8.000000}{9.600000}\selectfont \(\displaystyle {10^{-2}}\)}%
\end{pgfscope}%
\begin{pgfscope}%
\pgfpathrectangle{\pgfqpoint{0.594525in}{0.417642in}}{\pgfqpoint{3.432047in}{2.016277in}}%
\pgfusepath{clip}%
\pgfsetrectcap%
\pgfsetroundjoin%
\pgfsetlinewidth{0.803000pt}%
\definecolor{currentstroke}{rgb}{0.450000,0.450000,0.450000}%
\pgfsetstrokecolor{currentstroke}%
\pgfsetdash{}{0pt}%
\pgfpathmoveto{\pgfqpoint{1.531514in}{0.417642in}}%
\pgfpathlineto{\pgfqpoint{1.531514in}{2.433919in}}%
\pgfusepath{stroke}%
\end{pgfscope}%
\begin{pgfscope}%
\pgfsetbuttcap%
\pgfsetroundjoin%
\definecolor{currentfill}{rgb}{0.000000,0.000000,0.000000}%
\pgfsetfillcolor{currentfill}%
\pgfsetlinewidth{0.803000pt}%
\definecolor{currentstroke}{rgb}{0.000000,0.000000,0.000000}%
\pgfsetstrokecolor{currentstroke}%
\pgfsetdash{}{0pt}%
\pgfsys@defobject{currentmarker}{\pgfqpoint{0.000000in}{-0.048611in}}{\pgfqpoint{0.000000in}{0.000000in}}{%
\pgfpathmoveto{\pgfqpoint{0.000000in}{0.000000in}}%
\pgfpathlineto{\pgfqpoint{0.000000in}{-0.048611in}}%
\pgfusepath{stroke,fill}%
}%
\begin{pgfscope}%
\pgfsys@transformshift{1.531514in}{0.417642in}%
\pgfsys@useobject{currentmarker}{}%
\end{pgfscope}%
\end{pgfscope}%
\begin{pgfscope}%
\definecolor{textcolor}{rgb}{0.000000,0.000000,0.000000}%
\pgfsetstrokecolor{textcolor}%
\pgfsetfillcolor{textcolor}%
\pgftext[x=1.531514in,y=0.320420in,,top]{\color{textcolor}\rmfamily\fontsize{8.000000}{9.600000}\selectfont \(\displaystyle {10^{-1}}\)}%
\end{pgfscope}%
\begin{pgfscope}%
\pgfpathrectangle{\pgfqpoint{0.594525in}{0.417642in}}{\pgfqpoint{3.432047in}{2.016277in}}%
\pgfusepath{clip}%
\pgfsetrectcap%
\pgfsetroundjoin%
\pgfsetlinewidth{0.803000pt}%
\definecolor{currentstroke}{rgb}{0.450000,0.450000,0.450000}%
\pgfsetstrokecolor{currentstroke}%
\pgfsetdash{}{0pt}%
\pgfpathmoveto{\pgfqpoint{2.312501in}{0.417642in}}%
\pgfpathlineto{\pgfqpoint{2.312501in}{2.433919in}}%
\pgfusepath{stroke}%
\end{pgfscope}%
\begin{pgfscope}%
\pgfsetbuttcap%
\pgfsetroundjoin%
\definecolor{currentfill}{rgb}{0.000000,0.000000,0.000000}%
\pgfsetfillcolor{currentfill}%
\pgfsetlinewidth{0.803000pt}%
\definecolor{currentstroke}{rgb}{0.000000,0.000000,0.000000}%
\pgfsetstrokecolor{currentstroke}%
\pgfsetdash{}{0pt}%
\pgfsys@defobject{currentmarker}{\pgfqpoint{0.000000in}{-0.048611in}}{\pgfqpoint{0.000000in}{0.000000in}}{%
\pgfpathmoveto{\pgfqpoint{0.000000in}{0.000000in}}%
\pgfpathlineto{\pgfqpoint{0.000000in}{-0.048611in}}%
\pgfusepath{stroke,fill}%
}%
\begin{pgfscope}%
\pgfsys@transformshift{2.312501in}{0.417642in}%
\pgfsys@useobject{currentmarker}{}%
\end{pgfscope}%
\end{pgfscope}%
\begin{pgfscope}%
\definecolor{textcolor}{rgb}{0.000000,0.000000,0.000000}%
\pgfsetstrokecolor{textcolor}%
\pgfsetfillcolor{textcolor}%
\pgftext[x=2.312501in,y=0.320420in,,top]{\color{textcolor}\rmfamily\fontsize{8.000000}{9.600000}\selectfont \(\displaystyle {10^{0}}\)}%
\end{pgfscope}%
\begin{pgfscope}%
\pgfpathrectangle{\pgfqpoint{0.594525in}{0.417642in}}{\pgfqpoint{3.432047in}{2.016277in}}%
\pgfusepath{clip}%
\pgfsetrectcap%
\pgfsetroundjoin%
\pgfsetlinewidth{0.803000pt}%
\definecolor{currentstroke}{rgb}{0.450000,0.450000,0.450000}%
\pgfsetstrokecolor{currentstroke}%
\pgfsetdash{}{0pt}%
\pgfpathmoveto{\pgfqpoint{3.093488in}{0.417642in}}%
\pgfpathlineto{\pgfqpoint{3.093488in}{2.433919in}}%
\pgfusepath{stroke}%
\end{pgfscope}%
\begin{pgfscope}%
\pgfsetbuttcap%
\pgfsetroundjoin%
\definecolor{currentfill}{rgb}{0.000000,0.000000,0.000000}%
\pgfsetfillcolor{currentfill}%
\pgfsetlinewidth{0.803000pt}%
\definecolor{currentstroke}{rgb}{0.000000,0.000000,0.000000}%
\pgfsetstrokecolor{currentstroke}%
\pgfsetdash{}{0pt}%
\pgfsys@defobject{currentmarker}{\pgfqpoint{0.000000in}{-0.048611in}}{\pgfqpoint{0.000000in}{0.000000in}}{%
\pgfpathmoveto{\pgfqpoint{0.000000in}{0.000000in}}%
\pgfpathlineto{\pgfqpoint{0.000000in}{-0.048611in}}%
\pgfusepath{stroke,fill}%
}%
\begin{pgfscope}%
\pgfsys@transformshift{3.093488in}{0.417642in}%
\pgfsys@useobject{currentmarker}{}%
\end{pgfscope}%
\end{pgfscope}%
\begin{pgfscope}%
\definecolor{textcolor}{rgb}{0.000000,0.000000,0.000000}%
\pgfsetstrokecolor{textcolor}%
\pgfsetfillcolor{textcolor}%
\pgftext[x=3.093488in,y=0.320420in,,top]{\color{textcolor}\rmfamily\fontsize{8.000000}{9.600000}\selectfont \(\displaystyle {10^{1}}\)}%
\end{pgfscope}%
\begin{pgfscope}%
\pgfpathrectangle{\pgfqpoint{0.594525in}{0.417642in}}{\pgfqpoint{3.432047in}{2.016277in}}%
\pgfusepath{clip}%
\pgfsetrectcap%
\pgfsetroundjoin%
\pgfsetlinewidth{0.803000pt}%
\definecolor{currentstroke}{rgb}{0.450000,0.450000,0.450000}%
\pgfsetstrokecolor{currentstroke}%
\pgfsetdash{}{0pt}%
\pgfpathmoveto{\pgfqpoint{3.874475in}{0.417642in}}%
\pgfpathlineto{\pgfqpoint{3.874475in}{2.433919in}}%
\pgfusepath{stroke}%
\end{pgfscope}%
\begin{pgfscope}%
\pgfsetbuttcap%
\pgfsetroundjoin%
\definecolor{currentfill}{rgb}{0.000000,0.000000,0.000000}%
\pgfsetfillcolor{currentfill}%
\pgfsetlinewidth{0.803000pt}%
\definecolor{currentstroke}{rgb}{0.000000,0.000000,0.000000}%
\pgfsetstrokecolor{currentstroke}%
\pgfsetdash{}{0pt}%
\pgfsys@defobject{currentmarker}{\pgfqpoint{0.000000in}{-0.048611in}}{\pgfqpoint{0.000000in}{0.000000in}}{%
\pgfpathmoveto{\pgfqpoint{0.000000in}{0.000000in}}%
\pgfpathlineto{\pgfqpoint{0.000000in}{-0.048611in}}%
\pgfusepath{stroke,fill}%
}%
\begin{pgfscope}%
\pgfsys@transformshift{3.874475in}{0.417642in}%
\pgfsys@useobject{currentmarker}{}%
\end{pgfscope}%
\end{pgfscope}%
\begin{pgfscope}%
\definecolor{textcolor}{rgb}{0.000000,0.000000,0.000000}%
\pgfsetstrokecolor{textcolor}%
\pgfsetfillcolor{textcolor}%
\pgftext[x=3.874475in,y=0.320420in,,top]{\color{textcolor}\rmfamily\fontsize{8.000000}{9.600000}\selectfont \(\displaystyle {10^{2}}\)}%
\end{pgfscope}%
\begin{pgfscope}%
\pgfpathrectangle{\pgfqpoint{0.594525in}{0.417642in}}{\pgfqpoint{3.432047in}{2.016277in}}%
\pgfusepath{clip}%
\pgfsetrectcap%
\pgfsetroundjoin%
\pgfsetlinewidth{0.803000pt}%
\definecolor{currentstroke}{rgb}{0.850000,0.850000,0.850000}%
\pgfsetstrokecolor{currentstroke}%
\pgfsetdash{}{0pt}%
\pgfpathmoveto{\pgfqpoint{0.629550in}{0.417642in}}%
\pgfpathlineto{\pgfqpoint{0.629550in}{2.433919in}}%
\pgfusepath{stroke}%
\end{pgfscope}%
\begin{pgfscope}%
\pgfsetbuttcap%
\pgfsetroundjoin%
\definecolor{currentfill}{rgb}{0.000000,0.000000,0.000000}%
\pgfsetfillcolor{currentfill}%
\pgfsetlinewidth{0.602250pt}%
\definecolor{currentstroke}{rgb}{0.000000,0.000000,0.000000}%
\pgfsetstrokecolor{currentstroke}%
\pgfsetdash{}{0pt}%
\pgfsys@defobject{currentmarker}{\pgfqpoint{0.000000in}{-0.027778in}}{\pgfqpoint{0.000000in}{0.000000in}}{%
\pgfpathmoveto{\pgfqpoint{0.000000in}{0.000000in}}%
\pgfpathlineto{\pgfqpoint{0.000000in}{-0.027778in}}%
\pgfusepath{stroke,fill}%
}%
\begin{pgfscope}%
\pgfsys@transformshift{0.629550in}{0.417642in}%
\pgfsys@useobject{currentmarker}{}%
\end{pgfscope}%
\end{pgfscope}%
\begin{pgfscope}%
\pgfpathrectangle{\pgfqpoint{0.594525in}{0.417642in}}{\pgfqpoint{3.432047in}{2.016277in}}%
\pgfusepath{clip}%
\pgfsetrectcap%
\pgfsetroundjoin%
\pgfsetlinewidth{0.803000pt}%
\definecolor{currentstroke}{rgb}{0.850000,0.850000,0.850000}%
\pgfsetstrokecolor{currentstroke}%
\pgfsetdash{}{0pt}%
\pgfpathmoveto{\pgfqpoint{0.674841in}{0.417642in}}%
\pgfpathlineto{\pgfqpoint{0.674841in}{2.433919in}}%
\pgfusepath{stroke}%
\end{pgfscope}%
\begin{pgfscope}%
\pgfsetbuttcap%
\pgfsetroundjoin%
\definecolor{currentfill}{rgb}{0.000000,0.000000,0.000000}%
\pgfsetfillcolor{currentfill}%
\pgfsetlinewidth{0.602250pt}%
\definecolor{currentstroke}{rgb}{0.000000,0.000000,0.000000}%
\pgfsetstrokecolor{currentstroke}%
\pgfsetdash{}{0pt}%
\pgfsys@defobject{currentmarker}{\pgfqpoint{0.000000in}{-0.027778in}}{\pgfqpoint{0.000000in}{0.000000in}}{%
\pgfpathmoveto{\pgfqpoint{0.000000in}{0.000000in}}%
\pgfpathlineto{\pgfqpoint{0.000000in}{-0.027778in}}%
\pgfusepath{stroke,fill}%
}%
\begin{pgfscope}%
\pgfsys@transformshift{0.674841in}{0.417642in}%
\pgfsys@useobject{currentmarker}{}%
\end{pgfscope}%
\end{pgfscope}%
\begin{pgfscope}%
\pgfpathrectangle{\pgfqpoint{0.594525in}{0.417642in}}{\pgfqpoint{3.432047in}{2.016277in}}%
\pgfusepath{clip}%
\pgfsetrectcap%
\pgfsetroundjoin%
\pgfsetlinewidth{0.803000pt}%
\definecolor{currentstroke}{rgb}{0.850000,0.850000,0.850000}%
\pgfsetstrokecolor{currentstroke}%
\pgfsetdash{}{0pt}%
\pgfpathmoveto{\pgfqpoint{0.714791in}{0.417642in}}%
\pgfpathlineto{\pgfqpoint{0.714791in}{2.433919in}}%
\pgfusepath{stroke}%
\end{pgfscope}%
\begin{pgfscope}%
\pgfsetbuttcap%
\pgfsetroundjoin%
\definecolor{currentfill}{rgb}{0.000000,0.000000,0.000000}%
\pgfsetfillcolor{currentfill}%
\pgfsetlinewidth{0.602250pt}%
\definecolor{currentstroke}{rgb}{0.000000,0.000000,0.000000}%
\pgfsetstrokecolor{currentstroke}%
\pgfsetdash{}{0pt}%
\pgfsys@defobject{currentmarker}{\pgfqpoint{0.000000in}{-0.027778in}}{\pgfqpoint{0.000000in}{0.000000in}}{%
\pgfpathmoveto{\pgfqpoint{0.000000in}{0.000000in}}%
\pgfpathlineto{\pgfqpoint{0.000000in}{-0.027778in}}%
\pgfusepath{stroke,fill}%
}%
\begin{pgfscope}%
\pgfsys@transformshift{0.714791in}{0.417642in}%
\pgfsys@useobject{currentmarker}{}%
\end{pgfscope}%
\end{pgfscope}%
\begin{pgfscope}%
\pgfpathrectangle{\pgfqpoint{0.594525in}{0.417642in}}{\pgfqpoint{3.432047in}{2.016277in}}%
\pgfusepath{clip}%
\pgfsetrectcap%
\pgfsetroundjoin%
\pgfsetlinewidth{0.803000pt}%
\definecolor{currentstroke}{rgb}{0.850000,0.850000,0.850000}%
\pgfsetstrokecolor{currentstroke}%
\pgfsetdash{}{0pt}%
\pgfpathmoveto{\pgfqpoint{0.985627in}{0.417642in}}%
\pgfpathlineto{\pgfqpoint{0.985627in}{2.433919in}}%
\pgfusepath{stroke}%
\end{pgfscope}%
\begin{pgfscope}%
\pgfsetbuttcap%
\pgfsetroundjoin%
\definecolor{currentfill}{rgb}{0.000000,0.000000,0.000000}%
\pgfsetfillcolor{currentfill}%
\pgfsetlinewidth{0.602250pt}%
\definecolor{currentstroke}{rgb}{0.000000,0.000000,0.000000}%
\pgfsetstrokecolor{currentstroke}%
\pgfsetdash{}{0pt}%
\pgfsys@defobject{currentmarker}{\pgfqpoint{0.000000in}{-0.027778in}}{\pgfqpoint{0.000000in}{0.000000in}}{%
\pgfpathmoveto{\pgfqpoint{0.000000in}{0.000000in}}%
\pgfpathlineto{\pgfqpoint{0.000000in}{-0.027778in}}%
\pgfusepath{stroke,fill}%
}%
\begin{pgfscope}%
\pgfsys@transformshift{0.985627in}{0.417642in}%
\pgfsys@useobject{currentmarker}{}%
\end{pgfscope}%
\end{pgfscope}%
\begin{pgfscope}%
\pgfpathrectangle{\pgfqpoint{0.594525in}{0.417642in}}{\pgfqpoint{3.432047in}{2.016277in}}%
\pgfusepath{clip}%
\pgfsetrectcap%
\pgfsetroundjoin%
\pgfsetlinewidth{0.803000pt}%
\definecolor{currentstroke}{rgb}{0.850000,0.850000,0.850000}%
\pgfsetstrokecolor{currentstroke}%
\pgfsetdash{}{0pt}%
\pgfpathmoveto{\pgfqpoint{1.123152in}{0.417642in}}%
\pgfpathlineto{\pgfqpoint{1.123152in}{2.433919in}}%
\pgfusepath{stroke}%
\end{pgfscope}%
\begin{pgfscope}%
\pgfsetbuttcap%
\pgfsetroundjoin%
\definecolor{currentfill}{rgb}{0.000000,0.000000,0.000000}%
\pgfsetfillcolor{currentfill}%
\pgfsetlinewidth{0.602250pt}%
\definecolor{currentstroke}{rgb}{0.000000,0.000000,0.000000}%
\pgfsetstrokecolor{currentstroke}%
\pgfsetdash{}{0pt}%
\pgfsys@defobject{currentmarker}{\pgfqpoint{0.000000in}{-0.027778in}}{\pgfqpoint{0.000000in}{0.000000in}}{%
\pgfpathmoveto{\pgfqpoint{0.000000in}{0.000000in}}%
\pgfpathlineto{\pgfqpoint{0.000000in}{-0.027778in}}%
\pgfusepath{stroke,fill}%
}%
\begin{pgfscope}%
\pgfsys@transformshift{1.123152in}{0.417642in}%
\pgfsys@useobject{currentmarker}{}%
\end{pgfscope}%
\end{pgfscope}%
\begin{pgfscope}%
\pgfpathrectangle{\pgfqpoint{0.594525in}{0.417642in}}{\pgfqpoint{3.432047in}{2.016277in}}%
\pgfusepath{clip}%
\pgfsetrectcap%
\pgfsetroundjoin%
\pgfsetlinewidth{0.803000pt}%
\definecolor{currentstroke}{rgb}{0.850000,0.850000,0.850000}%
\pgfsetstrokecolor{currentstroke}%
\pgfsetdash{}{0pt}%
\pgfpathmoveto{\pgfqpoint{1.220728in}{0.417642in}}%
\pgfpathlineto{\pgfqpoint{1.220728in}{2.433919in}}%
\pgfusepath{stroke}%
\end{pgfscope}%
\begin{pgfscope}%
\pgfsetbuttcap%
\pgfsetroundjoin%
\definecolor{currentfill}{rgb}{0.000000,0.000000,0.000000}%
\pgfsetfillcolor{currentfill}%
\pgfsetlinewidth{0.602250pt}%
\definecolor{currentstroke}{rgb}{0.000000,0.000000,0.000000}%
\pgfsetstrokecolor{currentstroke}%
\pgfsetdash{}{0pt}%
\pgfsys@defobject{currentmarker}{\pgfqpoint{0.000000in}{-0.027778in}}{\pgfqpoint{0.000000in}{0.000000in}}{%
\pgfpathmoveto{\pgfqpoint{0.000000in}{0.000000in}}%
\pgfpathlineto{\pgfqpoint{0.000000in}{-0.027778in}}%
\pgfusepath{stroke,fill}%
}%
\begin{pgfscope}%
\pgfsys@transformshift{1.220728in}{0.417642in}%
\pgfsys@useobject{currentmarker}{}%
\end{pgfscope}%
\end{pgfscope}%
\begin{pgfscope}%
\pgfpathrectangle{\pgfqpoint{0.594525in}{0.417642in}}{\pgfqpoint{3.432047in}{2.016277in}}%
\pgfusepath{clip}%
\pgfsetrectcap%
\pgfsetroundjoin%
\pgfsetlinewidth{0.803000pt}%
\definecolor{currentstroke}{rgb}{0.850000,0.850000,0.850000}%
\pgfsetstrokecolor{currentstroke}%
\pgfsetdash{}{0pt}%
\pgfpathmoveto{\pgfqpoint{1.296413in}{0.417642in}}%
\pgfpathlineto{\pgfqpoint{1.296413in}{2.433919in}}%
\pgfusepath{stroke}%
\end{pgfscope}%
\begin{pgfscope}%
\pgfsetbuttcap%
\pgfsetroundjoin%
\definecolor{currentfill}{rgb}{0.000000,0.000000,0.000000}%
\pgfsetfillcolor{currentfill}%
\pgfsetlinewidth{0.602250pt}%
\definecolor{currentstroke}{rgb}{0.000000,0.000000,0.000000}%
\pgfsetstrokecolor{currentstroke}%
\pgfsetdash{}{0pt}%
\pgfsys@defobject{currentmarker}{\pgfqpoint{0.000000in}{-0.027778in}}{\pgfqpoint{0.000000in}{0.000000in}}{%
\pgfpathmoveto{\pgfqpoint{0.000000in}{0.000000in}}%
\pgfpathlineto{\pgfqpoint{0.000000in}{-0.027778in}}%
\pgfusepath{stroke,fill}%
}%
\begin{pgfscope}%
\pgfsys@transformshift{1.296413in}{0.417642in}%
\pgfsys@useobject{currentmarker}{}%
\end{pgfscope}%
\end{pgfscope}%
\begin{pgfscope}%
\pgfpathrectangle{\pgfqpoint{0.594525in}{0.417642in}}{\pgfqpoint{3.432047in}{2.016277in}}%
\pgfusepath{clip}%
\pgfsetrectcap%
\pgfsetroundjoin%
\pgfsetlinewidth{0.803000pt}%
\definecolor{currentstroke}{rgb}{0.850000,0.850000,0.850000}%
\pgfsetstrokecolor{currentstroke}%
\pgfsetdash{}{0pt}%
\pgfpathmoveto{\pgfqpoint{1.358253in}{0.417642in}}%
\pgfpathlineto{\pgfqpoint{1.358253in}{2.433919in}}%
\pgfusepath{stroke}%
\end{pgfscope}%
\begin{pgfscope}%
\pgfsetbuttcap%
\pgfsetroundjoin%
\definecolor{currentfill}{rgb}{0.000000,0.000000,0.000000}%
\pgfsetfillcolor{currentfill}%
\pgfsetlinewidth{0.602250pt}%
\definecolor{currentstroke}{rgb}{0.000000,0.000000,0.000000}%
\pgfsetstrokecolor{currentstroke}%
\pgfsetdash{}{0pt}%
\pgfsys@defobject{currentmarker}{\pgfqpoint{0.000000in}{-0.027778in}}{\pgfqpoint{0.000000in}{0.000000in}}{%
\pgfpathmoveto{\pgfqpoint{0.000000in}{0.000000in}}%
\pgfpathlineto{\pgfqpoint{0.000000in}{-0.027778in}}%
\pgfusepath{stroke,fill}%
}%
\begin{pgfscope}%
\pgfsys@transformshift{1.358253in}{0.417642in}%
\pgfsys@useobject{currentmarker}{}%
\end{pgfscope}%
\end{pgfscope}%
\begin{pgfscope}%
\pgfpathrectangle{\pgfqpoint{0.594525in}{0.417642in}}{\pgfqpoint{3.432047in}{2.016277in}}%
\pgfusepath{clip}%
\pgfsetrectcap%
\pgfsetroundjoin%
\pgfsetlinewidth{0.803000pt}%
\definecolor{currentstroke}{rgb}{0.850000,0.850000,0.850000}%
\pgfsetstrokecolor{currentstroke}%
\pgfsetdash{}{0pt}%
\pgfpathmoveto{\pgfqpoint{1.410538in}{0.417642in}}%
\pgfpathlineto{\pgfqpoint{1.410538in}{2.433919in}}%
\pgfusepath{stroke}%
\end{pgfscope}%
\begin{pgfscope}%
\pgfsetbuttcap%
\pgfsetroundjoin%
\definecolor{currentfill}{rgb}{0.000000,0.000000,0.000000}%
\pgfsetfillcolor{currentfill}%
\pgfsetlinewidth{0.602250pt}%
\definecolor{currentstroke}{rgb}{0.000000,0.000000,0.000000}%
\pgfsetstrokecolor{currentstroke}%
\pgfsetdash{}{0pt}%
\pgfsys@defobject{currentmarker}{\pgfqpoint{0.000000in}{-0.027778in}}{\pgfqpoint{0.000000in}{0.000000in}}{%
\pgfpathmoveto{\pgfqpoint{0.000000in}{0.000000in}}%
\pgfpathlineto{\pgfqpoint{0.000000in}{-0.027778in}}%
\pgfusepath{stroke,fill}%
}%
\begin{pgfscope}%
\pgfsys@transformshift{1.410538in}{0.417642in}%
\pgfsys@useobject{currentmarker}{}%
\end{pgfscope}%
\end{pgfscope}%
\begin{pgfscope}%
\pgfpathrectangle{\pgfqpoint{0.594525in}{0.417642in}}{\pgfqpoint{3.432047in}{2.016277in}}%
\pgfusepath{clip}%
\pgfsetrectcap%
\pgfsetroundjoin%
\pgfsetlinewidth{0.803000pt}%
\definecolor{currentstroke}{rgb}{0.850000,0.850000,0.850000}%
\pgfsetstrokecolor{currentstroke}%
\pgfsetdash{}{0pt}%
\pgfpathmoveto{\pgfqpoint{1.455829in}{0.417642in}}%
\pgfpathlineto{\pgfqpoint{1.455829in}{2.433919in}}%
\pgfusepath{stroke}%
\end{pgfscope}%
\begin{pgfscope}%
\pgfsetbuttcap%
\pgfsetroundjoin%
\definecolor{currentfill}{rgb}{0.000000,0.000000,0.000000}%
\pgfsetfillcolor{currentfill}%
\pgfsetlinewidth{0.602250pt}%
\definecolor{currentstroke}{rgb}{0.000000,0.000000,0.000000}%
\pgfsetstrokecolor{currentstroke}%
\pgfsetdash{}{0pt}%
\pgfsys@defobject{currentmarker}{\pgfqpoint{0.000000in}{-0.027778in}}{\pgfqpoint{0.000000in}{0.000000in}}{%
\pgfpathmoveto{\pgfqpoint{0.000000in}{0.000000in}}%
\pgfpathlineto{\pgfqpoint{0.000000in}{-0.027778in}}%
\pgfusepath{stroke,fill}%
}%
\begin{pgfscope}%
\pgfsys@transformshift{1.455829in}{0.417642in}%
\pgfsys@useobject{currentmarker}{}%
\end{pgfscope}%
\end{pgfscope}%
\begin{pgfscope}%
\pgfpathrectangle{\pgfqpoint{0.594525in}{0.417642in}}{\pgfqpoint{3.432047in}{2.016277in}}%
\pgfusepath{clip}%
\pgfsetrectcap%
\pgfsetroundjoin%
\pgfsetlinewidth{0.803000pt}%
\definecolor{currentstroke}{rgb}{0.850000,0.850000,0.850000}%
\pgfsetstrokecolor{currentstroke}%
\pgfsetdash{}{0pt}%
\pgfpathmoveto{\pgfqpoint{1.495778in}{0.417642in}}%
\pgfpathlineto{\pgfqpoint{1.495778in}{2.433919in}}%
\pgfusepath{stroke}%
\end{pgfscope}%
\begin{pgfscope}%
\pgfsetbuttcap%
\pgfsetroundjoin%
\definecolor{currentfill}{rgb}{0.000000,0.000000,0.000000}%
\pgfsetfillcolor{currentfill}%
\pgfsetlinewidth{0.602250pt}%
\definecolor{currentstroke}{rgb}{0.000000,0.000000,0.000000}%
\pgfsetstrokecolor{currentstroke}%
\pgfsetdash{}{0pt}%
\pgfsys@defobject{currentmarker}{\pgfqpoint{0.000000in}{-0.027778in}}{\pgfqpoint{0.000000in}{0.000000in}}{%
\pgfpathmoveto{\pgfqpoint{0.000000in}{0.000000in}}%
\pgfpathlineto{\pgfqpoint{0.000000in}{-0.027778in}}%
\pgfusepath{stroke,fill}%
}%
\begin{pgfscope}%
\pgfsys@transformshift{1.495778in}{0.417642in}%
\pgfsys@useobject{currentmarker}{}%
\end{pgfscope}%
\end{pgfscope}%
\begin{pgfscope}%
\pgfpathrectangle{\pgfqpoint{0.594525in}{0.417642in}}{\pgfqpoint{3.432047in}{2.016277in}}%
\pgfusepath{clip}%
\pgfsetrectcap%
\pgfsetroundjoin%
\pgfsetlinewidth{0.803000pt}%
\definecolor{currentstroke}{rgb}{0.850000,0.850000,0.850000}%
\pgfsetstrokecolor{currentstroke}%
\pgfsetdash{}{0pt}%
\pgfpathmoveto{\pgfqpoint{1.766615in}{0.417642in}}%
\pgfpathlineto{\pgfqpoint{1.766615in}{2.433919in}}%
\pgfusepath{stroke}%
\end{pgfscope}%
\begin{pgfscope}%
\pgfsetbuttcap%
\pgfsetroundjoin%
\definecolor{currentfill}{rgb}{0.000000,0.000000,0.000000}%
\pgfsetfillcolor{currentfill}%
\pgfsetlinewidth{0.602250pt}%
\definecolor{currentstroke}{rgb}{0.000000,0.000000,0.000000}%
\pgfsetstrokecolor{currentstroke}%
\pgfsetdash{}{0pt}%
\pgfsys@defobject{currentmarker}{\pgfqpoint{0.000000in}{-0.027778in}}{\pgfqpoint{0.000000in}{0.000000in}}{%
\pgfpathmoveto{\pgfqpoint{0.000000in}{0.000000in}}%
\pgfpathlineto{\pgfqpoint{0.000000in}{-0.027778in}}%
\pgfusepath{stroke,fill}%
}%
\begin{pgfscope}%
\pgfsys@transformshift{1.766615in}{0.417642in}%
\pgfsys@useobject{currentmarker}{}%
\end{pgfscope}%
\end{pgfscope}%
\begin{pgfscope}%
\pgfpathrectangle{\pgfqpoint{0.594525in}{0.417642in}}{\pgfqpoint{3.432047in}{2.016277in}}%
\pgfusepath{clip}%
\pgfsetrectcap%
\pgfsetroundjoin%
\pgfsetlinewidth{0.803000pt}%
\definecolor{currentstroke}{rgb}{0.850000,0.850000,0.850000}%
\pgfsetstrokecolor{currentstroke}%
\pgfsetdash{}{0pt}%
\pgfpathmoveto{\pgfqpoint{1.904140in}{0.417642in}}%
\pgfpathlineto{\pgfqpoint{1.904140in}{2.433919in}}%
\pgfusepath{stroke}%
\end{pgfscope}%
\begin{pgfscope}%
\pgfsetbuttcap%
\pgfsetroundjoin%
\definecolor{currentfill}{rgb}{0.000000,0.000000,0.000000}%
\pgfsetfillcolor{currentfill}%
\pgfsetlinewidth{0.602250pt}%
\definecolor{currentstroke}{rgb}{0.000000,0.000000,0.000000}%
\pgfsetstrokecolor{currentstroke}%
\pgfsetdash{}{0pt}%
\pgfsys@defobject{currentmarker}{\pgfqpoint{0.000000in}{-0.027778in}}{\pgfqpoint{0.000000in}{0.000000in}}{%
\pgfpathmoveto{\pgfqpoint{0.000000in}{0.000000in}}%
\pgfpathlineto{\pgfqpoint{0.000000in}{-0.027778in}}%
\pgfusepath{stroke,fill}%
}%
\begin{pgfscope}%
\pgfsys@transformshift{1.904140in}{0.417642in}%
\pgfsys@useobject{currentmarker}{}%
\end{pgfscope}%
\end{pgfscope}%
\begin{pgfscope}%
\pgfpathrectangle{\pgfqpoint{0.594525in}{0.417642in}}{\pgfqpoint{3.432047in}{2.016277in}}%
\pgfusepath{clip}%
\pgfsetrectcap%
\pgfsetroundjoin%
\pgfsetlinewidth{0.803000pt}%
\definecolor{currentstroke}{rgb}{0.850000,0.850000,0.850000}%
\pgfsetstrokecolor{currentstroke}%
\pgfsetdash{}{0pt}%
\pgfpathmoveto{\pgfqpoint{2.001715in}{0.417642in}}%
\pgfpathlineto{\pgfqpoint{2.001715in}{2.433919in}}%
\pgfusepath{stroke}%
\end{pgfscope}%
\begin{pgfscope}%
\pgfsetbuttcap%
\pgfsetroundjoin%
\definecolor{currentfill}{rgb}{0.000000,0.000000,0.000000}%
\pgfsetfillcolor{currentfill}%
\pgfsetlinewidth{0.602250pt}%
\definecolor{currentstroke}{rgb}{0.000000,0.000000,0.000000}%
\pgfsetstrokecolor{currentstroke}%
\pgfsetdash{}{0pt}%
\pgfsys@defobject{currentmarker}{\pgfqpoint{0.000000in}{-0.027778in}}{\pgfqpoint{0.000000in}{0.000000in}}{%
\pgfpathmoveto{\pgfqpoint{0.000000in}{0.000000in}}%
\pgfpathlineto{\pgfqpoint{0.000000in}{-0.027778in}}%
\pgfusepath{stroke,fill}%
}%
\begin{pgfscope}%
\pgfsys@transformshift{2.001715in}{0.417642in}%
\pgfsys@useobject{currentmarker}{}%
\end{pgfscope}%
\end{pgfscope}%
\begin{pgfscope}%
\pgfpathrectangle{\pgfqpoint{0.594525in}{0.417642in}}{\pgfqpoint{3.432047in}{2.016277in}}%
\pgfusepath{clip}%
\pgfsetrectcap%
\pgfsetroundjoin%
\pgfsetlinewidth{0.803000pt}%
\definecolor{currentstroke}{rgb}{0.850000,0.850000,0.850000}%
\pgfsetstrokecolor{currentstroke}%
\pgfsetdash{}{0pt}%
\pgfpathmoveto{\pgfqpoint{2.077401in}{0.417642in}}%
\pgfpathlineto{\pgfqpoint{2.077401in}{2.433919in}}%
\pgfusepath{stroke}%
\end{pgfscope}%
\begin{pgfscope}%
\pgfsetbuttcap%
\pgfsetroundjoin%
\definecolor{currentfill}{rgb}{0.000000,0.000000,0.000000}%
\pgfsetfillcolor{currentfill}%
\pgfsetlinewidth{0.602250pt}%
\definecolor{currentstroke}{rgb}{0.000000,0.000000,0.000000}%
\pgfsetstrokecolor{currentstroke}%
\pgfsetdash{}{0pt}%
\pgfsys@defobject{currentmarker}{\pgfqpoint{0.000000in}{-0.027778in}}{\pgfqpoint{0.000000in}{0.000000in}}{%
\pgfpathmoveto{\pgfqpoint{0.000000in}{0.000000in}}%
\pgfpathlineto{\pgfqpoint{0.000000in}{-0.027778in}}%
\pgfusepath{stroke,fill}%
}%
\begin{pgfscope}%
\pgfsys@transformshift{2.077401in}{0.417642in}%
\pgfsys@useobject{currentmarker}{}%
\end{pgfscope}%
\end{pgfscope}%
\begin{pgfscope}%
\pgfpathrectangle{\pgfqpoint{0.594525in}{0.417642in}}{\pgfqpoint{3.432047in}{2.016277in}}%
\pgfusepath{clip}%
\pgfsetrectcap%
\pgfsetroundjoin%
\pgfsetlinewidth{0.803000pt}%
\definecolor{currentstroke}{rgb}{0.850000,0.850000,0.850000}%
\pgfsetstrokecolor{currentstroke}%
\pgfsetdash{}{0pt}%
\pgfpathmoveto{\pgfqpoint{2.139240in}{0.417642in}}%
\pgfpathlineto{\pgfqpoint{2.139240in}{2.433919in}}%
\pgfusepath{stroke}%
\end{pgfscope}%
\begin{pgfscope}%
\pgfsetbuttcap%
\pgfsetroundjoin%
\definecolor{currentfill}{rgb}{0.000000,0.000000,0.000000}%
\pgfsetfillcolor{currentfill}%
\pgfsetlinewidth{0.602250pt}%
\definecolor{currentstroke}{rgb}{0.000000,0.000000,0.000000}%
\pgfsetstrokecolor{currentstroke}%
\pgfsetdash{}{0pt}%
\pgfsys@defobject{currentmarker}{\pgfqpoint{0.000000in}{-0.027778in}}{\pgfqpoint{0.000000in}{0.000000in}}{%
\pgfpathmoveto{\pgfqpoint{0.000000in}{0.000000in}}%
\pgfpathlineto{\pgfqpoint{0.000000in}{-0.027778in}}%
\pgfusepath{stroke,fill}%
}%
\begin{pgfscope}%
\pgfsys@transformshift{2.139240in}{0.417642in}%
\pgfsys@useobject{currentmarker}{}%
\end{pgfscope}%
\end{pgfscope}%
\begin{pgfscope}%
\pgfpathrectangle{\pgfqpoint{0.594525in}{0.417642in}}{\pgfqpoint{3.432047in}{2.016277in}}%
\pgfusepath{clip}%
\pgfsetrectcap%
\pgfsetroundjoin%
\pgfsetlinewidth{0.803000pt}%
\definecolor{currentstroke}{rgb}{0.850000,0.850000,0.850000}%
\pgfsetstrokecolor{currentstroke}%
\pgfsetdash{}{0pt}%
\pgfpathmoveto{\pgfqpoint{2.191525in}{0.417642in}}%
\pgfpathlineto{\pgfqpoint{2.191525in}{2.433919in}}%
\pgfusepath{stroke}%
\end{pgfscope}%
\begin{pgfscope}%
\pgfsetbuttcap%
\pgfsetroundjoin%
\definecolor{currentfill}{rgb}{0.000000,0.000000,0.000000}%
\pgfsetfillcolor{currentfill}%
\pgfsetlinewidth{0.602250pt}%
\definecolor{currentstroke}{rgb}{0.000000,0.000000,0.000000}%
\pgfsetstrokecolor{currentstroke}%
\pgfsetdash{}{0pt}%
\pgfsys@defobject{currentmarker}{\pgfqpoint{0.000000in}{-0.027778in}}{\pgfqpoint{0.000000in}{0.000000in}}{%
\pgfpathmoveto{\pgfqpoint{0.000000in}{0.000000in}}%
\pgfpathlineto{\pgfqpoint{0.000000in}{-0.027778in}}%
\pgfusepath{stroke,fill}%
}%
\begin{pgfscope}%
\pgfsys@transformshift{2.191525in}{0.417642in}%
\pgfsys@useobject{currentmarker}{}%
\end{pgfscope}%
\end{pgfscope}%
\begin{pgfscope}%
\pgfpathrectangle{\pgfqpoint{0.594525in}{0.417642in}}{\pgfqpoint{3.432047in}{2.016277in}}%
\pgfusepath{clip}%
\pgfsetrectcap%
\pgfsetroundjoin%
\pgfsetlinewidth{0.803000pt}%
\definecolor{currentstroke}{rgb}{0.850000,0.850000,0.850000}%
\pgfsetstrokecolor{currentstroke}%
\pgfsetdash{}{0pt}%
\pgfpathmoveto{\pgfqpoint{2.236816in}{0.417642in}}%
\pgfpathlineto{\pgfqpoint{2.236816in}{2.433919in}}%
\pgfusepath{stroke}%
\end{pgfscope}%
\begin{pgfscope}%
\pgfsetbuttcap%
\pgfsetroundjoin%
\definecolor{currentfill}{rgb}{0.000000,0.000000,0.000000}%
\pgfsetfillcolor{currentfill}%
\pgfsetlinewidth{0.602250pt}%
\definecolor{currentstroke}{rgb}{0.000000,0.000000,0.000000}%
\pgfsetstrokecolor{currentstroke}%
\pgfsetdash{}{0pt}%
\pgfsys@defobject{currentmarker}{\pgfqpoint{0.000000in}{-0.027778in}}{\pgfqpoint{0.000000in}{0.000000in}}{%
\pgfpathmoveto{\pgfqpoint{0.000000in}{0.000000in}}%
\pgfpathlineto{\pgfqpoint{0.000000in}{-0.027778in}}%
\pgfusepath{stroke,fill}%
}%
\begin{pgfscope}%
\pgfsys@transformshift{2.236816in}{0.417642in}%
\pgfsys@useobject{currentmarker}{}%
\end{pgfscope}%
\end{pgfscope}%
\begin{pgfscope}%
\pgfpathrectangle{\pgfqpoint{0.594525in}{0.417642in}}{\pgfqpoint{3.432047in}{2.016277in}}%
\pgfusepath{clip}%
\pgfsetrectcap%
\pgfsetroundjoin%
\pgfsetlinewidth{0.803000pt}%
\definecolor{currentstroke}{rgb}{0.850000,0.850000,0.850000}%
\pgfsetstrokecolor{currentstroke}%
\pgfsetdash{}{0pt}%
\pgfpathmoveto{\pgfqpoint{2.276765in}{0.417642in}}%
\pgfpathlineto{\pgfqpoint{2.276765in}{2.433919in}}%
\pgfusepath{stroke}%
\end{pgfscope}%
\begin{pgfscope}%
\pgfsetbuttcap%
\pgfsetroundjoin%
\definecolor{currentfill}{rgb}{0.000000,0.000000,0.000000}%
\pgfsetfillcolor{currentfill}%
\pgfsetlinewidth{0.602250pt}%
\definecolor{currentstroke}{rgb}{0.000000,0.000000,0.000000}%
\pgfsetstrokecolor{currentstroke}%
\pgfsetdash{}{0pt}%
\pgfsys@defobject{currentmarker}{\pgfqpoint{0.000000in}{-0.027778in}}{\pgfqpoint{0.000000in}{0.000000in}}{%
\pgfpathmoveto{\pgfqpoint{0.000000in}{0.000000in}}%
\pgfpathlineto{\pgfqpoint{0.000000in}{-0.027778in}}%
\pgfusepath{stroke,fill}%
}%
\begin{pgfscope}%
\pgfsys@transformshift{2.276765in}{0.417642in}%
\pgfsys@useobject{currentmarker}{}%
\end{pgfscope}%
\end{pgfscope}%
\begin{pgfscope}%
\pgfpathrectangle{\pgfqpoint{0.594525in}{0.417642in}}{\pgfqpoint{3.432047in}{2.016277in}}%
\pgfusepath{clip}%
\pgfsetrectcap%
\pgfsetroundjoin%
\pgfsetlinewidth{0.803000pt}%
\definecolor{currentstroke}{rgb}{0.850000,0.850000,0.850000}%
\pgfsetstrokecolor{currentstroke}%
\pgfsetdash{}{0pt}%
\pgfpathmoveto{\pgfqpoint{2.547602in}{0.417642in}}%
\pgfpathlineto{\pgfqpoint{2.547602in}{2.433919in}}%
\pgfusepath{stroke}%
\end{pgfscope}%
\begin{pgfscope}%
\pgfsetbuttcap%
\pgfsetroundjoin%
\definecolor{currentfill}{rgb}{0.000000,0.000000,0.000000}%
\pgfsetfillcolor{currentfill}%
\pgfsetlinewidth{0.602250pt}%
\definecolor{currentstroke}{rgb}{0.000000,0.000000,0.000000}%
\pgfsetstrokecolor{currentstroke}%
\pgfsetdash{}{0pt}%
\pgfsys@defobject{currentmarker}{\pgfqpoint{0.000000in}{-0.027778in}}{\pgfqpoint{0.000000in}{0.000000in}}{%
\pgfpathmoveto{\pgfqpoint{0.000000in}{0.000000in}}%
\pgfpathlineto{\pgfqpoint{0.000000in}{-0.027778in}}%
\pgfusepath{stroke,fill}%
}%
\begin{pgfscope}%
\pgfsys@transformshift{2.547602in}{0.417642in}%
\pgfsys@useobject{currentmarker}{}%
\end{pgfscope}%
\end{pgfscope}%
\begin{pgfscope}%
\pgfpathrectangle{\pgfqpoint{0.594525in}{0.417642in}}{\pgfqpoint{3.432047in}{2.016277in}}%
\pgfusepath{clip}%
\pgfsetrectcap%
\pgfsetroundjoin%
\pgfsetlinewidth{0.803000pt}%
\definecolor{currentstroke}{rgb}{0.850000,0.850000,0.850000}%
\pgfsetstrokecolor{currentstroke}%
\pgfsetdash{}{0pt}%
\pgfpathmoveto{\pgfqpoint{2.685127in}{0.417642in}}%
\pgfpathlineto{\pgfqpoint{2.685127in}{2.433919in}}%
\pgfusepath{stroke}%
\end{pgfscope}%
\begin{pgfscope}%
\pgfsetbuttcap%
\pgfsetroundjoin%
\definecolor{currentfill}{rgb}{0.000000,0.000000,0.000000}%
\pgfsetfillcolor{currentfill}%
\pgfsetlinewidth{0.602250pt}%
\definecolor{currentstroke}{rgb}{0.000000,0.000000,0.000000}%
\pgfsetstrokecolor{currentstroke}%
\pgfsetdash{}{0pt}%
\pgfsys@defobject{currentmarker}{\pgfqpoint{0.000000in}{-0.027778in}}{\pgfqpoint{0.000000in}{0.000000in}}{%
\pgfpathmoveto{\pgfqpoint{0.000000in}{0.000000in}}%
\pgfpathlineto{\pgfqpoint{0.000000in}{-0.027778in}}%
\pgfusepath{stroke,fill}%
}%
\begin{pgfscope}%
\pgfsys@transformshift{2.685127in}{0.417642in}%
\pgfsys@useobject{currentmarker}{}%
\end{pgfscope}%
\end{pgfscope}%
\begin{pgfscope}%
\pgfpathrectangle{\pgfqpoint{0.594525in}{0.417642in}}{\pgfqpoint{3.432047in}{2.016277in}}%
\pgfusepath{clip}%
\pgfsetrectcap%
\pgfsetroundjoin%
\pgfsetlinewidth{0.803000pt}%
\definecolor{currentstroke}{rgb}{0.850000,0.850000,0.850000}%
\pgfsetstrokecolor{currentstroke}%
\pgfsetdash{}{0pt}%
\pgfpathmoveto{\pgfqpoint{2.782702in}{0.417642in}}%
\pgfpathlineto{\pgfqpoint{2.782702in}{2.433919in}}%
\pgfusepath{stroke}%
\end{pgfscope}%
\begin{pgfscope}%
\pgfsetbuttcap%
\pgfsetroundjoin%
\definecolor{currentfill}{rgb}{0.000000,0.000000,0.000000}%
\pgfsetfillcolor{currentfill}%
\pgfsetlinewidth{0.602250pt}%
\definecolor{currentstroke}{rgb}{0.000000,0.000000,0.000000}%
\pgfsetstrokecolor{currentstroke}%
\pgfsetdash{}{0pt}%
\pgfsys@defobject{currentmarker}{\pgfqpoint{0.000000in}{-0.027778in}}{\pgfqpoint{0.000000in}{0.000000in}}{%
\pgfpathmoveto{\pgfqpoint{0.000000in}{0.000000in}}%
\pgfpathlineto{\pgfqpoint{0.000000in}{-0.027778in}}%
\pgfusepath{stroke,fill}%
}%
\begin{pgfscope}%
\pgfsys@transformshift{2.782702in}{0.417642in}%
\pgfsys@useobject{currentmarker}{}%
\end{pgfscope}%
\end{pgfscope}%
\begin{pgfscope}%
\pgfpathrectangle{\pgfqpoint{0.594525in}{0.417642in}}{\pgfqpoint{3.432047in}{2.016277in}}%
\pgfusepath{clip}%
\pgfsetrectcap%
\pgfsetroundjoin%
\pgfsetlinewidth{0.803000pt}%
\definecolor{currentstroke}{rgb}{0.850000,0.850000,0.850000}%
\pgfsetstrokecolor{currentstroke}%
\pgfsetdash{}{0pt}%
\pgfpathmoveto{\pgfqpoint{2.858388in}{0.417642in}}%
\pgfpathlineto{\pgfqpoint{2.858388in}{2.433919in}}%
\pgfusepath{stroke}%
\end{pgfscope}%
\begin{pgfscope}%
\pgfsetbuttcap%
\pgfsetroundjoin%
\definecolor{currentfill}{rgb}{0.000000,0.000000,0.000000}%
\pgfsetfillcolor{currentfill}%
\pgfsetlinewidth{0.602250pt}%
\definecolor{currentstroke}{rgb}{0.000000,0.000000,0.000000}%
\pgfsetstrokecolor{currentstroke}%
\pgfsetdash{}{0pt}%
\pgfsys@defobject{currentmarker}{\pgfqpoint{0.000000in}{-0.027778in}}{\pgfqpoint{0.000000in}{0.000000in}}{%
\pgfpathmoveto{\pgfqpoint{0.000000in}{0.000000in}}%
\pgfpathlineto{\pgfqpoint{0.000000in}{-0.027778in}}%
\pgfusepath{stroke,fill}%
}%
\begin{pgfscope}%
\pgfsys@transformshift{2.858388in}{0.417642in}%
\pgfsys@useobject{currentmarker}{}%
\end{pgfscope}%
\end{pgfscope}%
\begin{pgfscope}%
\pgfpathrectangle{\pgfqpoint{0.594525in}{0.417642in}}{\pgfqpoint{3.432047in}{2.016277in}}%
\pgfusepath{clip}%
\pgfsetrectcap%
\pgfsetroundjoin%
\pgfsetlinewidth{0.803000pt}%
\definecolor{currentstroke}{rgb}{0.850000,0.850000,0.850000}%
\pgfsetstrokecolor{currentstroke}%
\pgfsetdash{}{0pt}%
\pgfpathmoveto{\pgfqpoint{2.920227in}{0.417642in}}%
\pgfpathlineto{\pgfqpoint{2.920227in}{2.433919in}}%
\pgfusepath{stroke}%
\end{pgfscope}%
\begin{pgfscope}%
\pgfsetbuttcap%
\pgfsetroundjoin%
\definecolor{currentfill}{rgb}{0.000000,0.000000,0.000000}%
\pgfsetfillcolor{currentfill}%
\pgfsetlinewidth{0.602250pt}%
\definecolor{currentstroke}{rgb}{0.000000,0.000000,0.000000}%
\pgfsetstrokecolor{currentstroke}%
\pgfsetdash{}{0pt}%
\pgfsys@defobject{currentmarker}{\pgfqpoint{0.000000in}{-0.027778in}}{\pgfqpoint{0.000000in}{0.000000in}}{%
\pgfpathmoveto{\pgfqpoint{0.000000in}{0.000000in}}%
\pgfpathlineto{\pgfqpoint{0.000000in}{-0.027778in}}%
\pgfusepath{stroke,fill}%
}%
\begin{pgfscope}%
\pgfsys@transformshift{2.920227in}{0.417642in}%
\pgfsys@useobject{currentmarker}{}%
\end{pgfscope}%
\end{pgfscope}%
\begin{pgfscope}%
\pgfpathrectangle{\pgfqpoint{0.594525in}{0.417642in}}{\pgfqpoint{3.432047in}{2.016277in}}%
\pgfusepath{clip}%
\pgfsetrectcap%
\pgfsetroundjoin%
\pgfsetlinewidth{0.803000pt}%
\definecolor{currentstroke}{rgb}{0.850000,0.850000,0.850000}%
\pgfsetstrokecolor{currentstroke}%
\pgfsetdash{}{0pt}%
\pgfpathmoveto{\pgfqpoint{2.972512in}{0.417642in}}%
\pgfpathlineto{\pgfqpoint{2.972512in}{2.433919in}}%
\pgfusepath{stroke}%
\end{pgfscope}%
\begin{pgfscope}%
\pgfsetbuttcap%
\pgfsetroundjoin%
\definecolor{currentfill}{rgb}{0.000000,0.000000,0.000000}%
\pgfsetfillcolor{currentfill}%
\pgfsetlinewidth{0.602250pt}%
\definecolor{currentstroke}{rgb}{0.000000,0.000000,0.000000}%
\pgfsetstrokecolor{currentstroke}%
\pgfsetdash{}{0pt}%
\pgfsys@defobject{currentmarker}{\pgfqpoint{0.000000in}{-0.027778in}}{\pgfqpoint{0.000000in}{0.000000in}}{%
\pgfpathmoveto{\pgfqpoint{0.000000in}{0.000000in}}%
\pgfpathlineto{\pgfqpoint{0.000000in}{-0.027778in}}%
\pgfusepath{stroke,fill}%
}%
\begin{pgfscope}%
\pgfsys@transformshift{2.972512in}{0.417642in}%
\pgfsys@useobject{currentmarker}{}%
\end{pgfscope}%
\end{pgfscope}%
\begin{pgfscope}%
\pgfpathrectangle{\pgfqpoint{0.594525in}{0.417642in}}{\pgfqpoint{3.432047in}{2.016277in}}%
\pgfusepath{clip}%
\pgfsetrectcap%
\pgfsetroundjoin%
\pgfsetlinewidth{0.803000pt}%
\definecolor{currentstroke}{rgb}{0.850000,0.850000,0.850000}%
\pgfsetstrokecolor{currentstroke}%
\pgfsetdash{}{0pt}%
\pgfpathmoveto{\pgfqpoint{3.017803in}{0.417642in}}%
\pgfpathlineto{\pgfqpoint{3.017803in}{2.433919in}}%
\pgfusepath{stroke}%
\end{pgfscope}%
\begin{pgfscope}%
\pgfsetbuttcap%
\pgfsetroundjoin%
\definecolor{currentfill}{rgb}{0.000000,0.000000,0.000000}%
\pgfsetfillcolor{currentfill}%
\pgfsetlinewidth{0.602250pt}%
\definecolor{currentstroke}{rgb}{0.000000,0.000000,0.000000}%
\pgfsetstrokecolor{currentstroke}%
\pgfsetdash{}{0pt}%
\pgfsys@defobject{currentmarker}{\pgfqpoint{0.000000in}{-0.027778in}}{\pgfqpoint{0.000000in}{0.000000in}}{%
\pgfpathmoveto{\pgfqpoint{0.000000in}{0.000000in}}%
\pgfpathlineto{\pgfqpoint{0.000000in}{-0.027778in}}%
\pgfusepath{stroke,fill}%
}%
\begin{pgfscope}%
\pgfsys@transformshift{3.017803in}{0.417642in}%
\pgfsys@useobject{currentmarker}{}%
\end{pgfscope}%
\end{pgfscope}%
\begin{pgfscope}%
\pgfpathrectangle{\pgfqpoint{0.594525in}{0.417642in}}{\pgfqpoint{3.432047in}{2.016277in}}%
\pgfusepath{clip}%
\pgfsetrectcap%
\pgfsetroundjoin%
\pgfsetlinewidth{0.803000pt}%
\definecolor{currentstroke}{rgb}{0.850000,0.850000,0.850000}%
\pgfsetstrokecolor{currentstroke}%
\pgfsetdash{}{0pt}%
\pgfpathmoveto{\pgfqpoint{3.057752in}{0.417642in}}%
\pgfpathlineto{\pgfqpoint{3.057752in}{2.433919in}}%
\pgfusepath{stroke}%
\end{pgfscope}%
\begin{pgfscope}%
\pgfsetbuttcap%
\pgfsetroundjoin%
\definecolor{currentfill}{rgb}{0.000000,0.000000,0.000000}%
\pgfsetfillcolor{currentfill}%
\pgfsetlinewidth{0.602250pt}%
\definecolor{currentstroke}{rgb}{0.000000,0.000000,0.000000}%
\pgfsetstrokecolor{currentstroke}%
\pgfsetdash{}{0pt}%
\pgfsys@defobject{currentmarker}{\pgfqpoint{0.000000in}{-0.027778in}}{\pgfqpoint{0.000000in}{0.000000in}}{%
\pgfpathmoveto{\pgfqpoint{0.000000in}{0.000000in}}%
\pgfpathlineto{\pgfqpoint{0.000000in}{-0.027778in}}%
\pgfusepath{stroke,fill}%
}%
\begin{pgfscope}%
\pgfsys@transformshift{3.057752in}{0.417642in}%
\pgfsys@useobject{currentmarker}{}%
\end{pgfscope}%
\end{pgfscope}%
\begin{pgfscope}%
\pgfpathrectangle{\pgfqpoint{0.594525in}{0.417642in}}{\pgfqpoint{3.432047in}{2.016277in}}%
\pgfusepath{clip}%
\pgfsetrectcap%
\pgfsetroundjoin%
\pgfsetlinewidth{0.803000pt}%
\definecolor{currentstroke}{rgb}{0.850000,0.850000,0.850000}%
\pgfsetstrokecolor{currentstroke}%
\pgfsetdash{}{0pt}%
\pgfpathmoveto{\pgfqpoint{3.328589in}{0.417642in}}%
\pgfpathlineto{\pgfqpoint{3.328589in}{2.433919in}}%
\pgfusepath{stroke}%
\end{pgfscope}%
\begin{pgfscope}%
\pgfsetbuttcap%
\pgfsetroundjoin%
\definecolor{currentfill}{rgb}{0.000000,0.000000,0.000000}%
\pgfsetfillcolor{currentfill}%
\pgfsetlinewidth{0.602250pt}%
\definecolor{currentstroke}{rgb}{0.000000,0.000000,0.000000}%
\pgfsetstrokecolor{currentstroke}%
\pgfsetdash{}{0pt}%
\pgfsys@defobject{currentmarker}{\pgfqpoint{0.000000in}{-0.027778in}}{\pgfqpoint{0.000000in}{0.000000in}}{%
\pgfpathmoveto{\pgfqpoint{0.000000in}{0.000000in}}%
\pgfpathlineto{\pgfqpoint{0.000000in}{-0.027778in}}%
\pgfusepath{stroke,fill}%
}%
\begin{pgfscope}%
\pgfsys@transformshift{3.328589in}{0.417642in}%
\pgfsys@useobject{currentmarker}{}%
\end{pgfscope}%
\end{pgfscope}%
\begin{pgfscope}%
\pgfpathrectangle{\pgfqpoint{0.594525in}{0.417642in}}{\pgfqpoint{3.432047in}{2.016277in}}%
\pgfusepath{clip}%
\pgfsetrectcap%
\pgfsetroundjoin%
\pgfsetlinewidth{0.803000pt}%
\definecolor{currentstroke}{rgb}{0.850000,0.850000,0.850000}%
\pgfsetstrokecolor{currentstroke}%
\pgfsetdash{}{0pt}%
\pgfpathmoveto{\pgfqpoint{3.466114in}{0.417642in}}%
\pgfpathlineto{\pgfqpoint{3.466114in}{2.433919in}}%
\pgfusepath{stroke}%
\end{pgfscope}%
\begin{pgfscope}%
\pgfsetbuttcap%
\pgfsetroundjoin%
\definecolor{currentfill}{rgb}{0.000000,0.000000,0.000000}%
\pgfsetfillcolor{currentfill}%
\pgfsetlinewidth{0.602250pt}%
\definecolor{currentstroke}{rgb}{0.000000,0.000000,0.000000}%
\pgfsetstrokecolor{currentstroke}%
\pgfsetdash{}{0pt}%
\pgfsys@defobject{currentmarker}{\pgfqpoint{0.000000in}{-0.027778in}}{\pgfqpoint{0.000000in}{0.000000in}}{%
\pgfpathmoveto{\pgfqpoint{0.000000in}{0.000000in}}%
\pgfpathlineto{\pgfqpoint{0.000000in}{-0.027778in}}%
\pgfusepath{stroke,fill}%
}%
\begin{pgfscope}%
\pgfsys@transformshift{3.466114in}{0.417642in}%
\pgfsys@useobject{currentmarker}{}%
\end{pgfscope}%
\end{pgfscope}%
\begin{pgfscope}%
\pgfpathrectangle{\pgfqpoint{0.594525in}{0.417642in}}{\pgfqpoint{3.432047in}{2.016277in}}%
\pgfusepath{clip}%
\pgfsetrectcap%
\pgfsetroundjoin%
\pgfsetlinewidth{0.803000pt}%
\definecolor{currentstroke}{rgb}{0.850000,0.850000,0.850000}%
\pgfsetstrokecolor{currentstroke}%
\pgfsetdash{}{0pt}%
\pgfpathmoveto{\pgfqpoint{3.563689in}{0.417642in}}%
\pgfpathlineto{\pgfqpoint{3.563689in}{2.433919in}}%
\pgfusepath{stroke}%
\end{pgfscope}%
\begin{pgfscope}%
\pgfsetbuttcap%
\pgfsetroundjoin%
\definecolor{currentfill}{rgb}{0.000000,0.000000,0.000000}%
\pgfsetfillcolor{currentfill}%
\pgfsetlinewidth{0.602250pt}%
\definecolor{currentstroke}{rgb}{0.000000,0.000000,0.000000}%
\pgfsetstrokecolor{currentstroke}%
\pgfsetdash{}{0pt}%
\pgfsys@defobject{currentmarker}{\pgfqpoint{0.000000in}{-0.027778in}}{\pgfqpoint{0.000000in}{0.000000in}}{%
\pgfpathmoveto{\pgfqpoint{0.000000in}{0.000000in}}%
\pgfpathlineto{\pgfqpoint{0.000000in}{-0.027778in}}%
\pgfusepath{stroke,fill}%
}%
\begin{pgfscope}%
\pgfsys@transformshift{3.563689in}{0.417642in}%
\pgfsys@useobject{currentmarker}{}%
\end{pgfscope}%
\end{pgfscope}%
\begin{pgfscope}%
\pgfpathrectangle{\pgfqpoint{0.594525in}{0.417642in}}{\pgfqpoint{3.432047in}{2.016277in}}%
\pgfusepath{clip}%
\pgfsetrectcap%
\pgfsetroundjoin%
\pgfsetlinewidth{0.803000pt}%
\definecolor{currentstroke}{rgb}{0.850000,0.850000,0.850000}%
\pgfsetstrokecolor{currentstroke}%
\pgfsetdash{}{0pt}%
\pgfpathmoveto{\pgfqpoint{3.639375in}{0.417642in}}%
\pgfpathlineto{\pgfqpoint{3.639375in}{2.433919in}}%
\pgfusepath{stroke}%
\end{pgfscope}%
\begin{pgfscope}%
\pgfsetbuttcap%
\pgfsetroundjoin%
\definecolor{currentfill}{rgb}{0.000000,0.000000,0.000000}%
\pgfsetfillcolor{currentfill}%
\pgfsetlinewidth{0.602250pt}%
\definecolor{currentstroke}{rgb}{0.000000,0.000000,0.000000}%
\pgfsetstrokecolor{currentstroke}%
\pgfsetdash{}{0pt}%
\pgfsys@defobject{currentmarker}{\pgfqpoint{0.000000in}{-0.027778in}}{\pgfqpoint{0.000000in}{0.000000in}}{%
\pgfpathmoveto{\pgfqpoint{0.000000in}{0.000000in}}%
\pgfpathlineto{\pgfqpoint{0.000000in}{-0.027778in}}%
\pgfusepath{stroke,fill}%
}%
\begin{pgfscope}%
\pgfsys@transformshift{3.639375in}{0.417642in}%
\pgfsys@useobject{currentmarker}{}%
\end{pgfscope}%
\end{pgfscope}%
\begin{pgfscope}%
\pgfpathrectangle{\pgfqpoint{0.594525in}{0.417642in}}{\pgfqpoint{3.432047in}{2.016277in}}%
\pgfusepath{clip}%
\pgfsetrectcap%
\pgfsetroundjoin%
\pgfsetlinewidth{0.803000pt}%
\definecolor{currentstroke}{rgb}{0.850000,0.850000,0.850000}%
\pgfsetstrokecolor{currentstroke}%
\pgfsetdash{}{0pt}%
\pgfpathmoveto{\pgfqpoint{3.701214in}{0.417642in}}%
\pgfpathlineto{\pgfqpoint{3.701214in}{2.433919in}}%
\pgfusepath{stroke}%
\end{pgfscope}%
\begin{pgfscope}%
\pgfsetbuttcap%
\pgfsetroundjoin%
\definecolor{currentfill}{rgb}{0.000000,0.000000,0.000000}%
\pgfsetfillcolor{currentfill}%
\pgfsetlinewidth{0.602250pt}%
\definecolor{currentstroke}{rgb}{0.000000,0.000000,0.000000}%
\pgfsetstrokecolor{currentstroke}%
\pgfsetdash{}{0pt}%
\pgfsys@defobject{currentmarker}{\pgfqpoint{0.000000in}{-0.027778in}}{\pgfqpoint{0.000000in}{0.000000in}}{%
\pgfpathmoveto{\pgfqpoint{0.000000in}{0.000000in}}%
\pgfpathlineto{\pgfqpoint{0.000000in}{-0.027778in}}%
\pgfusepath{stroke,fill}%
}%
\begin{pgfscope}%
\pgfsys@transformshift{3.701214in}{0.417642in}%
\pgfsys@useobject{currentmarker}{}%
\end{pgfscope}%
\end{pgfscope}%
\begin{pgfscope}%
\pgfpathrectangle{\pgfqpoint{0.594525in}{0.417642in}}{\pgfqpoint{3.432047in}{2.016277in}}%
\pgfusepath{clip}%
\pgfsetrectcap%
\pgfsetroundjoin%
\pgfsetlinewidth{0.803000pt}%
\definecolor{currentstroke}{rgb}{0.850000,0.850000,0.850000}%
\pgfsetstrokecolor{currentstroke}%
\pgfsetdash{}{0pt}%
\pgfpathmoveto{\pgfqpoint{3.753499in}{0.417642in}}%
\pgfpathlineto{\pgfqpoint{3.753499in}{2.433919in}}%
\pgfusepath{stroke}%
\end{pgfscope}%
\begin{pgfscope}%
\pgfsetbuttcap%
\pgfsetroundjoin%
\definecolor{currentfill}{rgb}{0.000000,0.000000,0.000000}%
\pgfsetfillcolor{currentfill}%
\pgfsetlinewidth{0.602250pt}%
\definecolor{currentstroke}{rgb}{0.000000,0.000000,0.000000}%
\pgfsetstrokecolor{currentstroke}%
\pgfsetdash{}{0pt}%
\pgfsys@defobject{currentmarker}{\pgfqpoint{0.000000in}{-0.027778in}}{\pgfqpoint{0.000000in}{0.000000in}}{%
\pgfpathmoveto{\pgfqpoint{0.000000in}{0.000000in}}%
\pgfpathlineto{\pgfqpoint{0.000000in}{-0.027778in}}%
\pgfusepath{stroke,fill}%
}%
\begin{pgfscope}%
\pgfsys@transformshift{3.753499in}{0.417642in}%
\pgfsys@useobject{currentmarker}{}%
\end{pgfscope}%
\end{pgfscope}%
\begin{pgfscope}%
\pgfpathrectangle{\pgfqpoint{0.594525in}{0.417642in}}{\pgfqpoint{3.432047in}{2.016277in}}%
\pgfusepath{clip}%
\pgfsetrectcap%
\pgfsetroundjoin%
\pgfsetlinewidth{0.803000pt}%
\definecolor{currentstroke}{rgb}{0.850000,0.850000,0.850000}%
\pgfsetstrokecolor{currentstroke}%
\pgfsetdash{}{0pt}%
\pgfpathmoveto{\pgfqpoint{3.798790in}{0.417642in}}%
\pgfpathlineto{\pgfqpoint{3.798790in}{2.433919in}}%
\pgfusepath{stroke}%
\end{pgfscope}%
\begin{pgfscope}%
\pgfsetbuttcap%
\pgfsetroundjoin%
\definecolor{currentfill}{rgb}{0.000000,0.000000,0.000000}%
\pgfsetfillcolor{currentfill}%
\pgfsetlinewidth{0.602250pt}%
\definecolor{currentstroke}{rgb}{0.000000,0.000000,0.000000}%
\pgfsetstrokecolor{currentstroke}%
\pgfsetdash{}{0pt}%
\pgfsys@defobject{currentmarker}{\pgfqpoint{0.000000in}{-0.027778in}}{\pgfqpoint{0.000000in}{0.000000in}}{%
\pgfpathmoveto{\pgfqpoint{0.000000in}{0.000000in}}%
\pgfpathlineto{\pgfqpoint{0.000000in}{-0.027778in}}%
\pgfusepath{stroke,fill}%
}%
\begin{pgfscope}%
\pgfsys@transformshift{3.798790in}{0.417642in}%
\pgfsys@useobject{currentmarker}{}%
\end{pgfscope}%
\end{pgfscope}%
\begin{pgfscope}%
\pgfpathrectangle{\pgfqpoint{0.594525in}{0.417642in}}{\pgfqpoint{3.432047in}{2.016277in}}%
\pgfusepath{clip}%
\pgfsetrectcap%
\pgfsetroundjoin%
\pgfsetlinewidth{0.803000pt}%
\definecolor{currentstroke}{rgb}{0.850000,0.850000,0.850000}%
\pgfsetstrokecolor{currentstroke}%
\pgfsetdash{}{0pt}%
\pgfpathmoveto{\pgfqpoint{3.838739in}{0.417642in}}%
\pgfpathlineto{\pgfqpoint{3.838739in}{2.433919in}}%
\pgfusepath{stroke}%
\end{pgfscope}%
\begin{pgfscope}%
\pgfsetbuttcap%
\pgfsetroundjoin%
\definecolor{currentfill}{rgb}{0.000000,0.000000,0.000000}%
\pgfsetfillcolor{currentfill}%
\pgfsetlinewidth{0.602250pt}%
\definecolor{currentstroke}{rgb}{0.000000,0.000000,0.000000}%
\pgfsetstrokecolor{currentstroke}%
\pgfsetdash{}{0pt}%
\pgfsys@defobject{currentmarker}{\pgfqpoint{0.000000in}{-0.027778in}}{\pgfqpoint{0.000000in}{0.000000in}}{%
\pgfpathmoveto{\pgfqpoint{0.000000in}{0.000000in}}%
\pgfpathlineto{\pgfqpoint{0.000000in}{-0.027778in}}%
\pgfusepath{stroke,fill}%
}%
\begin{pgfscope}%
\pgfsys@transformshift{3.838739in}{0.417642in}%
\pgfsys@useobject{currentmarker}{}%
\end{pgfscope}%
\end{pgfscope}%
\begin{pgfscope}%
\definecolor{textcolor}{rgb}{0.000000,0.000000,0.000000}%
\pgfsetstrokecolor{textcolor}%
\pgfsetfillcolor{textcolor}%
\pgftext[x=2.310548in,y=0.165003in,,top]{\color{textcolor}\rmfamily\fontsize{10.000000}{12.000000}\selectfont Frequency in \(\displaystyle \unit{\Hz}\)}%
\end{pgfscope}%
\begin{pgfscope}%
\pgfpathrectangle{\pgfqpoint{0.594525in}{0.417642in}}{\pgfqpoint{3.432047in}{2.016277in}}%
\pgfusepath{clip}%
\pgfsetrectcap%
\pgfsetroundjoin%
\pgfsetlinewidth{0.803000pt}%
\definecolor{currentstroke}{rgb}{0.450000,0.450000,0.450000}%
\pgfsetstrokecolor{currentstroke}%
\pgfsetdash{}{0pt}%
\pgfpathmoveto{\pgfqpoint{0.594525in}{0.517495in}}%
\pgfpathlineto{\pgfqpoint{4.026572in}{0.517495in}}%
\pgfusepath{stroke}%
\end{pgfscope}%
\begin{pgfscope}%
\pgfsetbuttcap%
\pgfsetroundjoin%
\definecolor{currentfill}{rgb}{0.000000,0.000000,0.000000}%
\pgfsetfillcolor{currentfill}%
\pgfsetlinewidth{0.803000pt}%
\definecolor{currentstroke}{rgb}{0.000000,0.000000,0.000000}%
\pgfsetstrokecolor{currentstroke}%
\pgfsetdash{}{0pt}%
\pgfsys@defobject{currentmarker}{\pgfqpoint{-0.048611in}{0.000000in}}{\pgfqpoint{-0.000000in}{0.000000in}}{%
\pgfpathmoveto{\pgfqpoint{-0.000000in}{0.000000in}}%
\pgfpathlineto{\pgfqpoint{-0.048611in}{0.000000in}}%
\pgfusepath{stroke,fill}%
}%
\begin{pgfscope}%
\pgfsys@transformshift{0.594525in}{0.517495in}%
\pgfsys@useobject{currentmarker}{}%
\end{pgfscope}%
\end{pgfscope}%
\begin{pgfscope}%
\definecolor{textcolor}{rgb}{0.000000,0.000000,0.000000}%
\pgfsetstrokecolor{textcolor}%
\pgfsetfillcolor{textcolor}%
\pgftext[x=0.241129in, y=0.478342in, left, base]{\color{textcolor}\rmfamily\fontsize{8.000000}{9.600000}\selectfont \(\displaystyle {10^{-6}}\)}%
\end{pgfscope}%
\begin{pgfscope}%
\pgfpathrectangle{\pgfqpoint{0.594525in}{0.417642in}}{\pgfqpoint{3.432047in}{2.016277in}}%
\pgfusepath{clip}%
\pgfsetrectcap%
\pgfsetroundjoin%
\pgfsetlinewidth{0.803000pt}%
\definecolor{currentstroke}{rgb}{0.450000,0.450000,0.450000}%
\pgfsetstrokecolor{currentstroke}%
\pgfsetdash{}{0pt}%
\pgfpathmoveto{\pgfqpoint{0.594525in}{0.836827in}}%
\pgfpathlineto{\pgfqpoint{4.026572in}{0.836827in}}%
\pgfusepath{stroke}%
\end{pgfscope}%
\begin{pgfscope}%
\pgfsetbuttcap%
\pgfsetroundjoin%
\definecolor{currentfill}{rgb}{0.000000,0.000000,0.000000}%
\pgfsetfillcolor{currentfill}%
\pgfsetlinewidth{0.803000pt}%
\definecolor{currentstroke}{rgb}{0.000000,0.000000,0.000000}%
\pgfsetstrokecolor{currentstroke}%
\pgfsetdash{}{0pt}%
\pgfsys@defobject{currentmarker}{\pgfqpoint{-0.048611in}{0.000000in}}{\pgfqpoint{-0.000000in}{0.000000in}}{%
\pgfpathmoveto{\pgfqpoint{-0.000000in}{0.000000in}}%
\pgfpathlineto{\pgfqpoint{-0.048611in}{0.000000in}}%
\pgfusepath{stroke,fill}%
}%
\begin{pgfscope}%
\pgfsys@transformshift{0.594525in}{0.836827in}%
\pgfsys@useobject{currentmarker}{}%
\end{pgfscope}%
\end{pgfscope}%
\begin{pgfscope}%
\definecolor{textcolor}{rgb}{0.000000,0.000000,0.000000}%
\pgfsetstrokecolor{textcolor}%
\pgfsetfillcolor{textcolor}%
\pgftext[x=0.241129in, y=0.797674in, left, base]{\color{textcolor}\rmfamily\fontsize{8.000000}{9.600000}\selectfont \(\displaystyle {10^{-5}}\)}%
\end{pgfscope}%
\begin{pgfscope}%
\pgfpathrectangle{\pgfqpoint{0.594525in}{0.417642in}}{\pgfqpoint{3.432047in}{2.016277in}}%
\pgfusepath{clip}%
\pgfsetrectcap%
\pgfsetroundjoin%
\pgfsetlinewidth{0.803000pt}%
\definecolor{currentstroke}{rgb}{0.450000,0.450000,0.450000}%
\pgfsetstrokecolor{currentstroke}%
\pgfsetdash{}{0pt}%
\pgfpathmoveto{\pgfqpoint{0.594525in}{1.156160in}}%
\pgfpathlineto{\pgfqpoint{4.026572in}{1.156160in}}%
\pgfusepath{stroke}%
\end{pgfscope}%
\begin{pgfscope}%
\pgfsetbuttcap%
\pgfsetroundjoin%
\definecolor{currentfill}{rgb}{0.000000,0.000000,0.000000}%
\pgfsetfillcolor{currentfill}%
\pgfsetlinewidth{0.803000pt}%
\definecolor{currentstroke}{rgb}{0.000000,0.000000,0.000000}%
\pgfsetstrokecolor{currentstroke}%
\pgfsetdash{}{0pt}%
\pgfsys@defobject{currentmarker}{\pgfqpoint{-0.048611in}{0.000000in}}{\pgfqpoint{-0.000000in}{0.000000in}}{%
\pgfpathmoveto{\pgfqpoint{-0.000000in}{0.000000in}}%
\pgfpathlineto{\pgfqpoint{-0.048611in}{0.000000in}}%
\pgfusepath{stroke,fill}%
}%
\begin{pgfscope}%
\pgfsys@transformshift{0.594525in}{1.156160in}%
\pgfsys@useobject{currentmarker}{}%
\end{pgfscope}%
\end{pgfscope}%
\begin{pgfscope}%
\definecolor{textcolor}{rgb}{0.000000,0.000000,0.000000}%
\pgfsetstrokecolor{textcolor}%
\pgfsetfillcolor{textcolor}%
\pgftext[x=0.241129in, y=1.117007in, left, base]{\color{textcolor}\rmfamily\fontsize{8.000000}{9.600000}\selectfont \(\displaystyle {10^{-4}}\)}%
\end{pgfscope}%
\begin{pgfscope}%
\pgfpathrectangle{\pgfqpoint{0.594525in}{0.417642in}}{\pgfqpoint{3.432047in}{2.016277in}}%
\pgfusepath{clip}%
\pgfsetrectcap%
\pgfsetroundjoin%
\pgfsetlinewidth{0.803000pt}%
\definecolor{currentstroke}{rgb}{0.450000,0.450000,0.450000}%
\pgfsetstrokecolor{currentstroke}%
\pgfsetdash{}{0pt}%
\pgfpathmoveto{\pgfqpoint{0.594525in}{1.475492in}}%
\pgfpathlineto{\pgfqpoint{4.026572in}{1.475492in}}%
\pgfusepath{stroke}%
\end{pgfscope}%
\begin{pgfscope}%
\pgfsetbuttcap%
\pgfsetroundjoin%
\definecolor{currentfill}{rgb}{0.000000,0.000000,0.000000}%
\pgfsetfillcolor{currentfill}%
\pgfsetlinewidth{0.803000pt}%
\definecolor{currentstroke}{rgb}{0.000000,0.000000,0.000000}%
\pgfsetstrokecolor{currentstroke}%
\pgfsetdash{}{0pt}%
\pgfsys@defobject{currentmarker}{\pgfqpoint{-0.048611in}{0.000000in}}{\pgfqpoint{-0.000000in}{0.000000in}}{%
\pgfpathmoveto{\pgfqpoint{-0.000000in}{0.000000in}}%
\pgfpathlineto{\pgfqpoint{-0.048611in}{0.000000in}}%
\pgfusepath{stroke,fill}%
}%
\begin{pgfscope}%
\pgfsys@transformshift{0.594525in}{1.475492in}%
\pgfsys@useobject{currentmarker}{}%
\end{pgfscope}%
\end{pgfscope}%
\begin{pgfscope}%
\definecolor{textcolor}{rgb}{0.000000,0.000000,0.000000}%
\pgfsetstrokecolor{textcolor}%
\pgfsetfillcolor{textcolor}%
\pgftext[x=0.241129in, y=1.436339in, left, base]{\color{textcolor}\rmfamily\fontsize{8.000000}{9.600000}\selectfont \(\displaystyle {10^{-3}}\)}%
\end{pgfscope}%
\begin{pgfscope}%
\pgfpathrectangle{\pgfqpoint{0.594525in}{0.417642in}}{\pgfqpoint{3.432047in}{2.016277in}}%
\pgfusepath{clip}%
\pgfsetrectcap%
\pgfsetroundjoin%
\pgfsetlinewidth{0.803000pt}%
\definecolor{currentstroke}{rgb}{0.450000,0.450000,0.450000}%
\pgfsetstrokecolor{currentstroke}%
\pgfsetdash{}{0pt}%
\pgfpathmoveto{\pgfqpoint{0.594525in}{1.794824in}}%
\pgfpathlineto{\pgfqpoint{4.026572in}{1.794824in}}%
\pgfusepath{stroke}%
\end{pgfscope}%
\begin{pgfscope}%
\pgfsetbuttcap%
\pgfsetroundjoin%
\definecolor{currentfill}{rgb}{0.000000,0.000000,0.000000}%
\pgfsetfillcolor{currentfill}%
\pgfsetlinewidth{0.803000pt}%
\definecolor{currentstroke}{rgb}{0.000000,0.000000,0.000000}%
\pgfsetstrokecolor{currentstroke}%
\pgfsetdash{}{0pt}%
\pgfsys@defobject{currentmarker}{\pgfqpoint{-0.048611in}{0.000000in}}{\pgfqpoint{-0.000000in}{0.000000in}}{%
\pgfpathmoveto{\pgfqpoint{-0.000000in}{0.000000in}}%
\pgfpathlineto{\pgfqpoint{-0.048611in}{0.000000in}}%
\pgfusepath{stroke,fill}%
}%
\begin{pgfscope}%
\pgfsys@transformshift{0.594525in}{1.794824in}%
\pgfsys@useobject{currentmarker}{}%
\end{pgfscope}%
\end{pgfscope}%
\begin{pgfscope}%
\definecolor{textcolor}{rgb}{0.000000,0.000000,0.000000}%
\pgfsetstrokecolor{textcolor}%
\pgfsetfillcolor{textcolor}%
\pgftext[x=0.241129in, y=1.755671in, left, base]{\color{textcolor}\rmfamily\fontsize{8.000000}{9.600000}\selectfont \(\displaystyle {10^{-2}}\)}%
\end{pgfscope}%
\begin{pgfscope}%
\pgfpathrectangle{\pgfqpoint{0.594525in}{0.417642in}}{\pgfqpoint{3.432047in}{2.016277in}}%
\pgfusepath{clip}%
\pgfsetrectcap%
\pgfsetroundjoin%
\pgfsetlinewidth{0.803000pt}%
\definecolor{currentstroke}{rgb}{0.450000,0.450000,0.450000}%
\pgfsetstrokecolor{currentstroke}%
\pgfsetdash{}{0pt}%
\pgfpathmoveto{\pgfqpoint{0.594525in}{2.114156in}}%
\pgfpathlineto{\pgfqpoint{4.026572in}{2.114156in}}%
\pgfusepath{stroke}%
\end{pgfscope}%
\begin{pgfscope}%
\pgfsetbuttcap%
\pgfsetroundjoin%
\definecolor{currentfill}{rgb}{0.000000,0.000000,0.000000}%
\pgfsetfillcolor{currentfill}%
\pgfsetlinewidth{0.803000pt}%
\definecolor{currentstroke}{rgb}{0.000000,0.000000,0.000000}%
\pgfsetstrokecolor{currentstroke}%
\pgfsetdash{}{0pt}%
\pgfsys@defobject{currentmarker}{\pgfqpoint{-0.048611in}{0.000000in}}{\pgfqpoint{-0.000000in}{0.000000in}}{%
\pgfpathmoveto{\pgfqpoint{-0.000000in}{0.000000in}}%
\pgfpathlineto{\pgfqpoint{-0.048611in}{0.000000in}}%
\pgfusepath{stroke,fill}%
}%
\begin{pgfscope}%
\pgfsys@transformshift{0.594525in}{2.114156in}%
\pgfsys@useobject{currentmarker}{}%
\end{pgfscope}%
\end{pgfscope}%
\begin{pgfscope}%
\definecolor{textcolor}{rgb}{0.000000,0.000000,0.000000}%
\pgfsetstrokecolor{textcolor}%
\pgfsetfillcolor{textcolor}%
\pgftext[x=0.241129in, y=2.075004in, left, base]{\color{textcolor}\rmfamily\fontsize{8.000000}{9.600000}\selectfont \(\displaystyle {10^{-1}}\)}%
\end{pgfscope}%
\begin{pgfscope}%
\pgfpathrectangle{\pgfqpoint{0.594525in}{0.417642in}}{\pgfqpoint{3.432047in}{2.016277in}}%
\pgfusepath{clip}%
\pgfsetrectcap%
\pgfsetroundjoin%
\pgfsetlinewidth{0.803000pt}%
\definecolor{currentstroke}{rgb}{0.450000,0.450000,0.450000}%
\pgfsetstrokecolor{currentstroke}%
\pgfsetdash{}{0pt}%
\pgfpathmoveto{\pgfqpoint{0.594525in}{2.433489in}}%
\pgfpathlineto{\pgfqpoint{4.026572in}{2.433489in}}%
\pgfusepath{stroke}%
\end{pgfscope}%
\begin{pgfscope}%
\pgfsetbuttcap%
\pgfsetroundjoin%
\definecolor{currentfill}{rgb}{0.000000,0.000000,0.000000}%
\pgfsetfillcolor{currentfill}%
\pgfsetlinewidth{0.803000pt}%
\definecolor{currentstroke}{rgb}{0.000000,0.000000,0.000000}%
\pgfsetstrokecolor{currentstroke}%
\pgfsetdash{}{0pt}%
\pgfsys@defobject{currentmarker}{\pgfqpoint{-0.048611in}{0.000000in}}{\pgfqpoint{-0.000000in}{0.000000in}}{%
\pgfpathmoveto{\pgfqpoint{-0.000000in}{0.000000in}}%
\pgfpathlineto{\pgfqpoint{-0.048611in}{0.000000in}}%
\pgfusepath{stroke,fill}%
}%
\begin{pgfscope}%
\pgfsys@transformshift{0.594525in}{2.433489in}%
\pgfsys@useobject{currentmarker}{}%
\end{pgfscope}%
\end{pgfscope}%
\begin{pgfscope}%
\definecolor{textcolor}{rgb}{0.000000,0.000000,0.000000}%
\pgfsetstrokecolor{textcolor}%
\pgfsetfillcolor{textcolor}%
\pgftext[x=0.321376in, y=2.394336in, left, base]{\color{textcolor}\rmfamily\fontsize{8.000000}{9.600000}\selectfont \(\displaystyle {10^{0}}\)}%
\end{pgfscope}%
\begin{pgfscope}%
\pgfpathrectangle{\pgfqpoint{0.594525in}{0.417642in}}{\pgfqpoint{3.432047in}{2.016277in}}%
\pgfusepath{clip}%
\pgfsetrectcap%
\pgfsetroundjoin%
\pgfsetlinewidth{0.803000pt}%
\definecolor{currentstroke}{rgb}{0.850000,0.850000,0.850000}%
\pgfsetstrokecolor{currentstroke}%
\pgfsetdash{}{0pt}%
\pgfpathmoveto{\pgfqpoint{0.594525in}{0.421366in}}%
\pgfpathlineto{\pgfqpoint{4.026572in}{0.421366in}}%
\pgfusepath{stroke}%
\end{pgfscope}%
\begin{pgfscope}%
\pgfsetbuttcap%
\pgfsetroundjoin%
\definecolor{currentfill}{rgb}{0.000000,0.000000,0.000000}%
\pgfsetfillcolor{currentfill}%
\pgfsetlinewidth{0.602250pt}%
\definecolor{currentstroke}{rgb}{0.000000,0.000000,0.000000}%
\pgfsetstrokecolor{currentstroke}%
\pgfsetdash{}{0pt}%
\pgfsys@defobject{currentmarker}{\pgfqpoint{-0.027778in}{0.000000in}}{\pgfqpoint{-0.000000in}{0.000000in}}{%
\pgfpathmoveto{\pgfqpoint{-0.000000in}{0.000000in}}%
\pgfpathlineto{\pgfqpoint{-0.027778in}{0.000000in}}%
\pgfusepath{stroke,fill}%
}%
\begin{pgfscope}%
\pgfsys@transformshift{0.594525in}{0.421366in}%
\pgfsys@useobject{currentmarker}{}%
\end{pgfscope}%
\end{pgfscope}%
\begin{pgfscope}%
\pgfpathrectangle{\pgfqpoint{0.594525in}{0.417642in}}{\pgfqpoint{3.432047in}{2.016277in}}%
\pgfusepath{clip}%
\pgfsetrectcap%
\pgfsetroundjoin%
\pgfsetlinewidth{0.803000pt}%
\definecolor{currentstroke}{rgb}{0.850000,0.850000,0.850000}%
\pgfsetstrokecolor{currentstroke}%
\pgfsetdash{}{0pt}%
\pgfpathmoveto{\pgfqpoint{0.594525in}{0.446651in}}%
\pgfpathlineto{\pgfqpoint{4.026572in}{0.446651in}}%
\pgfusepath{stroke}%
\end{pgfscope}%
\begin{pgfscope}%
\pgfsetbuttcap%
\pgfsetroundjoin%
\definecolor{currentfill}{rgb}{0.000000,0.000000,0.000000}%
\pgfsetfillcolor{currentfill}%
\pgfsetlinewidth{0.602250pt}%
\definecolor{currentstroke}{rgb}{0.000000,0.000000,0.000000}%
\pgfsetstrokecolor{currentstroke}%
\pgfsetdash{}{0pt}%
\pgfsys@defobject{currentmarker}{\pgfqpoint{-0.027778in}{0.000000in}}{\pgfqpoint{-0.000000in}{0.000000in}}{%
\pgfpathmoveto{\pgfqpoint{-0.000000in}{0.000000in}}%
\pgfpathlineto{\pgfqpoint{-0.027778in}{0.000000in}}%
\pgfusepath{stroke,fill}%
}%
\begin{pgfscope}%
\pgfsys@transformshift{0.594525in}{0.446651in}%
\pgfsys@useobject{currentmarker}{}%
\end{pgfscope}%
\end{pgfscope}%
\begin{pgfscope}%
\pgfpathrectangle{\pgfqpoint{0.594525in}{0.417642in}}{\pgfqpoint{3.432047in}{2.016277in}}%
\pgfusepath{clip}%
\pgfsetrectcap%
\pgfsetroundjoin%
\pgfsetlinewidth{0.803000pt}%
\definecolor{currentstroke}{rgb}{0.850000,0.850000,0.850000}%
\pgfsetstrokecolor{currentstroke}%
\pgfsetdash{}{0pt}%
\pgfpathmoveto{\pgfqpoint{0.594525in}{0.468030in}}%
\pgfpathlineto{\pgfqpoint{4.026572in}{0.468030in}}%
\pgfusepath{stroke}%
\end{pgfscope}%
\begin{pgfscope}%
\pgfsetbuttcap%
\pgfsetroundjoin%
\definecolor{currentfill}{rgb}{0.000000,0.000000,0.000000}%
\pgfsetfillcolor{currentfill}%
\pgfsetlinewidth{0.602250pt}%
\definecolor{currentstroke}{rgb}{0.000000,0.000000,0.000000}%
\pgfsetstrokecolor{currentstroke}%
\pgfsetdash{}{0pt}%
\pgfsys@defobject{currentmarker}{\pgfqpoint{-0.027778in}{0.000000in}}{\pgfqpoint{-0.000000in}{0.000000in}}{%
\pgfpathmoveto{\pgfqpoint{-0.000000in}{0.000000in}}%
\pgfpathlineto{\pgfqpoint{-0.027778in}{0.000000in}}%
\pgfusepath{stroke,fill}%
}%
\begin{pgfscope}%
\pgfsys@transformshift{0.594525in}{0.468030in}%
\pgfsys@useobject{currentmarker}{}%
\end{pgfscope}%
\end{pgfscope}%
\begin{pgfscope}%
\pgfpathrectangle{\pgfqpoint{0.594525in}{0.417642in}}{\pgfqpoint{3.432047in}{2.016277in}}%
\pgfusepath{clip}%
\pgfsetrectcap%
\pgfsetroundjoin%
\pgfsetlinewidth{0.803000pt}%
\definecolor{currentstroke}{rgb}{0.850000,0.850000,0.850000}%
\pgfsetstrokecolor{currentstroke}%
\pgfsetdash{}{0pt}%
\pgfpathmoveto{\pgfqpoint{0.594525in}{0.486548in}}%
\pgfpathlineto{\pgfqpoint{4.026572in}{0.486548in}}%
\pgfusepath{stroke}%
\end{pgfscope}%
\begin{pgfscope}%
\pgfsetbuttcap%
\pgfsetroundjoin%
\definecolor{currentfill}{rgb}{0.000000,0.000000,0.000000}%
\pgfsetfillcolor{currentfill}%
\pgfsetlinewidth{0.602250pt}%
\definecolor{currentstroke}{rgb}{0.000000,0.000000,0.000000}%
\pgfsetstrokecolor{currentstroke}%
\pgfsetdash{}{0pt}%
\pgfsys@defobject{currentmarker}{\pgfqpoint{-0.027778in}{0.000000in}}{\pgfqpoint{-0.000000in}{0.000000in}}{%
\pgfpathmoveto{\pgfqpoint{-0.000000in}{0.000000in}}%
\pgfpathlineto{\pgfqpoint{-0.027778in}{0.000000in}}%
\pgfusepath{stroke,fill}%
}%
\begin{pgfscope}%
\pgfsys@transformshift{0.594525in}{0.486548in}%
\pgfsys@useobject{currentmarker}{}%
\end{pgfscope}%
\end{pgfscope}%
\begin{pgfscope}%
\pgfpathrectangle{\pgfqpoint{0.594525in}{0.417642in}}{\pgfqpoint{3.432047in}{2.016277in}}%
\pgfusepath{clip}%
\pgfsetrectcap%
\pgfsetroundjoin%
\pgfsetlinewidth{0.803000pt}%
\definecolor{currentstroke}{rgb}{0.850000,0.850000,0.850000}%
\pgfsetstrokecolor{currentstroke}%
\pgfsetdash{}{0pt}%
\pgfpathmoveto{\pgfqpoint{0.594525in}{0.502883in}}%
\pgfpathlineto{\pgfqpoint{4.026572in}{0.502883in}}%
\pgfusepath{stroke}%
\end{pgfscope}%
\begin{pgfscope}%
\pgfsetbuttcap%
\pgfsetroundjoin%
\definecolor{currentfill}{rgb}{0.000000,0.000000,0.000000}%
\pgfsetfillcolor{currentfill}%
\pgfsetlinewidth{0.602250pt}%
\definecolor{currentstroke}{rgb}{0.000000,0.000000,0.000000}%
\pgfsetstrokecolor{currentstroke}%
\pgfsetdash{}{0pt}%
\pgfsys@defobject{currentmarker}{\pgfqpoint{-0.027778in}{0.000000in}}{\pgfqpoint{-0.000000in}{0.000000in}}{%
\pgfpathmoveto{\pgfqpoint{-0.000000in}{0.000000in}}%
\pgfpathlineto{\pgfqpoint{-0.027778in}{0.000000in}}%
\pgfusepath{stroke,fill}%
}%
\begin{pgfscope}%
\pgfsys@transformshift{0.594525in}{0.502883in}%
\pgfsys@useobject{currentmarker}{}%
\end{pgfscope}%
\end{pgfscope}%
\begin{pgfscope}%
\pgfpathrectangle{\pgfqpoint{0.594525in}{0.417642in}}{\pgfqpoint{3.432047in}{2.016277in}}%
\pgfusepath{clip}%
\pgfsetrectcap%
\pgfsetroundjoin%
\pgfsetlinewidth{0.803000pt}%
\definecolor{currentstroke}{rgb}{0.850000,0.850000,0.850000}%
\pgfsetstrokecolor{currentstroke}%
\pgfsetdash{}{0pt}%
\pgfpathmoveto{\pgfqpoint{0.594525in}{0.613623in}}%
\pgfpathlineto{\pgfqpoint{4.026572in}{0.613623in}}%
\pgfusepath{stroke}%
\end{pgfscope}%
\begin{pgfscope}%
\pgfsetbuttcap%
\pgfsetroundjoin%
\definecolor{currentfill}{rgb}{0.000000,0.000000,0.000000}%
\pgfsetfillcolor{currentfill}%
\pgfsetlinewidth{0.602250pt}%
\definecolor{currentstroke}{rgb}{0.000000,0.000000,0.000000}%
\pgfsetstrokecolor{currentstroke}%
\pgfsetdash{}{0pt}%
\pgfsys@defobject{currentmarker}{\pgfqpoint{-0.027778in}{0.000000in}}{\pgfqpoint{-0.000000in}{0.000000in}}{%
\pgfpathmoveto{\pgfqpoint{-0.000000in}{0.000000in}}%
\pgfpathlineto{\pgfqpoint{-0.027778in}{0.000000in}}%
\pgfusepath{stroke,fill}%
}%
\begin{pgfscope}%
\pgfsys@transformshift{0.594525in}{0.613623in}%
\pgfsys@useobject{currentmarker}{}%
\end{pgfscope}%
\end{pgfscope}%
\begin{pgfscope}%
\pgfpathrectangle{\pgfqpoint{0.594525in}{0.417642in}}{\pgfqpoint{3.432047in}{2.016277in}}%
\pgfusepath{clip}%
\pgfsetrectcap%
\pgfsetroundjoin%
\pgfsetlinewidth{0.803000pt}%
\definecolor{currentstroke}{rgb}{0.850000,0.850000,0.850000}%
\pgfsetstrokecolor{currentstroke}%
\pgfsetdash{}{0pt}%
\pgfpathmoveto{\pgfqpoint{0.594525in}{0.669855in}}%
\pgfpathlineto{\pgfqpoint{4.026572in}{0.669855in}}%
\pgfusepath{stroke}%
\end{pgfscope}%
\begin{pgfscope}%
\pgfsetbuttcap%
\pgfsetroundjoin%
\definecolor{currentfill}{rgb}{0.000000,0.000000,0.000000}%
\pgfsetfillcolor{currentfill}%
\pgfsetlinewidth{0.602250pt}%
\definecolor{currentstroke}{rgb}{0.000000,0.000000,0.000000}%
\pgfsetstrokecolor{currentstroke}%
\pgfsetdash{}{0pt}%
\pgfsys@defobject{currentmarker}{\pgfqpoint{-0.027778in}{0.000000in}}{\pgfqpoint{-0.000000in}{0.000000in}}{%
\pgfpathmoveto{\pgfqpoint{-0.000000in}{0.000000in}}%
\pgfpathlineto{\pgfqpoint{-0.027778in}{0.000000in}}%
\pgfusepath{stroke,fill}%
}%
\begin{pgfscope}%
\pgfsys@transformshift{0.594525in}{0.669855in}%
\pgfsys@useobject{currentmarker}{}%
\end{pgfscope}%
\end{pgfscope}%
\begin{pgfscope}%
\pgfpathrectangle{\pgfqpoint{0.594525in}{0.417642in}}{\pgfqpoint{3.432047in}{2.016277in}}%
\pgfusepath{clip}%
\pgfsetrectcap%
\pgfsetroundjoin%
\pgfsetlinewidth{0.803000pt}%
\definecolor{currentstroke}{rgb}{0.850000,0.850000,0.850000}%
\pgfsetstrokecolor{currentstroke}%
\pgfsetdash{}{0pt}%
\pgfpathmoveto{\pgfqpoint{0.594525in}{0.709752in}}%
\pgfpathlineto{\pgfqpoint{4.026572in}{0.709752in}}%
\pgfusepath{stroke}%
\end{pgfscope}%
\begin{pgfscope}%
\pgfsetbuttcap%
\pgfsetroundjoin%
\definecolor{currentfill}{rgb}{0.000000,0.000000,0.000000}%
\pgfsetfillcolor{currentfill}%
\pgfsetlinewidth{0.602250pt}%
\definecolor{currentstroke}{rgb}{0.000000,0.000000,0.000000}%
\pgfsetstrokecolor{currentstroke}%
\pgfsetdash{}{0pt}%
\pgfsys@defobject{currentmarker}{\pgfqpoint{-0.027778in}{0.000000in}}{\pgfqpoint{-0.000000in}{0.000000in}}{%
\pgfpathmoveto{\pgfqpoint{-0.000000in}{0.000000in}}%
\pgfpathlineto{\pgfqpoint{-0.027778in}{0.000000in}}%
\pgfusepath{stroke,fill}%
}%
\begin{pgfscope}%
\pgfsys@transformshift{0.594525in}{0.709752in}%
\pgfsys@useobject{currentmarker}{}%
\end{pgfscope}%
\end{pgfscope}%
\begin{pgfscope}%
\pgfpathrectangle{\pgfqpoint{0.594525in}{0.417642in}}{\pgfqpoint{3.432047in}{2.016277in}}%
\pgfusepath{clip}%
\pgfsetrectcap%
\pgfsetroundjoin%
\pgfsetlinewidth{0.803000pt}%
\definecolor{currentstroke}{rgb}{0.850000,0.850000,0.850000}%
\pgfsetstrokecolor{currentstroke}%
\pgfsetdash{}{0pt}%
\pgfpathmoveto{\pgfqpoint{0.594525in}{0.740699in}}%
\pgfpathlineto{\pgfqpoint{4.026572in}{0.740699in}}%
\pgfusepath{stroke}%
\end{pgfscope}%
\begin{pgfscope}%
\pgfsetbuttcap%
\pgfsetroundjoin%
\definecolor{currentfill}{rgb}{0.000000,0.000000,0.000000}%
\pgfsetfillcolor{currentfill}%
\pgfsetlinewidth{0.602250pt}%
\definecolor{currentstroke}{rgb}{0.000000,0.000000,0.000000}%
\pgfsetstrokecolor{currentstroke}%
\pgfsetdash{}{0pt}%
\pgfsys@defobject{currentmarker}{\pgfqpoint{-0.027778in}{0.000000in}}{\pgfqpoint{-0.000000in}{0.000000in}}{%
\pgfpathmoveto{\pgfqpoint{-0.000000in}{0.000000in}}%
\pgfpathlineto{\pgfqpoint{-0.027778in}{0.000000in}}%
\pgfusepath{stroke,fill}%
}%
\begin{pgfscope}%
\pgfsys@transformshift{0.594525in}{0.740699in}%
\pgfsys@useobject{currentmarker}{}%
\end{pgfscope}%
\end{pgfscope}%
\begin{pgfscope}%
\pgfpathrectangle{\pgfqpoint{0.594525in}{0.417642in}}{\pgfqpoint{3.432047in}{2.016277in}}%
\pgfusepath{clip}%
\pgfsetrectcap%
\pgfsetroundjoin%
\pgfsetlinewidth{0.803000pt}%
\definecolor{currentstroke}{rgb}{0.850000,0.850000,0.850000}%
\pgfsetstrokecolor{currentstroke}%
\pgfsetdash{}{0pt}%
\pgfpathmoveto{\pgfqpoint{0.594525in}{0.765984in}}%
\pgfpathlineto{\pgfqpoint{4.026572in}{0.765984in}}%
\pgfusepath{stroke}%
\end{pgfscope}%
\begin{pgfscope}%
\pgfsetbuttcap%
\pgfsetroundjoin%
\definecolor{currentfill}{rgb}{0.000000,0.000000,0.000000}%
\pgfsetfillcolor{currentfill}%
\pgfsetlinewidth{0.602250pt}%
\definecolor{currentstroke}{rgb}{0.000000,0.000000,0.000000}%
\pgfsetstrokecolor{currentstroke}%
\pgfsetdash{}{0pt}%
\pgfsys@defobject{currentmarker}{\pgfqpoint{-0.027778in}{0.000000in}}{\pgfqpoint{-0.000000in}{0.000000in}}{%
\pgfpathmoveto{\pgfqpoint{-0.000000in}{0.000000in}}%
\pgfpathlineto{\pgfqpoint{-0.027778in}{0.000000in}}%
\pgfusepath{stroke,fill}%
}%
\begin{pgfscope}%
\pgfsys@transformshift{0.594525in}{0.765984in}%
\pgfsys@useobject{currentmarker}{}%
\end{pgfscope}%
\end{pgfscope}%
\begin{pgfscope}%
\pgfpathrectangle{\pgfqpoint{0.594525in}{0.417642in}}{\pgfqpoint{3.432047in}{2.016277in}}%
\pgfusepath{clip}%
\pgfsetrectcap%
\pgfsetroundjoin%
\pgfsetlinewidth{0.803000pt}%
\definecolor{currentstroke}{rgb}{0.850000,0.850000,0.850000}%
\pgfsetstrokecolor{currentstroke}%
\pgfsetdash{}{0pt}%
\pgfpathmoveto{\pgfqpoint{0.594525in}{0.787362in}}%
\pgfpathlineto{\pgfqpoint{4.026572in}{0.787362in}}%
\pgfusepath{stroke}%
\end{pgfscope}%
\begin{pgfscope}%
\pgfsetbuttcap%
\pgfsetroundjoin%
\definecolor{currentfill}{rgb}{0.000000,0.000000,0.000000}%
\pgfsetfillcolor{currentfill}%
\pgfsetlinewidth{0.602250pt}%
\definecolor{currentstroke}{rgb}{0.000000,0.000000,0.000000}%
\pgfsetstrokecolor{currentstroke}%
\pgfsetdash{}{0pt}%
\pgfsys@defobject{currentmarker}{\pgfqpoint{-0.027778in}{0.000000in}}{\pgfqpoint{-0.000000in}{0.000000in}}{%
\pgfpathmoveto{\pgfqpoint{-0.000000in}{0.000000in}}%
\pgfpathlineto{\pgfqpoint{-0.027778in}{0.000000in}}%
\pgfusepath{stroke,fill}%
}%
\begin{pgfscope}%
\pgfsys@transformshift{0.594525in}{0.787362in}%
\pgfsys@useobject{currentmarker}{}%
\end{pgfscope}%
\end{pgfscope}%
\begin{pgfscope}%
\pgfpathrectangle{\pgfqpoint{0.594525in}{0.417642in}}{\pgfqpoint{3.432047in}{2.016277in}}%
\pgfusepath{clip}%
\pgfsetrectcap%
\pgfsetroundjoin%
\pgfsetlinewidth{0.803000pt}%
\definecolor{currentstroke}{rgb}{0.850000,0.850000,0.850000}%
\pgfsetstrokecolor{currentstroke}%
\pgfsetdash{}{0pt}%
\pgfpathmoveto{\pgfqpoint{0.594525in}{0.805881in}}%
\pgfpathlineto{\pgfqpoint{4.026572in}{0.805881in}}%
\pgfusepath{stroke}%
\end{pgfscope}%
\begin{pgfscope}%
\pgfsetbuttcap%
\pgfsetroundjoin%
\definecolor{currentfill}{rgb}{0.000000,0.000000,0.000000}%
\pgfsetfillcolor{currentfill}%
\pgfsetlinewidth{0.602250pt}%
\definecolor{currentstroke}{rgb}{0.000000,0.000000,0.000000}%
\pgfsetstrokecolor{currentstroke}%
\pgfsetdash{}{0pt}%
\pgfsys@defobject{currentmarker}{\pgfqpoint{-0.027778in}{0.000000in}}{\pgfqpoint{-0.000000in}{0.000000in}}{%
\pgfpathmoveto{\pgfqpoint{-0.000000in}{0.000000in}}%
\pgfpathlineto{\pgfqpoint{-0.027778in}{0.000000in}}%
\pgfusepath{stroke,fill}%
}%
\begin{pgfscope}%
\pgfsys@transformshift{0.594525in}{0.805881in}%
\pgfsys@useobject{currentmarker}{}%
\end{pgfscope}%
\end{pgfscope}%
\begin{pgfscope}%
\pgfpathrectangle{\pgfqpoint{0.594525in}{0.417642in}}{\pgfqpoint{3.432047in}{2.016277in}}%
\pgfusepath{clip}%
\pgfsetrectcap%
\pgfsetroundjoin%
\pgfsetlinewidth{0.803000pt}%
\definecolor{currentstroke}{rgb}{0.850000,0.850000,0.850000}%
\pgfsetstrokecolor{currentstroke}%
\pgfsetdash{}{0pt}%
\pgfpathmoveto{\pgfqpoint{0.594525in}{0.822215in}}%
\pgfpathlineto{\pgfqpoint{4.026572in}{0.822215in}}%
\pgfusepath{stroke}%
\end{pgfscope}%
\begin{pgfscope}%
\pgfsetbuttcap%
\pgfsetroundjoin%
\definecolor{currentfill}{rgb}{0.000000,0.000000,0.000000}%
\pgfsetfillcolor{currentfill}%
\pgfsetlinewidth{0.602250pt}%
\definecolor{currentstroke}{rgb}{0.000000,0.000000,0.000000}%
\pgfsetstrokecolor{currentstroke}%
\pgfsetdash{}{0pt}%
\pgfsys@defobject{currentmarker}{\pgfqpoint{-0.027778in}{0.000000in}}{\pgfqpoint{-0.000000in}{0.000000in}}{%
\pgfpathmoveto{\pgfqpoint{-0.000000in}{0.000000in}}%
\pgfpathlineto{\pgfqpoint{-0.027778in}{0.000000in}}%
\pgfusepath{stroke,fill}%
}%
\begin{pgfscope}%
\pgfsys@transformshift{0.594525in}{0.822215in}%
\pgfsys@useobject{currentmarker}{}%
\end{pgfscope}%
\end{pgfscope}%
\begin{pgfscope}%
\pgfpathrectangle{\pgfqpoint{0.594525in}{0.417642in}}{\pgfqpoint{3.432047in}{2.016277in}}%
\pgfusepath{clip}%
\pgfsetrectcap%
\pgfsetroundjoin%
\pgfsetlinewidth{0.803000pt}%
\definecolor{currentstroke}{rgb}{0.850000,0.850000,0.850000}%
\pgfsetstrokecolor{currentstroke}%
\pgfsetdash{}{0pt}%
\pgfpathmoveto{\pgfqpoint{0.594525in}{0.932956in}}%
\pgfpathlineto{\pgfqpoint{4.026572in}{0.932956in}}%
\pgfusepath{stroke}%
\end{pgfscope}%
\begin{pgfscope}%
\pgfsetbuttcap%
\pgfsetroundjoin%
\definecolor{currentfill}{rgb}{0.000000,0.000000,0.000000}%
\pgfsetfillcolor{currentfill}%
\pgfsetlinewidth{0.602250pt}%
\definecolor{currentstroke}{rgb}{0.000000,0.000000,0.000000}%
\pgfsetstrokecolor{currentstroke}%
\pgfsetdash{}{0pt}%
\pgfsys@defobject{currentmarker}{\pgfqpoint{-0.027778in}{0.000000in}}{\pgfqpoint{-0.000000in}{0.000000in}}{%
\pgfpathmoveto{\pgfqpoint{-0.000000in}{0.000000in}}%
\pgfpathlineto{\pgfqpoint{-0.027778in}{0.000000in}}%
\pgfusepath{stroke,fill}%
}%
\begin{pgfscope}%
\pgfsys@transformshift{0.594525in}{0.932956in}%
\pgfsys@useobject{currentmarker}{}%
\end{pgfscope}%
\end{pgfscope}%
\begin{pgfscope}%
\pgfpathrectangle{\pgfqpoint{0.594525in}{0.417642in}}{\pgfqpoint{3.432047in}{2.016277in}}%
\pgfusepath{clip}%
\pgfsetrectcap%
\pgfsetroundjoin%
\pgfsetlinewidth{0.803000pt}%
\definecolor{currentstroke}{rgb}{0.850000,0.850000,0.850000}%
\pgfsetstrokecolor{currentstroke}%
\pgfsetdash{}{0pt}%
\pgfpathmoveto{\pgfqpoint{0.594525in}{0.989187in}}%
\pgfpathlineto{\pgfqpoint{4.026572in}{0.989187in}}%
\pgfusepath{stroke}%
\end{pgfscope}%
\begin{pgfscope}%
\pgfsetbuttcap%
\pgfsetroundjoin%
\definecolor{currentfill}{rgb}{0.000000,0.000000,0.000000}%
\pgfsetfillcolor{currentfill}%
\pgfsetlinewidth{0.602250pt}%
\definecolor{currentstroke}{rgb}{0.000000,0.000000,0.000000}%
\pgfsetstrokecolor{currentstroke}%
\pgfsetdash{}{0pt}%
\pgfsys@defobject{currentmarker}{\pgfqpoint{-0.027778in}{0.000000in}}{\pgfqpoint{-0.000000in}{0.000000in}}{%
\pgfpathmoveto{\pgfqpoint{-0.000000in}{0.000000in}}%
\pgfpathlineto{\pgfqpoint{-0.027778in}{0.000000in}}%
\pgfusepath{stroke,fill}%
}%
\begin{pgfscope}%
\pgfsys@transformshift{0.594525in}{0.989187in}%
\pgfsys@useobject{currentmarker}{}%
\end{pgfscope}%
\end{pgfscope}%
\begin{pgfscope}%
\pgfpathrectangle{\pgfqpoint{0.594525in}{0.417642in}}{\pgfqpoint{3.432047in}{2.016277in}}%
\pgfusepath{clip}%
\pgfsetrectcap%
\pgfsetroundjoin%
\pgfsetlinewidth{0.803000pt}%
\definecolor{currentstroke}{rgb}{0.850000,0.850000,0.850000}%
\pgfsetstrokecolor{currentstroke}%
\pgfsetdash{}{0pt}%
\pgfpathmoveto{\pgfqpoint{0.594525in}{1.029084in}}%
\pgfpathlineto{\pgfqpoint{4.026572in}{1.029084in}}%
\pgfusepath{stroke}%
\end{pgfscope}%
\begin{pgfscope}%
\pgfsetbuttcap%
\pgfsetroundjoin%
\definecolor{currentfill}{rgb}{0.000000,0.000000,0.000000}%
\pgfsetfillcolor{currentfill}%
\pgfsetlinewidth{0.602250pt}%
\definecolor{currentstroke}{rgb}{0.000000,0.000000,0.000000}%
\pgfsetstrokecolor{currentstroke}%
\pgfsetdash{}{0pt}%
\pgfsys@defobject{currentmarker}{\pgfqpoint{-0.027778in}{0.000000in}}{\pgfqpoint{-0.000000in}{0.000000in}}{%
\pgfpathmoveto{\pgfqpoint{-0.000000in}{0.000000in}}%
\pgfpathlineto{\pgfqpoint{-0.027778in}{0.000000in}}%
\pgfusepath{stroke,fill}%
}%
\begin{pgfscope}%
\pgfsys@transformshift{0.594525in}{1.029084in}%
\pgfsys@useobject{currentmarker}{}%
\end{pgfscope}%
\end{pgfscope}%
\begin{pgfscope}%
\pgfpathrectangle{\pgfqpoint{0.594525in}{0.417642in}}{\pgfqpoint{3.432047in}{2.016277in}}%
\pgfusepath{clip}%
\pgfsetrectcap%
\pgfsetroundjoin%
\pgfsetlinewidth{0.803000pt}%
\definecolor{currentstroke}{rgb}{0.850000,0.850000,0.850000}%
\pgfsetstrokecolor{currentstroke}%
\pgfsetdash{}{0pt}%
\pgfpathmoveto{\pgfqpoint{0.594525in}{1.060031in}}%
\pgfpathlineto{\pgfqpoint{4.026572in}{1.060031in}}%
\pgfusepath{stroke}%
\end{pgfscope}%
\begin{pgfscope}%
\pgfsetbuttcap%
\pgfsetroundjoin%
\definecolor{currentfill}{rgb}{0.000000,0.000000,0.000000}%
\pgfsetfillcolor{currentfill}%
\pgfsetlinewidth{0.602250pt}%
\definecolor{currentstroke}{rgb}{0.000000,0.000000,0.000000}%
\pgfsetstrokecolor{currentstroke}%
\pgfsetdash{}{0pt}%
\pgfsys@defobject{currentmarker}{\pgfqpoint{-0.027778in}{0.000000in}}{\pgfqpoint{-0.000000in}{0.000000in}}{%
\pgfpathmoveto{\pgfqpoint{-0.000000in}{0.000000in}}%
\pgfpathlineto{\pgfqpoint{-0.027778in}{0.000000in}}%
\pgfusepath{stroke,fill}%
}%
\begin{pgfscope}%
\pgfsys@transformshift{0.594525in}{1.060031in}%
\pgfsys@useobject{currentmarker}{}%
\end{pgfscope}%
\end{pgfscope}%
\begin{pgfscope}%
\pgfpathrectangle{\pgfqpoint{0.594525in}{0.417642in}}{\pgfqpoint{3.432047in}{2.016277in}}%
\pgfusepath{clip}%
\pgfsetrectcap%
\pgfsetroundjoin%
\pgfsetlinewidth{0.803000pt}%
\definecolor{currentstroke}{rgb}{0.850000,0.850000,0.850000}%
\pgfsetstrokecolor{currentstroke}%
\pgfsetdash{}{0pt}%
\pgfpathmoveto{\pgfqpoint{0.594525in}{1.085316in}}%
\pgfpathlineto{\pgfqpoint{4.026572in}{1.085316in}}%
\pgfusepath{stroke}%
\end{pgfscope}%
\begin{pgfscope}%
\pgfsetbuttcap%
\pgfsetroundjoin%
\definecolor{currentfill}{rgb}{0.000000,0.000000,0.000000}%
\pgfsetfillcolor{currentfill}%
\pgfsetlinewidth{0.602250pt}%
\definecolor{currentstroke}{rgb}{0.000000,0.000000,0.000000}%
\pgfsetstrokecolor{currentstroke}%
\pgfsetdash{}{0pt}%
\pgfsys@defobject{currentmarker}{\pgfqpoint{-0.027778in}{0.000000in}}{\pgfqpoint{-0.000000in}{0.000000in}}{%
\pgfpathmoveto{\pgfqpoint{-0.000000in}{0.000000in}}%
\pgfpathlineto{\pgfqpoint{-0.027778in}{0.000000in}}%
\pgfusepath{stroke,fill}%
}%
\begin{pgfscope}%
\pgfsys@transformshift{0.594525in}{1.085316in}%
\pgfsys@useobject{currentmarker}{}%
\end{pgfscope}%
\end{pgfscope}%
\begin{pgfscope}%
\pgfpathrectangle{\pgfqpoint{0.594525in}{0.417642in}}{\pgfqpoint{3.432047in}{2.016277in}}%
\pgfusepath{clip}%
\pgfsetrectcap%
\pgfsetroundjoin%
\pgfsetlinewidth{0.803000pt}%
\definecolor{currentstroke}{rgb}{0.850000,0.850000,0.850000}%
\pgfsetstrokecolor{currentstroke}%
\pgfsetdash{}{0pt}%
\pgfpathmoveto{\pgfqpoint{0.594525in}{1.106694in}}%
\pgfpathlineto{\pgfqpoint{4.026572in}{1.106694in}}%
\pgfusepath{stroke}%
\end{pgfscope}%
\begin{pgfscope}%
\pgfsetbuttcap%
\pgfsetroundjoin%
\definecolor{currentfill}{rgb}{0.000000,0.000000,0.000000}%
\pgfsetfillcolor{currentfill}%
\pgfsetlinewidth{0.602250pt}%
\definecolor{currentstroke}{rgb}{0.000000,0.000000,0.000000}%
\pgfsetstrokecolor{currentstroke}%
\pgfsetdash{}{0pt}%
\pgfsys@defobject{currentmarker}{\pgfqpoint{-0.027778in}{0.000000in}}{\pgfqpoint{-0.000000in}{0.000000in}}{%
\pgfpathmoveto{\pgfqpoint{-0.000000in}{0.000000in}}%
\pgfpathlineto{\pgfqpoint{-0.027778in}{0.000000in}}%
\pgfusepath{stroke,fill}%
}%
\begin{pgfscope}%
\pgfsys@transformshift{0.594525in}{1.106694in}%
\pgfsys@useobject{currentmarker}{}%
\end{pgfscope}%
\end{pgfscope}%
\begin{pgfscope}%
\pgfpathrectangle{\pgfqpoint{0.594525in}{0.417642in}}{\pgfqpoint{3.432047in}{2.016277in}}%
\pgfusepath{clip}%
\pgfsetrectcap%
\pgfsetroundjoin%
\pgfsetlinewidth{0.803000pt}%
\definecolor{currentstroke}{rgb}{0.850000,0.850000,0.850000}%
\pgfsetstrokecolor{currentstroke}%
\pgfsetdash{}{0pt}%
\pgfpathmoveto{\pgfqpoint{0.594525in}{1.125213in}}%
\pgfpathlineto{\pgfqpoint{4.026572in}{1.125213in}}%
\pgfusepath{stroke}%
\end{pgfscope}%
\begin{pgfscope}%
\pgfsetbuttcap%
\pgfsetroundjoin%
\definecolor{currentfill}{rgb}{0.000000,0.000000,0.000000}%
\pgfsetfillcolor{currentfill}%
\pgfsetlinewidth{0.602250pt}%
\definecolor{currentstroke}{rgb}{0.000000,0.000000,0.000000}%
\pgfsetstrokecolor{currentstroke}%
\pgfsetdash{}{0pt}%
\pgfsys@defobject{currentmarker}{\pgfqpoint{-0.027778in}{0.000000in}}{\pgfqpoint{-0.000000in}{0.000000in}}{%
\pgfpathmoveto{\pgfqpoint{-0.000000in}{0.000000in}}%
\pgfpathlineto{\pgfqpoint{-0.027778in}{0.000000in}}%
\pgfusepath{stroke,fill}%
}%
\begin{pgfscope}%
\pgfsys@transformshift{0.594525in}{1.125213in}%
\pgfsys@useobject{currentmarker}{}%
\end{pgfscope}%
\end{pgfscope}%
\begin{pgfscope}%
\pgfpathrectangle{\pgfqpoint{0.594525in}{0.417642in}}{\pgfqpoint{3.432047in}{2.016277in}}%
\pgfusepath{clip}%
\pgfsetrectcap%
\pgfsetroundjoin%
\pgfsetlinewidth{0.803000pt}%
\definecolor{currentstroke}{rgb}{0.850000,0.850000,0.850000}%
\pgfsetstrokecolor{currentstroke}%
\pgfsetdash{}{0pt}%
\pgfpathmoveto{\pgfqpoint{0.594525in}{1.141548in}}%
\pgfpathlineto{\pgfqpoint{4.026572in}{1.141548in}}%
\pgfusepath{stroke}%
\end{pgfscope}%
\begin{pgfscope}%
\pgfsetbuttcap%
\pgfsetroundjoin%
\definecolor{currentfill}{rgb}{0.000000,0.000000,0.000000}%
\pgfsetfillcolor{currentfill}%
\pgfsetlinewidth{0.602250pt}%
\definecolor{currentstroke}{rgb}{0.000000,0.000000,0.000000}%
\pgfsetstrokecolor{currentstroke}%
\pgfsetdash{}{0pt}%
\pgfsys@defobject{currentmarker}{\pgfqpoint{-0.027778in}{0.000000in}}{\pgfqpoint{-0.000000in}{0.000000in}}{%
\pgfpathmoveto{\pgfqpoint{-0.000000in}{0.000000in}}%
\pgfpathlineto{\pgfqpoint{-0.027778in}{0.000000in}}%
\pgfusepath{stroke,fill}%
}%
\begin{pgfscope}%
\pgfsys@transformshift{0.594525in}{1.141548in}%
\pgfsys@useobject{currentmarker}{}%
\end{pgfscope}%
\end{pgfscope}%
\begin{pgfscope}%
\pgfpathrectangle{\pgfqpoint{0.594525in}{0.417642in}}{\pgfqpoint{3.432047in}{2.016277in}}%
\pgfusepath{clip}%
\pgfsetrectcap%
\pgfsetroundjoin%
\pgfsetlinewidth{0.803000pt}%
\definecolor{currentstroke}{rgb}{0.850000,0.850000,0.850000}%
\pgfsetstrokecolor{currentstroke}%
\pgfsetdash{}{0pt}%
\pgfpathmoveto{\pgfqpoint{0.594525in}{1.252288in}}%
\pgfpathlineto{\pgfqpoint{4.026572in}{1.252288in}}%
\pgfusepath{stroke}%
\end{pgfscope}%
\begin{pgfscope}%
\pgfsetbuttcap%
\pgfsetroundjoin%
\definecolor{currentfill}{rgb}{0.000000,0.000000,0.000000}%
\pgfsetfillcolor{currentfill}%
\pgfsetlinewidth{0.602250pt}%
\definecolor{currentstroke}{rgb}{0.000000,0.000000,0.000000}%
\pgfsetstrokecolor{currentstroke}%
\pgfsetdash{}{0pt}%
\pgfsys@defobject{currentmarker}{\pgfqpoint{-0.027778in}{0.000000in}}{\pgfqpoint{-0.000000in}{0.000000in}}{%
\pgfpathmoveto{\pgfqpoint{-0.000000in}{0.000000in}}%
\pgfpathlineto{\pgfqpoint{-0.027778in}{0.000000in}}%
\pgfusepath{stroke,fill}%
}%
\begin{pgfscope}%
\pgfsys@transformshift{0.594525in}{1.252288in}%
\pgfsys@useobject{currentmarker}{}%
\end{pgfscope}%
\end{pgfscope}%
\begin{pgfscope}%
\pgfpathrectangle{\pgfqpoint{0.594525in}{0.417642in}}{\pgfqpoint{3.432047in}{2.016277in}}%
\pgfusepath{clip}%
\pgfsetrectcap%
\pgfsetroundjoin%
\pgfsetlinewidth{0.803000pt}%
\definecolor{currentstroke}{rgb}{0.850000,0.850000,0.850000}%
\pgfsetstrokecolor{currentstroke}%
\pgfsetdash{}{0pt}%
\pgfpathmoveto{\pgfqpoint{0.594525in}{1.308520in}}%
\pgfpathlineto{\pgfqpoint{4.026572in}{1.308520in}}%
\pgfusepath{stroke}%
\end{pgfscope}%
\begin{pgfscope}%
\pgfsetbuttcap%
\pgfsetroundjoin%
\definecolor{currentfill}{rgb}{0.000000,0.000000,0.000000}%
\pgfsetfillcolor{currentfill}%
\pgfsetlinewidth{0.602250pt}%
\definecolor{currentstroke}{rgb}{0.000000,0.000000,0.000000}%
\pgfsetstrokecolor{currentstroke}%
\pgfsetdash{}{0pt}%
\pgfsys@defobject{currentmarker}{\pgfqpoint{-0.027778in}{0.000000in}}{\pgfqpoint{-0.000000in}{0.000000in}}{%
\pgfpathmoveto{\pgfqpoint{-0.000000in}{0.000000in}}%
\pgfpathlineto{\pgfqpoint{-0.027778in}{0.000000in}}%
\pgfusepath{stroke,fill}%
}%
\begin{pgfscope}%
\pgfsys@transformshift{0.594525in}{1.308520in}%
\pgfsys@useobject{currentmarker}{}%
\end{pgfscope}%
\end{pgfscope}%
\begin{pgfscope}%
\pgfpathrectangle{\pgfqpoint{0.594525in}{0.417642in}}{\pgfqpoint{3.432047in}{2.016277in}}%
\pgfusepath{clip}%
\pgfsetrectcap%
\pgfsetroundjoin%
\pgfsetlinewidth{0.803000pt}%
\definecolor{currentstroke}{rgb}{0.850000,0.850000,0.850000}%
\pgfsetstrokecolor{currentstroke}%
\pgfsetdash{}{0pt}%
\pgfpathmoveto{\pgfqpoint{0.594525in}{1.348417in}}%
\pgfpathlineto{\pgfqpoint{4.026572in}{1.348417in}}%
\pgfusepath{stroke}%
\end{pgfscope}%
\begin{pgfscope}%
\pgfsetbuttcap%
\pgfsetroundjoin%
\definecolor{currentfill}{rgb}{0.000000,0.000000,0.000000}%
\pgfsetfillcolor{currentfill}%
\pgfsetlinewidth{0.602250pt}%
\definecolor{currentstroke}{rgb}{0.000000,0.000000,0.000000}%
\pgfsetstrokecolor{currentstroke}%
\pgfsetdash{}{0pt}%
\pgfsys@defobject{currentmarker}{\pgfqpoint{-0.027778in}{0.000000in}}{\pgfqpoint{-0.000000in}{0.000000in}}{%
\pgfpathmoveto{\pgfqpoint{-0.000000in}{0.000000in}}%
\pgfpathlineto{\pgfqpoint{-0.027778in}{0.000000in}}%
\pgfusepath{stroke,fill}%
}%
\begin{pgfscope}%
\pgfsys@transformshift{0.594525in}{1.348417in}%
\pgfsys@useobject{currentmarker}{}%
\end{pgfscope}%
\end{pgfscope}%
\begin{pgfscope}%
\pgfpathrectangle{\pgfqpoint{0.594525in}{0.417642in}}{\pgfqpoint{3.432047in}{2.016277in}}%
\pgfusepath{clip}%
\pgfsetrectcap%
\pgfsetroundjoin%
\pgfsetlinewidth{0.803000pt}%
\definecolor{currentstroke}{rgb}{0.850000,0.850000,0.850000}%
\pgfsetstrokecolor{currentstroke}%
\pgfsetdash{}{0pt}%
\pgfpathmoveto{\pgfqpoint{0.594525in}{1.379363in}}%
\pgfpathlineto{\pgfqpoint{4.026572in}{1.379363in}}%
\pgfusepath{stroke}%
\end{pgfscope}%
\begin{pgfscope}%
\pgfsetbuttcap%
\pgfsetroundjoin%
\definecolor{currentfill}{rgb}{0.000000,0.000000,0.000000}%
\pgfsetfillcolor{currentfill}%
\pgfsetlinewidth{0.602250pt}%
\definecolor{currentstroke}{rgb}{0.000000,0.000000,0.000000}%
\pgfsetstrokecolor{currentstroke}%
\pgfsetdash{}{0pt}%
\pgfsys@defobject{currentmarker}{\pgfqpoint{-0.027778in}{0.000000in}}{\pgfqpoint{-0.000000in}{0.000000in}}{%
\pgfpathmoveto{\pgfqpoint{-0.000000in}{0.000000in}}%
\pgfpathlineto{\pgfqpoint{-0.027778in}{0.000000in}}%
\pgfusepath{stroke,fill}%
}%
\begin{pgfscope}%
\pgfsys@transformshift{0.594525in}{1.379363in}%
\pgfsys@useobject{currentmarker}{}%
\end{pgfscope}%
\end{pgfscope}%
\begin{pgfscope}%
\pgfpathrectangle{\pgfqpoint{0.594525in}{0.417642in}}{\pgfqpoint{3.432047in}{2.016277in}}%
\pgfusepath{clip}%
\pgfsetrectcap%
\pgfsetroundjoin%
\pgfsetlinewidth{0.803000pt}%
\definecolor{currentstroke}{rgb}{0.850000,0.850000,0.850000}%
\pgfsetstrokecolor{currentstroke}%
\pgfsetdash{}{0pt}%
\pgfpathmoveto{\pgfqpoint{0.594525in}{1.404648in}}%
\pgfpathlineto{\pgfqpoint{4.026572in}{1.404648in}}%
\pgfusepath{stroke}%
\end{pgfscope}%
\begin{pgfscope}%
\pgfsetbuttcap%
\pgfsetroundjoin%
\definecolor{currentfill}{rgb}{0.000000,0.000000,0.000000}%
\pgfsetfillcolor{currentfill}%
\pgfsetlinewidth{0.602250pt}%
\definecolor{currentstroke}{rgb}{0.000000,0.000000,0.000000}%
\pgfsetstrokecolor{currentstroke}%
\pgfsetdash{}{0pt}%
\pgfsys@defobject{currentmarker}{\pgfqpoint{-0.027778in}{0.000000in}}{\pgfqpoint{-0.000000in}{0.000000in}}{%
\pgfpathmoveto{\pgfqpoint{-0.000000in}{0.000000in}}%
\pgfpathlineto{\pgfqpoint{-0.027778in}{0.000000in}}%
\pgfusepath{stroke,fill}%
}%
\begin{pgfscope}%
\pgfsys@transformshift{0.594525in}{1.404648in}%
\pgfsys@useobject{currentmarker}{}%
\end{pgfscope}%
\end{pgfscope}%
\begin{pgfscope}%
\pgfpathrectangle{\pgfqpoint{0.594525in}{0.417642in}}{\pgfqpoint{3.432047in}{2.016277in}}%
\pgfusepath{clip}%
\pgfsetrectcap%
\pgfsetroundjoin%
\pgfsetlinewidth{0.803000pt}%
\definecolor{currentstroke}{rgb}{0.850000,0.850000,0.850000}%
\pgfsetstrokecolor{currentstroke}%
\pgfsetdash{}{0pt}%
\pgfpathmoveto{\pgfqpoint{0.594525in}{1.426027in}}%
\pgfpathlineto{\pgfqpoint{4.026572in}{1.426027in}}%
\pgfusepath{stroke}%
\end{pgfscope}%
\begin{pgfscope}%
\pgfsetbuttcap%
\pgfsetroundjoin%
\definecolor{currentfill}{rgb}{0.000000,0.000000,0.000000}%
\pgfsetfillcolor{currentfill}%
\pgfsetlinewidth{0.602250pt}%
\definecolor{currentstroke}{rgb}{0.000000,0.000000,0.000000}%
\pgfsetstrokecolor{currentstroke}%
\pgfsetdash{}{0pt}%
\pgfsys@defobject{currentmarker}{\pgfqpoint{-0.027778in}{0.000000in}}{\pgfqpoint{-0.000000in}{0.000000in}}{%
\pgfpathmoveto{\pgfqpoint{-0.000000in}{0.000000in}}%
\pgfpathlineto{\pgfqpoint{-0.027778in}{0.000000in}}%
\pgfusepath{stroke,fill}%
}%
\begin{pgfscope}%
\pgfsys@transformshift{0.594525in}{1.426027in}%
\pgfsys@useobject{currentmarker}{}%
\end{pgfscope}%
\end{pgfscope}%
\begin{pgfscope}%
\pgfpathrectangle{\pgfqpoint{0.594525in}{0.417642in}}{\pgfqpoint{3.432047in}{2.016277in}}%
\pgfusepath{clip}%
\pgfsetrectcap%
\pgfsetroundjoin%
\pgfsetlinewidth{0.803000pt}%
\definecolor{currentstroke}{rgb}{0.850000,0.850000,0.850000}%
\pgfsetstrokecolor{currentstroke}%
\pgfsetdash{}{0pt}%
\pgfpathmoveto{\pgfqpoint{0.594525in}{1.444545in}}%
\pgfpathlineto{\pgfqpoint{4.026572in}{1.444545in}}%
\pgfusepath{stroke}%
\end{pgfscope}%
\begin{pgfscope}%
\pgfsetbuttcap%
\pgfsetroundjoin%
\definecolor{currentfill}{rgb}{0.000000,0.000000,0.000000}%
\pgfsetfillcolor{currentfill}%
\pgfsetlinewidth{0.602250pt}%
\definecolor{currentstroke}{rgb}{0.000000,0.000000,0.000000}%
\pgfsetstrokecolor{currentstroke}%
\pgfsetdash{}{0pt}%
\pgfsys@defobject{currentmarker}{\pgfqpoint{-0.027778in}{0.000000in}}{\pgfqpoint{-0.000000in}{0.000000in}}{%
\pgfpathmoveto{\pgfqpoint{-0.000000in}{0.000000in}}%
\pgfpathlineto{\pgfqpoint{-0.027778in}{0.000000in}}%
\pgfusepath{stroke,fill}%
}%
\begin{pgfscope}%
\pgfsys@transformshift{0.594525in}{1.444545in}%
\pgfsys@useobject{currentmarker}{}%
\end{pgfscope}%
\end{pgfscope}%
\begin{pgfscope}%
\pgfpathrectangle{\pgfqpoint{0.594525in}{0.417642in}}{\pgfqpoint{3.432047in}{2.016277in}}%
\pgfusepath{clip}%
\pgfsetrectcap%
\pgfsetroundjoin%
\pgfsetlinewidth{0.803000pt}%
\definecolor{currentstroke}{rgb}{0.850000,0.850000,0.850000}%
\pgfsetstrokecolor{currentstroke}%
\pgfsetdash{}{0pt}%
\pgfpathmoveto{\pgfqpoint{0.594525in}{1.460880in}}%
\pgfpathlineto{\pgfqpoint{4.026572in}{1.460880in}}%
\pgfusepath{stroke}%
\end{pgfscope}%
\begin{pgfscope}%
\pgfsetbuttcap%
\pgfsetroundjoin%
\definecolor{currentfill}{rgb}{0.000000,0.000000,0.000000}%
\pgfsetfillcolor{currentfill}%
\pgfsetlinewidth{0.602250pt}%
\definecolor{currentstroke}{rgb}{0.000000,0.000000,0.000000}%
\pgfsetstrokecolor{currentstroke}%
\pgfsetdash{}{0pt}%
\pgfsys@defobject{currentmarker}{\pgfqpoint{-0.027778in}{0.000000in}}{\pgfqpoint{-0.000000in}{0.000000in}}{%
\pgfpathmoveto{\pgfqpoint{-0.000000in}{0.000000in}}%
\pgfpathlineto{\pgfqpoint{-0.027778in}{0.000000in}}%
\pgfusepath{stroke,fill}%
}%
\begin{pgfscope}%
\pgfsys@transformshift{0.594525in}{1.460880in}%
\pgfsys@useobject{currentmarker}{}%
\end{pgfscope}%
\end{pgfscope}%
\begin{pgfscope}%
\pgfpathrectangle{\pgfqpoint{0.594525in}{0.417642in}}{\pgfqpoint{3.432047in}{2.016277in}}%
\pgfusepath{clip}%
\pgfsetrectcap%
\pgfsetroundjoin%
\pgfsetlinewidth{0.803000pt}%
\definecolor{currentstroke}{rgb}{0.850000,0.850000,0.850000}%
\pgfsetstrokecolor{currentstroke}%
\pgfsetdash{}{0pt}%
\pgfpathmoveto{\pgfqpoint{0.594525in}{1.571620in}}%
\pgfpathlineto{\pgfqpoint{4.026572in}{1.571620in}}%
\pgfusepath{stroke}%
\end{pgfscope}%
\begin{pgfscope}%
\pgfsetbuttcap%
\pgfsetroundjoin%
\definecolor{currentfill}{rgb}{0.000000,0.000000,0.000000}%
\pgfsetfillcolor{currentfill}%
\pgfsetlinewidth{0.602250pt}%
\definecolor{currentstroke}{rgb}{0.000000,0.000000,0.000000}%
\pgfsetstrokecolor{currentstroke}%
\pgfsetdash{}{0pt}%
\pgfsys@defobject{currentmarker}{\pgfqpoint{-0.027778in}{0.000000in}}{\pgfqpoint{-0.000000in}{0.000000in}}{%
\pgfpathmoveto{\pgfqpoint{-0.000000in}{0.000000in}}%
\pgfpathlineto{\pgfqpoint{-0.027778in}{0.000000in}}%
\pgfusepath{stroke,fill}%
}%
\begin{pgfscope}%
\pgfsys@transformshift{0.594525in}{1.571620in}%
\pgfsys@useobject{currentmarker}{}%
\end{pgfscope}%
\end{pgfscope}%
\begin{pgfscope}%
\pgfpathrectangle{\pgfqpoint{0.594525in}{0.417642in}}{\pgfqpoint{3.432047in}{2.016277in}}%
\pgfusepath{clip}%
\pgfsetrectcap%
\pgfsetroundjoin%
\pgfsetlinewidth{0.803000pt}%
\definecolor{currentstroke}{rgb}{0.850000,0.850000,0.850000}%
\pgfsetstrokecolor{currentstroke}%
\pgfsetdash{}{0pt}%
\pgfpathmoveto{\pgfqpoint{0.594525in}{1.627852in}}%
\pgfpathlineto{\pgfqpoint{4.026572in}{1.627852in}}%
\pgfusepath{stroke}%
\end{pgfscope}%
\begin{pgfscope}%
\pgfsetbuttcap%
\pgfsetroundjoin%
\definecolor{currentfill}{rgb}{0.000000,0.000000,0.000000}%
\pgfsetfillcolor{currentfill}%
\pgfsetlinewidth{0.602250pt}%
\definecolor{currentstroke}{rgb}{0.000000,0.000000,0.000000}%
\pgfsetstrokecolor{currentstroke}%
\pgfsetdash{}{0pt}%
\pgfsys@defobject{currentmarker}{\pgfqpoint{-0.027778in}{0.000000in}}{\pgfqpoint{-0.000000in}{0.000000in}}{%
\pgfpathmoveto{\pgfqpoint{-0.000000in}{0.000000in}}%
\pgfpathlineto{\pgfqpoint{-0.027778in}{0.000000in}}%
\pgfusepath{stroke,fill}%
}%
\begin{pgfscope}%
\pgfsys@transformshift{0.594525in}{1.627852in}%
\pgfsys@useobject{currentmarker}{}%
\end{pgfscope}%
\end{pgfscope}%
\begin{pgfscope}%
\pgfpathrectangle{\pgfqpoint{0.594525in}{0.417642in}}{\pgfqpoint{3.432047in}{2.016277in}}%
\pgfusepath{clip}%
\pgfsetrectcap%
\pgfsetroundjoin%
\pgfsetlinewidth{0.803000pt}%
\definecolor{currentstroke}{rgb}{0.850000,0.850000,0.850000}%
\pgfsetstrokecolor{currentstroke}%
\pgfsetdash{}{0pt}%
\pgfpathmoveto{\pgfqpoint{0.594525in}{1.667749in}}%
\pgfpathlineto{\pgfqpoint{4.026572in}{1.667749in}}%
\pgfusepath{stroke}%
\end{pgfscope}%
\begin{pgfscope}%
\pgfsetbuttcap%
\pgfsetroundjoin%
\definecolor{currentfill}{rgb}{0.000000,0.000000,0.000000}%
\pgfsetfillcolor{currentfill}%
\pgfsetlinewidth{0.602250pt}%
\definecolor{currentstroke}{rgb}{0.000000,0.000000,0.000000}%
\pgfsetstrokecolor{currentstroke}%
\pgfsetdash{}{0pt}%
\pgfsys@defobject{currentmarker}{\pgfqpoint{-0.027778in}{0.000000in}}{\pgfqpoint{-0.000000in}{0.000000in}}{%
\pgfpathmoveto{\pgfqpoint{-0.000000in}{0.000000in}}%
\pgfpathlineto{\pgfqpoint{-0.027778in}{0.000000in}}%
\pgfusepath{stroke,fill}%
}%
\begin{pgfscope}%
\pgfsys@transformshift{0.594525in}{1.667749in}%
\pgfsys@useobject{currentmarker}{}%
\end{pgfscope}%
\end{pgfscope}%
\begin{pgfscope}%
\pgfpathrectangle{\pgfqpoint{0.594525in}{0.417642in}}{\pgfqpoint{3.432047in}{2.016277in}}%
\pgfusepath{clip}%
\pgfsetrectcap%
\pgfsetroundjoin%
\pgfsetlinewidth{0.803000pt}%
\definecolor{currentstroke}{rgb}{0.850000,0.850000,0.850000}%
\pgfsetstrokecolor{currentstroke}%
\pgfsetdash{}{0pt}%
\pgfpathmoveto{\pgfqpoint{0.594525in}{1.698696in}}%
\pgfpathlineto{\pgfqpoint{4.026572in}{1.698696in}}%
\pgfusepath{stroke}%
\end{pgfscope}%
\begin{pgfscope}%
\pgfsetbuttcap%
\pgfsetroundjoin%
\definecolor{currentfill}{rgb}{0.000000,0.000000,0.000000}%
\pgfsetfillcolor{currentfill}%
\pgfsetlinewidth{0.602250pt}%
\definecolor{currentstroke}{rgb}{0.000000,0.000000,0.000000}%
\pgfsetstrokecolor{currentstroke}%
\pgfsetdash{}{0pt}%
\pgfsys@defobject{currentmarker}{\pgfqpoint{-0.027778in}{0.000000in}}{\pgfqpoint{-0.000000in}{0.000000in}}{%
\pgfpathmoveto{\pgfqpoint{-0.000000in}{0.000000in}}%
\pgfpathlineto{\pgfqpoint{-0.027778in}{0.000000in}}%
\pgfusepath{stroke,fill}%
}%
\begin{pgfscope}%
\pgfsys@transformshift{0.594525in}{1.698696in}%
\pgfsys@useobject{currentmarker}{}%
\end{pgfscope}%
\end{pgfscope}%
\begin{pgfscope}%
\pgfpathrectangle{\pgfqpoint{0.594525in}{0.417642in}}{\pgfqpoint{3.432047in}{2.016277in}}%
\pgfusepath{clip}%
\pgfsetrectcap%
\pgfsetroundjoin%
\pgfsetlinewidth{0.803000pt}%
\definecolor{currentstroke}{rgb}{0.850000,0.850000,0.850000}%
\pgfsetstrokecolor{currentstroke}%
\pgfsetdash{}{0pt}%
\pgfpathmoveto{\pgfqpoint{0.594525in}{1.723981in}}%
\pgfpathlineto{\pgfqpoint{4.026572in}{1.723981in}}%
\pgfusepath{stroke}%
\end{pgfscope}%
\begin{pgfscope}%
\pgfsetbuttcap%
\pgfsetroundjoin%
\definecolor{currentfill}{rgb}{0.000000,0.000000,0.000000}%
\pgfsetfillcolor{currentfill}%
\pgfsetlinewidth{0.602250pt}%
\definecolor{currentstroke}{rgb}{0.000000,0.000000,0.000000}%
\pgfsetstrokecolor{currentstroke}%
\pgfsetdash{}{0pt}%
\pgfsys@defobject{currentmarker}{\pgfqpoint{-0.027778in}{0.000000in}}{\pgfqpoint{-0.000000in}{0.000000in}}{%
\pgfpathmoveto{\pgfqpoint{-0.000000in}{0.000000in}}%
\pgfpathlineto{\pgfqpoint{-0.027778in}{0.000000in}}%
\pgfusepath{stroke,fill}%
}%
\begin{pgfscope}%
\pgfsys@transformshift{0.594525in}{1.723981in}%
\pgfsys@useobject{currentmarker}{}%
\end{pgfscope}%
\end{pgfscope}%
\begin{pgfscope}%
\pgfpathrectangle{\pgfqpoint{0.594525in}{0.417642in}}{\pgfqpoint{3.432047in}{2.016277in}}%
\pgfusepath{clip}%
\pgfsetrectcap%
\pgfsetroundjoin%
\pgfsetlinewidth{0.803000pt}%
\definecolor{currentstroke}{rgb}{0.850000,0.850000,0.850000}%
\pgfsetstrokecolor{currentstroke}%
\pgfsetdash{}{0pt}%
\pgfpathmoveto{\pgfqpoint{0.594525in}{1.745359in}}%
\pgfpathlineto{\pgfqpoint{4.026572in}{1.745359in}}%
\pgfusepath{stroke}%
\end{pgfscope}%
\begin{pgfscope}%
\pgfsetbuttcap%
\pgfsetroundjoin%
\definecolor{currentfill}{rgb}{0.000000,0.000000,0.000000}%
\pgfsetfillcolor{currentfill}%
\pgfsetlinewidth{0.602250pt}%
\definecolor{currentstroke}{rgb}{0.000000,0.000000,0.000000}%
\pgfsetstrokecolor{currentstroke}%
\pgfsetdash{}{0pt}%
\pgfsys@defobject{currentmarker}{\pgfqpoint{-0.027778in}{0.000000in}}{\pgfqpoint{-0.000000in}{0.000000in}}{%
\pgfpathmoveto{\pgfqpoint{-0.000000in}{0.000000in}}%
\pgfpathlineto{\pgfqpoint{-0.027778in}{0.000000in}}%
\pgfusepath{stroke,fill}%
}%
\begin{pgfscope}%
\pgfsys@transformshift{0.594525in}{1.745359in}%
\pgfsys@useobject{currentmarker}{}%
\end{pgfscope}%
\end{pgfscope}%
\begin{pgfscope}%
\pgfpathrectangle{\pgfqpoint{0.594525in}{0.417642in}}{\pgfqpoint{3.432047in}{2.016277in}}%
\pgfusepath{clip}%
\pgfsetrectcap%
\pgfsetroundjoin%
\pgfsetlinewidth{0.803000pt}%
\definecolor{currentstroke}{rgb}{0.850000,0.850000,0.850000}%
\pgfsetstrokecolor{currentstroke}%
\pgfsetdash{}{0pt}%
\pgfpathmoveto{\pgfqpoint{0.594525in}{1.763878in}}%
\pgfpathlineto{\pgfqpoint{4.026572in}{1.763878in}}%
\pgfusepath{stroke}%
\end{pgfscope}%
\begin{pgfscope}%
\pgfsetbuttcap%
\pgfsetroundjoin%
\definecolor{currentfill}{rgb}{0.000000,0.000000,0.000000}%
\pgfsetfillcolor{currentfill}%
\pgfsetlinewidth{0.602250pt}%
\definecolor{currentstroke}{rgb}{0.000000,0.000000,0.000000}%
\pgfsetstrokecolor{currentstroke}%
\pgfsetdash{}{0pt}%
\pgfsys@defobject{currentmarker}{\pgfqpoint{-0.027778in}{0.000000in}}{\pgfqpoint{-0.000000in}{0.000000in}}{%
\pgfpathmoveto{\pgfqpoint{-0.000000in}{0.000000in}}%
\pgfpathlineto{\pgfqpoint{-0.027778in}{0.000000in}}%
\pgfusepath{stroke,fill}%
}%
\begin{pgfscope}%
\pgfsys@transformshift{0.594525in}{1.763878in}%
\pgfsys@useobject{currentmarker}{}%
\end{pgfscope}%
\end{pgfscope}%
\begin{pgfscope}%
\pgfpathrectangle{\pgfqpoint{0.594525in}{0.417642in}}{\pgfqpoint{3.432047in}{2.016277in}}%
\pgfusepath{clip}%
\pgfsetrectcap%
\pgfsetroundjoin%
\pgfsetlinewidth{0.803000pt}%
\definecolor{currentstroke}{rgb}{0.850000,0.850000,0.850000}%
\pgfsetstrokecolor{currentstroke}%
\pgfsetdash{}{0pt}%
\pgfpathmoveto{\pgfqpoint{0.594525in}{1.780212in}}%
\pgfpathlineto{\pgfqpoint{4.026572in}{1.780212in}}%
\pgfusepath{stroke}%
\end{pgfscope}%
\begin{pgfscope}%
\pgfsetbuttcap%
\pgfsetroundjoin%
\definecolor{currentfill}{rgb}{0.000000,0.000000,0.000000}%
\pgfsetfillcolor{currentfill}%
\pgfsetlinewidth{0.602250pt}%
\definecolor{currentstroke}{rgb}{0.000000,0.000000,0.000000}%
\pgfsetstrokecolor{currentstroke}%
\pgfsetdash{}{0pt}%
\pgfsys@defobject{currentmarker}{\pgfqpoint{-0.027778in}{0.000000in}}{\pgfqpoint{-0.000000in}{0.000000in}}{%
\pgfpathmoveto{\pgfqpoint{-0.000000in}{0.000000in}}%
\pgfpathlineto{\pgfqpoint{-0.027778in}{0.000000in}}%
\pgfusepath{stroke,fill}%
}%
\begin{pgfscope}%
\pgfsys@transformshift{0.594525in}{1.780212in}%
\pgfsys@useobject{currentmarker}{}%
\end{pgfscope}%
\end{pgfscope}%
\begin{pgfscope}%
\pgfpathrectangle{\pgfqpoint{0.594525in}{0.417642in}}{\pgfqpoint{3.432047in}{2.016277in}}%
\pgfusepath{clip}%
\pgfsetrectcap%
\pgfsetroundjoin%
\pgfsetlinewidth{0.803000pt}%
\definecolor{currentstroke}{rgb}{0.850000,0.850000,0.850000}%
\pgfsetstrokecolor{currentstroke}%
\pgfsetdash{}{0pt}%
\pgfpathmoveto{\pgfqpoint{0.594525in}{1.890953in}}%
\pgfpathlineto{\pgfqpoint{4.026572in}{1.890953in}}%
\pgfusepath{stroke}%
\end{pgfscope}%
\begin{pgfscope}%
\pgfsetbuttcap%
\pgfsetroundjoin%
\definecolor{currentfill}{rgb}{0.000000,0.000000,0.000000}%
\pgfsetfillcolor{currentfill}%
\pgfsetlinewidth{0.602250pt}%
\definecolor{currentstroke}{rgb}{0.000000,0.000000,0.000000}%
\pgfsetstrokecolor{currentstroke}%
\pgfsetdash{}{0pt}%
\pgfsys@defobject{currentmarker}{\pgfqpoint{-0.027778in}{0.000000in}}{\pgfqpoint{-0.000000in}{0.000000in}}{%
\pgfpathmoveto{\pgfqpoint{-0.000000in}{0.000000in}}%
\pgfpathlineto{\pgfqpoint{-0.027778in}{0.000000in}}%
\pgfusepath{stroke,fill}%
}%
\begin{pgfscope}%
\pgfsys@transformshift{0.594525in}{1.890953in}%
\pgfsys@useobject{currentmarker}{}%
\end{pgfscope}%
\end{pgfscope}%
\begin{pgfscope}%
\pgfpathrectangle{\pgfqpoint{0.594525in}{0.417642in}}{\pgfqpoint{3.432047in}{2.016277in}}%
\pgfusepath{clip}%
\pgfsetrectcap%
\pgfsetroundjoin%
\pgfsetlinewidth{0.803000pt}%
\definecolor{currentstroke}{rgb}{0.850000,0.850000,0.850000}%
\pgfsetstrokecolor{currentstroke}%
\pgfsetdash{}{0pt}%
\pgfpathmoveto{\pgfqpoint{0.594525in}{1.947184in}}%
\pgfpathlineto{\pgfqpoint{4.026572in}{1.947184in}}%
\pgfusepath{stroke}%
\end{pgfscope}%
\begin{pgfscope}%
\pgfsetbuttcap%
\pgfsetroundjoin%
\definecolor{currentfill}{rgb}{0.000000,0.000000,0.000000}%
\pgfsetfillcolor{currentfill}%
\pgfsetlinewidth{0.602250pt}%
\definecolor{currentstroke}{rgb}{0.000000,0.000000,0.000000}%
\pgfsetstrokecolor{currentstroke}%
\pgfsetdash{}{0pt}%
\pgfsys@defobject{currentmarker}{\pgfqpoint{-0.027778in}{0.000000in}}{\pgfqpoint{-0.000000in}{0.000000in}}{%
\pgfpathmoveto{\pgfqpoint{-0.000000in}{0.000000in}}%
\pgfpathlineto{\pgfqpoint{-0.027778in}{0.000000in}}%
\pgfusepath{stroke,fill}%
}%
\begin{pgfscope}%
\pgfsys@transformshift{0.594525in}{1.947184in}%
\pgfsys@useobject{currentmarker}{}%
\end{pgfscope}%
\end{pgfscope}%
\begin{pgfscope}%
\pgfpathrectangle{\pgfqpoint{0.594525in}{0.417642in}}{\pgfqpoint{3.432047in}{2.016277in}}%
\pgfusepath{clip}%
\pgfsetrectcap%
\pgfsetroundjoin%
\pgfsetlinewidth{0.803000pt}%
\definecolor{currentstroke}{rgb}{0.850000,0.850000,0.850000}%
\pgfsetstrokecolor{currentstroke}%
\pgfsetdash{}{0pt}%
\pgfpathmoveto{\pgfqpoint{0.594525in}{1.987081in}}%
\pgfpathlineto{\pgfqpoint{4.026572in}{1.987081in}}%
\pgfusepath{stroke}%
\end{pgfscope}%
\begin{pgfscope}%
\pgfsetbuttcap%
\pgfsetroundjoin%
\definecolor{currentfill}{rgb}{0.000000,0.000000,0.000000}%
\pgfsetfillcolor{currentfill}%
\pgfsetlinewidth{0.602250pt}%
\definecolor{currentstroke}{rgb}{0.000000,0.000000,0.000000}%
\pgfsetstrokecolor{currentstroke}%
\pgfsetdash{}{0pt}%
\pgfsys@defobject{currentmarker}{\pgfqpoint{-0.027778in}{0.000000in}}{\pgfqpoint{-0.000000in}{0.000000in}}{%
\pgfpathmoveto{\pgfqpoint{-0.000000in}{0.000000in}}%
\pgfpathlineto{\pgfqpoint{-0.027778in}{0.000000in}}%
\pgfusepath{stroke,fill}%
}%
\begin{pgfscope}%
\pgfsys@transformshift{0.594525in}{1.987081in}%
\pgfsys@useobject{currentmarker}{}%
\end{pgfscope}%
\end{pgfscope}%
\begin{pgfscope}%
\pgfpathrectangle{\pgfqpoint{0.594525in}{0.417642in}}{\pgfqpoint{3.432047in}{2.016277in}}%
\pgfusepath{clip}%
\pgfsetrectcap%
\pgfsetroundjoin%
\pgfsetlinewidth{0.803000pt}%
\definecolor{currentstroke}{rgb}{0.850000,0.850000,0.850000}%
\pgfsetstrokecolor{currentstroke}%
\pgfsetdash{}{0pt}%
\pgfpathmoveto{\pgfqpoint{0.594525in}{2.018028in}}%
\pgfpathlineto{\pgfqpoint{4.026572in}{2.018028in}}%
\pgfusepath{stroke}%
\end{pgfscope}%
\begin{pgfscope}%
\pgfsetbuttcap%
\pgfsetroundjoin%
\definecolor{currentfill}{rgb}{0.000000,0.000000,0.000000}%
\pgfsetfillcolor{currentfill}%
\pgfsetlinewidth{0.602250pt}%
\definecolor{currentstroke}{rgb}{0.000000,0.000000,0.000000}%
\pgfsetstrokecolor{currentstroke}%
\pgfsetdash{}{0pt}%
\pgfsys@defobject{currentmarker}{\pgfqpoint{-0.027778in}{0.000000in}}{\pgfqpoint{-0.000000in}{0.000000in}}{%
\pgfpathmoveto{\pgfqpoint{-0.000000in}{0.000000in}}%
\pgfpathlineto{\pgfqpoint{-0.027778in}{0.000000in}}%
\pgfusepath{stroke,fill}%
}%
\begin{pgfscope}%
\pgfsys@transformshift{0.594525in}{2.018028in}%
\pgfsys@useobject{currentmarker}{}%
\end{pgfscope}%
\end{pgfscope}%
\begin{pgfscope}%
\pgfpathrectangle{\pgfqpoint{0.594525in}{0.417642in}}{\pgfqpoint{3.432047in}{2.016277in}}%
\pgfusepath{clip}%
\pgfsetrectcap%
\pgfsetroundjoin%
\pgfsetlinewidth{0.803000pt}%
\definecolor{currentstroke}{rgb}{0.850000,0.850000,0.850000}%
\pgfsetstrokecolor{currentstroke}%
\pgfsetdash{}{0pt}%
\pgfpathmoveto{\pgfqpoint{0.594525in}{2.043313in}}%
\pgfpathlineto{\pgfqpoint{4.026572in}{2.043313in}}%
\pgfusepath{stroke}%
\end{pgfscope}%
\begin{pgfscope}%
\pgfsetbuttcap%
\pgfsetroundjoin%
\definecolor{currentfill}{rgb}{0.000000,0.000000,0.000000}%
\pgfsetfillcolor{currentfill}%
\pgfsetlinewidth{0.602250pt}%
\definecolor{currentstroke}{rgb}{0.000000,0.000000,0.000000}%
\pgfsetstrokecolor{currentstroke}%
\pgfsetdash{}{0pt}%
\pgfsys@defobject{currentmarker}{\pgfqpoint{-0.027778in}{0.000000in}}{\pgfqpoint{-0.000000in}{0.000000in}}{%
\pgfpathmoveto{\pgfqpoint{-0.000000in}{0.000000in}}%
\pgfpathlineto{\pgfqpoint{-0.027778in}{0.000000in}}%
\pgfusepath{stroke,fill}%
}%
\begin{pgfscope}%
\pgfsys@transformshift{0.594525in}{2.043313in}%
\pgfsys@useobject{currentmarker}{}%
\end{pgfscope}%
\end{pgfscope}%
\begin{pgfscope}%
\pgfpathrectangle{\pgfqpoint{0.594525in}{0.417642in}}{\pgfqpoint{3.432047in}{2.016277in}}%
\pgfusepath{clip}%
\pgfsetrectcap%
\pgfsetroundjoin%
\pgfsetlinewidth{0.803000pt}%
\definecolor{currentstroke}{rgb}{0.850000,0.850000,0.850000}%
\pgfsetstrokecolor{currentstroke}%
\pgfsetdash{}{0pt}%
\pgfpathmoveto{\pgfqpoint{0.594525in}{2.064691in}}%
\pgfpathlineto{\pgfqpoint{4.026572in}{2.064691in}}%
\pgfusepath{stroke}%
\end{pgfscope}%
\begin{pgfscope}%
\pgfsetbuttcap%
\pgfsetroundjoin%
\definecolor{currentfill}{rgb}{0.000000,0.000000,0.000000}%
\pgfsetfillcolor{currentfill}%
\pgfsetlinewidth{0.602250pt}%
\definecolor{currentstroke}{rgb}{0.000000,0.000000,0.000000}%
\pgfsetstrokecolor{currentstroke}%
\pgfsetdash{}{0pt}%
\pgfsys@defobject{currentmarker}{\pgfqpoint{-0.027778in}{0.000000in}}{\pgfqpoint{-0.000000in}{0.000000in}}{%
\pgfpathmoveto{\pgfqpoint{-0.000000in}{0.000000in}}%
\pgfpathlineto{\pgfqpoint{-0.027778in}{0.000000in}}%
\pgfusepath{stroke,fill}%
}%
\begin{pgfscope}%
\pgfsys@transformshift{0.594525in}{2.064691in}%
\pgfsys@useobject{currentmarker}{}%
\end{pgfscope}%
\end{pgfscope}%
\begin{pgfscope}%
\pgfpathrectangle{\pgfqpoint{0.594525in}{0.417642in}}{\pgfqpoint{3.432047in}{2.016277in}}%
\pgfusepath{clip}%
\pgfsetrectcap%
\pgfsetroundjoin%
\pgfsetlinewidth{0.803000pt}%
\definecolor{currentstroke}{rgb}{0.850000,0.850000,0.850000}%
\pgfsetstrokecolor{currentstroke}%
\pgfsetdash{}{0pt}%
\pgfpathmoveto{\pgfqpoint{0.594525in}{2.083210in}}%
\pgfpathlineto{\pgfqpoint{4.026572in}{2.083210in}}%
\pgfusepath{stroke}%
\end{pgfscope}%
\begin{pgfscope}%
\pgfsetbuttcap%
\pgfsetroundjoin%
\definecolor{currentfill}{rgb}{0.000000,0.000000,0.000000}%
\pgfsetfillcolor{currentfill}%
\pgfsetlinewidth{0.602250pt}%
\definecolor{currentstroke}{rgb}{0.000000,0.000000,0.000000}%
\pgfsetstrokecolor{currentstroke}%
\pgfsetdash{}{0pt}%
\pgfsys@defobject{currentmarker}{\pgfqpoint{-0.027778in}{0.000000in}}{\pgfqpoint{-0.000000in}{0.000000in}}{%
\pgfpathmoveto{\pgfqpoint{-0.000000in}{0.000000in}}%
\pgfpathlineto{\pgfqpoint{-0.027778in}{0.000000in}}%
\pgfusepath{stroke,fill}%
}%
\begin{pgfscope}%
\pgfsys@transformshift{0.594525in}{2.083210in}%
\pgfsys@useobject{currentmarker}{}%
\end{pgfscope}%
\end{pgfscope}%
\begin{pgfscope}%
\pgfpathrectangle{\pgfqpoint{0.594525in}{0.417642in}}{\pgfqpoint{3.432047in}{2.016277in}}%
\pgfusepath{clip}%
\pgfsetrectcap%
\pgfsetroundjoin%
\pgfsetlinewidth{0.803000pt}%
\definecolor{currentstroke}{rgb}{0.850000,0.850000,0.850000}%
\pgfsetstrokecolor{currentstroke}%
\pgfsetdash{}{0pt}%
\pgfpathmoveto{\pgfqpoint{0.594525in}{2.099545in}}%
\pgfpathlineto{\pgfqpoint{4.026572in}{2.099545in}}%
\pgfusepath{stroke}%
\end{pgfscope}%
\begin{pgfscope}%
\pgfsetbuttcap%
\pgfsetroundjoin%
\definecolor{currentfill}{rgb}{0.000000,0.000000,0.000000}%
\pgfsetfillcolor{currentfill}%
\pgfsetlinewidth{0.602250pt}%
\definecolor{currentstroke}{rgb}{0.000000,0.000000,0.000000}%
\pgfsetstrokecolor{currentstroke}%
\pgfsetdash{}{0pt}%
\pgfsys@defobject{currentmarker}{\pgfqpoint{-0.027778in}{0.000000in}}{\pgfqpoint{-0.000000in}{0.000000in}}{%
\pgfpathmoveto{\pgfqpoint{-0.000000in}{0.000000in}}%
\pgfpathlineto{\pgfqpoint{-0.027778in}{0.000000in}}%
\pgfusepath{stroke,fill}%
}%
\begin{pgfscope}%
\pgfsys@transformshift{0.594525in}{2.099545in}%
\pgfsys@useobject{currentmarker}{}%
\end{pgfscope}%
\end{pgfscope}%
\begin{pgfscope}%
\pgfpathrectangle{\pgfqpoint{0.594525in}{0.417642in}}{\pgfqpoint{3.432047in}{2.016277in}}%
\pgfusepath{clip}%
\pgfsetrectcap%
\pgfsetroundjoin%
\pgfsetlinewidth{0.803000pt}%
\definecolor{currentstroke}{rgb}{0.850000,0.850000,0.850000}%
\pgfsetstrokecolor{currentstroke}%
\pgfsetdash{}{0pt}%
\pgfpathmoveto{\pgfqpoint{0.594525in}{2.210285in}}%
\pgfpathlineto{\pgfqpoint{4.026572in}{2.210285in}}%
\pgfusepath{stroke}%
\end{pgfscope}%
\begin{pgfscope}%
\pgfsetbuttcap%
\pgfsetroundjoin%
\definecolor{currentfill}{rgb}{0.000000,0.000000,0.000000}%
\pgfsetfillcolor{currentfill}%
\pgfsetlinewidth{0.602250pt}%
\definecolor{currentstroke}{rgb}{0.000000,0.000000,0.000000}%
\pgfsetstrokecolor{currentstroke}%
\pgfsetdash{}{0pt}%
\pgfsys@defobject{currentmarker}{\pgfqpoint{-0.027778in}{0.000000in}}{\pgfqpoint{-0.000000in}{0.000000in}}{%
\pgfpathmoveto{\pgfqpoint{-0.000000in}{0.000000in}}%
\pgfpathlineto{\pgfqpoint{-0.027778in}{0.000000in}}%
\pgfusepath{stroke,fill}%
}%
\begin{pgfscope}%
\pgfsys@transformshift{0.594525in}{2.210285in}%
\pgfsys@useobject{currentmarker}{}%
\end{pgfscope}%
\end{pgfscope}%
\begin{pgfscope}%
\pgfpathrectangle{\pgfqpoint{0.594525in}{0.417642in}}{\pgfqpoint{3.432047in}{2.016277in}}%
\pgfusepath{clip}%
\pgfsetrectcap%
\pgfsetroundjoin%
\pgfsetlinewidth{0.803000pt}%
\definecolor{currentstroke}{rgb}{0.850000,0.850000,0.850000}%
\pgfsetstrokecolor{currentstroke}%
\pgfsetdash{}{0pt}%
\pgfpathmoveto{\pgfqpoint{0.594525in}{2.266517in}}%
\pgfpathlineto{\pgfqpoint{4.026572in}{2.266517in}}%
\pgfusepath{stroke}%
\end{pgfscope}%
\begin{pgfscope}%
\pgfsetbuttcap%
\pgfsetroundjoin%
\definecolor{currentfill}{rgb}{0.000000,0.000000,0.000000}%
\pgfsetfillcolor{currentfill}%
\pgfsetlinewidth{0.602250pt}%
\definecolor{currentstroke}{rgb}{0.000000,0.000000,0.000000}%
\pgfsetstrokecolor{currentstroke}%
\pgfsetdash{}{0pt}%
\pgfsys@defobject{currentmarker}{\pgfqpoint{-0.027778in}{0.000000in}}{\pgfqpoint{-0.000000in}{0.000000in}}{%
\pgfpathmoveto{\pgfqpoint{-0.000000in}{0.000000in}}%
\pgfpathlineto{\pgfqpoint{-0.027778in}{0.000000in}}%
\pgfusepath{stroke,fill}%
}%
\begin{pgfscope}%
\pgfsys@transformshift{0.594525in}{2.266517in}%
\pgfsys@useobject{currentmarker}{}%
\end{pgfscope}%
\end{pgfscope}%
\begin{pgfscope}%
\pgfpathrectangle{\pgfqpoint{0.594525in}{0.417642in}}{\pgfqpoint{3.432047in}{2.016277in}}%
\pgfusepath{clip}%
\pgfsetrectcap%
\pgfsetroundjoin%
\pgfsetlinewidth{0.803000pt}%
\definecolor{currentstroke}{rgb}{0.850000,0.850000,0.850000}%
\pgfsetstrokecolor{currentstroke}%
\pgfsetdash{}{0pt}%
\pgfpathmoveto{\pgfqpoint{0.594525in}{2.306414in}}%
\pgfpathlineto{\pgfqpoint{4.026572in}{2.306414in}}%
\pgfusepath{stroke}%
\end{pgfscope}%
\begin{pgfscope}%
\pgfsetbuttcap%
\pgfsetroundjoin%
\definecolor{currentfill}{rgb}{0.000000,0.000000,0.000000}%
\pgfsetfillcolor{currentfill}%
\pgfsetlinewidth{0.602250pt}%
\definecolor{currentstroke}{rgb}{0.000000,0.000000,0.000000}%
\pgfsetstrokecolor{currentstroke}%
\pgfsetdash{}{0pt}%
\pgfsys@defobject{currentmarker}{\pgfqpoint{-0.027778in}{0.000000in}}{\pgfqpoint{-0.000000in}{0.000000in}}{%
\pgfpathmoveto{\pgfqpoint{-0.000000in}{0.000000in}}%
\pgfpathlineto{\pgfqpoint{-0.027778in}{0.000000in}}%
\pgfusepath{stroke,fill}%
}%
\begin{pgfscope}%
\pgfsys@transformshift{0.594525in}{2.306414in}%
\pgfsys@useobject{currentmarker}{}%
\end{pgfscope}%
\end{pgfscope}%
\begin{pgfscope}%
\pgfpathrectangle{\pgfqpoint{0.594525in}{0.417642in}}{\pgfqpoint{3.432047in}{2.016277in}}%
\pgfusepath{clip}%
\pgfsetrectcap%
\pgfsetroundjoin%
\pgfsetlinewidth{0.803000pt}%
\definecolor{currentstroke}{rgb}{0.850000,0.850000,0.850000}%
\pgfsetstrokecolor{currentstroke}%
\pgfsetdash{}{0pt}%
\pgfpathmoveto{\pgfqpoint{0.594525in}{2.337360in}}%
\pgfpathlineto{\pgfqpoint{4.026572in}{2.337360in}}%
\pgfusepath{stroke}%
\end{pgfscope}%
\begin{pgfscope}%
\pgfsetbuttcap%
\pgfsetroundjoin%
\definecolor{currentfill}{rgb}{0.000000,0.000000,0.000000}%
\pgfsetfillcolor{currentfill}%
\pgfsetlinewidth{0.602250pt}%
\definecolor{currentstroke}{rgb}{0.000000,0.000000,0.000000}%
\pgfsetstrokecolor{currentstroke}%
\pgfsetdash{}{0pt}%
\pgfsys@defobject{currentmarker}{\pgfqpoint{-0.027778in}{0.000000in}}{\pgfqpoint{-0.000000in}{0.000000in}}{%
\pgfpathmoveto{\pgfqpoint{-0.000000in}{0.000000in}}%
\pgfpathlineto{\pgfqpoint{-0.027778in}{0.000000in}}%
\pgfusepath{stroke,fill}%
}%
\begin{pgfscope}%
\pgfsys@transformshift{0.594525in}{2.337360in}%
\pgfsys@useobject{currentmarker}{}%
\end{pgfscope}%
\end{pgfscope}%
\begin{pgfscope}%
\pgfpathrectangle{\pgfqpoint{0.594525in}{0.417642in}}{\pgfqpoint{3.432047in}{2.016277in}}%
\pgfusepath{clip}%
\pgfsetrectcap%
\pgfsetroundjoin%
\pgfsetlinewidth{0.803000pt}%
\definecolor{currentstroke}{rgb}{0.850000,0.850000,0.850000}%
\pgfsetstrokecolor{currentstroke}%
\pgfsetdash{}{0pt}%
\pgfpathmoveto{\pgfqpoint{0.594525in}{2.362645in}}%
\pgfpathlineto{\pgfqpoint{4.026572in}{2.362645in}}%
\pgfusepath{stroke}%
\end{pgfscope}%
\begin{pgfscope}%
\pgfsetbuttcap%
\pgfsetroundjoin%
\definecolor{currentfill}{rgb}{0.000000,0.000000,0.000000}%
\pgfsetfillcolor{currentfill}%
\pgfsetlinewidth{0.602250pt}%
\definecolor{currentstroke}{rgb}{0.000000,0.000000,0.000000}%
\pgfsetstrokecolor{currentstroke}%
\pgfsetdash{}{0pt}%
\pgfsys@defobject{currentmarker}{\pgfqpoint{-0.027778in}{0.000000in}}{\pgfqpoint{-0.000000in}{0.000000in}}{%
\pgfpathmoveto{\pgfqpoint{-0.000000in}{0.000000in}}%
\pgfpathlineto{\pgfqpoint{-0.027778in}{0.000000in}}%
\pgfusepath{stroke,fill}%
}%
\begin{pgfscope}%
\pgfsys@transformshift{0.594525in}{2.362645in}%
\pgfsys@useobject{currentmarker}{}%
\end{pgfscope}%
\end{pgfscope}%
\begin{pgfscope}%
\pgfpathrectangle{\pgfqpoint{0.594525in}{0.417642in}}{\pgfqpoint{3.432047in}{2.016277in}}%
\pgfusepath{clip}%
\pgfsetrectcap%
\pgfsetroundjoin%
\pgfsetlinewidth{0.803000pt}%
\definecolor{currentstroke}{rgb}{0.850000,0.850000,0.850000}%
\pgfsetstrokecolor{currentstroke}%
\pgfsetdash{}{0pt}%
\pgfpathmoveto{\pgfqpoint{0.594525in}{2.384024in}}%
\pgfpathlineto{\pgfqpoint{4.026572in}{2.384024in}}%
\pgfusepath{stroke}%
\end{pgfscope}%
\begin{pgfscope}%
\pgfsetbuttcap%
\pgfsetroundjoin%
\definecolor{currentfill}{rgb}{0.000000,0.000000,0.000000}%
\pgfsetfillcolor{currentfill}%
\pgfsetlinewidth{0.602250pt}%
\definecolor{currentstroke}{rgb}{0.000000,0.000000,0.000000}%
\pgfsetstrokecolor{currentstroke}%
\pgfsetdash{}{0pt}%
\pgfsys@defobject{currentmarker}{\pgfqpoint{-0.027778in}{0.000000in}}{\pgfqpoint{-0.000000in}{0.000000in}}{%
\pgfpathmoveto{\pgfqpoint{-0.000000in}{0.000000in}}%
\pgfpathlineto{\pgfqpoint{-0.027778in}{0.000000in}}%
\pgfusepath{stroke,fill}%
}%
\begin{pgfscope}%
\pgfsys@transformshift{0.594525in}{2.384024in}%
\pgfsys@useobject{currentmarker}{}%
\end{pgfscope}%
\end{pgfscope}%
\begin{pgfscope}%
\pgfpathrectangle{\pgfqpoint{0.594525in}{0.417642in}}{\pgfqpoint{3.432047in}{2.016277in}}%
\pgfusepath{clip}%
\pgfsetrectcap%
\pgfsetroundjoin%
\pgfsetlinewidth{0.803000pt}%
\definecolor{currentstroke}{rgb}{0.850000,0.850000,0.850000}%
\pgfsetstrokecolor{currentstroke}%
\pgfsetdash{}{0pt}%
\pgfpathmoveto{\pgfqpoint{0.594525in}{2.402542in}}%
\pgfpathlineto{\pgfqpoint{4.026572in}{2.402542in}}%
\pgfusepath{stroke}%
\end{pgfscope}%
\begin{pgfscope}%
\pgfsetbuttcap%
\pgfsetroundjoin%
\definecolor{currentfill}{rgb}{0.000000,0.000000,0.000000}%
\pgfsetfillcolor{currentfill}%
\pgfsetlinewidth{0.602250pt}%
\definecolor{currentstroke}{rgb}{0.000000,0.000000,0.000000}%
\pgfsetstrokecolor{currentstroke}%
\pgfsetdash{}{0pt}%
\pgfsys@defobject{currentmarker}{\pgfqpoint{-0.027778in}{0.000000in}}{\pgfqpoint{-0.000000in}{0.000000in}}{%
\pgfpathmoveto{\pgfqpoint{-0.000000in}{0.000000in}}%
\pgfpathlineto{\pgfqpoint{-0.027778in}{0.000000in}}%
\pgfusepath{stroke,fill}%
}%
\begin{pgfscope}%
\pgfsys@transformshift{0.594525in}{2.402542in}%
\pgfsys@useobject{currentmarker}{}%
\end{pgfscope}%
\end{pgfscope}%
\begin{pgfscope}%
\pgfpathrectangle{\pgfqpoint{0.594525in}{0.417642in}}{\pgfqpoint{3.432047in}{2.016277in}}%
\pgfusepath{clip}%
\pgfsetrectcap%
\pgfsetroundjoin%
\pgfsetlinewidth{0.803000pt}%
\definecolor{currentstroke}{rgb}{0.850000,0.850000,0.850000}%
\pgfsetstrokecolor{currentstroke}%
\pgfsetdash{}{0pt}%
\pgfpathmoveto{\pgfqpoint{0.594525in}{2.418877in}}%
\pgfpathlineto{\pgfqpoint{4.026572in}{2.418877in}}%
\pgfusepath{stroke}%
\end{pgfscope}%
\begin{pgfscope}%
\pgfsetbuttcap%
\pgfsetroundjoin%
\definecolor{currentfill}{rgb}{0.000000,0.000000,0.000000}%
\pgfsetfillcolor{currentfill}%
\pgfsetlinewidth{0.602250pt}%
\definecolor{currentstroke}{rgb}{0.000000,0.000000,0.000000}%
\pgfsetstrokecolor{currentstroke}%
\pgfsetdash{}{0pt}%
\pgfsys@defobject{currentmarker}{\pgfqpoint{-0.027778in}{0.000000in}}{\pgfqpoint{-0.000000in}{0.000000in}}{%
\pgfpathmoveto{\pgfqpoint{-0.000000in}{0.000000in}}%
\pgfpathlineto{\pgfqpoint{-0.027778in}{0.000000in}}%
\pgfusepath{stroke,fill}%
}%
\begin{pgfscope}%
\pgfsys@transformshift{0.594525in}{2.418877in}%
\pgfsys@useobject{currentmarker}{}%
\end{pgfscope}%
\end{pgfscope}%
\begin{pgfscope}%
\definecolor{textcolor}{rgb}{0.000000,0.000000,0.000000}%
\pgfsetstrokecolor{textcolor}%
\pgfsetfillcolor{textcolor}%
\pgftext[x=0.185574in,y=1.425780in,,bottom,rotate=90.000000]{\color{textcolor}\rmfamily\fontsize{10.000000}{12.000000}\selectfont \(\displaystyle S_y(f)\) in \(\displaystyle \unit{1 \per \Hz}\)}%
\end{pgfscope}%
\begin{pgfscope}%
\pgfpathrectangle{\pgfqpoint{0.594525in}{0.417642in}}{\pgfqpoint{3.432047in}{2.016277in}}%
\pgfusepath{clip}%
\pgfsetbuttcap%
\pgfsetroundjoin%
\pgfsetlinewidth{1.505625pt}%
\definecolor{currentstroke}{rgb}{0.003922,0.450980,0.698039}%
\pgfsetstrokecolor{currentstroke}%
\pgfsetdash{{5.550000pt}{2.400000pt}}{0.000000pt}%
\pgfpathmoveto{\pgfqpoint{0.750527in}{1.947423in}}%
\pgfpathlineto{\pgfqpoint{1.690929in}{1.946274in}}%
\pgfpathlineto{\pgfqpoint{1.880739in}{1.943924in}}%
\pgfpathlineto{\pgfqpoint{2.001715in}{1.940370in}}%
\pgfpathlineto{\pgfqpoint{2.093949in}{1.935485in}}%
\pgfpathlineto{\pgfqpoint{2.171567in}{1.928998in}}%
\pgfpathlineto{\pgfqpoint{2.234689in}{1.921426in}}%
\pgfpathlineto{\pgfqpoint{2.289705in}{1.912626in}}%
\pgfpathlineto{\pgfqpoint{2.343283in}{1.901638in}}%
\pgfpathlineto{\pgfqpoint{2.390889in}{1.889543in}}%
\pgfpathlineto{\pgfqpoint{2.438527in}{1.874986in}}%
\pgfpathlineto{\pgfqpoint{2.485423in}{1.858085in}}%
\pgfpathlineto{\pgfqpoint{2.531985in}{1.838696in}}%
\pgfpathlineto{\pgfqpoint{2.578384in}{1.816803in}}%
\pgfpathlineto{\pgfqpoint{2.625316in}{1.792152in}}%
\pgfpathlineto{\pgfqpoint{2.672456in}{1.765047in}}%
\pgfpathlineto{\pgfqpoint{2.727080in}{1.731029in}}%
\pgfpathlineto{\pgfqpoint{2.790249in}{1.688740in}}%
\pgfpathlineto{\pgfqpoint{2.860754in}{1.638588in}}%
\pgfpathlineto{\pgfqpoint{2.946854in}{1.574265in}}%
\pgfpathlineto{\pgfqpoint{3.056241in}{1.489357in}}%
\pgfpathlineto{\pgfqpoint{3.205059in}{1.370671in}}%
\pgfpathlineto{\pgfqpoint{3.447823in}{1.173807in}}%
\pgfpathlineto{\pgfqpoint{3.870569in}{0.828580in}}%
\pgfpathlineto{\pgfqpoint{3.870569in}{0.828580in}}%
\pgfusepath{stroke}%
\end{pgfscope}%
\begin{pgfscope}%
\pgfpathrectangle{\pgfqpoint{0.594525in}{0.417642in}}{\pgfqpoint{3.432047in}{2.016277in}}%
\pgfusepath{clip}%
\pgfsetbuttcap%
\pgfsetroundjoin%
\definecolor{currentfill}{rgb}{0.003922,0.450980,0.698039}%
\pgfsetfillcolor{currentfill}%
\pgfsetlinewidth{1.003750pt}%
\definecolor{currentstroke}{rgb}{0.003922,0.450980,0.698039}%
\pgfsetstrokecolor{currentstroke}%
\pgfsetdash{}{0pt}%
\pgfsys@defobject{currentmarker}{\pgfqpoint{-0.006944in}{-0.006944in}}{\pgfqpoint{0.006944in}{0.006944in}}{%
\pgfpathmoveto{\pgfqpoint{0.000000in}{-0.006944in}}%
\pgfpathcurveto{\pgfqpoint{0.001842in}{-0.006944in}}{\pgfqpoint{0.003608in}{-0.006213in}}{\pgfqpoint{0.004910in}{-0.004910in}}%
\pgfpathcurveto{\pgfqpoint{0.006213in}{-0.003608in}}{\pgfqpoint{0.006944in}{-0.001842in}}{\pgfqpoint{0.006944in}{0.000000in}}%
\pgfpathcurveto{\pgfqpoint{0.006944in}{0.001842in}}{\pgfqpoint{0.006213in}{0.003608in}}{\pgfqpoint{0.004910in}{0.004910in}}%
\pgfpathcurveto{\pgfqpoint{0.003608in}{0.006213in}}{\pgfqpoint{0.001842in}{0.006944in}}{\pgfqpoint{0.000000in}{0.006944in}}%
\pgfpathcurveto{\pgfqpoint{-0.001842in}{0.006944in}}{\pgfqpoint{-0.003608in}{0.006213in}}{\pgfqpoint{-0.004910in}{0.004910in}}%
\pgfpathcurveto{\pgfqpoint{-0.006213in}{0.003608in}}{\pgfqpoint{-0.006944in}{0.001842in}}{\pgfqpoint{-0.006944in}{0.000000in}}%
\pgfpathcurveto{\pgfqpoint{-0.006944in}{-0.001842in}}{\pgfqpoint{-0.006213in}{-0.003608in}}{\pgfqpoint{-0.004910in}{-0.004910in}}%
\pgfpathcurveto{\pgfqpoint{-0.003608in}{-0.006213in}}{\pgfqpoint{-0.001842in}{-0.006944in}}{\pgfqpoint{0.000000in}{-0.006944in}}%
\pgfpathlineto{\pgfqpoint{0.000000in}{-0.006944in}}%
\pgfpathclose%
\pgfusepath{stroke,fill}%
}%
\begin{pgfscope}%
\pgfsys@transformshift{0.750527in}{1.928874in}%
\pgfsys@useobject{currentmarker}{}%
\end{pgfscope}%
\begin{pgfscope}%
\pgfsys@transformshift{0.985627in}{1.949594in}%
\pgfsys@useobject{currentmarker}{}%
\end{pgfscope}%
\begin{pgfscope}%
\pgfsys@transformshift{1.123152in}{1.957338in}%
\pgfsys@useobject{currentmarker}{}%
\end{pgfscope}%
\begin{pgfscope}%
\pgfsys@transformshift{1.220728in}{1.944865in}%
\pgfsys@useobject{currentmarker}{}%
\end{pgfscope}%
\begin{pgfscope}%
\pgfsys@transformshift{1.296413in}{1.941415in}%
\pgfsys@useobject{currentmarker}{}%
\end{pgfscope}%
\begin{pgfscope}%
\pgfsys@transformshift{1.358253in}{1.936367in}%
\pgfsys@useobject{currentmarker}{}%
\end{pgfscope}%
\begin{pgfscope}%
\pgfsys@transformshift{1.410538in}{1.941183in}%
\pgfsys@useobject{currentmarker}{}%
\end{pgfscope}%
\begin{pgfscope}%
\pgfsys@transformshift{1.455829in}{1.946675in}%
\pgfsys@useobject{currentmarker}{}%
\end{pgfscope}%
\begin{pgfscope}%
\pgfsys@transformshift{1.495778in}{1.950385in}%
\pgfsys@useobject{currentmarker}{}%
\end{pgfscope}%
\begin{pgfscope}%
\pgfsys@transformshift{1.531514in}{1.938377in}%
\pgfsys@useobject{currentmarker}{}%
\end{pgfscope}%
\begin{pgfscope}%
\pgfsys@transformshift{1.563841in}{1.929005in}%
\pgfsys@useobject{currentmarker}{}%
\end{pgfscope}%
\begin{pgfscope}%
\pgfsys@transformshift{1.593354in}{1.936773in}%
\pgfsys@useobject{currentmarker}{}%
\end{pgfscope}%
\begin{pgfscope}%
\pgfsys@transformshift{1.620502in}{1.939118in}%
\pgfsys@useobject{currentmarker}{}%
\end{pgfscope}%
\begin{pgfscope}%
\pgfsys@transformshift{1.645638in}{1.944714in}%
\pgfsys@useobject{currentmarker}{}%
\end{pgfscope}%
\begin{pgfscope}%
\pgfsys@transformshift{1.669039in}{1.952663in}%
\pgfsys@useobject{currentmarker}{}%
\end{pgfscope}%
\begin{pgfscope}%
\pgfsys@transformshift{1.690929in}{1.947396in}%
\pgfsys@useobject{currentmarker}{}%
\end{pgfscope}%
\begin{pgfscope}%
\pgfsys@transformshift{1.711492in}{1.926045in}%
\pgfsys@useobject{currentmarker}{}%
\end{pgfscope}%
\begin{pgfscope}%
\pgfsys@transformshift{1.730879in}{1.947583in}%
\pgfsys@useobject{currentmarker}{}%
\end{pgfscope}%
\begin{pgfscope}%
\pgfsys@transformshift{1.749217in}{1.940807in}%
\pgfsys@useobject{currentmarker}{}%
\end{pgfscope}%
\begin{pgfscope}%
\pgfsys@transformshift{1.766615in}{1.942597in}%
\pgfsys@useobject{currentmarker}{}%
\end{pgfscope}%
\begin{pgfscope}%
\pgfsys@transformshift{1.783163in}{1.941872in}%
\pgfsys@useobject{currentmarker}{}%
\end{pgfscope}%
\begin{pgfscope}%
\pgfsys@transformshift{1.798942in}{1.945008in}%
\pgfsys@useobject{currentmarker}{}%
\end{pgfscope}%
\begin{pgfscope}%
\pgfsys@transformshift{1.814019in}{1.951307in}%
\pgfsys@useobject{currentmarker}{}%
\end{pgfscope}%
\begin{pgfscope}%
\pgfsys@transformshift{1.828454in}{1.939010in}%
\pgfsys@useobject{currentmarker}{}%
\end{pgfscope}%
\begin{pgfscope}%
\pgfsys@transformshift{1.842300in}{1.921499in}%
\pgfsys@useobject{currentmarker}{}%
\end{pgfscope}%
\begin{pgfscope}%
\pgfsys@transformshift{1.855603in}{1.934294in}%
\pgfsys@useobject{currentmarker}{}%
\end{pgfscope}%
\begin{pgfscope}%
\pgfsys@transformshift{1.868404in}{1.940051in}%
\pgfsys@useobject{currentmarker}{}%
\end{pgfscope}%
\begin{pgfscope}%
\pgfsys@transformshift{1.880739in}{1.936995in}%
\pgfsys@useobject{currentmarker}{}%
\end{pgfscope}%
\begin{pgfscope}%
\pgfsys@transformshift{1.892641in}{1.925546in}%
\pgfsys@useobject{currentmarker}{}%
\end{pgfscope}%
\begin{pgfscope}%
\pgfsys@transformshift{1.904140in}{1.937562in}%
\pgfsys@useobject{currentmarker}{}%
\end{pgfscope}%
\begin{pgfscope}%
\pgfsys@transformshift{1.915261in}{1.926059in}%
\pgfsys@useobject{currentmarker}{}%
\end{pgfscope}%
\begin{pgfscope}%
\pgfsys@transformshift{1.926030in}{1.929280in}%
\pgfsys@useobject{currentmarker}{}%
\end{pgfscope}%
\begin{pgfscope}%
\pgfsys@transformshift{1.936467in}{1.931825in}%
\pgfsys@useobject{currentmarker}{}%
\end{pgfscope}%
\begin{pgfscope}%
\pgfsys@transformshift{1.946592in}{1.930487in}%
\pgfsys@useobject{currentmarker}{}%
\end{pgfscope}%
\begin{pgfscope}%
\pgfsys@transformshift{1.956424in}{1.943180in}%
\pgfsys@useobject{currentmarker}{}%
\end{pgfscope}%
\begin{pgfscope}%
\pgfsys@transformshift{1.965979in}{1.958029in}%
\pgfsys@useobject{currentmarker}{}%
\end{pgfscope}%
\begin{pgfscope}%
\pgfsys@transformshift{1.975272in}{1.950325in}%
\pgfsys@useobject{currentmarker}{}%
\end{pgfscope}%
\begin{pgfscope}%
\pgfsys@transformshift{1.984318in}{1.922211in}%
\pgfsys@useobject{currentmarker}{}%
\end{pgfscope}%
\begin{pgfscope}%
\pgfsys@transformshift{1.993128in}{1.924279in}%
\pgfsys@useobject{currentmarker}{}%
\end{pgfscope}%
\begin{pgfscope}%
\pgfsys@transformshift{2.001715in}{1.925673in}%
\pgfsys@useobject{currentmarker}{}%
\end{pgfscope}%
\begin{pgfscope}%
\pgfsys@transformshift{2.010090in}{1.922900in}%
\pgfsys@useobject{currentmarker}{}%
\end{pgfscope}%
\begin{pgfscope}%
\pgfsys@transformshift{2.018264in}{1.922806in}%
\pgfsys@useobject{currentmarker}{}%
\end{pgfscope}%
\begin{pgfscope}%
\pgfsys@transformshift{2.026245in}{1.926506in}%
\pgfsys@useobject{currentmarker}{}%
\end{pgfscope}%
\begin{pgfscope}%
\pgfsys@transformshift{2.034042in}{1.931899in}%
\pgfsys@useobject{currentmarker}{}%
\end{pgfscope}%
\begin{pgfscope}%
\pgfsys@transformshift{2.041665in}{1.938074in}%
\pgfsys@useobject{currentmarker}{}%
\end{pgfscope}%
\begin{pgfscope}%
\pgfsys@transformshift{2.049119in}{1.929735in}%
\pgfsys@useobject{currentmarker}{}%
\end{pgfscope}%
\begin{pgfscope}%
\pgfsys@transformshift{2.056414in}{1.944196in}%
\pgfsys@useobject{currentmarker}{}%
\end{pgfscope}%
\begin{pgfscope}%
\pgfsys@transformshift{2.063555in}{1.943092in}%
\pgfsys@useobject{currentmarker}{}%
\end{pgfscope}%
\begin{pgfscope}%
\pgfsys@transformshift{2.070548in}{1.924422in}%
\pgfsys@useobject{currentmarker}{}%
\end{pgfscope}%
\begin{pgfscope}%
\pgfsys@transformshift{2.077401in}{1.925785in}%
\pgfsys@useobject{currentmarker}{}%
\end{pgfscope}%
\begin{pgfscope}%
\pgfsys@transformshift{2.084117in}{1.925073in}%
\pgfsys@useobject{currentmarker}{}%
\end{pgfscope}%
\begin{pgfscope}%
\pgfsys@transformshift{2.093949in}{1.935161in}%
\pgfsys@useobject{currentmarker}{}%
\end{pgfscope}%
\begin{pgfscope}%
\pgfsys@transformshift{2.103504in}{1.943385in}%
\pgfsys@useobject{currentmarker}{}%
\end{pgfscope}%
\begin{pgfscope}%
\pgfsys@transformshift{2.109728in}{1.936490in}%
\pgfsys@useobject{currentmarker}{}%
\end{pgfscope}%
\begin{pgfscope}%
\pgfsys@transformshift{2.115839in}{1.930926in}%
\pgfsys@useobject{currentmarker}{}%
\end{pgfscope}%
\begin{pgfscope}%
\pgfsys@transformshift{2.124805in}{1.942939in}%
\pgfsys@useobject{currentmarker}{}%
\end{pgfscope}%
\begin{pgfscope}%
\pgfsys@transformshift{2.133540in}{1.936087in}%
\pgfsys@useobject{currentmarker}{}%
\end{pgfscope}%
\begin{pgfscope}%
\pgfsys@transformshift{2.139240in}{1.936039in}%
\pgfsys@useobject{currentmarker}{}%
\end{pgfscope}%
\begin{pgfscope}%
\pgfsys@transformshift{2.147615in}{1.931423in}%
\pgfsys@useobject{currentmarker}{}%
\end{pgfscope}%
\begin{pgfscope}%
\pgfsys@transformshift{2.155789in}{1.926676in}%
\pgfsys@useobject{currentmarker}{}%
\end{pgfscope}%
\begin{pgfscope}%
\pgfsys@transformshift{2.163770in}{1.935270in}%
\pgfsys@useobject{currentmarker}{}%
\end{pgfscope}%
\begin{pgfscope}%
\pgfsys@transformshift{2.171567in}{1.928472in}%
\pgfsys@useobject{currentmarker}{}%
\end{pgfscope}%
\begin{pgfscope}%
\pgfsys@transformshift{2.179190in}{1.912586in}%
\pgfsys@useobject{currentmarker}{}%
\end{pgfscope}%
\begin{pgfscope}%
\pgfsys@transformshift{2.186644in}{1.912375in}%
\pgfsys@useobject{currentmarker}{}%
\end{pgfscope}%
\begin{pgfscope}%
\pgfsys@transformshift{2.193939in}{1.924428in}%
\pgfsys@useobject{currentmarker}{}%
\end{pgfscope}%
\begin{pgfscope}%
\pgfsys@transformshift{2.203427in}{1.922481in}%
\pgfsys@useobject{currentmarker}{}%
\end{pgfscope}%
\begin{pgfscope}%
\pgfsys@transformshift{2.210373in}{1.942305in}%
\pgfsys@useobject{currentmarker}{}%
\end{pgfscope}%
\begin{pgfscope}%
\pgfsys@transformshift{2.217179in}{1.920947in}%
\pgfsys@useobject{currentmarker}{}%
\end{pgfscope}%
\begin{pgfscope}%
\pgfsys@transformshift{2.226047in}{1.912864in}%
\pgfsys@useobject{currentmarker}{}%
\end{pgfscope}%
\begin{pgfscope}%
\pgfsys@transformshift{2.234689in}{1.913822in}%
\pgfsys@useobject{currentmarker}{}%
\end{pgfscope}%
\begin{pgfscope}%
\pgfsys@transformshift{2.243116in}{1.902053in}%
\pgfsys@useobject{currentmarker}{}%
\end{pgfscope}%
\begin{pgfscope}%
\pgfsys@transformshift{2.251339in}{1.915023in}%
\pgfsys@useobject{currentmarker}{}%
\end{pgfscope}%
\begin{pgfscope}%
\pgfsys@transformshift{2.259368in}{1.927677in}%
\pgfsys@useobject{currentmarker}{}%
\end{pgfscope}%
\begin{pgfscope}%
\pgfsys@transformshift{2.267210in}{1.915267in}%
\pgfsys@useobject{currentmarker}{}%
\end{pgfscope}%
\begin{pgfscope}%
\pgfsys@transformshift{2.274876in}{1.911942in}%
\pgfsys@useobject{currentmarker}{}%
\end{pgfscope}%
\begin{pgfscope}%
\pgfsys@transformshift{2.282372in}{1.920130in}%
\pgfsys@useobject{currentmarker}{}%
\end{pgfscope}%
\begin{pgfscope}%
\pgfsys@transformshift{2.289705in}{1.912516in}%
\pgfsys@useobject{currentmarker}{}%
\end{pgfscope}%
\begin{pgfscope}%
\pgfsys@transformshift{2.296884in}{1.911121in}%
\pgfsys@useobject{currentmarker}{}%
\end{pgfscope}%
\begin{pgfscope}%
\pgfsys@transformshift{2.303914in}{1.911113in}%
\pgfsys@useobject{currentmarker}{}%
\end{pgfscope}%
\begin{pgfscope}%
\pgfsys@transformshift{2.312501in}{1.908461in}%
\pgfsys@useobject{currentmarker}{}%
\end{pgfscope}%
\begin{pgfscope}%
\pgfsys@transformshift{2.320876in}{1.906886in}%
\pgfsys@useobject{currentmarker}{}%
\end{pgfscope}%
\begin{pgfscope}%
\pgfsys@transformshift{2.327431in}{1.899330in}%
\pgfsys@useobject{currentmarker}{}%
\end{pgfscope}%
\begin{pgfscope}%
\pgfsys@transformshift{2.335450in}{1.903407in}%
\pgfsys@useobject{currentmarker}{}%
\end{pgfscope}%
\begin{pgfscope}%
\pgfsys@transformshift{2.343283in}{1.900802in}%
\pgfsys@useobject{currentmarker}{}%
\end{pgfscope}%
\begin{pgfscope}%
\pgfsys@transformshift{2.350940in}{1.906052in}%
\pgfsys@useobject{currentmarker}{}%
\end{pgfscope}%
\begin{pgfscope}%
\pgfsys@transformshift{2.359905in}{1.893294in}%
\pgfsys@useobject{currentmarker}{}%
\end{pgfscope}%
\begin{pgfscope}%
\pgfsys@transformshift{2.367200in}{1.879564in}%
\pgfsys@useobject{currentmarker}{}%
\end{pgfscope}%
\begin{pgfscope}%
\pgfsys@transformshift{2.374341in}{1.893276in}%
\pgfsys@useobject{currentmarker}{}%
\end{pgfscope}%
\begin{pgfscope}%
\pgfsys@transformshift{2.382716in}{1.880078in}%
\pgfsys@useobject{currentmarker}{}%
\end{pgfscope}%
\begin{pgfscope}%
\pgfsys@transformshift{2.390889in}{1.889316in}%
\pgfsys@useobject{currentmarker}{}%
\end{pgfscope}%
\begin{pgfscope}%
\pgfsys@transformshift{2.398870in}{1.887966in}%
\pgfsys@useobject{currentmarker}{}%
\end{pgfscope}%
\begin{pgfscope}%
\pgfsys@transformshift{2.406668in}{1.876536in}%
\pgfsys@useobject{currentmarker}{}%
\end{pgfscope}%
\begin{pgfscope}%
\pgfsys@transformshift{2.414290in}{1.869427in}%
\pgfsys@useobject{currentmarker}{}%
\end{pgfscope}%
\begin{pgfscope}%
\pgfsys@transformshift{2.421745in}{1.872920in}%
\pgfsys@useobject{currentmarker}{}%
\end{pgfscope}%
\begin{pgfscope}%
\pgfsys@transformshift{2.430240in}{1.865832in}%
\pgfsys@useobject{currentmarker}{}%
\end{pgfscope}%
\begin{pgfscope}%
\pgfsys@transformshift{2.438527in}{1.881926in}%
\pgfsys@useobject{currentmarker}{}%
\end{pgfscope}%
\begin{pgfscope}%
\pgfsys@transformshift{2.445473in}{1.873196in}%
\pgfsys@useobject{currentmarker}{}%
\end{pgfscope}%
\begin{pgfscope}%
\pgfsys@transformshift{2.453401in}{1.866123in}%
\pgfsys@useobject{currentmarker}{}%
\end{pgfscope}%
\begin{pgfscope}%
\pgfsys@transformshift{2.461148in}{1.859446in}%
\pgfsys@useobject{currentmarker}{}%
\end{pgfscope}%
\begin{pgfscope}%
\pgfsys@transformshift{2.468721in}{1.855960in}%
\pgfsys@useobject{currentmarker}{}%
\end{pgfscope}%
\begin{pgfscope}%
\pgfsys@transformshift{2.477175in}{1.859998in}%
\pgfsys@useobject{currentmarker}{}%
\end{pgfscope}%
\begin{pgfscope}%
\pgfsys@transformshift{2.485423in}{1.852361in}%
\pgfsys@useobject{currentmarker}{}%
\end{pgfscope}%
\begin{pgfscope}%
\pgfsys@transformshift{2.493475in}{1.859905in}%
\pgfsys@useobject{currentmarker}{}%
\end{pgfscope}%
\begin{pgfscope}%
\pgfsys@transformshift{2.501340in}{1.851489in}%
\pgfsys@useobject{currentmarker}{}%
\end{pgfscope}%
\begin{pgfscope}%
\pgfsys@transformshift{2.509027in}{1.854478in}%
\pgfsys@useobject{currentmarker}{}%
\end{pgfscope}%
\begin{pgfscope}%
\pgfsys@transformshift{2.516544in}{1.841466in}%
\pgfsys@useobject{currentmarker}{}%
\end{pgfscope}%
\begin{pgfscope}%
\pgfsys@transformshift{2.523898in}{1.845763in}%
\pgfsys@useobject{currentmarker}{}%
\end{pgfscope}%
\begin{pgfscope}%
\pgfsys@transformshift{2.531985in}{1.836507in}%
\pgfsys@useobject{currentmarker}{}%
\end{pgfscope}%
\begin{pgfscope}%
\pgfsys@transformshift{2.539883in}{1.822836in}%
\pgfsys@useobject{currentmarker}{}%
\end{pgfscope}%
\begin{pgfscope}%
\pgfsys@transformshift{2.547602in}{1.829430in}%
\pgfsys@useobject{currentmarker}{}%
\end{pgfscope}%
\begin{pgfscope}%
\pgfsys@transformshift{2.555149in}{1.831423in}%
\pgfsys@useobject{currentmarker}{}%
\end{pgfscope}%
\begin{pgfscope}%
\pgfsys@transformshift{2.562531in}{1.829970in}%
\pgfsys@useobject{currentmarker}{}%
\end{pgfscope}%
\begin{pgfscope}%
\pgfsys@transformshift{2.570550in}{1.807216in}%
\pgfsys@useobject{currentmarker}{}%
\end{pgfscope}%
\begin{pgfscope}%
\pgfsys@transformshift{2.578384in}{1.818835in}%
\pgfsys@useobject{currentmarker}{}%
\end{pgfscope}%
\begin{pgfscope}%
\pgfsys@transformshift{2.586040in}{1.809862in}%
\pgfsys@useobject{currentmarker}{}%
\end{pgfscope}%
\begin{pgfscope}%
\pgfsys@transformshift{2.594268in}{1.799094in}%
\pgfsys@useobject{currentmarker}{}%
\end{pgfscope}%
\begin{pgfscope}%
\pgfsys@transformshift{2.602300in}{1.799365in}%
\pgfsys@useobject{currentmarker}{}%
\end{pgfscope}%
\begin{pgfscope}%
\pgfsys@transformshift{2.610147in}{1.805535in}%
\pgfsys@useobject{currentmarker}{}%
\end{pgfscope}%
\begin{pgfscope}%
\pgfsys@transformshift{2.617816in}{1.793230in}%
\pgfsys@useobject{currentmarker}{}%
\end{pgfscope}%
\begin{pgfscope}%
\pgfsys@transformshift{2.625316in}{1.785349in}%
\pgfsys@useobject{currentmarker}{}%
\end{pgfscope}%
\begin{pgfscope}%
\pgfsys@transformshift{2.633313in}{1.786791in}%
\pgfsys@useobject{currentmarker}{}%
\end{pgfscope}%
\begin{pgfscope}%
\pgfsys@transformshift{2.641125in}{1.785999in}%
\pgfsys@useobject{currentmarker}{}%
\end{pgfscope}%
\begin{pgfscope}%
\pgfsys@transformshift{2.648762in}{1.784933in}%
\pgfsys@useobject{currentmarker}{}%
\end{pgfscope}%
\begin{pgfscope}%
\pgfsys@transformshift{2.656845in}{1.771944in}%
\pgfsys@useobject{currentmarker}{}%
\end{pgfscope}%
\begin{pgfscope}%
\pgfsys@transformshift{2.664741in}{1.775236in}%
\pgfsys@useobject{currentmarker}{}%
\end{pgfscope}%
\begin{pgfscope}%
\pgfsys@transformshift{2.672456in}{1.762444in}%
\pgfsys@useobject{currentmarker}{}%
\end{pgfscope}%
\begin{pgfscope}%
\pgfsys@transformshift{2.680574in}{1.769577in}%
\pgfsys@useobject{currentmarker}{}%
\end{pgfscope}%
\begin{pgfscope}%
\pgfsys@transformshift{2.688502in}{1.758800in}%
\pgfsys@useobject{currentmarker}{}%
\end{pgfscope}%
\begin{pgfscope}%
\pgfsys@transformshift{2.696248in}{1.757204in}%
\pgfsys@useobject{currentmarker}{}%
\end{pgfscope}%
\begin{pgfscope}%
\pgfsys@transformshift{2.703822in}{1.752826in}%
\pgfsys@useobject{currentmarker}{}%
\end{pgfscope}%
\begin{pgfscope}%
\pgfsys@transformshift{2.711753in}{1.735788in}%
\pgfsys@useobject{currentmarker}{}%
\end{pgfscope}%
\begin{pgfscope}%
\pgfsys@transformshift{2.719503in}{1.740777in}%
\pgfsys@useobject{currentmarker}{}%
\end{pgfscope}%
\begin{pgfscope}%
\pgfsys@transformshift{2.727080in}{1.730526in}%
\pgfsys@useobject{currentmarker}{}%
\end{pgfscope}%
\begin{pgfscope}%
\pgfsys@transformshift{2.734980in}{1.725072in}%
\pgfsys@useobject{currentmarker}{}%
\end{pgfscope}%
\begin{pgfscope}%
\pgfsys@transformshift{2.743177in}{1.718298in}%
\pgfsys@useobject{currentmarker}{}%
\end{pgfscope}%
\begin{pgfscope}%
\pgfsys@transformshift{2.751180in}{1.715113in}%
\pgfsys@useobject{currentmarker}{}%
\end{pgfscope}%
\begin{pgfscope}%
\pgfsys@transformshift{2.758998in}{1.702223in}%
\pgfsys@useobject{currentmarker}{}%
\end{pgfscope}%
\begin{pgfscope}%
\pgfsys@transformshift{2.766641in}{1.695459in}%
\pgfsys@useobject{currentmarker}{}%
\end{pgfscope}%
\begin{pgfscope}%
\pgfsys@transformshift{2.774115in}{1.701715in}%
\pgfsys@useobject{currentmarker}{}%
\end{pgfscope}%
\begin{pgfscope}%
\pgfsys@transformshift{2.782278in}{1.696444in}%
\pgfsys@useobject{currentmarker}{}%
\end{pgfscope}%
\begin{pgfscope}%
\pgfsys@transformshift{2.790249in}{1.692675in}%
\pgfsys@useobject{currentmarker}{}%
\end{pgfscope}%
\begin{pgfscope}%
\pgfsys@transformshift{2.798037in}{1.680659in}%
\pgfsys@useobject{currentmarker}{}%
\end{pgfscope}%
\begin{pgfscope}%
\pgfsys@transformshift{2.806047in}{1.675688in}%
\pgfsys@useobject{currentmarker}{}%
\end{pgfscope}%
\begin{pgfscope}%
\pgfsys@transformshift{2.813871in}{1.666706in}%
\pgfsys@useobject{currentmarker}{}%
\end{pgfscope}%
\begin{pgfscope}%
\pgfsys@transformshift{2.821519in}{1.670644in}%
\pgfsys@useobject{currentmarker}{}%
\end{pgfscope}%
\begin{pgfscope}%
\pgfsys@transformshift{2.829368in}{1.661094in}%
\pgfsys@useobject{currentmarker}{}%
\end{pgfscope}%
\begin{pgfscope}%
\pgfsys@transformshift{2.837040in}{1.652782in}%
\pgfsys@useobject{currentmarker}{}%
\end{pgfscope}%
\begin{pgfscope}%
\pgfsys@transformshift{2.844895in}{1.653153in}%
\pgfsys@useobject{currentmarker}{}%
\end{pgfscope}%
\begin{pgfscope}%
\pgfsys@transformshift{2.852917in}{1.649206in}%
\pgfsys@useobject{currentmarker}{}%
\end{pgfscope}%
\begin{pgfscope}%
\pgfsys@transformshift{2.860754in}{1.637217in}%
\pgfsys@useobject{currentmarker}{}%
\end{pgfscope}%
\begin{pgfscope}%
\pgfsys@transformshift{2.868413in}{1.628682in}%
\pgfsys@useobject{currentmarker}{}%
\end{pgfscope}%
\begin{pgfscope}%
\pgfsys@transformshift{2.876226in}{1.631635in}%
\pgfsys@useobject{currentmarker}{}%
\end{pgfscope}%
\begin{pgfscope}%
\pgfsys@transformshift{2.884177in}{1.622290in}%
\pgfsys@useobject{currentmarker}{}%
\end{pgfscope}%
\begin{pgfscope}%
\pgfsys@transformshift{2.891946in}{1.611425in}%
\pgfsys@useobject{currentmarker}{}%
\end{pgfscope}%
\begin{pgfscope}%
\pgfsys@transformshift{2.899841in}{1.602435in}%
\pgfsys@useobject{currentmarker}{}%
\end{pgfscope}%
\begin{pgfscope}%
\pgfsys@transformshift{2.907557in}{1.595553in}%
\pgfsys@useobject{currentmarker}{}%
\end{pgfscope}%
\begin{pgfscope}%
\pgfsys@transformshift{2.915388in}{1.600072in}%
\pgfsys@useobject{currentmarker}{}%
\end{pgfscope}%
\begin{pgfscope}%
\pgfsys@transformshift{2.923322in}{1.595865in}%
\pgfsys@useobject{currentmarker}{}%
\end{pgfscope}%
\begin{pgfscope}%
\pgfsys@transformshift{2.931075in}{1.587107in}%
\pgfsys@useobject{currentmarker}{}%
\end{pgfscope}%
\begin{pgfscope}%
\pgfsys@transformshift{2.938923in}{1.583499in}%
\pgfsys@useobject{currentmarker}{}%
\end{pgfscope}%
\begin{pgfscope}%
\pgfsys@transformshift{2.946854in}{1.577353in}%
\pgfsys@useobject{currentmarker}{}%
\end{pgfscope}%
\begin{pgfscope}%
\pgfsys@transformshift{2.954604in}{1.571811in}%
\pgfsys@useobject{currentmarker}{}%
\end{pgfscope}%
\begin{pgfscope}%
\pgfsys@transformshift{2.962430in}{1.570342in}%
\pgfsys@useobject{currentmarker}{}%
\end{pgfscope}%
\begin{pgfscope}%
\pgfsys@transformshift{2.970324in}{1.562496in}%
\pgfsys@useobject{currentmarker}{}%
\end{pgfscope}%
\begin{pgfscope}%
\pgfsys@transformshift{2.978039in}{1.556038in}%
\pgfsys@useobject{currentmarker}{}%
\end{pgfscope}%
\begin{pgfscope}%
\pgfsys@transformshift{2.985815in}{1.546345in}%
\pgfsys@useobject{currentmarker}{}%
\end{pgfscope}%
\begin{pgfscope}%
\pgfsys@transformshift{2.993644in}{1.535846in}%
\pgfsys@useobject{currentmarker}{}%
\end{pgfscope}%
\begin{pgfscope}%
\pgfsys@transformshift{3.001519in}{1.533106in}%
\pgfsys@useobject{currentmarker}{}%
\end{pgfscope}%
\begin{pgfscope}%
\pgfsys@transformshift{3.009433in}{1.528406in}%
\pgfsys@useobject{currentmarker}{}%
\end{pgfscope}%
\begin{pgfscope}%
\pgfsys@transformshift{3.017166in}{1.520273in}%
\pgfsys@useobject{currentmarker}{}%
\end{pgfscope}%
\begin{pgfscope}%
\pgfsys@transformshift{3.024935in}{1.515058in}%
\pgfsys@useobject{currentmarker}{}%
\end{pgfscope}%
\begin{pgfscope}%
\pgfsys@transformshift{3.032732in}{1.503170in}%
\pgfsys@useobject{currentmarker}{}%
\end{pgfscope}%
\begin{pgfscope}%
\pgfsys@transformshift{3.040553in}{1.502677in}%
\pgfsys@useobject{currentmarker}{}%
\end{pgfscope}%
\begin{pgfscope}%
\pgfsys@transformshift{3.048391in}{1.495652in}%
\pgfsys@useobject{currentmarker}{}%
\end{pgfscope}%
\begin{pgfscope}%
\pgfsys@transformshift{3.056241in}{1.492841in}%
\pgfsys@useobject{currentmarker}{}%
\end{pgfscope}%
\begin{pgfscope}%
\pgfsys@transformshift{3.064099in}{1.483616in}%
\pgfsys@useobject{currentmarker}{}%
\end{pgfscope}%
\begin{pgfscope}%
\pgfsys@transformshift{3.071960in}{1.478927in}%
\pgfsys@useobject{currentmarker}{}%
\end{pgfscope}%
\begin{pgfscope}%
\pgfsys@transformshift{3.079819in}{1.472494in}%
\pgfsys@useobject{currentmarker}{}%
\end{pgfscope}%
\begin{pgfscope}%
\pgfsys@transformshift{3.087673in}{1.470666in}%
\pgfsys@useobject{currentmarker}{}%
\end{pgfscope}%
\begin{pgfscope}%
\pgfsys@transformshift{3.095517in}{1.459565in}%
\pgfsys@useobject{currentmarker}{}%
\end{pgfscope}%
\begin{pgfscope}%
\pgfsys@transformshift{3.103349in}{1.456283in}%
\pgfsys@useobject{currentmarker}{}%
\end{pgfscope}%
\begin{pgfscope}%
\pgfsys@transformshift{3.111166in}{1.454025in}%
\pgfsys@useobject{currentmarker}{}%
\end{pgfscope}%
\begin{pgfscope}%
\pgfsys@transformshift{3.118963in}{1.440052in}%
\pgfsys@useobject{currentmarker}{}%
\end{pgfscope}%
\begin{pgfscope}%
\pgfsys@transformshift{3.126739in}{1.441429in}%
\pgfsys@useobject{currentmarker}{}%
\end{pgfscope}%
\begin{pgfscope}%
\pgfsys@transformshift{3.134491in}{1.428415in}%
\pgfsys@useobject{currentmarker}{}%
\end{pgfscope}%
\begin{pgfscope}%
\pgfsys@transformshift{3.142364in}{1.423594in}%
\pgfsys@useobject{currentmarker}{}%
\end{pgfscope}%
\begin{pgfscope}%
\pgfsys@transformshift{3.150202in}{1.413662in}%
\pgfsys@useobject{currentmarker}{}%
\end{pgfscope}%
\begin{pgfscope}%
\pgfsys@transformshift{3.158002in}{1.408565in}%
\pgfsys@useobject{currentmarker}{}%
\end{pgfscope}%
\begin{pgfscope}%
\pgfsys@transformshift{3.165902in}{1.401889in}%
\pgfsys@useobject{currentmarker}{}%
\end{pgfscope}%
\begin{pgfscope}%
\pgfsys@transformshift{3.173756in}{1.393385in}%
\pgfsys@useobject{currentmarker}{}%
\end{pgfscope}%
\begin{pgfscope}%
\pgfsys@transformshift{3.181562in}{1.391538in}%
\pgfsys@useobject{currentmarker}{}%
\end{pgfscope}%
\begin{pgfscope}%
\pgfsys@transformshift{3.189449in}{1.388507in}%
\pgfsys@useobject{currentmarker}{}%
\end{pgfscope}%
\begin{pgfscope}%
\pgfsys@transformshift{3.197281in}{1.377593in}%
\pgfsys@useobject{currentmarker}{}%
\end{pgfscope}%
\begin{pgfscope}%
\pgfsys@transformshift{3.205059in}{1.373192in}%
\pgfsys@useobject{currentmarker}{}%
\end{pgfscope}%
\begin{pgfscope}%
\pgfsys@transformshift{3.212901in}{1.367013in}%
\pgfsys@useobject{currentmarker}{}%
\end{pgfscope}%
\begin{pgfscope}%
\pgfsys@transformshift{3.220799in}{1.358061in}%
\pgfsys@useobject{currentmarker}{}%
\end{pgfscope}%
\begin{pgfscope}%
\pgfsys@transformshift{3.228631in}{1.354471in}%
\pgfsys@useobject{currentmarker}{}%
\end{pgfscope}%
\begin{pgfscope}%
\pgfsys@transformshift{3.236397in}{1.350627in}%
\pgfsys@useobject{currentmarker}{}%
\end{pgfscope}%
\begin{pgfscope}%
\pgfsys@transformshift{3.244207in}{1.344372in}%
\pgfsys@useobject{currentmarker}{}%
\end{pgfscope}%
\begin{pgfscope}%
\pgfsys@transformshift{3.252054in}{1.334880in}%
\pgfsys@useobject{currentmarker}{}%
\end{pgfscope}%
\begin{pgfscope}%
\pgfsys@transformshift{3.259932in}{1.330755in}%
\pgfsys@useobject{currentmarker}{}%
\end{pgfscope}%
\begin{pgfscope}%
\pgfsys@transformshift{3.267732in}{1.327173in}%
\pgfsys@useobject{currentmarker}{}%
\end{pgfscope}%
\begin{pgfscope}%
\pgfsys@transformshift{3.275554in}{1.315546in}%
\pgfsys@useobject{currentmarker}{}%
\end{pgfscope}%
\begin{pgfscope}%
\pgfsys@transformshift{3.283395in}{1.309213in}%
\pgfsys@useobject{currentmarker}{}%
\end{pgfscope}%
\begin{pgfscope}%
\pgfsys@transformshift{3.291153in}{1.307280in}%
\pgfsys@useobject{currentmarker}{}%
\end{pgfscope}%
\begin{pgfscope}%
\pgfsys@transformshift{3.299015in}{1.299628in}%
\pgfsys@useobject{currentmarker}{}%
\end{pgfscope}%
\begin{pgfscope}%
\pgfsys@transformshift{3.306880in}{1.291580in}%
\pgfsys@useobject{currentmarker}{}%
\end{pgfscope}%
\begin{pgfscope}%
\pgfsys@transformshift{3.314655in}{1.287993in}%
\pgfsys@useobject{currentmarker}{}%
\end{pgfscope}%
\begin{pgfscope}%
\pgfsys@transformshift{3.322514in}{1.280271in}%
\pgfsys@useobject{currentmarker}{}%
\end{pgfscope}%
\begin{pgfscope}%
\pgfsys@transformshift{3.330365in}{1.272103in}%
\pgfsys@useobject{currentmarker}{}%
\end{pgfscope}%
\begin{pgfscope}%
\pgfsys@transformshift{3.338203in}{1.271662in}%
\pgfsys@useobject{currentmarker}{}%
\end{pgfscope}%
\begin{pgfscope}%
\pgfsys@transformshift{3.346025in}{1.258757in}%
\pgfsys@useobject{currentmarker}{}%
\end{pgfscope}%
\begin{pgfscope}%
\pgfsys@transformshift{3.353828in}{1.259672in}%
\pgfsys@useobject{currentmarker}{}%
\end{pgfscope}%
\begin{pgfscope}%
\pgfsys@transformshift{3.361686in}{1.248067in}%
\pgfsys@useobject{currentmarker}{}%
\end{pgfscope}%
\begin{pgfscope}%
\pgfsys@transformshift{3.369517in}{1.248281in}%
\pgfsys@useobject{currentmarker}{}%
\end{pgfscope}%
\begin{pgfscope}%
\pgfsys@transformshift{3.377318in}{1.238893in}%
\pgfsys@useobject{currentmarker}{}%
\end{pgfscope}%
\begin{pgfscope}%
\pgfsys@transformshift{3.385159in}{1.228444in}%
\pgfsys@useobject{currentmarker}{}%
\end{pgfscope}%
\begin{pgfscope}%
\pgfsys@transformshift{3.393033in}{1.224075in}%
\pgfsys@useobject{currentmarker}{}%
\end{pgfscope}%
\begin{pgfscope}%
\pgfsys@transformshift{3.400865in}{1.218029in}%
\pgfsys@useobject{currentmarker}{}%
\end{pgfscope}%
\begin{pgfscope}%
\pgfsys@transformshift{3.408655in}{1.212633in}%
\pgfsys@useobject{currentmarker}{}%
\end{pgfscope}%
\begin{pgfscope}%
\pgfsys@transformshift{3.416466in}{1.210181in}%
\pgfsys@useobject{currentmarker}{}%
\end{pgfscope}%
\begin{pgfscope}%
\pgfsys@transformshift{3.424294in}{1.204504in}%
\pgfsys@useobject{currentmarker}{}%
\end{pgfscope}%
\begin{pgfscope}%
\pgfsys@transformshift{3.432132in}{1.198012in}%
\pgfsys@useobject{currentmarker}{}%
\end{pgfscope}%
\begin{pgfscope}%
\pgfsys@transformshift{3.439976in}{1.192516in}%
\pgfsys@useobject{currentmarker}{}%
\end{pgfscope}%
\begin{pgfscope}%
\pgfsys@transformshift{3.447823in}{1.181366in}%
\pgfsys@useobject{currentmarker}{}%
\end{pgfscope}%
\begin{pgfscope}%
\pgfsys@transformshift{3.455666in}{1.176764in}%
\pgfsys@useobject{currentmarker}{}%
\end{pgfscope}%
\begin{pgfscope}%
\pgfsys@transformshift{3.463447in}{1.172225in}%
\pgfsys@useobject{currentmarker}{}%
\end{pgfscope}%
\begin{pgfscope}%
\pgfsys@transformshift{3.471275in}{1.167378in}%
\pgfsys@useobject{currentmarker}{}%
\end{pgfscope}%
\begin{pgfscope}%
\pgfsys@transformshift{3.479145in}{1.163834in}%
\pgfsys@useobject{currentmarker}{}%
\end{pgfscope}%
\begin{pgfscope}%
\pgfsys@transformshift{3.486942in}{1.158801in}%
\pgfsys@useobject{currentmarker}{}%
\end{pgfscope}%
\begin{pgfscope}%
\pgfsys@transformshift{3.494773in}{1.148010in}%
\pgfsys@useobject{currentmarker}{}%
\end{pgfscope}%
\begin{pgfscope}%
\pgfsys@transformshift{3.502629in}{1.142291in}%
\pgfsys@useobject{currentmarker}{}%
\end{pgfscope}%
\begin{pgfscope}%
\pgfsys@transformshift{3.510457in}{1.138598in}%
\pgfsys@useobject{currentmarker}{}%
\end{pgfscope}%
\begin{pgfscope}%
\pgfsys@transformshift{3.518253in}{1.128641in}%
\pgfsys@useobject{currentmarker}{}%
\end{pgfscope}%
\begin{pgfscope}%
\pgfsys@transformshift{3.526064in}{1.122716in}%
\pgfsys@useobject{currentmarker}{}%
\end{pgfscope}%
\begin{pgfscope}%
\pgfsys@transformshift{3.533930in}{1.119541in}%
\pgfsys@useobject{currentmarker}{}%
\end{pgfscope}%
\begin{pgfscope}%
\pgfsys@transformshift{3.541754in}{1.115778in}%
\pgfsys@useobject{currentmarker}{}%
\end{pgfscope}%
\begin{pgfscope}%
\pgfsys@transformshift{3.549578in}{1.109153in}%
\pgfsys@useobject{currentmarker}{}%
\end{pgfscope}%
\begin{pgfscope}%
\pgfsys@transformshift{3.557399in}{1.103011in}%
\pgfsys@useobject{currentmarker}{}%
\end{pgfscope}%
\begin{pgfscope}%
\pgfsys@transformshift{3.565212in}{1.097825in}%
\pgfsys@useobject{currentmarker}{}%
\end{pgfscope}%
\begin{pgfscope}%
\pgfsys@transformshift{3.573056in}{1.091347in}%
\pgfsys@useobject{currentmarker}{}%
\end{pgfscope}%
\begin{pgfscope}%
\pgfsys@transformshift{3.580883in}{1.085471in}%
\pgfsys@useobject{currentmarker}{}%
\end{pgfscope}%
\begin{pgfscope}%
\pgfsys@transformshift{3.588731in}{1.078181in}%
\pgfsys@useobject{currentmarker}{}%
\end{pgfscope}%
\begin{pgfscope}%
\pgfsys@transformshift{3.596556in}{1.073649in}%
\pgfsys@useobject{currentmarker}{}%
\end{pgfscope}%
\begin{pgfscope}%
\pgfsys@transformshift{3.604354in}{1.066865in}%
\pgfsys@useobject{currentmarker}{}%
\end{pgfscope}%
\begin{pgfscope}%
\pgfsys@transformshift{3.612198in}{1.067836in}%
\pgfsys@useobject{currentmarker}{}%
\end{pgfscope}%
\begin{pgfscope}%
\pgfsys@transformshift{3.620044in}{1.057106in}%
\pgfsys@useobject{currentmarker}{}%
\end{pgfscope}%
\begin{pgfscope}%
\pgfsys@transformshift{3.627888in}{1.056021in}%
\pgfsys@useobject{currentmarker}{}%
\end{pgfscope}%
\begin{pgfscope}%
\pgfsys@transformshift{3.635726in}{1.052030in}%
\pgfsys@useobject{currentmarker}{}%
\end{pgfscope}%
\begin{pgfscope}%
\pgfsys@transformshift{3.643555in}{1.045165in}%
\pgfsys@useobject{currentmarker}{}%
\end{pgfscope}%
\begin{pgfscope}%
\pgfsys@transformshift{3.651371in}{1.038030in}%
\pgfsys@useobject{currentmarker}{}%
\end{pgfscope}%
\begin{pgfscope}%
\pgfsys@transformshift{3.659170in}{1.036278in}%
\pgfsys@useobject{currentmarker}{}%
\end{pgfscope}%
\begin{pgfscope}%
\pgfsys@transformshift{3.667014in}{1.032882in}%
\pgfsys@useobject{currentmarker}{}%
\end{pgfscope}%
\begin{pgfscope}%
\pgfsys@transformshift{3.674863in}{1.027090in}%
\pgfsys@useobject{currentmarker}{}%
\end{pgfscope}%
\begin{pgfscope}%
\pgfsys@transformshift{3.682684in}{1.021139in}%
\pgfsys@useobject{currentmarker}{}%
\end{pgfscope}%
\begin{pgfscope}%
\pgfsys@transformshift{3.690504in}{1.015861in}%
\pgfsys@useobject{currentmarker}{}%
\end{pgfscope}%
\begin{pgfscope}%
\pgfsys@transformshift{3.698319in}{1.009210in}%
\pgfsys@useobject{currentmarker}{}%
\end{pgfscope}%
\begin{pgfscope}%
\pgfsys@transformshift{3.706153in}{1.006736in}%
\pgfsys@useobject{currentmarker}{}%
\end{pgfscope}%
\begin{pgfscope}%
\pgfsys@transformshift{3.714000in}{1.003398in}%
\pgfsys@useobject{currentmarker}{}%
\end{pgfscope}%
\begin{pgfscope}%
\pgfsys@transformshift{3.721830in}{0.996407in}%
\pgfsys@useobject{currentmarker}{}%
\end{pgfscope}%
\begin{pgfscope}%
\pgfsys@transformshift{3.729665in}{0.992891in}%
\pgfsys@useobject{currentmarker}{}%
\end{pgfscope}%
\begin{pgfscope}%
\pgfsys@transformshift{3.737501in}{0.991977in}%
\pgfsys@useobject{currentmarker}{}%
\end{pgfscope}%
\begin{pgfscope}%
\pgfsys@transformshift{3.745309in}{0.986370in}%
\pgfsys@useobject{currentmarker}{}%
\end{pgfscope}%
\begin{pgfscope}%
\pgfsys@transformshift{3.753135in}{0.984132in}%
\pgfsys@useobject{currentmarker}{}%
\end{pgfscope}%
\begin{pgfscope}%
\pgfsys@transformshift{3.760975in}{0.975872in}%
\pgfsys@useobject{currentmarker}{}%
\end{pgfscope}%
\begin{pgfscope}%
\pgfsys@transformshift{3.768799in}{0.977165in}%
\pgfsys@useobject{currentmarker}{}%
\end{pgfscope}%
\begin{pgfscope}%
\pgfsys@transformshift{3.776628in}{0.973750in}%
\pgfsys@useobject{currentmarker}{}%
\end{pgfscope}%
\begin{pgfscope}%
\pgfsys@transformshift{3.784458in}{0.970796in}%
\pgfsys@useobject{currentmarker}{}%
\end{pgfscope}%
\begin{pgfscope}%
\pgfsys@transformshift{3.792284in}{0.966927in}%
\pgfsys@useobject{currentmarker}{}%
\end{pgfscope}%
\begin{pgfscope}%
\pgfsys@transformshift{3.800123in}{0.966400in}%
\pgfsys@useobject{currentmarker}{}%
\end{pgfscope}%
\begin{pgfscope}%
\pgfsys@transformshift{3.807950in}{0.960010in}%
\pgfsys@useobject{currentmarker}{}%
\end{pgfscope}%
\begin{pgfscope}%
\pgfsys@transformshift{3.815762in}{0.960040in}%
\pgfsys@useobject{currentmarker}{}%
\end{pgfscope}%
\begin{pgfscope}%
\pgfsys@transformshift{3.823596in}{0.957662in}%
\pgfsys@useobject{currentmarker}{}%
\end{pgfscope}%
\begin{pgfscope}%
\pgfsys@transformshift{3.831425in}{0.955426in}%
\pgfsys@useobject{currentmarker}{}%
\end{pgfscope}%
\begin{pgfscope}%
\pgfsys@transformshift{3.839267in}{0.953319in}%
\pgfsys@useobject{currentmarker}{}%
\end{pgfscope}%
\begin{pgfscope}%
\pgfsys@transformshift{3.847096in}{0.952455in}%
\pgfsys@useobject{currentmarker}{}%
\end{pgfscope}%
\begin{pgfscope}%
\pgfsys@transformshift{3.854911in}{0.952616in}%
\pgfsys@useobject{currentmarker}{}%
\end{pgfscope}%
\begin{pgfscope}%
\pgfsys@transformshift{3.862743in}{0.953465in}%
\pgfsys@useobject{currentmarker}{}%
\end{pgfscope}%
\begin{pgfscope}%
\pgfsys@transformshift{3.870569in}{0.949998in}%
\pgfsys@useobject{currentmarker}{}%
\end{pgfscope}%
\end{pgfscope}%
\begin{pgfscope}%
\pgfpathrectangle{\pgfqpoint{0.594525in}{0.417642in}}{\pgfqpoint{3.432047in}{2.016277in}}%
\pgfusepath{clip}%
\pgfsetbuttcap%
\pgfsetroundjoin%
\pgfsetlinewidth{1.505625pt}%
\definecolor{currentstroke}{rgb}{0.007843,0.619608,0.450980}%
\pgfsetstrokecolor{currentstroke}%
\pgfsetdash{{5.550000pt}{2.400000pt}}{0.000000pt}%
\pgfpathmoveto{\pgfqpoint{0.750527in}{2.337223in}}%
\pgfpathlineto{\pgfqpoint{1.123152in}{2.336134in}}%
\pgfpathlineto{\pgfqpoint{1.296413in}{2.333980in}}%
\pgfpathlineto{\pgfqpoint{1.410538in}{2.330810in}}%
\pgfpathlineto{\pgfqpoint{1.495778in}{2.326694in}}%
\pgfpathlineto{\pgfqpoint{1.563841in}{2.321715in}}%
\pgfpathlineto{\pgfqpoint{1.620502in}{2.315967in}}%
\pgfpathlineto{\pgfqpoint{1.690929in}{2.306119in}}%
\pgfpathlineto{\pgfqpoint{1.749217in}{2.295095in}}%
\pgfpathlineto{\pgfqpoint{1.798942in}{2.283207in}}%
\pgfpathlineto{\pgfqpoint{1.842300in}{2.270725in}}%
\pgfpathlineto{\pgfqpoint{1.892641in}{2.253549in}}%
\pgfpathlineto{\pgfqpoint{1.936467in}{2.236139in}}%
\pgfpathlineto{\pgfqpoint{1.984318in}{2.214500in}}%
\pgfpathlineto{\pgfqpoint{2.034042in}{2.189188in}}%
\pgfpathlineto{\pgfqpoint{2.084117in}{2.160989in}}%
\pgfpathlineto{\pgfqpoint{2.139240in}{2.127143in}}%
\pgfpathlineto{\pgfqpoint{2.193939in}{2.091074in}}%
\pgfpathlineto{\pgfqpoint{2.259368in}{2.045298in}}%
\pgfpathlineto{\pgfqpoint{2.335450in}{1.989322in}}%
\pgfpathlineto{\pgfqpoint{2.430240in}{1.916719in}}%
\pgfpathlineto{\pgfqpoint{2.555149in}{1.818099in}}%
\pgfpathlineto{\pgfqpoint{2.743177in}{1.666552in}}%
\pgfpathlineto{\pgfqpoint{3.118963in}{1.360230in}}%
\pgfpathlineto{\pgfqpoint{3.870569in}{0.745712in}}%
\pgfpathlineto{\pgfqpoint{3.870569in}{0.745712in}}%
\pgfusepath{stroke}%
\end{pgfscope}%
\begin{pgfscope}%
\pgfpathrectangle{\pgfqpoint{0.594525in}{0.417642in}}{\pgfqpoint{3.432047in}{2.016277in}}%
\pgfusepath{clip}%
\pgfsetbuttcap%
\pgfsetroundjoin%
\definecolor{currentfill}{rgb}{0.007843,0.619608,0.450980}%
\pgfsetfillcolor{currentfill}%
\pgfsetlinewidth{1.003750pt}%
\definecolor{currentstroke}{rgb}{0.007843,0.619608,0.450980}%
\pgfsetstrokecolor{currentstroke}%
\pgfsetdash{}{0pt}%
\pgfsys@defobject{currentmarker}{\pgfqpoint{-0.006944in}{-0.006944in}}{\pgfqpoint{0.006944in}{0.006944in}}{%
\pgfpathmoveto{\pgfqpoint{0.000000in}{-0.006944in}}%
\pgfpathcurveto{\pgfqpoint{0.001842in}{-0.006944in}}{\pgfqpoint{0.003608in}{-0.006213in}}{\pgfqpoint{0.004910in}{-0.004910in}}%
\pgfpathcurveto{\pgfqpoint{0.006213in}{-0.003608in}}{\pgfqpoint{0.006944in}{-0.001842in}}{\pgfqpoint{0.006944in}{0.000000in}}%
\pgfpathcurveto{\pgfqpoint{0.006944in}{0.001842in}}{\pgfqpoint{0.006213in}{0.003608in}}{\pgfqpoint{0.004910in}{0.004910in}}%
\pgfpathcurveto{\pgfqpoint{0.003608in}{0.006213in}}{\pgfqpoint{0.001842in}{0.006944in}}{\pgfqpoint{0.000000in}{0.006944in}}%
\pgfpathcurveto{\pgfqpoint{-0.001842in}{0.006944in}}{\pgfqpoint{-0.003608in}{0.006213in}}{\pgfqpoint{-0.004910in}{0.004910in}}%
\pgfpathcurveto{\pgfqpoint{-0.006213in}{0.003608in}}{\pgfqpoint{-0.006944in}{0.001842in}}{\pgfqpoint{-0.006944in}{0.000000in}}%
\pgfpathcurveto{\pgfqpoint{-0.006944in}{-0.001842in}}{\pgfqpoint{-0.006213in}{-0.003608in}}{\pgfqpoint{-0.004910in}{-0.004910in}}%
\pgfpathcurveto{\pgfqpoint{-0.003608in}{-0.006213in}}{\pgfqpoint{-0.001842in}{-0.006944in}}{\pgfqpoint{0.000000in}{-0.006944in}}%
\pgfpathlineto{\pgfqpoint{0.000000in}{-0.006944in}}%
\pgfpathclose%
\pgfusepath{stroke,fill}%
}%
\begin{pgfscope}%
\pgfsys@transformshift{0.750527in}{2.312872in}%
\pgfsys@useobject{currentmarker}{}%
\end{pgfscope}%
\begin{pgfscope}%
\pgfsys@transformshift{0.985627in}{2.341687in}%
\pgfsys@useobject{currentmarker}{}%
\end{pgfscope}%
\begin{pgfscope}%
\pgfsys@transformshift{1.123152in}{2.327223in}%
\pgfsys@useobject{currentmarker}{}%
\end{pgfscope}%
\begin{pgfscope}%
\pgfsys@transformshift{1.220728in}{2.322175in}%
\pgfsys@useobject{currentmarker}{}%
\end{pgfscope}%
\begin{pgfscope}%
\pgfsys@transformshift{1.296413in}{2.331902in}%
\pgfsys@useobject{currentmarker}{}%
\end{pgfscope}%
\begin{pgfscope}%
\pgfsys@transformshift{1.358253in}{2.334081in}%
\pgfsys@useobject{currentmarker}{}%
\end{pgfscope}%
\begin{pgfscope}%
\pgfsys@transformshift{1.410538in}{2.342270in}%
\pgfsys@useobject{currentmarker}{}%
\end{pgfscope}%
\begin{pgfscope}%
\pgfsys@transformshift{1.455829in}{2.329571in}%
\pgfsys@useobject{currentmarker}{}%
\end{pgfscope}%
\begin{pgfscope}%
\pgfsys@transformshift{1.495778in}{2.333138in}%
\pgfsys@useobject{currentmarker}{}%
\end{pgfscope}%
\begin{pgfscope}%
\pgfsys@transformshift{1.531514in}{2.324519in}%
\pgfsys@useobject{currentmarker}{}%
\end{pgfscope}%
\begin{pgfscope}%
\pgfsys@transformshift{1.563841in}{2.321269in}%
\pgfsys@useobject{currentmarker}{}%
\end{pgfscope}%
\begin{pgfscope}%
\pgfsys@transformshift{1.593354in}{2.319662in}%
\pgfsys@useobject{currentmarker}{}%
\end{pgfscope}%
\begin{pgfscope}%
\pgfsys@transformshift{1.620502in}{2.318968in}%
\pgfsys@useobject{currentmarker}{}%
\end{pgfscope}%
\begin{pgfscope}%
\pgfsys@transformshift{1.645638in}{2.303840in}%
\pgfsys@useobject{currentmarker}{}%
\end{pgfscope}%
\begin{pgfscope}%
\pgfsys@transformshift{1.669039in}{2.293263in}%
\pgfsys@useobject{currentmarker}{}%
\end{pgfscope}%
\begin{pgfscope}%
\pgfsys@transformshift{1.690929in}{2.299007in}%
\pgfsys@useobject{currentmarker}{}%
\end{pgfscope}%
\begin{pgfscope}%
\pgfsys@transformshift{1.711492in}{2.302164in}%
\pgfsys@useobject{currentmarker}{}%
\end{pgfscope}%
\begin{pgfscope}%
\pgfsys@transformshift{1.730879in}{2.294600in}%
\pgfsys@useobject{currentmarker}{}%
\end{pgfscope}%
\begin{pgfscope}%
\pgfsys@transformshift{1.749217in}{2.299021in}%
\pgfsys@useobject{currentmarker}{}%
\end{pgfscope}%
\begin{pgfscope}%
\pgfsys@transformshift{1.766615in}{2.298245in}%
\pgfsys@useobject{currentmarker}{}%
\end{pgfscope}%
\begin{pgfscope}%
\pgfsys@transformshift{1.783163in}{2.270928in}%
\pgfsys@useobject{currentmarker}{}%
\end{pgfscope}%
\begin{pgfscope}%
\pgfsys@transformshift{1.798942in}{2.276321in}%
\pgfsys@useobject{currentmarker}{}%
\end{pgfscope}%
\begin{pgfscope}%
\pgfsys@transformshift{1.814019in}{2.276563in}%
\pgfsys@useobject{currentmarker}{}%
\end{pgfscope}%
\begin{pgfscope}%
\pgfsys@transformshift{1.828454in}{2.283492in}%
\pgfsys@useobject{currentmarker}{}%
\end{pgfscope}%
\begin{pgfscope}%
\pgfsys@transformshift{1.842300in}{2.290871in}%
\pgfsys@useobject{currentmarker}{}%
\end{pgfscope}%
\begin{pgfscope}%
\pgfsys@transformshift{1.855603in}{2.256449in}%
\pgfsys@useobject{currentmarker}{}%
\end{pgfscope}%
\begin{pgfscope}%
\pgfsys@transformshift{1.868404in}{2.252423in}%
\pgfsys@useobject{currentmarker}{}%
\end{pgfscope}%
\begin{pgfscope}%
\pgfsys@transformshift{1.880739in}{2.270300in}%
\pgfsys@useobject{currentmarker}{}%
\end{pgfscope}%
\begin{pgfscope}%
\pgfsys@transformshift{1.892641in}{2.262432in}%
\pgfsys@useobject{currentmarker}{}%
\end{pgfscope}%
\begin{pgfscope}%
\pgfsys@transformshift{1.904140in}{2.248129in}%
\pgfsys@useobject{currentmarker}{}%
\end{pgfscope}%
\begin{pgfscope}%
\pgfsys@transformshift{1.915261in}{2.253258in}%
\pgfsys@useobject{currentmarker}{}%
\end{pgfscope}%
\begin{pgfscope}%
\pgfsys@transformshift{1.926030in}{2.243927in}%
\pgfsys@useobject{currentmarker}{}%
\end{pgfscope}%
\begin{pgfscope}%
\pgfsys@transformshift{1.936467in}{2.225273in}%
\pgfsys@useobject{currentmarker}{}%
\end{pgfscope}%
\begin{pgfscope}%
\pgfsys@transformshift{1.946592in}{2.226196in}%
\pgfsys@useobject{currentmarker}{}%
\end{pgfscope}%
\begin{pgfscope}%
\pgfsys@transformshift{1.956424in}{2.246161in}%
\pgfsys@useobject{currentmarker}{}%
\end{pgfscope}%
\begin{pgfscope}%
\pgfsys@transformshift{1.965979in}{2.246572in}%
\pgfsys@useobject{currentmarker}{}%
\end{pgfscope}%
\begin{pgfscope}%
\pgfsys@transformshift{1.975272in}{2.221757in}%
\pgfsys@useobject{currentmarker}{}%
\end{pgfscope}%
\begin{pgfscope}%
\pgfsys@transformshift{1.984318in}{2.219103in}%
\pgfsys@useobject{currentmarker}{}%
\end{pgfscope}%
\begin{pgfscope}%
\pgfsys@transformshift{1.993128in}{2.212058in}%
\pgfsys@useobject{currentmarker}{}%
\end{pgfscope}%
\begin{pgfscope}%
\pgfsys@transformshift{2.001715in}{2.210222in}%
\pgfsys@useobject{currentmarker}{}%
\end{pgfscope}%
\begin{pgfscope}%
\pgfsys@transformshift{2.010090in}{2.210599in}%
\pgfsys@useobject{currentmarker}{}%
\end{pgfscope}%
\begin{pgfscope}%
\pgfsys@transformshift{2.018264in}{2.185799in}%
\pgfsys@useobject{currentmarker}{}%
\end{pgfscope}%
\begin{pgfscope}%
\pgfsys@transformshift{2.026245in}{2.188264in}%
\pgfsys@useobject{currentmarker}{}%
\end{pgfscope}%
\begin{pgfscope}%
\pgfsys@transformshift{2.034042in}{2.192759in}%
\pgfsys@useobject{currentmarker}{}%
\end{pgfscope}%
\begin{pgfscope}%
\pgfsys@transformshift{2.041665in}{2.183839in}%
\pgfsys@useobject{currentmarker}{}%
\end{pgfscope}%
\begin{pgfscope}%
\pgfsys@transformshift{2.049119in}{2.173392in}%
\pgfsys@useobject{currentmarker}{}%
\end{pgfscope}%
\begin{pgfscope}%
\pgfsys@transformshift{2.056414in}{2.163033in}%
\pgfsys@useobject{currentmarker}{}%
\end{pgfscope}%
\begin{pgfscope}%
\pgfsys@transformshift{2.063555in}{2.180108in}%
\pgfsys@useobject{currentmarker}{}%
\end{pgfscope}%
\begin{pgfscope}%
\pgfsys@transformshift{2.070548in}{2.176949in}%
\pgfsys@useobject{currentmarker}{}%
\end{pgfscope}%
\begin{pgfscope}%
\pgfsys@transformshift{2.077401in}{2.173534in}%
\pgfsys@useobject{currentmarker}{}%
\end{pgfscope}%
\begin{pgfscope}%
\pgfsys@transformshift{2.084117in}{2.158367in}%
\pgfsys@useobject{currentmarker}{}%
\end{pgfscope}%
\begin{pgfscope}%
\pgfsys@transformshift{2.093949in}{2.155150in}%
\pgfsys@useobject{currentmarker}{}%
\end{pgfscope}%
\begin{pgfscope}%
\pgfsys@transformshift{2.103504in}{2.153829in}%
\pgfsys@useobject{currentmarker}{}%
\end{pgfscope}%
\begin{pgfscope}%
\pgfsys@transformshift{2.109728in}{2.141879in}%
\pgfsys@useobject{currentmarker}{}%
\end{pgfscope}%
\begin{pgfscope}%
\pgfsys@transformshift{2.115839in}{2.132821in}%
\pgfsys@useobject{currentmarker}{}%
\end{pgfscope}%
\begin{pgfscope}%
\pgfsys@transformshift{2.124805in}{2.142289in}%
\pgfsys@useobject{currentmarker}{}%
\end{pgfscope}%
\begin{pgfscope}%
\pgfsys@transformshift{2.133540in}{2.113769in}%
\pgfsys@useobject{currentmarker}{}%
\end{pgfscope}%
\begin{pgfscope}%
\pgfsys@transformshift{2.139240in}{2.124439in}%
\pgfsys@useobject{currentmarker}{}%
\end{pgfscope}%
\begin{pgfscope}%
\pgfsys@transformshift{2.147615in}{2.116199in}%
\pgfsys@useobject{currentmarker}{}%
\end{pgfscope}%
\begin{pgfscope}%
\pgfsys@transformshift{2.155789in}{2.116386in}%
\pgfsys@useobject{currentmarker}{}%
\end{pgfscope}%
\begin{pgfscope}%
\pgfsys@transformshift{2.163770in}{2.091014in}%
\pgfsys@useobject{currentmarker}{}%
\end{pgfscope}%
\begin{pgfscope}%
\pgfsys@transformshift{2.171567in}{2.095018in}%
\pgfsys@useobject{currentmarker}{}%
\end{pgfscope}%
\begin{pgfscope}%
\pgfsys@transformshift{2.179190in}{2.115256in}%
\pgfsys@useobject{currentmarker}{}%
\end{pgfscope}%
\begin{pgfscope}%
\pgfsys@transformshift{2.186644in}{2.107445in}%
\pgfsys@useobject{currentmarker}{}%
\end{pgfscope}%
\begin{pgfscope}%
\pgfsys@transformshift{2.193939in}{2.078225in}%
\pgfsys@useobject{currentmarker}{}%
\end{pgfscope}%
\begin{pgfscope}%
\pgfsys@transformshift{2.203427in}{2.082310in}%
\pgfsys@useobject{currentmarker}{}%
\end{pgfscope}%
\begin{pgfscope}%
\pgfsys@transformshift{2.210373in}{2.084450in}%
\pgfsys@useobject{currentmarker}{}%
\end{pgfscope}%
\begin{pgfscope}%
\pgfsys@transformshift{2.217179in}{2.067392in}%
\pgfsys@useobject{currentmarker}{}%
\end{pgfscope}%
\begin{pgfscope}%
\pgfsys@transformshift{2.226047in}{2.073761in}%
\pgfsys@useobject{currentmarker}{}%
\end{pgfscope}%
\begin{pgfscope}%
\pgfsys@transformshift{2.234689in}{2.084131in}%
\pgfsys@useobject{currentmarker}{}%
\end{pgfscope}%
\begin{pgfscope}%
\pgfsys@transformshift{2.243116in}{2.067875in}%
\pgfsys@useobject{currentmarker}{}%
\end{pgfscope}%
\begin{pgfscope}%
\pgfsys@transformshift{2.251339in}{2.064152in}%
\pgfsys@useobject{currentmarker}{}%
\end{pgfscope}%
\begin{pgfscope}%
\pgfsys@transformshift{2.259368in}{2.048809in}%
\pgfsys@useobject{currentmarker}{}%
\end{pgfscope}%
\begin{pgfscope}%
\pgfsys@transformshift{2.267210in}{2.035917in}%
\pgfsys@useobject{currentmarker}{}%
\end{pgfscope}%
\begin{pgfscope}%
\pgfsys@transformshift{2.274876in}{2.037395in}%
\pgfsys@useobject{currentmarker}{}%
\end{pgfscope}%
\begin{pgfscope}%
\pgfsys@transformshift{2.282372in}{2.033088in}%
\pgfsys@useobject{currentmarker}{}%
\end{pgfscope}%
\begin{pgfscope}%
\pgfsys@transformshift{2.289705in}{2.014323in}%
\pgfsys@useobject{currentmarker}{}%
\end{pgfscope}%
\begin{pgfscope}%
\pgfsys@transformshift{2.296884in}{2.005601in}%
\pgfsys@useobject{currentmarker}{}%
\end{pgfscope}%
\begin{pgfscope}%
\pgfsys@transformshift{2.303914in}{2.010100in}%
\pgfsys@useobject{currentmarker}{}%
\end{pgfscope}%
\begin{pgfscope}%
\pgfsys@transformshift{2.312501in}{1.998748in}%
\pgfsys@useobject{currentmarker}{}%
\end{pgfscope}%
\begin{pgfscope}%
\pgfsys@transformshift{2.320876in}{2.003698in}%
\pgfsys@useobject{currentmarker}{}%
\end{pgfscope}%
\begin{pgfscope}%
\pgfsys@transformshift{2.327431in}{1.996319in}%
\pgfsys@useobject{currentmarker}{}%
\end{pgfscope}%
\begin{pgfscope}%
\pgfsys@transformshift{2.335450in}{1.993753in}%
\pgfsys@useobject{currentmarker}{}%
\end{pgfscope}%
\begin{pgfscope}%
\pgfsys@transformshift{2.343283in}{1.978344in}%
\pgfsys@useobject{currentmarker}{}%
\end{pgfscope}%
\begin{pgfscope}%
\pgfsys@transformshift{2.350940in}{1.976436in}%
\pgfsys@useobject{currentmarker}{}%
\end{pgfscope}%
\begin{pgfscope}%
\pgfsys@transformshift{2.359905in}{1.975299in}%
\pgfsys@useobject{currentmarker}{}%
\end{pgfscope}%
\begin{pgfscope}%
\pgfsys@transformshift{2.367200in}{1.969444in}%
\pgfsys@useobject{currentmarker}{}%
\end{pgfscope}%
\begin{pgfscope}%
\pgfsys@transformshift{2.374341in}{1.966938in}%
\pgfsys@useobject{currentmarker}{}%
\end{pgfscope}%
\begin{pgfscope}%
\pgfsys@transformshift{2.382716in}{1.955844in}%
\pgfsys@useobject{currentmarker}{}%
\end{pgfscope}%
\begin{pgfscope}%
\pgfsys@transformshift{2.390889in}{1.934756in}%
\pgfsys@useobject{currentmarker}{}%
\end{pgfscope}%
\begin{pgfscope}%
\pgfsys@transformshift{2.398870in}{1.935840in}%
\pgfsys@useobject{currentmarker}{}%
\end{pgfscope}%
\begin{pgfscope}%
\pgfsys@transformshift{2.406668in}{1.929706in}%
\pgfsys@useobject{currentmarker}{}%
\end{pgfscope}%
\begin{pgfscope}%
\pgfsys@transformshift{2.414290in}{1.922728in}%
\pgfsys@useobject{currentmarker}{}%
\end{pgfscope}%
\begin{pgfscope}%
\pgfsys@transformshift{2.421745in}{1.934868in}%
\pgfsys@useobject{currentmarker}{}%
\end{pgfscope}%
\begin{pgfscope}%
\pgfsys@transformshift{2.430240in}{1.921215in}%
\pgfsys@useobject{currentmarker}{}%
\end{pgfscope}%
\begin{pgfscope}%
\pgfsys@transformshift{2.438527in}{1.904432in}%
\pgfsys@useobject{currentmarker}{}%
\end{pgfscope}%
\begin{pgfscope}%
\pgfsys@transformshift{2.445473in}{1.906030in}%
\pgfsys@useobject{currentmarker}{}%
\end{pgfscope}%
\begin{pgfscope}%
\pgfsys@transformshift{2.453401in}{1.901020in}%
\pgfsys@useobject{currentmarker}{}%
\end{pgfscope}%
\begin{pgfscope}%
\pgfsys@transformshift{2.461148in}{1.892082in}%
\pgfsys@useobject{currentmarker}{}%
\end{pgfscope}%
\begin{pgfscope}%
\pgfsys@transformshift{2.468721in}{1.886508in}%
\pgfsys@useobject{currentmarker}{}%
\end{pgfscope}%
\begin{pgfscope}%
\pgfsys@transformshift{2.477175in}{1.879606in}%
\pgfsys@useobject{currentmarker}{}%
\end{pgfscope}%
\begin{pgfscope}%
\pgfsys@transformshift{2.485423in}{1.871759in}%
\pgfsys@useobject{currentmarker}{}%
\end{pgfscope}%
\begin{pgfscope}%
\pgfsys@transformshift{2.493475in}{1.877308in}%
\pgfsys@useobject{currentmarker}{}%
\end{pgfscope}%
\begin{pgfscope}%
\pgfsys@transformshift{2.501340in}{1.854179in}%
\pgfsys@useobject{currentmarker}{}%
\end{pgfscope}%
\begin{pgfscope}%
\pgfsys@transformshift{2.509027in}{1.858872in}%
\pgfsys@useobject{currentmarker}{}%
\end{pgfscope}%
\begin{pgfscope}%
\pgfsys@transformshift{2.516544in}{1.860354in}%
\pgfsys@useobject{currentmarker}{}%
\end{pgfscope}%
\begin{pgfscope}%
\pgfsys@transformshift{2.523898in}{1.847614in}%
\pgfsys@useobject{currentmarker}{}%
\end{pgfscope}%
\begin{pgfscope}%
\pgfsys@transformshift{2.531985in}{1.832204in}%
\pgfsys@useobject{currentmarker}{}%
\end{pgfscope}%
\begin{pgfscope}%
\pgfsys@transformshift{2.539883in}{1.830552in}%
\pgfsys@useobject{currentmarker}{}%
\end{pgfscope}%
\begin{pgfscope}%
\pgfsys@transformshift{2.547602in}{1.824957in}%
\pgfsys@useobject{currentmarker}{}%
\end{pgfscope}%
\begin{pgfscope}%
\pgfsys@transformshift{2.555149in}{1.813347in}%
\pgfsys@useobject{currentmarker}{}%
\end{pgfscope}%
\begin{pgfscope}%
\pgfsys@transformshift{2.562531in}{1.809799in}%
\pgfsys@useobject{currentmarker}{}%
\end{pgfscope}%
\begin{pgfscope}%
\pgfsys@transformshift{2.570550in}{1.801004in}%
\pgfsys@useobject{currentmarker}{}%
\end{pgfscope}%
\begin{pgfscope}%
\pgfsys@transformshift{2.578384in}{1.803959in}%
\pgfsys@useobject{currentmarker}{}%
\end{pgfscope}%
\begin{pgfscope}%
\pgfsys@transformshift{2.586040in}{1.795614in}%
\pgfsys@useobject{currentmarker}{}%
\end{pgfscope}%
\begin{pgfscope}%
\pgfsys@transformshift{2.594268in}{1.790854in}%
\pgfsys@useobject{currentmarker}{}%
\end{pgfscope}%
\begin{pgfscope}%
\pgfsys@transformshift{2.602300in}{1.784354in}%
\pgfsys@useobject{currentmarker}{}%
\end{pgfscope}%
\begin{pgfscope}%
\pgfsys@transformshift{2.610147in}{1.770069in}%
\pgfsys@useobject{currentmarker}{}%
\end{pgfscope}%
\begin{pgfscope}%
\pgfsys@transformshift{2.617816in}{1.763158in}%
\pgfsys@useobject{currentmarker}{}%
\end{pgfscope}%
\begin{pgfscope}%
\pgfsys@transformshift{2.625316in}{1.762591in}%
\pgfsys@useobject{currentmarker}{}%
\end{pgfscope}%
\begin{pgfscope}%
\pgfsys@transformshift{2.633313in}{1.751288in}%
\pgfsys@useobject{currentmarker}{}%
\end{pgfscope}%
\begin{pgfscope}%
\pgfsys@transformshift{2.641125in}{1.757589in}%
\pgfsys@useobject{currentmarker}{}%
\end{pgfscope}%
\begin{pgfscope}%
\pgfsys@transformshift{2.648762in}{1.742875in}%
\pgfsys@useobject{currentmarker}{}%
\end{pgfscope}%
\begin{pgfscope}%
\pgfsys@transformshift{2.656845in}{1.734751in}%
\pgfsys@useobject{currentmarker}{}%
\end{pgfscope}%
\begin{pgfscope}%
\pgfsys@transformshift{2.664741in}{1.735067in}%
\pgfsys@useobject{currentmarker}{}%
\end{pgfscope}%
\begin{pgfscope}%
\pgfsys@transformshift{2.672456in}{1.731989in}%
\pgfsys@useobject{currentmarker}{}%
\end{pgfscope}%
\begin{pgfscope}%
\pgfsys@transformshift{2.680574in}{1.716662in}%
\pgfsys@useobject{currentmarker}{}%
\end{pgfscope}%
\begin{pgfscope}%
\pgfsys@transformshift{2.688502in}{1.708710in}%
\pgfsys@useobject{currentmarker}{}%
\end{pgfscope}%
\begin{pgfscope}%
\pgfsys@transformshift{2.696248in}{1.700337in}%
\pgfsys@useobject{currentmarker}{}%
\end{pgfscope}%
\begin{pgfscope}%
\pgfsys@transformshift{2.703822in}{1.697143in}%
\pgfsys@useobject{currentmarker}{}%
\end{pgfscope}%
\begin{pgfscope}%
\pgfsys@transformshift{2.711753in}{1.688899in}%
\pgfsys@useobject{currentmarker}{}%
\end{pgfscope}%
\begin{pgfscope}%
\pgfsys@transformshift{2.719503in}{1.692055in}%
\pgfsys@useobject{currentmarker}{}%
\end{pgfscope}%
\begin{pgfscope}%
\pgfsys@transformshift{2.727080in}{1.679544in}%
\pgfsys@useobject{currentmarker}{}%
\end{pgfscope}%
\begin{pgfscope}%
\pgfsys@transformshift{2.734980in}{1.672842in}%
\pgfsys@useobject{currentmarker}{}%
\end{pgfscope}%
\begin{pgfscope}%
\pgfsys@transformshift{2.743177in}{1.664884in}%
\pgfsys@useobject{currentmarker}{}%
\end{pgfscope}%
\begin{pgfscope}%
\pgfsys@transformshift{2.751180in}{1.668670in}%
\pgfsys@useobject{currentmarker}{}%
\end{pgfscope}%
\begin{pgfscope}%
\pgfsys@transformshift{2.758998in}{1.647537in}%
\pgfsys@useobject{currentmarker}{}%
\end{pgfscope}%
\begin{pgfscope}%
\pgfsys@transformshift{2.766641in}{1.641199in}%
\pgfsys@useobject{currentmarker}{}%
\end{pgfscope}%
\begin{pgfscope}%
\pgfsys@transformshift{2.774115in}{1.635412in}%
\pgfsys@useobject{currentmarker}{}%
\end{pgfscope}%
\begin{pgfscope}%
\pgfsys@transformshift{2.782278in}{1.638310in}%
\pgfsys@useobject{currentmarker}{}%
\end{pgfscope}%
\begin{pgfscope}%
\pgfsys@transformshift{2.790249in}{1.629929in}%
\pgfsys@useobject{currentmarker}{}%
\end{pgfscope}%
\begin{pgfscope}%
\pgfsys@transformshift{2.798037in}{1.614088in}%
\pgfsys@useobject{currentmarker}{}%
\end{pgfscope}%
\begin{pgfscope}%
\pgfsys@transformshift{2.806047in}{1.614473in}%
\pgfsys@useobject{currentmarker}{}%
\end{pgfscope}%
\begin{pgfscope}%
\pgfsys@transformshift{2.813871in}{1.600726in}%
\pgfsys@useobject{currentmarker}{}%
\end{pgfscope}%
\begin{pgfscope}%
\pgfsys@transformshift{2.821519in}{1.599082in}%
\pgfsys@useobject{currentmarker}{}%
\end{pgfscope}%
\begin{pgfscope}%
\pgfsys@transformshift{2.829368in}{1.599589in}%
\pgfsys@useobject{currentmarker}{}%
\end{pgfscope}%
\begin{pgfscope}%
\pgfsys@transformshift{2.837040in}{1.589361in}%
\pgfsys@useobject{currentmarker}{}%
\end{pgfscope}%
\begin{pgfscope}%
\pgfsys@transformshift{2.844895in}{1.591960in}%
\pgfsys@useobject{currentmarker}{}%
\end{pgfscope}%
\begin{pgfscope}%
\pgfsys@transformshift{2.852917in}{1.575274in}%
\pgfsys@useobject{currentmarker}{}%
\end{pgfscope}%
\begin{pgfscope}%
\pgfsys@transformshift{2.860754in}{1.570242in}%
\pgfsys@useobject{currentmarker}{}%
\end{pgfscope}%
\begin{pgfscope}%
\pgfsys@transformshift{2.868413in}{1.564665in}%
\pgfsys@useobject{currentmarker}{}%
\end{pgfscope}%
\begin{pgfscope}%
\pgfsys@transformshift{2.876226in}{1.549913in}%
\pgfsys@useobject{currentmarker}{}%
\end{pgfscope}%
\begin{pgfscope}%
\pgfsys@transformshift{2.884177in}{1.558970in}%
\pgfsys@useobject{currentmarker}{}%
\end{pgfscope}%
\begin{pgfscope}%
\pgfsys@transformshift{2.891946in}{1.542427in}%
\pgfsys@useobject{currentmarker}{}%
\end{pgfscope}%
\begin{pgfscope}%
\pgfsys@transformshift{2.899841in}{1.546029in}%
\pgfsys@useobject{currentmarker}{}%
\end{pgfscope}%
\begin{pgfscope}%
\pgfsys@transformshift{2.907557in}{1.530025in}%
\pgfsys@useobject{currentmarker}{}%
\end{pgfscope}%
\begin{pgfscope}%
\pgfsys@transformshift{2.915388in}{1.529997in}%
\pgfsys@useobject{currentmarker}{}%
\end{pgfscope}%
\begin{pgfscope}%
\pgfsys@transformshift{2.923322in}{1.519984in}%
\pgfsys@useobject{currentmarker}{}%
\end{pgfscope}%
\begin{pgfscope}%
\pgfsys@transformshift{2.931075in}{1.517926in}%
\pgfsys@useobject{currentmarker}{}%
\end{pgfscope}%
\begin{pgfscope}%
\pgfsys@transformshift{2.938923in}{1.512908in}%
\pgfsys@useobject{currentmarker}{}%
\end{pgfscope}%
\begin{pgfscope}%
\pgfsys@transformshift{2.946854in}{1.499336in}%
\pgfsys@useobject{currentmarker}{}%
\end{pgfscope}%
\begin{pgfscope}%
\pgfsys@transformshift{2.954604in}{1.492570in}%
\pgfsys@useobject{currentmarker}{}%
\end{pgfscope}%
\begin{pgfscope}%
\pgfsys@transformshift{2.962430in}{1.488204in}%
\pgfsys@useobject{currentmarker}{}%
\end{pgfscope}%
\begin{pgfscope}%
\pgfsys@transformshift{2.970324in}{1.485822in}%
\pgfsys@useobject{currentmarker}{}%
\end{pgfscope}%
\begin{pgfscope}%
\pgfsys@transformshift{2.978039in}{1.477367in}%
\pgfsys@useobject{currentmarker}{}%
\end{pgfscope}%
\begin{pgfscope}%
\pgfsys@transformshift{2.985815in}{1.467965in}%
\pgfsys@useobject{currentmarker}{}%
\end{pgfscope}%
\begin{pgfscope}%
\pgfsys@transformshift{2.993644in}{1.466007in}%
\pgfsys@useobject{currentmarker}{}%
\end{pgfscope}%
\begin{pgfscope}%
\pgfsys@transformshift{3.001519in}{1.452237in}%
\pgfsys@useobject{currentmarker}{}%
\end{pgfscope}%
\begin{pgfscope}%
\pgfsys@transformshift{3.009433in}{1.455087in}%
\pgfsys@useobject{currentmarker}{}%
\end{pgfscope}%
\begin{pgfscope}%
\pgfsys@transformshift{3.017166in}{1.444148in}%
\pgfsys@useobject{currentmarker}{}%
\end{pgfscope}%
\begin{pgfscope}%
\pgfsys@transformshift{3.024935in}{1.436237in}%
\pgfsys@useobject{currentmarker}{}%
\end{pgfscope}%
\begin{pgfscope}%
\pgfsys@transformshift{3.032732in}{1.428828in}%
\pgfsys@useobject{currentmarker}{}%
\end{pgfscope}%
\begin{pgfscope}%
\pgfsys@transformshift{3.040553in}{1.432246in}%
\pgfsys@useobject{currentmarker}{}%
\end{pgfscope}%
\begin{pgfscope}%
\pgfsys@transformshift{3.048391in}{1.418699in}%
\pgfsys@useobject{currentmarker}{}%
\end{pgfscope}%
\begin{pgfscope}%
\pgfsys@transformshift{3.056241in}{1.411626in}%
\pgfsys@useobject{currentmarker}{}%
\end{pgfscope}%
\begin{pgfscope}%
\pgfsys@transformshift{3.064099in}{1.400894in}%
\pgfsys@useobject{currentmarker}{}%
\end{pgfscope}%
\begin{pgfscope}%
\pgfsys@transformshift{3.071960in}{1.398123in}%
\pgfsys@useobject{currentmarker}{}%
\end{pgfscope}%
\begin{pgfscope}%
\pgfsys@transformshift{3.079819in}{1.394982in}%
\pgfsys@useobject{currentmarker}{}%
\end{pgfscope}%
\begin{pgfscope}%
\pgfsys@transformshift{3.087673in}{1.386823in}%
\pgfsys@useobject{currentmarker}{}%
\end{pgfscope}%
\begin{pgfscope}%
\pgfsys@transformshift{3.095517in}{1.380144in}%
\pgfsys@useobject{currentmarker}{}%
\end{pgfscope}%
\begin{pgfscope}%
\pgfsys@transformshift{3.103349in}{1.376718in}%
\pgfsys@useobject{currentmarker}{}%
\end{pgfscope}%
\begin{pgfscope}%
\pgfsys@transformshift{3.111166in}{1.368263in}%
\pgfsys@useobject{currentmarker}{}%
\end{pgfscope}%
\begin{pgfscope}%
\pgfsys@transformshift{3.118963in}{1.368114in}%
\pgfsys@useobject{currentmarker}{}%
\end{pgfscope}%
\begin{pgfscope}%
\pgfsys@transformshift{3.126739in}{1.353066in}%
\pgfsys@useobject{currentmarker}{}%
\end{pgfscope}%
\begin{pgfscope}%
\pgfsys@transformshift{3.134491in}{1.347703in}%
\pgfsys@useobject{currentmarker}{}%
\end{pgfscope}%
\begin{pgfscope}%
\pgfsys@transformshift{3.142364in}{1.343550in}%
\pgfsys@useobject{currentmarker}{}%
\end{pgfscope}%
\begin{pgfscope}%
\pgfsys@transformshift{3.150202in}{1.334570in}%
\pgfsys@useobject{currentmarker}{}%
\end{pgfscope}%
\begin{pgfscope}%
\pgfsys@transformshift{3.158002in}{1.325484in}%
\pgfsys@useobject{currentmarker}{}%
\end{pgfscope}%
\begin{pgfscope}%
\pgfsys@transformshift{3.165902in}{1.322108in}%
\pgfsys@useobject{currentmarker}{}%
\end{pgfscope}%
\begin{pgfscope}%
\pgfsys@transformshift{3.173756in}{1.317947in}%
\pgfsys@useobject{currentmarker}{}%
\end{pgfscope}%
\begin{pgfscope}%
\pgfsys@transformshift{3.181562in}{1.305417in}%
\pgfsys@useobject{currentmarker}{}%
\end{pgfscope}%
\begin{pgfscope}%
\pgfsys@transformshift{3.189449in}{1.305961in}%
\pgfsys@useobject{currentmarker}{}%
\end{pgfscope}%
\begin{pgfscope}%
\pgfsys@transformshift{3.197281in}{1.300219in}%
\pgfsys@useobject{currentmarker}{}%
\end{pgfscope}%
\begin{pgfscope}%
\pgfsys@transformshift{3.205059in}{1.289502in}%
\pgfsys@useobject{currentmarker}{}%
\end{pgfscope}%
\begin{pgfscope}%
\pgfsys@transformshift{3.212901in}{1.288984in}%
\pgfsys@useobject{currentmarker}{}%
\end{pgfscope}%
\begin{pgfscope}%
\pgfsys@transformshift{3.220799in}{1.281645in}%
\pgfsys@useobject{currentmarker}{}%
\end{pgfscope}%
\begin{pgfscope}%
\pgfsys@transformshift{3.228631in}{1.275754in}%
\pgfsys@useobject{currentmarker}{}%
\end{pgfscope}%
\begin{pgfscope}%
\pgfsys@transformshift{3.236397in}{1.267001in}%
\pgfsys@useobject{currentmarker}{}%
\end{pgfscope}%
\begin{pgfscope}%
\pgfsys@transformshift{3.244207in}{1.264374in}%
\pgfsys@useobject{currentmarker}{}%
\end{pgfscope}%
\begin{pgfscope}%
\pgfsys@transformshift{3.252054in}{1.255234in}%
\pgfsys@useobject{currentmarker}{}%
\end{pgfscope}%
\begin{pgfscope}%
\pgfsys@transformshift{3.259932in}{1.250595in}%
\pgfsys@useobject{currentmarker}{}%
\end{pgfscope}%
\begin{pgfscope}%
\pgfsys@transformshift{3.267732in}{1.242916in}%
\pgfsys@useobject{currentmarker}{}%
\end{pgfscope}%
\begin{pgfscope}%
\pgfsys@transformshift{3.275554in}{1.236193in}%
\pgfsys@useobject{currentmarker}{}%
\end{pgfscope}%
\begin{pgfscope}%
\pgfsys@transformshift{3.283395in}{1.228386in}%
\pgfsys@useobject{currentmarker}{}%
\end{pgfscope}%
\begin{pgfscope}%
\pgfsys@transformshift{3.291153in}{1.225093in}%
\pgfsys@useobject{currentmarker}{}%
\end{pgfscope}%
\begin{pgfscope}%
\pgfsys@transformshift{3.299015in}{1.217284in}%
\pgfsys@useobject{currentmarker}{}%
\end{pgfscope}%
\begin{pgfscope}%
\pgfsys@transformshift{3.306880in}{1.211545in}%
\pgfsys@useobject{currentmarker}{}%
\end{pgfscope}%
\begin{pgfscope}%
\pgfsys@transformshift{3.314655in}{1.203353in}%
\pgfsys@useobject{currentmarker}{}%
\end{pgfscope}%
\begin{pgfscope}%
\pgfsys@transformshift{3.322514in}{1.193682in}%
\pgfsys@useobject{currentmarker}{}%
\end{pgfscope}%
\begin{pgfscope}%
\pgfsys@transformshift{3.330365in}{1.189060in}%
\pgfsys@useobject{currentmarker}{}%
\end{pgfscope}%
\begin{pgfscope}%
\pgfsys@transformshift{3.338203in}{1.184324in}%
\pgfsys@useobject{currentmarker}{}%
\end{pgfscope}%
\begin{pgfscope}%
\pgfsys@transformshift{3.346025in}{1.178167in}%
\pgfsys@useobject{currentmarker}{}%
\end{pgfscope}%
\begin{pgfscope}%
\pgfsys@transformshift{3.353828in}{1.173559in}%
\pgfsys@useobject{currentmarker}{}%
\end{pgfscope}%
\begin{pgfscope}%
\pgfsys@transformshift{3.361686in}{1.167920in}%
\pgfsys@useobject{currentmarker}{}%
\end{pgfscope}%
\begin{pgfscope}%
\pgfsys@transformshift{3.369517in}{1.160531in}%
\pgfsys@useobject{currentmarker}{}%
\end{pgfscope}%
\begin{pgfscope}%
\pgfsys@transformshift{3.377318in}{1.157265in}%
\pgfsys@useobject{currentmarker}{}%
\end{pgfscope}%
\begin{pgfscope}%
\pgfsys@transformshift{3.385159in}{1.146412in}%
\pgfsys@useobject{currentmarker}{}%
\end{pgfscope}%
\begin{pgfscope}%
\pgfsys@transformshift{3.393033in}{1.143415in}%
\pgfsys@useobject{currentmarker}{}%
\end{pgfscope}%
\begin{pgfscope}%
\pgfsys@transformshift{3.400865in}{1.138258in}%
\pgfsys@useobject{currentmarker}{}%
\end{pgfscope}%
\begin{pgfscope}%
\pgfsys@transformshift{3.408655in}{1.127803in}%
\pgfsys@useobject{currentmarker}{}%
\end{pgfscope}%
\begin{pgfscope}%
\pgfsys@transformshift{3.416466in}{1.123552in}%
\pgfsys@useobject{currentmarker}{}%
\end{pgfscope}%
\begin{pgfscope}%
\pgfsys@transformshift{3.424294in}{1.119173in}%
\pgfsys@useobject{currentmarker}{}%
\end{pgfscope}%
\begin{pgfscope}%
\pgfsys@transformshift{3.432132in}{1.113106in}%
\pgfsys@useobject{currentmarker}{}%
\end{pgfscope}%
\begin{pgfscope}%
\pgfsys@transformshift{3.439976in}{1.109386in}%
\pgfsys@useobject{currentmarker}{}%
\end{pgfscope}%
\begin{pgfscope}%
\pgfsys@transformshift{3.447823in}{1.100879in}%
\pgfsys@useobject{currentmarker}{}%
\end{pgfscope}%
\begin{pgfscope}%
\pgfsys@transformshift{3.455666in}{1.094077in}%
\pgfsys@useobject{currentmarker}{}%
\end{pgfscope}%
\begin{pgfscope}%
\pgfsys@transformshift{3.463447in}{1.089718in}%
\pgfsys@useobject{currentmarker}{}%
\end{pgfscope}%
\begin{pgfscope}%
\pgfsys@transformshift{3.471275in}{1.080117in}%
\pgfsys@useobject{currentmarker}{}%
\end{pgfscope}%
\begin{pgfscope}%
\pgfsys@transformshift{3.479145in}{1.077536in}%
\pgfsys@useobject{currentmarker}{}%
\end{pgfscope}%
\begin{pgfscope}%
\pgfsys@transformshift{3.486942in}{1.072077in}%
\pgfsys@useobject{currentmarker}{}%
\end{pgfscope}%
\begin{pgfscope}%
\pgfsys@transformshift{3.494773in}{1.068861in}%
\pgfsys@useobject{currentmarker}{}%
\end{pgfscope}%
\begin{pgfscope}%
\pgfsys@transformshift{3.502629in}{1.057550in}%
\pgfsys@useobject{currentmarker}{}%
\end{pgfscope}%
\begin{pgfscope}%
\pgfsys@transformshift{3.510457in}{1.053673in}%
\pgfsys@useobject{currentmarker}{}%
\end{pgfscope}%
\begin{pgfscope}%
\pgfsys@transformshift{3.518253in}{1.050134in}%
\pgfsys@useobject{currentmarker}{}%
\end{pgfscope}%
\begin{pgfscope}%
\pgfsys@transformshift{3.526064in}{1.043666in}%
\pgfsys@useobject{currentmarker}{}%
\end{pgfscope}%
\begin{pgfscope}%
\pgfsys@transformshift{3.533930in}{1.037843in}%
\pgfsys@useobject{currentmarker}{}%
\end{pgfscope}%
\begin{pgfscope}%
\pgfsys@transformshift{3.541754in}{1.033860in}%
\pgfsys@useobject{currentmarker}{}%
\end{pgfscope}%
\begin{pgfscope}%
\pgfsys@transformshift{3.549578in}{1.026047in}%
\pgfsys@useobject{currentmarker}{}%
\end{pgfscope}%
\begin{pgfscope}%
\pgfsys@transformshift{3.557399in}{1.017522in}%
\pgfsys@useobject{currentmarker}{}%
\end{pgfscope}%
\begin{pgfscope}%
\pgfsys@transformshift{3.565212in}{1.015202in}%
\pgfsys@useobject{currentmarker}{}%
\end{pgfscope}%
\begin{pgfscope}%
\pgfsys@transformshift{3.573056in}{1.007339in}%
\pgfsys@useobject{currentmarker}{}%
\end{pgfscope}%
\begin{pgfscope}%
\pgfsys@transformshift{3.580883in}{1.004465in}%
\pgfsys@useobject{currentmarker}{}%
\end{pgfscope}%
\begin{pgfscope}%
\pgfsys@transformshift{3.588731in}{0.993937in}%
\pgfsys@useobject{currentmarker}{}%
\end{pgfscope}%
\begin{pgfscope}%
\pgfsys@transformshift{3.596556in}{0.992775in}%
\pgfsys@useobject{currentmarker}{}%
\end{pgfscope}%
\begin{pgfscope}%
\pgfsys@transformshift{3.604354in}{0.986978in}%
\pgfsys@useobject{currentmarker}{}%
\end{pgfscope}%
\begin{pgfscope}%
\pgfsys@transformshift{3.612198in}{0.983277in}%
\pgfsys@useobject{currentmarker}{}%
\end{pgfscope}%
\begin{pgfscope}%
\pgfsys@transformshift{3.620044in}{0.976143in}%
\pgfsys@useobject{currentmarker}{}%
\end{pgfscope}%
\begin{pgfscope}%
\pgfsys@transformshift{3.627888in}{0.973304in}%
\pgfsys@useobject{currentmarker}{}%
\end{pgfscope}%
\begin{pgfscope}%
\pgfsys@transformshift{3.635726in}{0.966732in}%
\pgfsys@useobject{currentmarker}{}%
\end{pgfscope}%
\begin{pgfscope}%
\pgfsys@transformshift{3.643555in}{0.960707in}%
\pgfsys@useobject{currentmarker}{}%
\end{pgfscope}%
\begin{pgfscope}%
\pgfsys@transformshift{3.651371in}{0.955502in}%
\pgfsys@useobject{currentmarker}{}%
\end{pgfscope}%
\begin{pgfscope}%
\pgfsys@transformshift{3.659170in}{0.953343in}%
\pgfsys@useobject{currentmarker}{}%
\end{pgfscope}%
\begin{pgfscope}%
\pgfsys@transformshift{3.667014in}{0.943937in}%
\pgfsys@useobject{currentmarker}{}%
\end{pgfscope}%
\begin{pgfscope}%
\pgfsys@transformshift{3.674863in}{0.942055in}%
\pgfsys@useobject{currentmarker}{}%
\end{pgfscope}%
\begin{pgfscope}%
\pgfsys@transformshift{3.682684in}{0.938829in}%
\pgfsys@useobject{currentmarker}{}%
\end{pgfscope}%
\begin{pgfscope}%
\pgfsys@transformshift{3.690504in}{0.932359in}%
\pgfsys@useobject{currentmarker}{}%
\end{pgfscope}%
\begin{pgfscope}%
\pgfsys@transformshift{3.698319in}{0.928450in}%
\pgfsys@useobject{currentmarker}{}%
\end{pgfscope}%
\begin{pgfscope}%
\pgfsys@transformshift{3.706153in}{0.923952in}%
\pgfsys@useobject{currentmarker}{}%
\end{pgfscope}%
\begin{pgfscope}%
\pgfsys@transformshift{3.714000in}{0.921470in}%
\pgfsys@useobject{currentmarker}{}%
\end{pgfscope}%
\begin{pgfscope}%
\pgfsys@transformshift{3.721830in}{0.915759in}%
\pgfsys@useobject{currentmarker}{}%
\end{pgfscope}%
\begin{pgfscope}%
\pgfsys@transformshift{3.729665in}{0.910090in}%
\pgfsys@useobject{currentmarker}{}%
\end{pgfscope}%
\begin{pgfscope}%
\pgfsys@transformshift{3.737501in}{0.907638in}%
\pgfsys@useobject{currentmarker}{}%
\end{pgfscope}%
\begin{pgfscope}%
\pgfsys@transformshift{3.745309in}{0.902834in}%
\pgfsys@useobject{currentmarker}{}%
\end{pgfscope}%
\begin{pgfscope}%
\pgfsys@transformshift{3.753135in}{0.900327in}%
\pgfsys@useobject{currentmarker}{}%
\end{pgfscope}%
\begin{pgfscope}%
\pgfsys@transformshift{3.760975in}{0.895926in}%
\pgfsys@useobject{currentmarker}{}%
\end{pgfscope}%
\begin{pgfscope}%
\pgfsys@transformshift{3.768799in}{0.895633in}%
\pgfsys@useobject{currentmarker}{}%
\end{pgfscope}%
\begin{pgfscope}%
\pgfsys@transformshift{3.776628in}{0.890559in}%
\pgfsys@useobject{currentmarker}{}%
\end{pgfscope}%
\begin{pgfscope}%
\pgfsys@transformshift{3.784458in}{0.886817in}%
\pgfsys@useobject{currentmarker}{}%
\end{pgfscope}%
\begin{pgfscope}%
\pgfsys@transformshift{3.792284in}{0.883504in}%
\pgfsys@useobject{currentmarker}{}%
\end{pgfscope}%
\begin{pgfscope}%
\pgfsys@transformshift{3.800123in}{0.880033in}%
\pgfsys@useobject{currentmarker}{}%
\end{pgfscope}%
\begin{pgfscope}%
\pgfsys@transformshift{3.807950in}{0.876751in}%
\pgfsys@useobject{currentmarker}{}%
\end{pgfscope}%
\begin{pgfscope}%
\pgfsys@transformshift{3.815762in}{0.876603in}%
\pgfsys@useobject{currentmarker}{}%
\end{pgfscope}%
\begin{pgfscope}%
\pgfsys@transformshift{3.823596in}{0.874755in}%
\pgfsys@useobject{currentmarker}{}%
\end{pgfscope}%
\begin{pgfscope}%
\pgfsys@transformshift{3.831425in}{0.872852in}%
\pgfsys@useobject{currentmarker}{}%
\end{pgfscope}%
\begin{pgfscope}%
\pgfsys@transformshift{3.839267in}{0.870759in}%
\pgfsys@useobject{currentmarker}{}%
\end{pgfscope}%
\begin{pgfscope}%
\pgfsys@transformshift{3.847096in}{0.868349in}%
\pgfsys@useobject{currentmarker}{}%
\end{pgfscope}%
\begin{pgfscope}%
\pgfsys@transformshift{3.854911in}{0.868101in}%
\pgfsys@useobject{currentmarker}{}%
\end{pgfscope}%
\begin{pgfscope}%
\pgfsys@transformshift{3.862743in}{0.868406in}%
\pgfsys@useobject{currentmarker}{}%
\end{pgfscope}%
\begin{pgfscope}%
\pgfsys@transformshift{3.870569in}{0.866777in}%
\pgfsys@useobject{currentmarker}{}%
\end{pgfscope}%
\end{pgfscope}%
\begin{pgfscope}%
\pgfpathrectangle{\pgfqpoint{0.594525in}{0.417642in}}{\pgfqpoint{3.432047in}{2.016277in}}%
\pgfusepath{clip}%
\pgfsetbuttcap%
\pgfsetroundjoin%
\pgfsetlinewidth{1.505625pt}%
\definecolor{currentstroke}{rgb}{0.835294,0.368627,0.000000}%
\pgfsetstrokecolor{currentstroke}%
\pgfsetdash{{5.550000pt}{2.400000pt}}{0.000000pt}%
\pgfpathmoveto{\pgfqpoint{0.750527in}{2.266308in}}%
\pgfpathlineto{\pgfqpoint{0.985627in}{2.264961in}}%
\pgfpathlineto{\pgfqpoint{1.123152in}{2.262746in}}%
\pgfpathlineto{\pgfqpoint{1.220728in}{2.259703in}}%
\pgfpathlineto{\pgfqpoint{1.296413in}{2.255885in}}%
\pgfpathlineto{\pgfqpoint{1.358253in}{2.251358in}}%
\pgfpathlineto{\pgfqpoint{1.410538in}{2.246191in}}%
\pgfpathlineto{\pgfqpoint{1.455829in}{2.240460in}}%
\pgfpathlineto{\pgfqpoint{1.531514in}{2.227600in}}%
\pgfpathlineto{\pgfqpoint{1.593354in}{2.213346in}}%
\pgfpathlineto{\pgfqpoint{1.645638in}{2.198197in}}%
\pgfpathlineto{\pgfqpoint{1.690929in}{2.182554in}}%
\pgfpathlineto{\pgfqpoint{1.730879in}{2.166727in}}%
\pgfpathlineto{\pgfqpoint{1.783163in}{2.143119in}}%
\pgfpathlineto{\pgfqpoint{1.828454in}{2.120094in}}%
\pgfpathlineto{\pgfqpoint{1.880739in}{2.090744in}}%
\pgfpathlineto{\pgfqpoint{1.936467in}{2.056542in}}%
\pgfpathlineto{\pgfqpoint{1.993128in}{2.019139in}}%
\pgfpathlineto{\pgfqpoint{2.063555in}{1.969686in}}%
\pgfpathlineto{\pgfqpoint{2.139240in}{1.913782in}}%
\pgfpathlineto{\pgfqpoint{2.234689in}{1.840491in}}%
\pgfpathlineto{\pgfqpoint{2.367200in}{1.735651in}}%
\pgfpathlineto{\pgfqpoint{2.570550in}{1.571478in}}%
\pgfpathlineto{\pgfqpoint{2.993644in}{1.226334in}}%
\pgfpathlineto{\pgfqpoint{3.870569in}{0.509291in}}%
\pgfpathlineto{\pgfqpoint{3.870569in}{0.509291in}}%
\pgfusepath{stroke}%
\end{pgfscope}%
\begin{pgfscope}%
\pgfpathrectangle{\pgfqpoint{0.594525in}{0.417642in}}{\pgfqpoint{3.432047in}{2.016277in}}%
\pgfusepath{clip}%
\pgfsetbuttcap%
\pgfsetroundjoin%
\definecolor{currentfill}{rgb}{0.835294,0.368627,0.000000}%
\pgfsetfillcolor{currentfill}%
\pgfsetlinewidth{1.003750pt}%
\definecolor{currentstroke}{rgb}{0.835294,0.368627,0.000000}%
\pgfsetstrokecolor{currentstroke}%
\pgfsetdash{}{0pt}%
\pgfsys@defobject{currentmarker}{\pgfqpoint{-0.006944in}{-0.006944in}}{\pgfqpoint{0.006944in}{0.006944in}}{%
\pgfpathmoveto{\pgfqpoint{0.000000in}{-0.006944in}}%
\pgfpathcurveto{\pgfqpoint{0.001842in}{-0.006944in}}{\pgfqpoint{0.003608in}{-0.006213in}}{\pgfqpoint{0.004910in}{-0.004910in}}%
\pgfpathcurveto{\pgfqpoint{0.006213in}{-0.003608in}}{\pgfqpoint{0.006944in}{-0.001842in}}{\pgfqpoint{0.006944in}{0.000000in}}%
\pgfpathcurveto{\pgfqpoint{0.006944in}{0.001842in}}{\pgfqpoint{0.006213in}{0.003608in}}{\pgfqpoint{0.004910in}{0.004910in}}%
\pgfpathcurveto{\pgfqpoint{0.003608in}{0.006213in}}{\pgfqpoint{0.001842in}{0.006944in}}{\pgfqpoint{0.000000in}{0.006944in}}%
\pgfpathcurveto{\pgfqpoint{-0.001842in}{0.006944in}}{\pgfqpoint{-0.003608in}{0.006213in}}{\pgfqpoint{-0.004910in}{0.004910in}}%
\pgfpathcurveto{\pgfqpoint{-0.006213in}{0.003608in}}{\pgfqpoint{-0.006944in}{0.001842in}}{\pgfqpoint{-0.006944in}{0.000000in}}%
\pgfpathcurveto{\pgfqpoint{-0.006944in}{-0.001842in}}{\pgfqpoint{-0.006213in}{-0.003608in}}{\pgfqpoint{-0.004910in}{-0.004910in}}%
\pgfpathcurveto{\pgfqpoint{-0.003608in}{-0.006213in}}{\pgfqpoint{-0.001842in}{-0.006944in}}{\pgfqpoint{0.000000in}{-0.006944in}}%
\pgfpathlineto{\pgfqpoint{0.000000in}{-0.006944in}}%
\pgfpathclose%
\pgfusepath{stroke,fill}%
}%
\begin{pgfscope}%
\pgfsys@transformshift{0.750527in}{2.242922in}%
\pgfsys@useobject{currentmarker}{}%
\end{pgfscope}%
\begin{pgfscope}%
\pgfsys@transformshift{0.985627in}{2.267711in}%
\pgfsys@useobject{currentmarker}{}%
\end{pgfscope}%
\begin{pgfscope}%
\pgfsys@transformshift{1.123152in}{2.258194in}%
\pgfsys@useobject{currentmarker}{}%
\end{pgfscope}%
\begin{pgfscope}%
\pgfsys@transformshift{1.220728in}{2.255756in}%
\pgfsys@useobject{currentmarker}{}%
\end{pgfscope}%
\begin{pgfscope}%
\pgfsys@transformshift{1.296413in}{2.258359in}%
\pgfsys@useobject{currentmarker}{}%
\end{pgfscope}%
\begin{pgfscope}%
\pgfsys@transformshift{1.358253in}{2.248696in}%
\pgfsys@useobject{currentmarker}{}%
\end{pgfscope}%
\begin{pgfscope}%
\pgfsys@transformshift{1.410538in}{2.230399in}%
\pgfsys@useobject{currentmarker}{}%
\end{pgfscope}%
\begin{pgfscope}%
\pgfsys@transformshift{1.455829in}{2.232375in}%
\pgfsys@useobject{currentmarker}{}%
\end{pgfscope}%
\begin{pgfscope}%
\pgfsys@transformshift{1.495778in}{2.224078in}%
\pgfsys@useobject{currentmarker}{}%
\end{pgfscope}%
\begin{pgfscope}%
\pgfsys@transformshift{1.531514in}{2.216072in}%
\pgfsys@useobject{currentmarker}{}%
\end{pgfscope}%
\begin{pgfscope}%
\pgfsys@transformshift{1.563841in}{2.204274in}%
\pgfsys@useobject{currentmarker}{}%
\end{pgfscope}%
\begin{pgfscope}%
\pgfsys@transformshift{1.593354in}{2.184573in}%
\pgfsys@useobject{currentmarker}{}%
\end{pgfscope}%
\begin{pgfscope}%
\pgfsys@transformshift{1.620502in}{2.178099in}%
\pgfsys@useobject{currentmarker}{}%
\end{pgfscope}%
\begin{pgfscope}%
\pgfsys@transformshift{1.645638in}{2.195228in}%
\pgfsys@useobject{currentmarker}{}%
\end{pgfscope}%
\begin{pgfscope}%
\pgfsys@transformshift{1.669039in}{2.193206in}%
\pgfsys@useobject{currentmarker}{}%
\end{pgfscope}%
\begin{pgfscope}%
\pgfsys@transformshift{1.690929in}{2.173130in}%
\pgfsys@useobject{currentmarker}{}%
\end{pgfscope}%
\begin{pgfscope}%
\pgfsys@transformshift{1.711492in}{2.155337in}%
\pgfsys@useobject{currentmarker}{}%
\end{pgfscope}%
\begin{pgfscope}%
\pgfsys@transformshift{1.730879in}{2.155788in}%
\pgfsys@useobject{currentmarker}{}%
\end{pgfscope}%
\begin{pgfscope}%
\pgfsys@transformshift{1.749217in}{2.156894in}%
\pgfsys@useobject{currentmarker}{}%
\end{pgfscope}%
\begin{pgfscope}%
\pgfsys@transformshift{1.766615in}{2.144229in}%
\pgfsys@useobject{currentmarker}{}%
\end{pgfscope}%
\begin{pgfscope}%
\pgfsys@transformshift{1.783163in}{2.137224in}%
\pgfsys@useobject{currentmarker}{}%
\end{pgfscope}%
\begin{pgfscope}%
\pgfsys@transformshift{1.798942in}{2.144611in}%
\pgfsys@useobject{currentmarker}{}%
\end{pgfscope}%
\begin{pgfscope}%
\pgfsys@transformshift{1.814019in}{2.130034in}%
\pgfsys@useobject{currentmarker}{}%
\end{pgfscope}%
\begin{pgfscope}%
\pgfsys@transformshift{1.828454in}{2.124470in}%
\pgfsys@useobject{currentmarker}{}%
\end{pgfscope}%
\begin{pgfscope}%
\pgfsys@transformshift{1.842300in}{2.108289in}%
\pgfsys@useobject{currentmarker}{}%
\end{pgfscope}%
\begin{pgfscope}%
\pgfsys@transformshift{1.855603in}{2.102036in}%
\pgfsys@useobject{currentmarker}{}%
\end{pgfscope}%
\begin{pgfscope}%
\pgfsys@transformshift{1.868404in}{2.104159in}%
\pgfsys@useobject{currentmarker}{}%
\end{pgfscope}%
\begin{pgfscope}%
\pgfsys@transformshift{1.880739in}{2.082110in}%
\pgfsys@useobject{currentmarker}{}%
\end{pgfscope}%
\begin{pgfscope}%
\pgfsys@transformshift{1.892641in}{2.077393in}%
\pgfsys@useobject{currentmarker}{}%
\end{pgfscope}%
\begin{pgfscope}%
\pgfsys@transformshift{1.904140in}{2.080732in}%
\pgfsys@useobject{currentmarker}{}%
\end{pgfscope}%
\begin{pgfscope}%
\pgfsys@transformshift{1.915261in}{2.066059in}%
\pgfsys@useobject{currentmarker}{}%
\end{pgfscope}%
\begin{pgfscope}%
\pgfsys@transformshift{1.926030in}{2.043936in}%
\pgfsys@useobject{currentmarker}{}%
\end{pgfscope}%
\begin{pgfscope}%
\pgfsys@transformshift{1.936467in}{2.034250in}%
\pgfsys@useobject{currentmarker}{}%
\end{pgfscope}%
\begin{pgfscope}%
\pgfsys@transformshift{1.946592in}{2.039148in}%
\pgfsys@useobject{currentmarker}{}%
\end{pgfscope}%
\begin{pgfscope}%
\pgfsys@transformshift{1.956424in}{2.038467in}%
\pgfsys@useobject{currentmarker}{}%
\end{pgfscope}%
\begin{pgfscope}%
\pgfsys@transformshift{1.965979in}{2.047961in}%
\pgfsys@useobject{currentmarker}{}%
\end{pgfscope}%
\begin{pgfscope}%
\pgfsys@transformshift{1.975272in}{2.029116in}%
\pgfsys@useobject{currentmarker}{}%
\end{pgfscope}%
\begin{pgfscope}%
\pgfsys@transformshift{1.984318in}{2.025141in}%
\pgfsys@useobject{currentmarker}{}%
\end{pgfscope}%
\begin{pgfscope}%
\pgfsys@transformshift{1.993128in}{2.023049in}%
\pgfsys@useobject{currentmarker}{}%
\end{pgfscope}%
\begin{pgfscope}%
\pgfsys@transformshift{2.001715in}{2.014879in}%
\pgfsys@useobject{currentmarker}{}%
\end{pgfscope}%
\begin{pgfscope}%
\pgfsys@transformshift{2.010090in}{2.009596in}%
\pgfsys@useobject{currentmarker}{}%
\end{pgfscope}%
\begin{pgfscope}%
\pgfsys@transformshift{2.018264in}{1.992857in}%
\pgfsys@useobject{currentmarker}{}%
\end{pgfscope}%
\begin{pgfscope}%
\pgfsys@transformshift{2.026245in}{1.981543in}%
\pgfsys@useobject{currentmarker}{}%
\end{pgfscope}%
\begin{pgfscope}%
\pgfsys@transformshift{2.034042in}{1.980208in}%
\pgfsys@useobject{currentmarker}{}%
\end{pgfscope}%
\begin{pgfscope}%
\pgfsys@transformshift{2.041665in}{1.982414in}%
\pgfsys@useobject{currentmarker}{}%
\end{pgfscope}%
\begin{pgfscope}%
\pgfsys@transformshift{2.049119in}{1.991603in}%
\pgfsys@useobject{currentmarker}{}%
\end{pgfscope}%
\begin{pgfscope}%
\pgfsys@transformshift{2.056414in}{1.955989in}%
\pgfsys@useobject{currentmarker}{}%
\end{pgfscope}%
\begin{pgfscope}%
\pgfsys@transformshift{2.063555in}{1.949986in}%
\pgfsys@useobject{currentmarker}{}%
\end{pgfscope}%
\begin{pgfscope}%
\pgfsys@transformshift{2.070548in}{1.954281in}%
\pgfsys@useobject{currentmarker}{}%
\end{pgfscope}%
\begin{pgfscope}%
\pgfsys@transformshift{2.077401in}{1.951415in}%
\pgfsys@useobject{currentmarker}{}%
\end{pgfscope}%
\begin{pgfscope}%
\pgfsys@transformshift{2.084117in}{1.939417in}%
\pgfsys@useobject{currentmarker}{}%
\end{pgfscope}%
\begin{pgfscope}%
\pgfsys@transformshift{2.093949in}{1.950091in}%
\pgfsys@useobject{currentmarker}{}%
\end{pgfscope}%
\begin{pgfscope}%
\pgfsys@transformshift{2.103504in}{1.936985in}%
\pgfsys@useobject{currentmarker}{}%
\end{pgfscope}%
\begin{pgfscope}%
\pgfsys@transformshift{2.109728in}{1.928830in}%
\pgfsys@useobject{currentmarker}{}%
\end{pgfscope}%
\begin{pgfscope}%
\pgfsys@transformshift{2.115839in}{1.912373in}%
\pgfsys@useobject{currentmarker}{}%
\end{pgfscope}%
\begin{pgfscope}%
\pgfsys@transformshift{2.124805in}{1.910191in}%
\pgfsys@useobject{currentmarker}{}%
\end{pgfscope}%
\begin{pgfscope}%
\pgfsys@transformshift{2.133540in}{1.919928in}%
\pgfsys@useobject{currentmarker}{}%
\end{pgfscope}%
\begin{pgfscope}%
\pgfsys@transformshift{2.139240in}{1.906125in}%
\pgfsys@useobject{currentmarker}{}%
\end{pgfscope}%
\begin{pgfscope}%
\pgfsys@transformshift{2.147615in}{1.915295in}%
\pgfsys@useobject{currentmarker}{}%
\end{pgfscope}%
\begin{pgfscope}%
\pgfsys@transformshift{2.155789in}{1.912846in}%
\pgfsys@useobject{currentmarker}{}%
\end{pgfscope}%
\begin{pgfscope}%
\pgfsys@transformshift{2.163770in}{1.915975in}%
\pgfsys@useobject{currentmarker}{}%
\end{pgfscope}%
\begin{pgfscope}%
\pgfsys@transformshift{2.171567in}{1.907866in}%
\pgfsys@useobject{currentmarker}{}%
\end{pgfscope}%
\begin{pgfscope}%
\pgfsys@transformshift{2.179190in}{1.890533in}%
\pgfsys@useobject{currentmarker}{}%
\end{pgfscope}%
\begin{pgfscope}%
\pgfsys@transformshift{2.186644in}{1.877368in}%
\pgfsys@useobject{currentmarker}{}%
\end{pgfscope}%
\begin{pgfscope}%
\pgfsys@transformshift{2.193939in}{1.892414in}%
\pgfsys@useobject{currentmarker}{}%
\end{pgfscope}%
\begin{pgfscope}%
\pgfsys@transformshift{2.203427in}{1.896018in}%
\pgfsys@useobject{currentmarker}{}%
\end{pgfscope}%
\begin{pgfscope}%
\pgfsys@transformshift{2.210373in}{1.880621in}%
\pgfsys@useobject{currentmarker}{}%
\end{pgfscope}%
\begin{pgfscope}%
\pgfsys@transformshift{2.217179in}{1.862126in}%
\pgfsys@useobject{currentmarker}{}%
\end{pgfscope}%
\begin{pgfscope}%
\pgfsys@transformshift{2.226047in}{1.862298in}%
\pgfsys@useobject{currentmarker}{}%
\end{pgfscope}%
\begin{pgfscope}%
\pgfsys@transformshift{2.234689in}{1.850041in}%
\pgfsys@useobject{currentmarker}{}%
\end{pgfscope}%
\begin{pgfscope}%
\pgfsys@transformshift{2.243116in}{1.836644in}%
\pgfsys@useobject{currentmarker}{}%
\end{pgfscope}%
\begin{pgfscope}%
\pgfsys@transformshift{2.251339in}{1.839732in}%
\pgfsys@useobject{currentmarker}{}%
\end{pgfscope}%
\begin{pgfscope}%
\pgfsys@transformshift{2.259368in}{1.820793in}%
\pgfsys@useobject{currentmarker}{}%
\end{pgfscope}%
\begin{pgfscope}%
\pgfsys@transformshift{2.267210in}{1.826608in}%
\pgfsys@useobject{currentmarker}{}%
\end{pgfscope}%
\begin{pgfscope}%
\pgfsys@transformshift{2.274876in}{1.817028in}%
\pgfsys@useobject{currentmarker}{}%
\end{pgfscope}%
\begin{pgfscope}%
\pgfsys@transformshift{2.282372in}{1.816160in}%
\pgfsys@useobject{currentmarker}{}%
\end{pgfscope}%
\begin{pgfscope}%
\pgfsys@transformshift{2.289705in}{1.795115in}%
\pgfsys@useobject{currentmarker}{}%
\end{pgfscope}%
\begin{pgfscope}%
\pgfsys@transformshift{2.296884in}{1.784473in}%
\pgfsys@useobject{currentmarker}{}%
\end{pgfscope}%
\begin{pgfscope}%
\pgfsys@transformshift{2.303914in}{1.799479in}%
\pgfsys@useobject{currentmarker}{}%
\end{pgfscope}%
\begin{pgfscope}%
\pgfsys@transformshift{2.312501in}{1.783885in}%
\pgfsys@useobject{currentmarker}{}%
\end{pgfscope}%
\begin{pgfscope}%
\pgfsys@transformshift{2.320876in}{1.782203in}%
\pgfsys@useobject{currentmarker}{}%
\end{pgfscope}%
\begin{pgfscope}%
\pgfsys@transformshift{2.327431in}{1.766910in}%
\pgfsys@useobject{currentmarker}{}%
\end{pgfscope}%
\begin{pgfscope}%
\pgfsys@transformshift{2.335450in}{1.763729in}%
\pgfsys@useobject{currentmarker}{}%
\end{pgfscope}%
\begin{pgfscope}%
\pgfsys@transformshift{2.343283in}{1.759899in}%
\pgfsys@useobject{currentmarker}{}%
\end{pgfscope}%
\begin{pgfscope}%
\pgfsys@transformshift{2.350940in}{1.750471in}%
\pgfsys@useobject{currentmarker}{}%
\end{pgfscope}%
\begin{pgfscope}%
\pgfsys@transformshift{2.359905in}{1.747630in}%
\pgfsys@useobject{currentmarker}{}%
\end{pgfscope}%
\begin{pgfscope}%
\pgfsys@transformshift{2.367200in}{1.759412in}%
\pgfsys@useobject{currentmarker}{}%
\end{pgfscope}%
\begin{pgfscope}%
\pgfsys@transformshift{2.374341in}{1.739285in}%
\pgfsys@useobject{currentmarker}{}%
\end{pgfscope}%
\begin{pgfscope}%
\pgfsys@transformshift{2.382716in}{1.729515in}%
\pgfsys@useobject{currentmarker}{}%
\end{pgfscope}%
\begin{pgfscope}%
\pgfsys@transformshift{2.390889in}{1.718967in}%
\pgfsys@useobject{currentmarker}{}%
\end{pgfscope}%
\begin{pgfscope}%
\pgfsys@transformshift{2.398870in}{1.718020in}%
\pgfsys@useobject{currentmarker}{}%
\end{pgfscope}%
\begin{pgfscope}%
\pgfsys@transformshift{2.406668in}{1.709188in}%
\pgfsys@useobject{currentmarker}{}%
\end{pgfscope}%
\begin{pgfscope}%
\pgfsys@transformshift{2.414290in}{1.720707in}%
\pgfsys@useobject{currentmarker}{}%
\end{pgfscope}%
\begin{pgfscope}%
\pgfsys@transformshift{2.421745in}{1.695925in}%
\pgfsys@useobject{currentmarker}{}%
\end{pgfscope}%
\begin{pgfscope}%
\pgfsys@transformshift{2.430240in}{1.677145in}%
\pgfsys@useobject{currentmarker}{}%
\end{pgfscope}%
\begin{pgfscope}%
\pgfsys@transformshift{2.438527in}{1.674665in}%
\pgfsys@useobject{currentmarker}{}%
\end{pgfscope}%
\begin{pgfscope}%
\pgfsys@transformshift{2.445473in}{1.671102in}%
\pgfsys@useobject{currentmarker}{}%
\end{pgfscope}%
\begin{pgfscope}%
\pgfsys@transformshift{2.453401in}{1.685078in}%
\pgfsys@useobject{currentmarker}{}%
\end{pgfscope}%
\begin{pgfscope}%
\pgfsys@transformshift{2.461148in}{1.652248in}%
\pgfsys@useobject{currentmarker}{}%
\end{pgfscope}%
\begin{pgfscope}%
\pgfsys@transformshift{2.468721in}{1.661166in}%
\pgfsys@useobject{currentmarker}{}%
\end{pgfscope}%
\begin{pgfscope}%
\pgfsys@transformshift{2.477175in}{1.654755in}%
\pgfsys@useobject{currentmarker}{}%
\end{pgfscope}%
\begin{pgfscope}%
\pgfsys@transformshift{2.485423in}{1.647235in}%
\pgfsys@useobject{currentmarker}{}%
\end{pgfscope}%
\begin{pgfscope}%
\pgfsys@transformshift{2.493475in}{1.627739in}%
\pgfsys@useobject{currentmarker}{}%
\end{pgfscope}%
\begin{pgfscope}%
\pgfsys@transformshift{2.501340in}{1.629400in}%
\pgfsys@useobject{currentmarker}{}%
\end{pgfscope}%
\begin{pgfscope}%
\pgfsys@transformshift{2.509027in}{1.615493in}%
\pgfsys@useobject{currentmarker}{}%
\end{pgfscope}%
\begin{pgfscope}%
\pgfsys@transformshift{2.516544in}{1.602405in}%
\pgfsys@useobject{currentmarker}{}%
\end{pgfscope}%
\begin{pgfscope}%
\pgfsys@transformshift{2.523898in}{1.596418in}%
\pgfsys@useobject{currentmarker}{}%
\end{pgfscope}%
\begin{pgfscope}%
\pgfsys@transformshift{2.531985in}{1.602197in}%
\pgfsys@useobject{currentmarker}{}%
\end{pgfscope}%
\begin{pgfscope}%
\pgfsys@transformshift{2.539883in}{1.600115in}%
\pgfsys@useobject{currentmarker}{}%
\end{pgfscope}%
\begin{pgfscope}%
\pgfsys@transformshift{2.547602in}{1.595470in}%
\pgfsys@useobject{currentmarker}{}%
\end{pgfscope}%
\begin{pgfscope}%
\pgfsys@transformshift{2.555149in}{1.575269in}%
\pgfsys@useobject{currentmarker}{}%
\end{pgfscope}%
\begin{pgfscope}%
\pgfsys@transformshift{2.562531in}{1.577498in}%
\pgfsys@useobject{currentmarker}{}%
\end{pgfscope}%
\begin{pgfscope}%
\pgfsys@transformshift{2.570550in}{1.572853in}%
\pgfsys@useobject{currentmarker}{}%
\end{pgfscope}%
\begin{pgfscope}%
\pgfsys@transformshift{2.578384in}{1.564076in}%
\pgfsys@useobject{currentmarker}{}%
\end{pgfscope}%
\begin{pgfscope}%
\pgfsys@transformshift{2.586040in}{1.562440in}%
\pgfsys@useobject{currentmarker}{}%
\end{pgfscope}%
\begin{pgfscope}%
\pgfsys@transformshift{2.594268in}{1.559649in}%
\pgfsys@useobject{currentmarker}{}%
\end{pgfscope}%
\begin{pgfscope}%
\pgfsys@transformshift{2.602300in}{1.557994in}%
\pgfsys@useobject{currentmarker}{}%
\end{pgfscope}%
\begin{pgfscope}%
\pgfsys@transformshift{2.610147in}{1.543039in}%
\pgfsys@useobject{currentmarker}{}%
\end{pgfscope}%
\begin{pgfscope}%
\pgfsys@transformshift{2.617816in}{1.537611in}%
\pgfsys@useobject{currentmarker}{}%
\end{pgfscope}%
\begin{pgfscope}%
\pgfsys@transformshift{2.625316in}{1.526578in}%
\pgfsys@useobject{currentmarker}{}%
\end{pgfscope}%
\begin{pgfscope}%
\pgfsys@transformshift{2.633313in}{1.526751in}%
\pgfsys@useobject{currentmarker}{}%
\end{pgfscope}%
\begin{pgfscope}%
\pgfsys@transformshift{2.641125in}{1.517120in}%
\pgfsys@useobject{currentmarker}{}%
\end{pgfscope}%
\begin{pgfscope}%
\pgfsys@transformshift{2.648762in}{1.520533in}%
\pgfsys@useobject{currentmarker}{}%
\end{pgfscope}%
\begin{pgfscope}%
\pgfsys@transformshift{2.656845in}{1.495295in}%
\pgfsys@useobject{currentmarker}{}%
\end{pgfscope}%
\begin{pgfscope}%
\pgfsys@transformshift{2.664741in}{1.494986in}%
\pgfsys@useobject{currentmarker}{}%
\end{pgfscope}%
\begin{pgfscope}%
\pgfsys@transformshift{2.672456in}{1.493313in}%
\pgfsys@useobject{currentmarker}{}%
\end{pgfscope}%
\begin{pgfscope}%
\pgfsys@transformshift{2.680574in}{1.488856in}%
\pgfsys@useobject{currentmarker}{}%
\end{pgfscope}%
\begin{pgfscope}%
\pgfsys@transformshift{2.688502in}{1.470535in}%
\pgfsys@useobject{currentmarker}{}%
\end{pgfscope}%
\begin{pgfscope}%
\pgfsys@transformshift{2.696248in}{1.469258in}%
\pgfsys@useobject{currentmarker}{}%
\end{pgfscope}%
\begin{pgfscope}%
\pgfsys@transformshift{2.703822in}{1.462565in}%
\pgfsys@useobject{currentmarker}{}%
\end{pgfscope}%
\begin{pgfscope}%
\pgfsys@transformshift{2.711753in}{1.466159in}%
\pgfsys@useobject{currentmarker}{}%
\end{pgfscope}%
\begin{pgfscope}%
\pgfsys@transformshift{2.719503in}{1.454392in}%
\pgfsys@useobject{currentmarker}{}%
\end{pgfscope}%
\begin{pgfscope}%
\pgfsys@transformshift{2.727080in}{1.453268in}%
\pgfsys@useobject{currentmarker}{}%
\end{pgfscope}%
\begin{pgfscope}%
\pgfsys@transformshift{2.734980in}{1.441298in}%
\pgfsys@useobject{currentmarker}{}%
\end{pgfscope}%
\begin{pgfscope}%
\pgfsys@transformshift{2.743177in}{1.443303in}%
\pgfsys@useobject{currentmarker}{}%
\end{pgfscope}%
\begin{pgfscope}%
\pgfsys@transformshift{2.751180in}{1.426114in}%
\pgfsys@useobject{currentmarker}{}%
\end{pgfscope}%
\begin{pgfscope}%
\pgfsys@transformshift{2.758998in}{1.415950in}%
\pgfsys@useobject{currentmarker}{}%
\end{pgfscope}%
\begin{pgfscope}%
\pgfsys@transformshift{2.766641in}{1.413140in}%
\pgfsys@useobject{currentmarker}{}%
\end{pgfscope}%
\begin{pgfscope}%
\pgfsys@transformshift{2.774115in}{1.417712in}%
\pgfsys@useobject{currentmarker}{}%
\end{pgfscope}%
\begin{pgfscope}%
\pgfsys@transformshift{2.782278in}{1.395287in}%
\pgfsys@useobject{currentmarker}{}%
\end{pgfscope}%
\begin{pgfscope}%
\pgfsys@transformshift{2.790249in}{1.398763in}%
\pgfsys@useobject{currentmarker}{}%
\end{pgfscope}%
\begin{pgfscope}%
\pgfsys@transformshift{2.798037in}{1.382873in}%
\pgfsys@useobject{currentmarker}{}%
\end{pgfscope}%
\begin{pgfscope}%
\pgfsys@transformshift{2.806047in}{1.382456in}%
\pgfsys@useobject{currentmarker}{}%
\end{pgfscope}%
\begin{pgfscope}%
\pgfsys@transformshift{2.813871in}{1.375556in}%
\pgfsys@useobject{currentmarker}{}%
\end{pgfscope}%
\begin{pgfscope}%
\pgfsys@transformshift{2.821519in}{1.376608in}%
\pgfsys@useobject{currentmarker}{}%
\end{pgfscope}%
\begin{pgfscope}%
\pgfsys@transformshift{2.829368in}{1.364229in}%
\pgfsys@useobject{currentmarker}{}%
\end{pgfscope}%
\begin{pgfscope}%
\pgfsys@transformshift{2.837040in}{1.351091in}%
\pgfsys@useobject{currentmarker}{}%
\end{pgfscope}%
\begin{pgfscope}%
\pgfsys@transformshift{2.844895in}{1.351069in}%
\pgfsys@useobject{currentmarker}{}%
\end{pgfscope}%
\begin{pgfscope}%
\pgfsys@transformshift{2.852917in}{1.346645in}%
\pgfsys@useobject{currentmarker}{}%
\end{pgfscope}%
\begin{pgfscope}%
\pgfsys@transformshift{2.860754in}{1.335580in}%
\pgfsys@useobject{currentmarker}{}%
\end{pgfscope}%
\begin{pgfscope}%
\pgfsys@transformshift{2.868413in}{1.324545in}%
\pgfsys@useobject{currentmarker}{}%
\end{pgfscope}%
\begin{pgfscope}%
\pgfsys@transformshift{2.876226in}{1.322872in}%
\pgfsys@useobject{currentmarker}{}%
\end{pgfscope}%
\begin{pgfscope}%
\pgfsys@transformshift{2.884177in}{1.315381in}%
\pgfsys@useobject{currentmarker}{}%
\end{pgfscope}%
\begin{pgfscope}%
\pgfsys@transformshift{2.891946in}{1.310877in}%
\pgfsys@useobject{currentmarker}{}%
\end{pgfscope}%
\begin{pgfscope}%
\pgfsys@transformshift{2.899841in}{1.311630in}%
\pgfsys@useobject{currentmarker}{}%
\end{pgfscope}%
\begin{pgfscope}%
\pgfsys@transformshift{2.907557in}{1.298928in}%
\pgfsys@useobject{currentmarker}{}%
\end{pgfscope}%
\begin{pgfscope}%
\pgfsys@transformshift{2.915388in}{1.293082in}%
\pgfsys@useobject{currentmarker}{}%
\end{pgfscope}%
\begin{pgfscope}%
\pgfsys@transformshift{2.923322in}{1.290849in}%
\pgfsys@useobject{currentmarker}{}%
\end{pgfscope}%
\begin{pgfscope}%
\pgfsys@transformshift{2.931075in}{1.281180in}%
\pgfsys@useobject{currentmarker}{}%
\end{pgfscope}%
\begin{pgfscope}%
\pgfsys@transformshift{2.938923in}{1.275929in}%
\pgfsys@useobject{currentmarker}{}%
\end{pgfscope}%
\begin{pgfscope}%
\pgfsys@transformshift{2.946854in}{1.266086in}%
\pgfsys@useobject{currentmarker}{}%
\end{pgfscope}%
\begin{pgfscope}%
\pgfsys@transformshift{2.954604in}{1.257792in}%
\pgfsys@useobject{currentmarker}{}%
\end{pgfscope}%
\begin{pgfscope}%
\pgfsys@transformshift{2.962430in}{1.255149in}%
\pgfsys@useobject{currentmarker}{}%
\end{pgfscope}%
\begin{pgfscope}%
\pgfsys@transformshift{2.970324in}{1.250399in}%
\pgfsys@useobject{currentmarker}{}%
\end{pgfscope}%
\begin{pgfscope}%
\pgfsys@transformshift{2.978039in}{1.246859in}%
\pgfsys@useobject{currentmarker}{}%
\end{pgfscope}%
\begin{pgfscope}%
\pgfsys@transformshift{2.985815in}{1.236412in}%
\pgfsys@useobject{currentmarker}{}%
\end{pgfscope}%
\begin{pgfscope}%
\pgfsys@transformshift{2.993644in}{1.236908in}%
\pgfsys@useobject{currentmarker}{}%
\end{pgfscope}%
\begin{pgfscope}%
\pgfsys@transformshift{3.001519in}{1.215458in}%
\pgfsys@useobject{currentmarker}{}%
\end{pgfscope}%
\begin{pgfscope}%
\pgfsys@transformshift{3.009433in}{1.216849in}%
\pgfsys@useobject{currentmarker}{}%
\end{pgfscope}%
\begin{pgfscope}%
\pgfsys@transformshift{3.017166in}{1.204607in}%
\pgfsys@useobject{currentmarker}{}%
\end{pgfscope}%
\begin{pgfscope}%
\pgfsys@transformshift{3.024935in}{1.204762in}%
\pgfsys@useobject{currentmarker}{}%
\end{pgfscope}%
\begin{pgfscope}%
\pgfsys@transformshift{3.032732in}{1.199263in}%
\pgfsys@useobject{currentmarker}{}%
\end{pgfscope}%
\begin{pgfscope}%
\pgfsys@transformshift{3.040553in}{1.195386in}%
\pgfsys@useobject{currentmarker}{}%
\end{pgfscope}%
\begin{pgfscope}%
\pgfsys@transformshift{3.048391in}{1.192342in}%
\pgfsys@useobject{currentmarker}{}%
\end{pgfscope}%
\begin{pgfscope}%
\pgfsys@transformshift{3.056241in}{1.181419in}%
\pgfsys@useobject{currentmarker}{}%
\end{pgfscope}%
\begin{pgfscope}%
\pgfsys@transformshift{3.064099in}{1.173918in}%
\pgfsys@useobject{currentmarker}{}%
\end{pgfscope}%
\begin{pgfscope}%
\pgfsys@transformshift{3.071960in}{1.169274in}%
\pgfsys@useobject{currentmarker}{}%
\end{pgfscope}%
\begin{pgfscope}%
\pgfsys@transformshift{3.079819in}{1.156078in}%
\pgfsys@useobject{currentmarker}{}%
\end{pgfscope}%
\begin{pgfscope}%
\pgfsys@transformshift{3.087673in}{1.150215in}%
\pgfsys@useobject{currentmarker}{}%
\end{pgfscope}%
\begin{pgfscope}%
\pgfsys@transformshift{3.095517in}{1.138747in}%
\pgfsys@useobject{currentmarker}{}%
\end{pgfscope}%
\begin{pgfscope}%
\pgfsys@transformshift{3.103349in}{1.140384in}%
\pgfsys@useobject{currentmarker}{}%
\end{pgfscope}%
\begin{pgfscope}%
\pgfsys@transformshift{3.111166in}{1.129719in}%
\pgfsys@useobject{currentmarker}{}%
\end{pgfscope}%
\begin{pgfscope}%
\pgfsys@transformshift{3.118963in}{1.130123in}%
\pgfsys@useobject{currentmarker}{}%
\end{pgfscope}%
\begin{pgfscope}%
\pgfsys@transformshift{3.126739in}{1.124494in}%
\pgfsys@useobject{currentmarker}{}%
\end{pgfscope}%
\begin{pgfscope}%
\pgfsys@transformshift{3.134491in}{1.115415in}%
\pgfsys@useobject{currentmarker}{}%
\end{pgfscope}%
\begin{pgfscope}%
\pgfsys@transformshift{3.142364in}{1.103524in}%
\pgfsys@useobject{currentmarker}{}%
\end{pgfscope}%
\begin{pgfscope}%
\pgfsys@transformshift{3.150202in}{1.096645in}%
\pgfsys@useobject{currentmarker}{}%
\end{pgfscope}%
\begin{pgfscope}%
\pgfsys@transformshift{3.158002in}{1.089515in}%
\pgfsys@useobject{currentmarker}{}%
\end{pgfscope}%
\begin{pgfscope}%
\pgfsys@transformshift{3.165902in}{1.085629in}%
\pgfsys@useobject{currentmarker}{}%
\end{pgfscope}%
\begin{pgfscope}%
\pgfsys@transformshift{3.173756in}{1.079553in}%
\pgfsys@useobject{currentmarker}{}%
\end{pgfscope}%
\begin{pgfscope}%
\pgfsys@transformshift{3.181562in}{1.073667in}%
\pgfsys@useobject{currentmarker}{}%
\end{pgfscope}%
\begin{pgfscope}%
\pgfsys@transformshift{3.189449in}{1.068513in}%
\pgfsys@useobject{currentmarker}{}%
\end{pgfscope}%
\begin{pgfscope}%
\pgfsys@transformshift{3.197281in}{1.064873in}%
\pgfsys@useobject{currentmarker}{}%
\end{pgfscope}%
\begin{pgfscope}%
\pgfsys@transformshift{3.205059in}{1.058285in}%
\pgfsys@useobject{currentmarker}{}%
\end{pgfscope}%
\begin{pgfscope}%
\pgfsys@transformshift{3.212901in}{1.055008in}%
\pgfsys@useobject{currentmarker}{}%
\end{pgfscope}%
\begin{pgfscope}%
\pgfsys@transformshift{3.220799in}{1.047326in}%
\pgfsys@useobject{currentmarker}{}%
\end{pgfscope}%
\begin{pgfscope}%
\pgfsys@transformshift{3.228631in}{1.043061in}%
\pgfsys@useobject{currentmarker}{}%
\end{pgfscope}%
\begin{pgfscope}%
\pgfsys@transformshift{3.236397in}{1.033133in}%
\pgfsys@useobject{currentmarker}{}%
\end{pgfscope}%
\begin{pgfscope}%
\pgfsys@transformshift{3.244207in}{1.022663in}%
\pgfsys@useobject{currentmarker}{}%
\end{pgfscope}%
\begin{pgfscope}%
\pgfsys@transformshift{3.252054in}{1.017629in}%
\pgfsys@useobject{currentmarker}{}%
\end{pgfscope}%
\begin{pgfscope}%
\pgfsys@transformshift{3.259932in}{1.015853in}%
\pgfsys@useobject{currentmarker}{}%
\end{pgfscope}%
\begin{pgfscope}%
\pgfsys@transformshift{3.267732in}{1.003607in}%
\pgfsys@useobject{currentmarker}{}%
\end{pgfscope}%
\begin{pgfscope}%
\pgfsys@transformshift{3.275554in}{0.996605in}%
\pgfsys@useobject{currentmarker}{}%
\end{pgfscope}%
\begin{pgfscope}%
\pgfsys@transformshift{3.283395in}{0.999848in}%
\pgfsys@useobject{currentmarker}{}%
\end{pgfscope}%
\begin{pgfscope}%
\pgfsys@transformshift{3.291153in}{0.987825in}%
\pgfsys@useobject{currentmarker}{}%
\end{pgfscope}%
\begin{pgfscope}%
\pgfsys@transformshift{3.299015in}{0.980948in}%
\pgfsys@useobject{currentmarker}{}%
\end{pgfscope}%
\begin{pgfscope}%
\pgfsys@transformshift{3.306880in}{0.972393in}%
\pgfsys@useobject{currentmarker}{}%
\end{pgfscope}%
\begin{pgfscope}%
\pgfsys@transformshift{3.314655in}{0.971079in}%
\pgfsys@useobject{currentmarker}{}%
\end{pgfscope}%
\begin{pgfscope}%
\pgfsys@transformshift{3.322514in}{0.967335in}%
\pgfsys@useobject{currentmarker}{}%
\end{pgfscope}%
\begin{pgfscope}%
\pgfsys@transformshift{3.330365in}{0.959972in}%
\pgfsys@useobject{currentmarker}{}%
\end{pgfscope}%
\begin{pgfscope}%
\pgfsys@transformshift{3.338203in}{0.953885in}%
\pgfsys@useobject{currentmarker}{}%
\end{pgfscope}%
\begin{pgfscope}%
\pgfsys@transformshift{3.346025in}{0.942309in}%
\pgfsys@useobject{currentmarker}{}%
\end{pgfscope}%
\begin{pgfscope}%
\pgfsys@transformshift{3.353828in}{0.935709in}%
\pgfsys@useobject{currentmarker}{}%
\end{pgfscope}%
\begin{pgfscope}%
\pgfsys@transformshift{3.361686in}{0.926802in}%
\pgfsys@useobject{currentmarker}{}%
\end{pgfscope}%
\begin{pgfscope}%
\pgfsys@transformshift{3.369517in}{0.930654in}%
\pgfsys@useobject{currentmarker}{}%
\end{pgfscope}%
\begin{pgfscope}%
\pgfsys@transformshift{3.377318in}{0.924878in}%
\pgfsys@useobject{currentmarker}{}%
\end{pgfscope}%
\begin{pgfscope}%
\pgfsys@transformshift{3.385159in}{0.914307in}%
\pgfsys@useobject{currentmarker}{}%
\end{pgfscope}%
\begin{pgfscope}%
\pgfsys@transformshift{3.393033in}{0.905580in}%
\pgfsys@useobject{currentmarker}{}%
\end{pgfscope}%
\begin{pgfscope}%
\pgfsys@transformshift{3.400865in}{0.899063in}%
\pgfsys@useobject{currentmarker}{}%
\end{pgfscope}%
\begin{pgfscope}%
\pgfsys@transformshift{3.408655in}{0.899178in}%
\pgfsys@useobject{currentmarker}{}%
\end{pgfscope}%
\begin{pgfscope}%
\pgfsys@transformshift{3.416466in}{0.890228in}%
\pgfsys@useobject{currentmarker}{}%
\end{pgfscope}%
\begin{pgfscope}%
\pgfsys@transformshift{3.424294in}{0.877738in}%
\pgfsys@useobject{currentmarker}{}%
\end{pgfscope}%
\begin{pgfscope}%
\pgfsys@transformshift{3.432132in}{0.878056in}%
\pgfsys@useobject{currentmarker}{}%
\end{pgfscope}%
\begin{pgfscope}%
\pgfsys@transformshift{3.439976in}{0.874204in}%
\pgfsys@useobject{currentmarker}{}%
\end{pgfscope}%
\begin{pgfscope}%
\pgfsys@transformshift{3.447823in}{0.863637in}%
\pgfsys@useobject{currentmarker}{}%
\end{pgfscope}%
\begin{pgfscope}%
\pgfsys@transformshift{3.455666in}{0.855131in}%
\pgfsys@useobject{currentmarker}{}%
\end{pgfscope}%
\begin{pgfscope}%
\pgfsys@transformshift{3.463447in}{0.857028in}%
\pgfsys@useobject{currentmarker}{}%
\end{pgfscope}%
\begin{pgfscope}%
\pgfsys@transformshift{3.471275in}{0.849301in}%
\pgfsys@useobject{currentmarker}{}%
\end{pgfscope}%
\begin{pgfscope}%
\pgfsys@transformshift{3.479145in}{0.842214in}%
\pgfsys@useobject{currentmarker}{}%
\end{pgfscope}%
\begin{pgfscope}%
\pgfsys@transformshift{3.486942in}{0.839563in}%
\pgfsys@useobject{currentmarker}{}%
\end{pgfscope}%
\begin{pgfscope}%
\pgfsys@transformshift{3.494773in}{0.827045in}%
\pgfsys@useobject{currentmarker}{}%
\end{pgfscope}%
\begin{pgfscope}%
\pgfsys@transformshift{3.502629in}{0.825559in}%
\pgfsys@useobject{currentmarker}{}%
\end{pgfscope}%
\begin{pgfscope}%
\pgfsys@transformshift{3.510457in}{0.820198in}%
\pgfsys@useobject{currentmarker}{}%
\end{pgfscope}%
\begin{pgfscope}%
\pgfsys@transformshift{3.518253in}{0.815070in}%
\pgfsys@useobject{currentmarker}{}%
\end{pgfscope}%
\begin{pgfscope}%
\pgfsys@transformshift{3.526064in}{0.806851in}%
\pgfsys@useobject{currentmarker}{}%
\end{pgfscope}%
\begin{pgfscope}%
\pgfsys@transformshift{3.533930in}{0.802045in}%
\pgfsys@useobject{currentmarker}{}%
\end{pgfscope}%
\begin{pgfscope}%
\pgfsys@transformshift{3.541754in}{0.796165in}%
\pgfsys@useobject{currentmarker}{}%
\end{pgfscope}%
\begin{pgfscope}%
\pgfsys@transformshift{3.549578in}{0.795657in}%
\pgfsys@useobject{currentmarker}{}%
\end{pgfscope}%
\begin{pgfscope}%
\pgfsys@transformshift{3.557399in}{0.782476in}%
\pgfsys@useobject{currentmarker}{}%
\end{pgfscope}%
\begin{pgfscope}%
\pgfsys@transformshift{3.565212in}{0.778372in}%
\pgfsys@useobject{currentmarker}{}%
\end{pgfscope}%
\begin{pgfscope}%
\pgfsys@transformshift{3.573056in}{0.773704in}%
\pgfsys@useobject{currentmarker}{}%
\end{pgfscope}%
\begin{pgfscope}%
\pgfsys@transformshift{3.580883in}{0.769087in}%
\pgfsys@useobject{currentmarker}{}%
\end{pgfscope}%
\begin{pgfscope}%
\pgfsys@transformshift{3.588731in}{0.759799in}%
\pgfsys@useobject{currentmarker}{}%
\end{pgfscope}%
\begin{pgfscope}%
\pgfsys@transformshift{3.596556in}{0.761066in}%
\pgfsys@useobject{currentmarker}{}%
\end{pgfscope}%
\begin{pgfscope}%
\pgfsys@transformshift{3.604354in}{0.749908in}%
\pgfsys@useobject{currentmarker}{}%
\end{pgfscope}%
\begin{pgfscope}%
\pgfsys@transformshift{3.612198in}{0.747835in}%
\pgfsys@useobject{currentmarker}{}%
\end{pgfscope}%
\begin{pgfscope}%
\pgfsys@transformshift{3.620044in}{0.738252in}%
\pgfsys@useobject{currentmarker}{}%
\end{pgfscope}%
\begin{pgfscope}%
\pgfsys@transformshift{3.627888in}{0.736554in}%
\pgfsys@useobject{currentmarker}{}%
\end{pgfscope}%
\begin{pgfscope}%
\pgfsys@transformshift{3.635726in}{0.734400in}%
\pgfsys@useobject{currentmarker}{}%
\end{pgfscope}%
\begin{pgfscope}%
\pgfsys@transformshift{3.643555in}{0.726742in}%
\pgfsys@useobject{currentmarker}{}%
\end{pgfscope}%
\begin{pgfscope}%
\pgfsys@transformshift{3.651371in}{0.721791in}%
\pgfsys@useobject{currentmarker}{}%
\end{pgfscope}%
\begin{pgfscope}%
\pgfsys@transformshift{3.659170in}{0.716195in}%
\pgfsys@useobject{currentmarker}{}%
\end{pgfscope}%
\begin{pgfscope}%
\pgfsys@transformshift{3.667014in}{0.709154in}%
\pgfsys@useobject{currentmarker}{}%
\end{pgfscope}%
\begin{pgfscope}%
\pgfsys@transformshift{3.674863in}{0.706190in}%
\pgfsys@useobject{currentmarker}{}%
\end{pgfscope}%
\begin{pgfscope}%
\pgfsys@transformshift{3.682684in}{0.704662in}%
\pgfsys@useobject{currentmarker}{}%
\end{pgfscope}%
\begin{pgfscope}%
\pgfsys@transformshift{3.690504in}{0.699259in}%
\pgfsys@useobject{currentmarker}{}%
\end{pgfscope}%
\begin{pgfscope}%
\pgfsys@transformshift{3.698319in}{0.697705in}%
\pgfsys@useobject{currentmarker}{}%
\end{pgfscope}%
\begin{pgfscope}%
\pgfsys@transformshift{3.706153in}{0.689456in}%
\pgfsys@useobject{currentmarker}{}%
\end{pgfscope}%
\begin{pgfscope}%
\pgfsys@transformshift{3.714000in}{0.685822in}%
\pgfsys@useobject{currentmarker}{}%
\end{pgfscope}%
\begin{pgfscope}%
\pgfsys@transformshift{3.721830in}{0.680301in}%
\pgfsys@useobject{currentmarker}{}%
\end{pgfscope}%
\begin{pgfscope}%
\pgfsys@transformshift{3.729665in}{0.680140in}%
\pgfsys@useobject{currentmarker}{}%
\end{pgfscope}%
\begin{pgfscope}%
\pgfsys@transformshift{3.737501in}{0.674910in}%
\pgfsys@useobject{currentmarker}{}%
\end{pgfscope}%
\begin{pgfscope}%
\pgfsys@transformshift{3.745309in}{0.668317in}%
\pgfsys@useobject{currentmarker}{}%
\end{pgfscope}%
\begin{pgfscope}%
\pgfsys@transformshift{3.753135in}{0.667080in}%
\pgfsys@useobject{currentmarker}{}%
\end{pgfscope}%
\begin{pgfscope}%
\pgfsys@transformshift{3.760975in}{0.663041in}%
\pgfsys@useobject{currentmarker}{}%
\end{pgfscope}%
\begin{pgfscope}%
\pgfsys@transformshift{3.768799in}{0.655285in}%
\pgfsys@useobject{currentmarker}{}%
\end{pgfscope}%
\begin{pgfscope}%
\pgfsys@transformshift{3.776628in}{0.657072in}%
\pgfsys@useobject{currentmarker}{}%
\end{pgfscope}%
\begin{pgfscope}%
\pgfsys@transformshift{3.784458in}{0.651905in}%
\pgfsys@useobject{currentmarker}{}%
\end{pgfscope}%
\begin{pgfscope}%
\pgfsys@transformshift{3.792284in}{0.649929in}%
\pgfsys@useobject{currentmarker}{}%
\end{pgfscope}%
\begin{pgfscope}%
\pgfsys@transformshift{3.800123in}{0.644844in}%
\pgfsys@useobject{currentmarker}{}%
\end{pgfscope}%
\begin{pgfscope}%
\pgfsys@transformshift{3.807950in}{0.646798in}%
\pgfsys@useobject{currentmarker}{}%
\end{pgfscope}%
\begin{pgfscope}%
\pgfsys@transformshift{3.815762in}{0.646343in}%
\pgfsys@useobject{currentmarker}{}%
\end{pgfscope}%
\begin{pgfscope}%
\pgfsys@transformshift{3.823596in}{0.639796in}%
\pgfsys@useobject{currentmarker}{}%
\end{pgfscope}%
\begin{pgfscope}%
\pgfsys@transformshift{3.831425in}{0.641296in}%
\pgfsys@useobject{currentmarker}{}%
\end{pgfscope}%
\begin{pgfscope}%
\pgfsys@transformshift{3.839267in}{0.635313in}%
\pgfsys@useobject{currentmarker}{}%
\end{pgfscope}%
\begin{pgfscope}%
\pgfsys@transformshift{3.847096in}{0.636731in}%
\pgfsys@useobject{currentmarker}{}%
\end{pgfscope}%
\begin{pgfscope}%
\pgfsys@transformshift{3.854911in}{0.635184in}%
\pgfsys@useobject{currentmarker}{}%
\end{pgfscope}%
\begin{pgfscope}%
\pgfsys@transformshift{3.862743in}{0.633221in}%
\pgfsys@useobject{currentmarker}{}%
\end{pgfscope}%
\begin{pgfscope}%
\pgfsys@transformshift{3.870569in}{0.632988in}%
\pgfsys@useobject{currentmarker}{}%
\end{pgfscope}%
\end{pgfscope}%
\begin{pgfscope}%
\pgfsetrectcap%
\pgfsetmiterjoin%
\pgfsetlinewidth{0.803000pt}%
\definecolor{currentstroke}{rgb}{0.000000,0.000000,0.000000}%
\pgfsetstrokecolor{currentstroke}%
\pgfsetdash{}{0pt}%
\pgfpathmoveto{\pgfqpoint{0.594525in}{0.417642in}}%
\pgfpathlineto{\pgfqpoint{0.594525in}{2.433919in}}%
\pgfusepath{stroke}%
\end{pgfscope}%
\begin{pgfscope}%
\pgfsetrectcap%
\pgfsetmiterjoin%
\pgfsetlinewidth{0.803000pt}%
\definecolor{currentstroke}{rgb}{0.000000,0.000000,0.000000}%
\pgfsetstrokecolor{currentstroke}%
\pgfsetdash{}{0pt}%
\pgfpathmoveto{\pgfqpoint{4.026572in}{0.417642in}}%
\pgfpathlineto{\pgfqpoint{4.026572in}{2.433919in}}%
\pgfusepath{stroke}%
\end{pgfscope}%
\begin{pgfscope}%
\pgfsetrectcap%
\pgfsetmiterjoin%
\pgfsetlinewidth{0.803000pt}%
\definecolor{currentstroke}{rgb}{0.000000,0.000000,0.000000}%
\pgfsetstrokecolor{currentstroke}%
\pgfsetdash{}{0pt}%
\pgfpathmoveto{\pgfqpoint{0.594525in}{0.417642in}}%
\pgfpathlineto{\pgfqpoint{4.026572in}{0.417642in}}%
\pgfusepath{stroke}%
\end{pgfscope}%
\begin{pgfscope}%
\pgfsetrectcap%
\pgfsetmiterjoin%
\pgfsetlinewidth{0.803000pt}%
\definecolor{currentstroke}{rgb}{0.000000,0.000000,0.000000}%
\pgfsetstrokecolor{currentstroke}%
\pgfsetdash{}{0pt}%
\pgfpathmoveto{\pgfqpoint{0.594525in}{2.433919in}}%
\pgfpathlineto{\pgfqpoint{4.026572in}{2.433919in}}%
\pgfusepath{stroke}%
\end{pgfscope}%
\begin{pgfscope}%
\pgfsetbuttcap%
\pgfsetmiterjoin%
\definecolor{currentfill}{rgb}{1.000000,1.000000,1.000000}%
\pgfsetfillcolor{currentfill}%
\pgfsetfillopacity{0.800000}%
\pgfsetlinewidth{1.003750pt}%
\definecolor{currentstroke}{rgb}{0.800000,0.800000,0.800000}%
\pgfsetstrokecolor{currentstroke}%
\pgfsetstrokeopacity{0.800000}%
\pgfsetdash{}{0pt}%
\pgfpathmoveto{\pgfqpoint{0.672303in}{0.473198in}}%
\pgfpathlineto{\pgfqpoint{1.512613in}{0.473198in}}%
\pgfpathquadraticcurveto{\pgfqpoint{1.534835in}{0.473198in}}{\pgfqpoint{1.534835in}{0.495420in}}%
\pgfpathlineto{\pgfqpoint{1.534835in}{0.948975in}}%
\pgfpathquadraticcurveto{\pgfqpoint{1.534835in}{0.971197in}}{\pgfqpoint{1.512613in}{0.971197in}}%
\pgfpathlineto{\pgfqpoint{0.672303in}{0.971197in}}%
\pgfpathquadraticcurveto{\pgfqpoint{0.650080in}{0.971197in}}{\pgfqpoint{0.650080in}{0.948975in}}%
\pgfpathlineto{\pgfqpoint{0.650080in}{0.495420in}}%
\pgfpathquadraticcurveto{\pgfqpoint{0.650080in}{0.473198in}}{\pgfqpoint{0.672303in}{0.473198in}}%
\pgfpathlineto{\pgfqpoint{0.672303in}{0.473198in}}%
\pgfpathclose%
\pgfusepath{stroke,fill}%
\end{pgfscope}%
\begin{pgfscope}%
\pgfsetbuttcap%
\pgfsetroundjoin%
\pgfsetlinewidth{1.505625pt}%
\definecolor{currentstroke}{rgb}{0.003922,0.450980,0.698039}%
\pgfsetstrokecolor{currentstroke}%
\pgfsetdash{{5.550000pt}{2.400000pt}}{0.000000pt}%
\pgfpathmoveto{\pgfqpoint{0.694525in}{0.887864in}}%
\pgfpathlineto{\pgfqpoint{0.805636in}{0.887864in}}%
\pgfpathlineto{\pgfqpoint{0.916747in}{0.887864in}}%
\pgfusepath{stroke}%
\end{pgfscope}%
\begin{pgfscope}%
\definecolor{textcolor}{rgb}{0.000000,0.000000,0.000000}%
\pgfsetstrokecolor{textcolor}%
\pgfsetfillcolor{textcolor}%
\pgftext[x=1.005636in,y=0.848975in,left,base]{\color{textcolor}\rmfamily\fontsize{8.000000}{9.600000}\selectfont \(\displaystyle \bar\tau_1=\qty{0.1}{\s}\)}%
\end{pgfscope}%
\begin{pgfscope}%
\pgfsetbuttcap%
\pgfsetroundjoin%
\pgfsetlinewidth{1.505625pt}%
\definecolor{currentstroke}{rgb}{0.007843,0.619608,0.450980}%
\pgfsetstrokecolor{currentstroke}%
\pgfsetdash{{5.550000pt}{2.400000pt}}{0.000000pt}%
\pgfpathmoveto{\pgfqpoint{0.694525in}{0.732975in}}%
\pgfpathlineto{\pgfqpoint{0.805636in}{0.732975in}}%
\pgfpathlineto{\pgfqpoint{0.916747in}{0.732975in}}%
\pgfusepath{stroke}%
\end{pgfscope}%
\begin{pgfscope}%
\definecolor{textcolor}{rgb}{0.000000,0.000000,0.000000}%
\pgfsetstrokecolor{textcolor}%
\pgfsetfillcolor{textcolor}%
\pgftext[x=1.005636in,y=0.694086in,left,base]{\color{textcolor}\rmfamily\fontsize{8.000000}{9.600000}\selectfont \(\displaystyle \bar\tau_1=\qty{1}{\s}\)}%
\end{pgfscope}%
\begin{pgfscope}%
\pgfsetbuttcap%
\pgfsetroundjoin%
\pgfsetlinewidth{1.505625pt}%
\definecolor{currentstroke}{rgb}{0.835294,0.368627,0.000000}%
\pgfsetstrokecolor{currentstroke}%
\pgfsetdash{{5.550000pt}{2.400000pt}}{0.000000pt}%
\pgfpathmoveto{\pgfqpoint{0.694525in}{0.578086in}}%
\pgfpathlineto{\pgfqpoint{0.805636in}{0.578086in}}%
\pgfpathlineto{\pgfqpoint{0.916747in}{0.578086in}}%
\pgfusepath{stroke}%
\end{pgfscope}%
\begin{pgfscope}%
\definecolor{textcolor}{rgb}{0.000000,0.000000,0.000000}%
\pgfsetstrokecolor{textcolor}%
\pgfsetfillcolor{textcolor}%
\pgftext[x=1.005636in,y=0.539197in,left,base]{\color{textcolor}\rmfamily\fontsize{8.000000}{9.600000}\selectfont \(\displaystyle \bar\tau_1=\qty{10}{\s}\)}%
\end{pgfscope}%
\end{pgfpicture}%
\makeatother%
\endgroup%
% data/simulations/sim_burst_noise.py
        } % scalebox
        \caption{Power spectral density}
        \label{fig:burst_noise_psd}
    \end{subfigure}
    \begin{subfigure}{0.8\linewidth}
        \centering
        \scalebox{1}{%
            %% Creator: Matplotlib, PGF backend
%%
%% To include the figure in your LaTeX document, write
%%   \input{<filename>.pgf}
%%
%% Make sure the required packages are loaded in your preamble
%%   \usepackage{pgf}
%%
%% Also ensure that all the required font packages are loaded; for instance,
%% the lmodern package is sometimes necessary when using math font.
%%   \usepackage{lmodern}
%%
%% Figures using additional raster images can only be included by \input if
%% they are in the same directory as the main LaTeX file. For loading figures
%% from other directories you can use the `import` package
%%   \usepackage{import}
%%
%% and then include the figures with
%%   \import{<path to file>}{<filename>.pgf}
%%
%% Matplotlib used the following preamble
%%   \usepackage{siunitx}
%%   \sisetup{per-mode = symbol}%
%%   \usepackage{fontspec}
%%   \makeatletter\@ifpackageloaded{underscore}{}{\usepackage[strings]{underscore}}\makeatother
%%
\begingroup%
\makeatletter%
\begin{pgfpicture}%
\pgfpathrectangle{\pgfpointorigin}{\pgfqpoint{4.068242in}{2.514312in}}%
\pgfusepath{use as bounding box, clip}%
\begin{pgfscope}%
\pgfsetbuttcap%
\pgfsetmiterjoin%
\definecolor{currentfill}{rgb}{1.000000,1.000000,1.000000}%
\pgfsetfillcolor{currentfill}%
\pgfsetlinewidth{0.000000pt}%
\definecolor{currentstroke}{rgb}{1.000000,1.000000,1.000000}%
\pgfsetstrokecolor{currentstroke}%
\pgfsetdash{}{0pt}%
\pgfpathmoveto{\pgfqpoint{0.000000in}{0.000000in}}%
\pgfpathlineto{\pgfqpoint{4.068242in}{0.000000in}}%
\pgfpathlineto{\pgfqpoint{4.068242in}{2.514312in}}%
\pgfpathlineto{\pgfqpoint{0.000000in}{2.514312in}}%
\pgfpathlineto{\pgfqpoint{0.000000in}{0.000000in}}%
\pgfpathclose%
\pgfusepath{fill}%
\end{pgfscope}%
\begin{pgfscope}%
\pgfsetbuttcap%
\pgfsetmiterjoin%
\definecolor{currentfill}{rgb}{1.000000,1.000000,1.000000}%
\pgfsetfillcolor{currentfill}%
\pgfsetlinewidth{0.000000pt}%
\definecolor{currentstroke}{rgb}{0.000000,0.000000,0.000000}%
\pgfsetstrokecolor{currentstroke}%
\pgfsetstrokeopacity{0.000000}%
\pgfsetdash{}{0pt}%
\pgfpathmoveto{\pgfqpoint{0.589510in}{0.417642in}}%
\pgfpathlineto{\pgfqpoint{4.026572in}{0.417642in}}%
\pgfpathlineto{\pgfqpoint{4.026572in}{2.472642in}}%
\pgfpathlineto{\pgfqpoint{0.589510in}{2.472642in}}%
\pgfpathlineto{\pgfqpoint{0.589510in}{0.417642in}}%
\pgfpathclose%
\pgfusepath{fill}%
\end{pgfscope}%
\begin{pgfscope}%
\pgfpathrectangle{\pgfqpoint{0.589510in}{0.417642in}}{\pgfqpoint{3.437062in}{2.055000in}}%
\pgfusepath{clip}%
\pgfsetrectcap%
\pgfsetroundjoin%
\pgfsetlinewidth{0.803000pt}%
\definecolor{currentstroke}{rgb}{0.450000,0.450000,0.450000}%
\pgfsetstrokecolor{currentstroke}%
\pgfsetdash{}{0pt}%
\pgfpathmoveto{\pgfqpoint{0.745740in}{0.417642in}}%
\pgfpathlineto{\pgfqpoint{0.745740in}{2.472642in}}%
\pgfusepath{stroke}%
\end{pgfscope}%
\begin{pgfscope}%
\pgfsetbuttcap%
\pgfsetroundjoin%
\definecolor{currentfill}{rgb}{0.000000,0.000000,0.000000}%
\pgfsetfillcolor{currentfill}%
\pgfsetlinewidth{0.803000pt}%
\definecolor{currentstroke}{rgb}{0.000000,0.000000,0.000000}%
\pgfsetstrokecolor{currentstroke}%
\pgfsetdash{}{0pt}%
\pgfsys@defobject{currentmarker}{\pgfqpoint{0.000000in}{-0.048611in}}{\pgfqpoint{0.000000in}{0.000000in}}{%
\pgfpathmoveto{\pgfqpoint{0.000000in}{0.000000in}}%
\pgfpathlineto{\pgfqpoint{0.000000in}{-0.048611in}}%
\pgfusepath{stroke,fill}%
}%
\begin{pgfscope}%
\pgfsys@transformshift{0.745740in}{0.417642in}%
\pgfsys@useobject{currentmarker}{}%
\end{pgfscope}%
\end{pgfscope}%
\begin{pgfscope}%
\definecolor{textcolor}{rgb}{0.000000,0.000000,0.000000}%
\pgfsetstrokecolor{textcolor}%
\pgfsetfillcolor{textcolor}%
\pgftext[x=0.745740in,y=0.320420in,,top]{\color{textcolor}\rmfamily\fontsize{8.000000}{9.600000}\selectfont \(\displaystyle {10^{-2}}\)}%
\end{pgfscope}%
\begin{pgfscope}%
\pgfpathrectangle{\pgfqpoint{0.589510in}{0.417642in}}{\pgfqpoint{3.437062in}{2.055000in}}%
\pgfusepath{clip}%
\pgfsetrectcap%
\pgfsetroundjoin%
\pgfsetlinewidth{0.803000pt}%
\definecolor{currentstroke}{rgb}{0.450000,0.450000,0.450000}%
\pgfsetstrokecolor{currentstroke}%
\pgfsetdash{}{0pt}%
\pgfpathmoveto{\pgfqpoint{1.526890in}{0.417642in}}%
\pgfpathlineto{\pgfqpoint{1.526890in}{2.472642in}}%
\pgfusepath{stroke}%
\end{pgfscope}%
\begin{pgfscope}%
\pgfsetbuttcap%
\pgfsetroundjoin%
\definecolor{currentfill}{rgb}{0.000000,0.000000,0.000000}%
\pgfsetfillcolor{currentfill}%
\pgfsetlinewidth{0.803000pt}%
\definecolor{currentstroke}{rgb}{0.000000,0.000000,0.000000}%
\pgfsetstrokecolor{currentstroke}%
\pgfsetdash{}{0pt}%
\pgfsys@defobject{currentmarker}{\pgfqpoint{0.000000in}{-0.048611in}}{\pgfqpoint{0.000000in}{0.000000in}}{%
\pgfpathmoveto{\pgfqpoint{0.000000in}{0.000000in}}%
\pgfpathlineto{\pgfqpoint{0.000000in}{-0.048611in}}%
\pgfusepath{stroke,fill}%
}%
\begin{pgfscope}%
\pgfsys@transformshift{1.526890in}{0.417642in}%
\pgfsys@useobject{currentmarker}{}%
\end{pgfscope}%
\end{pgfscope}%
\begin{pgfscope}%
\definecolor{textcolor}{rgb}{0.000000,0.000000,0.000000}%
\pgfsetstrokecolor{textcolor}%
\pgfsetfillcolor{textcolor}%
\pgftext[x=1.526890in,y=0.320420in,,top]{\color{textcolor}\rmfamily\fontsize{8.000000}{9.600000}\selectfont \(\displaystyle {10^{-1}}\)}%
\end{pgfscope}%
\begin{pgfscope}%
\pgfpathrectangle{\pgfqpoint{0.589510in}{0.417642in}}{\pgfqpoint{3.437062in}{2.055000in}}%
\pgfusepath{clip}%
\pgfsetrectcap%
\pgfsetroundjoin%
\pgfsetlinewidth{0.803000pt}%
\definecolor{currentstroke}{rgb}{0.450000,0.450000,0.450000}%
\pgfsetstrokecolor{currentstroke}%
\pgfsetdash{}{0pt}%
\pgfpathmoveto{\pgfqpoint{2.308041in}{0.417642in}}%
\pgfpathlineto{\pgfqpoint{2.308041in}{2.472642in}}%
\pgfusepath{stroke}%
\end{pgfscope}%
\begin{pgfscope}%
\pgfsetbuttcap%
\pgfsetroundjoin%
\definecolor{currentfill}{rgb}{0.000000,0.000000,0.000000}%
\pgfsetfillcolor{currentfill}%
\pgfsetlinewidth{0.803000pt}%
\definecolor{currentstroke}{rgb}{0.000000,0.000000,0.000000}%
\pgfsetstrokecolor{currentstroke}%
\pgfsetdash{}{0pt}%
\pgfsys@defobject{currentmarker}{\pgfqpoint{0.000000in}{-0.048611in}}{\pgfqpoint{0.000000in}{0.000000in}}{%
\pgfpathmoveto{\pgfqpoint{0.000000in}{0.000000in}}%
\pgfpathlineto{\pgfqpoint{0.000000in}{-0.048611in}}%
\pgfusepath{stroke,fill}%
}%
\begin{pgfscope}%
\pgfsys@transformshift{2.308041in}{0.417642in}%
\pgfsys@useobject{currentmarker}{}%
\end{pgfscope}%
\end{pgfscope}%
\begin{pgfscope}%
\definecolor{textcolor}{rgb}{0.000000,0.000000,0.000000}%
\pgfsetstrokecolor{textcolor}%
\pgfsetfillcolor{textcolor}%
\pgftext[x=2.308041in,y=0.320420in,,top]{\color{textcolor}\rmfamily\fontsize{8.000000}{9.600000}\selectfont \(\displaystyle {10^{0}}\)}%
\end{pgfscope}%
\begin{pgfscope}%
\pgfpathrectangle{\pgfqpoint{0.589510in}{0.417642in}}{\pgfqpoint{3.437062in}{2.055000in}}%
\pgfusepath{clip}%
\pgfsetrectcap%
\pgfsetroundjoin%
\pgfsetlinewidth{0.803000pt}%
\definecolor{currentstroke}{rgb}{0.450000,0.450000,0.450000}%
\pgfsetstrokecolor{currentstroke}%
\pgfsetdash{}{0pt}%
\pgfpathmoveto{\pgfqpoint{3.089191in}{0.417642in}}%
\pgfpathlineto{\pgfqpoint{3.089191in}{2.472642in}}%
\pgfusepath{stroke}%
\end{pgfscope}%
\begin{pgfscope}%
\pgfsetbuttcap%
\pgfsetroundjoin%
\definecolor{currentfill}{rgb}{0.000000,0.000000,0.000000}%
\pgfsetfillcolor{currentfill}%
\pgfsetlinewidth{0.803000pt}%
\definecolor{currentstroke}{rgb}{0.000000,0.000000,0.000000}%
\pgfsetstrokecolor{currentstroke}%
\pgfsetdash{}{0pt}%
\pgfsys@defobject{currentmarker}{\pgfqpoint{0.000000in}{-0.048611in}}{\pgfqpoint{0.000000in}{0.000000in}}{%
\pgfpathmoveto{\pgfqpoint{0.000000in}{0.000000in}}%
\pgfpathlineto{\pgfqpoint{0.000000in}{-0.048611in}}%
\pgfusepath{stroke,fill}%
}%
\begin{pgfscope}%
\pgfsys@transformshift{3.089191in}{0.417642in}%
\pgfsys@useobject{currentmarker}{}%
\end{pgfscope}%
\end{pgfscope}%
\begin{pgfscope}%
\definecolor{textcolor}{rgb}{0.000000,0.000000,0.000000}%
\pgfsetstrokecolor{textcolor}%
\pgfsetfillcolor{textcolor}%
\pgftext[x=3.089191in,y=0.320420in,,top]{\color{textcolor}\rmfamily\fontsize{8.000000}{9.600000}\selectfont \(\displaystyle {10^{1}}\)}%
\end{pgfscope}%
\begin{pgfscope}%
\pgfpathrectangle{\pgfqpoint{0.589510in}{0.417642in}}{\pgfqpoint{3.437062in}{2.055000in}}%
\pgfusepath{clip}%
\pgfsetrectcap%
\pgfsetroundjoin%
\pgfsetlinewidth{0.803000pt}%
\definecolor{currentstroke}{rgb}{0.450000,0.450000,0.450000}%
\pgfsetstrokecolor{currentstroke}%
\pgfsetdash{}{0pt}%
\pgfpathmoveto{\pgfqpoint{3.870342in}{0.417642in}}%
\pgfpathlineto{\pgfqpoint{3.870342in}{2.472642in}}%
\pgfusepath{stroke}%
\end{pgfscope}%
\begin{pgfscope}%
\pgfsetbuttcap%
\pgfsetroundjoin%
\definecolor{currentfill}{rgb}{0.000000,0.000000,0.000000}%
\pgfsetfillcolor{currentfill}%
\pgfsetlinewidth{0.803000pt}%
\definecolor{currentstroke}{rgb}{0.000000,0.000000,0.000000}%
\pgfsetstrokecolor{currentstroke}%
\pgfsetdash{}{0pt}%
\pgfsys@defobject{currentmarker}{\pgfqpoint{0.000000in}{-0.048611in}}{\pgfqpoint{0.000000in}{0.000000in}}{%
\pgfpathmoveto{\pgfqpoint{0.000000in}{0.000000in}}%
\pgfpathlineto{\pgfqpoint{0.000000in}{-0.048611in}}%
\pgfusepath{stroke,fill}%
}%
\begin{pgfscope}%
\pgfsys@transformshift{3.870342in}{0.417642in}%
\pgfsys@useobject{currentmarker}{}%
\end{pgfscope}%
\end{pgfscope}%
\begin{pgfscope}%
\definecolor{textcolor}{rgb}{0.000000,0.000000,0.000000}%
\pgfsetstrokecolor{textcolor}%
\pgfsetfillcolor{textcolor}%
\pgftext[x=3.870342in,y=0.320420in,,top]{\color{textcolor}\rmfamily\fontsize{8.000000}{9.600000}\selectfont \(\displaystyle {10^{2}}\)}%
\end{pgfscope}%
\begin{pgfscope}%
\pgfpathrectangle{\pgfqpoint{0.589510in}{0.417642in}}{\pgfqpoint{3.437062in}{2.055000in}}%
\pgfusepath{clip}%
\pgfsetrectcap%
\pgfsetroundjoin%
\pgfsetlinewidth{0.803000pt}%
\definecolor{currentstroke}{rgb}{0.850000,0.850000,0.850000}%
\pgfsetstrokecolor{currentstroke}%
\pgfsetdash{}{0pt}%
\pgfpathmoveto{\pgfqpoint{0.624738in}{0.417642in}}%
\pgfpathlineto{\pgfqpoint{0.624738in}{2.472642in}}%
\pgfusepath{stroke}%
\end{pgfscope}%
\begin{pgfscope}%
\pgfsetbuttcap%
\pgfsetroundjoin%
\definecolor{currentfill}{rgb}{0.000000,0.000000,0.000000}%
\pgfsetfillcolor{currentfill}%
\pgfsetlinewidth{0.602250pt}%
\definecolor{currentstroke}{rgb}{0.000000,0.000000,0.000000}%
\pgfsetstrokecolor{currentstroke}%
\pgfsetdash{}{0pt}%
\pgfsys@defobject{currentmarker}{\pgfqpoint{0.000000in}{-0.027778in}}{\pgfqpoint{0.000000in}{0.000000in}}{%
\pgfpathmoveto{\pgfqpoint{0.000000in}{0.000000in}}%
\pgfpathlineto{\pgfqpoint{0.000000in}{-0.027778in}}%
\pgfusepath{stroke,fill}%
}%
\begin{pgfscope}%
\pgfsys@transformshift{0.624738in}{0.417642in}%
\pgfsys@useobject{currentmarker}{}%
\end{pgfscope}%
\end{pgfscope}%
\begin{pgfscope}%
\pgfpathrectangle{\pgfqpoint{0.589510in}{0.417642in}}{\pgfqpoint{3.437062in}{2.055000in}}%
\pgfusepath{clip}%
\pgfsetrectcap%
\pgfsetroundjoin%
\pgfsetlinewidth{0.803000pt}%
\definecolor{currentstroke}{rgb}{0.850000,0.850000,0.850000}%
\pgfsetstrokecolor{currentstroke}%
\pgfsetdash{}{0pt}%
\pgfpathmoveto{\pgfqpoint{0.670039in}{0.417642in}}%
\pgfpathlineto{\pgfqpoint{0.670039in}{2.472642in}}%
\pgfusepath{stroke}%
\end{pgfscope}%
\begin{pgfscope}%
\pgfsetbuttcap%
\pgfsetroundjoin%
\definecolor{currentfill}{rgb}{0.000000,0.000000,0.000000}%
\pgfsetfillcolor{currentfill}%
\pgfsetlinewidth{0.602250pt}%
\definecolor{currentstroke}{rgb}{0.000000,0.000000,0.000000}%
\pgfsetstrokecolor{currentstroke}%
\pgfsetdash{}{0pt}%
\pgfsys@defobject{currentmarker}{\pgfqpoint{0.000000in}{-0.027778in}}{\pgfqpoint{0.000000in}{0.000000in}}{%
\pgfpathmoveto{\pgfqpoint{0.000000in}{0.000000in}}%
\pgfpathlineto{\pgfqpoint{0.000000in}{-0.027778in}}%
\pgfusepath{stroke,fill}%
}%
\begin{pgfscope}%
\pgfsys@transformshift{0.670039in}{0.417642in}%
\pgfsys@useobject{currentmarker}{}%
\end{pgfscope}%
\end{pgfscope}%
\begin{pgfscope}%
\pgfpathrectangle{\pgfqpoint{0.589510in}{0.417642in}}{\pgfqpoint{3.437062in}{2.055000in}}%
\pgfusepath{clip}%
\pgfsetrectcap%
\pgfsetroundjoin%
\pgfsetlinewidth{0.803000pt}%
\definecolor{currentstroke}{rgb}{0.850000,0.850000,0.850000}%
\pgfsetstrokecolor{currentstroke}%
\pgfsetdash{}{0pt}%
\pgfpathmoveto{\pgfqpoint{0.709996in}{0.417642in}}%
\pgfpathlineto{\pgfqpoint{0.709996in}{2.472642in}}%
\pgfusepath{stroke}%
\end{pgfscope}%
\begin{pgfscope}%
\pgfsetbuttcap%
\pgfsetroundjoin%
\definecolor{currentfill}{rgb}{0.000000,0.000000,0.000000}%
\pgfsetfillcolor{currentfill}%
\pgfsetlinewidth{0.602250pt}%
\definecolor{currentstroke}{rgb}{0.000000,0.000000,0.000000}%
\pgfsetstrokecolor{currentstroke}%
\pgfsetdash{}{0pt}%
\pgfsys@defobject{currentmarker}{\pgfqpoint{0.000000in}{-0.027778in}}{\pgfqpoint{0.000000in}{0.000000in}}{%
\pgfpathmoveto{\pgfqpoint{0.000000in}{0.000000in}}%
\pgfpathlineto{\pgfqpoint{0.000000in}{-0.027778in}}%
\pgfusepath{stroke,fill}%
}%
\begin{pgfscope}%
\pgfsys@transformshift{0.709996in}{0.417642in}%
\pgfsys@useobject{currentmarker}{}%
\end{pgfscope}%
\end{pgfscope}%
\begin{pgfscope}%
\pgfpathrectangle{\pgfqpoint{0.589510in}{0.417642in}}{\pgfqpoint{3.437062in}{2.055000in}}%
\pgfusepath{clip}%
\pgfsetrectcap%
\pgfsetroundjoin%
\pgfsetlinewidth{0.803000pt}%
\definecolor{currentstroke}{rgb}{0.850000,0.850000,0.850000}%
\pgfsetstrokecolor{currentstroke}%
\pgfsetdash{}{0pt}%
\pgfpathmoveto{\pgfqpoint{0.980890in}{0.417642in}}%
\pgfpathlineto{\pgfqpoint{0.980890in}{2.472642in}}%
\pgfusepath{stroke}%
\end{pgfscope}%
\begin{pgfscope}%
\pgfsetbuttcap%
\pgfsetroundjoin%
\definecolor{currentfill}{rgb}{0.000000,0.000000,0.000000}%
\pgfsetfillcolor{currentfill}%
\pgfsetlinewidth{0.602250pt}%
\definecolor{currentstroke}{rgb}{0.000000,0.000000,0.000000}%
\pgfsetstrokecolor{currentstroke}%
\pgfsetdash{}{0pt}%
\pgfsys@defobject{currentmarker}{\pgfqpoint{0.000000in}{-0.027778in}}{\pgfqpoint{0.000000in}{0.000000in}}{%
\pgfpathmoveto{\pgfqpoint{0.000000in}{0.000000in}}%
\pgfpathlineto{\pgfqpoint{0.000000in}{-0.027778in}}%
\pgfusepath{stroke,fill}%
}%
\begin{pgfscope}%
\pgfsys@transformshift{0.980890in}{0.417642in}%
\pgfsys@useobject{currentmarker}{}%
\end{pgfscope}%
\end{pgfscope}%
\begin{pgfscope}%
\pgfpathrectangle{\pgfqpoint{0.589510in}{0.417642in}}{\pgfqpoint{3.437062in}{2.055000in}}%
\pgfusepath{clip}%
\pgfsetrectcap%
\pgfsetroundjoin%
\pgfsetlinewidth{0.803000pt}%
\definecolor{currentstroke}{rgb}{0.850000,0.850000,0.850000}%
\pgfsetstrokecolor{currentstroke}%
\pgfsetdash{}{0pt}%
\pgfpathmoveto{\pgfqpoint{1.118443in}{0.417642in}}%
\pgfpathlineto{\pgfqpoint{1.118443in}{2.472642in}}%
\pgfusepath{stroke}%
\end{pgfscope}%
\begin{pgfscope}%
\pgfsetbuttcap%
\pgfsetroundjoin%
\definecolor{currentfill}{rgb}{0.000000,0.000000,0.000000}%
\pgfsetfillcolor{currentfill}%
\pgfsetlinewidth{0.602250pt}%
\definecolor{currentstroke}{rgb}{0.000000,0.000000,0.000000}%
\pgfsetstrokecolor{currentstroke}%
\pgfsetdash{}{0pt}%
\pgfsys@defobject{currentmarker}{\pgfqpoint{0.000000in}{-0.027778in}}{\pgfqpoint{0.000000in}{0.000000in}}{%
\pgfpathmoveto{\pgfqpoint{0.000000in}{0.000000in}}%
\pgfpathlineto{\pgfqpoint{0.000000in}{-0.027778in}}%
\pgfusepath{stroke,fill}%
}%
\begin{pgfscope}%
\pgfsys@transformshift{1.118443in}{0.417642in}%
\pgfsys@useobject{currentmarker}{}%
\end{pgfscope}%
\end{pgfscope}%
\begin{pgfscope}%
\pgfpathrectangle{\pgfqpoint{0.589510in}{0.417642in}}{\pgfqpoint{3.437062in}{2.055000in}}%
\pgfusepath{clip}%
\pgfsetrectcap%
\pgfsetroundjoin%
\pgfsetlinewidth{0.803000pt}%
\definecolor{currentstroke}{rgb}{0.850000,0.850000,0.850000}%
\pgfsetstrokecolor{currentstroke}%
\pgfsetdash{}{0pt}%
\pgfpathmoveto{\pgfqpoint{1.216039in}{0.417642in}}%
\pgfpathlineto{\pgfqpoint{1.216039in}{2.472642in}}%
\pgfusepath{stroke}%
\end{pgfscope}%
\begin{pgfscope}%
\pgfsetbuttcap%
\pgfsetroundjoin%
\definecolor{currentfill}{rgb}{0.000000,0.000000,0.000000}%
\pgfsetfillcolor{currentfill}%
\pgfsetlinewidth{0.602250pt}%
\definecolor{currentstroke}{rgb}{0.000000,0.000000,0.000000}%
\pgfsetstrokecolor{currentstroke}%
\pgfsetdash{}{0pt}%
\pgfsys@defobject{currentmarker}{\pgfqpoint{0.000000in}{-0.027778in}}{\pgfqpoint{0.000000in}{0.000000in}}{%
\pgfpathmoveto{\pgfqpoint{0.000000in}{0.000000in}}%
\pgfpathlineto{\pgfqpoint{0.000000in}{-0.027778in}}%
\pgfusepath{stroke,fill}%
}%
\begin{pgfscope}%
\pgfsys@transformshift{1.216039in}{0.417642in}%
\pgfsys@useobject{currentmarker}{}%
\end{pgfscope}%
\end{pgfscope}%
\begin{pgfscope}%
\pgfpathrectangle{\pgfqpoint{0.589510in}{0.417642in}}{\pgfqpoint{3.437062in}{2.055000in}}%
\pgfusepath{clip}%
\pgfsetrectcap%
\pgfsetroundjoin%
\pgfsetlinewidth{0.803000pt}%
\definecolor{currentstroke}{rgb}{0.850000,0.850000,0.850000}%
\pgfsetstrokecolor{currentstroke}%
\pgfsetdash{}{0pt}%
\pgfpathmoveto{\pgfqpoint{1.291741in}{0.417642in}}%
\pgfpathlineto{\pgfqpoint{1.291741in}{2.472642in}}%
\pgfusepath{stroke}%
\end{pgfscope}%
\begin{pgfscope}%
\pgfsetbuttcap%
\pgfsetroundjoin%
\definecolor{currentfill}{rgb}{0.000000,0.000000,0.000000}%
\pgfsetfillcolor{currentfill}%
\pgfsetlinewidth{0.602250pt}%
\definecolor{currentstroke}{rgb}{0.000000,0.000000,0.000000}%
\pgfsetstrokecolor{currentstroke}%
\pgfsetdash{}{0pt}%
\pgfsys@defobject{currentmarker}{\pgfqpoint{0.000000in}{-0.027778in}}{\pgfqpoint{0.000000in}{0.000000in}}{%
\pgfpathmoveto{\pgfqpoint{0.000000in}{0.000000in}}%
\pgfpathlineto{\pgfqpoint{0.000000in}{-0.027778in}}%
\pgfusepath{stroke,fill}%
}%
\begin{pgfscope}%
\pgfsys@transformshift{1.291741in}{0.417642in}%
\pgfsys@useobject{currentmarker}{}%
\end{pgfscope}%
\end{pgfscope}%
\begin{pgfscope}%
\pgfpathrectangle{\pgfqpoint{0.589510in}{0.417642in}}{\pgfqpoint{3.437062in}{2.055000in}}%
\pgfusepath{clip}%
\pgfsetrectcap%
\pgfsetroundjoin%
\pgfsetlinewidth{0.803000pt}%
\definecolor{currentstroke}{rgb}{0.850000,0.850000,0.850000}%
\pgfsetstrokecolor{currentstroke}%
\pgfsetdash{}{0pt}%
\pgfpathmoveto{\pgfqpoint{1.353593in}{0.417642in}}%
\pgfpathlineto{\pgfqpoint{1.353593in}{2.472642in}}%
\pgfusepath{stroke}%
\end{pgfscope}%
\begin{pgfscope}%
\pgfsetbuttcap%
\pgfsetroundjoin%
\definecolor{currentfill}{rgb}{0.000000,0.000000,0.000000}%
\pgfsetfillcolor{currentfill}%
\pgfsetlinewidth{0.602250pt}%
\definecolor{currentstroke}{rgb}{0.000000,0.000000,0.000000}%
\pgfsetstrokecolor{currentstroke}%
\pgfsetdash{}{0pt}%
\pgfsys@defobject{currentmarker}{\pgfqpoint{0.000000in}{-0.027778in}}{\pgfqpoint{0.000000in}{0.000000in}}{%
\pgfpathmoveto{\pgfqpoint{0.000000in}{0.000000in}}%
\pgfpathlineto{\pgfqpoint{0.000000in}{-0.027778in}}%
\pgfusepath{stroke,fill}%
}%
\begin{pgfscope}%
\pgfsys@transformshift{1.353593in}{0.417642in}%
\pgfsys@useobject{currentmarker}{}%
\end{pgfscope}%
\end{pgfscope}%
\begin{pgfscope}%
\pgfpathrectangle{\pgfqpoint{0.589510in}{0.417642in}}{\pgfqpoint{3.437062in}{2.055000in}}%
\pgfusepath{clip}%
\pgfsetrectcap%
\pgfsetroundjoin%
\pgfsetlinewidth{0.803000pt}%
\definecolor{currentstroke}{rgb}{0.850000,0.850000,0.850000}%
\pgfsetstrokecolor{currentstroke}%
\pgfsetdash{}{0pt}%
\pgfpathmoveto{\pgfqpoint{1.405889in}{0.417642in}}%
\pgfpathlineto{\pgfqpoint{1.405889in}{2.472642in}}%
\pgfusepath{stroke}%
\end{pgfscope}%
\begin{pgfscope}%
\pgfsetbuttcap%
\pgfsetroundjoin%
\definecolor{currentfill}{rgb}{0.000000,0.000000,0.000000}%
\pgfsetfillcolor{currentfill}%
\pgfsetlinewidth{0.602250pt}%
\definecolor{currentstroke}{rgb}{0.000000,0.000000,0.000000}%
\pgfsetstrokecolor{currentstroke}%
\pgfsetdash{}{0pt}%
\pgfsys@defobject{currentmarker}{\pgfqpoint{0.000000in}{-0.027778in}}{\pgfqpoint{0.000000in}{0.000000in}}{%
\pgfpathmoveto{\pgfqpoint{0.000000in}{0.000000in}}%
\pgfpathlineto{\pgfqpoint{0.000000in}{-0.027778in}}%
\pgfusepath{stroke,fill}%
}%
\begin{pgfscope}%
\pgfsys@transformshift{1.405889in}{0.417642in}%
\pgfsys@useobject{currentmarker}{}%
\end{pgfscope}%
\end{pgfscope}%
\begin{pgfscope}%
\pgfpathrectangle{\pgfqpoint{0.589510in}{0.417642in}}{\pgfqpoint{3.437062in}{2.055000in}}%
\pgfusepath{clip}%
\pgfsetrectcap%
\pgfsetroundjoin%
\pgfsetlinewidth{0.803000pt}%
\definecolor{currentstroke}{rgb}{0.850000,0.850000,0.850000}%
\pgfsetstrokecolor{currentstroke}%
\pgfsetdash{}{0pt}%
\pgfpathmoveto{\pgfqpoint{1.451189in}{0.417642in}}%
\pgfpathlineto{\pgfqpoint{1.451189in}{2.472642in}}%
\pgfusepath{stroke}%
\end{pgfscope}%
\begin{pgfscope}%
\pgfsetbuttcap%
\pgfsetroundjoin%
\definecolor{currentfill}{rgb}{0.000000,0.000000,0.000000}%
\pgfsetfillcolor{currentfill}%
\pgfsetlinewidth{0.602250pt}%
\definecolor{currentstroke}{rgb}{0.000000,0.000000,0.000000}%
\pgfsetstrokecolor{currentstroke}%
\pgfsetdash{}{0pt}%
\pgfsys@defobject{currentmarker}{\pgfqpoint{0.000000in}{-0.027778in}}{\pgfqpoint{0.000000in}{0.000000in}}{%
\pgfpathmoveto{\pgfqpoint{0.000000in}{0.000000in}}%
\pgfpathlineto{\pgfqpoint{0.000000in}{-0.027778in}}%
\pgfusepath{stroke,fill}%
}%
\begin{pgfscope}%
\pgfsys@transformshift{1.451189in}{0.417642in}%
\pgfsys@useobject{currentmarker}{}%
\end{pgfscope}%
\end{pgfscope}%
\begin{pgfscope}%
\pgfpathrectangle{\pgfqpoint{0.589510in}{0.417642in}}{\pgfqpoint{3.437062in}{2.055000in}}%
\pgfusepath{clip}%
\pgfsetrectcap%
\pgfsetroundjoin%
\pgfsetlinewidth{0.803000pt}%
\definecolor{currentstroke}{rgb}{0.850000,0.850000,0.850000}%
\pgfsetstrokecolor{currentstroke}%
\pgfsetdash{}{0pt}%
\pgfpathmoveto{\pgfqpoint{1.491147in}{0.417642in}}%
\pgfpathlineto{\pgfqpoint{1.491147in}{2.472642in}}%
\pgfusepath{stroke}%
\end{pgfscope}%
\begin{pgfscope}%
\pgfsetbuttcap%
\pgfsetroundjoin%
\definecolor{currentfill}{rgb}{0.000000,0.000000,0.000000}%
\pgfsetfillcolor{currentfill}%
\pgfsetlinewidth{0.602250pt}%
\definecolor{currentstroke}{rgb}{0.000000,0.000000,0.000000}%
\pgfsetstrokecolor{currentstroke}%
\pgfsetdash{}{0pt}%
\pgfsys@defobject{currentmarker}{\pgfqpoint{0.000000in}{-0.027778in}}{\pgfqpoint{0.000000in}{0.000000in}}{%
\pgfpathmoveto{\pgfqpoint{0.000000in}{0.000000in}}%
\pgfpathlineto{\pgfqpoint{0.000000in}{-0.027778in}}%
\pgfusepath{stroke,fill}%
}%
\begin{pgfscope}%
\pgfsys@transformshift{1.491147in}{0.417642in}%
\pgfsys@useobject{currentmarker}{}%
\end{pgfscope}%
\end{pgfscope}%
\begin{pgfscope}%
\pgfpathrectangle{\pgfqpoint{0.589510in}{0.417642in}}{\pgfqpoint{3.437062in}{2.055000in}}%
\pgfusepath{clip}%
\pgfsetrectcap%
\pgfsetroundjoin%
\pgfsetlinewidth{0.803000pt}%
\definecolor{currentstroke}{rgb}{0.850000,0.850000,0.850000}%
\pgfsetstrokecolor{currentstroke}%
\pgfsetdash{}{0pt}%
\pgfpathmoveto{\pgfqpoint{1.762040in}{0.417642in}}%
\pgfpathlineto{\pgfqpoint{1.762040in}{2.472642in}}%
\pgfusepath{stroke}%
\end{pgfscope}%
\begin{pgfscope}%
\pgfsetbuttcap%
\pgfsetroundjoin%
\definecolor{currentfill}{rgb}{0.000000,0.000000,0.000000}%
\pgfsetfillcolor{currentfill}%
\pgfsetlinewidth{0.602250pt}%
\definecolor{currentstroke}{rgb}{0.000000,0.000000,0.000000}%
\pgfsetstrokecolor{currentstroke}%
\pgfsetdash{}{0pt}%
\pgfsys@defobject{currentmarker}{\pgfqpoint{0.000000in}{-0.027778in}}{\pgfqpoint{0.000000in}{0.000000in}}{%
\pgfpathmoveto{\pgfqpoint{0.000000in}{0.000000in}}%
\pgfpathlineto{\pgfqpoint{0.000000in}{-0.027778in}}%
\pgfusepath{stroke,fill}%
}%
\begin{pgfscope}%
\pgfsys@transformshift{1.762040in}{0.417642in}%
\pgfsys@useobject{currentmarker}{}%
\end{pgfscope}%
\end{pgfscope}%
\begin{pgfscope}%
\pgfpathrectangle{\pgfqpoint{0.589510in}{0.417642in}}{\pgfqpoint{3.437062in}{2.055000in}}%
\pgfusepath{clip}%
\pgfsetrectcap%
\pgfsetroundjoin%
\pgfsetlinewidth{0.803000pt}%
\definecolor{currentstroke}{rgb}{0.850000,0.850000,0.850000}%
\pgfsetstrokecolor{currentstroke}%
\pgfsetdash{}{0pt}%
\pgfpathmoveto{\pgfqpoint{1.899594in}{0.417642in}}%
\pgfpathlineto{\pgfqpoint{1.899594in}{2.472642in}}%
\pgfusepath{stroke}%
\end{pgfscope}%
\begin{pgfscope}%
\pgfsetbuttcap%
\pgfsetroundjoin%
\definecolor{currentfill}{rgb}{0.000000,0.000000,0.000000}%
\pgfsetfillcolor{currentfill}%
\pgfsetlinewidth{0.602250pt}%
\definecolor{currentstroke}{rgb}{0.000000,0.000000,0.000000}%
\pgfsetstrokecolor{currentstroke}%
\pgfsetdash{}{0pt}%
\pgfsys@defobject{currentmarker}{\pgfqpoint{0.000000in}{-0.027778in}}{\pgfqpoint{0.000000in}{0.000000in}}{%
\pgfpathmoveto{\pgfqpoint{0.000000in}{0.000000in}}%
\pgfpathlineto{\pgfqpoint{0.000000in}{-0.027778in}}%
\pgfusepath{stroke,fill}%
}%
\begin{pgfscope}%
\pgfsys@transformshift{1.899594in}{0.417642in}%
\pgfsys@useobject{currentmarker}{}%
\end{pgfscope}%
\end{pgfscope}%
\begin{pgfscope}%
\pgfpathrectangle{\pgfqpoint{0.589510in}{0.417642in}}{\pgfqpoint{3.437062in}{2.055000in}}%
\pgfusepath{clip}%
\pgfsetrectcap%
\pgfsetroundjoin%
\pgfsetlinewidth{0.803000pt}%
\definecolor{currentstroke}{rgb}{0.850000,0.850000,0.850000}%
\pgfsetstrokecolor{currentstroke}%
\pgfsetdash{}{0pt}%
\pgfpathmoveto{\pgfqpoint{1.997190in}{0.417642in}}%
\pgfpathlineto{\pgfqpoint{1.997190in}{2.472642in}}%
\pgfusepath{stroke}%
\end{pgfscope}%
\begin{pgfscope}%
\pgfsetbuttcap%
\pgfsetroundjoin%
\definecolor{currentfill}{rgb}{0.000000,0.000000,0.000000}%
\pgfsetfillcolor{currentfill}%
\pgfsetlinewidth{0.602250pt}%
\definecolor{currentstroke}{rgb}{0.000000,0.000000,0.000000}%
\pgfsetstrokecolor{currentstroke}%
\pgfsetdash{}{0pt}%
\pgfsys@defobject{currentmarker}{\pgfqpoint{0.000000in}{-0.027778in}}{\pgfqpoint{0.000000in}{0.000000in}}{%
\pgfpathmoveto{\pgfqpoint{0.000000in}{0.000000in}}%
\pgfpathlineto{\pgfqpoint{0.000000in}{-0.027778in}}%
\pgfusepath{stroke,fill}%
}%
\begin{pgfscope}%
\pgfsys@transformshift{1.997190in}{0.417642in}%
\pgfsys@useobject{currentmarker}{}%
\end{pgfscope}%
\end{pgfscope}%
\begin{pgfscope}%
\pgfpathrectangle{\pgfqpoint{0.589510in}{0.417642in}}{\pgfqpoint{3.437062in}{2.055000in}}%
\pgfusepath{clip}%
\pgfsetrectcap%
\pgfsetroundjoin%
\pgfsetlinewidth{0.803000pt}%
\definecolor{currentstroke}{rgb}{0.850000,0.850000,0.850000}%
\pgfsetstrokecolor{currentstroke}%
\pgfsetdash{}{0pt}%
\pgfpathmoveto{\pgfqpoint{2.072891in}{0.417642in}}%
\pgfpathlineto{\pgfqpoint{2.072891in}{2.472642in}}%
\pgfusepath{stroke}%
\end{pgfscope}%
\begin{pgfscope}%
\pgfsetbuttcap%
\pgfsetroundjoin%
\definecolor{currentfill}{rgb}{0.000000,0.000000,0.000000}%
\pgfsetfillcolor{currentfill}%
\pgfsetlinewidth{0.602250pt}%
\definecolor{currentstroke}{rgb}{0.000000,0.000000,0.000000}%
\pgfsetstrokecolor{currentstroke}%
\pgfsetdash{}{0pt}%
\pgfsys@defobject{currentmarker}{\pgfqpoint{0.000000in}{-0.027778in}}{\pgfqpoint{0.000000in}{0.000000in}}{%
\pgfpathmoveto{\pgfqpoint{0.000000in}{0.000000in}}%
\pgfpathlineto{\pgfqpoint{0.000000in}{-0.027778in}}%
\pgfusepath{stroke,fill}%
}%
\begin{pgfscope}%
\pgfsys@transformshift{2.072891in}{0.417642in}%
\pgfsys@useobject{currentmarker}{}%
\end{pgfscope}%
\end{pgfscope}%
\begin{pgfscope}%
\pgfpathrectangle{\pgfqpoint{0.589510in}{0.417642in}}{\pgfqpoint{3.437062in}{2.055000in}}%
\pgfusepath{clip}%
\pgfsetrectcap%
\pgfsetroundjoin%
\pgfsetlinewidth{0.803000pt}%
\definecolor{currentstroke}{rgb}{0.850000,0.850000,0.850000}%
\pgfsetstrokecolor{currentstroke}%
\pgfsetdash{}{0pt}%
\pgfpathmoveto{\pgfqpoint{2.134743in}{0.417642in}}%
\pgfpathlineto{\pgfqpoint{2.134743in}{2.472642in}}%
\pgfusepath{stroke}%
\end{pgfscope}%
\begin{pgfscope}%
\pgfsetbuttcap%
\pgfsetroundjoin%
\definecolor{currentfill}{rgb}{0.000000,0.000000,0.000000}%
\pgfsetfillcolor{currentfill}%
\pgfsetlinewidth{0.602250pt}%
\definecolor{currentstroke}{rgb}{0.000000,0.000000,0.000000}%
\pgfsetstrokecolor{currentstroke}%
\pgfsetdash{}{0pt}%
\pgfsys@defobject{currentmarker}{\pgfqpoint{0.000000in}{-0.027778in}}{\pgfqpoint{0.000000in}{0.000000in}}{%
\pgfpathmoveto{\pgfqpoint{0.000000in}{0.000000in}}%
\pgfpathlineto{\pgfqpoint{0.000000in}{-0.027778in}}%
\pgfusepath{stroke,fill}%
}%
\begin{pgfscope}%
\pgfsys@transformshift{2.134743in}{0.417642in}%
\pgfsys@useobject{currentmarker}{}%
\end{pgfscope}%
\end{pgfscope}%
\begin{pgfscope}%
\pgfpathrectangle{\pgfqpoint{0.589510in}{0.417642in}}{\pgfqpoint{3.437062in}{2.055000in}}%
\pgfusepath{clip}%
\pgfsetrectcap%
\pgfsetroundjoin%
\pgfsetlinewidth{0.803000pt}%
\definecolor{currentstroke}{rgb}{0.850000,0.850000,0.850000}%
\pgfsetstrokecolor{currentstroke}%
\pgfsetdash{}{0pt}%
\pgfpathmoveto{\pgfqpoint{2.187039in}{0.417642in}}%
\pgfpathlineto{\pgfqpoint{2.187039in}{2.472642in}}%
\pgfusepath{stroke}%
\end{pgfscope}%
\begin{pgfscope}%
\pgfsetbuttcap%
\pgfsetroundjoin%
\definecolor{currentfill}{rgb}{0.000000,0.000000,0.000000}%
\pgfsetfillcolor{currentfill}%
\pgfsetlinewidth{0.602250pt}%
\definecolor{currentstroke}{rgb}{0.000000,0.000000,0.000000}%
\pgfsetstrokecolor{currentstroke}%
\pgfsetdash{}{0pt}%
\pgfsys@defobject{currentmarker}{\pgfqpoint{0.000000in}{-0.027778in}}{\pgfqpoint{0.000000in}{0.000000in}}{%
\pgfpathmoveto{\pgfqpoint{0.000000in}{0.000000in}}%
\pgfpathlineto{\pgfqpoint{0.000000in}{-0.027778in}}%
\pgfusepath{stroke,fill}%
}%
\begin{pgfscope}%
\pgfsys@transformshift{2.187039in}{0.417642in}%
\pgfsys@useobject{currentmarker}{}%
\end{pgfscope}%
\end{pgfscope}%
\begin{pgfscope}%
\pgfpathrectangle{\pgfqpoint{0.589510in}{0.417642in}}{\pgfqpoint{3.437062in}{2.055000in}}%
\pgfusepath{clip}%
\pgfsetrectcap%
\pgfsetroundjoin%
\pgfsetlinewidth{0.803000pt}%
\definecolor{currentstroke}{rgb}{0.850000,0.850000,0.850000}%
\pgfsetstrokecolor{currentstroke}%
\pgfsetdash{}{0pt}%
\pgfpathmoveto{\pgfqpoint{2.232339in}{0.417642in}}%
\pgfpathlineto{\pgfqpoint{2.232339in}{2.472642in}}%
\pgfusepath{stroke}%
\end{pgfscope}%
\begin{pgfscope}%
\pgfsetbuttcap%
\pgfsetroundjoin%
\definecolor{currentfill}{rgb}{0.000000,0.000000,0.000000}%
\pgfsetfillcolor{currentfill}%
\pgfsetlinewidth{0.602250pt}%
\definecolor{currentstroke}{rgb}{0.000000,0.000000,0.000000}%
\pgfsetstrokecolor{currentstroke}%
\pgfsetdash{}{0pt}%
\pgfsys@defobject{currentmarker}{\pgfqpoint{0.000000in}{-0.027778in}}{\pgfqpoint{0.000000in}{0.000000in}}{%
\pgfpathmoveto{\pgfqpoint{0.000000in}{0.000000in}}%
\pgfpathlineto{\pgfqpoint{0.000000in}{-0.027778in}}%
\pgfusepath{stroke,fill}%
}%
\begin{pgfscope}%
\pgfsys@transformshift{2.232339in}{0.417642in}%
\pgfsys@useobject{currentmarker}{}%
\end{pgfscope}%
\end{pgfscope}%
\begin{pgfscope}%
\pgfpathrectangle{\pgfqpoint{0.589510in}{0.417642in}}{\pgfqpoint{3.437062in}{2.055000in}}%
\pgfusepath{clip}%
\pgfsetrectcap%
\pgfsetroundjoin%
\pgfsetlinewidth{0.803000pt}%
\definecolor{currentstroke}{rgb}{0.850000,0.850000,0.850000}%
\pgfsetstrokecolor{currentstroke}%
\pgfsetdash{}{0pt}%
\pgfpathmoveto{\pgfqpoint{2.272297in}{0.417642in}}%
\pgfpathlineto{\pgfqpoint{2.272297in}{2.472642in}}%
\pgfusepath{stroke}%
\end{pgfscope}%
\begin{pgfscope}%
\pgfsetbuttcap%
\pgfsetroundjoin%
\definecolor{currentfill}{rgb}{0.000000,0.000000,0.000000}%
\pgfsetfillcolor{currentfill}%
\pgfsetlinewidth{0.602250pt}%
\definecolor{currentstroke}{rgb}{0.000000,0.000000,0.000000}%
\pgfsetstrokecolor{currentstroke}%
\pgfsetdash{}{0pt}%
\pgfsys@defobject{currentmarker}{\pgfqpoint{0.000000in}{-0.027778in}}{\pgfqpoint{0.000000in}{0.000000in}}{%
\pgfpathmoveto{\pgfqpoint{0.000000in}{0.000000in}}%
\pgfpathlineto{\pgfqpoint{0.000000in}{-0.027778in}}%
\pgfusepath{stroke,fill}%
}%
\begin{pgfscope}%
\pgfsys@transformshift{2.272297in}{0.417642in}%
\pgfsys@useobject{currentmarker}{}%
\end{pgfscope}%
\end{pgfscope}%
\begin{pgfscope}%
\pgfpathrectangle{\pgfqpoint{0.589510in}{0.417642in}}{\pgfqpoint{3.437062in}{2.055000in}}%
\pgfusepath{clip}%
\pgfsetrectcap%
\pgfsetroundjoin%
\pgfsetlinewidth{0.803000pt}%
\definecolor{currentstroke}{rgb}{0.850000,0.850000,0.850000}%
\pgfsetstrokecolor{currentstroke}%
\pgfsetdash{}{0pt}%
\pgfpathmoveto{\pgfqpoint{2.543190in}{0.417642in}}%
\pgfpathlineto{\pgfqpoint{2.543190in}{2.472642in}}%
\pgfusepath{stroke}%
\end{pgfscope}%
\begin{pgfscope}%
\pgfsetbuttcap%
\pgfsetroundjoin%
\definecolor{currentfill}{rgb}{0.000000,0.000000,0.000000}%
\pgfsetfillcolor{currentfill}%
\pgfsetlinewidth{0.602250pt}%
\definecolor{currentstroke}{rgb}{0.000000,0.000000,0.000000}%
\pgfsetstrokecolor{currentstroke}%
\pgfsetdash{}{0pt}%
\pgfsys@defobject{currentmarker}{\pgfqpoint{0.000000in}{-0.027778in}}{\pgfqpoint{0.000000in}{0.000000in}}{%
\pgfpathmoveto{\pgfqpoint{0.000000in}{0.000000in}}%
\pgfpathlineto{\pgfqpoint{0.000000in}{-0.027778in}}%
\pgfusepath{stroke,fill}%
}%
\begin{pgfscope}%
\pgfsys@transformshift{2.543190in}{0.417642in}%
\pgfsys@useobject{currentmarker}{}%
\end{pgfscope}%
\end{pgfscope}%
\begin{pgfscope}%
\pgfpathrectangle{\pgfqpoint{0.589510in}{0.417642in}}{\pgfqpoint{3.437062in}{2.055000in}}%
\pgfusepath{clip}%
\pgfsetrectcap%
\pgfsetroundjoin%
\pgfsetlinewidth{0.803000pt}%
\definecolor{currentstroke}{rgb}{0.850000,0.850000,0.850000}%
\pgfsetstrokecolor{currentstroke}%
\pgfsetdash{}{0pt}%
\pgfpathmoveto{\pgfqpoint{2.680744in}{0.417642in}}%
\pgfpathlineto{\pgfqpoint{2.680744in}{2.472642in}}%
\pgfusepath{stroke}%
\end{pgfscope}%
\begin{pgfscope}%
\pgfsetbuttcap%
\pgfsetroundjoin%
\definecolor{currentfill}{rgb}{0.000000,0.000000,0.000000}%
\pgfsetfillcolor{currentfill}%
\pgfsetlinewidth{0.602250pt}%
\definecolor{currentstroke}{rgb}{0.000000,0.000000,0.000000}%
\pgfsetstrokecolor{currentstroke}%
\pgfsetdash{}{0pt}%
\pgfsys@defobject{currentmarker}{\pgfqpoint{0.000000in}{-0.027778in}}{\pgfqpoint{0.000000in}{0.000000in}}{%
\pgfpathmoveto{\pgfqpoint{0.000000in}{0.000000in}}%
\pgfpathlineto{\pgfqpoint{0.000000in}{-0.027778in}}%
\pgfusepath{stroke,fill}%
}%
\begin{pgfscope}%
\pgfsys@transformshift{2.680744in}{0.417642in}%
\pgfsys@useobject{currentmarker}{}%
\end{pgfscope}%
\end{pgfscope}%
\begin{pgfscope}%
\pgfpathrectangle{\pgfqpoint{0.589510in}{0.417642in}}{\pgfqpoint{3.437062in}{2.055000in}}%
\pgfusepath{clip}%
\pgfsetrectcap%
\pgfsetroundjoin%
\pgfsetlinewidth{0.803000pt}%
\definecolor{currentstroke}{rgb}{0.850000,0.850000,0.850000}%
\pgfsetstrokecolor{currentstroke}%
\pgfsetdash{}{0pt}%
\pgfpathmoveto{\pgfqpoint{2.778340in}{0.417642in}}%
\pgfpathlineto{\pgfqpoint{2.778340in}{2.472642in}}%
\pgfusepath{stroke}%
\end{pgfscope}%
\begin{pgfscope}%
\pgfsetbuttcap%
\pgfsetroundjoin%
\definecolor{currentfill}{rgb}{0.000000,0.000000,0.000000}%
\pgfsetfillcolor{currentfill}%
\pgfsetlinewidth{0.602250pt}%
\definecolor{currentstroke}{rgb}{0.000000,0.000000,0.000000}%
\pgfsetstrokecolor{currentstroke}%
\pgfsetdash{}{0pt}%
\pgfsys@defobject{currentmarker}{\pgfqpoint{0.000000in}{-0.027778in}}{\pgfqpoint{0.000000in}{0.000000in}}{%
\pgfpathmoveto{\pgfqpoint{0.000000in}{0.000000in}}%
\pgfpathlineto{\pgfqpoint{0.000000in}{-0.027778in}}%
\pgfusepath{stroke,fill}%
}%
\begin{pgfscope}%
\pgfsys@transformshift{2.778340in}{0.417642in}%
\pgfsys@useobject{currentmarker}{}%
\end{pgfscope}%
\end{pgfscope}%
\begin{pgfscope}%
\pgfpathrectangle{\pgfqpoint{0.589510in}{0.417642in}}{\pgfqpoint{3.437062in}{2.055000in}}%
\pgfusepath{clip}%
\pgfsetrectcap%
\pgfsetroundjoin%
\pgfsetlinewidth{0.803000pt}%
\definecolor{currentstroke}{rgb}{0.850000,0.850000,0.850000}%
\pgfsetstrokecolor{currentstroke}%
\pgfsetdash{}{0pt}%
\pgfpathmoveto{\pgfqpoint{2.854041in}{0.417642in}}%
\pgfpathlineto{\pgfqpoint{2.854041in}{2.472642in}}%
\pgfusepath{stroke}%
\end{pgfscope}%
\begin{pgfscope}%
\pgfsetbuttcap%
\pgfsetroundjoin%
\definecolor{currentfill}{rgb}{0.000000,0.000000,0.000000}%
\pgfsetfillcolor{currentfill}%
\pgfsetlinewidth{0.602250pt}%
\definecolor{currentstroke}{rgb}{0.000000,0.000000,0.000000}%
\pgfsetstrokecolor{currentstroke}%
\pgfsetdash{}{0pt}%
\pgfsys@defobject{currentmarker}{\pgfqpoint{0.000000in}{-0.027778in}}{\pgfqpoint{0.000000in}{0.000000in}}{%
\pgfpathmoveto{\pgfqpoint{0.000000in}{0.000000in}}%
\pgfpathlineto{\pgfqpoint{0.000000in}{-0.027778in}}%
\pgfusepath{stroke,fill}%
}%
\begin{pgfscope}%
\pgfsys@transformshift{2.854041in}{0.417642in}%
\pgfsys@useobject{currentmarker}{}%
\end{pgfscope}%
\end{pgfscope}%
\begin{pgfscope}%
\pgfpathrectangle{\pgfqpoint{0.589510in}{0.417642in}}{\pgfqpoint{3.437062in}{2.055000in}}%
\pgfusepath{clip}%
\pgfsetrectcap%
\pgfsetroundjoin%
\pgfsetlinewidth{0.803000pt}%
\definecolor{currentstroke}{rgb}{0.850000,0.850000,0.850000}%
\pgfsetstrokecolor{currentstroke}%
\pgfsetdash{}{0pt}%
\pgfpathmoveto{\pgfqpoint{2.915894in}{0.417642in}}%
\pgfpathlineto{\pgfqpoint{2.915894in}{2.472642in}}%
\pgfusepath{stroke}%
\end{pgfscope}%
\begin{pgfscope}%
\pgfsetbuttcap%
\pgfsetroundjoin%
\definecolor{currentfill}{rgb}{0.000000,0.000000,0.000000}%
\pgfsetfillcolor{currentfill}%
\pgfsetlinewidth{0.602250pt}%
\definecolor{currentstroke}{rgb}{0.000000,0.000000,0.000000}%
\pgfsetstrokecolor{currentstroke}%
\pgfsetdash{}{0pt}%
\pgfsys@defobject{currentmarker}{\pgfqpoint{0.000000in}{-0.027778in}}{\pgfqpoint{0.000000in}{0.000000in}}{%
\pgfpathmoveto{\pgfqpoint{0.000000in}{0.000000in}}%
\pgfpathlineto{\pgfqpoint{0.000000in}{-0.027778in}}%
\pgfusepath{stroke,fill}%
}%
\begin{pgfscope}%
\pgfsys@transformshift{2.915894in}{0.417642in}%
\pgfsys@useobject{currentmarker}{}%
\end{pgfscope}%
\end{pgfscope}%
\begin{pgfscope}%
\pgfpathrectangle{\pgfqpoint{0.589510in}{0.417642in}}{\pgfqpoint{3.437062in}{2.055000in}}%
\pgfusepath{clip}%
\pgfsetrectcap%
\pgfsetroundjoin%
\pgfsetlinewidth{0.803000pt}%
\definecolor{currentstroke}{rgb}{0.850000,0.850000,0.850000}%
\pgfsetstrokecolor{currentstroke}%
\pgfsetdash{}{0pt}%
\pgfpathmoveto{\pgfqpoint{2.968189in}{0.417642in}}%
\pgfpathlineto{\pgfqpoint{2.968189in}{2.472642in}}%
\pgfusepath{stroke}%
\end{pgfscope}%
\begin{pgfscope}%
\pgfsetbuttcap%
\pgfsetroundjoin%
\definecolor{currentfill}{rgb}{0.000000,0.000000,0.000000}%
\pgfsetfillcolor{currentfill}%
\pgfsetlinewidth{0.602250pt}%
\definecolor{currentstroke}{rgb}{0.000000,0.000000,0.000000}%
\pgfsetstrokecolor{currentstroke}%
\pgfsetdash{}{0pt}%
\pgfsys@defobject{currentmarker}{\pgfqpoint{0.000000in}{-0.027778in}}{\pgfqpoint{0.000000in}{0.000000in}}{%
\pgfpathmoveto{\pgfqpoint{0.000000in}{0.000000in}}%
\pgfpathlineto{\pgfqpoint{0.000000in}{-0.027778in}}%
\pgfusepath{stroke,fill}%
}%
\begin{pgfscope}%
\pgfsys@transformshift{2.968189in}{0.417642in}%
\pgfsys@useobject{currentmarker}{}%
\end{pgfscope}%
\end{pgfscope}%
\begin{pgfscope}%
\pgfpathrectangle{\pgfqpoint{0.589510in}{0.417642in}}{\pgfqpoint{3.437062in}{2.055000in}}%
\pgfusepath{clip}%
\pgfsetrectcap%
\pgfsetroundjoin%
\pgfsetlinewidth{0.803000pt}%
\definecolor{currentstroke}{rgb}{0.850000,0.850000,0.850000}%
\pgfsetstrokecolor{currentstroke}%
\pgfsetdash{}{0pt}%
\pgfpathmoveto{\pgfqpoint{3.013490in}{0.417642in}}%
\pgfpathlineto{\pgfqpoint{3.013490in}{2.472642in}}%
\pgfusepath{stroke}%
\end{pgfscope}%
\begin{pgfscope}%
\pgfsetbuttcap%
\pgfsetroundjoin%
\definecolor{currentfill}{rgb}{0.000000,0.000000,0.000000}%
\pgfsetfillcolor{currentfill}%
\pgfsetlinewidth{0.602250pt}%
\definecolor{currentstroke}{rgb}{0.000000,0.000000,0.000000}%
\pgfsetstrokecolor{currentstroke}%
\pgfsetdash{}{0pt}%
\pgfsys@defobject{currentmarker}{\pgfqpoint{0.000000in}{-0.027778in}}{\pgfqpoint{0.000000in}{0.000000in}}{%
\pgfpathmoveto{\pgfqpoint{0.000000in}{0.000000in}}%
\pgfpathlineto{\pgfqpoint{0.000000in}{-0.027778in}}%
\pgfusepath{stroke,fill}%
}%
\begin{pgfscope}%
\pgfsys@transformshift{3.013490in}{0.417642in}%
\pgfsys@useobject{currentmarker}{}%
\end{pgfscope}%
\end{pgfscope}%
\begin{pgfscope}%
\pgfpathrectangle{\pgfqpoint{0.589510in}{0.417642in}}{\pgfqpoint{3.437062in}{2.055000in}}%
\pgfusepath{clip}%
\pgfsetrectcap%
\pgfsetroundjoin%
\pgfsetlinewidth{0.803000pt}%
\definecolor{currentstroke}{rgb}{0.850000,0.850000,0.850000}%
\pgfsetstrokecolor{currentstroke}%
\pgfsetdash{}{0pt}%
\pgfpathmoveto{\pgfqpoint{3.053448in}{0.417642in}}%
\pgfpathlineto{\pgfqpoint{3.053448in}{2.472642in}}%
\pgfusepath{stroke}%
\end{pgfscope}%
\begin{pgfscope}%
\pgfsetbuttcap%
\pgfsetroundjoin%
\definecolor{currentfill}{rgb}{0.000000,0.000000,0.000000}%
\pgfsetfillcolor{currentfill}%
\pgfsetlinewidth{0.602250pt}%
\definecolor{currentstroke}{rgb}{0.000000,0.000000,0.000000}%
\pgfsetstrokecolor{currentstroke}%
\pgfsetdash{}{0pt}%
\pgfsys@defobject{currentmarker}{\pgfqpoint{0.000000in}{-0.027778in}}{\pgfqpoint{0.000000in}{0.000000in}}{%
\pgfpathmoveto{\pgfqpoint{0.000000in}{0.000000in}}%
\pgfpathlineto{\pgfqpoint{0.000000in}{-0.027778in}}%
\pgfusepath{stroke,fill}%
}%
\begin{pgfscope}%
\pgfsys@transformshift{3.053448in}{0.417642in}%
\pgfsys@useobject{currentmarker}{}%
\end{pgfscope}%
\end{pgfscope}%
\begin{pgfscope}%
\pgfpathrectangle{\pgfqpoint{0.589510in}{0.417642in}}{\pgfqpoint{3.437062in}{2.055000in}}%
\pgfusepath{clip}%
\pgfsetrectcap%
\pgfsetroundjoin%
\pgfsetlinewidth{0.803000pt}%
\definecolor{currentstroke}{rgb}{0.850000,0.850000,0.850000}%
\pgfsetstrokecolor{currentstroke}%
\pgfsetdash{}{0pt}%
\pgfpathmoveto{\pgfqpoint{3.324341in}{0.417642in}}%
\pgfpathlineto{\pgfqpoint{3.324341in}{2.472642in}}%
\pgfusepath{stroke}%
\end{pgfscope}%
\begin{pgfscope}%
\pgfsetbuttcap%
\pgfsetroundjoin%
\definecolor{currentfill}{rgb}{0.000000,0.000000,0.000000}%
\pgfsetfillcolor{currentfill}%
\pgfsetlinewidth{0.602250pt}%
\definecolor{currentstroke}{rgb}{0.000000,0.000000,0.000000}%
\pgfsetstrokecolor{currentstroke}%
\pgfsetdash{}{0pt}%
\pgfsys@defobject{currentmarker}{\pgfqpoint{0.000000in}{-0.027778in}}{\pgfqpoint{0.000000in}{0.000000in}}{%
\pgfpathmoveto{\pgfqpoint{0.000000in}{0.000000in}}%
\pgfpathlineto{\pgfqpoint{0.000000in}{-0.027778in}}%
\pgfusepath{stroke,fill}%
}%
\begin{pgfscope}%
\pgfsys@transformshift{3.324341in}{0.417642in}%
\pgfsys@useobject{currentmarker}{}%
\end{pgfscope}%
\end{pgfscope}%
\begin{pgfscope}%
\pgfpathrectangle{\pgfqpoint{0.589510in}{0.417642in}}{\pgfqpoint{3.437062in}{2.055000in}}%
\pgfusepath{clip}%
\pgfsetrectcap%
\pgfsetroundjoin%
\pgfsetlinewidth{0.803000pt}%
\definecolor{currentstroke}{rgb}{0.850000,0.850000,0.850000}%
\pgfsetstrokecolor{currentstroke}%
\pgfsetdash{}{0pt}%
\pgfpathmoveto{\pgfqpoint{3.461895in}{0.417642in}}%
\pgfpathlineto{\pgfqpoint{3.461895in}{2.472642in}}%
\pgfusepath{stroke}%
\end{pgfscope}%
\begin{pgfscope}%
\pgfsetbuttcap%
\pgfsetroundjoin%
\definecolor{currentfill}{rgb}{0.000000,0.000000,0.000000}%
\pgfsetfillcolor{currentfill}%
\pgfsetlinewidth{0.602250pt}%
\definecolor{currentstroke}{rgb}{0.000000,0.000000,0.000000}%
\pgfsetstrokecolor{currentstroke}%
\pgfsetdash{}{0pt}%
\pgfsys@defobject{currentmarker}{\pgfqpoint{0.000000in}{-0.027778in}}{\pgfqpoint{0.000000in}{0.000000in}}{%
\pgfpathmoveto{\pgfqpoint{0.000000in}{0.000000in}}%
\pgfpathlineto{\pgfqpoint{0.000000in}{-0.027778in}}%
\pgfusepath{stroke,fill}%
}%
\begin{pgfscope}%
\pgfsys@transformshift{3.461895in}{0.417642in}%
\pgfsys@useobject{currentmarker}{}%
\end{pgfscope}%
\end{pgfscope}%
\begin{pgfscope}%
\pgfpathrectangle{\pgfqpoint{0.589510in}{0.417642in}}{\pgfqpoint{3.437062in}{2.055000in}}%
\pgfusepath{clip}%
\pgfsetrectcap%
\pgfsetroundjoin%
\pgfsetlinewidth{0.803000pt}%
\definecolor{currentstroke}{rgb}{0.850000,0.850000,0.850000}%
\pgfsetstrokecolor{currentstroke}%
\pgfsetdash{}{0pt}%
\pgfpathmoveto{\pgfqpoint{3.559491in}{0.417642in}}%
\pgfpathlineto{\pgfqpoint{3.559491in}{2.472642in}}%
\pgfusepath{stroke}%
\end{pgfscope}%
\begin{pgfscope}%
\pgfsetbuttcap%
\pgfsetroundjoin%
\definecolor{currentfill}{rgb}{0.000000,0.000000,0.000000}%
\pgfsetfillcolor{currentfill}%
\pgfsetlinewidth{0.602250pt}%
\definecolor{currentstroke}{rgb}{0.000000,0.000000,0.000000}%
\pgfsetstrokecolor{currentstroke}%
\pgfsetdash{}{0pt}%
\pgfsys@defobject{currentmarker}{\pgfqpoint{0.000000in}{-0.027778in}}{\pgfqpoint{0.000000in}{0.000000in}}{%
\pgfpathmoveto{\pgfqpoint{0.000000in}{0.000000in}}%
\pgfpathlineto{\pgfqpoint{0.000000in}{-0.027778in}}%
\pgfusepath{stroke,fill}%
}%
\begin{pgfscope}%
\pgfsys@transformshift{3.559491in}{0.417642in}%
\pgfsys@useobject{currentmarker}{}%
\end{pgfscope}%
\end{pgfscope}%
\begin{pgfscope}%
\pgfpathrectangle{\pgfqpoint{0.589510in}{0.417642in}}{\pgfqpoint{3.437062in}{2.055000in}}%
\pgfusepath{clip}%
\pgfsetrectcap%
\pgfsetroundjoin%
\pgfsetlinewidth{0.803000pt}%
\definecolor{currentstroke}{rgb}{0.850000,0.850000,0.850000}%
\pgfsetstrokecolor{currentstroke}%
\pgfsetdash{}{0pt}%
\pgfpathmoveto{\pgfqpoint{3.635192in}{0.417642in}}%
\pgfpathlineto{\pgfqpoint{3.635192in}{2.472642in}}%
\pgfusepath{stroke}%
\end{pgfscope}%
\begin{pgfscope}%
\pgfsetbuttcap%
\pgfsetroundjoin%
\definecolor{currentfill}{rgb}{0.000000,0.000000,0.000000}%
\pgfsetfillcolor{currentfill}%
\pgfsetlinewidth{0.602250pt}%
\definecolor{currentstroke}{rgb}{0.000000,0.000000,0.000000}%
\pgfsetstrokecolor{currentstroke}%
\pgfsetdash{}{0pt}%
\pgfsys@defobject{currentmarker}{\pgfqpoint{0.000000in}{-0.027778in}}{\pgfqpoint{0.000000in}{0.000000in}}{%
\pgfpathmoveto{\pgfqpoint{0.000000in}{0.000000in}}%
\pgfpathlineto{\pgfqpoint{0.000000in}{-0.027778in}}%
\pgfusepath{stroke,fill}%
}%
\begin{pgfscope}%
\pgfsys@transformshift{3.635192in}{0.417642in}%
\pgfsys@useobject{currentmarker}{}%
\end{pgfscope}%
\end{pgfscope}%
\begin{pgfscope}%
\pgfpathrectangle{\pgfqpoint{0.589510in}{0.417642in}}{\pgfqpoint{3.437062in}{2.055000in}}%
\pgfusepath{clip}%
\pgfsetrectcap%
\pgfsetroundjoin%
\pgfsetlinewidth{0.803000pt}%
\definecolor{currentstroke}{rgb}{0.850000,0.850000,0.850000}%
\pgfsetstrokecolor{currentstroke}%
\pgfsetdash{}{0pt}%
\pgfpathmoveto{\pgfqpoint{3.697044in}{0.417642in}}%
\pgfpathlineto{\pgfqpoint{3.697044in}{2.472642in}}%
\pgfusepath{stroke}%
\end{pgfscope}%
\begin{pgfscope}%
\pgfsetbuttcap%
\pgfsetroundjoin%
\definecolor{currentfill}{rgb}{0.000000,0.000000,0.000000}%
\pgfsetfillcolor{currentfill}%
\pgfsetlinewidth{0.602250pt}%
\definecolor{currentstroke}{rgb}{0.000000,0.000000,0.000000}%
\pgfsetstrokecolor{currentstroke}%
\pgfsetdash{}{0pt}%
\pgfsys@defobject{currentmarker}{\pgfqpoint{0.000000in}{-0.027778in}}{\pgfqpoint{0.000000in}{0.000000in}}{%
\pgfpathmoveto{\pgfqpoint{0.000000in}{0.000000in}}%
\pgfpathlineto{\pgfqpoint{0.000000in}{-0.027778in}}%
\pgfusepath{stroke,fill}%
}%
\begin{pgfscope}%
\pgfsys@transformshift{3.697044in}{0.417642in}%
\pgfsys@useobject{currentmarker}{}%
\end{pgfscope}%
\end{pgfscope}%
\begin{pgfscope}%
\pgfpathrectangle{\pgfqpoint{0.589510in}{0.417642in}}{\pgfqpoint{3.437062in}{2.055000in}}%
\pgfusepath{clip}%
\pgfsetrectcap%
\pgfsetroundjoin%
\pgfsetlinewidth{0.803000pt}%
\definecolor{currentstroke}{rgb}{0.850000,0.850000,0.850000}%
\pgfsetstrokecolor{currentstroke}%
\pgfsetdash{}{0pt}%
\pgfpathmoveto{\pgfqpoint{3.749340in}{0.417642in}}%
\pgfpathlineto{\pgfqpoint{3.749340in}{2.472642in}}%
\pgfusepath{stroke}%
\end{pgfscope}%
\begin{pgfscope}%
\pgfsetbuttcap%
\pgfsetroundjoin%
\definecolor{currentfill}{rgb}{0.000000,0.000000,0.000000}%
\pgfsetfillcolor{currentfill}%
\pgfsetlinewidth{0.602250pt}%
\definecolor{currentstroke}{rgb}{0.000000,0.000000,0.000000}%
\pgfsetstrokecolor{currentstroke}%
\pgfsetdash{}{0pt}%
\pgfsys@defobject{currentmarker}{\pgfqpoint{0.000000in}{-0.027778in}}{\pgfqpoint{0.000000in}{0.000000in}}{%
\pgfpathmoveto{\pgfqpoint{0.000000in}{0.000000in}}%
\pgfpathlineto{\pgfqpoint{0.000000in}{-0.027778in}}%
\pgfusepath{stroke,fill}%
}%
\begin{pgfscope}%
\pgfsys@transformshift{3.749340in}{0.417642in}%
\pgfsys@useobject{currentmarker}{}%
\end{pgfscope}%
\end{pgfscope}%
\begin{pgfscope}%
\pgfpathrectangle{\pgfqpoint{0.589510in}{0.417642in}}{\pgfqpoint{3.437062in}{2.055000in}}%
\pgfusepath{clip}%
\pgfsetrectcap%
\pgfsetroundjoin%
\pgfsetlinewidth{0.803000pt}%
\definecolor{currentstroke}{rgb}{0.850000,0.850000,0.850000}%
\pgfsetstrokecolor{currentstroke}%
\pgfsetdash{}{0pt}%
\pgfpathmoveto{\pgfqpoint{3.794640in}{0.417642in}}%
\pgfpathlineto{\pgfqpoint{3.794640in}{2.472642in}}%
\pgfusepath{stroke}%
\end{pgfscope}%
\begin{pgfscope}%
\pgfsetbuttcap%
\pgfsetroundjoin%
\definecolor{currentfill}{rgb}{0.000000,0.000000,0.000000}%
\pgfsetfillcolor{currentfill}%
\pgfsetlinewidth{0.602250pt}%
\definecolor{currentstroke}{rgb}{0.000000,0.000000,0.000000}%
\pgfsetstrokecolor{currentstroke}%
\pgfsetdash{}{0pt}%
\pgfsys@defobject{currentmarker}{\pgfqpoint{0.000000in}{-0.027778in}}{\pgfqpoint{0.000000in}{0.000000in}}{%
\pgfpathmoveto{\pgfqpoint{0.000000in}{0.000000in}}%
\pgfpathlineto{\pgfqpoint{0.000000in}{-0.027778in}}%
\pgfusepath{stroke,fill}%
}%
\begin{pgfscope}%
\pgfsys@transformshift{3.794640in}{0.417642in}%
\pgfsys@useobject{currentmarker}{}%
\end{pgfscope}%
\end{pgfscope}%
\begin{pgfscope}%
\pgfpathrectangle{\pgfqpoint{0.589510in}{0.417642in}}{\pgfqpoint{3.437062in}{2.055000in}}%
\pgfusepath{clip}%
\pgfsetrectcap%
\pgfsetroundjoin%
\pgfsetlinewidth{0.803000pt}%
\definecolor{currentstroke}{rgb}{0.850000,0.850000,0.850000}%
\pgfsetstrokecolor{currentstroke}%
\pgfsetdash{}{0pt}%
\pgfpathmoveto{\pgfqpoint{3.834598in}{0.417642in}}%
\pgfpathlineto{\pgfqpoint{3.834598in}{2.472642in}}%
\pgfusepath{stroke}%
\end{pgfscope}%
\begin{pgfscope}%
\pgfsetbuttcap%
\pgfsetroundjoin%
\definecolor{currentfill}{rgb}{0.000000,0.000000,0.000000}%
\pgfsetfillcolor{currentfill}%
\pgfsetlinewidth{0.602250pt}%
\definecolor{currentstroke}{rgb}{0.000000,0.000000,0.000000}%
\pgfsetstrokecolor{currentstroke}%
\pgfsetdash{}{0pt}%
\pgfsys@defobject{currentmarker}{\pgfqpoint{0.000000in}{-0.027778in}}{\pgfqpoint{0.000000in}{0.000000in}}{%
\pgfpathmoveto{\pgfqpoint{0.000000in}{0.000000in}}%
\pgfpathlineto{\pgfqpoint{0.000000in}{-0.027778in}}%
\pgfusepath{stroke,fill}%
}%
\begin{pgfscope}%
\pgfsys@transformshift{3.834598in}{0.417642in}%
\pgfsys@useobject{currentmarker}{}%
\end{pgfscope}%
\end{pgfscope}%
\begin{pgfscope}%
\definecolor{textcolor}{rgb}{0.000000,0.000000,0.000000}%
\pgfsetstrokecolor{textcolor}%
\pgfsetfillcolor{textcolor}%
\pgftext[x=2.308041in,y=0.165003in,,top]{\color{textcolor}\rmfamily\fontsize{10.000000}{12.000000}\selectfont \(\displaystyle \tau\) in \unit{\second}}%
\end{pgfscope}%
\begin{pgfscope}%
\pgfpathrectangle{\pgfqpoint{0.589510in}{0.417642in}}{\pgfqpoint{3.437062in}{2.055000in}}%
\pgfusepath{clip}%
\pgfsetrectcap%
\pgfsetroundjoin%
\pgfsetlinewidth{0.803000pt}%
\definecolor{currentstroke}{rgb}{0.450000,0.450000,0.450000}%
\pgfsetstrokecolor{currentstroke}%
\pgfsetdash{}{0pt}%
\pgfpathmoveto{\pgfqpoint{0.589510in}{1.728810in}}%
\pgfpathlineto{\pgfqpoint{4.026572in}{1.728810in}}%
\pgfusepath{stroke}%
\end{pgfscope}%
\begin{pgfscope}%
\pgfsetbuttcap%
\pgfsetroundjoin%
\definecolor{currentfill}{rgb}{0.000000,0.000000,0.000000}%
\pgfsetfillcolor{currentfill}%
\pgfsetlinewidth{0.803000pt}%
\definecolor{currentstroke}{rgb}{0.000000,0.000000,0.000000}%
\pgfsetstrokecolor{currentstroke}%
\pgfsetdash{}{0pt}%
\pgfsys@defobject{currentmarker}{\pgfqpoint{-0.048611in}{0.000000in}}{\pgfqpoint{-0.000000in}{0.000000in}}{%
\pgfpathmoveto{\pgfqpoint{-0.000000in}{0.000000in}}%
\pgfpathlineto{\pgfqpoint{-0.048611in}{0.000000in}}%
\pgfusepath{stroke,fill}%
}%
\begin{pgfscope}%
\pgfsys@transformshift{0.589510in}{1.728810in}%
\pgfsys@useobject{currentmarker}{}%
\end{pgfscope}%
\end{pgfscope}%
\begin{pgfscope}%
\definecolor{textcolor}{rgb}{0.000000,0.000000,0.000000}%
\pgfsetstrokecolor{textcolor}%
\pgfsetfillcolor{textcolor}%
\pgftext[x=0.236114in, y=1.689657in, left, base]{\color{textcolor}\rmfamily\fontsize{8.000000}{9.600000}\selectfont \(\displaystyle {10^{-1}}\)}%
\end{pgfscope}%
\begin{pgfscope}%
\pgfpathrectangle{\pgfqpoint{0.589510in}{0.417642in}}{\pgfqpoint{3.437062in}{2.055000in}}%
\pgfusepath{clip}%
\pgfsetrectcap%
\pgfsetroundjoin%
\pgfsetlinewidth{0.803000pt}%
\definecolor{currentstroke}{rgb}{0.850000,0.850000,0.850000}%
\pgfsetstrokecolor{currentstroke}%
\pgfsetdash{}{0pt}%
\pgfpathmoveto{\pgfqpoint{0.589510in}{0.795369in}}%
\pgfpathlineto{\pgfqpoint{4.026572in}{0.795369in}}%
\pgfusepath{stroke}%
\end{pgfscope}%
\begin{pgfscope}%
\pgfsetbuttcap%
\pgfsetroundjoin%
\definecolor{currentfill}{rgb}{0.000000,0.000000,0.000000}%
\pgfsetfillcolor{currentfill}%
\pgfsetlinewidth{0.602250pt}%
\definecolor{currentstroke}{rgb}{0.000000,0.000000,0.000000}%
\pgfsetstrokecolor{currentstroke}%
\pgfsetdash{}{0pt}%
\pgfsys@defobject{currentmarker}{\pgfqpoint{-0.027778in}{0.000000in}}{\pgfqpoint{-0.000000in}{0.000000in}}{%
\pgfpathmoveto{\pgfqpoint{-0.000000in}{0.000000in}}%
\pgfpathlineto{\pgfqpoint{-0.027778in}{0.000000in}}%
\pgfusepath{stroke,fill}%
}%
\begin{pgfscope}%
\pgfsys@transformshift{0.589510in}{0.795369in}%
\pgfsys@useobject{currentmarker}{}%
\end{pgfscope}%
\end{pgfscope}%
\begin{pgfscope}%
\pgfpathrectangle{\pgfqpoint{0.589510in}{0.417642in}}{\pgfqpoint{3.437062in}{2.055000in}}%
\pgfusepath{clip}%
\pgfsetrectcap%
\pgfsetroundjoin%
\pgfsetlinewidth{0.803000pt}%
\definecolor{currentstroke}{rgb}{0.850000,0.850000,0.850000}%
\pgfsetstrokecolor{currentstroke}%
\pgfsetdash{}{0pt}%
\pgfpathmoveto{\pgfqpoint{0.589510in}{1.030531in}}%
\pgfpathlineto{\pgfqpoint{4.026572in}{1.030531in}}%
\pgfusepath{stroke}%
\end{pgfscope}%
\begin{pgfscope}%
\pgfsetbuttcap%
\pgfsetroundjoin%
\definecolor{currentfill}{rgb}{0.000000,0.000000,0.000000}%
\pgfsetfillcolor{currentfill}%
\pgfsetlinewidth{0.602250pt}%
\definecolor{currentstroke}{rgb}{0.000000,0.000000,0.000000}%
\pgfsetstrokecolor{currentstroke}%
\pgfsetdash{}{0pt}%
\pgfsys@defobject{currentmarker}{\pgfqpoint{-0.027778in}{0.000000in}}{\pgfqpoint{-0.000000in}{0.000000in}}{%
\pgfpathmoveto{\pgfqpoint{-0.000000in}{0.000000in}}%
\pgfpathlineto{\pgfqpoint{-0.027778in}{0.000000in}}%
\pgfusepath{stroke,fill}%
}%
\begin{pgfscope}%
\pgfsys@transformshift{0.589510in}{1.030531in}%
\pgfsys@useobject{currentmarker}{}%
\end{pgfscope}%
\end{pgfscope}%
\begin{pgfscope}%
\pgfpathrectangle{\pgfqpoint{0.589510in}{0.417642in}}{\pgfqpoint{3.437062in}{2.055000in}}%
\pgfusepath{clip}%
\pgfsetrectcap%
\pgfsetroundjoin%
\pgfsetlinewidth{0.803000pt}%
\definecolor{currentstroke}{rgb}{0.850000,0.850000,0.850000}%
\pgfsetstrokecolor{currentstroke}%
\pgfsetdash{}{0pt}%
\pgfpathmoveto{\pgfqpoint{0.589510in}{1.197380in}}%
\pgfpathlineto{\pgfqpoint{4.026572in}{1.197380in}}%
\pgfusepath{stroke}%
\end{pgfscope}%
\begin{pgfscope}%
\pgfsetbuttcap%
\pgfsetroundjoin%
\definecolor{currentfill}{rgb}{0.000000,0.000000,0.000000}%
\pgfsetfillcolor{currentfill}%
\pgfsetlinewidth{0.602250pt}%
\definecolor{currentstroke}{rgb}{0.000000,0.000000,0.000000}%
\pgfsetstrokecolor{currentstroke}%
\pgfsetdash{}{0pt}%
\pgfsys@defobject{currentmarker}{\pgfqpoint{-0.027778in}{0.000000in}}{\pgfqpoint{-0.000000in}{0.000000in}}{%
\pgfpathmoveto{\pgfqpoint{-0.000000in}{0.000000in}}%
\pgfpathlineto{\pgfqpoint{-0.027778in}{0.000000in}}%
\pgfusepath{stroke,fill}%
}%
\begin{pgfscope}%
\pgfsys@transformshift{0.589510in}{1.197380in}%
\pgfsys@useobject{currentmarker}{}%
\end{pgfscope}%
\end{pgfscope}%
\begin{pgfscope}%
\pgfpathrectangle{\pgfqpoint{0.589510in}{0.417642in}}{\pgfqpoint{3.437062in}{2.055000in}}%
\pgfusepath{clip}%
\pgfsetrectcap%
\pgfsetroundjoin%
\pgfsetlinewidth{0.803000pt}%
\definecolor{currentstroke}{rgb}{0.850000,0.850000,0.850000}%
\pgfsetstrokecolor{currentstroke}%
\pgfsetdash{}{0pt}%
\pgfpathmoveto{\pgfqpoint{0.589510in}{1.326799in}}%
\pgfpathlineto{\pgfqpoint{4.026572in}{1.326799in}}%
\pgfusepath{stroke}%
\end{pgfscope}%
\begin{pgfscope}%
\pgfsetbuttcap%
\pgfsetroundjoin%
\definecolor{currentfill}{rgb}{0.000000,0.000000,0.000000}%
\pgfsetfillcolor{currentfill}%
\pgfsetlinewidth{0.602250pt}%
\definecolor{currentstroke}{rgb}{0.000000,0.000000,0.000000}%
\pgfsetstrokecolor{currentstroke}%
\pgfsetdash{}{0pt}%
\pgfsys@defobject{currentmarker}{\pgfqpoint{-0.027778in}{0.000000in}}{\pgfqpoint{-0.000000in}{0.000000in}}{%
\pgfpathmoveto{\pgfqpoint{-0.000000in}{0.000000in}}%
\pgfpathlineto{\pgfqpoint{-0.027778in}{0.000000in}}%
\pgfusepath{stroke,fill}%
}%
\begin{pgfscope}%
\pgfsys@transformshift{0.589510in}{1.326799in}%
\pgfsys@useobject{currentmarker}{}%
\end{pgfscope}%
\end{pgfscope}%
\begin{pgfscope}%
\pgfpathrectangle{\pgfqpoint{0.589510in}{0.417642in}}{\pgfqpoint{3.437062in}{2.055000in}}%
\pgfusepath{clip}%
\pgfsetrectcap%
\pgfsetroundjoin%
\pgfsetlinewidth{0.803000pt}%
\definecolor{currentstroke}{rgb}{0.850000,0.850000,0.850000}%
\pgfsetstrokecolor{currentstroke}%
\pgfsetdash{}{0pt}%
\pgfpathmoveto{\pgfqpoint{0.589510in}{1.432542in}}%
\pgfpathlineto{\pgfqpoint{4.026572in}{1.432542in}}%
\pgfusepath{stroke}%
\end{pgfscope}%
\begin{pgfscope}%
\pgfsetbuttcap%
\pgfsetroundjoin%
\definecolor{currentfill}{rgb}{0.000000,0.000000,0.000000}%
\pgfsetfillcolor{currentfill}%
\pgfsetlinewidth{0.602250pt}%
\definecolor{currentstroke}{rgb}{0.000000,0.000000,0.000000}%
\pgfsetstrokecolor{currentstroke}%
\pgfsetdash{}{0pt}%
\pgfsys@defobject{currentmarker}{\pgfqpoint{-0.027778in}{0.000000in}}{\pgfqpoint{-0.000000in}{0.000000in}}{%
\pgfpathmoveto{\pgfqpoint{-0.000000in}{0.000000in}}%
\pgfpathlineto{\pgfqpoint{-0.027778in}{0.000000in}}%
\pgfusepath{stroke,fill}%
}%
\begin{pgfscope}%
\pgfsys@transformshift{0.589510in}{1.432542in}%
\pgfsys@useobject{currentmarker}{}%
\end{pgfscope}%
\end{pgfscope}%
\begin{pgfscope}%
\pgfpathrectangle{\pgfqpoint{0.589510in}{0.417642in}}{\pgfqpoint{3.437062in}{2.055000in}}%
\pgfusepath{clip}%
\pgfsetrectcap%
\pgfsetroundjoin%
\pgfsetlinewidth{0.803000pt}%
\definecolor{currentstroke}{rgb}{0.850000,0.850000,0.850000}%
\pgfsetstrokecolor{currentstroke}%
\pgfsetdash{}{0pt}%
\pgfpathmoveto{\pgfqpoint{0.589510in}{1.521946in}}%
\pgfpathlineto{\pgfqpoint{4.026572in}{1.521946in}}%
\pgfusepath{stroke}%
\end{pgfscope}%
\begin{pgfscope}%
\pgfsetbuttcap%
\pgfsetroundjoin%
\definecolor{currentfill}{rgb}{0.000000,0.000000,0.000000}%
\pgfsetfillcolor{currentfill}%
\pgfsetlinewidth{0.602250pt}%
\definecolor{currentstroke}{rgb}{0.000000,0.000000,0.000000}%
\pgfsetstrokecolor{currentstroke}%
\pgfsetdash{}{0pt}%
\pgfsys@defobject{currentmarker}{\pgfqpoint{-0.027778in}{0.000000in}}{\pgfqpoint{-0.000000in}{0.000000in}}{%
\pgfpathmoveto{\pgfqpoint{-0.000000in}{0.000000in}}%
\pgfpathlineto{\pgfqpoint{-0.027778in}{0.000000in}}%
\pgfusepath{stroke,fill}%
}%
\begin{pgfscope}%
\pgfsys@transformshift{0.589510in}{1.521946in}%
\pgfsys@useobject{currentmarker}{}%
\end{pgfscope}%
\end{pgfscope}%
\begin{pgfscope}%
\pgfpathrectangle{\pgfqpoint{0.589510in}{0.417642in}}{\pgfqpoint{3.437062in}{2.055000in}}%
\pgfusepath{clip}%
\pgfsetrectcap%
\pgfsetroundjoin%
\pgfsetlinewidth{0.803000pt}%
\definecolor{currentstroke}{rgb}{0.850000,0.850000,0.850000}%
\pgfsetstrokecolor{currentstroke}%
\pgfsetdash{}{0pt}%
\pgfpathmoveto{\pgfqpoint{0.589510in}{1.599392in}}%
\pgfpathlineto{\pgfqpoint{4.026572in}{1.599392in}}%
\pgfusepath{stroke}%
\end{pgfscope}%
\begin{pgfscope}%
\pgfsetbuttcap%
\pgfsetroundjoin%
\definecolor{currentfill}{rgb}{0.000000,0.000000,0.000000}%
\pgfsetfillcolor{currentfill}%
\pgfsetlinewidth{0.602250pt}%
\definecolor{currentstroke}{rgb}{0.000000,0.000000,0.000000}%
\pgfsetstrokecolor{currentstroke}%
\pgfsetdash{}{0pt}%
\pgfsys@defobject{currentmarker}{\pgfqpoint{-0.027778in}{0.000000in}}{\pgfqpoint{-0.000000in}{0.000000in}}{%
\pgfpathmoveto{\pgfqpoint{-0.000000in}{0.000000in}}%
\pgfpathlineto{\pgfqpoint{-0.027778in}{0.000000in}}%
\pgfusepath{stroke,fill}%
}%
\begin{pgfscope}%
\pgfsys@transformshift{0.589510in}{1.599392in}%
\pgfsys@useobject{currentmarker}{}%
\end{pgfscope}%
\end{pgfscope}%
\begin{pgfscope}%
\pgfpathrectangle{\pgfqpoint{0.589510in}{0.417642in}}{\pgfqpoint{3.437062in}{2.055000in}}%
\pgfusepath{clip}%
\pgfsetrectcap%
\pgfsetroundjoin%
\pgfsetlinewidth{0.803000pt}%
\definecolor{currentstroke}{rgb}{0.850000,0.850000,0.850000}%
\pgfsetstrokecolor{currentstroke}%
\pgfsetdash{}{0pt}%
\pgfpathmoveto{\pgfqpoint{0.589510in}{1.667703in}}%
\pgfpathlineto{\pgfqpoint{4.026572in}{1.667703in}}%
\pgfusepath{stroke}%
\end{pgfscope}%
\begin{pgfscope}%
\pgfsetbuttcap%
\pgfsetroundjoin%
\definecolor{currentfill}{rgb}{0.000000,0.000000,0.000000}%
\pgfsetfillcolor{currentfill}%
\pgfsetlinewidth{0.602250pt}%
\definecolor{currentstroke}{rgb}{0.000000,0.000000,0.000000}%
\pgfsetstrokecolor{currentstroke}%
\pgfsetdash{}{0pt}%
\pgfsys@defobject{currentmarker}{\pgfqpoint{-0.027778in}{0.000000in}}{\pgfqpoint{-0.000000in}{0.000000in}}{%
\pgfpathmoveto{\pgfqpoint{-0.000000in}{0.000000in}}%
\pgfpathlineto{\pgfqpoint{-0.027778in}{0.000000in}}%
\pgfusepath{stroke,fill}%
}%
\begin{pgfscope}%
\pgfsys@transformshift{0.589510in}{1.667703in}%
\pgfsys@useobject{currentmarker}{}%
\end{pgfscope}%
\end{pgfscope}%
\begin{pgfscope}%
\pgfpathrectangle{\pgfqpoint{0.589510in}{0.417642in}}{\pgfqpoint{3.437062in}{2.055000in}}%
\pgfusepath{clip}%
\pgfsetrectcap%
\pgfsetroundjoin%
\pgfsetlinewidth{0.803000pt}%
\definecolor{currentstroke}{rgb}{0.850000,0.850000,0.850000}%
\pgfsetstrokecolor{currentstroke}%
\pgfsetdash{}{0pt}%
\pgfpathmoveto{\pgfqpoint{0.589510in}{2.130821in}}%
\pgfpathlineto{\pgfqpoint{4.026572in}{2.130821in}}%
\pgfusepath{stroke}%
\end{pgfscope}%
\begin{pgfscope}%
\pgfsetbuttcap%
\pgfsetroundjoin%
\definecolor{currentfill}{rgb}{0.000000,0.000000,0.000000}%
\pgfsetfillcolor{currentfill}%
\pgfsetlinewidth{0.602250pt}%
\definecolor{currentstroke}{rgb}{0.000000,0.000000,0.000000}%
\pgfsetstrokecolor{currentstroke}%
\pgfsetdash{}{0pt}%
\pgfsys@defobject{currentmarker}{\pgfqpoint{-0.027778in}{0.000000in}}{\pgfqpoint{-0.000000in}{0.000000in}}{%
\pgfpathmoveto{\pgfqpoint{-0.000000in}{0.000000in}}%
\pgfpathlineto{\pgfqpoint{-0.027778in}{0.000000in}}%
\pgfusepath{stroke,fill}%
}%
\begin{pgfscope}%
\pgfsys@transformshift{0.589510in}{2.130821in}%
\pgfsys@useobject{currentmarker}{}%
\end{pgfscope}%
\end{pgfscope}%
\begin{pgfscope}%
\pgfpathrectangle{\pgfqpoint{0.589510in}{0.417642in}}{\pgfqpoint{3.437062in}{2.055000in}}%
\pgfusepath{clip}%
\pgfsetrectcap%
\pgfsetroundjoin%
\pgfsetlinewidth{0.803000pt}%
\definecolor{currentstroke}{rgb}{0.850000,0.850000,0.850000}%
\pgfsetstrokecolor{currentstroke}%
\pgfsetdash{}{0pt}%
\pgfpathmoveto{\pgfqpoint{0.589510in}{2.365983in}}%
\pgfpathlineto{\pgfqpoint{4.026572in}{2.365983in}}%
\pgfusepath{stroke}%
\end{pgfscope}%
\begin{pgfscope}%
\pgfsetbuttcap%
\pgfsetroundjoin%
\definecolor{currentfill}{rgb}{0.000000,0.000000,0.000000}%
\pgfsetfillcolor{currentfill}%
\pgfsetlinewidth{0.602250pt}%
\definecolor{currentstroke}{rgb}{0.000000,0.000000,0.000000}%
\pgfsetstrokecolor{currentstroke}%
\pgfsetdash{}{0pt}%
\pgfsys@defobject{currentmarker}{\pgfqpoint{-0.027778in}{0.000000in}}{\pgfqpoint{-0.000000in}{0.000000in}}{%
\pgfpathmoveto{\pgfqpoint{-0.000000in}{0.000000in}}%
\pgfpathlineto{\pgfqpoint{-0.027778in}{0.000000in}}%
\pgfusepath{stroke,fill}%
}%
\begin{pgfscope}%
\pgfsys@transformshift{0.589510in}{2.365983in}%
\pgfsys@useobject{currentmarker}{}%
\end{pgfscope}%
\end{pgfscope}%
\begin{pgfscope}%
\definecolor{textcolor}{rgb}{0.000000,0.000000,0.000000}%
\pgfsetstrokecolor{textcolor}%
\pgfsetfillcolor{textcolor}%
\pgftext[x=0.180559in,y=1.445142in,,bottom,rotate=90.000000]{\color{textcolor}\rmfamily\fontsize{10.000000}{12.000000}\selectfont ADEV \(\displaystyle \sigma_A(\tau)\)}%
\end{pgfscope}%
\begin{pgfscope}%
\pgfpathrectangle{\pgfqpoint{0.589510in}{0.417642in}}{\pgfqpoint{3.437062in}{2.055000in}}%
\pgfusepath{clip}%
\pgfsetbuttcap%
\pgfsetroundjoin%
\pgfsetlinewidth{1.505625pt}%
\definecolor{currentstroke}{rgb}{0.003922,0.450980,0.698039}%
\pgfsetstrokecolor{currentstroke}%
\pgfsetdash{{5.550000pt}{2.400000pt}}{0.000000pt}%
\pgfpathmoveto{\pgfqpoint{0.745740in}{1.559906in}}%
\pgfpathlineto{\pgfqpoint{0.883294in}{1.665824in}}%
\pgfpathlineto{\pgfqpoint{1.056591in}{1.790983in}}%
\pgfpathlineto{\pgfqpoint{1.216039in}{1.893684in}}%
\pgfpathlineto{\pgfqpoint{1.380747in}{1.980727in}}%
\pgfpathlineto{\pgfqpoint{1.559224in}{2.043219in}}%
\pgfpathlineto{\pgfqpoint{1.726296in}{2.061342in}}%
\pgfpathlineto{\pgfqpoint{1.893892in}{2.033564in}}%
\pgfpathlineto{\pgfqpoint{2.059042in}{1.963607in}}%
\pgfpathlineto{\pgfqpoint{2.223750in}{1.863241in}}%
\pgfpathlineto{\pgfqpoint{2.389127in}{1.744648in}}%
\pgfpathlineto{\pgfqpoint{2.554041in}{1.616807in}}%
\pgfpathlineto{\pgfqpoint{2.718686in}{1.483847in}}%
\pgfpathlineto{\pgfqpoint{2.883588in}{1.347581in}}%
\pgfpathlineto{\pgfqpoint{3.047937in}{1.209936in}}%
\pgfpathlineto{\pgfqpoint{3.212425in}{1.071070in}}%
\pgfpathlineto{\pgfqpoint{3.376951in}{0.931498in}}%
\pgfpathlineto{\pgfqpoint{3.541419in}{0.791565in}}%
\pgfpathlineto{\pgfqpoint{3.705862in}{0.651400in}}%
\pgfpathlineto{\pgfqpoint{3.870342in}{0.511051in}}%
\pgfusepath{stroke}%
\end{pgfscope}%
\begin{pgfscope}%
\pgfpathrectangle{\pgfqpoint{0.589510in}{0.417642in}}{\pgfqpoint{3.437062in}{2.055000in}}%
\pgfusepath{clip}%
\pgfsetbuttcap%
\pgfsetroundjoin%
\definecolor{currentfill}{rgb}{0.003922,0.450980,0.698039}%
\pgfsetfillcolor{currentfill}%
\pgfsetlinewidth{1.003750pt}%
\definecolor{currentstroke}{rgb}{0.003922,0.450980,0.698039}%
\pgfsetstrokecolor{currentstroke}%
\pgfsetdash{}{0pt}%
\pgfsys@defobject{currentmarker}{\pgfqpoint{-0.020833in}{-0.020833in}}{\pgfqpoint{0.020833in}{0.020833in}}{%
\pgfpathmoveto{\pgfqpoint{0.000000in}{-0.020833in}}%
\pgfpathcurveto{\pgfqpoint{0.005525in}{-0.020833in}}{\pgfqpoint{0.010825in}{-0.018638in}}{\pgfqpoint{0.014731in}{-0.014731in}}%
\pgfpathcurveto{\pgfqpoint{0.018638in}{-0.010825in}}{\pgfqpoint{0.020833in}{-0.005525in}}{\pgfqpoint{0.020833in}{0.000000in}}%
\pgfpathcurveto{\pgfqpoint{0.020833in}{0.005525in}}{\pgfqpoint{0.018638in}{0.010825in}}{\pgfqpoint{0.014731in}{0.014731in}}%
\pgfpathcurveto{\pgfqpoint{0.010825in}{0.018638in}}{\pgfqpoint{0.005525in}{0.020833in}}{\pgfqpoint{0.000000in}{0.020833in}}%
\pgfpathcurveto{\pgfqpoint{-0.005525in}{0.020833in}}{\pgfqpoint{-0.010825in}{0.018638in}}{\pgfqpoint{-0.014731in}{0.014731in}}%
\pgfpathcurveto{\pgfqpoint{-0.018638in}{0.010825in}}{\pgfqpoint{-0.020833in}{0.005525in}}{\pgfqpoint{-0.020833in}{0.000000in}}%
\pgfpathcurveto{\pgfqpoint{-0.020833in}{-0.005525in}}{\pgfqpoint{-0.018638in}{-0.010825in}}{\pgfqpoint{-0.014731in}{-0.014731in}}%
\pgfpathcurveto{\pgfqpoint{-0.010825in}{-0.018638in}}{\pgfqpoint{-0.005525in}{-0.020833in}}{\pgfqpoint{0.000000in}{-0.020833in}}%
\pgfpathlineto{\pgfqpoint{0.000000in}{-0.020833in}}%
\pgfpathclose%
\pgfusepath{stroke,fill}%
}%
\begin{pgfscope}%
\pgfsys@transformshift{0.745740in}{1.595988in}%
\pgfsys@useobject{currentmarker}{}%
\end{pgfscope}%
\begin{pgfscope}%
\pgfsys@transformshift{0.883294in}{1.682684in}%
\pgfsys@useobject{currentmarker}{}%
\end{pgfscope}%
\begin{pgfscope}%
\pgfsys@transformshift{1.056591in}{1.797122in}%
\pgfsys@useobject{currentmarker}{}%
\end{pgfscope}%
\begin{pgfscope}%
\pgfsys@transformshift{1.216039in}{1.895780in}%
\pgfsys@useobject{currentmarker}{}%
\end{pgfscope}%
\begin{pgfscope}%
\pgfsys@transformshift{1.380747in}{1.980802in}%
\pgfsys@useobject{currentmarker}{}%
\end{pgfscope}%
\begin{pgfscope}%
\pgfsys@transformshift{1.559224in}{2.042134in}%
\pgfsys@useobject{currentmarker}{}%
\end{pgfscope}%
\begin{pgfscope}%
\pgfsys@transformshift{1.726296in}{2.059848in}%
\pgfsys@useobject{currentmarker}{}%
\end{pgfscope}%
\begin{pgfscope}%
\pgfsys@transformshift{1.893892in}{2.032510in}%
\pgfsys@useobject{currentmarker}{}%
\end{pgfscope}%
\begin{pgfscope}%
\pgfsys@transformshift{2.059042in}{1.963610in}%
\pgfsys@useobject{currentmarker}{}%
\end{pgfscope}%
\begin{pgfscope}%
\pgfsys@transformshift{2.223750in}{1.862324in}%
\pgfsys@useobject{currentmarker}{}%
\end{pgfscope}%
\begin{pgfscope}%
\pgfsys@transformshift{2.389127in}{1.742498in}%
\pgfsys@useobject{currentmarker}{}%
\end{pgfscope}%
\begin{pgfscope}%
\pgfsys@transformshift{2.554041in}{1.614059in}%
\pgfsys@useobject{currentmarker}{}%
\end{pgfscope}%
\begin{pgfscope}%
\pgfsys@transformshift{2.718686in}{1.480139in}%
\pgfsys@useobject{currentmarker}{}%
\end{pgfscope}%
\begin{pgfscope}%
\pgfsys@transformshift{2.883588in}{1.344195in}%
\pgfsys@useobject{currentmarker}{}%
\end{pgfscope}%
\begin{pgfscope}%
\pgfsys@transformshift{3.047937in}{1.211691in}%
\pgfsys@useobject{currentmarker}{}%
\end{pgfscope}%
\begin{pgfscope}%
\pgfsys@transformshift{3.212425in}{1.071840in}%
\pgfsys@useobject{currentmarker}{}%
\end{pgfscope}%
\begin{pgfscope}%
\pgfsys@transformshift{3.376951in}{0.932165in}%
\pgfsys@useobject{currentmarker}{}%
\end{pgfscope}%
\begin{pgfscope}%
\pgfsys@transformshift{3.541419in}{0.793422in}%
\pgfsys@useobject{currentmarker}{}%
\end{pgfscope}%
\begin{pgfscope}%
\pgfsys@transformshift{3.705862in}{0.660521in}%
\pgfsys@useobject{currentmarker}{}%
\end{pgfscope}%
\begin{pgfscope}%
\pgfsys@transformshift{3.870342in}{0.518723in}%
\pgfsys@useobject{currentmarker}{}%
\end{pgfscope}%
\end{pgfscope}%
\begin{pgfscope}%
\pgfpathrectangle{\pgfqpoint{0.589510in}{0.417642in}}{\pgfqpoint{3.437062in}{2.055000in}}%
\pgfusepath{clip}%
\pgfsetbuttcap%
\pgfsetroundjoin%
\pgfsetlinewidth{1.505625pt}%
\definecolor{currentstroke}{rgb}{0.007843,0.619608,0.450980}%
\pgfsetstrokecolor{currentstroke}%
\pgfsetdash{{5.550000pt}{2.400000pt}}{0.000000pt}%
\pgfpathmoveto{\pgfqpoint{0.745740in}{1.405882in}}%
\pgfpathlineto{\pgfqpoint{0.883294in}{1.521298in}}%
\pgfpathlineto{\pgfqpoint{1.056591in}{1.665114in}}%
\pgfpathlineto{\pgfqpoint{1.216039in}{1.794963in}}%
\pgfpathlineto{\pgfqpoint{1.380747in}{1.925088in}}%
\pgfpathlineto{\pgfqpoint{1.559224in}{2.058696in}}%
\pgfpathlineto{\pgfqpoint{1.726296in}{2.172645in}}%
\pgfpathlineto{\pgfqpoint{1.893892in}{2.270086in}}%
\pgfpathlineto{\pgfqpoint{2.059042in}{2.341547in}}%
\pgfpathlineto{\pgfqpoint{2.223750in}{2.379022in}}%
\pgfpathlineto{\pgfqpoint{2.389127in}{2.374190in}}%
\pgfpathlineto{\pgfqpoint{2.554041in}{2.324611in}}%
\pgfpathlineto{\pgfqpoint{2.718686in}{2.237995in}}%
\pgfpathlineto{\pgfqpoint{2.883588in}{2.127384in}}%
\pgfpathlineto{\pgfqpoint{3.047937in}{2.004124in}}%
\pgfpathlineto{\pgfqpoint{3.212425in}{1.873652in}}%
\pgfpathlineto{\pgfqpoint{3.376951in}{1.739083in}}%
\pgfpathlineto{\pgfqpoint{3.541419in}{1.602169in}}%
\pgfpathlineto{\pgfqpoint{3.705862in}{1.463842in}}%
\pgfpathlineto{\pgfqpoint{3.870342in}{1.324616in}}%
\pgfusepath{stroke}%
\end{pgfscope}%
\begin{pgfscope}%
\pgfpathrectangle{\pgfqpoint{0.589510in}{0.417642in}}{\pgfqpoint{3.437062in}{2.055000in}}%
\pgfusepath{clip}%
\pgfsetbuttcap%
\pgfsetroundjoin%
\definecolor{currentfill}{rgb}{0.007843,0.619608,0.450980}%
\pgfsetfillcolor{currentfill}%
\pgfsetlinewidth{1.003750pt}%
\definecolor{currentstroke}{rgb}{0.007843,0.619608,0.450980}%
\pgfsetstrokecolor{currentstroke}%
\pgfsetdash{}{0pt}%
\pgfsys@defobject{currentmarker}{\pgfqpoint{-0.020833in}{-0.020833in}}{\pgfqpoint{0.020833in}{0.020833in}}{%
\pgfpathmoveto{\pgfqpoint{0.000000in}{-0.020833in}}%
\pgfpathcurveto{\pgfqpoint{0.005525in}{-0.020833in}}{\pgfqpoint{0.010825in}{-0.018638in}}{\pgfqpoint{0.014731in}{-0.014731in}}%
\pgfpathcurveto{\pgfqpoint{0.018638in}{-0.010825in}}{\pgfqpoint{0.020833in}{-0.005525in}}{\pgfqpoint{0.020833in}{0.000000in}}%
\pgfpathcurveto{\pgfqpoint{0.020833in}{0.005525in}}{\pgfqpoint{0.018638in}{0.010825in}}{\pgfqpoint{0.014731in}{0.014731in}}%
\pgfpathcurveto{\pgfqpoint{0.010825in}{0.018638in}}{\pgfqpoint{0.005525in}{0.020833in}}{\pgfqpoint{0.000000in}{0.020833in}}%
\pgfpathcurveto{\pgfqpoint{-0.005525in}{0.020833in}}{\pgfqpoint{-0.010825in}{0.018638in}}{\pgfqpoint{-0.014731in}{0.014731in}}%
\pgfpathcurveto{\pgfqpoint{-0.018638in}{0.010825in}}{\pgfqpoint{-0.020833in}{0.005525in}}{\pgfqpoint{-0.020833in}{0.000000in}}%
\pgfpathcurveto{\pgfqpoint{-0.020833in}{-0.005525in}}{\pgfqpoint{-0.018638in}{-0.010825in}}{\pgfqpoint{-0.014731in}{-0.014731in}}%
\pgfpathcurveto{\pgfqpoint{-0.010825in}{-0.018638in}}{\pgfqpoint{-0.005525in}{-0.020833in}}{\pgfqpoint{0.000000in}{-0.020833in}}%
\pgfpathlineto{\pgfqpoint{0.000000in}{-0.020833in}}%
\pgfpathclose%
\pgfusepath{stroke,fill}%
}%
\begin{pgfscope}%
\pgfsys@transformshift{0.745740in}{1.439250in}%
\pgfsys@useobject{currentmarker}{}%
\end{pgfscope}%
\begin{pgfscope}%
\pgfsys@transformshift{0.883294in}{1.536085in}%
\pgfsys@useobject{currentmarker}{}%
\end{pgfscope}%
\begin{pgfscope}%
\pgfsys@transformshift{1.056591in}{1.669922in}%
\pgfsys@useobject{currentmarker}{}%
\end{pgfscope}%
\begin{pgfscope}%
\pgfsys@transformshift{1.216039in}{1.796284in}%
\pgfsys@useobject{currentmarker}{}%
\end{pgfscope}%
\begin{pgfscope}%
\pgfsys@transformshift{1.380747in}{1.925105in}%
\pgfsys@useobject{currentmarker}{}%
\end{pgfscope}%
\begin{pgfscope}%
\pgfsys@transformshift{1.559224in}{2.058323in}%
\pgfsys@useobject{currentmarker}{}%
\end{pgfscope}%
\begin{pgfscope}%
\pgfsys@transformshift{1.726296in}{2.172197in}%
\pgfsys@useobject{currentmarker}{}%
\end{pgfscope}%
\begin{pgfscope}%
\pgfsys@transformshift{1.893892in}{2.269642in}%
\pgfsys@useobject{currentmarker}{}%
\end{pgfscope}%
\begin{pgfscope}%
\pgfsys@transformshift{2.059042in}{2.341414in}%
\pgfsys@useobject{currentmarker}{}%
\end{pgfscope}%
\begin{pgfscope}%
\pgfsys@transformshift{2.223750in}{2.379233in}%
\pgfsys@useobject{currentmarker}{}%
\end{pgfscope}%
\begin{pgfscope}%
\pgfsys@transformshift{2.389127in}{2.374711in}%
\pgfsys@useobject{currentmarker}{}%
\end{pgfscope}%
\begin{pgfscope}%
\pgfsys@transformshift{2.554041in}{2.326537in}%
\pgfsys@useobject{currentmarker}{}%
\end{pgfscope}%
\begin{pgfscope}%
\pgfsys@transformshift{2.718686in}{2.239534in}%
\pgfsys@useobject{currentmarker}{}%
\end{pgfscope}%
\begin{pgfscope}%
\pgfsys@transformshift{2.883588in}{2.125773in}%
\pgfsys@useobject{currentmarker}{}%
\end{pgfscope}%
\begin{pgfscope}%
\pgfsys@transformshift{3.047937in}{2.002365in}%
\pgfsys@useobject{currentmarker}{}%
\end{pgfscope}%
\begin{pgfscope}%
\pgfsys@transformshift{3.212425in}{1.870829in}%
\pgfsys@useobject{currentmarker}{}%
\end{pgfscope}%
\begin{pgfscope}%
\pgfsys@transformshift{3.376951in}{1.732480in}%
\pgfsys@useobject{currentmarker}{}%
\end{pgfscope}%
\begin{pgfscope}%
\pgfsys@transformshift{3.541419in}{1.599369in}%
\pgfsys@useobject{currentmarker}{}%
\end{pgfscope}%
\begin{pgfscope}%
\pgfsys@transformshift{3.705862in}{1.470071in}%
\pgfsys@useobject{currentmarker}{}%
\end{pgfscope}%
\begin{pgfscope}%
\pgfsys@transformshift{3.870342in}{1.322559in}%
\pgfsys@useobject{currentmarker}{}%
\end{pgfscope}%
\end{pgfscope}%
\begin{pgfscope}%
\pgfpathrectangle{\pgfqpoint{0.589510in}{0.417642in}}{\pgfqpoint{3.437062in}{2.055000in}}%
\pgfusepath{clip}%
\pgfsetbuttcap%
\pgfsetroundjoin%
\pgfsetlinewidth{1.505625pt}%
\definecolor{currentstroke}{rgb}{0.835294,0.368627,0.000000}%
\pgfsetstrokecolor{currentstroke}%
\pgfsetdash{{5.550000pt}{2.400000pt}}{0.000000pt}%
\pgfpathmoveto{\pgfqpoint{0.745740in}{0.913474in}}%
\pgfpathlineto{\pgfqpoint{0.883294in}{1.029862in}}%
\pgfpathlineto{\pgfqpoint{1.056591in}{1.175613in}}%
\pgfpathlineto{\pgfqpoint{1.216039in}{1.308344in}}%
\pgfpathlineto{\pgfqpoint{1.380747in}{1.443219in}}%
\pgfpathlineto{\pgfqpoint{1.559224in}{1.585203in}}%
\pgfpathlineto{\pgfqpoint{1.726296in}{1.711754in}}%
\pgfpathlineto{\pgfqpoint{1.893892in}{1.828794in}}%
\pgfpathlineto{\pgfqpoint{2.059042in}{1.929055in}}%
\pgfpathlineto{\pgfqpoint{2.223750in}{2.006719in}}%
\pgfpathlineto{\pgfqpoint{2.389127in}{2.052968in}}%
\pgfpathlineto{\pgfqpoint{2.554041in}{2.058337in}}%
\pgfpathlineto{\pgfqpoint{2.718686in}{2.018832in}}%
\pgfpathlineto{\pgfqpoint{2.883588in}{1.939550in}}%
\pgfpathlineto{\pgfqpoint{3.047937in}{1.833671in}}%
\pgfpathlineto{\pgfqpoint{3.212425in}{1.712667in}}%
\pgfpathlineto{\pgfqpoint{3.376951in}{1.583476in}}%
\pgfpathlineto{\pgfqpoint{3.541419in}{1.449716in}}%
\pgfpathlineto{\pgfqpoint{3.705862in}{1.313276in}}%
\pgfpathlineto{\pgfqpoint{3.870342in}{1.175191in}}%
\pgfusepath{stroke}%
\end{pgfscope}%
\begin{pgfscope}%
\pgfpathrectangle{\pgfqpoint{0.589510in}{0.417642in}}{\pgfqpoint{3.437062in}{2.055000in}}%
\pgfusepath{clip}%
\pgfsetbuttcap%
\pgfsetroundjoin%
\definecolor{currentfill}{rgb}{0.835294,0.368627,0.000000}%
\pgfsetfillcolor{currentfill}%
\pgfsetlinewidth{1.003750pt}%
\definecolor{currentstroke}{rgb}{0.835294,0.368627,0.000000}%
\pgfsetstrokecolor{currentstroke}%
\pgfsetdash{}{0pt}%
\pgfsys@defobject{currentmarker}{\pgfqpoint{-0.020833in}{-0.020833in}}{\pgfqpoint{0.020833in}{0.020833in}}{%
\pgfpathmoveto{\pgfqpoint{0.000000in}{-0.020833in}}%
\pgfpathcurveto{\pgfqpoint{0.005525in}{-0.020833in}}{\pgfqpoint{0.010825in}{-0.018638in}}{\pgfqpoint{0.014731in}{-0.014731in}}%
\pgfpathcurveto{\pgfqpoint{0.018638in}{-0.010825in}}{\pgfqpoint{0.020833in}{-0.005525in}}{\pgfqpoint{0.020833in}{0.000000in}}%
\pgfpathcurveto{\pgfqpoint{0.020833in}{0.005525in}}{\pgfqpoint{0.018638in}{0.010825in}}{\pgfqpoint{0.014731in}{0.014731in}}%
\pgfpathcurveto{\pgfqpoint{0.010825in}{0.018638in}}{\pgfqpoint{0.005525in}{0.020833in}}{\pgfqpoint{0.000000in}{0.020833in}}%
\pgfpathcurveto{\pgfqpoint{-0.005525in}{0.020833in}}{\pgfqpoint{-0.010825in}{0.018638in}}{\pgfqpoint{-0.014731in}{0.014731in}}%
\pgfpathcurveto{\pgfqpoint{-0.018638in}{0.010825in}}{\pgfqpoint{-0.020833in}{0.005525in}}{\pgfqpoint{-0.020833in}{0.000000in}}%
\pgfpathcurveto{\pgfqpoint{-0.020833in}{-0.005525in}}{\pgfqpoint{-0.018638in}{-0.010825in}}{\pgfqpoint{-0.014731in}{-0.014731in}}%
\pgfpathcurveto{\pgfqpoint{-0.010825in}{-0.018638in}}{\pgfqpoint{-0.005525in}{-0.020833in}}{\pgfqpoint{0.000000in}{-0.020833in}}%
\pgfpathlineto{\pgfqpoint{0.000000in}{-0.020833in}}%
\pgfpathclose%
\pgfusepath{stroke,fill}%
}%
\begin{pgfscope}%
\pgfsys@transformshift{0.745740in}{0.947471in}%
\pgfsys@useobject{currentmarker}{}%
\end{pgfscope}%
\begin{pgfscope}%
\pgfsys@transformshift{0.883294in}{1.045335in}%
\pgfsys@useobject{currentmarker}{}%
\end{pgfscope}%
\begin{pgfscope}%
\pgfsys@transformshift{1.056591in}{1.181110in}%
\pgfsys@useobject{currentmarker}{}%
\end{pgfscope}%
\begin{pgfscope}%
\pgfsys@transformshift{1.216039in}{1.310404in}%
\pgfsys@useobject{currentmarker}{}%
\end{pgfscope}%
\begin{pgfscope}%
\pgfsys@transformshift{1.380747in}{1.444139in}%
\pgfsys@useobject{currentmarker}{}%
\end{pgfscope}%
\begin{pgfscope}%
\pgfsys@transformshift{1.559224in}{1.585886in}%
\pgfsys@useobject{currentmarker}{}%
\end{pgfscope}%
\begin{pgfscope}%
\pgfsys@transformshift{1.726296in}{1.712219in}%
\pgfsys@useobject{currentmarker}{}%
\end{pgfscope}%
\begin{pgfscope}%
\pgfsys@transformshift{1.893892in}{1.828510in}%
\pgfsys@useobject{currentmarker}{}%
\end{pgfscope}%
\begin{pgfscope}%
\pgfsys@transformshift{2.059042in}{1.928187in}%
\pgfsys@useobject{currentmarker}{}%
\end{pgfscope}%
\begin{pgfscope}%
\pgfsys@transformshift{2.223750in}{2.005395in}%
\pgfsys@useobject{currentmarker}{}%
\end{pgfscope}%
\begin{pgfscope}%
\pgfsys@transformshift{2.389127in}{2.051102in}%
\pgfsys@useobject{currentmarker}{}%
\end{pgfscope}%
\begin{pgfscope}%
\pgfsys@transformshift{2.554041in}{2.055275in}%
\pgfsys@useobject{currentmarker}{}%
\end{pgfscope}%
\begin{pgfscope}%
\pgfsys@transformshift{2.718686in}{2.013974in}%
\pgfsys@useobject{currentmarker}{}%
\end{pgfscope}%
\begin{pgfscope}%
\pgfsys@transformshift{2.883588in}{1.934223in}%
\pgfsys@useobject{currentmarker}{}%
\end{pgfscope}%
\begin{pgfscope}%
\pgfsys@transformshift{3.047937in}{1.829737in}%
\pgfsys@useobject{currentmarker}{}%
\end{pgfscope}%
\begin{pgfscope}%
\pgfsys@transformshift{3.212425in}{1.710686in}%
\pgfsys@useobject{currentmarker}{}%
\end{pgfscope}%
\begin{pgfscope}%
\pgfsys@transformshift{3.376951in}{1.583962in}%
\pgfsys@useobject{currentmarker}{}%
\end{pgfscope}%
\begin{pgfscope}%
\pgfsys@transformshift{3.541419in}{1.447015in}%
\pgfsys@useobject{currentmarker}{}%
\end{pgfscope}%
\begin{pgfscope}%
\pgfsys@transformshift{3.705862in}{1.301664in}%
\pgfsys@useobject{currentmarker}{}%
\end{pgfscope}%
\begin{pgfscope}%
\pgfsys@transformshift{3.870342in}{1.156630in}%
\pgfsys@useobject{currentmarker}{}%
\end{pgfscope}%
\end{pgfscope}%
\begin{pgfscope}%
\pgfsetrectcap%
\pgfsetmiterjoin%
\pgfsetlinewidth{0.803000pt}%
\definecolor{currentstroke}{rgb}{0.000000,0.000000,0.000000}%
\pgfsetstrokecolor{currentstroke}%
\pgfsetdash{}{0pt}%
\pgfpathmoveto{\pgfqpoint{0.589510in}{0.417642in}}%
\pgfpathlineto{\pgfqpoint{0.589510in}{2.472642in}}%
\pgfusepath{stroke}%
\end{pgfscope}%
\begin{pgfscope}%
\pgfsetrectcap%
\pgfsetmiterjoin%
\pgfsetlinewidth{0.803000pt}%
\definecolor{currentstroke}{rgb}{0.000000,0.000000,0.000000}%
\pgfsetstrokecolor{currentstroke}%
\pgfsetdash{}{0pt}%
\pgfpathmoveto{\pgfqpoint{4.026572in}{0.417642in}}%
\pgfpathlineto{\pgfqpoint{4.026572in}{2.472642in}}%
\pgfusepath{stroke}%
\end{pgfscope}%
\begin{pgfscope}%
\pgfsetrectcap%
\pgfsetmiterjoin%
\pgfsetlinewidth{0.803000pt}%
\definecolor{currentstroke}{rgb}{0.000000,0.000000,0.000000}%
\pgfsetstrokecolor{currentstroke}%
\pgfsetdash{}{0pt}%
\pgfpathmoveto{\pgfqpoint{0.589510in}{0.417642in}}%
\pgfpathlineto{\pgfqpoint{4.026572in}{0.417642in}}%
\pgfusepath{stroke}%
\end{pgfscope}%
\begin{pgfscope}%
\pgfsetrectcap%
\pgfsetmiterjoin%
\pgfsetlinewidth{0.803000pt}%
\definecolor{currentstroke}{rgb}{0.000000,0.000000,0.000000}%
\pgfsetstrokecolor{currentstroke}%
\pgfsetdash{}{0pt}%
\pgfpathmoveto{\pgfqpoint{0.589510in}{2.472642in}}%
\pgfpathlineto{\pgfqpoint{4.026572in}{2.472642in}}%
\pgfusepath{stroke}%
\end{pgfscope}%
\begin{pgfscope}%
\pgfsetbuttcap%
\pgfsetmiterjoin%
\definecolor{currentfill}{rgb}{1.000000,1.000000,1.000000}%
\pgfsetfillcolor{currentfill}%
\pgfsetfillopacity{0.800000}%
\pgfsetlinewidth{1.003750pt}%
\definecolor{currentstroke}{rgb}{0.800000,0.800000,0.800000}%
\pgfsetstrokecolor{currentstroke}%
\pgfsetstrokeopacity{0.800000}%
\pgfsetdash{}{0pt}%
\pgfpathmoveto{\pgfqpoint{3.108484in}{1.919086in}}%
\pgfpathlineto{\pgfqpoint{3.948794in}{1.919086in}}%
\pgfpathquadraticcurveto{\pgfqpoint{3.971016in}{1.919086in}}{\pgfqpoint{3.971016in}{1.941309in}}%
\pgfpathlineto{\pgfqpoint{3.971016in}{2.394864in}}%
\pgfpathquadraticcurveto{\pgfqpoint{3.971016in}{2.417086in}}{\pgfqpoint{3.948794in}{2.417086in}}%
\pgfpathlineto{\pgfqpoint{3.108484in}{2.417086in}}%
\pgfpathquadraticcurveto{\pgfqpoint{3.086261in}{2.417086in}}{\pgfqpoint{3.086261in}{2.394864in}}%
\pgfpathlineto{\pgfqpoint{3.086261in}{1.941309in}}%
\pgfpathquadraticcurveto{\pgfqpoint{3.086261in}{1.919086in}}{\pgfqpoint{3.108484in}{1.919086in}}%
\pgfpathlineto{\pgfqpoint{3.108484in}{1.919086in}}%
\pgfpathclose%
\pgfusepath{stroke,fill}%
\end{pgfscope}%
\begin{pgfscope}%
\pgfsetbuttcap%
\pgfsetroundjoin%
\definecolor{currentfill}{rgb}{0.003922,0.450980,0.698039}%
\pgfsetfillcolor{currentfill}%
\pgfsetlinewidth{1.003750pt}%
\definecolor{currentstroke}{rgb}{0.003922,0.450980,0.698039}%
\pgfsetstrokecolor{currentstroke}%
\pgfsetdash{}{0pt}%
\pgfsys@defobject{currentmarker}{\pgfqpoint{-0.020833in}{-0.020833in}}{\pgfqpoint{0.020833in}{0.020833in}}{%
\pgfpathmoveto{\pgfqpoint{0.000000in}{-0.020833in}}%
\pgfpathcurveto{\pgfqpoint{0.005525in}{-0.020833in}}{\pgfqpoint{0.010825in}{-0.018638in}}{\pgfqpoint{0.014731in}{-0.014731in}}%
\pgfpathcurveto{\pgfqpoint{0.018638in}{-0.010825in}}{\pgfqpoint{0.020833in}{-0.005525in}}{\pgfqpoint{0.020833in}{0.000000in}}%
\pgfpathcurveto{\pgfqpoint{0.020833in}{0.005525in}}{\pgfqpoint{0.018638in}{0.010825in}}{\pgfqpoint{0.014731in}{0.014731in}}%
\pgfpathcurveto{\pgfqpoint{0.010825in}{0.018638in}}{\pgfqpoint{0.005525in}{0.020833in}}{\pgfqpoint{0.000000in}{0.020833in}}%
\pgfpathcurveto{\pgfqpoint{-0.005525in}{0.020833in}}{\pgfqpoint{-0.010825in}{0.018638in}}{\pgfqpoint{-0.014731in}{0.014731in}}%
\pgfpathcurveto{\pgfqpoint{-0.018638in}{0.010825in}}{\pgfqpoint{-0.020833in}{0.005525in}}{\pgfqpoint{-0.020833in}{0.000000in}}%
\pgfpathcurveto{\pgfqpoint{-0.020833in}{-0.005525in}}{\pgfqpoint{-0.018638in}{-0.010825in}}{\pgfqpoint{-0.014731in}{-0.014731in}}%
\pgfpathcurveto{\pgfqpoint{-0.010825in}{-0.018638in}}{\pgfqpoint{-0.005525in}{-0.020833in}}{\pgfqpoint{0.000000in}{-0.020833in}}%
\pgfpathlineto{\pgfqpoint{0.000000in}{-0.020833in}}%
\pgfpathclose%
\pgfusepath{stroke,fill}%
}%
\begin{pgfscope}%
\pgfsys@transformshift{3.241817in}{2.333753in}%
\pgfsys@useobject{currentmarker}{}%
\end{pgfscope}%
\end{pgfscope}%
\begin{pgfscope}%
\definecolor{textcolor}{rgb}{0.000000,0.000000,0.000000}%
\pgfsetstrokecolor{textcolor}%
\pgfsetfillcolor{textcolor}%
\pgftext[x=3.441817in,y=2.294864in,left,base]{\color{textcolor}\rmfamily\fontsize{8.000000}{9.600000}\selectfont \(\displaystyle \bar\tau_1=\qty{0.1}{\s}\)}%
\end{pgfscope}%
\begin{pgfscope}%
\pgfsetbuttcap%
\pgfsetroundjoin%
\definecolor{currentfill}{rgb}{0.007843,0.619608,0.450980}%
\pgfsetfillcolor{currentfill}%
\pgfsetlinewidth{1.003750pt}%
\definecolor{currentstroke}{rgb}{0.007843,0.619608,0.450980}%
\pgfsetstrokecolor{currentstroke}%
\pgfsetdash{}{0pt}%
\pgfsys@defobject{currentmarker}{\pgfqpoint{-0.020833in}{-0.020833in}}{\pgfqpoint{0.020833in}{0.020833in}}{%
\pgfpathmoveto{\pgfqpoint{0.000000in}{-0.020833in}}%
\pgfpathcurveto{\pgfqpoint{0.005525in}{-0.020833in}}{\pgfqpoint{0.010825in}{-0.018638in}}{\pgfqpoint{0.014731in}{-0.014731in}}%
\pgfpathcurveto{\pgfqpoint{0.018638in}{-0.010825in}}{\pgfqpoint{0.020833in}{-0.005525in}}{\pgfqpoint{0.020833in}{0.000000in}}%
\pgfpathcurveto{\pgfqpoint{0.020833in}{0.005525in}}{\pgfqpoint{0.018638in}{0.010825in}}{\pgfqpoint{0.014731in}{0.014731in}}%
\pgfpathcurveto{\pgfqpoint{0.010825in}{0.018638in}}{\pgfqpoint{0.005525in}{0.020833in}}{\pgfqpoint{0.000000in}{0.020833in}}%
\pgfpathcurveto{\pgfqpoint{-0.005525in}{0.020833in}}{\pgfqpoint{-0.010825in}{0.018638in}}{\pgfqpoint{-0.014731in}{0.014731in}}%
\pgfpathcurveto{\pgfqpoint{-0.018638in}{0.010825in}}{\pgfqpoint{-0.020833in}{0.005525in}}{\pgfqpoint{-0.020833in}{0.000000in}}%
\pgfpathcurveto{\pgfqpoint{-0.020833in}{-0.005525in}}{\pgfqpoint{-0.018638in}{-0.010825in}}{\pgfqpoint{-0.014731in}{-0.014731in}}%
\pgfpathcurveto{\pgfqpoint{-0.010825in}{-0.018638in}}{\pgfqpoint{-0.005525in}{-0.020833in}}{\pgfqpoint{0.000000in}{-0.020833in}}%
\pgfpathlineto{\pgfqpoint{0.000000in}{-0.020833in}}%
\pgfpathclose%
\pgfusepath{stroke,fill}%
}%
\begin{pgfscope}%
\pgfsys@transformshift{3.241817in}{2.178864in}%
\pgfsys@useobject{currentmarker}{}%
\end{pgfscope}%
\end{pgfscope}%
\begin{pgfscope}%
\definecolor{textcolor}{rgb}{0.000000,0.000000,0.000000}%
\pgfsetstrokecolor{textcolor}%
\pgfsetfillcolor{textcolor}%
\pgftext[x=3.441817in,y=2.139975in,left,base]{\color{textcolor}\rmfamily\fontsize{8.000000}{9.600000}\selectfont \(\displaystyle \bar\tau_1=\qty{1}{\s}\)}%
\end{pgfscope}%
\begin{pgfscope}%
\pgfsetbuttcap%
\pgfsetroundjoin%
\definecolor{currentfill}{rgb}{0.835294,0.368627,0.000000}%
\pgfsetfillcolor{currentfill}%
\pgfsetlinewidth{1.003750pt}%
\definecolor{currentstroke}{rgb}{0.835294,0.368627,0.000000}%
\pgfsetstrokecolor{currentstroke}%
\pgfsetdash{}{0pt}%
\pgfsys@defobject{currentmarker}{\pgfqpoint{-0.020833in}{-0.020833in}}{\pgfqpoint{0.020833in}{0.020833in}}{%
\pgfpathmoveto{\pgfqpoint{0.000000in}{-0.020833in}}%
\pgfpathcurveto{\pgfqpoint{0.005525in}{-0.020833in}}{\pgfqpoint{0.010825in}{-0.018638in}}{\pgfqpoint{0.014731in}{-0.014731in}}%
\pgfpathcurveto{\pgfqpoint{0.018638in}{-0.010825in}}{\pgfqpoint{0.020833in}{-0.005525in}}{\pgfqpoint{0.020833in}{0.000000in}}%
\pgfpathcurveto{\pgfqpoint{0.020833in}{0.005525in}}{\pgfqpoint{0.018638in}{0.010825in}}{\pgfqpoint{0.014731in}{0.014731in}}%
\pgfpathcurveto{\pgfqpoint{0.010825in}{0.018638in}}{\pgfqpoint{0.005525in}{0.020833in}}{\pgfqpoint{0.000000in}{0.020833in}}%
\pgfpathcurveto{\pgfqpoint{-0.005525in}{0.020833in}}{\pgfqpoint{-0.010825in}{0.018638in}}{\pgfqpoint{-0.014731in}{0.014731in}}%
\pgfpathcurveto{\pgfqpoint{-0.018638in}{0.010825in}}{\pgfqpoint{-0.020833in}{0.005525in}}{\pgfqpoint{-0.020833in}{0.000000in}}%
\pgfpathcurveto{\pgfqpoint{-0.020833in}{-0.005525in}}{\pgfqpoint{-0.018638in}{-0.010825in}}{\pgfqpoint{-0.014731in}{-0.014731in}}%
\pgfpathcurveto{\pgfqpoint{-0.010825in}{-0.018638in}}{\pgfqpoint{-0.005525in}{-0.020833in}}{\pgfqpoint{0.000000in}{-0.020833in}}%
\pgfpathlineto{\pgfqpoint{0.000000in}{-0.020833in}}%
\pgfpathclose%
\pgfusepath{stroke,fill}%
}%
\begin{pgfscope}%
\pgfsys@transformshift{3.241817in}{2.023975in}%
\pgfsys@useobject{currentmarker}{}%
\end{pgfscope}%
\end{pgfscope}%
\begin{pgfscope}%
\definecolor{textcolor}{rgb}{0.000000,0.000000,0.000000}%
\pgfsetstrokecolor{textcolor}%
\pgfsetfillcolor{textcolor}%
\pgftext[x=3.441817in,y=1.985086in,left,base]{\color{textcolor}\rmfamily\fontsize{8.000000}{9.600000}\selectfont \(\displaystyle \bar\tau_1=\qty{10}{\s}\)}%
\end{pgfscope}%
\end{pgfpicture}%
\makeatother%
\endgroup%
% data/simulations/sim_burst_noise.py
        } % scalebox
        \caption{Allan deviation}
        \label{fig:burst_noise_adev}
    \end{subfigure}
    \caption{Different representations of burst noise for different $\bar \tau_1$ and fixed $\bar \tau_0 = \qty{1}{\s}$.}
    \label{fig:burst_noise_simulated}
\end{figure}

The burst noise equations can be used to gain further insight into other types of noise. The first one is Shot noise, which is commonly found in photodetectors and lasers. Here, electrons or photons are created at discrete intervals resulting in an instantaneous signal. This means that the lifetime of the upper level is very short in comparison to the lower level ($\tau_1 \ll \tau_0$) equation \ref{eqn:burst_noise_psd} becomes:
\begin{align}
    S_{Shot}(\omega) = S_{\tau_1 \ll \tau_0}(\omega) &= 4 \Delta y^2 \frac{\tau_1}{\tau_0} \frac{\frac{1}{\bar \tau_1}}{\left(\frac{1}{\bar \tau_1}\right)^2 + \omega^2}\nonumber\\
    &= 4 \Delta y^2 \frac{1}{\tau_0} \frac{1}{\frac{1}{\tau_1^2}+\omega^2}\\
    \overset{\omega \ll 1/\tau_0}&{\approx} 4 \Delta y^2 \frac{\tau_1^2}{\tau_0} = \text{const.}
\end{align}

Typically, a very large number of such events happen. When not counting single events, but rather a stream, the relation $\omega \ll 1/\tau_0$ is valid and hence the result is a white spectrum as $S_{Shot}(\omega)$ is constant with respect to $\omega$ --- just as observed in photodetectors and lasers.

The other interesting occurrence is a case where many trap sites with different time constants are contributing to the noise. This can change the shape of the spectrum from $f^{-2}$ to $f^{-1}$ and is discussed in the next section.

\clearpage
\subsubsection{Flicker Noise}%
\label{sec:flicker_noise}
Flicker noise is also called $\frac 1 f$-noise and it can be observed in many naturally occurring phenomena. Its origin is not clear, even though there have been many explanations. An overview can be found in \cite{flicker_noise_overview, flicker_noise_overview2, origins_1_f_noise}. This section concentrates on flicker noise in electronic devices. In thick-film resistors, for example, it was shown to extend over at least 6 decades without any visible flattening \cite{1_f_noise_thick_film}. In transistors, flicker noise is caused by the existence of generation-recombination noise or burst noise discussed in the previous section \cite{origins_1_f_noise}. If there are many uncorrelated trap sites which contribute to the total noise, the envelope of the noise spectral density changes from $\frac{1}{f^2}$ to $\frac{1}{f^1}$ as shown in figure \ref{fig:flicker_noise_evelope}
\begin{figure}[hb]
    \centering
    %% Creator: Matplotlib, PGF backend
%%
%% To include the figure in your LaTeX document, write
%%   \input{<filename>.pgf}
%%
%% Make sure the required packages are loaded in your preamble
%%   \usepackage{pgf}
%%
%% Also ensure that all the required font packages are loaded; for instance,
%% the lmodern package is sometimes necessary when using math font.
%%   \usepackage{lmodern}
%%
%% Figures using additional raster images can only be included by \input if
%% they are in the same directory as the main LaTeX file. For loading figures
%% from other directories you can use the `import` package
%%   \usepackage{import}
%%
%% and then include the figures with
%%   \import{<path to file>}{<filename>.pgf}
%%
%% Matplotlib used the following preamble
%%   \usepackage{siunitx}
%%   \usepackage{fontspec}
%%   \makeatletter\@ifpackageloaded{underscore}{}{\usepackage[strings]{underscore}}\makeatother
%%
\begingroup%
\makeatletter%
\begin{pgfpicture}%
\pgfpathrectangle{\pgfpointorigin}{\pgfqpoint{4.060000in}{2.510000in}}%
\pgfusepath{use as bounding box, clip}%
\begin{pgfscope}%
\pgfsetbuttcap%
\pgfsetmiterjoin%
\definecolor{currentfill}{rgb}{1.000000,1.000000,1.000000}%
\pgfsetfillcolor{currentfill}%
\pgfsetlinewidth{0.000000pt}%
\definecolor{currentstroke}{rgb}{1.000000,1.000000,1.000000}%
\pgfsetstrokecolor{currentstroke}%
\pgfsetdash{}{0pt}%
\pgfpathmoveto{\pgfqpoint{0.000000in}{0.000000in}}%
\pgfpathlineto{\pgfqpoint{4.060000in}{0.000000in}}%
\pgfpathlineto{\pgfqpoint{4.060000in}{2.510000in}}%
\pgfpathlineto{\pgfqpoint{0.000000in}{2.510000in}}%
\pgfpathlineto{\pgfqpoint{0.000000in}{0.000000in}}%
\pgfpathclose%
\pgfusepath{fill}%
\end{pgfscope}%
\begin{pgfscope}%
\pgfsetbuttcap%
\pgfsetmiterjoin%
\definecolor{currentfill}{rgb}{1.000000,1.000000,1.000000}%
\pgfsetfillcolor{currentfill}%
\pgfsetlinewidth{0.000000pt}%
\definecolor{currentstroke}{rgb}{0.000000,0.000000,0.000000}%
\pgfsetstrokecolor{currentstroke}%
\pgfsetstrokeopacity{0.000000}%
\pgfsetdash{}{0pt}%
\pgfpathmoveto{\pgfqpoint{0.594525in}{0.417642in}}%
\pgfpathlineto{\pgfqpoint{4.018330in}{0.417642in}}%
\pgfpathlineto{\pgfqpoint{4.018330in}{2.429177in}}%
\pgfpathlineto{\pgfqpoint{0.594525in}{2.429177in}}%
\pgfpathlineto{\pgfqpoint{0.594525in}{0.417642in}}%
\pgfpathclose%
\pgfusepath{fill}%
\end{pgfscope}%
\begin{pgfscope}%
\pgfpathrectangle{\pgfqpoint{0.594525in}{0.417642in}}{\pgfqpoint{3.423805in}{2.011535in}}%
\pgfusepath{clip}%
\pgfsetrectcap%
\pgfsetroundjoin%
\pgfsetlinewidth{0.803000pt}%
\definecolor{currentstroke}{rgb}{0.450000,0.450000,0.450000}%
\pgfsetstrokecolor{currentstroke}%
\pgfsetdash{}{0pt}%
\pgfpathmoveto{\pgfqpoint{0.750152in}{0.417642in}}%
\pgfpathlineto{\pgfqpoint{0.750152in}{2.429177in}}%
\pgfusepath{stroke}%
\end{pgfscope}%
\begin{pgfscope}%
\pgfsetbuttcap%
\pgfsetroundjoin%
\definecolor{currentfill}{rgb}{0.000000,0.000000,0.000000}%
\pgfsetfillcolor{currentfill}%
\pgfsetlinewidth{0.803000pt}%
\definecolor{currentstroke}{rgb}{0.000000,0.000000,0.000000}%
\pgfsetstrokecolor{currentstroke}%
\pgfsetdash{}{0pt}%
\pgfsys@defobject{currentmarker}{\pgfqpoint{0.000000in}{-0.048611in}}{\pgfqpoint{0.000000in}{0.000000in}}{%
\pgfpathmoveto{\pgfqpoint{0.000000in}{0.000000in}}%
\pgfpathlineto{\pgfqpoint{0.000000in}{-0.048611in}}%
\pgfusepath{stroke,fill}%
}%
\begin{pgfscope}%
\pgfsys@transformshift{0.750152in}{0.417642in}%
\pgfsys@useobject{currentmarker}{}%
\end{pgfscope}%
\end{pgfscope}%
\begin{pgfscope}%
\definecolor{textcolor}{rgb}{0.000000,0.000000,0.000000}%
\pgfsetstrokecolor{textcolor}%
\pgfsetfillcolor{textcolor}%
\pgftext[x=0.750152in,y=0.320420in,,top]{\color{textcolor}\rmfamily\fontsize{8.000000}{9.600000}\selectfont \(\displaystyle {10^{-2}}\)}%
\end{pgfscope}%
\begin{pgfscope}%
\pgfpathrectangle{\pgfqpoint{0.594525in}{0.417642in}}{\pgfqpoint{3.423805in}{2.011535in}}%
\pgfusepath{clip}%
\pgfsetrectcap%
\pgfsetroundjoin%
\pgfsetlinewidth{0.803000pt}%
\definecolor{currentstroke}{rgb}{0.450000,0.450000,0.450000}%
\pgfsetstrokecolor{currentstroke}%
\pgfsetdash{}{0pt}%
\pgfpathmoveto{\pgfqpoint{1.528290in}{0.417642in}}%
\pgfpathlineto{\pgfqpoint{1.528290in}{2.429177in}}%
\pgfusepath{stroke}%
\end{pgfscope}%
\begin{pgfscope}%
\pgfsetbuttcap%
\pgfsetroundjoin%
\definecolor{currentfill}{rgb}{0.000000,0.000000,0.000000}%
\pgfsetfillcolor{currentfill}%
\pgfsetlinewidth{0.803000pt}%
\definecolor{currentstroke}{rgb}{0.000000,0.000000,0.000000}%
\pgfsetstrokecolor{currentstroke}%
\pgfsetdash{}{0pt}%
\pgfsys@defobject{currentmarker}{\pgfqpoint{0.000000in}{-0.048611in}}{\pgfqpoint{0.000000in}{0.000000in}}{%
\pgfpathmoveto{\pgfqpoint{0.000000in}{0.000000in}}%
\pgfpathlineto{\pgfqpoint{0.000000in}{-0.048611in}}%
\pgfusepath{stroke,fill}%
}%
\begin{pgfscope}%
\pgfsys@transformshift{1.528290in}{0.417642in}%
\pgfsys@useobject{currentmarker}{}%
\end{pgfscope}%
\end{pgfscope}%
\begin{pgfscope}%
\definecolor{textcolor}{rgb}{0.000000,0.000000,0.000000}%
\pgfsetstrokecolor{textcolor}%
\pgfsetfillcolor{textcolor}%
\pgftext[x=1.528290in,y=0.320420in,,top]{\color{textcolor}\rmfamily\fontsize{8.000000}{9.600000}\selectfont \(\displaystyle {10^{-1}}\)}%
\end{pgfscope}%
\begin{pgfscope}%
\pgfpathrectangle{\pgfqpoint{0.594525in}{0.417642in}}{\pgfqpoint{3.423805in}{2.011535in}}%
\pgfusepath{clip}%
\pgfsetrectcap%
\pgfsetroundjoin%
\pgfsetlinewidth{0.803000pt}%
\definecolor{currentstroke}{rgb}{0.450000,0.450000,0.450000}%
\pgfsetstrokecolor{currentstroke}%
\pgfsetdash{}{0pt}%
\pgfpathmoveto{\pgfqpoint{2.306427in}{0.417642in}}%
\pgfpathlineto{\pgfqpoint{2.306427in}{2.429177in}}%
\pgfusepath{stroke}%
\end{pgfscope}%
\begin{pgfscope}%
\pgfsetbuttcap%
\pgfsetroundjoin%
\definecolor{currentfill}{rgb}{0.000000,0.000000,0.000000}%
\pgfsetfillcolor{currentfill}%
\pgfsetlinewidth{0.803000pt}%
\definecolor{currentstroke}{rgb}{0.000000,0.000000,0.000000}%
\pgfsetstrokecolor{currentstroke}%
\pgfsetdash{}{0pt}%
\pgfsys@defobject{currentmarker}{\pgfqpoint{0.000000in}{-0.048611in}}{\pgfqpoint{0.000000in}{0.000000in}}{%
\pgfpathmoveto{\pgfqpoint{0.000000in}{0.000000in}}%
\pgfpathlineto{\pgfqpoint{0.000000in}{-0.048611in}}%
\pgfusepath{stroke,fill}%
}%
\begin{pgfscope}%
\pgfsys@transformshift{2.306427in}{0.417642in}%
\pgfsys@useobject{currentmarker}{}%
\end{pgfscope}%
\end{pgfscope}%
\begin{pgfscope}%
\definecolor{textcolor}{rgb}{0.000000,0.000000,0.000000}%
\pgfsetstrokecolor{textcolor}%
\pgfsetfillcolor{textcolor}%
\pgftext[x=2.306427in,y=0.320420in,,top]{\color{textcolor}\rmfamily\fontsize{8.000000}{9.600000}\selectfont \(\displaystyle {10^{0}}\)}%
\end{pgfscope}%
\begin{pgfscope}%
\pgfpathrectangle{\pgfqpoint{0.594525in}{0.417642in}}{\pgfqpoint{3.423805in}{2.011535in}}%
\pgfusepath{clip}%
\pgfsetrectcap%
\pgfsetroundjoin%
\pgfsetlinewidth{0.803000pt}%
\definecolor{currentstroke}{rgb}{0.450000,0.450000,0.450000}%
\pgfsetstrokecolor{currentstroke}%
\pgfsetdash{}{0pt}%
\pgfpathmoveto{\pgfqpoint{3.084565in}{0.417642in}}%
\pgfpathlineto{\pgfqpoint{3.084565in}{2.429177in}}%
\pgfusepath{stroke}%
\end{pgfscope}%
\begin{pgfscope}%
\pgfsetbuttcap%
\pgfsetroundjoin%
\definecolor{currentfill}{rgb}{0.000000,0.000000,0.000000}%
\pgfsetfillcolor{currentfill}%
\pgfsetlinewidth{0.803000pt}%
\definecolor{currentstroke}{rgb}{0.000000,0.000000,0.000000}%
\pgfsetstrokecolor{currentstroke}%
\pgfsetdash{}{0pt}%
\pgfsys@defobject{currentmarker}{\pgfqpoint{0.000000in}{-0.048611in}}{\pgfqpoint{0.000000in}{0.000000in}}{%
\pgfpathmoveto{\pgfqpoint{0.000000in}{0.000000in}}%
\pgfpathlineto{\pgfqpoint{0.000000in}{-0.048611in}}%
\pgfusepath{stroke,fill}%
}%
\begin{pgfscope}%
\pgfsys@transformshift{3.084565in}{0.417642in}%
\pgfsys@useobject{currentmarker}{}%
\end{pgfscope}%
\end{pgfscope}%
\begin{pgfscope}%
\definecolor{textcolor}{rgb}{0.000000,0.000000,0.000000}%
\pgfsetstrokecolor{textcolor}%
\pgfsetfillcolor{textcolor}%
\pgftext[x=3.084565in,y=0.320420in,,top]{\color{textcolor}\rmfamily\fontsize{8.000000}{9.600000}\selectfont \(\displaystyle {10^{1}}\)}%
\end{pgfscope}%
\begin{pgfscope}%
\pgfpathrectangle{\pgfqpoint{0.594525in}{0.417642in}}{\pgfqpoint{3.423805in}{2.011535in}}%
\pgfusepath{clip}%
\pgfsetrectcap%
\pgfsetroundjoin%
\pgfsetlinewidth{0.803000pt}%
\definecolor{currentstroke}{rgb}{0.450000,0.450000,0.450000}%
\pgfsetstrokecolor{currentstroke}%
\pgfsetdash{}{0pt}%
\pgfpathmoveto{\pgfqpoint{3.862702in}{0.417642in}}%
\pgfpathlineto{\pgfqpoint{3.862702in}{2.429177in}}%
\pgfusepath{stroke}%
\end{pgfscope}%
\begin{pgfscope}%
\pgfsetbuttcap%
\pgfsetroundjoin%
\definecolor{currentfill}{rgb}{0.000000,0.000000,0.000000}%
\pgfsetfillcolor{currentfill}%
\pgfsetlinewidth{0.803000pt}%
\definecolor{currentstroke}{rgb}{0.000000,0.000000,0.000000}%
\pgfsetstrokecolor{currentstroke}%
\pgfsetdash{}{0pt}%
\pgfsys@defobject{currentmarker}{\pgfqpoint{0.000000in}{-0.048611in}}{\pgfqpoint{0.000000in}{0.000000in}}{%
\pgfpathmoveto{\pgfqpoint{0.000000in}{0.000000in}}%
\pgfpathlineto{\pgfqpoint{0.000000in}{-0.048611in}}%
\pgfusepath{stroke,fill}%
}%
\begin{pgfscope}%
\pgfsys@transformshift{3.862702in}{0.417642in}%
\pgfsys@useobject{currentmarker}{}%
\end{pgfscope}%
\end{pgfscope}%
\begin{pgfscope}%
\definecolor{textcolor}{rgb}{0.000000,0.000000,0.000000}%
\pgfsetstrokecolor{textcolor}%
\pgfsetfillcolor{textcolor}%
\pgftext[x=3.862702in,y=0.320420in,,top]{\color{textcolor}\rmfamily\fontsize{8.000000}{9.600000}\selectfont \(\displaystyle {10^{2}}\)}%
\end{pgfscope}%
\begin{pgfscope}%
\pgfpathrectangle{\pgfqpoint{0.594525in}{0.417642in}}{\pgfqpoint{3.423805in}{2.011535in}}%
\pgfusepath{clip}%
\pgfsetrectcap%
\pgfsetroundjoin%
\pgfsetlinewidth{0.803000pt}%
\definecolor{currentstroke}{rgb}{0.850000,0.850000,0.850000}%
\pgfsetstrokecolor{currentstroke}%
\pgfsetdash{}{0pt}%
\pgfpathmoveto{\pgfqpoint{0.629617in}{0.417642in}}%
\pgfpathlineto{\pgfqpoint{0.629617in}{2.429177in}}%
\pgfusepath{stroke}%
\end{pgfscope}%
\begin{pgfscope}%
\pgfsetbuttcap%
\pgfsetroundjoin%
\definecolor{currentfill}{rgb}{0.000000,0.000000,0.000000}%
\pgfsetfillcolor{currentfill}%
\pgfsetlinewidth{0.602250pt}%
\definecolor{currentstroke}{rgb}{0.000000,0.000000,0.000000}%
\pgfsetstrokecolor{currentstroke}%
\pgfsetdash{}{0pt}%
\pgfsys@defobject{currentmarker}{\pgfqpoint{0.000000in}{-0.027778in}}{\pgfqpoint{0.000000in}{0.000000in}}{%
\pgfpathmoveto{\pgfqpoint{0.000000in}{0.000000in}}%
\pgfpathlineto{\pgfqpoint{0.000000in}{-0.027778in}}%
\pgfusepath{stroke,fill}%
}%
\begin{pgfscope}%
\pgfsys@transformshift{0.629617in}{0.417642in}%
\pgfsys@useobject{currentmarker}{}%
\end{pgfscope}%
\end{pgfscope}%
\begin{pgfscope}%
\pgfpathrectangle{\pgfqpoint{0.594525in}{0.417642in}}{\pgfqpoint{3.423805in}{2.011535in}}%
\pgfusepath{clip}%
\pgfsetrectcap%
\pgfsetroundjoin%
\pgfsetlinewidth{0.803000pt}%
\definecolor{currentstroke}{rgb}{0.850000,0.850000,0.850000}%
\pgfsetstrokecolor{currentstroke}%
\pgfsetdash{}{0pt}%
\pgfpathmoveto{\pgfqpoint{0.674743in}{0.417642in}}%
\pgfpathlineto{\pgfqpoint{0.674743in}{2.429177in}}%
\pgfusepath{stroke}%
\end{pgfscope}%
\begin{pgfscope}%
\pgfsetbuttcap%
\pgfsetroundjoin%
\definecolor{currentfill}{rgb}{0.000000,0.000000,0.000000}%
\pgfsetfillcolor{currentfill}%
\pgfsetlinewidth{0.602250pt}%
\definecolor{currentstroke}{rgb}{0.000000,0.000000,0.000000}%
\pgfsetstrokecolor{currentstroke}%
\pgfsetdash{}{0pt}%
\pgfsys@defobject{currentmarker}{\pgfqpoint{0.000000in}{-0.027778in}}{\pgfqpoint{0.000000in}{0.000000in}}{%
\pgfpathmoveto{\pgfqpoint{0.000000in}{0.000000in}}%
\pgfpathlineto{\pgfqpoint{0.000000in}{-0.027778in}}%
\pgfusepath{stroke,fill}%
}%
\begin{pgfscope}%
\pgfsys@transformshift{0.674743in}{0.417642in}%
\pgfsys@useobject{currentmarker}{}%
\end{pgfscope}%
\end{pgfscope}%
\begin{pgfscope}%
\pgfpathrectangle{\pgfqpoint{0.594525in}{0.417642in}}{\pgfqpoint{3.423805in}{2.011535in}}%
\pgfusepath{clip}%
\pgfsetrectcap%
\pgfsetroundjoin%
\pgfsetlinewidth{0.803000pt}%
\definecolor{currentstroke}{rgb}{0.850000,0.850000,0.850000}%
\pgfsetstrokecolor{currentstroke}%
\pgfsetdash{}{0pt}%
\pgfpathmoveto{\pgfqpoint{0.714547in}{0.417642in}}%
\pgfpathlineto{\pgfqpoint{0.714547in}{2.429177in}}%
\pgfusepath{stroke}%
\end{pgfscope}%
\begin{pgfscope}%
\pgfsetbuttcap%
\pgfsetroundjoin%
\definecolor{currentfill}{rgb}{0.000000,0.000000,0.000000}%
\pgfsetfillcolor{currentfill}%
\pgfsetlinewidth{0.602250pt}%
\definecolor{currentstroke}{rgb}{0.000000,0.000000,0.000000}%
\pgfsetstrokecolor{currentstroke}%
\pgfsetdash{}{0pt}%
\pgfsys@defobject{currentmarker}{\pgfqpoint{0.000000in}{-0.027778in}}{\pgfqpoint{0.000000in}{0.000000in}}{%
\pgfpathmoveto{\pgfqpoint{0.000000in}{0.000000in}}%
\pgfpathlineto{\pgfqpoint{0.000000in}{-0.027778in}}%
\pgfusepath{stroke,fill}%
}%
\begin{pgfscope}%
\pgfsys@transformshift{0.714547in}{0.417642in}%
\pgfsys@useobject{currentmarker}{}%
\end{pgfscope}%
\end{pgfscope}%
\begin{pgfscope}%
\pgfpathrectangle{\pgfqpoint{0.594525in}{0.417642in}}{\pgfqpoint{3.423805in}{2.011535in}}%
\pgfusepath{clip}%
\pgfsetrectcap%
\pgfsetroundjoin%
\pgfsetlinewidth{0.803000pt}%
\definecolor{currentstroke}{rgb}{0.850000,0.850000,0.850000}%
\pgfsetstrokecolor{currentstroke}%
\pgfsetdash{}{0pt}%
\pgfpathmoveto{\pgfqpoint{0.984395in}{0.417642in}}%
\pgfpathlineto{\pgfqpoint{0.984395in}{2.429177in}}%
\pgfusepath{stroke}%
\end{pgfscope}%
\begin{pgfscope}%
\pgfsetbuttcap%
\pgfsetroundjoin%
\definecolor{currentfill}{rgb}{0.000000,0.000000,0.000000}%
\pgfsetfillcolor{currentfill}%
\pgfsetlinewidth{0.602250pt}%
\definecolor{currentstroke}{rgb}{0.000000,0.000000,0.000000}%
\pgfsetstrokecolor{currentstroke}%
\pgfsetdash{}{0pt}%
\pgfsys@defobject{currentmarker}{\pgfqpoint{0.000000in}{-0.027778in}}{\pgfqpoint{0.000000in}{0.000000in}}{%
\pgfpathmoveto{\pgfqpoint{0.000000in}{0.000000in}}%
\pgfpathlineto{\pgfqpoint{0.000000in}{-0.027778in}}%
\pgfusepath{stroke,fill}%
}%
\begin{pgfscope}%
\pgfsys@transformshift{0.984395in}{0.417642in}%
\pgfsys@useobject{currentmarker}{}%
\end{pgfscope}%
\end{pgfscope}%
\begin{pgfscope}%
\pgfpathrectangle{\pgfqpoint{0.594525in}{0.417642in}}{\pgfqpoint{3.423805in}{2.011535in}}%
\pgfusepath{clip}%
\pgfsetrectcap%
\pgfsetroundjoin%
\pgfsetlinewidth{0.803000pt}%
\definecolor{currentstroke}{rgb}{0.850000,0.850000,0.850000}%
\pgfsetstrokecolor{currentstroke}%
\pgfsetdash{}{0pt}%
\pgfpathmoveto{\pgfqpoint{1.121418in}{0.417642in}}%
\pgfpathlineto{\pgfqpoint{1.121418in}{2.429177in}}%
\pgfusepath{stroke}%
\end{pgfscope}%
\begin{pgfscope}%
\pgfsetbuttcap%
\pgfsetroundjoin%
\definecolor{currentfill}{rgb}{0.000000,0.000000,0.000000}%
\pgfsetfillcolor{currentfill}%
\pgfsetlinewidth{0.602250pt}%
\definecolor{currentstroke}{rgb}{0.000000,0.000000,0.000000}%
\pgfsetstrokecolor{currentstroke}%
\pgfsetdash{}{0pt}%
\pgfsys@defobject{currentmarker}{\pgfqpoint{0.000000in}{-0.027778in}}{\pgfqpoint{0.000000in}{0.000000in}}{%
\pgfpathmoveto{\pgfqpoint{0.000000in}{0.000000in}}%
\pgfpathlineto{\pgfqpoint{0.000000in}{-0.027778in}}%
\pgfusepath{stroke,fill}%
}%
\begin{pgfscope}%
\pgfsys@transformshift{1.121418in}{0.417642in}%
\pgfsys@useobject{currentmarker}{}%
\end{pgfscope}%
\end{pgfscope}%
\begin{pgfscope}%
\pgfpathrectangle{\pgfqpoint{0.594525in}{0.417642in}}{\pgfqpoint{3.423805in}{2.011535in}}%
\pgfusepath{clip}%
\pgfsetrectcap%
\pgfsetroundjoin%
\pgfsetlinewidth{0.803000pt}%
\definecolor{currentstroke}{rgb}{0.850000,0.850000,0.850000}%
\pgfsetstrokecolor{currentstroke}%
\pgfsetdash{}{0pt}%
\pgfpathmoveto{\pgfqpoint{1.218638in}{0.417642in}}%
\pgfpathlineto{\pgfqpoint{1.218638in}{2.429177in}}%
\pgfusepath{stroke}%
\end{pgfscope}%
\begin{pgfscope}%
\pgfsetbuttcap%
\pgfsetroundjoin%
\definecolor{currentfill}{rgb}{0.000000,0.000000,0.000000}%
\pgfsetfillcolor{currentfill}%
\pgfsetlinewidth{0.602250pt}%
\definecolor{currentstroke}{rgb}{0.000000,0.000000,0.000000}%
\pgfsetstrokecolor{currentstroke}%
\pgfsetdash{}{0pt}%
\pgfsys@defobject{currentmarker}{\pgfqpoint{0.000000in}{-0.027778in}}{\pgfqpoint{0.000000in}{0.000000in}}{%
\pgfpathmoveto{\pgfqpoint{0.000000in}{0.000000in}}%
\pgfpathlineto{\pgfqpoint{0.000000in}{-0.027778in}}%
\pgfusepath{stroke,fill}%
}%
\begin{pgfscope}%
\pgfsys@transformshift{1.218638in}{0.417642in}%
\pgfsys@useobject{currentmarker}{}%
\end{pgfscope}%
\end{pgfscope}%
\begin{pgfscope}%
\pgfpathrectangle{\pgfqpoint{0.594525in}{0.417642in}}{\pgfqpoint{3.423805in}{2.011535in}}%
\pgfusepath{clip}%
\pgfsetrectcap%
\pgfsetroundjoin%
\pgfsetlinewidth{0.803000pt}%
\definecolor{currentstroke}{rgb}{0.850000,0.850000,0.850000}%
\pgfsetstrokecolor{currentstroke}%
\pgfsetdash{}{0pt}%
\pgfpathmoveto{\pgfqpoint{1.294047in}{0.417642in}}%
\pgfpathlineto{\pgfqpoint{1.294047in}{2.429177in}}%
\pgfusepath{stroke}%
\end{pgfscope}%
\begin{pgfscope}%
\pgfsetbuttcap%
\pgfsetroundjoin%
\definecolor{currentfill}{rgb}{0.000000,0.000000,0.000000}%
\pgfsetfillcolor{currentfill}%
\pgfsetlinewidth{0.602250pt}%
\definecolor{currentstroke}{rgb}{0.000000,0.000000,0.000000}%
\pgfsetstrokecolor{currentstroke}%
\pgfsetdash{}{0pt}%
\pgfsys@defobject{currentmarker}{\pgfqpoint{0.000000in}{-0.027778in}}{\pgfqpoint{0.000000in}{0.000000in}}{%
\pgfpathmoveto{\pgfqpoint{0.000000in}{0.000000in}}%
\pgfpathlineto{\pgfqpoint{0.000000in}{-0.027778in}}%
\pgfusepath{stroke,fill}%
}%
\begin{pgfscope}%
\pgfsys@transformshift{1.294047in}{0.417642in}%
\pgfsys@useobject{currentmarker}{}%
\end{pgfscope}%
\end{pgfscope}%
\begin{pgfscope}%
\pgfpathrectangle{\pgfqpoint{0.594525in}{0.417642in}}{\pgfqpoint{3.423805in}{2.011535in}}%
\pgfusepath{clip}%
\pgfsetrectcap%
\pgfsetroundjoin%
\pgfsetlinewidth{0.803000pt}%
\definecolor{currentstroke}{rgb}{0.850000,0.850000,0.850000}%
\pgfsetstrokecolor{currentstroke}%
\pgfsetdash{}{0pt}%
\pgfpathmoveto{\pgfqpoint{1.355661in}{0.417642in}}%
\pgfpathlineto{\pgfqpoint{1.355661in}{2.429177in}}%
\pgfusepath{stroke}%
\end{pgfscope}%
\begin{pgfscope}%
\pgfsetbuttcap%
\pgfsetroundjoin%
\definecolor{currentfill}{rgb}{0.000000,0.000000,0.000000}%
\pgfsetfillcolor{currentfill}%
\pgfsetlinewidth{0.602250pt}%
\definecolor{currentstroke}{rgb}{0.000000,0.000000,0.000000}%
\pgfsetstrokecolor{currentstroke}%
\pgfsetdash{}{0pt}%
\pgfsys@defobject{currentmarker}{\pgfqpoint{0.000000in}{-0.027778in}}{\pgfqpoint{0.000000in}{0.000000in}}{%
\pgfpathmoveto{\pgfqpoint{0.000000in}{0.000000in}}%
\pgfpathlineto{\pgfqpoint{0.000000in}{-0.027778in}}%
\pgfusepath{stroke,fill}%
}%
\begin{pgfscope}%
\pgfsys@transformshift{1.355661in}{0.417642in}%
\pgfsys@useobject{currentmarker}{}%
\end{pgfscope}%
\end{pgfscope}%
\begin{pgfscope}%
\pgfpathrectangle{\pgfqpoint{0.594525in}{0.417642in}}{\pgfqpoint{3.423805in}{2.011535in}}%
\pgfusepath{clip}%
\pgfsetrectcap%
\pgfsetroundjoin%
\pgfsetlinewidth{0.803000pt}%
\definecolor{currentstroke}{rgb}{0.850000,0.850000,0.850000}%
\pgfsetstrokecolor{currentstroke}%
\pgfsetdash{}{0pt}%
\pgfpathmoveto{\pgfqpoint{1.407755in}{0.417642in}}%
\pgfpathlineto{\pgfqpoint{1.407755in}{2.429177in}}%
\pgfusepath{stroke}%
\end{pgfscope}%
\begin{pgfscope}%
\pgfsetbuttcap%
\pgfsetroundjoin%
\definecolor{currentfill}{rgb}{0.000000,0.000000,0.000000}%
\pgfsetfillcolor{currentfill}%
\pgfsetlinewidth{0.602250pt}%
\definecolor{currentstroke}{rgb}{0.000000,0.000000,0.000000}%
\pgfsetstrokecolor{currentstroke}%
\pgfsetdash{}{0pt}%
\pgfsys@defobject{currentmarker}{\pgfqpoint{0.000000in}{-0.027778in}}{\pgfqpoint{0.000000in}{0.000000in}}{%
\pgfpathmoveto{\pgfqpoint{0.000000in}{0.000000in}}%
\pgfpathlineto{\pgfqpoint{0.000000in}{-0.027778in}}%
\pgfusepath{stroke,fill}%
}%
\begin{pgfscope}%
\pgfsys@transformshift{1.407755in}{0.417642in}%
\pgfsys@useobject{currentmarker}{}%
\end{pgfscope}%
\end{pgfscope}%
\begin{pgfscope}%
\pgfpathrectangle{\pgfqpoint{0.594525in}{0.417642in}}{\pgfqpoint{3.423805in}{2.011535in}}%
\pgfusepath{clip}%
\pgfsetrectcap%
\pgfsetroundjoin%
\pgfsetlinewidth{0.803000pt}%
\definecolor{currentstroke}{rgb}{0.850000,0.850000,0.850000}%
\pgfsetstrokecolor{currentstroke}%
\pgfsetdash{}{0pt}%
\pgfpathmoveto{\pgfqpoint{1.452880in}{0.417642in}}%
\pgfpathlineto{\pgfqpoint{1.452880in}{2.429177in}}%
\pgfusepath{stroke}%
\end{pgfscope}%
\begin{pgfscope}%
\pgfsetbuttcap%
\pgfsetroundjoin%
\definecolor{currentfill}{rgb}{0.000000,0.000000,0.000000}%
\pgfsetfillcolor{currentfill}%
\pgfsetlinewidth{0.602250pt}%
\definecolor{currentstroke}{rgb}{0.000000,0.000000,0.000000}%
\pgfsetstrokecolor{currentstroke}%
\pgfsetdash{}{0pt}%
\pgfsys@defobject{currentmarker}{\pgfqpoint{0.000000in}{-0.027778in}}{\pgfqpoint{0.000000in}{0.000000in}}{%
\pgfpathmoveto{\pgfqpoint{0.000000in}{0.000000in}}%
\pgfpathlineto{\pgfqpoint{0.000000in}{-0.027778in}}%
\pgfusepath{stroke,fill}%
}%
\begin{pgfscope}%
\pgfsys@transformshift{1.452880in}{0.417642in}%
\pgfsys@useobject{currentmarker}{}%
\end{pgfscope}%
\end{pgfscope}%
\begin{pgfscope}%
\pgfpathrectangle{\pgfqpoint{0.594525in}{0.417642in}}{\pgfqpoint{3.423805in}{2.011535in}}%
\pgfusepath{clip}%
\pgfsetrectcap%
\pgfsetroundjoin%
\pgfsetlinewidth{0.803000pt}%
\definecolor{currentstroke}{rgb}{0.850000,0.850000,0.850000}%
\pgfsetstrokecolor{currentstroke}%
\pgfsetdash{}{0pt}%
\pgfpathmoveto{\pgfqpoint{1.492684in}{0.417642in}}%
\pgfpathlineto{\pgfqpoint{1.492684in}{2.429177in}}%
\pgfusepath{stroke}%
\end{pgfscope}%
\begin{pgfscope}%
\pgfsetbuttcap%
\pgfsetroundjoin%
\definecolor{currentfill}{rgb}{0.000000,0.000000,0.000000}%
\pgfsetfillcolor{currentfill}%
\pgfsetlinewidth{0.602250pt}%
\definecolor{currentstroke}{rgb}{0.000000,0.000000,0.000000}%
\pgfsetstrokecolor{currentstroke}%
\pgfsetdash{}{0pt}%
\pgfsys@defobject{currentmarker}{\pgfqpoint{0.000000in}{-0.027778in}}{\pgfqpoint{0.000000in}{0.000000in}}{%
\pgfpathmoveto{\pgfqpoint{0.000000in}{0.000000in}}%
\pgfpathlineto{\pgfqpoint{0.000000in}{-0.027778in}}%
\pgfusepath{stroke,fill}%
}%
\begin{pgfscope}%
\pgfsys@transformshift{1.492684in}{0.417642in}%
\pgfsys@useobject{currentmarker}{}%
\end{pgfscope}%
\end{pgfscope}%
\begin{pgfscope}%
\pgfpathrectangle{\pgfqpoint{0.594525in}{0.417642in}}{\pgfqpoint{3.423805in}{2.011535in}}%
\pgfusepath{clip}%
\pgfsetrectcap%
\pgfsetroundjoin%
\pgfsetlinewidth{0.803000pt}%
\definecolor{currentstroke}{rgb}{0.850000,0.850000,0.850000}%
\pgfsetstrokecolor{currentstroke}%
\pgfsetdash{}{0pt}%
\pgfpathmoveto{\pgfqpoint{1.762533in}{0.417642in}}%
\pgfpathlineto{\pgfqpoint{1.762533in}{2.429177in}}%
\pgfusepath{stroke}%
\end{pgfscope}%
\begin{pgfscope}%
\pgfsetbuttcap%
\pgfsetroundjoin%
\definecolor{currentfill}{rgb}{0.000000,0.000000,0.000000}%
\pgfsetfillcolor{currentfill}%
\pgfsetlinewidth{0.602250pt}%
\definecolor{currentstroke}{rgb}{0.000000,0.000000,0.000000}%
\pgfsetstrokecolor{currentstroke}%
\pgfsetdash{}{0pt}%
\pgfsys@defobject{currentmarker}{\pgfqpoint{0.000000in}{-0.027778in}}{\pgfqpoint{0.000000in}{0.000000in}}{%
\pgfpathmoveto{\pgfqpoint{0.000000in}{0.000000in}}%
\pgfpathlineto{\pgfqpoint{0.000000in}{-0.027778in}}%
\pgfusepath{stroke,fill}%
}%
\begin{pgfscope}%
\pgfsys@transformshift{1.762533in}{0.417642in}%
\pgfsys@useobject{currentmarker}{}%
\end{pgfscope}%
\end{pgfscope}%
\begin{pgfscope}%
\pgfpathrectangle{\pgfqpoint{0.594525in}{0.417642in}}{\pgfqpoint{3.423805in}{2.011535in}}%
\pgfusepath{clip}%
\pgfsetrectcap%
\pgfsetroundjoin%
\pgfsetlinewidth{0.803000pt}%
\definecolor{currentstroke}{rgb}{0.850000,0.850000,0.850000}%
\pgfsetstrokecolor{currentstroke}%
\pgfsetdash{}{0pt}%
\pgfpathmoveto{\pgfqpoint{1.899556in}{0.417642in}}%
\pgfpathlineto{\pgfqpoint{1.899556in}{2.429177in}}%
\pgfusepath{stroke}%
\end{pgfscope}%
\begin{pgfscope}%
\pgfsetbuttcap%
\pgfsetroundjoin%
\definecolor{currentfill}{rgb}{0.000000,0.000000,0.000000}%
\pgfsetfillcolor{currentfill}%
\pgfsetlinewidth{0.602250pt}%
\definecolor{currentstroke}{rgb}{0.000000,0.000000,0.000000}%
\pgfsetstrokecolor{currentstroke}%
\pgfsetdash{}{0pt}%
\pgfsys@defobject{currentmarker}{\pgfqpoint{0.000000in}{-0.027778in}}{\pgfqpoint{0.000000in}{0.000000in}}{%
\pgfpathmoveto{\pgfqpoint{0.000000in}{0.000000in}}%
\pgfpathlineto{\pgfqpoint{0.000000in}{-0.027778in}}%
\pgfusepath{stroke,fill}%
}%
\begin{pgfscope}%
\pgfsys@transformshift{1.899556in}{0.417642in}%
\pgfsys@useobject{currentmarker}{}%
\end{pgfscope}%
\end{pgfscope}%
\begin{pgfscope}%
\pgfpathrectangle{\pgfqpoint{0.594525in}{0.417642in}}{\pgfqpoint{3.423805in}{2.011535in}}%
\pgfusepath{clip}%
\pgfsetrectcap%
\pgfsetroundjoin%
\pgfsetlinewidth{0.803000pt}%
\definecolor{currentstroke}{rgb}{0.850000,0.850000,0.850000}%
\pgfsetstrokecolor{currentstroke}%
\pgfsetdash{}{0pt}%
\pgfpathmoveto{\pgfqpoint{1.996775in}{0.417642in}}%
\pgfpathlineto{\pgfqpoint{1.996775in}{2.429177in}}%
\pgfusepath{stroke}%
\end{pgfscope}%
\begin{pgfscope}%
\pgfsetbuttcap%
\pgfsetroundjoin%
\definecolor{currentfill}{rgb}{0.000000,0.000000,0.000000}%
\pgfsetfillcolor{currentfill}%
\pgfsetlinewidth{0.602250pt}%
\definecolor{currentstroke}{rgb}{0.000000,0.000000,0.000000}%
\pgfsetstrokecolor{currentstroke}%
\pgfsetdash{}{0pt}%
\pgfsys@defobject{currentmarker}{\pgfqpoint{0.000000in}{-0.027778in}}{\pgfqpoint{0.000000in}{0.000000in}}{%
\pgfpathmoveto{\pgfqpoint{0.000000in}{0.000000in}}%
\pgfpathlineto{\pgfqpoint{0.000000in}{-0.027778in}}%
\pgfusepath{stroke,fill}%
}%
\begin{pgfscope}%
\pgfsys@transformshift{1.996775in}{0.417642in}%
\pgfsys@useobject{currentmarker}{}%
\end{pgfscope}%
\end{pgfscope}%
\begin{pgfscope}%
\pgfpathrectangle{\pgfqpoint{0.594525in}{0.417642in}}{\pgfqpoint{3.423805in}{2.011535in}}%
\pgfusepath{clip}%
\pgfsetrectcap%
\pgfsetroundjoin%
\pgfsetlinewidth{0.803000pt}%
\definecolor{currentstroke}{rgb}{0.850000,0.850000,0.850000}%
\pgfsetstrokecolor{currentstroke}%
\pgfsetdash{}{0pt}%
\pgfpathmoveto{\pgfqpoint{2.072185in}{0.417642in}}%
\pgfpathlineto{\pgfqpoint{2.072185in}{2.429177in}}%
\pgfusepath{stroke}%
\end{pgfscope}%
\begin{pgfscope}%
\pgfsetbuttcap%
\pgfsetroundjoin%
\definecolor{currentfill}{rgb}{0.000000,0.000000,0.000000}%
\pgfsetfillcolor{currentfill}%
\pgfsetlinewidth{0.602250pt}%
\definecolor{currentstroke}{rgb}{0.000000,0.000000,0.000000}%
\pgfsetstrokecolor{currentstroke}%
\pgfsetdash{}{0pt}%
\pgfsys@defobject{currentmarker}{\pgfqpoint{0.000000in}{-0.027778in}}{\pgfqpoint{0.000000in}{0.000000in}}{%
\pgfpathmoveto{\pgfqpoint{0.000000in}{0.000000in}}%
\pgfpathlineto{\pgfqpoint{0.000000in}{-0.027778in}}%
\pgfusepath{stroke,fill}%
}%
\begin{pgfscope}%
\pgfsys@transformshift{2.072185in}{0.417642in}%
\pgfsys@useobject{currentmarker}{}%
\end{pgfscope}%
\end{pgfscope}%
\begin{pgfscope}%
\pgfpathrectangle{\pgfqpoint{0.594525in}{0.417642in}}{\pgfqpoint{3.423805in}{2.011535in}}%
\pgfusepath{clip}%
\pgfsetrectcap%
\pgfsetroundjoin%
\pgfsetlinewidth{0.803000pt}%
\definecolor{currentstroke}{rgb}{0.850000,0.850000,0.850000}%
\pgfsetstrokecolor{currentstroke}%
\pgfsetdash{}{0pt}%
\pgfpathmoveto{\pgfqpoint{2.133799in}{0.417642in}}%
\pgfpathlineto{\pgfqpoint{2.133799in}{2.429177in}}%
\pgfusepath{stroke}%
\end{pgfscope}%
\begin{pgfscope}%
\pgfsetbuttcap%
\pgfsetroundjoin%
\definecolor{currentfill}{rgb}{0.000000,0.000000,0.000000}%
\pgfsetfillcolor{currentfill}%
\pgfsetlinewidth{0.602250pt}%
\definecolor{currentstroke}{rgb}{0.000000,0.000000,0.000000}%
\pgfsetstrokecolor{currentstroke}%
\pgfsetdash{}{0pt}%
\pgfsys@defobject{currentmarker}{\pgfqpoint{0.000000in}{-0.027778in}}{\pgfqpoint{0.000000in}{0.000000in}}{%
\pgfpathmoveto{\pgfqpoint{0.000000in}{0.000000in}}%
\pgfpathlineto{\pgfqpoint{0.000000in}{-0.027778in}}%
\pgfusepath{stroke,fill}%
}%
\begin{pgfscope}%
\pgfsys@transformshift{2.133799in}{0.417642in}%
\pgfsys@useobject{currentmarker}{}%
\end{pgfscope}%
\end{pgfscope}%
\begin{pgfscope}%
\pgfpathrectangle{\pgfqpoint{0.594525in}{0.417642in}}{\pgfqpoint{3.423805in}{2.011535in}}%
\pgfusepath{clip}%
\pgfsetrectcap%
\pgfsetroundjoin%
\pgfsetlinewidth{0.803000pt}%
\definecolor{currentstroke}{rgb}{0.850000,0.850000,0.850000}%
\pgfsetstrokecolor{currentstroke}%
\pgfsetdash{}{0pt}%
\pgfpathmoveto{\pgfqpoint{2.185892in}{0.417642in}}%
\pgfpathlineto{\pgfqpoint{2.185892in}{2.429177in}}%
\pgfusepath{stroke}%
\end{pgfscope}%
\begin{pgfscope}%
\pgfsetbuttcap%
\pgfsetroundjoin%
\definecolor{currentfill}{rgb}{0.000000,0.000000,0.000000}%
\pgfsetfillcolor{currentfill}%
\pgfsetlinewidth{0.602250pt}%
\definecolor{currentstroke}{rgb}{0.000000,0.000000,0.000000}%
\pgfsetstrokecolor{currentstroke}%
\pgfsetdash{}{0pt}%
\pgfsys@defobject{currentmarker}{\pgfqpoint{0.000000in}{-0.027778in}}{\pgfqpoint{0.000000in}{0.000000in}}{%
\pgfpathmoveto{\pgfqpoint{0.000000in}{0.000000in}}%
\pgfpathlineto{\pgfqpoint{0.000000in}{-0.027778in}}%
\pgfusepath{stroke,fill}%
}%
\begin{pgfscope}%
\pgfsys@transformshift{2.185892in}{0.417642in}%
\pgfsys@useobject{currentmarker}{}%
\end{pgfscope}%
\end{pgfscope}%
\begin{pgfscope}%
\pgfpathrectangle{\pgfqpoint{0.594525in}{0.417642in}}{\pgfqpoint{3.423805in}{2.011535in}}%
\pgfusepath{clip}%
\pgfsetrectcap%
\pgfsetroundjoin%
\pgfsetlinewidth{0.803000pt}%
\definecolor{currentstroke}{rgb}{0.850000,0.850000,0.850000}%
\pgfsetstrokecolor{currentstroke}%
\pgfsetdash{}{0pt}%
\pgfpathmoveto{\pgfqpoint{2.231018in}{0.417642in}}%
\pgfpathlineto{\pgfqpoint{2.231018in}{2.429177in}}%
\pgfusepath{stroke}%
\end{pgfscope}%
\begin{pgfscope}%
\pgfsetbuttcap%
\pgfsetroundjoin%
\definecolor{currentfill}{rgb}{0.000000,0.000000,0.000000}%
\pgfsetfillcolor{currentfill}%
\pgfsetlinewidth{0.602250pt}%
\definecolor{currentstroke}{rgb}{0.000000,0.000000,0.000000}%
\pgfsetstrokecolor{currentstroke}%
\pgfsetdash{}{0pt}%
\pgfsys@defobject{currentmarker}{\pgfqpoint{0.000000in}{-0.027778in}}{\pgfqpoint{0.000000in}{0.000000in}}{%
\pgfpathmoveto{\pgfqpoint{0.000000in}{0.000000in}}%
\pgfpathlineto{\pgfqpoint{0.000000in}{-0.027778in}}%
\pgfusepath{stroke,fill}%
}%
\begin{pgfscope}%
\pgfsys@transformshift{2.231018in}{0.417642in}%
\pgfsys@useobject{currentmarker}{}%
\end{pgfscope}%
\end{pgfscope}%
\begin{pgfscope}%
\pgfpathrectangle{\pgfqpoint{0.594525in}{0.417642in}}{\pgfqpoint{3.423805in}{2.011535in}}%
\pgfusepath{clip}%
\pgfsetrectcap%
\pgfsetroundjoin%
\pgfsetlinewidth{0.803000pt}%
\definecolor{currentstroke}{rgb}{0.850000,0.850000,0.850000}%
\pgfsetstrokecolor{currentstroke}%
\pgfsetdash{}{0pt}%
\pgfpathmoveto{\pgfqpoint{2.270822in}{0.417642in}}%
\pgfpathlineto{\pgfqpoint{2.270822in}{2.429177in}}%
\pgfusepath{stroke}%
\end{pgfscope}%
\begin{pgfscope}%
\pgfsetbuttcap%
\pgfsetroundjoin%
\definecolor{currentfill}{rgb}{0.000000,0.000000,0.000000}%
\pgfsetfillcolor{currentfill}%
\pgfsetlinewidth{0.602250pt}%
\definecolor{currentstroke}{rgb}{0.000000,0.000000,0.000000}%
\pgfsetstrokecolor{currentstroke}%
\pgfsetdash{}{0pt}%
\pgfsys@defobject{currentmarker}{\pgfqpoint{0.000000in}{-0.027778in}}{\pgfqpoint{0.000000in}{0.000000in}}{%
\pgfpathmoveto{\pgfqpoint{0.000000in}{0.000000in}}%
\pgfpathlineto{\pgfqpoint{0.000000in}{-0.027778in}}%
\pgfusepath{stroke,fill}%
}%
\begin{pgfscope}%
\pgfsys@transformshift{2.270822in}{0.417642in}%
\pgfsys@useobject{currentmarker}{}%
\end{pgfscope}%
\end{pgfscope}%
\begin{pgfscope}%
\pgfpathrectangle{\pgfqpoint{0.594525in}{0.417642in}}{\pgfqpoint{3.423805in}{2.011535in}}%
\pgfusepath{clip}%
\pgfsetrectcap%
\pgfsetroundjoin%
\pgfsetlinewidth{0.803000pt}%
\definecolor{currentstroke}{rgb}{0.850000,0.850000,0.850000}%
\pgfsetstrokecolor{currentstroke}%
\pgfsetdash{}{0pt}%
\pgfpathmoveto{\pgfqpoint{2.540670in}{0.417642in}}%
\pgfpathlineto{\pgfqpoint{2.540670in}{2.429177in}}%
\pgfusepath{stroke}%
\end{pgfscope}%
\begin{pgfscope}%
\pgfsetbuttcap%
\pgfsetroundjoin%
\definecolor{currentfill}{rgb}{0.000000,0.000000,0.000000}%
\pgfsetfillcolor{currentfill}%
\pgfsetlinewidth{0.602250pt}%
\definecolor{currentstroke}{rgb}{0.000000,0.000000,0.000000}%
\pgfsetstrokecolor{currentstroke}%
\pgfsetdash{}{0pt}%
\pgfsys@defobject{currentmarker}{\pgfqpoint{0.000000in}{-0.027778in}}{\pgfqpoint{0.000000in}{0.000000in}}{%
\pgfpathmoveto{\pgfqpoint{0.000000in}{0.000000in}}%
\pgfpathlineto{\pgfqpoint{0.000000in}{-0.027778in}}%
\pgfusepath{stroke,fill}%
}%
\begin{pgfscope}%
\pgfsys@transformshift{2.540670in}{0.417642in}%
\pgfsys@useobject{currentmarker}{}%
\end{pgfscope}%
\end{pgfscope}%
\begin{pgfscope}%
\pgfpathrectangle{\pgfqpoint{0.594525in}{0.417642in}}{\pgfqpoint{3.423805in}{2.011535in}}%
\pgfusepath{clip}%
\pgfsetrectcap%
\pgfsetroundjoin%
\pgfsetlinewidth{0.803000pt}%
\definecolor{currentstroke}{rgb}{0.850000,0.850000,0.850000}%
\pgfsetstrokecolor{currentstroke}%
\pgfsetdash{}{0pt}%
\pgfpathmoveto{\pgfqpoint{2.677693in}{0.417642in}}%
\pgfpathlineto{\pgfqpoint{2.677693in}{2.429177in}}%
\pgfusepath{stroke}%
\end{pgfscope}%
\begin{pgfscope}%
\pgfsetbuttcap%
\pgfsetroundjoin%
\definecolor{currentfill}{rgb}{0.000000,0.000000,0.000000}%
\pgfsetfillcolor{currentfill}%
\pgfsetlinewidth{0.602250pt}%
\definecolor{currentstroke}{rgb}{0.000000,0.000000,0.000000}%
\pgfsetstrokecolor{currentstroke}%
\pgfsetdash{}{0pt}%
\pgfsys@defobject{currentmarker}{\pgfqpoint{0.000000in}{-0.027778in}}{\pgfqpoint{0.000000in}{0.000000in}}{%
\pgfpathmoveto{\pgfqpoint{0.000000in}{0.000000in}}%
\pgfpathlineto{\pgfqpoint{0.000000in}{-0.027778in}}%
\pgfusepath{stroke,fill}%
}%
\begin{pgfscope}%
\pgfsys@transformshift{2.677693in}{0.417642in}%
\pgfsys@useobject{currentmarker}{}%
\end{pgfscope}%
\end{pgfscope}%
\begin{pgfscope}%
\pgfpathrectangle{\pgfqpoint{0.594525in}{0.417642in}}{\pgfqpoint{3.423805in}{2.011535in}}%
\pgfusepath{clip}%
\pgfsetrectcap%
\pgfsetroundjoin%
\pgfsetlinewidth{0.803000pt}%
\definecolor{currentstroke}{rgb}{0.850000,0.850000,0.850000}%
\pgfsetstrokecolor{currentstroke}%
\pgfsetdash{}{0pt}%
\pgfpathmoveto{\pgfqpoint{2.774913in}{0.417642in}}%
\pgfpathlineto{\pgfqpoint{2.774913in}{2.429177in}}%
\pgfusepath{stroke}%
\end{pgfscope}%
\begin{pgfscope}%
\pgfsetbuttcap%
\pgfsetroundjoin%
\definecolor{currentfill}{rgb}{0.000000,0.000000,0.000000}%
\pgfsetfillcolor{currentfill}%
\pgfsetlinewidth{0.602250pt}%
\definecolor{currentstroke}{rgb}{0.000000,0.000000,0.000000}%
\pgfsetstrokecolor{currentstroke}%
\pgfsetdash{}{0pt}%
\pgfsys@defobject{currentmarker}{\pgfqpoint{0.000000in}{-0.027778in}}{\pgfqpoint{0.000000in}{0.000000in}}{%
\pgfpathmoveto{\pgfqpoint{0.000000in}{0.000000in}}%
\pgfpathlineto{\pgfqpoint{0.000000in}{-0.027778in}}%
\pgfusepath{stroke,fill}%
}%
\begin{pgfscope}%
\pgfsys@transformshift{2.774913in}{0.417642in}%
\pgfsys@useobject{currentmarker}{}%
\end{pgfscope}%
\end{pgfscope}%
\begin{pgfscope}%
\pgfpathrectangle{\pgfqpoint{0.594525in}{0.417642in}}{\pgfqpoint{3.423805in}{2.011535in}}%
\pgfusepath{clip}%
\pgfsetrectcap%
\pgfsetroundjoin%
\pgfsetlinewidth{0.803000pt}%
\definecolor{currentstroke}{rgb}{0.850000,0.850000,0.850000}%
\pgfsetstrokecolor{currentstroke}%
\pgfsetdash{}{0pt}%
\pgfpathmoveto{\pgfqpoint{2.850322in}{0.417642in}}%
\pgfpathlineto{\pgfqpoint{2.850322in}{2.429177in}}%
\pgfusepath{stroke}%
\end{pgfscope}%
\begin{pgfscope}%
\pgfsetbuttcap%
\pgfsetroundjoin%
\definecolor{currentfill}{rgb}{0.000000,0.000000,0.000000}%
\pgfsetfillcolor{currentfill}%
\pgfsetlinewidth{0.602250pt}%
\definecolor{currentstroke}{rgb}{0.000000,0.000000,0.000000}%
\pgfsetstrokecolor{currentstroke}%
\pgfsetdash{}{0pt}%
\pgfsys@defobject{currentmarker}{\pgfqpoint{0.000000in}{-0.027778in}}{\pgfqpoint{0.000000in}{0.000000in}}{%
\pgfpathmoveto{\pgfqpoint{0.000000in}{0.000000in}}%
\pgfpathlineto{\pgfqpoint{0.000000in}{-0.027778in}}%
\pgfusepath{stroke,fill}%
}%
\begin{pgfscope}%
\pgfsys@transformshift{2.850322in}{0.417642in}%
\pgfsys@useobject{currentmarker}{}%
\end{pgfscope}%
\end{pgfscope}%
\begin{pgfscope}%
\pgfpathrectangle{\pgfqpoint{0.594525in}{0.417642in}}{\pgfqpoint{3.423805in}{2.011535in}}%
\pgfusepath{clip}%
\pgfsetrectcap%
\pgfsetroundjoin%
\pgfsetlinewidth{0.803000pt}%
\definecolor{currentstroke}{rgb}{0.850000,0.850000,0.850000}%
\pgfsetstrokecolor{currentstroke}%
\pgfsetdash{}{0pt}%
\pgfpathmoveto{\pgfqpoint{2.911936in}{0.417642in}}%
\pgfpathlineto{\pgfqpoint{2.911936in}{2.429177in}}%
\pgfusepath{stroke}%
\end{pgfscope}%
\begin{pgfscope}%
\pgfsetbuttcap%
\pgfsetroundjoin%
\definecolor{currentfill}{rgb}{0.000000,0.000000,0.000000}%
\pgfsetfillcolor{currentfill}%
\pgfsetlinewidth{0.602250pt}%
\definecolor{currentstroke}{rgb}{0.000000,0.000000,0.000000}%
\pgfsetstrokecolor{currentstroke}%
\pgfsetdash{}{0pt}%
\pgfsys@defobject{currentmarker}{\pgfqpoint{0.000000in}{-0.027778in}}{\pgfqpoint{0.000000in}{0.000000in}}{%
\pgfpathmoveto{\pgfqpoint{0.000000in}{0.000000in}}%
\pgfpathlineto{\pgfqpoint{0.000000in}{-0.027778in}}%
\pgfusepath{stroke,fill}%
}%
\begin{pgfscope}%
\pgfsys@transformshift{2.911936in}{0.417642in}%
\pgfsys@useobject{currentmarker}{}%
\end{pgfscope}%
\end{pgfscope}%
\begin{pgfscope}%
\pgfpathrectangle{\pgfqpoint{0.594525in}{0.417642in}}{\pgfqpoint{3.423805in}{2.011535in}}%
\pgfusepath{clip}%
\pgfsetrectcap%
\pgfsetroundjoin%
\pgfsetlinewidth{0.803000pt}%
\definecolor{currentstroke}{rgb}{0.850000,0.850000,0.850000}%
\pgfsetstrokecolor{currentstroke}%
\pgfsetdash{}{0pt}%
\pgfpathmoveto{\pgfqpoint{2.964030in}{0.417642in}}%
\pgfpathlineto{\pgfqpoint{2.964030in}{2.429177in}}%
\pgfusepath{stroke}%
\end{pgfscope}%
\begin{pgfscope}%
\pgfsetbuttcap%
\pgfsetroundjoin%
\definecolor{currentfill}{rgb}{0.000000,0.000000,0.000000}%
\pgfsetfillcolor{currentfill}%
\pgfsetlinewidth{0.602250pt}%
\definecolor{currentstroke}{rgb}{0.000000,0.000000,0.000000}%
\pgfsetstrokecolor{currentstroke}%
\pgfsetdash{}{0pt}%
\pgfsys@defobject{currentmarker}{\pgfqpoint{0.000000in}{-0.027778in}}{\pgfqpoint{0.000000in}{0.000000in}}{%
\pgfpathmoveto{\pgfqpoint{0.000000in}{0.000000in}}%
\pgfpathlineto{\pgfqpoint{0.000000in}{-0.027778in}}%
\pgfusepath{stroke,fill}%
}%
\begin{pgfscope}%
\pgfsys@transformshift{2.964030in}{0.417642in}%
\pgfsys@useobject{currentmarker}{}%
\end{pgfscope}%
\end{pgfscope}%
\begin{pgfscope}%
\pgfpathrectangle{\pgfqpoint{0.594525in}{0.417642in}}{\pgfqpoint{3.423805in}{2.011535in}}%
\pgfusepath{clip}%
\pgfsetrectcap%
\pgfsetroundjoin%
\pgfsetlinewidth{0.803000pt}%
\definecolor{currentstroke}{rgb}{0.850000,0.850000,0.850000}%
\pgfsetstrokecolor{currentstroke}%
\pgfsetdash{}{0pt}%
\pgfpathmoveto{\pgfqpoint{3.009156in}{0.417642in}}%
\pgfpathlineto{\pgfqpoint{3.009156in}{2.429177in}}%
\pgfusepath{stroke}%
\end{pgfscope}%
\begin{pgfscope}%
\pgfsetbuttcap%
\pgfsetroundjoin%
\definecolor{currentfill}{rgb}{0.000000,0.000000,0.000000}%
\pgfsetfillcolor{currentfill}%
\pgfsetlinewidth{0.602250pt}%
\definecolor{currentstroke}{rgb}{0.000000,0.000000,0.000000}%
\pgfsetstrokecolor{currentstroke}%
\pgfsetdash{}{0pt}%
\pgfsys@defobject{currentmarker}{\pgfqpoint{0.000000in}{-0.027778in}}{\pgfqpoint{0.000000in}{0.000000in}}{%
\pgfpathmoveto{\pgfqpoint{0.000000in}{0.000000in}}%
\pgfpathlineto{\pgfqpoint{0.000000in}{-0.027778in}}%
\pgfusepath{stroke,fill}%
}%
\begin{pgfscope}%
\pgfsys@transformshift{3.009156in}{0.417642in}%
\pgfsys@useobject{currentmarker}{}%
\end{pgfscope}%
\end{pgfscope}%
\begin{pgfscope}%
\pgfpathrectangle{\pgfqpoint{0.594525in}{0.417642in}}{\pgfqpoint{3.423805in}{2.011535in}}%
\pgfusepath{clip}%
\pgfsetrectcap%
\pgfsetroundjoin%
\pgfsetlinewidth{0.803000pt}%
\definecolor{currentstroke}{rgb}{0.850000,0.850000,0.850000}%
\pgfsetstrokecolor{currentstroke}%
\pgfsetdash{}{0pt}%
\pgfpathmoveto{\pgfqpoint{3.048959in}{0.417642in}}%
\pgfpathlineto{\pgfqpoint{3.048959in}{2.429177in}}%
\pgfusepath{stroke}%
\end{pgfscope}%
\begin{pgfscope}%
\pgfsetbuttcap%
\pgfsetroundjoin%
\definecolor{currentfill}{rgb}{0.000000,0.000000,0.000000}%
\pgfsetfillcolor{currentfill}%
\pgfsetlinewidth{0.602250pt}%
\definecolor{currentstroke}{rgb}{0.000000,0.000000,0.000000}%
\pgfsetstrokecolor{currentstroke}%
\pgfsetdash{}{0pt}%
\pgfsys@defobject{currentmarker}{\pgfqpoint{0.000000in}{-0.027778in}}{\pgfqpoint{0.000000in}{0.000000in}}{%
\pgfpathmoveto{\pgfqpoint{0.000000in}{0.000000in}}%
\pgfpathlineto{\pgfqpoint{0.000000in}{-0.027778in}}%
\pgfusepath{stroke,fill}%
}%
\begin{pgfscope}%
\pgfsys@transformshift{3.048959in}{0.417642in}%
\pgfsys@useobject{currentmarker}{}%
\end{pgfscope}%
\end{pgfscope}%
\begin{pgfscope}%
\pgfpathrectangle{\pgfqpoint{0.594525in}{0.417642in}}{\pgfqpoint{3.423805in}{2.011535in}}%
\pgfusepath{clip}%
\pgfsetrectcap%
\pgfsetroundjoin%
\pgfsetlinewidth{0.803000pt}%
\definecolor{currentstroke}{rgb}{0.850000,0.850000,0.850000}%
\pgfsetstrokecolor{currentstroke}%
\pgfsetdash{}{0pt}%
\pgfpathmoveto{\pgfqpoint{3.318808in}{0.417642in}}%
\pgfpathlineto{\pgfqpoint{3.318808in}{2.429177in}}%
\pgfusepath{stroke}%
\end{pgfscope}%
\begin{pgfscope}%
\pgfsetbuttcap%
\pgfsetroundjoin%
\definecolor{currentfill}{rgb}{0.000000,0.000000,0.000000}%
\pgfsetfillcolor{currentfill}%
\pgfsetlinewidth{0.602250pt}%
\definecolor{currentstroke}{rgb}{0.000000,0.000000,0.000000}%
\pgfsetstrokecolor{currentstroke}%
\pgfsetdash{}{0pt}%
\pgfsys@defobject{currentmarker}{\pgfqpoint{0.000000in}{-0.027778in}}{\pgfqpoint{0.000000in}{0.000000in}}{%
\pgfpathmoveto{\pgfqpoint{0.000000in}{0.000000in}}%
\pgfpathlineto{\pgfqpoint{0.000000in}{-0.027778in}}%
\pgfusepath{stroke,fill}%
}%
\begin{pgfscope}%
\pgfsys@transformshift{3.318808in}{0.417642in}%
\pgfsys@useobject{currentmarker}{}%
\end{pgfscope}%
\end{pgfscope}%
\begin{pgfscope}%
\pgfpathrectangle{\pgfqpoint{0.594525in}{0.417642in}}{\pgfqpoint{3.423805in}{2.011535in}}%
\pgfusepath{clip}%
\pgfsetrectcap%
\pgfsetroundjoin%
\pgfsetlinewidth{0.803000pt}%
\definecolor{currentstroke}{rgb}{0.850000,0.850000,0.850000}%
\pgfsetstrokecolor{currentstroke}%
\pgfsetdash{}{0pt}%
\pgfpathmoveto{\pgfqpoint{3.455831in}{0.417642in}}%
\pgfpathlineto{\pgfqpoint{3.455831in}{2.429177in}}%
\pgfusepath{stroke}%
\end{pgfscope}%
\begin{pgfscope}%
\pgfsetbuttcap%
\pgfsetroundjoin%
\definecolor{currentfill}{rgb}{0.000000,0.000000,0.000000}%
\pgfsetfillcolor{currentfill}%
\pgfsetlinewidth{0.602250pt}%
\definecolor{currentstroke}{rgb}{0.000000,0.000000,0.000000}%
\pgfsetstrokecolor{currentstroke}%
\pgfsetdash{}{0pt}%
\pgfsys@defobject{currentmarker}{\pgfqpoint{0.000000in}{-0.027778in}}{\pgfqpoint{0.000000in}{0.000000in}}{%
\pgfpathmoveto{\pgfqpoint{0.000000in}{0.000000in}}%
\pgfpathlineto{\pgfqpoint{0.000000in}{-0.027778in}}%
\pgfusepath{stroke,fill}%
}%
\begin{pgfscope}%
\pgfsys@transformshift{3.455831in}{0.417642in}%
\pgfsys@useobject{currentmarker}{}%
\end{pgfscope}%
\end{pgfscope}%
\begin{pgfscope}%
\pgfpathrectangle{\pgfqpoint{0.594525in}{0.417642in}}{\pgfqpoint{3.423805in}{2.011535in}}%
\pgfusepath{clip}%
\pgfsetrectcap%
\pgfsetroundjoin%
\pgfsetlinewidth{0.803000pt}%
\definecolor{currentstroke}{rgb}{0.850000,0.850000,0.850000}%
\pgfsetstrokecolor{currentstroke}%
\pgfsetdash{}{0pt}%
\pgfpathmoveto{\pgfqpoint{3.553050in}{0.417642in}}%
\pgfpathlineto{\pgfqpoint{3.553050in}{2.429177in}}%
\pgfusepath{stroke}%
\end{pgfscope}%
\begin{pgfscope}%
\pgfsetbuttcap%
\pgfsetroundjoin%
\definecolor{currentfill}{rgb}{0.000000,0.000000,0.000000}%
\pgfsetfillcolor{currentfill}%
\pgfsetlinewidth{0.602250pt}%
\definecolor{currentstroke}{rgb}{0.000000,0.000000,0.000000}%
\pgfsetstrokecolor{currentstroke}%
\pgfsetdash{}{0pt}%
\pgfsys@defobject{currentmarker}{\pgfqpoint{0.000000in}{-0.027778in}}{\pgfqpoint{0.000000in}{0.000000in}}{%
\pgfpathmoveto{\pgfqpoint{0.000000in}{0.000000in}}%
\pgfpathlineto{\pgfqpoint{0.000000in}{-0.027778in}}%
\pgfusepath{stroke,fill}%
}%
\begin{pgfscope}%
\pgfsys@transformshift{3.553050in}{0.417642in}%
\pgfsys@useobject{currentmarker}{}%
\end{pgfscope}%
\end{pgfscope}%
\begin{pgfscope}%
\pgfpathrectangle{\pgfqpoint{0.594525in}{0.417642in}}{\pgfqpoint{3.423805in}{2.011535in}}%
\pgfusepath{clip}%
\pgfsetrectcap%
\pgfsetroundjoin%
\pgfsetlinewidth{0.803000pt}%
\definecolor{currentstroke}{rgb}{0.850000,0.850000,0.850000}%
\pgfsetstrokecolor{currentstroke}%
\pgfsetdash{}{0pt}%
\pgfpathmoveto{\pgfqpoint{3.628460in}{0.417642in}}%
\pgfpathlineto{\pgfqpoint{3.628460in}{2.429177in}}%
\pgfusepath{stroke}%
\end{pgfscope}%
\begin{pgfscope}%
\pgfsetbuttcap%
\pgfsetroundjoin%
\definecolor{currentfill}{rgb}{0.000000,0.000000,0.000000}%
\pgfsetfillcolor{currentfill}%
\pgfsetlinewidth{0.602250pt}%
\definecolor{currentstroke}{rgb}{0.000000,0.000000,0.000000}%
\pgfsetstrokecolor{currentstroke}%
\pgfsetdash{}{0pt}%
\pgfsys@defobject{currentmarker}{\pgfqpoint{0.000000in}{-0.027778in}}{\pgfqpoint{0.000000in}{0.000000in}}{%
\pgfpathmoveto{\pgfqpoint{0.000000in}{0.000000in}}%
\pgfpathlineto{\pgfqpoint{0.000000in}{-0.027778in}}%
\pgfusepath{stroke,fill}%
}%
\begin{pgfscope}%
\pgfsys@transformshift{3.628460in}{0.417642in}%
\pgfsys@useobject{currentmarker}{}%
\end{pgfscope}%
\end{pgfscope}%
\begin{pgfscope}%
\pgfpathrectangle{\pgfqpoint{0.594525in}{0.417642in}}{\pgfqpoint{3.423805in}{2.011535in}}%
\pgfusepath{clip}%
\pgfsetrectcap%
\pgfsetroundjoin%
\pgfsetlinewidth{0.803000pt}%
\definecolor{currentstroke}{rgb}{0.850000,0.850000,0.850000}%
\pgfsetstrokecolor{currentstroke}%
\pgfsetdash{}{0pt}%
\pgfpathmoveto{\pgfqpoint{3.690074in}{0.417642in}}%
\pgfpathlineto{\pgfqpoint{3.690074in}{2.429177in}}%
\pgfusepath{stroke}%
\end{pgfscope}%
\begin{pgfscope}%
\pgfsetbuttcap%
\pgfsetroundjoin%
\definecolor{currentfill}{rgb}{0.000000,0.000000,0.000000}%
\pgfsetfillcolor{currentfill}%
\pgfsetlinewidth{0.602250pt}%
\definecolor{currentstroke}{rgb}{0.000000,0.000000,0.000000}%
\pgfsetstrokecolor{currentstroke}%
\pgfsetdash{}{0pt}%
\pgfsys@defobject{currentmarker}{\pgfqpoint{0.000000in}{-0.027778in}}{\pgfqpoint{0.000000in}{0.000000in}}{%
\pgfpathmoveto{\pgfqpoint{0.000000in}{0.000000in}}%
\pgfpathlineto{\pgfqpoint{0.000000in}{-0.027778in}}%
\pgfusepath{stroke,fill}%
}%
\begin{pgfscope}%
\pgfsys@transformshift{3.690074in}{0.417642in}%
\pgfsys@useobject{currentmarker}{}%
\end{pgfscope}%
\end{pgfscope}%
\begin{pgfscope}%
\pgfpathrectangle{\pgfqpoint{0.594525in}{0.417642in}}{\pgfqpoint{3.423805in}{2.011535in}}%
\pgfusepath{clip}%
\pgfsetrectcap%
\pgfsetroundjoin%
\pgfsetlinewidth{0.803000pt}%
\definecolor{currentstroke}{rgb}{0.850000,0.850000,0.850000}%
\pgfsetstrokecolor{currentstroke}%
\pgfsetdash{}{0pt}%
\pgfpathmoveto{\pgfqpoint{3.742167in}{0.417642in}}%
\pgfpathlineto{\pgfqpoint{3.742167in}{2.429177in}}%
\pgfusepath{stroke}%
\end{pgfscope}%
\begin{pgfscope}%
\pgfsetbuttcap%
\pgfsetroundjoin%
\definecolor{currentfill}{rgb}{0.000000,0.000000,0.000000}%
\pgfsetfillcolor{currentfill}%
\pgfsetlinewidth{0.602250pt}%
\definecolor{currentstroke}{rgb}{0.000000,0.000000,0.000000}%
\pgfsetstrokecolor{currentstroke}%
\pgfsetdash{}{0pt}%
\pgfsys@defobject{currentmarker}{\pgfqpoint{0.000000in}{-0.027778in}}{\pgfqpoint{0.000000in}{0.000000in}}{%
\pgfpathmoveto{\pgfqpoint{0.000000in}{0.000000in}}%
\pgfpathlineto{\pgfqpoint{0.000000in}{-0.027778in}}%
\pgfusepath{stroke,fill}%
}%
\begin{pgfscope}%
\pgfsys@transformshift{3.742167in}{0.417642in}%
\pgfsys@useobject{currentmarker}{}%
\end{pgfscope}%
\end{pgfscope}%
\begin{pgfscope}%
\pgfpathrectangle{\pgfqpoint{0.594525in}{0.417642in}}{\pgfqpoint{3.423805in}{2.011535in}}%
\pgfusepath{clip}%
\pgfsetrectcap%
\pgfsetroundjoin%
\pgfsetlinewidth{0.803000pt}%
\definecolor{currentstroke}{rgb}{0.850000,0.850000,0.850000}%
\pgfsetstrokecolor{currentstroke}%
\pgfsetdash{}{0pt}%
\pgfpathmoveto{\pgfqpoint{3.787293in}{0.417642in}}%
\pgfpathlineto{\pgfqpoint{3.787293in}{2.429177in}}%
\pgfusepath{stroke}%
\end{pgfscope}%
\begin{pgfscope}%
\pgfsetbuttcap%
\pgfsetroundjoin%
\definecolor{currentfill}{rgb}{0.000000,0.000000,0.000000}%
\pgfsetfillcolor{currentfill}%
\pgfsetlinewidth{0.602250pt}%
\definecolor{currentstroke}{rgb}{0.000000,0.000000,0.000000}%
\pgfsetstrokecolor{currentstroke}%
\pgfsetdash{}{0pt}%
\pgfsys@defobject{currentmarker}{\pgfqpoint{0.000000in}{-0.027778in}}{\pgfqpoint{0.000000in}{0.000000in}}{%
\pgfpathmoveto{\pgfqpoint{0.000000in}{0.000000in}}%
\pgfpathlineto{\pgfqpoint{0.000000in}{-0.027778in}}%
\pgfusepath{stroke,fill}%
}%
\begin{pgfscope}%
\pgfsys@transformshift{3.787293in}{0.417642in}%
\pgfsys@useobject{currentmarker}{}%
\end{pgfscope}%
\end{pgfscope}%
\begin{pgfscope}%
\pgfpathrectangle{\pgfqpoint{0.594525in}{0.417642in}}{\pgfqpoint{3.423805in}{2.011535in}}%
\pgfusepath{clip}%
\pgfsetrectcap%
\pgfsetroundjoin%
\pgfsetlinewidth{0.803000pt}%
\definecolor{currentstroke}{rgb}{0.850000,0.850000,0.850000}%
\pgfsetstrokecolor{currentstroke}%
\pgfsetdash{}{0pt}%
\pgfpathmoveto{\pgfqpoint{3.827097in}{0.417642in}}%
\pgfpathlineto{\pgfqpoint{3.827097in}{2.429177in}}%
\pgfusepath{stroke}%
\end{pgfscope}%
\begin{pgfscope}%
\pgfsetbuttcap%
\pgfsetroundjoin%
\definecolor{currentfill}{rgb}{0.000000,0.000000,0.000000}%
\pgfsetfillcolor{currentfill}%
\pgfsetlinewidth{0.602250pt}%
\definecolor{currentstroke}{rgb}{0.000000,0.000000,0.000000}%
\pgfsetstrokecolor{currentstroke}%
\pgfsetdash{}{0pt}%
\pgfsys@defobject{currentmarker}{\pgfqpoint{0.000000in}{-0.027778in}}{\pgfqpoint{0.000000in}{0.000000in}}{%
\pgfpathmoveto{\pgfqpoint{0.000000in}{0.000000in}}%
\pgfpathlineto{\pgfqpoint{0.000000in}{-0.027778in}}%
\pgfusepath{stroke,fill}%
}%
\begin{pgfscope}%
\pgfsys@transformshift{3.827097in}{0.417642in}%
\pgfsys@useobject{currentmarker}{}%
\end{pgfscope}%
\end{pgfscope}%
\begin{pgfscope}%
\definecolor{textcolor}{rgb}{0.000000,0.000000,0.000000}%
\pgfsetstrokecolor{textcolor}%
\pgfsetfillcolor{textcolor}%
\pgftext[x=2.306427in,y=0.165003in,,top]{\color{textcolor}\rmfamily\fontsize{10.000000}{12.000000}\selectfont Frequency in \(\displaystyle \unit{\Hz}\)}%
\end{pgfscope}%
\begin{pgfscope}%
\pgfpathrectangle{\pgfqpoint{0.594525in}{0.417642in}}{\pgfqpoint{3.423805in}{2.011535in}}%
\pgfusepath{clip}%
\pgfsetrectcap%
\pgfsetroundjoin%
\pgfsetlinewidth{0.803000pt}%
\definecolor{currentstroke}{rgb}{0.450000,0.450000,0.450000}%
\pgfsetstrokecolor{currentstroke}%
\pgfsetdash{}{0pt}%
\pgfpathmoveto{\pgfqpoint{0.594525in}{0.417642in}}%
\pgfpathlineto{\pgfqpoint{4.018330in}{0.417642in}}%
\pgfusepath{stroke}%
\end{pgfscope}%
\begin{pgfscope}%
\pgfsetbuttcap%
\pgfsetroundjoin%
\definecolor{currentfill}{rgb}{0.000000,0.000000,0.000000}%
\pgfsetfillcolor{currentfill}%
\pgfsetlinewidth{0.803000pt}%
\definecolor{currentstroke}{rgb}{0.000000,0.000000,0.000000}%
\pgfsetstrokecolor{currentstroke}%
\pgfsetdash{}{0pt}%
\pgfsys@defobject{currentmarker}{\pgfqpoint{-0.048611in}{0.000000in}}{\pgfqpoint{-0.000000in}{0.000000in}}{%
\pgfpathmoveto{\pgfqpoint{-0.000000in}{0.000000in}}%
\pgfpathlineto{\pgfqpoint{-0.048611in}{0.000000in}}%
\pgfusepath{stroke,fill}%
}%
\begin{pgfscope}%
\pgfsys@transformshift{0.594525in}{0.417642in}%
\pgfsys@useobject{currentmarker}{}%
\end{pgfscope}%
\end{pgfscope}%
\begin{pgfscope}%
\definecolor{textcolor}{rgb}{0.000000,0.000000,0.000000}%
\pgfsetstrokecolor{textcolor}%
\pgfsetfillcolor{textcolor}%
\pgftext[x=0.241129in, y=0.378489in, left, base]{\color{textcolor}\rmfamily\fontsize{8.000000}{9.600000}\selectfont \(\displaystyle {10^{-4}}\)}%
\end{pgfscope}%
\begin{pgfscope}%
\pgfpathrectangle{\pgfqpoint{0.594525in}{0.417642in}}{\pgfqpoint{3.423805in}{2.011535in}}%
\pgfusepath{clip}%
\pgfsetrectcap%
\pgfsetroundjoin%
\pgfsetlinewidth{0.803000pt}%
\definecolor{currentstroke}{rgb}{0.450000,0.450000,0.450000}%
\pgfsetstrokecolor{currentstroke}%
\pgfsetdash{}{0pt}%
\pgfpathmoveto{\pgfqpoint{0.594525in}{0.819949in}}%
\pgfpathlineto{\pgfqpoint{4.018330in}{0.819949in}}%
\pgfusepath{stroke}%
\end{pgfscope}%
\begin{pgfscope}%
\pgfsetbuttcap%
\pgfsetroundjoin%
\definecolor{currentfill}{rgb}{0.000000,0.000000,0.000000}%
\pgfsetfillcolor{currentfill}%
\pgfsetlinewidth{0.803000pt}%
\definecolor{currentstroke}{rgb}{0.000000,0.000000,0.000000}%
\pgfsetstrokecolor{currentstroke}%
\pgfsetdash{}{0pt}%
\pgfsys@defobject{currentmarker}{\pgfqpoint{-0.048611in}{0.000000in}}{\pgfqpoint{-0.000000in}{0.000000in}}{%
\pgfpathmoveto{\pgfqpoint{-0.000000in}{0.000000in}}%
\pgfpathlineto{\pgfqpoint{-0.048611in}{0.000000in}}%
\pgfusepath{stroke,fill}%
}%
\begin{pgfscope}%
\pgfsys@transformshift{0.594525in}{0.819949in}%
\pgfsys@useobject{currentmarker}{}%
\end{pgfscope}%
\end{pgfscope}%
\begin{pgfscope}%
\definecolor{textcolor}{rgb}{0.000000,0.000000,0.000000}%
\pgfsetstrokecolor{textcolor}%
\pgfsetfillcolor{textcolor}%
\pgftext[x=0.241129in, y=0.780796in, left, base]{\color{textcolor}\rmfamily\fontsize{8.000000}{9.600000}\selectfont \(\displaystyle {10^{-3}}\)}%
\end{pgfscope}%
\begin{pgfscope}%
\pgfpathrectangle{\pgfqpoint{0.594525in}{0.417642in}}{\pgfqpoint{3.423805in}{2.011535in}}%
\pgfusepath{clip}%
\pgfsetrectcap%
\pgfsetroundjoin%
\pgfsetlinewidth{0.803000pt}%
\definecolor{currentstroke}{rgb}{0.450000,0.450000,0.450000}%
\pgfsetstrokecolor{currentstroke}%
\pgfsetdash{}{0pt}%
\pgfpathmoveto{\pgfqpoint{0.594525in}{1.222256in}}%
\pgfpathlineto{\pgfqpoint{4.018330in}{1.222256in}}%
\pgfusepath{stroke}%
\end{pgfscope}%
\begin{pgfscope}%
\pgfsetbuttcap%
\pgfsetroundjoin%
\definecolor{currentfill}{rgb}{0.000000,0.000000,0.000000}%
\pgfsetfillcolor{currentfill}%
\pgfsetlinewidth{0.803000pt}%
\definecolor{currentstroke}{rgb}{0.000000,0.000000,0.000000}%
\pgfsetstrokecolor{currentstroke}%
\pgfsetdash{}{0pt}%
\pgfsys@defobject{currentmarker}{\pgfqpoint{-0.048611in}{0.000000in}}{\pgfqpoint{-0.000000in}{0.000000in}}{%
\pgfpathmoveto{\pgfqpoint{-0.000000in}{0.000000in}}%
\pgfpathlineto{\pgfqpoint{-0.048611in}{0.000000in}}%
\pgfusepath{stroke,fill}%
}%
\begin{pgfscope}%
\pgfsys@transformshift{0.594525in}{1.222256in}%
\pgfsys@useobject{currentmarker}{}%
\end{pgfscope}%
\end{pgfscope}%
\begin{pgfscope}%
\definecolor{textcolor}{rgb}{0.000000,0.000000,0.000000}%
\pgfsetstrokecolor{textcolor}%
\pgfsetfillcolor{textcolor}%
\pgftext[x=0.241129in, y=1.183103in, left, base]{\color{textcolor}\rmfamily\fontsize{8.000000}{9.600000}\selectfont \(\displaystyle {10^{-2}}\)}%
\end{pgfscope}%
\begin{pgfscope}%
\pgfpathrectangle{\pgfqpoint{0.594525in}{0.417642in}}{\pgfqpoint{3.423805in}{2.011535in}}%
\pgfusepath{clip}%
\pgfsetrectcap%
\pgfsetroundjoin%
\pgfsetlinewidth{0.803000pt}%
\definecolor{currentstroke}{rgb}{0.450000,0.450000,0.450000}%
\pgfsetstrokecolor{currentstroke}%
\pgfsetdash{}{0pt}%
\pgfpathmoveto{\pgfqpoint{0.594525in}{1.624563in}}%
\pgfpathlineto{\pgfqpoint{4.018330in}{1.624563in}}%
\pgfusepath{stroke}%
\end{pgfscope}%
\begin{pgfscope}%
\pgfsetbuttcap%
\pgfsetroundjoin%
\definecolor{currentfill}{rgb}{0.000000,0.000000,0.000000}%
\pgfsetfillcolor{currentfill}%
\pgfsetlinewidth{0.803000pt}%
\definecolor{currentstroke}{rgb}{0.000000,0.000000,0.000000}%
\pgfsetstrokecolor{currentstroke}%
\pgfsetdash{}{0pt}%
\pgfsys@defobject{currentmarker}{\pgfqpoint{-0.048611in}{0.000000in}}{\pgfqpoint{-0.000000in}{0.000000in}}{%
\pgfpathmoveto{\pgfqpoint{-0.000000in}{0.000000in}}%
\pgfpathlineto{\pgfqpoint{-0.048611in}{0.000000in}}%
\pgfusepath{stroke,fill}%
}%
\begin{pgfscope}%
\pgfsys@transformshift{0.594525in}{1.624563in}%
\pgfsys@useobject{currentmarker}{}%
\end{pgfscope}%
\end{pgfscope}%
\begin{pgfscope}%
\definecolor{textcolor}{rgb}{0.000000,0.000000,0.000000}%
\pgfsetstrokecolor{textcolor}%
\pgfsetfillcolor{textcolor}%
\pgftext[x=0.241129in, y=1.585410in, left, base]{\color{textcolor}\rmfamily\fontsize{8.000000}{9.600000}\selectfont \(\displaystyle {10^{-1}}\)}%
\end{pgfscope}%
\begin{pgfscope}%
\pgfpathrectangle{\pgfqpoint{0.594525in}{0.417642in}}{\pgfqpoint{3.423805in}{2.011535in}}%
\pgfusepath{clip}%
\pgfsetrectcap%
\pgfsetroundjoin%
\pgfsetlinewidth{0.803000pt}%
\definecolor{currentstroke}{rgb}{0.450000,0.450000,0.450000}%
\pgfsetstrokecolor{currentstroke}%
\pgfsetdash{}{0pt}%
\pgfpathmoveto{\pgfqpoint{0.594525in}{2.026870in}}%
\pgfpathlineto{\pgfqpoint{4.018330in}{2.026870in}}%
\pgfusepath{stroke}%
\end{pgfscope}%
\begin{pgfscope}%
\pgfsetbuttcap%
\pgfsetroundjoin%
\definecolor{currentfill}{rgb}{0.000000,0.000000,0.000000}%
\pgfsetfillcolor{currentfill}%
\pgfsetlinewidth{0.803000pt}%
\definecolor{currentstroke}{rgb}{0.000000,0.000000,0.000000}%
\pgfsetstrokecolor{currentstroke}%
\pgfsetdash{}{0pt}%
\pgfsys@defobject{currentmarker}{\pgfqpoint{-0.048611in}{0.000000in}}{\pgfqpoint{-0.000000in}{0.000000in}}{%
\pgfpathmoveto{\pgfqpoint{-0.000000in}{0.000000in}}%
\pgfpathlineto{\pgfqpoint{-0.048611in}{0.000000in}}%
\pgfusepath{stroke,fill}%
}%
\begin{pgfscope}%
\pgfsys@transformshift{0.594525in}{2.026870in}%
\pgfsys@useobject{currentmarker}{}%
\end{pgfscope}%
\end{pgfscope}%
\begin{pgfscope}%
\definecolor{textcolor}{rgb}{0.000000,0.000000,0.000000}%
\pgfsetstrokecolor{textcolor}%
\pgfsetfillcolor{textcolor}%
\pgftext[x=0.321376in, y=1.987717in, left, base]{\color{textcolor}\rmfamily\fontsize{8.000000}{9.600000}\selectfont \(\displaystyle {10^{0}}\)}%
\end{pgfscope}%
\begin{pgfscope}%
\pgfpathrectangle{\pgfqpoint{0.594525in}{0.417642in}}{\pgfqpoint{3.423805in}{2.011535in}}%
\pgfusepath{clip}%
\pgfsetrectcap%
\pgfsetroundjoin%
\pgfsetlinewidth{0.803000pt}%
\definecolor{currentstroke}{rgb}{0.450000,0.450000,0.450000}%
\pgfsetstrokecolor{currentstroke}%
\pgfsetdash{}{0pt}%
\pgfpathmoveto{\pgfqpoint{0.594525in}{2.429177in}}%
\pgfpathlineto{\pgfqpoint{4.018330in}{2.429177in}}%
\pgfusepath{stroke}%
\end{pgfscope}%
\begin{pgfscope}%
\pgfsetbuttcap%
\pgfsetroundjoin%
\definecolor{currentfill}{rgb}{0.000000,0.000000,0.000000}%
\pgfsetfillcolor{currentfill}%
\pgfsetlinewidth{0.803000pt}%
\definecolor{currentstroke}{rgb}{0.000000,0.000000,0.000000}%
\pgfsetstrokecolor{currentstroke}%
\pgfsetdash{}{0pt}%
\pgfsys@defobject{currentmarker}{\pgfqpoint{-0.048611in}{0.000000in}}{\pgfqpoint{-0.000000in}{0.000000in}}{%
\pgfpathmoveto{\pgfqpoint{-0.000000in}{0.000000in}}%
\pgfpathlineto{\pgfqpoint{-0.048611in}{0.000000in}}%
\pgfusepath{stroke,fill}%
}%
\begin{pgfscope}%
\pgfsys@transformshift{0.594525in}{2.429177in}%
\pgfsys@useobject{currentmarker}{}%
\end{pgfscope}%
\end{pgfscope}%
\begin{pgfscope}%
\definecolor{textcolor}{rgb}{0.000000,0.000000,0.000000}%
\pgfsetstrokecolor{textcolor}%
\pgfsetfillcolor{textcolor}%
\pgftext[x=0.321376in, y=2.390024in, left, base]{\color{textcolor}\rmfamily\fontsize{8.000000}{9.600000}\selectfont \(\displaystyle {10^{1}}\)}%
\end{pgfscope}%
\begin{pgfscope}%
\pgfpathrectangle{\pgfqpoint{0.594525in}{0.417642in}}{\pgfqpoint{3.423805in}{2.011535in}}%
\pgfusepath{clip}%
\pgfsetrectcap%
\pgfsetroundjoin%
\pgfsetlinewidth{0.803000pt}%
\definecolor{currentstroke}{rgb}{0.850000,0.850000,0.850000}%
\pgfsetstrokecolor{currentstroke}%
\pgfsetdash{}{0pt}%
\pgfpathmoveto{\pgfqpoint{0.594525in}{0.538748in}}%
\pgfpathlineto{\pgfqpoint{4.018330in}{0.538748in}}%
\pgfusepath{stroke}%
\end{pgfscope}%
\begin{pgfscope}%
\pgfsetbuttcap%
\pgfsetroundjoin%
\definecolor{currentfill}{rgb}{0.000000,0.000000,0.000000}%
\pgfsetfillcolor{currentfill}%
\pgfsetlinewidth{0.602250pt}%
\definecolor{currentstroke}{rgb}{0.000000,0.000000,0.000000}%
\pgfsetstrokecolor{currentstroke}%
\pgfsetdash{}{0pt}%
\pgfsys@defobject{currentmarker}{\pgfqpoint{-0.027778in}{0.000000in}}{\pgfqpoint{-0.000000in}{0.000000in}}{%
\pgfpathmoveto{\pgfqpoint{-0.000000in}{0.000000in}}%
\pgfpathlineto{\pgfqpoint{-0.027778in}{0.000000in}}%
\pgfusepath{stroke,fill}%
}%
\begin{pgfscope}%
\pgfsys@transformshift{0.594525in}{0.538748in}%
\pgfsys@useobject{currentmarker}{}%
\end{pgfscope}%
\end{pgfscope}%
\begin{pgfscope}%
\pgfpathrectangle{\pgfqpoint{0.594525in}{0.417642in}}{\pgfqpoint{3.423805in}{2.011535in}}%
\pgfusepath{clip}%
\pgfsetrectcap%
\pgfsetroundjoin%
\pgfsetlinewidth{0.803000pt}%
\definecolor{currentstroke}{rgb}{0.850000,0.850000,0.850000}%
\pgfsetstrokecolor{currentstroke}%
\pgfsetdash{}{0pt}%
\pgfpathmoveto{\pgfqpoint{0.594525in}{0.609591in}}%
\pgfpathlineto{\pgfqpoint{4.018330in}{0.609591in}}%
\pgfusepath{stroke}%
\end{pgfscope}%
\begin{pgfscope}%
\pgfsetbuttcap%
\pgfsetroundjoin%
\definecolor{currentfill}{rgb}{0.000000,0.000000,0.000000}%
\pgfsetfillcolor{currentfill}%
\pgfsetlinewidth{0.602250pt}%
\definecolor{currentstroke}{rgb}{0.000000,0.000000,0.000000}%
\pgfsetstrokecolor{currentstroke}%
\pgfsetdash{}{0pt}%
\pgfsys@defobject{currentmarker}{\pgfqpoint{-0.027778in}{0.000000in}}{\pgfqpoint{-0.000000in}{0.000000in}}{%
\pgfpathmoveto{\pgfqpoint{-0.000000in}{0.000000in}}%
\pgfpathlineto{\pgfqpoint{-0.027778in}{0.000000in}}%
\pgfusepath{stroke,fill}%
}%
\begin{pgfscope}%
\pgfsys@transformshift{0.594525in}{0.609591in}%
\pgfsys@useobject{currentmarker}{}%
\end{pgfscope}%
\end{pgfscope}%
\begin{pgfscope}%
\pgfpathrectangle{\pgfqpoint{0.594525in}{0.417642in}}{\pgfqpoint{3.423805in}{2.011535in}}%
\pgfusepath{clip}%
\pgfsetrectcap%
\pgfsetroundjoin%
\pgfsetlinewidth{0.803000pt}%
\definecolor{currentstroke}{rgb}{0.850000,0.850000,0.850000}%
\pgfsetstrokecolor{currentstroke}%
\pgfsetdash{}{0pt}%
\pgfpathmoveto{\pgfqpoint{0.594525in}{0.659855in}}%
\pgfpathlineto{\pgfqpoint{4.018330in}{0.659855in}}%
\pgfusepath{stroke}%
\end{pgfscope}%
\begin{pgfscope}%
\pgfsetbuttcap%
\pgfsetroundjoin%
\definecolor{currentfill}{rgb}{0.000000,0.000000,0.000000}%
\pgfsetfillcolor{currentfill}%
\pgfsetlinewidth{0.602250pt}%
\definecolor{currentstroke}{rgb}{0.000000,0.000000,0.000000}%
\pgfsetstrokecolor{currentstroke}%
\pgfsetdash{}{0pt}%
\pgfsys@defobject{currentmarker}{\pgfqpoint{-0.027778in}{0.000000in}}{\pgfqpoint{-0.000000in}{0.000000in}}{%
\pgfpathmoveto{\pgfqpoint{-0.000000in}{0.000000in}}%
\pgfpathlineto{\pgfqpoint{-0.027778in}{0.000000in}}%
\pgfusepath{stroke,fill}%
}%
\begin{pgfscope}%
\pgfsys@transformshift{0.594525in}{0.659855in}%
\pgfsys@useobject{currentmarker}{}%
\end{pgfscope}%
\end{pgfscope}%
\begin{pgfscope}%
\pgfpathrectangle{\pgfqpoint{0.594525in}{0.417642in}}{\pgfqpoint{3.423805in}{2.011535in}}%
\pgfusepath{clip}%
\pgfsetrectcap%
\pgfsetroundjoin%
\pgfsetlinewidth{0.803000pt}%
\definecolor{currentstroke}{rgb}{0.850000,0.850000,0.850000}%
\pgfsetstrokecolor{currentstroke}%
\pgfsetdash{}{0pt}%
\pgfpathmoveto{\pgfqpoint{0.594525in}{0.698843in}}%
\pgfpathlineto{\pgfqpoint{4.018330in}{0.698843in}}%
\pgfusepath{stroke}%
\end{pgfscope}%
\begin{pgfscope}%
\pgfsetbuttcap%
\pgfsetroundjoin%
\definecolor{currentfill}{rgb}{0.000000,0.000000,0.000000}%
\pgfsetfillcolor{currentfill}%
\pgfsetlinewidth{0.602250pt}%
\definecolor{currentstroke}{rgb}{0.000000,0.000000,0.000000}%
\pgfsetstrokecolor{currentstroke}%
\pgfsetdash{}{0pt}%
\pgfsys@defobject{currentmarker}{\pgfqpoint{-0.027778in}{0.000000in}}{\pgfqpoint{-0.000000in}{0.000000in}}{%
\pgfpathmoveto{\pgfqpoint{-0.000000in}{0.000000in}}%
\pgfpathlineto{\pgfqpoint{-0.027778in}{0.000000in}}%
\pgfusepath{stroke,fill}%
}%
\begin{pgfscope}%
\pgfsys@transformshift{0.594525in}{0.698843in}%
\pgfsys@useobject{currentmarker}{}%
\end{pgfscope}%
\end{pgfscope}%
\begin{pgfscope}%
\pgfpathrectangle{\pgfqpoint{0.594525in}{0.417642in}}{\pgfqpoint{3.423805in}{2.011535in}}%
\pgfusepath{clip}%
\pgfsetrectcap%
\pgfsetroundjoin%
\pgfsetlinewidth{0.803000pt}%
\definecolor{currentstroke}{rgb}{0.850000,0.850000,0.850000}%
\pgfsetstrokecolor{currentstroke}%
\pgfsetdash{}{0pt}%
\pgfpathmoveto{\pgfqpoint{0.594525in}{0.730698in}}%
\pgfpathlineto{\pgfqpoint{4.018330in}{0.730698in}}%
\pgfusepath{stroke}%
\end{pgfscope}%
\begin{pgfscope}%
\pgfsetbuttcap%
\pgfsetroundjoin%
\definecolor{currentfill}{rgb}{0.000000,0.000000,0.000000}%
\pgfsetfillcolor{currentfill}%
\pgfsetlinewidth{0.602250pt}%
\definecolor{currentstroke}{rgb}{0.000000,0.000000,0.000000}%
\pgfsetstrokecolor{currentstroke}%
\pgfsetdash{}{0pt}%
\pgfsys@defobject{currentmarker}{\pgfqpoint{-0.027778in}{0.000000in}}{\pgfqpoint{-0.000000in}{0.000000in}}{%
\pgfpathmoveto{\pgfqpoint{-0.000000in}{0.000000in}}%
\pgfpathlineto{\pgfqpoint{-0.027778in}{0.000000in}}%
\pgfusepath{stroke,fill}%
}%
\begin{pgfscope}%
\pgfsys@transformshift{0.594525in}{0.730698in}%
\pgfsys@useobject{currentmarker}{}%
\end{pgfscope}%
\end{pgfscope}%
\begin{pgfscope}%
\pgfpathrectangle{\pgfqpoint{0.594525in}{0.417642in}}{\pgfqpoint{3.423805in}{2.011535in}}%
\pgfusepath{clip}%
\pgfsetrectcap%
\pgfsetroundjoin%
\pgfsetlinewidth{0.803000pt}%
\definecolor{currentstroke}{rgb}{0.850000,0.850000,0.850000}%
\pgfsetstrokecolor{currentstroke}%
\pgfsetdash{}{0pt}%
\pgfpathmoveto{\pgfqpoint{0.594525in}{0.757631in}}%
\pgfpathlineto{\pgfqpoint{4.018330in}{0.757631in}}%
\pgfusepath{stroke}%
\end{pgfscope}%
\begin{pgfscope}%
\pgfsetbuttcap%
\pgfsetroundjoin%
\definecolor{currentfill}{rgb}{0.000000,0.000000,0.000000}%
\pgfsetfillcolor{currentfill}%
\pgfsetlinewidth{0.602250pt}%
\definecolor{currentstroke}{rgb}{0.000000,0.000000,0.000000}%
\pgfsetstrokecolor{currentstroke}%
\pgfsetdash{}{0pt}%
\pgfsys@defobject{currentmarker}{\pgfqpoint{-0.027778in}{0.000000in}}{\pgfqpoint{-0.000000in}{0.000000in}}{%
\pgfpathmoveto{\pgfqpoint{-0.000000in}{0.000000in}}%
\pgfpathlineto{\pgfqpoint{-0.027778in}{0.000000in}}%
\pgfusepath{stroke,fill}%
}%
\begin{pgfscope}%
\pgfsys@transformshift{0.594525in}{0.757631in}%
\pgfsys@useobject{currentmarker}{}%
\end{pgfscope}%
\end{pgfscope}%
\begin{pgfscope}%
\pgfpathrectangle{\pgfqpoint{0.594525in}{0.417642in}}{\pgfqpoint{3.423805in}{2.011535in}}%
\pgfusepath{clip}%
\pgfsetrectcap%
\pgfsetroundjoin%
\pgfsetlinewidth{0.803000pt}%
\definecolor{currentstroke}{rgb}{0.850000,0.850000,0.850000}%
\pgfsetstrokecolor{currentstroke}%
\pgfsetdash{}{0pt}%
\pgfpathmoveto{\pgfqpoint{0.594525in}{0.780961in}}%
\pgfpathlineto{\pgfqpoint{4.018330in}{0.780961in}}%
\pgfusepath{stroke}%
\end{pgfscope}%
\begin{pgfscope}%
\pgfsetbuttcap%
\pgfsetroundjoin%
\definecolor{currentfill}{rgb}{0.000000,0.000000,0.000000}%
\pgfsetfillcolor{currentfill}%
\pgfsetlinewidth{0.602250pt}%
\definecolor{currentstroke}{rgb}{0.000000,0.000000,0.000000}%
\pgfsetstrokecolor{currentstroke}%
\pgfsetdash{}{0pt}%
\pgfsys@defobject{currentmarker}{\pgfqpoint{-0.027778in}{0.000000in}}{\pgfqpoint{-0.000000in}{0.000000in}}{%
\pgfpathmoveto{\pgfqpoint{-0.000000in}{0.000000in}}%
\pgfpathlineto{\pgfqpoint{-0.027778in}{0.000000in}}%
\pgfusepath{stroke,fill}%
}%
\begin{pgfscope}%
\pgfsys@transformshift{0.594525in}{0.780961in}%
\pgfsys@useobject{currentmarker}{}%
\end{pgfscope}%
\end{pgfscope}%
\begin{pgfscope}%
\pgfpathrectangle{\pgfqpoint{0.594525in}{0.417642in}}{\pgfqpoint{3.423805in}{2.011535in}}%
\pgfusepath{clip}%
\pgfsetrectcap%
\pgfsetroundjoin%
\pgfsetlinewidth{0.803000pt}%
\definecolor{currentstroke}{rgb}{0.850000,0.850000,0.850000}%
\pgfsetstrokecolor{currentstroke}%
\pgfsetdash{}{0pt}%
\pgfpathmoveto{\pgfqpoint{0.594525in}{0.801540in}}%
\pgfpathlineto{\pgfqpoint{4.018330in}{0.801540in}}%
\pgfusepath{stroke}%
\end{pgfscope}%
\begin{pgfscope}%
\pgfsetbuttcap%
\pgfsetroundjoin%
\definecolor{currentfill}{rgb}{0.000000,0.000000,0.000000}%
\pgfsetfillcolor{currentfill}%
\pgfsetlinewidth{0.602250pt}%
\definecolor{currentstroke}{rgb}{0.000000,0.000000,0.000000}%
\pgfsetstrokecolor{currentstroke}%
\pgfsetdash{}{0pt}%
\pgfsys@defobject{currentmarker}{\pgfqpoint{-0.027778in}{0.000000in}}{\pgfqpoint{-0.000000in}{0.000000in}}{%
\pgfpathmoveto{\pgfqpoint{-0.000000in}{0.000000in}}%
\pgfpathlineto{\pgfqpoint{-0.027778in}{0.000000in}}%
\pgfusepath{stroke,fill}%
}%
\begin{pgfscope}%
\pgfsys@transformshift{0.594525in}{0.801540in}%
\pgfsys@useobject{currentmarker}{}%
\end{pgfscope}%
\end{pgfscope}%
\begin{pgfscope}%
\pgfpathrectangle{\pgfqpoint{0.594525in}{0.417642in}}{\pgfqpoint{3.423805in}{2.011535in}}%
\pgfusepath{clip}%
\pgfsetrectcap%
\pgfsetroundjoin%
\pgfsetlinewidth{0.803000pt}%
\definecolor{currentstroke}{rgb}{0.850000,0.850000,0.850000}%
\pgfsetstrokecolor{currentstroke}%
\pgfsetdash{}{0pt}%
\pgfpathmoveto{\pgfqpoint{0.594525in}{0.941055in}}%
\pgfpathlineto{\pgfqpoint{4.018330in}{0.941055in}}%
\pgfusepath{stroke}%
\end{pgfscope}%
\begin{pgfscope}%
\pgfsetbuttcap%
\pgfsetroundjoin%
\definecolor{currentfill}{rgb}{0.000000,0.000000,0.000000}%
\pgfsetfillcolor{currentfill}%
\pgfsetlinewidth{0.602250pt}%
\definecolor{currentstroke}{rgb}{0.000000,0.000000,0.000000}%
\pgfsetstrokecolor{currentstroke}%
\pgfsetdash{}{0pt}%
\pgfsys@defobject{currentmarker}{\pgfqpoint{-0.027778in}{0.000000in}}{\pgfqpoint{-0.000000in}{0.000000in}}{%
\pgfpathmoveto{\pgfqpoint{-0.000000in}{0.000000in}}%
\pgfpathlineto{\pgfqpoint{-0.027778in}{0.000000in}}%
\pgfusepath{stroke,fill}%
}%
\begin{pgfscope}%
\pgfsys@transformshift{0.594525in}{0.941055in}%
\pgfsys@useobject{currentmarker}{}%
\end{pgfscope}%
\end{pgfscope}%
\begin{pgfscope}%
\pgfpathrectangle{\pgfqpoint{0.594525in}{0.417642in}}{\pgfqpoint{3.423805in}{2.011535in}}%
\pgfusepath{clip}%
\pgfsetrectcap%
\pgfsetroundjoin%
\pgfsetlinewidth{0.803000pt}%
\definecolor{currentstroke}{rgb}{0.850000,0.850000,0.850000}%
\pgfsetstrokecolor{currentstroke}%
\pgfsetdash{}{0pt}%
\pgfpathmoveto{\pgfqpoint{0.594525in}{1.011898in}}%
\pgfpathlineto{\pgfqpoint{4.018330in}{1.011898in}}%
\pgfusepath{stroke}%
\end{pgfscope}%
\begin{pgfscope}%
\pgfsetbuttcap%
\pgfsetroundjoin%
\definecolor{currentfill}{rgb}{0.000000,0.000000,0.000000}%
\pgfsetfillcolor{currentfill}%
\pgfsetlinewidth{0.602250pt}%
\definecolor{currentstroke}{rgb}{0.000000,0.000000,0.000000}%
\pgfsetstrokecolor{currentstroke}%
\pgfsetdash{}{0pt}%
\pgfsys@defobject{currentmarker}{\pgfqpoint{-0.027778in}{0.000000in}}{\pgfqpoint{-0.000000in}{0.000000in}}{%
\pgfpathmoveto{\pgfqpoint{-0.000000in}{0.000000in}}%
\pgfpathlineto{\pgfqpoint{-0.027778in}{0.000000in}}%
\pgfusepath{stroke,fill}%
}%
\begin{pgfscope}%
\pgfsys@transformshift{0.594525in}{1.011898in}%
\pgfsys@useobject{currentmarker}{}%
\end{pgfscope}%
\end{pgfscope}%
\begin{pgfscope}%
\pgfpathrectangle{\pgfqpoint{0.594525in}{0.417642in}}{\pgfqpoint{3.423805in}{2.011535in}}%
\pgfusepath{clip}%
\pgfsetrectcap%
\pgfsetroundjoin%
\pgfsetlinewidth{0.803000pt}%
\definecolor{currentstroke}{rgb}{0.850000,0.850000,0.850000}%
\pgfsetstrokecolor{currentstroke}%
\pgfsetdash{}{0pt}%
\pgfpathmoveto{\pgfqpoint{0.594525in}{1.062162in}}%
\pgfpathlineto{\pgfqpoint{4.018330in}{1.062162in}}%
\pgfusepath{stroke}%
\end{pgfscope}%
\begin{pgfscope}%
\pgfsetbuttcap%
\pgfsetroundjoin%
\definecolor{currentfill}{rgb}{0.000000,0.000000,0.000000}%
\pgfsetfillcolor{currentfill}%
\pgfsetlinewidth{0.602250pt}%
\definecolor{currentstroke}{rgb}{0.000000,0.000000,0.000000}%
\pgfsetstrokecolor{currentstroke}%
\pgfsetdash{}{0pt}%
\pgfsys@defobject{currentmarker}{\pgfqpoint{-0.027778in}{0.000000in}}{\pgfqpoint{-0.000000in}{0.000000in}}{%
\pgfpathmoveto{\pgfqpoint{-0.000000in}{0.000000in}}%
\pgfpathlineto{\pgfqpoint{-0.027778in}{0.000000in}}%
\pgfusepath{stroke,fill}%
}%
\begin{pgfscope}%
\pgfsys@transformshift{0.594525in}{1.062162in}%
\pgfsys@useobject{currentmarker}{}%
\end{pgfscope}%
\end{pgfscope}%
\begin{pgfscope}%
\pgfpathrectangle{\pgfqpoint{0.594525in}{0.417642in}}{\pgfqpoint{3.423805in}{2.011535in}}%
\pgfusepath{clip}%
\pgfsetrectcap%
\pgfsetroundjoin%
\pgfsetlinewidth{0.803000pt}%
\definecolor{currentstroke}{rgb}{0.850000,0.850000,0.850000}%
\pgfsetstrokecolor{currentstroke}%
\pgfsetdash{}{0pt}%
\pgfpathmoveto{\pgfqpoint{0.594525in}{1.101150in}}%
\pgfpathlineto{\pgfqpoint{4.018330in}{1.101150in}}%
\pgfusepath{stroke}%
\end{pgfscope}%
\begin{pgfscope}%
\pgfsetbuttcap%
\pgfsetroundjoin%
\definecolor{currentfill}{rgb}{0.000000,0.000000,0.000000}%
\pgfsetfillcolor{currentfill}%
\pgfsetlinewidth{0.602250pt}%
\definecolor{currentstroke}{rgb}{0.000000,0.000000,0.000000}%
\pgfsetstrokecolor{currentstroke}%
\pgfsetdash{}{0pt}%
\pgfsys@defobject{currentmarker}{\pgfqpoint{-0.027778in}{0.000000in}}{\pgfqpoint{-0.000000in}{0.000000in}}{%
\pgfpathmoveto{\pgfqpoint{-0.000000in}{0.000000in}}%
\pgfpathlineto{\pgfqpoint{-0.027778in}{0.000000in}}%
\pgfusepath{stroke,fill}%
}%
\begin{pgfscope}%
\pgfsys@transformshift{0.594525in}{1.101150in}%
\pgfsys@useobject{currentmarker}{}%
\end{pgfscope}%
\end{pgfscope}%
\begin{pgfscope}%
\pgfpathrectangle{\pgfqpoint{0.594525in}{0.417642in}}{\pgfqpoint{3.423805in}{2.011535in}}%
\pgfusepath{clip}%
\pgfsetrectcap%
\pgfsetroundjoin%
\pgfsetlinewidth{0.803000pt}%
\definecolor{currentstroke}{rgb}{0.850000,0.850000,0.850000}%
\pgfsetstrokecolor{currentstroke}%
\pgfsetdash{}{0pt}%
\pgfpathmoveto{\pgfqpoint{0.594525in}{1.133005in}}%
\pgfpathlineto{\pgfqpoint{4.018330in}{1.133005in}}%
\pgfusepath{stroke}%
\end{pgfscope}%
\begin{pgfscope}%
\pgfsetbuttcap%
\pgfsetroundjoin%
\definecolor{currentfill}{rgb}{0.000000,0.000000,0.000000}%
\pgfsetfillcolor{currentfill}%
\pgfsetlinewidth{0.602250pt}%
\definecolor{currentstroke}{rgb}{0.000000,0.000000,0.000000}%
\pgfsetstrokecolor{currentstroke}%
\pgfsetdash{}{0pt}%
\pgfsys@defobject{currentmarker}{\pgfqpoint{-0.027778in}{0.000000in}}{\pgfqpoint{-0.000000in}{0.000000in}}{%
\pgfpathmoveto{\pgfqpoint{-0.000000in}{0.000000in}}%
\pgfpathlineto{\pgfqpoint{-0.027778in}{0.000000in}}%
\pgfusepath{stroke,fill}%
}%
\begin{pgfscope}%
\pgfsys@transformshift{0.594525in}{1.133005in}%
\pgfsys@useobject{currentmarker}{}%
\end{pgfscope}%
\end{pgfscope}%
\begin{pgfscope}%
\pgfpathrectangle{\pgfqpoint{0.594525in}{0.417642in}}{\pgfqpoint{3.423805in}{2.011535in}}%
\pgfusepath{clip}%
\pgfsetrectcap%
\pgfsetroundjoin%
\pgfsetlinewidth{0.803000pt}%
\definecolor{currentstroke}{rgb}{0.850000,0.850000,0.850000}%
\pgfsetstrokecolor{currentstroke}%
\pgfsetdash{}{0pt}%
\pgfpathmoveto{\pgfqpoint{0.594525in}{1.159938in}}%
\pgfpathlineto{\pgfqpoint{4.018330in}{1.159938in}}%
\pgfusepath{stroke}%
\end{pgfscope}%
\begin{pgfscope}%
\pgfsetbuttcap%
\pgfsetroundjoin%
\definecolor{currentfill}{rgb}{0.000000,0.000000,0.000000}%
\pgfsetfillcolor{currentfill}%
\pgfsetlinewidth{0.602250pt}%
\definecolor{currentstroke}{rgb}{0.000000,0.000000,0.000000}%
\pgfsetstrokecolor{currentstroke}%
\pgfsetdash{}{0pt}%
\pgfsys@defobject{currentmarker}{\pgfqpoint{-0.027778in}{0.000000in}}{\pgfqpoint{-0.000000in}{0.000000in}}{%
\pgfpathmoveto{\pgfqpoint{-0.000000in}{0.000000in}}%
\pgfpathlineto{\pgfqpoint{-0.027778in}{0.000000in}}%
\pgfusepath{stroke,fill}%
}%
\begin{pgfscope}%
\pgfsys@transformshift{0.594525in}{1.159938in}%
\pgfsys@useobject{currentmarker}{}%
\end{pgfscope}%
\end{pgfscope}%
\begin{pgfscope}%
\pgfpathrectangle{\pgfqpoint{0.594525in}{0.417642in}}{\pgfqpoint{3.423805in}{2.011535in}}%
\pgfusepath{clip}%
\pgfsetrectcap%
\pgfsetroundjoin%
\pgfsetlinewidth{0.803000pt}%
\definecolor{currentstroke}{rgb}{0.850000,0.850000,0.850000}%
\pgfsetstrokecolor{currentstroke}%
\pgfsetdash{}{0pt}%
\pgfpathmoveto{\pgfqpoint{0.594525in}{1.183268in}}%
\pgfpathlineto{\pgfqpoint{4.018330in}{1.183268in}}%
\pgfusepath{stroke}%
\end{pgfscope}%
\begin{pgfscope}%
\pgfsetbuttcap%
\pgfsetroundjoin%
\definecolor{currentfill}{rgb}{0.000000,0.000000,0.000000}%
\pgfsetfillcolor{currentfill}%
\pgfsetlinewidth{0.602250pt}%
\definecolor{currentstroke}{rgb}{0.000000,0.000000,0.000000}%
\pgfsetstrokecolor{currentstroke}%
\pgfsetdash{}{0pt}%
\pgfsys@defobject{currentmarker}{\pgfqpoint{-0.027778in}{0.000000in}}{\pgfqpoint{-0.000000in}{0.000000in}}{%
\pgfpathmoveto{\pgfqpoint{-0.000000in}{0.000000in}}%
\pgfpathlineto{\pgfqpoint{-0.027778in}{0.000000in}}%
\pgfusepath{stroke,fill}%
}%
\begin{pgfscope}%
\pgfsys@transformshift{0.594525in}{1.183268in}%
\pgfsys@useobject{currentmarker}{}%
\end{pgfscope}%
\end{pgfscope}%
\begin{pgfscope}%
\pgfpathrectangle{\pgfqpoint{0.594525in}{0.417642in}}{\pgfqpoint{3.423805in}{2.011535in}}%
\pgfusepath{clip}%
\pgfsetrectcap%
\pgfsetroundjoin%
\pgfsetlinewidth{0.803000pt}%
\definecolor{currentstroke}{rgb}{0.850000,0.850000,0.850000}%
\pgfsetstrokecolor{currentstroke}%
\pgfsetdash{}{0pt}%
\pgfpathmoveto{\pgfqpoint{0.594525in}{1.203847in}}%
\pgfpathlineto{\pgfqpoint{4.018330in}{1.203847in}}%
\pgfusepath{stroke}%
\end{pgfscope}%
\begin{pgfscope}%
\pgfsetbuttcap%
\pgfsetroundjoin%
\definecolor{currentfill}{rgb}{0.000000,0.000000,0.000000}%
\pgfsetfillcolor{currentfill}%
\pgfsetlinewidth{0.602250pt}%
\definecolor{currentstroke}{rgb}{0.000000,0.000000,0.000000}%
\pgfsetstrokecolor{currentstroke}%
\pgfsetdash{}{0pt}%
\pgfsys@defobject{currentmarker}{\pgfqpoint{-0.027778in}{0.000000in}}{\pgfqpoint{-0.000000in}{0.000000in}}{%
\pgfpathmoveto{\pgfqpoint{-0.000000in}{0.000000in}}%
\pgfpathlineto{\pgfqpoint{-0.027778in}{0.000000in}}%
\pgfusepath{stroke,fill}%
}%
\begin{pgfscope}%
\pgfsys@transformshift{0.594525in}{1.203847in}%
\pgfsys@useobject{currentmarker}{}%
\end{pgfscope}%
\end{pgfscope}%
\begin{pgfscope}%
\pgfpathrectangle{\pgfqpoint{0.594525in}{0.417642in}}{\pgfqpoint{3.423805in}{2.011535in}}%
\pgfusepath{clip}%
\pgfsetrectcap%
\pgfsetroundjoin%
\pgfsetlinewidth{0.803000pt}%
\definecolor{currentstroke}{rgb}{0.850000,0.850000,0.850000}%
\pgfsetstrokecolor{currentstroke}%
\pgfsetdash{}{0pt}%
\pgfpathmoveto{\pgfqpoint{0.594525in}{1.343363in}}%
\pgfpathlineto{\pgfqpoint{4.018330in}{1.343363in}}%
\pgfusepath{stroke}%
\end{pgfscope}%
\begin{pgfscope}%
\pgfsetbuttcap%
\pgfsetroundjoin%
\definecolor{currentfill}{rgb}{0.000000,0.000000,0.000000}%
\pgfsetfillcolor{currentfill}%
\pgfsetlinewidth{0.602250pt}%
\definecolor{currentstroke}{rgb}{0.000000,0.000000,0.000000}%
\pgfsetstrokecolor{currentstroke}%
\pgfsetdash{}{0pt}%
\pgfsys@defobject{currentmarker}{\pgfqpoint{-0.027778in}{0.000000in}}{\pgfqpoint{-0.000000in}{0.000000in}}{%
\pgfpathmoveto{\pgfqpoint{-0.000000in}{0.000000in}}%
\pgfpathlineto{\pgfqpoint{-0.027778in}{0.000000in}}%
\pgfusepath{stroke,fill}%
}%
\begin{pgfscope}%
\pgfsys@transformshift{0.594525in}{1.343363in}%
\pgfsys@useobject{currentmarker}{}%
\end{pgfscope}%
\end{pgfscope}%
\begin{pgfscope}%
\pgfpathrectangle{\pgfqpoint{0.594525in}{0.417642in}}{\pgfqpoint{3.423805in}{2.011535in}}%
\pgfusepath{clip}%
\pgfsetrectcap%
\pgfsetroundjoin%
\pgfsetlinewidth{0.803000pt}%
\definecolor{currentstroke}{rgb}{0.850000,0.850000,0.850000}%
\pgfsetstrokecolor{currentstroke}%
\pgfsetdash{}{0pt}%
\pgfpathmoveto{\pgfqpoint{0.594525in}{1.414205in}}%
\pgfpathlineto{\pgfqpoint{4.018330in}{1.414205in}}%
\pgfusepath{stroke}%
\end{pgfscope}%
\begin{pgfscope}%
\pgfsetbuttcap%
\pgfsetroundjoin%
\definecolor{currentfill}{rgb}{0.000000,0.000000,0.000000}%
\pgfsetfillcolor{currentfill}%
\pgfsetlinewidth{0.602250pt}%
\definecolor{currentstroke}{rgb}{0.000000,0.000000,0.000000}%
\pgfsetstrokecolor{currentstroke}%
\pgfsetdash{}{0pt}%
\pgfsys@defobject{currentmarker}{\pgfqpoint{-0.027778in}{0.000000in}}{\pgfqpoint{-0.000000in}{0.000000in}}{%
\pgfpathmoveto{\pgfqpoint{-0.000000in}{0.000000in}}%
\pgfpathlineto{\pgfqpoint{-0.027778in}{0.000000in}}%
\pgfusepath{stroke,fill}%
}%
\begin{pgfscope}%
\pgfsys@transformshift{0.594525in}{1.414205in}%
\pgfsys@useobject{currentmarker}{}%
\end{pgfscope}%
\end{pgfscope}%
\begin{pgfscope}%
\pgfpathrectangle{\pgfqpoint{0.594525in}{0.417642in}}{\pgfqpoint{3.423805in}{2.011535in}}%
\pgfusepath{clip}%
\pgfsetrectcap%
\pgfsetroundjoin%
\pgfsetlinewidth{0.803000pt}%
\definecolor{currentstroke}{rgb}{0.850000,0.850000,0.850000}%
\pgfsetstrokecolor{currentstroke}%
\pgfsetdash{}{0pt}%
\pgfpathmoveto{\pgfqpoint{0.594525in}{1.464469in}}%
\pgfpathlineto{\pgfqpoint{4.018330in}{1.464469in}}%
\pgfusepath{stroke}%
\end{pgfscope}%
\begin{pgfscope}%
\pgfsetbuttcap%
\pgfsetroundjoin%
\definecolor{currentfill}{rgb}{0.000000,0.000000,0.000000}%
\pgfsetfillcolor{currentfill}%
\pgfsetlinewidth{0.602250pt}%
\definecolor{currentstroke}{rgb}{0.000000,0.000000,0.000000}%
\pgfsetstrokecolor{currentstroke}%
\pgfsetdash{}{0pt}%
\pgfsys@defobject{currentmarker}{\pgfqpoint{-0.027778in}{0.000000in}}{\pgfqpoint{-0.000000in}{0.000000in}}{%
\pgfpathmoveto{\pgfqpoint{-0.000000in}{0.000000in}}%
\pgfpathlineto{\pgfqpoint{-0.027778in}{0.000000in}}%
\pgfusepath{stroke,fill}%
}%
\begin{pgfscope}%
\pgfsys@transformshift{0.594525in}{1.464469in}%
\pgfsys@useobject{currentmarker}{}%
\end{pgfscope}%
\end{pgfscope}%
\begin{pgfscope}%
\pgfpathrectangle{\pgfqpoint{0.594525in}{0.417642in}}{\pgfqpoint{3.423805in}{2.011535in}}%
\pgfusepath{clip}%
\pgfsetrectcap%
\pgfsetroundjoin%
\pgfsetlinewidth{0.803000pt}%
\definecolor{currentstroke}{rgb}{0.850000,0.850000,0.850000}%
\pgfsetstrokecolor{currentstroke}%
\pgfsetdash{}{0pt}%
\pgfpathmoveto{\pgfqpoint{0.594525in}{1.503457in}}%
\pgfpathlineto{\pgfqpoint{4.018330in}{1.503457in}}%
\pgfusepath{stroke}%
\end{pgfscope}%
\begin{pgfscope}%
\pgfsetbuttcap%
\pgfsetroundjoin%
\definecolor{currentfill}{rgb}{0.000000,0.000000,0.000000}%
\pgfsetfillcolor{currentfill}%
\pgfsetlinewidth{0.602250pt}%
\definecolor{currentstroke}{rgb}{0.000000,0.000000,0.000000}%
\pgfsetstrokecolor{currentstroke}%
\pgfsetdash{}{0pt}%
\pgfsys@defobject{currentmarker}{\pgfqpoint{-0.027778in}{0.000000in}}{\pgfqpoint{-0.000000in}{0.000000in}}{%
\pgfpathmoveto{\pgfqpoint{-0.000000in}{0.000000in}}%
\pgfpathlineto{\pgfqpoint{-0.027778in}{0.000000in}}%
\pgfusepath{stroke,fill}%
}%
\begin{pgfscope}%
\pgfsys@transformshift{0.594525in}{1.503457in}%
\pgfsys@useobject{currentmarker}{}%
\end{pgfscope}%
\end{pgfscope}%
\begin{pgfscope}%
\pgfpathrectangle{\pgfqpoint{0.594525in}{0.417642in}}{\pgfqpoint{3.423805in}{2.011535in}}%
\pgfusepath{clip}%
\pgfsetrectcap%
\pgfsetroundjoin%
\pgfsetlinewidth{0.803000pt}%
\definecolor{currentstroke}{rgb}{0.850000,0.850000,0.850000}%
\pgfsetstrokecolor{currentstroke}%
\pgfsetdash{}{0pt}%
\pgfpathmoveto{\pgfqpoint{0.594525in}{1.535312in}}%
\pgfpathlineto{\pgfqpoint{4.018330in}{1.535312in}}%
\pgfusepath{stroke}%
\end{pgfscope}%
\begin{pgfscope}%
\pgfsetbuttcap%
\pgfsetroundjoin%
\definecolor{currentfill}{rgb}{0.000000,0.000000,0.000000}%
\pgfsetfillcolor{currentfill}%
\pgfsetlinewidth{0.602250pt}%
\definecolor{currentstroke}{rgb}{0.000000,0.000000,0.000000}%
\pgfsetstrokecolor{currentstroke}%
\pgfsetdash{}{0pt}%
\pgfsys@defobject{currentmarker}{\pgfqpoint{-0.027778in}{0.000000in}}{\pgfqpoint{-0.000000in}{0.000000in}}{%
\pgfpathmoveto{\pgfqpoint{-0.000000in}{0.000000in}}%
\pgfpathlineto{\pgfqpoint{-0.027778in}{0.000000in}}%
\pgfusepath{stroke,fill}%
}%
\begin{pgfscope}%
\pgfsys@transformshift{0.594525in}{1.535312in}%
\pgfsys@useobject{currentmarker}{}%
\end{pgfscope}%
\end{pgfscope}%
\begin{pgfscope}%
\pgfpathrectangle{\pgfqpoint{0.594525in}{0.417642in}}{\pgfqpoint{3.423805in}{2.011535in}}%
\pgfusepath{clip}%
\pgfsetrectcap%
\pgfsetroundjoin%
\pgfsetlinewidth{0.803000pt}%
\definecolor{currentstroke}{rgb}{0.850000,0.850000,0.850000}%
\pgfsetstrokecolor{currentstroke}%
\pgfsetdash{}{0pt}%
\pgfpathmoveto{\pgfqpoint{0.594525in}{1.562245in}}%
\pgfpathlineto{\pgfqpoint{4.018330in}{1.562245in}}%
\pgfusepath{stroke}%
\end{pgfscope}%
\begin{pgfscope}%
\pgfsetbuttcap%
\pgfsetroundjoin%
\definecolor{currentfill}{rgb}{0.000000,0.000000,0.000000}%
\pgfsetfillcolor{currentfill}%
\pgfsetlinewidth{0.602250pt}%
\definecolor{currentstroke}{rgb}{0.000000,0.000000,0.000000}%
\pgfsetstrokecolor{currentstroke}%
\pgfsetdash{}{0pt}%
\pgfsys@defobject{currentmarker}{\pgfqpoint{-0.027778in}{0.000000in}}{\pgfqpoint{-0.000000in}{0.000000in}}{%
\pgfpathmoveto{\pgfqpoint{-0.000000in}{0.000000in}}%
\pgfpathlineto{\pgfqpoint{-0.027778in}{0.000000in}}%
\pgfusepath{stroke,fill}%
}%
\begin{pgfscope}%
\pgfsys@transformshift{0.594525in}{1.562245in}%
\pgfsys@useobject{currentmarker}{}%
\end{pgfscope}%
\end{pgfscope}%
\begin{pgfscope}%
\pgfpathrectangle{\pgfqpoint{0.594525in}{0.417642in}}{\pgfqpoint{3.423805in}{2.011535in}}%
\pgfusepath{clip}%
\pgfsetrectcap%
\pgfsetroundjoin%
\pgfsetlinewidth{0.803000pt}%
\definecolor{currentstroke}{rgb}{0.850000,0.850000,0.850000}%
\pgfsetstrokecolor{currentstroke}%
\pgfsetdash{}{0pt}%
\pgfpathmoveto{\pgfqpoint{0.594525in}{1.585576in}}%
\pgfpathlineto{\pgfqpoint{4.018330in}{1.585576in}}%
\pgfusepath{stroke}%
\end{pgfscope}%
\begin{pgfscope}%
\pgfsetbuttcap%
\pgfsetroundjoin%
\definecolor{currentfill}{rgb}{0.000000,0.000000,0.000000}%
\pgfsetfillcolor{currentfill}%
\pgfsetlinewidth{0.602250pt}%
\definecolor{currentstroke}{rgb}{0.000000,0.000000,0.000000}%
\pgfsetstrokecolor{currentstroke}%
\pgfsetdash{}{0pt}%
\pgfsys@defobject{currentmarker}{\pgfqpoint{-0.027778in}{0.000000in}}{\pgfqpoint{-0.000000in}{0.000000in}}{%
\pgfpathmoveto{\pgfqpoint{-0.000000in}{0.000000in}}%
\pgfpathlineto{\pgfqpoint{-0.027778in}{0.000000in}}%
\pgfusepath{stroke,fill}%
}%
\begin{pgfscope}%
\pgfsys@transformshift{0.594525in}{1.585576in}%
\pgfsys@useobject{currentmarker}{}%
\end{pgfscope}%
\end{pgfscope}%
\begin{pgfscope}%
\pgfpathrectangle{\pgfqpoint{0.594525in}{0.417642in}}{\pgfqpoint{3.423805in}{2.011535in}}%
\pgfusepath{clip}%
\pgfsetrectcap%
\pgfsetroundjoin%
\pgfsetlinewidth{0.803000pt}%
\definecolor{currentstroke}{rgb}{0.850000,0.850000,0.850000}%
\pgfsetstrokecolor{currentstroke}%
\pgfsetdash{}{0pt}%
\pgfpathmoveto{\pgfqpoint{0.594525in}{1.606155in}}%
\pgfpathlineto{\pgfqpoint{4.018330in}{1.606155in}}%
\pgfusepath{stroke}%
\end{pgfscope}%
\begin{pgfscope}%
\pgfsetbuttcap%
\pgfsetroundjoin%
\definecolor{currentfill}{rgb}{0.000000,0.000000,0.000000}%
\pgfsetfillcolor{currentfill}%
\pgfsetlinewidth{0.602250pt}%
\definecolor{currentstroke}{rgb}{0.000000,0.000000,0.000000}%
\pgfsetstrokecolor{currentstroke}%
\pgfsetdash{}{0pt}%
\pgfsys@defobject{currentmarker}{\pgfqpoint{-0.027778in}{0.000000in}}{\pgfqpoint{-0.000000in}{0.000000in}}{%
\pgfpathmoveto{\pgfqpoint{-0.000000in}{0.000000in}}%
\pgfpathlineto{\pgfqpoint{-0.027778in}{0.000000in}}%
\pgfusepath{stroke,fill}%
}%
\begin{pgfscope}%
\pgfsys@transformshift{0.594525in}{1.606155in}%
\pgfsys@useobject{currentmarker}{}%
\end{pgfscope}%
\end{pgfscope}%
\begin{pgfscope}%
\pgfpathrectangle{\pgfqpoint{0.594525in}{0.417642in}}{\pgfqpoint{3.423805in}{2.011535in}}%
\pgfusepath{clip}%
\pgfsetrectcap%
\pgfsetroundjoin%
\pgfsetlinewidth{0.803000pt}%
\definecolor{currentstroke}{rgb}{0.850000,0.850000,0.850000}%
\pgfsetstrokecolor{currentstroke}%
\pgfsetdash{}{0pt}%
\pgfpathmoveto{\pgfqpoint{0.594525in}{1.745670in}}%
\pgfpathlineto{\pgfqpoint{4.018330in}{1.745670in}}%
\pgfusepath{stroke}%
\end{pgfscope}%
\begin{pgfscope}%
\pgfsetbuttcap%
\pgfsetroundjoin%
\definecolor{currentfill}{rgb}{0.000000,0.000000,0.000000}%
\pgfsetfillcolor{currentfill}%
\pgfsetlinewidth{0.602250pt}%
\definecolor{currentstroke}{rgb}{0.000000,0.000000,0.000000}%
\pgfsetstrokecolor{currentstroke}%
\pgfsetdash{}{0pt}%
\pgfsys@defobject{currentmarker}{\pgfqpoint{-0.027778in}{0.000000in}}{\pgfqpoint{-0.000000in}{0.000000in}}{%
\pgfpathmoveto{\pgfqpoint{-0.000000in}{0.000000in}}%
\pgfpathlineto{\pgfqpoint{-0.027778in}{0.000000in}}%
\pgfusepath{stroke,fill}%
}%
\begin{pgfscope}%
\pgfsys@transformshift{0.594525in}{1.745670in}%
\pgfsys@useobject{currentmarker}{}%
\end{pgfscope}%
\end{pgfscope}%
\begin{pgfscope}%
\pgfpathrectangle{\pgfqpoint{0.594525in}{0.417642in}}{\pgfqpoint{3.423805in}{2.011535in}}%
\pgfusepath{clip}%
\pgfsetrectcap%
\pgfsetroundjoin%
\pgfsetlinewidth{0.803000pt}%
\definecolor{currentstroke}{rgb}{0.850000,0.850000,0.850000}%
\pgfsetstrokecolor{currentstroke}%
\pgfsetdash{}{0pt}%
\pgfpathmoveto{\pgfqpoint{0.594525in}{1.816512in}}%
\pgfpathlineto{\pgfqpoint{4.018330in}{1.816512in}}%
\pgfusepath{stroke}%
\end{pgfscope}%
\begin{pgfscope}%
\pgfsetbuttcap%
\pgfsetroundjoin%
\definecolor{currentfill}{rgb}{0.000000,0.000000,0.000000}%
\pgfsetfillcolor{currentfill}%
\pgfsetlinewidth{0.602250pt}%
\definecolor{currentstroke}{rgb}{0.000000,0.000000,0.000000}%
\pgfsetstrokecolor{currentstroke}%
\pgfsetdash{}{0pt}%
\pgfsys@defobject{currentmarker}{\pgfqpoint{-0.027778in}{0.000000in}}{\pgfqpoint{-0.000000in}{0.000000in}}{%
\pgfpathmoveto{\pgfqpoint{-0.000000in}{0.000000in}}%
\pgfpathlineto{\pgfqpoint{-0.027778in}{0.000000in}}%
\pgfusepath{stroke,fill}%
}%
\begin{pgfscope}%
\pgfsys@transformshift{0.594525in}{1.816512in}%
\pgfsys@useobject{currentmarker}{}%
\end{pgfscope}%
\end{pgfscope}%
\begin{pgfscope}%
\pgfpathrectangle{\pgfqpoint{0.594525in}{0.417642in}}{\pgfqpoint{3.423805in}{2.011535in}}%
\pgfusepath{clip}%
\pgfsetrectcap%
\pgfsetroundjoin%
\pgfsetlinewidth{0.803000pt}%
\definecolor{currentstroke}{rgb}{0.850000,0.850000,0.850000}%
\pgfsetstrokecolor{currentstroke}%
\pgfsetdash{}{0pt}%
\pgfpathmoveto{\pgfqpoint{0.594525in}{1.866776in}}%
\pgfpathlineto{\pgfqpoint{4.018330in}{1.866776in}}%
\pgfusepath{stroke}%
\end{pgfscope}%
\begin{pgfscope}%
\pgfsetbuttcap%
\pgfsetroundjoin%
\definecolor{currentfill}{rgb}{0.000000,0.000000,0.000000}%
\pgfsetfillcolor{currentfill}%
\pgfsetlinewidth{0.602250pt}%
\definecolor{currentstroke}{rgb}{0.000000,0.000000,0.000000}%
\pgfsetstrokecolor{currentstroke}%
\pgfsetdash{}{0pt}%
\pgfsys@defobject{currentmarker}{\pgfqpoint{-0.027778in}{0.000000in}}{\pgfqpoint{-0.000000in}{0.000000in}}{%
\pgfpathmoveto{\pgfqpoint{-0.000000in}{0.000000in}}%
\pgfpathlineto{\pgfqpoint{-0.027778in}{0.000000in}}%
\pgfusepath{stroke,fill}%
}%
\begin{pgfscope}%
\pgfsys@transformshift{0.594525in}{1.866776in}%
\pgfsys@useobject{currentmarker}{}%
\end{pgfscope}%
\end{pgfscope}%
\begin{pgfscope}%
\pgfpathrectangle{\pgfqpoint{0.594525in}{0.417642in}}{\pgfqpoint{3.423805in}{2.011535in}}%
\pgfusepath{clip}%
\pgfsetrectcap%
\pgfsetroundjoin%
\pgfsetlinewidth{0.803000pt}%
\definecolor{currentstroke}{rgb}{0.850000,0.850000,0.850000}%
\pgfsetstrokecolor{currentstroke}%
\pgfsetdash{}{0pt}%
\pgfpathmoveto{\pgfqpoint{0.594525in}{1.905764in}}%
\pgfpathlineto{\pgfqpoint{4.018330in}{1.905764in}}%
\pgfusepath{stroke}%
\end{pgfscope}%
\begin{pgfscope}%
\pgfsetbuttcap%
\pgfsetroundjoin%
\definecolor{currentfill}{rgb}{0.000000,0.000000,0.000000}%
\pgfsetfillcolor{currentfill}%
\pgfsetlinewidth{0.602250pt}%
\definecolor{currentstroke}{rgb}{0.000000,0.000000,0.000000}%
\pgfsetstrokecolor{currentstroke}%
\pgfsetdash{}{0pt}%
\pgfsys@defobject{currentmarker}{\pgfqpoint{-0.027778in}{0.000000in}}{\pgfqpoint{-0.000000in}{0.000000in}}{%
\pgfpathmoveto{\pgfqpoint{-0.000000in}{0.000000in}}%
\pgfpathlineto{\pgfqpoint{-0.027778in}{0.000000in}}%
\pgfusepath{stroke,fill}%
}%
\begin{pgfscope}%
\pgfsys@transformshift{0.594525in}{1.905764in}%
\pgfsys@useobject{currentmarker}{}%
\end{pgfscope}%
\end{pgfscope}%
\begin{pgfscope}%
\pgfpathrectangle{\pgfqpoint{0.594525in}{0.417642in}}{\pgfqpoint{3.423805in}{2.011535in}}%
\pgfusepath{clip}%
\pgfsetrectcap%
\pgfsetroundjoin%
\pgfsetlinewidth{0.803000pt}%
\definecolor{currentstroke}{rgb}{0.850000,0.850000,0.850000}%
\pgfsetstrokecolor{currentstroke}%
\pgfsetdash{}{0pt}%
\pgfpathmoveto{\pgfqpoint{0.594525in}{1.937619in}}%
\pgfpathlineto{\pgfqpoint{4.018330in}{1.937619in}}%
\pgfusepath{stroke}%
\end{pgfscope}%
\begin{pgfscope}%
\pgfsetbuttcap%
\pgfsetroundjoin%
\definecolor{currentfill}{rgb}{0.000000,0.000000,0.000000}%
\pgfsetfillcolor{currentfill}%
\pgfsetlinewidth{0.602250pt}%
\definecolor{currentstroke}{rgb}{0.000000,0.000000,0.000000}%
\pgfsetstrokecolor{currentstroke}%
\pgfsetdash{}{0pt}%
\pgfsys@defobject{currentmarker}{\pgfqpoint{-0.027778in}{0.000000in}}{\pgfqpoint{-0.000000in}{0.000000in}}{%
\pgfpathmoveto{\pgfqpoint{-0.000000in}{0.000000in}}%
\pgfpathlineto{\pgfqpoint{-0.027778in}{0.000000in}}%
\pgfusepath{stroke,fill}%
}%
\begin{pgfscope}%
\pgfsys@transformshift{0.594525in}{1.937619in}%
\pgfsys@useobject{currentmarker}{}%
\end{pgfscope}%
\end{pgfscope}%
\begin{pgfscope}%
\pgfpathrectangle{\pgfqpoint{0.594525in}{0.417642in}}{\pgfqpoint{3.423805in}{2.011535in}}%
\pgfusepath{clip}%
\pgfsetrectcap%
\pgfsetroundjoin%
\pgfsetlinewidth{0.803000pt}%
\definecolor{currentstroke}{rgb}{0.850000,0.850000,0.850000}%
\pgfsetstrokecolor{currentstroke}%
\pgfsetdash{}{0pt}%
\pgfpathmoveto{\pgfqpoint{0.594525in}{1.964552in}}%
\pgfpathlineto{\pgfqpoint{4.018330in}{1.964552in}}%
\pgfusepath{stroke}%
\end{pgfscope}%
\begin{pgfscope}%
\pgfsetbuttcap%
\pgfsetroundjoin%
\definecolor{currentfill}{rgb}{0.000000,0.000000,0.000000}%
\pgfsetfillcolor{currentfill}%
\pgfsetlinewidth{0.602250pt}%
\definecolor{currentstroke}{rgb}{0.000000,0.000000,0.000000}%
\pgfsetstrokecolor{currentstroke}%
\pgfsetdash{}{0pt}%
\pgfsys@defobject{currentmarker}{\pgfqpoint{-0.027778in}{0.000000in}}{\pgfqpoint{-0.000000in}{0.000000in}}{%
\pgfpathmoveto{\pgfqpoint{-0.000000in}{0.000000in}}%
\pgfpathlineto{\pgfqpoint{-0.027778in}{0.000000in}}%
\pgfusepath{stroke,fill}%
}%
\begin{pgfscope}%
\pgfsys@transformshift{0.594525in}{1.964552in}%
\pgfsys@useobject{currentmarker}{}%
\end{pgfscope}%
\end{pgfscope}%
\begin{pgfscope}%
\pgfpathrectangle{\pgfqpoint{0.594525in}{0.417642in}}{\pgfqpoint{3.423805in}{2.011535in}}%
\pgfusepath{clip}%
\pgfsetrectcap%
\pgfsetroundjoin%
\pgfsetlinewidth{0.803000pt}%
\definecolor{currentstroke}{rgb}{0.850000,0.850000,0.850000}%
\pgfsetstrokecolor{currentstroke}%
\pgfsetdash{}{0pt}%
\pgfpathmoveto{\pgfqpoint{0.594525in}{1.987883in}}%
\pgfpathlineto{\pgfqpoint{4.018330in}{1.987883in}}%
\pgfusepath{stroke}%
\end{pgfscope}%
\begin{pgfscope}%
\pgfsetbuttcap%
\pgfsetroundjoin%
\definecolor{currentfill}{rgb}{0.000000,0.000000,0.000000}%
\pgfsetfillcolor{currentfill}%
\pgfsetlinewidth{0.602250pt}%
\definecolor{currentstroke}{rgb}{0.000000,0.000000,0.000000}%
\pgfsetstrokecolor{currentstroke}%
\pgfsetdash{}{0pt}%
\pgfsys@defobject{currentmarker}{\pgfqpoint{-0.027778in}{0.000000in}}{\pgfqpoint{-0.000000in}{0.000000in}}{%
\pgfpathmoveto{\pgfqpoint{-0.000000in}{0.000000in}}%
\pgfpathlineto{\pgfqpoint{-0.027778in}{0.000000in}}%
\pgfusepath{stroke,fill}%
}%
\begin{pgfscope}%
\pgfsys@transformshift{0.594525in}{1.987883in}%
\pgfsys@useobject{currentmarker}{}%
\end{pgfscope}%
\end{pgfscope}%
\begin{pgfscope}%
\pgfpathrectangle{\pgfqpoint{0.594525in}{0.417642in}}{\pgfqpoint{3.423805in}{2.011535in}}%
\pgfusepath{clip}%
\pgfsetrectcap%
\pgfsetroundjoin%
\pgfsetlinewidth{0.803000pt}%
\definecolor{currentstroke}{rgb}{0.850000,0.850000,0.850000}%
\pgfsetstrokecolor{currentstroke}%
\pgfsetdash{}{0pt}%
\pgfpathmoveto{\pgfqpoint{0.594525in}{2.008462in}}%
\pgfpathlineto{\pgfqpoint{4.018330in}{2.008462in}}%
\pgfusepath{stroke}%
\end{pgfscope}%
\begin{pgfscope}%
\pgfsetbuttcap%
\pgfsetroundjoin%
\definecolor{currentfill}{rgb}{0.000000,0.000000,0.000000}%
\pgfsetfillcolor{currentfill}%
\pgfsetlinewidth{0.602250pt}%
\definecolor{currentstroke}{rgb}{0.000000,0.000000,0.000000}%
\pgfsetstrokecolor{currentstroke}%
\pgfsetdash{}{0pt}%
\pgfsys@defobject{currentmarker}{\pgfqpoint{-0.027778in}{0.000000in}}{\pgfqpoint{-0.000000in}{0.000000in}}{%
\pgfpathmoveto{\pgfqpoint{-0.000000in}{0.000000in}}%
\pgfpathlineto{\pgfqpoint{-0.027778in}{0.000000in}}%
\pgfusepath{stroke,fill}%
}%
\begin{pgfscope}%
\pgfsys@transformshift{0.594525in}{2.008462in}%
\pgfsys@useobject{currentmarker}{}%
\end{pgfscope}%
\end{pgfscope}%
\begin{pgfscope}%
\pgfpathrectangle{\pgfqpoint{0.594525in}{0.417642in}}{\pgfqpoint{3.423805in}{2.011535in}}%
\pgfusepath{clip}%
\pgfsetrectcap%
\pgfsetroundjoin%
\pgfsetlinewidth{0.803000pt}%
\definecolor{currentstroke}{rgb}{0.850000,0.850000,0.850000}%
\pgfsetstrokecolor{currentstroke}%
\pgfsetdash{}{0pt}%
\pgfpathmoveto{\pgfqpoint{0.594525in}{2.147977in}}%
\pgfpathlineto{\pgfqpoint{4.018330in}{2.147977in}}%
\pgfusepath{stroke}%
\end{pgfscope}%
\begin{pgfscope}%
\pgfsetbuttcap%
\pgfsetroundjoin%
\definecolor{currentfill}{rgb}{0.000000,0.000000,0.000000}%
\pgfsetfillcolor{currentfill}%
\pgfsetlinewidth{0.602250pt}%
\definecolor{currentstroke}{rgb}{0.000000,0.000000,0.000000}%
\pgfsetstrokecolor{currentstroke}%
\pgfsetdash{}{0pt}%
\pgfsys@defobject{currentmarker}{\pgfqpoint{-0.027778in}{0.000000in}}{\pgfqpoint{-0.000000in}{0.000000in}}{%
\pgfpathmoveto{\pgfqpoint{-0.000000in}{0.000000in}}%
\pgfpathlineto{\pgfqpoint{-0.027778in}{0.000000in}}%
\pgfusepath{stroke,fill}%
}%
\begin{pgfscope}%
\pgfsys@transformshift{0.594525in}{2.147977in}%
\pgfsys@useobject{currentmarker}{}%
\end{pgfscope}%
\end{pgfscope}%
\begin{pgfscope}%
\pgfpathrectangle{\pgfqpoint{0.594525in}{0.417642in}}{\pgfqpoint{3.423805in}{2.011535in}}%
\pgfusepath{clip}%
\pgfsetrectcap%
\pgfsetroundjoin%
\pgfsetlinewidth{0.803000pt}%
\definecolor{currentstroke}{rgb}{0.850000,0.850000,0.850000}%
\pgfsetstrokecolor{currentstroke}%
\pgfsetdash{}{0pt}%
\pgfpathmoveto{\pgfqpoint{0.594525in}{2.218819in}}%
\pgfpathlineto{\pgfqpoint{4.018330in}{2.218819in}}%
\pgfusepath{stroke}%
\end{pgfscope}%
\begin{pgfscope}%
\pgfsetbuttcap%
\pgfsetroundjoin%
\definecolor{currentfill}{rgb}{0.000000,0.000000,0.000000}%
\pgfsetfillcolor{currentfill}%
\pgfsetlinewidth{0.602250pt}%
\definecolor{currentstroke}{rgb}{0.000000,0.000000,0.000000}%
\pgfsetstrokecolor{currentstroke}%
\pgfsetdash{}{0pt}%
\pgfsys@defobject{currentmarker}{\pgfqpoint{-0.027778in}{0.000000in}}{\pgfqpoint{-0.000000in}{0.000000in}}{%
\pgfpathmoveto{\pgfqpoint{-0.000000in}{0.000000in}}%
\pgfpathlineto{\pgfqpoint{-0.027778in}{0.000000in}}%
\pgfusepath{stroke,fill}%
}%
\begin{pgfscope}%
\pgfsys@transformshift{0.594525in}{2.218819in}%
\pgfsys@useobject{currentmarker}{}%
\end{pgfscope}%
\end{pgfscope}%
\begin{pgfscope}%
\pgfpathrectangle{\pgfqpoint{0.594525in}{0.417642in}}{\pgfqpoint{3.423805in}{2.011535in}}%
\pgfusepath{clip}%
\pgfsetrectcap%
\pgfsetroundjoin%
\pgfsetlinewidth{0.803000pt}%
\definecolor{currentstroke}{rgb}{0.850000,0.850000,0.850000}%
\pgfsetstrokecolor{currentstroke}%
\pgfsetdash{}{0pt}%
\pgfpathmoveto{\pgfqpoint{0.594525in}{2.269083in}}%
\pgfpathlineto{\pgfqpoint{4.018330in}{2.269083in}}%
\pgfusepath{stroke}%
\end{pgfscope}%
\begin{pgfscope}%
\pgfsetbuttcap%
\pgfsetroundjoin%
\definecolor{currentfill}{rgb}{0.000000,0.000000,0.000000}%
\pgfsetfillcolor{currentfill}%
\pgfsetlinewidth{0.602250pt}%
\definecolor{currentstroke}{rgb}{0.000000,0.000000,0.000000}%
\pgfsetstrokecolor{currentstroke}%
\pgfsetdash{}{0pt}%
\pgfsys@defobject{currentmarker}{\pgfqpoint{-0.027778in}{0.000000in}}{\pgfqpoint{-0.000000in}{0.000000in}}{%
\pgfpathmoveto{\pgfqpoint{-0.000000in}{0.000000in}}%
\pgfpathlineto{\pgfqpoint{-0.027778in}{0.000000in}}%
\pgfusepath{stroke,fill}%
}%
\begin{pgfscope}%
\pgfsys@transformshift{0.594525in}{2.269083in}%
\pgfsys@useobject{currentmarker}{}%
\end{pgfscope}%
\end{pgfscope}%
\begin{pgfscope}%
\pgfpathrectangle{\pgfqpoint{0.594525in}{0.417642in}}{\pgfqpoint{3.423805in}{2.011535in}}%
\pgfusepath{clip}%
\pgfsetrectcap%
\pgfsetroundjoin%
\pgfsetlinewidth{0.803000pt}%
\definecolor{currentstroke}{rgb}{0.850000,0.850000,0.850000}%
\pgfsetstrokecolor{currentstroke}%
\pgfsetdash{}{0pt}%
\pgfpathmoveto{\pgfqpoint{0.594525in}{2.308071in}}%
\pgfpathlineto{\pgfqpoint{4.018330in}{2.308071in}}%
\pgfusepath{stroke}%
\end{pgfscope}%
\begin{pgfscope}%
\pgfsetbuttcap%
\pgfsetroundjoin%
\definecolor{currentfill}{rgb}{0.000000,0.000000,0.000000}%
\pgfsetfillcolor{currentfill}%
\pgfsetlinewidth{0.602250pt}%
\definecolor{currentstroke}{rgb}{0.000000,0.000000,0.000000}%
\pgfsetstrokecolor{currentstroke}%
\pgfsetdash{}{0pt}%
\pgfsys@defobject{currentmarker}{\pgfqpoint{-0.027778in}{0.000000in}}{\pgfqpoint{-0.000000in}{0.000000in}}{%
\pgfpathmoveto{\pgfqpoint{-0.000000in}{0.000000in}}%
\pgfpathlineto{\pgfqpoint{-0.027778in}{0.000000in}}%
\pgfusepath{stroke,fill}%
}%
\begin{pgfscope}%
\pgfsys@transformshift{0.594525in}{2.308071in}%
\pgfsys@useobject{currentmarker}{}%
\end{pgfscope}%
\end{pgfscope}%
\begin{pgfscope}%
\pgfpathrectangle{\pgfqpoint{0.594525in}{0.417642in}}{\pgfqpoint{3.423805in}{2.011535in}}%
\pgfusepath{clip}%
\pgfsetrectcap%
\pgfsetroundjoin%
\pgfsetlinewidth{0.803000pt}%
\definecolor{currentstroke}{rgb}{0.850000,0.850000,0.850000}%
\pgfsetstrokecolor{currentstroke}%
\pgfsetdash{}{0pt}%
\pgfpathmoveto{\pgfqpoint{0.594525in}{2.339926in}}%
\pgfpathlineto{\pgfqpoint{4.018330in}{2.339926in}}%
\pgfusepath{stroke}%
\end{pgfscope}%
\begin{pgfscope}%
\pgfsetbuttcap%
\pgfsetroundjoin%
\definecolor{currentfill}{rgb}{0.000000,0.000000,0.000000}%
\pgfsetfillcolor{currentfill}%
\pgfsetlinewidth{0.602250pt}%
\definecolor{currentstroke}{rgb}{0.000000,0.000000,0.000000}%
\pgfsetstrokecolor{currentstroke}%
\pgfsetdash{}{0pt}%
\pgfsys@defobject{currentmarker}{\pgfqpoint{-0.027778in}{0.000000in}}{\pgfqpoint{-0.000000in}{0.000000in}}{%
\pgfpathmoveto{\pgfqpoint{-0.000000in}{0.000000in}}%
\pgfpathlineto{\pgfqpoint{-0.027778in}{0.000000in}}%
\pgfusepath{stroke,fill}%
}%
\begin{pgfscope}%
\pgfsys@transformshift{0.594525in}{2.339926in}%
\pgfsys@useobject{currentmarker}{}%
\end{pgfscope}%
\end{pgfscope}%
\begin{pgfscope}%
\pgfpathrectangle{\pgfqpoint{0.594525in}{0.417642in}}{\pgfqpoint{3.423805in}{2.011535in}}%
\pgfusepath{clip}%
\pgfsetrectcap%
\pgfsetroundjoin%
\pgfsetlinewidth{0.803000pt}%
\definecolor{currentstroke}{rgb}{0.850000,0.850000,0.850000}%
\pgfsetstrokecolor{currentstroke}%
\pgfsetdash{}{0pt}%
\pgfpathmoveto{\pgfqpoint{0.594525in}{2.366859in}}%
\pgfpathlineto{\pgfqpoint{4.018330in}{2.366859in}}%
\pgfusepath{stroke}%
\end{pgfscope}%
\begin{pgfscope}%
\pgfsetbuttcap%
\pgfsetroundjoin%
\definecolor{currentfill}{rgb}{0.000000,0.000000,0.000000}%
\pgfsetfillcolor{currentfill}%
\pgfsetlinewidth{0.602250pt}%
\definecolor{currentstroke}{rgb}{0.000000,0.000000,0.000000}%
\pgfsetstrokecolor{currentstroke}%
\pgfsetdash{}{0pt}%
\pgfsys@defobject{currentmarker}{\pgfqpoint{-0.027778in}{0.000000in}}{\pgfqpoint{-0.000000in}{0.000000in}}{%
\pgfpathmoveto{\pgfqpoint{-0.000000in}{0.000000in}}%
\pgfpathlineto{\pgfqpoint{-0.027778in}{0.000000in}}%
\pgfusepath{stroke,fill}%
}%
\begin{pgfscope}%
\pgfsys@transformshift{0.594525in}{2.366859in}%
\pgfsys@useobject{currentmarker}{}%
\end{pgfscope}%
\end{pgfscope}%
\begin{pgfscope}%
\pgfpathrectangle{\pgfqpoint{0.594525in}{0.417642in}}{\pgfqpoint{3.423805in}{2.011535in}}%
\pgfusepath{clip}%
\pgfsetrectcap%
\pgfsetroundjoin%
\pgfsetlinewidth{0.803000pt}%
\definecolor{currentstroke}{rgb}{0.850000,0.850000,0.850000}%
\pgfsetstrokecolor{currentstroke}%
\pgfsetdash{}{0pt}%
\pgfpathmoveto{\pgfqpoint{0.594525in}{2.390190in}}%
\pgfpathlineto{\pgfqpoint{4.018330in}{2.390190in}}%
\pgfusepath{stroke}%
\end{pgfscope}%
\begin{pgfscope}%
\pgfsetbuttcap%
\pgfsetroundjoin%
\definecolor{currentfill}{rgb}{0.000000,0.000000,0.000000}%
\pgfsetfillcolor{currentfill}%
\pgfsetlinewidth{0.602250pt}%
\definecolor{currentstroke}{rgb}{0.000000,0.000000,0.000000}%
\pgfsetstrokecolor{currentstroke}%
\pgfsetdash{}{0pt}%
\pgfsys@defobject{currentmarker}{\pgfqpoint{-0.027778in}{0.000000in}}{\pgfqpoint{-0.000000in}{0.000000in}}{%
\pgfpathmoveto{\pgfqpoint{-0.000000in}{0.000000in}}%
\pgfpathlineto{\pgfqpoint{-0.027778in}{0.000000in}}%
\pgfusepath{stroke,fill}%
}%
\begin{pgfscope}%
\pgfsys@transformshift{0.594525in}{2.390190in}%
\pgfsys@useobject{currentmarker}{}%
\end{pgfscope}%
\end{pgfscope}%
\begin{pgfscope}%
\pgfpathrectangle{\pgfqpoint{0.594525in}{0.417642in}}{\pgfqpoint{3.423805in}{2.011535in}}%
\pgfusepath{clip}%
\pgfsetrectcap%
\pgfsetroundjoin%
\pgfsetlinewidth{0.803000pt}%
\definecolor{currentstroke}{rgb}{0.850000,0.850000,0.850000}%
\pgfsetstrokecolor{currentstroke}%
\pgfsetdash{}{0pt}%
\pgfpathmoveto{\pgfqpoint{0.594525in}{2.410769in}}%
\pgfpathlineto{\pgfqpoint{4.018330in}{2.410769in}}%
\pgfusepath{stroke}%
\end{pgfscope}%
\begin{pgfscope}%
\pgfsetbuttcap%
\pgfsetroundjoin%
\definecolor{currentfill}{rgb}{0.000000,0.000000,0.000000}%
\pgfsetfillcolor{currentfill}%
\pgfsetlinewidth{0.602250pt}%
\definecolor{currentstroke}{rgb}{0.000000,0.000000,0.000000}%
\pgfsetstrokecolor{currentstroke}%
\pgfsetdash{}{0pt}%
\pgfsys@defobject{currentmarker}{\pgfqpoint{-0.027778in}{0.000000in}}{\pgfqpoint{-0.000000in}{0.000000in}}{%
\pgfpathmoveto{\pgfqpoint{-0.000000in}{0.000000in}}%
\pgfpathlineto{\pgfqpoint{-0.027778in}{0.000000in}}%
\pgfusepath{stroke,fill}%
}%
\begin{pgfscope}%
\pgfsys@transformshift{0.594525in}{2.410769in}%
\pgfsys@useobject{currentmarker}{}%
\end{pgfscope}%
\end{pgfscope}%
\begin{pgfscope}%
\definecolor{textcolor}{rgb}{0.000000,0.000000,0.000000}%
\pgfsetstrokecolor{textcolor}%
\pgfsetfillcolor{textcolor}%
\pgftext[x=0.185574in,y=1.423410in,,bottom,rotate=90.000000]{\color{textcolor}\rmfamily\fontsize{10.000000}{12.000000}\selectfont  \(\displaystyle S_y(f)\) in \(\displaystyle \unit{1 \per \Hz}\)}%
\end{pgfscope}%
\begin{pgfscope}%
\pgfpathrectangle{\pgfqpoint{0.594525in}{0.417642in}}{\pgfqpoint{3.423805in}{2.011535in}}%
\pgfusepath{clip}%
\pgfsetbuttcap%
\pgfsetroundjoin%
\pgfsetlinewidth{1.505625pt}%
\definecolor{currentstroke}{rgb}{0.000000,0.447059,0.698039}%
\pgfsetstrokecolor{currentstroke}%
\pgfsetdash{{5.550000pt}{2.400000pt}}{0.000000pt}%
\pgfpathmoveto{\pgfqpoint{0.750152in}{1.101150in}}%
\pgfpathlineto{\pgfqpoint{0.913971in}{1.101150in}}%
\pgfpathlineto{\pgfqpoint{1.077789in}{1.101149in}}%
\pgfpathlineto{\pgfqpoint{1.241608in}{1.101149in}}%
\pgfpathlineto{\pgfqpoint{1.405426in}{1.101149in}}%
\pgfpathlineto{\pgfqpoint{1.569244in}{1.101147in}}%
\pgfpathlineto{\pgfqpoint{1.733063in}{1.101144in}}%
\pgfpathlineto{\pgfqpoint{1.896881in}{1.101134in}}%
\pgfpathlineto{\pgfqpoint{2.060700in}{1.101109in}}%
\pgfpathlineto{\pgfqpoint{2.224518in}{1.101043in}}%
\pgfpathlineto{\pgfqpoint{2.388337in}{1.100870in}}%
\pgfpathlineto{\pgfqpoint{2.552155in}{1.100413in}}%
\pgfpathlineto{\pgfqpoint{2.715973in}{1.099214in}}%
\pgfpathlineto{\pgfqpoint{2.879792in}{1.096091in}}%
\pgfpathlineto{\pgfqpoint{3.043610in}{1.088116in}}%
\pgfpathlineto{\pgfqpoint{3.207429in}{1.068682in}}%
\pgfpathlineto{\pgfqpoint{3.371247in}{1.025885in}}%
\pgfpathlineto{\pgfqpoint{3.535066in}{0.946760in}}%
\pgfpathlineto{\pgfqpoint{3.698884in}{0.829160in}}%
\pgfpathlineto{\pgfqpoint{3.862702in}{0.684273in}}%
\pgfusepath{stroke}%
\end{pgfscope}%
\begin{pgfscope}%
\pgfpathrectangle{\pgfqpoint{0.594525in}{0.417642in}}{\pgfqpoint{3.423805in}{2.011535in}}%
\pgfusepath{clip}%
\pgfsetbuttcap%
\pgfsetroundjoin%
\pgfsetlinewidth{1.505625pt}%
\definecolor{currentstroke}{rgb}{0.000000,0.619608,0.450980}%
\pgfsetstrokecolor{currentstroke}%
\pgfsetdash{{5.550000pt}{2.400000pt}}{0.000000pt}%
\pgfpathmoveto{\pgfqpoint{0.750152in}{1.503455in}}%
\pgfpathlineto{\pgfqpoint{0.913971in}{1.503452in}}%
\pgfpathlineto{\pgfqpoint{1.077789in}{1.503445in}}%
\pgfpathlineto{\pgfqpoint{1.241608in}{1.503425in}}%
\pgfpathlineto{\pgfqpoint{1.405426in}{1.503373in}}%
\pgfpathlineto{\pgfqpoint{1.569244in}{1.503237in}}%
\pgfpathlineto{\pgfqpoint{1.733063in}{1.502878in}}%
\pgfpathlineto{\pgfqpoint{1.896881in}{1.501936in}}%
\pgfpathlineto{\pgfqpoint{2.060700in}{1.499475in}}%
\pgfpathlineto{\pgfqpoint{2.224518in}{1.493147in}}%
\pgfpathlineto{\pgfqpoint{2.388337in}{1.477486in}}%
\pgfpathlineto{\pgfqpoint{2.552155in}{1.441876in}}%
\pgfpathlineto{\pgfqpoint{2.715973in}{1.372655in}}%
\pgfpathlineto{\pgfqpoint{2.879792in}{1.263993in}}%
\pgfpathlineto{\pgfqpoint{3.043610in}{1.124575in}}%
\pgfpathlineto{\pgfqpoint{3.207429in}{0.968046in}}%
\pgfpathlineto{\pgfqpoint{3.371247in}{0.803791in}}%
\pgfpathlineto{\pgfqpoint{3.535066in}{0.636387in}}%
\pgfpathlineto{\pgfqpoint{3.698884in}{0.467755in}}%
\pgfpathlineto{\pgfqpoint{3.757118in}{0.407642in}}%
\pgfusepath{stroke}%
\end{pgfscope}%
\begin{pgfscope}%
\pgfpathrectangle{\pgfqpoint{0.594525in}{0.417642in}}{\pgfqpoint{3.423805in}{2.011535in}}%
\pgfusepath{clip}%
\pgfsetbuttcap%
\pgfsetroundjoin%
\pgfsetlinewidth{1.505625pt}%
\definecolor{currentstroke}{rgb}{0.835294,0.368627,0.000000}%
\pgfsetstrokecolor{currentstroke}%
\pgfsetdash{{5.550000pt}{2.400000pt}}{0.000000pt}%
\pgfpathmoveto{\pgfqpoint{0.750152in}{1.905591in}}%
\pgfpathlineto{\pgfqpoint{0.913971in}{1.905310in}}%
\pgfpathlineto{\pgfqpoint{1.077789in}{1.904569in}}%
\pgfpathlineto{\pgfqpoint{1.241608in}{1.902631in}}%
\pgfpathlineto{\pgfqpoint{1.405426in}{1.897622in}}%
\pgfpathlineto{\pgfqpoint{1.569244in}{1.885066in}}%
\pgfpathlineto{\pgfqpoint{1.733063in}{1.855727in}}%
\pgfpathlineto{\pgfqpoint{1.896881in}{1.795997in}}%
\pgfpathlineto{\pgfqpoint{2.060700in}{1.696883in}}%
\pgfpathlineto{\pgfqpoint{2.224518in}{1.563834in}}%
\pgfpathlineto{\pgfqpoint{2.388337in}{1.410478in}}%
\pgfpathlineto{\pgfqpoint{2.552155in}{1.247574in}}%
\pgfpathlineto{\pgfqpoint{2.715973in}{1.080708in}}%
\pgfpathlineto{\pgfqpoint{2.879792in}{0.912283in}}%
\pgfpathlineto{\pgfqpoint{3.043610in}{0.743258in}}%
\pgfpathlineto{\pgfqpoint{3.207429in}{0.574006in}}%
\pgfpathlineto{\pgfqpoint{3.368369in}{0.407642in}}%
\pgfusepath{stroke}%
\end{pgfscope}%
\begin{pgfscope}%
\pgfpathrectangle{\pgfqpoint{0.594525in}{0.417642in}}{\pgfqpoint{3.423805in}{2.011535in}}%
\pgfusepath{clip}%
\pgfsetbuttcap%
\pgfsetroundjoin%
\pgfsetlinewidth{1.505625pt}%
\definecolor{currentstroke}{rgb}{0.800000,0.474510,0.654902}%
\pgfsetstrokecolor{currentstroke}%
\pgfsetdash{{5.550000pt}{2.400000pt}}{0.000000pt}%
\pgfpathmoveto{\pgfqpoint{0.750152in}{2.291625in}}%
\pgfpathlineto{\pgfqpoint{0.913971in}{2.267659in}}%
\pgfpathlineto{\pgfqpoint{1.077789in}{2.216791in}}%
\pgfpathlineto{\pgfqpoint{1.241608in}{2.127610in}}%
\pgfpathlineto{\pgfqpoint{1.405426in}{2.001847in}}%
\pgfpathlineto{\pgfqpoint{1.569244in}{1.852341in}}%
\pgfpathlineto{\pgfqpoint{1.733063in}{1.691125in}}%
\pgfpathlineto{\pgfqpoint{1.896881in}{1.524937in}}%
\pgfpathlineto{\pgfqpoint{2.060700in}{1.356775in}}%
\pgfpathlineto{\pgfqpoint{2.224518in}{1.187852in}}%
\pgfpathlineto{\pgfqpoint{2.388337in}{1.018638in}}%
\pgfpathlineto{\pgfqpoint{2.552155in}{0.849313in}}%
\pgfpathlineto{\pgfqpoint{2.715973in}{0.679946in}}%
\pgfpathlineto{\pgfqpoint{2.879792in}{0.510563in}}%
\pgfpathlineto{\pgfqpoint{2.979329in}{0.407642in}}%
\pgfusepath{stroke}%
\end{pgfscope}%
\begin{pgfscope}%
\pgfpathrectangle{\pgfqpoint{0.594525in}{0.417642in}}{\pgfqpoint{3.423805in}{2.011535in}}%
\pgfusepath{clip}%
\pgfsetrectcap%
\pgfsetroundjoin%
\pgfsetlinewidth{1.505625pt}%
\definecolor{currentstroke}{rgb}{0.000000,0.000000,0.000000}%
\pgfsetstrokecolor{currentstroke}%
\pgfsetdash{}{0pt}%
\pgfpathmoveto{\pgfqpoint{0.750152in}{2.311714in}}%
\pgfpathlineto{\pgfqpoint{0.913971in}{2.290485in}}%
\pgfpathlineto{\pgfqpoint{1.077789in}{2.246597in}}%
\pgfpathlineto{\pgfqpoint{1.241608in}{2.174364in}}%
\pgfpathlineto{\pgfqpoint{1.405426in}{2.085506in}}%
\pgfpathlineto{\pgfqpoint{1.569244in}{2.002007in}}%
\pgfpathlineto{\pgfqpoint{1.733063in}{1.930696in}}%
\pgfpathlineto{\pgfqpoint{1.896881in}{1.856833in}}%
\pgfpathlineto{\pgfqpoint{2.060700in}{1.767596in}}%
\pgfpathlineto{\pgfqpoint{2.224518in}{1.671721in}}%
\pgfpathlineto{\pgfqpoint{2.388337in}{1.586817in}}%
\pgfpathlineto{\pgfqpoint{2.552155in}{1.513206in}}%
\pgfpathlineto{\pgfqpoint{2.715973in}{1.433444in}}%
\pgfpathlineto{\pgfqpoint{2.879792in}{1.338240in}}%
\pgfpathlineto{\pgfqpoint{3.043610in}{1.239973in}}%
\pgfpathlineto{\pgfqpoint{3.207429in}{1.153723in}}%
\pgfpathlineto{\pgfqpoint{3.371247in}{1.073322in}}%
\pgfpathlineto{\pgfqpoint{3.535066in}{0.976858in}}%
\pgfpathlineto{\pgfqpoint{3.698884in}{0.852102in}}%
\pgfpathlineto{\pgfqpoint{3.862702in}{0.704407in}}%
\pgfusepath{stroke}%
\end{pgfscope}%
\begin{pgfscope}%
\pgfsetrectcap%
\pgfsetmiterjoin%
\pgfsetlinewidth{0.803000pt}%
\definecolor{currentstroke}{rgb}{0.000000,0.000000,0.000000}%
\pgfsetstrokecolor{currentstroke}%
\pgfsetdash{}{0pt}%
\pgfpathmoveto{\pgfqpoint{0.594525in}{0.417642in}}%
\pgfpathlineto{\pgfqpoint{0.594525in}{2.429177in}}%
\pgfusepath{stroke}%
\end{pgfscope}%
\begin{pgfscope}%
\pgfsetrectcap%
\pgfsetmiterjoin%
\pgfsetlinewidth{0.803000pt}%
\definecolor{currentstroke}{rgb}{0.000000,0.000000,0.000000}%
\pgfsetstrokecolor{currentstroke}%
\pgfsetdash{}{0pt}%
\pgfpathmoveto{\pgfqpoint{4.018330in}{0.417642in}}%
\pgfpathlineto{\pgfqpoint{4.018330in}{2.429177in}}%
\pgfusepath{stroke}%
\end{pgfscope}%
\begin{pgfscope}%
\pgfsetrectcap%
\pgfsetmiterjoin%
\pgfsetlinewidth{0.803000pt}%
\definecolor{currentstroke}{rgb}{0.000000,0.000000,0.000000}%
\pgfsetstrokecolor{currentstroke}%
\pgfsetdash{}{0pt}%
\pgfpathmoveto{\pgfqpoint{0.594525in}{0.417642in}}%
\pgfpathlineto{\pgfqpoint{4.018330in}{0.417642in}}%
\pgfusepath{stroke}%
\end{pgfscope}%
\begin{pgfscope}%
\pgfsetrectcap%
\pgfsetmiterjoin%
\pgfsetlinewidth{0.803000pt}%
\definecolor{currentstroke}{rgb}{0.000000,0.000000,0.000000}%
\pgfsetstrokecolor{currentstroke}%
\pgfsetdash{}{0pt}%
\pgfpathmoveto{\pgfqpoint{0.594525in}{2.429177in}}%
\pgfpathlineto{\pgfqpoint{4.018330in}{2.429177in}}%
\pgfusepath{stroke}%
\end{pgfscope}%
\begin{pgfscope}%
\pgfsetbuttcap%
\pgfsetmiterjoin%
\definecolor{currentfill}{rgb}{1.000000,1.000000,1.000000}%
\pgfsetfillcolor{currentfill}%
\pgfsetfillopacity{0.800000}%
\pgfsetlinewidth{1.003750pt}%
\definecolor{currentstroke}{rgb}{0.800000,0.800000,0.800000}%
\pgfsetstrokecolor{currentstroke}%
\pgfsetstrokeopacity{0.800000}%
\pgfsetdash{}{0pt}%
\pgfpathmoveto{\pgfqpoint{0.672303in}{0.473197in}}%
\pgfpathlineto{\pgfqpoint{1.839313in}{0.473197in}}%
\pgfpathquadraticcurveto{\pgfqpoint{1.861536in}{0.473197in}}{\pgfqpoint{1.861536in}{0.495420in}}%
\pgfpathlineto{\pgfqpoint{1.861536in}{1.258752in}}%
\pgfpathquadraticcurveto{\pgfqpoint{1.861536in}{1.280975in}}{\pgfqpoint{1.839313in}{1.280975in}}%
\pgfpathlineto{\pgfqpoint{0.672303in}{1.280975in}}%
\pgfpathquadraticcurveto{\pgfqpoint{0.650080in}{1.280975in}}{\pgfqpoint{0.650080in}{1.258752in}}%
\pgfpathlineto{\pgfqpoint{0.650080in}{0.495420in}}%
\pgfpathquadraticcurveto{\pgfqpoint{0.650080in}{0.473197in}}{\pgfqpoint{0.672303in}{0.473197in}}%
\pgfpathlineto{\pgfqpoint{0.672303in}{0.473197in}}%
\pgfpathclose%
\pgfusepath{stroke,fill}%
\end{pgfscope}%
\begin{pgfscope}%
\pgfsetbuttcap%
\pgfsetroundjoin%
\pgfsetlinewidth{1.505625pt}%
\definecolor{currentstroke}{rgb}{0.000000,0.447059,0.698039}%
\pgfsetstrokecolor{currentstroke}%
\pgfsetdash{{5.550000pt}{2.400000pt}}{0.000000pt}%
\pgfpathmoveto{\pgfqpoint{0.694525in}{1.197641in}}%
\pgfpathlineto{\pgfqpoint{0.805636in}{1.197641in}}%
\pgfpathlineto{\pgfqpoint{0.916747in}{1.197641in}}%
\pgfusepath{stroke}%
\end{pgfscope}%
\begin{pgfscope}%
\definecolor{textcolor}{rgb}{0.000000,0.000000,0.000000}%
\pgfsetstrokecolor{textcolor}%
\pgfsetfillcolor{textcolor}%
\pgftext[x=1.005636in,y=1.158752in,left,base]{\color{textcolor}\rmfamily\fontsize{8.000000}{9.600000}\selectfont \(\displaystyle \bar\tau_1=\tau_0=\qty{0.01}{\s}\)}%
\end{pgfscope}%
\begin{pgfscope}%
\pgfsetbuttcap%
\pgfsetroundjoin%
\pgfsetlinewidth{1.505625pt}%
\definecolor{currentstroke}{rgb}{0.000000,0.619608,0.450980}%
\pgfsetstrokecolor{currentstroke}%
\pgfsetdash{{5.550000pt}{2.400000pt}}{0.000000pt}%
\pgfpathmoveto{\pgfqpoint{0.694525in}{1.042752in}}%
\pgfpathlineto{\pgfqpoint{0.805636in}{1.042752in}}%
\pgfpathlineto{\pgfqpoint{0.916747in}{1.042752in}}%
\pgfusepath{stroke}%
\end{pgfscope}%
\begin{pgfscope}%
\definecolor{textcolor}{rgb}{0.000000,0.000000,0.000000}%
\pgfsetstrokecolor{textcolor}%
\pgfsetfillcolor{textcolor}%
\pgftext[x=1.005636in,y=1.003864in,left,base]{\color{textcolor}\rmfamily\fontsize{8.000000}{9.600000}\selectfont \(\displaystyle \bar\tau_1=\tau_0=\qty{0.1}{\s}\)}%
\end{pgfscope}%
\begin{pgfscope}%
\pgfsetbuttcap%
\pgfsetroundjoin%
\pgfsetlinewidth{1.505625pt}%
\definecolor{currentstroke}{rgb}{0.835294,0.368627,0.000000}%
\pgfsetstrokecolor{currentstroke}%
\pgfsetdash{{5.550000pt}{2.400000pt}}{0.000000pt}%
\pgfpathmoveto{\pgfqpoint{0.694525in}{0.887864in}}%
\pgfpathlineto{\pgfqpoint{0.805636in}{0.887864in}}%
\pgfpathlineto{\pgfqpoint{0.916747in}{0.887864in}}%
\pgfusepath{stroke}%
\end{pgfscope}%
\begin{pgfscope}%
\definecolor{textcolor}{rgb}{0.000000,0.000000,0.000000}%
\pgfsetstrokecolor{textcolor}%
\pgfsetfillcolor{textcolor}%
\pgftext[x=1.005636in,y=0.848975in,left,base]{\color{textcolor}\rmfamily\fontsize{8.000000}{9.600000}\selectfont \(\displaystyle \bar\tau_1=\tau_0=\qty{1}{\s}\)}%
\end{pgfscope}%
\begin{pgfscope}%
\pgfsetbuttcap%
\pgfsetroundjoin%
\pgfsetlinewidth{1.505625pt}%
\definecolor{currentstroke}{rgb}{0.800000,0.474510,0.654902}%
\pgfsetstrokecolor{currentstroke}%
\pgfsetdash{{5.550000pt}{2.400000pt}}{0.000000pt}%
\pgfpathmoveto{\pgfqpoint{0.694525in}{0.732975in}}%
\pgfpathlineto{\pgfqpoint{0.805636in}{0.732975in}}%
\pgfpathlineto{\pgfqpoint{0.916747in}{0.732975in}}%
\pgfusepath{stroke}%
\end{pgfscope}%
\begin{pgfscope}%
\definecolor{textcolor}{rgb}{0.000000,0.000000,0.000000}%
\pgfsetstrokecolor{textcolor}%
\pgfsetfillcolor{textcolor}%
\pgftext[x=1.005636in,y=0.694086in,left,base]{\color{textcolor}\rmfamily\fontsize{8.000000}{9.600000}\selectfont \(\displaystyle \bar\tau_1=\tau_0=\qty{10}{\s}\)}%
\end{pgfscope}%
\begin{pgfscope}%
\pgfsetrectcap%
\pgfsetroundjoin%
\pgfsetlinewidth{1.505625pt}%
\definecolor{currentstroke}{rgb}{0.000000,0.000000,0.000000}%
\pgfsetstrokecolor{currentstroke}%
\pgfsetdash{}{0pt}%
\pgfpathmoveto{\pgfqpoint{0.694525in}{0.578086in}}%
\pgfpathlineto{\pgfqpoint{0.805636in}{0.578086in}}%
\pgfpathlineto{\pgfqpoint{0.916747in}{0.578086in}}%
\pgfusepath{stroke}%
\end{pgfscope}%
\begin{pgfscope}%
\definecolor{textcolor}{rgb}{0.000000,0.000000,0.000000}%
\pgfsetstrokecolor{textcolor}%
\pgfsetfillcolor{textcolor}%
\pgftext[x=1.005636in,y=0.539197in,left,base]{\color{textcolor}\rmfamily\fontsize{8.000000}{9.600000}\selectfont Envelope}%
\end{pgfscope}%
\end{pgfpicture}%
\makeatother%
\endgroup%
% data/simulations/sim_flicker_noise_envelope.py
    \caption{Multiple overlapping Lorentzian noise sources forming a $\frac 1 f$-like shape.}
    \label{fig:flicker_noise_evelope}
\end{figure}

Given that no trap site can store an electron indefinitely, the number of trap sites $N$ with a certain time constant $\frac 1 2 \bar \tau = \bar \tau_0 = \bar \tau_1$ must decline when going to longer time scales. Assuming $N$ is inversely proportional to the time constant $\bar \tau$
\begin{equation}
    N(\tau) \propto \frac{1}{\bar \tau}\,, \label{eqn:flicker_noise_weight_function}
\end{equation}
which can be motivated if the trapping process is thermally activated \cite{1_f_noise_motivation} and using equation \ref{eqn:burst_noise_lorentzian} from the previous section, multiplying the weight function \ref{eqn:flicker_noise_weight_function} and integrating over all possible storage times gives:
\begin{align}
    S(\omega) &= \lim_{t \to \infty} \int_0^t N(\bar \tau) \, 4 R_{xx}(0) \frac{\bar \tau}{1 + \omega^2 \bar \tau^2} \, d\bar\tau \nonumber\\
    \overset{\bar \tau_0 = \bar \tau_1}&{=} 4 R_{xx}(0)\, C_N \lim_{t \to \infty} \int_0^t \frac{1}{1 + \omega^2 \bar\tau^2} \, d\bar\tau \nonumber\\
    &= \frac{4 R_{xx}(0)\, C_N}{\omega} \lim_{t \to \infty}  \arctan{\bar\tau \omega} \Big|_{\bar\tau=0}^t \nonumber\\
    &= \frac{4 R_{xx}(0)\, C_N}{\omega} \cdot \frac{\pi}{2} \nonumber\\
    &= \frac{2 \pi R_{xx}(0)\, C_N}{\omega}\\
    S(f) &= h_{-1} f^{-1}
\end{align}

$C_N$ is the proportionality constant of \ref{eqn:flicker_noise_weight_function} and $h_{-1}$ is the power coefficient introduced in \ref{eqn:power_law}. This shows that for a large number of distributed trap sites, a noise spectrum of $f^{-1}$ is found.

Using equation \ref{eqn:psd_to_adev}, the Allan variance can be calculated from the power spectral density:
\begin{align}
    \sigma_A^2(\tau) &= 2 h_{-1} \int_0^\infty \frac{1}{f} \frac{\sin^4\left( \pi f \tau \right)}{(\pi f \tau)^2}\,df \nonumber\\
    &=2 \ln 2 \, h_{-1}
\end{align}

Again, using the \textit{AllanTools} library \cite{allantools}, flicker noise was simulated to give an impression of its properties.
\begin{figure}[ht]
    \centering
    \begin{subfigure}{0.32\linewidth}
        \centering
        \scalebox{0.75}{%
            %% Creator: Matplotlib, PGF backend
%%
%% To include the figure in your LaTeX document, write
%%   \input{<filename>.pgf}
%%
%% Make sure the required packages are loaded in your preamble
%%   \usepackage{pgf}
%%
%% Also ensure that all the required font packages are loaded; for instance,
%% the lmodern package is sometimes necessary when using math font.
%%   \usepackage{lmodern}
%%
%% Figures using additional raster images can only be included by \input if
%% they are in the same directory as the main LaTeX file. For loading figures
%% from other directories you can use the `import` package
%%   \usepackage{import}
%%
%% and then include the figures with
%%   \import{<path to file>}{<filename>.pgf}
%%
%% Matplotlib used the following preamble
%%   \usepackage{siunitx}
%%   \usepackage{fontspec}
%%   \makeatletter\@ifpackageloaded{underscore}{}{\usepackage[strings]{underscore}}\makeatother
%%
\begingroup%
\makeatletter%
\begin{pgfpicture}%
\pgfpathrectangle{\pgfpointorigin}{\pgfqpoint{2.440945in}{1.830709in}}%
\pgfusepath{use as bounding box, clip}%
\begin{pgfscope}%
\pgfsetbuttcap%
\pgfsetmiterjoin%
\definecolor{currentfill}{rgb}{1.000000,1.000000,1.000000}%
\pgfsetfillcolor{currentfill}%
\pgfsetlinewidth{0.000000pt}%
\definecolor{currentstroke}{rgb}{1.000000,1.000000,1.000000}%
\pgfsetstrokecolor{currentstroke}%
\pgfsetdash{}{0pt}%
\pgfpathmoveto{\pgfqpoint{0.000000in}{0.000000in}}%
\pgfpathlineto{\pgfqpoint{2.440945in}{0.000000in}}%
\pgfpathlineto{\pgfqpoint{2.440945in}{1.830709in}}%
\pgfpathlineto{\pgfqpoint{0.000000in}{1.830709in}}%
\pgfpathlineto{\pgfqpoint{0.000000in}{0.000000in}}%
\pgfpathclose%
\pgfusepath{fill}%
\end{pgfscope}%
\begin{pgfscope}%
\pgfsetbuttcap%
\pgfsetmiterjoin%
\definecolor{currentfill}{rgb}{1.000000,1.000000,1.000000}%
\pgfsetfillcolor{currentfill}%
\pgfsetlinewidth{0.000000pt}%
\definecolor{currentstroke}{rgb}{0.000000,0.000000,0.000000}%
\pgfsetstrokecolor{currentstroke}%
\pgfsetstrokeopacity{0.000000}%
\pgfsetdash{}{0pt}%
\pgfpathmoveto{\pgfqpoint{0.530716in}{0.416447in}}%
\pgfpathlineto{\pgfqpoint{2.399275in}{0.416447in}}%
\pgfpathlineto{\pgfqpoint{2.399275in}{1.750483in}}%
\pgfpathlineto{\pgfqpoint{0.530716in}{1.750483in}}%
\pgfpathlineto{\pgfqpoint{0.530716in}{0.416447in}}%
\pgfpathclose%
\pgfusepath{fill}%
\end{pgfscope}%
\begin{pgfscope}%
\pgfpathrectangle{\pgfqpoint{0.530716in}{0.416447in}}{\pgfqpoint{1.868559in}{1.334036in}}%
\pgfusepath{clip}%
\pgfsetrectcap%
\pgfsetroundjoin%
\pgfsetlinewidth{0.803000pt}%
\definecolor{currentstroke}{rgb}{0.450000,0.450000,0.450000}%
\pgfsetstrokecolor{currentstroke}%
\pgfsetdash{}{0pt}%
\pgfpathmoveto{\pgfqpoint{0.615651in}{0.416447in}}%
\pgfpathlineto{\pgfqpoint{0.615651in}{1.750483in}}%
\pgfusepath{stroke}%
\end{pgfscope}%
\begin{pgfscope}%
\pgfsetbuttcap%
\pgfsetroundjoin%
\definecolor{currentfill}{rgb}{0.000000,0.000000,0.000000}%
\pgfsetfillcolor{currentfill}%
\pgfsetlinewidth{0.803000pt}%
\definecolor{currentstroke}{rgb}{0.000000,0.000000,0.000000}%
\pgfsetstrokecolor{currentstroke}%
\pgfsetdash{}{0pt}%
\pgfsys@defobject{currentmarker}{\pgfqpoint{0.000000in}{-0.048611in}}{\pgfqpoint{0.000000in}{0.000000in}}{%
\pgfpathmoveto{\pgfqpoint{0.000000in}{0.000000in}}%
\pgfpathlineto{\pgfqpoint{0.000000in}{-0.048611in}}%
\pgfusepath{stroke,fill}%
}%
\begin{pgfscope}%
\pgfsys@transformshift{0.615651in}{0.416447in}%
\pgfsys@useobject{currentmarker}{}%
\end{pgfscope}%
\end{pgfscope}%
\begin{pgfscope}%
\definecolor{textcolor}{rgb}{0.000000,0.000000,0.000000}%
\pgfsetstrokecolor{textcolor}%
\pgfsetfillcolor{textcolor}%
\pgftext[x=0.615651in,y=0.319225in,,top]{\color{textcolor}\rmfamily\fontsize{8.000000}{9.600000}\selectfont \(\displaystyle {0}\)}%
\end{pgfscope}%
\begin{pgfscope}%
\pgfpathrectangle{\pgfqpoint{0.530716in}{0.416447in}}{\pgfqpoint{1.868559in}{1.334036in}}%
\pgfusepath{clip}%
\pgfsetrectcap%
\pgfsetroundjoin%
\pgfsetlinewidth{0.803000pt}%
\definecolor{currentstroke}{rgb}{0.450000,0.450000,0.450000}%
\pgfsetstrokecolor{currentstroke}%
\pgfsetdash{}{0pt}%
\pgfpathmoveto{\pgfqpoint{1.134113in}{0.416447in}}%
\pgfpathlineto{\pgfqpoint{1.134113in}{1.750483in}}%
\pgfusepath{stroke}%
\end{pgfscope}%
\begin{pgfscope}%
\pgfsetbuttcap%
\pgfsetroundjoin%
\definecolor{currentfill}{rgb}{0.000000,0.000000,0.000000}%
\pgfsetfillcolor{currentfill}%
\pgfsetlinewidth{0.803000pt}%
\definecolor{currentstroke}{rgb}{0.000000,0.000000,0.000000}%
\pgfsetstrokecolor{currentstroke}%
\pgfsetdash{}{0pt}%
\pgfsys@defobject{currentmarker}{\pgfqpoint{0.000000in}{-0.048611in}}{\pgfqpoint{0.000000in}{0.000000in}}{%
\pgfpathmoveto{\pgfqpoint{0.000000in}{0.000000in}}%
\pgfpathlineto{\pgfqpoint{0.000000in}{-0.048611in}}%
\pgfusepath{stroke,fill}%
}%
\begin{pgfscope}%
\pgfsys@transformshift{1.134113in}{0.416447in}%
\pgfsys@useobject{currentmarker}{}%
\end{pgfscope}%
\end{pgfscope}%
\begin{pgfscope}%
\definecolor{textcolor}{rgb}{0.000000,0.000000,0.000000}%
\pgfsetstrokecolor{textcolor}%
\pgfsetfillcolor{textcolor}%
\pgftext[x=1.134113in,y=0.319225in,,top]{\color{textcolor}\rmfamily\fontsize{8.000000}{9.600000}\selectfont \(\displaystyle {5000}\)}%
\end{pgfscope}%
\begin{pgfscope}%
\pgfpathrectangle{\pgfqpoint{0.530716in}{0.416447in}}{\pgfqpoint{1.868559in}{1.334036in}}%
\pgfusepath{clip}%
\pgfsetrectcap%
\pgfsetroundjoin%
\pgfsetlinewidth{0.803000pt}%
\definecolor{currentstroke}{rgb}{0.450000,0.450000,0.450000}%
\pgfsetstrokecolor{currentstroke}%
\pgfsetdash{}{0pt}%
\pgfpathmoveto{\pgfqpoint{1.652575in}{0.416447in}}%
\pgfpathlineto{\pgfqpoint{1.652575in}{1.750483in}}%
\pgfusepath{stroke}%
\end{pgfscope}%
\begin{pgfscope}%
\pgfsetbuttcap%
\pgfsetroundjoin%
\definecolor{currentfill}{rgb}{0.000000,0.000000,0.000000}%
\pgfsetfillcolor{currentfill}%
\pgfsetlinewidth{0.803000pt}%
\definecolor{currentstroke}{rgb}{0.000000,0.000000,0.000000}%
\pgfsetstrokecolor{currentstroke}%
\pgfsetdash{}{0pt}%
\pgfsys@defobject{currentmarker}{\pgfqpoint{0.000000in}{-0.048611in}}{\pgfqpoint{0.000000in}{0.000000in}}{%
\pgfpathmoveto{\pgfqpoint{0.000000in}{0.000000in}}%
\pgfpathlineto{\pgfqpoint{0.000000in}{-0.048611in}}%
\pgfusepath{stroke,fill}%
}%
\begin{pgfscope}%
\pgfsys@transformshift{1.652575in}{0.416447in}%
\pgfsys@useobject{currentmarker}{}%
\end{pgfscope}%
\end{pgfscope}%
\begin{pgfscope}%
\definecolor{textcolor}{rgb}{0.000000,0.000000,0.000000}%
\pgfsetstrokecolor{textcolor}%
\pgfsetfillcolor{textcolor}%
\pgftext[x=1.652575in,y=0.319225in,,top]{\color{textcolor}\rmfamily\fontsize{8.000000}{9.600000}\selectfont \(\displaystyle {10000}\)}%
\end{pgfscope}%
\begin{pgfscope}%
\pgfpathrectangle{\pgfqpoint{0.530716in}{0.416447in}}{\pgfqpoint{1.868559in}{1.334036in}}%
\pgfusepath{clip}%
\pgfsetrectcap%
\pgfsetroundjoin%
\pgfsetlinewidth{0.803000pt}%
\definecolor{currentstroke}{rgb}{0.450000,0.450000,0.450000}%
\pgfsetstrokecolor{currentstroke}%
\pgfsetdash{}{0pt}%
\pgfpathmoveto{\pgfqpoint{2.171037in}{0.416447in}}%
\pgfpathlineto{\pgfqpoint{2.171037in}{1.750483in}}%
\pgfusepath{stroke}%
\end{pgfscope}%
\begin{pgfscope}%
\pgfsetbuttcap%
\pgfsetroundjoin%
\definecolor{currentfill}{rgb}{0.000000,0.000000,0.000000}%
\pgfsetfillcolor{currentfill}%
\pgfsetlinewidth{0.803000pt}%
\definecolor{currentstroke}{rgb}{0.000000,0.000000,0.000000}%
\pgfsetstrokecolor{currentstroke}%
\pgfsetdash{}{0pt}%
\pgfsys@defobject{currentmarker}{\pgfqpoint{0.000000in}{-0.048611in}}{\pgfqpoint{0.000000in}{0.000000in}}{%
\pgfpathmoveto{\pgfqpoint{0.000000in}{0.000000in}}%
\pgfpathlineto{\pgfqpoint{0.000000in}{-0.048611in}}%
\pgfusepath{stroke,fill}%
}%
\begin{pgfscope}%
\pgfsys@transformshift{2.171037in}{0.416447in}%
\pgfsys@useobject{currentmarker}{}%
\end{pgfscope}%
\end{pgfscope}%
\begin{pgfscope}%
\definecolor{textcolor}{rgb}{0.000000,0.000000,0.000000}%
\pgfsetstrokecolor{textcolor}%
\pgfsetfillcolor{textcolor}%
\pgftext[x=2.171037in,y=0.319225in,,top]{\color{textcolor}\rmfamily\fontsize{8.000000}{9.600000}\selectfont \(\displaystyle {15000}\)}%
\end{pgfscope}%
\begin{pgfscope}%
\definecolor{textcolor}{rgb}{0.000000,0.000000,0.000000}%
\pgfsetstrokecolor{textcolor}%
\pgfsetfillcolor{textcolor}%
\pgftext[x=1.464996in,y=0.165003in,,top]{\color{textcolor}\rmfamily\fontsize{10.000000}{12.000000}\selectfont Time in \unit{\second}}%
\end{pgfscope}%
\begin{pgfscope}%
\pgfpathrectangle{\pgfqpoint{0.530716in}{0.416447in}}{\pgfqpoint{1.868559in}{1.334036in}}%
\pgfusepath{clip}%
\pgfsetrectcap%
\pgfsetroundjoin%
\pgfsetlinewidth{0.803000pt}%
\definecolor{currentstroke}{rgb}{0.450000,0.450000,0.450000}%
\pgfsetstrokecolor{currentstroke}%
\pgfsetdash{}{0pt}%
\pgfpathmoveto{\pgfqpoint{0.530716in}{0.416447in}}%
\pgfpathlineto{\pgfqpoint{2.399275in}{0.416447in}}%
\pgfusepath{stroke}%
\end{pgfscope}%
\begin{pgfscope}%
\pgfsetbuttcap%
\pgfsetroundjoin%
\definecolor{currentfill}{rgb}{0.000000,0.000000,0.000000}%
\pgfsetfillcolor{currentfill}%
\pgfsetlinewidth{0.803000pt}%
\definecolor{currentstroke}{rgb}{0.000000,0.000000,0.000000}%
\pgfsetstrokecolor{currentstroke}%
\pgfsetdash{}{0pt}%
\pgfsys@defobject{currentmarker}{\pgfqpoint{-0.048611in}{0.000000in}}{\pgfqpoint{-0.000000in}{0.000000in}}{%
\pgfpathmoveto{\pgfqpoint{-0.000000in}{0.000000in}}%
\pgfpathlineto{\pgfqpoint{-0.048611in}{0.000000in}}%
\pgfusepath{stroke,fill}%
}%
\begin{pgfscope}%
\pgfsys@transformshift{0.530716in}{0.416447in}%
\pgfsys@useobject{currentmarker}{}%
\end{pgfscope}%
\end{pgfscope}%
\begin{pgfscope}%
\definecolor{textcolor}{rgb}{0.000000,0.000000,0.000000}%
\pgfsetstrokecolor{textcolor}%
\pgfsetfillcolor{textcolor}%
\pgftext[x=0.223614in, y=0.377892in, left, base]{\color{textcolor}\rmfamily\fontsize{8.000000}{9.600000}\selectfont \(\displaystyle {\ensuremath{-}15}\)}%
\end{pgfscope}%
\begin{pgfscope}%
\pgfpathrectangle{\pgfqpoint{0.530716in}{0.416447in}}{\pgfqpoint{1.868559in}{1.334036in}}%
\pgfusepath{clip}%
\pgfsetrectcap%
\pgfsetroundjoin%
\pgfsetlinewidth{0.803000pt}%
\definecolor{currentstroke}{rgb}{0.450000,0.450000,0.450000}%
\pgfsetstrokecolor{currentstroke}%
\pgfsetdash{}{0pt}%
\pgfpathmoveto{\pgfqpoint{0.530716in}{0.638787in}}%
\pgfpathlineto{\pgfqpoint{2.399275in}{0.638787in}}%
\pgfusepath{stroke}%
\end{pgfscope}%
\begin{pgfscope}%
\pgfsetbuttcap%
\pgfsetroundjoin%
\definecolor{currentfill}{rgb}{0.000000,0.000000,0.000000}%
\pgfsetfillcolor{currentfill}%
\pgfsetlinewidth{0.803000pt}%
\definecolor{currentstroke}{rgb}{0.000000,0.000000,0.000000}%
\pgfsetstrokecolor{currentstroke}%
\pgfsetdash{}{0pt}%
\pgfsys@defobject{currentmarker}{\pgfqpoint{-0.048611in}{0.000000in}}{\pgfqpoint{-0.000000in}{0.000000in}}{%
\pgfpathmoveto{\pgfqpoint{-0.000000in}{0.000000in}}%
\pgfpathlineto{\pgfqpoint{-0.048611in}{0.000000in}}%
\pgfusepath{stroke,fill}%
}%
\begin{pgfscope}%
\pgfsys@transformshift{0.530716in}{0.638787in}%
\pgfsys@useobject{currentmarker}{}%
\end{pgfscope}%
\end{pgfscope}%
\begin{pgfscope}%
\definecolor{textcolor}{rgb}{0.000000,0.000000,0.000000}%
\pgfsetstrokecolor{textcolor}%
\pgfsetfillcolor{textcolor}%
\pgftext[x=0.223614in, y=0.600231in, left, base]{\color{textcolor}\rmfamily\fontsize{8.000000}{9.600000}\selectfont \(\displaystyle {\ensuremath{-}10}\)}%
\end{pgfscope}%
\begin{pgfscope}%
\pgfpathrectangle{\pgfqpoint{0.530716in}{0.416447in}}{\pgfqpoint{1.868559in}{1.334036in}}%
\pgfusepath{clip}%
\pgfsetrectcap%
\pgfsetroundjoin%
\pgfsetlinewidth{0.803000pt}%
\definecolor{currentstroke}{rgb}{0.450000,0.450000,0.450000}%
\pgfsetstrokecolor{currentstroke}%
\pgfsetdash{}{0pt}%
\pgfpathmoveto{\pgfqpoint{0.530716in}{0.861126in}}%
\pgfpathlineto{\pgfqpoint{2.399275in}{0.861126in}}%
\pgfusepath{stroke}%
\end{pgfscope}%
\begin{pgfscope}%
\pgfsetbuttcap%
\pgfsetroundjoin%
\definecolor{currentfill}{rgb}{0.000000,0.000000,0.000000}%
\pgfsetfillcolor{currentfill}%
\pgfsetlinewidth{0.803000pt}%
\definecolor{currentstroke}{rgb}{0.000000,0.000000,0.000000}%
\pgfsetstrokecolor{currentstroke}%
\pgfsetdash{}{0pt}%
\pgfsys@defobject{currentmarker}{\pgfqpoint{-0.048611in}{0.000000in}}{\pgfqpoint{-0.000000in}{0.000000in}}{%
\pgfpathmoveto{\pgfqpoint{-0.000000in}{0.000000in}}%
\pgfpathlineto{\pgfqpoint{-0.048611in}{0.000000in}}%
\pgfusepath{stroke,fill}%
}%
\begin{pgfscope}%
\pgfsys@transformshift{0.530716in}{0.861126in}%
\pgfsys@useobject{currentmarker}{}%
\end{pgfscope}%
\end{pgfscope}%
\begin{pgfscope}%
\definecolor{textcolor}{rgb}{0.000000,0.000000,0.000000}%
\pgfsetstrokecolor{textcolor}%
\pgfsetfillcolor{textcolor}%
\pgftext[x=0.282643in, y=0.822570in, left, base]{\color{textcolor}\rmfamily\fontsize{8.000000}{9.600000}\selectfont \(\displaystyle {\ensuremath{-}5}\)}%
\end{pgfscope}%
\begin{pgfscope}%
\pgfpathrectangle{\pgfqpoint{0.530716in}{0.416447in}}{\pgfqpoint{1.868559in}{1.334036in}}%
\pgfusepath{clip}%
\pgfsetrectcap%
\pgfsetroundjoin%
\pgfsetlinewidth{0.803000pt}%
\definecolor{currentstroke}{rgb}{0.450000,0.450000,0.450000}%
\pgfsetstrokecolor{currentstroke}%
\pgfsetdash{}{0pt}%
\pgfpathmoveto{\pgfqpoint{0.530716in}{1.083465in}}%
\pgfpathlineto{\pgfqpoint{2.399275in}{1.083465in}}%
\pgfusepath{stroke}%
\end{pgfscope}%
\begin{pgfscope}%
\pgfsetbuttcap%
\pgfsetroundjoin%
\definecolor{currentfill}{rgb}{0.000000,0.000000,0.000000}%
\pgfsetfillcolor{currentfill}%
\pgfsetlinewidth{0.803000pt}%
\definecolor{currentstroke}{rgb}{0.000000,0.000000,0.000000}%
\pgfsetstrokecolor{currentstroke}%
\pgfsetdash{}{0pt}%
\pgfsys@defobject{currentmarker}{\pgfqpoint{-0.048611in}{0.000000in}}{\pgfqpoint{-0.000000in}{0.000000in}}{%
\pgfpathmoveto{\pgfqpoint{-0.000000in}{0.000000in}}%
\pgfpathlineto{\pgfqpoint{-0.048611in}{0.000000in}}%
\pgfusepath{stroke,fill}%
}%
\begin{pgfscope}%
\pgfsys@transformshift{0.530716in}{1.083465in}%
\pgfsys@useobject{currentmarker}{}%
\end{pgfscope}%
\end{pgfscope}%
\begin{pgfscope}%
\definecolor{textcolor}{rgb}{0.000000,0.000000,0.000000}%
\pgfsetstrokecolor{textcolor}%
\pgfsetfillcolor{textcolor}%
\pgftext[x=0.374465in, y=1.044910in, left, base]{\color{textcolor}\rmfamily\fontsize{8.000000}{9.600000}\selectfont \(\displaystyle {0}\)}%
\end{pgfscope}%
\begin{pgfscope}%
\pgfpathrectangle{\pgfqpoint{0.530716in}{0.416447in}}{\pgfqpoint{1.868559in}{1.334036in}}%
\pgfusepath{clip}%
\pgfsetrectcap%
\pgfsetroundjoin%
\pgfsetlinewidth{0.803000pt}%
\definecolor{currentstroke}{rgb}{0.450000,0.450000,0.450000}%
\pgfsetstrokecolor{currentstroke}%
\pgfsetdash{}{0pt}%
\pgfpathmoveto{\pgfqpoint{0.530716in}{1.305805in}}%
\pgfpathlineto{\pgfqpoint{2.399275in}{1.305805in}}%
\pgfusepath{stroke}%
\end{pgfscope}%
\begin{pgfscope}%
\pgfsetbuttcap%
\pgfsetroundjoin%
\definecolor{currentfill}{rgb}{0.000000,0.000000,0.000000}%
\pgfsetfillcolor{currentfill}%
\pgfsetlinewidth{0.803000pt}%
\definecolor{currentstroke}{rgb}{0.000000,0.000000,0.000000}%
\pgfsetstrokecolor{currentstroke}%
\pgfsetdash{}{0pt}%
\pgfsys@defobject{currentmarker}{\pgfqpoint{-0.048611in}{0.000000in}}{\pgfqpoint{-0.000000in}{0.000000in}}{%
\pgfpathmoveto{\pgfqpoint{-0.000000in}{0.000000in}}%
\pgfpathlineto{\pgfqpoint{-0.048611in}{0.000000in}}%
\pgfusepath{stroke,fill}%
}%
\begin{pgfscope}%
\pgfsys@transformshift{0.530716in}{1.305805in}%
\pgfsys@useobject{currentmarker}{}%
\end{pgfscope}%
\end{pgfscope}%
\begin{pgfscope}%
\definecolor{textcolor}{rgb}{0.000000,0.000000,0.000000}%
\pgfsetstrokecolor{textcolor}%
\pgfsetfillcolor{textcolor}%
\pgftext[x=0.374465in, y=1.267249in, left, base]{\color{textcolor}\rmfamily\fontsize{8.000000}{9.600000}\selectfont \(\displaystyle {5}\)}%
\end{pgfscope}%
\begin{pgfscope}%
\pgfpathrectangle{\pgfqpoint{0.530716in}{0.416447in}}{\pgfqpoint{1.868559in}{1.334036in}}%
\pgfusepath{clip}%
\pgfsetrectcap%
\pgfsetroundjoin%
\pgfsetlinewidth{0.803000pt}%
\definecolor{currentstroke}{rgb}{0.450000,0.450000,0.450000}%
\pgfsetstrokecolor{currentstroke}%
\pgfsetdash{}{0pt}%
\pgfpathmoveto{\pgfqpoint{0.530716in}{1.528144in}}%
\pgfpathlineto{\pgfqpoint{2.399275in}{1.528144in}}%
\pgfusepath{stroke}%
\end{pgfscope}%
\begin{pgfscope}%
\pgfsetbuttcap%
\pgfsetroundjoin%
\definecolor{currentfill}{rgb}{0.000000,0.000000,0.000000}%
\pgfsetfillcolor{currentfill}%
\pgfsetlinewidth{0.803000pt}%
\definecolor{currentstroke}{rgb}{0.000000,0.000000,0.000000}%
\pgfsetstrokecolor{currentstroke}%
\pgfsetdash{}{0pt}%
\pgfsys@defobject{currentmarker}{\pgfqpoint{-0.048611in}{0.000000in}}{\pgfqpoint{-0.000000in}{0.000000in}}{%
\pgfpathmoveto{\pgfqpoint{-0.000000in}{0.000000in}}%
\pgfpathlineto{\pgfqpoint{-0.048611in}{0.000000in}}%
\pgfusepath{stroke,fill}%
}%
\begin{pgfscope}%
\pgfsys@transformshift{0.530716in}{1.528144in}%
\pgfsys@useobject{currentmarker}{}%
\end{pgfscope}%
\end{pgfscope}%
\begin{pgfscope}%
\definecolor{textcolor}{rgb}{0.000000,0.000000,0.000000}%
\pgfsetstrokecolor{textcolor}%
\pgfsetfillcolor{textcolor}%
\pgftext[x=0.315437in, y=1.489588in, left, base]{\color{textcolor}\rmfamily\fontsize{8.000000}{9.600000}\selectfont \(\displaystyle {10}\)}%
\end{pgfscope}%
\begin{pgfscope}%
\pgfpathrectangle{\pgfqpoint{0.530716in}{0.416447in}}{\pgfqpoint{1.868559in}{1.334036in}}%
\pgfusepath{clip}%
\pgfsetrectcap%
\pgfsetroundjoin%
\pgfsetlinewidth{0.803000pt}%
\definecolor{currentstroke}{rgb}{0.450000,0.450000,0.450000}%
\pgfsetstrokecolor{currentstroke}%
\pgfsetdash{}{0pt}%
\pgfpathmoveto{\pgfqpoint{0.530716in}{1.750483in}}%
\pgfpathlineto{\pgfqpoint{2.399275in}{1.750483in}}%
\pgfusepath{stroke}%
\end{pgfscope}%
\begin{pgfscope}%
\pgfsetbuttcap%
\pgfsetroundjoin%
\definecolor{currentfill}{rgb}{0.000000,0.000000,0.000000}%
\pgfsetfillcolor{currentfill}%
\pgfsetlinewidth{0.803000pt}%
\definecolor{currentstroke}{rgb}{0.000000,0.000000,0.000000}%
\pgfsetstrokecolor{currentstroke}%
\pgfsetdash{}{0pt}%
\pgfsys@defobject{currentmarker}{\pgfqpoint{-0.048611in}{0.000000in}}{\pgfqpoint{-0.000000in}{0.000000in}}{%
\pgfpathmoveto{\pgfqpoint{-0.000000in}{0.000000in}}%
\pgfpathlineto{\pgfqpoint{-0.048611in}{0.000000in}}%
\pgfusepath{stroke,fill}%
}%
\begin{pgfscope}%
\pgfsys@transformshift{0.530716in}{1.750483in}%
\pgfsys@useobject{currentmarker}{}%
\end{pgfscope}%
\end{pgfscope}%
\begin{pgfscope}%
\definecolor{textcolor}{rgb}{0.000000,0.000000,0.000000}%
\pgfsetstrokecolor{textcolor}%
\pgfsetfillcolor{textcolor}%
\pgftext[x=0.315437in, y=1.711928in, left, base]{\color{textcolor}\rmfamily\fontsize{8.000000}{9.600000}\selectfont \(\displaystyle {15}\)}%
\end{pgfscope}%
\begin{pgfscope}%
\definecolor{textcolor}{rgb}{0.000000,0.000000,0.000000}%
\pgfsetstrokecolor{textcolor}%
\pgfsetfillcolor{textcolor}%
\pgftext[x=0.168059in,y=1.083465in,,bottom,rotate=90.000000]{\color{textcolor}\rmfamily\fontsize{10.000000}{12.000000}\selectfont Ampl. in arb. unit}%
\end{pgfscope}%
\begin{pgfscope}%
\pgfpathrectangle{\pgfqpoint{0.530716in}{0.416447in}}{\pgfqpoint{1.868559in}{1.334036in}}%
\pgfusepath{clip}%
\pgfsetrectcap%
\pgfsetroundjoin%
\pgfsetlinewidth{1.505625pt}%
\definecolor{currentstroke}{rgb}{0.007843,0.619608,0.450980}%
\pgfsetstrokecolor{currentstroke}%
\pgfsetdash{}{0pt}%
\pgfpathmoveto{\pgfqpoint{0.615651in}{1.090835in}}%
\pgfpathlineto{\pgfqpoint{0.616169in}{1.224404in}}%
\pgfpathlineto{\pgfqpoint{0.617517in}{0.913531in}}%
\pgfpathlineto{\pgfqpoint{0.620628in}{0.883035in}}%
\pgfpathlineto{\pgfqpoint{0.621250in}{1.077957in}}%
\pgfpathlineto{\pgfqpoint{0.623116in}{1.184730in}}%
\pgfpathlineto{\pgfqpoint{0.625398in}{0.930360in}}%
\pgfpathlineto{\pgfqpoint{0.627264in}{1.163139in}}%
\pgfpathlineto{\pgfqpoint{0.629027in}{0.948910in}}%
\pgfpathlineto{\pgfqpoint{0.630168in}{1.107285in}}%
\pgfpathlineto{\pgfqpoint{0.631619in}{0.989560in}}%
\pgfpathlineto{\pgfqpoint{0.634108in}{1.293997in}}%
\pgfpathlineto{\pgfqpoint{0.635145in}{0.995255in}}%
\pgfpathlineto{\pgfqpoint{0.637219in}{1.358064in}}%
\pgfpathlineto{\pgfqpoint{0.638670in}{1.057734in}}%
\pgfpathlineto{\pgfqpoint{0.641677in}{1.255821in}}%
\pgfpathlineto{\pgfqpoint{0.642714in}{0.901445in}}%
\pgfpathlineto{\pgfqpoint{0.644996in}{1.212147in}}%
\pgfpathlineto{\pgfqpoint{0.645825in}{1.013616in}}%
\pgfpathlineto{\pgfqpoint{0.648106in}{1.266381in}}%
\pgfpathlineto{\pgfqpoint{0.649247in}{1.083558in}}%
\pgfpathlineto{\pgfqpoint{0.651528in}{0.967643in}}%
\pgfpathlineto{\pgfqpoint{0.654328in}{1.329390in}}%
\pgfpathlineto{\pgfqpoint{0.655157in}{1.013870in}}%
\pgfpathlineto{\pgfqpoint{0.656402in}{1.246876in}}%
\pgfpathlineto{\pgfqpoint{0.658268in}{0.998176in}}%
\pgfpathlineto{\pgfqpoint{0.659927in}{1.313396in}}%
\pgfpathlineto{\pgfqpoint{0.661586in}{1.001843in}}%
\pgfpathlineto{\pgfqpoint{0.664490in}{0.888152in}}%
\pgfpathlineto{\pgfqpoint{0.665112in}{1.300870in}}%
\pgfpathlineto{\pgfqpoint{0.667289in}{0.983735in}}%
\pgfpathlineto{\pgfqpoint{0.669363in}{1.094726in}}%
\pgfpathlineto{\pgfqpoint{0.670815in}{0.881785in}}%
\pgfpathlineto{\pgfqpoint{0.673718in}{1.147334in}}%
\pgfpathlineto{\pgfqpoint{0.674444in}{0.856092in}}%
\pgfpathlineto{\pgfqpoint{0.676000in}{1.187764in}}%
\pgfpathlineto{\pgfqpoint{0.678592in}{0.940476in}}%
\pgfpathlineto{\pgfqpoint{0.680251in}{1.306281in}}%
\pgfpathlineto{\pgfqpoint{0.682532in}{0.850376in}}%
\pgfpathlineto{\pgfqpoint{0.683362in}{1.214418in}}%
\pgfpathlineto{\pgfqpoint{0.684813in}{0.918363in}}%
\pgfpathlineto{\pgfqpoint{0.687198in}{1.129400in}}%
\pgfpathlineto{\pgfqpoint{0.688546in}{0.907351in}}%
\pgfpathlineto{\pgfqpoint{0.689791in}{1.084641in}}%
\pgfpathlineto{\pgfqpoint{0.691553in}{0.896279in}}%
\pgfpathlineto{\pgfqpoint{0.693524in}{0.844032in}}%
\pgfpathlineto{\pgfqpoint{0.695286in}{1.200860in}}%
\pgfpathlineto{\pgfqpoint{0.697879in}{0.984884in}}%
\pgfpathlineto{\pgfqpoint{0.700782in}{1.306939in}}%
\pgfpathlineto{\pgfqpoint{0.702337in}{1.062406in}}%
\pgfpathlineto{\pgfqpoint{0.704826in}{1.025878in}}%
\pgfpathlineto{\pgfqpoint{0.705656in}{1.296207in}}%
\pgfpathlineto{\pgfqpoint{0.708248in}{1.403043in}}%
\pgfpathlineto{\pgfqpoint{0.709492in}{1.183707in}}%
\pgfpathlineto{\pgfqpoint{0.712085in}{1.362467in}}%
\pgfpathlineto{\pgfqpoint{0.713018in}{1.091484in}}%
\pgfpathlineto{\pgfqpoint{0.715818in}{1.381309in}}%
\pgfpathlineto{\pgfqpoint{0.716958in}{0.972396in}}%
\pgfpathlineto{\pgfqpoint{0.717995in}{1.169742in}}%
\pgfpathlineto{\pgfqpoint{0.720484in}{1.391036in}}%
\pgfpathlineto{\pgfqpoint{0.722454in}{1.140069in}}%
\pgfpathlineto{\pgfqpoint{0.724320in}{1.371597in}}%
\pgfpathlineto{\pgfqpoint{0.726083in}{1.063419in}}%
\pgfpathlineto{\pgfqpoint{0.727846in}{1.307389in}}%
\pgfpathlineto{\pgfqpoint{0.729712in}{1.059103in}}%
\pgfpathlineto{\pgfqpoint{0.731268in}{1.420710in}}%
\pgfpathlineto{\pgfqpoint{0.732408in}{1.255043in}}%
\pgfpathlineto{\pgfqpoint{0.735104in}{1.078852in}}%
\pgfpathlineto{\pgfqpoint{0.736452in}{1.398101in}}%
\pgfpathlineto{\pgfqpoint{0.737800in}{1.134483in}}%
\pgfpathlineto{\pgfqpoint{0.739874in}{1.283820in}}%
\pgfpathlineto{\pgfqpoint{0.741222in}{1.101298in}}%
\pgfpathlineto{\pgfqpoint{0.743711in}{1.411161in}}%
\pgfpathlineto{\pgfqpoint{0.745059in}{1.094926in}}%
\pgfpathlineto{\pgfqpoint{0.747858in}{1.374212in}}%
\pgfpathlineto{\pgfqpoint{0.749206in}{1.007888in}}%
\pgfpathlineto{\pgfqpoint{0.749932in}{1.278078in}}%
\pgfpathlineto{\pgfqpoint{0.752939in}{1.088519in}}%
\pgfpathlineto{\pgfqpoint{0.754910in}{1.315609in}}%
\pgfpathlineto{\pgfqpoint{0.756050in}{0.951544in}}%
\pgfpathlineto{\pgfqpoint{0.757917in}{0.978494in}}%
\pgfpathlineto{\pgfqpoint{0.759368in}{1.338949in}}%
\pgfpathlineto{\pgfqpoint{0.760716in}{1.104732in}}%
\pgfpathlineto{\pgfqpoint{0.762064in}{1.391249in}}%
\pgfpathlineto{\pgfqpoint{0.763827in}{1.224746in}}%
\pgfpathlineto{\pgfqpoint{0.766212in}{1.496517in}}%
\pgfpathlineto{\pgfqpoint{0.767353in}{1.225538in}}%
\pgfpathlineto{\pgfqpoint{0.769841in}{1.100613in}}%
\pgfpathlineto{\pgfqpoint{0.770878in}{1.487394in}}%
\pgfpathlineto{\pgfqpoint{0.773885in}{1.495872in}}%
\pgfpathlineto{\pgfqpoint{0.775130in}{1.024533in}}%
\pgfpathlineto{\pgfqpoint{0.776167in}{1.349766in}}%
\pgfpathlineto{\pgfqpoint{0.780522in}{0.960311in}}%
\pgfpathlineto{\pgfqpoint{0.783010in}{1.366589in}}%
\pgfpathlineto{\pgfqpoint{0.783632in}{1.130158in}}%
\pgfpathlineto{\pgfqpoint{0.786536in}{1.474046in}}%
\pgfpathlineto{\pgfqpoint{0.787884in}{1.198281in}}%
\pgfpathlineto{\pgfqpoint{0.789543in}{1.409163in}}%
\pgfpathlineto{\pgfqpoint{0.791098in}{1.212785in}}%
\pgfpathlineto{\pgfqpoint{0.792550in}{1.462078in}}%
\pgfpathlineto{\pgfqpoint{0.793794in}{1.286469in}}%
\pgfpathlineto{\pgfqpoint{0.797009in}{1.177588in}}%
\pgfpathlineto{\pgfqpoint{0.798046in}{1.492867in}}%
\pgfpathlineto{\pgfqpoint{0.800327in}{1.131117in}}%
\pgfpathlineto{\pgfqpoint{0.802193in}{1.358797in}}%
\pgfpathlineto{\pgfqpoint{0.802712in}{1.043050in}}%
\pgfpathlineto{\pgfqpoint{0.805511in}{1.267714in}}%
\pgfpathlineto{\pgfqpoint{0.806445in}{1.055804in}}%
\pgfpathlineto{\pgfqpoint{0.808000in}{1.297860in}}%
\pgfpathlineto{\pgfqpoint{0.810281in}{1.121133in}}%
\pgfpathlineto{\pgfqpoint{0.812770in}{1.300681in}}%
\pgfpathlineto{\pgfqpoint{0.814636in}{1.035685in}}%
\pgfpathlineto{\pgfqpoint{0.815051in}{1.285312in}}%
\pgfpathlineto{\pgfqpoint{0.818369in}{1.080136in}}%
\pgfpathlineto{\pgfqpoint{0.818473in}{1.354794in}}%
\pgfpathlineto{\pgfqpoint{0.820962in}{1.128969in}}%
\pgfpathlineto{\pgfqpoint{0.822413in}{1.397416in}}%
\pgfpathlineto{\pgfqpoint{0.825420in}{0.940180in}}%
\pgfpathlineto{\pgfqpoint{0.826561in}{1.412514in}}%
\pgfpathlineto{\pgfqpoint{0.827494in}{1.118370in}}%
\pgfpathlineto{\pgfqpoint{0.830294in}{1.303104in}}%
\pgfpathlineto{\pgfqpoint{0.831538in}{1.024478in}}%
\pgfpathlineto{\pgfqpoint{0.832886in}{1.222844in}}%
\pgfpathlineto{\pgfqpoint{0.835168in}{0.938125in}}%
\pgfpathlineto{\pgfqpoint{0.837138in}{1.292407in}}%
\pgfpathlineto{\pgfqpoint{0.838382in}{1.031799in}}%
\pgfpathlineto{\pgfqpoint{0.839937in}{1.303716in}}%
\pgfpathlineto{\pgfqpoint{0.842011in}{1.076446in}}%
\pgfpathlineto{\pgfqpoint{0.843670in}{1.282272in}}%
\pgfpathlineto{\pgfqpoint{0.846574in}{1.093233in}}%
\pgfpathlineto{\pgfqpoint{0.848025in}{1.316811in}}%
\pgfpathlineto{\pgfqpoint{0.849684in}{0.994749in}}%
\pgfpathlineto{\pgfqpoint{0.850410in}{1.254812in}}%
\pgfpathlineto{\pgfqpoint{0.851966in}{1.019277in}}%
\pgfpathlineto{\pgfqpoint{0.854558in}{1.441825in}}%
\pgfpathlineto{\pgfqpoint{0.856010in}{1.194968in}}%
\pgfpathlineto{\pgfqpoint{0.858395in}{1.307362in}}%
\pgfpathlineto{\pgfqpoint{0.860468in}{1.045037in}}%
\pgfpathlineto{\pgfqpoint{0.861817in}{1.312465in}}%
\pgfpathlineto{\pgfqpoint{0.864098in}{0.965324in}}%
\pgfpathlineto{\pgfqpoint{0.864927in}{1.209141in}}%
\pgfpathlineto{\pgfqpoint{0.866897in}{1.079421in}}%
\pgfpathlineto{\pgfqpoint{0.868660in}{1.275006in}}%
\pgfpathlineto{\pgfqpoint{0.869697in}{1.053391in}}%
\pgfpathlineto{\pgfqpoint{0.871771in}{1.379902in}}%
\pgfpathlineto{\pgfqpoint{0.873637in}{1.102086in}}%
\pgfpathlineto{\pgfqpoint{0.875400in}{1.345193in}}%
\pgfpathlineto{\pgfqpoint{0.876748in}{1.092158in}}%
\pgfpathlineto{\pgfqpoint{0.879859in}{1.359429in}}%
\pgfpathlineto{\pgfqpoint{0.880585in}{1.043209in}}%
\pgfpathlineto{\pgfqpoint{0.882970in}{1.351137in}}%
\pgfpathlineto{\pgfqpoint{0.885873in}{0.963056in}}%
\pgfpathlineto{\pgfqpoint{0.887532in}{1.323545in}}%
\pgfpathlineto{\pgfqpoint{0.890021in}{1.100902in}}%
\pgfpathlineto{\pgfqpoint{0.890850in}{1.292909in}}%
\pgfpathlineto{\pgfqpoint{0.892924in}{1.030776in}}%
\pgfpathlineto{\pgfqpoint{0.895724in}{1.404378in}}%
\pgfpathlineto{\pgfqpoint{0.897072in}{1.144856in}}%
\pgfpathlineto{\pgfqpoint{0.898005in}{1.426076in}}%
\pgfpathlineto{\pgfqpoint{0.900079in}{1.217935in}}%
\pgfpathlineto{\pgfqpoint{0.902257in}{1.341782in}}%
\pgfpathlineto{\pgfqpoint{0.903086in}{1.149588in}}%
\pgfpathlineto{\pgfqpoint{0.905575in}{1.033690in}}%
\pgfpathlineto{\pgfqpoint{0.907337in}{1.298269in}}%
\pgfpathlineto{\pgfqpoint{0.909826in}{1.449369in}}%
\pgfpathlineto{\pgfqpoint{0.910759in}{1.152740in}}%
\pgfpathlineto{\pgfqpoint{0.913041in}{1.266306in}}%
\pgfpathlineto{\pgfqpoint{0.914596in}{0.874379in}}%
\pgfpathlineto{\pgfqpoint{0.915737in}{1.357100in}}%
\pgfpathlineto{\pgfqpoint{0.917499in}{1.078707in}}%
\pgfpathlineto{\pgfqpoint{0.919884in}{1.360825in}}%
\pgfpathlineto{\pgfqpoint{0.921129in}{0.990552in}}%
\pgfpathlineto{\pgfqpoint{0.923721in}{1.004756in}}%
\pgfpathlineto{\pgfqpoint{0.925587in}{1.322001in}}%
\pgfpathlineto{\pgfqpoint{0.926624in}{1.052247in}}%
\pgfpathlineto{\pgfqpoint{0.927765in}{1.236144in}}%
\pgfpathlineto{\pgfqpoint{0.929735in}{0.931857in}}%
\pgfpathlineto{\pgfqpoint{0.931705in}{1.174975in}}%
\pgfpathlineto{\pgfqpoint{0.933261in}{0.944865in}}%
\pgfpathlineto{\pgfqpoint{0.934920in}{1.260249in}}%
\pgfpathlineto{\pgfqpoint{0.938030in}{1.043615in}}%
\pgfpathlineto{\pgfqpoint{0.938964in}{1.239363in}}%
\pgfpathlineto{\pgfqpoint{0.940312in}{1.010609in}}%
\pgfpathlineto{\pgfqpoint{0.942178in}{1.122663in}}%
\pgfpathlineto{\pgfqpoint{0.943734in}{0.874231in}}%
\pgfpathlineto{\pgfqpoint{0.946533in}{1.256252in}}%
\pgfpathlineto{\pgfqpoint{0.947155in}{0.968559in}}%
\pgfpathlineto{\pgfqpoint{0.948918in}{1.297434in}}%
\pgfpathlineto{\pgfqpoint{0.950681in}{0.997429in}}%
\pgfpathlineto{\pgfqpoint{0.954103in}{1.329736in}}%
\pgfpathlineto{\pgfqpoint{0.954725in}{1.133639in}}%
\pgfpathlineto{\pgfqpoint{0.956799in}{1.020157in}}%
\pgfpathlineto{\pgfqpoint{0.958562in}{1.301478in}}%
\pgfpathlineto{\pgfqpoint{0.960013in}{1.101719in}}%
\pgfpathlineto{\pgfqpoint{0.962606in}{1.327901in}}%
\pgfpathlineto{\pgfqpoint{0.964265in}{0.940133in}}%
\pgfpathlineto{\pgfqpoint{0.965302in}{1.259839in}}%
\pgfpathlineto{\pgfqpoint{0.967168in}{1.078432in}}%
\pgfpathlineto{\pgfqpoint{0.969035in}{1.312580in}}%
\pgfpathlineto{\pgfqpoint{0.970590in}{1.063735in}}%
\pgfpathlineto{\pgfqpoint{0.971834in}{1.369674in}}%
\pgfpathlineto{\pgfqpoint{0.973701in}{0.946407in}}%
\pgfpathlineto{\pgfqpoint{0.976293in}{1.230888in}}%
\pgfpathlineto{\pgfqpoint{0.977330in}{1.009711in}}%
\pgfpathlineto{\pgfqpoint{0.979715in}{1.249575in}}%
\pgfpathlineto{\pgfqpoint{0.980648in}{0.972000in}}%
\pgfpathlineto{\pgfqpoint{0.983344in}{1.343537in}}%
\pgfpathlineto{\pgfqpoint{0.984899in}{0.998915in}}%
\pgfpathlineto{\pgfqpoint{0.986973in}{1.360968in}}%
\pgfpathlineto{\pgfqpoint{0.987699in}{1.158975in}}%
\pgfpathlineto{\pgfqpoint{0.990810in}{1.298318in}}%
\pgfpathlineto{\pgfqpoint{0.991743in}{1.096533in}}%
\pgfpathlineto{\pgfqpoint{0.993506in}{1.054834in}}%
\pgfpathlineto{\pgfqpoint{0.994958in}{1.275380in}}%
\pgfpathlineto{\pgfqpoint{0.997135in}{1.155838in}}%
\pgfpathlineto{\pgfqpoint{0.998483in}{1.401011in}}%
\pgfpathlineto{\pgfqpoint{1.000972in}{1.480987in}}%
\pgfpathlineto{\pgfqpoint{1.002527in}{1.072067in}}%
\pgfpathlineto{\pgfqpoint{1.004912in}{1.012967in}}%
\pgfpathlineto{\pgfqpoint{1.006467in}{1.260891in}}%
\pgfpathlineto{\pgfqpoint{1.007297in}{0.923455in}}%
\pgfpathlineto{\pgfqpoint{1.009267in}{1.113085in}}%
\pgfpathlineto{\pgfqpoint{1.011134in}{0.874643in}}%
\pgfpathlineto{\pgfqpoint{1.012482in}{1.096356in}}%
\pgfpathlineto{\pgfqpoint{1.014348in}{1.156946in}}%
\pgfpathlineto{\pgfqpoint{1.015903in}{0.958630in}}%
\pgfpathlineto{\pgfqpoint{1.018496in}{0.859939in}}%
\pgfpathlineto{\pgfqpoint{1.020155in}{1.161625in}}%
\pgfpathlineto{\pgfqpoint{1.021399in}{0.949365in}}%
\pgfpathlineto{\pgfqpoint{1.023680in}{1.233788in}}%
\pgfpathlineto{\pgfqpoint{1.025028in}{1.012819in}}%
\pgfpathlineto{\pgfqpoint{1.028450in}{1.455961in}}%
\pgfpathlineto{\pgfqpoint{1.030006in}{0.996643in}}%
\pgfpathlineto{\pgfqpoint{1.032391in}{1.172293in}}%
\pgfpathlineto{\pgfqpoint{1.033739in}{0.823967in}}%
\pgfpathlineto{\pgfqpoint{1.036331in}{1.193482in}}%
\pgfpathlineto{\pgfqpoint{1.038301in}{0.910785in}}%
\pgfpathlineto{\pgfqpoint{1.038820in}{1.121431in}}%
\pgfpathlineto{\pgfqpoint{1.041930in}{1.218383in}}%
\pgfpathlineto{\pgfqpoint{1.042656in}{0.988642in}}%
\pgfpathlineto{\pgfqpoint{1.044937in}{1.273692in}}%
\pgfpathlineto{\pgfqpoint{1.047219in}{0.990386in}}%
\pgfpathlineto{\pgfqpoint{1.048774in}{1.188734in}}%
\pgfpathlineto{\pgfqpoint{1.050848in}{0.798244in}}%
\pgfpathlineto{\pgfqpoint{1.052714in}{1.057435in}}%
\pgfpathlineto{\pgfqpoint{1.053129in}{0.766032in}}%
\pgfpathlineto{\pgfqpoint{1.054996in}{1.082565in}}%
\pgfpathlineto{\pgfqpoint{1.056966in}{1.137066in}}%
\pgfpathlineto{\pgfqpoint{1.058728in}{0.823645in}}%
\pgfpathlineto{\pgfqpoint{1.060180in}{1.112169in}}%
\pgfpathlineto{\pgfqpoint{1.061943in}{0.801660in}}%
\pgfpathlineto{\pgfqpoint{1.064328in}{1.110968in}}%
\pgfpathlineto{\pgfqpoint{1.065365in}{0.849851in}}%
\pgfpathlineto{\pgfqpoint{1.068164in}{1.132772in}}%
\pgfpathlineto{\pgfqpoint{1.069305in}{0.846715in}}%
\pgfpathlineto{\pgfqpoint{1.071897in}{1.085716in}}%
\pgfpathlineto{\pgfqpoint{1.073764in}{0.778716in}}%
\pgfpathlineto{\pgfqpoint{1.074179in}{1.024447in}}%
\pgfpathlineto{\pgfqpoint{1.076564in}{0.819791in}}%
\pgfpathlineto{\pgfqpoint{1.079467in}{1.152387in}}%
\pgfpathlineto{\pgfqpoint{1.082370in}{0.805926in}}%
\pgfpathlineto{\pgfqpoint{1.082993in}{1.024530in}}%
\pgfpathlineto{\pgfqpoint{1.084755in}{1.150579in}}%
\pgfpathlineto{\pgfqpoint{1.086518in}{0.954872in}}%
\pgfpathlineto{\pgfqpoint{1.088488in}{1.118497in}}%
\pgfpathlineto{\pgfqpoint{1.090044in}{0.935634in}}%
\pgfpathlineto{\pgfqpoint{1.092843in}{1.239214in}}%
\pgfpathlineto{\pgfqpoint{1.093673in}{0.954234in}}%
\pgfpathlineto{\pgfqpoint{1.096576in}{0.725583in}}%
\pgfpathlineto{\pgfqpoint{1.097198in}{1.117089in}}%
\pgfpathlineto{\pgfqpoint{1.099894in}{0.948740in}}%
\pgfpathlineto{\pgfqpoint{1.101035in}{1.138488in}}%
\pgfpathlineto{\pgfqpoint{1.103005in}{0.916619in}}%
\pgfpathlineto{\pgfqpoint{1.104561in}{1.235430in}}%
\pgfpathlineto{\pgfqpoint{1.106220in}{0.992595in}}%
\pgfpathlineto{\pgfqpoint{1.108086in}{1.135410in}}%
\pgfpathlineto{\pgfqpoint{1.110886in}{1.119021in}}%
\pgfpathlineto{\pgfqpoint{1.111093in}{0.824042in}}%
\pgfpathlineto{\pgfqpoint{1.113063in}{0.779698in}}%
\pgfpathlineto{\pgfqpoint{1.114930in}{1.022809in}}%
\pgfpathlineto{\pgfqpoint{1.118041in}{0.875334in}}%
\pgfpathlineto{\pgfqpoint{1.118974in}{1.183609in}}%
\pgfpathlineto{\pgfqpoint{1.120011in}{0.866368in}}%
\pgfpathlineto{\pgfqpoint{1.121670in}{1.125013in}}%
\pgfpathlineto{\pgfqpoint{1.123847in}{0.867907in}}%
\pgfpathlineto{\pgfqpoint{1.125714in}{1.176358in}}%
\pgfpathlineto{\pgfqpoint{1.128514in}{1.002077in}}%
\pgfpathlineto{\pgfqpoint{1.129136in}{1.186281in}}%
\pgfpathlineto{\pgfqpoint{1.131417in}{0.888974in}}%
\pgfpathlineto{\pgfqpoint{1.133698in}{1.288233in}}%
\pgfpathlineto{\pgfqpoint{1.134217in}{0.968643in}}%
\pgfpathlineto{\pgfqpoint{1.137120in}{0.978633in}}%
\pgfpathlineto{\pgfqpoint{1.137742in}{1.329446in}}%
\pgfpathlineto{\pgfqpoint{1.139712in}{1.015468in}}%
\pgfpathlineto{\pgfqpoint{1.142201in}{1.213615in}}%
\pgfpathlineto{\pgfqpoint{1.143549in}{0.964351in}}%
\pgfpathlineto{\pgfqpoint{1.145519in}{1.223225in}}%
\pgfpathlineto{\pgfqpoint{1.146452in}{1.004140in}}%
\pgfpathlineto{\pgfqpoint{1.149563in}{1.273017in}}%
\pgfpathlineto{\pgfqpoint{1.150704in}{0.994572in}}%
\pgfpathlineto{\pgfqpoint{1.152155in}{1.163086in}}%
\pgfpathlineto{\pgfqpoint{1.154126in}{0.913606in}}%
\pgfpathlineto{\pgfqpoint{1.155162in}{1.208142in}}%
\pgfpathlineto{\pgfqpoint{1.157133in}{1.011729in}}%
\pgfpathlineto{\pgfqpoint{1.159310in}{1.233984in}}%
\pgfpathlineto{\pgfqpoint{1.160762in}{0.892150in}}%
\pgfpathlineto{\pgfqpoint{1.162317in}{1.177023in}}%
\pgfpathlineto{\pgfqpoint{1.165635in}{1.233912in}}%
\pgfpathlineto{\pgfqpoint{1.166776in}{0.992655in}}%
\pgfpathlineto{\pgfqpoint{1.168539in}{1.143085in}}%
\pgfpathlineto{\pgfqpoint{1.170094in}{0.842603in}}%
\pgfpathlineto{\pgfqpoint{1.172272in}{1.075173in}}%
\pgfpathlineto{\pgfqpoint{1.174346in}{0.826284in}}%
\pgfpathlineto{\pgfqpoint{1.175175in}{1.025248in}}%
\pgfpathlineto{\pgfqpoint{1.177456in}{0.825914in}}%
\pgfpathlineto{\pgfqpoint{1.178182in}{1.014912in}}%
\pgfpathlineto{\pgfqpoint{1.180256in}{0.841758in}}%
\pgfpathlineto{\pgfqpoint{1.183782in}{1.159188in}}%
\pgfpathlineto{\pgfqpoint{1.185648in}{0.854101in}}%
\pgfpathlineto{\pgfqpoint{1.187307in}{1.082087in}}%
\pgfpathlineto{\pgfqpoint{1.189796in}{0.929481in}}%
\pgfpathlineto{\pgfqpoint{1.190625in}{1.149205in}}%
\pgfpathlineto{\pgfqpoint{1.193632in}{0.874486in}}%
\pgfpathlineto{\pgfqpoint{1.194358in}{1.164661in}}%
\pgfpathlineto{\pgfqpoint{1.195706in}{0.954888in}}%
\pgfpathlineto{\pgfqpoint{1.197469in}{1.189984in}}%
\pgfpathlineto{\pgfqpoint{1.200580in}{0.875140in}}%
\pgfpathlineto{\pgfqpoint{1.201306in}{1.159546in}}%
\pgfpathlineto{\pgfqpoint{1.203794in}{1.255580in}}%
\pgfpathlineto{\pgfqpoint{1.205039in}{0.930527in}}%
\pgfpathlineto{\pgfqpoint{1.207527in}{1.103951in}}%
\pgfpathlineto{\pgfqpoint{1.209290in}{0.812602in}}%
\pgfpathlineto{\pgfqpoint{1.210431in}{1.086989in}}%
\pgfpathlineto{\pgfqpoint{1.211882in}{0.854884in}}%
\pgfpathlineto{\pgfqpoint{1.213852in}{1.025860in}}%
\pgfpathlineto{\pgfqpoint{1.216030in}{0.773450in}}%
\pgfpathlineto{\pgfqpoint{1.218104in}{1.168317in}}%
\pgfpathlineto{\pgfqpoint{1.218830in}{0.916341in}}%
\pgfpathlineto{\pgfqpoint{1.221733in}{1.072355in}}%
\pgfpathlineto{\pgfqpoint{1.222770in}{0.678438in}}%
\pgfpathlineto{\pgfqpoint{1.224636in}{0.906200in}}%
\pgfpathlineto{\pgfqpoint{1.226088in}{0.654046in}}%
\pgfpathlineto{\pgfqpoint{1.228058in}{0.610328in}}%
\pgfpathlineto{\pgfqpoint{1.229614in}{0.896987in}}%
\pgfpathlineto{\pgfqpoint{1.232413in}{0.768162in}}%
\pgfpathlineto{\pgfqpoint{1.233658in}{1.070793in}}%
\pgfpathlineto{\pgfqpoint{1.234902in}{0.872571in}}%
\pgfpathlineto{\pgfqpoint{1.237287in}{1.119819in}}%
\pgfpathlineto{\pgfqpoint{1.238428in}{0.850384in}}%
\pgfpathlineto{\pgfqpoint{1.239983in}{1.213973in}}%
\pgfpathlineto{\pgfqpoint{1.241538in}{0.840210in}}%
\pgfpathlineto{\pgfqpoint{1.243716in}{1.288257in}}%
\pgfpathlineto{\pgfqpoint{1.245790in}{0.960859in}}%
\pgfpathlineto{\pgfqpoint{1.247034in}{1.113527in}}%
\pgfpathlineto{\pgfqpoint{1.249626in}{0.958865in}}%
\pgfpathlineto{\pgfqpoint{1.250871in}{1.174360in}}%
\pgfpathlineto{\pgfqpoint{1.254604in}{0.878479in}}%
\pgfpathlineto{\pgfqpoint{1.257300in}{1.138945in}}%
\pgfpathlineto{\pgfqpoint{1.257818in}{0.891364in}}%
\pgfpathlineto{\pgfqpoint{1.260825in}{1.127050in}}%
\pgfpathlineto{\pgfqpoint{1.261862in}{0.839763in}}%
\pgfpathlineto{\pgfqpoint{1.263832in}{1.163965in}}%
\pgfpathlineto{\pgfqpoint{1.264454in}{0.913030in}}%
\pgfpathlineto{\pgfqpoint{1.267047in}{1.121504in}}%
\pgfpathlineto{\pgfqpoint{1.268913in}{0.895541in}}%
\pgfpathlineto{\pgfqpoint{1.270468in}{1.230643in}}%
\pgfpathlineto{\pgfqpoint{1.271505in}{1.019045in}}%
\pgfpathlineto{\pgfqpoint{1.274927in}{0.849414in}}%
\pgfpathlineto{\pgfqpoint{1.275549in}{1.185753in}}%
\pgfpathlineto{\pgfqpoint{1.278349in}{0.934567in}}%
\pgfpathlineto{\pgfqpoint{1.279697in}{1.104354in}}%
\pgfpathlineto{\pgfqpoint{1.280423in}{0.921586in}}%
\pgfpathlineto{\pgfqpoint{1.282082in}{1.141741in}}%
\pgfpathlineto{\pgfqpoint{1.284156in}{0.752256in}}%
\pgfpathlineto{\pgfqpoint{1.286852in}{0.679347in}}%
\pgfpathlineto{\pgfqpoint{1.287993in}{1.013191in}}%
\pgfpathlineto{\pgfqpoint{1.289859in}{0.866241in}}%
\pgfpathlineto{\pgfqpoint{1.291207in}{1.083682in}}%
\pgfpathlineto{\pgfqpoint{1.293903in}{0.787605in}}%
\pgfpathlineto{\pgfqpoint{1.294836in}{0.993449in}}%
\pgfpathlineto{\pgfqpoint{1.297014in}{1.156598in}}%
\pgfpathlineto{\pgfqpoint{1.298154in}{0.881959in}}%
\pgfpathlineto{\pgfqpoint{1.301265in}{1.092789in}}%
\pgfpathlineto{\pgfqpoint{1.302509in}{0.835140in}}%
\pgfpathlineto{\pgfqpoint{1.304169in}{1.001871in}}%
\pgfpathlineto{\pgfqpoint{1.306450in}{0.839773in}}%
\pgfpathlineto{\pgfqpoint{1.307694in}{1.031752in}}%
\pgfpathlineto{\pgfqpoint{1.309042in}{0.804220in}}%
\pgfpathlineto{\pgfqpoint{1.311531in}{1.072734in}}%
\pgfpathlineto{\pgfqpoint{1.312671in}{0.891172in}}%
\pgfpathlineto{\pgfqpoint{1.314227in}{1.236064in}}%
\pgfpathlineto{\pgfqpoint{1.315989in}{0.711002in}}%
\pgfpathlineto{\pgfqpoint{1.317337in}{1.128276in}}%
\pgfpathlineto{\pgfqpoint{1.320345in}{1.165313in}}%
\pgfpathlineto{\pgfqpoint{1.322418in}{0.710759in}}%
\pgfpathlineto{\pgfqpoint{1.322833in}{1.022187in}}%
\pgfpathlineto{\pgfqpoint{1.324492in}{0.785643in}}%
\pgfpathlineto{\pgfqpoint{1.327085in}{1.118041in}}%
\pgfpathlineto{\pgfqpoint{1.328847in}{0.794367in}}%
\pgfpathlineto{\pgfqpoint{1.330092in}{1.164991in}}%
\pgfpathlineto{\pgfqpoint{1.332062in}{0.700176in}}%
\pgfpathlineto{\pgfqpoint{1.333306in}{0.967345in}}%
\pgfpathlineto{\pgfqpoint{1.336624in}{0.801211in}}%
\pgfpathlineto{\pgfqpoint{1.337039in}{1.080068in}}%
\pgfpathlineto{\pgfqpoint{1.339942in}{0.840617in}}%
\pgfpathlineto{\pgfqpoint{1.340668in}{1.110074in}}%
\pgfpathlineto{\pgfqpoint{1.343468in}{1.244258in}}%
\pgfpathlineto{\pgfqpoint{1.345334in}{0.846376in}}%
\pgfpathlineto{\pgfqpoint{1.345749in}{0.995950in}}%
\pgfpathlineto{\pgfqpoint{1.348549in}{0.915337in}}%
\pgfpathlineto{\pgfqpoint{1.349378in}{1.211144in}}%
\pgfpathlineto{\pgfqpoint{1.350415in}{0.878569in}}%
\pgfpathlineto{\pgfqpoint{1.352593in}{0.794578in}}%
\pgfpathlineto{\pgfqpoint{1.354770in}{1.066149in}}%
\pgfpathlineto{\pgfqpoint{1.355911in}{0.797267in}}%
\pgfpathlineto{\pgfqpoint{1.357052in}{1.000686in}}%
\pgfpathlineto{\pgfqpoint{1.359644in}{1.091404in}}%
\pgfpathlineto{\pgfqpoint{1.360681in}{0.850446in}}%
\pgfpathlineto{\pgfqpoint{1.363170in}{1.168804in}}%
\pgfpathlineto{\pgfqpoint{1.364310in}{0.838609in}}%
\pgfpathlineto{\pgfqpoint{1.365347in}{1.081261in}}%
\pgfpathlineto{\pgfqpoint{1.367214in}{0.810657in}}%
\pgfpathlineto{\pgfqpoint{1.368665in}{1.193012in}}%
\pgfpathlineto{\pgfqpoint{1.371258in}{1.191851in}}%
\pgfpathlineto{\pgfqpoint{1.373020in}{0.790256in}}%
\pgfpathlineto{\pgfqpoint{1.374472in}{0.844606in}}%
\pgfpathlineto{\pgfqpoint{1.375820in}{1.218727in}}%
\pgfpathlineto{\pgfqpoint{1.377479in}{0.887944in}}%
\pgfpathlineto{\pgfqpoint{1.379346in}{1.214338in}}%
\pgfpathlineto{\pgfqpoint{1.382353in}{0.943204in}}%
\pgfpathlineto{\pgfqpoint{1.384323in}{1.255174in}}%
\pgfpathlineto{\pgfqpoint{1.386086in}{0.892173in}}%
\pgfpathlineto{\pgfqpoint{1.387122in}{1.211218in}}%
\pgfpathlineto{\pgfqpoint{1.389404in}{0.862741in}}%
\pgfpathlineto{\pgfqpoint{1.390648in}{1.118897in}}%
\pgfpathlineto{\pgfqpoint{1.393033in}{0.853572in}}%
\pgfpathlineto{\pgfqpoint{1.394381in}{1.154350in}}%
\pgfpathlineto{\pgfqpoint{1.396247in}{0.813402in}}%
\pgfpathlineto{\pgfqpoint{1.398321in}{1.113061in}}%
\pgfpathlineto{\pgfqpoint{1.398736in}{0.950112in}}%
\pgfpathlineto{\pgfqpoint{1.401121in}{0.773594in}}%
\pgfpathlineto{\pgfqpoint{1.401847in}{1.023894in}}%
\pgfpathlineto{\pgfqpoint{1.404750in}{0.669597in}}%
\pgfpathlineto{\pgfqpoint{1.405372in}{0.967296in}}%
\pgfpathlineto{\pgfqpoint{1.407135in}{1.114528in}}%
\pgfpathlineto{\pgfqpoint{1.408691in}{0.841976in}}%
\pgfpathlineto{\pgfqpoint{1.411594in}{0.659284in}}%
\pgfpathlineto{\pgfqpoint{1.413564in}{1.080034in}}%
\pgfpathlineto{\pgfqpoint{1.415119in}{0.873115in}}%
\pgfpathlineto{\pgfqpoint{1.417608in}{0.866050in}}%
\pgfpathlineto{\pgfqpoint{1.419060in}{1.155071in}}%
\pgfpathlineto{\pgfqpoint{1.420097in}{0.950570in}}%
\pgfpathlineto{\pgfqpoint{1.422585in}{1.262264in}}%
\pgfpathlineto{\pgfqpoint{1.424244in}{0.932680in}}%
\pgfpathlineto{\pgfqpoint{1.425178in}{1.167246in}}%
\pgfpathlineto{\pgfqpoint{1.427355in}{0.990220in}}%
\pgfpathlineto{\pgfqpoint{1.429014in}{1.180358in}}%
\pgfpathlineto{\pgfqpoint{1.430362in}{0.791118in}}%
\pgfpathlineto{\pgfqpoint{1.431814in}{1.293213in}}%
\pgfpathlineto{\pgfqpoint{1.434095in}{0.903070in}}%
\pgfpathlineto{\pgfqpoint{1.435858in}{1.074862in}}%
\pgfpathlineto{\pgfqpoint{1.437724in}{0.953736in}}%
\pgfpathlineto{\pgfqpoint{1.438658in}{1.210078in}}%
\pgfpathlineto{\pgfqpoint{1.440213in}{0.973345in}}%
\pgfpathlineto{\pgfqpoint{1.441976in}{1.235727in}}%
\pgfpathlineto{\pgfqpoint{1.443531in}{1.048381in}}%
\pgfpathlineto{\pgfqpoint{1.444983in}{1.340767in}}%
\pgfpathlineto{\pgfqpoint{1.447575in}{0.863249in}}%
\pgfpathlineto{\pgfqpoint{1.448716in}{1.232031in}}%
\pgfpathlineto{\pgfqpoint{1.450375in}{0.953385in}}%
\pgfpathlineto{\pgfqpoint{1.452449in}{1.131921in}}%
\pgfpathlineto{\pgfqpoint{1.454004in}{0.926183in}}%
\pgfpathlineto{\pgfqpoint{1.455145in}{1.205195in}}%
\pgfpathlineto{\pgfqpoint{1.456700in}{0.819142in}}%
\pgfpathlineto{\pgfqpoint{1.458878in}{1.122669in}}%
\pgfpathlineto{\pgfqpoint{1.459915in}{0.864917in}}%
\pgfpathlineto{\pgfqpoint{1.462300in}{1.167378in}}%
\pgfpathlineto{\pgfqpoint{1.463751in}{0.893586in}}%
\pgfpathlineto{\pgfqpoint{1.465618in}{1.312434in}}%
\pgfpathlineto{\pgfqpoint{1.466655in}{0.936791in}}%
\pgfpathlineto{\pgfqpoint{1.468625in}{1.146548in}}%
\pgfpathlineto{\pgfqpoint{1.471217in}{0.911915in}}%
\pgfpathlineto{\pgfqpoint{1.472150in}{1.147454in}}%
\pgfpathlineto{\pgfqpoint{1.473498in}{0.912123in}}%
\pgfpathlineto{\pgfqpoint{1.475780in}{1.266207in}}%
\pgfpathlineto{\pgfqpoint{1.476816in}{1.067067in}}%
\pgfpathlineto{\pgfqpoint{1.478787in}{1.246293in}}%
\pgfpathlineto{\pgfqpoint{1.480757in}{0.996937in}}%
\pgfpathlineto{\pgfqpoint{1.481794in}{1.269364in}}%
\pgfpathlineto{\pgfqpoint{1.483764in}{0.966986in}}%
\pgfpathlineto{\pgfqpoint{1.484801in}{1.163736in}}%
\pgfpathlineto{\pgfqpoint{1.486978in}{1.248287in}}%
\pgfpathlineto{\pgfqpoint{1.489467in}{0.960684in}}%
\pgfpathlineto{\pgfqpoint{1.490815in}{1.138993in}}%
\pgfpathlineto{\pgfqpoint{1.492889in}{0.910175in}}%
\pgfpathlineto{\pgfqpoint{1.493511in}{1.138582in}}%
\pgfpathlineto{\pgfqpoint{1.494755in}{0.988933in}}%
\pgfpathlineto{\pgfqpoint{1.496933in}{1.250640in}}%
\pgfpathlineto{\pgfqpoint{1.499421in}{0.946407in}}%
\pgfpathlineto{\pgfqpoint{1.500769in}{1.234096in}}%
\pgfpathlineto{\pgfqpoint{1.502221in}{1.023011in}}%
\pgfpathlineto{\pgfqpoint{1.503051in}{1.225736in}}%
\pgfpathlineto{\pgfqpoint{1.505021in}{0.952364in}}%
\pgfpathlineto{\pgfqpoint{1.507509in}{1.219856in}}%
\pgfpathlineto{\pgfqpoint{1.508961in}{0.950744in}}%
\pgfpathlineto{\pgfqpoint{1.509791in}{1.200782in}}%
\pgfpathlineto{\pgfqpoint{1.512072in}{0.923593in}}%
\pgfpathlineto{\pgfqpoint{1.513524in}{1.240199in}}%
\pgfpathlineto{\pgfqpoint{1.515805in}{1.005508in}}%
\pgfpathlineto{\pgfqpoint{1.517257in}{1.360242in}}%
\pgfpathlineto{\pgfqpoint{1.518916in}{1.099185in}}%
\pgfpathlineto{\pgfqpoint{1.520367in}{1.371837in}}%
\pgfpathlineto{\pgfqpoint{1.521301in}{1.061946in}}%
\pgfpathlineto{\pgfqpoint{1.523893in}{1.279221in}}%
\pgfpathlineto{\pgfqpoint{1.526174in}{1.299071in}}%
\pgfpathlineto{\pgfqpoint{1.527522in}{0.962344in}}%
\pgfpathlineto{\pgfqpoint{1.527937in}{1.213425in}}%
\pgfpathlineto{\pgfqpoint{1.530218in}{1.255852in}}%
\pgfpathlineto{\pgfqpoint{1.531255in}{1.060126in}}%
\pgfpathlineto{\pgfqpoint{1.533225in}{1.250922in}}%
\pgfpathlineto{\pgfqpoint{1.534573in}{0.908266in}}%
\pgfpathlineto{\pgfqpoint{1.536751in}{1.103077in}}%
\pgfpathlineto{\pgfqpoint{1.539136in}{0.935360in}}%
\pgfpathlineto{\pgfqpoint{1.540069in}{1.147399in}}%
\pgfpathlineto{\pgfqpoint{1.541313in}{0.933615in}}%
\pgfpathlineto{\pgfqpoint{1.542869in}{1.142290in}}%
\pgfpathlineto{\pgfqpoint{1.545150in}{0.973051in}}%
\pgfpathlineto{\pgfqpoint{1.547224in}{1.238117in}}%
\pgfpathlineto{\pgfqpoint{1.548053in}{1.055410in}}%
\pgfpathlineto{\pgfqpoint{1.550438in}{0.937030in}}%
\pgfpathlineto{\pgfqpoint{1.551579in}{1.167793in}}%
\pgfpathlineto{\pgfqpoint{1.554793in}{0.932694in}}%
\pgfpathlineto{\pgfqpoint{1.556556in}{1.270580in}}%
\pgfpathlineto{\pgfqpoint{1.558837in}{1.053584in}}%
\pgfpathlineto{\pgfqpoint{1.560185in}{1.326906in}}%
\pgfpathlineto{\pgfqpoint{1.561326in}{1.106223in}}%
\pgfpathlineto{\pgfqpoint{1.563814in}{1.071095in}}%
\pgfpathlineto{\pgfqpoint{1.565266in}{1.318394in}}%
\pgfpathlineto{\pgfqpoint{1.566096in}{1.138171in}}%
\pgfpathlineto{\pgfqpoint{1.569206in}{1.220412in}}%
\pgfpathlineto{\pgfqpoint{1.570036in}{0.952618in}}%
\pgfpathlineto{\pgfqpoint{1.571177in}{1.193417in}}%
\pgfpathlineto{\pgfqpoint{1.573562in}{0.919525in}}%
\pgfpathlineto{\pgfqpoint{1.575532in}{1.131526in}}%
\pgfpathlineto{\pgfqpoint{1.576465in}{0.889062in}}%
\pgfpathlineto{\pgfqpoint{1.577813in}{1.082890in}}%
\pgfpathlineto{\pgfqpoint{1.580924in}{1.242316in}}%
\pgfpathlineto{\pgfqpoint{1.582375in}{0.865922in}}%
\pgfpathlineto{\pgfqpoint{1.582998in}{1.123480in}}%
\pgfpathlineto{\pgfqpoint{1.585486in}{0.944999in}}%
\pgfpathlineto{\pgfqpoint{1.586212in}{1.228009in}}%
\pgfpathlineto{\pgfqpoint{1.587767in}{0.911422in}}%
\pgfpathlineto{\pgfqpoint{1.589323in}{1.181616in}}%
\pgfpathlineto{\pgfqpoint{1.591811in}{0.979263in}}%
\pgfpathlineto{\pgfqpoint{1.592745in}{1.162320in}}%
\pgfpathlineto{\pgfqpoint{1.594404in}{0.968964in}}%
\pgfpathlineto{\pgfqpoint{1.597100in}{1.144395in}}%
\pgfpathlineto{\pgfqpoint{1.597929in}{0.878464in}}%
\pgfpathlineto{\pgfqpoint{1.599277in}{1.125101in}}%
\pgfpathlineto{\pgfqpoint{1.602181in}{1.001938in}}%
\pgfpathlineto{\pgfqpoint{1.603010in}{1.352643in}}%
\pgfpathlineto{\pgfqpoint{1.604566in}{1.053414in}}%
\pgfpathlineto{\pgfqpoint{1.606328in}{1.227073in}}%
\pgfpathlineto{\pgfqpoint{1.608195in}{0.920212in}}%
\pgfpathlineto{\pgfqpoint{1.610269in}{1.161980in}}%
\pgfpathlineto{\pgfqpoint{1.610995in}{0.916722in}}%
\pgfpathlineto{\pgfqpoint{1.614105in}{1.261949in}}%
\pgfpathlineto{\pgfqpoint{1.615142in}{0.958590in}}%
\pgfpathlineto{\pgfqpoint{1.616698in}{1.247793in}}%
\pgfpathlineto{\pgfqpoint{1.618564in}{0.983149in}}%
\pgfpathlineto{\pgfqpoint{1.620534in}{1.221803in}}%
\pgfpathlineto{\pgfqpoint{1.620845in}{1.046814in}}%
\pgfpathlineto{\pgfqpoint{1.622504in}{1.090881in}}%
\pgfpathlineto{\pgfqpoint{1.624475in}{0.938823in}}%
\pgfpathlineto{\pgfqpoint{1.627171in}{1.108654in}}%
\pgfpathlineto{\pgfqpoint{1.628622in}{0.740044in}}%
\pgfpathlineto{\pgfqpoint{1.629141in}{1.029920in}}%
\pgfpathlineto{\pgfqpoint{1.631940in}{1.202790in}}%
\pgfpathlineto{\pgfqpoint{1.632563in}{0.942363in}}%
\pgfpathlineto{\pgfqpoint{1.635570in}{1.135705in}}%
\pgfpathlineto{\pgfqpoint{1.635881in}{0.918382in}}%
\pgfpathlineto{\pgfqpoint{1.637851in}{1.124774in}}%
\pgfpathlineto{\pgfqpoint{1.639095in}{0.956867in}}%
\pgfpathlineto{\pgfqpoint{1.641791in}{0.981466in}}%
\pgfpathlineto{\pgfqpoint{1.643035in}{1.245560in}}%
\pgfpathlineto{\pgfqpoint{1.644383in}{1.002555in}}%
\pgfpathlineto{\pgfqpoint{1.646561in}{1.257074in}}%
\pgfpathlineto{\pgfqpoint{1.647598in}{0.991403in}}%
\pgfpathlineto{\pgfqpoint{1.649983in}{1.222186in}}%
\pgfpathlineto{\pgfqpoint{1.651123in}{0.893261in}}%
\pgfpathlineto{\pgfqpoint{1.652368in}{1.055373in}}%
\pgfpathlineto{\pgfqpoint{1.655064in}{1.245571in}}%
\pgfpathlineto{\pgfqpoint{1.655997in}{0.985286in}}%
\pgfpathlineto{\pgfqpoint{1.657552in}{1.138347in}}%
\pgfpathlineto{\pgfqpoint{1.659834in}{0.955596in}}%
\pgfpathlineto{\pgfqpoint{1.661078in}{1.139707in}}%
\pgfpathlineto{\pgfqpoint{1.663048in}{0.926881in}}%
\pgfpathlineto{\pgfqpoint{1.664915in}{1.167244in}}%
\pgfpathlineto{\pgfqpoint{1.665640in}{0.860646in}}%
\pgfpathlineto{\pgfqpoint{1.668336in}{0.781049in}}%
\pgfpathlineto{\pgfqpoint{1.668959in}{0.978968in}}%
\pgfpathlineto{\pgfqpoint{1.671655in}{1.197006in}}%
\pgfpathlineto{\pgfqpoint{1.672277in}{1.014414in}}%
\pgfpathlineto{\pgfqpoint{1.674973in}{0.855016in}}%
\pgfpathlineto{\pgfqpoint{1.675802in}{1.119596in}}%
\pgfpathlineto{\pgfqpoint{1.678913in}{0.942246in}}%
\pgfpathlineto{\pgfqpoint{1.680676in}{1.157885in}}%
\pgfpathlineto{\pgfqpoint{1.683164in}{0.850168in}}%
\pgfpathlineto{\pgfqpoint{1.683994in}{1.090810in}}%
\pgfpathlineto{\pgfqpoint{1.686897in}{1.220077in}}%
\pgfpathlineto{\pgfqpoint{1.687831in}{0.940412in}}%
\pgfpathlineto{\pgfqpoint{1.689282in}{1.227516in}}%
\pgfpathlineto{\pgfqpoint{1.690630in}{0.938033in}}%
\pgfpathlineto{\pgfqpoint{1.692704in}{1.162397in}}%
\pgfpathlineto{\pgfqpoint{1.693845in}{0.971671in}}%
\pgfpathlineto{\pgfqpoint{1.696022in}{1.184056in}}%
\pgfpathlineto{\pgfqpoint{1.697681in}{1.011969in}}%
\pgfpathlineto{\pgfqpoint{1.699859in}{1.221656in}}%
\pgfpathlineto{\pgfqpoint{1.700792in}{0.962828in}}%
\pgfpathlineto{\pgfqpoint{1.702244in}{1.196848in}}%
\pgfpathlineto{\pgfqpoint{1.703903in}{0.905102in}}%
\pgfpathlineto{\pgfqpoint{1.706288in}{1.143447in}}%
\pgfpathlineto{\pgfqpoint{1.708258in}{0.869976in}}%
\pgfpathlineto{\pgfqpoint{1.709191in}{1.037928in}}%
\pgfpathlineto{\pgfqpoint{1.711473in}{1.139199in}}%
\pgfpathlineto{\pgfqpoint{1.713132in}{0.848842in}}%
\pgfpathlineto{\pgfqpoint{1.714687in}{1.132705in}}%
\pgfpathlineto{\pgfqpoint{1.716657in}{0.952498in}}%
\pgfpathlineto{\pgfqpoint{1.717798in}{1.236323in}}%
\pgfpathlineto{\pgfqpoint{1.720183in}{0.910421in}}%
\pgfpathlineto{\pgfqpoint{1.720494in}{1.119206in}}%
\pgfpathlineto{\pgfqpoint{1.723605in}{0.865201in}}%
\pgfpathlineto{\pgfqpoint{1.724745in}{1.201690in}}%
\pgfpathlineto{\pgfqpoint{1.726612in}{0.896753in}}%
\pgfpathlineto{\pgfqpoint{1.727026in}{1.074704in}}%
\pgfpathlineto{\pgfqpoint{1.728685in}{0.651980in}}%
\pgfpathlineto{\pgfqpoint{1.731900in}{0.807961in}}%
\pgfpathlineto{\pgfqpoint{1.732729in}{1.254990in}}%
\pgfpathlineto{\pgfqpoint{1.734181in}{0.957905in}}%
\pgfpathlineto{\pgfqpoint{1.735840in}{1.079561in}}%
\pgfpathlineto{\pgfqpoint{1.737707in}{0.901530in}}%
\pgfpathlineto{\pgfqpoint{1.738951in}{1.080599in}}%
\pgfpathlineto{\pgfqpoint{1.741336in}{0.844511in}}%
\pgfpathlineto{\pgfqpoint{1.743099in}{1.108783in}}%
\pgfpathlineto{\pgfqpoint{1.743825in}{0.881505in}}%
\pgfpathlineto{\pgfqpoint{1.745276in}{1.018197in}}%
\pgfpathlineto{\pgfqpoint{1.747039in}{0.795431in}}%
\pgfpathlineto{\pgfqpoint{1.749735in}{0.714817in}}%
\pgfpathlineto{\pgfqpoint{1.750876in}{1.034553in}}%
\pgfpathlineto{\pgfqpoint{1.752016in}{0.719619in}}%
\pgfpathlineto{\pgfqpoint{1.754712in}{1.065436in}}%
\pgfpathlineto{\pgfqpoint{1.755542in}{0.882933in}}%
\pgfpathlineto{\pgfqpoint{1.758030in}{0.866653in}}%
\pgfpathlineto{\pgfqpoint{1.758756in}{1.100569in}}%
\pgfpathlineto{\pgfqpoint{1.760415in}{0.853057in}}%
\pgfpathlineto{\pgfqpoint{1.763215in}{1.071465in}}%
\pgfpathlineto{\pgfqpoint{1.764252in}{0.811350in}}%
\pgfpathlineto{\pgfqpoint{1.765185in}{1.013243in}}%
\pgfpathlineto{\pgfqpoint{1.767570in}{0.833595in}}%
\pgfpathlineto{\pgfqpoint{1.769126in}{1.108651in}}%
\pgfpathlineto{\pgfqpoint{1.770681in}{0.839715in}}%
\pgfpathlineto{\pgfqpoint{1.773170in}{1.151873in}}%
\pgfpathlineto{\pgfqpoint{1.774829in}{0.800241in}}%
\pgfpathlineto{\pgfqpoint{1.775658in}{1.032662in}}%
\pgfpathlineto{\pgfqpoint{1.777110in}{0.896296in}}%
\pgfpathlineto{\pgfqpoint{1.778562in}{1.160904in}}%
\pgfpathlineto{\pgfqpoint{1.780635in}{0.937662in}}%
\pgfpathlineto{\pgfqpoint{1.782087in}{1.144866in}}%
\pgfpathlineto{\pgfqpoint{1.783435in}{0.863479in}}%
\pgfpathlineto{\pgfqpoint{1.785198in}{1.228049in}}%
\pgfpathlineto{\pgfqpoint{1.788620in}{0.785098in}}%
\pgfpathlineto{\pgfqpoint{1.791212in}{1.183239in}}%
\pgfpathlineto{\pgfqpoint{1.792042in}{0.982499in}}%
\pgfpathlineto{\pgfqpoint{1.793493in}{1.195617in}}%
\pgfpathlineto{\pgfqpoint{1.795049in}{0.888704in}}%
\pgfpathlineto{\pgfqpoint{1.796811in}{1.042313in}}%
\pgfpathlineto{\pgfqpoint{1.799818in}{1.014155in}}%
\pgfpathlineto{\pgfqpoint{1.800337in}{0.708407in}}%
\pgfpathlineto{\pgfqpoint{1.803137in}{0.733539in}}%
\pgfpathlineto{\pgfqpoint{1.804277in}{1.105454in}}%
\pgfpathlineto{\pgfqpoint{1.806558in}{0.762368in}}%
\pgfpathlineto{\pgfqpoint{1.807388in}{1.028495in}}%
\pgfpathlineto{\pgfqpoint{1.809047in}{1.145663in}}%
\pgfpathlineto{\pgfqpoint{1.811432in}{0.830877in}}%
\pgfpathlineto{\pgfqpoint{1.812054in}{1.126796in}}%
\pgfpathlineto{\pgfqpoint{1.814439in}{0.919937in}}%
\pgfpathlineto{\pgfqpoint{1.815269in}{1.101212in}}%
\pgfpathlineto{\pgfqpoint{1.818068in}{1.205812in}}%
\pgfpathlineto{\pgfqpoint{1.819209in}{0.896135in}}%
\pgfpathlineto{\pgfqpoint{1.821490in}{1.230531in}}%
\pgfpathlineto{\pgfqpoint{1.822838in}{0.980019in}}%
\pgfpathlineto{\pgfqpoint{1.823875in}{1.277131in}}%
\pgfpathlineto{\pgfqpoint{1.825638in}{0.942428in}}%
\pgfpathlineto{\pgfqpoint{1.827919in}{1.189166in}}%
\pgfpathlineto{\pgfqpoint{1.828645in}{0.919471in}}%
\pgfpathlineto{\pgfqpoint{1.829993in}{1.196019in}}%
\pgfpathlineto{\pgfqpoint{1.831859in}{1.025352in}}%
\pgfpathlineto{\pgfqpoint{1.834659in}{1.189385in}}%
\pgfpathlineto{\pgfqpoint{1.836007in}{0.898593in}}%
\pgfpathlineto{\pgfqpoint{1.836526in}{1.055362in}}%
\pgfpathlineto{\pgfqpoint{1.839740in}{0.898989in}}%
\pgfpathlineto{\pgfqpoint{1.840777in}{1.125278in}}%
\pgfpathlineto{\pgfqpoint{1.842643in}{0.906117in}}%
\pgfpathlineto{\pgfqpoint{1.843266in}{1.107831in}}%
\pgfpathlineto{\pgfqpoint{1.844821in}{0.878326in}}%
\pgfpathlineto{\pgfqpoint{1.846584in}{1.063431in}}%
\pgfpathlineto{\pgfqpoint{1.849695in}{0.805003in}}%
\pgfpathlineto{\pgfqpoint{1.850006in}{1.067465in}}%
\pgfpathlineto{\pgfqpoint{1.851976in}{0.933101in}}%
\pgfpathlineto{\pgfqpoint{1.854050in}{1.238760in}}%
\pgfpathlineto{\pgfqpoint{1.854983in}{0.781777in}}%
\pgfpathlineto{\pgfqpoint{1.856435in}{1.050297in}}%
\pgfpathlineto{\pgfqpoint{1.858094in}{0.801408in}}%
\pgfpathlineto{\pgfqpoint{1.859856in}{0.997265in}}%
\pgfpathlineto{\pgfqpoint{1.862656in}{1.151151in}}%
\pgfpathlineto{\pgfqpoint{1.863693in}{0.838968in}}%
\pgfpathlineto{\pgfqpoint{1.865248in}{1.177381in}}%
\pgfpathlineto{\pgfqpoint{1.867530in}{0.977963in}}%
\pgfpathlineto{\pgfqpoint{1.868152in}{1.201177in}}%
\pgfpathlineto{\pgfqpoint{1.870329in}{0.983358in}}%
\pgfpathlineto{\pgfqpoint{1.871366in}{1.145367in}}%
\pgfpathlineto{\pgfqpoint{1.873336in}{1.233924in}}%
\pgfpathlineto{\pgfqpoint{1.875410in}{0.940733in}}%
\pgfpathlineto{\pgfqpoint{1.877484in}{1.258053in}}%
\pgfpathlineto{\pgfqpoint{1.878314in}{0.854991in}}%
\pgfpathlineto{\pgfqpoint{1.880699in}{1.158342in}}%
\pgfpathlineto{\pgfqpoint{1.881839in}{0.925710in}}%
\pgfpathlineto{\pgfqpoint{1.885261in}{1.355615in}}%
\pgfpathlineto{\pgfqpoint{1.887853in}{0.905773in}}%
\pgfpathlineto{\pgfqpoint{1.888890in}{1.155224in}}%
\pgfpathlineto{\pgfqpoint{1.890031in}{0.942803in}}%
\pgfpathlineto{\pgfqpoint{1.892001in}{1.301810in}}%
\pgfpathlineto{\pgfqpoint{1.893764in}{0.955829in}}%
\pgfpathlineto{\pgfqpoint{1.895008in}{1.089037in}}%
\pgfpathlineto{\pgfqpoint{1.896356in}{0.841145in}}%
\pgfpathlineto{\pgfqpoint{1.899363in}{1.204781in}}%
\pgfpathlineto{\pgfqpoint{1.899985in}{0.950405in}}%
\pgfpathlineto{\pgfqpoint{1.902785in}{1.153772in}}%
\pgfpathlineto{\pgfqpoint{1.903615in}{0.924564in}}%
\pgfpathlineto{\pgfqpoint{1.904548in}{1.151267in}}%
\pgfpathlineto{\pgfqpoint{1.907555in}{0.877717in}}%
\pgfpathlineto{\pgfqpoint{1.908592in}{1.171384in}}%
\pgfpathlineto{\pgfqpoint{1.909525in}{0.968588in}}%
\pgfpathlineto{\pgfqpoint{1.912532in}{0.905132in}}%
\pgfpathlineto{\pgfqpoint{1.913465in}{1.163475in}}%
\pgfpathlineto{\pgfqpoint{1.915332in}{0.939731in}}%
\pgfpathlineto{\pgfqpoint{1.916887in}{1.031132in}}%
\pgfpathlineto{\pgfqpoint{1.917821in}{0.831071in}}%
\pgfpathlineto{\pgfqpoint{1.921035in}{0.859777in}}%
\pgfpathlineto{\pgfqpoint{1.921346in}{1.123950in}}%
\pgfpathlineto{\pgfqpoint{1.922901in}{0.935203in}}%
\pgfpathlineto{\pgfqpoint{1.924561in}{0.893948in}}%
\pgfpathlineto{\pgfqpoint{1.926842in}{1.188653in}}%
\pgfpathlineto{\pgfqpoint{1.929227in}{0.982009in}}%
\pgfpathlineto{\pgfqpoint{1.930575in}{1.142264in}}%
\pgfpathlineto{\pgfqpoint{1.931093in}{0.991096in}}%
\pgfpathlineto{\pgfqpoint{1.933271in}{1.207275in}}%
\pgfpathlineto{\pgfqpoint{1.934722in}{0.971345in}}%
\pgfpathlineto{\pgfqpoint{1.937211in}{1.238728in}}%
\pgfpathlineto{\pgfqpoint{1.938041in}{0.976795in}}%
\pgfpathlineto{\pgfqpoint{1.940011in}{1.206129in}}%
\pgfpathlineto{\pgfqpoint{1.941462in}{0.956343in}}%
\pgfpathlineto{\pgfqpoint{1.942810in}{1.215920in}}%
\pgfpathlineto{\pgfqpoint{1.944677in}{1.029014in}}%
\pgfpathlineto{\pgfqpoint{1.947580in}{1.219725in}}%
\pgfpathlineto{\pgfqpoint{1.948513in}{0.910845in}}%
\pgfpathlineto{\pgfqpoint{1.949861in}{1.082407in}}%
\pgfpathlineto{\pgfqpoint{1.951209in}{0.898748in}}%
\pgfpathlineto{\pgfqpoint{1.953387in}{1.073044in}}%
\pgfpathlineto{\pgfqpoint{1.955565in}{0.799235in}}%
\pgfpathlineto{\pgfqpoint{1.955979in}{1.022202in}}%
\pgfpathlineto{\pgfqpoint{1.958572in}{1.137666in}}%
\pgfpathlineto{\pgfqpoint{1.959297in}{0.973508in}}%
\pgfpathlineto{\pgfqpoint{1.961060in}{1.143334in}}%
\pgfpathlineto{\pgfqpoint{1.962823in}{0.846178in}}%
\pgfpathlineto{\pgfqpoint{1.965208in}{0.790995in}}%
\pgfpathlineto{\pgfqpoint{1.965934in}{1.102087in}}%
\pgfpathlineto{\pgfqpoint{1.968111in}{0.996918in}}%
\pgfpathlineto{\pgfqpoint{1.970393in}{1.377470in}}%
\pgfpathlineto{\pgfqpoint{1.971326in}{1.149471in}}%
\pgfpathlineto{\pgfqpoint{1.973503in}{1.302681in}}%
\pgfpathlineto{\pgfqpoint{1.974540in}{0.932589in}}%
\pgfpathlineto{\pgfqpoint{1.975992in}{1.160916in}}%
\pgfpathlineto{\pgfqpoint{1.977755in}{1.044077in}}%
\pgfpathlineto{\pgfqpoint{1.979829in}{1.192989in}}%
\pgfpathlineto{\pgfqpoint{1.980866in}{0.982022in}}%
\pgfpathlineto{\pgfqpoint{1.982732in}{1.271757in}}%
\pgfpathlineto{\pgfqpoint{1.984806in}{1.067841in}}%
\pgfpathlineto{\pgfqpoint{1.986880in}{1.461395in}}%
\pgfpathlineto{\pgfqpoint{1.988746in}{0.985712in}}%
\pgfpathlineto{\pgfqpoint{1.989265in}{1.244040in}}%
\pgfpathlineto{\pgfqpoint{1.992375in}{1.087452in}}%
\pgfpathlineto{\pgfqpoint{1.994034in}{1.321637in}}%
\pgfpathlineto{\pgfqpoint{1.995382in}{1.012007in}}%
\pgfpathlineto{\pgfqpoint{1.996316in}{1.300794in}}%
\pgfpathlineto{\pgfqpoint{1.998286in}{1.327642in}}%
\pgfpathlineto{\pgfqpoint{1.999634in}{0.975836in}}%
\pgfpathlineto{\pgfqpoint{2.001604in}{0.917943in}}%
\pgfpathlineto{\pgfqpoint{2.003263in}{1.212115in}}%
\pgfpathlineto{\pgfqpoint{2.005648in}{0.952691in}}%
\pgfpathlineto{\pgfqpoint{2.006581in}{1.175991in}}%
\pgfpathlineto{\pgfqpoint{2.008759in}{0.899519in}}%
\pgfpathlineto{\pgfqpoint{2.009588in}{1.308855in}}%
\pgfpathlineto{\pgfqpoint{2.010833in}{1.042996in}}%
\pgfpathlineto{\pgfqpoint{2.013321in}{1.218926in}}%
\pgfpathlineto{\pgfqpoint{2.014980in}{1.038262in}}%
\pgfpathlineto{\pgfqpoint{2.015914in}{1.273445in}}%
\pgfpathlineto{\pgfqpoint{2.018091in}{1.051547in}}%
\pgfpathlineto{\pgfqpoint{2.019024in}{1.324291in}}%
\pgfpathlineto{\pgfqpoint{2.023898in}{0.806048in}}%
\pgfpathlineto{\pgfqpoint{2.024209in}{1.065132in}}%
\pgfpathlineto{\pgfqpoint{2.026490in}{1.278014in}}%
\pgfpathlineto{\pgfqpoint{2.027631in}{0.985032in}}%
\pgfpathlineto{\pgfqpoint{2.030223in}{0.925601in}}%
\pgfpathlineto{\pgfqpoint{2.032193in}{1.295608in}}%
\pgfpathlineto{\pgfqpoint{2.032297in}{1.080997in}}%
\pgfpathlineto{\pgfqpoint{2.035097in}{1.411855in}}%
\pgfpathlineto{\pgfqpoint{2.037689in}{0.989596in}}%
\pgfpathlineto{\pgfqpoint{2.039970in}{1.323112in}}%
\pgfpathlineto{\pgfqpoint{2.040903in}{0.939445in}}%
\pgfpathlineto{\pgfqpoint{2.042355in}{1.271595in}}%
\pgfpathlineto{\pgfqpoint{2.045051in}{0.952610in}}%
\pgfpathlineto{\pgfqpoint{2.046295in}{1.202301in}}%
\pgfpathlineto{\pgfqpoint{2.047229in}{1.014722in}}%
\pgfpathlineto{\pgfqpoint{2.049199in}{1.332641in}}%
\pgfpathlineto{\pgfqpoint{2.050547in}{1.244661in}}%
\pgfpathlineto{\pgfqpoint{2.053243in}{1.358497in}}%
\pgfpathlineto{\pgfqpoint{2.054383in}{1.114145in}}%
\pgfpathlineto{\pgfqpoint{2.056457in}{1.328996in}}%
\pgfpathlineto{\pgfqpoint{2.057391in}{0.984309in}}%
\pgfpathlineto{\pgfqpoint{2.059361in}{1.358768in}}%
\pgfpathlineto{\pgfqpoint{2.061227in}{0.905938in}}%
\pgfpathlineto{\pgfqpoint{2.062368in}{1.367171in}}%
\pgfpathlineto{\pgfqpoint{2.064027in}{1.139823in}}%
\pgfpathlineto{\pgfqpoint{2.066515in}{1.037709in}}%
\pgfpathlineto{\pgfqpoint{2.067656in}{1.385826in}}%
\pgfpathlineto{\pgfqpoint{2.069730in}{1.042915in}}%
\pgfpathlineto{\pgfqpoint{2.070559in}{1.305054in}}%
\pgfpathlineto{\pgfqpoint{2.072737in}{1.022273in}}%
\pgfpathlineto{\pgfqpoint{2.074915in}{1.238075in}}%
\pgfpathlineto{\pgfqpoint{2.076781in}{0.996742in}}%
\pgfpathlineto{\pgfqpoint{2.078233in}{1.264007in}}%
\pgfpathlineto{\pgfqpoint{2.079062in}{0.910637in}}%
\pgfpathlineto{\pgfqpoint{2.080929in}{1.166798in}}%
\pgfpathlineto{\pgfqpoint{2.082277in}{0.890606in}}%
\pgfpathlineto{\pgfqpoint{2.085180in}{1.355271in}}%
\pgfpathlineto{\pgfqpoint{2.085699in}{1.032270in}}%
\pgfpathlineto{\pgfqpoint{2.088187in}{0.995324in}}%
\pgfpathlineto{\pgfqpoint{2.090676in}{1.355956in}}%
\pgfpathlineto{\pgfqpoint{2.092335in}{1.080815in}}%
\pgfpathlineto{\pgfqpoint{2.093994in}{1.385932in}}%
\pgfpathlineto{\pgfqpoint{2.095549in}{1.091997in}}%
\pgfpathlineto{\pgfqpoint{2.098245in}{1.364197in}}%
\pgfpathlineto{\pgfqpoint{2.098971in}{1.095019in}}%
\pgfpathlineto{\pgfqpoint{2.100319in}{1.296696in}}%
\pgfpathlineto{\pgfqpoint{2.102289in}{1.020153in}}%
\pgfpathlineto{\pgfqpoint{2.104052in}{1.279738in}}%
\pgfpathlineto{\pgfqpoint{2.105711in}{1.041255in}}%
\pgfpathlineto{\pgfqpoint{2.107059in}{1.361226in}}%
\pgfpathlineto{\pgfqpoint{2.110377in}{0.992789in}}%
\pgfpathlineto{\pgfqpoint{2.112348in}{1.277940in}}%
\pgfpathlineto{\pgfqpoint{2.114421in}{1.018593in}}%
\pgfpathlineto{\pgfqpoint{2.115562in}{1.275429in}}%
\pgfpathlineto{\pgfqpoint{2.117947in}{0.964890in}}%
\pgfpathlineto{\pgfqpoint{2.119088in}{1.224243in}}%
\pgfpathlineto{\pgfqpoint{2.120332in}{1.032934in}}%
\pgfpathlineto{\pgfqpoint{2.121991in}{1.336374in}}%
\pgfpathlineto{\pgfqpoint{2.123754in}{1.106061in}}%
\pgfpathlineto{\pgfqpoint{2.125516in}{1.209530in}}%
\pgfpathlineto{\pgfqpoint{2.127487in}{0.833034in}}%
\pgfpathlineto{\pgfqpoint{2.128938in}{0.856774in}}%
\pgfpathlineto{\pgfqpoint{2.131634in}{1.398621in}}%
\pgfpathlineto{\pgfqpoint{2.131945in}{1.089106in}}%
\pgfpathlineto{\pgfqpoint{2.135056in}{0.868537in}}%
\pgfpathlineto{\pgfqpoint{2.136404in}{1.314833in}}%
\pgfpathlineto{\pgfqpoint{2.137026in}{1.114345in}}%
\pgfpathlineto{\pgfqpoint{2.138582in}{1.228009in}}%
\pgfpathlineto{\pgfqpoint{2.141381in}{1.008544in}}%
\pgfpathlineto{\pgfqpoint{2.142418in}{1.277447in}}%
\pgfpathlineto{\pgfqpoint{2.143870in}{1.028340in}}%
\pgfpathlineto{\pgfqpoint{2.146255in}{0.978576in}}%
\pgfpathlineto{\pgfqpoint{2.146773in}{1.268955in}}%
\pgfpathlineto{\pgfqpoint{2.149469in}{1.045291in}}%
\pgfpathlineto{\pgfqpoint{2.151543in}{0.993040in}}%
\pgfpathlineto{\pgfqpoint{2.153099in}{1.277622in}}%
\pgfpathlineto{\pgfqpoint{2.153721in}{1.091272in}}%
\pgfpathlineto{\pgfqpoint{2.155173in}{1.274594in}}%
\pgfpathlineto{\pgfqpoint{2.156728in}{1.075790in}}%
\pgfpathlineto{\pgfqpoint{2.159009in}{1.342929in}}%
\pgfpathlineto{\pgfqpoint{2.160979in}{1.021135in}}%
\pgfpathlineto{\pgfqpoint{2.161809in}{1.299920in}}%
\pgfpathlineto{\pgfqpoint{2.163468in}{1.073667in}}%
\pgfpathlineto{\pgfqpoint{2.165749in}{1.303322in}}%
\pgfpathlineto{\pgfqpoint{2.167512in}{1.366945in}}%
\pgfpathlineto{\pgfqpoint{2.169067in}{1.079986in}}%
\pgfpathlineto{\pgfqpoint{2.170001in}{1.337970in}}%
\pgfpathlineto{\pgfqpoint{2.171763in}{1.142813in}}%
\pgfpathlineto{\pgfqpoint{2.174356in}{1.306037in}}%
\pgfpathlineto{\pgfqpoint{2.175289in}{1.013840in}}%
\pgfpathlineto{\pgfqpoint{2.177881in}{1.247096in}}%
\pgfpathlineto{\pgfqpoint{2.179851in}{0.934205in}}%
\pgfpathlineto{\pgfqpoint{2.180785in}{1.161046in}}%
\pgfpathlineto{\pgfqpoint{2.182340in}{1.263403in}}%
\pgfpathlineto{\pgfqpoint{2.184518in}{1.016749in}}%
\pgfpathlineto{\pgfqpoint{2.185347in}{1.240711in}}%
\pgfpathlineto{\pgfqpoint{2.187732in}{0.820743in}}%
\pgfpathlineto{\pgfqpoint{2.188976in}{1.193903in}}%
\pgfpathlineto{\pgfqpoint{2.190739in}{0.924805in}}%
\pgfpathlineto{\pgfqpoint{2.192502in}{1.360206in}}%
\pgfpathlineto{\pgfqpoint{2.194057in}{1.153124in}}%
\pgfpathlineto{\pgfqpoint{2.195716in}{1.396054in}}%
\pgfpathlineto{\pgfqpoint{2.197790in}{1.111256in}}%
\pgfpathlineto{\pgfqpoint{2.198723in}{1.364664in}}%
\pgfpathlineto{\pgfqpoint{2.200486in}{1.040049in}}%
\pgfpathlineto{\pgfqpoint{2.202871in}{0.923231in}}%
\pgfpathlineto{\pgfqpoint{2.204323in}{1.278832in}}%
\pgfpathlineto{\pgfqpoint{2.205774in}{0.898522in}}%
\pgfpathlineto{\pgfqpoint{2.206500in}{1.172290in}}%
\pgfpathlineto{\pgfqpoint{2.208263in}{0.873948in}}%
\pgfpathlineto{\pgfqpoint{2.210026in}{1.113105in}}%
\pgfpathlineto{\pgfqpoint{2.212618in}{0.989012in}}%
\pgfpathlineto{\pgfqpoint{2.213551in}{1.164036in}}%
\pgfpathlineto{\pgfqpoint{2.216040in}{1.011099in}}%
\pgfpathlineto{\pgfqpoint{2.216455in}{1.222193in}}%
\pgfpathlineto{\pgfqpoint{2.218114in}{1.322157in}}%
\pgfpathlineto{\pgfqpoint{2.221328in}{1.024584in}}%
\pgfpathlineto{\pgfqpoint{2.221432in}{1.272578in}}%
\pgfpathlineto{\pgfqpoint{2.223921in}{1.364394in}}%
\pgfpathlineto{\pgfqpoint{2.225061in}{1.062634in}}%
\pgfpathlineto{\pgfqpoint{2.226513in}{1.310992in}}%
\pgfpathlineto{\pgfqpoint{2.229002in}{0.816384in}}%
\pgfpathlineto{\pgfqpoint{2.229727in}{1.089761in}}%
\pgfpathlineto{\pgfqpoint{2.232009in}{0.914069in}}%
\pgfpathlineto{\pgfqpoint{2.233460in}{1.174430in}}%
\pgfpathlineto{\pgfqpoint{2.235742in}{1.274460in}}%
\pgfpathlineto{\pgfqpoint{2.236467in}{0.961370in}}%
\pgfpathlineto{\pgfqpoint{2.238645in}{1.103428in}}%
\pgfpathlineto{\pgfqpoint{2.241237in}{0.943501in}}%
\pgfpathlineto{\pgfqpoint{2.241756in}{1.199971in}}%
\pgfpathlineto{\pgfqpoint{2.244452in}{0.905772in}}%
\pgfpathlineto{\pgfqpoint{2.245696in}{1.123682in}}%
\pgfpathlineto{\pgfqpoint{2.246629in}{0.794525in}}%
\pgfpathlineto{\pgfqpoint{2.248496in}{1.154406in}}%
\pgfpathlineto{\pgfqpoint{2.250155in}{1.223502in}}%
\pgfpathlineto{\pgfqpoint{2.253058in}{0.856964in}}%
\pgfpathlineto{\pgfqpoint{2.255547in}{1.174620in}}%
\pgfpathlineto{\pgfqpoint{2.256584in}{0.980665in}}%
\pgfpathlineto{\pgfqpoint{2.258347in}{1.367862in}}%
\pgfpathlineto{\pgfqpoint{2.259695in}{1.111768in}}%
\pgfpathlineto{\pgfqpoint{2.262391in}{1.064706in}}%
\pgfpathlineto{\pgfqpoint{2.263116in}{1.321804in}}%
\pgfpathlineto{\pgfqpoint{2.264775in}{1.097642in}}%
\pgfpathlineto{\pgfqpoint{2.266746in}{1.245190in}}%
\pgfpathlineto{\pgfqpoint{2.268301in}{1.021753in}}%
\pgfpathlineto{\pgfqpoint{2.271827in}{1.380352in}}%
\pgfpathlineto{\pgfqpoint{2.273589in}{1.053112in}}%
\pgfpathlineto{\pgfqpoint{2.275145in}{1.290664in}}%
\pgfpathlineto{\pgfqpoint{2.276389in}{1.118361in}}%
\pgfpathlineto{\pgfqpoint{2.278255in}{1.293946in}}%
\pgfpathlineto{\pgfqpoint{2.279707in}{1.085986in}}%
\pgfpathlineto{\pgfqpoint{2.281470in}{1.255170in}}%
\pgfpathlineto{\pgfqpoint{2.283855in}{0.987623in}}%
\pgfpathlineto{\pgfqpoint{2.284684in}{1.303874in}}%
\pgfpathlineto{\pgfqpoint{2.286551in}{1.398031in}}%
\pgfpathlineto{\pgfqpoint{2.288417in}{1.101503in}}%
\pgfpathlineto{\pgfqpoint{2.289869in}{1.329294in}}%
\pgfpathlineto{\pgfqpoint{2.291736in}{0.977359in}}%
\pgfpathlineto{\pgfqpoint{2.293498in}{1.216424in}}%
\pgfpathlineto{\pgfqpoint{2.294535in}{1.032700in}}%
\pgfpathlineto{\pgfqpoint{2.296713in}{1.396095in}}%
\pgfpathlineto{\pgfqpoint{2.298994in}{1.082464in}}%
\pgfpathlineto{\pgfqpoint{2.299512in}{1.361641in}}%
\pgfpathlineto{\pgfqpoint{2.301275in}{0.964996in}}%
\pgfpathlineto{\pgfqpoint{2.303556in}{1.278544in}}%
\pgfpathlineto{\pgfqpoint{2.304386in}{1.105328in}}%
\pgfpathlineto{\pgfqpoint{2.306978in}{0.927455in}}%
\pgfpathlineto{\pgfqpoint{2.307808in}{1.200502in}}%
\pgfpathlineto{\pgfqpoint{2.310815in}{1.189224in}}%
\pgfpathlineto{\pgfqpoint{2.312267in}{0.857182in}}%
\pgfpathlineto{\pgfqpoint{2.313615in}{1.173604in}}%
\pgfpathlineto{\pgfqpoint{2.314340in}{1.040327in}}%
\pgfpathlineto{\pgfqpoint{2.314340in}{1.040327in}}%
\pgfusepath{stroke}%
\end{pgfscope}%
\begin{pgfscope}%
\pgfsetrectcap%
\pgfsetmiterjoin%
\pgfsetlinewidth{0.803000pt}%
\definecolor{currentstroke}{rgb}{0.000000,0.000000,0.000000}%
\pgfsetstrokecolor{currentstroke}%
\pgfsetdash{}{0pt}%
\pgfpathmoveto{\pgfqpoint{0.530716in}{0.416447in}}%
\pgfpathlineto{\pgfqpoint{0.530716in}{1.750483in}}%
\pgfusepath{stroke}%
\end{pgfscope}%
\begin{pgfscope}%
\pgfsetrectcap%
\pgfsetmiterjoin%
\pgfsetlinewidth{0.803000pt}%
\definecolor{currentstroke}{rgb}{0.000000,0.000000,0.000000}%
\pgfsetstrokecolor{currentstroke}%
\pgfsetdash{}{0pt}%
\pgfpathmoveto{\pgfqpoint{2.399275in}{0.416447in}}%
\pgfpathlineto{\pgfqpoint{2.399275in}{1.750483in}}%
\pgfusepath{stroke}%
\end{pgfscope}%
\begin{pgfscope}%
\pgfsetrectcap%
\pgfsetmiterjoin%
\pgfsetlinewidth{0.803000pt}%
\definecolor{currentstroke}{rgb}{0.000000,0.000000,0.000000}%
\pgfsetstrokecolor{currentstroke}%
\pgfsetdash{}{0pt}%
\pgfpathmoveto{\pgfqpoint{0.530716in}{0.416447in}}%
\pgfpathlineto{\pgfqpoint{2.399275in}{0.416447in}}%
\pgfusepath{stroke}%
\end{pgfscope}%
\begin{pgfscope}%
\pgfsetrectcap%
\pgfsetmiterjoin%
\pgfsetlinewidth{0.803000pt}%
\definecolor{currentstroke}{rgb}{0.000000,0.000000,0.000000}%
\pgfsetstrokecolor{currentstroke}%
\pgfsetdash{}{0pt}%
\pgfpathmoveto{\pgfqpoint{0.530716in}{1.750483in}}%
\pgfpathlineto{\pgfqpoint{2.399275in}{1.750483in}}%
\pgfusepath{stroke}%
\end{pgfscope}%
\begin{pgfscope}%
\pgfsetbuttcap%
\pgfsetmiterjoin%
\definecolor{currentfill}{rgb}{1.000000,1.000000,1.000000}%
\pgfsetfillcolor{currentfill}%
\pgfsetfillopacity{0.800000}%
\pgfsetlinewidth{1.003750pt}%
\definecolor{currentstroke}{rgb}{0.800000,0.800000,0.800000}%
\pgfsetstrokecolor{currentstroke}%
\pgfsetstrokeopacity{0.800000}%
\pgfsetdash{}{0pt}%
\pgfpathmoveto{\pgfqpoint{0.608494in}{1.506706in}}%
\pgfpathlineto{\pgfqpoint{1.608827in}{1.506706in}}%
\pgfpathquadraticcurveto{\pgfqpoint{1.631049in}{1.506706in}}{\pgfqpoint{1.631049in}{1.528928in}}%
\pgfpathlineto{\pgfqpoint{1.631049in}{1.672705in}}%
\pgfpathquadraticcurveto{\pgfqpoint{1.631049in}{1.694928in}}{\pgfqpoint{1.608827in}{1.694928in}}%
\pgfpathlineto{\pgfqpoint{0.608494in}{1.694928in}}%
\pgfpathquadraticcurveto{\pgfqpoint{0.586272in}{1.694928in}}{\pgfqpoint{0.586272in}{1.672705in}}%
\pgfpathlineto{\pgfqpoint{0.586272in}{1.528928in}}%
\pgfpathquadraticcurveto{\pgfqpoint{0.586272in}{1.506706in}}{\pgfqpoint{0.608494in}{1.506706in}}%
\pgfpathlineto{\pgfqpoint{0.608494in}{1.506706in}}%
\pgfpathclose%
\pgfusepath{stroke,fill}%
\end{pgfscope}%
\begin{pgfscope}%
\pgfsetrectcap%
\pgfsetroundjoin%
\pgfsetlinewidth{1.505625pt}%
\definecolor{currentstroke}{rgb}{0.007843,0.619608,0.450980}%
\pgfsetstrokecolor{currentstroke}%
\pgfsetdash{}{0pt}%
\pgfpathmoveto{\pgfqpoint{0.630716in}{1.611594in}}%
\pgfpathlineto{\pgfqpoint{0.741827in}{1.611594in}}%
\pgfpathlineto{\pgfqpoint{0.852938in}{1.611594in}}%
\pgfusepath{stroke}%
\end{pgfscope}%
\begin{pgfscope}%
\definecolor{textcolor}{rgb}{0.000000,0.000000,0.000000}%
\pgfsetstrokecolor{textcolor}%
\pgfsetfillcolor{textcolor}%
\pgftext[x=0.941827in,y=1.572705in,left,base]{\color{textcolor}\rmfamily\fontsize{8.000000}{9.600000}\selectfont Flicker noise}%
\end{pgfscope}%
\end{pgfpicture}%
\makeatother%
\endgroup%
% data/simulations/sim_allan_variance.py
        } % scalebox
        \caption{Time domain}
        \label{fig:flicker_noise_time}
    \end{subfigure}
    \begin{subfigure}{0.32\linewidth}
        \centering
        \scalebox{0.75}{%
            %% Creator: Matplotlib, PGF backend
%%
%% To include the figure in your LaTeX document, write
%%   \input{<filename>.pgf}
%%
%% Make sure the required packages are loaded in your preamble
%%   \usepackage{pgf}
%%
%% Also ensure that all the required font packages are loaded; for instance,
%% the lmodern package is sometimes necessary when using math font.
%%   \usepackage{lmodern}
%%
%% Figures using additional raster images can only be included by \input if
%% they are in the same directory as the main LaTeX file. For loading figures
%% from other directories you can use the `import` package
%%   \usepackage{import}
%%
%% and then include the figures with
%%   \import{<path to file>}{<filename>.pgf}
%%
%% Matplotlib used the following preamble
%%   \usepackage{siunitx}
%%   \usepackage{fontspec}
%%   \makeatletter\@ifpackageloaded{underscore}{}{\usepackage[strings]{underscore}}\makeatother
%%
\begingroup%
\makeatletter%
\begin{pgfpicture}%
\pgfpathrectangle{\pgfpointorigin}{\pgfqpoint{2.440945in}{1.830709in}}%
\pgfusepath{use as bounding box, clip}%
\begin{pgfscope}%
\pgfsetbuttcap%
\pgfsetmiterjoin%
\definecolor{currentfill}{rgb}{1.000000,1.000000,1.000000}%
\pgfsetfillcolor{currentfill}%
\pgfsetlinewidth{0.000000pt}%
\definecolor{currentstroke}{rgb}{1.000000,1.000000,1.000000}%
\pgfsetstrokecolor{currentstroke}%
\pgfsetdash{}{0pt}%
\pgfpathmoveto{\pgfqpoint{0.000000in}{0.000000in}}%
\pgfpathlineto{\pgfqpoint{2.440945in}{0.000000in}}%
\pgfpathlineto{\pgfqpoint{2.440945in}{1.830709in}}%
\pgfpathlineto{\pgfqpoint{0.000000in}{1.830709in}}%
\pgfpathlineto{\pgfqpoint{0.000000in}{0.000000in}}%
\pgfpathclose%
\pgfusepath{fill}%
\end{pgfscope}%
\begin{pgfscope}%
\pgfsetbuttcap%
\pgfsetmiterjoin%
\definecolor{currentfill}{rgb}{1.000000,1.000000,1.000000}%
\pgfsetfillcolor{currentfill}%
\pgfsetlinewidth{0.000000pt}%
\definecolor{currentstroke}{rgb}{0.000000,0.000000,0.000000}%
\pgfsetstrokecolor{currentstroke}%
\pgfsetstrokeopacity{0.000000}%
\pgfsetdash{}{0pt}%
\pgfpathmoveto{\pgfqpoint{0.514278in}{0.417642in}}%
\pgfpathlineto{\pgfqpoint{2.399275in}{0.417642in}}%
\pgfpathlineto{\pgfqpoint{2.399275in}{1.789039in}}%
\pgfpathlineto{\pgfqpoint{0.514278in}{1.789039in}}%
\pgfpathlineto{\pgfqpoint{0.514278in}{0.417642in}}%
\pgfpathclose%
\pgfusepath{fill}%
\end{pgfscope}%
\begin{pgfscope}%
\pgfpathrectangle{\pgfqpoint{0.514278in}{0.417642in}}{\pgfqpoint{1.884996in}{1.371397in}}%
\pgfusepath{clip}%
\pgfsetrectcap%
\pgfsetroundjoin%
\pgfsetlinewidth{0.803000pt}%
\definecolor{currentstroke}{rgb}{0.450000,0.450000,0.450000}%
\pgfsetstrokecolor{currentstroke}%
\pgfsetdash{}{0pt}%
\pgfpathmoveto{\pgfqpoint{0.916836in}{0.417642in}}%
\pgfpathlineto{\pgfqpoint{0.916836in}{1.789039in}}%
\pgfusepath{stroke}%
\end{pgfscope}%
\begin{pgfscope}%
\pgfsetbuttcap%
\pgfsetroundjoin%
\definecolor{currentfill}{rgb}{0.000000,0.000000,0.000000}%
\pgfsetfillcolor{currentfill}%
\pgfsetlinewidth{0.803000pt}%
\definecolor{currentstroke}{rgb}{0.000000,0.000000,0.000000}%
\pgfsetstrokecolor{currentstroke}%
\pgfsetdash{}{0pt}%
\pgfsys@defobject{currentmarker}{\pgfqpoint{0.000000in}{-0.048611in}}{\pgfqpoint{0.000000in}{0.000000in}}{%
\pgfpathmoveto{\pgfqpoint{0.000000in}{0.000000in}}%
\pgfpathlineto{\pgfqpoint{0.000000in}{-0.048611in}}%
\pgfusepath{stroke,fill}%
}%
\begin{pgfscope}%
\pgfsys@transformshift{0.916836in}{0.417642in}%
\pgfsys@useobject{currentmarker}{}%
\end{pgfscope}%
\end{pgfscope}%
\begin{pgfscope}%
\definecolor{textcolor}{rgb}{0.000000,0.000000,0.000000}%
\pgfsetstrokecolor{textcolor}%
\pgfsetfillcolor{textcolor}%
\pgftext[x=0.916836in,y=0.320420in,,top]{\color{textcolor}\rmfamily\fontsize{8.000000}{9.600000}\selectfont \(\displaystyle {10^{-3}}\)}%
\end{pgfscope}%
\begin{pgfscope}%
\pgfpathrectangle{\pgfqpoint{0.514278in}{0.417642in}}{\pgfqpoint{1.884996in}{1.371397in}}%
\pgfusepath{clip}%
\pgfsetrectcap%
\pgfsetroundjoin%
\pgfsetlinewidth{0.803000pt}%
\definecolor{currentstroke}{rgb}{0.450000,0.450000,0.450000}%
\pgfsetstrokecolor{currentstroke}%
\pgfsetdash{}{0pt}%
\pgfpathmoveto{\pgfqpoint{1.434391in}{0.417642in}}%
\pgfpathlineto{\pgfqpoint{1.434391in}{1.789039in}}%
\pgfusepath{stroke}%
\end{pgfscope}%
\begin{pgfscope}%
\pgfsetbuttcap%
\pgfsetroundjoin%
\definecolor{currentfill}{rgb}{0.000000,0.000000,0.000000}%
\pgfsetfillcolor{currentfill}%
\pgfsetlinewidth{0.803000pt}%
\definecolor{currentstroke}{rgb}{0.000000,0.000000,0.000000}%
\pgfsetstrokecolor{currentstroke}%
\pgfsetdash{}{0pt}%
\pgfsys@defobject{currentmarker}{\pgfqpoint{0.000000in}{-0.048611in}}{\pgfqpoint{0.000000in}{0.000000in}}{%
\pgfpathmoveto{\pgfqpoint{0.000000in}{0.000000in}}%
\pgfpathlineto{\pgfqpoint{0.000000in}{-0.048611in}}%
\pgfusepath{stroke,fill}%
}%
\begin{pgfscope}%
\pgfsys@transformshift{1.434391in}{0.417642in}%
\pgfsys@useobject{currentmarker}{}%
\end{pgfscope}%
\end{pgfscope}%
\begin{pgfscope}%
\definecolor{textcolor}{rgb}{0.000000,0.000000,0.000000}%
\pgfsetstrokecolor{textcolor}%
\pgfsetfillcolor{textcolor}%
\pgftext[x=1.434391in,y=0.320420in,,top]{\color{textcolor}\rmfamily\fontsize{8.000000}{9.600000}\selectfont \(\displaystyle {10^{-2}}\)}%
\end{pgfscope}%
\begin{pgfscope}%
\pgfpathrectangle{\pgfqpoint{0.514278in}{0.417642in}}{\pgfqpoint{1.884996in}{1.371397in}}%
\pgfusepath{clip}%
\pgfsetrectcap%
\pgfsetroundjoin%
\pgfsetlinewidth{0.803000pt}%
\definecolor{currentstroke}{rgb}{0.450000,0.450000,0.450000}%
\pgfsetstrokecolor{currentstroke}%
\pgfsetdash{}{0pt}%
\pgfpathmoveto{\pgfqpoint{1.951947in}{0.417642in}}%
\pgfpathlineto{\pgfqpoint{1.951947in}{1.789039in}}%
\pgfusepath{stroke}%
\end{pgfscope}%
\begin{pgfscope}%
\pgfsetbuttcap%
\pgfsetroundjoin%
\definecolor{currentfill}{rgb}{0.000000,0.000000,0.000000}%
\pgfsetfillcolor{currentfill}%
\pgfsetlinewidth{0.803000pt}%
\definecolor{currentstroke}{rgb}{0.000000,0.000000,0.000000}%
\pgfsetstrokecolor{currentstroke}%
\pgfsetdash{}{0pt}%
\pgfsys@defobject{currentmarker}{\pgfqpoint{0.000000in}{-0.048611in}}{\pgfqpoint{0.000000in}{0.000000in}}{%
\pgfpathmoveto{\pgfqpoint{0.000000in}{0.000000in}}%
\pgfpathlineto{\pgfqpoint{0.000000in}{-0.048611in}}%
\pgfusepath{stroke,fill}%
}%
\begin{pgfscope}%
\pgfsys@transformshift{1.951947in}{0.417642in}%
\pgfsys@useobject{currentmarker}{}%
\end{pgfscope}%
\end{pgfscope}%
\begin{pgfscope}%
\definecolor{textcolor}{rgb}{0.000000,0.000000,0.000000}%
\pgfsetstrokecolor{textcolor}%
\pgfsetfillcolor{textcolor}%
\pgftext[x=1.951947in,y=0.320420in,,top]{\color{textcolor}\rmfamily\fontsize{8.000000}{9.600000}\selectfont \(\displaystyle {10^{-1}}\)}%
\end{pgfscope}%
\begin{pgfscope}%
\pgfpathrectangle{\pgfqpoint{0.514278in}{0.417642in}}{\pgfqpoint{1.884996in}{1.371397in}}%
\pgfusepath{clip}%
\pgfsetrectcap%
\pgfsetroundjoin%
\pgfsetlinewidth{0.803000pt}%
\definecolor{currentstroke}{rgb}{0.850000,0.850000,0.850000}%
\pgfsetstrokecolor{currentstroke}%
\pgfsetdash{}{0pt}%
\pgfpathmoveto{\pgfqpoint{0.555080in}{0.417642in}}%
\pgfpathlineto{\pgfqpoint{0.555080in}{1.789039in}}%
\pgfusepath{stroke}%
\end{pgfscope}%
\begin{pgfscope}%
\pgfsetbuttcap%
\pgfsetroundjoin%
\definecolor{currentfill}{rgb}{0.000000,0.000000,0.000000}%
\pgfsetfillcolor{currentfill}%
\pgfsetlinewidth{0.602250pt}%
\definecolor{currentstroke}{rgb}{0.000000,0.000000,0.000000}%
\pgfsetstrokecolor{currentstroke}%
\pgfsetdash{}{0pt}%
\pgfsys@defobject{currentmarker}{\pgfqpoint{0.000000in}{-0.027778in}}{\pgfqpoint{0.000000in}{0.000000in}}{%
\pgfpathmoveto{\pgfqpoint{0.000000in}{0.000000in}}%
\pgfpathlineto{\pgfqpoint{0.000000in}{-0.027778in}}%
\pgfusepath{stroke,fill}%
}%
\begin{pgfscope}%
\pgfsys@transformshift{0.555080in}{0.417642in}%
\pgfsys@useobject{currentmarker}{}%
\end{pgfscope}%
\end{pgfscope}%
\begin{pgfscope}%
\pgfpathrectangle{\pgfqpoint{0.514278in}{0.417642in}}{\pgfqpoint{1.884996in}{1.371397in}}%
\pgfusepath{clip}%
\pgfsetrectcap%
\pgfsetroundjoin%
\pgfsetlinewidth{0.803000pt}%
\definecolor{currentstroke}{rgb}{0.850000,0.850000,0.850000}%
\pgfsetstrokecolor{currentstroke}%
\pgfsetdash{}{0pt}%
\pgfpathmoveto{\pgfqpoint{0.646217in}{0.417642in}}%
\pgfpathlineto{\pgfqpoint{0.646217in}{1.789039in}}%
\pgfusepath{stroke}%
\end{pgfscope}%
\begin{pgfscope}%
\pgfsetbuttcap%
\pgfsetroundjoin%
\definecolor{currentfill}{rgb}{0.000000,0.000000,0.000000}%
\pgfsetfillcolor{currentfill}%
\pgfsetlinewidth{0.602250pt}%
\definecolor{currentstroke}{rgb}{0.000000,0.000000,0.000000}%
\pgfsetstrokecolor{currentstroke}%
\pgfsetdash{}{0pt}%
\pgfsys@defobject{currentmarker}{\pgfqpoint{0.000000in}{-0.027778in}}{\pgfqpoint{0.000000in}{0.000000in}}{%
\pgfpathmoveto{\pgfqpoint{0.000000in}{0.000000in}}%
\pgfpathlineto{\pgfqpoint{0.000000in}{-0.027778in}}%
\pgfusepath{stroke,fill}%
}%
\begin{pgfscope}%
\pgfsys@transformshift{0.646217in}{0.417642in}%
\pgfsys@useobject{currentmarker}{}%
\end{pgfscope}%
\end{pgfscope}%
\begin{pgfscope}%
\pgfpathrectangle{\pgfqpoint{0.514278in}{0.417642in}}{\pgfqpoint{1.884996in}{1.371397in}}%
\pgfusepath{clip}%
\pgfsetrectcap%
\pgfsetroundjoin%
\pgfsetlinewidth{0.803000pt}%
\definecolor{currentstroke}{rgb}{0.850000,0.850000,0.850000}%
\pgfsetstrokecolor{currentstroke}%
\pgfsetdash{}{0pt}%
\pgfpathmoveto{\pgfqpoint{0.710880in}{0.417642in}}%
\pgfpathlineto{\pgfqpoint{0.710880in}{1.789039in}}%
\pgfusepath{stroke}%
\end{pgfscope}%
\begin{pgfscope}%
\pgfsetbuttcap%
\pgfsetroundjoin%
\definecolor{currentfill}{rgb}{0.000000,0.000000,0.000000}%
\pgfsetfillcolor{currentfill}%
\pgfsetlinewidth{0.602250pt}%
\definecolor{currentstroke}{rgb}{0.000000,0.000000,0.000000}%
\pgfsetstrokecolor{currentstroke}%
\pgfsetdash{}{0pt}%
\pgfsys@defobject{currentmarker}{\pgfqpoint{0.000000in}{-0.027778in}}{\pgfqpoint{0.000000in}{0.000000in}}{%
\pgfpathmoveto{\pgfqpoint{0.000000in}{0.000000in}}%
\pgfpathlineto{\pgfqpoint{0.000000in}{-0.027778in}}%
\pgfusepath{stroke,fill}%
}%
\begin{pgfscope}%
\pgfsys@transformshift{0.710880in}{0.417642in}%
\pgfsys@useobject{currentmarker}{}%
\end{pgfscope}%
\end{pgfscope}%
\begin{pgfscope}%
\pgfpathrectangle{\pgfqpoint{0.514278in}{0.417642in}}{\pgfqpoint{1.884996in}{1.371397in}}%
\pgfusepath{clip}%
\pgfsetrectcap%
\pgfsetroundjoin%
\pgfsetlinewidth{0.803000pt}%
\definecolor{currentstroke}{rgb}{0.850000,0.850000,0.850000}%
\pgfsetstrokecolor{currentstroke}%
\pgfsetdash{}{0pt}%
\pgfpathmoveto{\pgfqpoint{0.761036in}{0.417642in}}%
\pgfpathlineto{\pgfqpoint{0.761036in}{1.789039in}}%
\pgfusepath{stroke}%
\end{pgfscope}%
\begin{pgfscope}%
\pgfsetbuttcap%
\pgfsetroundjoin%
\definecolor{currentfill}{rgb}{0.000000,0.000000,0.000000}%
\pgfsetfillcolor{currentfill}%
\pgfsetlinewidth{0.602250pt}%
\definecolor{currentstroke}{rgb}{0.000000,0.000000,0.000000}%
\pgfsetstrokecolor{currentstroke}%
\pgfsetdash{}{0pt}%
\pgfsys@defobject{currentmarker}{\pgfqpoint{0.000000in}{-0.027778in}}{\pgfqpoint{0.000000in}{0.000000in}}{%
\pgfpathmoveto{\pgfqpoint{0.000000in}{0.000000in}}%
\pgfpathlineto{\pgfqpoint{0.000000in}{-0.027778in}}%
\pgfusepath{stroke,fill}%
}%
\begin{pgfscope}%
\pgfsys@transformshift{0.761036in}{0.417642in}%
\pgfsys@useobject{currentmarker}{}%
\end{pgfscope}%
\end{pgfscope}%
\begin{pgfscope}%
\pgfpathrectangle{\pgfqpoint{0.514278in}{0.417642in}}{\pgfqpoint{1.884996in}{1.371397in}}%
\pgfusepath{clip}%
\pgfsetrectcap%
\pgfsetroundjoin%
\pgfsetlinewidth{0.803000pt}%
\definecolor{currentstroke}{rgb}{0.850000,0.850000,0.850000}%
\pgfsetstrokecolor{currentstroke}%
\pgfsetdash{}{0pt}%
\pgfpathmoveto{\pgfqpoint{0.802017in}{0.417642in}}%
\pgfpathlineto{\pgfqpoint{0.802017in}{1.789039in}}%
\pgfusepath{stroke}%
\end{pgfscope}%
\begin{pgfscope}%
\pgfsetbuttcap%
\pgfsetroundjoin%
\definecolor{currentfill}{rgb}{0.000000,0.000000,0.000000}%
\pgfsetfillcolor{currentfill}%
\pgfsetlinewidth{0.602250pt}%
\definecolor{currentstroke}{rgb}{0.000000,0.000000,0.000000}%
\pgfsetstrokecolor{currentstroke}%
\pgfsetdash{}{0pt}%
\pgfsys@defobject{currentmarker}{\pgfqpoint{0.000000in}{-0.027778in}}{\pgfqpoint{0.000000in}{0.000000in}}{%
\pgfpathmoveto{\pgfqpoint{0.000000in}{0.000000in}}%
\pgfpathlineto{\pgfqpoint{0.000000in}{-0.027778in}}%
\pgfusepath{stroke,fill}%
}%
\begin{pgfscope}%
\pgfsys@transformshift{0.802017in}{0.417642in}%
\pgfsys@useobject{currentmarker}{}%
\end{pgfscope}%
\end{pgfscope}%
\begin{pgfscope}%
\pgfpathrectangle{\pgfqpoint{0.514278in}{0.417642in}}{\pgfqpoint{1.884996in}{1.371397in}}%
\pgfusepath{clip}%
\pgfsetrectcap%
\pgfsetroundjoin%
\pgfsetlinewidth{0.803000pt}%
\definecolor{currentstroke}{rgb}{0.850000,0.850000,0.850000}%
\pgfsetstrokecolor{currentstroke}%
\pgfsetdash{}{0pt}%
\pgfpathmoveto{\pgfqpoint{0.836665in}{0.417642in}}%
\pgfpathlineto{\pgfqpoint{0.836665in}{1.789039in}}%
\pgfusepath{stroke}%
\end{pgfscope}%
\begin{pgfscope}%
\pgfsetbuttcap%
\pgfsetroundjoin%
\definecolor{currentfill}{rgb}{0.000000,0.000000,0.000000}%
\pgfsetfillcolor{currentfill}%
\pgfsetlinewidth{0.602250pt}%
\definecolor{currentstroke}{rgb}{0.000000,0.000000,0.000000}%
\pgfsetstrokecolor{currentstroke}%
\pgfsetdash{}{0pt}%
\pgfsys@defobject{currentmarker}{\pgfqpoint{0.000000in}{-0.027778in}}{\pgfqpoint{0.000000in}{0.000000in}}{%
\pgfpathmoveto{\pgfqpoint{0.000000in}{0.000000in}}%
\pgfpathlineto{\pgfqpoint{0.000000in}{-0.027778in}}%
\pgfusepath{stroke,fill}%
}%
\begin{pgfscope}%
\pgfsys@transformshift{0.836665in}{0.417642in}%
\pgfsys@useobject{currentmarker}{}%
\end{pgfscope}%
\end{pgfscope}%
\begin{pgfscope}%
\pgfpathrectangle{\pgfqpoint{0.514278in}{0.417642in}}{\pgfqpoint{1.884996in}{1.371397in}}%
\pgfusepath{clip}%
\pgfsetrectcap%
\pgfsetroundjoin%
\pgfsetlinewidth{0.803000pt}%
\definecolor{currentstroke}{rgb}{0.850000,0.850000,0.850000}%
\pgfsetstrokecolor{currentstroke}%
\pgfsetdash{}{0pt}%
\pgfpathmoveto{\pgfqpoint{0.866679in}{0.417642in}}%
\pgfpathlineto{\pgfqpoint{0.866679in}{1.789039in}}%
\pgfusepath{stroke}%
\end{pgfscope}%
\begin{pgfscope}%
\pgfsetbuttcap%
\pgfsetroundjoin%
\definecolor{currentfill}{rgb}{0.000000,0.000000,0.000000}%
\pgfsetfillcolor{currentfill}%
\pgfsetlinewidth{0.602250pt}%
\definecolor{currentstroke}{rgb}{0.000000,0.000000,0.000000}%
\pgfsetstrokecolor{currentstroke}%
\pgfsetdash{}{0pt}%
\pgfsys@defobject{currentmarker}{\pgfqpoint{0.000000in}{-0.027778in}}{\pgfqpoint{0.000000in}{0.000000in}}{%
\pgfpathmoveto{\pgfqpoint{0.000000in}{0.000000in}}%
\pgfpathlineto{\pgfqpoint{0.000000in}{-0.027778in}}%
\pgfusepath{stroke,fill}%
}%
\begin{pgfscope}%
\pgfsys@transformshift{0.866679in}{0.417642in}%
\pgfsys@useobject{currentmarker}{}%
\end{pgfscope}%
\end{pgfscope}%
\begin{pgfscope}%
\pgfpathrectangle{\pgfqpoint{0.514278in}{0.417642in}}{\pgfqpoint{1.884996in}{1.371397in}}%
\pgfusepath{clip}%
\pgfsetrectcap%
\pgfsetroundjoin%
\pgfsetlinewidth{0.803000pt}%
\definecolor{currentstroke}{rgb}{0.850000,0.850000,0.850000}%
\pgfsetstrokecolor{currentstroke}%
\pgfsetdash{}{0pt}%
\pgfpathmoveto{\pgfqpoint{0.893154in}{0.417642in}}%
\pgfpathlineto{\pgfqpoint{0.893154in}{1.789039in}}%
\pgfusepath{stroke}%
\end{pgfscope}%
\begin{pgfscope}%
\pgfsetbuttcap%
\pgfsetroundjoin%
\definecolor{currentfill}{rgb}{0.000000,0.000000,0.000000}%
\pgfsetfillcolor{currentfill}%
\pgfsetlinewidth{0.602250pt}%
\definecolor{currentstroke}{rgb}{0.000000,0.000000,0.000000}%
\pgfsetstrokecolor{currentstroke}%
\pgfsetdash{}{0pt}%
\pgfsys@defobject{currentmarker}{\pgfqpoint{0.000000in}{-0.027778in}}{\pgfqpoint{0.000000in}{0.000000in}}{%
\pgfpathmoveto{\pgfqpoint{0.000000in}{0.000000in}}%
\pgfpathlineto{\pgfqpoint{0.000000in}{-0.027778in}}%
\pgfusepath{stroke,fill}%
}%
\begin{pgfscope}%
\pgfsys@transformshift{0.893154in}{0.417642in}%
\pgfsys@useobject{currentmarker}{}%
\end{pgfscope}%
\end{pgfscope}%
\begin{pgfscope}%
\pgfpathrectangle{\pgfqpoint{0.514278in}{0.417642in}}{\pgfqpoint{1.884996in}{1.371397in}}%
\pgfusepath{clip}%
\pgfsetrectcap%
\pgfsetroundjoin%
\pgfsetlinewidth{0.803000pt}%
\definecolor{currentstroke}{rgb}{0.850000,0.850000,0.850000}%
\pgfsetstrokecolor{currentstroke}%
\pgfsetdash{}{0pt}%
\pgfpathmoveto{\pgfqpoint{1.072635in}{0.417642in}}%
\pgfpathlineto{\pgfqpoint{1.072635in}{1.789039in}}%
\pgfusepath{stroke}%
\end{pgfscope}%
\begin{pgfscope}%
\pgfsetbuttcap%
\pgfsetroundjoin%
\definecolor{currentfill}{rgb}{0.000000,0.000000,0.000000}%
\pgfsetfillcolor{currentfill}%
\pgfsetlinewidth{0.602250pt}%
\definecolor{currentstroke}{rgb}{0.000000,0.000000,0.000000}%
\pgfsetstrokecolor{currentstroke}%
\pgfsetdash{}{0pt}%
\pgfsys@defobject{currentmarker}{\pgfqpoint{0.000000in}{-0.027778in}}{\pgfqpoint{0.000000in}{0.000000in}}{%
\pgfpathmoveto{\pgfqpoint{0.000000in}{0.000000in}}%
\pgfpathlineto{\pgfqpoint{0.000000in}{-0.027778in}}%
\pgfusepath{stroke,fill}%
}%
\begin{pgfscope}%
\pgfsys@transformshift{1.072635in}{0.417642in}%
\pgfsys@useobject{currentmarker}{}%
\end{pgfscope}%
\end{pgfscope}%
\begin{pgfscope}%
\pgfpathrectangle{\pgfqpoint{0.514278in}{0.417642in}}{\pgfqpoint{1.884996in}{1.371397in}}%
\pgfusepath{clip}%
\pgfsetrectcap%
\pgfsetroundjoin%
\pgfsetlinewidth{0.803000pt}%
\definecolor{currentstroke}{rgb}{0.850000,0.850000,0.850000}%
\pgfsetstrokecolor{currentstroke}%
\pgfsetdash{}{0pt}%
\pgfpathmoveto{\pgfqpoint{1.163773in}{0.417642in}}%
\pgfpathlineto{\pgfqpoint{1.163773in}{1.789039in}}%
\pgfusepath{stroke}%
\end{pgfscope}%
\begin{pgfscope}%
\pgfsetbuttcap%
\pgfsetroundjoin%
\definecolor{currentfill}{rgb}{0.000000,0.000000,0.000000}%
\pgfsetfillcolor{currentfill}%
\pgfsetlinewidth{0.602250pt}%
\definecolor{currentstroke}{rgb}{0.000000,0.000000,0.000000}%
\pgfsetstrokecolor{currentstroke}%
\pgfsetdash{}{0pt}%
\pgfsys@defobject{currentmarker}{\pgfqpoint{0.000000in}{-0.027778in}}{\pgfqpoint{0.000000in}{0.000000in}}{%
\pgfpathmoveto{\pgfqpoint{0.000000in}{0.000000in}}%
\pgfpathlineto{\pgfqpoint{0.000000in}{-0.027778in}}%
\pgfusepath{stroke,fill}%
}%
\begin{pgfscope}%
\pgfsys@transformshift{1.163773in}{0.417642in}%
\pgfsys@useobject{currentmarker}{}%
\end{pgfscope}%
\end{pgfscope}%
\begin{pgfscope}%
\pgfpathrectangle{\pgfqpoint{0.514278in}{0.417642in}}{\pgfqpoint{1.884996in}{1.371397in}}%
\pgfusepath{clip}%
\pgfsetrectcap%
\pgfsetroundjoin%
\pgfsetlinewidth{0.803000pt}%
\definecolor{currentstroke}{rgb}{0.850000,0.850000,0.850000}%
\pgfsetstrokecolor{currentstroke}%
\pgfsetdash{}{0pt}%
\pgfpathmoveto{\pgfqpoint{1.228435in}{0.417642in}}%
\pgfpathlineto{\pgfqpoint{1.228435in}{1.789039in}}%
\pgfusepath{stroke}%
\end{pgfscope}%
\begin{pgfscope}%
\pgfsetbuttcap%
\pgfsetroundjoin%
\definecolor{currentfill}{rgb}{0.000000,0.000000,0.000000}%
\pgfsetfillcolor{currentfill}%
\pgfsetlinewidth{0.602250pt}%
\definecolor{currentstroke}{rgb}{0.000000,0.000000,0.000000}%
\pgfsetstrokecolor{currentstroke}%
\pgfsetdash{}{0pt}%
\pgfsys@defobject{currentmarker}{\pgfqpoint{0.000000in}{-0.027778in}}{\pgfqpoint{0.000000in}{0.000000in}}{%
\pgfpathmoveto{\pgfqpoint{0.000000in}{0.000000in}}%
\pgfpathlineto{\pgfqpoint{0.000000in}{-0.027778in}}%
\pgfusepath{stroke,fill}%
}%
\begin{pgfscope}%
\pgfsys@transformshift{1.228435in}{0.417642in}%
\pgfsys@useobject{currentmarker}{}%
\end{pgfscope}%
\end{pgfscope}%
\begin{pgfscope}%
\pgfpathrectangle{\pgfqpoint{0.514278in}{0.417642in}}{\pgfqpoint{1.884996in}{1.371397in}}%
\pgfusepath{clip}%
\pgfsetrectcap%
\pgfsetroundjoin%
\pgfsetlinewidth{0.803000pt}%
\definecolor{currentstroke}{rgb}{0.850000,0.850000,0.850000}%
\pgfsetstrokecolor{currentstroke}%
\pgfsetdash{}{0pt}%
\pgfpathmoveto{\pgfqpoint{1.278592in}{0.417642in}}%
\pgfpathlineto{\pgfqpoint{1.278592in}{1.789039in}}%
\pgfusepath{stroke}%
\end{pgfscope}%
\begin{pgfscope}%
\pgfsetbuttcap%
\pgfsetroundjoin%
\definecolor{currentfill}{rgb}{0.000000,0.000000,0.000000}%
\pgfsetfillcolor{currentfill}%
\pgfsetlinewidth{0.602250pt}%
\definecolor{currentstroke}{rgb}{0.000000,0.000000,0.000000}%
\pgfsetstrokecolor{currentstroke}%
\pgfsetdash{}{0pt}%
\pgfsys@defobject{currentmarker}{\pgfqpoint{0.000000in}{-0.027778in}}{\pgfqpoint{0.000000in}{0.000000in}}{%
\pgfpathmoveto{\pgfqpoint{0.000000in}{0.000000in}}%
\pgfpathlineto{\pgfqpoint{0.000000in}{-0.027778in}}%
\pgfusepath{stroke,fill}%
}%
\begin{pgfscope}%
\pgfsys@transformshift{1.278592in}{0.417642in}%
\pgfsys@useobject{currentmarker}{}%
\end{pgfscope}%
\end{pgfscope}%
\begin{pgfscope}%
\pgfpathrectangle{\pgfqpoint{0.514278in}{0.417642in}}{\pgfqpoint{1.884996in}{1.371397in}}%
\pgfusepath{clip}%
\pgfsetrectcap%
\pgfsetroundjoin%
\pgfsetlinewidth{0.803000pt}%
\definecolor{currentstroke}{rgb}{0.850000,0.850000,0.850000}%
\pgfsetstrokecolor{currentstroke}%
\pgfsetdash{}{0pt}%
\pgfpathmoveto{\pgfqpoint{1.319572in}{0.417642in}}%
\pgfpathlineto{\pgfqpoint{1.319572in}{1.789039in}}%
\pgfusepath{stroke}%
\end{pgfscope}%
\begin{pgfscope}%
\pgfsetbuttcap%
\pgfsetroundjoin%
\definecolor{currentfill}{rgb}{0.000000,0.000000,0.000000}%
\pgfsetfillcolor{currentfill}%
\pgfsetlinewidth{0.602250pt}%
\definecolor{currentstroke}{rgb}{0.000000,0.000000,0.000000}%
\pgfsetstrokecolor{currentstroke}%
\pgfsetdash{}{0pt}%
\pgfsys@defobject{currentmarker}{\pgfqpoint{0.000000in}{-0.027778in}}{\pgfqpoint{0.000000in}{0.000000in}}{%
\pgfpathmoveto{\pgfqpoint{0.000000in}{0.000000in}}%
\pgfpathlineto{\pgfqpoint{0.000000in}{-0.027778in}}%
\pgfusepath{stroke,fill}%
}%
\begin{pgfscope}%
\pgfsys@transformshift{1.319572in}{0.417642in}%
\pgfsys@useobject{currentmarker}{}%
\end{pgfscope}%
\end{pgfscope}%
\begin{pgfscope}%
\pgfpathrectangle{\pgfqpoint{0.514278in}{0.417642in}}{\pgfqpoint{1.884996in}{1.371397in}}%
\pgfusepath{clip}%
\pgfsetrectcap%
\pgfsetroundjoin%
\pgfsetlinewidth{0.803000pt}%
\definecolor{currentstroke}{rgb}{0.850000,0.850000,0.850000}%
\pgfsetstrokecolor{currentstroke}%
\pgfsetdash{}{0pt}%
\pgfpathmoveto{\pgfqpoint{1.354221in}{0.417642in}}%
\pgfpathlineto{\pgfqpoint{1.354221in}{1.789039in}}%
\pgfusepath{stroke}%
\end{pgfscope}%
\begin{pgfscope}%
\pgfsetbuttcap%
\pgfsetroundjoin%
\definecolor{currentfill}{rgb}{0.000000,0.000000,0.000000}%
\pgfsetfillcolor{currentfill}%
\pgfsetlinewidth{0.602250pt}%
\definecolor{currentstroke}{rgb}{0.000000,0.000000,0.000000}%
\pgfsetstrokecolor{currentstroke}%
\pgfsetdash{}{0pt}%
\pgfsys@defobject{currentmarker}{\pgfqpoint{0.000000in}{-0.027778in}}{\pgfqpoint{0.000000in}{0.000000in}}{%
\pgfpathmoveto{\pgfqpoint{0.000000in}{0.000000in}}%
\pgfpathlineto{\pgfqpoint{0.000000in}{-0.027778in}}%
\pgfusepath{stroke,fill}%
}%
\begin{pgfscope}%
\pgfsys@transformshift{1.354221in}{0.417642in}%
\pgfsys@useobject{currentmarker}{}%
\end{pgfscope}%
\end{pgfscope}%
\begin{pgfscope}%
\pgfpathrectangle{\pgfqpoint{0.514278in}{0.417642in}}{\pgfqpoint{1.884996in}{1.371397in}}%
\pgfusepath{clip}%
\pgfsetrectcap%
\pgfsetroundjoin%
\pgfsetlinewidth{0.803000pt}%
\definecolor{currentstroke}{rgb}{0.850000,0.850000,0.850000}%
\pgfsetstrokecolor{currentstroke}%
\pgfsetdash{}{0pt}%
\pgfpathmoveto{\pgfqpoint{1.384235in}{0.417642in}}%
\pgfpathlineto{\pgfqpoint{1.384235in}{1.789039in}}%
\pgfusepath{stroke}%
\end{pgfscope}%
\begin{pgfscope}%
\pgfsetbuttcap%
\pgfsetroundjoin%
\definecolor{currentfill}{rgb}{0.000000,0.000000,0.000000}%
\pgfsetfillcolor{currentfill}%
\pgfsetlinewidth{0.602250pt}%
\definecolor{currentstroke}{rgb}{0.000000,0.000000,0.000000}%
\pgfsetstrokecolor{currentstroke}%
\pgfsetdash{}{0pt}%
\pgfsys@defobject{currentmarker}{\pgfqpoint{0.000000in}{-0.027778in}}{\pgfqpoint{0.000000in}{0.000000in}}{%
\pgfpathmoveto{\pgfqpoint{0.000000in}{0.000000in}}%
\pgfpathlineto{\pgfqpoint{0.000000in}{-0.027778in}}%
\pgfusepath{stroke,fill}%
}%
\begin{pgfscope}%
\pgfsys@transformshift{1.384235in}{0.417642in}%
\pgfsys@useobject{currentmarker}{}%
\end{pgfscope}%
\end{pgfscope}%
\begin{pgfscope}%
\pgfpathrectangle{\pgfqpoint{0.514278in}{0.417642in}}{\pgfqpoint{1.884996in}{1.371397in}}%
\pgfusepath{clip}%
\pgfsetrectcap%
\pgfsetroundjoin%
\pgfsetlinewidth{0.803000pt}%
\definecolor{currentstroke}{rgb}{0.850000,0.850000,0.850000}%
\pgfsetstrokecolor{currentstroke}%
\pgfsetdash{}{0pt}%
\pgfpathmoveto{\pgfqpoint{1.410709in}{0.417642in}}%
\pgfpathlineto{\pgfqpoint{1.410709in}{1.789039in}}%
\pgfusepath{stroke}%
\end{pgfscope}%
\begin{pgfscope}%
\pgfsetbuttcap%
\pgfsetroundjoin%
\definecolor{currentfill}{rgb}{0.000000,0.000000,0.000000}%
\pgfsetfillcolor{currentfill}%
\pgfsetlinewidth{0.602250pt}%
\definecolor{currentstroke}{rgb}{0.000000,0.000000,0.000000}%
\pgfsetstrokecolor{currentstroke}%
\pgfsetdash{}{0pt}%
\pgfsys@defobject{currentmarker}{\pgfqpoint{0.000000in}{-0.027778in}}{\pgfqpoint{0.000000in}{0.000000in}}{%
\pgfpathmoveto{\pgfqpoint{0.000000in}{0.000000in}}%
\pgfpathlineto{\pgfqpoint{0.000000in}{-0.027778in}}%
\pgfusepath{stroke,fill}%
}%
\begin{pgfscope}%
\pgfsys@transformshift{1.410709in}{0.417642in}%
\pgfsys@useobject{currentmarker}{}%
\end{pgfscope}%
\end{pgfscope}%
\begin{pgfscope}%
\pgfpathrectangle{\pgfqpoint{0.514278in}{0.417642in}}{\pgfqpoint{1.884996in}{1.371397in}}%
\pgfusepath{clip}%
\pgfsetrectcap%
\pgfsetroundjoin%
\pgfsetlinewidth{0.803000pt}%
\definecolor{currentstroke}{rgb}{0.850000,0.850000,0.850000}%
\pgfsetstrokecolor{currentstroke}%
\pgfsetdash{}{0pt}%
\pgfpathmoveto{\pgfqpoint{1.590191in}{0.417642in}}%
\pgfpathlineto{\pgfqpoint{1.590191in}{1.789039in}}%
\pgfusepath{stroke}%
\end{pgfscope}%
\begin{pgfscope}%
\pgfsetbuttcap%
\pgfsetroundjoin%
\definecolor{currentfill}{rgb}{0.000000,0.000000,0.000000}%
\pgfsetfillcolor{currentfill}%
\pgfsetlinewidth{0.602250pt}%
\definecolor{currentstroke}{rgb}{0.000000,0.000000,0.000000}%
\pgfsetstrokecolor{currentstroke}%
\pgfsetdash{}{0pt}%
\pgfsys@defobject{currentmarker}{\pgfqpoint{0.000000in}{-0.027778in}}{\pgfqpoint{0.000000in}{0.000000in}}{%
\pgfpathmoveto{\pgfqpoint{0.000000in}{0.000000in}}%
\pgfpathlineto{\pgfqpoint{0.000000in}{-0.027778in}}%
\pgfusepath{stroke,fill}%
}%
\begin{pgfscope}%
\pgfsys@transformshift{1.590191in}{0.417642in}%
\pgfsys@useobject{currentmarker}{}%
\end{pgfscope}%
\end{pgfscope}%
\begin{pgfscope}%
\pgfpathrectangle{\pgfqpoint{0.514278in}{0.417642in}}{\pgfqpoint{1.884996in}{1.371397in}}%
\pgfusepath{clip}%
\pgfsetrectcap%
\pgfsetroundjoin%
\pgfsetlinewidth{0.803000pt}%
\definecolor{currentstroke}{rgb}{0.850000,0.850000,0.850000}%
\pgfsetstrokecolor{currentstroke}%
\pgfsetdash{}{0pt}%
\pgfpathmoveto{\pgfqpoint{1.681328in}{0.417642in}}%
\pgfpathlineto{\pgfqpoint{1.681328in}{1.789039in}}%
\pgfusepath{stroke}%
\end{pgfscope}%
\begin{pgfscope}%
\pgfsetbuttcap%
\pgfsetroundjoin%
\definecolor{currentfill}{rgb}{0.000000,0.000000,0.000000}%
\pgfsetfillcolor{currentfill}%
\pgfsetlinewidth{0.602250pt}%
\definecolor{currentstroke}{rgb}{0.000000,0.000000,0.000000}%
\pgfsetstrokecolor{currentstroke}%
\pgfsetdash{}{0pt}%
\pgfsys@defobject{currentmarker}{\pgfqpoint{0.000000in}{-0.027778in}}{\pgfqpoint{0.000000in}{0.000000in}}{%
\pgfpathmoveto{\pgfqpoint{0.000000in}{0.000000in}}%
\pgfpathlineto{\pgfqpoint{0.000000in}{-0.027778in}}%
\pgfusepath{stroke,fill}%
}%
\begin{pgfscope}%
\pgfsys@transformshift{1.681328in}{0.417642in}%
\pgfsys@useobject{currentmarker}{}%
\end{pgfscope}%
\end{pgfscope}%
\begin{pgfscope}%
\pgfpathrectangle{\pgfqpoint{0.514278in}{0.417642in}}{\pgfqpoint{1.884996in}{1.371397in}}%
\pgfusepath{clip}%
\pgfsetrectcap%
\pgfsetroundjoin%
\pgfsetlinewidth{0.803000pt}%
\definecolor{currentstroke}{rgb}{0.850000,0.850000,0.850000}%
\pgfsetstrokecolor{currentstroke}%
\pgfsetdash{}{0pt}%
\pgfpathmoveto{\pgfqpoint{1.745991in}{0.417642in}}%
\pgfpathlineto{\pgfqpoint{1.745991in}{1.789039in}}%
\pgfusepath{stroke}%
\end{pgfscope}%
\begin{pgfscope}%
\pgfsetbuttcap%
\pgfsetroundjoin%
\definecolor{currentfill}{rgb}{0.000000,0.000000,0.000000}%
\pgfsetfillcolor{currentfill}%
\pgfsetlinewidth{0.602250pt}%
\definecolor{currentstroke}{rgb}{0.000000,0.000000,0.000000}%
\pgfsetstrokecolor{currentstroke}%
\pgfsetdash{}{0pt}%
\pgfsys@defobject{currentmarker}{\pgfqpoint{0.000000in}{-0.027778in}}{\pgfqpoint{0.000000in}{0.000000in}}{%
\pgfpathmoveto{\pgfqpoint{0.000000in}{0.000000in}}%
\pgfpathlineto{\pgfqpoint{0.000000in}{-0.027778in}}%
\pgfusepath{stroke,fill}%
}%
\begin{pgfscope}%
\pgfsys@transformshift{1.745991in}{0.417642in}%
\pgfsys@useobject{currentmarker}{}%
\end{pgfscope}%
\end{pgfscope}%
\begin{pgfscope}%
\pgfpathrectangle{\pgfqpoint{0.514278in}{0.417642in}}{\pgfqpoint{1.884996in}{1.371397in}}%
\pgfusepath{clip}%
\pgfsetrectcap%
\pgfsetroundjoin%
\pgfsetlinewidth{0.803000pt}%
\definecolor{currentstroke}{rgb}{0.850000,0.850000,0.850000}%
\pgfsetstrokecolor{currentstroke}%
\pgfsetdash{}{0pt}%
\pgfpathmoveto{\pgfqpoint{1.796147in}{0.417642in}}%
\pgfpathlineto{\pgfqpoint{1.796147in}{1.789039in}}%
\pgfusepath{stroke}%
\end{pgfscope}%
\begin{pgfscope}%
\pgfsetbuttcap%
\pgfsetroundjoin%
\definecolor{currentfill}{rgb}{0.000000,0.000000,0.000000}%
\pgfsetfillcolor{currentfill}%
\pgfsetlinewidth{0.602250pt}%
\definecolor{currentstroke}{rgb}{0.000000,0.000000,0.000000}%
\pgfsetstrokecolor{currentstroke}%
\pgfsetdash{}{0pt}%
\pgfsys@defobject{currentmarker}{\pgfqpoint{0.000000in}{-0.027778in}}{\pgfqpoint{0.000000in}{0.000000in}}{%
\pgfpathmoveto{\pgfqpoint{0.000000in}{0.000000in}}%
\pgfpathlineto{\pgfqpoint{0.000000in}{-0.027778in}}%
\pgfusepath{stroke,fill}%
}%
\begin{pgfscope}%
\pgfsys@transformshift{1.796147in}{0.417642in}%
\pgfsys@useobject{currentmarker}{}%
\end{pgfscope}%
\end{pgfscope}%
\begin{pgfscope}%
\pgfpathrectangle{\pgfqpoint{0.514278in}{0.417642in}}{\pgfqpoint{1.884996in}{1.371397in}}%
\pgfusepath{clip}%
\pgfsetrectcap%
\pgfsetroundjoin%
\pgfsetlinewidth{0.803000pt}%
\definecolor{currentstroke}{rgb}{0.850000,0.850000,0.850000}%
\pgfsetstrokecolor{currentstroke}%
\pgfsetdash{}{0pt}%
\pgfpathmoveto{\pgfqpoint{1.837128in}{0.417642in}}%
\pgfpathlineto{\pgfqpoint{1.837128in}{1.789039in}}%
\pgfusepath{stroke}%
\end{pgfscope}%
\begin{pgfscope}%
\pgfsetbuttcap%
\pgfsetroundjoin%
\definecolor{currentfill}{rgb}{0.000000,0.000000,0.000000}%
\pgfsetfillcolor{currentfill}%
\pgfsetlinewidth{0.602250pt}%
\definecolor{currentstroke}{rgb}{0.000000,0.000000,0.000000}%
\pgfsetstrokecolor{currentstroke}%
\pgfsetdash{}{0pt}%
\pgfsys@defobject{currentmarker}{\pgfqpoint{0.000000in}{-0.027778in}}{\pgfqpoint{0.000000in}{0.000000in}}{%
\pgfpathmoveto{\pgfqpoint{0.000000in}{0.000000in}}%
\pgfpathlineto{\pgfqpoint{0.000000in}{-0.027778in}}%
\pgfusepath{stroke,fill}%
}%
\begin{pgfscope}%
\pgfsys@transformshift{1.837128in}{0.417642in}%
\pgfsys@useobject{currentmarker}{}%
\end{pgfscope}%
\end{pgfscope}%
\begin{pgfscope}%
\pgfpathrectangle{\pgfqpoint{0.514278in}{0.417642in}}{\pgfqpoint{1.884996in}{1.371397in}}%
\pgfusepath{clip}%
\pgfsetrectcap%
\pgfsetroundjoin%
\pgfsetlinewidth{0.803000pt}%
\definecolor{currentstroke}{rgb}{0.850000,0.850000,0.850000}%
\pgfsetstrokecolor{currentstroke}%
\pgfsetdash{}{0pt}%
\pgfpathmoveto{\pgfqpoint{1.871777in}{0.417642in}}%
\pgfpathlineto{\pgfqpoint{1.871777in}{1.789039in}}%
\pgfusepath{stroke}%
\end{pgfscope}%
\begin{pgfscope}%
\pgfsetbuttcap%
\pgfsetroundjoin%
\definecolor{currentfill}{rgb}{0.000000,0.000000,0.000000}%
\pgfsetfillcolor{currentfill}%
\pgfsetlinewidth{0.602250pt}%
\definecolor{currentstroke}{rgb}{0.000000,0.000000,0.000000}%
\pgfsetstrokecolor{currentstroke}%
\pgfsetdash{}{0pt}%
\pgfsys@defobject{currentmarker}{\pgfqpoint{0.000000in}{-0.027778in}}{\pgfqpoint{0.000000in}{0.000000in}}{%
\pgfpathmoveto{\pgfqpoint{0.000000in}{0.000000in}}%
\pgfpathlineto{\pgfqpoint{0.000000in}{-0.027778in}}%
\pgfusepath{stroke,fill}%
}%
\begin{pgfscope}%
\pgfsys@transformshift{1.871777in}{0.417642in}%
\pgfsys@useobject{currentmarker}{}%
\end{pgfscope}%
\end{pgfscope}%
\begin{pgfscope}%
\pgfpathrectangle{\pgfqpoint{0.514278in}{0.417642in}}{\pgfqpoint{1.884996in}{1.371397in}}%
\pgfusepath{clip}%
\pgfsetrectcap%
\pgfsetroundjoin%
\pgfsetlinewidth{0.803000pt}%
\definecolor{currentstroke}{rgb}{0.850000,0.850000,0.850000}%
\pgfsetstrokecolor{currentstroke}%
\pgfsetdash{}{0pt}%
\pgfpathmoveto{\pgfqpoint{1.901791in}{0.417642in}}%
\pgfpathlineto{\pgfqpoint{1.901791in}{1.789039in}}%
\pgfusepath{stroke}%
\end{pgfscope}%
\begin{pgfscope}%
\pgfsetbuttcap%
\pgfsetroundjoin%
\definecolor{currentfill}{rgb}{0.000000,0.000000,0.000000}%
\pgfsetfillcolor{currentfill}%
\pgfsetlinewidth{0.602250pt}%
\definecolor{currentstroke}{rgb}{0.000000,0.000000,0.000000}%
\pgfsetstrokecolor{currentstroke}%
\pgfsetdash{}{0pt}%
\pgfsys@defobject{currentmarker}{\pgfqpoint{0.000000in}{-0.027778in}}{\pgfqpoint{0.000000in}{0.000000in}}{%
\pgfpathmoveto{\pgfqpoint{0.000000in}{0.000000in}}%
\pgfpathlineto{\pgfqpoint{0.000000in}{-0.027778in}}%
\pgfusepath{stroke,fill}%
}%
\begin{pgfscope}%
\pgfsys@transformshift{1.901791in}{0.417642in}%
\pgfsys@useobject{currentmarker}{}%
\end{pgfscope}%
\end{pgfscope}%
\begin{pgfscope}%
\pgfpathrectangle{\pgfqpoint{0.514278in}{0.417642in}}{\pgfqpoint{1.884996in}{1.371397in}}%
\pgfusepath{clip}%
\pgfsetrectcap%
\pgfsetroundjoin%
\pgfsetlinewidth{0.803000pt}%
\definecolor{currentstroke}{rgb}{0.850000,0.850000,0.850000}%
\pgfsetstrokecolor{currentstroke}%
\pgfsetdash{}{0pt}%
\pgfpathmoveto{\pgfqpoint{1.928265in}{0.417642in}}%
\pgfpathlineto{\pgfqpoint{1.928265in}{1.789039in}}%
\pgfusepath{stroke}%
\end{pgfscope}%
\begin{pgfscope}%
\pgfsetbuttcap%
\pgfsetroundjoin%
\definecolor{currentfill}{rgb}{0.000000,0.000000,0.000000}%
\pgfsetfillcolor{currentfill}%
\pgfsetlinewidth{0.602250pt}%
\definecolor{currentstroke}{rgb}{0.000000,0.000000,0.000000}%
\pgfsetstrokecolor{currentstroke}%
\pgfsetdash{}{0pt}%
\pgfsys@defobject{currentmarker}{\pgfqpoint{0.000000in}{-0.027778in}}{\pgfqpoint{0.000000in}{0.000000in}}{%
\pgfpathmoveto{\pgfqpoint{0.000000in}{0.000000in}}%
\pgfpathlineto{\pgfqpoint{0.000000in}{-0.027778in}}%
\pgfusepath{stroke,fill}%
}%
\begin{pgfscope}%
\pgfsys@transformshift{1.928265in}{0.417642in}%
\pgfsys@useobject{currentmarker}{}%
\end{pgfscope}%
\end{pgfscope}%
\begin{pgfscope}%
\pgfpathrectangle{\pgfqpoint{0.514278in}{0.417642in}}{\pgfqpoint{1.884996in}{1.371397in}}%
\pgfusepath{clip}%
\pgfsetrectcap%
\pgfsetroundjoin%
\pgfsetlinewidth{0.803000pt}%
\definecolor{currentstroke}{rgb}{0.850000,0.850000,0.850000}%
\pgfsetstrokecolor{currentstroke}%
\pgfsetdash{}{0pt}%
\pgfpathmoveto{\pgfqpoint{2.107747in}{0.417642in}}%
\pgfpathlineto{\pgfqpoint{2.107747in}{1.789039in}}%
\pgfusepath{stroke}%
\end{pgfscope}%
\begin{pgfscope}%
\pgfsetbuttcap%
\pgfsetroundjoin%
\definecolor{currentfill}{rgb}{0.000000,0.000000,0.000000}%
\pgfsetfillcolor{currentfill}%
\pgfsetlinewidth{0.602250pt}%
\definecolor{currentstroke}{rgb}{0.000000,0.000000,0.000000}%
\pgfsetstrokecolor{currentstroke}%
\pgfsetdash{}{0pt}%
\pgfsys@defobject{currentmarker}{\pgfqpoint{0.000000in}{-0.027778in}}{\pgfqpoint{0.000000in}{0.000000in}}{%
\pgfpathmoveto{\pgfqpoint{0.000000in}{0.000000in}}%
\pgfpathlineto{\pgfqpoint{0.000000in}{-0.027778in}}%
\pgfusepath{stroke,fill}%
}%
\begin{pgfscope}%
\pgfsys@transformshift{2.107747in}{0.417642in}%
\pgfsys@useobject{currentmarker}{}%
\end{pgfscope}%
\end{pgfscope}%
\begin{pgfscope}%
\pgfpathrectangle{\pgfqpoint{0.514278in}{0.417642in}}{\pgfqpoint{1.884996in}{1.371397in}}%
\pgfusepath{clip}%
\pgfsetrectcap%
\pgfsetroundjoin%
\pgfsetlinewidth{0.803000pt}%
\definecolor{currentstroke}{rgb}{0.850000,0.850000,0.850000}%
\pgfsetstrokecolor{currentstroke}%
\pgfsetdash{}{0pt}%
\pgfpathmoveto{\pgfqpoint{2.198884in}{0.417642in}}%
\pgfpathlineto{\pgfqpoint{2.198884in}{1.789039in}}%
\pgfusepath{stroke}%
\end{pgfscope}%
\begin{pgfscope}%
\pgfsetbuttcap%
\pgfsetroundjoin%
\definecolor{currentfill}{rgb}{0.000000,0.000000,0.000000}%
\pgfsetfillcolor{currentfill}%
\pgfsetlinewidth{0.602250pt}%
\definecolor{currentstroke}{rgb}{0.000000,0.000000,0.000000}%
\pgfsetstrokecolor{currentstroke}%
\pgfsetdash{}{0pt}%
\pgfsys@defobject{currentmarker}{\pgfqpoint{0.000000in}{-0.027778in}}{\pgfqpoint{0.000000in}{0.000000in}}{%
\pgfpathmoveto{\pgfqpoint{0.000000in}{0.000000in}}%
\pgfpathlineto{\pgfqpoint{0.000000in}{-0.027778in}}%
\pgfusepath{stroke,fill}%
}%
\begin{pgfscope}%
\pgfsys@transformshift{2.198884in}{0.417642in}%
\pgfsys@useobject{currentmarker}{}%
\end{pgfscope}%
\end{pgfscope}%
\begin{pgfscope}%
\pgfpathrectangle{\pgfqpoint{0.514278in}{0.417642in}}{\pgfqpoint{1.884996in}{1.371397in}}%
\pgfusepath{clip}%
\pgfsetrectcap%
\pgfsetroundjoin%
\pgfsetlinewidth{0.803000pt}%
\definecolor{currentstroke}{rgb}{0.850000,0.850000,0.850000}%
\pgfsetstrokecolor{currentstroke}%
\pgfsetdash{}{0pt}%
\pgfpathmoveto{\pgfqpoint{2.263547in}{0.417642in}}%
\pgfpathlineto{\pgfqpoint{2.263547in}{1.789039in}}%
\pgfusepath{stroke}%
\end{pgfscope}%
\begin{pgfscope}%
\pgfsetbuttcap%
\pgfsetroundjoin%
\definecolor{currentfill}{rgb}{0.000000,0.000000,0.000000}%
\pgfsetfillcolor{currentfill}%
\pgfsetlinewidth{0.602250pt}%
\definecolor{currentstroke}{rgb}{0.000000,0.000000,0.000000}%
\pgfsetstrokecolor{currentstroke}%
\pgfsetdash{}{0pt}%
\pgfsys@defobject{currentmarker}{\pgfqpoint{0.000000in}{-0.027778in}}{\pgfqpoint{0.000000in}{0.000000in}}{%
\pgfpathmoveto{\pgfqpoint{0.000000in}{0.000000in}}%
\pgfpathlineto{\pgfqpoint{0.000000in}{-0.027778in}}%
\pgfusepath{stroke,fill}%
}%
\begin{pgfscope}%
\pgfsys@transformshift{2.263547in}{0.417642in}%
\pgfsys@useobject{currentmarker}{}%
\end{pgfscope}%
\end{pgfscope}%
\begin{pgfscope}%
\pgfpathrectangle{\pgfqpoint{0.514278in}{0.417642in}}{\pgfqpoint{1.884996in}{1.371397in}}%
\pgfusepath{clip}%
\pgfsetrectcap%
\pgfsetroundjoin%
\pgfsetlinewidth{0.803000pt}%
\definecolor{currentstroke}{rgb}{0.850000,0.850000,0.850000}%
\pgfsetstrokecolor{currentstroke}%
\pgfsetdash{}{0pt}%
\pgfpathmoveto{\pgfqpoint{2.313703in}{0.417642in}}%
\pgfpathlineto{\pgfqpoint{2.313703in}{1.789039in}}%
\pgfusepath{stroke}%
\end{pgfscope}%
\begin{pgfscope}%
\pgfsetbuttcap%
\pgfsetroundjoin%
\definecolor{currentfill}{rgb}{0.000000,0.000000,0.000000}%
\pgfsetfillcolor{currentfill}%
\pgfsetlinewidth{0.602250pt}%
\definecolor{currentstroke}{rgb}{0.000000,0.000000,0.000000}%
\pgfsetstrokecolor{currentstroke}%
\pgfsetdash{}{0pt}%
\pgfsys@defobject{currentmarker}{\pgfqpoint{0.000000in}{-0.027778in}}{\pgfqpoint{0.000000in}{0.000000in}}{%
\pgfpathmoveto{\pgfqpoint{0.000000in}{0.000000in}}%
\pgfpathlineto{\pgfqpoint{0.000000in}{-0.027778in}}%
\pgfusepath{stroke,fill}%
}%
\begin{pgfscope}%
\pgfsys@transformshift{2.313703in}{0.417642in}%
\pgfsys@useobject{currentmarker}{}%
\end{pgfscope}%
\end{pgfscope}%
\begin{pgfscope}%
\pgfpathrectangle{\pgfqpoint{0.514278in}{0.417642in}}{\pgfqpoint{1.884996in}{1.371397in}}%
\pgfusepath{clip}%
\pgfsetrectcap%
\pgfsetroundjoin%
\pgfsetlinewidth{0.803000pt}%
\definecolor{currentstroke}{rgb}{0.850000,0.850000,0.850000}%
\pgfsetstrokecolor{currentstroke}%
\pgfsetdash{}{0pt}%
\pgfpathmoveto{\pgfqpoint{2.354684in}{0.417642in}}%
\pgfpathlineto{\pgfqpoint{2.354684in}{1.789039in}}%
\pgfusepath{stroke}%
\end{pgfscope}%
\begin{pgfscope}%
\pgfsetbuttcap%
\pgfsetroundjoin%
\definecolor{currentfill}{rgb}{0.000000,0.000000,0.000000}%
\pgfsetfillcolor{currentfill}%
\pgfsetlinewidth{0.602250pt}%
\definecolor{currentstroke}{rgb}{0.000000,0.000000,0.000000}%
\pgfsetstrokecolor{currentstroke}%
\pgfsetdash{}{0pt}%
\pgfsys@defobject{currentmarker}{\pgfqpoint{0.000000in}{-0.027778in}}{\pgfqpoint{0.000000in}{0.000000in}}{%
\pgfpathmoveto{\pgfqpoint{0.000000in}{0.000000in}}%
\pgfpathlineto{\pgfqpoint{0.000000in}{-0.027778in}}%
\pgfusepath{stroke,fill}%
}%
\begin{pgfscope}%
\pgfsys@transformshift{2.354684in}{0.417642in}%
\pgfsys@useobject{currentmarker}{}%
\end{pgfscope}%
\end{pgfscope}%
\begin{pgfscope}%
\pgfpathrectangle{\pgfqpoint{0.514278in}{0.417642in}}{\pgfqpoint{1.884996in}{1.371397in}}%
\pgfusepath{clip}%
\pgfsetrectcap%
\pgfsetroundjoin%
\pgfsetlinewidth{0.803000pt}%
\definecolor{currentstroke}{rgb}{0.850000,0.850000,0.850000}%
\pgfsetstrokecolor{currentstroke}%
\pgfsetdash{}{0pt}%
\pgfpathmoveto{\pgfqpoint{2.389333in}{0.417642in}}%
\pgfpathlineto{\pgfqpoint{2.389333in}{1.789039in}}%
\pgfusepath{stroke}%
\end{pgfscope}%
\begin{pgfscope}%
\pgfsetbuttcap%
\pgfsetroundjoin%
\definecolor{currentfill}{rgb}{0.000000,0.000000,0.000000}%
\pgfsetfillcolor{currentfill}%
\pgfsetlinewidth{0.602250pt}%
\definecolor{currentstroke}{rgb}{0.000000,0.000000,0.000000}%
\pgfsetstrokecolor{currentstroke}%
\pgfsetdash{}{0pt}%
\pgfsys@defobject{currentmarker}{\pgfqpoint{0.000000in}{-0.027778in}}{\pgfqpoint{0.000000in}{0.000000in}}{%
\pgfpathmoveto{\pgfqpoint{0.000000in}{0.000000in}}%
\pgfpathlineto{\pgfqpoint{0.000000in}{-0.027778in}}%
\pgfusepath{stroke,fill}%
}%
\begin{pgfscope}%
\pgfsys@transformshift{2.389333in}{0.417642in}%
\pgfsys@useobject{currentmarker}{}%
\end{pgfscope}%
\end{pgfscope}%
\begin{pgfscope}%
\definecolor{textcolor}{rgb}{0.000000,0.000000,0.000000}%
\pgfsetstrokecolor{textcolor}%
\pgfsetfillcolor{textcolor}%
\pgftext[x=1.456777in,y=0.165003in,,top]{\color{textcolor}\rmfamily\fontsize{10.000000}{12.000000}\selectfont Frequency in \(\displaystyle \unit{\Hz}\)}%
\end{pgfscope}%
\begin{pgfscope}%
\pgfpathrectangle{\pgfqpoint{0.514278in}{0.417642in}}{\pgfqpoint{1.884996in}{1.371397in}}%
\pgfusepath{clip}%
\pgfsetrectcap%
\pgfsetroundjoin%
\pgfsetlinewidth{0.803000pt}%
\definecolor{currentstroke}{rgb}{0.450000,0.450000,0.450000}%
\pgfsetstrokecolor{currentstroke}%
\pgfsetdash{}{0pt}%
\pgfpathmoveto{\pgfqpoint{0.514278in}{0.640670in}}%
\pgfpathlineto{\pgfqpoint{2.399275in}{0.640670in}}%
\pgfusepath{stroke}%
\end{pgfscope}%
\begin{pgfscope}%
\pgfsetbuttcap%
\pgfsetroundjoin%
\definecolor{currentfill}{rgb}{0.000000,0.000000,0.000000}%
\pgfsetfillcolor{currentfill}%
\pgfsetlinewidth{0.803000pt}%
\definecolor{currentstroke}{rgb}{0.000000,0.000000,0.000000}%
\pgfsetstrokecolor{currentstroke}%
\pgfsetdash{}{0pt}%
\pgfsys@defobject{currentmarker}{\pgfqpoint{-0.048611in}{0.000000in}}{\pgfqpoint{-0.000000in}{0.000000in}}{%
\pgfpathmoveto{\pgfqpoint{-0.000000in}{0.000000in}}%
\pgfpathlineto{\pgfqpoint{-0.048611in}{0.000000in}}%
\pgfusepath{stroke,fill}%
}%
\begin{pgfscope}%
\pgfsys@transformshift{0.514278in}{0.640670in}%
\pgfsys@useobject{currentmarker}{}%
\end{pgfscope}%
\end{pgfscope}%
\begin{pgfscope}%
\definecolor{textcolor}{rgb}{0.000000,0.000000,0.000000}%
\pgfsetstrokecolor{textcolor}%
\pgfsetfillcolor{textcolor}%
\pgftext[x=0.241129in, y=0.601518in, left, base]{\color{textcolor}\rmfamily\fontsize{8.000000}{9.600000}\selectfont \(\displaystyle {10^{0}}\)}%
\end{pgfscope}%
\begin{pgfscope}%
\pgfpathrectangle{\pgfqpoint{0.514278in}{0.417642in}}{\pgfqpoint{1.884996in}{1.371397in}}%
\pgfusepath{clip}%
\pgfsetrectcap%
\pgfsetroundjoin%
\pgfsetlinewidth{0.803000pt}%
\definecolor{currentstroke}{rgb}{0.450000,0.450000,0.450000}%
\pgfsetstrokecolor{currentstroke}%
\pgfsetdash{}{0pt}%
\pgfpathmoveto{\pgfqpoint{0.514278in}{0.983520in}}%
\pgfpathlineto{\pgfqpoint{2.399275in}{0.983520in}}%
\pgfusepath{stroke}%
\end{pgfscope}%
\begin{pgfscope}%
\pgfsetbuttcap%
\pgfsetroundjoin%
\definecolor{currentfill}{rgb}{0.000000,0.000000,0.000000}%
\pgfsetfillcolor{currentfill}%
\pgfsetlinewidth{0.803000pt}%
\definecolor{currentstroke}{rgb}{0.000000,0.000000,0.000000}%
\pgfsetstrokecolor{currentstroke}%
\pgfsetdash{}{0pt}%
\pgfsys@defobject{currentmarker}{\pgfqpoint{-0.048611in}{0.000000in}}{\pgfqpoint{-0.000000in}{0.000000in}}{%
\pgfpathmoveto{\pgfqpoint{-0.000000in}{0.000000in}}%
\pgfpathlineto{\pgfqpoint{-0.048611in}{0.000000in}}%
\pgfusepath{stroke,fill}%
}%
\begin{pgfscope}%
\pgfsys@transformshift{0.514278in}{0.983520in}%
\pgfsys@useobject{currentmarker}{}%
\end{pgfscope}%
\end{pgfscope}%
\begin{pgfscope}%
\definecolor{textcolor}{rgb}{0.000000,0.000000,0.000000}%
\pgfsetstrokecolor{textcolor}%
\pgfsetfillcolor{textcolor}%
\pgftext[x=0.241129in, y=0.944367in, left, base]{\color{textcolor}\rmfamily\fontsize{8.000000}{9.600000}\selectfont \(\displaystyle {10^{2}}\)}%
\end{pgfscope}%
\begin{pgfscope}%
\pgfpathrectangle{\pgfqpoint{0.514278in}{0.417642in}}{\pgfqpoint{1.884996in}{1.371397in}}%
\pgfusepath{clip}%
\pgfsetrectcap%
\pgfsetroundjoin%
\pgfsetlinewidth{0.803000pt}%
\definecolor{currentstroke}{rgb}{0.450000,0.450000,0.450000}%
\pgfsetstrokecolor{currentstroke}%
\pgfsetdash{}{0pt}%
\pgfpathmoveto{\pgfqpoint{0.514278in}{1.326369in}}%
\pgfpathlineto{\pgfqpoint{2.399275in}{1.326369in}}%
\pgfusepath{stroke}%
\end{pgfscope}%
\begin{pgfscope}%
\pgfsetbuttcap%
\pgfsetroundjoin%
\definecolor{currentfill}{rgb}{0.000000,0.000000,0.000000}%
\pgfsetfillcolor{currentfill}%
\pgfsetlinewidth{0.803000pt}%
\definecolor{currentstroke}{rgb}{0.000000,0.000000,0.000000}%
\pgfsetstrokecolor{currentstroke}%
\pgfsetdash{}{0pt}%
\pgfsys@defobject{currentmarker}{\pgfqpoint{-0.048611in}{0.000000in}}{\pgfqpoint{-0.000000in}{0.000000in}}{%
\pgfpathmoveto{\pgfqpoint{-0.000000in}{0.000000in}}%
\pgfpathlineto{\pgfqpoint{-0.048611in}{0.000000in}}%
\pgfusepath{stroke,fill}%
}%
\begin{pgfscope}%
\pgfsys@transformshift{0.514278in}{1.326369in}%
\pgfsys@useobject{currentmarker}{}%
\end{pgfscope}%
\end{pgfscope}%
\begin{pgfscope}%
\definecolor{textcolor}{rgb}{0.000000,0.000000,0.000000}%
\pgfsetstrokecolor{textcolor}%
\pgfsetfillcolor{textcolor}%
\pgftext[x=0.241129in, y=1.287216in, left, base]{\color{textcolor}\rmfamily\fontsize{8.000000}{9.600000}\selectfont \(\displaystyle {10^{4}}\)}%
\end{pgfscope}%
\begin{pgfscope}%
\pgfpathrectangle{\pgfqpoint{0.514278in}{0.417642in}}{\pgfqpoint{1.884996in}{1.371397in}}%
\pgfusepath{clip}%
\pgfsetrectcap%
\pgfsetroundjoin%
\pgfsetlinewidth{0.803000pt}%
\definecolor{currentstroke}{rgb}{0.450000,0.450000,0.450000}%
\pgfsetstrokecolor{currentstroke}%
\pgfsetdash{}{0pt}%
\pgfpathmoveto{\pgfqpoint{0.514278in}{1.669218in}}%
\pgfpathlineto{\pgfqpoint{2.399275in}{1.669218in}}%
\pgfusepath{stroke}%
\end{pgfscope}%
\begin{pgfscope}%
\pgfsetbuttcap%
\pgfsetroundjoin%
\definecolor{currentfill}{rgb}{0.000000,0.000000,0.000000}%
\pgfsetfillcolor{currentfill}%
\pgfsetlinewidth{0.803000pt}%
\definecolor{currentstroke}{rgb}{0.000000,0.000000,0.000000}%
\pgfsetstrokecolor{currentstroke}%
\pgfsetdash{}{0pt}%
\pgfsys@defobject{currentmarker}{\pgfqpoint{-0.048611in}{0.000000in}}{\pgfqpoint{-0.000000in}{0.000000in}}{%
\pgfpathmoveto{\pgfqpoint{-0.000000in}{0.000000in}}%
\pgfpathlineto{\pgfqpoint{-0.048611in}{0.000000in}}%
\pgfusepath{stroke,fill}%
}%
\begin{pgfscope}%
\pgfsys@transformshift{0.514278in}{1.669218in}%
\pgfsys@useobject{currentmarker}{}%
\end{pgfscope}%
\end{pgfscope}%
\begin{pgfscope}%
\definecolor{textcolor}{rgb}{0.000000,0.000000,0.000000}%
\pgfsetstrokecolor{textcolor}%
\pgfsetfillcolor{textcolor}%
\pgftext[x=0.241129in, y=1.630065in, left, base]{\color{textcolor}\rmfamily\fontsize{8.000000}{9.600000}\selectfont \(\displaystyle {10^{6}}\)}%
\end{pgfscope}%
\begin{pgfscope}%
\definecolor{textcolor}{rgb}{0.000000,0.000000,0.000000}%
\pgfsetstrokecolor{textcolor}%
\pgfsetfillcolor{textcolor}%
\pgftext[x=0.185574in,y=1.103340in,,bottom,rotate=90.000000]{\color{textcolor}\rmfamily\fontsize{10.000000}{12.000000}\selectfont \(\displaystyle S_y(f)\) in \(\displaystyle \unit{1 \per \Hz}\)}%
\end{pgfscope}%
\begin{pgfscope}%
\pgfpathrectangle{\pgfqpoint{0.514278in}{0.417642in}}{\pgfqpoint{1.884996in}{1.371397in}}%
\pgfusepath{clip}%
\pgfsetbuttcap%
\pgfsetroundjoin%
\pgfsetlinewidth{1.505625pt}%
\definecolor{currentstroke}{rgb}{0.007843,0.619608,0.450980}%
\pgfsetstrokecolor{currentstroke}%
\pgfsetdash{{5.550000pt}{2.400000pt}}{0.000000pt}%
\pgfpathmoveto{\pgfqpoint{0.599960in}{1.235582in}}%
\pgfpathlineto{\pgfqpoint{0.755760in}{1.183978in}}%
\pgfpathlineto{\pgfqpoint{0.846897in}{1.153792in}}%
\pgfpathlineto{\pgfqpoint{0.911560in}{1.132374in}}%
\pgfpathlineto{\pgfqpoint{0.961716in}{1.115761in}}%
\pgfpathlineto{\pgfqpoint{1.002697in}{1.102188in}}%
\pgfpathlineto{\pgfqpoint{1.037345in}{1.090712in}}%
\pgfpathlineto{\pgfqpoint{1.067360in}{1.080770in}}%
\pgfpathlineto{\pgfqpoint{1.093834in}{1.072002in}}%
\pgfpathlineto{\pgfqpoint{1.117516in}{1.064158in}}%
\pgfpathlineto{\pgfqpoint{1.138939in}{1.057062in}}%
\pgfpathlineto{\pgfqpoint{1.158497in}{1.050584in}}%
\pgfpathlineto{\pgfqpoint{1.176488in}{1.044625in}}%
\pgfpathlineto{\pgfqpoint{1.193145in}{1.039108in}}%
\pgfpathlineto{\pgfqpoint{1.208653in}{1.033971in}}%
\pgfpathlineto{\pgfqpoint{1.223159in}{1.029166in}}%
\pgfpathlineto{\pgfqpoint{1.236786in}{1.024653in}}%
\pgfpathlineto{\pgfqpoint{1.249634in}{1.020398in}}%
\pgfpathlineto{\pgfqpoint{1.261786in}{1.016372in}}%
\pgfpathlineto{\pgfqpoint{1.273316in}{1.012554in}}%
\pgfpathlineto{\pgfqpoint{1.284282in}{1.008921in}}%
\pgfpathlineto{\pgfqpoint{1.294739in}{1.005458in}}%
\pgfpathlineto{\pgfqpoint{1.309564in}{1.000547in}}%
\pgfpathlineto{\pgfqpoint{1.323472in}{0.995941in}}%
\pgfpathlineto{\pgfqpoint{1.332288in}{0.993021in}}%
\pgfpathlineto{\pgfqpoint{1.340771in}{0.990211in}}%
\pgfpathlineto{\pgfqpoint{1.348945in}{0.987504in}}%
\pgfpathlineto{\pgfqpoint{1.360675in}{0.983618in}}%
\pgfpathlineto{\pgfqpoint{1.371823in}{0.979926in}}%
\pgfpathlineto{\pgfqpoint{1.382444in}{0.976408in}}%
\pgfpathlineto{\pgfqpoint{1.392586in}{0.973049in}}%
\pgfpathlineto{\pgfqpoint{1.402290in}{0.969835in}}%
\pgfpathlineto{\pgfqpoint{1.414609in}{0.965754in}}%
\pgfpathlineto{\pgfqpoint{1.426288in}{0.961886in}}%
\pgfpathlineto{\pgfqpoint{1.434666in}{0.959111in}}%
\pgfpathlineto{\pgfqpoint{1.442742in}{0.956436in}}%
\pgfpathlineto{\pgfqpoint{1.453078in}{0.953013in}}%
\pgfpathlineto{\pgfqpoint{1.465364in}{0.948943in}}%
\pgfpathlineto{\pgfqpoint{1.477013in}{0.945085in}}%
\pgfpathlineto{\pgfqpoint{1.485916in}{0.942136in}}%
\pgfpathlineto{\pgfqpoint{1.496570in}{0.938607in}}%
\pgfpathlineto{\pgfqpoint{1.506743in}{0.935238in}}%
\pgfpathlineto{\pgfqpoint{1.516475in}{0.932015in}}%
\pgfpathlineto{\pgfqpoint{1.527623in}{0.928322in}}%
\pgfpathlineto{\pgfqpoint{1.538244in}{0.924804in}}%
\pgfpathlineto{\pgfqpoint{1.548386in}{0.921445in}}%
\pgfpathlineto{\pgfqpoint{1.559667in}{0.917708in}}%
\pgfpathlineto{\pgfqpoint{1.570409in}{0.914151in}}%
\pgfpathlineto{\pgfqpoint{1.580661in}{0.910755in}}%
\pgfpathlineto{\pgfqpoint{1.590465in}{0.907507in}}%
\pgfpathlineto{\pgfqpoint{1.599860in}{0.904396in}}%
\pgfpathlineto{\pgfqpoint{1.611390in}{0.900577in}}%
\pgfpathlineto{\pgfqpoint{1.622356in}{0.896945in}}%
\pgfpathlineto{\pgfqpoint{1.631675in}{0.893858in}}%
\pgfpathlineto{\pgfqpoint{1.641716in}{0.890532in}}%
\pgfpathlineto{\pgfqpoint{1.652370in}{0.887003in}}%
\pgfpathlineto{\pgfqpoint{1.663535in}{0.883305in}}%
\pgfpathlineto{\pgfqpoint{1.674171in}{0.879782in}}%
\pgfpathlineto{\pgfqpoint{1.684327in}{0.876419in}}%
\pgfpathlineto{\pgfqpoint{1.694907in}{0.872914in}}%
\pgfpathlineto{\pgfqpoint{1.705010in}{0.869568in}}%
\pgfpathlineto{\pgfqpoint{1.715467in}{0.866105in}}%
\pgfpathlineto{\pgfqpoint{1.726209in}{0.862547in}}%
\pgfpathlineto{\pgfqpoint{1.736461in}{0.859151in}}%
\pgfpathlineto{\pgfqpoint{1.746950in}{0.855677in}}%
\pgfpathlineto{\pgfqpoint{1.756971in}{0.852358in}}%
\pgfpathlineto{\pgfqpoint{1.767189in}{0.848973in}}%
\pgfpathlineto{\pgfqpoint{1.778156in}{0.845341in}}%
\pgfpathlineto{\pgfqpoint{1.788612in}{0.841877in}}%
\pgfpathlineto{\pgfqpoint{1.798604in}{0.838568in}}%
\pgfpathlineto{\pgfqpoint{1.808690in}{0.835227in}}%
\pgfpathlineto{\pgfqpoint{1.819335in}{0.831701in}}%
\pgfpathlineto{\pgfqpoint{1.829971in}{0.828178in}}%
\pgfpathlineto{\pgfqpoint{1.840127in}{0.824815in}}%
\pgfpathlineto{\pgfqpoint{1.850275in}{0.821453in}}%
\pgfpathlineto{\pgfqpoint{1.860810in}{0.817964in}}%
\pgfpathlineto{\pgfqpoint{1.871266in}{0.814501in}}%
\pgfpathlineto{\pgfqpoint{1.881634in}{0.811067in}}%
\pgfpathlineto{\pgfqpoint{1.892260in}{0.807547in}}%
\pgfpathlineto{\pgfqpoint{1.902749in}{0.804073in}}%
\pgfpathlineto{\pgfqpoint{1.913097in}{0.800646in}}%
\pgfpathlineto{\pgfqpoint{1.923613in}{0.797163in}}%
\pgfpathlineto{\pgfqpoint{1.933956in}{0.793737in}}%
\pgfpathlineto{\pgfqpoint{1.944128in}{0.790367in}}%
\pgfpathlineto{\pgfqpoint{1.954404in}{0.786964in}}%
\pgfpathlineto{\pgfqpoint{1.965008in}{0.783452in}}%
\pgfpathlineto{\pgfqpoint{1.975629in}{0.779934in}}%
\pgfpathlineto{\pgfqpoint{1.986007in}{0.776496in}}%
\pgfpathlineto{\pgfqpoint{1.996378in}{0.773061in}}%
\pgfpathlineto{\pgfqpoint{2.006721in}{0.769635in}}%
\pgfpathlineto{\pgfqpoint{2.017021in}{0.766224in}}%
\pgfpathlineto{\pgfqpoint{2.027459in}{0.762767in}}%
\pgfpathlineto{\pgfqpoint{2.037808in}{0.759339in}}%
\pgfpathlineto{\pgfqpoint{2.048239in}{0.755884in}}%
\pgfpathlineto{\pgfqpoint{2.058720in}{0.752412in}}%
\pgfpathlineto{\pgfqpoint{2.069060in}{0.748988in}}%
\pgfpathlineto{\pgfqpoint{2.079412in}{0.745559in}}%
\pgfpathlineto{\pgfqpoint{2.089756in}{0.742133in}}%
\pgfpathlineto{\pgfqpoint{2.100212in}{0.738669in}}%
\pgfpathlineto{\pgfqpoint{2.110610in}{0.735225in}}%
\pgfpathlineto{\pgfqpoint{2.121067in}{0.731762in}}%
\pgfpathlineto{\pgfqpoint{2.131552in}{0.728289in}}%
\pgfpathlineto{\pgfqpoint{2.141925in}{0.724853in}}%
\pgfpathlineto{\pgfqpoint{2.152290in}{0.721420in}}%
\pgfpathlineto{\pgfqpoint{2.162629in}{0.717996in}}%
\pgfpathlineto{\pgfqpoint{2.173026in}{0.714552in}}%
\pgfpathlineto{\pgfqpoint{2.183455in}{0.711098in}}%
\pgfpathlineto{\pgfqpoint{2.193889in}{0.707642in}}%
\pgfpathlineto{\pgfqpoint{2.204307in}{0.704191in}}%
\pgfpathlineto{\pgfqpoint{2.214690in}{0.700752in}}%
\pgfpathlineto{\pgfqpoint{2.225104in}{0.697303in}}%
\pgfpathlineto{\pgfqpoint{2.235523in}{0.693852in}}%
\pgfpathlineto{\pgfqpoint{2.245927in}{0.690406in}}%
\pgfpathlineto{\pgfqpoint{2.256295in}{0.686971in}}%
\pgfpathlineto{\pgfqpoint{2.266681in}{0.683532in}}%
\pgfpathlineto{\pgfqpoint{2.277125in}{0.680072in}}%
\pgfpathlineto{\pgfqpoint{2.287537in}{0.676624in}}%
\pgfpathlineto{\pgfqpoint{2.297901in}{0.673191in}}%
\pgfpathlineto{\pgfqpoint{2.308315in}{0.669742in}}%
\pgfpathlineto{\pgfqpoint{2.313593in}{0.667993in}}%
\pgfusepath{stroke}%
\end{pgfscope}%
\begin{pgfscope}%
\pgfpathrectangle{\pgfqpoint{0.514278in}{0.417642in}}{\pgfqpoint{1.884996in}{1.371397in}}%
\pgfusepath{clip}%
\pgfsetbuttcap%
\pgfsetroundjoin%
\definecolor{currentfill}{rgb}{0.007843,0.619608,0.450980}%
\pgfsetfillcolor{currentfill}%
\pgfsetlinewidth{1.003750pt}%
\definecolor{currentstroke}{rgb}{0.007843,0.619608,0.450980}%
\pgfsetstrokecolor{currentstroke}%
\pgfsetdash{}{0pt}%
\pgfsys@defobject{currentmarker}{\pgfqpoint{-0.006944in}{-0.006944in}}{\pgfqpoint{0.006944in}{0.006944in}}{%
\pgfpathmoveto{\pgfqpoint{0.000000in}{-0.006944in}}%
\pgfpathcurveto{\pgfqpoint{0.001842in}{-0.006944in}}{\pgfqpoint{0.003608in}{-0.006213in}}{\pgfqpoint{0.004910in}{-0.004910in}}%
\pgfpathcurveto{\pgfqpoint{0.006213in}{-0.003608in}}{\pgfqpoint{0.006944in}{-0.001842in}}{\pgfqpoint{0.006944in}{0.000000in}}%
\pgfpathcurveto{\pgfqpoint{0.006944in}{0.001842in}}{\pgfqpoint{0.006213in}{0.003608in}}{\pgfqpoint{0.004910in}{0.004910in}}%
\pgfpathcurveto{\pgfqpoint{0.003608in}{0.006213in}}{\pgfqpoint{0.001842in}{0.006944in}}{\pgfqpoint{0.000000in}{0.006944in}}%
\pgfpathcurveto{\pgfqpoint{-0.001842in}{0.006944in}}{\pgfqpoint{-0.003608in}{0.006213in}}{\pgfqpoint{-0.004910in}{0.004910in}}%
\pgfpathcurveto{\pgfqpoint{-0.006213in}{0.003608in}}{\pgfqpoint{-0.006944in}{0.001842in}}{\pgfqpoint{-0.006944in}{0.000000in}}%
\pgfpathcurveto{\pgfqpoint{-0.006944in}{-0.001842in}}{\pgfqpoint{-0.006213in}{-0.003608in}}{\pgfqpoint{-0.004910in}{-0.004910in}}%
\pgfpathcurveto{\pgfqpoint{-0.003608in}{-0.006213in}}{\pgfqpoint{-0.001842in}{-0.006944in}}{\pgfqpoint{0.000000in}{-0.006944in}}%
\pgfpathlineto{\pgfqpoint{0.000000in}{-0.006944in}}%
\pgfpathclose%
\pgfusepath{stroke,fill}%
}%
\begin{pgfscope}%
\pgfsys@transformshift{0.599960in}{1.223928in}%
\pgfsys@useobject{currentmarker}{}%
\end{pgfscope}%
\begin{pgfscope}%
\pgfsys@transformshift{0.755760in}{1.133873in}%
\pgfsys@useobject{currentmarker}{}%
\end{pgfscope}%
\begin{pgfscope}%
\pgfsys@transformshift{0.846897in}{1.112772in}%
\pgfsys@useobject{currentmarker}{}%
\end{pgfscope}%
\begin{pgfscope}%
\pgfsys@transformshift{0.911560in}{1.087150in}%
\pgfsys@useobject{currentmarker}{}%
\end{pgfscope}%
\begin{pgfscope}%
\pgfsys@transformshift{0.961716in}{1.077979in}%
\pgfsys@useobject{currentmarker}{}%
\end{pgfscope}%
\begin{pgfscope}%
\pgfsys@transformshift{1.002697in}{1.028039in}%
\pgfsys@useobject{currentmarker}{}%
\end{pgfscope}%
\begin{pgfscope}%
\pgfsys@transformshift{1.037345in}{1.099417in}%
\pgfsys@useobject{currentmarker}{}%
\end{pgfscope}%
\begin{pgfscope}%
\pgfsys@transformshift{1.067360in}{1.110978in}%
\pgfsys@useobject{currentmarker}{}%
\end{pgfscope}%
\begin{pgfscope}%
\pgfsys@transformshift{1.093834in}{1.091707in}%
\pgfsys@useobject{currentmarker}{}%
\end{pgfscope}%
\begin{pgfscope}%
\pgfsys@transformshift{1.117516in}{1.074621in}%
\pgfsys@useobject{currentmarker}{}%
\end{pgfscope}%
\begin{pgfscope}%
\pgfsys@transformshift{1.138939in}{1.038922in}%
\pgfsys@useobject{currentmarker}{}%
\end{pgfscope}%
\begin{pgfscope}%
\pgfsys@transformshift{1.158497in}{1.072460in}%
\pgfsys@useobject{currentmarker}{}%
\end{pgfscope}%
\begin{pgfscope}%
\pgfsys@transformshift{1.176488in}{1.045045in}%
\pgfsys@useobject{currentmarker}{}%
\end{pgfscope}%
\begin{pgfscope}%
\pgfsys@transformshift{1.193145in}{0.976377in}%
\pgfsys@useobject{currentmarker}{}%
\end{pgfscope}%
\begin{pgfscope}%
\pgfsys@transformshift{1.208653in}{1.042403in}%
\pgfsys@useobject{currentmarker}{}%
\end{pgfscope}%
\begin{pgfscope}%
\pgfsys@transformshift{1.223159in}{1.066147in}%
\pgfsys@useobject{currentmarker}{}%
\end{pgfscope}%
\begin{pgfscope}%
\pgfsys@transformshift{1.236786in}{1.041717in}%
\pgfsys@useobject{currentmarker}{}%
\end{pgfscope}%
\begin{pgfscope}%
\pgfsys@transformshift{1.249634in}{0.934765in}%
\pgfsys@useobject{currentmarker}{}%
\end{pgfscope}%
\begin{pgfscope}%
\pgfsys@transformshift{1.261786in}{0.984899in}%
\pgfsys@useobject{currentmarker}{}%
\end{pgfscope}%
\begin{pgfscope}%
\pgfsys@transformshift{1.273316in}{1.050658in}%
\pgfsys@useobject{currentmarker}{}%
\end{pgfscope}%
\begin{pgfscope}%
\pgfsys@transformshift{1.284282in}{1.066325in}%
\pgfsys@useobject{currentmarker}{}%
\end{pgfscope}%
\begin{pgfscope}%
\pgfsys@transformshift{1.294739in}{1.028193in}%
\pgfsys@useobject{currentmarker}{}%
\end{pgfscope}%
\begin{pgfscope}%
\pgfsys@transformshift{1.309564in}{0.985648in}%
\pgfsys@useobject{currentmarker}{}%
\end{pgfscope}%
\begin{pgfscope}%
\pgfsys@transformshift{1.323472in}{0.981024in}%
\pgfsys@useobject{currentmarker}{}%
\end{pgfscope}%
\begin{pgfscope}%
\pgfsys@transformshift{1.332288in}{0.960523in}%
\pgfsys@useobject{currentmarker}{}%
\end{pgfscope}%
\begin{pgfscope}%
\pgfsys@transformshift{1.340771in}{0.954804in}%
\pgfsys@useobject{currentmarker}{}%
\end{pgfscope}%
\begin{pgfscope}%
\pgfsys@transformshift{1.348945in}{1.001306in}%
\pgfsys@useobject{currentmarker}{}%
\end{pgfscope}%
\begin{pgfscope}%
\pgfsys@transformshift{1.360675in}{0.988125in}%
\pgfsys@useobject{currentmarker}{}%
\end{pgfscope}%
\begin{pgfscope}%
\pgfsys@transformshift{1.371823in}{0.987385in}%
\pgfsys@useobject{currentmarker}{}%
\end{pgfscope}%
\begin{pgfscope}%
\pgfsys@transformshift{1.382444in}{0.963934in}%
\pgfsys@useobject{currentmarker}{}%
\end{pgfscope}%
\begin{pgfscope}%
\pgfsys@transformshift{1.392586in}{0.992313in}%
\pgfsys@useobject{currentmarker}{}%
\end{pgfscope}%
\begin{pgfscope}%
\pgfsys@transformshift{1.402290in}{0.971395in}%
\pgfsys@useobject{currentmarker}{}%
\end{pgfscope}%
\begin{pgfscope}%
\pgfsys@transformshift{1.414609in}{0.960609in}%
\pgfsys@useobject{currentmarker}{}%
\end{pgfscope}%
\begin{pgfscope}%
\pgfsys@transformshift{1.426288in}{0.928696in}%
\pgfsys@useobject{currentmarker}{}%
\end{pgfscope}%
\begin{pgfscope}%
\pgfsys@transformshift{1.434666in}{0.975645in}%
\pgfsys@useobject{currentmarker}{}%
\end{pgfscope}%
\begin{pgfscope}%
\pgfsys@transformshift{1.442742in}{0.996958in}%
\pgfsys@useobject{currentmarker}{}%
\end{pgfscope}%
\begin{pgfscope}%
\pgfsys@transformshift{1.453078in}{0.924735in}%
\pgfsys@useobject{currentmarker}{}%
\end{pgfscope}%
\begin{pgfscope}%
\pgfsys@transformshift{1.465364in}{0.930732in}%
\pgfsys@useobject{currentmarker}{}%
\end{pgfscope}%
\begin{pgfscope}%
\pgfsys@transformshift{1.477013in}{0.920723in}%
\pgfsys@useobject{currentmarker}{}%
\end{pgfscope}%
\begin{pgfscope}%
\pgfsys@transformshift{1.485916in}{0.857641in}%
\pgfsys@useobject{currentmarker}{}%
\end{pgfscope}%
\begin{pgfscope}%
\pgfsys@transformshift{1.496570in}{0.947202in}%
\pgfsys@useobject{currentmarker}{}%
\end{pgfscope}%
\begin{pgfscope}%
\pgfsys@transformshift{1.506743in}{0.966057in}%
\pgfsys@useobject{currentmarker}{}%
\end{pgfscope}%
\begin{pgfscope}%
\pgfsys@transformshift{1.516475in}{0.944591in}%
\pgfsys@useobject{currentmarker}{}%
\end{pgfscope}%
\begin{pgfscope}%
\pgfsys@transformshift{1.527623in}{0.885441in}%
\pgfsys@useobject{currentmarker}{}%
\end{pgfscope}%
\begin{pgfscope}%
\pgfsys@transformshift{1.538244in}{0.885741in}%
\pgfsys@useobject{currentmarker}{}%
\end{pgfscope}%
\begin{pgfscope}%
\pgfsys@transformshift{1.548386in}{0.925663in}%
\pgfsys@useobject{currentmarker}{}%
\end{pgfscope}%
\begin{pgfscope}%
\pgfsys@transformshift{1.559667in}{0.910685in}%
\pgfsys@useobject{currentmarker}{}%
\end{pgfscope}%
\begin{pgfscope}%
\pgfsys@transformshift{1.570409in}{0.909248in}%
\pgfsys@useobject{currentmarker}{}%
\end{pgfscope}%
\begin{pgfscope}%
\pgfsys@transformshift{1.580661in}{0.886260in}%
\pgfsys@useobject{currentmarker}{}%
\end{pgfscope}%
\begin{pgfscope}%
\pgfsys@transformshift{1.590465in}{0.888013in}%
\pgfsys@useobject{currentmarker}{}%
\end{pgfscope}%
\begin{pgfscope}%
\pgfsys@transformshift{1.599860in}{0.860927in}%
\pgfsys@useobject{currentmarker}{}%
\end{pgfscope}%
\begin{pgfscope}%
\pgfsys@transformshift{1.611390in}{0.869364in}%
\pgfsys@useobject{currentmarker}{}%
\end{pgfscope}%
\begin{pgfscope}%
\pgfsys@transformshift{1.622356in}{0.884202in}%
\pgfsys@useobject{currentmarker}{}%
\end{pgfscope}%
\begin{pgfscope}%
\pgfsys@transformshift{1.631675in}{0.862991in}%
\pgfsys@useobject{currentmarker}{}%
\end{pgfscope}%
\begin{pgfscope}%
\pgfsys@transformshift{1.641716in}{0.872746in}%
\pgfsys@useobject{currentmarker}{}%
\end{pgfscope}%
\begin{pgfscope}%
\pgfsys@transformshift{1.652370in}{0.892831in}%
\pgfsys@useobject{currentmarker}{}%
\end{pgfscope}%
\begin{pgfscope}%
\pgfsys@transformshift{1.663535in}{0.864944in}%
\pgfsys@useobject{currentmarker}{}%
\end{pgfscope}%
\begin{pgfscope}%
\pgfsys@transformshift{1.674171in}{0.868548in}%
\pgfsys@useobject{currentmarker}{}%
\end{pgfscope}%
\begin{pgfscope}%
\pgfsys@transformshift{1.684327in}{0.861120in}%
\pgfsys@useobject{currentmarker}{}%
\end{pgfscope}%
\begin{pgfscope}%
\pgfsys@transformshift{1.694907in}{0.866134in}%
\pgfsys@useobject{currentmarker}{}%
\end{pgfscope}%
\begin{pgfscope}%
\pgfsys@transformshift{1.705010in}{0.893510in}%
\pgfsys@useobject{currentmarker}{}%
\end{pgfscope}%
\begin{pgfscope}%
\pgfsys@transformshift{1.715467in}{0.866045in}%
\pgfsys@useobject{currentmarker}{}%
\end{pgfscope}%
\begin{pgfscope}%
\pgfsys@transformshift{1.726209in}{0.867177in}%
\pgfsys@useobject{currentmarker}{}%
\end{pgfscope}%
\begin{pgfscope}%
\pgfsys@transformshift{1.736461in}{0.869886in}%
\pgfsys@useobject{currentmarker}{}%
\end{pgfscope}%
\begin{pgfscope}%
\pgfsys@transformshift{1.746950in}{0.847004in}%
\pgfsys@useobject{currentmarker}{}%
\end{pgfscope}%
\begin{pgfscope}%
\pgfsys@transformshift{1.756971in}{0.788155in}%
\pgfsys@useobject{currentmarker}{}%
\end{pgfscope}%
\begin{pgfscope}%
\pgfsys@transformshift{1.767189in}{0.852985in}%
\pgfsys@useobject{currentmarker}{}%
\end{pgfscope}%
\begin{pgfscope}%
\pgfsys@transformshift{1.778156in}{0.861807in}%
\pgfsys@useobject{currentmarker}{}%
\end{pgfscope}%
\begin{pgfscope}%
\pgfsys@transformshift{1.788612in}{0.869935in}%
\pgfsys@useobject{currentmarker}{}%
\end{pgfscope}%
\begin{pgfscope}%
\pgfsys@transformshift{1.798604in}{0.831057in}%
\pgfsys@useobject{currentmarker}{}%
\end{pgfscope}%
\begin{pgfscope}%
\pgfsys@transformshift{1.808690in}{0.826776in}%
\pgfsys@useobject{currentmarker}{}%
\end{pgfscope}%
\begin{pgfscope}%
\pgfsys@transformshift{1.819335in}{0.807243in}%
\pgfsys@useobject{currentmarker}{}%
\end{pgfscope}%
\begin{pgfscope}%
\pgfsys@transformshift{1.829971in}{0.831039in}%
\pgfsys@useobject{currentmarker}{}%
\end{pgfscope}%
\begin{pgfscope}%
\pgfsys@transformshift{1.840127in}{0.820789in}%
\pgfsys@useobject{currentmarker}{}%
\end{pgfscope}%
\begin{pgfscope}%
\pgfsys@transformshift{1.850275in}{0.835396in}%
\pgfsys@useobject{currentmarker}{}%
\end{pgfscope}%
\begin{pgfscope}%
\pgfsys@transformshift{1.860810in}{0.795415in}%
\pgfsys@useobject{currentmarker}{}%
\end{pgfscope}%
\begin{pgfscope}%
\pgfsys@transformshift{1.871266in}{0.804096in}%
\pgfsys@useobject{currentmarker}{}%
\end{pgfscope}%
\begin{pgfscope}%
\pgfsys@transformshift{1.881634in}{0.811750in}%
\pgfsys@useobject{currentmarker}{}%
\end{pgfscope}%
\begin{pgfscope}%
\pgfsys@transformshift{1.892260in}{0.812710in}%
\pgfsys@useobject{currentmarker}{}%
\end{pgfscope}%
\begin{pgfscope}%
\pgfsys@transformshift{1.902749in}{0.830434in}%
\pgfsys@useobject{currentmarker}{}%
\end{pgfscope}%
\begin{pgfscope}%
\pgfsys@transformshift{1.913097in}{0.792247in}%
\pgfsys@useobject{currentmarker}{}%
\end{pgfscope}%
\begin{pgfscope}%
\pgfsys@transformshift{1.923613in}{0.782758in}%
\pgfsys@useobject{currentmarker}{}%
\end{pgfscope}%
\begin{pgfscope}%
\pgfsys@transformshift{1.933956in}{0.776822in}%
\pgfsys@useobject{currentmarker}{}%
\end{pgfscope}%
\begin{pgfscope}%
\pgfsys@transformshift{1.944128in}{0.791340in}%
\pgfsys@useobject{currentmarker}{}%
\end{pgfscope}%
\begin{pgfscope}%
\pgfsys@transformshift{1.954404in}{0.800167in}%
\pgfsys@useobject{currentmarker}{}%
\end{pgfscope}%
\begin{pgfscope}%
\pgfsys@transformshift{1.965008in}{0.784040in}%
\pgfsys@useobject{currentmarker}{}%
\end{pgfscope}%
\begin{pgfscope}%
\pgfsys@transformshift{1.975629in}{0.774833in}%
\pgfsys@useobject{currentmarker}{}%
\end{pgfscope}%
\begin{pgfscope}%
\pgfsys@transformshift{1.986007in}{0.778107in}%
\pgfsys@useobject{currentmarker}{}%
\end{pgfscope}%
\begin{pgfscope}%
\pgfsys@transformshift{1.996378in}{0.778122in}%
\pgfsys@useobject{currentmarker}{}%
\end{pgfscope}%
\begin{pgfscope}%
\pgfsys@transformshift{2.006721in}{0.767873in}%
\pgfsys@useobject{currentmarker}{}%
\end{pgfscope}%
\begin{pgfscope}%
\pgfsys@transformshift{2.017021in}{0.771471in}%
\pgfsys@useobject{currentmarker}{}%
\end{pgfscope}%
\begin{pgfscope}%
\pgfsys@transformshift{2.027459in}{0.768142in}%
\pgfsys@useobject{currentmarker}{}%
\end{pgfscope}%
\begin{pgfscope}%
\pgfsys@transformshift{2.037808in}{0.766022in}%
\pgfsys@useobject{currentmarker}{}%
\end{pgfscope}%
\begin{pgfscope}%
\pgfsys@transformshift{2.048239in}{0.767397in}%
\pgfsys@useobject{currentmarker}{}%
\end{pgfscope}%
\begin{pgfscope}%
\pgfsys@transformshift{2.058720in}{0.757491in}%
\pgfsys@useobject{currentmarker}{}%
\end{pgfscope}%
\begin{pgfscope}%
\pgfsys@transformshift{2.069060in}{0.747619in}%
\pgfsys@useobject{currentmarker}{}%
\end{pgfscope}%
\begin{pgfscope}%
\pgfsys@transformshift{2.079412in}{0.753371in}%
\pgfsys@useobject{currentmarker}{}%
\end{pgfscope}%
\begin{pgfscope}%
\pgfsys@transformshift{2.089756in}{0.744035in}%
\pgfsys@useobject{currentmarker}{}%
\end{pgfscope}%
\begin{pgfscope}%
\pgfsys@transformshift{2.100212in}{0.758842in}%
\pgfsys@useobject{currentmarker}{}%
\end{pgfscope}%
\begin{pgfscope}%
\pgfsys@transformshift{2.110610in}{0.746118in}%
\pgfsys@useobject{currentmarker}{}%
\end{pgfscope}%
\begin{pgfscope}%
\pgfsys@transformshift{2.121067in}{0.740097in}%
\pgfsys@useobject{currentmarker}{}%
\end{pgfscope}%
\begin{pgfscope}%
\pgfsys@transformshift{2.131552in}{0.734766in}%
\pgfsys@useobject{currentmarker}{}%
\end{pgfscope}%
\begin{pgfscope}%
\pgfsys@transformshift{2.141925in}{0.734322in}%
\pgfsys@useobject{currentmarker}{}%
\end{pgfscope}%
\begin{pgfscope}%
\pgfsys@transformshift{2.152290in}{0.727832in}%
\pgfsys@useobject{currentmarker}{}%
\end{pgfscope}%
\begin{pgfscope}%
\pgfsys@transformshift{2.162629in}{0.725064in}%
\pgfsys@useobject{currentmarker}{}%
\end{pgfscope}%
\begin{pgfscope}%
\pgfsys@transformshift{2.173026in}{0.724340in}%
\pgfsys@useobject{currentmarker}{}%
\end{pgfscope}%
\begin{pgfscope}%
\pgfsys@transformshift{2.183455in}{0.726605in}%
\pgfsys@useobject{currentmarker}{}%
\end{pgfscope}%
\begin{pgfscope}%
\pgfsys@transformshift{2.193889in}{0.723148in}%
\pgfsys@useobject{currentmarker}{}%
\end{pgfscope}%
\begin{pgfscope}%
\pgfsys@transformshift{2.204307in}{0.708706in}%
\pgfsys@useobject{currentmarker}{}%
\end{pgfscope}%
\begin{pgfscope}%
\pgfsys@transformshift{2.214690in}{0.703887in}%
\pgfsys@useobject{currentmarker}{}%
\end{pgfscope}%
\begin{pgfscope}%
\pgfsys@transformshift{2.225104in}{0.715935in}%
\pgfsys@useobject{currentmarker}{}%
\end{pgfscope}%
\begin{pgfscope}%
\pgfsys@transformshift{2.235523in}{0.717164in}%
\pgfsys@useobject{currentmarker}{}%
\end{pgfscope}%
\begin{pgfscope}%
\pgfsys@transformshift{2.245927in}{0.702726in}%
\pgfsys@useobject{currentmarker}{}%
\end{pgfscope}%
\begin{pgfscope}%
\pgfsys@transformshift{2.256295in}{0.705242in}%
\pgfsys@useobject{currentmarker}{}%
\end{pgfscope}%
\begin{pgfscope}%
\pgfsys@transformshift{2.266681in}{0.709932in}%
\pgfsys@useobject{currentmarker}{}%
\end{pgfscope}%
\begin{pgfscope}%
\pgfsys@transformshift{2.277125in}{0.705993in}%
\pgfsys@useobject{currentmarker}{}%
\end{pgfscope}%
\begin{pgfscope}%
\pgfsys@transformshift{2.287537in}{0.712018in}%
\pgfsys@useobject{currentmarker}{}%
\end{pgfscope}%
\begin{pgfscope}%
\pgfsys@transformshift{2.297901in}{0.696968in}%
\pgfsys@useobject{currentmarker}{}%
\end{pgfscope}%
\begin{pgfscope}%
\pgfsys@transformshift{2.308315in}{0.697962in}%
\pgfsys@useobject{currentmarker}{}%
\end{pgfscope}%
\begin{pgfscope}%
\pgfsys@transformshift{2.313593in}{0.684136in}%
\pgfsys@useobject{currentmarker}{}%
\end{pgfscope}%
\end{pgfscope}%
\begin{pgfscope}%
\pgfsetrectcap%
\pgfsetmiterjoin%
\pgfsetlinewidth{0.803000pt}%
\definecolor{currentstroke}{rgb}{0.000000,0.000000,0.000000}%
\pgfsetstrokecolor{currentstroke}%
\pgfsetdash{}{0pt}%
\pgfpathmoveto{\pgfqpoint{0.514278in}{0.417642in}}%
\pgfpathlineto{\pgfqpoint{0.514278in}{1.789039in}}%
\pgfusepath{stroke}%
\end{pgfscope}%
\begin{pgfscope}%
\pgfsetrectcap%
\pgfsetmiterjoin%
\pgfsetlinewidth{0.803000pt}%
\definecolor{currentstroke}{rgb}{0.000000,0.000000,0.000000}%
\pgfsetstrokecolor{currentstroke}%
\pgfsetdash{}{0pt}%
\pgfpathmoveto{\pgfqpoint{2.399275in}{0.417642in}}%
\pgfpathlineto{\pgfqpoint{2.399275in}{1.789039in}}%
\pgfusepath{stroke}%
\end{pgfscope}%
\begin{pgfscope}%
\pgfsetrectcap%
\pgfsetmiterjoin%
\pgfsetlinewidth{0.803000pt}%
\definecolor{currentstroke}{rgb}{0.000000,0.000000,0.000000}%
\pgfsetstrokecolor{currentstroke}%
\pgfsetdash{}{0pt}%
\pgfpathmoveto{\pgfqpoint{0.514278in}{0.417642in}}%
\pgfpathlineto{\pgfqpoint{2.399275in}{0.417642in}}%
\pgfusepath{stroke}%
\end{pgfscope}%
\begin{pgfscope}%
\pgfsetrectcap%
\pgfsetmiterjoin%
\pgfsetlinewidth{0.803000pt}%
\definecolor{currentstroke}{rgb}{0.000000,0.000000,0.000000}%
\pgfsetstrokecolor{currentstroke}%
\pgfsetdash{}{0pt}%
\pgfpathmoveto{\pgfqpoint{0.514278in}{1.789039in}}%
\pgfpathlineto{\pgfqpoint{2.399275in}{1.789039in}}%
\pgfusepath{stroke}%
\end{pgfscope}%
\begin{pgfscope}%
\pgfsetbuttcap%
\pgfsetmiterjoin%
\definecolor{currentfill}{rgb}{1.000000,1.000000,1.000000}%
\pgfsetfillcolor{currentfill}%
\pgfsetfillopacity{0.800000}%
\pgfsetlinewidth{1.003750pt}%
\definecolor{currentstroke}{rgb}{0.800000,0.800000,0.800000}%
\pgfsetstrokecolor{currentstroke}%
\pgfsetstrokeopacity{0.800000}%
\pgfsetdash{}{0pt}%
\pgfpathmoveto{\pgfqpoint{1.552717in}{1.517728in}}%
\pgfpathlineto{\pgfqpoint{2.321497in}{1.517728in}}%
\pgfpathquadraticcurveto{\pgfqpoint{2.343719in}{1.517728in}}{\pgfqpoint{2.343719in}{1.539950in}}%
\pgfpathlineto{\pgfqpoint{2.343719in}{1.711261in}}%
\pgfpathquadraticcurveto{\pgfqpoint{2.343719in}{1.733483in}}{\pgfqpoint{2.321497in}{1.733483in}}%
\pgfpathlineto{\pgfqpoint{1.552717in}{1.733483in}}%
\pgfpathquadraticcurveto{\pgfqpoint{1.530494in}{1.733483in}}{\pgfqpoint{1.530494in}{1.711261in}}%
\pgfpathlineto{\pgfqpoint{1.530494in}{1.539950in}}%
\pgfpathquadraticcurveto{\pgfqpoint{1.530494in}{1.517728in}}{\pgfqpoint{1.552717in}{1.517728in}}%
\pgfpathlineto{\pgfqpoint{1.552717in}{1.517728in}}%
\pgfpathclose%
\pgfusepath{stroke,fill}%
\end{pgfscope}%
\begin{pgfscope}%
\pgfsetbuttcap%
\pgfsetroundjoin%
\pgfsetlinewidth{1.505625pt}%
\definecolor{currentstroke}{rgb}{0.007843,0.619608,0.450980}%
\pgfsetstrokecolor{currentstroke}%
\pgfsetdash{{5.550000pt}{2.400000pt}}{0.000000pt}%
\pgfpathmoveto{\pgfqpoint{1.574939in}{1.628067in}}%
\pgfpathlineto{\pgfqpoint{1.686050in}{1.628067in}}%
\pgfpathlineto{\pgfqpoint{1.797161in}{1.628067in}}%
\pgfusepath{stroke}%
\end{pgfscope}%
\begin{pgfscope}%
\definecolor{textcolor}{rgb}{0.000000,0.000000,0.000000}%
\pgfsetstrokecolor{textcolor}%
\pgfsetfillcolor{textcolor}%
\pgftext[x=1.886050in,y=1.589178in,left,base]{\color{textcolor}\rmfamily\fontsize{8.000000}{9.600000}\selectfont \(\displaystyle h_{-1}f^{-1}\)}%
\end{pgfscope}%
\end{pgfpicture}%
\makeatother%
\endgroup%
% data/simulations/sim_allan_variance.py
        } % scalebox
        \caption{Power spectral density}
        \label{fig:flicker_noise_psd}
    \end{subfigure}
    \begin{subfigure}{0.32\linewidth}
        \centering
        \scalebox{0.75}{%
            %% Creator: Matplotlib, PGF backend
%%
%% To include the figure in your LaTeX document, write
%%   \input{<filename>.pgf}
%%
%% Make sure the required packages are loaded in your preamble
%%   \usepackage{pgf}
%%
%% Also ensure that all the required font packages are loaded; for instance,
%% the lmodern package is sometimes necessary when using math font.
%%   \usepackage{lmodern}
%%
%% Figures using additional raster images can only be included by \input if
%% they are in the same directory as the main LaTeX file. For loading figures
%% from other directories you can use the `import` package
%%   \usepackage{import}
%%
%% and then include the figures with
%%   \import{<path to file>}{<filename>.pgf}
%%
%% Matplotlib used the following preamble
%%   \usepackage{siunitx}
%%   \usepackage{fontspec}
%%   \makeatletter\@ifpackageloaded{underscore}{}{\usepackage[strings]{underscore}}\makeatother
%%
\begingroup%
\makeatletter%
\begin{pgfpicture}%
\pgfpathrectangle{\pgfpointorigin}{\pgfqpoint{2.440000in}{1.830000in}}%
\pgfusepath{use as bounding box, clip}%
\begin{pgfscope}%
\pgfsetbuttcap%
\pgfsetmiterjoin%
\definecolor{currentfill}{rgb}{1.000000,1.000000,1.000000}%
\pgfsetfillcolor{currentfill}%
\pgfsetlinewidth{0.000000pt}%
\definecolor{currentstroke}{rgb}{1.000000,1.000000,1.000000}%
\pgfsetstrokecolor{currentstroke}%
\pgfsetdash{}{0pt}%
\pgfpathmoveto{\pgfqpoint{0.000000in}{0.000000in}}%
\pgfpathlineto{\pgfqpoint{2.440000in}{0.000000in}}%
\pgfpathlineto{\pgfqpoint{2.440000in}{1.830000in}}%
\pgfpathlineto{\pgfqpoint{0.000000in}{1.830000in}}%
\pgfpathlineto{\pgfqpoint{0.000000in}{0.000000in}}%
\pgfpathclose%
\pgfusepath{fill}%
\end{pgfscope}%
\begin{pgfscope}%
\pgfsetbuttcap%
\pgfsetmiterjoin%
\definecolor{currentfill}{rgb}{1.000000,1.000000,1.000000}%
\pgfsetfillcolor{currentfill}%
\pgfsetlinewidth{0.000000pt}%
\definecolor{currentstroke}{rgb}{0.000000,0.000000,0.000000}%
\pgfsetstrokecolor{currentstroke}%
\pgfsetstrokeopacity{0.000000}%
\pgfsetdash{}{0pt}%
\pgfpathmoveto{\pgfqpoint{0.589510in}{0.417642in}}%
\pgfpathlineto{\pgfqpoint{2.398330in}{0.417642in}}%
\pgfpathlineto{\pgfqpoint{2.398330in}{1.788330in}}%
\pgfpathlineto{\pgfqpoint{0.589510in}{1.788330in}}%
\pgfpathlineto{\pgfqpoint{0.589510in}{0.417642in}}%
\pgfpathclose%
\pgfusepath{fill}%
\end{pgfscope}%
\begin{pgfscope}%
\pgfpathrectangle{\pgfqpoint{0.589510in}{0.417642in}}{\pgfqpoint{1.808820in}{1.370688in}}%
\pgfusepath{clip}%
\pgfsetrectcap%
\pgfsetroundjoin%
\pgfsetlinewidth{0.803000pt}%
\definecolor{currentstroke}{rgb}{0.450000,0.450000,0.450000}%
\pgfsetstrokecolor{currentstroke}%
\pgfsetdash{}{0pt}%
\pgfpathmoveto{\pgfqpoint{0.671729in}{0.417642in}}%
\pgfpathlineto{\pgfqpoint{0.671729in}{1.788330in}}%
\pgfusepath{stroke}%
\end{pgfscope}%
\begin{pgfscope}%
\pgfsetbuttcap%
\pgfsetroundjoin%
\definecolor{currentfill}{rgb}{0.000000,0.000000,0.000000}%
\pgfsetfillcolor{currentfill}%
\pgfsetlinewidth{0.803000pt}%
\definecolor{currentstroke}{rgb}{0.000000,0.000000,0.000000}%
\pgfsetstrokecolor{currentstroke}%
\pgfsetdash{}{0pt}%
\pgfsys@defobject{currentmarker}{\pgfqpoint{0.000000in}{-0.048611in}}{\pgfqpoint{0.000000in}{0.000000in}}{%
\pgfpathmoveto{\pgfqpoint{0.000000in}{0.000000in}}%
\pgfpathlineto{\pgfqpoint{0.000000in}{-0.048611in}}%
\pgfusepath{stroke,fill}%
}%
\begin{pgfscope}%
\pgfsys@transformshift{0.671729in}{0.417642in}%
\pgfsys@useobject{currentmarker}{}%
\end{pgfscope}%
\end{pgfscope}%
\begin{pgfscope}%
\definecolor{textcolor}{rgb}{0.000000,0.000000,0.000000}%
\pgfsetstrokecolor{textcolor}%
\pgfsetfillcolor{textcolor}%
\pgftext[x=0.671729in,y=0.320420in,,top]{\color{textcolor}\rmfamily\fontsize{8.000000}{9.600000}\selectfont \(\displaystyle {10^{0}}\)}%
\end{pgfscope}%
\begin{pgfscope}%
\pgfpathrectangle{\pgfqpoint{0.589510in}{0.417642in}}{\pgfqpoint{1.808820in}{1.370688in}}%
\pgfusepath{clip}%
\pgfsetrectcap%
\pgfsetroundjoin%
\pgfsetlinewidth{0.803000pt}%
\definecolor{currentstroke}{rgb}{0.450000,0.450000,0.450000}%
\pgfsetstrokecolor{currentstroke}%
\pgfsetdash{}{0pt}%
\pgfpathmoveto{\pgfqpoint{1.128240in}{0.417642in}}%
\pgfpathlineto{\pgfqpoint{1.128240in}{1.788330in}}%
\pgfusepath{stroke}%
\end{pgfscope}%
\begin{pgfscope}%
\pgfsetbuttcap%
\pgfsetroundjoin%
\definecolor{currentfill}{rgb}{0.000000,0.000000,0.000000}%
\pgfsetfillcolor{currentfill}%
\pgfsetlinewidth{0.803000pt}%
\definecolor{currentstroke}{rgb}{0.000000,0.000000,0.000000}%
\pgfsetstrokecolor{currentstroke}%
\pgfsetdash{}{0pt}%
\pgfsys@defobject{currentmarker}{\pgfqpoint{0.000000in}{-0.048611in}}{\pgfqpoint{0.000000in}{0.000000in}}{%
\pgfpathmoveto{\pgfqpoint{0.000000in}{0.000000in}}%
\pgfpathlineto{\pgfqpoint{0.000000in}{-0.048611in}}%
\pgfusepath{stroke,fill}%
}%
\begin{pgfscope}%
\pgfsys@transformshift{1.128240in}{0.417642in}%
\pgfsys@useobject{currentmarker}{}%
\end{pgfscope}%
\end{pgfscope}%
\begin{pgfscope}%
\definecolor{textcolor}{rgb}{0.000000,0.000000,0.000000}%
\pgfsetstrokecolor{textcolor}%
\pgfsetfillcolor{textcolor}%
\pgftext[x=1.128240in,y=0.320420in,,top]{\color{textcolor}\rmfamily\fontsize{8.000000}{9.600000}\selectfont \(\displaystyle {10^{1}}\)}%
\end{pgfscope}%
\begin{pgfscope}%
\pgfpathrectangle{\pgfqpoint{0.589510in}{0.417642in}}{\pgfqpoint{1.808820in}{1.370688in}}%
\pgfusepath{clip}%
\pgfsetrectcap%
\pgfsetroundjoin%
\pgfsetlinewidth{0.803000pt}%
\definecolor{currentstroke}{rgb}{0.450000,0.450000,0.450000}%
\pgfsetstrokecolor{currentstroke}%
\pgfsetdash{}{0pt}%
\pgfpathmoveto{\pgfqpoint{1.584752in}{0.417642in}}%
\pgfpathlineto{\pgfqpoint{1.584752in}{1.788330in}}%
\pgfusepath{stroke}%
\end{pgfscope}%
\begin{pgfscope}%
\pgfsetbuttcap%
\pgfsetroundjoin%
\definecolor{currentfill}{rgb}{0.000000,0.000000,0.000000}%
\pgfsetfillcolor{currentfill}%
\pgfsetlinewidth{0.803000pt}%
\definecolor{currentstroke}{rgb}{0.000000,0.000000,0.000000}%
\pgfsetstrokecolor{currentstroke}%
\pgfsetdash{}{0pt}%
\pgfsys@defobject{currentmarker}{\pgfqpoint{0.000000in}{-0.048611in}}{\pgfqpoint{0.000000in}{0.000000in}}{%
\pgfpathmoveto{\pgfqpoint{0.000000in}{0.000000in}}%
\pgfpathlineto{\pgfqpoint{0.000000in}{-0.048611in}}%
\pgfusepath{stroke,fill}%
}%
\begin{pgfscope}%
\pgfsys@transformshift{1.584752in}{0.417642in}%
\pgfsys@useobject{currentmarker}{}%
\end{pgfscope}%
\end{pgfscope}%
\begin{pgfscope}%
\definecolor{textcolor}{rgb}{0.000000,0.000000,0.000000}%
\pgfsetstrokecolor{textcolor}%
\pgfsetfillcolor{textcolor}%
\pgftext[x=1.584752in,y=0.320420in,,top]{\color{textcolor}\rmfamily\fontsize{8.000000}{9.600000}\selectfont \(\displaystyle {10^{2}}\)}%
\end{pgfscope}%
\begin{pgfscope}%
\pgfpathrectangle{\pgfqpoint{0.589510in}{0.417642in}}{\pgfqpoint{1.808820in}{1.370688in}}%
\pgfusepath{clip}%
\pgfsetrectcap%
\pgfsetroundjoin%
\pgfsetlinewidth{0.803000pt}%
\definecolor{currentstroke}{rgb}{0.450000,0.450000,0.450000}%
\pgfsetstrokecolor{currentstroke}%
\pgfsetdash{}{0pt}%
\pgfpathmoveto{\pgfqpoint{2.041264in}{0.417642in}}%
\pgfpathlineto{\pgfqpoint{2.041264in}{1.788330in}}%
\pgfusepath{stroke}%
\end{pgfscope}%
\begin{pgfscope}%
\pgfsetbuttcap%
\pgfsetroundjoin%
\definecolor{currentfill}{rgb}{0.000000,0.000000,0.000000}%
\pgfsetfillcolor{currentfill}%
\pgfsetlinewidth{0.803000pt}%
\definecolor{currentstroke}{rgb}{0.000000,0.000000,0.000000}%
\pgfsetstrokecolor{currentstroke}%
\pgfsetdash{}{0pt}%
\pgfsys@defobject{currentmarker}{\pgfqpoint{0.000000in}{-0.048611in}}{\pgfqpoint{0.000000in}{0.000000in}}{%
\pgfpathmoveto{\pgfqpoint{0.000000in}{0.000000in}}%
\pgfpathlineto{\pgfqpoint{0.000000in}{-0.048611in}}%
\pgfusepath{stroke,fill}%
}%
\begin{pgfscope}%
\pgfsys@transformshift{2.041264in}{0.417642in}%
\pgfsys@useobject{currentmarker}{}%
\end{pgfscope}%
\end{pgfscope}%
\begin{pgfscope}%
\definecolor{textcolor}{rgb}{0.000000,0.000000,0.000000}%
\pgfsetstrokecolor{textcolor}%
\pgfsetfillcolor{textcolor}%
\pgftext[x=2.041264in,y=0.320420in,,top]{\color{textcolor}\rmfamily\fontsize{8.000000}{9.600000}\selectfont \(\displaystyle {10^{3}}\)}%
\end{pgfscope}%
\begin{pgfscope}%
\pgfpathrectangle{\pgfqpoint{0.589510in}{0.417642in}}{\pgfqpoint{1.808820in}{1.370688in}}%
\pgfusepath{clip}%
\pgfsetrectcap%
\pgfsetroundjoin%
\pgfsetlinewidth{0.803000pt}%
\definecolor{currentstroke}{rgb}{0.850000,0.850000,0.850000}%
\pgfsetstrokecolor{currentstroke}%
\pgfsetdash{}{0pt}%
\pgfpathmoveto{\pgfqpoint{0.601014in}{0.417642in}}%
\pgfpathlineto{\pgfqpoint{0.601014in}{1.788330in}}%
\pgfusepath{stroke}%
\end{pgfscope}%
\begin{pgfscope}%
\pgfsetbuttcap%
\pgfsetroundjoin%
\definecolor{currentfill}{rgb}{0.000000,0.000000,0.000000}%
\pgfsetfillcolor{currentfill}%
\pgfsetlinewidth{0.602250pt}%
\definecolor{currentstroke}{rgb}{0.000000,0.000000,0.000000}%
\pgfsetstrokecolor{currentstroke}%
\pgfsetdash{}{0pt}%
\pgfsys@defobject{currentmarker}{\pgfqpoint{0.000000in}{-0.027778in}}{\pgfqpoint{0.000000in}{0.000000in}}{%
\pgfpathmoveto{\pgfqpoint{0.000000in}{0.000000in}}%
\pgfpathlineto{\pgfqpoint{0.000000in}{-0.027778in}}%
\pgfusepath{stroke,fill}%
}%
\begin{pgfscope}%
\pgfsys@transformshift{0.601014in}{0.417642in}%
\pgfsys@useobject{currentmarker}{}%
\end{pgfscope}%
\end{pgfscope}%
\begin{pgfscope}%
\pgfpathrectangle{\pgfqpoint{0.589510in}{0.417642in}}{\pgfqpoint{1.808820in}{1.370688in}}%
\pgfusepath{clip}%
\pgfsetrectcap%
\pgfsetroundjoin%
\pgfsetlinewidth{0.803000pt}%
\definecolor{currentstroke}{rgb}{0.850000,0.850000,0.850000}%
\pgfsetstrokecolor{currentstroke}%
\pgfsetdash{}{0pt}%
\pgfpathmoveto{\pgfqpoint{0.627488in}{0.417642in}}%
\pgfpathlineto{\pgfqpoint{0.627488in}{1.788330in}}%
\pgfusepath{stroke}%
\end{pgfscope}%
\begin{pgfscope}%
\pgfsetbuttcap%
\pgfsetroundjoin%
\definecolor{currentfill}{rgb}{0.000000,0.000000,0.000000}%
\pgfsetfillcolor{currentfill}%
\pgfsetlinewidth{0.602250pt}%
\definecolor{currentstroke}{rgb}{0.000000,0.000000,0.000000}%
\pgfsetstrokecolor{currentstroke}%
\pgfsetdash{}{0pt}%
\pgfsys@defobject{currentmarker}{\pgfqpoint{0.000000in}{-0.027778in}}{\pgfqpoint{0.000000in}{0.000000in}}{%
\pgfpathmoveto{\pgfqpoint{0.000000in}{0.000000in}}%
\pgfpathlineto{\pgfqpoint{0.000000in}{-0.027778in}}%
\pgfusepath{stroke,fill}%
}%
\begin{pgfscope}%
\pgfsys@transformshift{0.627488in}{0.417642in}%
\pgfsys@useobject{currentmarker}{}%
\end{pgfscope}%
\end{pgfscope}%
\begin{pgfscope}%
\pgfpathrectangle{\pgfqpoint{0.589510in}{0.417642in}}{\pgfqpoint{1.808820in}{1.370688in}}%
\pgfusepath{clip}%
\pgfsetrectcap%
\pgfsetroundjoin%
\pgfsetlinewidth{0.803000pt}%
\definecolor{currentstroke}{rgb}{0.850000,0.850000,0.850000}%
\pgfsetstrokecolor{currentstroke}%
\pgfsetdash{}{0pt}%
\pgfpathmoveto{\pgfqpoint{0.650840in}{0.417642in}}%
\pgfpathlineto{\pgfqpoint{0.650840in}{1.788330in}}%
\pgfusepath{stroke}%
\end{pgfscope}%
\begin{pgfscope}%
\pgfsetbuttcap%
\pgfsetroundjoin%
\definecolor{currentfill}{rgb}{0.000000,0.000000,0.000000}%
\pgfsetfillcolor{currentfill}%
\pgfsetlinewidth{0.602250pt}%
\definecolor{currentstroke}{rgb}{0.000000,0.000000,0.000000}%
\pgfsetstrokecolor{currentstroke}%
\pgfsetdash{}{0pt}%
\pgfsys@defobject{currentmarker}{\pgfqpoint{0.000000in}{-0.027778in}}{\pgfqpoint{0.000000in}{0.000000in}}{%
\pgfpathmoveto{\pgfqpoint{0.000000in}{0.000000in}}%
\pgfpathlineto{\pgfqpoint{0.000000in}{-0.027778in}}%
\pgfusepath{stroke,fill}%
}%
\begin{pgfscope}%
\pgfsys@transformshift{0.650840in}{0.417642in}%
\pgfsys@useobject{currentmarker}{}%
\end{pgfscope}%
\end{pgfscope}%
\begin{pgfscope}%
\pgfpathrectangle{\pgfqpoint{0.589510in}{0.417642in}}{\pgfqpoint{1.808820in}{1.370688in}}%
\pgfusepath{clip}%
\pgfsetrectcap%
\pgfsetroundjoin%
\pgfsetlinewidth{0.803000pt}%
\definecolor{currentstroke}{rgb}{0.850000,0.850000,0.850000}%
\pgfsetstrokecolor{currentstroke}%
\pgfsetdash{}{0pt}%
\pgfpathmoveto{\pgfqpoint{0.809153in}{0.417642in}}%
\pgfpathlineto{\pgfqpoint{0.809153in}{1.788330in}}%
\pgfusepath{stroke}%
\end{pgfscope}%
\begin{pgfscope}%
\pgfsetbuttcap%
\pgfsetroundjoin%
\definecolor{currentfill}{rgb}{0.000000,0.000000,0.000000}%
\pgfsetfillcolor{currentfill}%
\pgfsetlinewidth{0.602250pt}%
\definecolor{currentstroke}{rgb}{0.000000,0.000000,0.000000}%
\pgfsetstrokecolor{currentstroke}%
\pgfsetdash{}{0pt}%
\pgfsys@defobject{currentmarker}{\pgfqpoint{0.000000in}{-0.027778in}}{\pgfqpoint{0.000000in}{0.000000in}}{%
\pgfpathmoveto{\pgfqpoint{0.000000in}{0.000000in}}%
\pgfpathlineto{\pgfqpoint{0.000000in}{-0.027778in}}%
\pgfusepath{stroke,fill}%
}%
\begin{pgfscope}%
\pgfsys@transformshift{0.809153in}{0.417642in}%
\pgfsys@useobject{currentmarker}{}%
\end{pgfscope}%
\end{pgfscope}%
\begin{pgfscope}%
\pgfpathrectangle{\pgfqpoint{0.589510in}{0.417642in}}{\pgfqpoint{1.808820in}{1.370688in}}%
\pgfusepath{clip}%
\pgfsetrectcap%
\pgfsetroundjoin%
\pgfsetlinewidth{0.803000pt}%
\definecolor{currentstroke}{rgb}{0.850000,0.850000,0.850000}%
\pgfsetstrokecolor{currentstroke}%
\pgfsetdash{}{0pt}%
\pgfpathmoveto{\pgfqpoint{0.889540in}{0.417642in}}%
\pgfpathlineto{\pgfqpoint{0.889540in}{1.788330in}}%
\pgfusepath{stroke}%
\end{pgfscope}%
\begin{pgfscope}%
\pgfsetbuttcap%
\pgfsetroundjoin%
\definecolor{currentfill}{rgb}{0.000000,0.000000,0.000000}%
\pgfsetfillcolor{currentfill}%
\pgfsetlinewidth{0.602250pt}%
\definecolor{currentstroke}{rgb}{0.000000,0.000000,0.000000}%
\pgfsetstrokecolor{currentstroke}%
\pgfsetdash{}{0pt}%
\pgfsys@defobject{currentmarker}{\pgfqpoint{0.000000in}{-0.027778in}}{\pgfqpoint{0.000000in}{0.000000in}}{%
\pgfpathmoveto{\pgfqpoint{0.000000in}{0.000000in}}%
\pgfpathlineto{\pgfqpoint{0.000000in}{-0.027778in}}%
\pgfusepath{stroke,fill}%
}%
\begin{pgfscope}%
\pgfsys@transformshift{0.889540in}{0.417642in}%
\pgfsys@useobject{currentmarker}{}%
\end{pgfscope}%
\end{pgfscope}%
\begin{pgfscope}%
\pgfpathrectangle{\pgfqpoint{0.589510in}{0.417642in}}{\pgfqpoint{1.808820in}{1.370688in}}%
\pgfusepath{clip}%
\pgfsetrectcap%
\pgfsetroundjoin%
\pgfsetlinewidth{0.803000pt}%
\definecolor{currentstroke}{rgb}{0.850000,0.850000,0.850000}%
\pgfsetstrokecolor{currentstroke}%
\pgfsetdash{}{0pt}%
\pgfpathmoveto{\pgfqpoint{0.946576in}{0.417642in}}%
\pgfpathlineto{\pgfqpoint{0.946576in}{1.788330in}}%
\pgfusepath{stroke}%
\end{pgfscope}%
\begin{pgfscope}%
\pgfsetbuttcap%
\pgfsetroundjoin%
\definecolor{currentfill}{rgb}{0.000000,0.000000,0.000000}%
\pgfsetfillcolor{currentfill}%
\pgfsetlinewidth{0.602250pt}%
\definecolor{currentstroke}{rgb}{0.000000,0.000000,0.000000}%
\pgfsetstrokecolor{currentstroke}%
\pgfsetdash{}{0pt}%
\pgfsys@defobject{currentmarker}{\pgfqpoint{0.000000in}{-0.027778in}}{\pgfqpoint{0.000000in}{0.000000in}}{%
\pgfpathmoveto{\pgfqpoint{0.000000in}{0.000000in}}%
\pgfpathlineto{\pgfqpoint{0.000000in}{-0.027778in}}%
\pgfusepath{stroke,fill}%
}%
\begin{pgfscope}%
\pgfsys@transformshift{0.946576in}{0.417642in}%
\pgfsys@useobject{currentmarker}{}%
\end{pgfscope}%
\end{pgfscope}%
\begin{pgfscope}%
\pgfpathrectangle{\pgfqpoint{0.589510in}{0.417642in}}{\pgfqpoint{1.808820in}{1.370688in}}%
\pgfusepath{clip}%
\pgfsetrectcap%
\pgfsetroundjoin%
\pgfsetlinewidth{0.803000pt}%
\definecolor{currentstroke}{rgb}{0.850000,0.850000,0.850000}%
\pgfsetstrokecolor{currentstroke}%
\pgfsetdash{}{0pt}%
\pgfpathmoveto{\pgfqpoint{0.990817in}{0.417642in}}%
\pgfpathlineto{\pgfqpoint{0.990817in}{1.788330in}}%
\pgfusepath{stroke}%
\end{pgfscope}%
\begin{pgfscope}%
\pgfsetbuttcap%
\pgfsetroundjoin%
\definecolor{currentfill}{rgb}{0.000000,0.000000,0.000000}%
\pgfsetfillcolor{currentfill}%
\pgfsetlinewidth{0.602250pt}%
\definecolor{currentstroke}{rgb}{0.000000,0.000000,0.000000}%
\pgfsetstrokecolor{currentstroke}%
\pgfsetdash{}{0pt}%
\pgfsys@defobject{currentmarker}{\pgfqpoint{0.000000in}{-0.027778in}}{\pgfqpoint{0.000000in}{0.000000in}}{%
\pgfpathmoveto{\pgfqpoint{0.000000in}{0.000000in}}%
\pgfpathlineto{\pgfqpoint{0.000000in}{-0.027778in}}%
\pgfusepath{stroke,fill}%
}%
\begin{pgfscope}%
\pgfsys@transformshift{0.990817in}{0.417642in}%
\pgfsys@useobject{currentmarker}{}%
\end{pgfscope}%
\end{pgfscope}%
\begin{pgfscope}%
\pgfpathrectangle{\pgfqpoint{0.589510in}{0.417642in}}{\pgfqpoint{1.808820in}{1.370688in}}%
\pgfusepath{clip}%
\pgfsetrectcap%
\pgfsetroundjoin%
\pgfsetlinewidth{0.803000pt}%
\definecolor{currentstroke}{rgb}{0.850000,0.850000,0.850000}%
\pgfsetstrokecolor{currentstroke}%
\pgfsetdash{}{0pt}%
\pgfpathmoveto{\pgfqpoint{1.026964in}{0.417642in}}%
\pgfpathlineto{\pgfqpoint{1.026964in}{1.788330in}}%
\pgfusepath{stroke}%
\end{pgfscope}%
\begin{pgfscope}%
\pgfsetbuttcap%
\pgfsetroundjoin%
\definecolor{currentfill}{rgb}{0.000000,0.000000,0.000000}%
\pgfsetfillcolor{currentfill}%
\pgfsetlinewidth{0.602250pt}%
\definecolor{currentstroke}{rgb}{0.000000,0.000000,0.000000}%
\pgfsetstrokecolor{currentstroke}%
\pgfsetdash{}{0pt}%
\pgfsys@defobject{currentmarker}{\pgfqpoint{0.000000in}{-0.027778in}}{\pgfqpoint{0.000000in}{0.000000in}}{%
\pgfpathmoveto{\pgfqpoint{0.000000in}{0.000000in}}%
\pgfpathlineto{\pgfqpoint{0.000000in}{-0.027778in}}%
\pgfusepath{stroke,fill}%
}%
\begin{pgfscope}%
\pgfsys@transformshift{1.026964in}{0.417642in}%
\pgfsys@useobject{currentmarker}{}%
\end{pgfscope}%
\end{pgfscope}%
\begin{pgfscope}%
\pgfpathrectangle{\pgfqpoint{0.589510in}{0.417642in}}{\pgfqpoint{1.808820in}{1.370688in}}%
\pgfusepath{clip}%
\pgfsetrectcap%
\pgfsetroundjoin%
\pgfsetlinewidth{0.803000pt}%
\definecolor{currentstroke}{rgb}{0.850000,0.850000,0.850000}%
\pgfsetstrokecolor{currentstroke}%
\pgfsetdash{}{0pt}%
\pgfpathmoveto{\pgfqpoint{1.057526in}{0.417642in}}%
\pgfpathlineto{\pgfqpoint{1.057526in}{1.788330in}}%
\pgfusepath{stroke}%
\end{pgfscope}%
\begin{pgfscope}%
\pgfsetbuttcap%
\pgfsetroundjoin%
\definecolor{currentfill}{rgb}{0.000000,0.000000,0.000000}%
\pgfsetfillcolor{currentfill}%
\pgfsetlinewidth{0.602250pt}%
\definecolor{currentstroke}{rgb}{0.000000,0.000000,0.000000}%
\pgfsetstrokecolor{currentstroke}%
\pgfsetdash{}{0pt}%
\pgfsys@defobject{currentmarker}{\pgfqpoint{0.000000in}{-0.027778in}}{\pgfqpoint{0.000000in}{0.000000in}}{%
\pgfpathmoveto{\pgfqpoint{0.000000in}{0.000000in}}%
\pgfpathlineto{\pgfqpoint{0.000000in}{-0.027778in}}%
\pgfusepath{stroke,fill}%
}%
\begin{pgfscope}%
\pgfsys@transformshift{1.057526in}{0.417642in}%
\pgfsys@useobject{currentmarker}{}%
\end{pgfscope}%
\end{pgfscope}%
\begin{pgfscope}%
\pgfpathrectangle{\pgfqpoint{0.589510in}{0.417642in}}{\pgfqpoint{1.808820in}{1.370688in}}%
\pgfusepath{clip}%
\pgfsetrectcap%
\pgfsetroundjoin%
\pgfsetlinewidth{0.803000pt}%
\definecolor{currentstroke}{rgb}{0.850000,0.850000,0.850000}%
\pgfsetstrokecolor{currentstroke}%
\pgfsetdash{}{0pt}%
\pgfpathmoveto{\pgfqpoint{1.084000in}{0.417642in}}%
\pgfpathlineto{\pgfqpoint{1.084000in}{1.788330in}}%
\pgfusepath{stroke}%
\end{pgfscope}%
\begin{pgfscope}%
\pgfsetbuttcap%
\pgfsetroundjoin%
\definecolor{currentfill}{rgb}{0.000000,0.000000,0.000000}%
\pgfsetfillcolor{currentfill}%
\pgfsetlinewidth{0.602250pt}%
\definecolor{currentstroke}{rgb}{0.000000,0.000000,0.000000}%
\pgfsetstrokecolor{currentstroke}%
\pgfsetdash{}{0pt}%
\pgfsys@defobject{currentmarker}{\pgfqpoint{0.000000in}{-0.027778in}}{\pgfqpoint{0.000000in}{0.000000in}}{%
\pgfpathmoveto{\pgfqpoint{0.000000in}{0.000000in}}%
\pgfpathlineto{\pgfqpoint{0.000000in}{-0.027778in}}%
\pgfusepath{stroke,fill}%
}%
\begin{pgfscope}%
\pgfsys@transformshift{1.084000in}{0.417642in}%
\pgfsys@useobject{currentmarker}{}%
\end{pgfscope}%
\end{pgfscope}%
\begin{pgfscope}%
\pgfpathrectangle{\pgfqpoint{0.589510in}{0.417642in}}{\pgfqpoint{1.808820in}{1.370688in}}%
\pgfusepath{clip}%
\pgfsetrectcap%
\pgfsetroundjoin%
\pgfsetlinewidth{0.803000pt}%
\definecolor{currentstroke}{rgb}{0.850000,0.850000,0.850000}%
\pgfsetstrokecolor{currentstroke}%
\pgfsetdash{}{0pt}%
\pgfpathmoveto{\pgfqpoint{1.107352in}{0.417642in}}%
\pgfpathlineto{\pgfqpoint{1.107352in}{1.788330in}}%
\pgfusepath{stroke}%
\end{pgfscope}%
\begin{pgfscope}%
\pgfsetbuttcap%
\pgfsetroundjoin%
\definecolor{currentfill}{rgb}{0.000000,0.000000,0.000000}%
\pgfsetfillcolor{currentfill}%
\pgfsetlinewidth{0.602250pt}%
\definecolor{currentstroke}{rgb}{0.000000,0.000000,0.000000}%
\pgfsetstrokecolor{currentstroke}%
\pgfsetdash{}{0pt}%
\pgfsys@defobject{currentmarker}{\pgfqpoint{0.000000in}{-0.027778in}}{\pgfqpoint{0.000000in}{0.000000in}}{%
\pgfpathmoveto{\pgfqpoint{0.000000in}{0.000000in}}%
\pgfpathlineto{\pgfqpoint{0.000000in}{-0.027778in}}%
\pgfusepath{stroke,fill}%
}%
\begin{pgfscope}%
\pgfsys@transformshift{1.107352in}{0.417642in}%
\pgfsys@useobject{currentmarker}{}%
\end{pgfscope}%
\end{pgfscope}%
\begin{pgfscope}%
\pgfpathrectangle{\pgfqpoint{0.589510in}{0.417642in}}{\pgfqpoint{1.808820in}{1.370688in}}%
\pgfusepath{clip}%
\pgfsetrectcap%
\pgfsetroundjoin%
\pgfsetlinewidth{0.803000pt}%
\definecolor{currentstroke}{rgb}{0.850000,0.850000,0.850000}%
\pgfsetstrokecolor{currentstroke}%
\pgfsetdash{}{0pt}%
\pgfpathmoveto{\pgfqpoint{1.265664in}{0.417642in}}%
\pgfpathlineto{\pgfqpoint{1.265664in}{1.788330in}}%
\pgfusepath{stroke}%
\end{pgfscope}%
\begin{pgfscope}%
\pgfsetbuttcap%
\pgfsetroundjoin%
\definecolor{currentfill}{rgb}{0.000000,0.000000,0.000000}%
\pgfsetfillcolor{currentfill}%
\pgfsetlinewidth{0.602250pt}%
\definecolor{currentstroke}{rgb}{0.000000,0.000000,0.000000}%
\pgfsetstrokecolor{currentstroke}%
\pgfsetdash{}{0pt}%
\pgfsys@defobject{currentmarker}{\pgfqpoint{0.000000in}{-0.027778in}}{\pgfqpoint{0.000000in}{0.000000in}}{%
\pgfpathmoveto{\pgfqpoint{0.000000in}{0.000000in}}%
\pgfpathlineto{\pgfqpoint{0.000000in}{-0.027778in}}%
\pgfusepath{stroke,fill}%
}%
\begin{pgfscope}%
\pgfsys@transformshift{1.265664in}{0.417642in}%
\pgfsys@useobject{currentmarker}{}%
\end{pgfscope}%
\end{pgfscope}%
\begin{pgfscope}%
\pgfpathrectangle{\pgfqpoint{0.589510in}{0.417642in}}{\pgfqpoint{1.808820in}{1.370688in}}%
\pgfusepath{clip}%
\pgfsetrectcap%
\pgfsetroundjoin%
\pgfsetlinewidth{0.803000pt}%
\definecolor{currentstroke}{rgb}{0.850000,0.850000,0.850000}%
\pgfsetstrokecolor{currentstroke}%
\pgfsetdash{}{0pt}%
\pgfpathmoveto{\pgfqpoint{1.346052in}{0.417642in}}%
\pgfpathlineto{\pgfqpoint{1.346052in}{1.788330in}}%
\pgfusepath{stroke}%
\end{pgfscope}%
\begin{pgfscope}%
\pgfsetbuttcap%
\pgfsetroundjoin%
\definecolor{currentfill}{rgb}{0.000000,0.000000,0.000000}%
\pgfsetfillcolor{currentfill}%
\pgfsetlinewidth{0.602250pt}%
\definecolor{currentstroke}{rgb}{0.000000,0.000000,0.000000}%
\pgfsetstrokecolor{currentstroke}%
\pgfsetdash{}{0pt}%
\pgfsys@defobject{currentmarker}{\pgfqpoint{0.000000in}{-0.027778in}}{\pgfqpoint{0.000000in}{0.000000in}}{%
\pgfpathmoveto{\pgfqpoint{0.000000in}{0.000000in}}%
\pgfpathlineto{\pgfqpoint{0.000000in}{-0.027778in}}%
\pgfusepath{stroke,fill}%
}%
\begin{pgfscope}%
\pgfsys@transformshift{1.346052in}{0.417642in}%
\pgfsys@useobject{currentmarker}{}%
\end{pgfscope}%
\end{pgfscope}%
\begin{pgfscope}%
\pgfpathrectangle{\pgfqpoint{0.589510in}{0.417642in}}{\pgfqpoint{1.808820in}{1.370688in}}%
\pgfusepath{clip}%
\pgfsetrectcap%
\pgfsetroundjoin%
\pgfsetlinewidth{0.803000pt}%
\definecolor{currentstroke}{rgb}{0.850000,0.850000,0.850000}%
\pgfsetstrokecolor{currentstroke}%
\pgfsetdash{}{0pt}%
\pgfpathmoveto{\pgfqpoint{1.403088in}{0.417642in}}%
\pgfpathlineto{\pgfqpoint{1.403088in}{1.788330in}}%
\pgfusepath{stroke}%
\end{pgfscope}%
\begin{pgfscope}%
\pgfsetbuttcap%
\pgfsetroundjoin%
\definecolor{currentfill}{rgb}{0.000000,0.000000,0.000000}%
\pgfsetfillcolor{currentfill}%
\pgfsetlinewidth{0.602250pt}%
\definecolor{currentstroke}{rgb}{0.000000,0.000000,0.000000}%
\pgfsetstrokecolor{currentstroke}%
\pgfsetdash{}{0pt}%
\pgfsys@defobject{currentmarker}{\pgfqpoint{0.000000in}{-0.027778in}}{\pgfqpoint{0.000000in}{0.000000in}}{%
\pgfpathmoveto{\pgfqpoint{0.000000in}{0.000000in}}%
\pgfpathlineto{\pgfqpoint{0.000000in}{-0.027778in}}%
\pgfusepath{stroke,fill}%
}%
\begin{pgfscope}%
\pgfsys@transformshift{1.403088in}{0.417642in}%
\pgfsys@useobject{currentmarker}{}%
\end{pgfscope}%
\end{pgfscope}%
\begin{pgfscope}%
\pgfpathrectangle{\pgfqpoint{0.589510in}{0.417642in}}{\pgfqpoint{1.808820in}{1.370688in}}%
\pgfusepath{clip}%
\pgfsetrectcap%
\pgfsetroundjoin%
\pgfsetlinewidth{0.803000pt}%
\definecolor{currentstroke}{rgb}{0.850000,0.850000,0.850000}%
\pgfsetstrokecolor{currentstroke}%
\pgfsetdash{}{0pt}%
\pgfpathmoveto{\pgfqpoint{1.447328in}{0.417642in}}%
\pgfpathlineto{\pgfqpoint{1.447328in}{1.788330in}}%
\pgfusepath{stroke}%
\end{pgfscope}%
\begin{pgfscope}%
\pgfsetbuttcap%
\pgfsetroundjoin%
\definecolor{currentfill}{rgb}{0.000000,0.000000,0.000000}%
\pgfsetfillcolor{currentfill}%
\pgfsetlinewidth{0.602250pt}%
\definecolor{currentstroke}{rgb}{0.000000,0.000000,0.000000}%
\pgfsetstrokecolor{currentstroke}%
\pgfsetdash{}{0pt}%
\pgfsys@defobject{currentmarker}{\pgfqpoint{0.000000in}{-0.027778in}}{\pgfqpoint{0.000000in}{0.000000in}}{%
\pgfpathmoveto{\pgfqpoint{0.000000in}{0.000000in}}%
\pgfpathlineto{\pgfqpoint{0.000000in}{-0.027778in}}%
\pgfusepath{stroke,fill}%
}%
\begin{pgfscope}%
\pgfsys@transformshift{1.447328in}{0.417642in}%
\pgfsys@useobject{currentmarker}{}%
\end{pgfscope}%
\end{pgfscope}%
\begin{pgfscope}%
\pgfpathrectangle{\pgfqpoint{0.589510in}{0.417642in}}{\pgfqpoint{1.808820in}{1.370688in}}%
\pgfusepath{clip}%
\pgfsetrectcap%
\pgfsetroundjoin%
\pgfsetlinewidth{0.803000pt}%
\definecolor{currentstroke}{rgb}{0.850000,0.850000,0.850000}%
\pgfsetstrokecolor{currentstroke}%
\pgfsetdash{}{0pt}%
\pgfpathmoveto{\pgfqpoint{1.483475in}{0.417642in}}%
\pgfpathlineto{\pgfqpoint{1.483475in}{1.788330in}}%
\pgfusepath{stroke}%
\end{pgfscope}%
\begin{pgfscope}%
\pgfsetbuttcap%
\pgfsetroundjoin%
\definecolor{currentfill}{rgb}{0.000000,0.000000,0.000000}%
\pgfsetfillcolor{currentfill}%
\pgfsetlinewidth{0.602250pt}%
\definecolor{currentstroke}{rgb}{0.000000,0.000000,0.000000}%
\pgfsetstrokecolor{currentstroke}%
\pgfsetdash{}{0pt}%
\pgfsys@defobject{currentmarker}{\pgfqpoint{0.000000in}{-0.027778in}}{\pgfqpoint{0.000000in}{0.000000in}}{%
\pgfpathmoveto{\pgfqpoint{0.000000in}{0.000000in}}%
\pgfpathlineto{\pgfqpoint{0.000000in}{-0.027778in}}%
\pgfusepath{stroke,fill}%
}%
\begin{pgfscope}%
\pgfsys@transformshift{1.483475in}{0.417642in}%
\pgfsys@useobject{currentmarker}{}%
\end{pgfscope}%
\end{pgfscope}%
\begin{pgfscope}%
\pgfpathrectangle{\pgfqpoint{0.589510in}{0.417642in}}{\pgfqpoint{1.808820in}{1.370688in}}%
\pgfusepath{clip}%
\pgfsetrectcap%
\pgfsetroundjoin%
\pgfsetlinewidth{0.803000pt}%
\definecolor{currentstroke}{rgb}{0.850000,0.850000,0.850000}%
\pgfsetstrokecolor{currentstroke}%
\pgfsetdash{}{0pt}%
\pgfpathmoveto{\pgfqpoint{1.514037in}{0.417642in}}%
\pgfpathlineto{\pgfqpoint{1.514037in}{1.788330in}}%
\pgfusepath{stroke}%
\end{pgfscope}%
\begin{pgfscope}%
\pgfsetbuttcap%
\pgfsetroundjoin%
\definecolor{currentfill}{rgb}{0.000000,0.000000,0.000000}%
\pgfsetfillcolor{currentfill}%
\pgfsetlinewidth{0.602250pt}%
\definecolor{currentstroke}{rgb}{0.000000,0.000000,0.000000}%
\pgfsetstrokecolor{currentstroke}%
\pgfsetdash{}{0pt}%
\pgfsys@defobject{currentmarker}{\pgfqpoint{0.000000in}{-0.027778in}}{\pgfqpoint{0.000000in}{0.000000in}}{%
\pgfpathmoveto{\pgfqpoint{0.000000in}{0.000000in}}%
\pgfpathlineto{\pgfqpoint{0.000000in}{-0.027778in}}%
\pgfusepath{stroke,fill}%
}%
\begin{pgfscope}%
\pgfsys@transformshift{1.514037in}{0.417642in}%
\pgfsys@useobject{currentmarker}{}%
\end{pgfscope}%
\end{pgfscope}%
\begin{pgfscope}%
\pgfpathrectangle{\pgfqpoint{0.589510in}{0.417642in}}{\pgfqpoint{1.808820in}{1.370688in}}%
\pgfusepath{clip}%
\pgfsetrectcap%
\pgfsetroundjoin%
\pgfsetlinewidth{0.803000pt}%
\definecolor{currentstroke}{rgb}{0.850000,0.850000,0.850000}%
\pgfsetstrokecolor{currentstroke}%
\pgfsetdash{}{0pt}%
\pgfpathmoveto{\pgfqpoint{1.540511in}{0.417642in}}%
\pgfpathlineto{\pgfqpoint{1.540511in}{1.788330in}}%
\pgfusepath{stroke}%
\end{pgfscope}%
\begin{pgfscope}%
\pgfsetbuttcap%
\pgfsetroundjoin%
\definecolor{currentfill}{rgb}{0.000000,0.000000,0.000000}%
\pgfsetfillcolor{currentfill}%
\pgfsetlinewidth{0.602250pt}%
\definecolor{currentstroke}{rgb}{0.000000,0.000000,0.000000}%
\pgfsetstrokecolor{currentstroke}%
\pgfsetdash{}{0pt}%
\pgfsys@defobject{currentmarker}{\pgfqpoint{0.000000in}{-0.027778in}}{\pgfqpoint{0.000000in}{0.000000in}}{%
\pgfpathmoveto{\pgfqpoint{0.000000in}{0.000000in}}%
\pgfpathlineto{\pgfqpoint{0.000000in}{-0.027778in}}%
\pgfusepath{stroke,fill}%
}%
\begin{pgfscope}%
\pgfsys@transformshift{1.540511in}{0.417642in}%
\pgfsys@useobject{currentmarker}{}%
\end{pgfscope}%
\end{pgfscope}%
\begin{pgfscope}%
\pgfpathrectangle{\pgfqpoint{0.589510in}{0.417642in}}{\pgfqpoint{1.808820in}{1.370688in}}%
\pgfusepath{clip}%
\pgfsetrectcap%
\pgfsetroundjoin%
\pgfsetlinewidth{0.803000pt}%
\definecolor{currentstroke}{rgb}{0.850000,0.850000,0.850000}%
\pgfsetstrokecolor{currentstroke}%
\pgfsetdash{}{0pt}%
\pgfpathmoveto{\pgfqpoint{1.563863in}{0.417642in}}%
\pgfpathlineto{\pgfqpoint{1.563863in}{1.788330in}}%
\pgfusepath{stroke}%
\end{pgfscope}%
\begin{pgfscope}%
\pgfsetbuttcap%
\pgfsetroundjoin%
\definecolor{currentfill}{rgb}{0.000000,0.000000,0.000000}%
\pgfsetfillcolor{currentfill}%
\pgfsetlinewidth{0.602250pt}%
\definecolor{currentstroke}{rgb}{0.000000,0.000000,0.000000}%
\pgfsetstrokecolor{currentstroke}%
\pgfsetdash{}{0pt}%
\pgfsys@defobject{currentmarker}{\pgfqpoint{0.000000in}{-0.027778in}}{\pgfqpoint{0.000000in}{0.000000in}}{%
\pgfpathmoveto{\pgfqpoint{0.000000in}{0.000000in}}%
\pgfpathlineto{\pgfqpoint{0.000000in}{-0.027778in}}%
\pgfusepath{stroke,fill}%
}%
\begin{pgfscope}%
\pgfsys@transformshift{1.563863in}{0.417642in}%
\pgfsys@useobject{currentmarker}{}%
\end{pgfscope}%
\end{pgfscope}%
\begin{pgfscope}%
\pgfpathrectangle{\pgfqpoint{0.589510in}{0.417642in}}{\pgfqpoint{1.808820in}{1.370688in}}%
\pgfusepath{clip}%
\pgfsetrectcap%
\pgfsetroundjoin%
\pgfsetlinewidth{0.803000pt}%
\definecolor{currentstroke}{rgb}{0.850000,0.850000,0.850000}%
\pgfsetstrokecolor{currentstroke}%
\pgfsetdash{}{0pt}%
\pgfpathmoveto{\pgfqpoint{1.722176in}{0.417642in}}%
\pgfpathlineto{\pgfqpoint{1.722176in}{1.788330in}}%
\pgfusepath{stroke}%
\end{pgfscope}%
\begin{pgfscope}%
\pgfsetbuttcap%
\pgfsetroundjoin%
\definecolor{currentfill}{rgb}{0.000000,0.000000,0.000000}%
\pgfsetfillcolor{currentfill}%
\pgfsetlinewidth{0.602250pt}%
\definecolor{currentstroke}{rgb}{0.000000,0.000000,0.000000}%
\pgfsetstrokecolor{currentstroke}%
\pgfsetdash{}{0pt}%
\pgfsys@defobject{currentmarker}{\pgfqpoint{0.000000in}{-0.027778in}}{\pgfqpoint{0.000000in}{0.000000in}}{%
\pgfpathmoveto{\pgfqpoint{0.000000in}{0.000000in}}%
\pgfpathlineto{\pgfqpoint{0.000000in}{-0.027778in}}%
\pgfusepath{stroke,fill}%
}%
\begin{pgfscope}%
\pgfsys@transformshift{1.722176in}{0.417642in}%
\pgfsys@useobject{currentmarker}{}%
\end{pgfscope}%
\end{pgfscope}%
\begin{pgfscope}%
\pgfpathrectangle{\pgfqpoint{0.589510in}{0.417642in}}{\pgfqpoint{1.808820in}{1.370688in}}%
\pgfusepath{clip}%
\pgfsetrectcap%
\pgfsetroundjoin%
\pgfsetlinewidth{0.803000pt}%
\definecolor{currentstroke}{rgb}{0.850000,0.850000,0.850000}%
\pgfsetstrokecolor{currentstroke}%
\pgfsetdash{}{0pt}%
\pgfpathmoveto{\pgfqpoint{1.802563in}{0.417642in}}%
\pgfpathlineto{\pgfqpoint{1.802563in}{1.788330in}}%
\pgfusepath{stroke}%
\end{pgfscope}%
\begin{pgfscope}%
\pgfsetbuttcap%
\pgfsetroundjoin%
\definecolor{currentfill}{rgb}{0.000000,0.000000,0.000000}%
\pgfsetfillcolor{currentfill}%
\pgfsetlinewidth{0.602250pt}%
\definecolor{currentstroke}{rgb}{0.000000,0.000000,0.000000}%
\pgfsetstrokecolor{currentstroke}%
\pgfsetdash{}{0pt}%
\pgfsys@defobject{currentmarker}{\pgfqpoint{0.000000in}{-0.027778in}}{\pgfqpoint{0.000000in}{0.000000in}}{%
\pgfpathmoveto{\pgfqpoint{0.000000in}{0.000000in}}%
\pgfpathlineto{\pgfqpoint{0.000000in}{-0.027778in}}%
\pgfusepath{stroke,fill}%
}%
\begin{pgfscope}%
\pgfsys@transformshift{1.802563in}{0.417642in}%
\pgfsys@useobject{currentmarker}{}%
\end{pgfscope}%
\end{pgfscope}%
\begin{pgfscope}%
\pgfpathrectangle{\pgfqpoint{0.589510in}{0.417642in}}{\pgfqpoint{1.808820in}{1.370688in}}%
\pgfusepath{clip}%
\pgfsetrectcap%
\pgfsetroundjoin%
\pgfsetlinewidth{0.803000pt}%
\definecolor{currentstroke}{rgb}{0.850000,0.850000,0.850000}%
\pgfsetstrokecolor{currentstroke}%
\pgfsetdash{}{0pt}%
\pgfpathmoveto{\pgfqpoint{1.859599in}{0.417642in}}%
\pgfpathlineto{\pgfqpoint{1.859599in}{1.788330in}}%
\pgfusepath{stroke}%
\end{pgfscope}%
\begin{pgfscope}%
\pgfsetbuttcap%
\pgfsetroundjoin%
\definecolor{currentfill}{rgb}{0.000000,0.000000,0.000000}%
\pgfsetfillcolor{currentfill}%
\pgfsetlinewidth{0.602250pt}%
\definecolor{currentstroke}{rgb}{0.000000,0.000000,0.000000}%
\pgfsetstrokecolor{currentstroke}%
\pgfsetdash{}{0pt}%
\pgfsys@defobject{currentmarker}{\pgfqpoint{0.000000in}{-0.027778in}}{\pgfqpoint{0.000000in}{0.000000in}}{%
\pgfpathmoveto{\pgfqpoint{0.000000in}{0.000000in}}%
\pgfpathlineto{\pgfqpoint{0.000000in}{-0.027778in}}%
\pgfusepath{stroke,fill}%
}%
\begin{pgfscope}%
\pgfsys@transformshift{1.859599in}{0.417642in}%
\pgfsys@useobject{currentmarker}{}%
\end{pgfscope}%
\end{pgfscope}%
\begin{pgfscope}%
\pgfpathrectangle{\pgfqpoint{0.589510in}{0.417642in}}{\pgfqpoint{1.808820in}{1.370688in}}%
\pgfusepath{clip}%
\pgfsetrectcap%
\pgfsetroundjoin%
\pgfsetlinewidth{0.803000pt}%
\definecolor{currentstroke}{rgb}{0.850000,0.850000,0.850000}%
\pgfsetstrokecolor{currentstroke}%
\pgfsetdash{}{0pt}%
\pgfpathmoveto{\pgfqpoint{1.903840in}{0.417642in}}%
\pgfpathlineto{\pgfqpoint{1.903840in}{1.788330in}}%
\pgfusepath{stroke}%
\end{pgfscope}%
\begin{pgfscope}%
\pgfsetbuttcap%
\pgfsetroundjoin%
\definecolor{currentfill}{rgb}{0.000000,0.000000,0.000000}%
\pgfsetfillcolor{currentfill}%
\pgfsetlinewidth{0.602250pt}%
\definecolor{currentstroke}{rgb}{0.000000,0.000000,0.000000}%
\pgfsetstrokecolor{currentstroke}%
\pgfsetdash{}{0pt}%
\pgfsys@defobject{currentmarker}{\pgfqpoint{0.000000in}{-0.027778in}}{\pgfqpoint{0.000000in}{0.000000in}}{%
\pgfpathmoveto{\pgfqpoint{0.000000in}{0.000000in}}%
\pgfpathlineto{\pgfqpoint{0.000000in}{-0.027778in}}%
\pgfusepath{stroke,fill}%
}%
\begin{pgfscope}%
\pgfsys@transformshift{1.903840in}{0.417642in}%
\pgfsys@useobject{currentmarker}{}%
\end{pgfscope}%
\end{pgfscope}%
\begin{pgfscope}%
\pgfpathrectangle{\pgfqpoint{0.589510in}{0.417642in}}{\pgfqpoint{1.808820in}{1.370688in}}%
\pgfusepath{clip}%
\pgfsetrectcap%
\pgfsetroundjoin%
\pgfsetlinewidth{0.803000pt}%
\definecolor{currentstroke}{rgb}{0.850000,0.850000,0.850000}%
\pgfsetstrokecolor{currentstroke}%
\pgfsetdash{}{0pt}%
\pgfpathmoveto{\pgfqpoint{1.939987in}{0.417642in}}%
\pgfpathlineto{\pgfqpoint{1.939987in}{1.788330in}}%
\pgfusepath{stroke}%
\end{pgfscope}%
\begin{pgfscope}%
\pgfsetbuttcap%
\pgfsetroundjoin%
\definecolor{currentfill}{rgb}{0.000000,0.000000,0.000000}%
\pgfsetfillcolor{currentfill}%
\pgfsetlinewidth{0.602250pt}%
\definecolor{currentstroke}{rgb}{0.000000,0.000000,0.000000}%
\pgfsetstrokecolor{currentstroke}%
\pgfsetdash{}{0pt}%
\pgfsys@defobject{currentmarker}{\pgfqpoint{0.000000in}{-0.027778in}}{\pgfqpoint{0.000000in}{0.000000in}}{%
\pgfpathmoveto{\pgfqpoint{0.000000in}{0.000000in}}%
\pgfpathlineto{\pgfqpoint{0.000000in}{-0.027778in}}%
\pgfusepath{stroke,fill}%
}%
\begin{pgfscope}%
\pgfsys@transformshift{1.939987in}{0.417642in}%
\pgfsys@useobject{currentmarker}{}%
\end{pgfscope}%
\end{pgfscope}%
\begin{pgfscope}%
\pgfpathrectangle{\pgfqpoint{0.589510in}{0.417642in}}{\pgfqpoint{1.808820in}{1.370688in}}%
\pgfusepath{clip}%
\pgfsetrectcap%
\pgfsetroundjoin%
\pgfsetlinewidth{0.803000pt}%
\definecolor{currentstroke}{rgb}{0.850000,0.850000,0.850000}%
\pgfsetstrokecolor{currentstroke}%
\pgfsetdash{}{0pt}%
\pgfpathmoveto{\pgfqpoint{1.970549in}{0.417642in}}%
\pgfpathlineto{\pgfqpoint{1.970549in}{1.788330in}}%
\pgfusepath{stroke}%
\end{pgfscope}%
\begin{pgfscope}%
\pgfsetbuttcap%
\pgfsetroundjoin%
\definecolor{currentfill}{rgb}{0.000000,0.000000,0.000000}%
\pgfsetfillcolor{currentfill}%
\pgfsetlinewidth{0.602250pt}%
\definecolor{currentstroke}{rgb}{0.000000,0.000000,0.000000}%
\pgfsetstrokecolor{currentstroke}%
\pgfsetdash{}{0pt}%
\pgfsys@defobject{currentmarker}{\pgfqpoint{0.000000in}{-0.027778in}}{\pgfqpoint{0.000000in}{0.000000in}}{%
\pgfpathmoveto{\pgfqpoint{0.000000in}{0.000000in}}%
\pgfpathlineto{\pgfqpoint{0.000000in}{-0.027778in}}%
\pgfusepath{stroke,fill}%
}%
\begin{pgfscope}%
\pgfsys@transformshift{1.970549in}{0.417642in}%
\pgfsys@useobject{currentmarker}{}%
\end{pgfscope}%
\end{pgfscope}%
\begin{pgfscope}%
\pgfpathrectangle{\pgfqpoint{0.589510in}{0.417642in}}{\pgfqpoint{1.808820in}{1.370688in}}%
\pgfusepath{clip}%
\pgfsetrectcap%
\pgfsetroundjoin%
\pgfsetlinewidth{0.803000pt}%
\definecolor{currentstroke}{rgb}{0.850000,0.850000,0.850000}%
\pgfsetstrokecolor{currentstroke}%
\pgfsetdash{}{0pt}%
\pgfpathmoveto{\pgfqpoint{1.997023in}{0.417642in}}%
\pgfpathlineto{\pgfqpoint{1.997023in}{1.788330in}}%
\pgfusepath{stroke}%
\end{pgfscope}%
\begin{pgfscope}%
\pgfsetbuttcap%
\pgfsetroundjoin%
\definecolor{currentfill}{rgb}{0.000000,0.000000,0.000000}%
\pgfsetfillcolor{currentfill}%
\pgfsetlinewidth{0.602250pt}%
\definecolor{currentstroke}{rgb}{0.000000,0.000000,0.000000}%
\pgfsetstrokecolor{currentstroke}%
\pgfsetdash{}{0pt}%
\pgfsys@defobject{currentmarker}{\pgfqpoint{0.000000in}{-0.027778in}}{\pgfqpoint{0.000000in}{0.000000in}}{%
\pgfpathmoveto{\pgfqpoint{0.000000in}{0.000000in}}%
\pgfpathlineto{\pgfqpoint{0.000000in}{-0.027778in}}%
\pgfusepath{stroke,fill}%
}%
\begin{pgfscope}%
\pgfsys@transformshift{1.997023in}{0.417642in}%
\pgfsys@useobject{currentmarker}{}%
\end{pgfscope}%
\end{pgfscope}%
\begin{pgfscope}%
\pgfpathrectangle{\pgfqpoint{0.589510in}{0.417642in}}{\pgfqpoint{1.808820in}{1.370688in}}%
\pgfusepath{clip}%
\pgfsetrectcap%
\pgfsetroundjoin%
\pgfsetlinewidth{0.803000pt}%
\definecolor{currentstroke}{rgb}{0.850000,0.850000,0.850000}%
\pgfsetstrokecolor{currentstroke}%
\pgfsetdash{}{0pt}%
\pgfpathmoveto{\pgfqpoint{2.020375in}{0.417642in}}%
\pgfpathlineto{\pgfqpoint{2.020375in}{1.788330in}}%
\pgfusepath{stroke}%
\end{pgfscope}%
\begin{pgfscope}%
\pgfsetbuttcap%
\pgfsetroundjoin%
\definecolor{currentfill}{rgb}{0.000000,0.000000,0.000000}%
\pgfsetfillcolor{currentfill}%
\pgfsetlinewidth{0.602250pt}%
\definecolor{currentstroke}{rgb}{0.000000,0.000000,0.000000}%
\pgfsetstrokecolor{currentstroke}%
\pgfsetdash{}{0pt}%
\pgfsys@defobject{currentmarker}{\pgfqpoint{0.000000in}{-0.027778in}}{\pgfqpoint{0.000000in}{0.000000in}}{%
\pgfpathmoveto{\pgfqpoint{0.000000in}{0.000000in}}%
\pgfpathlineto{\pgfqpoint{0.000000in}{-0.027778in}}%
\pgfusepath{stroke,fill}%
}%
\begin{pgfscope}%
\pgfsys@transformshift{2.020375in}{0.417642in}%
\pgfsys@useobject{currentmarker}{}%
\end{pgfscope}%
\end{pgfscope}%
\begin{pgfscope}%
\pgfpathrectangle{\pgfqpoint{0.589510in}{0.417642in}}{\pgfqpoint{1.808820in}{1.370688in}}%
\pgfusepath{clip}%
\pgfsetrectcap%
\pgfsetroundjoin%
\pgfsetlinewidth{0.803000pt}%
\definecolor{currentstroke}{rgb}{0.850000,0.850000,0.850000}%
\pgfsetstrokecolor{currentstroke}%
\pgfsetdash{}{0pt}%
\pgfpathmoveto{\pgfqpoint{2.178687in}{0.417642in}}%
\pgfpathlineto{\pgfqpoint{2.178687in}{1.788330in}}%
\pgfusepath{stroke}%
\end{pgfscope}%
\begin{pgfscope}%
\pgfsetbuttcap%
\pgfsetroundjoin%
\definecolor{currentfill}{rgb}{0.000000,0.000000,0.000000}%
\pgfsetfillcolor{currentfill}%
\pgfsetlinewidth{0.602250pt}%
\definecolor{currentstroke}{rgb}{0.000000,0.000000,0.000000}%
\pgfsetstrokecolor{currentstroke}%
\pgfsetdash{}{0pt}%
\pgfsys@defobject{currentmarker}{\pgfqpoint{0.000000in}{-0.027778in}}{\pgfqpoint{0.000000in}{0.000000in}}{%
\pgfpathmoveto{\pgfqpoint{0.000000in}{0.000000in}}%
\pgfpathlineto{\pgfqpoint{0.000000in}{-0.027778in}}%
\pgfusepath{stroke,fill}%
}%
\begin{pgfscope}%
\pgfsys@transformshift{2.178687in}{0.417642in}%
\pgfsys@useobject{currentmarker}{}%
\end{pgfscope}%
\end{pgfscope}%
\begin{pgfscope}%
\pgfpathrectangle{\pgfqpoint{0.589510in}{0.417642in}}{\pgfqpoint{1.808820in}{1.370688in}}%
\pgfusepath{clip}%
\pgfsetrectcap%
\pgfsetroundjoin%
\pgfsetlinewidth{0.803000pt}%
\definecolor{currentstroke}{rgb}{0.850000,0.850000,0.850000}%
\pgfsetstrokecolor{currentstroke}%
\pgfsetdash{}{0pt}%
\pgfpathmoveto{\pgfqpoint{2.259075in}{0.417642in}}%
\pgfpathlineto{\pgfqpoint{2.259075in}{1.788330in}}%
\pgfusepath{stroke}%
\end{pgfscope}%
\begin{pgfscope}%
\pgfsetbuttcap%
\pgfsetroundjoin%
\definecolor{currentfill}{rgb}{0.000000,0.000000,0.000000}%
\pgfsetfillcolor{currentfill}%
\pgfsetlinewidth{0.602250pt}%
\definecolor{currentstroke}{rgb}{0.000000,0.000000,0.000000}%
\pgfsetstrokecolor{currentstroke}%
\pgfsetdash{}{0pt}%
\pgfsys@defobject{currentmarker}{\pgfqpoint{0.000000in}{-0.027778in}}{\pgfqpoint{0.000000in}{0.000000in}}{%
\pgfpathmoveto{\pgfqpoint{0.000000in}{0.000000in}}%
\pgfpathlineto{\pgfqpoint{0.000000in}{-0.027778in}}%
\pgfusepath{stroke,fill}%
}%
\begin{pgfscope}%
\pgfsys@transformshift{2.259075in}{0.417642in}%
\pgfsys@useobject{currentmarker}{}%
\end{pgfscope}%
\end{pgfscope}%
\begin{pgfscope}%
\pgfpathrectangle{\pgfqpoint{0.589510in}{0.417642in}}{\pgfqpoint{1.808820in}{1.370688in}}%
\pgfusepath{clip}%
\pgfsetrectcap%
\pgfsetroundjoin%
\pgfsetlinewidth{0.803000pt}%
\definecolor{currentstroke}{rgb}{0.850000,0.850000,0.850000}%
\pgfsetstrokecolor{currentstroke}%
\pgfsetdash{}{0pt}%
\pgfpathmoveto{\pgfqpoint{2.316111in}{0.417642in}}%
\pgfpathlineto{\pgfqpoint{2.316111in}{1.788330in}}%
\pgfusepath{stroke}%
\end{pgfscope}%
\begin{pgfscope}%
\pgfsetbuttcap%
\pgfsetroundjoin%
\definecolor{currentfill}{rgb}{0.000000,0.000000,0.000000}%
\pgfsetfillcolor{currentfill}%
\pgfsetlinewidth{0.602250pt}%
\definecolor{currentstroke}{rgb}{0.000000,0.000000,0.000000}%
\pgfsetstrokecolor{currentstroke}%
\pgfsetdash{}{0pt}%
\pgfsys@defobject{currentmarker}{\pgfqpoint{0.000000in}{-0.027778in}}{\pgfqpoint{0.000000in}{0.000000in}}{%
\pgfpathmoveto{\pgfqpoint{0.000000in}{0.000000in}}%
\pgfpathlineto{\pgfqpoint{0.000000in}{-0.027778in}}%
\pgfusepath{stroke,fill}%
}%
\begin{pgfscope}%
\pgfsys@transformshift{2.316111in}{0.417642in}%
\pgfsys@useobject{currentmarker}{}%
\end{pgfscope}%
\end{pgfscope}%
\begin{pgfscope}%
\pgfpathrectangle{\pgfqpoint{0.589510in}{0.417642in}}{\pgfqpoint{1.808820in}{1.370688in}}%
\pgfusepath{clip}%
\pgfsetrectcap%
\pgfsetroundjoin%
\pgfsetlinewidth{0.803000pt}%
\definecolor{currentstroke}{rgb}{0.850000,0.850000,0.850000}%
\pgfsetstrokecolor{currentstroke}%
\pgfsetdash{}{0pt}%
\pgfpathmoveto{\pgfqpoint{2.360351in}{0.417642in}}%
\pgfpathlineto{\pgfqpoint{2.360351in}{1.788330in}}%
\pgfusepath{stroke}%
\end{pgfscope}%
\begin{pgfscope}%
\pgfsetbuttcap%
\pgfsetroundjoin%
\definecolor{currentfill}{rgb}{0.000000,0.000000,0.000000}%
\pgfsetfillcolor{currentfill}%
\pgfsetlinewidth{0.602250pt}%
\definecolor{currentstroke}{rgb}{0.000000,0.000000,0.000000}%
\pgfsetstrokecolor{currentstroke}%
\pgfsetdash{}{0pt}%
\pgfsys@defobject{currentmarker}{\pgfqpoint{0.000000in}{-0.027778in}}{\pgfqpoint{0.000000in}{0.000000in}}{%
\pgfpathmoveto{\pgfqpoint{0.000000in}{0.000000in}}%
\pgfpathlineto{\pgfqpoint{0.000000in}{-0.027778in}}%
\pgfusepath{stroke,fill}%
}%
\begin{pgfscope}%
\pgfsys@transformshift{2.360351in}{0.417642in}%
\pgfsys@useobject{currentmarker}{}%
\end{pgfscope}%
\end{pgfscope}%
\begin{pgfscope}%
\pgfpathrectangle{\pgfqpoint{0.589510in}{0.417642in}}{\pgfqpoint{1.808820in}{1.370688in}}%
\pgfusepath{clip}%
\pgfsetrectcap%
\pgfsetroundjoin%
\pgfsetlinewidth{0.803000pt}%
\definecolor{currentstroke}{rgb}{0.850000,0.850000,0.850000}%
\pgfsetstrokecolor{currentstroke}%
\pgfsetdash{}{0pt}%
\pgfpathmoveto{\pgfqpoint{2.396499in}{0.417642in}}%
\pgfpathlineto{\pgfqpoint{2.396499in}{1.788330in}}%
\pgfusepath{stroke}%
\end{pgfscope}%
\begin{pgfscope}%
\pgfsetbuttcap%
\pgfsetroundjoin%
\definecolor{currentfill}{rgb}{0.000000,0.000000,0.000000}%
\pgfsetfillcolor{currentfill}%
\pgfsetlinewidth{0.602250pt}%
\definecolor{currentstroke}{rgb}{0.000000,0.000000,0.000000}%
\pgfsetstrokecolor{currentstroke}%
\pgfsetdash{}{0pt}%
\pgfsys@defobject{currentmarker}{\pgfqpoint{0.000000in}{-0.027778in}}{\pgfqpoint{0.000000in}{0.000000in}}{%
\pgfpathmoveto{\pgfqpoint{0.000000in}{0.000000in}}%
\pgfpathlineto{\pgfqpoint{0.000000in}{-0.027778in}}%
\pgfusepath{stroke,fill}%
}%
\begin{pgfscope}%
\pgfsys@transformshift{2.396499in}{0.417642in}%
\pgfsys@useobject{currentmarker}{}%
\end{pgfscope}%
\end{pgfscope}%
\begin{pgfscope}%
\definecolor{textcolor}{rgb}{0.000000,0.000000,0.000000}%
\pgfsetstrokecolor{textcolor}%
\pgfsetfillcolor{textcolor}%
\pgftext[x=1.493920in,y=0.165003in,,top]{\color{textcolor}\rmfamily\fontsize{10.000000}{12.000000}\selectfont \(\displaystyle \tau\) in \unit{\second}}%
\end{pgfscope}%
\begin{pgfscope}%
\pgfpathrectangle{\pgfqpoint{0.589510in}{0.417642in}}{\pgfqpoint{1.808820in}{1.370688in}}%
\pgfusepath{clip}%
\pgfsetrectcap%
\pgfsetroundjoin%
\pgfsetlinewidth{0.803000pt}%
\definecolor{currentstroke}{rgb}{0.450000,0.450000,0.450000}%
\pgfsetstrokecolor{currentstroke}%
\pgfsetdash{}{0pt}%
\pgfpathmoveto{\pgfqpoint{0.589510in}{0.417642in}}%
\pgfpathlineto{\pgfqpoint{2.398330in}{0.417642in}}%
\pgfusepath{stroke}%
\end{pgfscope}%
\begin{pgfscope}%
\pgfsetbuttcap%
\pgfsetroundjoin%
\definecolor{currentfill}{rgb}{0.000000,0.000000,0.000000}%
\pgfsetfillcolor{currentfill}%
\pgfsetlinewidth{0.803000pt}%
\definecolor{currentstroke}{rgb}{0.000000,0.000000,0.000000}%
\pgfsetstrokecolor{currentstroke}%
\pgfsetdash{}{0pt}%
\pgfsys@defobject{currentmarker}{\pgfqpoint{-0.048611in}{0.000000in}}{\pgfqpoint{-0.000000in}{0.000000in}}{%
\pgfpathmoveto{\pgfqpoint{-0.000000in}{0.000000in}}%
\pgfpathlineto{\pgfqpoint{-0.048611in}{0.000000in}}%
\pgfusepath{stroke,fill}%
}%
\begin{pgfscope}%
\pgfsys@transformshift{0.589510in}{0.417642in}%
\pgfsys@useobject{currentmarker}{}%
\end{pgfscope}%
\end{pgfscope}%
\begin{pgfscope}%
\definecolor{textcolor}{rgb}{0.000000,0.000000,0.000000}%
\pgfsetstrokecolor{textcolor}%
\pgfsetfillcolor{textcolor}%
\pgftext[x=0.236114in, y=0.378489in, left, base]{\color{textcolor}\rmfamily\fontsize{8.000000}{9.600000}\selectfont \(\displaystyle {10^{-2}}\)}%
\end{pgfscope}%
\begin{pgfscope}%
\pgfpathrectangle{\pgfqpoint{0.589510in}{0.417642in}}{\pgfqpoint{1.808820in}{1.370688in}}%
\pgfusepath{clip}%
\pgfsetrectcap%
\pgfsetroundjoin%
\pgfsetlinewidth{0.803000pt}%
\definecolor{currentstroke}{rgb}{0.450000,0.450000,0.450000}%
\pgfsetstrokecolor{currentstroke}%
\pgfsetdash{}{0pt}%
\pgfpathmoveto{\pgfqpoint{0.589510in}{0.826865in}}%
\pgfpathlineto{\pgfqpoint{2.398330in}{0.826865in}}%
\pgfusepath{stroke}%
\end{pgfscope}%
\begin{pgfscope}%
\pgfsetbuttcap%
\pgfsetroundjoin%
\definecolor{currentfill}{rgb}{0.000000,0.000000,0.000000}%
\pgfsetfillcolor{currentfill}%
\pgfsetlinewidth{0.803000pt}%
\definecolor{currentstroke}{rgb}{0.000000,0.000000,0.000000}%
\pgfsetstrokecolor{currentstroke}%
\pgfsetdash{}{0pt}%
\pgfsys@defobject{currentmarker}{\pgfqpoint{-0.048611in}{0.000000in}}{\pgfqpoint{-0.000000in}{0.000000in}}{%
\pgfpathmoveto{\pgfqpoint{-0.000000in}{0.000000in}}%
\pgfpathlineto{\pgfqpoint{-0.048611in}{0.000000in}}%
\pgfusepath{stroke,fill}%
}%
\begin{pgfscope}%
\pgfsys@transformshift{0.589510in}{0.826865in}%
\pgfsys@useobject{currentmarker}{}%
\end{pgfscope}%
\end{pgfscope}%
\begin{pgfscope}%
\definecolor{textcolor}{rgb}{0.000000,0.000000,0.000000}%
\pgfsetstrokecolor{textcolor}%
\pgfsetfillcolor{textcolor}%
\pgftext[x=0.316361in, y=0.787713in, left, base]{\color{textcolor}\rmfamily\fontsize{8.000000}{9.600000}\selectfont \(\displaystyle {10^{0}}\)}%
\end{pgfscope}%
\begin{pgfscope}%
\pgfpathrectangle{\pgfqpoint{0.589510in}{0.417642in}}{\pgfqpoint{1.808820in}{1.370688in}}%
\pgfusepath{clip}%
\pgfsetrectcap%
\pgfsetroundjoin%
\pgfsetlinewidth{0.803000pt}%
\definecolor{currentstroke}{rgb}{0.450000,0.450000,0.450000}%
\pgfsetstrokecolor{currentstroke}%
\pgfsetdash{}{0pt}%
\pgfpathmoveto{\pgfqpoint{0.589510in}{1.236089in}}%
\pgfpathlineto{\pgfqpoint{2.398330in}{1.236089in}}%
\pgfusepath{stroke}%
\end{pgfscope}%
\begin{pgfscope}%
\pgfsetbuttcap%
\pgfsetroundjoin%
\definecolor{currentfill}{rgb}{0.000000,0.000000,0.000000}%
\pgfsetfillcolor{currentfill}%
\pgfsetlinewidth{0.803000pt}%
\definecolor{currentstroke}{rgb}{0.000000,0.000000,0.000000}%
\pgfsetstrokecolor{currentstroke}%
\pgfsetdash{}{0pt}%
\pgfsys@defobject{currentmarker}{\pgfqpoint{-0.048611in}{0.000000in}}{\pgfqpoint{-0.000000in}{0.000000in}}{%
\pgfpathmoveto{\pgfqpoint{-0.000000in}{0.000000in}}%
\pgfpathlineto{\pgfqpoint{-0.048611in}{0.000000in}}%
\pgfusepath{stroke,fill}%
}%
\begin{pgfscope}%
\pgfsys@transformshift{0.589510in}{1.236089in}%
\pgfsys@useobject{currentmarker}{}%
\end{pgfscope}%
\end{pgfscope}%
\begin{pgfscope}%
\definecolor{textcolor}{rgb}{0.000000,0.000000,0.000000}%
\pgfsetstrokecolor{textcolor}%
\pgfsetfillcolor{textcolor}%
\pgftext[x=0.316361in, y=1.196936in, left, base]{\color{textcolor}\rmfamily\fontsize{8.000000}{9.600000}\selectfont \(\displaystyle {10^{2}}\)}%
\end{pgfscope}%
\begin{pgfscope}%
\pgfpathrectangle{\pgfqpoint{0.589510in}{0.417642in}}{\pgfqpoint{1.808820in}{1.370688in}}%
\pgfusepath{clip}%
\pgfsetrectcap%
\pgfsetroundjoin%
\pgfsetlinewidth{0.803000pt}%
\definecolor{currentstroke}{rgb}{0.450000,0.450000,0.450000}%
\pgfsetstrokecolor{currentstroke}%
\pgfsetdash{}{0pt}%
\pgfpathmoveto{\pgfqpoint{0.589510in}{1.645313in}}%
\pgfpathlineto{\pgfqpoint{2.398330in}{1.645313in}}%
\pgfusepath{stroke}%
\end{pgfscope}%
\begin{pgfscope}%
\pgfsetbuttcap%
\pgfsetroundjoin%
\definecolor{currentfill}{rgb}{0.000000,0.000000,0.000000}%
\pgfsetfillcolor{currentfill}%
\pgfsetlinewidth{0.803000pt}%
\definecolor{currentstroke}{rgb}{0.000000,0.000000,0.000000}%
\pgfsetstrokecolor{currentstroke}%
\pgfsetdash{}{0pt}%
\pgfsys@defobject{currentmarker}{\pgfqpoint{-0.048611in}{0.000000in}}{\pgfqpoint{-0.000000in}{0.000000in}}{%
\pgfpathmoveto{\pgfqpoint{-0.000000in}{0.000000in}}%
\pgfpathlineto{\pgfqpoint{-0.048611in}{0.000000in}}%
\pgfusepath{stroke,fill}%
}%
\begin{pgfscope}%
\pgfsys@transformshift{0.589510in}{1.645313in}%
\pgfsys@useobject{currentmarker}{}%
\end{pgfscope}%
\end{pgfscope}%
\begin{pgfscope}%
\definecolor{textcolor}{rgb}{0.000000,0.000000,0.000000}%
\pgfsetstrokecolor{textcolor}%
\pgfsetfillcolor{textcolor}%
\pgftext[x=0.316361in, y=1.606160in, left, base]{\color{textcolor}\rmfamily\fontsize{8.000000}{9.600000}\selectfont \(\displaystyle {10^{4}}\)}%
\end{pgfscope}%
\begin{pgfscope}%
\definecolor{textcolor}{rgb}{0.000000,0.000000,0.000000}%
\pgfsetstrokecolor{textcolor}%
\pgfsetfillcolor{textcolor}%
\pgftext[x=0.180559in,y=1.102986in,,bottom,rotate=90.000000]{\color{textcolor}\rmfamily\fontsize{10.000000}{12.000000}\selectfont ADEV \(\displaystyle \sigma_A(\tau)\)}%
\end{pgfscope}%
\begin{pgfscope}%
\pgfpathrectangle{\pgfqpoint{0.589510in}{0.417642in}}{\pgfqpoint{1.808820in}{1.370688in}}%
\pgfusepath{clip}%
\pgfsetbuttcap%
\pgfsetroundjoin%
\pgfsetlinewidth{1.505625pt}%
\definecolor{currentstroke}{rgb}{0.007843,0.619608,0.450980}%
\pgfsetstrokecolor{currentstroke}%
\pgfsetdash{{5.550000pt}{2.400000pt}}{0.000000pt}%
\pgfpathmoveto{\pgfqpoint{0.671729in}{0.826865in}}%
\pgfpathlineto{\pgfqpoint{0.809153in}{0.826865in}}%
\pgfpathlineto{\pgfqpoint{0.946576in}{0.826865in}}%
\pgfpathlineto{\pgfqpoint{1.128240in}{0.826865in}}%
\pgfpathlineto{\pgfqpoint{1.265664in}{0.826865in}}%
\pgfpathlineto{\pgfqpoint{1.403088in}{0.826865in}}%
\pgfpathlineto{\pgfqpoint{1.584752in}{0.826865in}}%
\pgfpathlineto{\pgfqpoint{1.722176in}{0.826865in}}%
\pgfpathlineto{\pgfqpoint{1.859599in}{0.826865in}}%
\pgfpathlineto{\pgfqpoint{2.041264in}{0.826865in}}%
\pgfpathlineto{\pgfqpoint{2.178687in}{0.826865in}}%
\pgfpathlineto{\pgfqpoint{2.316111in}{0.826865in}}%
\pgfusepath{stroke}%
\end{pgfscope}%
\begin{pgfscope}%
\pgfpathrectangle{\pgfqpoint{0.589510in}{0.417642in}}{\pgfqpoint{1.808820in}{1.370688in}}%
\pgfusepath{clip}%
\pgfsetbuttcap%
\pgfsetroundjoin%
\definecolor{currentfill}{rgb}{0.007843,0.619608,0.450980}%
\pgfsetfillcolor{currentfill}%
\pgfsetlinewidth{1.003750pt}%
\definecolor{currentstroke}{rgb}{0.007843,0.619608,0.450980}%
\pgfsetstrokecolor{currentstroke}%
\pgfsetdash{}{0pt}%
\pgfsys@defobject{currentmarker}{\pgfqpoint{-0.020833in}{-0.020833in}}{\pgfqpoint{0.020833in}{0.020833in}}{%
\pgfpathmoveto{\pgfqpoint{0.000000in}{-0.020833in}}%
\pgfpathcurveto{\pgfqpoint{0.005525in}{-0.020833in}}{\pgfqpoint{0.010825in}{-0.018638in}}{\pgfqpoint{0.014731in}{-0.014731in}}%
\pgfpathcurveto{\pgfqpoint{0.018638in}{-0.010825in}}{\pgfqpoint{0.020833in}{-0.005525in}}{\pgfqpoint{0.020833in}{0.000000in}}%
\pgfpathcurveto{\pgfqpoint{0.020833in}{0.005525in}}{\pgfqpoint{0.018638in}{0.010825in}}{\pgfqpoint{0.014731in}{0.014731in}}%
\pgfpathcurveto{\pgfqpoint{0.010825in}{0.018638in}}{\pgfqpoint{0.005525in}{0.020833in}}{\pgfqpoint{0.000000in}{0.020833in}}%
\pgfpathcurveto{\pgfqpoint{-0.005525in}{0.020833in}}{\pgfqpoint{-0.010825in}{0.018638in}}{\pgfqpoint{-0.014731in}{0.014731in}}%
\pgfpathcurveto{\pgfqpoint{-0.018638in}{0.010825in}}{\pgfqpoint{-0.020833in}{0.005525in}}{\pgfqpoint{-0.020833in}{0.000000in}}%
\pgfpathcurveto{\pgfqpoint{-0.020833in}{-0.005525in}}{\pgfqpoint{-0.018638in}{-0.010825in}}{\pgfqpoint{-0.014731in}{-0.014731in}}%
\pgfpathcurveto{\pgfqpoint{-0.010825in}{-0.018638in}}{\pgfqpoint{-0.005525in}{-0.020833in}}{\pgfqpoint{0.000000in}{-0.020833in}}%
\pgfpathlineto{\pgfqpoint{0.000000in}{-0.020833in}}%
\pgfpathclose%
\pgfusepath{stroke,fill}%
}%
\begin{pgfscope}%
\pgfsys@transformshift{0.671729in}{0.843745in}%
\pgfsys@useobject{currentmarker}{}%
\end{pgfscope}%
\begin{pgfscope}%
\pgfsys@transformshift{0.809153in}{0.833670in}%
\pgfsys@useobject{currentmarker}{}%
\end{pgfscope}%
\begin{pgfscope}%
\pgfsys@transformshift{0.946576in}{0.828583in}%
\pgfsys@useobject{currentmarker}{}%
\end{pgfscope}%
\begin{pgfscope}%
\pgfsys@transformshift{1.128240in}{0.824543in}%
\pgfsys@useobject{currentmarker}{}%
\end{pgfscope}%
\begin{pgfscope}%
\pgfsys@transformshift{1.265664in}{0.822909in}%
\pgfsys@useobject{currentmarker}{}%
\end{pgfscope}%
\begin{pgfscope}%
\pgfsys@transformshift{1.403088in}{0.826945in}%
\pgfsys@useobject{currentmarker}{}%
\end{pgfscope}%
\begin{pgfscope}%
\pgfsys@transformshift{1.584752in}{0.827471in}%
\pgfsys@useobject{currentmarker}{}%
\end{pgfscope}%
\begin{pgfscope}%
\pgfsys@transformshift{1.722176in}{0.820728in}%
\pgfsys@useobject{currentmarker}{}%
\end{pgfscope}%
\begin{pgfscope}%
\pgfsys@transformshift{1.859599in}{0.809931in}%
\pgfsys@useobject{currentmarker}{}%
\end{pgfscope}%
\begin{pgfscope}%
\pgfsys@transformshift{2.041264in}{0.819062in}%
\pgfsys@useobject{currentmarker}{}%
\end{pgfscope}%
\begin{pgfscope}%
\pgfsys@transformshift{2.178687in}{0.851857in}%
\pgfsys@useobject{currentmarker}{}%
\end{pgfscope}%
\begin{pgfscope}%
\pgfsys@transformshift{2.316111in}{0.835620in}%
\pgfsys@useobject{currentmarker}{}%
\end{pgfscope}%
\end{pgfscope}%
\begin{pgfscope}%
\pgfsetrectcap%
\pgfsetmiterjoin%
\pgfsetlinewidth{0.803000pt}%
\definecolor{currentstroke}{rgb}{0.000000,0.000000,0.000000}%
\pgfsetstrokecolor{currentstroke}%
\pgfsetdash{}{0pt}%
\pgfpathmoveto{\pgfqpoint{0.589510in}{0.417642in}}%
\pgfpathlineto{\pgfqpoint{0.589510in}{1.788330in}}%
\pgfusepath{stroke}%
\end{pgfscope}%
\begin{pgfscope}%
\pgfsetrectcap%
\pgfsetmiterjoin%
\pgfsetlinewidth{0.803000pt}%
\definecolor{currentstroke}{rgb}{0.000000,0.000000,0.000000}%
\pgfsetstrokecolor{currentstroke}%
\pgfsetdash{}{0pt}%
\pgfpathmoveto{\pgfqpoint{2.398330in}{0.417642in}}%
\pgfpathlineto{\pgfqpoint{2.398330in}{1.788330in}}%
\pgfusepath{stroke}%
\end{pgfscope}%
\begin{pgfscope}%
\pgfsetrectcap%
\pgfsetmiterjoin%
\pgfsetlinewidth{0.803000pt}%
\definecolor{currentstroke}{rgb}{0.000000,0.000000,0.000000}%
\pgfsetstrokecolor{currentstroke}%
\pgfsetdash{}{0pt}%
\pgfpathmoveto{\pgfqpoint{0.589510in}{0.417642in}}%
\pgfpathlineto{\pgfqpoint{2.398330in}{0.417642in}}%
\pgfusepath{stroke}%
\end{pgfscope}%
\begin{pgfscope}%
\pgfsetrectcap%
\pgfsetmiterjoin%
\pgfsetlinewidth{0.803000pt}%
\definecolor{currentstroke}{rgb}{0.000000,0.000000,0.000000}%
\pgfsetstrokecolor{currentstroke}%
\pgfsetdash{}{0pt}%
\pgfpathmoveto{\pgfqpoint{0.589510in}{1.788330in}}%
\pgfpathlineto{\pgfqpoint{2.398330in}{1.788330in}}%
\pgfusepath{stroke}%
\end{pgfscope}%
\begin{pgfscope}%
\pgfsetbuttcap%
\pgfsetmiterjoin%
\definecolor{currentfill}{rgb}{1.000000,1.000000,1.000000}%
\pgfsetfillcolor{currentfill}%
\pgfsetfillopacity{0.800000}%
\pgfsetlinewidth{1.003750pt}%
\definecolor{currentstroke}{rgb}{0.800000,0.800000,0.800000}%
\pgfsetstrokecolor{currentstroke}%
\pgfsetstrokeopacity{0.800000}%
\pgfsetdash{}{0pt}%
\pgfpathmoveto{\pgfqpoint{1.211763in}{1.471662in}}%
\pgfpathlineto{\pgfqpoint{2.320552in}{1.471662in}}%
\pgfpathquadraticcurveto{\pgfqpoint{2.342774in}{1.471662in}}{\pgfqpoint{2.342774in}{1.493884in}}%
\pgfpathlineto{\pgfqpoint{2.342774in}{1.710552in}}%
\pgfpathquadraticcurveto{\pgfqpoint{2.342774in}{1.732774in}}{\pgfqpoint{2.320552in}{1.732774in}}%
\pgfpathlineto{\pgfqpoint{1.211763in}{1.732774in}}%
\pgfpathquadraticcurveto{\pgfqpoint{1.189541in}{1.732774in}}{\pgfqpoint{1.189541in}{1.710552in}}%
\pgfpathlineto{\pgfqpoint{1.189541in}{1.493884in}}%
\pgfpathquadraticcurveto{\pgfqpoint{1.189541in}{1.471662in}}{\pgfqpoint{1.211763in}{1.471662in}}%
\pgfpathlineto{\pgfqpoint{1.211763in}{1.471662in}}%
\pgfpathclose%
\pgfusepath{stroke,fill}%
\end{pgfscope}%
\begin{pgfscope}%
\pgfsetbuttcap%
\pgfsetroundjoin%
\pgfsetlinewidth{1.505625pt}%
\definecolor{currentstroke}{rgb}{0.007843,0.619608,0.450980}%
\pgfsetstrokecolor{currentstroke}%
\pgfsetdash{{5.550000pt}{2.400000pt}}{0.000000pt}%
\pgfpathmoveto{\pgfqpoint{1.233985in}{1.601717in}}%
\pgfpathlineto{\pgfqpoint{1.345096in}{1.601717in}}%
\pgfpathlineto{\pgfqpoint{1.456207in}{1.601717in}}%
\pgfusepath{stroke}%
\end{pgfscope}%
\begin{pgfscope}%
\definecolor{textcolor}{rgb}{0.000000,0.000000,0.000000}%
\pgfsetstrokecolor{textcolor}%
\pgfsetfillcolor{textcolor}%
\pgftext[x=1.545096in,y=1.562828in,left,base]{\color{textcolor}\rmfamily\fontsize{8.000000}{9.600000}\selectfont \(\displaystyle \propto\sqrt{h_{-1}}\tau^{+0.0}\)}%
\end{pgfscope}%
\end{pgfpicture}%
\makeatother%
\endgroup%
% data/simulations/sim_allan_variance.py
        } % scalebox
        \caption{Allan deviation}
        \label{fig:flicker_noise_adev}
    \end{subfigure}
    \caption{Different representations of flicker noise.}
    \label{fig:flicker_noise_simulated}
\end{figure}

While it is not immediately evident from the power spectral density, the Allan deviation plot explains very well, why additional filtering does not affect flicker noise. No matter how long the integration time, the variance will still be the same.

The small wiggles at longer $\tau$ are typical end-of-data errors caused by spectral leakage, because there are insufficient samples to average over \cite{adev_long_tau}. As it was discussed above, the Allan deviation can only be estimated using equation \ref{eqn:adev_estimator} given a limited number of samples. Therefore, at $\frac{\tau}{2}$ there are only $2$ samples left, so there is no averaging possible to improve the estimate of the Allan deviation, which causes the oscillations at low frequencies or large $\tau$.

As a last remark, a commonly used definition in combination with flicker noise is the corner frequency $f_c$. The corner frequency appears in situations where there is both flicker and white noise present. It is the crossover point in frequency, where the flicker noise is equal compared to the white noise.
\begin{equation}
    f_c = \frac{h_{-1}}{h_0} \label{eqn:corner_frequency}
\end{equation}
It can be graphically extracted from the power spectral density plot by drawing a line trough the flicker noise and the white noise and finding the intersection. This can be seen in figure \ref{fig:adev_example_psd} on page \pageref{fig:adev_example_psd}. The corner frequency can be found where the horizontal dashed blue and green line meet.

\subsubsection{Random Walk}%
\label{sec:random_walk}
Random walk noise can be attributed to environmental factors such as temperature \cite{random_walk_fm} and diffusion processes, the latter contributing to the ageing effect seen in semiconductors.
It is a process, where in each time step the change is randomly determined to be either a positve or negative step with equal probability and a fixed step size. Its mean is
\begin{equation}
    \langle y_n \rangle = \langle e_1 + e_2 + \dots e_n \rangle = \underbrace{\langle e_1 \rangle}_{=\,0} + \langle e_2 \rangle + \dots + \langle e_n \rangle = 0 \, ,
\end{equation}
but its variance
\begin{equation}
    \sigma_y^2 = \langle y_n^2 \rangle - \underbrace{\langle y_n \rangle}_{=\,0} = \sigma_{e_1}^2 + \sigma_{e_2}^2 + \dots \sigma_{e_n}^2 = n \sigma_e^2
\end{equation}
goes with $n$ (or $t$). It therefore is not a stationary process as can also be seen in figure \ref{fig:random_walk_adev}.

The power spectral density can be calculated \cite{psd_to_adev,noise_generation} to be
\begin{equation}
    S(f) = h_{-2} \frac{1}{f^2}
\end{equation}
and the Allan deviation can again be calculated from the spectral density
\begin{align}
    \sigma_A^2(\tau) &= 2 h_{-2} \int_0^\infty \frac{1}{f^2} \frac{\sin^4\left( \pi f \tau \right)}{(\pi f \tau)^2}\,df \nonumber\\
    &=\frac{2}{3} \pi^2 h_{-2}\, \tau
\end{align}

The \textit{AllanTools} library \cite{allantools} can then be used to simulate the random walk.
\begin{figure}[ht]
    \centering
    \begin{subfigure}{0.32\linewidth}
        \centering
        \scalebox{0.75}{%
            %% Creator: Matplotlib, PGF backend
%%
%% To include the figure in your LaTeX document, write
%%   \input{<filename>.pgf}
%%
%% Make sure the required packages are loaded in your preamble
%%   \usepackage{pgf}
%%
%% Also ensure that all the required font packages are loaded; for instance,
%% the lmodern package is sometimes necessary when using math font.
%%   \usepackage{lmodern}
%%
%% Figures using additional raster images can only be included by \input if
%% they are in the same directory as the main LaTeX file. For loading figures
%% from other directories you can use the `import` package
%%   \usepackage{import}
%%
%% and then include the figures with
%%   \import{<path to file>}{<filename>.pgf}
%%
%% Matplotlib used the following preamble
%%   \usepackage{siunitx}
%%   \sisetup{per-mode = symbol}%
%%   \usepackage{fontspec}
%%   \makeatletter\@ifpackageloaded{underscore}{}{\usepackage[strings]{underscore}}\makeatother
%%
\begingroup%
\makeatletter%
\begin{pgfpicture}%
\pgfpathrectangle{\pgfpointorigin}{\pgfqpoint{2.440945in}{1.830709in}}%
\pgfusepath{use as bounding box, clip}%
\begin{pgfscope}%
\pgfsetbuttcap%
\pgfsetmiterjoin%
\definecolor{currentfill}{rgb}{1.000000,1.000000,1.000000}%
\pgfsetfillcolor{currentfill}%
\pgfsetlinewidth{0.000000pt}%
\definecolor{currentstroke}{rgb}{1.000000,1.000000,1.000000}%
\pgfsetstrokecolor{currentstroke}%
\pgfsetdash{}{0pt}%
\pgfpathmoveto{\pgfqpoint{0.000000in}{0.000000in}}%
\pgfpathlineto{\pgfqpoint{2.440945in}{0.000000in}}%
\pgfpathlineto{\pgfqpoint{2.440945in}{1.830709in}}%
\pgfpathlineto{\pgfqpoint{0.000000in}{1.830709in}}%
\pgfpathlineto{\pgfqpoint{0.000000in}{0.000000in}}%
\pgfpathclose%
\pgfusepath{fill}%
\end{pgfscope}%
\begin{pgfscope}%
\pgfsetbuttcap%
\pgfsetmiterjoin%
\definecolor{currentfill}{rgb}{1.000000,1.000000,1.000000}%
\pgfsetfillcolor{currentfill}%
\pgfsetlinewidth{0.000000pt}%
\definecolor{currentstroke}{rgb}{0.000000,0.000000,0.000000}%
\pgfsetstrokecolor{currentstroke}%
\pgfsetstrokeopacity{0.000000}%
\pgfsetdash{}{0pt}%
\pgfpathmoveto{\pgfqpoint{0.589745in}{0.416447in}}%
\pgfpathlineto{\pgfqpoint{2.399275in}{0.416447in}}%
\pgfpathlineto{\pgfqpoint{2.399275in}{1.789039in}}%
\pgfpathlineto{\pgfqpoint{0.589745in}{1.789039in}}%
\pgfpathlineto{\pgfqpoint{0.589745in}{0.416447in}}%
\pgfpathclose%
\pgfusepath{fill}%
\end{pgfscope}%
\begin{pgfscope}%
\pgfpathrectangle{\pgfqpoint{0.589745in}{0.416447in}}{\pgfqpoint{1.809530in}{1.372591in}}%
\pgfusepath{clip}%
\pgfsetrectcap%
\pgfsetroundjoin%
\pgfsetlinewidth{0.803000pt}%
\definecolor{currentstroke}{rgb}{0.450000,0.450000,0.450000}%
\pgfsetstrokecolor{currentstroke}%
\pgfsetdash{}{0pt}%
\pgfpathmoveto{\pgfqpoint{0.671996in}{0.416447in}}%
\pgfpathlineto{\pgfqpoint{0.671996in}{1.789039in}}%
\pgfusepath{stroke}%
\end{pgfscope}%
\begin{pgfscope}%
\pgfsetbuttcap%
\pgfsetroundjoin%
\definecolor{currentfill}{rgb}{0.000000,0.000000,0.000000}%
\pgfsetfillcolor{currentfill}%
\pgfsetlinewidth{0.803000pt}%
\definecolor{currentstroke}{rgb}{0.000000,0.000000,0.000000}%
\pgfsetstrokecolor{currentstroke}%
\pgfsetdash{}{0pt}%
\pgfsys@defobject{currentmarker}{\pgfqpoint{0.000000in}{-0.048611in}}{\pgfqpoint{0.000000in}{0.000000in}}{%
\pgfpathmoveto{\pgfqpoint{0.000000in}{0.000000in}}%
\pgfpathlineto{\pgfqpoint{0.000000in}{-0.048611in}}%
\pgfusepath{stroke,fill}%
}%
\begin{pgfscope}%
\pgfsys@transformshift{0.671996in}{0.416447in}%
\pgfsys@useobject{currentmarker}{}%
\end{pgfscope}%
\end{pgfscope}%
\begin{pgfscope}%
\definecolor{textcolor}{rgb}{0.000000,0.000000,0.000000}%
\pgfsetstrokecolor{textcolor}%
\pgfsetfillcolor{textcolor}%
\pgftext[x=0.671996in,y=0.319225in,,top]{\color{textcolor}\rmfamily\fontsize{8.000000}{9.600000}\selectfont \(\displaystyle {0}\)}%
\end{pgfscope}%
\begin{pgfscope}%
\pgfpathrectangle{\pgfqpoint{0.589745in}{0.416447in}}{\pgfqpoint{1.809530in}{1.372591in}}%
\pgfusepath{clip}%
\pgfsetrectcap%
\pgfsetroundjoin%
\pgfsetlinewidth{0.803000pt}%
\definecolor{currentstroke}{rgb}{0.450000,0.450000,0.450000}%
\pgfsetstrokecolor{currentstroke}%
\pgfsetdash{}{0pt}%
\pgfpathmoveto{\pgfqpoint{1.174080in}{0.416447in}}%
\pgfpathlineto{\pgfqpoint{1.174080in}{1.789039in}}%
\pgfusepath{stroke}%
\end{pgfscope}%
\begin{pgfscope}%
\pgfsetbuttcap%
\pgfsetroundjoin%
\definecolor{currentfill}{rgb}{0.000000,0.000000,0.000000}%
\pgfsetfillcolor{currentfill}%
\pgfsetlinewidth{0.803000pt}%
\definecolor{currentstroke}{rgb}{0.000000,0.000000,0.000000}%
\pgfsetstrokecolor{currentstroke}%
\pgfsetdash{}{0pt}%
\pgfsys@defobject{currentmarker}{\pgfqpoint{0.000000in}{-0.048611in}}{\pgfqpoint{0.000000in}{0.000000in}}{%
\pgfpathmoveto{\pgfqpoint{0.000000in}{0.000000in}}%
\pgfpathlineto{\pgfqpoint{0.000000in}{-0.048611in}}%
\pgfusepath{stroke,fill}%
}%
\begin{pgfscope}%
\pgfsys@transformshift{1.174080in}{0.416447in}%
\pgfsys@useobject{currentmarker}{}%
\end{pgfscope}%
\end{pgfscope}%
\begin{pgfscope}%
\definecolor{textcolor}{rgb}{0.000000,0.000000,0.000000}%
\pgfsetstrokecolor{textcolor}%
\pgfsetfillcolor{textcolor}%
\pgftext[x=1.174080in,y=0.319225in,,top]{\color{textcolor}\rmfamily\fontsize{8.000000}{9.600000}\selectfont \(\displaystyle {5000}\)}%
\end{pgfscope}%
\begin{pgfscope}%
\pgfpathrectangle{\pgfqpoint{0.589745in}{0.416447in}}{\pgfqpoint{1.809530in}{1.372591in}}%
\pgfusepath{clip}%
\pgfsetrectcap%
\pgfsetroundjoin%
\pgfsetlinewidth{0.803000pt}%
\definecolor{currentstroke}{rgb}{0.450000,0.450000,0.450000}%
\pgfsetstrokecolor{currentstroke}%
\pgfsetdash{}{0pt}%
\pgfpathmoveto{\pgfqpoint{1.676164in}{0.416447in}}%
\pgfpathlineto{\pgfqpoint{1.676164in}{1.789039in}}%
\pgfusepath{stroke}%
\end{pgfscope}%
\begin{pgfscope}%
\pgfsetbuttcap%
\pgfsetroundjoin%
\definecolor{currentfill}{rgb}{0.000000,0.000000,0.000000}%
\pgfsetfillcolor{currentfill}%
\pgfsetlinewidth{0.803000pt}%
\definecolor{currentstroke}{rgb}{0.000000,0.000000,0.000000}%
\pgfsetstrokecolor{currentstroke}%
\pgfsetdash{}{0pt}%
\pgfsys@defobject{currentmarker}{\pgfqpoint{0.000000in}{-0.048611in}}{\pgfqpoint{0.000000in}{0.000000in}}{%
\pgfpathmoveto{\pgfqpoint{0.000000in}{0.000000in}}%
\pgfpathlineto{\pgfqpoint{0.000000in}{-0.048611in}}%
\pgfusepath{stroke,fill}%
}%
\begin{pgfscope}%
\pgfsys@transformshift{1.676164in}{0.416447in}%
\pgfsys@useobject{currentmarker}{}%
\end{pgfscope}%
\end{pgfscope}%
\begin{pgfscope}%
\definecolor{textcolor}{rgb}{0.000000,0.000000,0.000000}%
\pgfsetstrokecolor{textcolor}%
\pgfsetfillcolor{textcolor}%
\pgftext[x=1.676164in,y=0.319225in,,top]{\color{textcolor}\rmfamily\fontsize{8.000000}{9.600000}\selectfont \(\displaystyle {10000}\)}%
\end{pgfscope}%
\begin{pgfscope}%
\pgfpathrectangle{\pgfqpoint{0.589745in}{0.416447in}}{\pgfqpoint{1.809530in}{1.372591in}}%
\pgfusepath{clip}%
\pgfsetrectcap%
\pgfsetroundjoin%
\pgfsetlinewidth{0.803000pt}%
\definecolor{currentstroke}{rgb}{0.450000,0.450000,0.450000}%
\pgfsetstrokecolor{currentstroke}%
\pgfsetdash{}{0pt}%
\pgfpathmoveto{\pgfqpoint{2.178248in}{0.416447in}}%
\pgfpathlineto{\pgfqpoint{2.178248in}{1.789039in}}%
\pgfusepath{stroke}%
\end{pgfscope}%
\begin{pgfscope}%
\pgfsetbuttcap%
\pgfsetroundjoin%
\definecolor{currentfill}{rgb}{0.000000,0.000000,0.000000}%
\pgfsetfillcolor{currentfill}%
\pgfsetlinewidth{0.803000pt}%
\definecolor{currentstroke}{rgb}{0.000000,0.000000,0.000000}%
\pgfsetstrokecolor{currentstroke}%
\pgfsetdash{}{0pt}%
\pgfsys@defobject{currentmarker}{\pgfqpoint{0.000000in}{-0.048611in}}{\pgfqpoint{0.000000in}{0.000000in}}{%
\pgfpathmoveto{\pgfqpoint{0.000000in}{0.000000in}}%
\pgfpathlineto{\pgfqpoint{0.000000in}{-0.048611in}}%
\pgfusepath{stroke,fill}%
}%
\begin{pgfscope}%
\pgfsys@transformshift{2.178248in}{0.416447in}%
\pgfsys@useobject{currentmarker}{}%
\end{pgfscope}%
\end{pgfscope}%
\begin{pgfscope}%
\definecolor{textcolor}{rgb}{0.000000,0.000000,0.000000}%
\pgfsetstrokecolor{textcolor}%
\pgfsetfillcolor{textcolor}%
\pgftext[x=2.178248in,y=0.319225in,,top]{\color{textcolor}\rmfamily\fontsize{8.000000}{9.600000}\selectfont \(\displaystyle {15000}\)}%
\end{pgfscope}%
\begin{pgfscope}%
\definecolor{textcolor}{rgb}{0.000000,0.000000,0.000000}%
\pgfsetstrokecolor{textcolor}%
\pgfsetfillcolor{textcolor}%
\pgftext[x=1.494510in,y=0.165003in,,top]{\color{textcolor}\rmfamily\fontsize{10.000000}{12.000000}\selectfont Time in \unit{\second}}%
\end{pgfscope}%
\begin{pgfscope}%
\pgfpathrectangle{\pgfqpoint{0.589745in}{0.416447in}}{\pgfqpoint{1.809530in}{1.372591in}}%
\pgfusepath{clip}%
\pgfsetrectcap%
\pgfsetroundjoin%
\pgfsetlinewidth{0.803000pt}%
\definecolor{currentstroke}{rgb}{0.450000,0.450000,0.450000}%
\pgfsetstrokecolor{currentstroke}%
\pgfsetdash{}{0pt}%
\pgfpathmoveto{\pgfqpoint{0.589745in}{0.416447in}}%
\pgfpathlineto{\pgfqpoint{2.399275in}{0.416447in}}%
\pgfusepath{stroke}%
\end{pgfscope}%
\begin{pgfscope}%
\pgfsetbuttcap%
\pgfsetroundjoin%
\definecolor{currentfill}{rgb}{0.000000,0.000000,0.000000}%
\pgfsetfillcolor{currentfill}%
\pgfsetlinewidth{0.803000pt}%
\definecolor{currentstroke}{rgb}{0.000000,0.000000,0.000000}%
\pgfsetstrokecolor{currentstroke}%
\pgfsetdash{}{0pt}%
\pgfsys@defobject{currentmarker}{\pgfqpoint{-0.048611in}{0.000000in}}{\pgfqpoint{-0.000000in}{0.000000in}}{%
\pgfpathmoveto{\pgfqpoint{-0.000000in}{0.000000in}}%
\pgfpathlineto{\pgfqpoint{-0.048611in}{0.000000in}}%
\pgfusepath{stroke,fill}%
}%
\begin{pgfscope}%
\pgfsys@transformshift{0.589745in}{0.416447in}%
\pgfsys@useobject{currentmarker}{}%
\end{pgfscope}%
\end{pgfscope}%
\begin{pgfscope}%
\definecolor{textcolor}{rgb}{0.000000,0.000000,0.000000}%
\pgfsetstrokecolor{textcolor}%
\pgfsetfillcolor{textcolor}%
\pgftext[x=0.223614in, y=0.377892in, left, base]{\color{textcolor}\rmfamily\fontsize{8.000000}{9.600000}\selectfont \(\displaystyle {\ensuremath{-}200}\)}%
\end{pgfscope}%
\begin{pgfscope}%
\pgfpathrectangle{\pgfqpoint{0.589745in}{0.416447in}}{\pgfqpoint{1.809530in}{1.372591in}}%
\pgfusepath{clip}%
\pgfsetrectcap%
\pgfsetroundjoin%
\pgfsetlinewidth{0.803000pt}%
\definecolor{currentstroke}{rgb}{0.450000,0.450000,0.450000}%
\pgfsetstrokecolor{currentstroke}%
\pgfsetdash{}{0pt}%
\pgfpathmoveto{\pgfqpoint{0.589745in}{0.721468in}}%
\pgfpathlineto{\pgfqpoint{2.399275in}{0.721468in}}%
\pgfusepath{stroke}%
\end{pgfscope}%
\begin{pgfscope}%
\pgfsetbuttcap%
\pgfsetroundjoin%
\definecolor{currentfill}{rgb}{0.000000,0.000000,0.000000}%
\pgfsetfillcolor{currentfill}%
\pgfsetlinewidth{0.803000pt}%
\definecolor{currentstroke}{rgb}{0.000000,0.000000,0.000000}%
\pgfsetstrokecolor{currentstroke}%
\pgfsetdash{}{0pt}%
\pgfsys@defobject{currentmarker}{\pgfqpoint{-0.048611in}{0.000000in}}{\pgfqpoint{-0.000000in}{0.000000in}}{%
\pgfpathmoveto{\pgfqpoint{-0.000000in}{0.000000in}}%
\pgfpathlineto{\pgfqpoint{-0.048611in}{0.000000in}}%
\pgfusepath{stroke,fill}%
}%
\begin{pgfscope}%
\pgfsys@transformshift{0.589745in}{0.721468in}%
\pgfsys@useobject{currentmarker}{}%
\end{pgfscope}%
\end{pgfscope}%
\begin{pgfscope}%
\definecolor{textcolor}{rgb}{0.000000,0.000000,0.000000}%
\pgfsetstrokecolor{textcolor}%
\pgfsetfillcolor{textcolor}%
\pgftext[x=0.223614in, y=0.682912in, left, base]{\color{textcolor}\rmfamily\fontsize{8.000000}{9.600000}\selectfont \(\displaystyle {\ensuremath{-}100}\)}%
\end{pgfscope}%
\begin{pgfscope}%
\pgfpathrectangle{\pgfqpoint{0.589745in}{0.416447in}}{\pgfqpoint{1.809530in}{1.372591in}}%
\pgfusepath{clip}%
\pgfsetrectcap%
\pgfsetroundjoin%
\pgfsetlinewidth{0.803000pt}%
\definecolor{currentstroke}{rgb}{0.450000,0.450000,0.450000}%
\pgfsetstrokecolor{currentstroke}%
\pgfsetdash{}{0pt}%
\pgfpathmoveto{\pgfqpoint{0.589745in}{1.026488in}}%
\pgfpathlineto{\pgfqpoint{2.399275in}{1.026488in}}%
\pgfusepath{stroke}%
\end{pgfscope}%
\begin{pgfscope}%
\pgfsetbuttcap%
\pgfsetroundjoin%
\definecolor{currentfill}{rgb}{0.000000,0.000000,0.000000}%
\pgfsetfillcolor{currentfill}%
\pgfsetlinewidth{0.803000pt}%
\definecolor{currentstroke}{rgb}{0.000000,0.000000,0.000000}%
\pgfsetstrokecolor{currentstroke}%
\pgfsetdash{}{0pt}%
\pgfsys@defobject{currentmarker}{\pgfqpoint{-0.048611in}{0.000000in}}{\pgfqpoint{-0.000000in}{0.000000in}}{%
\pgfpathmoveto{\pgfqpoint{-0.000000in}{0.000000in}}%
\pgfpathlineto{\pgfqpoint{-0.048611in}{0.000000in}}%
\pgfusepath{stroke,fill}%
}%
\begin{pgfscope}%
\pgfsys@transformshift{0.589745in}{1.026488in}%
\pgfsys@useobject{currentmarker}{}%
\end{pgfscope}%
\end{pgfscope}%
\begin{pgfscope}%
\definecolor{textcolor}{rgb}{0.000000,0.000000,0.000000}%
\pgfsetstrokecolor{textcolor}%
\pgfsetfillcolor{textcolor}%
\pgftext[x=0.433494in, y=0.987932in, left, base]{\color{textcolor}\rmfamily\fontsize{8.000000}{9.600000}\selectfont \(\displaystyle {0}\)}%
\end{pgfscope}%
\begin{pgfscope}%
\pgfpathrectangle{\pgfqpoint{0.589745in}{0.416447in}}{\pgfqpoint{1.809530in}{1.372591in}}%
\pgfusepath{clip}%
\pgfsetrectcap%
\pgfsetroundjoin%
\pgfsetlinewidth{0.803000pt}%
\definecolor{currentstroke}{rgb}{0.450000,0.450000,0.450000}%
\pgfsetstrokecolor{currentstroke}%
\pgfsetdash{}{0pt}%
\pgfpathmoveto{\pgfqpoint{0.589745in}{1.331508in}}%
\pgfpathlineto{\pgfqpoint{2.399275in}{1.331508in}}%
\pgfusepath{stroke}%
\end{pgfscope}%
\begin{pgfscope}%
\pgfsetbuttcap%
\pgfsetroundjoin%
\definecolor{currentfill}{rgb}{0.000000,0.000000,0.000000}%
\pgfsetfillcolor{currentfill}%
\pgfsetlinewidth{0.803000pt}%
\definecolor{currentstroke}{rgb}{0.000000,0.000000,0.000000}%
\pgfsetstrokecolor{currentstroke}%
\pgfsetdash{}{0pt}%
\pgfsys@defobject{currentmarker}{\pgfqpoint{-0.048611in}{0.000000in}}{\pgfqpoint{-0.000000in}{0.000000in}}{%
\pgfpathmoveto{\pgfqpoint{-0.000000in}{0.000000in}}%
\pgfpathlineto{\pgfqpoint{-0.048611in}{0.000000in}}%
\pgfusepath{stroke,fill}%
}%
\begin{pgfscope}%
\pgfsys@transformshift{0.589745in}{1.331508in}%
\pgfsys@useobject{currentmarker}{}%
\end{pgfscope}%
\end{pgfscope}%
\begin{pgfscope}%
\definecolor{textcolor}{rgb}{0.000000,0.000000,0.000000}%
\pgfsetstrokecolor{textcolor}%
\pgfsetfillcolor{textcolor}%
\pgftext[x=0.315437in, y=1.292953in, left, base]{\color{textcolor}\rmfamily\fontsize{8.000000}{9.600000}\selectfont \(\displaystyle {100}\)}%
\end{pgfscope}%
\begin{pgfscope}%
\pgfpathrectangle{\pgfqpoint{0.589745in}{0.416447in}}{\pgfqpoint{1.809530in}{1.372591in}}%
\pgfusepath{clip}%
\pgfsetrectcap%
\pgfsetroundjoin%
\pgfsetlinewidth{0.803000pt}%
\definecolor{currentstroke}{rgb}{0.450000,0.450000,0.450000}%
\pgfsetstrokecolor{currentstroke}%
\pgfsetdash{}{0pt}%
\pgfpathmoveto{\pgfqpoint{0.589745in}{1.636529in}}%
\pgfpathlineto{\pgfqpoint{2.399275in}{1.636529in}}%
\pgfusepath{stroke}%
\end{pgfscope}%
\begin{pgfscope}%
\pgfsetbuttcap%
\pgfsetroundjoin%
\definecolor{currentfill}{rgb}{0.000000,0.000000,0.000000}%
\pgfsetfillcolor{currentfill}%
\pgfsetlinewidth{0.803000pt}%
\definecolor{currentstroke}{rgb}{0.000000,0.000000,0.000000}%
\pgfsetstrokecolor{currentstroke}%
\pgfsetdash{}{0pt}%
\pgfsys@defobject{currentmarker}{\pgfqpoint{-0.048611in}{0.000000in}}{\pgfqpoint{-0.000000in}{0.000000in}}{%
\pgfpathmoveto{\pgfqpoint{-0.000000in}{0.000000in}}%
\pgfpathlineto{\pgfqpoint{-0.048611in}{0.000000in}}%
\pgfusepath{stroke,fill}%
}%
\begin{pgfscope}%
\pgfsys@transformshift{0.589745in}{1.636529in}%
\pgfsys@useobject{currentmarker}{}%
\end{pgfscope}%
\end{pgfscope}%
\begin{pgfscope}%
\definecolor{textcolor}{rgb}{0.000000,0.000000,0.000000}%
\pgfsetstrokecolor{textcolor}%
\pgfsetfillcolor{textcolor}%
\pgftext[x=0.315437in, y=1.597973in, left, base]{\color{textcolor}\rmfamily\fontsize{8.000000}{9.600000}\selectfont \(\displaystyle {200}\)}%
\end{pgfscope}%
\begin{pgfscope}%
\definecolor{textcolor}{rgb}{0.000000,0.000000,0.000000}%
\pgfsetstrokecolor{textcolor}%
\pgfsetfillcolor{textcolor}%
\pgftext[x=0.168059in,y=1.102743in,,bottom,rotate=90.000000]{\color{textcolor}\rmfamily\fontsize{10.000000}{12.000000}\selectfont Ampl. in arb. unit}%
\end{pgfscope}%
\begin{pgfscope}%
\pgfpathrectangle{\pgfqpoint{0.589745in}{0.416447in}}{\pgfqpoint{1.809530in}{1.372591in}}%
\pgfusepath{clip}%
\pgfsetrectcap%
\pgfsetroundjoin%
\pgfsetlinewidth{1.505625pt}%
\definecolor{currentstroke}{rgb}{0.835294,0.368627,0.000000}%
\pgfsetstrokecolor{currentstroke}%
\pgfsetdash{}{0pt}%
\pgfpathmoveto{\pgfqpoint{0.671996in}{1.028382in}}%
\pgfpathlineto{\pgfqpoint{0.672799in}{1.050160in}}%
\pgfpathlineto{\pgfqpoint{0.673804in}{1.008388in}}%
\pgfpathlineto{\pgfqpoint{0.677117in}{0.963029in}}%
\pgfpathlineto{\pgfqpoint{0.677419in}{0.976628in}}%
\pgfpathlineto{\pgfqpoint{0.679226in}{0.994781in}}%
\pgfpathlineto{\pgfqpoint{0.682239in}{0.955876in}}%
\pgfpathlineto{\pgfqpoint{0.684448in}{0.990350in}}%
\pgfpathlineto{\pgfqpoint{0.686657in}{0.956920in}}%
\pgfpathlineto{\pgfqpoint{0.689971in}{1.008738in}}%
\pgfpathlineto{\pgfqpoint{0.691879in}{0.983409in}}%
\pgfpathlineto{\pgfqpoint{0.693284in}{1.033325in}}%
\pgfpathlineto{\pgfqpoint{0.695293in}{1.017473in}}%
\pgfpathlineto{\pgfqpoint{0.697301in}{1.038142in}}%
\pgfpathlineto{\pgfqpoint{0.698506in}{1.001274in}}%
\pgfpathlineto{\pgfqpoint{0.700414in}{1.017613in}}%
\pgfpathlineto{\pgfqpoint{0.701217in}{1.000983in}}%
\pgfpathlineto{\pgfqpoint{0.705435in}{1.059540in}}%
\pgfpathlineto{\pgfqpoint{0.706941in}{1.026195in}}%
\pgfpathlineto{\pgfqpoint{0.709452in}{1.068101in}}%
\pgfpathlineto{\pgfqpoint{0.710456in}{1.051311in}}%
\pgfpathlineto{\pgfqpoint{0.711460in}{1.072581in}}%
\pgfpathlineto{\pgfqpoint{0.713267in}{1.050411in}}%
\pgfpathlineto{\pgfqpoint{0.714874in}{1.109093in}}%
\pgfpathlineto{\pgfqpoint{0.719292in}{1.027176in}}%
\pgfpathlineto{\pgfqpoint{0.719995in}{1.068644in}}%
\pgfpathlineto{\pgfqpoint{0.722204in}{1.059532in}}%
\pgfpathlineto{\pgfqpoint{0.724414in}{1.028869in}}%
\pgfpathlineto{\pgfqpoint{0.725819in}{0.981630in}}%
\pgfpathlineto{\pgfqpoint{0.728330in}{0.995090in}}%
\pgfpathlineto{\pgfqpoint{0.729033in}{0.969724in}}%
\pgfpathlineto{\pgfqpoint{0.730639in}{0.992702in}}%
\pgfpathlineto{\pgfqpoint{0.732949in}{0.986556in}}%
\pgfpathlineto{\pgfqpoint{0.734556in}{1.038679in}}%
\pgfpathlineto{\pgfqpoint{0.736765in}{0.989943in}}%
\pgfpathlineto{\pgfqpoint{0.737970in}{1.015469in}}%
\pgfpathlineto{\pgfqpoint{0.739777in}{0.975342in}}%
\pgfpathlineto{\pgfqpoint{0.741284in}{0.998304in}}%
\pgfpathlineto{\pgfqpoint{0.742991in}{0.974466in}}%
\pgfpathlineto{\pgfqpoint{0.743895in}{0.990290in}}%
\pgfpathlineto{\pgfqpoint{0.746104in}{0.990028in}}%
\pgfpathlineto{\pgfqpoint{0.747409in}{0.938439in}}%
\pgfpathlineto{\pgfqpoint{0.749618in}{0.996588in}}%
\pgfpathlineto{\pgfqpoint{0.751627in}{0.983846in}}%
\pgfpathlineto{\pgfqpoint{0.753334in}{0.986717in}}%
\pgfpathlineto{\pgfqpoint{0.754840in}{1.033789in}}%
\pgfpathlineto{\pgfqpoint{0.755744in}{1.007055in}}%
\pgfpathlineto{\pgfqpoint{0.758355in}{1.018001in}}%
\pgfpathlineto{\pgfqpoint{0.760062in}{1.039121in}}%
\pgfpathlineto{\pgfqpoint{0.761769in}{1.111956in}}%
\pgfpathlineto{\pgfqpoint{0.765384in}{1.156039in}}%
\pgfpathlineto{\pgfqpoint{0.767492in}{1.121262in}}%
\pgfpathlineto{\pgfqpoint{0.768999in}{1.172235in}}%
\pgfpathlineto{\pgfqpoint{0.770103in}{1.126633in}}%
\pgfpathlineto{\pgfqpoint{0.771810in}{1.115731in}}%
\pgfpathlineto{\pgfqpoint{0.773718in}{1.178864in}}%
\pgfpathlineto{\pgfqpoint{0.776028in}{1.182389in}}%
\pgfpathlineto{\pgfqpoint{0.777233in}{1.215041in}}%
\pgfpathlineto{\pgfqpoint{0.780045in}{1.183131in}}%
\pgfpathlineto{\pgfqpoint{0.782455in}{1.203215in}}%
\pgfpathlineto{\pgfqpoint{0.785969in}{1.283414in}}%
\pgfpathlineto{\pgfqpoint{0.787676in}{1.252093in}}%
\pgfpathlineto{\pgfqpoint{0.789484in}{1.302433in}}%
\pgfpathlineto{\pgfqpoint{0.791793in}{1.256289in}}%
\pgfpathlineto{\pgfqpoint{0.794203in}{1.256062in}}%
\pgfpathlineto{\pgfqpoint{0.796312in}{1.307795in}}%
\pgfpathlineto{\pgfqpoint{0.797617in}{1.271131in}}%
\pgfpathlineto{\pgfqpoint{0.799224in}{1.265744in}}%
\pgfpathlineto{\pgfqpoint{0.800730in}{1.303382in}}%
\pgfpathlineto{\pgfqpoint{0.801935in}{1.287507in}}%
\pgfpathlineto{\pgfqpoint{0.803743in}{1.311649in}}%
\pgfpathlineto{\pgfqpoint{0.806856in}{1.308349in}}%
\pgfpathlineto{\pgfqpoint{0.808262in}{1.249344in}}%
\pgfpathlineto{\pgfqpoint{0.809768in}{1.234533in}}%
\pgfpathlineto{\pgfqpoint{0.811174in}{1.279436in}}%
\pgfpathlineto{\pgfqpoint{0.812479in}{1.269969in}}%
\pgfpathlineto{\pgfqpoint{0.814588in}{1.327813in}}%
\pgfpathlineto{\pgfqpoint{0.816094in}{1.314642in}}%
\pgfpathlineto{\pgfqpoint{0.817902in}{1.367676in}}%
\pgfpathlineto{\pgfqpoint{0.818906in}{1.358873in}}%
\pgfpathlineto{\pgfqpoint{0.821316in}{1.355364in}}%
\pgfpathlineto{\pgfqpoint{0.822521in}{1.419301in}}%
\pgfpathlineto{\pgfqpoint{0.825232in}{1.444840in}}%
\pgfpathlineto{\pgfqpoint{0.826437in}{1.396328in}}%
\pgfpathlineto{\pgfqpoint{0.827441in}{1.418052in}}%
\pgfpathlineto{\pgfqpoint{0.829249in}{1.408509in}}%
\pgfpathlineto{\pgfqpoint{0.832060in}{1.352404in}}%
\pgfpathlineto{\pgfqpoint{0.833667in}{1.353689in}}%
\pgfpathlineto{\pgfqpoint{0.838588in}{1.443224in}}%
\pgfpathlineto{\pgfqpoint{0.839893in}{1.424151in}}%
\pgfpathlineto{\pgfqpoint{0.843307in}{1.478097in}}%
\pgfpathlineto{\pgfqpoint{0.844512in}{1.471360in}}%
\pgfpathlineto{\pgfqpoint{0.846320in}{1.473487in}}%
\pgfpathlineto{\pgfqpoint{0.849734in}{1.531018in}}%
\pgfpathlineto{\pgfqpoint{0.851340in}{1.483831in}}%
\pgfpathlineto{\pgfqpoint{0.853951in}{1.475102in}}%
\pgfpathlineto{\pgfqpoint{0.854955in}{1.442567in}}%
\pgfpathlineto{\pgfqpoint{0.857165in}{1.432908in}}%
\pgfpathlineto{\pgfqpoint{0.858872in}{1.465357in}}%
\pgfpathlineto{\pgfqpoint{0.859876in}{1.449097in}}%
\pgfpathlineto{\pgfqpoint{0.863190in}{1.468953in}}%
\pgfpathlineto{\pgfqpoint{0.863491in}{1.448393in}}%
\pgfpathlineto{\pgfqpoint{0.865399in}{1.433213in}}%
\pgfpathlineto{\pgfqpoint{0.868311in}{1.448486in}}%
\pgfpathlineto{\pgfqpoint{0.869114in}{1.482138in}}%
\pgfpathlineto{\pgfqpoint{0.870922in}{1.468953in}}%
\pgfpathlineto{\pgfqpoint{0.872227in}{1.512453in}}%
\pgfpathlineto{\pgfqpoint{0.875139in}{1.440697in}}%
\pgfpathlineto{\pgfqpoint{0.876344in}{1.482798in}}%
\pgfpathlineto{\pgfqpoint{0.878353in}{1.444191in}}%
\pgfpathlineto{\pgfqpoint{0.880060in}{1.466788in}}%
\pgfpathlineto{\pgfqpoint{0.881566in}{1.436556in}}%
\pgfpathlineto{\pgfqpoint{0.882369in}{1.453445in}}%
\pgfpathlineto{\pgfqpoint{0.884679in}{1.416311in}}%
\pgfpathlineto{\pgfqpoint{0.886486in}{1.454866in}}%
\pgfpathlineto{\pgfqpoint{0.887792in}{1.427612in}}%
\pgfpathlineto{\pgfqpoint{0.889198in}{1.446298in}}%
\pgfpathlineto{\pgfqpoint{0.891808in}{1.428288in}}%
\pgfpathlineto{\pgfqpoint{0.893013in}{1.445594in}}%
\pgfpathlineto{\pgfqpoint{0.895423in}{1.457889in}}%
\pgfpathlineto{\pgfqpoint{0.896628in}{1.435393in}}%
\pgfpathlineto{\pgfqpoint{0.897934in}{1.453342in}}%
\pgfpathlineto{\pgfqpoint{0.901649in}{1.411022in}}%
\pgfpathlineto{\pgfqpoint{0.903155in}{1.432137in}}%
\pgfpathlineto{\pgfqpoint{0.904461in}{1.490675in}}%
\pgfpathlineto{\pgfqpoint{0.907072in}{1.504084in}}%
\pgfpathlineto{\pgfqpoint{0.909080in}{1.474221in}}%
\pgfpathlineto{\pgfqpoint{0.910386in}{1.486989in}}%
\pgfpathlineto{\pgfqpoint{0.913197in}{1.418903in}}%
\pgfpathlineto{\pgfqpoint{0.915005in}{1.450529in}}%
\pgfpathlineto{\pgfqpoint{0.917013in}{1.455788in}}%
\pgfpathlineto{\pgfqpoint{0.918218in}{1.430496in}}%
\pgfpathlineto{\pgfqpoint{0.921130in}{1.485158in}}%
\pgfpathlineto{\pgfqpoint{0.922134in}{1.464130in}}%
\pgfpathlineto{\pgfqpoint{0.923641in}{1.487643in}}%
\pgfpathlineto{\pgfqpoint{0.924846in}{1.464594in}}%
\pgfpathlineto{\pgfqpoint{0.926653in}{1.501245in}}%
\pgfpathlineto{\pgfqpoint{0.929565in}{1.467042in}}%
\pgfpathlineto{\pgfqpoint{0.931373in}{1.512224in}}%
\pgfpathlineto{\pgfqpoint{0.931774in}{1.496649in}}%
\pgfpathlineto{\pgfqpoint{0.933682in}{1.459377in}}%
\pgfpathlineto{\pgfqpoint{0.936494in}{1.521512in}}%
\pgfpathlineto{\pgfqpoint{0.937699in}{1.505337in}}%
\pgfpathlineto{\pgfqpoint{0.938904in}{1.525365in}}%
\pgfpathlineto{\pgfqpoint{0.940711in}{1.487793in}}%
\pgfpathlineto{\pgfqpoint{0.942318in}{1.488486in}}%
\pgfpathlineto{\pgfqpoint{0.943523in}{1.537586in}}%
\pgfpathlineto{\pgfqpoint{0.945230in}{1.542311in}}%
\pgfpathlineto{\pgfqpoint{0.947339in}{1.608473in}}%
\pgfpathlineto{\pgfqpoint{0.949749in}{1.603955in}}%
\pgfpathlineto{\pgfqpoint{0.952862in}{1.534210in}}%
\pgfpathlineto{\pgfqpoint{0.954167in}{1.530786in}}%
\pgfpathlineto{\pgfqpoint{0.956878in}{1.582461in}}%
\pgfpathlineto{\pgfqpoint{0.958083in}{1.555660in}}%
\pgfpathlineto{\pgfqpoint{0.960192in}{1.558678in}}%
\pgfpathlineto{\pgfqpoint{0.961698in}{1.487920in}}%
\pgfpathlineto{\pgfqpoint{0.962602in}{1.513358in}}%
\pgfpathlineto{\pgfqpoint{0.964309in}{1.494956in}}%
\pgfpathlineto{\pgfqpoint{0.966619in}{1.551913in}}%
\pgfpathlineto{\pgfqpoint{0.968025in}{1.512647in}}%
\pgfpathlineto{\pgfqpoint{0.970334in}{1.503784in}}%
\pgfpathlineto{\pgfqpoint{0.971439in}{1.539301in}}%
\pgfpathlineto{\pgfqpoint{0.972744in}{1.539295in}}%
\pgfpathlineto{\pgfqpoint{0.976058in}{1.489378in}}%
\pgfpathlineto{\pgfqpoint{0.979573in}{1.439862in}}%
\pgfpathlineto{\pgfqpoint{0.981179in}{1.486773in}}%
\pgfpathlineto{\pgfqpoint{0.983489in}{1.492912in}}%
\pgfpathlineto{\pgfqpoint{0.984493in}{1.464048in}}%
\pgfpathlineto{\pgfqpoint{0.987204in}{1.475706in}}%
\pgfpathlineto{\pgfqpoint{0.990920in}{1.375985in}}%
\pgfpathlineto{\pgfqpoint{0.992426in}{1.413554in}}%
\pgfpathlineto{\pgfqpoint{0.993832in}{1.384876in}}%
\pgfpathlineto{\pgfqpoint{0.994736in}{1.426587in}}%
\pgfpathlineto{\pgfqpoint{0.996443in}{1.408110in}}%
\pgfpathlineto{\pgfqpoint{1.001162in}{1.471423in}}%
\pgfpathlineto{\pgfqpoint{1.003070in}{1.440875in}}%
\pgfpathlineto{\pgfqpoint{1.004476in}{1.472041in}}%
\pgfpathlineto{\pgfqpoint{1.004978in}{1.461496in}}%
\pgfpathlineto{\pgfqpoint{1.008091in}{1.503035in}}%
\pgfpathlineto{\pgfqpoint{1.009698in}{1.447446in}}%
\pgfpathlineto{\pgfqpoint{1.011204in}{1.447175in}}%
\pgfpathlineto{\pgfqpoint{1.011806in}{1.474315in}}%
\pgfpathlineto{\pgfqpoint{1.014518in}{1.499617in}}%
\pgfpathlineto{\pgfqpoint{1.015723in}{1.478897in}}%
\pgfpathlineto{\pgfqpoint{1.016928in}{1.532884in}}%
\pgfpathlineto{\pgfqpoint{1.018735in}{1.487847in}}%
\pgfpathlineto{\pgfqpoint{1.020543in}{1.509778in}}%
\pgfpathlineto{\pgfqpoint{1.022451in}{1.470820in}}%
\pgfpathlineto{\pgfqpoint{1.025162in}{1.490717in}}%
\pgfpathlineto{\pgfqpoint{1.026266in}{1.457367in}}%
\pgfpathlineto{\pgfqpoint{1.028074in}{1.514357in}}%
\pgfpathlineto{\pgfqpoint{1.029580in}{1.484648in}}%
\pgfpathlineto{\pgfqpoint{1.030584in}{1.516197in}}%
\pgfpathlineto{\pgfqpoint{1.032291in}{1.505736in}}%
\pgfpathlineto{\pgfqpoint{1.034099in}{1.536619in}}%
\pgfpathlineto{\pgfqpoint{1.035806in}{1.535506in}}%
\pgfpathlineto{\pgfqpoint{1.037915in}{1.507509in}}%
\pgfpathlineto{\pgfqpoint{1.039220in}{1.507487in}}%
\pgfpathlineto{\pgfqpoint{1.041931in}{1.536696in}}%
\pgfpathlineto{\pgfqpoint{1.043538in}{1.595170in}}%
\pgfpathlineto{\pgfqpoint{1.045546in}{1.615871in}}%
\pgfpathlineto{\pgfqpoint{1.046651in}{1.576768in}}%
\pgfpathlineto{\pgfqpoint{1.048961in}{1.542125in}}%
\pgfpathlineto{\pgfqpoint{1.050467in}{1.565408in}}%
\pgfpathlineto{\pgfqpoint{1.051772in}{1.505963in}}%
\pgfpathlineto{\pgfqpoint{1.053178in}{1.509923in}}%
\pgfpathlineto{\pgfqpoint{1.055488in}{1.437115in}}%
\pgfpathlineto{\pgfqpoint{1.056693in}{1.448674in}}%
\pgfpathlineto{\pgfqpoint{1.058701in}{1.441163in}}%
\pgfpathlineto{\pgfqpoint{1.060006in}{1.409929in}}%
\pgfpathlineto{\pgfqpoint{1.062919in}{1.379515in}}%
\pgfpathlineto{\pgfqpoint{1.064124in}{1.414242in}}%
\pgfpathlineto{\pgfqpoint{1.065128in}{1.391692in}}%
\pgfpathlineto{\pgfqpoint{1.067136in}{1.428266in}}%
\pgfpathlineto{\pgfqpoint{1.068642in}{1.405846in}}%
\pgfpathlineto{\pgfqpoint{1.071755in}{1.475065in}}%
\pgfpathlineto{\pgfqpoint{1.073261in}{1.424196in}}%
\pgfpathlineto{\pgfqpoint{1.076575in}{1.399480in}}%
\pgfpathlineto{\pgfqpoint{1.076977in}{1.366641in}}%
\pgfpathlineto{\pgfqpoint{1.079688in}{1.406090in}}%
\pgfpathlineto{\pgfqpoint{1.081496in}{1.368484in}}%
\pgfpathlineto{\pgfqpoint{1.083002in}{1.390385in}}%
\pgfpathlineto{\pgfqpoint{1.083705in}{1.372606in}}%
\pgfpathlineto{\pgfqpoint{1.086215in}{1.374139in}}%
\pgfpathlineto{\pgfqpoint{1.087922in}{1.418781in}}%
\pgfpathlineto{\pgfqpoint{1.088625in}{1.396656in}}%
\pgfpathlineto{\pgfqpoint{1.090433in}{1.396206in}}%
\pgfpathlineto{\pgfqpoint{1.092843in}{1.393743in}}%
\pgfpathlineto{\pgfqpoint{1.095654in}{1.290070in}}%
\pgfpathlineto{\pgfqpoint{1.098165in}{1.294178in}}%
\pgfpathlineto{\pgfqpoint{1.100274in}{1.325778in}}%
\pgfpathlineto{\pgfqpoint{1.101378in}{1.278390in}}%
\pgfpathlineto{\pgfqpoint{1.102684in}{1.303671in}}%
\pgfpathlineto{\pgfqpoint{1.104792in}{1.238290in}}%
\pgfpathlineto{\pgfqpoint{1.107102in}{1.260167in}}%
\pgfpathlineto{\pgfqpoint{1.109010in}{1.219246in}}%
\pgfpathlineto{\pgfqpoint{1.110215in}{1.244345in}}%
\pgfpathlineto{\pgfqpoint{1.110918in}{1.221507in}}%
\pgfpathlineto{\pgfqpoint{1.113428in}{1.206574in}}%
\pgfpathlineto{\pgfqpoint{1.114633in}{1.231352in}}%
\pgfpathlineto{\pgfqpoint{1.118349in}{1.153282in}}%
\pgfpathlineto{\pgfqpoint{1.119554in}{1.153523in}}%
\pgfpathlineto{\pgfqpoint{1.121261in}{1.209198in}}%
\pgfpathlineto{\pgfqpoint{1.123470in}{1.204160in}}%
\pgfpathlineto{\pgfqpoint{1.125779in}{1.156658in}}%
\pgfpathlineto{\pgfqpoint{1.126282in}{1.188200in}}%
\pgfpathlineto{\pgfqpoint{1.128089in}{1.180588in}}%
\pgfpathlineto{\pgfqpoint{1.130599in}{1.198954in}}%
\pgfpathlineto{\pgfqpoint{1.131403in}{1.184135in}}%
\pgfpathlineto{\pgfqpoint{1.134114in}{1.226481in}}%
\pgfpathlineto{\pgfqpoint{1.136323in}{1.217661in}}%
\pgfpathlineto{\pgfqpoint{1.137729in}{1.165468in}}%
\pgfpathlineto{\pgfqpoint{1.138332in}{1.188500in}}%
\pgfpathlineto{\pgfqpoint{1.141344in}{1.178205in}}%
\pgfpathlineto{\pgfqpoint{1.142047in}{1.197090in}}%
\pgfpathlineto{\pgfqpoint{1.143955in}{1.173356in}}%
\pgfpathlineto{\pgfqpoint{1.145963in}{1.227555in}}%
\pgfpathlineto{\pgfqpoint{1.147068in}{1.203889in}}%
\pgfpathlineto{\pgfqpoint{1.148474in}{1.220746in}}%
\pgfpathlineto{\pgfqpoint{1.151587in}{1.206983in}}%
\pgfpathlineto{\pgfqpoint{1.152189in}{1.173327in}}%
\pgfpathlineto{\pgfqpoint{1.154499in}{1.127436in}}%
\pgfpathlineto{\pgfqpoint{1.156708in}{1.137973in}}%
\pgfpathlineto{\pgfqpoint{1.157612in}{1.121034in}}%
\pgfpathlineto{\pgfqpoint{1.159720in}{1.150076in}}%
\pgfpathlineto{\pgfqpoint{1.160423in}{1.122385in}}%
\pgfpathlineto{\pgfqpoint{1.162833in}{1.147802in}}%
\pgfpathlineto{\pgfqpoint{1.164942in}{1.098646in}}%
\pgfpathlineto{\pgfqpoint{1.166348in}{1.146187in}}%
\pgfpathlineto{\pgfqpoint{1.167151in}{1.137178in}}%
\pgfpathlineto{\pgfqpoint{1.169360in}{1.169407in}}%
\pgfpathlineto{\pgfqpoint{1.171971in}{1.133395in}}%
\pgfpathlineto{\pgfqpoint{1.173779in}{1.178946in}}%
\pgfpathlineto{\pgfqpoint{1.174482in}{1.155741in}}%
\pgfpathlineto{\pgfqpoint{1.176992in}{1.155095in}}%
\pgfpathlineto{\pgfqpoint{1.177996in}{1.206015in}}%
\pgfpathlineto{\pgfqpoint{1.180306in}{1.185446in}}%
\pgfpathlineto{\pgfqpoint{1.180908in}{1.198371in}}%
\pgfpathlineto{\pgfqpoint{1.182816in}{1.213476in}}%
\pgfpathlineto{\pgfqpoint{1.185025in}{1.190660in}}%
\pgfpathlineto{\pgfqpoint{1.186732in}{1.187343in}}%
\pgfpathlineto{\pgfqpoint{1.189042in}{1.215508in}}%
\pgfpathlineto{\pgfqpoint{1.190548in}{1.214918in}}%
\pgfpathlineto{\pgfqpoint{1.191352in}{1.188916in}}%
\pgfpathlineto{\pgfqpoint{1.193862in}{1.165823in}}%
\pgfpathlineto{\pgfqpoint{1.194665in}{1.194213in}}%
\pgfpathlineto{\pgfqpoint{1.196372in}{1.180284in}}%
\pgfpathlineto{\pgfqpoint{1.198481in}{1.228042in}}%
\pgfpathlineto{\pgfqpoint{1.199887in}{1.179900in}}%
\pgfpathlineto{\pgfqpoint{1.201393in}{1.216189in}}%
\pgfpathlineto{\pgfqpoint{1.204607in}{1.244571in}}%
\pgfpathlineto{\pgfqpoint{1.205711in}{1.219208in}}%
\pgfpathlineto{\pgfqpoint{1.206916in}{1.223489in}}%
\pgfpathlineto{\pgfqpoint{1.208925in}{1.174514in}}%
\pgfpathlineto{\pgfqpoint{1.211335in}{1.144399in}}%
\pgfpathlineto{\pgfqpoint{1.212740in}{1.161641in}}%
\pgfpathlineto{\pgfqpoint{1.213544in}{1.126150in}}%
\pgfpathlineto{\pgfqpoint{1.215652in}{1.115154in}}%
\pgfpathlineto{\pgfqpoint{1.218765in}{1.067327in}}%
\pgfpathlineto{\pgfqpoint{1.220975in}{1.069277in}}%
\pgfpathlineto{\pgfqpoint{1.222280in}{1.098013in}}%
\pgfpathlineto{\pgfqpoint{1.223585in}{1.101671in}}%
\pgfpathlineto{\pgfqpoint{1.226698in}{1.053406in}}%
\pgfpathlineto{\pgfqpoint{1.228405in}{1.096440in}}%
\pgfpathlineto{\pgfqpoint{1.228707in}{1.084638in}}%
\pgfpathlineto{\pgfqpoint{1.231016in}{1.113279in}}%
\pgfpathlineto{\pgfqpoint{1.232020in}{1.076230in}}%
\pgfpathlineto{\pgfqpoint{1.235334in}{1.110496in}}%
\pgfpathlineto{\pgfqpoint{1.236439in}{1.082863in}}%
\pgfpathlineto{\pgfqpoint{1.237845in}{1.111997in}}%
\pgfpathlineto{\pgfqpoint{1.238849in}{1.091113in}}%
\pgfpathlineto{\pgfqpoint{1.242062in}{1.127060in}}%
\pgfpathlineto{\pgfqpoint{1.243568in}{1.085421in}}%
\pgfpathlineto{\pgfqpoint{1.244271in}{1.093728in}}%
\pgfpathlineto{\pgfqpoint{1.246079in}{1.094918in}}%
\pgfpathlineto{\pgfqpoint{1.247384in}{1.062650in}}%
\pgfpathlineto{\pgfqpoint{1.249794in}{1.032432in}}%
\pgfpathlineto{\pgfqpoint{1.252405in}{1.017360in}}%
\pgfpathlineto{\pgfqpoint{1.253008in}{1.042951in}}%
\pgfpathlineto{\pgfqpoint{1.255518in}{1.055169in}}%
\pgfpathlineto{\pgfqpoint{1.256120in}{1.031493in}}%
\pgfpathlineto{\pgfqpoint{1.258028in}{1.020722in}}%
\pgfpathlineto{\pgfqpoint{1.261744in}{0.936060in}}%
\pgfpathlineto{\pgfqpoint{1.265058in}{0.857218in}}%
\pgfpathlineto{\pgfqpoint{1.267668in}{0.847558in}}%
\pgfpathlineto{\pgfqpoint{1.268271in}{0.875230in}}%
\pgfpathlineto{\pgfqpoint{1.269576in}{0.854532in}}%
\pgfpathlineto{\pgfqpoint{1.273995in}{0.937688in}}%
\pgfpathlineto{\pgfqpoint{1.275099in}{0.903673in}}%
\pgfpathlineto{\pgfqpoint{1.277007in}{0.948726in}}%
\pgfpathlineto{\pgfqpoint{1.278112in}{0.902784in}}%
\pgfpathlineto{\pgfqpoint{1.280723in}{0.990180in}}%
\pgfpathlineto{\pgfqpoint{1.282731in}{1.001967in}}%
\pgfpathlineto{\pgfqpoint{1.284840in}{0.975425in}}%
\pgfpathlineto{\pgfqpoint{1.285040in}{0.993391in}}%
\pgfpathlineto{\pgfqpoint{1.286647in}{0.975199in}}%
\pgfpathlineto{\pgfqpoint{1.289459in}{0.993476in}}%
\pgfpathlineto{\pgfqpoint{1.290764in}{0.963511in}}%
\pgfpathlineto{\pgfqpoint{1.293375in}{0.983902in}}%
\pgfpathlineto{\pgfqpoint{1.294580in}{0.948053in}}%
\pgfpathlineto{\pgfqpoint{1.295584in}{0.972862in}}%
\pgfpathlineto{\pgfqpoint{1.297894in}{0.946215in}}%
\pgfpathlineto{\pgfqpoint{1.299802in}{0.964695in}}%
\pgfpathlineto{\pgfqpoint{1.300404in}{0.944187in}}%
\pgfpathlineto{\pgfqpoint{1.303618in}{0.975300in}}%
\pgfpathlineto{\pgfqpoint{1.304622in}{0.952959in}}%
\pgfpathlineto{\pgfqpoint{1.306831in}{0.995901in}}%
\pgfpathlineto{\pgfqpoint{1.307132in}{0.983110in}}%
\pgfpathlineto{\pgfqpoint{1.309542in}{0.994464in}}%
\pgfpathlineto{\pgfqpoint{1.310546in}{0.963021in}}%
\pgfpathlineto{\pgfqpoint{1.312555in}{1.000053in}}%
\pgfpathlineto{\pgfqpoint{1.314764in}{0.962607in}}%
\pgfpathlineto{\pgfqpoint{1.317274in}{0.970348in}}%
\pgfpathlineto{\pgfqpoint{1.320186in}{0.883459in}}%
\pgfpathlineto{\pgfqpoint{1.322898in}{0.837076in}}%
\pgfpathlineto{\pgfqpoint{1.325709in}{0.887322in}}%
\pgfpathlineto{\pgfqpoint{1.326814in}{0.871764in}}%
\pgfpathlineto{\pgfqpoint{1.328119in}{0.898262in}}%
\pgfpathlineto{\pgfqpoint{1.329324in}{0.851760in}}%
\pgfpathlineto{\pgfqpoint{1.332136in}{0.907239in}}%
\pgfpathlineto{\pgfqpoint{1.332939in}{0.880761in}}%
\pgfpathlineto{\pgfqpoint{1.335450in}{0.864891in}}%
\pgfpathlineto{\pgfqpoint{1.336353in}{0.902078in}}%
\pgfpathlineto{\pgfqpoint{1.338663in}{0.846935in}}%
\pgfpathlineto{\pgfqpoint{1.340872in}{0.869714in}}%
\pgfpathlineto{\pgfqpoint{1.341274in}{0.840102in}}%
\pgfpathlineto{\pgfqpoint{1.343182in}{0.847296in}}%
\pgfpathlineto{\pgfqpoint{1.345291in}{0.801427in}}%
\pgfpathlineto{\pgfqpoint{1.348504in}{0.869384in}}%
\pgfpathlineto{\pgfqpoint{1.350211in}{0.798934in}}%
\pgfpathlineto{\pgfqpoint{1.351516in}{0.841236in}}%
\pgfpathlineto{\pgfqpoint{1.354629in}{0.858492in}}%
\pgfpathlineto{\pgfqpoint{1.356437in}{0.784280in}}%
\pgfpathlineto{\pgfqpoint{1.357140in}{0.803925in}}%
\pgfpathlineto{\pgfqpoint{1.358847in}{0.763057in}}%
\pgfpathlineto{\pgfqpoint{1.361357in}{0.813848in}}%
\pgfpathlineto{\pgfqpoint{1.363366in}{0.757704in}}%
\pgfpathlineto{\pgfqpoint{1.363868in}{0.787371in}}%
\pgfpathlineto{\pgfqpoint{1.365776in}{0.735035in}}%
\pgfpathlineto{\pgfqpoint{1.366981in}{0.760654in}}%
\pgfpathlineto{\pgfqpoint{1.370395in}{0.713782in}}%
\pgfpathlineto{\pgfqpoint{1.372604in}{0.783180in}}%
\pgfpathlineto{\pgfqpoint{1.374913in}{0.773549in}}%
\pgfpathlineto{\pgfqpoint{1.376821in}{0.838249in}}%
\pgfpathlineto{\pgfqpoint{1.378328in}{0.828675in}}%
\pgfpathlineto{\pgfqpoint{1.379733in}{0.783538in}}%
\pgfpathlineto{\pgfqpoint{1.382947in}{0.837344in}}%
\pgfpathlineto{\pgfqpoint{1.383549in}{0.807796in}}%
\pgfpathlineto{\pgfqpoint{1.385859in}{0.785589in}}%
\pgfpathlineto{\pgfqpoint{1.388168in}{0.799240in}}%
\pgfpathlineto{\pgfqpoint{1.389574in}{0.767216in}}%
\pgfpathlineto{\pgfqpoint{1.391382in}{0.751807in}}%
\pgfpathlineto{\pgfqpoint{1.392486in}{0.781044in}}%
\pgfpathlineto{\pgfqpoint{1.393792in}{0.760591in}}%
\pgfpathlineto{\pgfqpoint{1.395901in}{0.789110in}}%
\pgfpathlineto{\pgfqpoint{1.397106in}{0.763963in}}%
\pgfpathlineto{\pgfqpoint{1.398411in}{0.790263in}}%
\pgfpathlineto{\pgfqpoint{1.400018in}{0.753367in}}%
\pgfpathlineto{\pgfqpoint{1.401223in}{0.810095in}}%
\pgfpathlineto{\pgfqpoint{1.403833in}{0.834806in}}%
\pgfpathlineto{\pgfqpoint{1.406344in}{0.758372in}}%
\pgfpathlineto{\pgfqpoint{1.408453in}{0.809103in}}%
\pgfpathlineto{\pgfqpoint{1.409758in}{0.781256in}}%
\pgfpathlineto{\pgfqpoint{1.411566in}{0.824825in}}%
\pgfpathlineto{\pgfqpoint{1.412469in}{0.815033in}}%
\pgfpathlineto{\pgfqpoint{1.414076in}{0.841226in}}%
\pgfpathlineto{\pgfqpoint{1.416386in}{0.860335in}}%
\pgfpathlineto{\pgfqpoint{1.418796in}{0.801384in}}%
\pgfpathlineto{\pgfqpoint{1.419800in}{0.840051in}}%
\pgfpathlineto{\pgfqpoint{1.421306in}{0.800585in}}%
\pgfpathlineto{\pgfqpoint{1.422511in}{0.816833in}}%
\pgfpathlineto{\pgfqpoint{1.424921in}{0.795460in}}%
\pgfpathlineto{\pgfqpoint{1.426628in}{0.810772in}}%
\pgfpathlineto{\pgfqpoint{1.428938in}{0.761148in}}%
\pgfpathlineto{\pgfqpoint{1.430143in}{0.803163in}}%
\pgfpathlineto{\pgfqpoint{1.432954in}{0.744895in}}%
\pgfpathlineto{\pgfqpoint{1.434461in}{0.764591in}}%
\pgfpathlineto{\pgfqpoint{1.437373in}{0.702493in}}%
\pgfpathlineto{\pgfqpoint{1.438477in}{0.743120in}}%
\pgfpathlineto{\pgfqpoint{1.439984in}{0.710493in}}%
\pgfpathlineto{\pgfqpoint{1.441590in}{0.732853in}}%
\pgfpathlineto{\pgfqpoint{1.443297in}{0.678747in}}%
\pgfpathlineto{\pgfqpoint{1.445105in}{0.721444in}}%
\pgfpathlineto{\pgfqpoint{1.446310in}{0.704742in}}%
\pgfpathlineto{\pgfqpoint{1.448619in}{0.706572in}}%
\pgfpathlineto{\pgfqpoint{1.452736in}{0.792912in}}%
\pgfpathlineto{\pgfqpoint{1.454644in}{0.799048in}}%
\pgfpathlineto{\pgfqpoint{1.456954in}{0.768328in}}%
\pgfpathlineto{\pgfqpoint{1.457657in}{0.787370in}}%
\pgfpathlineto{\pgfqpoint{1.460469in}{0.801491in}}%
\pgfpathlineto{\pgfqpoint{1.461674in}{0.761410in}}%
\pgfpathlineto{\pgfqpoint{1.462577in}{0.811372in}}%
\pgfpathlineto{\pgfqpoint{1.464284in}{0.814237in}}%
\pgfpathlineto{\pgfqpoint{1.465489in}{0.768245in}}%
\pgfpathlineto{\pgfqpoint{1.467297in}{0.793749in}}%
\pgfpathlineto{\pgfqpoint{1.469707in}{0.795745in}}%
\pgfpathlineto{\pgfqpoint{1.470912in}{0.829466in}}%
\pgfpathlineto{\pgfqpoint{1.473422in}{0.824753in}}%
\pgfpathlineto{\pgfqpoint{1.474025in}{0.832572in}}%
\pgfpathlineto{\pgfqpoint{1.476134in}{0.902358in}}%
\pgfpathlineto{\pgfqpoint{1.478041in}{0.854129in}}%
\pgfpathlineto{\pgfqpoint{1.479749in}{0.881071in}}%
\pgfpathlineto{\pgfqpoint{1.480652in}{0.848032in}}%
\pgfpathlineto{\pgfqpoint{1.482359in}{0.857490in}}%
\pgfpathlineto{\pgfqpoint{1.484267in}{0.833707in}}%
\pgfpathlineto{\pgfqpoint{1.485171in}{0.864612in}}%
\pgfpathlineto{\pgfqpoint{1.486878in}{0.789323in}}%
\pgfpathlineto{\pgfqpoint{1.488585in}{0.807767in}}%
\pgfpathlineto{\pgfqpoint{1.489589in}{0.773407in}}%
\pgfpathlineto{\pgfqpoint{1.491698in}{0.773683in}}%
\pgfpathlineto{\pgfqpoint{1.493104in}{0.822970in}}%
\pgfpathlineto{\pgfqpoint{1.495213in}{0.848407in}}%
\pgfpathlineto{\pgfqpoint{1.497322in}{0.790273in}}%
\pgfpathlineto{\pgfqpoint{1.498225in}{0.814808in}}%
\pgfpathlineto{\pgfqpoint{1.500635in}{0.796700in}}%
\pgfpathlineto{\pgfqpoint{1.501439in}{0.818036in}}%
\pgfpathlineto{\pgfqpoint{1.502744in}{0.789827in}}%
\pgfpathlineto{\pgfqpoint{1.504351in}{0.785875in}}%
\pgfpathlineto{\pgfqpoint{1.505857in}{0.851034in}}%
\pgfpathlineto{\pgfqpoint{1.507464in}{0.842589in}}%
\pgfpathlineto{\pgfqpoint{1.508869in}{0.884412in}}%
\pgfpathlineto{\pgfqpoint{1.510476in}{0.868907in}}%
\pgfpathlineto{\pgfqpoint{1.512284in}{0.897647in}}%
\pgfpathlineto{\pgfqpoint{1.513890in}{0.872677in}}%
\pgfpathlineto{\pgfqpoint{1.515798in}{0.907878in}}%
\pgfpathlineto{\pgfqpoint{1.518409in}{0.872228in}}%
\pgfpathlineto{\pgfqpoint{1.519514in}{0.879258in}}%
\pgfpathlineto{\pgfqpoint{1.520819in}{0.859689in}}%
\pgfpathlineto{\pgfqpoint{1.523129in}{0.851648in}}%
\pgfpathlineto{\pgfqpoint{1.524032in}{0.874145in}}%
\pgfpathlineto{\pgfqpoint{1.525237in}{0.864680in}}%
\pgfpathlineto{\pgfqpoint{1.527346in}{0.892339in}}%
\pgfpathlineto{\pgfqpoint{1.528551in}{0.859506in}}%
\pgfpathlineto{\pgfqpoint{1.531764in}{0.918549in}}%
\pgfpathlineto{\pgfqpoint{1.533672in}{0.891412in}}%
\pgfpathlineto{\pgfqpoint{1.535982in}{0.916923in}}%
\pgfpathlineto{\pgfqpoint{1.537087in}{0.889137in}}%
\pgfpathlineto{\pgfqpoint{1.538292in}{0.922504in}}%
\pgfpathlineto{\pgfqpoint{1.540501in}{0.891754in}}%
\pgfpathlineto{\pgfqpoint{1.541103in}{0.893004in}}%
\pgfpathlineto{\pgfqpoint{1.543312in}{0.949599in}}%
\pgfpathlineto{\pgfqpoint{1.544216in}{0.938902in}}%
\pgfpathlineto{\pgfqpoint{1.548333in}{1.003202in}}%
\pgfpathlineto{\pgfqpoint{1.549538in}{0.974690in}}%
\pgfpathlineto{\pgfqpoint{1.553756in}{1.026721in}}%
\pgfpathlineto{\pgfqpoint{1.555162in}{0.984015in}}%
\pgfpathlineto{\pgfqpoint{1.555463in}{1.001669in}}%
\pgfpathlineto{\pgfqpoint{1.557973in}{1.013919in}}%
\pgfpathlineto{\pgfqpoint{1.558777in}{0.991117in}}%
\pgfpathlineto{\pgfqpoint{1.560584in}{1.006508in}}%
\pgfpathlineto{\pgfqpoint{1.561889in}{0.953867in}}%
\pgfpathlineto{\pgfqpoint{1.563697in}{0.927154in}}%
\pgfpathlineto{\pgfqpoint{1.565304in}{0.940964in}}%
\pgfpathlineto{\pgfqpoint{1.566810in}{0.920784in}}%
\pgfpathlineto{\pgfqpoint{1.569220in}{0.906119in}}%
\pgfpathlineto{\pgfqpoint{1.569923in}{0.937940in}}%
\pgfpathlineto{\pgfqpoint{1.572232in}{0.918363in}}%
\pgfpathlineto{\pgfqpoint{1.575847in}{0.989569in}}%
\pgfpathlineto{\pgfqpoint{1.577454in}{0.946394in}}%
\pgfpathlineto{\pgfqpoint{1.579864in}{0.976966in}}%
\pgfpathlineto{\pgfqpoint{1.581471in}{0.939886in}}%
\pgfpathlineto{\pgfqpoint{1.583680in}{0.951653in}}%
\pgfpathlineto{\pgfqpoint{1.584383in}{0.987396in}}%
\pgfpathlineto{\pgfqpoint{1.586190in}{0.985869in}}%
\pgfpathlineto{\pgfqpoint{1.587596in}{1.043182in}}%
\pgfpathlineto{\pgfqpoint{1.590508in}{1.033220in}}%
\pgfpathlineto{\pgfqpoint{1.592115in}{1.073534in}}%
\pgfpathlineto{\pgfqpoint{1.592416in}{1.061294in}}%
\pgfpathlineto{\pgfqpoint{1.595429in}{1.046704in}}%
\pgfpathlineto{\pgfqpoint{1.596835in}{1.007669in}}%
\pgfpathlineto{\pgfqpoint{1.597939in}{1.026317in}}%
\pgfpathlineto{\pgfqpoint{1.601454in}{0.952448in}}%
\pgfpathlineto{\pgfqpoint{1.603060in}{0.971765in}}%
\pgfpathlineto{\pgfqpoint{1.604265in}{0.947832in}}%
\pgfpathlineto{\pgfqpoint{1.606776in}{0.974973in}}%
\pgfpathlineto{\pgfqpoint{1.608382in}{0.928875in}}%
\pgfpathlineto{\pgfqpoint{1.609387in}{0.955388in}}%
\pgfpathlineto{\pgfqpoint{1.611194in}{0.954639in}}%
\pgfpathlineto{\pgfqpoint{1.612700in}{0.990138in}}%
\pgfpathlineto{\pgfqpoint{1.614407in}{0.941419in}}%
\pgfpathlineto{\pgfqpoint{1.615412in}{0.974365in}}%
\pgfpathlineto{\pgfqpoint{1.617320in}{0.958645in}}%
\pgfpathlineto{\pgfqpoint{1.618324in}{0.978972in}}%
\pgfpathlineto{\pgfqpoint{1.620633in}{0.944532in}}%
\pgfpathlineto{\pgfqpoint{1.622441in}{0.963203in}}%
\pgfpathlineto{\pgfqpoint{1.623244in}{0.936031in}}%
\pgfpathlineto{\pgfqpoint{1.625353in}{0.939660in}}%
\pgfpathlineto{\pgfqpoint{1.628165in}{1.005843in}}%
\pgfpathlineto{\pgfqpoint{1.629972in}{0.985715in}}%
\pgfpathlineto{\pgfqpoint{1.632683in}{1.006987in}}%
\pgfpathlineto{\pgfqpoint{1.634692in}{0.954303in}}%
\pgfpathlineto{\pgfqpoint{1.636399in}{0.974921in}}%
\pgfpathlineto{\pgfqpoint{1.638407in}{0.944780in}}%
\pgfpathlineto{\pgfqpoint{1.639010in}{0.974739in}}%
\pgfpathlineto{\pgfqpoint{1.640616in}{0.959250in}}%
\pgfpathlineto{\pgfqpoint{1.642424in}{0.994994in}}%
\pgfpathlineto{\pgfqpoint{1.644633in}{0.968476in}}%
\pgfpathlineto{\pgfqpoint{1.646641in}{0.996288in}}%
\pgfpathlineto{\pgfqpoint{1.647244in}{0.973017in}}%
\pgfpathlineto{\pgfqpoint{1.648951in}{0.950976in}}%
\pgfpathlineto{\pgfqpoint{1.651562in}{0.956764in}}%
\pgfpathlineto{\pgfqpoint{1.653369in}{0.910632in}}%
\pgfpathlineto{\pgfqpoint{1.654072in}{0.931619in}}%
\pgfpathlineto{\pgfqpoint{1.655378in}{0.917697in}}%
\pgfpathlineto{\pgfqpoint{1.656683in}{0.936941in}}%
\pgfpathlineto{\pgfqpoint{1.659294in}{0.888459in}}%
\pgfpathlineto{\pgfqpoint{1.660097in}{0.906527in}}%
\pgfpathlineto{\pgfqpoint{1.661704in}{0.885067in}}%
\pgfpathlineto{\pgfqpoint{1.663612in}{0.915885in}}%
\pgfpathlineto{\pgfqpoint{1.665720in}{0.896680in}}%
\pgfpathlineto{\pgfqpoint{1.667628in}{0.956787in}}%
\pgfpathlineto{\pgfqpoint{1.668231in}{0.940134in}}%
\pgfpathlineto{\pgfqpoint{1.670340in}{0.952761in}}%
\pgfpathlineto{\pgfqpoint{1.671344in}{0.928538in}}%
\pgfpathlineto{\pgfqpoint{1.673653in}{0.953887in}}%
\pgfpathlineto{\pgfqpoint{1.674959in}{0.904553in}}%
\pgfpathlineto{\pgfqpoint{1.676867in}{0.904802in}}%
\pgfpathlineto{\pgfqpoint{1.678674in}{0.948136in}}%
\pgfpathlineto{\pgfqpoint{1.679779in}{0.930173in}}%
\pgfpathlineto{\pgfqpoint{1.680984in}{0.947184in}}%
\pgfpathlineto{\pgfqpoint{1.683193in}{0.913650in}}%
\pgfpathlineto{\pgfqpoint{1.685000in}{0.928024in}}%
\pgfpathlineto{\pgfqpoint{1.686607in}{0.895741in}}%
\pgfpathlineto{\pgfqpoint{1.688314in}{0.916031in}}%
\pgfpathlineto{\pgfqpoint{1.691427in}{0.823187in}}%
\pgfpathlineto{\pgfqpoint{1.692532in}{0.808235in}}%
\pgfpathlineto{\pgfqpoint{1.696749in}{0.891675in}}%
\pgfpathlineto{\pgfqpoint{1.697854in}{0.858605in}}%
\pgfpathlineto{\pgfqpoint{1.698657in}{0.876539in}}%
\pgfpathlineto{\pgfqpoint{1.700063in}{0.849061in}}%
\pgfpathlineto{\pgfqpoint{1.702172in}{0.855250in}}%
\pgfpathlineto{\pgfqpoint{1.704682in}{0.898004in}}%
\pgfpathlineto{\pgfqpoint{1.705787in}{0.858958in}}%
\pgfpathlineto{\pgfqpoint{1.706490in}{0.885159in}}%
\pgfpathlineto{\pgfqpoint{1.709402in}{0.917078in}}%
\pgfpathlineto{\pgfqpoint{1.710506in}{0.890967in}}%
\pgfpathlineto{\pgfqpoint{1.711812in}{0.913077in}}%
\pgfpathlineto{\pgfqpoint{1.713017in}{0.881290in}}%
\pgfpathlineto{\pgfqpoint{1.714523in}{0.908561in}}%
\pgfpathlineto{\pgfqpoint{1.716330in}{0.899777in}}%
\pgfpathlineto{\pgfqpoint{1.718339in}{0.924898in}}%
\pgfpathlineto{\pgfqpoint{1.720548in}{0.912912in}}%
\pgfpathlineto{\pgfqpoint{1.722155in}{0.931263in}}%
\pgfpathlineto{\pgfqpoint{1.722858in}{0.901759in}}%
\pgfpathlineto{\pgfqpoint{1.724966in}{0.931536in}}%
\pgfpathlineto{\pgfqpoint{1.725870in}{0.895038in}}%
\pgfpathlineto{\pgfqpoint{1.728180in}{0.907900in}}%
\pgfpathlineto{\pgfqpoint{1.729585in}{0.902454in}}%
\pgfpathlineto{\pgfqpoint{1.731795in}{0.847440in}}%
\pgfpathlineto{\pgfqpoint{1.733200in}{0.875344in}}%
\pgfpathlineto{\pgfqpoint{1.734807in}{0.835755in}}%
\pgfpathlineto{\pgfqpoint{1.736313in}{0.863947in}}%
\pgfpathlineto{\pgfqpoint{1.737016in}{0.842685in}}%
\pgfpathlineto{\pgfqpoint{1.739326in}{0.892110in}}%
\pgfpathlineto{\pgfqpoint{1.741736in}{0.852284in}}%
\pgfpathlineto{\pgfqpoint{1.743041in}{0.877732in}}%
\pgfpathlineto{\pgfqpoint{1.744949in}{0.844746in}}%
\pgfpathlineto{\pgfqpoint{1.746154in}{0.873503in}}%
\pgfpathlineto{\pgfqpoint{1.747460in}{0.875588in}}%
\pgfpathlineto{\pgfqpoint{1.748765in}{0.854126in}}%
\pgfpathlineto{\pgfqpoint{1.749870in}{0.787937in}}%
\pgfpathlineto{\pgfqpoint{1.752280in}{0.754497in}}%
\pgfpathlineto{\pgfqpoint{1.753786in}{0.826013in}}%
\pgfpathlineto{\pgfqpoint{1.754690in}{0.804832in}}%
\pgfpathlineto{\pgfqpoint{1.756798in}{0.813710in}}%
\pgfpathlineto{\pgfqpoint{1.758305in}{0.810795in}}%
\pgfpathlineto{\pgfqpoint{1.760614in}{0.766831in}}%
\pgfpathlineto{\pgfqpoint{1.761116in}{0.780399in}}%
\pgfpathlineto{\pgfqpoint{1.762723in}{0.750704in}}%
\pgfpathlineto{\pgfqpoint{1.765736in}{0.771429in}}%
\pgfpathlineto{\pgfqpoint{1.767041in}{0.727234in}}%
\pgfpathlineto{\pgfqpoint{1.768848in}{0.714137in}}%
\pgfpathlineto{\pgfqpoint{1.770556in}{0.673107in}}%
\pgfpathlineto{\pgfqpoint{1.771761in}{0.702627in}}%
\pgfpathlineto{\pgfqpoint{1.772664in}{0.664298in}}%
\pgfpathlineto{\pgfqpoint{1.774673in}{0.675345in}}%
\pgfpathlineto{\pgfqpoint{1.775576in}{0.703210in}}%
\pgfpathlineto{\pgfqpoint{1.778288in}{0.695334in}}%
\pgfpathlineto{\pgfqpoint{1.779091in}{0.715745in}}%
\pgfpathlineto{\pgfqpoint{1.780698in}{0.686688in}}%
\pgfpathlineto{\pgfqpoint{1.782003in}{0.713226in}}%
\pgfpathlineto{\pgfqpoint{1.784313in}{0.673696in}}%
\pgfpathlineto{\pgfqpoint{1.785216in}{0.704127in}}%
\pgfpathlineto{\pgfqpoint{1.787526in}{0.671850in}}%
\pgfpathlineto{\pgfqpoint{1.789032in}{0.707869in}}%
\pgfpathlineto{\pgfqpoint{1.791944in}{0.660098in}}%
\pgfpathlineto{\pgfqpoint{1.794254in}{0.701232in}}%
\pgfpathlineto{\pgfqpoint{1.796162in}{0.662994in}}%
\pgfpathlineto{\pgfqpoint{1.798371in}{0.713904in}}%
\pgfpathlineto{\pgfqpoint{1.800178in}{0.699538in}}%
\pgfpathlineto{\pgfqpoint{1.801584in}{0.724509in}}%
\pgfpathlineto{\pgfqpoint{1.802890in}{0.688810in}}%
\pgfpathlineto{\pgfqpoint{1.804597in}{0.749592in}}%
\pgfpathlineto{\pgfqpoint{1.806706in}{0.732758in}}%
\pgfpathlineto{\pgfqpoint{1.808111in}{0.670335in}}%
\pgfpathlineto{\pgfqpoint{1.809316in}{0.699104in}}%
\pgfpathlineto{\pgfqpoint{1.811224in}{0.701121in}}%
\pgfpathlineto{\pgfqpoint{1.812630in}{0.748043in}}%
\pgfpathlineto{\pgfqpoint{1.815442in}{0.726855in}}%
\pgfpathlineto{\pgfqpoint{1.816948in}{0.683713in}}%
\pgfpathlineto{\pgfqpoint{1.818756in}{0.689411in}}%
\pgfpathlineto{\pgfqpoint{1.820161in}{0.633556in}}%
\pgfpathlineto{\pgfqpoint{1.821969in}{0.608616in}}%
\pgfpathlineto{\pgfqpoint{1.822371in}{0.639243in}}%
\pgfpathlineto{\pgfqpoint{1.824379in}{0.653536in}}%
\pgfpathlineto{\pgfqpoint{1.825484in}{0.612669in}}%
\pgfpathlineto{\pgfqpoint{1.827693in}{0.667109in}}%
\pgfpathlineto{\pgfqpoint{1.830002in}{0.633577in}}%
\pgfpathlineto{\pgfqpoint{1.832312in}{0.703980in}}%
\pgfpathlineto{\pgfqpoint{1.834320in}{0.687472in}}%
\pgfpathlineto{\pgfqpoint{1.836429in}{0.713556in}}%
\pgfpathlineto{\pgfqpoint{1.837634in}{0.674905in}}%
\pgfpathlineto{\pgfqpoint{1.839441in}{0.690910in}}%
\pgfpathlineto{\pgfqpoint{1.840144in}{0.739780in}}%
\pgfpathlineto{\pgfqpoint{1.842052in}{0.770399in}}%
\pgfpathlineto{\pgfqpoint{1.844563in}{0.728110in}}%
\pgfpathlineto{\pgfqpoint{1.845969in}{0.746472in}}%
\pgfpathlineto{\pgfqpoint{1.846973in}{0.718447in}}%
\pgfpathlineto{\pgfqpoint{1.848178in}{0.759496in}}%
\pgfpathlineto{\pgfqpoint{1.849784in}{0.746876in}}%
\pgfpathlineto{\pgfqpoint{1.851391in}{0.769939in}}%
\pgfpathlineto{\pgfqpoint{1.854203in}{0.761469in}}%
\pgfpathlineto{\pgfqpoint{1.855408in}{0.719810in}}%
\pgfpathlineto{\pgfqpoint{1.856211in}{0.752916in}}%
\pgfpathlineto{\pgfqpoint{1.858019in}{0.721419in}}%
\pgfpathlineto{\pgfqpoint{1.859424in}{0.755207in}}%
\pgfpathlineto{\pgfqpoint{1.861734in}{0.728233in}}%
\pgfpathlineto{\pgfqpoint{1.863140in}{0.736128in}}%
\pgfpathlineto{\pgfqpoint{1.864546in}{0.714169in}}%
\pgfpathlineto{\pgfqpoint{1.865951in}{0.715655in}}%
\pgfpathlineto{\pgfqpoint{1.868261in}{0.667608in}}%
\pgfpathlineto{\pgfqpoint{1.869868in}{0.682825in}}%
\pgfpathlineto{\pgfqpoint{1.871474in}{0.732849in}}%
\pgfpathlineto{\pgfqpoint{1.872278in}{0.685394in}}%
\pgfpathlineto{\pgfqpoint{1.875089in}{0.633366in}}%
\pgfpathlineto{\pgfqpoint{1.875792in}{0.666583in}}%
\pgfpathlineto{\pgfqpoint{1.877098in}{0.638210in}}%
\pgfpathlineto{\pgfqpoint{1.879608in}{0.664546in}}%
\pgfpathlineto{\pgfqpoint{1.880612in}{0.646551in}}%
\pgfpathlineto{\pgfqpoint{1.882119in}{0.690567in}}%
\pgfpathlineto{\pgfqpoint{1.883223in}{0.677983in}}%
\pgfpathlineto{\pgfqpoint{1.885734in}{0.726374in}}%
\pgfpathlineto{\pgfqpoint{1.887139in}{0.709147in}}%
\pgfpathlineto{\pgfqpoint{1.890051in}{0.748988in}}%
\pgfpathlineto{\pgfqpoint{1.892461in}{0.735042in}}%
\pgfpathlineto{\pgfqpoint{1.894068in}{0.762783in}}%
\pgfpathlineto{\pgfqpoint{1.894771in}{0.714070in}}%
\pgfpathlineto{\pgfqpoint{1.897583in}{0.745831in}}%
\pgfpathlineto{\pgfqpoint{1.898185in}{0.731277in}}%
\pgfpathlineto{\pgfqpoint{1.900495in}{0.743227in}}%
\pgfpathlineto{\pgfqpoint{1.902302in}{0.795378in}}%
\pgfpathlineto{\pgfqpoint{1.904009in}{0.756015in}}%
\pgfpathlineto{\pgfqpoint{1.905616in}{0.771268in}}%
\pgfpathlineto{\pgfqpoint{1.906118in}{0.749458in}}%
\pgfpathlineto{\pgfqpoint{1.909131in}{0.819571in}}%
\pgfpathlineto{\pgfqpoint{1.912344in}{0.748919in}}%
\pgfpathlineto{\pgfqpoint{1.914252in}{0.773291in}}%
\pgfpathlineto{\pgfqpoint{1.915959in}{0.753436in}}%
\pgfpathlineto{\pgfqpoint{1.917365in}{0.779093in}}%
\pgfpathlineto{\pgfqpoint{1.919674in}{0.749928in}}%
\pgfpathlineto{\pgfqpoint{1.920177in}{0.762923in}}%
\pgfpathlineto{\pgfqpoint{1.923089in}{0.726455in}}%
\pgfpathlineto{\pgfqpoint{1.924595in}{0.758595in}}%
\pgfpathlineto{\pgfqpoint{1.925298in}{0.741890in}}%
\pgfpathlineto{\pgfqpoint{1.926804in}{0.752245in}}%
\pgfpathlineto{\pgfqpoint{1.928310in}{0.725859in}}%
\pgfpathlineto{\pgfqpoint{1.930419in}{0.745444in}}%
\pgfpathlineto{\pgfqpoint{1.932427in}{0.714709in}}%
\pgfpathlineto{\pgfqpoint{1.934134in}{0.662444in}}%
\pgfpathlineto{\pgfqpoint{1.936143in}{0.651644in}}%
\pgfpathlineto{\pgfqpoint{1.937147in}{0.691141in}}%
\pgfpathlineto{\pgfqpoint{1.939055in}{0.718044in}}%
\pgfpathlineto{\pgfqpoint{1.940059in}{0.692417in}}%
\pgfpathlineto{\pgfqpoint{1.941364in}{0.704660in}}%
\pgfpathlineto{\pgfqpoint{1.943072in}{0.757742in}}%
\pgfpathlineto{\pgfqpoint{1.944578in}{0.725780in}}%
\pgfpathlineto{\pgfqpoint{1.947389in}{0.726706in}}%
\pgfpathlineto{\pgfqpoint{1.948996in}{0.751883in}}%
\pgfpathlineto{\pgfqpoint{1.950402in}{0.733273in}}%
\pgfpathlineto{\pgfqpoint{1.951808in}{0.767070in}}%
\pgfpathlineto{\pgfqpoint{1.952912in}{0.746868in}}%
\pgfpathlineto{\pgfqpoint{1.954519in}{0.781616in}}%
\pgfpathlineto{\pgfqpoint{1.956025in}{0.753822in}}%
\pgfpathlineto{\pgfqpoint{1.957230in}{0.787577in}}%
\pgfpathlineto{\pgfqpoint{1.959038in}{0.771298in}}%
\pgfpathlineto{\pgfqpoint{1.961849in}{0.786781in}}%
\pgfpathlineto{\pgfqpoint{1.962151in}{0.766369in}}%
\pgfpathlineto{\pgfqpoint{1.964561in}{0.761356in}}%
\pgfpathlineto{\pgfqpoint{1.965464in}{0.734045in}}%
\pgfpathlineto{\pgfqpoint{1.967975in}{0.707160in}}%
\pgfpathlineto{\pgfqpoint{1.968879in}{0.733075in}}%
\pgfpathlineto{\pgfqpoint{1.970385in}{0.685888in}}%
\pgfpathlineto{\pgfqpoint{1.973598in}{0.723449in}}%
\pgfpathlineto{\pgfqpoint{1.974904in}{0.730953in}}%
\pgfpathlineto{\pgfqpoint{1.978920in}{0.653650in}}%
\pgfpathlineto{\pgfqpoint{1.984042in}{0.783598in}}%
\pgfpathlineto{\pgfqpoint{1.985347in}{0.807724in}}%
\pgfpathlineto{\pgfqpoint{1.987355in}{0.817600in}}%
\pgfpathlineto{\pgfqpoint{1.988259in}{0.780046in}}%
\pgfpathlineto{\pgfqpoint{1.989364in}{0.807548in}}%
\pgfpathlineto{\pgfqpoint{1.991071in}{0.790334in}}%
\pgfpathlineto{\pgfqpoint{1.992677in}{0.806072in}}%
\pgfpathlineto{\pgfqpoint{1.994184in}{0.786636in}}%
\pgfpathlineto{\pgfqpoint{1.996192in}{0.831888in}}%
\pgfpathlineto{\pgfqpoint{1.997899in}{0.818424in}}%
\pgfpathlineto{\pgfqpoint{1.999907in}{0.881408in}}%
\pgfpathlineto{\pgfqpoint{2.001916in}{0.853554in}}%
\pgfpathlineto{\pgfqpoint{2.002217in}{0.873367in}}%
\pgfpathlineto{\pgfqpoint{2.004828in}{0.891369in}}%
\pgfpathlineto{\pgfqpoint{2.005832in}{0.868190in}}%
\pgfpathlineto{\pgfqpoint{2.007137in}{0.917343in}}%
\pgfpathlineto{\pgfqpoint{2.008744in}{0.882464in}}%
\pgfpathlineto{\pgfqpoint{2.010953in}{0.922508in}}%
\pgfpathlineto{\pgfqpoint{2.012460in}{0.873967in}}%
\pgfpathlineto{\pgfqpoint{2.014568in}{0.849402in}}%
\pgfpathlineto{\pgfqpoint{2.016275in}{0.882418in}}%
\pgfpathlineto{\pgfqpoint{2.018083in}{0.853289in}}%
\pgfpathlineto{\pgfqpoint{2.018987in}{0.872247in}}%
\pgfpathlineto{\pgfqpoint{2.021095in}{0.837927in}}%
\pgfpathlineto{\pgfqpoint{2.022100in}{0.881142in}}%
\pgfpathlineto{\pgfqpoint{2.023104in}{0.864128in}}%
\pgfpathlineto{\pgfqpoint{2.025212in}{0.859681in}}%
\pgfpathlineto{\pgfqpoint{2.028426in}{0.909856in}}%
\pgfpathlineto{\pgfqpoint{2.030133in}{0.892938in}}%
\pgfpathlineto{\pgfqpoint{2.031037in}{0.939134in}}%
\pgfpathlineto{\pgfqpoint{2.032844in}{0.948852in}}%
\pgfpathlineto{\pgfqpoint{2.035756in}{0.860870in}}%
\pgfpathlineto{\pgfqpoint{2.037162in}{0.870833in}}%
\pgfpathlineto{\pgfqpoint{2.038267in}{0.914710in}}%
\pgfpathlineto{\pgfqpoint{2.040375in}{0.914100in}}%
\pgfpathlineto{\pgfqpoint{2.042384in}{0.854339in}}%
\pgfpathlineto{\pgfqpoint{2.046802in}{0.992337in}}%
\pgfpathlineto{\pgfqpoint{2.047304in}{0.995098in}}%
\pgfpathlineto{\pgfqpoint{2.049112in}{0.950859in}}%
\pgfpathlineto{\pgfqpoint{2.051321in}{0.995100in}}%
\pgfpathlineto{\pgfqpoint{2.052225in}{0.950454in}}%
\pgfpathlineto{\pgfqpoint{2.053731in}{0.977475in}}%
\pgfpathlineto{\pgfqpoint{2.055438in}{0.931350in}}%
\pgfpathlineto{\pgfqpoint{2.057747in}{0.917556in}}%
\pgfpathlineto{\pgfqpoint{2.059655in}{0.940730in}}%
\pgfpathlineto{\pgfqpoint{2.060258in}{0.977212in}}%
\pgfpathlineto{\pgfqpoint{2.062567in}{1.019131in}}%
\pgfpathlineto{\pgfqpoint{2.063572in}{1.011488in}}%
\pgfpathlineto{\pgfqpoint{2.064877in}{1.050274in}}%
\pgfpathlineto{\pgfqpoint{2.067287in}{1.054274in}}%
\pgfpathlineto{\pgfqpoint{2.069396in}{0.994116in}}%
\pgfpathlineto{\pgfqpoint{2.070099in}{1.035483in}}%
\pgfpathlineto{\pgfqpoint{2.072107in}{0.976016in}}%
\pgfpathlineto{\pgfqpoint{2.073011in}{1.033934in}}%
\pgfpathlineto{\pgfqpoint{2.075722in}{1.049594in}}%
\pgfpathlineto{\pgfqpoint{2.077128in}{1.022735in}}%
\pgfpathlineto{\pgfqpoint{2.078132in}{1.054634in}}%
\pgfpathlineto{\pgfqpoint{2.080140in}{1.032318in}}%
\pgfpathlineto{\pgfqpoint{2.080944in}{1.061022in}}%
\pgfpathlineto{\pgfqpoint{2.083053in}{1.024424in}}%
\pgfpathlineto{\pgfqpoint{2.084358in}{1.038877in}}%
\pgfpathlineto{\pgfqpoint{2.088575in}{1.031322in}}%
\pgfpathlineto{\pgfqpoint{2.089178in}{0.995612in}}%
\pgfpathlineto{\pgfqpoint{2.091387in}{0.996096in}}%
\pgfpathlineto{\pgfqpoint{2.092391in}{0.957582in}}%
\pgfpathlineto{\pgfqpoint{2.095103in}{1.034971in}}%
\pgfpathlineto{\pgfqpoint{2.095705in}{1.008889in}}%
\pgfpathlineto{\pgfqpoint{2.098818in}{0.992236in}}%
\pgfpathlineto{\pgfqpoint{2.100425in}{1.047576in}}%
\pgfpathlineto{\pgfqpoint{2.102132in}{1.043634in}}%
\pgfpathlineto{\pgfqpoint{2.104943in}{1.111677in}}%
\pgfpathlineto{\pgfqpoint{2.108056in}{1.108287in}}%
\pgfpathlineto{\pgfqpoint{2.109161in}{1.076777in}}%
\pgfpathlineto{\pgfqpoint{2.110065in}{1.098977in}}%
\pgfpathlineto{\pgfqpoint{2.111671in}{1.052237in}}%
\pgfpathlineto{\pgfqpoint{2.113579in}{1.083155in}}%
\pgfpathlineto{\pgfqpoint{2.114985in}{1.055119in}}%
\pgfpathlineto{\pgfqpoint{2.116391in}{1.114132in}}%
\pgfpathlineto{\pgfqpoint{2.118901in}{1.104964in}}%
\pgfpathlineto{\pgfqpoint{2.119504in}{1.078705in}}%
\pgfpathlineto{\pgfqpoint{2.121010in}{1.066332in}}%
\pgfpathlineto{\pgfqpoint{2.123219in}{1.098072in}}%
\pgfpathlineto{\pgfqpoint{2.124625in}{1.100493in}}%
\pgfpathlineto{\pgfqpoint{2.126834in}{1.057208in}}%
\pgfpathlineto{\pgfqpoint{2.128340in}{1.084077in}}%
\pgfpathlineto{\pgfqpoint{2.129345in}{1.059883in}}%
\pgfpathlineto{\pgfqpoint{2.130750in}{1.111782in}}%
\pgfpathlineto{\pgfqpoint{2.133261in}{1.110320in}}%
\pgfpathlineto{\pgfqpoint{2.134667in}{1.087989in}}%
\pgfpathlineto{\pgfqpoint{2.137780in}{1.008509in}}%
\pgfpathlineto{\pgfqpoint{2.140089in}{1.087339in}}%
\pgfpathlineto{\pgfqpoint{2.140390in}{1.072291in}}%
\pgfpathlineto{\pgfqpoint{2.143001in}{1.104976in}}%
\pgfpathlineto{\pgfqpoint{2.143704in}{1.068998in}}%
\pgfpathlineto{\pgfqpoint{2.146315in}{1.099256in}}%
\pgfpathlineto{\pgfqpoint{2.148022in}{1.084112in}}%
\pgfpathlineto{\pgfqpoint{2.149328in}{1.111518in}}%
\pgfpathlineto{\pgfqpoint{2.149930in}{1.092185in}}%
\pgfpathlineto{\pgfqpoint{2.152641in}{1.106852in}}%
\pgfpathlineto{\pgfqpoint{2.154348in}{1.075782in}}%
\pgfpathlineto{\pgfqpoint{2.154750in}{1.098965in}}%
\pgfpathlineto{\pgfqpoint{2.157763in}{1.098063in}}%
\pgfpathlineto{\pgfqpoint{2.159369in}{1.060627in}}%
\pgfpathlineto{\pgfqpoint{2.160876in}{1.095286in}}%
\pgfpathlineto{\pgfqpoint{2.162482in}{1.104816in}}%
\pgfpathlineto{\pgfqpoint{2.164290in}{1.084471in}}%
\pgfpathlineto{\pgfqpoint{2.166901in}{1.146568in}}%
\pgfpathlineto{\pgfqpoint{2.168507in}{1.106111in}}%
\pgfpathlineto{\pgfqpoint{2.169813in}{1.139359in}}%
\pgfpathlineto{\pgfqpoint{2.170917in}{1.120859in}}%
\pgfpathlineto{\pgfqpoint{2.172524in}{1.125006in}}%
\pgfpathlineto{\pgfqpoint{2.175536in}{1.179375in}}%
\pgfpathlineto{\pgfqpoint{2.176340in}{1.155904in}}%
\pgfpathlineto{\pgfqpoint{2.177846in}{1.197888in}}%
\pgfpathlineto{\pgfqpoint{2.179151in}{1.181507in}}%
\pgfpathlineto{\pgfqpoint{2.181762in}{1.206155in}}%
\pgfpathlineto{\pgfqpoint{2.183871in}{1.154919in}}%
\pgfpathlineto{\pgfqpoint{2.185578in}{1.186917in}}%
\pgfpathlineto{\pgfqpoint{2.187285in}{1.125246in}}%
\pgfpathlineto{\pgfqpoint{2.189293in}{1.160679in}}%
\pgfpathlineto{\pgfqpoint{2.191402in}{1.133400in}}%
\pgfpathlineto{\pgfqpoint{2.192206in}{1.157638in}}%
\pgfpathlineto{\pgfqpoint{2.194415in}{1.103866in}}%
\pgfpathlineto{\pgfqpoint{2.196122in}{1.125116in}}%
\pgfpathlineto{\pgfqpoint{2.197728in}{1.098282in}}%
\pgfpathlineto{\pgfqpoint{2.199737in}{1.167868in}}%
\pgfpathlineto{\pgfqpoint{2.201544in}{1.161704in}}%
\pgfpathlineto{\pgfqpoint{2.202950in}{1.197011in}}%
\pgfpathlineto{\pgfqpoint{2.205661in}{1.223174in}}%
\pgfpathlineto{\pgfqpoint{2.206163in}{1.191367in}}%
\pgfpathlineto{\pgfqpoint{2.207770in}{1.185786in}}%
\pgfpathlineto{\pgfqpoint{2.209778in}{1.136783in}}%
\pgfpathlineto{\pgfqpoint{2.211486in}{1.151871in}}%
\pgfpathlineto{\pgfqpoint{2.212992in}{1.137764in}}%
\pgfpathlineto{\pgfqpoint{2.215301in}{1.069810in}}%
\pgfpathlineto{\pgfqpoint{2.216105in}{1.088246in}}%
\pgfpathlineto{\pgfqpoint{2.218515in}{1.078381in}}%
\pgfpathlineto{\pgfqpoint{2.219418in}{1.100430in}}%
\pgfpathlineto{\pgfqpoint{2.221929in}{1.087200in}}%
\pgfpathlineto{\pgfqpoint{2.222632in}{1.115621in}}%
\pgfpathlineto{\pgfqpoint{2.224941in}{1.155072in}}%
\pgfpathlineto{\pgfqpoint{2.227753in}{1.129014in}}%
\pgfpathlineto{\pgfqpoint{2.230364in}{1.194820in}}%
\pgfpathlineto{\pgfqpoint{2.231971in}{1.193783in}}%
\pgfpathlineto{\pgfqpoint{2.234682in}{1.092459in}}%
\pgfpathlineto{\pgfqpoint{2.235083in}{1.107395in}}%
\pgfpathlineto{\pgfqpoint{2.237594in}{1.069199in}}%
\pgfpathlineto{\pgfqpoint{2.239502in}{1.081984in}}%
\pgfpathlineto{\pgfqpoint{2.240908in}{1.116069in}}%
\pgfpathlineto{\pgfqpoint{2.242916in}{1.108756in}}%
\pgfpathlineto{\pgfqpoint{2.243217in}{1.088606in}}%
\pgfpathlineto{\pgfqpoint{2.246230in}{1.065813in}}%
\pgfpathlineto{\pgfqpoint{2.246933in}{1.087966in}}%
\pgfpathlineto{\pgfqpoint{2.249343in}{1.055396in}}%
\pgfpathlineto{\pgfqpoint{2.250548in}{1.072288in}}%
\pgfpathlineto{\pgfqpoint{2.251552in}{1.026070in}}%
\pgfpathlineto{\pgfqpoint{2.252857in}{1.029506in}}%
\pgfpathlineto{\pgfqpoint{2.254966in}{1.080631in}}%
\pgfpathlineto{\pgfqpoint{2.256874in}{1.067588in}}%
\pgfpathlineto{\pgfqpoint{2.257677in}{1.036677in}}%
\pgfpathlineto{\pgfqpoint{2.259485in}{1.021429in}}%
\pgfpathlineto{\pgfqpoint{2.261895in}{1.040883in}}%
\pgfpathlineto{\pgfqpoint{2.263501in}{1.115450in}}%
\pgfpathlineto{\pgfqpoint{2.264104in}{1.090197in}}%
\pgfpathlineto{\pgfqpoint{2.266715in}{1.099001in}}%
\pgfpathlineto{\pgfqpoint{2.267719in}{1.138077in}}%
\pgfpathlineto{\pgfqpoint{2.269326in}{1.135472in}}%
\pgfpathlineto{\pgfqpoint{2.270631in}{1.106404in}}%
\pgfpathlineto{\pgfqpoint{2.272539in}{1.102261in}}%
\pgfpathlineto{\pgfqpoint{2.276254in}{1.165513in}}%
\pgfpathlineto{\pgfqpoint{2.277560in}{1.138127in}}%
\pgfpathlineto{\pgfqpoint{2.279669in}{1.175828in}}%
\pgfpathlineto{\pgfqpoint{2.280271in}{1.163236in}}%
\pgfpathlineto{\pgfqpoint{2.282179in}{1.200424in}}%
\pgfpathlineto{\pgfqpoint{2.283886in}{1.179650in}}%
\pgfpathlineto{\pgfqpoint{2.285191in}{1.192929in}}%
\pgfpathlineto{\pgfqpoint{2.287501in}{1.158101in}}%
\pgfpathlineto{\pgfqpoint{2.291618in}{1.252204in}}%
\pgfpathlineto{\pgfqpoint{2.292924in}{1.208837in}}%
\pgfpathlineto{\pgfqpoint{2.294530in}{1.230181in}}%
\pgfpathlineto{\pgfqpoint{2.296237in}{1.187048in}}%
\pgfpathlineto{\pgfqpoint{2.298547in}{1.189315in}}%
\pgfpathlineto{\pgfqpoint{2.299953in}{1.241444in}}%
\pgfpathlineto{\pgfqpoint{2.302162in}{1.230870in}}%
\pgfpathlineto{\pgfqpoint{2.302664in}{1.254321in}}%
\pgfpathlineto{\pgfqpoint{2.304371in}{1.220753in}}%
\pgfpathlineto{\pgfqpoint{2.305777in}{1.246369in}}%
\pgfpathlineto{\pgfqpoint{2.308287in}{1.236272in}}%
\pgfpathlineto{\pgfqpoint{2.309894in}{1.194737in}}%
\pgfpathlineto{\pgfqpoint{2.311802in}{1.223731in}}%
\pgfpathlineto{\pgfqpoint{2.313609in}{1.213605in}}%
\pgfpathlineto{\pgfqpoint{2.315015in}{1.154837in}}%
\pgfpathlineto{\pgfqpoint{2.316321in}{1.171124in}}%
\pgfpathlineto{\pgfqpoint{2.317024in}{1.153482in}}%
\pgfpathlineto{\pgfqpoint{2.317024in}{1.153482in}}%
\pgfusepath{stroke}%
\end{pgfscope}%
\begin{pgfscope}%
\pgfsetrectcap%
\pgfsetmiterjoin%
\pgfsetlinewidth{0.803000pt}%
\definecolor{currentstroke}{rgb}{0.000000,0.000000,0.000000}%
\pgfsetstrokecolor{currentstroke}%
\pgfsetdash{}{0pt}%
\pgfpathmoveto{\pgfqpoint{0.589745in}{0.416447in}}%
\pgfpathlineto{\pgfqpoint{0.589745in}{1.789039in}}%
\pgfusepath{stroke}%
\end{pgfscope}%
\begin{pgfscope}%
\pgfsetrectcap%
\pgfsetmiterjoin%
\pgfsetlinewidth{0.803000pt}%
\definecolor{currentstroke}{rgb}{0.000000,0.000000,0.000000}%
\pgfsetstrokecolor{currentstroke}%
\pgfsetdash{}{0pt}%
\pgfpathmoveto{\pgfqpoint{2.399275in}{0.416447in}}%
\pgfpathlineto{\pgfqpoint{2.399275in}{1.789039in}}%
\pgfusepath{stroke}%
\end{pgfscope}%
\begin{pgfscope}%
\pgfsetrectcap%
\pgfsetmiterjoin%
\pgfsetlinewidth{0.803000pt}%
\definecolor{currentstroke}{rgb}{0.000000,0.000000,0.000000}%
\pgfsetstrokecolor{currentstroke}%
\pgfsetdash{}{0pt}%
\pgfpathmoveto{\pgfqpoint{0.589745in}{0.416447in}}%
\pgfpathlineto{\pgfqpoint{2.399275in}{0.416447in}}%
\pgfusepath{stroke}%
\end{pgfscope}%
\begin{pgfscope}%
\pgfsetrectcap%
\pgfsetmiterjoin%
\pgfsetlinewidth{0.803000pt}%
\definecolor{currentstroke}{rgb}{0.000000,0.000000,0.000000}%
\pgfsetstrokecolor{currentstroke}%
\pgfsetdash{}{0pt}%
\pgfpathmoveto{\pgfqpoint{0.589745in}{1.789039in}}%
\pgfpathlineto{\pgfqpoint{2.399275in}{1.789039in}}%
\pgfusepath{stroke}%
\end{pgfscope}%
\begin{pgfscope}%
\pgfsetbuttcap%
\pgfsetmiterjoin%
\definecolor{currentfill}{rgb}{1.000000,1.000000,1.000000}%
\pgfsetfillcolor{currentfill}%
\pgfsetfillopacity{0.800000}%
\pgfsetlinewidth{1.003750pt}%
\definecolor{currentstroke}{rgb}{0.800000,0.800000,0.800000}%
\pgfsetstrokecolor{currentstroke}%
\pgfsetstrokeopacity{0.800000}%
\pgfsetdash{}{0pt}%
\pgfpathmoveto{\pgfqpoint{0.667523in}{1.545261in}}%
\pgfpathlineto{\pgfqpoint{1.732745in}{1.545261in}}%
\pgfpathquadraticcurveto{\pgfqpoint{1.754967in}{1.545261in}}{\pgfqpoint{1.754967in}{1.567483in}}%
\pgfpathlineto{\pgfqpoint{1.754967in}{1.711261in}}%
\pgfpathquadraticcurveto{\pgfqpoint{1.754967in}{1.733483in}}{\pgfqpoint{1.732745in}{1.733483in}}%
\pgfpathlineto{\pgfqpoint{0.667523in}{1.733483in}}%
\pgfpathquadraticcurveto{\pgfqpoint{0.645300in}{1.733483in}}{\pgfqpoint{0.645300in}{1.711261in}}%
\pgfpathlineto{\pgfqpoint{0.645300in}{1.567483in}}%
\pgfpathquadraticcurveto{\pgfqpoint{0.645300in}{1.545261in}}{\pgfqpoint{0.667523in}{1.545261in}}%
\pgfpathlineto{\pgfqpoint{0.667523in}{1.545261in}}%
\pgfpathclose%
\pgfusepath{stroke,fill}%
\end{pgfscope}%
\begin{pgfscope}%
\pgfsetrectcap%
\pgfsetroundjoin%
\pgfsetlinewidth{1.505625pt}%
\definecolor{currentstroke}{rgb}{0.835294,0.368627,0.000000}%
\pgfsetstrokecolor{currentstroke}%
\pgfsetdash{}{0pt}%
\pgfpathmoveto{\pgfqpoint{0.689745in}{1.650150in}}%
\pgfpathlineto{\pgfqpoint{0.800856in}{1.650150in}}%
\pgfpathlineto{\pgfqpoint{0.911967in}{1.650150in}}%
\pgfusepath{stroke}%
\end{pgfscope}%
\begin{pgfscope}%
\definecolor{textcolor}{rgb}{0.000000,0.000000,0.000000}%
\pgfsetstrokecolor{textcolor}%
\pgfsetfillcolor{textcolor}%
\pgftext[x=1.000856in,y=1.611261in,left,base]{\color{textcolor}\rmfamily\fontsize{8.000000}{9.600000}\selectfont Random walk}%
\end{pgfscope}%
\end{pgfpicture}%
\makeatother%
\endgroup%
% data/simulations/sim_allan_variance.py
        } % scalebox
        \caption{Time domain}
        \label{fig:random_walk_time}
    \end{subfigure}
    \begin{subfigure}{0.32\linewidth}
        \centering
        \scalebox{0.75}{%
            %% Creator: Matplotlib, PGF backend
%%
%% To include the figure in your LaTeX document, write
%%   \input{<filename>.pgf}
%%
%% Make sure the required packages are loaded in your preamble
%%   \usepackage{pgf}
%%
%% Also ensure that all the required font packages are loaded; for instance,
%% the lmodern package is sometimes necessary when using math font.
%%   \usepackage{lmodern}
%%
%% Figures using additional raster images can only be included by \input if
%% they are in the same directory as the main LaTeX file. For loading figures
%% from other directories you can use the `import` package
%%   \usepackage{import}
%%
%% and then include the figures with
%%   \import{<path to file>}{<filename>.pgf}
%%
%% Matplotlib used the following preamble
%%   \usepackage{siunitx}
%%   \usepackage{fontspec}
%%   \makeatletter\@ifpackageloaded{underscore}{}{\usepackage[strings]{underscore}}\makeatother
%%
\begingroup%
\makeatletter%
\begin{pgfpicture}%
\pgfpathrectangle{\pgfpointorigin}{\pgfqpoint{2.440000in}{1.830000in}}%
\pgfusepath{use as bounding box, clip}%
\begin{pgfscope}%
\pgfsetbuttcap%
\pgfsetmiterjoin%
\definecolor{currentfill}{rgb}{1.000000,1.000000,1.000000}%
\pgfsetfillcolor{currentfill}%
\pgfsetlinewidth{0.000000pt}%
\definecolor{currentstroke}{rgb}{1.000000,1.000000,1.000000}%
\pgfsetstrokecolor{currentstroke}%
\pgfsetdash{}{0pt}%
\pgfpathmoveto{\pgfqpoint{0.000000in}{0.000000in}}%
\pgfpathlineto{\pgfqpoint{2.440000in}{0.000000in}}%
\pgfpathlineto{\pgfqpoint{2.440000in}{1.830000in}}%
\pgfpathlineto{\pgfqpoint{0.000000in}{1.830000in}}%
\pgfpathlineto{\pgfqpoint{0.000000in}{0.000000in}}%
\pgfpathclose%
\pgfusepath{fill}%
\end{pgfscope}%
\begin{pgfscope}%
\pgfsetbuttcap%
\pgfsetmiterjoin%
\definecolor{currentfill}{rgb}{1.000000,1.000000,1.000000}%
\pgfsetfillcolor{currentfill}%
\pgfsetlinewidth{0.000000pt}%
\definecolor{currentstroke}{rgb}{0.000000,0.000000,0.000000}%
\pgfsetstrokecolor{currentstroke}%
\pgfsetstrokeopacity{0.000000}%
\pgfsetdash{}{0pt}%
\pgfpathmoveto{\pgfqpoint{0.514278in}{0.417642in}}%
\pgfpathlineto{\pgfqpoint{2.398330in}{0.417642in}}%
\pgfpathlineto{\pgfqpoint{2.398330in}{1.788330in}}%
\pgfpathlineto{\pgfqpoint{0.514278in}{1.788330in}}%
\pgfpathlineto{\pgfqpoint{0.514278in}{0.417642in}}%
\pgfpathclose%
\pgfusepath{fill}%
\end{pgfscope}%
\begin{pgfscope}%
\pgfpathrectangle{\pgfqpoint{0.514278in}{0.417642in}}{\pgfqpoint{1.884052in}{1.370688in}}%
\pgfusepath{clip}%
\pgfsetrectcap%
\pgfsetroundjoin%
\pgfsetlinewidth{0.803000pt}%
\definecolor{currentstroke}{rgb}{0.450000,0.450000,0.450000}%
\pgfsetstrokecolor{currentstroke}%
\pgfsetdash{}{0pt}%
\pgfpathmoveto{\pgfqpoint{0.916624in}{0.417642in}}%
\pgfpathlineto{\pgfqpoint{0.916624in}{1.788330in}}%
\pgfusepath{stroke}%
\end{pgfscope}%
\begin{pgfscope}%
\pgfsetbuttcap%
\pgfsetroundjoin%
\definecolor{currentfill}{rgb}{0.000000,0.000000,0.000000}%
\pgfsetfillcolor{currentfill}%
\pgfsetlinewidth{0.803000pt}%
\definecolor{currentstroke}{rgb}{0.000000,0.000000,0.000000}%
\pgfsetstrokecolor{currentstroke}%
\pgfsetdash{}{0pt}%
\pgfsys@defobject{currentmarker}{\pgfqpoint{0.000000in}{-0.048611in}}{\pgfqpoint{0.000000in}{0.000000in}}{%
\pgfpathmoveto{\pgfqpoint{0.000000in}{0.000000in}}%
\pgfpathlineto{\pgfqpoint{0.000000in}{-0.048611in}}%
\pgfusepath{stroke,fill}%
}%
\begin{pgfscope}%
\pgfsys@transformshift{0.916624in}{0.417642in}%
\pgfsys@useobject{currentmarker}{}%
\end{pgfscope}%
\end{pgfscope}%
\begin{pgfscope}%
\definecolor{textcolor}{rgb}{0.000000,0.000000,0.000000}%
\pgfsetstrokecolor{textcolor}%
\pgfsetfillcolor{textcolor}%
\pgftext[x=0.916624in,y=0.320420in,,top]{\color{textcolor}\rmfamily\fontsize{8.000000}{9.600000}\selectfont \(\displaystyle {10^{-3}}\)}%
\end{pgfscope}%
\begin{pgfscope}%
\pgfpathrectangle{\pgfqpoint{0.514278in}{0.417642in}}{\pgfqpoint{1.884052in}{1.370688in}}%
\pgfusepath{clip}%
\pgfsetrectcap%
\pgfsetroundjoin%
\pgfsetlinewidth{0.803000pt}%
\definecolor{currentstroke}{rgb}{0.450000,0.450000,0.450000}%
\pgfsetstrokecolor{currentstroke}%
\pgfsetdash{}{0pt}%
\pgfpathmoveto{\pgfqpoint{1.433903in}{0.417642in}}%
\pgfpathlineto{\pgfqpoint{1.433903in}{1.788330in}}%
\pgfusepath{stroke}%
\end{pgfscope}%
\begin{pgfscope}%
\pgfsetbuttcap%
\pgfsetroundjoin%
\definecolor{currentfill}{rgb}{0.000000,0.000000,0.000000}%
\pgfsetfillcolor{currentfill}%
\pgfsetlinewidth{0.803000pt}%
\definecolor{currentstroke}{rgb}{0.000000,0.000000,0.000000}%
\pgfsetstrokecolor{currentstroke}%
\pgfsetdash{}{0pt}%
\pgfsys@defobject{currentmarker}{\pgfqpoint{0.000000in}{-0.048611in}}{\pgfqpoint{0.000000in}{0.000000in}}{%
\pgfpathmoveto{\pgfqpoint{0.000000in}{0.000000in}}%
\pgfpathlineto{\pgfqpoint{0.000000in}{-0.048611in}}%
\pgfusepath{stroke,fill}%
}%
\begin{pgfscope}%
\pgfsys@transformshift{1.433903in}{0.417642in}%
\pgfsys@useobject{currentmarker}{}%
\end{pgfscope}%
\end{pgfscope}%
\begin{pgfscope}%
\definecolor{textcolor}{rgb}{0.000000,0.000000,0.000000}%
\pgfsetstrokecolor{textcolor}%
\pgfsetfillcolor{textcolor}%
\pgftext[x=1.433903in,y=0.320420in,,top]{\color{textcolor}\rmfamily\fontsize{8.000000}{9.600000}\selectfont \(\displaystyle {10^{-2}}\)}%
\end{pgfscope}%
\begin{pgfscope}%
\pgfpathrectangle{\pgfqpoint{0.514278in}{0.417642in}}{\pgfqpoint{1.884052in}{1.370688in}}%
\pgfusepath{clip}%
\pgfsetrectcap%
\pgfsetroundjoin%
\pgfsetlinewidth{0.803000pt}%
\definecolor{currentstroke}{rgb}{0.450000,0.450000,0.450000}%
\pgfsetstrokecolor{currentstroke}%
\pgfsetdash{}{0pt}%
\pgfpathmoveto{\pgfqpoint{1.951183in}{0.417642in}}%
\pgfpathlineto{\pgfqpoint{1.951183in}{1.788330in}}%
\pgfusepath{stroke}%
\end{pgfscope}%
\begin{pgfscope}%
\pgfsetbuttcap%
\pgfsetroundjoin%
\definecolor{currentfill}{rgb}{0.000000,0.000000,0.000000}%
\pgfsetfillcolor{currentfill}%
\pgfsetlinewidth{0.803000pt}%
\definecolor{currentstroke}{rgb}{0.000000,0.000000,0.000000}%
\pgfsetstrokecolor{currentstroke}%
\pgfsetdash{}{0pt}%
\pgfsys@defobject{currentmarker}{\pgfqpoint{0.000000in}{-0.048611in}}{\pgfqpoint{0.000000in}{0.000000in}}{%
\pgfpathmoveto{\pgfqpoint{0.000000in}{0.000000in}}%
\pgfpathlineto{\pgfqpoint{0.000000in}{-0.048611in}}%
\pgfusepath{stroke,fill}%
}%
\begin{pgfscope}%
\pgfsys@transformshift{1.951183in}{0.417642in}%
\pgfsys@useobject{currentmarker}{}%
\end{pgfscope}%
\end{pgfscope}%
\begin{pgfscope}%
\definecolor{textcolor}{rgb}{0.000000,0.000000,0.000000}%
\pgfsetstrokecolor{textcolor}%
\pgfsetfillcolor{textcolor}%
\pgftext[x=1.951183in,y=0.320420in,,top]{\color{textcolor}\rmfamily\fontsize{8.000000}{9.600000}\selectfont \(\displaystyle {10^{-1}}\)}%
\end{pgfscope}%
\begin{pgfscope}%
\pgfpathrectangle{\pgfqpoint{0.514278in}{0.417642in}}{\pgfqpoint{1.884052in}{1.370688in}}%
\pgfusepath{clip}%
\pgfsetrectcap%
\pgfsetroundjoin%
\pgfsetlinewidth{0.803000pt}%
\definecolor{currentstroke}{rgb}{0.850000,0.850000,0.850000}%
\pgfsetstrokecolor{currentstroke}%
\pgfsetdash{}{0pt}%
\pgfpathmoveto{\pgfqpoint{0.555061in}{0.417642in}}%
\pgfpathlineto{\pgfqpoint{0.555061in}{1.788330in}}%
\pgfusepath{stroke}%
\end{pgfscope}%
\begin{pgfscope}%
\pgfsetbuttcap%
\pgfsetroundjoin%
\definecolor{currentfill}{rgb}{0.000000,0.000000,0.000000}%
\pgfsetfillcolor{currentfill}%
\pgfsetlinewidth{0.602250pt}%
\definecolor{currentstroke}{rgb}{0.000000,0.000000,0.000000}%
\pgfsetstrokecolor{currentstroke}%
\pgfsetdash{}{0pt}%
\pgfsys@defobject{currentmarker}{\pgfqpoint{0.000000in}{-0.027778in}}{\pgfqpoint{0.000000in}{0.000000in}}{%
\pgfpathmoveto{\pgfqpoint{0.000000in}{0.000000in}}%
\pgfpathlineto{\pgfqpoint{0.000000in}{-0.027778in}}%
\pgfusepath{stroke,fill}%
}%
\begin{pgfscope}%
\pgfsys@transformshift{0.555061in}{0.417642in}%
\pgfsys@useobject{currentmarker}{}%
\end{pgfscope}%
\end{pgfscope}%
\begin{pgfscope}%
\pgfpathrectangle{\pgfqpoint{0.514278in}{0.417642in}}{\pgfqpoint{1.884052in}{1.370688in}}%
\pgfusepath{clip}%
\pgfsetrectcap%
\pgfsetroundjoin%
\pgfsetlinewidth{0.803000pt}%
\definecolor{currentstroke}{rgb}{0.850000,0.850000,0.850000}%
\pgfsetstrokecolor{currentstroke}%
\pgfsetdash{}{0pt}%
\pgfpathmoveto{\pgfqpoint{0.646149in}{0.417642in}}%
\pgfpathlineto{\pgfqpoint{0.646149in}{1.788330in}}%
\pgfusepath{stroke}%
\end{pgfscope}%
\begin{pgfscope}%
\pgfsetbuttcap%
\pgfsetroundjoin%
\definecolor{currentfill}{rgb}{0.000000,0.000000,0.000000}%
\pgfsetfillcolor{currentfill}%
\pgfsetlinewidth{0.602250pt}%
\definecolor{currentstroke}{rgb}{0.000000,0.000000,0.000000}%
\pgfsetstrokecolor{currentstroke}%
\pgfsetdash{}{0pt}%
\pgfsys@defobject{currentmarker}{\pgfqpoint{0.000000in}{-0.027778in}}{\pgfqpoint{0.000000in}{0.000000in}}{%
\pgfpathmoveto{\pgfqpoint{0.000000in}{0.000000in}}%
\pgfpathlineto{\pgfqpoint{0.000000in}{-0.027778in}}%
\pgfusepath{stroke,fill}%
}%
\begin{pgfscope}%
\pgfsys@transformshift{0.646149in}{0.417642in}%
\pgfsys@useobject{currentmarker}{}%
\end{pgfscope}%
\end{pgfscope}%
\begin{pgfscope}%
\pgfpathrectangle{\pgfqpoint{0.514278in}{0.417642in}}{\pgfqpoint{1.884052in}{1.370688in}}%
\pgfusepath{clip}%
\pgfsetrectcap%
\pgfsetroundjoin%
\pgfsetlinewidth{0.803000pt}%
\definecolor{currentstroke}{rgb}{0.850000,0.850000,0.850000}%
\pgfsetstrokecolor{currentstroke}%
\pgfsetdash{}{0pt}%
\pgfpathmoveto{\pgfqpoint{0.710777in}{0.417642in}}%
\pgfpathlineto{\pgfqpoint{0.710777in}{1.788330in}}%
\pgfusepath{stroke}%
\end{pgfscope}%
\begin{pgfscope}%
\pgfsetbuttcap%
\pgfsetroundjoin%
\definecolor{currentfill}{rgb}{0.000000,0.000000,0.000000}%
\pgfsetfillcolor{currentfill}%
\pgfsetlinewidth{0.602250pt}%
\definecolor{currentstroke}{rgb}{0.000000,0.000000,0.000000}%
\pgfsetstrokecolor{currentstroke}%
\pgfsetdash{}{0pt}%
\pgfsys@defobject{currentmarker}{\pgfqpoint{0.000000in}{-0.027778in}}{\pgfqpoint{0.000000in}{0.000000in}}{%
\pgfpathmoveto{\pgfqpoint{0.000000in}{0.000000in}}%
\pgfpathlineto{\pgfqpoint{0.000000in}{-0.027778in}}%
\pgfusepath{stroke,fill}%
}%
\begin{pgfscope}%
\pgfsys@transformshift{0.710777in}{0.417642in}%
\pgfsys@useobject{currentmarker}{}%
\end{pgfscope}%
\end{pgfscope}%
\begin{pgfscope}%
\pgfpathrectangle{\pgfqpoint{0.514278in}{0.417642in}}{\pgfqpoint{1.884052in}{1.370688in}}%
\pgfusepath{clip}%
\pgfsetrectcap%
\pgfsetroundjoin%
\pgfsetlinewidth{0.803000pt}%
\definecolor{currentstroke}{rgb}{0.850000,0.850000,0.850000}%
\pgfsetstrokecolor{currentstroke}%
\pgfsetdash{}{0pt}%
\pgfpathmoveto{\pgfqpoint{0.760907in}{0.417642in}}%
\pgfpathlineto{\pgfqpoint{0.760907in}{1.788330in}}%
\pgfusepath{stroke}%
\end{pgfscope}%
\begin{pgfscope}%
\pgfsetbuttcap%
\pgfsetroundjoin%
\definecolor{currentfill}{rgb}{0.000000,0.000000,0.000000}%
\pgfsetfillcolor{currentfill}%
\pgfsetlinewidth{0.602250pt}%
\definecolor{currentstroke}{rgb}{0.000000,0.000000,0.000000}%
\pgfsetstrokecolor{currentstroke}%
\pgfsetdash{}{0pt}%
\pgfsys@defobject{currentmarker}{\pgfqpoint{0.000000in}{-0.027778in}}{\pgfqpoint{0.000000in}{0.000000in}}{%
\pgfpathmoveto{\pgfqpoint{0.000000in}{0.000000in}}%
\pgfpathlineto{\pgfqpoint{0.000000in}{-0.027778in}}%
\pgfusepath{stroke,fill}%
}%
\begin{pgfscope}%
\pgfsys@transformshift{0.760907in}{0.417642in}%
\pgfsys@useobject{currentmarker}{}%
\end{pgfscope}%
\end{pgfscope}%
\begin{pgfscope}%
\pgfpathrectangle{\pgfqpoint{0.514278in}{0.417642in}}{\pgfqpoint{1.884052in}{1.370688in}}%
\pgfusepath{clip}%
\pgfsetrectcap%
\pgfsetroundjoin%
\pgfsetlinewidth{0.803000pt}%
\definecolor{currentstroke}{rgb}{0.850000,0.850000,0.850000}%
\pgfsetstrokecolor{currentstroke}%
\pgfsetdash{}{0pt}%
\pgfpathmoveto{\pgfqpoint{0.801866in}{0.417642in}}%
\pgfpathlineto{\pgfqpoint{0.801866in}{1.788330in}}%
\pgfusepath{stroke}%
\end{pgfscope}%
\begin{pgfscope}%
\pgfsetbuttcap%
\pgfsetroundjoin%
\definecolor{currentfill}{rgb}{0.000000,0.000000,0.000000}%
\pgfsetfillcolor{currentfill}%
\pgfsetlinewidth{0.602250pt}%
\definecolor{currentstroke}{rgb}{0.000000,0.000000,0.000000}%
\pgfsetstrokecolor{currentstroke}%
\pgfsetdash{}{0pt}%
\pgfsys@defobject{currentmarker}{\pgfqpoint{0.000000in}{-0.027778in}}{\pgfqpoint{0.000000in}{0.000000in}}{%
\pgfpathmoveto{\pgfqpoint{0.000000in}{0.000000in}}%
\pgfpathlineto{\pgfqpoint{0.000000in}{-0.027778in}}%
\pgfusepath{stroke,fill}%
}%
\begin{pgfscope}%
\pgfsys@transformshift{0.801866in}{0.417642in}%
\pgfsys@useobject{currentmarker}{}%
\end{pgfscope}%
\end{pgfscope}%
\begin{pgfscope}%
\pgfpathrectangle{\pgfqpoint{0.514278in}{0.417642in}}{\pgfqpoint{1.884052in}{1.370688in}}%
\pgfusepath{clip}%
\pgfsetrectcap%
\pgfsetroundjoin%
\pgfsetlinewidth{0.803000pt}%
\definecolor{currentstroke}{rgb}{0.850000,0.850000,0.850000}%
\pgfsetstrokecolor{currentstroke}%
\pgfsetdash{}{0pt}%
\pgfpathmoveto{\pgfqpoint{0.836496in}{0.417642in}}%
\pgfpathlineto{\pgfqpoint{0.836496in}{1.788330in}}%
\pgfusepath{stroke}%
\end{pgfscope}%
\begin{pgfscope}%
\pgfsetbuttcap%
\pgfsetroundjoin%
\definecolor{currentfill}{rgb}{0.000000,0.000000,0.000000}%
\pgfsetfillcolor{currentfill}%
\pgfsetlinewidth{0.602250pt}%
\definecolor{currentstroke}{rgb}{0.000000,0.000000,0.000000}%
\pgfsetstrokecolor{currentstroke}%
\pgfsetdash{}{0pt}%
\pgfsys@defobject{currentmarker}{\pgfqpoint{0.000000in}{-0.027778in}}{\pgfqpoint{0.000000in}{0.000000in}}{%
\pgfpathmoveto{\pgfqpoint{0.000000in}{0.000000in}}%
\pgfpathlineto{\pgfqpoint{0.000000in}{-0.027778in}}%
\pgfusepath{stroke,fill}%
}%
\begin{pgfscope}%
\pgfsys@transformshift{0.836496in}{0.417642in}%
\pgfsys@useobject{currentmarker}{}%
\end{pgfscope}%
\end{pgfscope}%
\begin{pgfscope}%
\pgfpathrectangle{\pgfqpoint{0.514278in}{0.417642in}}{\pgfqpoint{1.884052in}{1.370688in}}%
\pgfusepath{clip}%
\pgfsetrectcap%
\pgfsetroundjoin%
\pgfsetlinewidth{0.803000pt}%
\definecolor{currentstroke}{rgb}{0.850000,0.850000,0.850000}%
\pgfsetstrokecolor{currentstroke}%
\pgfsetdash{}{0pt}%
\pgfpathmoveto{\pgfqpoint{0.866494in}{0.417642in}}%
\pgfpathlineto{\pgfqpoint{0.866494in}{1.788330in}}%
\pgfusepath{stroke}%
\end{pgfscope}%
\begin{pgfscope}%
\pgfsetbuttcap%
\pgfsetroundjoin%
\definecolor{currentfill}{rgb}{0.000000,0.000000,0.000000}%
\pgfsetfillcolor{currentfill}%
\pgfsetlinewidth{0.602250pt}%
\definecolor{currentstroke}{rgb}{0.000000,0.000000,0.000000}%
\pgfsetstrokecolor{currentstroke}%
\pgfsetdash{}{0pt}%
\pgfsys@defobject{currentmarker}{\pgfqpoint{0.000000in}{-0.027778in}}{\pgfqpoint{0.000000in}{0.000000in}}{%
\pgfpathmoveto{\pgfqpoint{0.000000in}{0.000000in}}%
\pgfpathlineto{\pgfqpoint{0.000000in}{-0.027778in}}%
\pgfusepath{stroke,fill}%
}%
\begin{pgfscope}%
\pgfsys@transformshift{0.866494in}{0.417642in}%
\pgfsys@useobject{currentmarker}{}%
\end{pgfscope}%
\end{pgfscope}%
\begin{pgfscope}%
\pgfpathrectangle{\pgfqpoint{0.514278in}{0.417642in}}{\pgfqpoint{1.884052in}{1.370688in}}%
\pgfusepath{clip}%
\pgfsetrectcap%
\pgfsetroundjoin%
\pgfsetlinewidth{0.803000pt}%
\definecolor{currentstroke}{rgb}{0.850000,0.850000,0.850000}%
\pgfsetstrokecolor{currentstroke}%
\pgfsetdash{}{0pt}%
\pgfpathmoveto{\pgfqpoint{0.892954in}{0.417642in}}%
\pgfpathlineto{\pgfqpoint{0.892954in}{1.788330in}}%
\pgfusepath{stroke}%
\end{pgfscope}%
\begin{pgfscope}%
\pgfsetbuttcap%
\pgfsetroundjoin%
\definecolor{currentfill}{rgb}{0.000000,0.000000,0.000000}%
\pgfsetfillcolor{currentfill}%
\pgfsetlinewidth{0.602250pt}%
\definecolor{currentstroke}{rgb}{0.000000,0.000000,0.000000}%
\pgfsetstrokecolor{currentstroke}%
\pgfsetdash{}{0pt}%
\pgfsys@defobject{currentmarker}{\pgfqpoint{0.000000in}{-0.027778in}}{\pgfqpoint{0.000000in}{0.000000in}}{%
\pgfpathmoveto{\pgfqpoint{0.000000in}{0.000000in}}%
\pgfpathlineto{\pgfqpoint{0.000000in}{-0.027778in}}%
\pgfusepath{stroke,fill}%
}%
\begin{pgfscope}%
\pgfsys@transformshift{0.892954in}{0.417642in}%
\pgfsys@useobject{currentmarker}{}%
\end{pgfscope}%
\end{pgfscope}%
\begin{pgfscope}%
\pgfpathrectangle{\pgfqpoint{0.514278in}{0.417642in}}{\pgfqpoint{1.884052in}{1.370688in}}%
\pgfusepath{clip}%
\pgfsetrectcap%
\pgfsetroundjoin%
\pgfsetlinewidth{0.803000pt}%
\definecolor{currentstroke}{rgb}{0.850000,0.850000,0.850000}%
\pgfsetstrokecolor{currentstroke}%
\pgfsetdash{}{0pt}%
\pgfpathmoveto{\pgfqpoint{1.072340in}{0.417642in}}%
\pgfpathlineto{\pgfqpoint{1.072340in}{1.788330in}}%
\pgfusepath{stroke}%
\end{pgfscope}%
\begin{pgfscope}%
\pgfsetbuttcap%
\pgfsetroundjoin%
\definecolor{currentfill}{rgb}{0.000000,0.000000,0.000000}%
\pgfsetfillcolor{currentfill}%
\pgfsetlinewidth{0.602250pt}%
\definecolor{currentstroke}{rgb}{0.000000,0.000000,0.000000}%
\pgfsetstrokecolor{currentstroke}%
\pgfsetdash{}{0pt}%
\pgfsys@defobject{currentmarker}{\pgfqpoint{0.000000in}{-0.027778in}}{\pgfqpoint{0.000000in}{0.000000in}}{%
\pgfpathmoveto{\pgfqpoint{0.000000in}{0.000000in}}%
\pgfpathlineto{\pgfqpoint{0.000000in}{-0.027778in}}%
\pgfusepath{stroke,fill}%
}%
\begin{pgfscope}%
\pgfsys@transformshift{1.072340in}{0.417642in}%
\pgfsys@useobject{currentmarker}{}%
\end{pgfscope}%
\end{pgfscope}%
\begin{pgfscope}%
\pgfpathrectangle{\pgfqpoint{0.514278in}{0.417642in}}{\pgfqpoint{1.884052in}{1.370688in}}%
\pgfusepath{clip}%
\pgfsetrectcap%
\pgfsetroundjoin%
\pgfsetlinewidth{0.803000pt}%
\definecolor{currentstroke}{rgb}{0.850000,0.850000,0.850000}%
\pgfsetstrokecolor{currentstroke}%
\pgfsetdash{}{0pt}%
\pgfpathmoveto{\pgfqpoint{1.163429in}{0.417642in}}%
\pgfpathlineto{\pgfqpoint{1.163429in}{1.788330in}}%
\pgfusepath{stroke}%
\end{pgfscope}%
\begin{pgfscope}%
\pgfsetbuttcap%
\pgfsetroundjoin%
\definecolor{currentfill}{rgb}{0.000000,0.000000,0.000000}%
\pgfsetfillcolor{currentfill}%
\pgfsetlinewidth{0.602250pt}%
\definecolor{currentstroke}{rgb}{0.000000,0.000000,0.000000}%
\pgfsetstrokecolor{currentstroke}%
\pgfsetdash{}{0pt}%
\pgfsys@defobject{currentmarker}{\pgfqpoint{0.000000in}{-0.027778in}}{\pgfqpoint{0.000000in}{0.000000in}}{%
\pgfpathmoveto{\pgfqpoint{0.000000in}{0.000000in}}%
\pgfpathlineto{\pgfqpoint{0.000000in}{-0.027778in}}%
\pgfusepath{stroke,fill}%
}%
\begin{pgfscope}%
\pgfsys@transformshift{1.163429in}{0.417642in}%
\pgfsys@useobject{currentmarker}{}%
\end{pgfscope}%
\end{pgfscope}%
\begin{pgfscope}%
\pgfpathrectangle{\pgfqpoint{0.514278in}{0.417642in}}{\pgfqpoint{1.884052in}{1.370688in}}%
\pgfusepath{clip}%
\pgfsetrectcap%
\pgfsetroundjoin%
\pgfsetlinewidth{0.803000pt}%
\definecolor{currentstroke}{rgb}{0.850000,0.850000,0.850000}%
\pgfsetstrokecolor{currentstroke}%
\pgfsetdash{}{0pt}%
\pgfpathmoveto{\pgfqpoint{1.228057in}{0.417642in}}%
\pgfpathlineto{\pgfqpoint{1.228057in}{1.788330in}}%
\pgfusepath{stroke}%
\end{pgfscope}%
\begin{pgfscope}%
\pgfsetbuttcap%
\pgfsetroundjoin%
\definecolor{currentfill}{rgb}{0.000000,0.000000,0.000000}%
\pgfsetfillcolor{currentfill}%
\pgfsetlinewidth{0.602250pt}%
\definecolor{currentstroke}{rgb}{0.000000,0.000000,0.000000}%
\pgfsetstrokecolor{currentstroke}%
\pgfsetdash{}{0pt}%
\pgfsys@defobject{currentmarker}{\pgfqpoint{0.000000in}{-0.027778in}}{\pgfqpoint{0.000000in}{0.000000in}}{%
\pgfpathmoveto{\pgfqpoint{0.000000in}{0.000000in}}%
\pgfpathlineto{\pgfqpoint{0.000000in}{-0.027778in}}%
\pgfusepath{stroke,fill}%
}%
\begin{pgfscope}%
\pgfsys@transformshift{1.228057in}{0.417642in}%
\pgfsys@useobject{currentmarker}{}%
\end{pgfscope}%
\end{pgfscope}%
\begin{pgfscope}%
\pgfpathrectangle{\pgfqpoint{0.514278in}{0.417642in}}{\pgfqpoint{1.884052in}{1.370688in}}%
\pgfusepath{clip}%
\pgfsetrectcap%
\pgfsetroundjoin%
\pgfsetlinewidth{0.803000pt}%
\definecolor{currentstroke}{rgb}{0.850000,0.850000,0.850000}%
\pgfsetstrokecolor{currentstroke}%
\pgfsetdash{}{0pt}%
\pgfpathmoveto{\pgfqpoint{1.278187in}{0.417642in}}%
\pgfpathlineto{\pgfqpoint{1.278187in}{1.788330in}}%
\pgfusepath{stroke}%
\end{pgfscope}%
\begin{pgfscope}%
\pgfsetbuttcap%
\pgfsetroundjoin%
\definecolor{currentfill}{rgb}{0.000000,0.000000,0.000000}%
\pgfsetfillcolor{currentfill}%
\pgfsetlinewidth{0.602250pt}%
\definecolor{currentstroke}{rgb}{0.000000,0.000000,0.000000}%
\pgfsetstrokecolor{currentstroke}%
\pgfsetdash{}{0pt}%
\pgfsys@defobject{currentmarker}{\pgfqpoint{0.000000in}{-0.027778in}}{\pgfqpoint{0.000000in}{0.000000in}}{%
\pgfpathmoveto{\pgfqpoint{0.000000in}{0.000000in}}%
\pgfpathlineto{\pgfqpoint{0.000000in}{-0.027778in}}%
\pgfusepath{stroke,fill}%
}%
\begin{pgfscope}%
\pgfsys@transformshift{1.278187in}{0.417642in}%
\pgfsys@useobject{currentmarker}{}%
\end{pgfscope}%
\end{pgfscope}%
\begin{pgfscope}%
\pgfpathrectangle{\pgfqpoint{0.514278in}{0.417642in}}{\pgfqpoint{1.884052in}{1.370688in}}%
\pgfusepath{clip}%
\pgfsetrectcap%
\pgfsetroundjoin%
\pgfsetlinewidth{0.803000pt}%
\definecolor{currentstroke}{rgb}{0.850000,0.850000,0.850000}%
\pgfsetstrokecolor{currentstroke}%
\pgfsetdash{}{0pt}%
\pgfpathmoveto{\pgfqpoint{1.319146in}{0.417642in}}%
\pgfpathlineto{\pgfqpoint{1.319146in}{1.788330in}}%
\pgfusepath{stroke}%
\end{pgfscope}%
\begin{pgfscope}%
\pgfsetbuttcap%
\pgfsetroundjoin%
\definecolor{currentfill}{rgb}{0.000000,0.000000,0.000000}%
\pgfsetfillcolor{currentfill}%
\pgfsetlinewidth{0.602250pt}%
\definecolor{currentstroke}{rgb}{0.000000,0.000000,0.000000}%
\pgfsetstrokecolor{currentstroke}%
\pgfsetdash{}{0pt}%
\pgfsys@defobject{currentmarker}{\pgfqpoint{0.000000in}{-0.027778in}}{\pgfqpoint{0.000000in}{0.000000in}}{%
\pgfpathmoveto{\pgfqpoint{0.000000in}{0.000000in}}%
\pgfpathlineto{\pgfqpoint{0.000000in}{-0.027778in}}%
\pgfusepath{stroke,fill}%
}%
\begin{pgfscope}%
\pgfsys@transformshift{1.319146in}{0.417642in}%
\pgfsys@useobject{currentmarker}{}%
\end{pgfscope}%
\end{pgfscope}%
\begin{pgfscope}%
\pgfpathrectangle{\pgfqpoint{0.514278in}{0.417642in}}{\pgfqpoint{1.884052in}{1.370688in}}%
\pgfusepath{clip}%
\pgfsetrectcap%
\pgfsetroundjoin%
\pgfsetlinewidth{0.803000pt}%
\definecolor{currentstroke}{rgb}{0.850000,0.850000,0.850000}%
\pgfsetstrokecolor{currentstroke}%
\pgfsetdash{}{0pt}%
\pgfpathmoveto{\pgfqpoint{1.353776in}{0.417642in}}%
\pgfpathlineto{\pgfqpoint{1.353776in}{1.788330in}}%
\pgfusepath{stroke}%
\end{pgfscope}%
\begin{pgfscope}%
\pgfsetbuttcap%
\pgfsetroundjoin%
\definecolor{currentfill}{rgb}{0.000000,0.000000,0.000000}%
\pgfsetfillcolor{currentfill}%
\pgfsetlinewidth{0.602250pt}%
\definecolor{currentstroke}{rgb}{0.000000,0.000000,0.000000}%
\pgfsetstrokecolor{currentstroke}%
\pgfsetdash{}{0pt}%
\pgfsys@defobject{currentmarker}{\pgfqpoint{0.000000in}{-0.027778in}}{\pgfqpoint{0.000000in}{0.000000in}}{%
\pgfpathmoveto{\pgfqpoint{0.000000in}{0.000000in}}%
\pgfpathlineto{\pgfqpoint{0.000000in}{-0.027778in}}%
\pgfusepath{stroke,fill}%
}%
\begin{pgfscope}%
\pgfsys@transformshift{1.353776in}{0.417642in}%
\pgfsys@useobject{currentmarker}{}%
\end{pgfscope}%
\end{pgfscope}%
\begin{pgfscope}%
\pgfpathrectangle{\pgfqpoint{0.514278in}{0.417642in}}{\pgfqpoint{1.884052in}{1.370688in}}%
\pgfusepath{clip}%
\pgfsetrectcap%
\pgfsetroundjoin%
\pgfsetlinewidth{0.803000pt}%
\definecolor{currentstroke}{rgb}{0.850000,0.850000,0.850000}%
\pgfsetstrokecolor{currentstroke}%
\pgfsetdash{}{0pt}%
\pgfpathmoveto{\pgfqpoint{1.383774in}{0.417642in}}%
\pgfpathlineto{\pgfqpoint{1.383774in}{1.788330in}}%
\pgfusepath{stroke}%
\end{pgfscope}%
\begin{pgfscope}%
\pgfsetbuttcap%
\pgfsetroundjoin%
\definecolor{currentfill}{rgb}{0.000000,0.000000,0.000000}%
\pgfsetfillcolor{currentfill}%
\pgfsetlinewidth{0.602250pt}%
\definecolor{currentstroke}{rgb}{0.000000,0.000000,0.000000}%
\pgfsetstrokecolor{currentstroke}%
\pgfsetdash{}{0pt}%
\pgfsys@defobject{currentmarker}{\pgfqpoint{0.000000in}{-0.027778in}}{\pgfqpoint{0.000000in}{0.000000in}}{%
\pgfpathmoveto{\pgfqpoint{0.000000in}{0.000000in}}%
\pgfpathlineto{\pgfqpoint{0.000000in}{-0.027778in}}%
\pgfusepath{stroke,fill}%
}%
\begin{pgfscope}%
\pgfsys@transformshift{1.383774in}{0.417642in}%
\pgfsys@useobject{currentmarker}{}%
\end{pgfscope}%
\end{pgfscope}%
\begin{pgfscope}%
\pgfpathrectangle{\pgfqpoint{0.514278in}{0.417642in}}{\pgfqpoint{1.884052in}{1.370688in}}%
\pgfusepath{clip}%
\pgfsetrectcap%
\pgfsetroundjoin%
\pgfsetlinewidth{0.803000pt}%
\definecolor{currentstroke}{rgb}{0.850000,0.850000,0.850000}%
\pgfsetstrokecolor{currentstroke}%
\pgfsetdash{}{0pt}%
\pgfpathmoveto{\pgfqpoint{1.410234in}{0.417642in}}%
\pgfpathlineto{\pgfqpoint{1.410234in}{1.788330in}}%
\pgfusepath{stroke}%
\end{pgfscope}%
\begin{pgfscope}%
\pgfsetbuttcap%
\pgfsetroundjoin%
\definecolor{currentfill}{rgb}{0.000000,0.000000,0.000000}%
\pgfsetfillcolor{currentfill}%
\pgfsetlinewidth{0.602250pt}%
\definecolor{currentstroke}{rgb}{0.000000,0.000000,0.000000}%
\pgfsetstrokecolor{currentstroke}%
\pgfsetdash{}{0pt}%
\pgfsys@defobject{currentmarker}{\pgfqpoint{0.000000in}{-0.027778in}}{\pgfqpoint{0.000000in}{0.000000in}}{%
\pgfpathmoveto{\pgfqpoint{0.000000in}{0.000000in}}%
\pgfpathlineto{\pgfqpoint{0.000000in}{-0.027778in}}%
\pgfusepath{stroke,fill}%
}%
\begin{pgfscope}%
\pgfsys@transformshift{1.410234in}{0.417642in}%
\pgfsys@useobject{currentmarker}{}%
\end{pgfscope}%
\end{pgfscope}%
\begin{pgfscope}%
\pgfpathrectangle{\pgfqpoint{0.514278in}{0.417642in}}{\pgfqpoint{1.884052in}{1.370688in}}%
\pgfusepath{clip}%
\pgfsetrectcap%
\pgfsetroundjoin%
\pgfsetlinewidth{0.803000pt}%
\definecolor{currentstroke}{rgb}{0.850000,0.850000,0.850000}%
\pgfsetstrokecolor{currentstroke}%
\pgfsetdash{}{0pt}%
\pgfpathmoveto{\pgfqpoint{1.589620in}{0.417642in}}%
\pgfpathlineto{\pgfqpoint{1.589620in}{1.788330in}}%
\pgfusepath{stroke}%
\end{pgfscope}%
\begin{pgfscope}%
\pgfsetbuttcap%
\pgfsetroundjoin%
\definecolor{currentfill}{rgb}{0.000000,0.000000,0.000000}%
\pgfsetfillcolor{currentfill}%
\pgfsetlinewidth{0.602250pt}%
\definecolor{currentstroke}{rgb}{0.000000,0.000000,0.000000}%
\pgfsetstrokecolor{currentstroke}%
\pgfsetdash{}{0pt}%
\pgfsys@defobject{currentmarker}{\pgfqpoint{0.000000in}{-0.027778in}}{\pgfqpoint{0.000000in}{0.000000in}}{%
\pgfpathmoveto{\pgfqpoint{0.000000in}{0.000000in}}%
\pgfpathlineto{\pgfqpoint{0.000000in}{-0.027778in}}%
\pgfusepath{stroke,fill}%
}%
\begin{pgfscope}%
\pgfsys@transformshift{1.589620in}{0.417642in}%
\pgfsys@useobject{currentmarker}{}%
\end{pgfscope}%
\end{pgfscope}%
\begin{pgfscope}%
\pgfpathrectangle{\pgfqpoint{0.514278in}{0.417642in}}{\pgfqpoint{1.884052in}{1.370688in}}%
\pgfusepath{clip}%
\pgfsetrectcap%
\pgfsetroundjoin%
\pgfsetlinewidth{0.803000pt}%
\definecolor{currentstroke}{rgb}{0.850000,0.850000,0.850000}%
\pgfsetstrokecolor{currentstroke}%
\pgfsetdash{}{0pt}%
\pgfpathmoveto{\pgfqpoint{1.680709in}{0.417642in}}%
\pgfpathlineto{\pgfqpoint{1.680709in}{1.788330in}}%
\pgfusepath{stroke}%
\end{pgfscope}%
\begin{pgfscope}%
\pgfsetbuttcap%
\pgfsetroundjoin%
\definecolor{currentfill}{rgb}{0.000000,0.000000,0.000000}%
\pgfsetfillcolor{currentfill}%
\pgfsetlinewidth{0.602250pt}%
\definecolor{currentstroke}{rgb}{0.000000,0.000000,0.000000}%
\pgfsetstrokecolor{currentstroke}%
\pgfsetdash{}{0pt}%
\pgfsys@defobject{currentmarker}{\pgfqpoint{0.000000in}{-0.027778in}}{\pgfqpoint{0.000000in}{0.000000in}}{%
\pgfpathmoveto{\pgfqpoint{0.000000in}{0.000000in}}%
\pgfpathlineto{\pgfqpoint{0.000000in}{-0.027778in}}%
\pgfusepath{stroke,fill}%
}%
\begin{pgfscope}%
\pgfsys@transformshift{1.680709in}{0.417642in}%
\pgfsys@useobject{currentmarker}{}%
\end{pgfscope}%
\end{pgfscope}%
\begin{pgfscope}%
\pgfpathrectangle{\pgfqpoint{0.514278in}{0.417642in}}{\pgfqpoint{1.884052in}{1.370688in}}%
\pgfusepath{clip}%
\pgfsetrectcap%
\pgfsetroundjoin%
\pgfsetlinewidth{0.803000pt}%
\definecolor{currentstroke}{rgb}{0.850000,0.850000,0.850000}%
\pgfsetstrokecolor{currentstroke}%
\pgfsetdash{}{0pt}%
\pgfpathmoveto{\pgfqpoint{1.745337in}{0.417642in}}%
\pgfpathlineto{\pgfqpoint{1.745337in}{1.788330in}}%
\pgfusepath{stroke}%
\end{pgfscope}%
\begin{pgfscope}%
\pgfsetbuttcap%
\pgfsetroundjoin%
\definecolor{currentfill}{rgb}{0.000000,0.000000,0.000000}%
\pgfsetfillcolor{currentfill}%
\pgfsetlinewidth{0.602250pt}%
\definecolor{currentstroke}{rgb}{0.000000,0.000000,0.000000}%
\pgfsetstrokecolor{currentstroke}%
\pgfsetdash{}{0pt}%
\pgfsys@defobject{currentmarker}{\pgfqpoint{0.000000in}{-0.027778in}}{\pgfqpoint{0.000000in}{0.000000in}}{%
\pgfpathmoveto{\pgfqpoint{0.000000in}{0.000000in}}%
\pgfpathlineto{\pgfqpoint{0.000000in}{-0.027778in}}%
\pgfusepath{stroke,fill}%
}%
\begin{pgfscope}%
\pgfsys@transformshift{1.745337in}{0.417642in}%
\pgfsys@useobject{currentmarker}{}%
\end{pgfscope}%
\end{pgfscope}%
\begin{pgfscope}%
\pgfpathrectangle{\pgfqpoint{0.514278in}{0.417642in}}{\pgfqpoint{1.884052in}{1.370688in}}%
\pgfusepath{clip}%
\pgfsetrectcap%
\pgfsetroundjoin%
\pgfsetlinewidth{0.803000pt}%
\definecolor{currentstroke}{rgb}{0.850000,0.850000,0.850000}%
\pgfsetstrokecolor{currentstroke}%
\pgfsetdash{}{0pt}%
\pgfpathmoveto{\pgfqpoint{1.795466in}{0.417642in}}%
\pgfpathlineto{\pgfqpoint{1.795466in}{1.788330in}}%
\pgfusepath{stroke}%
\end{pgfscope}%
\begin{pgfscope}%
\pgfsetbuttcap%
\pgfsetroundjoin%
\definecolor{currentfill}{rgb}{0.000000,0.000000,0.000000}%
\pgfsetfillcolor{currentfill}%
\pgfsetlinewidth{0.602250pt}%
\definecolor{currentstroke}{rgb}{0.000000,0.000000,0.000000}%
\pgfsetstrokecolor{currentstroke}%
\pgfsetdash{}{0pt}%
\pgfsys@defobject{currentmarker}{\pgfqpoint{0.000000in}{-0.027778in}}{\pgfqpoint{0.000000in}{0.000000in}}{%
\pgfpathmoveto{\pgfqpoint{0.000000in}{0.000000in}}%
\pgfpathlineto{\pgfqpoint{0.000000in}{-0.027778in}}%
\pgfusepath{stroke,fill}%
}%
\begin{pgfscope}%
\pgfsys@transformshift{1.795466in}{0.417642in}%
\pgfsys@useobject{currentmarker}{}%
\end{pgfscope}%
\end{pgfscope}%
\begin{pgfscope}%
\pgfpathrectangle{\pgfqpoint{0.514278in}{0.417642in}}{\pgfqpoint{1.884052in}{1.370688in}}%
\pgfusepath{clip}%
\pgfsetrectcap%
\pgfsetroundjoin%
\pgfsetlinewidth{0.803000pt}%
\definecolor{currentstroke}{rgb}{0.850000,0.850000,0.850000}%
\pgfsetstrokecolor{currentstroke}%
\pgfsetdash{}{0pt}%
\pgfpathmoveto{\pgfqpoint{1.836425in}{0.417642in}}%
\pgfpathlineto{\pgfqpoint{1.836425in}{1.788330in}}%
\pgfusepath{stroke}%
\end{pgfscope}%
\begin{pgfscope}%
\pgfsetbuttcap%
\pgfsetroundjoin%
\definecolor{currentfill}{rgb}{0.000000,0.000000,0.000000}%
\pgfsetfillcolor{currentfill}%
\pgfsetlinewidth{0.602250pt}%
\definecolor{currentstroke}{rgb}{0.000000,0.000000,0.000000}%
\pgfsetstrokecolor{currentstroke}%
\pgfsetdash{}{0pt}%
\pgfsys@defobject{currentmarker}{\pgfqpoint{0.000000in}{-0.027778in}}{\pgfqpoint{0.000000in}{0.000000in}}{%
\pgfpathmoveto{\pgfqpoint{0.000000in}{0.000000in}}%
\pgfpathlineto{\pgfqpoint{0.000000in}{-0.027778in}}%
\pgfusepath{stroke,fill}%
}%
\begin{pgfscope}%
\pgfsys@transformshift{1.836425in}{0.417642in}%
\pgfsys@useobject{currentmarker}{}%
\end{pgfscope}%
\end{pgfscope}%
\begin{pgfscope}%
\pgfpathrectangle{\pgfqpoint{0.514278in}{0.417642in}}{\pgfqpoint{1.884052in}{1.370688in}}%
\pgfusepath{clip}%
\pgfsetrectcap%
\pgfsetroundjoin%
\pgfsetlinewidth{0.803000pt}%
\definecolor{currentstroke}{rgb}{0.850000,0.850000,0.850000}%
\pgfsetstrokecolor{currentstroke}%
\pgfsetdash{}{0pt}%
\pgfpathmoveto{\pgfqpoint{1.871056in}{0.417642in}}%
\pgfpathlineto{\pgfqpoint{1.871056in}{1.788330in}}%
\pgfusepath{stroke}%
\end{pgfscope}%
\begin{pgfscope}%
\pgfsetbuttcap%
\pgfsetroundjoin%
\definecolor{currentfill}{rgb}{0.000000,0.000000,0.000000}%
\pgfsetfillcolor{currentfill}%
\pgfsetlinewidth{0.602250pt}%
\definecolor{currentstroke}{rgb}{0.000000,0.000000,0.000000}%
\pgfsetstrokecolor{currentstroke}%
\pgfsetdash{}{0pt}%
\pgfsys@defobject{currentmarker}{\pgfqpoint{0.000000in}{-0.027778in}}{\pgfqpoint{0.000000in}{0.000000in}}{%
\pgfpathmoveto{\pgfqpoint{0.000000in}{0.000000in}}%
\pgfpathlineto{\pgfqpoint{0.000000in}{-0.027778in}}%
\pgfusepath{stroke,fill}%
}%
\begin{pgfscope}%
\pgfsys@transformshift{1.871056in}{0.417642in}%
\pgfsys@useobject{currentmarker}{}%
\end{pgfscope}%
\end{pgfscope}%
\begin{pgfscope}%
\pgfpathrectangle{\pgfqpoint{0.514278in}{0.417642in}}{\pgfqpoint{1.884052in}{1.370688in}}%
\pgfusepath{clip}%
\pgfsetrectcap%
\pgfsetroundjoin%
\pgfsetlinewidth{0.803000pt}%
\definecolor{currentstroke}{rgb}{0.850000,0.850000,0.850000}%
\pgfsetstrokecolor{currentstroke}%
\pgfsetdash{}{0pt}%
\pgfpathmoveto{\pgfqpoint{1.901054in}{0.417642in}}%
\pgfpathlineto{\pgfqpoint{1.901054in}{1.788330in}}%
\pgfusepath{stroke}%
\end{pgfscope}%
\begin{pgfscope}%
\pgfsetbuttcap%
\pgfsetroundjoin%
\definecolor{currentfill}{rgb}{0.000000,0.000000,0.000000}%
\pgfsetfillcolor{currentfill}%
\pgfsetlinewidth{0.602250pt}%
\definecolor{currentstroke}{rgb}{0.000000,0.000000,0.000000}%
\pgfsetstrokecolor{currentstroke}%
\pgfsetdash{}{0pt}%
\pgfsys@defobject{currentmarker}{\pgfqpoint{0.000000in}{-0.027778in}}{\pgfqpoint{0.000000in}{0.000000in}}{%
\pgfpathmoveto{\pgfqpoint{0.000000in}{0.000000in}}%
\pgfpathlineto{\pgfqpoint{0.000000in}{-0.027778in}}%
\pgfusepath{stroke,fill}%
}%
\begin{pgfscope}%
\pgfsys@transformshift{1.901054in}{0.417642in}%
\pgfsys@useobject{currentmarker}{}%
\end{pgfscope}%
\end{pgfscope}%
\begin{pgfscope}%
\pgfpathrectangle{\pgfqpoint{0.514278in}{0.417642in}}{\pgfqpoint{1.884052in}{1.370688in}}%
\pgfusepath{clip}%
\pgfsetrectcap%
\pgfsetroundjoin%
\pgfsetlinewidth{0.803000pt}%
\definecolor{currentstroke}{rgb}{0.850000,0.850000,0.850000}%
\pgfsetstrokecolor{currentstroke}%
\pgfsetdash{}{0pt}%
\pgfpathmoveto{\pgfqpoint{1.927514in}{0.417642in}}%
\pgfpathlineto{\pgfqpoint{1.927514in}{1.788330in}}%
\pgfusepath{stroke}%
\end{pgfscope}%
\begin{pgfscope}%
\pgfsetbuttcap%
\pgfsetroundjoin%
\definecolor{currentfill}{rgb}{0.000000,0.000000,0.000000}%
\pgfsetfillcolor{currentfill}%
\pgfsetlinewidth{0.602250pt}%
\definecolor{currentstroke}{rgb}{0.000000,0.000000,0.000000}%
\pgfsetstrokecolor{currentstroke}%
\pgfsetdash{}{0pt}%
\pgfsys@defobject{currentmarker}{\pgfqpoint{0.000000in}{-0.027778in}}{\pgfqpoint{0.000000in}{0.000000in}}{%
\pgfpathmoveto{\pgfqpoint{0.000000in}{0.000000in}}%
\pgfpathlineto{\pgfqpoint{0.000000in}{-0.027778in}}%
\pgfusepath{stroke,fill}%
}%
\begin{pgfscope}%
\pgfsys@transformshift{1.927514in}{0.417642in}%
\pgfsys@useobject{currentmarker}{}%
\end{pgfscope}%
\end{pgfscope}%
\begin{pgfscope}%
\pgfpathrectangle{\pgfqpoint{0.514278in}{0.417642in}}{\pgfqpoint{1.884052in}{1.370688in}}%
\pgfusepath{clip}%
\pgfsetrectcap%
\pgfsetroundjoin%
\pgfsetlinewidth{0.803000pt}%
\definecolor{currentstroke}{rgb}{0.850000,0.850000,0.850000}%
\pgfsetstrokecolor{currentstroke}%
\pgfsetdash{}{0pt}%
\pgfpathmoveto{\pgfqpoint{2.106900in}{0.417642in}}%
\pgfpathlineto{\pgfqpoint{2.106900in}{1.788330in}}%
\pgfusepath{stroke}%
\end{pgfscope}%
\begin{pgfscope}%
\pgfsetbuttcap%
\pgfsetroundjoin%
\definecolor{currentfill}{rgb}{0.000000,0.000000,0.000000}%
\pgfsetfillcolor{currentfill}%
\pgfsetlinewidth{0.602250pt}%
\definecolor{currentstroke}{rgb}{0.000000,0.000000,0.000000}%
\pgfsetstrokecolor{currentstroke}%
\pgfsetdash{}{0pt}%
\pgfsys@defobject{currentmarker}{\pgfqpoint{0.000000in}{-0.027778in}}{\pgfqpoint{0.000000in}{0.000000in}}{%
\pgfpathmoveto{\pgfqpoint{0.000000in}{0.000000in}}%
\pgfpathlineto{\pgfqpoint{0.000000in}{-0.027778in}}%
\pgfusepath{stroke,fill}%
}%
\begin{pgfscope}%
\pgfsys@transformshift{2.106900in}{0.417642in}%
\pgfsys@useobject{currentmarker}{}%
\end{pgfscope}%
\end{pgfscope}%
\begin{pgfscope}%
\pgfpathrectangle{\pgfqpoint{0.514278in}{0.417642in}}{\pgfqpoint{1.884052in}{1.370688in}}%
\pgfusepath{clip}%
\pgfsetrectcap%
\pgfsetroundjoin%
\pgfsetlinewidth{0.803000pt}%
\definecolor{currentstroke}{rgb}{0.850000,0.850000,0.850000}%
\pgfsetstrokecolor{currentstroke}%
\pgfsetdash{}{0pt}%
\pgfpathmoveto{\pgfqpoint{2.197988in}{0.417642in}}%
\pgfpathlineto{\pgfqpoint{2.197988in}{1.788330in}}%
\pgfusepath{stroke}%
\end{pgfscope}%
\begin{pgfscope}%
\pgfsetbuttcap%
\pgfsetroundjoin%
\definecolor{currentfill}{rgb}{0.000000,0.000000,0.000000}%
\pgfsetfillcolor{currentfill}%
\pgfsetlinewidth{0.602250pt}%
\definecolor{currentstroke}{rgb}{0.000000,0.000000,0.000000}%
\pgfsetstrokecolor{currentstroke}%
\pgfsetdash{}{0pt}%
\pgfsys@defobject{currentmarker}{\pgfqpoint{0.000000in}{-0.027778in}}{\pgfqpoint{0.000000in}{0.000000in}}{%
\pgfpathmoveto{\pgfqpoint{0.000000in}{0.000000in}}%
\pgfpathlineto{\pgfqpoint{0.000000in}{-0.027778in}}%
\pgfusepath{stroke,fill}%
}%
\begin{pgfscope}%
\pgfsys@transformshift{2.197988in}{0.417642in}%
\pgfsys@useobject{currentmarker}{}%
\end{pgfscope}%
\end{pgfscope}%
\begin{pgfscope}%
\pgfpathrectangle{\pgfqpoint{0.514278in}{0.417642in}}{\pgfqpoint{1.884052in}{1.370688in}}%
\pgfusepath{clip}%
\pgfsetrectcap%
\pgfsetroundjoin%
\pgfsetlinewidth{0.803000pt}%
\definecolor{currentstroke}{rgb}{0.850000,0.850000,0.850000}%
\pgfsetstrokecolor{currentstroke}%
\pgfsetdash{}{0pt}%
\pgfpathmoveto{\pgfqpoint{2.262617in}{0.417642in}}%
\pgfpathlineto{\pgfqpoint{2.262617in}{1.788330in}}%
\pgfusepath{stroke}%
\end{pgfscope}%
\begin{pgfscope}%
\pgfsetbuttcap%
\pgfsetroundjoin%
\definecolor{currentfill}{rgb}{0.000000,0.000000,0.000000}%
\pgfsetfillcolor{currentfill}%
\pgfsetlinewidth{0.602250pt}%
\definecolor{currentstroke}{rgb}{0.000000,0.000000,0.000000}%
\pgfsetstrokecolor{currentstroke}%
\pgfsetdash{}{0pt}%
\pgfsys@defobject{currentmarker}{\pgfqpoint{0.000000in}{-0.027778in}}{\pgfqpoint{0.000000in}{0.000000in}}{%
\pgfpathmoveto{\pgfqpoint{0.000000in}{0.000000in}}%
\pgfpathlineto{\pgfqpoint{0.000000in}{-0.027778in}}%
\pgfusepath{stroke,fill}%
}%
\begin{pgfscope}%
\pgfsys@transformshift{2.262617in}{0.417642in}%
\pgfsys@useobject{currentmarker}{}%
\end{pgfscope}%
\end{pgfscope}%
\begin{pgfscope}%
\pgfpathrectangle{\pgfqpoint{0.514278in}{0.417642in}}{\pgfqpoint{1.884052in}{1.370688in}}%
\pgfusepath{clip}%
\pgfsetrectcap%
\pgfsetroundjoin%
\pgfsetlinewidth{0.803000pt}%
\definecolor{currentstroke}{rgb}{0.850000,0.850000,0.850000}%
\pgfsetstrokecolor{currentstroke}%
\pgfsetdash{}{0pt}%
\pgfpathmoveto{\pgfqpoint{2.312746in}{0.417642in}}%
\pgfpathlineto{\pgfqpoint{2.312746in}{1.788330in}}%
\pgfusepath{stroke}%
\end{pgfscope}%
\begin{pgfscope}%
\pgfsetbuttcap%
\pgfsetroundjoin%
\definecolor{currentfill}{rgb}{0.000000,0.000000,0.000000}%
\pgfsetfillcolor{currentfill}%
\pgfsetlinewidth{0.602250pt}%
\definecolor{currentstroke}{rgb}{0.000000,0.000000,0.000000}%
\pgfsetstrokecolor{currentstroke}%
\pgfsetdash{}{0pt}%
\pgfsys@defobject{currentmarker}{\pgfqpoint{0.000000in}{-0.027778in}}{\pgfqpoint{0.000000in}{0.000000in}}{%
\pgfpathmoveto{\pgfqpoint{0.000000in}{0.000000in}}%
\pgfpathlineto{\pgfqpoint{0.000000in}{-0.027778in}}%
\pgfusepath{stroke,fill}%
}%
\begin{pgfscope}%
\pgfsys@transformshift{2.312746in}{0.417642in}%
\pgfsys@useobject{currentmarker}{}%
\end{pgfscope}%
\end{pgfscope}%
\begin{pgfscope}%
\pgfpathrectangle{\pgfqpoint{0.514278in}{0.417642in}}{\pgfqpoint{1.884052in}{1.370688in}}%
\pgfusepath{clip}%
\pgfsetrectcap%
\pgfsetroundjoin%
\pgfsetlinewidth{0.803000pt}%
\definecolor{currentstroke}{rgb}{0.850000,0.850000,0.850000}%
\pgfsetstrokecolor{currentstroke}%
\pgfsetdash{}{0pt}%
\pgfpathmoveto{\pgfqpoint{2.353705in}{0.417642in}}%
\pgfpathlineto{\pgfqpoint{2.353705in}{1.788330in}}%
\pgfusepath{stroke}%
\end{pgfscope}%
\begin{pgfscope}%
\pgfsetbuttcap%
\pgfsetroundjoin%
\definecolor{currentfill}{rgb}{0.000000,0.000000,0.000000}%
\pgfsetfillcolor{currentfill}%
\pgfsetlinewidth{0.602250pt}%
\definecolor{currentstroke}{rgb}{0.000000,0.000000,0.000000}%
\pgfsetstrokecolor{currentstroke}%
\pgfsetdash{}{0pt}%
\pgfsys@defobject{currentmarker}{\pgfqpoint{0.000000in}{-0.027778in}}{\pgfqpoint{0.000000in}{0.000000in}}{%
\pgfpathmoveto{\pgfqpoint{0.000000in}{0.000000in}}%
\pgfpathlineto{\pgfqpoint{0.000000in}{-0.027778in}}%
\pgfusepath{stroke,fill}%
}%
\begin{pgfscope}%
\pgfsys@transformshift{2.353705in}{0.417642in}%
\pgfsys@useobject{currentmarker}{}%
\end{pgfscope}%
\end{pgfscope}%
\begin{pgfscope}%
\pgfpathrectangle{\pgfqpoint{0.514278in}{0.417642in}}{\pgfqpoint{1.884052in}{1.370688in}}%
\pgfusepath{clip}%
\pgfsetrectcap%
\pgfsetroundjoin%
\pgfsetlinewidth{0.803000pt}%
\definecolor{currentstroke}{rgb}{0.850000,0.850000,0.850000}%
\pgfsetstrokecolor{currentstroke}%
\pgfsetdash{}{0pt}%
\pgfpathmoveto{\pgfqpoint{2.388335in}{0.417642in}}%
\pgfpathlineto{\pgfqpoint{2.388335in}{1.788330in}}%
\pgfusepath{stroke}%
\end{pgfscope}%
\begin{pgfscope}%
\pgfsetbuttcap%
\pgfsetroundjoin%
\definecolor{currentfill}{rgb}{0.000000,0.000000,0.000000}%
\pgfsetfillcolor{currentfill}%
\pgfsetlinewidth{0.602250pt}%
\definecolor{currentstroke}{rgb}{0.000000,0.000000,0.000000}%
\pgfsetstrokecolor{currentstroke}%
\pgfsetdash{}{0pt}%
\pgfsys@defobject{currentmarker}{\pgfqpoint{0.000000in}{-0.027778in}}{\pgfqpoint{0.000000in}{0.000000in}}{%
\pgfpathmoveto{\pgfqpoint{0.000000in}{0.000000in}}%
\pgfpathlineto{\pgfqpoint{0.000000in}{-0.027778in}}%
\pgfusepath{stroke,fill}%
}%
\begin{pgfscope}%
\pgfsys@transformshift{2.388335in}{0.417642in}%
\pgfsys@useobject{currentmarker}{}%
\end{pgfscope}%
\end{pgfscope}%
\begin{pgfscope}%
\definecolor{textcolor}{rgb}{0.000000,0.000000,0.000000}%
\pgfsetstrokecolor{textcolor}%
\pgfsetfillcolor{textcolor}%
\pgftext[x=1.456304in,y=0.165003in,,top]{\color{textcolor}\rmfamily\fontsize{10.000000}{12.000000}\selectfont Frequency in \(\displaystyle \unit{\Hz}\)}%
\end{pgfscope}%
\begin{pgfscope}%
\pgfpathrectangle{\pgfqpoint{0.514278in}{0.417642in}}{\pgfqpoint{1.884052in}{1.370688in}}%
\pgfusepath{clip}%
\pgfsetrectcap%
\pgfsetroundjoin%
\pgfsetlinewidth{0.803000pt}%
\definecolor{currentstroke}{rgb}{0.450000,0.450000,0.450000}%
\pgfsetstrokecolor{currentstroke}%
\pgfsetdash{}{0pt}%
\pgfpathmoveto{\pgfqpoint{0.514278in}{0.640555in}}%
\pgfpathlineto{\pgfqpoint{2.398330in}{0.640555in}}%
\pgfusepath{stroke}%
\end{pgfscope}%
\begin{pgfscope}%
\pgfsetbuttcap%
\pgfsetroundjoin%
\definecolor{currentfill}{rgb}{0.000000,0.000000,0.000000}%
\pgfsetfillcolor{currentfill}%
\pgfsetlinewidth{0.803000pt}%
\definecolor{currentstroke}{rgb}{0.000000,0.000000,0.000000}%
\pgfsetstrokecolor{currentstroke}%
\pgfsetdash{}{0pt}%
\pgfsys@defobject{currentmarker}{\pgfqpoint{-0.048611in}{0.000000in}}{\pgfqpoint{-0.000000in}{0.000000in}}{%
\pgfpathmoveto{\pgfqpoint{-0.000000in}{0.000000in}}%
\pgfpathlineto{\pgfqpoint{-0.048611in}{0.000000in}}%
\pgfusepath{stroke,fill}%
}%
\begin{pgfscope}%
\pgfsys@transformshift{0.514278in}{0.640555in}%
\pgfsys@useobject{currentmarker}{}%
\end{pgfscope}%
\end{pgfscope}%
\begin{pgfscope}%
\definecolor{textcolor}{rgb}{0.000000,0.000000,0.000000}%
\pgfsetstrokecolor{textcolor}%
\pgfsetfillcolor{textcolor}%
\pgftext[x=0.241129in, y=0.601402in, left, base]{\color{textcolor}\rmfamily\fontsize{8.000000}{9.600000}\selectfont \(\displaystyle {10^{0}}\)}%
\end{pgfscope}%
\begin{pgfscope}%
\pgfpathrectangle{\pgfqpoint{0.514278in}{0.417642in}}{\pgfqpoint{1.884052in}{1.370688in}}%
\pgfusepath{clip}%
\pgfsetrectcap%
\pgfsetroundjoin%
\pgfsetlinewidth{0.803000pt}%
\definecolor{currentstroke}{rgb}{0.450000,0.450000,0.450000}%
\pgfsetstrokecolor{currentstroke}%
\pgfsetdash{}{0pt}%
\pgfpathmoveto{\pgfqpoint{0.514278in}{0.983227in}}%
\pgfpathlineto{\pgfqpoint{2.398330in}{0.983227in}}%
\pgfusepath{stroke}%
\end{pgfscope}%
\begin{pgfscope}%
\pgfsetbuttcap%
\pgfsetroundjoin%
\definecolor{currentfill}{rgb}{0.000000,0.000000,0.000000}%
\pgfsetfillcolor{currentfill}%
\pgfsetlinewidth{0.803000pt}%
\definecolor{currentstroke}{rgb}{0.000000,0.000000,0.000000}%
\pgfsetstrokecolor{currentstroke}%
\pgfsetdash{}{0pt}%
\pgfsys@defobject{currentmarker}{\pgfqpoint{-0.048611in}{0.000000in}}{\pgfqpoint{-0.000000in}{0.000000in}}{%
\pgfpathmoveto{\pgfqpoint{-0.000000in}{0.000000in}}%
\pgfpathlineto{\pgfqpoint{-0.048611in}{0.000000in}}%
\pgfusepath{stroke,fill}%
}%
\begin{pgfscope}%
\pgfsys@transformshift{0.514278in}{0.983227in}%
\pgfsys@useobject{currentmarker}{}%
\end{pgfscope}%
\end{pgfscope}%
\begin{pgfscope}%
\definecolor{textcolor}{rgb}{0.000000,0.000000,0.000000}%
\pgfsetstrokecolor{textcolor}%
\pgfsetfillcolor{textcolor}%
\pgftext[x=0.241129in, y=0.944074in, left, base]{\color{textcolor}\rmfamily\fontsize{8.000000}{9.600000}\selectfont \(\displaystyle {10^{2}}\)}%
\end{pgfscope}%
\begin{pgfscope}%
\pgfpathrectangle{\pgfqpoint{0.514278in}{0.417642in}}{\pgfqpoint{1.884052in}{1.370688in}}%
\pgfusepath{clip}%
\pgfsetrectcap%
\pgfsetroundjoin%
\pgfsetlinewidth{0.803000pt}%
\definecolor{currentstroke}{rgb}{0.450000,0.450000,0.450000}%
\pgfsetstrokecolor{currentstroke}%
\pgfsetdash{}{0pt}%
\pgfpathmoveto{\pgfqpoint{0.514278in}{1.325899in}}%
\pgfpathlineto{\pgfqpoint{2.398330in}{1.325899in}}%
\pgfusepath{stroke}%
\end{pgfscope}%
\begin{pgfscope}%
\pgfsetbuttcap%
\pgfsetroundjoin%
\definecolor{currentfill}{rgb}{0.000000,0.000000,0.000000}%
\pgfsetfillcolor{currentfill}%
\pgfsetlinewidth{0.803000pt}%
\definecolor{currentstroke}{rgb}{0.000000,0.000000,0.000000}%
\pgfsetstrokecolor{currentstroke}%
\pgfsetdash{}{0pt}%
\pgfsys@defobject{currentmarker}{\pgfqpoint{-0.048611in}{0.000000in}}{\pgfqpoint{-0.000000in}{0.000000in}}{%
\pgfpathmoveto{\pgfqpoint{-0.000000in}{0.000000in}}%
\pgfpathlineto{\pgfqpoint{-0.048611in}{0.000000in}}%
\pgfusepath{stroke,fill}%
}%
\begin{pgfscope}%
\pgfsys@transformshift{0.514278in}{1.325899in}%
\pgfsys@useobject{currentmarker}{}%
\end{pgfscope}%
\end{pgfscope}%
\begin{pgfscope}%
\definecolor{textcolor}{rgb}{0.000000,0.000000,0.000000}%
\pgfsetstrokecolor{textcolor}%
\pgfsetfillcolor{textcolor}%
\pgftext[x=0.241129in, y=1.286746in, left, base]{\color{textcolor}\rmfamily\fontsize{8.000000}{9.600000}\selectfont \(\displaystyle {10^{4}}\)}%
\end{pgfscope}%
\begin{pgfscope}%
\pgfpathrectangle{\pgfqpoint{0.514278in}{0.417642in}}{\pgfqpoint{1.884052in}{1.370688in}}%
\pgfusepath{clip}%
\pgfsetrectcap%
\pgfsetroundjoin%
\pgfsetlinewidth{0.803000pt}%
\definecolor{currentstroke}{rgb}{0.450000,0.450000,0.450000}%
\pgfsetstrokecolor{currentstroke}%
\pgfsetdash{}{0pt}%
\pgfpathmoveto{\pgfqpoint{0.514278in}{1.668571in}}%
\pgfpathlineto{\pgfqpoint{2.398330in}{1.668571in}}%
\pgfusepath{stroke}%
\end{pgfscope}%
\begin{pgfscope}%
\pgfsetbuttcap%
\pgfsetroundjoin%
\definecolor{currentfill}{rgb}{0.000000,0.000000,0.000000}%
\pgfsetfillcolor{currentfill}%
\pgfsetlinewidth{0.803000pt}%
\definecolor{currentstroke}{rgb}{0.000000,0.000000,0.000000}%
\pgfsetstrokecolor{currentstroke}%
\pgfsetdash{}{0pt}%
\pgfsys@defobject{currentmarker}{\pgfqpoint{-0.048611in}{0.000000in}}{\pgfqpoint{-0.000000in}{0.000000in}}{%
\pgfpathmoveto{\pgfqpoint{-0.000000in}{0.000000in}}%
\pgfpathlineto{\pgfqpoint{-0.048611in}{0.000000in}}%
\pgfusepath{stroke,fill}%
}%
\begin{pgfscope}%
\pgfsys@transformshift{0.514278in}{1.668571in}%
\pgfsys@useobject{currentmarker}{}%
\end{pgfscope}%
\end{pgfscope}%
\begin{pgfscope}%
\definecolor{textcolor}{rgb}{0.000000,0.000000,0.000000}%
\pgfsetstrokecolor{textcolor}%
\pgfsetfillcolor{textcolor}%
\pgftext[x=0.241129in, y=1.629418in, left, base]{\color{textcolor}\rmfamily\fontsize{8.000000}{9.600000}\selectfont \(\displaystyle {10^{6}}\)}%
\end{pgfscope}%
\begin{pgfscope}%
\definecolor{textcolor}{rgb}{0.000000,0.000000,0.000000}%
\pgfsetstrokecolor{textcolor}%
\pgfsetfillcolor{textcolor}%
\pgftext[x=0.185574in,y=1.102986in,,bottom,rotate=90.000000]{\color{textcolor}\rmfamily\fontsize{10.000000}{12.000000}\selectfont  \(\displaystyle S_y(f)\) in \(\displaystyle \unit{1 \per \Hz}\)}%
\end{pgfscope}%
\begin{pgfscope}%
\pgfpathrectangle{\pgfqpoint{0.514278in}{0.417642in}}{\pgfqpoint{1.884052in}{1.370688in}}%
\pgfusepath{clip}%
\pgfsetbuttcap%
\pgfsetroundjoin%
\pgfsetlinewidth{1.505625pt}%
\definecolor{currentstroke}{rgb}{0.835294,0.368627,0.000000}%
\pgfsetstrokecolor{currentstroke}%
\pgfsetdash{{5.550000pt}{2.400000pt}}{0.000000pt}%
\pgfpathmoveto{\pgfqpoint{0.599917in}{1.738185in}}%
\pgfpathlineto{\pgfqpoint{2.312691in}{0.603558in}}%
\pgfpathlineto{\pgfqpoint{2.312691in}{0.603558in}}%
\pgfusepath{stroke}%
\end{pgfscope}%
\begin{pgfscope}%
\pgfpathrectangle{\pgfqpoint{0.514278in}{0.417642in}}{\pgfqpoint{1.884052in}{1.370688in}}%
\pgfusepath{clip}%
\pgfsetbuttcap%
\pgfsetroundjoin%
\definecolor{currentfill}{rgb}{0.835294,0.368627,0.000000}%
\pgfsetfillcolor{currentfill}%
\pgfsetlinewidth{1.003750pt}%
\definecolor{currentstroke}{rgb}{0.835294,0.368627,0.000000}%
\pgfsetstrokecolor{currentstroke}%
\pgfsetdash{}{0pt}%
\pgfsys@defobject{currentmarker}{\pgfqpoint{-0.006944in}{-0.006944in}}{\pgfqpoint{0.006944in}{0.006944in}}{%
\pgfpathmoveto{\pgfqpoint{0.000000in}{-0.006944in}}%
\pgfpathcurveto{\pgfqpoint{0.001842in}{-0.006944in}}{\pgfqpoint{0.003608in}{-0.006213in}}{\pgfqpoint{0.004910in}{-0.004910in}}%
\pgfpathcurveto{\pgfqpoint{0.006213in}{-0.003608in}}{\pgfqpoint{0.006944in}{-0.001842in}}{\pgfqpoint{0.006944in}{0.000000in}}%
\pgfpathcurveto{\pgfqpoint{0.006944in}{0.001842in}}{\pgfqpoint{0.006213in}{0.003608in}}{\pgfqpoint{0.004910in}{0.004910in}}%
\pgfpathcurveto{\pgfqpoint{0.003608in}{0.006213in}}{\pgfqpoint{0.001842in}{0.006944in}}{\pgfqpoint{0.000000in}{0.006944in}}%
\pgfpathcurveto{\pgfqpoint{-0.001842in}{0.006944in}}{\pgfqpoint{-0.003608in}{0.006213in}}{\pgfqpoint{-0.004910in}{0.004910in}}%
\pgfpathcurveto{\pgfqpoint{-0.006213in}{0.003608in}}{\pgfqpoint{-0.006944in}{0.001842in}}{\pgfqpoint{-0.006944in}{0.000000in}}%
\pgfpathcurveto{\pgfqpoint{-0.006944in}{-0.001842in}}{\pgfqpoint{-0.006213in}{-0.003608in}}{\pgfqpoint{-0.004910in}{-0.004910in}}%
\pgfpathcurveto{\pgfqpoint{-0.003608in}{-0.006213in}}{\pgfqpoint{-0.001842in}{-0.006944in}}{\pgfqpoint{0.000000in}{-0.006944in}}%
\pgfpathlineto{\pgfqpoint{0.000000in}{-0.006944in}}%
\pgfpathclose%
\pgfusepath{stroke,fill}%
}%
\begin{pgfscope}%
\pgfsys@transformshift{-226.701573in}{1.699596in}%
\pgfsys@useobject{currentmarker}{}%
\end{pgfscope}%
\begin{pgfscope}%
\pgfsys@transformshift{0.599917in}{1.767760in}%
\pgfsys@useobject{currentmarker}{}%
\end{pgfscope}%
\begin{pgfscope}%
\pgfsys@transformshift{0.755634in}{1.633926in}%
\pgfsys@useobject{currentmarker}{}%
\end{pgfscope}%
\begin{pgfscope}%
\pgfsys@transformshift{0.846722in}{1.542294in}%
\pgfsys@useobject{currentmarker}{}%
\end{pgfscope}%
\begin{pgfscope}%
\pgfsys@transformshift{0.911351in}{1.484904in}%
\pgfsys@useobject{currentmarker}{}%
\end{pgfscope}%
\begin{pgfscope}%
\pgfsys@transformshift{0.961480in}{1.467589in}%
\pgfsys@useobject{currentmarker}{}%
\end{pgfscope}%
\begin{pgfscope}%
\pgfsys@transformshift{1.002439in}{1.396364in}%
\pgfsys@useobject{currentmarker}{}%
\end{pgfscope}%
\begin{pgfscope}%
\pgfsys@transformshift{1.037069in}{1.455179in}%
\pgfsys@useobject{currentmarker}{}%
\end{pgfscope}%
\begin{pgfscope}%
\pgfsys@transformshift{1.067067in}{1.459825in}%
\pgfsys@useobject{currentmarker}{}%
\end{pgfscope}%
\begin{pgfscope}%
\pgfsys@transformshift{1.093527in}{1.432011in}%
\pgfsys@useobject{currentmarker}{}%
\end{pgfscope}%
\begin{pgfscope}%
\pgfsys@transformshift{1.117197in}{1.407185in}%
\pgfsys@useobject{currentmarker}{}%
\end{pgfscope}%
\begin{pgfscope}%
\pgfsys@transformshift{1.138608in}{1.363179in}%
\pgfsys@useobject{currentmarker}{}%
\end{pgfscope}%
\begin{pgfscope}%
\pgfsys@transformshift{1.158156in}{1.390307in}%
\pgfsys@useobject{currentmarker}{}%
\end{pgfscope}%
\begin{pgfscope}%
\pgfsys@transformshift{1.176137in}{1.358330in}%
\pgfsys@useobject{currentmarker}{}%
\end{pgfscope}%
\begin{pgfscope}%
\pgfsys@transformshift{1.192786in}{1.282640in}%
\pgfsys@useobject{currentmarker}{}%
\end{pgfscope}%
\begin{pgfscope}%
\pgfsys@transformshift{1.208285in}{1.342931in}%
\pgfsys@useobject{currentmarker}{}%
\end{pgfscope}%
\begin{pgfscope}%
\pgfsys@transformshift{1.222784in}{1.362450in}%
\pgfsys@useobject{currentmarker}{}%
\end{pgfscope}%
\begin{pgfscope}%
\pgfsys@transformshift{1.236403in}{1.334753in}%
\pgfsys@useobject{currentmarker}{}%
\end{pgfscope}%
\begin{pgfscope}%
\pgfsys@transformshift{1.249244in}{1.223887in}%
\pgfsys@useobject{currentmarker}{}%
\end{pgfscope}%
\begin{pgfscope}%
\pgfsys@transformshift{1.261390in}{1.267876in}%
\pgfsys@useobject{currentmarker}{}%
\end{pgfscope}%
\begin{pgfscope}%
\pgfsys@transformshift{1.272914in}{1.329337in}%
\pgfsys@useobject{currentmarker}{}%
\end{pgfscope}%
\begin{pgfscope}%
\pgfsys@transformshift{1.283874in}{1.343124in}%
\pgfsys@useobject{currentmarker}{}%
\end{pgfscope}%
\begin{pgfscope}%
\pgfsys@transformshift{1.294325in}{1.301182in}%
\pgfsys@useobject{currentmarker}{}%
\end{pgfscope}%
\begin{pgfscope}%
\pgfsys@transformshift{1.304311in}{1.275590in}%
\pgfsys@useobject{currentmarker}{}%
\end{pgfscope}%
\begin{pgfscope}%
\pgfsys@transformshift{1.313872in}{1.224897in}%
\pgfsys@useobject{currentmarker}{}%
\end{pgfscope}%
\begin{pgfscope}%
\pgfsys@transformshift{1.323043in}{1.244226in}%
\pgfsys@useobject{currentmarker}{}%
\end{pgfscope}%
\begin{pgfscope}%
\pgfsys@transformshift{1.331854in}{1.221070in}%
\pgfsys@useobject{currentmarker}{}%
\end{pgfscope}%
\begin{pgfscope}%
\pgfsys@transformshift{1.340333in}{1.212021in}%
\pgfsys@useobject{currentmarker}{}%
\end{pgfscope}%
\begin{pgfscope}%
\pgfsys@transformshift{1.348503in}{1.255859in}%
\pgfsys@useobject{currentmarker}{}%
\end{pgfscope}%
\begin{pgfscope}%
\pgfsys@transformshift{1.356386in}{1.259043in}%
\pgfsys@useobject{currentmarker}{}%
\end{pgfscope}%
\begin{pgfscope}%
\pgfsys@transformshift{1.364002in}{1.213137in}%
\pgfsys@useobject{currentmarker}{}%
\end{pgfscope}%
\begin{pgfscope}%
\pgfsys@transformshift{1.371368in}{1.234630in}%
\pgfsys@useobject{currentmarker}{}%
\end{pgfscope}%
\begin{pgfscope}%
\pgfsys@transformshift{1.378501in}{1.216343in}%
\pgfsys@useobject{currentmarker}{}%
\end{pgfscope}%
\begin{pgfscope}%
\pgfsys@transformshift{1.385414in}{1.197913in}%
\pgfsys@useobject{currentmarker}{}%
\end{pgfscope}%
\begin{pgfscope}%
\pgfsys@transformshift{1.392120in}{1.232938in}%
\pgfsys@useobject{currentmarker}{}%
\end{pgfscope}%
\begin{pgfscope}%
\pgfsys@transformshift{1.398632in}{1.191697in}%
\pgfsys@useobject{currentmarker}{}%
\end{pgfscope}%
\begin{pgfscope}%
\pgfsys@transformshift{1.404961in}{1.221783in}%
\pgfsys@useobject{currentmarker}{}%
\end{pgfscope}%
\begin{pgfscope}%
\pgfsys@transformshift{1.411116in}{1.208746in}%
\pgfsys@useobject{currentmarker}{}%
\end{pgfscope}%
\begin{pgfscope}%
\pgfsys@transformshift{1.417107in}{1.176183in}%
\pgfsys@useobject{currentmarker}{}%
\end{pgfscope}%
\begin{pgfscope}%
\pgfsys@transformshift{1.422943in}{1.131458in}%
\pgfsys@useobject{currentmarker}{}%
\end{pgfscope}%
\begin{pgfscope}%
\pgfsys@transformshift{1.428630in}{1.176332in}%
\pgfsys@useobject{currentmarker}{}%
\end{pgfscope}%
\begin{pgfscope}%
\pgfsys@transformshift{1.434178in}{1.202293in}%
\pgfsys@useobject{currentmarker}{}%
\end{pgfscope}%
\begin{pgfscope}%
\pgfsys@transformshift{1.439591in}{1.213066in}%
\pgfsys@useobject{currentmarker}{}%
\end{pgfscope}%
\begin{pgfscope}%
\pgfsys@transformshift{1.444877in}{1.227325in}%
\pgfsys@useobject{currentmarker}{}%
\end{pgfscope}%
\begin{pgfscope}%
\pgfsys@transformshift{1.450042in}{1.175888in}%
\pgfsys@useobject{currentmarker}{}%
\end{pgfscope}%
\begin{pgfscope}%
\pgfsys@transformshift{1.455090in}{1.095359in}%
\pgfsys@useobject{currentmarker}{}%
\end{pgfscope}%
\begin{pgfscope}%
\pgfsys@transformshift{1.460028in}{1.120846in}%
\pgfsys@useobject{currentmarker}{}%
\end{pgfscope}%
\begin{pgfscope}%
\pgfsys@transformshift{1.464859in}{1.142584in}%
\pgfsys@useobject{currentmarker}{}%
\end{pgfscope}%
\begin{pgfscope}%
\pgfsys@transformshift{1.469589in}{1.168559in}%
\pgfsys@useobject{currentmarker}{}%
\end{pgfscope}%
\begin{pgfscope}%
\pgfsys@transformshift{1.474221in}{1.154005in}%
\pgfsys@useobject{currentmarker}{}%
\end{pgfscope}%
\begin{pgfscope}%
\pgfsys@transformshift{1.478760in}{1.105622in}%
\pgfsys@useobject{currentmarker}{}%
\end{pgfscope}%
\begin{pgfscope}%
\pgfsys@transformshift{1.483209in}{1.045925in}%
\pgfsys@useobject{currentmarker}{}%
\end{pgfscope}%
\begin{pgfscope}%
\pgfsys@transformshift{1.487571in}{1.082888in}%
\pgfsys@useobject{currentmarker}{}%
\end{pgfscope}%
\begin{pgfscope}%
\pgfsys@transformshift{1.491850in}{1.148928in}%
\pgfsys@useobject{currentmarker}{}%
\end{pgfscope}%
\begin{pgfscope}%
\pgfsys@transformshift{1.496049in}{1.150320in}%
\pgfsys@useobject{currentmarker}{}%
\end{pgfscope}%
\begin{pgfscope}%
\pgfsys@transformshift{1.500172in}{1.159522in}%
\pgfsys@useobject{currentmarker}{}%
\end{pgfscope}%
\begin{pgfscope}%
\pgfsys@transformshift{1.504219in}{1.163355in}%
\pgfsys@useobject{currentmarker}{}%
\end{pgfscope}%
\begin{pgfscope}%
\pgfsys@transformshift{1.508196in}{1.173362in}%
\pgfsys@useobject{currentmarker}{}%
\end{pgfscope}%
\begin{pgfscope}%
\pgfsys@transformshift{1.512103in}{1.164352in}%
\pgfsys@useobject{currentmarker}{}%
\end{pgfscope}%
\begin{pgfscope}%
\pgfsys@transformshift{1.515943in}{1.141812in}%
\pgfsys@useobject{currentmarker}{}%
\end{pgfscope}%
\begin{pgfscope}%
\pgfsys@transformshift{1.519719in}{1.120098in}%
\pgfsys@useobject{currentmarker}{}%
\end{pgfscope}%
\begin{pgfscope}%
\pgfsys@transformshift{1.523432in}{1.112526in}%
\pgfsys@useobject{currentmarker}{}%
\end{pgfscope}%
\begin{pgfscope}%
\pgfsys@transformshift{1.527085in}{1.055989in}%
\pgfsys@useobject{currentmarker}{}%
\end{pgfscope}%
\begin{pgfscope}%
\pgfsys@transformshift{1.530680in}{1.062603in}%
\pgfsys@useobject{currentmarker}{}%
\end{pgfscope}%
\begin{pgfscope}%
\pgfsys@transformshift{1.534217in}{1.096690in}%
\pgfsys@useobject{currentmarker}{}%
\end{pgfscope}%
\begin{pgfscope}%
\pgfsys@transformshift{1.537700in}{1.068067in}%
\pgfsys@useobject{currentmarker}{}%
\end{pgfscope}%
\begin{pgfscope}%
\pgfsys@transformshift{1.541130in}{1.065379in}%
\pgfsys@useobject{currentmarker}{}%
\end{pgfscope}%
\begin{pgfscope}%
\pgfsys@transformshift{1.544509in}{1.095491in}%
\pgfsys@useobject{currentmarker}{}%
\end{pgfscope}%
\begin{pgfscope}%
\pgfsys@transformshift{1.547837in}{1.127129in}%
\pgfsys@useobject{currentmarker}{}%
\end{pgfscope}%
\begin{pgfscope}%
\pgfsys@transformshift{1.551117in}{1.117168in}%
\pgfsys@useobject{currentmarker}{}%
\end{pgfscope}%
\begin{pgfscope}%
\pgfsys@transformshift{1.554349in}{1.048005in}%
\pgfsys@useobject{currentmarker}{}%
\end{pgfscope}%
\begin{pgfscope}%
\pgfsys@transformshift{1.557536in}{1.069736in}%
\pgfsys@useobject{currentmarker}{}%
\end{pgfscope}%
\begin{pgfscope}%
\pgfsys@transformshift{1.560678in}{1.118483in}%
\pgfsys@useobject{currentmarker}{}%
\end{pgfscope}%
\begin{pgfscope}%
\pgfsys@transformshift{1.563776in}{1.120149in}%
\pgfsys@useobject{currentmarker}{}%
\end{pgfscope}%
\begin{pgfscope}%
\pgfsys@transformshift{1.566833in}{1.081543in}%
\pgfsys@useobject{currentmarker}{}%
\end{pgfscope}%
\begin{pgfscope}%
\pgfsys@transformshift{1.569848in}{1.084658in}%
\pgfsys@useobject{currentmarker}{}%
\end{pgfscope}%
\begin{pgfscope}%
\pgfsys@transformshift{1.572824in}{1.103899in}%
\pgfsys@useobject{currentmarker}{}%
\end{pgfscope}%
\begin{pgfscope}%
\pgfsys@transformshift{1.575761in}{1.081469in}%
\pgfsys@useobject{currentmarker}{}%
\end{pgfscope}%
\begin{pgfscope}%
\pgfsys@transformshift{1.578659in}{1.031169in}%
\pgfsys@useobject{currentmarker}{}%
\end{pgfscope}%
\begin{pgfscope}%
\pgfsys@transformshift{1.581521in}{1.037846in}%
\pgfsys@useobject{currentmarker}{}%
\end{pgfscope}%
\begin{pgfscope}%
\pgfsys@transformshift{1.584347in}{1.089595in}%
\pgfsys@useobject{currentmarker}{}%
\end{pgfscope}%
\begin{pgfscope}%
\pgfsys@transformshift{1.587138in}{1.077670in}%
\pgfsys@useobject{currentmarker}{}%
\end{pgfscope}%
\begin{pgfscope}%
\pgfsys@transformshift{1.589894in}{1.064079in}%
\pgfsys@useobject{currentmarker}{}%
\end{pgfscope}%
\begin{pgfscope}%
\pgfsys@transformshift{1.592617in}{1.043820in}%
\pgfsys@useobject{currentmarker}{}%
\end{pgfscope}%
\begin{pgfscope}%
\pgfsys@transformshift{1.595308in}{1.024211in}%
\pgfsys@useobject{currentmarker}{}%
\end{pgfscope}%
\begin{pgfscope}%
\pgfsys@transformshift{1.597966in}{1.021445in}%
\pgfsys@useobject{currentmarker}{}%
\end{pgfscope}%
\begin{pgfscope}%
\pgfsys@transformshift{1.600594in}{1.057507in}%
\pgfsys@useobject{currentmarker}{}%
\end{pgfscope}%
\begin{pgfscope}%
\pgfsys@transformshift{1.603191in}{1.020981in}%
\pgfsys@useobject{currentmarker}{}%
\end{pgfscope}%
\begin{pgfscope}%
\pgfsys@transformshift{1.605759in}{1.041287in}%
\pgfsys@useobject{currentmarker}{}%
\end{pgfscope}%
\begin{pgfscope}%
\pgfsys@transformshift{1.608297in}{0.994575in}%
\pgfsys@useobject{currentmarker}{}%
\end{pgfscope}%
\begin{pgfscope}%
\pgfsys@transformshift{1.610807in}{0.997938in}%
\pgfsys@useobject{currentmarker}{}%
\end{pgfscope}%
\begin{pgfscope}%
\pgfsys@transformshift{1.613290in}{1.066445in}%
\pgfsys@useobject{currentmarker}{}%
\end{pgfscope}%
\begin{pgfscope}%
\pgfsys@transformshift{1.615745in}{1.056958in}%
\pgfsys@useobject{currentmarker}{}%
\end{pgfscope}%
\begin{pgfscope}%
\pgfsys@transformshift{1.618173in}{1.063108in}%
\pgfsys@useobject{currentmarker}{}%
\end{pgfscope}%
\begin{pgfscope}%
\pgfsys@transformshift{1.620576in}{1.051093in}%
\pgfsys@useobject{currentmarker}{}%
\end{pgfscope}%
\begin{pgfscope}%
\pgfsys@transformshift{1.622953in}{1.030392in}%
\pgfsys@useobject{currentmarker}{}%
\end{pgfscope}%
\begin{pgfscope}%
\pgfsys@transformshift{1.625306in}{1.046458in}%
\pgfsys@useobject{currentmarker}{}%
\end{pgfscope}%
\begin{pgfscope}%
\pgfsys@transformshift{1.627634in}{1.031185in}%
\pgfsys@useobject{currentmarker}{}%
\end{pgfscope}%
\begin{pgfscope}%
\pgfsys@transformshift{1.629938in}{1.014596in}%
\pgfsys@useobject{currentmarker}{}%
\end{pgfscope}%
\begin{pgfscope}%
\pgfsys@transformshift{1.632219in}{1.033804in}%
\pgfsys@useobject{currentmarker}{}%
\end{pgfscope}%
\begin{pgfscope}%
\pgfsys@transformshift{1.634477in}{1.016011in}%
\pgfsys@useobject{currentmarker}{}%
\end{pgfscope}%
\begin{pgfscope}%
\pgfsys@transformshift{1.636712in}{0.987823in}%
\pgfsys@useobject{currentmarker}{}%
\end{pgfscope}%
\begin{pgfscope}%
\pgfsys@transformshift{1.638925in}{1.021058in}%
\pgfsys@useobject{currentmarker}{}%
\end{pgfscope}%
\begin{pgfscope}%
\pgfsys@transformshift{1.641117in}{1.027398in}%
\pgfsys@useobject{currentmarker}{}%
\end{pgfscope}%
\begin{pgfscope}%
\pgfsys@transformshift{1.643288in}{1.065967in}%
\pgfsys@useobject{currentmarker}{}%
\end{pgfscope}%
\begin{pgfscope}%
\pgfsys@transformshift{1.645437in}{1.029419in}%
\pgfsys@useobject{currentmarker}{}%
\end{pgfscope}%
\begin{pgfscope}%
\pgfsys@transformshift{1.647567in}{1.031442in}%
\pgfsys@useobject{currentmarker}{}%
\end{pgfscope}%
\begin{pgfscope}%
\pgfsys@transformshift{1.649676in}{1.042601in}%
\pgfsys@useobject{currentmarker}{}%
\end{pgfscope}%
\begin{pgfscope}%
\pgfsys@transformshift{1.651766in}{1.046952in}%
\pgfsys@useobject{currentmarker}{}%
\end{pgfscope}%
\begin{pgfscope}%
\pgfsys@transformshift{1.653837in}{1.062189in}%
\pgfsys@useobject{currentmarker}{}%
\end{pgfscope}%
\begin{pgfscope}%
\pgfsys@transformshift{1.655888in}{1.049449in}%
\pgfsys@useobject{currentmarker}{}%
\end{pgfscope}%
\begin{pgfscope}%
\pgfsys@transformshift{1.657921in}{1.013267in}%
\pgfsys@useobject{currentmarker}{}%
\end{pgfscope}%
\begin{pgfscope}%
\pgfsys@transformshift{1.659936in}{1.039645in}%
\pgfsys@useobject{currentmarker}{}%
\end{pgfscope}%
\begin{pgfscope}%
\pgfsys@transformshift{1.661933in}{1.020337in}%
\pgfsys@useobject{currentmarker}{}%
\end{pgfscope}%
\begin{pgfscope}%
\pgfsys@transformshift{1.663912in}{1.018009in}%
\pgfsys@useobject{currentmarker}{}%
\end{pgfscope}%
\begin{pgfscope}%
\pgfsys@transformshift{1.665874in}{1.015334in}%
\pgfsys@useobject{currentmarker}{}%
\end{pgfscope}%
\begin{pgfscope}%
\pgfsys@transformshift{1.667819in}{0.975119in}%
\pgfsys@useobject{currentmarker}{}%
\end{pgfscope}%
\begin{pgfscope}%
\pgfsys@transformshift{1.669748in}{0.996764in}%
\pgfsys@useobject{currentmarker}{}%
\end{pgfscope}%
\begin{pgfscope}%
\pgfsys@transformshift{1.671660in}{1.042020in}%
\pgfsys@useobject{currentmarker}{}%
\end{pgfscope}%
\begin{pgfscope}%
\pgfsys@transformshift{1.673556in}{1.029084in}%
\pgfsys@useobject{currentmarker}{}%
\end{pgfscope}%
\begin{pgfscope}%
\pgfsys@transformshift{1.675435in}{0.999451in}%
\pgfsys@useobject{currentmarker}{}%
\end{pgfscope}%
\begin{pgfscope}%
\pgfsys@transformshift{1.677300in}{1.000164in}%
\pgfsys@useobject{currentmarker}{}%
\end{pgfscope}%
\begin{pgfscope}%
\pgfsys@transformshift{1.679149in}{1.017880in}%
\pgfsys@useobject{currentmarker}{}%
\end{pgfscope}%
\begin{pgfscope}%
\pgfsys@transformshift{1.680983in}{0.984477in}%
\pgfsys@useobject{currentmarker}{}%
\end{pgfscope}%
\begin{pgfscope}%
\pgfsys@transformshift{1.682802in}{1.013991in}%
\pgfsys@useobject{currentmarker}{}%
\end{pgfscope}%
\begin{pgfscope}%
\pgfsys@transformshift{1.684606in}{1.005423in}%
\pgfsys@useobject{currentmarker}{}%
\end{pgfscope}%
\begin{pgfscope}%
\pgfsys@transformshift{1.686396in}{1.010247in}%
\pgfsys@useobject{currentmarker}{}%
\end{pgfscope}%
\begin{pgfscope}%
\pgfsys@transformshift{1.688172in}{0.993147in}%
\pgfsys@useobject{currentmarker}{}%
\end{pgfscope}%
\begin{pgfscope}%
\pgfsys@transformshift{1.689934in}{1.005775in}%
\pgfsys@useobject{currentmarker}{}%
\end{pgfscope}%
\begin{pgfscope}%
\pgfsys@transformshift{1.691682in}{1.003888in}%
\pgfsys@useobject{currentmarker}{}%
\end{pgfscope}%
\begin{pgfscope}%
\pgfsys@transformshift{1.693417in}{0.976093in}%
\pgfsys@useobject{currentmarker}{}%
\end{pgfscope}%
\begin{pgfscope}%
\pgfsys@transformshift{1.695139in}{1.014675in}%
\pgfsys@useobject{currentmarker}{}%
\end{pgfscope}%
\begin{pgfscope}%
\pgfsys@transformshift{1.696847in}{1.028608in}%
\pgfsys@useobject{currentmarker}{}%
\end{pgfscope}%
\begin{pgfscope}%
\pgfsys@transformshift{1.698543in}{1.000062in}%
\pgfsys@useobject{currentmarker}{}%
\end{pgfscope}%
\begin{pgfscope}%
\pgfsys@transformshift{1.700225in}{1.002268in}%
\pgfsys@useobject{currentmarker}{}%
\end{pgfscope}%
\begin{pgfscope}%
\pgfsys@transformshift{1.701896in}{1.033007in}%
\pgfsys@useobject{currentmarker}{}%
\end{pgfscope}%
\begin{pgfscope}%
\pgfsys@transformshift{1.703554in}{1.021678in}%
\pgfsys@useobject{currentmarker}{}%
\end{pgfscope}%
\begin{pgfscope}%
\pgfsys@transformshift{1.705199in}{1.058368in}%
\pgfsys@useobject{currentmarker}{}%
\end{pgfscope}%
\begin{pgfscope}%
\pgfsys@transformshift{1.706833in}{1.042920in}%
\pgfsys@useobject{currentmarker}{}%
\end{pgfscope}%
\begin{pgfscope}%
\pgfsys@transformshift{1.708455in}{1.010079in}%
\pgfsys@useobject{currentmarker}{}%
\end{pgfscope}%
\begin{pgfscope}%
\pgfsys@transformshift{1.710066in}{1.016787in}%
\pgfsys@useobject{currentmarker}{}%
\end{pgfscope}%
\begin{pgfscope}%
\pgfsys@transformshift{1.711665in}{0.982408in}%
\pgfsys@useobject{currentmarker}{}%
\end{pgfscope}%
\begin{pgfscope}%
\pgfsys@transformshift{1.713252in}{1.052763in}%
\pgfsys@useobject{currentmarker}{}%
\end{pgfscope}%
\begin{pgfscope}%
\pgfsys@transformshift{1.714829in}{1.004967in}%
\pgfsys@useobject{currentmarker}{}%
\end{pgfscope}%
\begin{pgfscope}%
\pgfsys@transformshift{1.716394in}{1.010402in}%
\pgfsys@useobject{currentmarker}{}%
\end{pgfscope}%
\begin{pgfscope}%
\pgfsys@transformshift{1.717949in}{0.941492in}%
\pgfsys@useobject{currentmarker}{}%
\end{pgfscope}%
\begin{pgfscope}%
\pgfsys@transformshift{1.719493in}{0.898942in}%
\pgfsys@useobject{currentmarker}{}%
\end{pgfscope}%
\begin{pgfscope}%
\pgfsys@transformshift{1.721026in}{0.958275in}%
\pgfsys@useobject{currentmarker}{}%
\end{pgfscope}%
\begin{pgfscope}%
\pgfsys@transformshift{1.722550in}{0.991722in}%
\pgfsys@useobject{currentmarker}{}%
\end{pgfscope}%
\begin{pgfscope}%
\pgfsys@transformshift{1.724062in}{1.004222in}%
\pgfsys@useobject{currentmarker}{}%
\end{pgfscope}%
\begin{pgfscope}%
\pgfsys@transformshift{1.725565in}{1.027335in}%
\pgfsys@useobject{currentmarker}{}%
\end{pgfscope}%
\begin{pgfscope}%
\pgfsys@transformshift{1.727058in}{0.990523in}%
\pgfsys@useobject{currentmarker}{}%
\end{pgfscope}%
\begin{pgfscope}%
\pgfsys@transformshift{1.728541in}{0.989322in}%
\pgfsys@useobject{currentmarker}{}%
\end{pgfscope}%
\begin{pgfscope}%
\pgfsys@transformshift{1.730014in}{1.001958in}%
\pgfsys@useobject{currentmarker}{}%
\end{pgfscope}%
\begin{pgfscope}%
\pgfsys@transformshift{1.731477in}{0.952554in}%
\pgfsys@useobject{currentmarker}{}%
\end{pgfscope}%
\begin{pgfscope}%
\pgfsys@transformshift{1.732931in}{0.975463in}%
\pgfsys@useobject{currentmarker}{}%
\end{pgfscope}%
\begin{pgfscope}%
\pgfsys@transformshift{1.734376in}{0.988815in}%
\pgfsys@useobject{currentmarker}{}%
\end{pgfscope}%
\begin{pgfscope}%
\pgfsys@transformshift{1.735812in}{1.019218in}%
\pgfsys@useobject{currentmarker}{}%
\end{pgfscope}%
\begin{pgfscope}%
\pgfsys@transformshift{1.737238in}{1.034054in}%
\pgfsys@useobject{currentmarker}{}%
\end{pgfscope}%
\begin{pgfscope}%
\pgfsys@transformshift{1.738655in}{1.013071in}%
\pgfsys@useobject{currentmarker}{}%
\end{pgfscope}%
\begin{pgfscope}%
\pgfsys@transformshift{1.740064in}{0.949076in}%
\pgfsys@useobject{currentmarker}{}%
\end{pgfscope}%
\begin{pgfscope}%
\pgfsys@transformshift{1.741463in}{0.878230in}%
\pgfsys@useobject{currentmarker}{}%
\end{pgfscope}%
\begin{pgfscope}%
\pgfsys@transformshift{1.742854in}{0.939312in}%
\pgfsys@useobject{currentmarker}{}%
\end{pgfscope}%
\begin{pgfscope}%
\pgfsys@transformshift{1.744237in}{0.970803in}%
\pgfsys@useobject{currentmarker}{}%
\end{pgfscope}%
\begin{pgfscope}%
\pgfsys@transformshift{1.745611in}{1.004233in}%
\pgfsys@useobject{currentmarker}{}%
\end{pgfscope}%
\begin{pgfscope}%
\pgfsys@transformshift{1.746977in}{1.022592in}%
\pgfsys@useobject{currentmarker}{}%
\end{pgfscope}%
\begin{pgfscope}%
\pgfsys@transformshift{1.748334in}{0.985141in}%
\pgfsys@useobject{currentmarker}{}%
\end{pgfscope}%
\begin{pgfscope}%
\pgfsys@transformshift{1.749683in}{0.932980in}%
\pgfsys@useobject{currentmarker}{}%
\end{pgfscope}%
\begin{pgfscope}%
\pgfsys@transformshift{1.751025in}{0.933641in}%
\pgfsys@useobject{currentmarker}{}%
\end{pgfscope}%
\begin{pgfscope}%
\pgfsys@transformshift{1.752358in}{0.896854in}%
\pgfsys@useobject{currentmarker}{}%
\end{pgfscope}%
\begin{pgfscope}%
\pgfsys@transformshift{1.753683in}{0.908551in}%
\pgfsys@useobject{currentmarker}{}%
\end{pgfscope}%
\begin{pgfscope}%
\pgfsys@transformshift{1.755001in}{0.918177in}%
\pgfsys@useobject{currentmarker}{}%
\end{pgfscope}%
\begin{pgfscope}%
\pgfsys@transformshift{1.756311in}{0.944832in}%
\pgfsys@useobject{currentmarker}{}%
\end{pgfscope}%
\begin{pgfscope}%
\pgfsys@transformshift{1.757613in}{0.924943in}%
\pgfsys@useobject{currentmarker}{}%
\end{pgfscope}%
\begin{pgfscope}%
\pgfsys@transformshift{1.758908in}{0.873133in}%
\pgfsys@useobject{currentmarker}{}%
\end{pgfscope}%
\begin{pgfscope}%
\pgfsys@transformshift{1.760195in}{0.854998in}%
\pgfsys@useobject{currentmarker}{}%
\end{pgfscope}%
\begin{pgfscope}%
\pgfsys@transformshift{1.761475in}{0.927751in}%
\pgfsys@useobject{currentmarker}{}%
\end{pgfscope}%
\begin{pgfscope}%
\pgfsys@transformshift{1.762748in}{0.981997in}%
\pgfsys@useobject{currentmarker}{}%
\end{pgfscope}%
\begin{pgfscope}%
\pgfsys@transformshift{1.764014in}{0.959718in}%
\pgfsys@useobject{currentmarker}{}%
\end{pgfscope}%
\begin{pgfscope}%
\pgfsys@transformshift{1.765272in}{0.951327in}%
\pgfsys@useobject{currentmarker}{}%
\end{pgfscope}%
\begin{pgfscope}%
\pgfsys@transformshift{1.766524in}{0.992409in}%
\pgfsys@useobject{currentmarker}{}%
\end{pgfscope}%
\begin{pgfscope}%
\pgfsys@transformshift{1.767769in}{0.998812in}%
\pgfsys@useobject{currentmarker}{}%
\end{pgfscope}%
\begin{pgfscope}%
\pgfsys@transformshift{1.769006in}{0.977150in}%
\pgfsys@useobject{currentmarker}{}%
\end{pgfscope}%
\begin{pgfscope}%
\pgfsys@transformshift{1.770237in}{0.951963in}%
\pgfsys@useobject{currentmarker}{}%
\end{pgfscope}%
\begin{pgfscope}%
\pgfsys@transformshift{1.771462in}{0.957431in}%
\pgfsys@useobject{currentmarker}{}%
\end{pgfscope}%
\begin{pgfscope}%
\pgfsys@transformshift{1.772679in}{0.988688in}%
\pgfsys@useobject{currentmarker}{}%
\end{pgfscope}%
\begin{pgfscope}%
\pgfsys@transformshift{1.773890in}{0.959751in}%
\pgfsys@useobject{currentmarker}{}%
\end{pgfscope}%
\begin{pgfscope}%
\pgfsys@transformshift{1.775095in}{0.953720in}%
\pgfsys@useobject{currentmarker}{}%
\end{pgfscope}%
\begin{pgfscope}%
\pgfsys@transformshift{1.776293in}{0.996163in}%
\pgfsys@useobject{currentmarker}{}%
\end{pgfscope}%
\begin{pgfscope}%
\pgfsys@transformshift{1.777485in}{0.978138in}%
\pgfsys@useobject{currentmarker}{}%
\end{pgfscope}%
\begin{pgfscope}%
\pgfsys@transformshift{1.778670in}{0.942278in}%
\pgfsys@useobject{currentmarker}{}%
\end{pgfscope}%
\begin{pgfscope}%
\pgfsys@transformshift{1.779849in}{0.998451in}%
\pgfsys@useobject{currentmarker}{}%
\end{pgfscope}%
\begin{pgfscope}%
\pgfsys@transformshift{1.781023in}{0.987817in}%
\pgfsys@useobject{currentmarker}{}%
\end{pgfscope}%
\begin{pgfscope}%
\pgfsys@transformshift{1.782190in}{0.941817in}%
\pgfsys@useobject{currentmarker}{}%
\end{pgfscope}%
\begin{pgfscope}%
\pgfsys@transformshift{1.783351in}{0.954538in}%
\pgfsys@useobject{currentmarker}{}%
\end{pgfscope}%
\begin{pgfscope}%
\pgfsys@transformshift{1.784506in}{0.981581in}%
\pgfsys@useobject{currentmarker}{}%
\end{pgfscope}%
\begin{pgfscope}%
\pgfsys@transformshift{1.785655in}{0.902075in}%
\pgfsys@useobject{currentmarker}{}%
\end{pgfscope}%
\begin{pgfscope}%
\pgfsys@transformshift{1.786798in}{0.890611in}%
\pgfsys@useobject{currentmarker}{}%
\end{pgfscope}%
\begin{pgfscope}%
\pgfsys@transformshift{1.787936in}{0.961528in}%
\pgfsys@useobject{currentmarker}{}%
\end{pgfscope}%
\begin{pgfscope}%
\pgfsys@transformshift{1.789067in}{0.989306in}%
\pgfsys@useobject{currentmarker}{}%
\end{pgfscope}%
\begin{pgfscope}%
\pgfsys@transformshift{1.790193in}{0.993728in}%
\pgfsys@useobject{currentmarker}{}%
\end{pgfscope}%
\begin{pgfscope}%
\pgfsys@transformshift{1.791314in}{1.019221in}%
\pgfsys@useobject{currentmarker}{}%
\end{pgfscope}%
\begin{pgfscope}%
\pgfsys@transformshift{1.792429in}{1.019675in}%
\pgfsys@useobject{currentmarker}{}%
\end{pgfscope}%
\begin{pgfscope}%
\pgfsys@transformshift{1.793538in}{0.968329in}%
\pgfsys@useobject{currentmarker}{}%
\end{pgfscope}%
\begin{pgfscope}%
\pgfsys@transformshift{1.794642in}{0.927693in}%
\pgfsys@useobject{currentmarker}{}%
\end{pgfscope}%
\begin{pgfscope}%
\pgfsys@transformshift{1.795741in}{0.886227in}%
\pgfsys@useobject{currentmarker}{}%
\end{pgfscope}%
\begin{pgfscope}%
\pgfsys@transformshift{1.796834in}{0.911023in}%
\pgfsys@useobject{currentmarker}{}%
\end{pgfscope}%
\begin{pgfscope}%
\pgfsys@transformshift{1.797922in}{0.964434in}%
\pgfsys@useobject{currentmarker}{}%
\end{pgfscope}%
\begin{pgfscope}%
\pgfsys@transformshift{1.799004in}{0.942858in}%
\pgfsys@useobject{currentmarker}{}%
\end{pgfscope}%
\begin{pgfscope}%
\pgfsys@transformshift{1.800082in}{0.935556in}%
\pgfsys@useobject{currentmarker}{}%
\end{pgfscope}%
\begin{pgfscope}%
\pgfsys@transformshift{1.801154in}{0.955890in}%
\pgfsys@useobject{currentmarker}{}%
\end{pgfscope}%
\begin{pgfscope}%
\pgfsys@transformshift{1.802221in}{0.903562in}%
\pgfsys@useobject{currentmarker}{}%
\end{pgfscope}%
\begin{pgfscope}%
\pgfsys@transformshift{1.803284in}{0.918587in}%
\pgfsys@useobject{currentmarker}{}%
\end{pgfscope}%
\begin{pgfscope}%
\pgfsys@transformshift{1.804341in}{0.910437in}%
\pgfsys@useobject{currentmarker}{}%
\end{pgfscope}%
\begin{pgfscope}%
\pgfsys@transformshift{1.805393in}{0.914167in}%
\pgfsys@useobject{currentmarker}{}%
\end{pgfscope}%
\begin{pgfscope}%
\pgfsys@transformshift{1.806440in}{0.928218in}%
\pgfsys@useobject{currentmarker}{}%
\end{pgfscope}%
\begin{pgfscope}%
\pgfsys@transformshift{1.807483in}{0.928742in}%
\pgfsys@useobject{currentmarker}{}%
\end{pgfscope}%
\begin{pgfscope}%
\pgfsys@transformshift{1.808520in}{0.940043in}%
\pgfsys@useobject{currentmarker}{}%
\end{pgfscope}%
\begin{pgfscope}%
\pgfsys@transformshift{1.809553in}{0.965731in}%
\pgfsys@useobject{currentmarker}{}%
\end{pgfscope}%
\begin{pgfscope}%
\pgfsys@transformshift{1.810581in}{0.934750in}%
\pgfsys@useobject{currentmarker}{}%
\end{pgfscope}%
\begin{pgfscope}%
\pgfsys@transformshift{1.811605in}{0.932998in}%
\pgfsys@useobject{currentmarker}{}%
\end{pgfscope}%
\begin{pgfscope}%
\pgfsys@transformshift{1.812624in}{0.903523in}%
\pgfsys@useobject{currentmarker}{}%
\end{pgfscope}%
\begin{pgfscope}%
\pgfsys@transformshift{1.813638in}{0.894253in}%
\pgfsys@useobject{currentmarker}{}%
\end{pgfscope}%
\begin{pgfscope}%
\pgfsys@transformshift{1.814648in}{0.919857in}%
\pgfsys@useobject{currentmarker}{}%
\end{pgfscope}%
\begin{pgfscope}%
\pgfsys@transformshift{1.815653in}{0.943210in}%
\pgfsys@useobject{currentmarker}{}%
\end{pgfscope}%
\begin{pgfscope}%
\pgfsys@transformshift{1.816653in}{0.938640in}%
\pgfsys@useobject{currentmarker}{}%
\end{pgfscope}%
\begin{pgfscope}%
\pgfsys@transformshift{1.817650in}{0.922136in}%
\pgfsys@useobject{currentmarker}{}%
\end{pgfscope}%
\begin{pgfscope}%
\pgfsys@transformshift{1.818642in}{0.885553in}%
\pgfsys@useobject{currentmarker}{}%
\end{pgfscope}%
\begin{pgfscope}%
\pgfsys@transformshift{1.819629in}{0.882626in}%
\pgfsys@useobject{currentmarker}{}%
\end{pgfscope}%
\begin{pgfscope}%
\pgfsys@transformshift{1.820612in}{0.923199in}%
\pgfsys@useobject{currentmarker}{}%
\end{pgfscope}%
\begin{pgfscope}%
\pgfsys@transformshift{1.821591in}{0.871260in}%
\pgfsys@useobject{currentmarker}{}%
\end{pgfscope}%
\begin{pgfscope}%
\pgfsys@transformshift{1.822566in}{0.835198in}%
\pgfsys@useobject{currentmarker}{}%
\end{pgfscope}%
\begin{pgfscope}%
\pgfsys@transformshift{1.823536in}{0.896678in}%
\pgfsys@useobject{currentmarker}{}%
\end{pgfscope}%
\begin{pgfscope}%
\pgfsys@transformshift{1.824502in}{0.917862in}%
\pgfsys@useobject{currentmarker}{}%
\end{pgfscope}%
\begin{pgfscope}%
\pgfsys@transformshift{1.825464in}{0.900156in}%
\pgfsys@useobject{currentmarker}{}%
\end{pgfscope}%
\begin{pgfscope}%
\pgfsys@transformshift{1.826423in}{0.954410in}%
\pgfsys@useobject{currentmarker}{}%
\end{pgfscope}%
\begin{pgfscope}%
\pgfsys@transformshift{1.827376in}{0.967181in}%
\pgfsys@useobject{currentmarker}{}%
\end{pgfscope}%
\begin{pgfscope}%
\pgfsys@transformshift{1.828326in}{0.915909in}%
\pgfsys@useobject{currentmarker}{}%
\end{pgfscope}%
\begin{pgfscope}%
\pgfsys@transformshift{1.829272in}{0.922509in}%
\pgfsys@useobject{currentmarker}{}%
\end{pgfscope}%
\begin{pgfscope}%
\pgfsys@transformshift{1.830214in}{0.911325in}%
\pgfsys@useobject{currentmarker}{}%
\end{pgfscope}%
\begin{pgfscope}%
\pgfsys@transformshift{1.831152in}{0.930041in}%
\pgfsys@useobject{currentmarker}{}%
\end{pgfscope}%
\begin{pgfscope}%
\pgfsys@transformshift{1.832086in}{0.940203in}%
\pgfsys@useobject{currentmarker}{}%
\end{pgfscope}%
\begin{pgfscope}%
\pgfsys@transformshift{1.833017in}{0.926372in}%
\pgfsys@useobject{currentmarker}{}%
\end{pgfscope}%
\begin{pgfscope}%
\pgfsys@transformshift{1.833943in}{0.866135in}%
\pgfsys@useobject{currentmarker}{}%
\end{pgfscope}%
\begin{pgfscope}%
\pgfsys@transformshift{1.834866in}{0.878424in}%
\pgfsys@useobject{currentmarker}{}%
\end{pgfscope}%
\begin{pgfscope}%
\pgfsys@transformshift{1.835784in}{0.889490in}%
\pgfsys@useobject{currentmarker}{}%
\end{pgfscope}%
\begin{pgfscope}%
\pgfsys@transformshift{1.836699in}{0.877095in}%
\pgfsys@useobject{currentmarker}{}%
\end{pgfscope}%
\begin{pgfscope}%
\pgfsys@transformshift{1.837611in}{0.892343in}%
\pgfsys@useobject{currentmarker}{}%
\end{pgfscope}%
\begin{pgfscope}%
\pgfsys@transformshift{1.838518in}{0.940050in}%
\pgfsys@useobject{currentmarker}{}%
\end{pgfscope}%
\begin{pgfscope}%
\pgfsys@transformshift{1.839423in}{0.948179in}%
\pgfsys@useobject{currentmarker}{}%
\end{pgfscope}%
\begin{pgfscope}%
\pgfsys@transformshift{1.840323in}{0.882358in}%
\pgfsys@useobject{currentmarker}{}%
\end{pgfscope}%
\begin{pgfscope}%
\pgfsys@transformshift{1.841220in}{0.892817in}%
\pgfsys@useobject{currentmarker}{}%
\end{pgfscope}%
\begin{pgfscope}%
\pgfsys@transformshift{1.842113in}{0.925810in}%
\pgfsys@useobject{currentmarker}{}%
\end{pgfscope}%
\begin{pgfscope}%
\pgfsys@transformshift{1.843003in}{0.938740in}%
\pgfsys@useobject{currentmarker}{}%
\end{pgfscope}%
\begin{pgfscope}%
\pgfsys@transformshift{1.843889in}{0.929383in}%
\pgfsys@useobject{currentmarker}{}%
\end{pgfscope}%
\begin{pgfscope}%
\pgfsys@transformshift{1.844772in}{0.946552in}%
\pgfsys@useobject{currentmarker}{}%
\end{pgfscope}%
\begin{pgfscope}%
\pgfsys@transformshift{1.845651in}{0.954592in}%
\pgfsys@useobject{currentmarker}{}%
\end{pgfscope}%
\begin{pgfscope}%
\pgfsys@transformshift{1.846527in}{0.925979in}%
\pgfsys@useobject{currentmarker}{}%
\end{pgfscope}%
\begin{pgfscope}%
\pgfsys@transformshift{1.847399in}{0.918996in}%
\pgfsys@useobject{currentmarker}{}%
\end{pgfscope}%
\begin{pgfscope}%
\pgfsys@transformshift{1.848268in}{0.936402in}%
\pgfsys@useobject{currentmarker}{}%
\end{pgfscope}%
\begin{pgfscope}%
\pgfsys@transformshift{1.849134in}{0.893017in}%
\pgfsys@useobject{currentmarker}{}%
\end{pgfscope}%
\begin{pgfscope}%
\pgfsys@transformshift{1.849996in}{0.925095in}%
\pgfsys@useobject{currentmarker}{}%
\end{pgfscope}%
\begin{pgfscope}%
\pgfsys@transformshift{1.850855in}{0.953066in}%
\pgfsys@useobject{currentmarker}{}%
\end{pgfscope}%
\begin{pgfscope}%
\pgfsys@transformshift{1.851711in}{0.909335in}%
\pgfsys@useobject{currentmarker}{}%
\end{pgfscope}%
\begin{pgfscope}%
\pgfsys@transformshift{1.852564in}{0.909027in}%
\pgfsys@useobject{currentmarker}{}%
\end{pgfscope}%
\begin{pgfscope}%
\pgfsys@transformshift{1.853413in}{0.906683in}%
\pgfsys@useobject{currentmarker}{}%
\end{pgfscope}%
\begin{pgfscope}%
\pgfsys@transformshift{1.854259in}{0.882094in}%
\pgfsys@useobject{currentmarker}{}%
\end{pgfscope}%
\begin{pgfscope}%
\pgfsys@transformshift{1.855102in}{0.845605in}%
\pgfsys@useobject{currentmarker}{}%
\end{pgfscope}%
\begin{pgfscope}%
\pgfsys@transformshift{1.855942in}{0.822602in}%
\pgfsys@useobject{currentmarker}{}%
\end{pgfscope}%
\begin{pgfscope}%
\pgfsys@transformshift{1.856779in}{0.857045in}%
\pgfsys@useobject{currentmarker}{}%
\end{pgfscope}%
\begin{pgfscope}%
\pgfsys@transformshift{1.857612in}{0.882969in}%
\pgfsys@useobject{currentmarker}{}%
\end{pgfscope}%
\begin{pgfscope}%
\pgfsys@transformshift{1.858443in}{0.874862in}%
\pgfsys@useobject{currentmarker}{}%
\end{pgfscope}%
\begin{pgfscope}%
\pgfsys@transformshift{1.859270in}{0.892196in}%
\pgfsys@useobject{currentmarker}{}%
\end{pgfscope}%
\begin{pgfscope}%
\pgfsys@transformshift{1.860095in}{0.927409in}%
\pgfsys@useobject{currentmarker}{}%
\end{pgfscope}%
\begin{pgfscope}%
\pgfsys@transformshift{1.860916in}{0.862143in}%
\pgfsys@useobject{currentmarker}{}%
\end{pgfscope}%
\begin{pgfscope}%
\pgfsys@transformshift{1.861735in}{0.856254in}%
\pgfsys@useobject{currentmarker}{}%
\end{pgfscope}%
\begin{pgfscope}%
\pgfsys@transformshift{1.862550in}{0.867133in}%
\pgfsys@useobject{currentmarker}{}%
\end{pgfscope}%
\begin{pgfscope}%
\pgfsys@transformshift{1.863362in}{0.908099in}%
\pgfsys@useobject{currentmarker}{}%
\end{pgfscope}%
\begin{pgfscope}%
\pgfsys@transformshift{1.864172in}{0.897139in}%
\pgfsys@useobject{currentmarker}{}%
\end{pgfscope}%
\begin{pgfscope}%
\pgfsys@transformshift{1.864979in}{0.897752in}%
\pgfsys@useobject{currentmarker}{}%
\end{pgfscope}%
\begin{pgfscope}%
\pgfsys@transformshift{1.865782in}{0.892636in}%
\pgfsys@useobject{currentmarker}{}%
\end{pgfscope}%
\begin{pgfscope}%
\pgfsys@transformshift{1.866583in}{0.866430in}%
\pgfsys@useobject{currentmarker}{}%
\end{pgfscope}%
\begin{pgfscope}%
\pgfsys@transformshift{1.867381in}{0.902646in}%
\pgfsys@useobject{currentmarker}{}%
\end{pgfscope}%
\begin{pgfscope}%
\pgfsys@transformshift{1.868177in}{0.896592in}%
\pgfsys@useobject{currentmarker}{}%
\end{pgfscope}%
\begin{pgfscope}%
\pgfsys@transformshift{1.868969in}{0.931603in}%
\pgfsys@useobject{currentmarker}{}%
\end{pgfscope}%
\begin{pgfscope}%
\pgfsys@transformshift{1.869759in}{0.905184in}%
\pgfsys@useobject{currentmarker}{}%
\end{pgfscope}%
\begin{pgfscope}%
\pgfsys@transformshift{1.870546in}{0.815348in}%
\pgfsys@useobject{currentmarker}{}%
\end{pgfscope}%
\begin{pgfscope}%
\pgfsys@transformshift{1.871330in}{0.877727in}%
\pgfsys@useobject{currentmarker}{}%
\end{pgfscope}%
\begin{pgfscope}%
\pgfsys@transformshift{1.872111in}{0.900657in}%
\pgfsys@useobject{currentmarker}{}%
\end{pgfscope}%
\begin{pgfscope}%
\pgfsys@transformshift{1.872890in}{0.860195in}%
\pgfsys@useobject{currentmarker}{}%
\end{pgfscope}%
\begin{pgfscope}%
\pgfsys@transformshift{1.873666in}{0.841754in}%
\pgfsys@useobject{currentmarker}{}%
\end{pgfscope}%
\begin{pgfscope}%
\pgfsys@transformshift{1.874439in}{0.852890in}%
\pgfsys@useobject{currentmarker}{}%
\end{pgfscope}%
\begin{pgfscope}%
\pgfsys@transformshift{1.875210in}{0.907077in}%
\pgfsys@useobject{currentmarker}{}%
\end{pgfscope}%
\begin{pgfscope}%
\pgfsys@transformshift{1.875978in}{0.925623in}%
\pgfsys@useobject{currentmarker}{}%
\end{pgfscope}%
\begin{pgfscope}%
\pgfsys@transformshift{1.876743in}{0.919749in}%
\pgfsys@useobject{currentmarker}{}%
\end{pgfscope}%
\begin{pgfscope}%
\pgfsys@transformshift{1.877506in}{0.856119in}%
\pgfsys@useobject{currentmarker}{}%
\end{pgfscope}%
\begin{pgfscope}%
\pgfsys@transformshift{1.878266in}{0.852607in}%
\pgfsys@useobject{currentmarker}{}%
\end{pgfscope}%
\begin{pgfscope}%
\pgfsys@transformshift{1.879024in}{0.861972in}%
\pgfsys@useobject{currentmarker}{}%
\end{pgfscope}%
\begin{pgfscope}%
\pgfsys@transformshift{1.879779in}{0.895745in}%
\pgfsys@useobject{currentmarker}{}%
\end{pgfscope}%
\begin{pgfscope}%
\pgfsys@transformshift{1.880532in}{0.895235in}%
\pgfsys@useobject{currentmarker}{}%
\end{pgfscope}%
\begin{pgfscope}%
\pgfsys@transformshift{1.881282in}{0.862184in}%
\pgfsys@useobject{currentmarker}{}%
\end{pgfscope}%
\begin{pgfscope}%
\pgfsys@transformshift{1.882029in}{0.898176in}%
\pgfsys@useobject{currentmarker}{}%
\end{pgfscope}%
\begin{pgfscope}%
\pgfsys@transformshift{1.882774in}{0.894241in}%
\pgfsys@useobject{currentmarker}{}%
\end{pgfscope}%
\begin{pgfscope}%
\pgfsys@transformshift{1.883517in}{0.862124in}%
\pgfsys@useobject{currentmarker}{}%
\end{pgfscope}%
\begin{pgfscope}%
\pgfsys@transformshift{1.884257in}{0.913113in}%
\pgfsys@useobject{currentmarker}{}%
\end{pgfscope}%
\begin{pgfscope}%
\pgfsys@transformshift{1.884995in}{0.911665in}%
\pgfsys@useobject{currentmarker}{}%
\end{pgfscope}%
\begin{pgfscope}%
\pgfsys@transformshift{1.885730in}{0.868371in}%
\pgfsys@useobject{currentmarker}{}%
\end{pgfscope}%
\begin{pgfscope}%
\pgfsys@transformshift{1.886463in}{0.851790in}%
\pgfsys@useobject{currentmarker}{}%
\end{pgfscope}%
\begin{pgfscope}%
\pgfsys@transformshift{1.887194in}{0.830603in}%
\pgfsys@useobject{currentmarker}{}%
\end{pgfscope}%
\begin{pgfscope}%
\pgfsys@transformshift{1.887922in}{0.917230in}%
\pgfsys@useobject{currentmarker}{}%
\end{pgfscope}%
\begin{pgfscope}%
\pgfsys@transformshift{1.888648in}{0.929372in}%
\pgfsys@useobject{currentmarker}{}%
\end{pgfscope}%
\begin{pgfscope}%
\pgfsys@transformshift{1.889372in}{0.884223in}%
\pgfsys@useobject{currentmarker}{}%
\end{pgfscope}%
\begin{pgfscope}%
\pgfsys@transformshift{1.890093in}{0.866289in}%
\pgfsys@useobject{currentmarker}{}%
\end{pgfscope}%
\begin{pgfscope}%
\pgfsys@transformshift{1.890812in}{0.864875in}%
\pgfsys@useobject{currentmarker}{}%
\end{pgfscope}%
\begin{pgfscope}%
\pgfsys@transformshift{1.891528in}{0.920091in}%
\pgfsys@useobject{currentmarker}{}%
\end{pgfscope}%
\begin{pgfscope}%
\pgfsys@transformshift{1.892243in}{0.897719in}%
\pgfsys@useobject{currentmarker}{}%
\end{pgfscope}%
\begin{pgfscope}%
\pgfsys@transformshift{1.892955in}{0.860372in}%
\pgfsys@useobject{currentmarker}{}%
\end{pgfscope}%
\begin{pgfscope}%
\pgfsys@transformshift{1.893664in}{0.914282in}%
\pgfsys@useobject{currentmarker}{}%
\end{pgfscope}%
\begin{pgfscope}%
\pgfsys@transformshift{1.894372in}{0.909851in}%
\pgfsys@useobject{currentmarker}{}%
\end{pgfscope}%
\begin{pgfscope}%
\pgfsys@transformshift{1.895077in}{0.897261in}%
\pgfsys@useobject{currentmarker}{}%
\end{pgfscope}%
\begin{pgfscope}%
\pgfsys@transformshift{1.895780in}{0.820830in}%
\pgfsys@useobject{currentmarker}{}%
\end{pgfscope}%
\begin{pgfscope}%
\pgfsys@transformshift{1.896481in}{0.860261in}%
\pgfsys@useobject{currentmarker}{}%
\end{pgfscope}%
\begin{pgfscope}%
\pgfsys@transformshift{1.897180in}{0.869633in}%
\pgfsys@useobject{currentmarker}{}%
\end{pgfscope}%
\begin{pgfscope}%
\pgfsys@transformshift{1.897877in}{0.876122in}%
\pgfsys@useobject{currentmarker}{}%
\end{pgfscope}%
\begin{pgfscope}%
\pgfsys@transformshift{1.898571in}{0.870369in}%
\pgfsys@useobject{currentmarker}{}%
\end{pgfscope}%
\begin{pgfscope}%
\pgfsys@transformshift{1.899264in}{0.869295in}%
\pgfsys@useobject{currentmarker}{}%
\end{pgfscope}%
\begin{pgfscope}%
\pgfsys@transformshift{1.899954in}{0.919133in}%
\pgfsys@useobject{currentmarker}{}%
\end{pgfscope}%
\begin{pgfscope}%
\pgfsys@transformshift{1.900642in}{0.931295in}%
\pgfsys@useobject{currentmarker}{}%
\end{pgfscope}%
\begin{pgfscope}%
\pgfsys@transformshift{1.901328in}{0.881504in}%
\pgfsys@useobject{currentmarker}{}%
\end{pgfscope}%
\begin{pgfscope}%
\pgfsys@transformshift{1.902012in}{0.857289in}%
\pgfsys@useobject{currentmarker}{}%
\end{pgfscope}%
\begin{pgfscope}%
\pgfsys@transformshift{1.902693in}{0.882424in}%
\pgfsys@useobject{currentmarker}{}%
\end{pgfscope}%
\begin{pgfscope}%
\pgfsys@transformshift{1.903373in}{0.962077in}%
\pgfsys@useobject{currentmarker}{}%
\end{pgfscope}%
\begin{pgfscope}%
\pgfsys@transformshift{1.904051in}{0.951743in}%
\pgfsys@useobject{currentmarker}{}%
\end{pgfscope}%
\begin{pgfscope}%
\pgfsys@transformshift{1.904726in}{0.913622in}%
\pgfsys@useobject{currentmarker}{}%
\end{pgfscope}%
\begin{pgfscope}%
\pgfsys@transformshift{1.905400in}{0.915250in}%
\pgfsys@useobject{currentmarker}{}%
\end{pgfscope}%
\begin{pgfscope}%
\pgfsys@transformshift{1.906072in}{0.866016in}%
\pgfsys@useobject{currentmarker}{}%
\end{pgfscope}%
\begin{pgfscope}%
\pgfsys@transformshift{1.906741in}{0.848849in}%
\pgfsys@useobject{currentmarker}{}%
\end{pgfscope}%
\begin{pgfscope}%
\pgfsys@transformshift{1.907409in}{0.851163in}%
\pgfsys@useobject{currentmarker}{}%
\end{pgfscope}%
\begin{pgfscope}%
\pgfsys@transformshift{1.908075in}{0.828793in}%
\pgfsys@useobject{currentmarker}{}%
\end{pgfscope}%
\begin{pgfscope}%
\pgfsys@transformshift{1.908738in}{0.822618in}%
\pgfsys@useobject{currentmarker}{}%
\end{pgfscope}%
\begin{pgfscope}%
\pgfsys@transformshift{1.909400in}{0.834806in}%
\pgfsys@useobject{currentmarker}{}%
\end{pgfscope}%
\begin{pgfscope}%
\pgfsys@transformshift{1.910060in}{0.827132in}%
\pgfsys@useobject{currentmarker}{}%
\end{pgfscope}%
\begin{pgfscope}%
\pgfsys@transformshift{1.910717in}{0.829656in}%
\pgfsys@useobject{currentmarker}{}%
\end{pgfscope}%
\begin{pgfscope}%
\pgfsys@transformshift{1.911373in}{0.883877in}%
\pgfsys@useobject{currentmarker}{}%
\end{pgfscope}%
\begin{pgfscope}%
\pgfsys@transformshift{1.912027in}{0.878445in}%
\pgfsys@useobject{currentmarker}{}%
\end{pgfscope}%
\begin{pgfscope}%
\pgfsys@transformshift{1.912680in}{0.913145in}%
\pgfsys@useobject{currentmarker}{}%
\end{pgfscope}%
\begin{pgfscope}%
\pgfsys@transformshift{1.913330in}{0.923664in}%
\pgfsys@useobject{currentmarker}{}%
\end{pgfscope}%
\begin{pgfscope}%
\pgfsys@transformshift{1.913978in}{0.832269in}%
\pgfsys@useobject{currentmarker}{}%
\end{pgfscope}%
\begin{pgfscope}%
\pgfsys@transformshift{1.914625in}{0.881998in}%
\pgfsys@useobject{currentmarker}{}%
\end{pgfscope}%
\begin{pgfscope}%
\pgfsys@transformshift{1.915269in}{0.870560in}%
\pgfsys@useobject{currentmarker}{}%
\end{pgfscope}%
\begin{pgfscope}%
\pgfsys@transformshift{1.915912in}{0.852447in}%
\pgfsys@useobject{currentmarker}{}%
\end{pgfscope}%
\begin{pgfscope}%
\pgfsys@transformshift{1.916553in}{0.831464in}%
\pgfsys@useobject{currentmarker}{}%
\end{pgfscope}%
\begin{pgfscope}%
\pgfsys@transformshift{1.917192in}{0.760373in}%
\pgfsys@useobject{currentmarker}{}%
\end{pgfscope}%
\begin{pgfscope}%
\pgfsys@transformshift{1.917829in}{0.852771in}%
\pgfsys@useobject{currentmarker}{}%
\end{pgfscope}%
\begin{pgfscope}%
\pgfsys@transformshift{1.918465in}{0.883449in}%
\pgfsys@useobject{currentmarker}{}%
\end{pgfscope}%
\begin{pgfscope}%
\pgfsys@transformshift{1.919099in}{0.855862in}%
\pgfsys@useobject{currentmarker}{}%
\end{pgfscope}%
\begin{pgfscope}%
\pgfsys@transformshift{1.919731in}{0.854415in}%
\pgfsys@useobject{currentmarker}{}%
\end{pgfscope}%
\begin{pgfscope}%
\pgfsys@transformshift{1.920361in}{0.855146in}%
\pgfsys@useobject{currentmarker}{}%
\end{pgfscope}%
\begin{pgfscope}%
\pgfsys@transformshift{1.920989in}{0.845043in}%
\pgfsys@useobject{currentmarker}{}%
\end{pgfscope}%
\begin{pgfscope}%
\pgfsys@transformshift{1.921616in}{0.797923in}%
\pgfsys@useobject{currentmarker}{}%
\end{pgfscope}%
\begin{pgfscope}%
\pgfsys@transformshift{1.922241in}{0.851903in}%
\pgfsys@useobject{currentmarker}{}%
\end{pgfscope}%
\begin{pgfscope}%
\pgfsys@transformshift{1.922864in}{0.799815in}%
\pgfsys@useobject{currentmarker}{}%
\end{pgfscope}%
\begin{pgfscope}%
\pgfsys@transformshift{1.923485in}{0.862288in}%
\pgfsys@useobject{currentmarker}{}%
\end{pgfscope}%
\begin{pgfscope}%
\pgfsys@transformshift{1.924105in}{0.893396in}%
\pgfsys@useobject{currentmarker}{}%
\end{pgfscope}%
\begin{pgfscope}%
\pgfsys@transformshift{1.924723in}{0.863905in}%
\pgfsys@useobject{currentmarker}{}%
\end{pgfscope}%
\begin{pgfscope}%
\pgfsys@transformshift{1.925339in}{0.849477in}%
\pgfsys@useobject{currentmarker}{}%
\end{pgfscope}%
\begin{pgfscope}%
\pgfsys@transformshift{1.925954in}{0.811275in}%
\pgfsys@useobject{currentmarker}{}%
\end{pgfscope}%
\begin{pgfscope}%
\pgfsys@transformshift{1.926567in}{0.846296in}%
\pgfsys@useobject{currentmarker}{}%
\end{pgfscope}%
\begin{pgfscope}%
\pgfsys@transformshift{1.927178in}{0.817540in}%
\pgfsys@useobject{currentmarker}{}%
\end{pgfscope}%
\begin{pgfscope}%
\pgfsys@transformshift{1.927788in}{0.799489in}%
\pgfsys@useobject{currentmarker}{}%
\end{pgfscope}%
\begin{pgfscope}%
\pgfsys@transformshift{1.928396in}{0.781782in}%
\pgfsys@useobject{currentmarker}{}%
\end{pgfscope}%
\begin{pgfscope}%
\pgfsys@transformshift{1.929002in}{0.820781in}%
\pgfsys@useobject{currentmarker}{}%
\end{pgfscope}%
\begin{pgfscope}%
\pgfsys@transformshift{1.929607in}{0.871187in}%
\pgfsys@useobject{currentmarker}{}%
\end{pgfscope}%
\begin{pgfscope}%
\pgfsys@transformshift{1.930210in}{0.849613in}%
\pgfsys@useobject{currentmarker}{}%
\end{pgfscope}%
\begin{pgfscope}%
\pgfsys@transformshift{1.930811in}{0.817072in}%
\pgfsys@useobject{currentmarker}{}%
\end{pgfscope}%
\begin{pgfscope}%
\pgfsys@transformshift{1.931411in}{0.837572in}%
\pgfsys@useobject{currentmarker}{}%
\end{pgfscope}%
\begin{pgfscope}%
\pgfsys@transformshift{1.932010in}{0.834399in}%
\pgfsys@useobject{currentmarker}{}%
\end{pgfscope}%
\begin{pgfscope}%
\pgfsys@transformshift{1.932606in}{0.865978in}%
\pgfsys@useobject{currentmarker}{}%
\end{pgfscope}%
\begin{pgfscope}%
\pgfsys@transformshift{1.933201in}{0.844577in}%
\pgfsys@useobject{currentmarker}{}%
\end{pgfscope}%
\begin{pgfscope}%
\pgfsys@transformshift{1.933795in}{0.832464in}%
\pgfsys@useobject{currentmarker}{}%
\end{pgfscope}%
\begin{pgfscope}%
\pgfsys@transformshift{1.934387in}{0.848728in}%
\pgfsys@useobject{currentmarker}{}%
\end{pgfscope}%
\begin{pgfscope}%
\pgfsys@transformshift{1.934977in}{0.825778in}%
\pgfsys@useobject{currentmarker}{}%
\end{pgfscope}%
\begin{pgfscope}%
\pgfsys@transformshift{1.935566in}{0.754126in}%
\pgfsys@useobject{currentmarker}{}%
\end{pgfscope}%
\begin{pgfscope}%
\pgfsys@transformshift{1.936154in}{0.841103in}%
\pgfsys@useobject{currentmarker}{}%
\end{pgfscope}%
\begin{pgfscope}%
\pgfsys@transformshift{1.936739in}{0.841987in}%
\pgfsys@useobject{currentmarker}{}%
\end{pgfscope}%
\begin{pgfscope}%
\pgfsys@transformshift{1.937324in}{0.836999in}%
\pgfsys@useobject{currentmarker}{}%
\end{pgfscope}%
\begin{pgfscope}%
\pgfsys@transformshift{1.937906in}{0.875378in}%
\pgfsys@useobject{currentmarker}{}%
\end{pgfscope}%
\begin{pgfscope}%
\pgfsys@transformshift{1.938488in}{0.881792in}%
\pgfsys@useobject{currentmarker}{}%
\end{pgfscope}%
\begin{pgfscope}%
\pgfsys@transformshift{1.939067in}{0.889746in}%
\pgfsys@useobject{currentmarker}{}%
\end{pgfscope}%
\begin{pgfscope}%
\pgfsys@transformshift{1.939646in}{0.810120in}%
\pgfsys@useobject{currentmarker}{}%
\end{pgfscope}%
\begin{pgfscope}%
\pgfsys@transformshift{1.940222in}{0.817668in}%
\pgfsys@useobject{currentmarker}{}%
\end{pgfscope}%
\begin{pgfscope}%
\pgfsys@transformshift{1.940798in}{0.804678in}%
\pgfsys@useobject{currentmarker}{}%
\end{pgfscope}%
\begin{pgfscope}%
\pgfsys@transformshift{1.941371in}{0.831522in}%
\pgfsys@useobject{currentmarker}{}%
\end{pgfscope}%
\begin{pgfscope}%
\pgfsys@transformshift{1.941944in}{0.847745in}%
\pgfsys@useobject{currentmarker}{}%
\end{pgfscope}%
\begin{pgfscope}%
\pgfsys@transformshift{1.942515in}{0.852349in}%
\pgfsys@useobject{currentmarker}{}%
\end{pgfscope}%
\begin{pgfscope}%
\pgfsys@transformshift{1.943084in}{0.863436in}%
\pgfsys@useobject{currentmarker}{}%
\end{pgfscope}%
\begin{pgfscope}%
\pgfsys@transformshift{1.943652in}{0.855830in}%
\pgfsys@useobject{currentmarker}{}%
\end{pgfscope}%
\begin{pgfscope}%
\pgfsys@transformshift{1.944219in}{0.852371in}%
\pgfsys@useobject{currentmarker}{}%
\end{pgfscope}%
\begin{pgfscope}%
\pgfsys@transformshift{1.944784in}{0.848238in}%
\pgfsys@useobject{currentmarker}{}%
\end{pgfscope}%
\begin{pgfscope}%
\pgfsys@transformshift{1.945348in}{0.858159in}%
\pgfsys@useobject{currentmarker}{}%
\end{pgfscope}%
\begin{pgfscope}%
\pgfsys@transformshift{1.945910in}{0.862303in}%
\pgfsys@useobject{currentmarker}{}%
\end{pgfscope}%
\begin{pgfscope}%
\pgfsys@transformshift{1.946471in}{0.837512in}%
\pgfsys@useobject{currentmarker}{}%
\end{pgfscope}%
\begin{pgfscope}%
\pgfsys@transformshift{1.947031in}{0.821151in}%
\pgfsys@useobject{currentmarker}{}%
\end{pgfscope}%
\begin{pgfscope}%
\pgfsys@transformshift{1.947589in}{0.818575in}%
\pgfsys@useobject{currentmarker}{}%
\end{pgfscope}%
\begin{pgfscope}%
\pgfsys@transformshift{1.948145in}{0.881364in}%
\pgfsys@useobject{currentmarker}{}%
\end{pgfscope}%
\begin{pgfscope}%
\pgfsys@transformshift{1.948701in}{0.869961in}%
\pgfsys@useobject{currentmarker}{}%
\end{pgfscope}%
\begin{pgfscope}%
\pgfsys@transformshift{1.949255in}{0.854023in}%
\pgfsys@useobject{currentmarker}{}%
\end{pgfscope}%
\begin{pgfscope}%
\pgfsys@transformshift{1.949807in}{0.840069in}%
\pgfsys@useobject{currentmarker}{}%
\end{pgfscope}%
\begin{pgfscope}%
\pgfsys@transformshift{1.950359in}{0.861918in}%
\pgfsys@useobject{currentmarker}{}%
\end{pgfscope}%
\begin{pgfscope}%
\pgfsys@transformshift{1.950909in}{0.837782in}%
\pgfsys@useobject{currentmarker}{}%
\end{pgfscope}%
\begin{pgfscope}%
\pgfsys@transformshift{1.951457in}{0.837257in}%
\pgfsys@useobject{currentmarker}{}%
\end{pgfscope}%
\begin{pgfscope}%
\pgfsys@transformshift{1.952005in}{0.804909in}%
\pgfsys@useobject{currentmarker}{}%
\end{pgfscope}%
\begin{pgfscope}%
\pgfsys@transformshift{1.952550in}{0.835141in}%
\pgfsys@useobject{currentmarker}{}%
\end{pgfscope}%
\begin{pgfscope}%
\pgfsys@transformshift{1.953095in}{0.878779in}%
\pgfsys@useobject{currentmarker}{}%
\end{pgfscope}%
\begin{pgfscope}%
\pgfsys@transformshift{1.953638in}{0.876278in}%
\pgfsys@useobject{currentmarker}{}%
\end{pgfscope}%
\begin{pgfscope}%
\pgfsys@transformshift{1.954180in}{0.857814in}%
\pgfsys@useobject{currentmarker}{}%
\end{pgfscope}%
\begin{pgfscope}%
\pgfsys@transformshift{1.954721in}{0.894881in}%
\pgfsys@useobject{currentmarker}{}%
\end{pgfscope}%
\begin{pgfscope}%
\pgfsys@transformshift{1.955260in}{0.880440in}%
\pgfsys@useobject{currentmarker}{}%
\end{pgfscope}%
\begin{pgfscope}%
\pgfsys@transformshift{1.955799in}{0.833254in}%
\pgfsys@useobject{currentmarker}{}%
\end{pgfscope}%
\begin{pgfscope}%
\pgfsys@transformshift{1.956335in}{0.852752in}%
\pgfsys@useobject{currentmarker}{}%
\end{pgfscope}%
\begin{pgfscope}%
\pgfsys@transformshift{1.956871in}{0.872189in}%
\pgfsys@useobject{currentmarker}{}%
\end{pgfscope}%
\begin{pgfscope}%
\pgfsys@transformshift{1.957405in}{0.863373in}%
\pgfsys@useobject{currentmarker}{}%
\end{pgfscope}%
\begin{pgfscope}%
\pgfsys@transformshift{1.957938in}{0.803570in}%
\pgfsys@useobject{currentmarker}{}%
\end{pgfscope}%
\begin{pgfscope}%
\pgfsys@transformshift{1.958470in}{0.836460in}%
\pgfsys@useobject{currentmarker}{}%
\end{pgfscope}%
\begin{pgfscope}%
\pgfsys@transformshift{1.959000in}{0.879784in}%
\pgfsys@useobject{currentmarker}{}%
\end{pgfscope}%
\begin{pgfscope}%
\pgfsys@transformshift{1.959529in}{0.843010in}%
\pgfsys@useobject{currentmarker}{}%
\end{pgfscope}%
\begin{pgfscope}%
\pgfsys@transformshift{1.960057in}{0.816646in}%
\pgfsys@useobject{currentmarker}{}%
\end{pgfscope}%
\begin{pgfscope}%
\pgfsys@transformshift{1.960584in}{0.773861in}%
\pgfsys@useobject{currentmarker}{}%
\end{pgfscope}%
\begin{pgfscope}%
\pgfsys@transformshift{1.961110in}{0.805317in}%
\pgfsys@useobject{currentmarker}{}%
\end{pgfscope}%
\begin{pgfscope}%
\pgfsys@transformshift{1.961634in}{0.826616in}%
\pgfsys@useobject{currentmarker}{}%
\end{pgfscope}%
\begin{pgfscope}%
\pgfsys@transformshift{1.962157in}{0.815432in}%
\pgfsys@useobject{currentmarker}{}%
\end{pgfscope}%
\begin{pgfscope}%
\pgfsys@transformshift{1.962679in}{0.835468in}%
\pgfsys@useobject{currentmarker}{}%
\end{pgfscope}%
\begin{pgfscope}%
\pgfsys@transformshift{1.963199in}{0.858071in}%
\pgfsys@useobject{currentmarker}{}%
\end{pgfscope}%
\begin{pgfscope}%
\pgfsys@transformshift{1.963719in}{0.831046in}%
\pgfsys@useobject{currentmarker}{}%
\end{pgfscope}%
\begin{pgfscope}%
\pgfsys@transformshift{1.964237in}{0.819070in}%
\pgfsys@useobject{currentmarker}{}%
\end{pgfscope}%
\begin{pgfscope}%
\pgfsys@transformshift{1.964754in}{0.856447in}%
\pgfsys@useobject{currentmarker}{}%
\end{pgfscope}%
\begin{pgfscope}%
\pgfsys@transformshift{1.965270in}{0.860059in}%
\pgfsys@useobject{currentmarker}{}%
\end{pgfscope}%
\begin{pgfscope}%
\pgfsys@transformshift{1.965785in}{0.810766in}%
\pgfsys@useobject{currentmarker}{}%
\end{pgfscope}%
\begin{pgfscope}%
\pgfsys@transformshift{1.966298in}{0.813258in}%
\pgfsys@useobject{currentmarker}{}%
\end{pgfscope}%
\begin{pgfscope}%
\pgfsys@transformshift{1.966810in}{0.863284in}%
\pgfsys@useobject{currentmarker}{}%
\end{pgfscope}%
\begin{pgfscope}%
\pgfsys@transformshift{1.967322in}{0.874811in}%
\pgfsys@useobject{currentmarker}{}%
\end{pgfscope}%
\begin{pgfscope}%
\pgfsys@transformshift{1.967832in}{0.849959in}%
\pgfsys@useobject{currentmarker}{}%
\end{pgfscope}%
\begin{pgfscope}%
\pgfsys@transformshift{1.968340in}{0.800901in}%
\pgfsys@useobject{currentmarker}{}%
\end{pgfscope}%
\begin{pgfscope}%
\pgfsys@transformshift{1.968848in}{0.795933in}%
\pgfsys@useobject{currentmarker}{}%
\end{pgfscope}%
\begin{pgfscope}%
\pgfsys@transformshift{1.969355in}{0.831290in}%
\pgfsys@useobject{currentmarker}{}%
\end{pgfscope}%
\begin{pgfscope}%
\pgfsys@transformshift{1.969860in}{0.804019in}%
\pgfsys@useobject{currentmarker}{}%
\end{pgfscope}%
\begin{pgfscope}%
\pgfsys@transformshift{1.970364in}{0.836466in}%
\pgfsys@useobject{currentmarker}{}%
\end{pgfscope}%
\begin{pgfscope}%
\pgfsys@transformshift{1.970868in}{0.818526in}%
\pgfsys@useobject{currentmarker}{}%
\end{pgfscope}%
\begin{pgfscope}%
\pgfsys@transformshift{1.971370in}{0.796743in}%
\pgfsys@useobject{currentmarker}{}%
\end{pgfscope}%
\begin{pgfscope}%
\pgfsys@transformshift{1.971870in}{0.787864in}%
\pgfsys@useobject{currentmarker}{}%
\end{pgfscope}%
\begin{pgfscope}%
\pgfsys@transformshift{1.972370in}{0.782755in}%
\pgfsys@useobject{currentmarker}{}%
\end{pgfscope}%
\begin{pgfscope}%
\pgfsys@transformshift{1.972869in}{0.852076in}%
\pgfsys@useobject{currentmarker}{}%
\end{pgfscope}%
\begin{pgfscope}%
\pgfsys@transformshift{1.973366in}{0.823058in}%
\pgfsys@useobject{currentmarker}{}%
\end{pgfscope}%
\begin{pgfscope}%
\pgfsys@transformshift{1.973863in}{0.796900in}%
\pgfsys@useobject{currentmarker}{}%
\end{pgfscope}%
\begin{pgfscope}%
\pgfsys@transformshift{1.974358in}{0.822127in}%
\pgfsys@useobject{currentmarker}{}%
\end{pgfscope}%
\begin{pgfscope}%
\pgfsys@transformshift{1.974853in}{0.831811in}%
\pgfsys@useobject{currentmarker}{}%
\end{pgfscope}%
\begin{pgfscope}%
\pgfsys@transformshift{1.975346in}{0.852858in}%
\pgfsys@useobject{currentmarker}{}%
\end{pgfscope}%
\begin{pgfscope}%
\pgfsys@transformshift{1.975838in}{0.845188in}%
\pgfsys@useobject{currentmarker}{}%
\end{pgfscope}%
\begin{pgfscope}%
\pgfsys@transformshift{1.976329in}{0.855269in}%
\pgfsys@useobject{currentmarker}{}%
\end{pgfscope}%
\begin{pgfscope}%
\pgfsys@transformshift{1.976819in}{0.840067in}%
\pgfsys@useobject{currentmarker}{}%
\end{pgfscope}%
\begin{pgfscope}%
\pgfsys@transformshift{1.977308in}{0.828964in}%
\pgfsys@useobject{currentmarker}{}%
\end{pgfscope}%
\begin{pgfscope}%
\pgfsys@transformshift{1.977796in}{0.827034in}%
\pgfsys@useobject{currentmarker}{}%
\end{pgfscope}%
\begin{pgfscope}%
\pgfsys@transformshift{1.978282in}{0.807477in}%
\pgfsys@useobject{currentmarker}{}%
\end{pgfscope}%
\begin{pgfscope}%
\pgfsys@transformshift{1.978768in}{0.797191in}%
\pgfsys@useobject{currentmarker}{}%
\end{pgfscope}%
\begin{pgfscope}%
\pgfsys@transformshift{1.979253in}{0.825363in}%
\pgfsys@useobject{currentmarker}{}%
\end{pgfscope}%
\begin{pgfscope}%
\pgfsys@transformshift{1.979736in}{0.802362in}%
\pgfsys@useobject{currentmarker}{}%
\end{pgfscope}%
\begin{pgfscope}%
\pgfsys@transformshift{1.980219in}{0.810875in}%
\pgfsys@useobject{currentmarker}{}%
\end{pgfscope}%
\begin{pgfscope}%
\pgfsys@transformshift{1.980701in}{0.817469in}%
\pgfsys@useobject{currentmarker}{}%
\end{pgfscope}%
\begin{pgfscope}%
\pgfsys@transformshift{1.981181in}{0.857488in}%
\pgfsys@useobject{currentmarker}{}%
\end{pgfscope}%
\begin{pgfscope}%
\pgfsys@transformshift{1.981661in}{0.849323in}%
\pgfsys@useobject{currentmarker}{}%
\end{pgfscope}%
\begin{pgfscope}%
\pgfsys@transformshift{1.982139in}{0.848861in}%
\pgfsys@useobject{currentmarker}{}%
\end{pgfscope}%
\begin{pgfscope}%
\pgfsys@transformshift{1.982617in}{0.846113in}%
\pgfsys@useobject{currentmarker}{}%
\end{pgfscope}%
\begin{pgfscope}%
\pgfsys@transformshift{1.983093in}{0.824861in}%
\pgfsys@useobject{currentmarker}{}%
\end{pgfscope}%
\begin{pgfscope}%
\pgfsys@transformshift{1.983569in}{0.824708in}%
\pgfsys@useobject{currentmarker}{}%
\end{pgfscope}%
\begin{pgfscope}%
\pgfsys@transformshift{1.984043in}{0.798602in}%
\pgfsys@useobject{currentmarker}{}%
\end{pgfscope}%
\begin{pgfscope}%
\pgfsys@transformshift{1.984517in}{0.824694in}%
\pgfsys@useobject{currentmarker}{}%
\end{pgfscope}%
\begin{pgfscope}%
\pgfsys@transformshift{1.984989in}{0.839707in}%
\pgfsys@useobject{currentmarker}{}%
\end{pgfscope}%
\begin{pgfscope}%
\pgfsys@transformshift{1.985460in}{0.820366in}%
\pgfsys@useobject{currentmarker}{}%
\end{pgfscope}%
\begin{pgfscope}%
\pgfsys@transformshift{1.985931in}{0.791033in}%
\pgfsys@useobject{currentmarker}{}%
\end{pgfscope}%
\begin{pgfscope}%
\pgfsys@transformshift{1.986400in}{0.803211in}%
\pgfsys@useobject{currentmarker}{}%
\end{pgfscope}%
\begin{pgfscope}%
\pgfsys@transformshift{1.986869in}{0.770530in}%
\pgfsys@useobject{currentmarker}{}%
\end{pgfscope}%
\begin{pgfscope}%
\pgfsys@transformshift{1.987336in}{0.799032in}%
\pgfsys@useobject{currentmarker}{}%
\end{pgfscope}%
\begin{pgfscope}%
\pgfsys@transformshift{1.987803in}{0.830753in}%
\pgfsys@useobject{currentmarker}{}%
\end{pgfscope}%
\begin{pgfscope}%
\pgfsys@transformshift{1.988269in}{0.802493in}%
\pgfsys@useobject{currentmarker}{}%
\end{pgfscope}%
\begin{pgfscope}%
\pgfsys@transformshift{1.988733in}{0.762829in}%
\pgfsys@useobject{currentmarker}{}%
\end{pgfscope}%
\begin{pgfscope}%
\pgfsys@transformshift{1.989197in}{0.786480in}%
\pgfsys@useobject{currentmarker}{}%
\end{pgfscope}%
\begin{pgfscope}%
\pgfsys@transformshift{1.989660in}{0.849395in}%
\pgfsys@useobject{currentmarker}{}%
\end{pgfscope}%
\begin{pgfscope}%
\pgfsys@transformshift{1.990121in}{0.861601in}%
\pgfsys@useobject{currentmarker}{}%
\end{pgfscope}%
\begin{pgfscope}%
\pgfsys@transformshift{1.990582in}{0.844780in}%
\pgfsys@useobject{currentmarker}{}%
\end{pgfscope}%
\begin{pgfscope}%
\pgfsys@transformshift{1.991042in}{0.789673in}%
\pgfsys@useobject{currentmarker}{}%
\end{pgfscope}%
\begin{pgfscope}%
\pgfsys@transformshift{1.991501in}{0.773997in}%
\pgfsys@useobject{currentmarker}{}%
\end{pgfscope}%
\begin{pgfscope}%
\pgfsys@transformshift{1.991959in}{0.806476in}%
\pgfsys@useobject{currentmarker}{}%
\end{pgfscope}%
\begin{pgfscope}%
\pgfsys@transformshift{1.992416in}{0.823982in}%
\pgfsys@useobject{currentmarker}{}%
\end{pgfscope}%
\begin{pgfscope}%
\pgfsys@transformshift{1.992872in}{0.816315in}%
\pgfsys@useobject{currentmarker}{}%
\end{pgfscope}%
\begin{pgfscope}%
\pgfsys@transformshift{1.993328in}{0.840357in}%
\pgfsys@useobject{currentmarker}{}%
\end{pgfscope}%
\begin{pgfscope}%
\pgfsys@transformshift{1.993782in}{0.823913in}%
\pgfsys@useobject{currentmarker}{}%
\end{pgfscope}%
\begin{pgfscope}%
\pgfsys@transformshift{1.994235in}{0.804318in}%
\pgfsys@useobject{currentmarker}{}%
\end{pgfscope}%
\begin{pgfscope}%
\pgfsys@transformshift{1.994688in}{0.799733in}%
\pgfsys@useobject{currentmarker}{}%
\end{pgfscope}%
\begin{pgfscope}%
\pgfsys@transformshift{1.995139in}{0.863067in}%
\pgfsys@useobject{currentmarker}{}%
\end{pgfscope}%
\begin{pgfscope}%
\pgfsys@transformshift{1.995590in}{0.860490in}%
\pgfsys@useobject{currentmarker}{}%
\end{pgfscope}%
\begin{pgfscope}%
\pgfsys@transformshift{1.996040in}{0.809483in}%
\pgfsys@useobject{currentmarker}{}%
\end{pgfscope}%
\begin{pgfscope}%
\pgfsys@transformshift{1.996488in}{0.797447in}%
\pgfsys@useobject{currentmarker}{}%
\end{pgfscope}%
\begin{pgfscope}%
\pgfsys@transformshift{1.996936in}{0.815548in}%
\pgfsys@useobject{currentmarker}{}%
\end{pgfscope}%
\begin{pgfscope}%
\pgfsys@transformshift{1.997384in}{0.832481in}%
\pgfsys@useobject{currentmarker}{}%
\end{pgfscope}%
\begin{pgfscope}%
\pgfsys@transformshift{1.997830in}{0.805600in}%
\pgfsys@useobject{currentmarker}{}%
\end{pgfscope}%
\begin{pgfscope}%
\pgfsys@transformshift{1.998275in}{0.812903in}%
\pgfsys@useobject{currentmarker}{}%
\end{pgfscope}%
\begin{pgfscope}%
\pgfsys@transformshift{1.998719in}{0.843620in}%
\pgfsys@useobject{currentmarker}{}%
\end{pgfscope}%
\begin{pgfscope}%
\pgfsys@transformshift{1.999163in}{0.824199in}%
\pgfsys@useobject{currentmarker}{}%
\end{pgfscope}%
\begin{pgfscope}%
\pgfsys@transformshift{1.999606in}{0.765688in}%
\pgfsys@useobject{currentmarker}{}%
\end{pgfscope}%
\begin{pgfscope}%
\pgfsys@transformshift{2.000047in}{0.815875in}%
\pgfsys@useobject{currentmarker}{}%
\end{pgfscope}%
\begin{pgfscope}%
\pgfsys@transformshift{2.000488in}{0.816073in}%
\pgfsys@useobject{currentmarker}{}%
\end{pgfscope}%
\begin{pgfscope}%
\pgfsys@transformshift{2.000928in}{0.835400in}%
\pgfsys@useobject{currentmarker}{}%
\end{pgfscope}%
\begin{pgfscope}%
\pgfsys@transformshift{2.001368in}{0.803598in}%
\pgfsys@useobject{currentmarker}{}%
\end{pgfscope}%
\begin{pgfscope}%
\pgfsys@transformshift{2.001806in}{0.804704in}%
\pgfsys@useobject{currentmarker}{}%
\end{pgfscope}%
\begin{pgfscope}%
\pgfsys@transformshift{2.002243in}{0.848122in}%
\pgfsys@useobject{currentmarker}{}%
\end{pgfscope}%
\begin{pgfscope}%
\pgfsys@transformshift{2.002680in}{0.736194in}%
\pgfsys@useobject{currentmarker}{}%
\end{pgfscope}%
\begin{pgfscope}%
\pgfsys@transformshift{2.003116in}{0.748053in}%
\pgfsys@useobject{currentmarker}{}%
\end{pgfscope}%
\begin{pgfscope}%
\pgfsys@transformshift{2.003551in}{0.785164in}%
\pgfsys@useobject{currentmarker}{}%
\end{pgfscope}%
\begin{pgfscope}%
\pgfsys@transformshift{2.003985in}{0.846940in}%
\pgfsys@useobject{currentmarker}{}%
\end{pgfscope}%
\begin{pgfscope}%
\pgfsys@transformshift{2.004418in}{0.826073in}%
\pgfsys@useobject{currentmarker}{}%
\end{pgfscope}%
\begin{pgfscope}%
\pgfsys@transformshift{2.004851in}{0.732683in}%
\pgfsys@useobject{currentmarker}{}%
\end{pgfscope}%
\begin{pgfscope}%
\pgfsys@transformshift{2.005282in}{0.764420in}%
\pgfsys@useobject{currentmarker}{}%
\end{pgfscope}%
\begin{pgfscope}%
\pgfsys@transformshift{2.005713in}{0.750013in}%
\pgfsys@useobject{currentmarker}{}%
\end{pgfscope}%
\begin{pgfscope}%
\pgfsys@transformshift{2.006143in}{0.841089in}%
\pgfsys@useobject{currentmarker}{}%
\end{pgfscope}%
\begin{pgfscope}%
\pgfsys@transformshift{2.006572in}{0.796806in}%
\pgfsys@useobject{currentmarker}{}%
\end{pgfscope}%
\begin{pgfscope}%
\pgfsys@transformshift{2.007000in}{0.777390in}%
\pgfsys@useobject{currentmarker}{}%
\end{pgfscope}%
\begin{pgfscope}%
\pgfsys@transformshift{2.007428in}{0.816028in}%
\pgfsys@useobject{currentmarker}{}%
\end{pgfscope}%
\begin{pgfscope}%
\pgfsys@transformshift{2.007855in}{0.842749in}%
\pgfsys@useobject{currentmarker}{}%
\end{pgfscope}%
\begin{pgfscope}%
\pgfsys@transformshift{2.008280in}{0.837164in}%
\pgfsys@useobject{currentmarker}{}%
\end{pgfscope}%
\begin{pgfscope}%
\pgfsys@transformshift{2.008706in}{0.795140in}%
\pgfsys@useobject{currentmarker}{}%
\end{pgfscope}%
\begin{pgfscope}%
\pgfsys@transformshift{2.009130in}{0.760137in}%
\pgfsys@useobject{currentmarker}{}%
\end{pgfscope}%
\begin{pgfscope}%
\pgfsys@transformshift{2.009553in}{0.793978in}%
\pgfsys@useobject{currentmarker}{}%
\end{pgfscope}%
\begin{pgfscope}%
\pgfsys@transformshift{2.009976in}{0.836378in}%
\pgfsys@useobject{currentmarker}{}%
\end{pgfscope}%
\begin{pgfscope}%
\pgfsys@transformshift{2.010398in}{0.815180in}%
\pgfsys@useobject{currentmarker}{}%
\end{pgfscope}%
\begin{pgfscope}%
\pgfsys@transformshift{2.010819in}{0.780410in}%
\pgfsys@useobject{currentmarker}{}%
\end{pgfscope}%
\begin{pgfscope}%
\pgfsys@transformshift{2.011239in}{0.798333in}%
\pgfsys@useobject{currentmarker}{}%
\end{pgfscope}%
\begin{pgfscope}%
\pgfsys@transformshift{2.011659in}{0.824686in}%
\pgfsys@useobject{currentmarker}{}%
\end{pgfscope}%
\begin{pgfscope}%
\pgfsys@transformshift{2.012078in}{0.834952in}%
\pgfsys@useobject{currentmarker}{}%
\end{pgfscope}%
\begin{pgfscope}%
\pgfsys@transformshift{2.012495in}{0.791375in}%
\pgfsys@useobject{currentmarker}{}%
\end{pgfscope}%
\begin{pgfscope}%
\pgfsys@transformshift{2.012913in}{0.794212in}%
\pgfsys@useobject{currentmarker}{}%
\end{pgfscope}%
\begin{pgfscope}%
\pgfsys@transformshift{2.013329in}{0.783574in}%
\pgfsys@useobject{currentmarker}{}%
\end{pgfscope}%
\begin{pgfscope}%
\pgfsys@transformshift{2.013745in}{0.809414in}%
\pgfsys@useobject{currentmarker}{}%
\end{pgfscope}%
\begin{pgfscope}%
\pgfsys@transformshift{2.014160in}{0.802448in}%
\pgfsys@useobject{currentmarker}{}%
\end{pgfscope}%
\begin{pgfscope}%
\pgfsys@transformshift{2.014574in}{0.824433in}%
\pgfsys@useobject{currentmarker}{}%
\end{pgfscope}%
\begin{pgfscope}%
\pgfsys@transformshift{2.014987in}{0.829897in}%
\pgfsys@useobject{currentmarker}{}%
\end{pgfscope}%
\begin{pgfscope}%
\pgfsys@transformshift{2.015400in}{0.794305in}%
\pgfsys@useobject{currentmarker}{}%
\end{pgfscope}%
\begin{pgfscope}%
\pgfsys@transformshift{2.015811in}{0.769556in}%
\pgfsys@useobject{currentmarker}{}%
\end{pgfscope}%
\begin{pgfscope}%
\pgfsys@transformshift{2.016223in}{0.808688in}%
\pgfsys@useobject{currentmarker}{}%
\end{pgfscope}%
\begin{pgfscope}%
\pgfsys@transformshift{2.016633in}{0.812639in}%
\pgfsys@useobject{currentmarker}{}%
\end{pgfscope}%
\begin{pgfscope}%
\pgfsys@transformshift{2.017042in}{0.787931in}%
\pgfsys@useobject{currentmarker}{}%
\end{pgfscope}%
\begin{pgfscope}%
\pgfsys@transformshift{2.017451in}{0.772141in}%
\pgfsys@useobject{currentmarker}{}%
\end{pgfscope}%
\begin{pgfscope}%
\pgfsys@transformshift{2.017859in}{0.783414in}%
\pgfsys@useobject{currentmarker}{}%
\end{pgfscope}%
\begin{pgfscope}%
\pgfsys@transformshift{2.018267in}{0.762160in}%
\pgfsys@useobject{currentmarker}{}%
\end{pgfscope}%
\begin{pgfscope}%
\pgfsys@transformshift{2.018673in}{0.804974in}%
\pgfsys@useobject{currentmarker}{}%
\end{pgfscope}%
\begin{pgfscope}%
\pgfsys@transformshift{2.019079in}{0.846118in}%
\pgfsys@useobject{currentmarker}{}%
\end{pgfscope}%
\begin{pgfscope}%
\pgfsys@transformshift{2.019484in}{0.821338in}%
\pgfsys@useobject{currentmarker}{}%
\end{pgfscope}%
\begin{pgfscope}%
\pgfsys@transformshift{2.019889in}{0.811291in}%
\pgfsys@useobject{currentmarker}{}%
\end{pgfscope}%
\begin{pgfscope}%
\pgfsys@transformshift{2.020292in}{0.822839in}%
\pgfsys@useobject{currentmarker}{}%
\end{pgfscope}%
\begin{pgfscope}%
\pgfsys@transformshift{2.020695in}{0.818493in}%
\pgfsys@useobject{currentmarker}{}%
\end{pgfscope}%
\begin{pgfscope}%
\pgfsys@transformshift{2.021098in}{0.804460in}%
\pgfsys@useobject{currentmarker}{}%
\end{pgfscope}%
\begin{pgfscope}%
\pgfsys@transformshift{2.021499in}{0.817500in}%
\pgfsys@useobject{currentmarker}{}%
\end{pgfscope}%
\begin{pgfscope}%
\pgfsys@transformshift{2.021900in}{0.803444in}%
\pgfsys@useobject{currentmarker}{}%
\end{pgfscope}%
\begin{pgfscope}%
\pgfsys@transformshift{2.022300in}{0.787036in}%
\pgfsys@useobject{currentmarker}{}%
\end{pgfscope}%
\begin{pgfscope}%
\pgfsys@transformshift{2.022699in}{0.777913in}%
\pgfsys@useobject{currentmarker}{}%
\end{pgfscope}%
\begin{pgfscope}%
\pgfsys@transformshift{2.023098in}{0.817688in}%
\pgfsys@useobject{currentmarker}{}%
\end{pgfscope}%
\begin{pgfscope}%
\pgfsys@transformshift{2.023496in}{0.833024in}%
\pgfsys@useobject{currentmarker}{}%
\end{pgfscope}%
\begin{pgfscope}%
\pgfsys@transformshift{2.023893in}{0.814919in}%
\pgfsys@useobject{currentmarker}{}%
\end{pgfscope}%
\begin{pgfscope}%
\pgfsys@transformshift{2.024290in}{0.834457in}%
\pgfsys@useobject{currentmarker}{}%
\end{pgfscope}%
\begin{pgfscope}%
\pgfsys@transformshift{2.024686in}{0.786378in}%
\pgfsys@useobject{currentmarker}{}%
\end{pgfscope}%
\begin{pgfscope}%
\pgfsys@transformshift{2.025081in}{0.734385in}%
\pgfsys@useobject{currentmarker}{}%
\end{pgfscope}%
\begin{pgfscope}%
\pgfsys@transformshift{2.025475in}{0.760606in}%
\pgfsys@useobject{currentmarker}{}%
\end{pgfscope}%
\begin{pgfscope}%
\pgfsys@transformshift{2.025869in}{0.822223in}%
\pgfsys@useobject{currentmarker}{}%
\end{pgfscope}%
\begin{pgfscope}%
\pgfsys@transformshift{2.026262in}{0.846939in}%
\pgfsys@useobject{currentmarker}{}%
\end{pgfscope}%
\begin{pgfscope}%
\pgfsys@transformshift{2.026655in}{0.818413in}%
\pgfsys@useobject{currentmarker}{}%
\end{pgfscope}%
\begin{pgfscope}%
\pgfsys@transformshift{2.027046in}{0.772538in}%
\pgfsys@useobject{currentmarker}{}%
\end{pgfscope}%
\begin{pgfscope}%
\pgfsys@transformshift{2.027437in}{0.764981in}%
\pgfsys@useobject{currentmarker}{}%
\end{pgfscope}%
\begin{pgfscope}%
\pgfsys@transformshift{2.027828in}{0.817610in}%
\pgfsys@useobject{currentmarker}{}%
\end{pgfscope}%
\begin{pgfscope}%
\pgfsys@transformshift{2.028217in}{0.795965in}%
\pgfsys@useobject{currentmarker}{}%
\end{pgfscope}%
\begin{pgfscope}%
\pgfsys@transformshift{2.028606in}{0.797624in}%
\pgfsys@useobject{currentmarker}{}%
\end{pgfscope}%
\begin{pgfscope}%
\pgfsys@transformshift{2.028995in}{0.767084in}%
\pgfsys@useobject{currentmarker}{}%
\end{pgfscope}%
\begin{pgfscope}%
\pgfsys@transformshift{2.029382in}{0.758665in}%
\pgfsys@useobject{currentmarker}{}%
\end{pgfscope}%
\begin{pgfscope}%
\pgfsys@transformshift{2.029769in}{0.827693in}%
\pgfsys@useobject{currentmarker}{}%
\end{pgfscope}%
\begin{pgfscope}%
\pgfsys@transformshift{2.030156in}{0.814346in}%
\pgfsys@useobject{currentmarker}{}%
\end{pgfscope}%
\begin{pgfscope}%
\pgfsys@transformshift{2.030541in}{0.754448in}%
\pgfsys@useobject{currentmarker}{}%
\end{pgfscope}%
\begin{pgfscope}%
\pgfsys@transformshift{2.030926in}{0.745627in}%
\pgfsys@useobject{currentmarker}{}%
\end{pgfscope}%
\begin{pgfscope}%
\pgfsys@transformshift{2.031311in}{0.784945in}%
\pgfsys@useobject{currentmarker}{}%
\end{pgfscope}%
\begin{pgfscope}%
\pgfsys@transformshift{2.031694in}{0.815758in}%
\pgfsys@useobject{currentmarker}{}%
\end{pgfscope}%
\begin{pgfscope}%
\pgfsys@transformshift{2.032078in}{0.810155in}%
\pgfsys@useobject{currentmarker}{}%
\end{pgfscope}%
\begin{pgfscope}%
\pgfsys@transformshift{2.032460in}{0.790675in}%
\pgfsys@useobject{currentmarker}{}%
\end{pgfscope}%
\begin{pgfscope}%
\pgfsys@transformshift{2.032842in}{0.764484in}%
\pgfsys@useobject{currentmarker}{}%
\end{pgfscope}%
\begin{pgfscope}%
\pgfsys@transformshift{2.033223in}{0.811195in}%
\pgfsys@useobject{currentmarker}{}%
\end{pgfscope}%
\begin{pgfscope}%
\pgfsys@transformshift{2.033603in}{0.765842in}%
\pgfsys@useobject{currentmarker}{}%
\end{pgfscope}%
\begin{pgfscope}%
\pgfsys@transformshift{2.033983in}{0.794832in}%
\pgfsys@useobject{currentmarker}{}%
\end{pgfscope}%
\begin{pgfscope}%
\pgfsys@transformshift{2.034362in}{0.809335in}%
\pgfsys@useobject{currentmarker}{}%
\end{pgfscope}%
\begin{pgfscope}%
\pgfsys@transformshift{2.034741in}{0.744305in}%
\pgfsys@useobject{currentmarker}{}%
\end{pgfscope}%
\begin{pgfscope}%
\pgfsys@transformshift{2.035119in}{0.796932in}%
\pgfsys@useobject{currentmarker}{}%
\end{pgfscope}%
\begin{pgfscope}%
\pgfsys@transformshift{2.035496in}{0.807257in}%
\pgfsys@useobject{currentmarker}{}%
\end{pgfscope}%
\begin{pgfscope}%
\pgfsys@transformshift{2.035872in}{0.794995in}%
\pgfsys@useobject{currentmarker}{}%
\end{pgfscope}%
\begin{pgfscope}%
\pgfsys@transformshift{2.036248in}{0.837130in}%
\pgfsys@useobject{currentmarker}{}%
\end{pgfscope}%
\begin{pgfscope}%
\pgfsys@transformshift{2.036624in}{0.832889in}%
\pgfsys@useobject{currentmarker}{}%
\end{pgfscope}%
\begin{pgfscope}%
\pgfsys@transformshift{2.036998in}{0.791830in}%
\pgfsys@useobject{currentmarker}{}%
\end{pgfscope}%
\begin{pgfscope}%
\pgfsys@transformshift{2.037373in}{0.784453in}%
\pgfsys@useobject{currentmarker}{}%
\end{pgfscope}%
\begin{pgfscope}%
\pgfsys@transformshift{2.037746in}{0.797190in}%
\pgfsys@useobject{currentmarker}{}%
\end{pgfscope}%
\begin{pgfscope}%
\pgfsys@transformshift{2.038119in}{0.795050in}%
\pgfsys@useobject{currentmarker}{}%
\end{pgfscope}%
\begin{pgfscope}%
\pgfsys@transformshift{2.038491in}{0.798083in}%
\pgfsys@useobject{currentmarker}{}%
\end{pgfscope}%
\begin{pgfscope}%
\pgfsys@transformshift{2.038863in}{0.744900in}%
\pgfsys@useobject{currentmarker}{}%
\end{pgfscope}%
\begin{pgfscope}%
\pgfsys@transformshift{2.039234in}{0.779797in}%
\pgfsys@useobject{currentmarker}{}%
\end{pgfscope}%
\begin{pgfscope}%
\pgfsys@transformshift{2.039604in}{0.776970in}%
\pgfsys@useobject{currentmarker}{}%
\end{pgfscope}%
\begin{pgfscope}%
\pgfsys@transformshift{2.039974in}{0.765209in}%
\pgfsys@useobject{currentmarker}{}%
\end{pgfscope}%
\begin{pgfscope}%
\pgfsys@transformshift{2.040343in}{0.809163in}%
\pgfsys@useobject{currentmarker}{}%
\end{pgfscope}%
\begin{pgfscope}%
\pgfsys@transformshift{2.040712in}{0.776786in}%
\pgfsys@useobject{currentmarker}{}%
\end{pgfscope}%
\begin{pgfscope}%
\pgfsys@transformshift{2.041080in}{0.780541in}%
\pgfsys@useobject{currentmarker}{}%
\end{pgfscope}%
\begin{pgfscope}%
\pgfsys@transformshift{2.041447in}{0.800843in}%
\pgfsys@useobject{currentmarker}{}%
\end{pgfscope}%
\begin{pgfscope}%
\pgfsys@transformshift{2.041814in}{0.825914in}%
\pgfsys@useobject{currentmarker}{}%
\end{pgfscope}%
\begin{pgfscope}%
\pgfsys@transformshift{2.042180in}{0.864052in}%
\pgfsys@useobject{currentmarker}{}%
\end{pgfscope}%
\begin{pgfscope}%
\pgfsys@transformshift{2.042546in}{0.770022in}%
\pgfsys@useobject{currentmarker}{}%
\end{pgfscope}%
\begin{pgfscope}%
\pgfsys@transformshift{2.042911in}{0.778317in}%
\pgfsys@useobject{currentmarker}{}%
\end{pgfscope}%
\begin{pgfscope}%
\pgfsys@transformshift{2.043275in}{0.836923in}%
\pgfsys@useobject{currentmarker}{}%
\end{pgfscope}%
\begin{pgfscope}%
\pgfsys@transformshift{2.043639in}{0.803202in}%
\pgfsys@useobject{currentmarker}{}%
\end{pgfscope}%
\begin{pgfscope}%
\pgfsys@transformshift{2.044002in}{0.765283in}%
\pgfsys@useobject{currentmarker}{}%
\end{pgfscope}%
\begin{pgfscope}%
\pgfsys@transformshift{2.044365in}{0.813224in}%
\pgfsys@useobject{currentmarker}{}%
\end{pgfscope}%
\begin{pgfscope}%
\pgfsys@transformshift{2.044727in}{0.792821in}%
\pgfsys@useobject{currentmarker}{}%
\end{pgfscope}%
\begin{pgfscope}%
\pgfsys@transformshift{2.045088in}{0.757780in}%
\pgfsys@useobject{currentmarker}{}%
\end{pgfscope}%
\begin{pgfscope}%
\pgfsys@transformshift{2.045449in}{0.763163in}%
\pgfsys@useobject{currentmarker}{}%
\end{pgfscope}%
\begin{pgfscope}%
\pgfsys@transformshift{2.045809in}{0.783541in}%
\pgfsys@useobject{currentmarker}{}%
\end{pgfscope}%
\begin{pgfscope}%
\pgfsys@transformshift{2.046169in}{0.792656in}%
\pgfsys@useobject{currentmarker}{}%
\end{pgfscope}%
\begin{pgfscope}%
\pgfsys@transformshift{2.046528in}{0.771310in}%
\pgfsys@useobject{currentmarker}{}%
\end{pgfscope}%
\begin{pgfscope}%
\pgfsys@transformshift{2.046887in}{0.742986in}%
\pgfsys@useobject{currentmarker}{}%
\end{pgfscope}%
\begin{pgfscope}%
\pgfsys@transformshift{2.047245in}{0.788321in}%
\pgfsys@useobject{currentmarker}{}%
\end{pgfscope}%
\begin{pgfscope}%
\pgfsys@transformshift{2.047602in}{0.795158in}%
\pgfsys@useobject{currentmarker}{}%
\end{pgfscope}%
\begin{pgfscope}%
\pgfsys@transformshift{2.047959in}{0.795058in}%
\pgfsys@useobject{currentmarker}{}%
\end{pgfscope}%
\begin{pgfscope}%
\pgfsys@transformshift{2.048316in}{0.795739in}%
\pgfsys@useobject{currentmarker}{}%
\end{pgfscope}%
\begin{pgfscope}%
\pgfsys@transformshift{2.048671in}{0.792409in}%
\pgfsys@useobject{currentmarker}{}%
\end{pgfscope}%
\begin{pgfscope}%
\pgfsys@transformshift{2.049027in}{0.776553in}%
\pgfsys@useobject{currentmarker}{}%
\end{pgfscope}%
\begin{pgfscope}%
\pgfsys@transformshift{2.049381in}{0.784012in}%
\pgfsys@useobject{currentmarker}{}%
\end{pgfscope}%
\begin{pgfscope}%
\pgfsys@transformshift{2.049735in}{0.772137in}%
\pgfsys@useobject{currentmarker}{}%
\end{pgfscope}%
\begin{pgfscope}%
\pgfsys@transformshift{2.050089in}{0.791689in}%
\pgfsys@useobject{currentmarker}{}%
\end{pgfscope}%
\begin{pgfscope}%
\pgfsys@transformshift{2.050442in}{0.807334in}%
\pgfsys@useobject{currentmarker}{}%
\end{pgfscope}%
\begin{pgfscope}%
\pgfsys@transformshift{2.050794in}{0.752869in}%
\pgfsys@useobject{currentmarker}{}%
\end{pgfscope}%
\begin{pgfscope}%
\pgfsys@transformshift{2.051146in}{0.725112in}%
\pgfsys@useobject{currentmarker}{}%
\end{pgfscope}%
\begin{pgfscope}%
\pgfsys@transformshift{2.051497in}{0.812152in}%
\pgfsys@useobject{currentmarker}{}%
\end{pgfscope}%
\begin{pgfscope}%
\pgfsys@transformshift{2.051848in}{0.825674in}%
\pgfsys@useobject{currentmarker}{}%
\end{pgfscope}%
\begin{pgfscope}%
\pgfsys@transformshift{2.052198in}{0.767659in}%
\pgfsys@useobject{currentmarker}{}%
\end{pgfscope}%
\begin{pgfscope}%
\pgfsys@transformshift{2.052548in}{0.826811in}%
\pgfsys@useobject{currentmarker}{}%
\end{pgfscope}%
\begin{pgfscope}%
\pgfsys@transformshift{2.052897in}{0.828595in}%
\pgfsys@useobject{currentmarker}{}%
\end{pgfscope}%
\begin{pgfscope}%
\pgfsys@transformshift{2.053245in}{0.791855in}%
\pgfsys@useobject{currentmarker}{}%
\end{pgfscope}%
\begin{pgfscope}%
\pgfsys@transformshift{2.053593in}{0.729114in}%
\pgfsys@useobject{currentmarker}{}%
\end{pgfscope}%
\begin{pgfscope}%
\pgfsys@transformshift{2.053941in}{0.723759in}%
\pgfsys@useobject{currentmarker}{}%
\end{pgfscope}%
\begin{pgfscope}%
\pgfsys@transformshift{2.054288in}{0.748747in}%
\pgfsys@useobject{currentmarker}{}%
\end{pgfscope}%
\begin{pgfscope}%
\pgfsys@transformshift{2.054634in}{0.762472in}%
\pgfsys@useobject{currentmarker}{}%
\end{pgfscope}%
\begin{pgfscope}%
\pgfsys@transformshift{2.054980in}{0.785257in}%
\pgfsys@useobject{currentmarker}{}%
\end{pgfscope}%
\begin{pgfscope}%
\pgfsys@transformshift{2.055326in}{0.795376in}%
\pgfsys@useobject{currentmarker}{}%
\end{pgfscope}%
\begin{pgfscope}%
\pgfsys@transformshift{2.055670in}{0.804972in}%
\pgfsys@useobject{currentmarker}{}%
\end{pgfscope}%
\begin{pgfscope}%
\pgfsys@transformshift{2.056015in}{0.789617in}%
\pgfsys@useobject{currentmarker}{}%
\end{pgfscope}%
\begin{pgfscope}%
\pgfsys@transformshift{2.056358in}{0.813792in}%
\pgfsys@useobject{currentmarker}{}%
\end{pgfscope}%
\begin{pgfscope}%
\pgfsys@transformshift{2.056702in}{0.788919in}%
\pgfsys@useobject{currentmarker}{}%
\end{pgfscope}%
\begin{pgfscope}%
\pgfsys@transformshift{2.057044in}{0.727923in}%
\pgfsys@useobject{currentmarker}{}%
\end{pgfscope}%
\begin{pgfscope}%
\pgfsys@transformshift{2.057387in}{0.737615in}%
\pgfsys@useobject{currentmarker}{}%
\end{pgfscope}%
\begin{pgfscope}%
\pgfsys@transformshift{2.057728in}{0.782174in}%
\pgfsys@useobject{currentmarker}{}%
\end{pgfscope}%
\begin{pgfscope}%
\pgfsys@transformshift{2.058069in}{0.781884in}%
\pgfsys@useobject{currentmarker}{}%
\end{pgfscope}%
\begin{pgfscope}%
\pgfsys@transformshift{2.058410in}{0.721037in}%
\pgfsys@useobject{currentmarker}{}%
\end{pgfscope}%
\begin{pgfscope}%
\pgfsys@transformshift{2.058750in}{0.765744in}%
\pgfsys@useobject{currentmarker}{}%
\end{pgfscope}%
\begin{pgfscope}%
\pgfsys@transformshift{2.059090in}{0.808250in}%
\pgfsys@useobject{currentmarker}{}%
\end{pgfscope}%
\begin{pgfscope}%
\pgfsys@transformshift{2.059429in}{0.822864in}%
\pgfsys@useobject{currentmarker}{}%
\end{pgfscope}%
\begin{pgfscope}%
\pgfsys@transformshift{2.059767in}{0.763653in}%
\pgfsys@useobject{currentmarker}{}%
\end{pgfscope}%
\begin{pgfscope}%
\pgfsys@transformshift{2.060106in}{0.761226in}%
\pgfsys@useobject{currentmarker}{}%
\end{pgfscope}%
\begin{pgfscope}%
\pgfsys@transformshift{2.060443in}{0.788990in}%
\pgfsys@useobject{currentmarker}{}%
\end{pgfscope}%
\begin{pgfscope}%
\pgfsys@transformshift{2.060780in}{0.773484in}%
\pgfsys@useobject{currentmarker}{}%
\end{pgfscope}%
\begin{pgfscope}%
\pgfsys@transformshift{2.061117in}{0.790140in}%
\pgfsys@useobject{currentmarker}{}%
\end{pgfscope}%
\begin{pgfscope}%
\pgfsys@transformshift{2.061453in}{0.820679in}%
\pgfsys@useobject{currentmarker}{}%
\end{pgfscope}%
\begin{pgfscope}%
\pgfsys@transformshift{2.061788in}{0.758347in}%
\pgfsys@useobject{currentmarker}{}%
\end{pgfscope}%
\begin{pgfscope}%
\pgfsys@transformshift{2.062123in}{0.758432in}%
\pgfsys@useobject{currentmarker}{}%
\end{pgfscope}%
\begin{pgfscope}%
\pgfsys@transformshift{2.062458in}{0.787997in}%
\pgfsys@useobject{currentmarker}{}%
\end{pgfscope}%
\begin{pgfscope}%
\pgfsys@transformshift{2.062792in}{0.721743in}%
\pgfsys@useobject{currentmarker}{}%
\end{pgfscope}%
\begin{pgfscope}%
\pgfsys@transformshift{2.063126in}{0.800623in}%
\pgfsys@useobject{currentmarker}{}%
\end{pgfscope}%
\begin{pgfscope}%
\pgfsys@transformshift{2.063459in}{0.781085in}%
\pgfsys@useobject{currentmarker}{}%
\end{pgfscope}%
\begin{pgfscope}%
\pgfsys@transformshift{2.063791in}{0.756461in}%
\pgfsys@useobject{currentmarker}{}%
\end{pgfscope}%
\begin{pgfscope}%
\pgfsys@transformshift{2.064123in}{0.715672in}%
\pgfsys@useobject{currentmarker}{}%
\end{pgfscope}%
\begin{pgfscope}%
\pgfsys@transformshift{2.064455in}{0.772093in}%
\pgfsys@useobject{currentmarker}{}%
\end{pgfscope}%
\begin{pgfscope}%
\pgfsys@transformshift{2.064786in}{0.787890in}%
\pgfsys@useobject{currentmarker}{}%
\end{pgfscope}%
\begin{pgfscope}%
\pgfsys@transformshift{2.065117in}{0.787823in}%
\pgfsys@useobject{currentmarker}{}%
\end{pgfscope}%
\begin{pgfscope}%
\pgfsys@transformshift{2.065447in}{0.776960in}%
\pgfsys@useobject{currentmarker}{}%
\end{pgfscope}%
\begin{pgfscope}%
\pgfsys@transformshift{2.065776in}{0.764442in}%
\pgfsys@useobject{currentmarker}{}%
\end{pgfscope}%
\begin{pgfscope}%
\pgfsys@transformshift{2.066106in}{0.746445in}%
\pgfsys@useobject{currentmarker}{}%
\end{pgfscope}%
\begin{pgfscope}%
\pgfsys@transformshift{2.066434in}{0.785003in}%
\pgfsys@useobject{currentmarker}{}%
\end{pgfscope}%
\begin{pgfscope}%
\pgfsys@transformshift{2.066762in}{0.785275in}%
\pgfsys@useobject{currentmarker}{}%
\end{pgfscope}%
\begin{pgfscope}%
\pgfsys@transformshift{2.067090in}{0.766141in}%
\pgfsys@useobject{currentmarker}{}%
\end{pgfscope}%
\begin{pgfscope}%
\pgfsys@transformshift{2.067417in}{0.737749in}%
\pgfsys@useobject{currentmarker}{}%
\end{pgfscope}%
\begin{pgfscope}%
\pgfsys@transformshift{2.067744in}{0.729141in}%
\pgfsys@useobject{currentmarker}{}%
\end{pgfscope}%
\begin{pgfscope}%
\pgfsys@transformshift{2.068070in}{0.738890in}%
\pgfsys@useobject{currentmarker}{}%
\end{pgfscope}%
\begin{pgfscope}%
\pgfsys@transformshift{2.068396in}{0.748667in}%
\pgfsys@useobject{currentmarker}{}%
\end{pgfscope}%
\begin{pgfscope}%
\pgfsys@transformshift{2.068722in}{0.756254in}%
\pgfsys@useobject{currentmarker}{}%
\end{pgfscope}%
\begin{pgfscope}%
\pgfsys@transformshift{2.069046in}{0.778096in}%
\pgfsys@useobject{currentmarker}{}%
\end{pgfscope}%
\begin{pgfscope}%
\pgfsys@transformshift{2.069371in}{0.758567in}%
\pgfsys@useobject{currentmarker}{}%
\end{pgfscope}%
\begin{pgfscope}%
\pgfsys@transformshift{2.069695in}{0.745304in}%
\pgfsys@useobject{currentmarker}{}%
\end{pgfscope}%
\begin{pgfscope}%
\pgfsys@transformshift{2.070018in}{0.727923in}%
\pgfsys@useobject{currentmarker}{}%
\end{pgfscope}%
\begin{pgfscope}%
\pgfsys@transformshift{2.070341in}{0.760419in}%
\pgfsys@useobject{currentmarker}{}%
\end{pgfscope}%
\begin{pgfscope}%
\pgfsys@transformshift{2.070664in}{0.758535in}%
\pgfsys@useobject{currentmarker}{}%
\end{pgfscope}%
\begin{pgfscope}%
\pgfsys@transformshift{2.070986in}{0.752122in}%
\pgfsys@useobject{currentmarker}{}%
\end{pgfscope}%
\begin{pgfscope}%
\pgfsys@transformshift{2.071308in}{0.781840in}%
\pgfsys@useobject{currentmarker}{}%
\end{pgfscope}%
\begin{pgfscope}%
\pgfsys@transformshift{2.071629in}{0.800878in}%
\pgfsys@useobject{currentmarker}{}%
\end{pgfscope}%
\begin{pgfscope}%
\pgfsys@transformshift{2.071949in}{0.790472in}%
\pgfsys@useobject{currentmarker}{}%
\end{pgfscope}%
\begin{pgfscope}%
\pgfsys@transformshift{2.072270in}{0.747592in}%
\pgfsys@useobject{currentmarker}{}%
\end{pgfscope}%
\begin{pgfscope}%
\pgfsys@transformshift{2.072589in}{0.785789in}%
\pgfsys@useobject{currentmarker}{}%
\end{pgfscope}%
\begin{pgfscope}%
\pgfsys@transformshift{2.072909in}{0.787160in}%
\pgfsys@useobject{currentmarker}{}%
\end{pgfscope}%
\begin{pgfscope}%
\pgfsys@transformshift{2.073228in}{0.753818in}%
\pgfsys@useobject{currentmarker}{}%
\end{pgfscope}%
\begin{pgfscope}%
\pgfsys@transformshift{2.073546in}{0.765796in}%
\pgfsys@useobject{currentmarker}{}%
\end{pgfscope}%
\begin{pgfscope}%
\pgfsys@transformshift{2.073864in}{0.725444in}%
\pgfsys@useobject{currentmarker}{}%
\end{pgfscope}%
\begin{pgfscope}%
\pgfsys@transformshift{2.074182in}{0.767389in}%
\pgfsys@useobject{currentmarker}{}%
\end{pgfscope}%
\begin{pgfscope}%
\pgfsys@transformshift{2.074499in}{0.757386in}%
\pgfsys@useobject{currentmarker}{}%
\end{pgfscope}%
\begin{pgfscope}%
\pgfsys@transformshift{2.074815in}{0.739924in}%
\pgfsys@useobject{currentmarker}{}%
\end{pgfscope}%
\begin{pgfscope}%
\pgfsys@transformshift{2.075131in}{0.712449in}%
\pgfsys@useobject{currentmarker}{}%
\end{pgfscope}%
\begin{pgfscope}%
\pgfsys@transformshift{2.075447in}{0.771152in}%
\pgfsys@useobject{currentmarker}{}%
\end{pgfscope}%
\begin{pgfscope}%
\pgfsys@transformshift{2.075763in}{0.791699in}%
\pgfsys@useobject{currentmarker}{}%
\end{pgfscope}%
\begin{pgfscope}%
\pgfsys@transformshift{2.076077in}{0.778587in}%
\pgfsys@useobject{currentmarker}{}%
\end{pgfscope}%
\begin{pgfscope}%
\pgfsys@transformshift{2.076392in}{0.731044in}%
\pgfsys@useobject{currentmarker}{}%
\end{pgfscope}%
\begin{pgfscope}%
\pgfsys@transformshift{2.076706in}{0.719593in}%
\pgfsys@useobject{currentmarker}{}%
\end{pgfscope}%
\begin{pgfscope}%
\pgfsys@transformshift{2.077019in}{0.815000in}%
\pgfsys@useobject{currentmarker}{}%
\end{pgfscope}%
\begin{pgfscope}%
\pgfsys@transformshift{2.077332in}{0.810825in}%
\pgfsys@useobject{currentmarker}{}%
\end{pgfscope}%
\begin{pgfscope}%
\pgfsys@transformshift{2.077645in}{0.777372in}%
\pgfsys@useobject{currentmarker}{}%
\end{pgfscope}%
\begin{pgfscope}%
\pgfsys@transformshift{2.077957in}{0.761272in}%
\pgfsys@useobject{currentmarker}{}%
\end{pgfscope}%
\begin{pgfscope}%
\pgfsys@transformshift{2.078269in}{0.766970in}%
\pgfsys@useobject{currentmarker}{}%
\end{pgfscope}%
\begin{pgfscope}%
\pgfsys@transformshift{2.078581in}{0.785769in}%
\pgfsys@useobject{currentmarker}{}%
\end{pgfscope}%
\begin{pgfscope}%
\pgfsys@transformshift{2.078891in}{0.740198in}%
\pgfsys@useobject{currentmarker}{}%
\end{pgfscope}%
\begin{pgfscope}%
\pgfsys@transformshift{2.079202in}{0.800970in}%
\pgfsys@useobject{currentmarker}{}%
\end{pgfscope}%
\begin{pgfscope}%
\pgfsys@transformshift{2.079512in}{0.793799in}%
\pgfsys@useobject{currentmarker}{}%
\end{pgfscope}%
\begin{pgfscope}%
\pgfsys@transformshift{2.079822in}{0.807703in}%
\pgfsys@useobject{currentmarker}{}%
\end{pgfscope}%
\begin{pgfscope}%
\pgfsys@transformshift{2.080131in}{0.802201in}%
\pgfsys@useobject{currentmarker}{}%
\end{pgfscope}%
\begin{pgfscope}%
\pgfsys@transformshift{2.080440in}{0.746714in}%
\pgfsys@useobject{currentmarker}{}%
\end{pgfscope}%
\begin{pgfscope}%
\pgfsys@transformshift{2.080748in}{0.735326in}%
\pgfsys@useobject{currentmarker}{}%
\end{pgfscope}%
\begin{pgfscope}%
\pgfsys@transformshift{2.081056in}{0.751767in}%
\pgfsys@useobject{currentmarker}{}%
\end{pgfscope}%
\begin{pgfscope}%
\pgfsys@transformshift{2.081364in}{0.751211in}%
\pgfsys@useobject{currentmarker}{}%
\end{pgfscope}%
\begin{pgfscope}%
\pgfsys@transformshift{2.081671in}{0.733680in}%
\pgfsys@useobject{currentmarker}{}%
\end{pgfscope}%
\begin{pgfscope}%
\pgfsys@transformshift{2.081977in}{0.776505in}%
\pgfsys@useobject{currentmarker}{}%
\end{pgfscope}%
\begin{pgfscope}%
\pgfsys@transformshift{2.082284in}{0.786109in}%
\pgfsys@useobject{currentmarker}{}%
\end{pgfscope}%
\begin{pgfscope}%
\pgfsys@transformshift{2.082589in}{0.784557in}%
\pgfsys@useobject{currentmarker}{}%
\end{pgfscope}%
\begin{pgfscope}%
\pgfsys@transformshift{2.082895in}{0.781083in}%
\pgfsys@useobject{currentmarker}{}%
\end{pgfscope}%
\begin{pgfscope}%
\pgfsys@transformshift{2.083200in}{0.724494in}%
\pgfsys@useobject{currentmarker}{}%
\end{pgfscope}%
\begin{pgfscope}%
\pgfsys@transformshift{2.083505in}{0.754370in}%
\pgfsys@useobject{currentmarker}{}%
\end{pgfscope}%
\begin{pgfscope}%
\pgfsys@transformshift{2.083809in}{0.767576in}%
\pgfsys@useobject{currentmarker}{}%
\end{pgfscope}%
\begin{pgfscope}%
\pgfsys@transformshift{2.084113in}{0.722638in}%
\pgfsys@useobject{currentmarker}{}%
\end{pgfscope}%
\begin{pgfscope}%
\pgfsys@transformshift{2.084416in}{0.665345in}%
\pgfsys@useobject{currentmarker}{}%
\end{pgfscope}%
\begin{pgfscope}%
\pgfsys@transformshift{2.084719in}{0.773847in}%
\pgfsys@useobject{currentmarker}{}%
\end{pgfscope}%
\begin{pgfscope}%
\pgfsys@transformshift{2.085021in}{0.791958in}%
\pgfsys@useobject{currentmarker}{}%
\end{pgfscope}%
\begin{pgfscope}%
\pgfsys@transformshift{2.085324in}{0.770189in}%
\pgfsys@useobject{currentmarker}{}%
\end{pgfscope}%
\begin{pgfscope}%
\pgfsys@transformshift{2.085625in}{0.756130in}%
\pgfsys@useobject{currentmarker}{}%
\end{pgfscope}%
\begin{pgfscope}%
\pgfsys@transformshift{2.085927in}{0.759117in}%
\pgfsys@useobject{currentmarker}{}%
\end{pgfscope}%
\begin{pgfscope}%
\pgfsys@transformshift{2.086228in}{0.711966in}%
\pgfsys@useobject{currentmarker}{}%
\end{pgfscope}%
\begin{pgfscope}%
\pgfsys@transformshift{2.086528in}{0.743672in}%
\pgfsys@useobject{currentmarker}{}%
\end{pgfscope}%
\begin{pgfscope}%
\pgfsys@transformshift{2.086828in}{0.796596in}%
\pgfsys@useobject{currentmarker}{}%
\end{pgfscope}%
\begin{pgfscope}%
\pgfsys@transformshift{2.087128in}{0.758227in}%
\pgfsys@useobject{currentmarker}{}%
\end{pgfscope}%
\begin{pgfscope}%
\pgfsys@transformshift{2.087427in}{0.765954in}%
\pgfsys@useobject{currentmarker}{}%
\end{pgfscope}%
\begin{pgfscope}%
\pgfsys@transformshift{2.087726in}{0.793542in}%
\pgfsys@useobject{currentmarker}{}%
\end{pgfscope}%
\begin{pgfscope}%
\pgfsys@transformshift{2.088025in}{0.778983in}%
\pgfsys@useobject{currentmarker}{}%
\end{pgfscope}%
\begin{pgfscope}%
\pgfsys@transformshift{2.088323in}{0.776103in}%
\pgfsys@useobject{currentmarker}{}%
\end{pgfscope}%
\begin{pgfscope}%
\pgfsys@transformshift{2.088621in}{0.768464in}%
\pgfsys@useobject{currentmarker}{}%
\end{pgfscope}%
\begin{pgfscope}%
\pgfsys@transformshift{2.088918in}{0.717334in}%
\pgfsys@useobject{currentmarker}{}%
\end{pgfscope}%
\begin{pgfscope}%
\pgfsys@transformshift{2.089215in}{0.752356in}%
\pgfsys@useobject{currentmarker}{}%
\end{pgfscope}%
\begin{pgfscope}%
\pgfsys@transformshift{2.089512in}{0.752851in}%
\pgfsys@useobject{currentmarker}{}%
\end{pgfscope}%
\begin{pgfscope}%
\pgfsys@transformshift{2.089808in}{0.736706in}%
\pgfsys@useobject{currentmarker}{}%
\end{pgfscope}%
\begin{pgfscope}%
\pgfsys@transformshift{2.090104in}{0.786102in}%
\pgfsys@useobject{currentmarker}{}%
\end{pgfscope}%
\begin{pgfscope}%
\pgfsys@transformshift{2.090399in}{0.794158in}%
\pgfsys@useobject{currentmarker}{}%
\end{pgfscope}%
\begin{pgfscope}%
\pgfsys@transformshift{2.090694in}{0.705120in}%
\pgfsys@useobject{currentmarker}{}%
\end{pgfscope}%
\begin{pgfscope}%
\pgfsys@transformshift{2.090989in}{0.751445in}%
\pgfsys@useobject{currentmarker}{}%
\end{pgfscope}%
\begin{pgfscope}%
\pgfsys@transformshift{2.091283in}{0.710301in}%
\pgfsys@useobject{currentmarker}{}%
\end{pgfscope}%
\begin{pgfscope}%
\pgfsys@transformshift{2.091577in}{0.766715in}%
\pgfsys@useobject{currentmarker}{}%
\end{pgfscope}%
\begin{pgfscope}%
\pgfsys@transformshift{2.091870in}{0.763879in}%
\pgfsys@useobject{currentmarker}{}%
\end{pgfscope}%
\begin{pgfscope}%
\pgfsys@transformshift{2.092163in}{0.745374in}%
\pgfsys@useobject{currentmarker}{}%
\end{pgfscope}%
\begin{pgfscope}%
\pgfsys@transformshift{2.092456in}{0.768062in}%
\pgfsys@useobject{currentmarker}{}%
\end{pgfscope}%
\begin{pgfscope}%
\pgfsys@transformshift{2.092748in}{0.764653in}%
\pgfsys@useobject{currentmarker}{}%
\end{pgfscope}%
\begin{pgfscope}%
\pgfsys@transformshift{2.093040in}{0.711543in}%
\pgfsys@useobject{currentmarker}{}%
\end{pgfscope}%
\begin{pgfscope}%
\pgfsys@transformshift{2.093332in}{0.672255in}%
\pgfsys@useobject{currentmarker}{}%
\end{pgfscope}%
\begin{pgfscope}%
\pgfsys@transformshift{2.093623in}{0.748229in}%
\pgfsys@useobject{currentmarker}{}%
\end{pgfscope}%
\begin{pgfscope}%
\pgfsys@transformshift{2.093914in}{0.771720in}%
\pgfsys@useobject{currentmarker}{}%
\end{pgfscope}%
\begin{pgfscope}%
\pgfsys@transformshift{2.094204in}{0.776900in}%
\pgfsys@useobject{currentmarker}{}%
\end{pgfscope}%
\begin{pgfscope}%
\pgfsys@transformshift{2.094494in}{0.778248in}%
\pgfsys@useobject{currentmarker}{}%
\end{pgfscope}%
\begin{pgfscope}%
\pgfsys@transformshift{2.094784in}{0.752275in}%
\pgfsys@useobject{currentmarker}{}%
\end{pgfscope}%
\begin{pgfscope}%
\pgfsys@transformshift{2.095073in}{0.675818in}%
\pgfsys@useobject{currentmarker}{}%
\end{pgfscope}%
\begin{pgfscope}%
\pgfsys@transformshift{2.095362in}{0.728916in}%
\pgfsys@useobject{currentmarker}{}%
\end{pgfscope}%
\begin{pgfscope}%
\pgfsys@transformshift{2.095651in}{0.756213in}%
\pgfsys@useobject{currentmarker}{}%
\end{pgfscope}%
\begin{pgfscope}%
\pgfsys@transformshift{2.095939in}{0.780391in}%
\pgfsys@useobject{currentmarker}{}%
\end{pgfscope}%
\begin{pgfscope}%
\pgfsys@transformshift{2.096227in}{0.779043in}%
\pgfsys@useobject{currentmarker}{}%
\end{pgfscope}%
\begin{pgfscope}%
\pgfsys@transformshift{2.096514in}{0.762338in}%
\pgfsys@useobject{currentmarker}{}%
\end{pgfscope}%
\begin{pgfscope}%
\pgfsys@transformshift{2.096801in}{0.774066in}%
\pgfsys@useobject{currentmarker}{}%
\end{pgfscope}%
\begin{pgfscope}%
\pgfsys@transformshift{2.097088in}{0.786978in}%
\pgfsys@useobject{currentmarker}{}%
\end{pgfscope}%
\begin{pgfscope}%
\pgfsys@transformshift{2.097375in}{0.810369in}%
\pgfsys@useobject{currentmarker}{}%
\end{pgfscope}%
\begin{pgfscope}%
\pgfsys@transformshift{2.097661in}{0.802044in}%
\pgfsys@useobject{currentmarker}{}%
\end{pgfscope}%
\begin{pgfscope}%
\pgfsys@transformshift{2.097946in}{0.801201in}%
\pgfsys@useobject{currentmarker}{}%
\end{pgfscope}%
\begin{pgfscope}%
\pgfsys@transformshift{2.098231in}{0.766326in}%
\pgfsys@useobject{currentmarker}{}%
\end{pgfscope}%
\begin{pgfscope}%
\pgfsys@transformshift{2.098516in}{0.744037in}%
\pgfsys@useobject{currentmarker}{}%
\end{pgfscope}%
\begin{pgfscope}%
\pgfsys@transformshift{2.098801in}{0.762782in}%
\pgfsys@useobject{currentmarker}{}%
\end{pgfscope}%
\begin{pgfscope}%
\pgfsys@transformshift{2.099085in}{0.792498in}%
\pgfsys@useobject{currentmarker}{}%
\end{pgfscope}%
\begin{pgfscope}%
\pgfsys@transformshift{2.099369in}{0.805887in}%
\pgfsys@useobject{currentmarker}{}%
\end{pgfscope}%
\begin{pgfscope}%
\pgfsys@transformshift{2.099652in}{0.759720in}%
\pgfsys@useobject{currentmarker}{}%
\end{pgfscope}%
\begin{pgfscope}%
\pgfsys@transformshift{2.099936in}{0.744603in}%
\pgfsys@useobject{currentmarker}{}%
\end{pgfscope}%
\begin{pgfscope}%
\pgfsys@transformshift{2.100218in}{0.765481in}%
\pgfsys@useobject{currentmarker}{}%
\end{pgfscope}%
\begin{pgfscope}%
\pgfsys@transformshift{2.100501in}{0.773291in}%
\pgfsys@useobject{currentmarker}{}%
\end{pgfscope}%
\begin{pgfscope}%
\pgfsys@transformshift{2.100783in}{0.755713in}%
\pgfsys@useobject{currentmarker}{}%
\end{pgfscope}%
\begin{pgfscope}%
\pgfsys@transformshift{2.101064in}{0.785410in}%
\pgfsys@useobject{currentmarker}{}%
\end{pgfscope}%
\begin{pgfscope}%
\pgfsys@transformshift{2.101346in}{0.763174in}%
\pgfsys@useobject{currentmarker}{}%
\end{pgfscope}%
\begin{pgfscope}%
\pgfsys@transformshift{2.101627in}{0.737039in}%
\pgfsys@useobject{currentmarker}{}%
\end{pgfscope}%
\begin{pgfscope}%
\pgfsys@transformshift{2.101907in}{0.767780in}%
\pgfsys@useobject{currentmarker}{}%
\end{pgfscope}%
\begin{pgfscope}%
\pgfsys@transformshift{2.102188in}{0.759880in}%
\pgfsys@useobject{currentmarker}{}%
\end{pgfscope}%
\begin{pgfscope}%
\pgfsys@transformshift{2.102468in}{0.725297in}%
\pgfsys@useobject{currentmarker}{}%
\end{pgfscope}%
\begin{pgfscope}%
\pgfsys@transformshift{2.102747in}{0.717293in}%
\pgfsys@useobject{currentmarker}{}%
\end{pgfscope}%
\begin{pgfscope}%
\pgfsys@transformshift{2.103026in}{0.806572in}%
\pgfsys@useobject{currentmarker}{}%
\end{pgfscope}%
\begin{pgfscope}%
\pgfsys@transformshift{2.103305in}{0.795638in}%
\pgfsys@useobject{currentmarker}{}%
\end{pgfscope}%
\begin{pgfscope}%
\pgfsys@transformshift{2.103584in}{0.770456in}%
\pgfsys@useobject{currentmarker}{}%
\end{pgfscope}%
\begin{pgfscope}%
\pgfsys@transformshift{2.103862in}{0.771636in}%
\pgfsys@useobject{currentmarker}{}%
\end{pgfscope}%
\begin{pgfscope}%
\pgfsys@transformshift{2.104140in}{0.765007in}%
\pgfsys@useobject{currentmarker}{}%
\end{pgfscope}%
\begin{pgfscope}%
\pgfsys@transformshift{2.104417in}{0.674943in}%
\pgfsys@useobject{currentmarker}{}%
\end{pgfscope}%
\begin{pgfscope}%
\pgfsys@transformshift{2.104695in}{0.738306in}%
\pgfsys@useobject{currentmarker}{}%
\end{pgfscope}%
\begin{pgfscope}%
\pgfsys@transformshift{2.104972in}{0.763139in}%
\pgfsys@useobject{currentmarker}{}%
\end{pgfscope}%
\begin{pgfscope}%
\pgfsys@transformshift{2.105248in}{0.717427in}%
\pgfsys@useobject{currentmarker}{}%
\end{pgfscope}%
\begin{pgfscope}%
\pgfsys@transformshift{2.105524in}{0.707985in}%
\pgfsys@useobject{currentmarker}{}%
\end{pgfscope}%
\begin{pgfscope}%
\pgfsys@transformshift{2.105800in}{0.738873in}%
\pgfsys@useobject{currentmarker}{}%
\end{pgfscope}%
\begin{pgfscope}%
\pgfsys@transformshift{2.106075in}{0.771415in}%
\pgfsys@useobject{currentmarker}{}%
\end{pgfscope}%
\begin{pgfscope}%
\pgfsys@transformshift{2.106351in}{0.750304in}%
\pgfsys@useobject{currentmarker}{}%
\end{pgfscope}%
\begin{pgfscope}%
\pgfsys@transformshift{2.106625in}{0.761546in}%
\pgfsys@useobject{currentmarker}{}%
\end{pgfscope}%
\begin{pgfscope}%
\pgfsys@transformshift{2.106900in}{0.758811in}%
\pgfsys@useobject{currentmarker}{}%
\end{pgfscope}%
\begin{pgfscope}%
\pgfsys@transformshift{2.107174in}{0.747350in}%
\pgfsys@useobject{currentmarker}{}%
\end{pgfscope}%
\begin{pgfscope}%
\pgfsys@transformshift{2.107448in}{0.756182in}%
\pgfsys@useobject{currentmarker}{}%
\end{pgfscope}%
\begin{pgfscope}%
\pgfsys@transformshift{2.107721in}{0.733143in}%
\pgfsys@useobject{currentmarker}{}%
\end{pgfscope}%
\begin{pgfscope}%
\pgfsys@transformshift{2.107994in}{0.733651in}%
\pgfsys@useobject{currentmarker}{}%
\end{pgfscope}%
\begin{pgfscope}%
\pgfsys@transformshift{2.108267in}{0.744592in}%
\pgfsys@useobject{currentmarker}{}%
\end{pgfscope}%
\begin{pgfscope}%
\pgfsys@transformshift{2.108540in}{0.757944in}%
\pgfsys@useobject{currentmarker}{}%
\end{pgfscope}%
\begin{pgfscope}%
\pgfsys@transformshift{2.108812in}{0.721864in}%
\pgfsys@useobject{currentmarker}{}%
\end{pgfscope}%
\begin{pgfscope}%
\pgfsys@transformshift{2.109084in}{0.742827in}%
\pgfsys@useobject{currentmarker}{}%
\end{pgfscope}%
\begin{pgfscope}%
\pgfsys@transformshift{2.109355in}{0.796482in}%
\pgfsys@useobject{currentmarker}{}%
\end{pgfscope}%
\begin{pgfscope}%
\pgfsys@transformshift{2.109626in}{0.791489in}%
\pgfsys@useobject{currentmarker}{}%
\end{pgfscope}%
\begin{pgfscope}%
\pgfsys@transformshift{2.109897in}{0.743602in}%
\pgfsys@useobject{currentmarker}{}%
\end{pgfscope}%
\begin{pgfscope}%
\pgfsys@transformshift{2.110168in}{0.728055in}%
\pgfsys@useobject{currentmarker}{}%
\end{pgfscope}%
\begin{pgfscope}%
\pgfsys@transformshift{2.110438in}{0.750642in}%
\pgfsys@useobject{currentmarker}{}%
\end{pgfscope}%
\begin{pgfscope}%
\pgfsys@transformshift{2.110708in}{0.718617in}%
\pgfsys@useobject{currentmarker}{}%
\end{pgfscope}%
\begin{pgfscope}%
\pgfsys@transformshift{2.110977in}{0.668139in}%
\pgfsys@useobject{currentmarker}{}%
\end{pgfscope}%
\begin{pgfscope}%
\pgfsys@transformshift{2.111246in}{0.724750in}%
\pgfsys@useobject{currentmarker}{}%
\end{pgfscope}%
\begin{pgfscope}%
\pgfsys@transformshift{2.111515in}{0.775464in}%
\pgfsys@useobject{currentmarker}{}%
\end{pgfscope}%
\begin{pgfscope}%
\pgfsys@transformshift{2.111784in}{0.765697in}%
\pgfsys@useobject{currentmarker}{}%
\end{pgfscope}%
\begin{pgfscope}%
\pgfsys@transformshift{2.112052in}{0.766776in}%
\pgfsys@useobject{currentmarker}{}%
\end{pgfscope}%
\begin{pgfscope}%
\pgfsys@transformshift{2.112320in}{0.763934in}%
\pgfsys@useobject{currentmarker}{}%
\end{pgfscope}%
\begin{pgfscope}%
\pgfsys@transformshift{2.112588in}{0.745045in}%
\pgfsys@useobject{currentmarker}{}%
\end{pgfscope}%
\begin{pgfscope}%
\pgfsys@transformshift{2.112855in}{0.770596in}%
\pgfsys@useobject{currentmarker}{}%
\end{pgfscope}%
\begin{pgfscope}%
\pgfsys@transformshift{2.113122in}{0.737559in}%
\pgfsys@useobject{currentmarker}{}%
\end{pgfscope}%
\begin{pgfscope}%
\pgfsys@transformshift{2.113388in}{0.756427in}%
\pgfsys@useobject{currentmarker}{}%
\end{pgfscope}%
\begin{pgfscope}%
\pgfsys@transformshift{2.113655in}{0.745014in}%
\pgfsys@useobject{currentmarker}{}%
\end{pgfscope}%
\begin{pgfscope}%
\pgfsys@transformshift{2.113921in}{0.769913in}%
\pgfsys@useobject{currentmarker}{}%
\end{pgfscope}%
\begin{pgfscope}%
\pgfsys@transformshift{2.114187in}{0.808915in}%
\pgfsys@useobject{currentmarker}{}%
\end{pgfscope}%
\begin{pgfscope}%
\pgfsys@transformshift{2.114452in}{0.784133in}%
\pgfsys@useobject{currentmarker}{}%
\end{pgfscope}%
\begin{pgfscope}%
\pgfsys@transformshift{2.114717in}{0.722321in}%
\pgfsys@useobject{currentmarker}{}%
\end{pgfscope}%
\begin{pgfscope}%
\pgfsys@transformshift{2.114982in}{0.695509in}%
\pgfsys@useobject{currentmarker}{}%
\end{pgfscope}%
\begin{pgfscope}%
\pgfsys@transformshift{2.115246in}{0.722933in}%
\pgfsys@useobject{currentmarker}{}%
\end{pgfscope}%
\begin{pgfscope}%
\pgfsys@transformshift{2.115510in}{0.771400in}%
\pgfsys@useobject{currentmarker}{}%
\end{pgfscope}%
\begin{pgfscope}%
\pgfsys@transformshift{2.115774in}{0.745901in}%
\pgfsys@useobject{currentmarker}{}%
\end{pgfscope}%
\begin{pgfscope}%
\pgfsys@transformshift{2.116038in}{0.675524in}%
\pgfsys@useobject{currentmarker}{}%
\end{pgfscope}%
\begin{pgfscope}%
\pgfsys@transformshift{2.116301in}{0.690438in}%
\pgfsys@useobject{currentmarker}{}%
\end{pgfscope}%
\begin{pgfscope}%
\pgfsys@transformshift{2.116564in}{0.694552in}%
\pgfsys@useobject{currentmarker}{}%
\end{pgfscope}%
\begin{pgfscope}%
\pgfsys@transformshift{2.116826in}{0.735613in}%
\pgfsys@useobject{currentmarker}{}%
\end{pgfscope}%
\begin{pgfscope}%
\pgfsys@transformshift{2.117089in}{0.745796in}%
\pgfsys@useobject{currentmarker}{}%
\end{pgfscope}%
\begin{pgfscope}%
\pgfsys@transformshift{2.117351in}{0.757476in}%
\pgfsys@useobject{currentmarker}{}%
\end{pgfscope}%
\begin{pgfscope}%
\pgfsys@transformshift{2.117612in}{0.775756in}%
\pgfsys@useobject{currentmarker}{}%
\end{pgfscope}%
\begin{pgfscope}%
\pgfsys@transformshift{2.117874in}{0.744567in}%
\pgfsys@useobject{currentmarker}{}%
\end{pgfscope}%
\begin{pgfscope}%
\pgfsys@transformshift{2.118135in}{0.734748in}%
\pgfsys@useobject{currentmarker}{}%
\end{pgfscope}%
\begin{pgfscope}%
\pgfsys@transformshift{2.118396in}{0.725906in}%
\pgfsys@useobject{currentmarker}{}%
\end{pgfscope}%
\begin{pgfscope}%
\pgfsys@transformshift{2.118656in}{0.732736in}%
\pgfsys@useobject{currentmarker}{}%
\end{pgfscope}%
\begin{pgfscope}%
\pgfsys@transformshift{2.118916in}{0.719722in}%
\pgfsys@useobject{currentmarker}{}%
\end{pgfscope}%
\begin{pgfscope}%
\pgfsys@transformshift{2.119176in}{0.706597in}%
\pgfsys@useobject{currentmarker}{}%
\end{pgfscope}%
\begin{pgfscope}%
\pgfsys@transformshift{2.119436in}{0.717115in}%
\pgfsys@useobject{currentmarker}{}%
\end{pgfscope}%
\begin{pgfscope}%
\pgfsys@transformshift{2.119695in}{0.710174in}%
\pgfsys@useobject{currentmarker}{}%
\end{pgfscope}%
\begin{pgfscope}%
\pgfsys@transformshift{2.119954in}{0.701083in}%
\pgfsys@useobject{currentmarker}{}%
\end{pgfscope}%
\begin{pgfscope}%
\pgfsys@transformshift{2.120213in}{0.738727in}%
\pgfsys@useobject{currentmarker}{}%
\end{pgfscope}%
\begin{pgfscope}%
\pgfsys@transformshift{2.120471in}{0.754360in}%
\pgfsys@useobject{currentmarker}{}%
\end{pgfscope}%
\begin{pgfscope}%
\pgfsys@transformshift{2.120729in}{0.754275in}%
\pgfsys@useobject{currentmarker}{}%
\end{pgfscope}%
\begin{pgfscope}%
\pgfsys@transformshift{2.120987in}{0.733179in}%
\pgfsys@useobject{currentmarker}{}%
\end{pgfscope}%
\begin{pgfscope}%
\pgfsys@transformshift{2.121244in}{0.755059in}%
\pgfsys@useobject{currentmarker}{}%
\end{pgfscope}%
\begin{pgfscope}%
\pgfsys@transformshift{2.121501in}{0.791621in}%
\pgfsys@useobject{currentmarker}{}%
\end{pgfscope}%
\begin{pgfscope}%
\pgfsys@transformshift{2.121758in}{0.787327in}%
\pgfsys@useobject{currentmarker}{}%
\end{pgfscope}%
\begin{pgfscope}%
\pgfsys@transformshift{2.122015in}{0.743552in}%
\pgfsys@useobject{currentmarker}{}%
\end{pgfscope}%
\begin{pgfscope}%
\pgfsys@transformshift{2.122271in}{0.738259in}%
\pgfsys@useobject{currentmarker}{}%
\end{pgfscope}%
\begin{pgfscope}%
\pgfsys@transformshift{2.122527in}{0.782943in}%
\pgfsys@useobject{currentmarker}{}%
\end{pgfscope}%
\begin{pgfscope}%
\pgfsys@transformshift{2.122783in}{0.781467in}%
\pgfsys@useobject{currentmarker}{}%
\end{pgfscope}%
\begin{pgfscope}%
\pgfsys@transformshift{2.123038in}{0.771210in}%
\pgfsys@useobject{currentmarker}{}%
\end{pgfscope}%
\begin{pgfscope}%
\pgfsys@transformshift{2.123293in}{0.709657in}%
\pgfsys@useobject{currentmarker}{}%
\end{pgfscope}%
\begin{pgfscope}%
\pgfsys@transformshift{2.123548in}{0.739582in}%
\pgfsys@useobject{currentmarker}{}%
\end{pgfscope}%
\begin{pgfscope}%
\pgfsys@transformshift{2.123803in}{0.757784in}%
\pgfsys@useobject{currentmarker}{}%
\end{pgfscope}%
\begin{pgfscope}%
\pgfsys@transformshift{2.124057in}{0.748578in}%
\pgfsys@useobject{currentmarker}{}%
\end{pgfscope}%
\begin{pgfscope}%
\pgfsys@transformshift{2.124311in}{0.722612in}%
\pgfsys@useobject{currentmarker}{}%
\end{pgfscope}%
\begin{pgfscope}%
\pgfsys@transformshift{2.124565in}{0.731054in}%
\pgfsys@useobject{currentmarker}{}%
\end{pgfscope}%
\begin{pgfscope}%
\pgfsys@transformshift{2.124818in}{0.761786in}%
\pgfsys@useobject{currentmarker}{}%
\end{pgfscope}%
\begin{pgfscope}%
\pgfsys@transformshift{2.125071in}{0.768314in}%
\pgfsys@useobject{currentmarker}{}%
\end{pgfscope}%
\begin{pgfscope}%
\pgfsys@transformshift{2.125324in}{0.758108in}%
\pgfsys@useobject{currentmarker}{}%
\end{pgfscope}%
\begin{pgfscope}%
\pgfsys@transformshift{2.125577in}{0.719701in}%
\pgfsys@useobject{currentmarker}{}%
\end{pgfscope}%
\begin{pgfscope}%
\pgfsys@transformshift{2.125829in}{0.762119in}%
\pgfsys@useobject{currentmarker}{}%
\end{pgfscope}%
\begin{pgfscope}%
\pgfsys@transformshift{2.126081in}{0.734965in}%
\pgfsys@useobject{currentmarker}{}%
\end{pgfscope}%
\begin{pgfscope}%
\pgfsys@transformshift{2.126333in}{0.745933in}%
\pgfsys@useobject{currentmarker}{}%
\end{pgfscope}%
\begin{pgfscope}%
\pgfsys@transformshift{2.126584in}{0.773927in}%
\pgfsys@useobject{currentmarker}{}%
\end{pgfscope}%
\begin{pgfscope}%
\pgfsys@transformshift{2.126835in}{0.749195in}%
\pgfsys@useobject{currentmarker}{}%
\end{pgfscope}%
\begin{pgfscope}%
\pgfsys@transformshift{2.127086in}{0.726377in}%
\pgfsys@useobject{currentmarker}{}%
\end{pgfscope}%
\begin{pgfscope}%
\pgfsys@transformshift{2.127337in}{0.739536in}%
\pgfsys@useobject{currentmarker}{}%
\end{pgfscope}%
\begin{pgfscope}%
\pgfsys@transformshift{2.127587in}{0.742369in}%
\pgfsys@useobject{currentmarker}{}%
\end{pgfscope}%
\begin{pgfscope}%
\pgfsys@transformshift{2.127837in}{0.692092in}%
\pgfsys@useobject{currentmarker}{}%
\end{pgfscope}%
\begin{pgfscope}%
\pgfsys@transformshift{2.128087in}{0.583372in}%
\pgfsys@useobject{currentmarker}{}%
\end{pgfscope}%
\begin{pgfscope}%
\pgfsys@transformshift{2.128336in}{0.687074in}%
\pgfsys@useobject{currentmarker}{}%
\end{pgfscope}%
\begin{pgfscope}%
\pgfsys@transformshift{2.128586in}{0.700639in}%
\pgfsys@useobject{currentmarker}{}%
\end{pgfscope}%
\begin{pgfscope}%
\pgfsys@transformshift{2.128835in}{0.712495in}%
\pgfsys@useobject{currentmarker}{}%
\end{pgfscope}%
\begin{pgfscope}%
\pgfsys@transformshift{2.129083in}{0.733748in}%
\pgfsys@useobject{currentmarker}{}%
\end{pgfscope}%
\begin{pgfscope}%
\pgfsys@transformshift{2.129332in}{0.757149in}%
\pgfsys@useobject{currentmarker}{}%
\end{pgfscope}%
\begin{pgfscope}%
\pgfsys@transformshift{2.129580in}{0.755118in}%
\pgfsys@useobject{currentmarker}{}%
\end{pgfscope}%
\begin{pgfscope}%
\pgfsys@transformshift{2.129827in}{0.740099in}%
\pgfsys@useobject{currentmarker}{}%
\end{pgfscope}%
\begin{pgfscope}%
\pgfsys@transformshift{2.130075in}{0.757521in}%
\pgfsys@useobject{currentmarker}{}%
\end{pgfscope}%
\begin{pgfscope}%
\pgfsys@transformshift{2.130322in}{0.714419in}%
\pgfsys@useobject{currentmarker}{}%
\end{pgfscope}%
\begin{pgfscope}%
\pgfsys@transformshift{2.130569in}{0.692725in}%
\pgfsys@useobject{currentmarker}{}%
\end{pgfscope}%
\begin{pgfscope}%
\pgfsys@transformshift{2.130816in}{0.671637in}%
\pgfsys@useobject{currentmarker}{}%
\end{pgfscope}%
\begin{pgfscope}%
\pgfsys@transformshift{2.131062in}{0.712582in}%
\pgfsys@useobject{currentmarker}{}%
\end{pgfscope}%
\begin{pgfscope}%
\pgfsys@transformshift{2.131309in}{0.722571in}%
\pgfsys@useobject{currentmarker}{}%
\end{pgfscope}%
\begin{pgfscope}%
\pgfsys@transformshift{2.131555in}{0.728320in}%
\pgfsys@useobject{currentmarker}{}%
\end{pgfscope}%
\begin{pgfscope}%
\pgfsys@transformshift{2.131800in}{0.745514in}%
\pgfsys@useobject{currentmarker}{}%
\end{pgfscope}%
\begin{pgfscope}%
\pgfsys@transformshift{2.132046in}{0.718325in}%
\pgfsys@useobject{currentmarker}{}%
\end{pgfscope}%
\begin{pgfscope}%
\pgfsys@transformshift{2.132291in}{0.739599in}%
\pgfsys@useobject{currentmarker}{}%
\end{pgfscope}%
\begin{pgfscope}%
\pgfsys@transformshift{2.132536in}{0.764963in}%
\pgfsys@useobject{currentmarker}{}%
\end{pgfscope}%
\begin{pgfscope}%
\pgfsys@transformshift{2.132780in}{0.741206in}%
\pgfsys@useobject{currentmarker}{}%
\end{pgfscope}%
\begin{pgfscope}%
\pgfsys@transformshift{2.133025in}{0.693069in}%
\pgfsys@useobject{currentmarker}{}%
\end{pgfscope}%
\begin{pgfscope}%
\pgfsys@transformshift{2.133269in}{0.729560in}%
\pgfsys@useobject{currentmarker}{}%
\end{pgfscope}%
\begin{pgfscope}%
\pgfsys@transformshift{2.133512in}{0.698952in}%
\pgfsys@useobject{currentmarker}{}%
\end{pgfscope}%
\begin{pgfscope}%
\pgfsys@transformshift{2.133756in}{0.714272in}%
\pgfsys@useobject{currentmarker}{}%
\end{pgfscope}%
\begin{pgfscope}%
\pgfsys@transformshift{2.133999in}{0.739100in}%
\pgfsys@useobject{currentmarker}{}%
\end{pgfscope}%
\begin{pgfscope}%
\pgfsys@transformshift{2.134242in}{0.788243in}%
\pgfsys@useobject{currentmarker}{}%
\end{pgfscope}%
\begin{pgfscope}%
\pgfsys@transformshift{2.134485in}{0.766811in}%
\pgfsys@useobject{currentmarker}{}%
\end{pgfscope}%
\begin{pgfscope}%
\pgfsys@transformshift{2.134727in}{0.742117in}%
\pgfsys@useobject{currentmarker}{}%
\end{pgfscope}%
\begin{pgfscope}%
\pgfsys@transformshift{2.134970in}{0.772503in}%
\pgfsys@useobject{currentmarker}{}%
\end{pgfscope}%
\begin{pgfscope}%
\pgfsys@transformshift{2.135212in}{0.755034in}%
\pgfsys@useobject{currentmarker}{}%
\end{pgfscope}%
\begin{pgfscope}%
\pgfsys@transformshift{2.135453in}{0.756957in}%
\pgfsys@useobject{currentmarker}{}%
\end{pgfscope}%
\begin{pgfscope}%
\pgfsys@transformshift{2.135695in}{0.747327in}%
\pgfsys@useobject{currentmarker}{}%
\end{pgfscope}%
\begin{pgfscope}%
\pgfsys@transformshift{2.135936in}{0.745524in}%
\pgfsys@useobject{currentmarker}{}%
\end{pgfscope}%
\begin{pgfscope}%
\pgfsys@transformshift{2.136177in}{0.743153in}%
\pgfsys@useobject{currentmarker}{}%
\end{pgfscope}%
\begin{pgfscope}%
\pgfsys@transformshift{2.136417in}{0.784306in}%
\pgfsys@useobject{currentmarker}{}%
\end{pgfscope}%
\begin{pgfscope}%
\pgfsys@transformshift{2.136658in}{0.756540in}%
\pgfsys@useobject{currentmarker}{}%
\end{pgfscope}%
\begin{pgfscope}%
\pgfsys@transformshift{2.136898in}{0.734460in}%
\pgfsys@useobject{currentmarker}{}%
\end{pgfscope}%
\begin{pgfscope}%
\pgfsys@transformshift{2.137138in}{0.741341in}%
\pgfsys@useobject{currentmarker}{}%
\end{pgfscope}%
\begin{pgfscope}%
\pgfsys@transformshift{2.137377in}{0.710754in}%
\pgfsys@useobject{currentmarker}{}%
\end{pgfscope}%
\begin{pgfscope}%
\pgfsys@transformshift{2.137617in}{0.694449in}%
\pgfsys@useobject{currentmarker}{}%
\end{pgfscope}%
\begin{pgfscope}%
\pgfsys@transformshift{2.137856in}{0.678078in}%
\pgfsys@useobject{currentmarker}{}%
\end{pgfscope}%
\begin{pgfscope}%
\pgfsys@transformshift{2.138095in}{0.723370in}%
\pgfsys@useobject{currentmarker}{}%
\end{pgfscope}%
\begin{pgfscope}%
\pgfsys@transformshift{2.138333in}{0.715188in}%
\pgfsys@useobject{currentmarker}{}%
\end{pgfscope}%
\begin{pgfscope}%
\pgfsys@transformshift{2.138572in}{0.727830in}%
\pgfsys@useobject{currentmarker}{}%
\end{pgfscope}%
\begin{pgfscope}%
\pgfsys@transformshift{2.138810in}{0.747185in}%
\pgfsys@useobject{currentmarker}{}%
\end{pgfscope}%
\begin{pgfscope}%
\pgfsys@transformshift{2.139048in}{0.741907in}%
\pgfsys@useobject{currentmarker}{}%
\end{pgfscope}%
\begin{pgfscope}%
\pgfsys@transformshift{2.139285in}{0.745801in}%
\pgfsys@useobject{currentmarker}{}%
\end{pgfscope}%
\begin{pgfscope}%
\pgfsys@transformshift{2.139523in}{0.733556in}%
\pgfsys@useobject{currentmarker}{}%
\end{pgfscope}%
\begin{pgfscope}%
\pgfsys@transformshift{2.139760in}{0.769085in}%
\pgfsys@useobject{currentmarker}{}%
\end{pgfscope}%
\begin{pgfscope}%
\pgfsys@transformshift{2.139997in}{0.747306in}%
\pgfsys@useobject{currentmarker}{}%
\end{pgfscope}%
\begin{pgfscope}%
\pgfsys@transformshift{2.140233in}{0.717067in}%
\pgfsys@useobject{currentmarker}{}%
\end{pgfscope}%
\begin{pgfscope}%
\pgfsys@transformshift{2.140470in}{0.682604in}%
\pgfsys@useobject{currentmarker}{}%
\end{pgfscope}%
\begin{pgfscope}%
\pgfsys@transformshift{2.140706in}{0.677891in}%
\pgfsys@useobject{currentmarker}{}%
\end{pgfscope}%
\begin{pgfscope}%
\pgfsys@transformshift{2.140942in}{0.696633in}%
\pgfsys@useobject{currentmarker}{}%
\end{pgfscope}%
\begin{pgfscope}%
\pgfsys@transformshift{2.141177in}{0.686876in}%
\pgfsys@useobject{currentmarker}{}%
\end{pgfscope}%
\begin{pgfscope}%
\pgfsys@transformshift{2.141412in}{0.755205in}%
\pgfsys@useobject{currentmarker}{}%
\end{pgfscope}%
\begin{pgfscope}%
\pgfsys@transformshift{2.141648in}{0.759415in}%
\pgfsys@useobject{currentmarker}{}%
\end{pgfscope}%
\begin{pgfscope}%
\pgfsys@transformshift{2.141882in}{0.724966in}%
\pgfsys@useobject{currentmarker}{}%
\end{pgfscope}%
\begin{pgfscope}%
\pgfsys@transformshift{2.142117in}{0.723721in}%
\pgfsys@useobject{currentmarker}{}%
\end{pgfscope}%
\begin{pgfscope}%
\pgfsys@transformshift{2.142351in}{0.717700in}%
\pgfsys@useobject{currentmarker}{}%
\end{pgfscope}%
\begin{pgfscope}%
\pgfsys@transformshift{2.142586in}{0.678488in}%
\pgfsys@useobject{currentmarker}{}%
\end{pgfscope}%
\begin{pgfscope}%
\pgfsys@transformshift{2.142820in}{0.688144in}%
\pgfsys@useobject{currentmarker}{}%
\end{pgfscope}%
\begin{pgfscope}%
\pgfsys@transformshift{2.143053in}{0.729649in}%
\pgfsys@useobject{currentmarker}{}%
\end{pgfscope}%
\begin{pgfscope}%
\pgfsys@transformshift{2.143287in}{0.738245in}%
\pgfsys@useobject{currentmarker}{}%
\end{pgfscope}%
\begin{pgfscope}%
\pgfsys@transformshift{2.143520in}{0.751378in}%
\pgfsys@useobject{currentmarker}{}%
\end{pgfscope}%
\begin{pgfscope}%
\pgfsys@transformshift{2.143753in}{0.742542in}%
\pgfsys@useobject{currentmarker}{}%
\end{pgfscope}%
\begin{pgfscope}%
\pgfsys@transformshift{2.143985in}{0.727847in}%
\pgfsys@useobject{currentmarker}{}%
\end{pgfscope}%
\begin{pgfscope}%
\pgfsys@transformshift{2.144218in}{0.782439in}%
\pgfsys@useobject{currentmarker}{}%
\end{pgfscope}%
\begin{pgfscope}%
\pgfsys@transformshift{2.144450in}{0.777934in}%
\pgfsys@useobject{currentmarker}{}%
\end{pgfscope}%
\begin{pgfscope}%
\pgfsys@transformshift{2.144682in}{0.727502in}%
\pgfsys@useobject{currentmarker}{}%
\end{pgfscope}%
\begin{pgfscope}%
\pgfsys@transformshift{2.144914in}{0.724140in}%
\pgfsys@useobject{currentmarker}{}%
\end{pgfscope}%
\begin{pgfscope}%
\pgfsys@transformshift{2.145145in}{0.671094in}%
\pgfsys@useobject{currentmarker}{}%
\end{pgfscope}%
\begin{pgfscope}%
\pgfsys@transformshift{2.145376in}{0.679968in}%
\pgfsys@useobject{currentmarker}{}%
\end{pgfscope}%
\begin{pgfscope}%
\pgfsys@transformshift{2.145607in}{0.718964in}%
\pgfsys@useobject{currentmarker}{}%
\end{pgfscope}%
\begin{pgfscope}%
\pgfsys@transformshift{2.145838in}{0.738431in}%
\pgfsys@useobject{currentmarker}{}%
\end{pgfscope}%
\begin{pgfscope}%
\pgfsys@transformshift{2.146069in}{0.765222in}%
\pgfsys@useobject{currentmarker}{}%
\end{pgfscope}%
\begin{pgfscope}%
\pgfsys@transformshift{2.146299in}{0.742933in}%
\pgfsys@useobject{currentmarker}{}%
\end{pgfscope}%
\begin{pgfscope}%
\pgfsys@transformshift{2.146529in}{0.735651in}%
\pgfsys@useobject{currentmarker}{}%
\end{pgfscope}%
\begin{pgfscope}%
\pgfsys@transformshift{2.146759in}{0.708925in}%
\pgfsys@useobject{currentmarker}{}%
\end{pgfscope}%
\begin{pgfscope}%
\pgfsys@transformshift{2.146988in}{0.657875in}%
\pgfsys@useobject{currentmarker}{}%
\end{pgfscope}%
\begin{pgfscope}%
\pgfsys@transformshift{2.147218in}{0.649352in}%
\pgfsys@useobject{currentmarker}{}%
\end{pgfscope}%
\begin{pgfscope}%
\pgfsys@transformshift{2.147447in}{0.691531in}%
\pgfsys@useobject{currentmarker}{}%
\end{pgfscope}%
\begin{pgfscope}%
\pgfsys@transformshift{2.147676in}{0.698453in}%
\pgfsys@useobject{currentmarker}{}%
\end{pgfscope}%
\begin{pgfscope}%
\pgfsys@transformshift{2.147904in}{0.670853in}%
\pgfsys@useobject{currentmarker}{}%
\end{pgfscope}%
\begin{pgfscope}%
\pgfsys@transformshift{2.148133in}{0.702732in}%
\pgfsys@useobject{currentmarker}{}%
\end{pgfscope}%
\begin{pgfscope}%
\pgfsys@transformshift{2.148361in}{0.719129in}%
\pgfsys@useobject{currentmarker}{}%
\end{pgfscope}%
\begin{pgfscope}%
\pgfsys@transformshift{2.148589in}{0.720260in}%
\pgfsys@useobject{currentmarker}{}%
\end{pgfscope}%
\begin{pgfscope}%
\pgfsys@transformshift{2.148817in}{0.737484in}%
\pgfsys@useobject{currentmarker}{}%
\end{pgfscope}%
\begin{pgfscope}%
\pgfsys@transformshift{2.149044in}{0.687334in}%
\pgfsys@useobject{currentmarker}{}%
\end{pgfscope}%
\begin{pgfscope}%
\pgfsys@transformshift{2.149271in}{0.672022in}%
\pgfsys@useobject{currentmarker}{}%
\end{pgfscope}%
\begin{pgfscope}%
\pgfsys@transformshift{2.149499in}{0.718057in}%
\pgfsys@useobject{currentmarker}{}%
\end{pgfscope}%
\begin{pgfscope}%
\pgfsys@transformshift{2.149725in}{0.735866in}%
\pgfsys@useobject{currentmarker}{}%
\end{pgfscope}%
\begin{pgfscope}%
\pgfsys@transformshift{2.149952in}{0.719864in}%
\pgfsys@useobject{currentmarker}{}%
\end{pgfscope}%
\begin{pgfscope}%
\pgfsys@transformshift{2.150178in}{0.715389in}%
\pgfsys@useobject{currentmarker}{}%
\end{pgfscope}%
\begin{pgfscope}%
\pgfsys@transformshift{2.150404in}{0.732510in}%
\pgfsys@useobject{currentmarker}{}%
\end{pgfscope}%
\begin{pgfscope}%
\pgfsys@transformshift{2.150630in}{0.759267in}%
\pgfsys@useobject{currentmarker}{}%
\end{pgfscope}%
\begin{pgfscope}%
\pgfsys@transformshift{2.150856in}{0.772014in}%
\pgfsys@useobject{currentmarker}{}%
\end{pgfscope}%
\begin{pgfscope}%
\pgfsys@transformshift{2.151081in}{0.739052in}%
\pgfsys@useobject{currentmarker}{}%
\end{pgfscope}%
\begin{pgfscope}%
\pgfsys@transformshift{2.151307in}{0.744961in}%
\pgfsys@useobject{currentmarker}{}%
\end{pgfscope}%
\begin{pgfscope}%
\pgfsys@transformshift{2.151532in}{0.756346in}%
\pgfsys@useobject{currentmarker}{}%
\end{pgfscope}%
\begin{pgfscope}%
\pgfsys@transformshift{2.151756in}{0.786215in}%
\pgfsys@useobject{currentmarker}{}%
\end{pgfscope}%
\begin{pgfscope}%
\pgfsys@transformshift{2.151981in}{0.775746in}%
\pgfsys@useobject{currentmarker}{}%
\end{pgfscope}%
\begin{pgfscope}%
\pgfsys@transformshift{2.152205in}{0.736112in}%
\pgfsys@useobject{currentmarker}{}%
\end{pgfscope}%
\begin{pgfscope}%
\pgfsys@transformshift{2.152429in}{0.726876in}%
\pgfsys@useobject{currentmarker}{}%
\end{pgfscope}%
\begin{pgfscope}%
\pgfsys@transformshift{2.152653in}{0.734080in}%
\pgfsys@useobject{currentmarker}{}%
\end{pgfscope}%
\begin{pgfscope}%
\pgfsys@transformshift{2.152877in}{0.739071in}%
\pgfsys@useobject{currentmarker}{}%
\end{pgfscope}%
\begin{pgfscope}%
\pgfsys@transformshift{2.153100in}{0.745069in}%
\pgfsys@useobject{currentmarker}{}%
\end{pgfscope}%
\begin{pgfscope}%
\pgfsys@transformshift{2.153323in}{0.745703in}%
\pgfsys@useobject{currentmarker}{}%
\end{pgfscope}%
\begin{pgfscope}%
\pgfsys@transformshift{2.153546in}{0.724801in}%
\pgfsys@useobject{currentmarker}{}%
\end{pgfscope}%
\begin{pgfscope}%
\pgfsys@transformshift{2.153769in}{0.715722in}%
\pgfsys@useobject{currentmarker}{}%
\end{pgfscope}%
\begin{pgfscope}%
\pgfsys@transformshift{2.153992in}{0.698345in}%
\pgfsys@useobject{currentmarker}{}%
\end{pgfscope}%
\begin{pgfscope}%
\pgfsys@transformshift{2.154214in}{0.657475in}%
\pgfsys@useobject{currentmarker}{}%
\end{pgfscope}%
\begin{pgfscope}%
\pgfsys@transformshift{2.154436in}{0.664817in}%
\pgfsys@useobject{currentmarker}{}%
\end{pgfscope}%
\begin{pgfscope}%
\pgfsys@transformshift{2.154658in}{0.680556in}%
\pgfsys@useobject{currentmarker}{}%
\end{pgfscope}%
\begin{pgfscope}%
\pgfsys@transformshift{2.154880in}{0.672700in}%
\pgfsys@useobject{currentmarker}{}%
\end{pgfscope}%
\begin{pgfscope}%
\pgfsys@transformshift{2.155101in}{0.647495in}%
\pgfsys@useobject{currentmarker}{}%
\end{pgfscope}%
\begin{pgfscope}%
\pgfsys@transformshift{2.155322in}{0.682551in}%
\pgfsys@useobject{currentmarker}{}%
\end{pgfscope}%
\begin{pgfscope}%
\pgfsys@transformshift{2.155543in}{0.699665in}%
\pgfsys@useobject{currentmarker}{}%
\end{pgfscope}%
\begin{pgfscope}%
\pgfsys@transformshift{2.155764in}{0.728117in}%
\pgfsys@useobject{currentmarker}{}%
\end{pgfscope}%
\begin{pgfscope}%
\pgfsys@transformshift{2.155985in}{0.738862in}%
\pgfsys@useobject{currentmarker}{}%
\end{pgfscope}%
\begin{pgfscope}%
\pgfsys@transformshift{2.156205in}{0.756020in}%
\pgfsys@useobject{currentmarker}{}%
\end{pgfscope}%
\begin{pgfscope}%
\pgfsys@transformshift{2.156425in}{0.728128in}%
\pgfsys@useobject{currentmarker}{}%
\end{pgfscope}%
\begin{pgfscope}%
\pgfsys@transformshift{2.156645in}{0.706383in}%
\pgfsys@useobject{currentmarker}{}%
\end{pgfscope}%
\begin{pgfscope}%
\pgfsys@transformshift{2.156865in}{0.744930in}%
\pgfsys@useobject{currentmarker}{}%
\end{pgfscope}%
\begin{pgfscope}%
\pgfsys@transformshift{2.157084in}{0.736139in}%
\pgfsys@useobject{currentmarker}{}%
\end{pgfscope}%
\begin{pgfscope}%
\pgfsys@transformshift{2.157304in}{0.678594in}%
\pgfsys@useobject{currentmarker}{}%
\end{pgfscope}%
\begin{pgfscope}%
\pgfsys@transformshift{2.157523in}{0.720768in}%
\pgfsys@useobject{currentmarker}{}%
\end{pgfscope}%
\begin{pgfscope}%
\pgfsys@transformshift{2.157741in}{0.745489in}%
\pgfsys@useobject{currentmarker}{}%
\end{pgfscope}%
\begin{pgfscope}%
\pgfsys@transformshift{2.157960in}{0.757569in}%
\pgfsys@useobject{currentmarker}{}%
\end{pgfscope}%
\begin{pgfscope}%
\pgfsys@transformshift{2.158179in}{0.743203in}%
\pgfsys@useobject{currentmarker}{}%
\end{pgfscope}%
\begin{pgfscope}%
\pgfsys@transformshift{2.158397in}{0.707806in}%
\pgfsys@useobject{currentmarker}{}%
\end{pgfscope}%
\begin{pgfscope}%
\pgfsys@transformshift{2.158615in}{0.706890in}%
\pgfsys@useobject{currentmarker}{}%
\end{pgfscope}%
\begin{pgfscope}%
\pgfsys@transformshift{2.158833in}{0.696165in}%
\pgfsys@useobject{currentmarker}{}%
\end{pgfscope}%
\begin{pgfscope}%
\pgfsys@transformshift{2.159050in}{0.712572in}%
\pgfsys@useobject{currentmarker}{}%
\end{pgfscope}%
\begin{pgfscope}%
\pgfsys@transformshift{2.159268in}{0.746496in}%
\pgfsys@useobject{currentmarker}{}%
\end{pgfscope}%
\begin{pgfscope}%
\pgfsys@transformshift{2.159485in}{0.726192in}%
\pgfsys@useobject{currentmarker}{}%
\end{pgfscope}%
\begin{pgfscope}%
\pgfsys@transformshift{2.159702in}{0.727406in}%
\pgfsys@useobject{currentmarker}{}%
\end{pgfscope}%
\begin{pgfscope}%
\pgfsys@transformshift{2.159918in}{0.738646in}%
\pgfsys@useobject{currentmarker}{}%
\end{pgfscope}%
\begin{pgfscope}%
\pgfsys@transformshift{2.160135in}{0.740001in}%
\pgfsys@useobject{currentmarker}{}%
\end{pgfscope}%
\begin{pgfscope}%
\pgfsys@transformshift{2.160351in}{0.723584in}%
\pgfsys@useobject{currentmarker}{}%
\end{pgfscope}%
\begin{pgfscope}%
\pgfsys@transformshift{2.160567in}{0.656126in}%
\pgfsys@useobject{currentmarker}{}%
\end{pgfscope}%
\begin{pgfscope}%
\pgfsys@transformshift{2.160783in}{0.701592in}%
\pgfsys@useobject{currentmarker}{}%
\end{pgfscope}%
\begin{pgfscope}%
\pgfsys@transformshift{2.160999in}{0.663691in}%
\pgfsys@useobject{currentmarker}{}%
\end{pgfscope}%
\begin{pgfscope}%
\pgfsys@transformshift{2.161214in}{0.692026in}%
\pgfsys@useobject{currentmarker}{}%
\end{pgfscope}%
\begin{pgfscope}%
\pgfsys@transformshift{2.161430in}{0.770786in}%
\pgfsys@useobject{currentmarker}{}%
\end{pgfscope}%
\begin{pgfscope}%
\pgfsys@transformshift{2.161645in}{0.759329in}%
\pgfsys@useobject{currentmarker}{}%
\end{pgfscope}%
\begin{pgfscope}%
\pgfsys@transformshift{2.161860in}{0.710355in}%
\pgfsys@useobject{currentmarker}{}%
\end{pgfscope}%
\begin{pgfscope}%
\pgfsys@transformshift{2.162074in}{0.739331in}%
\pgfsys@useobject{currentmarker}{}%
\end{pgfscope}%
\begin{pgfscope}%
\pgfsys@transformshift{2.162289in}{0.708310in}%
\pgfsys@useobject{currentmarker}{}%
\end{pgfscope}%
\begin{pgfscope}%
\pgfsys@transformshift{2.162503in}{0.669930in}%
\pgfsys@useobject{currentmarker}{}%
\end{pgfscope}%
\begin{pgfscope}%
\pgfsys@transformshift{2.162717in}{0.728552in}%
\pgfsys@useobject{currentmarker}{}%
\end{pgfscope}%
\begin{pgfscope}%
\pgfsys@transformshift{2.162931in}{0.755829in}%
\pgfsys@useobject{currentmarker}{}%
\end{pgfscope}%
\begin{pgfscope}%
\pgfsys@transformshift{2.163145in}{0.713167in}%
\pgfsys@useobject{currentmarker}{}%
\end{pgfscope}%
\begin{pgfscope}%
\pgfsys@transformshift{2.163358in}{0.672702in}%
\pgfsys@useobject{currentmarker}{}%
\end{pgfscope}%
\begin{pgfscope}%
\pgfsys@transformshift{2.163571in}{0.708122in}%
\pgfsys@useobject{currentmarker}{}%
\end{pgfscope}%
\begin{pgfscope}%
\pgfsys@transformshift{2.163784in}{0.744918in}%
\pgfsys@useobject{currentmarker}{}%
\end{pgfscope}%
\begin{pgfscope}%
\pgfsys@transformshift{2.163997in}{0.745216in}%
\pgfsys@useobject{currentmarker}{}%
\end{pgfscope}%
\begin{pgfscope}%
\pgfsys@transformshift{2.164210in}{0.711256in}%
\pgfsys@useobject{currentmarker}{}%
\end{pgfscope}%
\begin{pgfscope}%
\pgfsys@transformshift{2.164422in}{0.651359in}%
\pgfsys@useobject{currentmarker}{}%
\end{pgfscope}%
\begin{pgfscope}%
\pgfsys@transformshift{2.164635in}{0.623871in}%
\pgfsys@useobject{currentmarker}{}%
\end{pgfscope}%
\begin{pgfscope}%
\pgfsys@transformshift{2.164847in}{0.634790in}%
\pgfsys@useobject{currentmarker}{}%
\end{pgfscope}%
\begin{pgfscope}%
\pgfsys@transformshift{2.165058in}{0.667293in}%
\pgfsys@useobject{currentmarker}{}%
\end{pgfscope}%
\begin{pgfscope}%
\pgfsys@transformshift{2.165270in}{0.696907in}%
\pgfsys@useobject{currentmarker}{}%
\end{pgfscope}%
\begin{pgfscope}%
\pgfsys@transformshift{2.165481in}{0.698575in}%
\pgfsys@useobject{currentmarker}{}%
\end{pgfscope}%
\begin{pgfscope}%
\pgfsys@transformshift{2.165693in}{0.673732in}%
\pgfsys@useobject{currentmarker}{}%
\end{pgfscope}%
\begin{pgfscope}%
\pgfsys@transformshift{2.165904in}{0.717383in}%
\pgfsys@useobject{currentmarker}{}%
\end{pgfscope}%
\begin{pgfscope}%
\pgfsys@transformshift{2.166115in}{0.734440in}%
\pgfsys@useobject{currentmarker}{}%
\end{pgfscope}%
\begin{pgfscope}%
\pgfsys@transformshift{2.166325in}{0.721465in}%
\pgfsys@useobject{currentmarker}{}%
\end{pgfscope}%
\begin{pgfscope}%
\pgfsys@transformshift{2.166536in}{0.727223in}%
\pgfsys@useobject{currentmarker}{}%
\end{pgfscope}%
\begin{pgfscope}%
\pgfsys@transformshift{2.166746in}{0.726318in}%
\pgfsys@useobject{currentmarker}{}%
\end{pgfscope}%
\begin{pgfscope}%
\pgfsys@transformshift{2.166956in}{0.705061in}%
\pgfsys@useobject{currentmarker}{}%
\end{pgfscope}%
\begin{pgfscope}%
\pgfsys@transformshift{2.167166in}{0.713422in}%
\pgfsys@useobject{currentmarker}{}%
\end{pgfscope}%
\begin{pgfscope}%
\pgfsys@transformshift{2.167375in}{0.721215in}%
\pgfsys@useobject{currentmarker}{}%
\end{pgfscope}%
\begin{pgfscope}%
\pgfsys@transformshift{2.167585in}{0.669433in}%
\pgfsys@useobject{currentmarker}{}%
\end{pgfscope}%
\begin{pgfscope}%
\pgfsys@transformshift{2.167794in}{0.703672in}%
\pgfsys@useobject{currentmarker}{}%
\end{pgfscope}%
\begin{pgfscope}%
\pgfsys@transformshift{2.168003in}{0.720923in}%
\pgfsys@useobject{currentmarker}{}%
\end{pgfscope}%
\begin{pgfscope}%
\pgfsys@transformshift{2.168212in}{0.773505in}%
\pgfsys@useobject{currentmarker}{}%
\end{pgfscope}%
\begin{pgfscope}%
\pgfsys@transformshift{2.168421in}{0.763206in}%
\pgfsys@useobject{currentmarker}{}%
\end{pgfscope}%
\begin{pgfscope}%
\pgfsys@transformshift{2.168629in}{0.720212in}%
\pgfsys@useobject{currentmarker}{}%
\end{pgfscope}%
\begin{pgfscope}%
\pgfsys@transformshift{2.168838in}{0.711242in}%
\pgfsys@useobject{currentmarker}{}%
\end{pgfscope}%
\begin{pgfscope}%
\pgfsys@transformshift{2.169046in}{0.668423in}%
\pgfsys@useobject{currentmarker}{}%
\end{pgfscope}%
\begin{pgfscope}%
\pgfsys@transformshift{2.169254in}{0.699231in}%
\pgfsys@useobject{currentmarker}{}%
\end{pgfscope}%
\begin{pgfscope}%
\pgfsys@transformshift{2.169461in}{0.696676in}%
\pgfsys@useobject{currentmarker}{}%
\end{pgfscope}%
\begin{pgfscope}%
\pgfsys@transformshift{2.169669in}{0.681883in}%
\pgfsys@useobject{currentmarker}{}%
\end{pgfscope}%
\begin{pgfscope}%
\pgfsys@transformshift{2.169876in}{0.688953in}%
\pgfsys@useobject{currentmarker}{}%
\end{pgfscope}%
\begin{pgfscope}%
\pgfsys@transformshift{2.170083in}{0.736737in}%
\pgfsys@useobject{currentmarker}{}%
\end{pgfscope}%
\begin{pgfscope}%
\pgfsys@transformshift{2.170290in}{0.726701in}%
\pgfsys@useobject{currentmarker}{}%
\end{pgfscope}%
\begin{pgfscope}%
\pgfsys@transformshift{2.170497in}{0.708945in}%
\pgfsys@useobject{currentmarker}{}%
\end{pgfscope}%
\begin{pgfscope}%
\pgfsys@transformshift{2.170704in}{0.754924in}%
\pgfsys@useobject{currentmarker}{}%
\end{pgfscope}%
\begin{pgfscope}%
\pgfsys@transformshift{2.170910in}{0.756449in}%
\pgfsys@useobject{currentmarker}{}%
\end{pgfscope}%
\begin{pgfscope}%
\pgfsys@transformshift{2.171116in}{0.669675in}%
\pgfsys@useobject{currentmarker}{}%
\end{pgfscope}%
\begin{pgfscope}%
\pgfsys@transformshift{2.171322in}{0.648596in}%
\pgfsys@useobject{currentmarker}{}%
\end{pgfscope}%
\begin{pgfscope}%
\pgfsys@transformshift{2.171528in}{0.720611in}%
\pgfsys@useobject{currentmarker}{}%
\end{pgfscope}%
\begin{pgfscope}%
\pgfsys@transformshift{2.171734in}{0.736635in}%
\pgfsys@useobject{currentmarker}{}%
\end{pgfscope}%
\begin{pgfscope}%
\pgfsys@transformshift{2.171939in}{0.700356in}%
\pgfsys@useobject{currentmarker}{}%
\end{pgfscope}%
\begin{pgfscope}%
\pgfsys@transformshift{2.172144in}{0.676777in}%
\pgfsys@useobject{currentmarker}{}%
\end{pgfscope}%
\begin{pgfscope}%
\pgfsys@transformshift{2.172350in}{0.722993in}%
\pgfsys@useobject{currentmarker}{}%
\end{pgfscope}%
\begin{pgfscope}%
\pgfsys@transformshift{2.172554in}{0.708050in}%
\pgfsys@useobject{currentmarker}{}%
\end{pgfscope}%
\begin{pgfscope}%
\pgfsys@transformshift{2.172759in}{0.671673in}%
\pgfsys@useobject{currentmarker}{}%
\end{pgfscope}%
\begin{pgfscope}%
\pgfsys@transformshift{2.172964in}{0.677384in}%
\pgfsys@useobject{currentmarker}{}%
\end{pgfscope}%
\begin{pgfscope}%
\pgfsys@transformshift{2.173168in}{0.733607in}%
\pgfsys@useobject{currentmarker}{}%
\end{pgfscope}%
\begin{pgfscope}%
\pgfsys@transformshift{2.173372in}{0.764656in}%
\pgfsys@useobject{currentmarker}{}%
\end{pgfscope}%
\begin{pgfscope}%
\pgfsys@transformshift{2.173576in}{0.706235in}%
\pgfsys@useobject{currentmarker}{}%
\end{pgfscope}%
\begin{pgfscope}%
\pgfsys@transformshift{2.173780in}{0.733161in}%
\pgfsys@useobject{currentmarker}{}%
\end{pgfscope}%
\begin{pgfscope}%
\pgfsys@transformshift{2.173983in}{0.742567in}%
\pgfsys@useobject{currentmarker}{}%
\end{pgfscope}%
\begin{pgfscope}%
\pgfsys@transformshift{2.174187in}{0.710698in}%
\pgfsys@useobject{currentmarker}{}%
\end{pgfscope}%
\begin{pgfscope}%
\pgfsys@transformshift{2.174390in}{0.650475in}%
\pgfsys@useobject{currentmarker}{}%
\end{pgfscope}%
\begin{pgfscope}%
\pgfsys@transformshift{2.174593in}{0.657691in}%
\pgfsys@useobject{currentmarker}{}%
\end{pgfscope}%
\begin{pgfscope}%
\pgfsys@transformshift{2.174796in}{0.691387in}%
\pgfsys@useobject{currentmarker}{}%
\end{pgfscope}%
\begin{pgfscope}%
\pgfsys@transformshift{2.174999in}{0.701248in}%
\pgfsys@useobject{currentmarker}{}%
\end{pgfscope}%
\begin{pgfscope}%
\pgfsys@transformshift{2.175201in}{0.698631in}%
\pgfsys@useobject{currentmarker}{}%
\end{pgfscope}%
\begin{pgfscope}%
\pgfsys@transformshift{2.175403in}{0.715662in}%
\pgfsys@useobject{currentmarker}{}%
\end{pgfscope}%
\begin{pgfscope}%
\pgfsys@transformshift{2.175605in}{0.734154in}%
\pgfsys@useobject{currentmarker}{}%
\end{pgfscope}%
\begin{pgfscope}%
\pgfsys@transformshift{2.175807in}{0.743521in}%
\pgfsys@useobject{currentmarker}{}%
\end{pgfscope}%
\begin{pgfscope}%
\pgfsys@transformshift{2.176009in}{0.739611in}%
\pgfsys@useobject{currentmarker}{}%
\end{pgfscope}%
\begin{pgfscope}%
\pgfsys@transformshift{2.176211in}{0.717308in}%
\pgfsys@useobject{currentmarker}{}%
\end{pgfscope}%
\begin{pgfscope}%
\pgfsys@transformshift{2.176412in}{0.728441in}%
\pgfsys@useobject{currentmarker}{}%
\end{pgfscope}%
\begin{pgfscope}%
\pgfsys@transformshift{2.176613in}{0.694546in}%
\pgfsys@useobject{currentmarker}{}%
\end{pgfscope}%
\begin{pgfscope}%
\pgfsys@transformshift{2.176814in}{0.673112in}%
\pgfsys@useobject{currentmarker}{}%
\end{pgfscope}%
\begin{pgfscope}%
\pgfsys@transformshift{2.177015in}{0.696367in}%
\pgfsys@useobject{currentmarker}{}%
\end{pgfscope}%
\begin{pgfscope}%
\pgfsys@transformshift{2.177216in}{0.705838in}%
\pgfsys@useobject{currentmarker}{}%
\end{pgfscope}%
\begin{pgfscope}%
\pgfsys@transformshift{2.177416in}{0.678078in}%
\pgfsys@useobject{currentmarker}{}%
\end{pgfscope}%
\begin{pgfscope}%
\pgfsys@transformshift{2.177617in}{0.681173in}%
\pgfsys@useobject{currentmarker}{}%
\end{pgfscope}%
\begin{pgfscope}%
\pgfsys@transformshift{2.177817in}{0.761714in}%
\pgfsys@useobject{currentmarker}{}%
\end{pgfscope}%
\begin{pgfscope}%
\pgfsys@transformshift{2.178017in}{0.732977in}%
\pgfsys@useobject{currentmarker}{}%
\end{pgfscope}%
\begin{pgfscope}%
\pgfsys@transformshift{2.178217in}{0.721785in}%
\pgfsys@useobject{currentmarker}{}%
\end{pgfscope}%
\begin{pgfscope}%
\pgfsys@transformshift{2.178416in}{0.685483in}%
\pgfsys@useobject{currentmarker}{}%
\end{pgfscope}%
\begin{pgfscope}%
\pgfsys@transformshift{2.178616in}{0.684326in}%
\pgfsys@useobject{currentmarker}{}%
\end{pgfscope}%
\begin{pgfscope}%
\pgfsys@transformshift{2.178815in}{0.718565in}%
\pgfsys@useobject{currentmarker}{}%
\end{pgfscope}%
\begin{pgfscope}%
\pgfsys@transformshift{2.179014in}{0.685271in}%
\pgfsys@useobject{currentmarker}{}%
\end{pgfscope}%
\begin{pgfscope}%
\pgfsys@transformshift{2.179213in}{0.664190in}%
\pgfsys@useobject{currentmarker}{}%
\end{pgfscope}%
\begin{pgfscope}%
\pgfsys@transformshift{2.179411in}{0.685953in}%
\pgfsys@useobject{currentmarker}{}%
\end{pgfscope}%
\begin{pgfscope}%
\pgfsys@transformshift{2.179610in}{0.738743in}%
\pgfsys@useobject{currentmarker}{}%
\end{pgfscope}%
\begin{pgfscope}%
\pgfsys@transformshift{2.179808in}{0.713178in}%
\pgfsys@useobject{currentmarker}{}%
\end{pgfscope}%
\begin{pgfscope}%
\pgfsys@transformshift{2.180007in}{0.715660in}%
\pgfsys@useobject{currentmarker}{}%
\end{pgfscope}%
\begin{pgfscope}%
\pgfsys@transformshift{2.180205in}{0.728192in}%
\pgfsys@useobject{currentmarker}{}%
\end{pgfscope}%
\begin{pgfscope}%
\pgfsys@transformshift{2.180402in}{0.695799in}%
\pgfsys@useobject{currentmarker}{}%
\end{pgfscope}%
\begin{pgfscope}%
\pgfsys@transformshift{2.180600in}{0.717860in}%
\pgfsys@useobject{currentmarker}{}%
\end{pgfscope}%
\begin{pgfscope}%
\pgfsys@transformshift{2.180798in}{0.734015in}%
\pgfsys@useobject{currentmarker}{}%
\end{pgfscope}%
\begin{pgfscope}%
\pgfsys@transformshift{2.180995in}{0.731213in}%
\pgfsys@useobject{currentmarker}{}%
\end{pgfscope}%
\begin{pgfscope}%
\pgfsys@transformshift{2.181192in}{0.710442in}%
\pgfsys@useobject{currentmarker}{}%
\end{pgfscope}%
\begin{pgfscope}%
\pgfsys@transformshift{2.181389in}{0.692507in}%
\pgfsys@useobject{currentmarker}{}%
\end{pgfscope}%
\begin{pgfscope}%
\pgfsys@transformshift{2.181586in}{0.695097in}%
\pgfsys@useobject{currentmarker}{}%
\end{pgfscope}%
\begin{pgfscope}%
\pgfsys@transformshift{2.181782in}{0.727766in}%
\pgfsys@useobject{currentmarker}{}%
\end{pgfscope}%
\begin{pgfscope}%
\pgfsys@transformshift{2.181979in}{0.672138in}%
\pgfsys@useobject{currentmarker}{}%
\end{pgfscope}%
\begin{pgfscope}%
\pgfsys@transformshift{2.182175in}{0.674989in}%
\pgfsys@useobject{currentmarker}{}%
\end{pgfscope}%
\begin{pgfscope}%
\pgfsys@transformshift{2.182371in}{0.742624in}%
\pgfsys@useobject{currentmarker}{}%
\end{pgfscope}%
\begin{pgfscope}%
\pgfsys@transformshift{2.182567in}{0.755067in}%
\pgfsys@useobject{currentmarker}{}%
\end{pgfscope}%
\begin{pgfscope}%
\pgfsys@transformshift{2.182763in}{0.717957in}%
\pgfsys@useobject{currentmarker}{}%
\end{pgfscope}%
\begin{pgfscope}%
\pgfsys@transformshift{2.182959in}{0.711630in}%
\pgfsys@useobject{currentmarker}{}%
\end{pgfscope}%
\begin{pgfscope}%
\pgfsys@transformshift{2.183154in}{0.715877in}%
\pgfsys@useobject{currentmarker}{}%
\end{pgfscope}%
\begin{pgfscope}%
\pgfsys@transformshift{2.183349in}{0.689947in}%
\pgfsys@useobject{currentmarker}{}%
\end{pgfscope}%
\begin{pgfscope}%
\pgfsys@transformshift{2.183544in}{0.668804in}%
\pgfsys@useobject{currentmarker}{}%
\end{pgfscope}%
\begin{pgfscope}%
\pgfsys@transformshift{2.183739in}{0.661677in}%
\pgfsys@useobject{currentmarker}{}%
\end{pgfscope}%
\begin{pgfscope}%
\pgfsys@transformshift{2.183934in}{0.680868in}%
\pgfsys@useobject{currentmarker}{}%
\end{pgfscope}%
\begin{pgfscope}%
\pgfsys@transformshift{2.184129in}{0.714945in}%
\pgfsys@useobject{currentmarker}{}%
\end{pgfscope}%
\begin{pgfscope}%
\pgfsys@transformshift{2.184323in}{0.715940in}%
\pgfsys@useobject{currentmarker}{}%
\end{pgfscope}%
\begin{pgfscope}%
\pgfsys@transformshift{2.184517in}{0.727992in}%
\pgfsys@useobject{currentmarker}{}%
\end{pgfscope}%
\begin{pgfscope}%
\pgfsys@transformshift{2.184711in}{0.730353in}%
\pgfsys@useobject{currentmarker}{}%
\end{pgfscope}%
\begin{pgfscope}%
\pgfsys@transformshift{2.184905in}{0.739834in}%
\pgfsys@useobject{currentmarker}{}%
\end{pgfscope}%
\begin{pgfscope}%
\pgfsys@transformshift{2.185099in}{0.713620in}%
\pgfsys@useobject{currentmarker}{}%
\end{pgfscope}%
\begin{pgfscope}%
\pgfsys@transformshift{2.185293in}{0.685257in}%
\pgfsys@useobject{currentmarker}{}%
\end{pgfscope}%
\begin{pgfscope}%
\pgfsys@transformshift{2.185486in}{0.681571in}%
\pgfsys@useobject{currentmarker}{}%
\end{pgfscope}%
\begin{pgfscope}%
\pgfsys@transformshift{2.185679in}{0.649910in}%
\pgfsys@useobject{currentmarker}{}%
\end{pgfscope}%
\begin{pgfscope}%
\pgfsys@transformshift{2.185872in}{0.652157in}%
\pgfsys@useobject{currentmarker}{}%
\end{pgfscope}%
\begin{pgfscope}%
\pgfsys@transformshift{2.186065in}{0.687677in}%
\pgfsys@useobject{currentmarker}{}%
\end{pgfscope}%
\begin{pgfscope}%
\pgfsys@transformshift{2.186258in}{0.717005in}%
\pgfsys@useobject{currentmarker}{}%
\end{pgfscope}%
\begin{pgfscope}%
\pgfsys@transformshift{2.186451in}{0.718910in}%
\pgfsys@useobject{currentmarker}{}%
\end{pgfscope}%
\begin{pgfscope}%
\pgfsys@transformshift{2.186643in}{0.732227in}%
\pgfsys@useobject{currentmarker}{}%
\end{pgfscope}%
\begin{pgfscope}%
\pgfsys@transformshift{2.186835in}{0.755536in}%
\pgfsys@useobject{currentmarker}{}%
\end{pgfscope}%
\begin{pgfscope}%
\pgfsys@transformshift{2.187028in}{0.753894in}%
\pgfsys@useobject{currentmarker}{}%
\end{pgfscope}%
\begin{pgfscope}%
\pgfsys@transformshift{2.187219in}{0.772044in}%
\pgfsys@useobject{currentmarker}{}%
\end{pgfscope}%
\begin{pgfscope}%
\pgfsys@transformshift{2.187411in}{0.751588in}%
\pgfsys@useobject{currentmarker}{}%
\end{pgfscope}%
\begin{pgfscope}%
\pgfsys@transformshift{2.187603in}{0.707936in}%
\pgfsys@useobject{currentmarker}{}%
\end{pgfscope}%
\begin{pgfscope}%
\pgfsys@transformshift{2.187794in}{0.736030in}%
\pgfsys@useobject{currentmarker}{}%
\end{pgfscope}%
\begin{pgfscope}%
\pgfsys@transformshift{2.187986in}{0.728339in}%
\pgfsys@useobject{currentmarker}{}%
\end{pgfscope}%
\begin{pgfscope}%
\pgfsys@transformshift{2.188177in}{0.699380in}%
\pgfsys@useobject{currentmarker}{}%
\end{pgfscope}%
\begin{pgfscope}%
\pgfsys@transformshift{2.188368in}{0.666017in}%
\pgfsys@useobject{currentmarker}{}%
\end{pgfscope}%
\begin{pgfscope}%
\pgfsys@transformshift{2.188558in}{0.680523in}%
\pgfsys@useobject{currentmarker}{}%
\end{pgfscope}%
\begin{pgfscope}%
\pgfsys@transformshift{2.188749in}{0.722615in}%
\pgfsys@useobject{currentmarker}{}%
\end{pgfscope}%
\begin{pgfscope}%
\pgfsys@transformshift{2.188939in}{0.705168in}%
\pgfsys@useobject{currentmarker}{}%
\end{pgfscope}%
\begin{pgfscope}%
\pgfsys@transformshift{2.189130in}{0.689519in}%
\pgfsys@useobject{currentmarker}{}%
\end{pgfscope}%
\begin{pgfscope}%
\pgfsys@transformshift{2.189320in}{0.688651in}%
\pgfsys@useobject{currentmarker}{}%
\end{pgfscope}%
\begin{pgfscope}%
\pgfsys@transformshift{2.189510in}{0.710229in}%
\pgfsys@useobject{currentmarker}{}%
\end{pgfscope}%
\begin{pgfscope}%
\pgfsys@transformshift{2.189700in}{0.716728in}%
\pgfsys@useobject{currentmarker}{}%
\end{pgfscope}%
\begin{pgfscope}%
\pgfsys@transformshift{2.189889in}{0.727247in}%
\pgfsys@useobject{currentmarker}{}%
\end{pgfscope}%
\begin{pgfscope}%
\pgfsys@transformshift{2.190079in}{0.704509in}%
\pgfsys@useobject{currentmarker}{}%
\end{pgfscope}%
\begin{pgfscope}%
\pgfsys@transformshift{2.190268in}{0.704351in}%
\pgfsys@useobject{currentmarker}{}%
\end{pgfscope}%
\begin{pgfscope}%
\pgfsys@transformshift{2.190457in}{0.662984in}%
\pgfsys@useobject{currentmarker}{}%
\end{pgfscope}%
\begin{pgfscope}%
\pgfsys@transformshift{2.190646in}{0.592196in}%
\pgfsys@useobject{currentmarker}{}%
\end{pgfscope}%
\begin{pgfscope}%
\pgfsys@transformshift{2.190835in}{0.662557in}%
\pgfsys@useobject{currentmarker}{}%
\end{pgfscope}%
\begin{pgfscope}%
\pgfsys@transformshift{2.191024in}{0.708106in}%
\pgfsys@useobject{currentmarker}{}%
\end{pgfscope}%
\begin{pgfscope}%
\pgfsys@transformshift{2.191213in}{0.702896in}%
\pgfsys@useobject{currentmarker}{}%
\end{pgfscope}%
\begin{pgfscope}%
\pgfsys@transformshift{2.191401in}{0.683284in}%
\pgfsys@useobject{currentmarker}{}%
\end{pgfscope}%
\begin{pgfscope}%
\pgfsys@transformshift{2.191589in}{0.655938in}%
\pgfsys@useobject{currentmarker}{}%
\end{pgfscope}%
\begin{pgfscope}%
\pgfsys@transformshift{2.191777in}{0.678317in}%
\pgfsys@useobject{currentmarker}{}%
\end{pgfscope}%
\begin{pgfscope}%
\pgfsys@transformshift{2.191965in}{0.715542in}%
\pgfsys@useobject{currentmarker}{}%
\end{pgfscope}%
\begin{pgfscope}%
\pgfsys@transformshift{2.192153in}{0.680849in}%
\pgfsys@useobject{currentmarker}{}%
\end{pgfscope}%
\begin{pgfscope}%
\pgfsys@transformshift{2.192340in}{0.724524in}%
\pgfsys@useobject{currentmarker}{}%
\end{pgfscope}%
\begin{pgfscope}%
\pgfsys@transformshift{2.192528in}{0.754026in}%
\pgfsys@useobject{currentmarker}{}%
\end{pgfscope}%
\begin{pgfscope}%
\pgfsys@transformshift{2.192715in}{0.712912in}%
\pgfsys@useobject{currentmarker}{}%
\end{pgfscope}%
\begin{pgfscope}%
\pgfsys@transformshift{2.192902in}{0.701435in}%
\pgfsys@useobject{currentmarker}{}%
\end{pgfscope}%
\begin{pgfscope}%
\pgfsys@transformshift{2.193089in}{0.722462in}%
\pgfsys@useobject{currentmarker}{}%
\end{pgfscope}%
\begin{pgfscope}%
\pgfsys@transformshift{2.193276in}{0.723242in}%
\pgfsys@useobject{currentmarker}{}%
\end{pgfscope}%
\begin{pgfscope}%
\pgfsys@transformshift{2.193463in}{0.695910in}%
\pgfsys@useobject{currentmarker}{}%
\end{pgfscope}%
\begin{pgfscope}%
\pgfsys@transformshift{2.193649in}{0.704246in}%
\pgfsys@useobject{currentmarker}{}%
\end{pgfscope}%
\begin{pgfscope}%
\pgfsys@transformshift{2.193836in}{0.719939in}%
\pgfsys@useobject{currentmarker}{}%
\end{pgfscope}%
\begin{pgfscope}%
\pgfsys@transformshift{2.194022in}{0.701344in}%
\pgfsys@useobject{currentmarker}{}%
\end{pgfscope}%
\begin{pgfscope}%
\pgfsys@transformshift{2.194208in}{0.698389in}%
\pgfsys@useobject{currentmarker}{}%
\end{pgfscope}%
\begin{pgfscope}%
\pgfsys@transformshift{2.194394in}{0.700112in}%
\pgfsys@useobject{currentmarker}{}%
\end{pgfscope}%
\begin{pgfscope}%
\pgfsys@transformshift{2.194580in}{0.717927in}%
\pgfsys@useobject{currentmarker}{}%
\end{pgfscope}%
\begin{pgfscope}%
\pgfsys@transformshift{2.194765in}{0.760663in}%
\pgfsys@useobject{currentmarker}{}%
\end{pgfscope}%
\begin{pgfscope}%
\pgfsys@transformshift{2.194951in}{0.742220in}%
\pgfsys@useobject{currentmarker}{}%
\end{pgfscope}%
\begin{pgfscope}%
\pgfsys@transformshift{2.195136in}{0.712015in}%
\pgfsys@useobject{currentmarker}{}%
\end{pgfscope}%
\begin{pgfscope}%
\pgfsys@transformshift{2.195321in}{0.719726in}%
\pgfsys@useobject{currentmarker}{}%
\end{pgfscope}%
\begin{pgfscope}%
\pgfsys@transformshift{2.195506in}{0.718760in}%
\pgfsys@useobject{currentmarker}{}%
\end{pgfscope}%
\begin{pgfscope}%
\pgfsys@transformshift{2.195691in}{0.631928in}%
\pgfsys@useobject{currentmarker}{}%
\end{pgfscope}%
\begin{pgfscope}%
\pgfsys@transformshift{2.195875in}{0.687643in}%
\pgfsys@useobject{currentmarker}{}%
\end{pgfscope}%
\begin{pgfscope}%
\pgfsys@transformshift{2.196060in}{0.693508in}%
\pgfsys@useobject{currentmarker}{}%
\end{pgfscope}%
\begin{pgfscope}%
\pgfsys@transformshift{2.196244in}{0.722062in}%
\pgfsys@useobject{currentmarker}{}%
\end{pgfscope}%
\begin{pgfscope}%
\pgfsys@transformshift{2.196429in}{0.704035in}%
\pgfsys@useobject{currentmarker}{}%
\end{pgfscope}%
\begin{pgfscope}%
\pgfsys@transformshift{2.196613in}{0.699288in}%
\pgfsys@useobject{currentmarker}{}%
\end{pgfscope}%
\begin{pgfscope}%
\pgfsys@transformshift{2.196797in}{0.711622in}%
\pgfsys@useobject{currentmarker}{}%
\end{pgfscope}%
\begin{pgfscope}%
\pgfsys@transformshift{2.196980in}{0.692619in}%
\pgfsys@useobject{currentmarker}{}%
\end{pgfscope}%
\begin{pgfscope}%
\pgfsys@transformshift{2.197164in}{0.760304in}%
\pgfsys@useobject{currentmarker}{}%
\end{pgfscope}%
\begin{pgfscope}%
\pgfsys@transformshift{2.197347in}{0.746247in}%
\pgfsys@useobject{currentmarker}{}%
\end{pgfscope}%
\begin{pgfscope}%
\pgfsys@transformshift{2.197531in}{0.729386in}%
\pgfsys@useobject{currentmarker}{}%
\end{pgfscope}%
\begin{pgfscope}%
\pgfsys@transformshift{2.197714in}{0.722816in}%
\pgfsys@useobject{currentmarker}{}%
\end{pgfscope}%
\begin{pgfscope}%
\pgfsys@transformshift{2.197897in}{0.701341in}%
\pgfsys@useobject{currentmarker}{}%
\end{pgfscope}%
\begin{pgfscope}%
\pgfsys@transformshift{2.198080in}{0.669729in}%
\pgfsys@useobject{currentmarker}{}%
\end{pgfscope}%
\begin{pgfscope}%
\pgfsys@transformshift{2.198262in}{0.706661in}%
\pgfsys@useobject{currentmarker}{}%
\end{pgfscope}%
\begin{pgfscope}%
\pgfsys@transformshift{2.198445in}{0.715244in}%
\pgfsys@useobject{currentmarker}{}%
\end{pgfscope}%
\begin{pgfscope}%
\pgfsys@transformshift{2.198627in}{0.706333in}%
\pgfsys@useobject{currentmarker}{}%
\end{pgfscope}%
\begin{pgfscope}%
\pgfsys@transformshift{2.198810in}{0.690146in}%
\pgfsys@useobject{currentmarker}{}%
\end{pgfscope}%
\begin{pgfscope}%
\pgfsys@transformshift{2.198992in}{0.690807in}%
\pgfsys@useobject{currentmarker}{}%
\end{pgfscope}%
\begin{pgfscope}%
\pgfsys@transformshift{2.199174in}{0.700704in}%
\pgfsys@useobject{currentmarker}{}%
\end{pgfscope}%
\begin{pgfscope}%
\pgfsys@transformshift{2.199356in}{0.699940in}%
\pgfsys@useobject{currentmarker}{}%
\end{pgfscope}%
\begin{pgfscope}%
\pgfsys@transformshift{2.199537in}{0.709103in}%
\pgfsys@useobject{currentmarker}{}%
\end{pgfscope}%
\begin{pgfscope}%
\pgfsys@transformshift{2.199719in}{0.701384in}%
\pgfsys@useobject{currentmarker}{}%
\end{pgfscope}%
\begin{pgfscope}%
\pgfsys@transformshift{2.199900in}{0.684330in}%
\pgfsys@useobject{currentmarker}{}%
\end{pgfscope}%
\begin{pgfscope}%
\pgfsys@transformshift{2.200081in}{0.686280in}%
\pgfsys@useobject{currentmarker}{}%
\end{pgfscope}%
\begin{pgfscope}%
\pgfsys@transformshift{2.200263in}{0.709757in}%
\pgfsys@useobject{currentmarker}{}%
\end{pgfscope}%
\begin{pgfscope}%
\pgfsys@transformshift{2.200444in}{0.709704in}%
\pgfsys@useobject{currentmarker}{}%
\end{pgfscope}%
\begin{pgfscope}%
\pgfsys@transformshift{2.200624in}{0.698326in}%
\pgfsys@useobject{currentmarker}{}%
\end{pgfscope}%
\begin{pgfscope}%
\pgfsys@transformshift{2.200805in}{0.660877in}%
\pgfsys@useobject{currentmarker}{}%
\end{pgfscope}%
\begin{pgfscope}%
\pgfsys@transformshift{2.200986in}{0.684121in}%
\pgfsys@useobject{currentmarker}{}%
\end{pgfscope}%
\begin{pgfscope}%
\pgfsys@transformshift{2.201166in}{0.711412in}%
\pgfsys@useobject{currentmarker}{}%
\end{pgfscope}%
\begin{pgfscope}%
\pgfsys@transformshift{2.201346in}{0.647488in}%
\pgfsys@useobject{currentmarker}{}%
\end{pgfscope}%
\begin{pgfscope}%
\pgfsys@transformshift{2.201526in}{0.623108in}%
\pgfsys@useobject{currentmarker}{}%
\end{pgfscope}%
\begin{pgfscope}%
\pgfsys@transformshift{2.201706in}{0.636178in}%
\pgfsys@useobject{currentmarker}{}%
\end{pgfscope}%
\begin{pgfscope}%
\pgfsys@transformshift{2.201886in}{0.650636in}%
\pgfsys@useobject{currentmarker}{}%
\end{pgfscope}%
\begin{pgfscope}%
\pgfsys@transformshift{2.202066in}{0.681083in}%
\pgfsys@useobject{currentmarker}{}%
\end{pgfscope}%
\begin{pgfscope}%
\pgfsys@transformshift{2.202245in}{0.646135in}%
\pgfsys@useobject{currentmarker}{}%
\end{pgfscope}%
\begin{pgfscope}%
\pgfsys@transformshift{2.202424in}{0.657928in}%
\pgfsys@useobject{currentmarker}{}%
\end{pgfscope}%
\begin{pgfscope}%
\pgfsys@transformshift{2.202604in}{0.673066in}%
\pgfsys@useobject{currentmarker}{}%
\end{pgfscope}%
\begin{pgfscope}%
\pgfsys@transformshift{2.202783in}{0.690799in}%
\pgfsys@useobject{currentmarker}{}%
\end{pgfscope}%
\begin{pgfscope}%
\pgfsys@transformshift{2.202962in}{0.703526in}%
\pgfsys@useobject{currentmarker}{}%
\end{pgfscope}%
\begin{pgfscope}%
\pgfsys@transformshift{2.203140in}{0.686310in}%
\pgfsys@useobject{currentmarker}{}%
\end{pgfscope}%
\begin{pgfscope}%
\pgfsys@transformshift{2.203319in}{0.658153in}%
\pgfsys@useobject{currentmarker}{}%
\end{pgfscope}%
\begin{pgfscope}%
\pgfsys@transformshift{2.203498in}{0.684317in}%
\pgfsys@useobject{currentmarker}{}%
\end{pgfscope}%
\begin{pgfscope}%
\pgfsys@transformshift{2.203676in}{0.743454in}%
\pgfsys@useobject{currentmarker}{}%
\end{pgfscope}%
\begin{pgfscope}%
\pgfsys@transformshift{2.203854in}{0.738394in}%
\pgfsys@useobject{currentmarker}{}%
\end{pgfscope}%
\begin{pgfscope}%
\pgfsys@transformshift{2.204032in}{0.718487in}%
\pgfsys@useobject{currentmarker}{}%
\end{pgfscope}%
\begin{pgfscope}%
\pgfsys@transformshift{2.204210in}{0.654393in}%
\pgfsys@useobject{currentmarker}{}%
\end{pgfscope}%
\begin{pgfscope}%
\pgfsys@transformshift{2.204388in}{0.665074in}%
\pgfsys@useobject{currentmarker}{}%
\end{pgfscope}%
\begin{pgfscope}%
\pgfsys@transformshift{2.204566in}{0.707318in}%
\pgfsys@useobject{currentmarker}{}%
\end{pgfscope}%
\begin{pgfscope}%
\pgfsys@transformshift{2.204743in}{0.721140in}%
\pgfsys@useobject{currentmarker}{}%
\end{pgfscope}%
\begin{pgfscope}%
\pgfsys@transformshift{2.204921in}{0.629012in}%
\pgfsys@useobject{currentmarker}{}%
\end{pgfscope}%
\begin{pgfscope}%
\pgfsys@transformshift{2.205098in}{0.731811in}%
\pgfsys@useobject{currentmarker}{}%
\end{pgfscope}%
\begin{pgfscope}%
\pgfsys@transformshift{2.205275in}{0.717617in}%
\pgfsys@useobject{currentmarker}{}%
\end{pgfscope}%
\begin{pgfscope}%
\pgfsys@transformshift{2.205452in}{0.698380in}%
\pgfsys@useobject{currentmarker}{}%
\end{pgfscope}%
\begin{pgfscope}%
\pgfsys@transformshift{2.205629in}{0.710271in}%
\pgfsys@useobject{currentmarker}{}%
\end{pgfscope}%
\begin{pgfscope}%
\pgfsys@transformshift{2.205805in}{0.687162in}%
\pgfsys@useobject{currentmarker}{}%
\end{pgfscope}%
\begin{pgfscope}%
\pgfsys@transformshift{2.205982in}{0.665303in}%
\pgfsys@useobject{currentmarker}{}%
\end{pgfscope}%
\begin{pgfscope}%
\pgfsys@transformshift{2.206158in}{0.722979in}%
\pgfsys@useobject{currentmarker}{}%
\end{pgfscope}%
\begin{pgfscope}%
\pgfsys@transformshift{2.206335in}{0.722266in}%
\pgfsys@useobject{currentmarker}{}%
\end{pgfscope}%
\begin{pgfscope}%
\pgfsys@transformshift{2.206511in}{0.682677in}%
\pgfsys@useobject{currentmarker}{}%
\end{pgfscope}%
\begin{pgfscope}%
\pgfsys@transformshift{2.206687in}{0.659216in}%
\pgfsys@useobject{currentmarker}{}%
\end{pgfscope}%
\begin{pgfscope}%
\pgfsys@transformshift{2.206863in}{0.691140in}%
\pgfsys@useobject{currentmarker}{}%
\end{pgfscope}%
\begin{pgfscope}%
\pgfsys@transformshift{2.207038in}{0.690426in}%
\pgfsys@useobject{currentmarker}{}%
\end{pgfscope}%
\begin{pgfscope}%
\pgfsys@transformshift{2.207214in}{0.672113in}%
\pgfsys@useobject{currentmarker}{}%
\end{pgfscope}%
\begin{pgfscope}%
\pgfsys@transformshift{2.207389in}{0.724117in}%
\pgfsys@useobject{currentmarker}{}%
\end{pgfscope}%
\begin{pgfscope}%
\pgfsys@transformshift{2.207565in}{0.727842in}%
\pgfsys@useobject{currentmarker}{}%
\end{pgfscope}%
\begin{pgfscope}%
\pgfsys@transformshift{2.207740in}{0.654429in}%
\pgfsys@useobject{currentmarker}{}%
\end{pgfscope}%
\begin{pgfscope}%
\pgfsys@transformshift{2.207915in}{0.645144in}%
\pgfsys@useobject{currentmarker}{}%
\end{pgfscope}%
\begin{pgfscope}%
\pgfsys@transformshift{2.208090in}{0.645824in}%
\pgfsys@useobject{currentmarker}{}%
\end{pgfscope}%
\begin{pgfscope}%
\pgfsys@transformshift{2.208264in}{0.636517in}%
\pgfsys@useobject{currentmarker}{}%
\end{pgfscope}%
\begin{pgfscope}%
\pgfsys@transformshift{2.208439in}{0.674486in}%
\pgfsys@useobject{currentmarker}{}%
\end{pgfscope}%
\begin{pgfscope}%
\pgfsys@transformshift{2.208614in}{0.614702in}%
\pgfsys@useobject{currentmarker}{}%
\end{pgfscope}%
\begin{pgfscope}%
\pgfsys@transformshift{2.208788in}{0.655269in}%
\pgfsys@useobject{currentmarker}{}%
\end{pgfscope}%
\begin{pgfscope}%
\pgfsys@transformshift{2.208962in}{0.665779in}%
\pgfsys@useobject{currentmarker}{}%
\end{pgfscope}%
\begin{pgfscope}%
\pgfsys@transformshift{2.209136in}{0.690538in}%
\pgfsys@useobject{currentmarker}{}%
\end{pgfscope}%
\begin{pgfscope}%
\pgfsys@transformshift{2.209310in}{0.706981in}%
\pgfsys@useobject{currentmarker}{}%
\end{pgfscope}%
\begin{pgfscope}%
\pgfsys@transformshift{2.209484in}{0.701144in}%
\pgfsys@useobject{currentmarker}{}%
\end{pgfscope}%
\begin{pgfscope}%
\pgfsys@transformshift{2.209658in}{0.699004in}%
\pgfsys@useobject{currentmarker}{}%
\end{pgfscope}%
\begin{pgfscope}%
\pgfsys@transformshift{2.209831in}{0.670171in}%
\pgfsys@useobject{currentmarker}{}%
\end{pgfscope}%
\begin{pgfscope}%
\pgfsys@transformshift{2.210005in}{0.691461in}%
\pgfsys@useobject{currentmarker}{}%
\end{pgfscope}%
\begin{pgfscope}%
\pgfsys@transformshift{2.210178in}{0.664451in}%
\pgfsys@useobject{currentmarker}{}%
\end{pgfscope}%
\begin{pgfscope}%
\pgfsys@transformshift{2.210351in}{0.637785in}%
\pgfsys@useobject{currentmarker}{}%
\end{pgfscope}%
\begin{pgfscope}%
\pgfsys@transformshift{2.210524in}{0.658172in}%
\pgfsys@useobject{currentmarker}{}%
\end{pgfscope}%
\begin{pgfscope}%
\pgfsys@transformshift{2.210697in}{0.662022in}%
\pgfsys@useobject{currentmarker}{}%
\end{pgfscope}%
\begin{pgfscope}%
\pgfsys@transformshift{2.210870in}{0.655558in}%
\pgfsys@useobject{currentmarker}{}%
\end{pgfscope}%
\begin{pgfscope}%
\pgfsys@transformshift{2.211042in}{0.665558in}%
\pgfsys@useobject{currentmarker}{}%
\end{pgfscope}%
\begin{pgfscope}%
\pgfsys@transformshift{2.211215in}{0.703880in}%
\pgfsys@useobject{currentmarker}{}%
\end{pgfscope}%
\begin{pgfscope}%
\pgfsys@transformshift{2.211387in}{0.698403in}%
\pgfsys@useobject{currentmarker}{}%
\end{pgfscope}%
\begin{pgfscope}%
\pgfsys@transformshift{2.211559in}{0.654520in}%
\pgfsys@useobject{currentmarker}{}%
\end{pgfscope}%
\begin{pgfscope}%
\pgfsys@transformshift{2.211731in}{0.651175in}%
\pgfsys@useobject{currentmarker}{}%
\end{pgfscope}%
\begin{pgfscope}%
\pgfsys@transformshift{2.211903in}{0.642618in}%
\pgfsys@useobject{currentmarker}{}%
\end{pgfscope}%
\begin{pgfscope}%
\pgfsys@transformshift{2.212075in}{0.604438in}%
\pgfsys@useobject{currentmarker}{}%
\end{pgfscope}%
\begin{pgfscope}%
\pgfsys@transformshift{2.212247in}{0.676310in}%
\pgfsys@useobject{currentmarker}{}%
\end{pgfscope}%
\begin{pgfscope}%
\pgfsys@transformshift{2.212418in}{0.692149in}%
\pgfsys@useobject{currentmarker}{}%
\end{pgfscope}%
\begin{pgfscope}%
\pgfsys@transformshift{2.212590in}{0.678494in}%
\pgfsys@useobject{currentmarker}{}%
\end{pgfscope}%
\begin{pgfscope}%
\pgfsys@transformshift{2.212761in}{0.663655in}%
\pgfsys@useobject{currentmarker}{}%
\end{pgfscope}%
\begin{pgfscope}%
\pgfsys@transformshift{2.212932in}{0.709323in}%
\pgfsys@useobject{currentmarker}{}%
\end{pgfscope}%
\begin{pgfscope}%
\pgfsys@transformshift{2.213103in}{0.707774in}%
\pgfsys@useobject{currentmarker}{}%
\end{pgfscope}%
\begin{pgfscope}%
\pgfsys@transformshift{2.213274in}{0.685196in}%
\pgfsys@useobject{currentmarker}{}%
\end{pgfscope}%
\begin{pgfscope}%
\pgfsys@transformshift{2.213445in}{0.715490in}%
\pgfsys@useobject{currentmarker}{}%
\end{pgfscope}%
\begin{pgfscope}%
\pgfsys@transformshift{2.213616in}{0.685996in}%
\pgfsys@useobject{currentmarker}{}%
\end{pgfscope}%
\begin{pgfscope}%
\pgfsys@transformshift{2.213786in}{0.690508in}%
\pgfsys@useobject{currentmarker}{}%
\end{pgfscope}%
\begin{pgfscope}%
\pgfsys@transformshift{2.213957in}{0.689029in}%
\pgfsys@useobject{currentmarker}{}%
\end{pgfscope}%
\begin{pgfscope}%
\pgfsys@transformshift{2.214127in}{0.695985in}%
\pgfsys@useobject{currentmarker}{}%
\end{pgfscope}%
\begin{pgfscope}%
\pgfsys@transformshift{2.214297in}{0.651903in}%
\pgfsys@useobject{currentmarker}{}%
\end{pgfscope}%
\begin{pgfscope}%
\pgfsys@transformshift{2.214467in}{0.661715in}%
\pgfsys@useobject{currentmarker}{}%
\end{pgfscope}%
\begin{pgfscope}%
\pgfsys@transformshift{2.214637in}{0.727454in}%
\pgfsys@useobject{currentmarker}{}%
\end{pgfscope}%
\begin{pgfscope}%
\pgfsys@transformshift{2.214807in}{0.703116in}%
\pgfsys@useobject{currentmarker}{}%
\end{pgfscope}%
\begin{pgfscope}%
\pgfsys@transformshift{2.214976in}{0.672621in}%
\pgfsys@useobject{currentmarker}{}%
\end{pgfscope}%
\begin{pgfscope}%
\pgfsys@transformshift{2.215146in}{0.670440in}%
\pgfsys@useobject{currentmarker}{}%
\end{pgfscope}%
\begin{pgfscope}%
\pgfsys@transformshift{2.215315in}{0.688599in}%
\pgfsys@useobject{currentmarker}{}%
\end{pgfscope}%
\begin{pgfscope}%
\pgfsys@transformshift{2.215484in}{0.676720in}%
\pgfsys@useobject{currentmarker}{}%
\end{pgfscope}%
\begin{pgfscope}%
\pgfsys@transformshift{2.215653in}{0.725742in}%
\pgfsys@useobject{currentmarker}{}%
\end{pgfscope}%
\begin{pgfscope}%
\pgfsys@transformshift{2.215822in}{0.729649in}%
\pgfsys@useobject{currentmarker}{}%
\end{pgfscope}%
\begin{pgfscope}%
\pgfsys@transformshift{2.215991in}{0.687232in}%
\pgfsys@useobject{currentmarker}{}%
\end{pgfscope}%
\begin{pgfscope}%
\pgfsys@transformshift{2.216160in}{0.659842in}%
\pgfsys@useobject{currentmarker}{}%
\end{pgfscope}%
\begin{pgfscope}%
\pgfsys@transformshift{2.216328in}{0.694122in}%
\pgfsys@useobject{currentmarker}{}%
\end{pgfscope}%
\begin{pgfscope}%
\pgfsys@transformshift{2.216497in}{0.681836in}%
\pgfsys@useobject{currentmarker}{}%
\end{pgfscope}%
\begin{pgfscope}%
\pgfsys@transformshift{2.216665in}{0.722855in}%
\pgfsys@useobject{currentmarker}{}%
\end{pgfscope}%
\begin{pgfscope}%
\pgfsys@transformshift{2.216833in}{0.713131in}%
\pgfsys@useobject{currentmarker}{}%
\end{pgfscope}%
\begin{pgfscope}%
\pgfsys@transformshift{2.217002in}{0.686684in}%
\pgfsys@useobject{currentmarker}{}%
\end{pgfscope}%
\begin{pgfscope}%
\pgfsys@transformshift{2.217170in}{0.670298in}%
\pgfsys@useobject{currentmarker}{}%
\end{pgfscope}%
\begin{pgfscope}%
\pgfsys@transformshift{2.217337in}{0.691544in}%
\pgfsys@useobject{currentmarker}{}%
\end{pgfscope}%
\begin{pgfscope}%
\pgfsys@transformshift{2.217505in}{0.707442in}%
\pgfsys@useobject{currentmarker}{}%
\end{pgfscope}%
\begin{pgfscope}%
\pgfsys@transformshift{2.217673in}{0.701476in}%
\pgfsys@useobject{currentmarker}{}%
\end{pgfscope}%
\begin{pgfscope}%
\pgfsys@transformshift{2.217840in}{0.742355in}%
\pgfsys@useobject{currentmarker}{}%
\end{pgfscope}%
\begin{pgfscope}%
\pgfsys@transformshift{2.218007in}{0.697654in}%
\pgfsys@useobject{currentmarker}{}%
\end{pgfscope}%
\begin{pgfscope}%
\pgfsys@transformshift{2.218175in}{0.662512in}%
\pgfsys@useobject{currentmarker}{}%
\end{pgfscope}%
\begin{pgfscope}%
\pgfsys@transformshift{2.218342in}{0.647427in}%
\pgfsys@useobject{currentmarker}{}%
\end{pgfscope}%
\begin{pgfscope}%
\pgfsys@transformshift{2.218509in}{0.586686in}%
\pgfsys@useobject{currentmarker}{}%
\end{pgfscope}%
\begin{pgfscope}%
\pgfsys@transformshift{2.218676in}{0.672205in}%
\pgfsys@useobject{currentmarker}{}%
\end{pgfscope}%
\begin{pgfscope}%
\pgfsys@transformshift{2.218842in}{0.658616in}%
\pgfsys@useobject{currentmarker}{}%
\end{pgfscope}%
\begin{pgfscope}%
\pgfsys@transformshift{2.219009in}{0.697068in}%
\pgfsys@useobject{currentmarker}{}%
\end{pgfscope}%
\begin{pgfscope}%
\pgfsys@transformshift{2.219175in}{0.679150in}%
\pgfsys@useobject{currentmarker}{}%
\end{pgfscope}%
\begin{pgfscope}%
\pgfsys@transformshift{2.219342in}{0.718839in}%
\pgfsys@useobject{currentmarker}{}%
\end{pgfscope}%
\begin{pgfscope}%
\pgfsys@transformshift{2.219508in}{0.751922in}%
\pgfsys@useobject{currentmarker}{}%
\end{pgfscope}%
\begin{pgfscope}%
\pgfsys@transformshift{2.219674in}{0.661995in}%
\pgfsys@useobject{currentmarker}{}%
\end{pgfscope}%
\begin{pgfscope}%
\pgfsys@transformshift{2.219840in}{0.678927in}%
\pgfsys@useobject{currentmarker}{}%
\end{pgfscope}%
\begin{pgfscope}%
\pgfsys@transformshift{2.220006in}{0.716956in}%
\pgfsys@useobject{currentmarker}{}%
\end{pgfscope}%
\begin{pgfscope}%
\pgfsys@transformshift{2.220172in}{0.730325in}%
\pgfsys@useobject{currentmarker}{}%
\end{pgfscope}%
\begin{pgfscope}%
\pgfsys@transformshift{2.220337in}{0.698822in}%
\pgfsys@useobject{currentmarker}{}%
\end{pgfscope}%
\begin{pgfscope}%
\pgfsys@transformshift{2.220503in}{0.652438in}%
\pgfsys@useobject{currentmarker}{}%
\end{pgfscope}%
\begin{pgfscope}%
\pgfsys@transformshift{2.220668in}{0.658983in}%
\pgfsys@useobject{currentmarker}{}%
\end{pgfscope}%
\begin{pgfscope}%
\pgfsys@transformshift{2.220833in}{0.662532in}%
\pgfsys@useobject{currentmarker}{}%
\end{pgfscope}%
\begin{pgfscope}%
\pgfsys@transformshift{2.220998in}{0.661507in}%
\pgfsys@useobject{currentmarker}{}%
\end{pgfscope}%
\begin{pgfscope}%
\pgfsys@transformshift{2.221163in}{0.691979in}%
\pgfsys@useobject{currentmarker}{}%
\end{pgfscope}%
\begin{pgfscope}%
\pgfsys@transformshift{2.221328in}{0.684149in}%
\pgfsys@useobject{currentmarker}{}%
\end{pgfscope}%
\begin{pgfscope}%
\pgfsys@transformshift{2.221493in}{0.676064in}%
\pgfsys@useobject{currentmarker}{}%
\end{pgfscope}%
\begin{pgfscope}%
\pgfsys@transformshift{2.221658in}{0.737285in}%
\pgfsys@useobject{currentmarker}{}%
\end{pgfscope}%
\begin{pgfscope}%
\pgfsys@transformshift{2.221822in}{0.695028in}%
\pgfsys@useobject{currentmarker}{}%
\end{pgfscope}%
\begin{pgfscope}%
\pgfsys@transformshift{2.221987in}{0.644811in}%
\pgfsys@useobject{currentmarker}{}%
\end{pgfscope}%
\begin{pgfscope}%
\pgfsys@transformshift{2.222151in}{0.661033in}%
\pgfsys@useobject{currentmarker}{}%
\end{pgfscope}%
\begin{pgfscope}%
\pgfsys@transformshift{2.222315in}{0.656753in}%
\pgfsys@useobject{currentmarker}{}%
\end{pgfscope}%
\begin{pgfscope}%
\pgfsys@transformshift{2.222479in}{0.659268in}%
\pgfsys@useobject{currentmarker}{}%
\end{pgfscope}%
\begin{pgfscope}%
\pgfsys@transformshift{2.222643in}{0.716246in}%
\pgfsys@useobject{currentmarker}{}%
\end{pgfscope}%
\begin{pgfscope}%
\pgfsys@transformshift{2.222807in}{0.726481in}%
\pgfsys@useobject{currentmarker}{}%
\end{pgfscope}%
\begin{pgfscope}%
\pgfsys@transformshift{2.222971in}{0.715042in}%
\pgfsys@useobject{currentmarker}{}%
\end{pgfscope}%
\begin{pgfscope}%
\pgfsys@transformshift{2.223134in}{0.700755in}%
\pgfsys@useobject{currentmarker}{}%
\end{pgfscope}%
\begin{pgfscope}%
\pgfsys@transformshift{2.223298in}{0.723611in}%
\pgfsys@useobject{currentmarker}{}%
\end{pgfscope}%
\begin{pgfscope}%
\pgfsys@transformshift{2.223461in}{0.713462in}%
\pgfsys@useobject{currentmarker}{}%
\end{pgfscope}%
\begin{pgfscope}%
\pgfsys@transformshift{2.223624in}{0.656997in}%
\pgfsys@useobject{currentmarker}{}%
\end{pgfscope}%
\begin{pgfscope}%
\pgfsys@transformshift{2.223787in}{0.685790in}%
\pgfsys@useobject{currentmarker}{}%
\end{pgfscope}%
\begin{pgfscope}%
\pgfsys@transformshift{2.223950in}{0.677426in}%
\pgfsys@useobject{currentmarker}{}%
\end{pgfscope}%
\begin{pgfscope}%
\pgfsys@transformshift{2.224113in}{0.701467in}%
\pgfsys@useobject{currentmarker}{}%
\end{pgfscope}%
\begin{pgfscope}%
\pgfsys@transformshift{2.224276in}{0.737747in}%
\pgfsys@useobject{currentmarker}{}%
\end{pgfscope}%
\begin{pgfscope}%
\pgfsys@transformshift{2.224438in}{0.756439in}%
\pgfsys@useobject{currentmarker}{}%
\end{pgfscope}%
\begin{pgfscope}%
\pgfsys@transformshift{2.224601in}{0.703901in}%
\pgfsys@useobject{currentmarker}{}%
\end{pgfscope}%
\begin{pgfscope}%
\pgfsys@transformshift{2.224763in}{0.681114in}%
\pgfsys@useobject{currentmarker}{}%
\end{pgfscope}%
\begin{pgfscope}%
\pgfsys@transformshift{2.224925in}{0.683200in}%
\pgfsys@useobject{currentmarker}{}%
\end{pgfscope}%
\begin{pgfscope}%
\pgfsys@transformshift{2.225088in}{0.702248in}%
\pgfsys@useobject{currentmarker}{}%
\end{pgfscope}%
\begin{pgfscope}%
\pgfsys@transformshift{2.225250in}{0.700620in}%
\pgfsys@useobject{currentmarker}{}%
\end{pgfscope}%
\begin{pgfscope}%
\pgfsys@transformshift{2.225412in}{0.645747in}%
\pgfsys@useobject{currentmarker}{}%
\end{pgfscope}%
\begin{pgfscope}%
\pgfsys@transformshift{2.225573in}{0.719457in}%
\pgfsys@useobject{currentmarker}{}%
\end{pgfscope}%
\begin{pgfscope}%
\pgfsys@transformshift{2.225735in}{0.720150in}%
\pgfsys@useobject{currentmarker}{}%
\end{pgfscope}%
\begin{pgfscope}%
\pgfsys@transformshift{2.225897in}{0.705578in}%
\pgfsys@useobject{currentmarker}{}%
\end{pgfscope}%
\begin{pgfscope}%
\pgfsys@transformshift{2.226058in}{0.677766in}%
\pgfsys@useobject{currentmarker}{}%
\end{pgfscope}%
\begin{pgfscope}%
\pgfsys@transformshift{2.226219in}{0.658175in}%
\pgfsys@useobject{currentmarker}{}%
\end{pgfscope}%
\begin{pgfscope}%
\pgfsys@transformshift{2.226381in}{0.684897in}%
\pgfsys@useobject{currentmarker}{}%
\end{pgfscope}%
\begin{pgfscope}%
\pgfsys@transformshift{2.226542in}{0.700898in}%
\pgfsys@useobject{currentmarker}{}%
\end{pgfscope}%
\begin{pgfscope}%
\pgfsys@transformshift{2.226703in}{0.620834in}%
\pgfsys@useobject{currentmarker}{}%
\end{pgfscope}%
\begin{pgfscope}%
\pgfsys@transformshift{2.226863in}{0.638387in}%
\pgfsys@useobject{currentmarker}{}%
\end{pgfscope}%
\begin{pgfscope}%
\pgfsys@transformshift{2.227024in}{0.654146in}%
\pgfsys@useobject{currentmarker}{}%
\end{pgfscope}%
\begin{pgfscope}%
\pgfsys@transformshift{2.227185in}{0.670980in}%
\pgfsys@useobject{currentmarker}{}%
\end{pgfscope}%
\begin{pgfscope}%
\pgfsys@transformshift{2.227345in}{0.664999in}%
\pgfsys@useobject{currentmarker}{}%
\end{pgfscope}%
\begin{pgfscope}%
\pgfsys@transformshift{2.227506in}{0.677720in}%
\pgfsys@useobject{currentmarker}{}%
\end{pgfscope}%
\begin{pgfscope}%
\pgfsys@transformshift{2.227666in}{0.683698in}%
\pgfsys@useobject{currentmarker}{}%
\end{pgfscope}%
\begin{pgfscope}%
\pgfsys@transformshift{2.227826in}{0.663858in}%
\pgfsys@useobject{currentmarker}{}%
\end{pgfscope}%
\begin{pgfscope}%
\pgfsys@transformshift{2.227986in}{0.658817in}%
\pgfsys@useobject{currentmarker}{}%
\end{pgfscope}%
\begin{pgfscope}%
\pgfsys@transformshift{2.228146in}{0.649575in}%
\pgfsys@useobject{currentmarker}{}%
\end{pgfscope}%
\begin{pgfscope}%
\pgfsys@transformshift{2.228306in}{0.694644in}%
\pgfsys@useobject{currentmarker}{}%
\end{pgfscope}%
\begin{pgfscope}%
\pgfsys@transformshift{2.228466in}{0.664077in}%
\pgfsys@useobject{currentmarker}{}%
\end{pgfscope}%
\begin{pgfscope}%
\pgfsys@transformshift{2.228625in}{0.692586in}%
\pgfsys@useobject{currentmarker}{}%
\end{pgfscope}%
\begin{pgfscope}%
\pgfsys@transformshift{2.228785in}{0.683773in}%
\pgfsys@useobject{currentmarker}{}%
\end{pgfscope}%
\begin{pgfscope}%
\pgfsys@transformshift{2.228944in}{0.722397in}%
\pgfsys@useobject{currentmarker}{}%
\end{pgfscope}%
\begin{pgfscope}%
\pgfsys@transformshift{2.229104in}{0.737867in}%
\pgfsys@useobject{currentmarker}{}%
\end{pgfscope}%
\begin{pgfscope}%
\pgfsys@transformshift{2.229263in}{0.748168in}%
\pgfsys@useobject{currentmarker}{}%
\end{pgfscope}%
\begin{pgfscope}%
\pgfsys@transformshift{2.229422in}{0.755335in}%
\pgfsys@useobject{currentmarker}{}%
\end{pgfscope}%
\begin{pgfscope}%
\pgfsys@transformshift{2.229581in}{0.745430in}%
\pgfsys@useobject{currentmarker}{}%
\end{pgfscope}%
\begin{pgfscope}%
\pgfsys@transformshift{2.229740in}{0.752728in}%
\pgfsys@useobject{currentmarker}{}%
\end{pgfscope}%
\begin{pgfscope}%
\pgfsys@transformshift{2.229898in}{0.722353in}%
\pgfsys@useobject{currentmarker}{}%
\end{pgfscope}%
\begin{pgfscope}%
\pgfsys@transformshift{2.230057in}{0.676680in}%
\pgfsys@useobject{currentmarker}{}%
\end{pgfscope}%
\begin{pgfscope}%
\pgfsys@transformshift{2.230215in}{0.694917in}%
\pgfsys@useobject{currentmarker}{}%
\end{pgfscope}%
\begin{pgfscope}%
\pgfsys@transformshift{2.230374in}{0.727930in}%
\pgfsys@useobject{currentmarker}{}%
\end{pgfscope}%
\begin{pgfscope}%
\pgfsys@transformshift{2.230532in}{0.660147in}%
\pgfsys@useobject{currentmarker}{}%
\end{pgfscope}%
\begin{pgfscope}%
\pgfsys@transformshift{2.230690in}{0.707979in}%
\pgfsys@useobject{currentmarker}{}%
\end{pgfscope}%
\begin{pgfscope}%
\pgfsys@transformshift{2.230848in}{0.695276in}%
\pgfsys@useobject{currentmarker}{}%
\end{pgfscope}%
\begin{pgfscope}%
\pgfsys@transformshift{2.231006in}{0.649705in}%
\pgfsys@useobject{currentmarker}{}%
\end{pgfscope}%
\begin{pgfscope}%
\pgfsys@transformshift{2.231164in}{0.704161in}%
\pgfsys@useobject{currentmarker}{}%
\end{pgfscope}%
\begin{pgfscope}%
\pgfsys@transformshift{2.231322in}{0.713497in}%
\pgfsys@useobject{currentmarker}{}%
\end{pgfscope}%
\begin{pgfscope}%
\pgfsys@transformshift{2.231479in}{0.672658in}%
\pgfsys@useobject{currentmarker}{}%
\end{pgfscope}%
\begin{pgfscope}%
\pgfsys@transformshift{2.231637in}{0.661265in}%
\pgfsys@useobject{currentmarker}{}%
\end{pgfscope}%
\begin{pgfscope}%
\pgfsys@transformshift{2.231794in}{0.689194in}%
\pgfsys@useobject{currentmarker}{}%
\end{pgfscope}%
\begin{pgfscope}%
\pgfsys@transformshift{2.231951in}{0.689508in}%
\pgfsys@useobject{currentmarker}{}%
\end{pgfscope}%
\begin{pgfscope}%
\pgfsys@transformshift{2.232109in}{0.687662in}%
\pgfsys@useobject{currentmarker}{}%
\end{pgfscope}%
\begin{pgfscope}%
\pgfsys@transformshift{2.232266in}{0.683328in}%
\pgfsys@useobject{currentmarker}{}%
\end{pgfscope}%
\begin{pgfscope}%
\pgfsys@transformshift{2.232423in}{0.661486in}%
\pgfsys@useobject{currentmarker}{}%
\end{pgfscope}%
\begin{pgfscope}%
\pgfsys@transformshift{2.232579in}{0.679501in}%
\pgfsys@useobject{currentmarker}{}%
\end{pgfscope}%
\begin{pgfscope}%
\pgfsys@transformshift{2.232736in}{0.681998in}%
\pgfsys@useobject{currentmarker}{}%
\end{pgfscope}%
\begin{pgfscope}%
\pgfsys@transformshift{2.232893in}{0.711393in}%
\pgfsys@useobject{currentmarker}{}%
\end{pgfscope}%
\begin{pgfscope}%
\pgfsys@transformshift{2.233049in}{0.704705in}%
\pgfsys@useobject{currentmarker}{}%
\end{pgfscope}%
\begin{pgfscope}%
\pgfsys@transformshift{2.233206in}{0.668906in}%
\pgfsys@useobject{currentmarker}{}%
\end{pgfscope}%
\begin{pgfscope}%
\pgfsys@transformshift{2.233362in}{0.699601in}%
\pgfsys@useobject{currentmarker}{}%
\end{pgfscope}%
\begin{pgfscope}%
\pgfsys@transformshift{2.233518in}{0.675201in}%
\pgfsys@useobject{currentmarker}{}%
\end{pgfscope}%
\begin{pgfscope}%
\pgfsys@transformshift{2.233674in}{0.667717in}%
\pgfsys@useobject{currentmarker}{}%
\end{pgfscope}%
\begin{pgfscope}%
\pgfsys@transformshift{2.233830in}{0.657621in}%
\pgfsys@useobject{currentmarker}{}%
\end{pgfscope}%
\begin{pgfscope}%
\pgfsys@transformshift{2.233986in}{0.666936in}%
\pgfsys@useobject{currentmarker}{}%
\end{pgfscope}%
\begin{pgfscope}%
\pgfsys@transformshift{2.234142in}{0.675027in}%
\pgfsys@useobject{currentmarker}{}%
\end{pgfscope}%
\begin{pgfscope}%
\pgfsys@transformshift{2.234297in}{0.717864in}%
\pgfsys@useobject{currentmarker}{}%
\end{pgfscope}%
\begin{pgfscope}%
\pgfsys@transformshift{2.234453in}{0.721882in}%
\pgfsys@useobject{currentmarker}{}%
\end{pgfscope}%
\begin{pgfscope}%
\pgfsys@transformshift{2.234608in}{0.728866in}%
\pgfsys@useobject{currentmarker}{}%
\end{pgfscope}%
\begin{pgfscope}%
\pgfsys@transformshift{2.234763in}{0.706888in}%
\pgfsys@useobject{currentmarker}{}%
\end{pgfscope}%
\begin{pgfscope}%
\pgfsys@transformshift{2.234919in}{0.679311in}%
\pgfsys@useobject{currentmarker}{}%
\end{pgfscope}%
\begin{pgfscope}%
\pgfsys@transformshift{2.235074in}{0.629745in}%
\pgfsys@useobject{currentmarker}{}%
\end{pgfscope}%
\begin{pgfscope}%
\pgfsys@transformshift{2.235229in}{0.732518in}%
\pgfsys@useobject{currentmarker}{}%
\end{pgfscope}%
\begin{pgfscope}%
\pgfsys@transformshift{2.235384in}{0.734973in}%
\pgfsys@useobject{currentmarker}{}%
\end{pgfscope}%
\begin{pgfscope}%
\pgfsys@transformshift{2.235538in}{0.636745in}%
\pgfsys@useobject{currentmarker}{}%
\end{pgfscope}%
\begin{pgfscope}%
\pgfsys@transformshift{2.235693in}{0.643513in}%
\pgfsys@useobject{currentmarker}{}%
\end{pgfscope}%
\begin{pgfscope}%
\pgfsys@transformshift{2.235848in}{0.658157in}%
\pgfsys@useobject{currentmarker}{}%
\end{pgfscope}%
\begin{pgfscope}%
\pgfsys@transformshift{2.236002in}{0.645197in}%
\pgfsys@useobject{currentmarker}{}%
\end{pgfscope}%
\begin{pgfscope}%
\pgfsys@transformshift{2.236156in}{0.705928in}%
\pgfsys@useobject{currentmarker}{}%
\end{pgfscope}%
\begin{pgfscope}%
\pgfsys@transformshift{2.236311in}{0.690141in}%
\pgfsys@useobject{currentmarker}{}%
\end{pgfscope}%
\begin{pgfscope}%
\pgfsys@transformshift{2.236465in}{0.685857in}%
\pgfsys@useobject{currentmarker}{}%
\end{pgfscope}%
\begin{pgfscope}%
\pgfsys@transformshift{2.236619in}{0.639104in}%
\pgfsys@useobject{currentmarker}{}%
\end{pgfscope}%
\begin{pgfscope}%
\pgfsys@transformshift{2.236773in}{0.668784in}%
\pgfsys@useobject{currentmarker}{}%
\end{pgfscope}%
\begin{pgfscope}%
\pgfsys@transformshift{2.236927in}{0.650435in}%
\pgfsys@useobject{currentmarker}{}%
\end{pgfscope}%
\begin{pgfscope}%
\pgfsys@transformshift{2.237080in}{0.660918in}%
\pgfsys@useobject{currentmarker}{}%
\end{pgfscope}%
\begin{pgfscope}%
\pgfsys@transformshift{2.237234in}{0.651859in}%
\pgfsys@useobject{currentmarker}{}%
\end{pgfscope}%
\begin{pgfscope}%
\pgfsys@transformshift{2.237387in}{0.670603in}%
\pgfsys@useobject{currentmarker}{}%
\end{pgfscope}%
\begin{pgfscope}%
\pgfsys@transformshift{2.237541in}{0.705735in}%
\pgfsys@useobject{currentmarker}{}%
\end{pgfscope}%
\begin{pgfscope}%
\pgfsys@transformshift{2.237694in}{0.675600in}%
\pgfsys@useobject{currentmarker}{}%
\end{pgfscope}%
\begin{pgfscope}%
\pgfsys@transformshift{2.237847in}{0.666518in}%
\pgfsys@useobject{currentmarker}{}%
\end{pgfscope}%
\begin{pgfscope}%
\pgfsys@transformshift{2.238000in}{0.697076in}%
\pgfsys@useobject{currentmarker}{}%
\end{pgfscope}%
\begin{pgfscope}%
\pgfsys@transformshift{2.238153in}{0.654820in}%
\pgfsys@useobject{currentmarker}{}%
\end{pgfscope}%
\begin{pgfscope}%
\pgfsys@transformshift{2.238306in}{0.653475in}%
\pgfsys@useobject{currentmarker}{}%
\end{pgfscope}%
\begin{pgfscope}%
\pgfsys@transformshift{2.238459in}{0.700927in}%
\pgfsys@useobject{currentmarker}{}%
\end{pgfscope}%
\begin{pgfscope}%
\pgfsys@transformshift{2.238612in}{0.676631in}%
\pgfsys@useobject{currentmarker}{}%
\end{pgfscope}%
\begin{pgfscope}%
\pgfsys@transformshift{2.238764in}{0.698705in}%
\pgfsys@useobject{currentmarker}{}%
\end{pgfscope}%
\begin{pgfscope}%
\pgfsys@transformshift{2.238917in}{0.716267in}%
\pgfsys@useobject{currentmarker}{}%
\end{pgfscope}%
\begin{pgfscope}%
\pgfsys@transformshift{2.239069in}{0.735411in}%
\pgfsys@useobject{currentmarker}{}%
\end{pgfscope}%
\begin{pgfscope}%
\pgfsys@transformshift{2.239221in}{0.689838in}%
\pgfsys@useobject{currentmarker}{}%
\end{pgfscope}%
\begin{pgfscope}%
\pgfsys@transformshift{2.239373in}{0.697758in}%
\pgfsys@useobject{currentmarker}{}%
\end{pgfscope}%
\begin{pgfscope}%
\pgfsys@transformshift{2.239525in}{0.713092in}%
\pgfsys@useobject{currentmarker}{}%
\end{pgfscope}%
\begin{pgfscope}%
\pgfsys@transformshift{2.239677in}{0.742158in}%
\pgfsys@useobject{currentmarker}{}%
\end{pgfscope}%
\begin{pgfscope}%
\pgfsys@transformshift{2.239829in}{0.725152in}%
\pgfsys@useobject{currentmarker}{}%
\end{pgfscope}%
\begin{pgfscope}%
\pgfsys@transformshift{2.239981in}{0.653973in}%
\pgfsys@useobject{currentmarker}{}%
\end{pgfscope}%
\begin{pgfscope}%
\pgfsys@transformshift{2.240133in}{0.638739in}%
\pgfsys@useobject{currentmarker}{}%
\end{pgfscope}%
\begin{pgfscope}%
\pgfsys@transformshift{2.240284in}{0.643816in}%
\pgfsys@useobject{currentmarker}{}%
\end{pgfscope}%
\begin{pgfscope}%
\pgfsys@transformshift{2.240436in}{0.598254in}%
\pgfsys@useobject{currentmarker}{}%
\end{pgfscope}%
\begin{pgfscope}%
\pgfsys@transformshift{2.240587in}{0.653877in}%
\pgfsys@useobject{currentmarker}{}%
\end{pgfscope}%
\begin{pgfscope}%
\pgfsys@transformshift{2.240738in}{0.641419in}%
\pgfsys@useobject{currentmarker}{}%
\end{pgfscope}%
\begin{pgfscope}%
\pgfsys@transformshift{2.240889in}{0.698568in}%
\pgfsys@useobject{currentmarker}{}%
\end{pgfscope}%
\begin{pgfscope}%
\pgfsys@transformshift{2.241040in}{0.689027in}%
\pgfsys@useobject{currentmarker}{}%
\end{pgfscope}%
\begin{pgfscope}%
\pgfsys@transformshift{2.241191in}{0.717154in}%
\pgfsys@useobject{currentmarker}{}%
\end{pgfscope}%
\begin{pgfscope}%
\pgfsys@transformshift{2.241342in}{0.742262in}%
\pgfsys@useobject{currentmarker}{}%
\end{pgfscope}%
\begin{pgfscope}%
\pgfsys@transformshift{2.241493in}{0.724503in}%
\pgfsys@useobject{currentmarker}{}%
\end{pgfscope}%
\begin{pgfscope}%
\pgfsys@transformshift{2.241643in}{0.702501in}%
\pgfsys@useobject{currentmarker}{}%
\end{pgfscope}%
\begin{pgfscope}%
\pgfsys@transformshift{2.241794in}{0.706021in}%
\pgfsys@useobject{currentmarker}{}%
\end{pgfscope}%
\begin{pgfscope}%
\pgfsys@transformshift{2.241944in}{0.685025in}%
\pgfsys@useobject{currentmarker}{}%
\end{pgfscope}%
\begin{pgfscope}%
\pgfsys@transformshift{2.242095in}{0.645130in}%
\pgfsys@useobject{currentmarker}{}%
\end{pgfscope}%
\begin{pgfscope}%
\pgfsys@transformshift{2.242245in}{0.637856in}%
\pgfsys@useobject{currentmarker}{}%
\end{pgfscope}%
\begin{pgfscope}%
\pgfsys@transformshift{2.242395in}{0.683687in}%
\pgfsys@useobject{currentmarker}{}%
\end{pgfscope}%
\begin{pgfscope}%
\pgfsys@transformshift{2.242545in}{0.667367in}%
\pgfsys@useobject{currentmarker}{}%
\end{pgfscope}%
\begin{pgfscope}%
\pgfsys@transformshift{2.242695in}{0.631481in}%
\pgfsys@useobject{currentmarker}{}%
\end{pgfscope}%
\begin{pgfscope}%
\pgfsys@transformshift{2.242845in}{0.653579in}%
\pgfsys@useobject{currentmarker}{}%
\end{pgfscope}%
\begin{pgfscope}%
\pgfsys@transformshift{2.242994in}{0.639135in}%
\pgfsys@useobject{currentmarker}{}%
\end{pgfscope}%
\begin{pgfscope}%
\pgfsys@transformshift{2.243144in}{0.647941in}%
\pgfsys@useobject{currentmarker}{}%
\end{pgfscope}%
\begin{pgfscope}%
\pgfsys@transformshift{2.243294in}{0.671264in}%
\pgfsys@useobject{currentmarker}{}%
\end{pgfscope}%
\begin{pgfscope}%
\pgfsys@transformshift{2.243443in}{0.715717in}%
\pgfsys@useobject{currentmarker}{}%
\end{pgfscope}%
\begin{pgfscope}%
\pgfsys@transformshift{2.243592in}{0.693670in}%
\pgfsys@useobject{currentmarker}{}%
\end{pgfscope}%
\begin{pgfscope}%
\pgfsys@transformshift{2.243742in}{0.694179in}%
\pgfsys@useobject{currentmarker}{}%
\end{pgfscope}%
\begin{pgfscope}%
\pgfsys@transformshift{2.243891in}{0.704584in}%
\pgfsys@useobject{currentmarker}{}%
\end{pgfscope}%
\begin{pgfscope}%
\pgfsys@transformshift{2.244040in}{0.692247in}%
\pgfsys@useobject{currentmarker}{}%
\end{pgfscope}%
\begin{pgfscope}%
\pgfsys@transformshift{2.244189in}{0.659932in}%
\pgfsys@useobject{currentmarker}{}%
\end{pgfscope}%
\begin{pgfscope}%
\pgfsys@transformshift{2.244337in}{0.646508in}%
\pgfsys@useobject{currentmarker}{}%
\end{pgfscope}%
\begin{pgfscope}%
\pgfsys@transformshift{2.244486in}{0.688065in}%
\pgfsys@useobject{currentmarker}{}%
\end{pgfscope}%
\begin{pgfscope}%
\pgfsys@transformshift{2.244635in}{0.667971in}%
\pgfsys@useobject{currentmarker}{}%
\end{pgfscope}%
\begin{pgfscope}%
\pgfsys@transformshift{2.244783in}{0.658482in}%
\pgfsys@useobject{currentmarker}{}%
\end{pgfscope}%
\begin{pgfscope}%
\pgfsys@transformshift{2.244932in}{0.640259in}%
\pgfsys@useobject{currentmarker}{}%
\end{pgfscope}%
\begin{pgfscope}%
\pgfsys@transformshift{2.245080in}{0.616577in}%
\pgfsys@useobject{currentmarker}{}%
\end{pgfscope}%
\begin{pgfscope}%
\pgfsys@transformshift{2.245228in}{0.659901in}%
\pgfsys@useobject{currentmarker}{}%
\end{pgfscope}%
\begin{pgfscope}%
\pgfsys@transformshift{2.245377in}{0.670329in}%
\pgfsys@useobject{currentmarker}{}%
\end{pgfscope}%
\begin{pgfscope}%
\pgfsys@transformshift{2.245525in}{0.711482in}%
\pgfsys@useobject{currentmarker}{}%
\end{pgfscope}%
\begin{pgfscope}%
\pgfsys@transformshift{2.245672in}{0.734372in}%
\pgfsys@useobject{currentmarker}{}%
\end{pgfscope}%
\begin{pgfscope}%
\pgfsys@transformshift{2.245820in}{0.687755in}%
\pgfsys@useobject{currentmarker}{}%
\end{pgfscope}%
\begin{pgfscope}%
\pgfsys@transformshift{2.245968in}{0.617986in}%
\pgfsys@useobject{currentmarker}{}%
\end{pgfscope}%
\begin{pgfscope}%
\pgfsys@transformshift{2.246116in}{0.632775in}%
\pgfsys@useobject{currentmarker}{}%
\end{pgfscope}%
\begin{pgfscope}%
\pgfsys@transformshift{2.246263in}{0.677654in}%
\pgfsys@useobject{currentmarker}{}%
\end{pgfscope}%
\begin{pgfscope}%
\pgfsys@transformshift{2.246411in}{0.668243in}%
\pgfsys@useobject{currentmarker}{}%
\end{pgfscope}%
\begin{pgfscope}%
\pgfsys@transformshift{2.246558in}{0.698607in}%
\pgfsys@useobject{currentmarker}{}%
\end{pgfscope}%
\begin{pgfscope}%
\pgfsys@transformshift{2.246705in}{0.670122in}%
\pgfsys@useobject{currentmarker}{}%
\end{pgfscope}%
\begin{pgfscope}%
\pgfsys@transformshift{2.246853in}{0.679737in}%
\pgfsys@useobject{currentmarker}{}%
\end{pgfscope}%
\begin{pgfscope}%
\pgfsys@transformshift{2.247000in}{0.724229in}%
\pgfsys@useobject{currentmarker}{}%
\end{pgfscope}%
\begin{pgfscope}%
\pgfsys@transformshift{2.247147in}{0.698804in}%
\pgfsys@useobject{currentmarker}{}%
\end{pgfscope}%
\begin{pgfscope}%
\pgfsys@transformshift{2.247293in}{0.651864in}%
\pgfsys@useobject{currentmarker}{}%
\end{pgfscope}%
\begin{pgfscope}%
\pgfsys@transformshift{2.247440in}{0.645893in}%
\pgfsys@useobject{currentmarker}{}%
\end{pgfscope}%
\begin{pgfscope}%
\pgfsys@transformshift{2.247587in}{0.656626in}%
\pgfsys@useobject{currentmarker}{}%
\end{pgfscope}%
\begin{pgfscope}%
\pgfsys@transformshift{2.247734in}{0.680341in}%
\pgfsys@useobject{currentmarker}{}%
\end{pgfscope}%
\begin{pgfscope}%
\pgfsys@transformshift{2.247880in}{0.687878in}%
\pgfsys@useobject{currentmarker}{}%
\end{pgfscope}%
\begin{pgfscope}%
\pgfsys@transformshift{2.248026in}{0.699692in}%
\pgfsys@useobject{currentmarker}{}%
\end{pgfscope}%
\begin{pgfscope}%
\pgfsys@transformshift{2.248173in}{0.638129in}%
\pgfsys@useobject{currentmarker}{}%
\end{pgfscope}%
\begin{pgfscope}%
\pgfsys@transformshift{2.248319in}{0.656535in}%
\pgfsys@useobject{currentmarker}{}%
\end{pgfscope}%
\begin{pgfscope}%
\pgfsys@transformshift{2.248465in}{0.654489in}%
\pgfsys@useobject{currentmarker}{}%
\end{pgfscope}%
\begin{pgfscope}%
\pgfsys@transformshift{2.248611in}{0.634004in}%
\pgfsys@useobject{currentmarker}{}%
\end{pgfscope}%
\begin{pgfscope}%
\pgfsys@transformshift{2.248757in}{0.651584in}%
\pgfsys@useobject{currentmarker}{}%
\end{pgfscope}%
\begin{pgfscope}%
\pgfsys@transformshift{2.248903in}{0.671580in}%
\pgfsys@useobject{currentmarker}{}%
\end{pgfscope}%
\begin{pgfscope}%
\pgfsys@transformshift{2.249049in}{0.691352in}%
\pgfsys@useobject{currentmarker}{}%
\end{pgfscope}%
\begin{pgfscope}%
\pgfsys@transformshift{2.249194in}{0.689325in}%
\pgfsys@useobject{currentmarker}{}%
\end{pgfscope}%
\begin{pgfscope}%
\pgfsys@transformshift{2.249340in}{0.700260in}%
\pgfsys@useobject{currentmarker}{}%
\end{pgfscope}%
\begin{pgfscope}%
\pgfsys@transformshift{2.249485in}{0.686621in}%
\pgfsys@useobject{currentmarker}{}%
\end{pgfscope}%
\begin{pgfscope}%
\pgfsys@transformshift{2.249631in}{0.652202in}%
\pgfsys@useobject{currentmarker}{}%
\end{pgfscope}%
\begin{pgfscope}%
\pgfsys@transformshift{2.249776in}{0.680983in}%
\pgfsys@useobject{currentmarker}{}%
\end{pgfscope}%
\begin{pgfscope}%
\pgfsys@transformshift{2.249921in}{0.690999in}%
\pgfsys@useobject{currentmarker}{}%
\end{pgfscope}%
\begin{pgfscope}%
\pgfsys@transformshift{2.250066in}{0.597394in}%
\pgfsys@useobject{currentmarker}{}%
\end{pgfscope}%
\begin{pgfscope}%
\pgfsys@transformshift{2.250211in}{0.615952in}%
\pgfsys@useobject{currentmarker}{}%
\end{pgfscope}%
\begin{pgfscope}%
\pgfsys@transformshift{2.250356in}{0.637657in}%
\pgfsys@useobject{currentmarker}{}%
\end{pgfscope}%
\begin{pgfscope}%
\pgfsys@transformshift{2.250501in}{0.709751in}%
\pgfsys@useobject{currentmarker}{}%
\end{pgfscope}%
\begin{pgfscope}%
\pgfsys@transformshift{2.250645in}{0.710290in}%
\pgfsys@useobject{currentmarker}{}%
\end{pgfscope}%
\begin{pgfscope}%
\pgfsys@transformshift{2.250790in}{0.683359in}%
\pgfsys@useobject{currentmarker}{}%
\end{pgfscope}%
\begin{pgfscope}%
\pgfsys@transformshift{2.250935in}{0.653658in}%
\pgfsys@useobject{currentmarker}{}%
\end{pgfscope}%
\begin{pgfscope}%
\pgfsys@transformshift{2.251079in}{0.689323in}%
\pgfsys@useobject{currentmarker}{}%
\end{pgfscope}%
\begin{pgfscope}%
\pgfsys@transformshift{2.251223in}{0.701257in}%
\pgfsys@useobject{currentmarker}{}%
\end{pgfscope}%
\begin{pgfscope}%
\pgfsys@transformshift{2.251368in}{0.679560in}%
\pgfsys@useobject{currentmarker}{}%
\end{pgfscope}%
\begin{pgfscope}%
\pgfsys@transformshift{2.251512in}{0.679012in}%
\pgfsys@useobject{currentmarker}{}%
\end{pgfscope}%
\begin{pgfscope}%
\pgfsys@transformshift{2.251656in}{0.693315in}%
\pgfsys@useobject{currentmarker}{}%
\end{pgfscope}%
\begin{pgfscope}%
\pgfsys@transformshift{2.251800in}{0.707314in}%
\pgfsys@useobject{currentmarker}{}%
\end{pgfscope}%
\begin{pgfscope}%
\pgfsys@transformshift{2.251944in}{0.695550in}%
\pgfsys@useobject{currentmarker}{}%
\end{pgfscope}%
\begin{pgfscope}%
\pgfsys@transformshift{2.252087in}{0.711409in}%
\pgfsys@useobject{currentmarker}{}%
\end{pgfscope}%
\begin{pgfscope}%
\pgfsys@transformshift{2.252231in}{0.710220in}%
\pgfsys@useobject{currentmarker}{}%
\end{pgfscope}%
\begin{pgfscope}%
\pgfsys@transformshift{2.252375in}{0.657986in}%
\pgfsys@useobject{currentmarker}{}%
\end{pgfscope}%
\begin{pgfscope}%
\pgfsys@transformshift{2.252518in}{0.654439in}%
\pgfsys@useobject{currentmarker}{}%
\end{pgfscope}%
\begin{pgfscope}%
\pgfsys@transformshift{2.252662in}{0.638055in}%
\pgfsys@useobject{currentmarker}{}%
\end{pgfscope}%
\begin{pgfscope}%
\pgfsys@transformshift{2.252805in}{0.676442in}%
\pgfsys@useobject{currentmarker}{}%
\end{pgfscope}%
\begin{pgfscope}%
\pgfsys@transformshift{2.252948in}{0.663980in}%
\pgfsys@useobject{currentmarker}{}%
\end{pgfscope}%
\begin{pgfscope}%
\pgfsys@transformshift{2.253091in}{0.676592in}%
\pgfsys@useobject{currentmarker}{}%
\end{pgfscope}%
\begin{pgfscope}%
\pgfsys@transformshift{2.253234in}{0.695366in}%
\pgfsys@useobject{currentmarker}{}%
\end{pgfscope}%
\begin{pgfscope}%
\pgfsys@transformshift{2.253377in}{0.660301in}%
\pgfsys@useobject{currentmarker}{}%
\end{pgfscope}%
\begin{pgfscope}%
\pgfsys@transformshift{2.253520in}{0.692952in}%
\pgfsys@useobject{currentmarker}{}%
\end{pgfscope}%
\begin{pgfscope}%
\pgfsys@transformshift{2.253663in}{0.724638in}%
\pgfsys@useobject{currentmarker}{}%
\end{pgfscope}%
\begin{pgfscope}%
\pgfsys@transformshift{2.253806in}{0.694144in}%
\pgfsys@useobject{currentmarker}{}%
\end{pgfscope}%
\begin{pgfscope}%
\pgfsys@transformshift{2.253948in}{0.662542in}%
\pgfsys@useobject{currentmarker}{}%
\end{pgfscope}%
\begin{pgfscope}%
\pgfsys@transformshift{2.254091in}{0.687305in}%
\pgfsys@useobject{currentmarker}{}%
\end{pgfscope}%
\begin{pgfscope}%
\pgfsys@transformshift{2.254233in}{0.679545in}%
\pgfsys@useobject{currentmarker}{}%
\end{pgfscope}%
\begin{pgfscope}%
\pgfsys@transformshift{2.254375in}{0.654846in}%
\pgfsys@useobject{currentmarker}{}%
\end{pgfscope}%
\begin{pgfscope}%
\pgfsys@transformshift{2.254518in}{0.640337in}%
\pgfsys@useobject{currentmarker}{}%
\end{pgfscope}%
\begin{pgfscope}%
\pgfsys@transformshift{2.254660in}{0.682372in}%
\pgfsys@useobject{currentmarker}{}%
\end{pgfscope}%
\begin{pgfscope}%
\pgfsys@transformshift{2.254802in}{0.714596in}%
\pgfsys@useobject{currentmarker}{}%
\end{pgfscope}%
\begin{pgfscope}%
\pgfsys@transformshift{2.254944in}{0.678984in}%
\pgfsys@useobject{currentmarker}{}%
\end{pgfscope}%
\begin{pgfscope}%
\pgfsys@transformshift{2.255086in}{0.672408in}%
\pgfsys@useobject{currentmarker}{}%
\end{pgfscope}%
\begin{pgfscope}%
\pgfsys@transformshift{2.255227in}{0.686225in}%
\pgfsys@useobject{currentmarker}{}%
\end{pgfscope}%
\begin{pgfscope}%
\pgfsys@transformshift{2.255369in}{0.689422in}%
\pgfsys@useobject{currentmarker}{}%
\end{pgfscope}%
\begin{pgfscope}%
\pgfsys@transformshift{2.255511in}{0.663292in}%
\pgfsys@useobject{currentmarker}{}%
\end{pgfscope}%
\begin{pgfscope}%
\pgfsys@transformshift{2.255652in}{0.700407in}%
\pgfsys@useobject{currentmarker}{}%
\end{pgfscope}%
\begin{pgfscope}%
\pgfsys@transformshift{2.255794in}{0.681218in}%
\pgfsys@useobject{currentmarker}{}%
\end{pgfscope}%
\begin{pgfscope}%
\pgfsys@transformshift{2.255935in}{0.699670in}%
\pgfsys@useobject{currentmarker}{}%
\end{pgfscope}%
\begin{pgfscope}%
\pgfsys@transformshift{2.256076in}{0.654761in}%
\pgfsys@useobject{currentmarker}{}%
\end{pgfscope}%
\begin{pgfscope}%
\pgfsys@transformshift{2.256217in}{0.662021in}%
\pgfsys@useobject{currentmarker}{}%
\end{pgfscope}%
\begin{pgfscope}%
\pgfsys@transformshift{2.256358in}{0.655766in}%
\pgfsys@useobject{currentmarker}{}%
\end{pgfscope}%
\begin{pgfscope}%
\pgfsys@transformshift{2.256499in}{0.630917in}%
\pgfsys@useobject{currentmarker}{}%
\end{pgfscope}%
\begin{pgfscope}%
\pgfsys@transformshift{2.256640in}{0.689770in}%
\pgfsys@useobject{currentmarker}{}%
\end{pgfscope}%
\begin{pgfscope}%
\pgfsys@transformshift{2.256781in}{0.654405in}%
\pgfsys@useobject{currentmarker}{}%
\end{pgfscope}%
\begin{pgfscope}%
\pgfsys@transformshift{2.256922in}{0.653953in}%
\pgfsys@useobject{currentmarker}{}%
\end{pgfscope}%
\begin{pgfscope}%
\pgfsys@transformshift{2.257062in}{0.696136in}%
\pgfsys@useobject{currentmarker}{}%
\end{pgfscope}%
\begin{pgfscope}%
\pgfsys@transformshift{2.257203in}{0.697490in}%
\pgfsys@useobject{currentmarker}{}%
\end{pgfscope}%
\begin{pgfscope}%
\pgfsys@transformshift{2.257343in}{0.672456in}%
\pgfsys@useobject{currentmarker}{}%
\end{pgfscope}%
\begin{pgfscope}%
\pgfsys@transformshift{2.257484in}{0.636366in}%
\pgfsys@useobject{currentmarker}{}%
\end{pgfscope}%
\begin{pgfscope}%
\pgfsys@transformshift{2.257624in}{0.620733in}%
\pgfsys@useobject{currentmarker}{}%
\end{pgfscope}%
\begin{pgfscope}%
\pgfsys@transformshift{2.257764in}{0.646307in}%
\pgfsys@useobject{currentmarker}{}%
\end{pgfscope}%
\begin{pgfscope}%
\pgfsys@transformshift{2.257904in}{0.643135in}%
\pgfsys@useobject{currentmarker}{}%
\end{pgfscope}%
\begin{pgfscope}%
\pgfsys@transformshift{2.258044in}{0.634727in}%
\pgfsys@useobject{currentmarker}{}%
\end{pgfscope}%
\begin{pgfscope}%
\pgfsys@transformshift{2.258184in}{0.674726in}%
\pgfsys@useobject{currentmarker}{}%
\end{pgfscope}%
\begin{pgfscope}%
\pgfsys@transformshift{2.258324in}{0.627905in}%
\pgfsys@useobject{currentmarker}{}%
\end{pgfscope}%
\begin{pgfscope}%
\pgfsys@transformshift{2.258464in}{0.590248in}%
\pgfsys@useobject{currentmarker}{}%
\end{pgfscope}%
\begin{pgfscope}%
\pgfsys@transformshift{2.258604in}{0.691105in}%
\pgfsys@useobject{currentmarker}{}%
\end{pgfscope}%
\begin{pgfscope}%
\pgfsys@transformshift{2.258743in}{0.707850in}%
\pgfsys@useobject{currentmarker}{}%
\end{pgfscope}%
\begin{pgfscope}%
\pgfsys@transformshift{2.258883in}{0.691628in}%
\pgfsys@useobject{currentmarker}{}%
\end{pgfscope}%
\begin{pgfscope}%
\pgfsys@transformshift{2.259022in}{0.727400in}%
\pgfsys@useobject{currentmarker}{}%
\end{pgfscope}%
\begin{pgfscope}%
\pgfsys@transformshift{2.259161in}{0.701337in}%
\pgfsys@useobject{currentmarker}{}%
\end{pgfscope}%
\begin{pgfscope}%
\pgfsys@transformshift{2.259301in}{0.675359in}%
\pgfsys@useobject{currentmarker}{}%
\end{pgfscope}%
\begin{pgfscope}%
\pgfsys@transformshift{2.259440in}{0.627953in}%
\pgfsys@useobject{currentmarker}{}%
\end{pgfscope}%
\begin{pgfscope}%
\pgfsys@transformshift{2.259579in}{0.629988in}%
\pgfsys@useobject{currentmarker}{}%
\end{pgfscope}%
\begin{pgfscope}%
\pgfsys@transformshift{2.259718in}{0.662198in}%
\pgfsys@useobject{currentmarker}{}%
\end{pgfscope}%
\begin{pgfscope}%
\pgfsys@transformshift{2.259857in}{0.668037in}%
\pgfsys@useobject{currentmarker}{}%
\end{pgfscope}%
\begin{pgfscope}%
\pgfsys@transformshift{2.259995in}{0.658715in}%
\pgfsys@useobject{currentmarker}{}%
\end{pgfscope}%
\begin{pgfscope}%
\pgfsys@transformshift{2.260134in}{0.688499in}%
\pgfsys@useobject{currentmarker}{}%
\end{pgfscope}%
\begin{pgfscope}%
\pgfsys@transformshift{2.260273in}{0.732446in}%
\pgfsys@useobject{currentmarker}{}%
\end{pgfscope}%
\begin{pgfscope}%
\pgfsys@transformshift{2.260411in}{0.687231in}%
\pgfsys@useobject{currentmarker}{}%
\end{pgfscope}%
\begin{pgfscope}%
\pgfsys@transformshift{2.260550in}{0.715975in}%
\pgfsys@useobject{currentmarker}{}%
\end{pgfscope}%
\begin{pgfscope}%
\pgfsys@transformshift{2.260688in}{0.718137in}%
\pgfsys@useobject{currentmarker}{}%
\end{pgfscope}%
\begin{pgfscope}%
\pgfsys@transformshift{2.260827in}{0.681819in}%
\pgfsys@useobject{currentmarker}{}%
\end{pgfscope}%
\begin{pgfscope}%
\pgfsys@transformshift{2.260965in}{0.639055in}%
\pgfsys@useobject{currentmarker}{}%
\end{pgfscope}%
\begin{pgfscope}%
\pgfsys@transformshift{2.261103in}{0.692238in}%
\pgfsys@useobject{currentmarker}{}%
\end{pgfscope}%
\begin{pgfscope}%
\pgfsys@transformshift{2.261241in}{0.725174in}%
\pgfsys@useobject{currentmarker}{}%
\end{pgfscope}%
\begin{pgfscope}%
\pgfsys@transformshift{2.261379in}{0.700106in}%
\pgfsys@useobject{currentmarker}{}%
\end{pgfscope}%
\begin{pgfscope}%
\pgfsys@transformshift{2.261517in}{0.679129in}%
\pgfsys@useobject{currentmarker}{}%
\end{pgfscope}%
\begin{pgfscope}%
\pgfsys@transformshift{2.261654in}{0.703659in}%
\pgfsys@useobject{currentmarker}{}%
\end{pgfscope}%
\begin{pgfscope}%
\pgfsys@transformshift{2.261792in}{0.687547in}%
\pgfsys@useobject{currentmarker}{}%
\end{pgfscope}%
\begin{pgfscope}%
\pgfsys@transformshift{2.261930in}{0.634297in}%
\pgfsys@useobject{currentmarker}{}%
\end{pgfscope}%
\begin{pgfscope}%
\pgfsys@transformshift{2.262067in}{0.681215in}%
\pgfsys@useobject{currentmarker}{}%
\end{pgfscope}%
\begin{pgfscope}%
\pgfsys@transformshift{2.262205in}{0.672779in}%
\pgfsys@useobject{currentmarker}{}%
\end{pgfscope}%
\begin{pgfscope}%
\pgfsys@transformshift{2.262342in}{0.659315in}%
\pgfsys@useobject{currentmarker}{}%
\end{pgfscope}%
\begin{pgfscope}%
\pgfsys@transformshift{2.262479in}{0.654322in}%
\pgfsys@useobject{currentmarker}{}%
\end{pgfscope}%
\begin{pgfscope}%
\pgfsys@transformshift{2.262617in}{0.681712in}%
\pgfsys@useobject{currentmarker}{}%
\end{pgfscope}%
\begin{pgfscope}%
\pgfsys@transformshift{2.262754in}{0.663223in}%
\pgfsys@useobject{currentmarker}{}%
\end{pgfscope}%
\begin{pgfscope}%
\pgfsys@transformshift{2.262891in}{0.636156in}%
\pgfsys@useobject{currentmarker}{}%
\end{pgfscope}%
\begin{pgfscope}%
\pgfsys@transformshift{2.263028in}{0.703908in}%
\pgfsys@useobject{currentmarker}{}%
\end{pgfscope}%
\begin{pgfscope}%
\pgfsys@transformshift{2.263165in}{0.692141in}%
\pgfsys@useobject{currentmarker}{}%
\end{pgfscope}%
\begin{pgfscope}%
\pgfsys@transformshift{2.263301in}{0.696151in}%
\pgfsys@useobject{currentmarker}{}%
\end{pgfscope}%
\begin{pgfscope}%
\pgfsys@transformshift{2.263438in}{0.680734in}%
\pgfsys@useobject{currentmarker}{}%
\end{pgfscope}%
\begin{pgfscope}%
\pgfsys@transformshift{2.263575in}{0.656500in}%
\pgfsys@useobject{currentmarker}{}%
\end{pgfscope}%
\begin{pgfscope}%
\pgfsys@transformshift{2.263711in}{0.657229in}%
\pgfsys@useobject{currentmarker}{}%
\end{pgfscope}%
\begin{pgfscope}%
\pgfsys@transformshift{2.263848in}{0.624708in}%
\pgfsys@useobject{currentmarker}{}%
\end{pgfscope}%
\begin{pgfscope}%
\pgfsys@transformshift{2.263984in}{0.593841in}%
\pgfsys@useobject{currentmarker}{}%
\end{pgfscope}%
\begin{pgfscope}%
\pgfsys@transformshift{2.264120in}{0.672873in}%
\pgfsys@useobject{currentmarker}{}%
\end{pgfscope}%
\begin{pgfscope}%
\pgfsys@transformshift{2.264256in}{0.645601in}%
\pgfsys@useobject{currentmarker}{}%
\end{pgfscope}%
\begin{pgfscope}%
\pgfsys@transformshift{2.264392in}{0.653257in}%
\pgfsys@useobject{currentmarker}{}%
\end{pgfscope}%
\begin{pgfscope}%
\pgfsys@transformshift{2.264529in}{0.721783in}%
\pgfsys@useobject{currentmarker}{}%
\end{pgfscope}%
\begin{pgfscope}%
\pgfsys@transformshift{2.264664in}{0.733285in}%
\pgfsys@useobject{currentmarker}{}%
\end{pgfscope}%
\begin{pgfscope}%
\pgfsys@transformshift{2.264800in}{0.722527in}%
\pgfsys@useobject{currentmarker}{}%
\end{pgfscope}%
\begin{pgfscope}%
\pgfsys@transformshift{2.264936in}{0.677941in}%
\pgfsys@useobject{currentmarker}{}%
\end{pgfscope}%
\begin{pgfscope}%
\pgfsys@transformshift{2.265072in}{0.714649in}%
\pgfsys@useobject{currentmarker}{}%
\end{pgfscope}%
\begin{pgfscope}%
\pgfsys@transformshift{2.265207in}{0.709469in}%
\pgfsys@useobject{currentmarker}{}%
\end{pgfscope}%
\begin{pgfscope}%
\pgfsys@transformshift{2.265343in}{0.612561in}%
\pgfsys@useobject{currentmarker}{}%
\end{pgfscope}%
\begin{pgfscope}%
\pgfsys@transformshift{2.265478in}{0.656899in}%
\pgfsys@useobject{currentmarker}{}%
\end{pgfscope}%
\begin{pgfscope}%
\pgfsys@transformshift{2.265614in}{0.705286in}%
\pgfsys@useobject{currentmarker}{}%
\end{pgfscope}%
\begin{pgfscope}%
\pgfsys@transformshift{2.265749in}{0.669695in}%
\pgfsys@useobject{currentmarker}{}%
\end{pgfscope}%
\begin{pgfscope}%
\pgfsys@transformshift{2.265884in}{0.648200in}%
\pgfsys@useobject{currentmarker}{}%
\end{pgfscope}%
\begin{pgfscope}%
\pgfsys@transformshift{2.266019in}{0.658233in}%
\pgfsys@useobject{currentmarker}{}%
\end{pgfscope}%
\begin{pgfscope}%
\pgfsys@transformshift{2.266154in}{0.676541in}%
\pgfsys@useobject{currentmarker}{}%
\end{pgfscope}%
\begin{pgfscope}%
\pgfsys@transformshift{2.266289in}{0.670689in}%
\pgfsys@useobject{currentmarker}{}%
\end{pgfscope}%
\begin{pgfscope}%
\pgfsys@transformshift{2.266424in}{0.676695in}%
\pgfsys@useobject{currentmarker}{}%
\end{pgfscope}%
\begin{pgfscope}%
\pgfsys@transformshift{2.266559in}{0.690134in}%
\pgfsys@useobject{currentmarker}{}%
\end{pgfscope}%
\begin{pgfscope}%
\pgfsys@transformshift{2.266694in}{0.717867in}%
\pgfsys@useobject{currentmarker}{}%
\end{pgfscope}%
\begin{pgfscope}%
\pgfsys@transformshift{2.266828in}{0.702821in}%
\pgfsys@useobject{currentmarker}{}%
\end{pgfscope}%
\begin{pgfscope}%
\pgfsys@transformshift{2.266963in}{0.704742in}%
\pgfsys@useobject{currentmarker}{}%
\end{pgfscope}%
\begin{pgfscope}%
\pgfsys@transformshift{2.267098in}{0.696943in}%
\pgfsys@useobject{currentmarker}{}%
\end{pgfscope}%
\begin{pgfscope}%
\pgfsys@transformshift{2.267232in}{0.703939in}%
\pgfsys@useobject{currentmarker}{}%
\end{pgfscope}%
\begin{pgfscope}%
\pgfsys@transformshift{2.267366in}{0.689822in}%
\pgfsys@useobject{currentmarker}{}%
\end{pgfscope}%
\begin{pgfscope}%
\pgfsys@transformshift{2.267501in}{0.645271in}%
\pgfsys@useobject{currentmarker}{}%
\end{pgfscope}%
\begin{pgfscope}%
\pgfsys@transformshift{2.267635in}{0.621526in}%
\pgfsys@useobject{currentmarker}{}%
\end{pgfscope}%
\begin{pgfscope}%
\pgfsys@transformshift{2.267769in}{0.675530in}%
\pgfsys@useobject{currentmarker}{}%
\end{pgfscope}%
\begin{pgfscope}%
\pgfsys@transformshift{2.267903in}{0.704719in}%
\pgfsys@useobject{currentmarker}{}%
\end{pgfscope}%
\begin{pgfscope}%
\pgfsys@transformshift{2.268037in}{0.608201in}%
\pgfsys@useobject{currentmarker}{}%
\end{pgfscope}%
\begin{pgfscope}%
\pgfsys@transformshift{2.268171in}{0.622905in}%
\pgfsys@useobject{currentmarker}{}%
\end{pgfscope}%
\begin{pgfscope}%
\pgfsys@transformshift{2.268304in}{0.697939in}%
\pgfsys@useobject{currentmarker}{}%
\end{pgfscope}%
\begin{pgfscope}%
\pgfsys@transformshift{2.268438in}{0.669543in}%
\pgfsys@useobject{currentmarker}{}%
\end{pgfscope}%
\begin{pgfscope}%
\pgfsys@transformshift{2.268572in}{0.695268in}%
\pgfsys@useobject{currentmarker}{}%
\end{pgfscope}%
\begin{pgfscope}%
\pgfsys@transformshift{2.268705in}{0.705057in}%
\pgfsys@useobject{currentmarker}{}%
\end{pgfscope}%
\begin{pgfscope}%
\pgfsys@transformshift{2.268839in}{0.684384in}%
\pgfsys@useobject{currentmarker}{}%
\end{pgfscope}%
\begin{pgfscope}%
\pgfsys@transformshift{2.268972in}{0.656603in}%
\pgfsys@useobject{currentmarker}{}%
\end{pgfscope}%
\begin{pgfscope}%
\pgfsys@transformshift{2.269105in}{0.630561in}%
\pgfsys@useobject{currentmarker}{}%
\end{pgfscope}%
\begin{pgfscope}%
\pgfsys@transformshift{2.269238in}{0.717822in}%
\pgfsys@useobject{currentmarker}{}%
\end{pgfscope}%
\begin{pgfscope}%
\pgfsys@transformshift{2.269371in}{0.737095in}%
\pgfsys@useobject{currentmarker}{}%
\end{pgfscope}%
\begin{pgfscope}%
\pgfsys@transformshift{2.269505in}{0.682879in}%
\pgfsys@useobject{currentmarker}{}%
\end{pgfscope}%
\begin{pgfscope}%
\pgfsys@transformshift{2.269638in}{0.665629in}%
\pgfsys@useobject{currentmarker}{}%
\end{pgfscope}%
\begin{pgfscope}%
\pgfsys@transformshift{2.269770in}{0.641309in}%
\pgfsys@useobject{currentmarker}{}%
\end{pgfscope}%
\begin{pgfscope}%
\pgfsys@transformshift{2.269903in}{0.675560in}%
\pgfsys@useobject{currentmarker}{}%
\end{pgfscope}%
\begin{pgfscope}%
\pgfsys@transformshift{2.270036in}{0.703418in}%
\pgfsys@useobject{currentmarker}{}%
\end{pgfscope}%
\begin{pgfscope}%
\pgfsys@transformshift{2.270169in}{0.618859in}%
\pgfsys@useobject{currentmarker}{}%
\end{pgfscope}%
\begin{pgfscope}%
\pgfsys@transformshift{2.270301in}{0.677442in}%
\pgfsys@useobject{currentmarker}{}%
\end{pgfscope}%
\begin{pgfscope}%
\pgfsys@transformshift{2.270434in}{0.684949in}%
\pgfsys@useobject{currentmarker}{}%
\end{pgfscope}%
\begin{pgfscope}%
\pgfsys@transformshift{2.270566in}{0.625947in}%
\pgfsys@useobject{currentmarker}{}%
\end{pgfscope}%
\begin{pgfscope}%
\pgfsys@transformshift{2.270698in}{0.662892in}%
\pgfsys@useobject{currentmarker}{}%
\end{pgfscope}%
\begin{pgfscope}%
\pgfsys@transformshift{2.270831in}{0.680748in}%
\pgfsys@useobject{currentmarker}{}%
\end{pgfscope}%
\begin{pgfscope}%
\pgfsys@transformshift{2.270963in}{0.710043in}%
\pgfsys@useobject{currentmarker}{}%
\end{pgfscope}%
\begin{pgfscope}%
\pgfsys@transformshift{2.271095in}{0.710242in}%
\pgfsys@useobject{currentmarker}{}%
\end{pgfscope}%
\begin{pgfscope}%
\pgfsys@transformshift{2.271227in}{0.639190in}%
\pgfsys@useobject{currentmarker}{}%
\end{pgfscope}%
\begin{pgfscope}%
\pgfsys@transformshift{2.271359in}{0.663887in}%
\pgfsys@useobject{currentmarker}{}%
\end{pgfscope}%
\begin{pgfscope}%
\pgfsys@transformshift{2.271491in}{0.706935in}%
\pgfsys@useobject{currentmarker}{}%
\end{pgfscope}%
\begin{pgfscope}%
\pgfsys@transformshift{2.271623in}{0.686287in}%
\pgfsys@useobject{currentmarker}{}%
\end{pgfscope}%
\begin{pgfscope}%
\pgfsys@transformshift{2.271754in}{0.687468in}%
\pgfsys@useobject{currentmarker}{}%
\end{pgfscope}%
\begin{pgfscope}%
\pgfsys@transformshift{2.271886in}{0.694717in}%
\pgfsys@useobject{currentmarker}{}%
\end{pgfscope}%
\begin{pgfscope}%
\pgfsys@transformshift{2.272018in}{0.689393in}%
\pgfsys@useobject{currentmarker}{}%
\end{pgfscope}%
\begin{pgfscope}%
\pgfsys@transformshift{2.272149in}{0.699422in}%
\pgfsys@useobject{currentmarker}{}%
\end{pgfscope}%
\begin{pgfscope}%
\pgfsys@transformshift{2.272281in}{0.689321in}%
\pgfsys@useobject{currentmarker}{}%
\end{pgfscope}%
\begin{pgfscope}%
\pgfsys@transformshift{2.272412in}{0.632581in}%
\pgfsys@useobject{currentmarker}{}%
\end{pgfscope}%
\begin{pgfscope}%
\pgfsys@transformshift{2.272543in}{0.621202in}%
\pgfsys@useobject{currentmarker}{}%
\end{pgfscope}%
\begin{pgfscope}%
\pgfsys@transformshift{2.272674in}{0.640186in}%
\pgfsys@useobject{currentmarker}{}%
\end{pgfscope}%
\begin{pgfscope}%
\pgfsys@transformshift{2.272805in}{0.663630in}%
\pgfsys@useobject{currentmarker}{}%
\end{pgfscope}%
\begin{pgfscope}%
\pgfsys@transformshift{2.272936in}{0.634707in}%
\pgfsys@useobject{currentmarker}{}%
\end{pgfscope}%
\begin{pgfscope}%
\pgfsys@transformshift{2.273067in}{0.634922in}%
\pgfsys@useobject{currentmarker}{}%
\end{pgfscope}%
\begin{pgfscope}%
\pgfsys@transformshift{2.273198in}{0.660453in}%
\pgfsys@useobject{currentmarker}{}%
\end{pgfscope}%
\begin{pgfscope}%
\pgfsys@transformshift{2.273329in}{0.694990in}%
\pgfsys@useobject{currentmarker}{}%
\end{pgfscope}%
\begin{pgfscope}%
\pgfsys@transformshift{2.273460in}{0.678135in}%
\pgfsys@useobject{currentmarker}{}%
\end{pgfscope}%
\begin{pgfscope}%
\pgfsys@transformshift{2.273590in}{0.640511in}%
\pgfsys@useobject{currentmarker}{}%
\end{pgfscope}%
\begin{pgfscope}%
\pgfsys@transformshift{2.273721in}{0.682518in}%
\pgfsys@useobject{currentmarker}{}%
\end{pgfscope}%
\begin{pgfscope}%
\pgfsys@transformshift{2.273852in}{0.659285in}%
\pgfsys@useobject{currentmarker}{}%
\end{pgfscope}%
\begin{pgfscope}%
\pgfsys@transformshift{2.273982in}{0.684687in}%
\pgfsys@useobject{currentmarker}{}%
\end{pgfscope}%
\begin{pgfscope}%
\pgfsys@transformshift{2.274112in}{0.683009in}%
\pgfsys@useobject{currentmarker}{}%
\end{pgfscope}%
\begin{pgfscope}%
\pgfsys@transformshift{2.274243in}{0.674116in}%
\pgfsys@useobject{currentmarker}{}%
\end{pgfscope}%
\begin{pgfscope}%
\pgfsys@transformshift{2.274373in}{0.638475in}%
\pgfsys@useobject{currentmarker}{}%
\end{pgfscope}%
\begin{pgfscope}%
\pgfsys@transformshift{2.274503in}{0.694480in}%
\pgfsys@useobject{currentmarker}{}%
\end{pgfscope}%
\begin{pgfscope}%
\pgfsys@transformshift{2.274633in}{0.709923in}%
\pgfsys@useobject{currentmarker}{}%
\end{pgfscope}%
\begin{pgfscope}%
\pgfsys@transformshift{2.274763in}{0.652359in}%
\pgfsys@useobject{currentmarker}{}%
\end{pgfscope}%
\begin{pgfscope}%
\pgfsys@transformshift{2.274893in}{0.657751in}%
\pgfsys@useobject{currentmarker}{}%
\end{pgfscope}%
\begin{pgfscope}%
\pgfsys@transformshift{2.275023in}{0.659568in}%
\pgfsys@useobject{currentmarker}{}%
\end{pgfscope}%
\begin{pgfscope}%
\pgfsys@transformshift{2.275152in}{0.628808in}%
\pgfsys@useobject{currentmarker}{}%
\end{pgfscope}%
\begin{pgfscope}%
\pgfsys@transformshift{2.275282in}{0.639788in}%
\pgfsys@useobject{currentmarker}{}%
\end{pgfscope}%
\begin{pgfscope}%
\pgfsys@transformshift{2.275412in}{0.671710in}%
\pgfsys@useobject{currentmarker}{}%
\end{pgfscope}%
\begin{pgfscope}%
\pgfsys@transformshift{2.275541in}{0.739571in}%
\pgfsys@useobject{currentmarker}{}%
\end{pgfscope}%
\begin{pgfscope}%
\pgfsys@transformshift{2.275671in}{0.705023in}%
\pgfsys@useobject{currentmarker}{}%
\end{pgfscope}%
\begin{pgfscope}%
\pgfsys@transformshift{2.275800in}{0.617422in}%
\pgfsys@useobject{currentmarker}{}%
\end{pgfscope}%
\begin{pgfscope}%
\pgfsys@transformshift{2.275929in}{0.621267in}%
\pgfsys@useobject{currentmarker}{}%
\end{pgfscope}%
\begin{pgfscope}%
\pgfsys@transformshift{2.276058in}{0.654245in}%
\pgfsys@useobject{currentmarker}{}%
\end{pgfscope}%
\begin{pgfscope}%
\pgfsys@transformshift{2.276188in}{0.675421in}%
\pgfsys@useobject{currentmarker}{}%
\end{pgfscope}%
\begin{pgfscope}%
\pgfsys@transformshift{2.276317in}{0.662926in}%
\pgfsys@useobject{currentmarker}{}%
\end{pgfscope}%
\begin{pgfscope}%
\pgfsys@transformshift{2.276446in}{0.593707in}%
\pgfsys@useobject{currentmarker}{}%
\end{pgfscope}%
\begin{pgfscope}%
\pgfsys@transformshift{2.276575in}{0.699276in}%
\pgfsys@useobject{currentmarker}{}%
\end{pgfscope}%
\begin{pgfscope}%
\pgfsys@transformshift{2.276703in}{0.757518in}%
\pgfsys@useobject{currentmarker}{}%
\end{pgfscope}%
\begin{pgfscope}%
\pgfsys@transformshift{2.276832in}{0.761603in}%
\pgfsys@useobject{currentmarker}{}%
\end{pgfscope}%
\begin{pgfscope}%
\pgfsys@transformshift{2.276961in}{0.685873in}%
\pgfsys@useobject{currentmarker}{}%
\end{pgfscope}%
\begin{pgfscope}%
\pgfsys@transformshift{2.277090in}{0.612118in}%
\pgfsys@useobject{currentmarker}{}%
\end{pgfscope}%
\begin{pgfscope}%
\pgfsys@transformshift{2.277218in}{0.645655in}%
\pgfsys@useobject{currentmarker}{}%
\end{pgfscope}%
\begin{pgfscope}%
\pgfsys@transformshift{2.277347in}{0.656503in}%
\pgfsys@useobject{currentmarker}{}%
\end{pgfscope}%
\begin{pgfscope}%
\pgfsys@transformshift{2.277475in}{0.655345in}%
\pgfsys@useobject{currentmarker}{}%
\end{pgfscope}%
\begin{pgfscope}%
\pgfsys@transformshift{2.277603in}{0.669952in}%
\pgfsys@useobject{currentmarker}{}%
\end{pgfscope}%
\begin{pgfscope}%
\pgfsys@transformshift{2.277732in}{0.685307in}%
\pgfsys@useobject{currentmarker}{}%
\end{pgfscope}%
\begin{pgfscope}%
\pgfsys@transformshift{2.277860in}{0.703907in}%
\pgfsys@useobject{currentmarker}{}%
\end{pgfscope}%
\begin{pgfscope}%
\pgfsys@transformshift{2.277988in}{0.640740in}%
\pgfsys@useobject{currentmarker}{}%
\end{pgfscope}%
\begin{pgfscope}%
\pgfsys@transformshift{2.278116in}{0.651054in}%
\pgfsys@useobject{currentmarker}{}%
\end{pgfscope}%
\begin{pgfscope}%
\pgfsys@transformshift{2.278244in}{0.675960in}%
\pgfsys@useobject{currentmarker}{}%
\end{pgfscope}%
\begin{pgfscope}%
\pgfsys@transformshift{2.278372in}{0.691841in}%
\pgfsys@useobject{currentmarker}{}%
\end{pgfscope}%
\begin{pgfscope}%
\pgfsys@transformshift{2.278500in}{0.673071in}%
\pgfsys@useobject{currentmarker}{}%
\end{pgfscope}%
\begin{pgfscope}%
\pgfsys@transformshift{2.278627in}{0.653595in}%
\pgfsys@useobject{currentmarker}{}%
\end{pgfscope}%
\begin{pgfscope}%
\pgfsys@transformshift{2.278755in}{0.675640in}%
\pgfsys@useobject{currentmarker}{}%
\end{pgfscope}%
\begin{pgfscope}%
\pgfsys@transformshift{2.278883in}{0.690476in}%
\pgfsys@useobject{currentmarker}{}%
\end{pgfscope}%
\begin{pgfscope}%
\pgfsys@transformshift{2.279010in}{0.693903in}%
\pgfsys@useobject{currentmarker}{}%
\end{pgfscope}%
\begin{pgfscope}%
\pgfsys@transformshift{2.279138in}{0.646390in}%
\pgfsys@useobject{currentmarker}{}%
\end{pgfscope}%
\begin{pgfscope}%
\pgfsys@transformshift{2.279265in}{0.631137in}%
\pgfsys@useobject{currentmarker}{}%
\end{pgfscope}%
\begin{pgfscope}%
\pgfsys@transformshift{2.279392in}{0.665355in}%
\pgfsys@useobject{currentmarker}{}%
\end{pgfscope}%
\begin{pgfscope}%
\pgfsys@transformshift{2.279520in}{0.703147in}%
\pgfsys@useobject{currentmarker}{}%
\end{pgfscope}%
\begin{pgfscope}%
\pgfsys@transformshift{2.279647in}{0.660187in}%
\pgfsys@useobject{currentmarker}{}%
\end{pgfscope}%
\begin{pgfscope}%
\pgfsys@transformshift{2.279774in}{0.617738in}%
\pgfsys@useobject{currentmarker}{}%
\end{pgfscope}%
\begin{pgfscope}%
\pgfsys@transformshift{2.279901in}{0.612342in}%
\pgfsys@useobject{currentmarker}{}%
\end{pgfscope}%
\begin{pgfscope}%
\pgfsys@transformshift{2.280028in}{0.659620in}%
\pgfsys@useobject{currentmarker}{}%
\end{pgfscope}%
\begin{pgfscope}%
\pgfsys@transformshift{2.280155in}{0.647861in}%
\pgfsys@useobject{currentmarker}{}%
\end{pgfscope}%
\begin{pgfscope}%
\pgfsys@transformshift{2.280282in}{0.675484in}%
\pgfsys@useobject{currentmarker}{}%
\end{pgfscope}%
\begin{pgfscope}%
\pgfsys@transformshift{2.280408in}{0.680495in}%
\pgfsys@useobject{currentmarker}{}%
\end{pgfscope}%
\begin{pgfscope}%
\pgfsys@transformshift{2.280535in}{0.693669in}%
\pgfsys@useobject{currentmarker}{}%
\end{pgfscope}%
\begin{pgfscope}%
\pgfsys@transformshift{2.280662in}{0.684703in}%
\pgfsys@useobject{currentmarker}{}%
\end{pgfscope}%
\begin{pgfscope}%
\pgfsys@transformshift{2.280788in}{0.684902in}%
\pgfsys@useobject{currentmarker}{}%
\end{pgfscope}%
\begin{pgfscope}%
\pgfsys@transformshift{2.280915in}{0.713158in}%
\pgfsys@useobject{currentmarker}{}%
\end{pgfscope}%
\begin{pgfscope}%
\pgfsys@transformshift{2.281041in}{0.689498in}%
\pgfsys@useobject{currentmarker}{}%
\end{pgfscope}%
\begin{pgfscope}%
\pgfsys@transformshift{2.281167in}{0.678358in}%
\pgfsys@useobject{currentmarker}{}%
\end{pgfscope}%
\begin{pgfscope}%
\pgfsys@transformshift{2.281294in}{0.685504in}%
\pgfsys@useobject{currentmarker}{}%
\end{pgfscope}%
\begin{pgfscope}%
\pgfsys@transformshift{2.281420in}{0.654222in}%
\pgfsys@useobject{currentmarker}{}%
\end{pgfscope}%
\begin{pgfscope}%
\pgfsys@transformshift{2.281546in}{0.628392in}%
\pgfsys@useobject{currentmarker}{}%
\end{pgfscope}%
\begin{pgfscope}%
\pgfsys@transformshift{2.281672in}{0.700803in}%
\pgfsys@useobject{currentmarker}{}%
\end{pgfscope}%
\begin{pgfscope}%
\pgfsys@transformshift{2.281798in}{0.685082in}%
\pgfsys@useobject{currentmarker}{}%
\end{pgfscope}%
\begin{pgfscope}%
\pgfsys@transformshift{2.281924in}{0.649362in}%
\pgfsys@useobject{currentmarker}{}%
\end{pgfscope}%
\begin{pgfscope}%
\pgfsys@transformshift{2.282050in}{0.642260in}%
\pgfsys@useobject{currentmarker}{}%
\end{pgfscope}%
\begin{pgfscope}%
\pgfsys@transformshift{2.282175in}{0.704298in}%
\pgfsys@useobject{currentmarker}{}%
\end{pgfscope}%
\begin{pgfscope}%
\pgfsys@transformshift{2.282301in}{0.703003in}%
\pgfsys@useobject{currentmarker}{}%
\end{pgfscope}%
\begin{pgfscope}%
\pgfsys@transformshift{2.282427in}{0.705209in}%
\pgfsys@useobject{currentmarker}{}%
\end{pgfscope}%
\begin{pgfscope}%
\pgfsys@transformshift{2.282552in}{0.677271in}%
\pgfsys@useobject{currentmarker}{}%
\end{pgfscope}%
\begin{pgfscope}%
\pgfsys@transformshift{2.282678in}{0.674395in}%
\pgfsys@useobject{currentmarker}{}%
\end{pgfscope}%
\begin{pgfscope}%
\pgfsys@transformshift{2.282803in}{0.643454in}%
\pgfsys@useobject{currentmarker}{}%
\end{pgfscope}%
\begin{pgfscope}%
\pgfsys@transformshift{2.282928in}{0.682233in}%
\pgfsys@useobject{currentmarker}{}%
\end{pgfscope}%
\begin{pgfscope}%
\pgfsys@transformshift{2.283054in}{0.685461in}%
\pgfsys@useobject{currentmarker}{}%
\end{pgfscope}%
\begin{pgfscope}%
\pgfsys@transformshift{2.283179in}{0.660369in}%
\pgfsys@useobject{currentmarker}{}%
\end{pgfscope}%
\begin{pgfscope}%
\pgfsys@transformshift{2.283304in}{0.672619in}%
\pgfsys@useobject{currentmarker}{}%
\end{pgfscope}%
\begin{pgfscope}%
\pgfsys@transformshift{2.283429in}{0.712954in}%
\pgfsys@useobject{currentmarker}{}%
\end{pgfscope}%
\begin{pgfscope}%
\pgfsys@transformshift{2.283554in}{0.682941in}%
\pgfsys@useobject{currentmarker}{}%
\end{pgfscope}%
\begin{pgfscope}%
\pgfsys@transformshift{2.283679in}{0.667730in}%
\pgfsys@useobject{currentmarker}{}%
\end{pgfscope}%
\begin{pgfscope}%
\pgfsys@transformshift{2.283804in}{0.708436in}%
\pgfsys@useobject{currentmarker}{}%
\end{pgfscope}%
\begin{pgfscope}%
\pgfsys@transformshift{2.283928in}{0.675085in}%
\pgfsys@useobject{currentmarker}{}%
\end{pgfscope}%
\begin{pgfscope}%
\pgfsys@transformshift{2.284053in}{0.683896in}%
\pgfsys@useobject{currentmarker}{}%
\end{pgfscope}%
\begin{pgfscope}%
\pgfsys@transformshift{2.284178in}{0.684381in}%
\pgfsys@useobject{currentmarker}{}%
\end{pgfscope}%
\begin{pgfscope}%
\pgfsys@transformshift{2.284302in}{0.695623in}%
\pgfsys@useobject{currentmarker}{}%
\end{pgfscope}%
\begin{pgfscope}%
\pgfsys@transformshift{2.284427in}{0.647550in}%
\pgfsys@useobject{currentmarker}{}%
\end{pgfscope}%
\begin{pgfscope}%
\pgfsys@transformshift{2.284551in}{0.667445in}%
\pgfsys@useobject{currentmarker}{}%
\end{pgfscope}%
\begin{pgfscope}%
\pgfsys@transformshift{2.284676in}{0.635571in}%
\pgfsys@useobject{currentmarker}{}%
\end{pgfscope}%
\begin{pgfscope}%
\pgfsys@transformshift{2.284800in}{0.602407in}%
\pgfsys@useobject{currentmarker}{}%
\end{pgfscope}%
\begin{pgfscope}%
\pgfsys@transformshift{2.284924in}{0.614202in}%
\pgfsys@useobject{currentmarker}{}%
\end{pgfscope}%
\begin{pgfscope}%
\pgfsys@transformshift{2.285048in}{0.620568in}%
\pgfsys@useobject{currentmarker}{}%
\end{pgfscope}%
\begin{pgfscope}%
\pgfsys@transformshift{2.285172in}{0.681380in}%
\pgfsys@useobject{currentmarker}{}%
\end{pgfscope}%
\begin{pgfscope}%
\pgfsys@transformshift{2.285296in}{0.707333in}%
\pgfsys@useobject{currentmarker}{}%
\end{pgfscope}%
\begin{pgfscope}%
\pgfsys@transformshift{2.285420in}{0.691262in}%
\pgfsys@useobject{currentmarker}{}%
\end{pgfscope}%
\begin{pgfscope}%
\pgfsys@transformshift{2.285544in}{0.670071in}%
\pgfsys@useobject{currentmarker}{}%
\end{pgfscope}%
\begin{pgfscope}%
\pgfsys@transformshift{2.285668in}{0.695140in}%
\pgfsys@useobject{currentmarker}{}%
\end{pgfscope}%
\begin{pgfscope}%
\pgfsys@transformshift{2.285792in}{0.690953in}%
\pgfsys@useobject{currentmarker}{}%
\end{pgfscope}%
\begin{pgfscope}%
\pgfsys@transformshift{2.285915in}{0.692511in}%
\pgfsys@useobject{currentmarker}{}%
\end{pgfscope}%
\begin{pgfscope}%
\pgfsys@transformshift{2.286039in}{0.670758in}%
\pgfsys@useobject{currentmarker}{}%
\end{pgfscope}%
\begin{pgfscope}%
\pgfsys@transformshift{2.286163in}{0.700373in}%
\pgfsys@useobject{currentmarker}{}%
\end{pgfscope}%
\begin{pgfscope}%
\pgfsys@transformshift{2.286286in}{0.717317in}%
\pgfsys@useobject{currentmarker}{}%
\end{pgfscope}%
\begin{pgfscope}%
\pgfsys@transformshift{2.286409in}{0.645983in}%
\pgfsys@useobject{currentmarker}{}%
\end{pgfscope}%
\begin{pgfscope}%
\pgfsys@transformshift{2.286533in}{0.668079in}%
\pgfsys@useobject{currentmarker}{}%
\end{pgfscope}%
\begin{pgfscope}%
\pgfsys@transformshift{2.286656in}{0.684820in}%
\pgfsys@useobject{currentmarker}{}%
\end{pgfscope}%
\begin{pgfscope}%
\pgfsys@transformshift{2.286779in}{0.691099in}%
\pgfsys@useobject{currentmarker}{}%
\end{pgfscope}%
\begin{pgfscope}%
\pgfsys@transformshift{2.286902in}{0.709386in}%
\pgfsys@useobject{currentmarker}{}%
\end{pgfscope}%
\begin{pgfscope}%
\pgfsys@transformshift{2.287025in}{0.679216in}%
\pgfsys@useobject{currentmarker}{}%
\end{pgfscope}%
\begin{pgfscope}%
\pgfsys@transformshift{2.287148in}{0.650438in}%
\pgfsys@useobject{currentmarker}{}%
\end{pgfscope}%
\begin{pgfscope}%
\pgfsys@transformshift{2.287271in}{0.628079in}%
\pgfsys@useobject{currentmarker}{}%
\end{pgfscope}%
\begin{pgfscope}%
\pgfsys@transformshift{2.287394in}{0.659483in}%
\pgfsys@useobject{currentmarker}{}%
\end{pgfscope}%
\begin{pgfscope}%
\pgfsys@transformshift{2.287517in}{0.707804in}%
\pgfsys@useobject{currentmarker}{}%
\end{pgfscope}%
\begin{pgfscope}%
\pgfsys@transformshift{2.287640in}{0.675398in}%
\pgfsys@useobject{currentmarker}{}%
\end{pgfscope}%
\begin{pgfscope}%
\pgfsys@transformshift{2.287762in}{0.672865in}%
\pgfsys@useobject{currentmarker}{}%
\end{pgfscope}%
\begin{pgfscope}%
\pgfsys@transformshift{2.287885in}{0.692482in}%
\pgfsys@useobject{currentmarker}{}%
\end{pgfscope}%
\begin{pgfscope}%
\pgfsys@transformshift{2.288007in}{0.679088in}%
\pgfsys@useobject{currentmarker}{}%
\end{pgfscope}%
\begin{pgfscope}%
\pgfsys@transformshift{2.288130in}{0.671585in}%
\pgfsys@useobject{currentmarker}{}%
\end{pgfscope}%
\begin{pgfscope}%
\pgfsys@transformshift{2.288252in}{0.674258in}%
\pgfsys@useobject{currentmarker}{}%
\end{pgfscope}%
\begin{pgfscope}%
\pgfsys@transformshift{2.288375in}{0.641660in}%
\pgfsys@useobject{currentmarker}{}%
\end{pgfscope}%
\begin{pgfscope}%
\pgfsys@transformshift{2.288497in}{0.690724in}%
\pgfsys@useobject{currentmarker}{}%
\end{pgfscope}%
\begin{pgfscope}%
\pgfsys@transformshift{2.288619in}{0.707782in}%
\pgfsys@useobject{currentmarker}{}%
\end{pgfscope}%
\begin{pgfscope}%
\pgfsys@transformshift{2.288741in}{0.676124in}%
\pgfsys@useobject{currentmarker}{}%
\end{pgfscope}%
\begin{pgfscope}%
\pgfsys@transformshift{2.288863in}{0.680647in}%
\pgfsys@useobject{currentmarker}{}%
\end{pgfscope}%
\begin{pgfscope}%
\pgfsys@transformshift{2.288985in}{0.666883in}%
\pgfsys@useobject{currentmarker}{}%
\end{pgfscope}%
\begin{pgfscope}%
\pgfsys@transformshift{2.289107in}{0.649389in}%
\pgfsys@useobject{currentmarker}{}%
\end{pgfscope}%
\begin{pgfscope}%
\pgfsys@transformshift{2.289229in}{0.672225in}%
\pgfsys@useobject{currentmarker}{}%
\end{pgfscope}%
\begin{pgfscope}%
\pgfsys@transformshift{2.289351in}{0.696815in}%
\pgfsys@useobject{currentmarker}{}%
\end{pgfscope}%
\begin{pgfscope}%
\pgfsys@transformshift{2.289473in}{0.686586in}%
\pgfsys@useobject{currentmarker}{}%
\end{pgfscope}%
\begin{pgfscope}%
\pgfsys@transformshift{2.289594in}{0.634180in}%
\pgfsys@useobject{currentmarker}{}%
\end{pgfscope}%
\begin{pgfscope}%
\pgfsys@transformshift{2.289716in}{0.661545in}%
\pgfsys@useobject{currentmarker}{}%
\end{pgfscope}%
\begin{pgfscope}%
\pgfsys@transformshift{2.289837in}{0.735895in}%
\pgfsys@useobject{currentmarker}{}%
\end{pgfscope}%
\begin{pgfscope}%
\pgfsys@transformshift{2.289959in}{0.751341in}%
\pgfsys@useobject{currentmarker}{}%
\end{pgfscope}%
\begin{pgfscope}%
\pgfsys@transformshift{2.290080in}{0.681450in}%
\pgfsys@useobject{currentmarker}{}%
\end{pgfscope}%
\begin{pgfscope}%
\pgfsys@transformshift{2.290202in}{0.695147in}%
\pgfsys@useobject{currentmarker}{}%
\end{pgfscope}%
\begin{pgfscope}%
\pgfsys@transformshift{2.290323in}{0.704108in}%
\pgfsys@useobject{currentmarker}{}%
\end{pgfscope}%
\begin{pgfscope}%
\pgfsys@transformshift{2.290444in}{0.643409in}%
\pgfsys@useobject{currentmarker}{}%
\end{pgfscope}%
\begin{pgfscope}%
\pgfsys@transformshift{2.290565in}{0.703679in}%
\pgfsys@useobject{currentmarker}{}%
\end{pgfscope}%
\begin{pgfscope}%
\pgfsys@transformshift{2.290686in}{0.693934in}%
\pgfsys@useobject{currentmarker}{}%
\end{pgfscope}%
\begin{pgfscope}%
\pgfsys@transformshift{2.290807in}{0.679671in}%
\pgfsys@useobject{currentmarker}{}%
\end{pgfscope}%
\begin{pgfscope}%
\pgfsys@transformshift{2.290928in}{0.681857in}%
\pgfsys@useobject{currentmarker}{}%
\end{pgfscope}%
\begin{pgfscope}%
\pgfsys@transformshift{2.291049in}{0.691992in}%
\pgfsys@useobject{currentmarker}{}%
\end{pgfscope}%
\begin{pgfscope}%
\pgfsys@transformshift{2.291170in}{0.674166in}%
\pgfsys@useobject{currentmarker}{}%
\end{pgfscope}%
\begin{pgfscope}%
\pgfsys@transformshift{2.291291in}{0.711859in}%
\pgfsys@useobject{currentmarker}{}%
\end{pgfscope}%
\begin{pgfscope}%
\pgfsys@transformshift{2.291411in}{0.668748in}%
\pgfsys@useobject{currentmarker}{}%
\end{pgfscope}%
\begin{pgfscope}%
\pgfsys@transformshift{2.291532in}{0.676178in}%
\pgfsys@useobject{currentmarker}{}%
\end{pgfscope}%
\begin{pgfscope}%
\pgfsys@transformshift{2.291653in}{0.698768in}%
\pgfsys@useobject{currentmarker}{}%
\end{pgfscope}%
\begin{pgfscope}%
\pgfsys@transformshift{2.291773in}{0.671036in}%
\pgfsys@useobject{currentmarker}{}%
\end{pgfscope}%
\begin{pgfscope}%
\pgfsys@transformshift{2.291893in}{0.657567in}%
\pgfsys@useobject{currentmarker}{}%
\end{pgfscope}%
\begin{pgfscope}%
\pgfsys@transformshift{2.292014in}{0.654721in}%
\pgfsys@useobject{currentmarker}{}%
\end{pgfscope}%
\begin{pgfscope}%
\pgfsys@transformshift{2.292134in}{0.675216in}%
\pgfsys@useobject{currentmarker}{}%
\end{pgfscope}%
\begin{pgfscope}%
\pgfsys@transformshift{2.292254in}{0.660652in}%
\pgfsys@useobject{currentmarker}{}%
\end{pgfscope}%
\begin{pgfscope}%
\pgfsys@transformshift{2.292374in}{0.650192in}%
\pgfsys@useobject{currentmarker}{}%
\end{pgfscope}%
\begin{pgfscope}%
\pgfsys@transformshift{2.292495in}{0.644563in}%
\pgfsys@useobject{currentmarker}{}%
\end{pgfscope}%
\begin{pgfscope}%
\pgfsys@transformshift{2.292615in}{0.683611in}%
\pgfsys@useobject{currentmarker}{}%
\end{pgfscope}%
\begin{pgfscope}%
\pgfsys@transformshift{2.292735in}{0.681995in}%
\pgfsys@useobject{currentmarker}{}%
\end{pgfscope}%
\begin{pgfscope}%
\pgfsys@transformshift{2.292855in}{0.667117in}%
\pgfsys@useobject{currentmarker}{}%
\end{pgfscope}%
\begin{pgfscope}%
\pgfsys@transformshift{2.292974in}{0.635375in}%
\pgfsys@useobject{currentmarker}{}%
\end{pgfscope}%
\begin{pgfscope}%
\pgfsys@transformshift{2.293094in}{0.627388in}%
\pgfsys@useobject{currentmarker}{}%
\end{pgfscope}%
\begin{pgfscope}%
\pgfsys@transformshift{2.293214in}{0.681910in}%
\pgfsys@useobject{currentmarker}{}%
\end{pgfscope}%
\begin{pgfscope}%
\pgfsys@transformshift{2.293334in}{0.682722in}%
\pgfsys@useobject{currentmarker}{}%
\end{pgfscope}%
\begin{pgfscope}%
\pgfsys@transformshift{2.293453in}{0.647924in}%
\pgfsys@useobject{currentmarker}{}%
\end{pgfscope}%
\begin{pgfscope}%
\pgfsys@transformshift{2.293573in}{0.694397in}%
\pgfsys@useobject{currentmarker}{}%
\end{pgfscope}%
\begin{pgfscope}%
\pgfsys@transformshift{2.293692in}{0.677924in}%
\pgfsys@useobject{currentmarker}{}%
\end{pgfscope}%
\begin{pgfscope}%
\pgfsys@transformshift{2.293812in}{0.647918in}%
\pgfsys@useobject{currentmarker}{}%
\end{pgfscope}%
\begin{pgfscope}%
\pgfsys@transformshift{2.293931in}{0.673262in}%
\pgfsys@useobject{currentmarker}{}%
\end{pgfscope}%
\begin{pgfscope}%
\pgfsys@transformshift{2.294050in}{0.682434in}%
\pgfsys@useobject{currentmarker}{}%
\end{pgfscope}%
\begin{pgfscope}%
\pgfsys@transformshift{2.294169in}{0.692400in}%
\pgfsys@useobject{currentmarker}{}%
\end{pgfscope}%
\begin{pgfscope}%
\pgfsys@transformshift{2.294288in}{0.710708in}%
\pgfsys@useobject{currentmarker}{}%
\end{pgfscope}%
\begin{pgfscope}%
\pgfsys@transformshift{2.294408in}{0.706811in}%
\pgfsys@useobject{currentmarker}{}%
\end{pgfscope}%
\begin{pgfscope}%
\pgfsys@transformshift{2.294527in}{0.672568in}%
\pgfsys@useobject{currentmarker}{}%
\end{pgfscope}%
\begin{pgfscope}%
\pgfsys@transformshift{2.294646in}{0.646445in}%
\pgfsys@useobject{currentmarker}{}%
\end{pgfscope}%
\begin{pgfscope}%
\pgfsys@transformshift{2.294764in}{0.661729in}%
\pgfsys@useobject{currentmarker}{}%
\end{pgfscope}%
\begin{pgfscope}%
\pgfsys@transformshift{2.294883in}{0.648276in}%
\pgfsys@useobject{currentmarker}{}%
\end{pgfscope}%
\begin{pgfscope}%
\pgfsys@transformshift{2.295002in}{0.638861in}%
\pgfsys@useobject{currentmarker}{}%
\end{pgfscope}%
\begin{pgfscope}%
\pgfsys@transformshift{2.295121in}{0.627525in}%
\pgfsys@useobject{currentmarker}{}%
\end{pgfscope}%
\begin{pgfscope}%
\pgfsys@transformshift{2.295239in}{0.680816in}%
\pgfsys@useobject{currentmarker}{}%
\end{pgfscope}%
\begin{pgfscope}%
\pgfsys@transformshift{2.295358in}{0.709967in}%
\pgfsys@useobject{currentmarker}{}%
\end{pgfscope}%
\begin{pgfscope}%
\pgfsys@transformshift{2.295476in}{0.700293in}%
\pgfsys@useobject{currentmarker}{}%
\end{pgfscope}%
\begin{pgfscope}%
\pgfsys@transformshift{2.295595in}{0.656387in}%
\pgfsys@useobject{currentmarker}{}%
\end{pgfscope}%
\begin{pgfscope}%
\pgfsys@transformshift{2.295713in}{0.662081in}%
\pgfsys@useobject{currentmarker}{}%
\end{pgfscope}%
\begin{pgfscope}%
\pgfsys@transformshift{2.295832in}{0.661520in}%
\pgfsys@useobject{currentmarker}{}%
\end{pgfscope}%
\begin{pgfscope}%
\pgfsys@transformshift{2.295950in}{0.680124in}%
\pgfsys@useobject{currentmarker}{}%
\end{pgfscope}%
\begin{pgfscope}%
\pgfsys@transformshift{2.296068in}{0.656054in}%
\pgfsys@useobject{currentmarker}{}%
\end{pgfscope}%
\begin{pgfscope}%
\pgfsys@transformshift{2.296186in}{0.652909in}%
\pgfsys@useobject{currentmarker}{}%
\end{pgfscope}%
\begin{pgfscope}%
\pgfsys@transformshift{2.296304in}{0.641789in}%
\pgfsys@useobject{currentmarker}{}%
\end{pgfscope}%
\begin{pgfscope}%
\pgfsys@transformshift{2.296422in}{0.603751in}%
\pgfsys@useobject{currentmarker}{}%
\end{pgfscope}%
\begin{pgfscope}%
\pgfsys@transformshift{2.296540in}{0.631273in}%
\pgfsys@useobject{currentmarker}{}%
\end{pgfscope}%
\begin{pgfscope}%
\pgfsys@transformshift{2.296658in}{0.691810in}%
\pgfsys@useobject{currentmarker}{}%
\end{pgfscope}%
\begin{pgfscope}%
\pgfsys@transformshift{2.296776in}{0.698668in}%
\pgfsys@useobject{currentmarker}{}%
\end{pgfscope}%
\begin{pgfscope}%
\pgfsys@transformshift{2.296894in}{0.674084in}%
\pgfsys@useobject{currentmarker}{}%
\end{pgfscope}%
\begin{pgfscope}%
\pgfsys@transformshift{2.297012in}{0.665989in}%
\pgfsys@useobject{currentmarker}{}%
\end{pgfscope}%
\begin{pgfscope}%
\pgfsys@transformshift{2.297129in}{0.609636in}%
\pgfsys@useobject{currentmarker}{}%
\end{pgfscope}%
\begin{pgfscope}%
\pgfsys@transformshift{2.297247in}{0.638547in}%
\pgfsys@useobject{currentmarker}{}%
\end{pgfscope}%
\begin{pgfscope}%
\pgfsys@transformshift{2.297364in}{0.690769in}%
\pgfsys@useobject{currentmarker}{}%
\end{pgfscope}%
\begin{pgfscope}%
\pgfsys@transformshift{2.297482in}{0.684271in}%
\pgfsys@useobject{currentmarker}{}%
\end{pgfscope}%
\begin{pgfscope}%
\pgfsys@transformshift{2.297599in}{0.648709in}%
\pgfsys@useobject{currentmarker}{}%
\end{pgfscope}%
\begin{pgfscope}%
\pgfsys@transformshift{2.297717in}{0.639670in}%
\pgfsys@useobject{currentmarker}{}%
\end{pgfscope}%
\begin{pgfscope}%
\pgfsys@transformshift{2.297834in}{0.648369in}%
\pgfsys@useobject{currentmarker}{}%
\end{pgfscope}%
\begin{pgfscope}%
\pgfsys@transformshift{2.297951in}{0.663953in}%
\pgfsys@useobject{currentmarker}{}%
\end{pgfscope}%
\begin{pgfscope}%
\pgfsys@transformshift{2.298068in}{0.654222in}%
\pgfsys@useobject{currentmarker}{}%
\end{pgfscope}%
\begin{pgfscope}%
\pgfsys@transformshift{2.298185in}{0.660754in}%
\pgfsys@useobject{currentmarker}{}%
\end{pgfscope}%
\begin{pgfscope}%
\pgfsys@transformshift{2.298302in}{0.679540in}%
\pgfsys@useobject{currentmarker}{}%
\end{pgfscope}%
\begin{pgfscope}%
\pgfsys@transformshift{2.298419in}{0.623709in}%
\pgfsys@useobject{currentmarker}{}%
\end{pgfscope}%
\begin{pgfscope}%
\pgfsys@transformshift{2.298536in}{0.636966in}%
\pgfsys@useobject{currentmarker}{}%
\end{pgfscope}%
\begin{pgfscope}%
\pgfsys@transformshift{2.298653in}{0.630659in}%
\pgfsys@useobject{currentmarker}{}%
\end{pgfscope}%
\begin{pgfscope}%
\pgfsys@transformshift{2.298770in}{0.626511in}%
\pgfsys@useobject{currentmarker}{}%
\end{pgfscope}%
\begin{pgfscope}%
\pgfsys@transformshift{2.298887in}{0.677257in}%
\pgfsys@useobject{currentmarker}{}%
\end{pgfscope}%
\begin{pgfscope}%
\pgfsys@transformshift{2.299003in}{0.660927in}%
\pgfsys@useobject{currentmarker}{}%
\end{pgfscope}%
\begin{pgfscope}%
\pgfsys@transformshift{2.299120in}{0.658376in}%
\pgfsys@useobject{currentmarker}{}%
\end{pgfscope}%
\begin{pgfscope}%
\pgfsys@transformshift{2.299236in}{0.690972in}%
\pgfsys@useobject{currentmarker}{}%
\end{pgfscope}%
\begin{pgfscope}%
\pgfsys@transformshift{2.299353in}{0.692994in}%
\pgfsys@useobject{currentmarker}{}%
\end{pgfscope}%
\begin{pgfscope}%
\pgfsys@transformshift{2.299469in}{0.692625in}%
\pgfsys@useobject{currentmarker}{}%
\end{pgfscope}%
\begin{pgfscope}%
\pgfsys@transformshift{2.299586in}{0.701255in}%
\pgfsys@useobject{currentmarker}{}%
\end{pgfscope}%
\begin{pgfscope}%
\pgfsys@transformshift{2.299702in}{0.679299in}%
\pgfsys@useobject{currentmarker}{}%
\end{pgfscope}%
\begin{pgfscope}%
\pgfsys@transformshift{2.299818in}{0.674972in}%
\pgfsys@useobject{currentmarker}{}%
\end{pgfscope}%
\begin{pgfscope}%
\pgfsys@transformshift{2.299934in}{0.639771in}%
\pgfsys@useobject{currentmarker}{}%
\end{pgfscope}%
\begin{pgfscope}%
\pgfsys@transformshift{2.300051in}{0.687561in}%
\pgfsys@useobject{currentmarker}{}%
\end{pgfscope}%
\begin{pgfscope}%
\pgfsys@transformshift{2.300167in}{0.683616in}%
\pgfsys@useobject{currentmarker}{}%
\end{pgfscope}%
\begin{pgfscope}%
\pgfsys@transformshift{2.300283in}{0.642821in}%
\pgfsys@useobject{currentmarker}{}%
\end{pgfscope}%
\begin{pgfscope}%
\pgfsys@transformshift{2.300399in}{0.669478in}%
\pgfsys@useobject{currentmarker}{}%
\end{pgfscope}%
\begin{pgfscope}%
\pgfsys@transformshift{2.300515in}{0.668758in}%
\pgfsys@useobject{currentmarker}{}%
\end{pgfscope}%
\begin{pgfscope}%
\pgfsys@transformshift{2.300630in}{0.589486in}%
\pgfsys@useobject{currentmarker}{}%
\end{pgfscope}%
\begin{pgfscope}%
\pgfsys@transformshift{2.300746in}{0.648213in}%
\pgfsys@useobject{currentmarker}{}%
\end{pgfscope}%
\begin{pgfscope}%
\pgfsys@transformshift{2.300862in}{0.651395in}%
\pgfsys@useobject{currentmarker}{}%
\end{pgfscope}%
\begin{pgfscope}%
\pgfsys@transformshift{2.300977in}{0.631733in}%
\pgfsys@useobject{currentmarker}{}%
\end{pgfscope}%
\begin{pgfscope}%
\pgfsys@transformshift{2.301093in}{0.668036in}%
\pgfsys@useobject{currentmarker}{}%
\end{pgfscope}%
\begin{pgfscope}%
\pgfsys@transformshift{2.301209in}{0.669826in}%
\pgfsys@useobject{currentmarker}{}%
\end{pgfscope}%
\begin{pgfscope}%
\pgfsys@transformshift{2.301324in}{0.682665in}%
\pgfsys@useobject{currentmarker}{}%
\end{pgfscope}%
\begin{pgfscope}%
\pgfsys@transformshift{2.301439in}{0.629785in}%
\pgfsys@useobject{currentmarker}{}%
\end{pgfscope}%
\begin{pgfscope}%
\pgfsys@transformshift{2.301555in}{0.690100in}%
\pgfsys@useobject{currentmarker}{}%
\end{pgfscope}%
\begin{pgfscope}%
\pgfsys@transformshift{2.301670in}{0.704950in}%
\pgfsys@useobject{currentmarker}{}%
\end{pgfscope}%
\begin{pgfscope}%
\pgfsys@transformshift{2.301785in}{0.656226in}%
\pgfsys@useobject{currentmarker}{}%
\end{pgfscope}%
\begin{pgfscope}%
\pgfsys@transformshift{2.301901in}{0.635383in}%
\pgfsys@useobject{currentmarker}{}%
\end{pgfscope}%
\begin{pgfscope}%
\pgfsys@transformshift{2.302016in}{0.590936in}%
\pgfsys@useobject{currentmarker}{}%
\end{pgfscope}%
\begin{pgfscope}%
\pgfsys@transformshift{2.302131in}{0.626375in}%
\pgfsys@useobject{currentmarker}{}%
\end{pgfscope}%
\begin{pgfscope}%
\pgfsys@transformshift{2.302246in}{0.681330in}%
\pgfsys@useobject{currentmarker}{}%
\end{pgfscope}%
\begin{pgfscope}%
\pgfsys@transformshift{2.302361in}{0.670729in}%
\pgfsys@useobject{currentmarker}{}%
\end{pgfscope}%
\begin{pgfscope}%
\pgfsys@transformshift{2.302476in}{0.722981in}%
\pgfsys@useobject{currentmarker}{}%
\end{pgfscope}%
\begin{pgfscope}%
\pgfsys@transformshift{2.302590in}{0.702981in}%
\pgfsys@useobject{currentmarker}{}%
\end{pgfscope}%
\begin{pgfscope}%
\pgfsys@transformshift{2.302705in}{0.655000in}%
\pgfsys@useobject{currentmarker}{}%
\end{pgfscope}%
\begin{pgfscope}%
\pgfsys@transformshift{2.302820in}{0.680883in}%
\pgfsys@useobject{currentmarker}{}%
\end{pgfscope}%
\begin{pgfscope}%
\pgfsys@transformshift{2.302934in}{0.670715in}%
\pgfsys@useobject{currentmarker}{}%
\end{pgfscope}%
\begin{pgfscope}%
\pgfsys@transformshift{2.303049in}{0.657788in}%
\pgfsys@useobject{currentmarker}{}%
\end{pgfscope}%
\begin{pgfscope}%
\pgfsys@transformshift{2.303164in}{0.674681in}%
\pgfsys@useobject{currentmarker}{}%
\end{pgfscope}%
\begin{pgfscope}%
\pgfsys@transformshift{2.303278in}{0.653235in}%
\pgfsys@useobject{currentmarker}{}%
\end{pgfscope}%
\begin{pgfscope}%
\pgfsys@transformshift{2.303392in}{0.673450in}%
\pgfsys@useobject{currentmarker}{}%
\end{pgfscope}%
\begin{pgfscope}%
\pgfsys@transformshift{2.303507in}{0.647342in}%
\pgfsys@useobject{currentmarker}{}%
\end{pgfscope}%
\begin{pgfscope}%
\pgfsys@transformshift{2.303621in}{0.672778in}%
\pgfsys@useobject{currentmarker}{}%
\end{pgfscope}%
\begin{pgfscope}%
\pgfsys@transformshift{2.303735in}{0.663151in}%
\pgfsys@useobject{currentmarker}{}%
\end{pgfscope}%
\begin{pgfscope}%
\pgfsys@transformshift{2.303850in}{0.660736in}%
\pgfsys@useobject{currentmarker}{}%
\end{pgfscope}%
\begin{pgfscope}%
\pgfsys@transformshift{2.303964in}{0.661568in}%
\pgfsys@useobject{currentmarker}{}%
\end{pgfscope}%
\begin{pgfscope}%
\pgfsys@transformshift{2.304078in}{0.681546in}%
\pgfsys@useobject{currentmarker}{}%
\end{pgfscope}%
\begin{pgfscope}%
\pgfsys@transformshift{2.304192in}{0.662191in}%
\pgfsys@useobject{currentmarker}{}%
\end{pgfscope}%
\begin{pgfscope}%
\pgfsys@transformshift{2.304306in}{0.663871in}%
\pgfsys@useobject{currentmarker}{}%
\end{pgfscope}%
\begin{pgfscope}%
\pgfsys@transformshift{2.304420in}{0.636050in}%
\pgfsys@useobject{currentmarker}{}%
\end{pgfscope}%
\begin{pgfscope}%
\pgfsys@transformshift{2.304533in}{0.598852in}%
\pgfsys@useobject{currentmarker}{}%
\end{pgfscope}%
\begin{pgfscope}%
\pgfsys@transformshift{2.304647in}{0.626567in}%
\pgfsys@useobject{currentmarker}{}%
\end{pgfscope}%
\begin{pgfscope}%
\pgfsys@transformshift{2.304761in}{0.641927in}%
\pgfsys@useobject{currentmarker}{}%
\end{pgfscope}%
\begin{pgfscope}%
\pgfsys@transformshift{2.304875in}{0.652052in}%
\pgfsys@useobject{currentmarker}{}%
\end{pgfscope}%
\begin{pgfscope}%
\pgfsys@transformshift{2.304988in}{0.664075in}%
\pgfsys@useobject{currentmarker}{}%
\end{pgfscope}%
\begin{pgfscope}%
\pgfsys@transformshift{2.305102in}{0.685435in}%
\pgfsys@useobject{currentmarker}{}%
\end{pgfscope}%
\begin{pgfscope}%
\pgfsys@transformshift{2.305215in}{0.682574in}%
\pgfsys@useobject{currentmarker}{}%
\end{pgfscope}%
\begin{pgfscope}%
\pgfsys@transformshift{2.305329in}{0.634659in}%
\pgfsys@useobject{currentmarker}{}%
\end{pgfscope}%
\begin{pgfscope}%
\pgfsys@transformshift{2.305442in}{0.635693in}%
\pgfsys@useobject{currentmarker}{}%
\end{pgfscope}%
\begin{pgfscope}%
\pgfsys@transformshift{2.305555in}{0.665719in}%
\pgfsys@useobject{currentmarker}{}%
\end{pgfscope}%
\begin{pgfscope}%
\pgfsys@transformshift{2.305669in}{0.701529in}%
\pgfsys@useobject{currentmarker}{}%
\end{pgfscope}%
\begin{pgfscope}%
\pgfsys@transformshift{2.305782in}{0.661021in}%
\pgfsys@useobject{currentmarker}{}%
\end{pgfscope}%
\begin{pgfscope}%
\pgfsys@transformshift{2.305895in}{0.699278in}%
\pgfsys@useobject{currentmarker}{}%
\end{pgfscope}%
\begin{pgfscope}%
\pgfsys@transformshift{2.306008in}{0.648960in}%
\pgfsys@useobject{currentmarker}{}%
\end{pgfscope}%
\begin{pgfscope}%
\pgfsys@transformshift{2.306121in}{0.650623in}%
\pgfsys@useobject{currentmarker}{}%
\end{pgfscope}%
\begin{pgfscope}%
\pgfsys@transformshift{2.306234in}{0.651458in}%
\pgfsys@useobject{currentmarker}{}%
\end{pgfscope}%
\begin{pgfscope}%
\pgfsys@transformshift{2.306347in}{0.613825in}%
\pgfsys@useobject{currentmarker}{}%
\end{pgfscope}%
\begin{pgfscope}%
\pgfsys@transformshift{2.306460in}{0.586209in}%
\pgfsys@useobject{currentmarker}{}%
\end{pgfscope}%
\begin{pgfscope}%
\pgfsys@transformshift{2.306573in}{0.670083in}%
\pgfsys@useobject{currentmarker}{}%
\end{pgfscope}%
\begin{pgfscope}%
\pgfsys@transformshift{2.306685in}{0.672948in}%
\pgfsys@useobject{currentmarker}{}%
\end{pgfscope}%
\begin{pgfscope}%
\pgfsys@transformshift{2.306798in}{0.645266in}%
\pgfsys@useobject{currentmarker}{}%
\end{pgfscope}%
\begin{pgfscope}%
\pgfsys@transformshift{2.306911in}{0.642729in}%
\pgfsys@useobject{currentmarker}{}%
\end{pgfscope}%
\begin{pgfscope}%
\pgfsys@transformshift{2.307023in}{0.668373in}%
\pgfsys@useobject{currentmarker}{}%
\end{pgfscope}%
\begin{pgfscope}%
\pgfsys@transformshift{2.307136in}{0.699892in}%
\pgfsys@useobject{currentmarker}{}%
\end{pgfscope}%
\begin{pgfscope}%
\pgfsys@transformshift{2.307248in}{0.719898in}%
\pgfsys@useobject{currentmarker}{}%
\end{pgfscope}%
\begin{pgfscope}%
\pgfsys@transformshift{2.307361in}{0.689919in}%
\pgfsys@useobject{currentmarker}{}%
\end{pgfscope}%
\begin{pgfscope}%
\pgfsys@transformshift{2.307473in}{0.687589in}%
\pgfsys@useobject{currentmarker}{}%
\end{pgfscope}%
\begin{pgfscope}%
\pgfsys@transformshift{2.307585in}{0.683352in}%
\pgfsys@useobject{currentmarker}{}%
\end{pgfscope}%
\begin{pgfscope}%
\pgfsys@transformshift{2.307698in}{0.662579in}%
\pgfsys@useobject{currentmarker}{}%
\end{pgfscope}%
\begin{pgfscope}%
\pgfsys@transformshift{2.307810in}{0.678877in}%
\pgfsys@useobject{currentmarker}{}%
\end{pgfscope}%
\begin{pgfscope}%
\pgfsys@transformshift{2.307922in}{0.610988in}%
\pgfsys@useobject{currentmarker}{}%
\end{pgfscope}%
\begin{pgfscope}%
\pgfsys@transformshift{2.308034in}{0.633629in}%
\pgfsys@useobject{currentmarker}{}%
\end{pgfscope}%
\begin{pgfscope}%
\pgfsys@transformshift{2.308146in}{0.634300in}%
\pgfsys@useobject{currentmarker}{}%
\end{pgfscope}%
\begin{pgfscope}%
\pgfsys@transformshift{2.308258in}{0.655192in}%
\pgfsys@useobject{currentmarker}{}%
\end{pgfscope}%
\begin{pgfscope}%
\pgfsys@transformshift{2.308370in}{0.666752in}%
\pgfsys@useobject{currentmarker}{}%
\end{pgfscope}%
\begin{pgfscope}%
\pgfsys@transformshift{2.308482in}{0.648257in}%
\pgfsys@useobject{currentmarker}{}%
\end{pgfscope}%
\begin{pgfscope}%
\pgfsys@transformshift{2.308594in}{0.666169in}%
\pgfsys@useobject{currentmarker}{}%
\end{pgfscope}%
\begin{pgfscope}%
\pgfsys@transformshift{2.308705in}{0.673702in}%
\pgfsys@useobject{currentmarker}{}%
\end{pgfscope}%
\begin{pgfscope}%
\pgfsys@transformshift{2.308817in}{0.616075in}%
\pgfsys@useobject{currentmarker}{}%
\end{pgfscope}%
\begin{pgfscope}%
\pgfsys@transformshift{2.308929in}{0.711626in}%
\pgfsys@useobject{currentmarker}{}%
\end{pgfscope}%
\begin{pgfscope}%
\pgfsys@transformshift{2.309040in}{0.692222in}%
\pgfsys@useobject{currentmarker}{}%
\end{pgfscope}%
\begin{pgfscope}%
\pgfsys@transformshift{2.309152in}{0.689041in}%
\pgfsys@useobject{currentmarker}{}%
\end{pgfscope}%
\begin{pgfscope}%
\pgfsys@transformshift{2.309263in}{0.700235in}%
\pgfsys@useobject{currentmarker}{}%
\end{pgfscope}%
\begin{pgfscope}%
\pgfsys@transformshift{2.309375in}{0.649461in}%
\pgfsys@useobject{currentmarker}{}%
\end{pgfscope}%
\begin{pgfscope}%
\pgfsys@transformshift{2.309486in}{0.677203in}%
\pgfsys@useobject{currentmarker}{}%
\end{pgfscope}%
\begin{pgfscope}%
\pgfsys@transformshift{2.309597in}{0.691380in}%
\pgfsys@useobject{currentmarker}{}%
\end{pgfscope}%
\begin{pgfscope}%
\pgfsys@transformshift{2.309708in}{0.638386in}%
\pgfsys@useobject{currentmarker}{}%
\end{pgfscope}%
\begin{pgfscope}%
\pgfsys@transformshift{2.309820in}{0.626961in}%
\pgfsys@useobject{currentmarker}{}%
\end{pgfscope}%
\begin{pgfscope}%
\pgfsys@transformshift{2.309931in}{0.677970in}%
\pgfsys@useobject{currentmarker}{}%
\end{pgfscope}%
\begin{pgfscope}%
\pgfsys@transformshift{2.310042in}{0.633506in}%
\pgfsys@useobject{currentmarker}{}%
\end{pgfscope}%
\begin{pgfscope}%
\pgfsys@transformshift{2.310153in}{0.605629in}%
\pgfsys@useobject{currentmarker}{}%
\end{pgfscope}%
\begin{pgfscope}%
\pgfsys@transformshift{2.310264in}{0.670713in}%
\pgfsys@useobject{currentmarker}{}%
\end{pgfscope}%
\begin{pgfscope}%
\pgfsys@transformshift{2.310375in}{0.670991in}%
\pgfsys@useobject{currentmarker}{}%
\end{pgfscope}%
\begin{pgfscope}%
\pgfsys@transformshift{2.310486in}{0.680648in}%
\pgfsys@useobject{currentmarker}{}%
\end{pgfscope}%
\begin{pgfscope}%
\pgfsys@transformshift{2.310596in}{0.602741in}%
\pgfsys@useobject{currentmarker}{}%
\end{pgfscope}%
\begin{pgfscope}%
\pgfsys@transformshift{2.310707in}{0.631625in}%
\pgfsys@useobject{currentmarker}{}%
\end{pgfscope}%
\begin{pgfscope}%
\pgfsys@transformshift{2.310818in}{0.629701in}%
\pgfsys@useobject{currentmarker}{}%
\end{pgfscope}%
\begin{pgfscope}%
\pgfsys@transformshift{2.310928in}{0.643290in}%
\pgfsys@useobject{currentmarker}{}%
\end{pgfscope}%
\begin{pgfscope}%
\pgfsys@transformshift{2.311039in}{0.642079in}%
\pgfsys@useobject{currentmarker}{}%
\end{pgfscope}%
\begin{pgfscope}%
\pgfsys@transformshift{2.311150in}{0.652611in}%
\pgfsys@useobject{currentmarker}{}%
\end{pgfscope}%
\begin{pgfscope}%
\pgfsys@transformshift{2.311260in}{0.602453in}%
\pgfsys@useobject{currentmarker}{}%
\end{pgfscope}%
\begin{pgfscope}%
\pgfsys@transformshift{2.311370in}{0.654620in}%
\pgfsys@useobject{currentmarker}{}%
\end{pgfscope}%
\begin{pgfscope}%
\pgfsys@transformshift{2.311481in}{0.641068in}%
\pgfsys@useobject{currentmarker}{}%
\end{pgfscope}%
\begin{pgfscope}%
\pgfsys@transformshift{2.311591in}{0.647847in}%
\pgfsys@useobject{currentmarker}{}%
\end{pgfscope}%
\begin{pgfscope}%
\pgfsys@transformshift{2.311701in}{0.647023in}%
\pgfsys@useobject{currentmarker}{}%
\end{pgfscope}%
\begin{pgfscope}%
\pgfsys@transformshift{2.311812in}{0.683638in}%
\pgfsys@useobject{currentmarker}{}%
\end{pgfscope}%
\begin{pgfscope}%
\pgfsys@transformshift{2.311922in}{0.675673in}%
\pgfsys@useobject{currentmarker}{}%
\end{pgfscope}%
\begin{pgfscope}%
\pgfsys@transformshift{2.312032in}{0.677050in}%
\pgfsys@useobject{currentmarker}{}%
\end{pgfscope}%
\begin{pgfscope}%
\pgfsys@transformshift{2.312142in}{0.715185in}%
\pgfsys@useobject{currentmarker}{}%
\end{pgfscope}%
\begin{pgfscope}%
\pgfsys@transformshift{2.312252in}{0.704471in}%
\pgfsys@useobject{currentmarker}{}%
\end{pgfscope}%
\begin{pgfscope}%
\pgfsys@transformshift{2.312362in}{0.719565in}%
\pgfsys@useobject{currentmarker}{}%
\end{pgfscope}%
\begin{pgfscope}%
\pgfsys@transformshift{2.312472in}{0.703206in}%
\pgfsys@useobject{currentmarker}{}%
\end{pgfscope}%
\begin{pgfscope}%
\pgfsys@transformshift{2.312582in}{0.667330in}%
\pgfsys@useobject{currentmarker}{}%
\end{pgfscope}%
\begin{pgfscope}%
\pgfsys@transformshift{2.312691in}{0.635971in}%
\pgfsys@useobject{currentmarker}{}%
\end{pgfscope}%
\end{pgfscope}%
\begin{pgfscope}%
\pgfsetrectcap%
\pgfsetmiterjoin%
\pgfsetlinewidth{0.803000pt}%
\definecolor{currentstroke}{rgb}{0.000000,0.000000,0.000000}%
\pgfsetstrokecolor{currentstroke}%
\pgfsetdash{}{0pt}%
\pgfpathmoveto{\pgfqpoint{0.514278in}{0.417642in}}%
\pgfpathlineto{\pgfqpoint{0.514278in}{1.788330in}}%
\pgfusepath{stroke}%
\end{pgfscope}%
\begin{pgfscope}%
\pgfsetrectcap%
\pgfsetmiterjoin%
\pgfsetlinewidth{0.803000pt}%
\definecolor{currentstroke}{rgb}{0.000000,0.000000,0.000000}%
\pgfsetstrokecolor{currentstroke}%
\pgfsetdash{}{0pt}%
\pgfpathmoveto{\pgfqpoint{2.398330in}{0.417642in}}%
\pgfpathlineto{\pgfqpoint{2.398330in}{1.788330in}}%
\pgfusepath{stroke}%
\end{pgfscope}%
\begin{pgfscope}%
\pgfsetrectcap%
\pgfsetmiterjoin%
\pgfsetlinewidth{0.803000pt}%
\definecolor{currentstroke}{rgb}{0.000000,0.000000,0.000000}%
\pgfsetstrokecolor{currentstroke}%
\pgfsetdash{}{0pt}%
\pgfpathmoveto{\pgfqpoint{0.514278in}{0.417642in}}%
\pgfpathlineto{\pgfqpoint{2.398330in}{0.417642in}}%
\pgfusepath{stroke}%
\end{pgfscope}%
\begin{pgfscope}%
\pgfsetrectcap%
\pgfsetmiterjoin%
\pgfsetlinewidth{0.803000pt}%
\definecolor{currentstroke}{rgb}{0.000000,0.000000,0.000000}%
\pgfsetstrokecolor{currentstroke}%
\pgfsetdash{}{0pt}%
\pgfpathmoveto{\pgfqpoint{0.514278in}{1.788330in}}%
\pgfpathlineto{\pgfqpoint{2.398330in}{1.788330in}}%
\pgfusepath{stroke}%
\end{pgfscope}%
\begin{pgfscope}%
\pgfsetbuttcap%
\pgfsetmiterjoin%
\definecolor{currentfill}{rgb}{1.000000,1.000000,1.000000}%
\pgfsetfillcolor{currentfill}%
\pgfsetfillopacity{0.800000}%
\pgfsetlinewidth{1.003750pt}%
\definecolor{currentstroke}{rgb}{0.800000,0.800000,0.800000}%
\pgfsetstrokecolor{currentstroke}%
\pgfsetstrokeopacity{0.800000}%
\pgfsetdash{}{0pt}%
\pgfpathmoveto{\pgfqpoint{1.551772in}{1.517019in}}%
\pgfpathlineto{\pgfqpoint{2.320552in}{1.517019in}}%
\pgfpathquadraticcurveto{\pgfqpoint{2.342774in}{1.517019in}}{\pgfqpoint{2.342774in}{1.539241in}}%
\pgfpathlineto{\pgfqpoint{2.342774in}{1.710552in}}%
\pgfpathquadraticcurveto{\pgfqpoint{2.342774in}{1.732774in}}{\pgfqpoint{2.320552in}{1.732774in}}%
\pgfpathlineto{\pgfqpoint{1.551772in}{1.732774in}}%
\pgfpathquadraticcurveto{\pgfqpoint{1.529549in}{1.732774in}}{\pgfqpoint{1.529549in}{1.710552in}}%
\pgfpathlineto{\pgfqpoint{1.529549in}{1.539241in}}%
\pgfpathquadraticcurveto{\pgfqpoint{1.529549in}{1.517019in}}{\pgfqpoint{1.551772in}{1.517019in}}%
\pgfpathlineto{\pgfqpoint{1.551772in}{1.517019in}}%
\pgfpathclose%
\pgfusepath{stroke,fill}%
\end{pgfscope}%
\begin{pgfscope}%
\pgfsetbuttcap%
\pgfsetroundjoin%
\pgfsetlinewidth{1.505625pt}%
\definecolor{currentstroke}{rgb}{0.835294,0.368627,0.000000}%
\pgfsetstrokecolor{currentstroke}%
\pgfsetdash{{5.550000pt}{2.400000pt}}{0.000000pt}%
\pgfpathmoveto{\pgfqpoint{1.573994in}{1.627358in}}%
\pgfpathlineto{\pgfqpoint{1.685105in}{1.627358in}}%
\pgfpathlineto{\pgfqpoint{1.796216in}{1.627358in}}%
\pgfusepath{stroke}%
\end{pgfscope}%
\begin{pgfscope}%
\definecolor{textcolor}{rgb}{0.000000,0.000000,0.000000}%
\pgfsetstrokecolor{textcolor}%
\pgfsetfillcolor{textcolor}%
\pgftext[x=1.885105in,y=1.588469in,left,base]{\color{textcolor}\rmfamily\fontsize{8.000000}{9.600000}\selectfont \(\displaystyle h_{-2}f^{-2}\)}%
\end{pgfscope}%
\end{pgfpicture}%
\makeatother%
\endgroup%
% data/simulations/sim_allan_variance.py
        } % scalebox
        \caption{Power spectral density}
        \label{fig:random_walk_psd}
    \end{subfigure}
    \begin{subfigure}{0.32\linewidth}
        \centering
        \scalebox{0.75}{%
            %% Creator: Matplotlib, PGF backend
%%
%% To include the figure in your LaTeX document, write
%%   \input{<filename>.pgf}
%%
%% Make sure the required packages are loaded in your preamble
%%   \usepackage{pgf}
%%
%% Also ensure that all the required font packages are loaded; for instance,
%% the lmodern package is sometimes necessary when using math font.
%%   \usepackage{lmodern}
%%
%% Figures using additional raster images can only be included by \input if
%% they are in the same directory as the main LaTeX file. For loading figures
%% from other directories you can use the `import` package
%%   \usepackage{import}
%%
%% and then include the figures with
%%   \import{<path to file>}{<filename>.pgf}
%%
%% Matplotlib used the following preamble
%%   \usepackage{siunitx}
%%   \sisetup{per-mode = symbol}%
%%   \usepackage{fontspec}
%%   \makeatletter\@ifpackageloaded{underscore}{}{\usepackage[strings]{underscore}}\makeatother
%%
\begingroup%
\makeatletter%
\begin{pgfpicture}%
\pgfpathrectangle{\pgfpointorigin}{\pgfqpoint{2.440945in}{1.830709in}}%
\pgfusepath{use as bounding box, clip}%
\begin{pgfscope}%
\pgfsetbuttcap%
\pgfsetmiterjoin%
\definecolor{currentfill}{rgb}{1.000000,1.000000,1.000000}%
\pgfsetfillcolor{currentfill}%
\pgfsetlinewidth{0.000000pt}%
\definecolor{currentstroke}{rgb}{1.000000,1.000000,1.000000}%
\pgfsetstrokecolor{currentstroke}%
\pgfsetdash{}{0pt}%
\pgfpathmoveto{\pgfqpoint{0.000000in}{0.000000in}}%
\pgfpathlineto{\pgfqpoint{2.440945in}{0.000000in}}%
\pgfpathlineto{\pgfqpoint{2.440945in}{1.830709in}}%
\pgfpathlineto{\pgfqpoint{0.000000in}{1.830709in}}%
\pgfpathlineto{\pgfqpoint{0.000000in}{0.000000in}}%
\pgfpathclose%
\pgfusepath{fill}%
\end{pgfscope}%
\begin{pgfscope}%
\pgfsetbuttcap%
\pgfsetmiterjoin%
\definecolor{currentfill}{rgb}{1.000000,1.000000,1.000000}%
\pgfsetfillcolor{currentfill}%
\pgfsetlinewidth{0.000000pt}%
\definecolor{currentstroke}{rgb}{0.000000,0.000000,0.000000}%
\pgfsetstrokecolor{currentstroke}%
\pgfsetstrokeopacity{0.000000}%
\pgfsetdash{}{0pt}%
\pgfpathmoveto{\pgfqpoint{0.589510in}{0.417642in}}%
\pgfpathlineto{\pgfqpoint{2.399275in}{0.417642in}}%
\pgfpathlineto{\pgfqpoint{2.399275in}{1.789039in}}%
\pgfpathlineto{\pgfqpoint{0.589510in}{1.789039in}}%
\pgfpathlineto{\pgfqpoint{0.589510in}{0.417642in}}%
\pgfpathclose%
\pgfusepath{fill}%
\end{pgfscope}%
\begin{pgfscope}%
\pgfpathrectangle{\pgfqpoint{0.589510in}{0.417642in}}{\pgfqpoint{1.809765in}{1.371397in}}%
\pgfusepath{clip}%
\pgfsetrectcap%
\pgfsetroundjoin%
\pgfsetlinewidth{0.803000pt}%
\definecolor{currentstroke}{rgb}{0.450000,0.450000,0.450000}%
\pgfsetstrokecolor{currentstroke}%
\pgfsetdash{}{0pt}%
\pgfpathmoveto{\pgfqpoint{0.671772in}{0.417642in}}%
\pgfpathlineto{\pgfqpoint{0.671772in}{1.789039in}}%
\pgfusepath{stroke}%
\end{pgfscope}%
\begin{pgfscope}%
\pgfsetbuttcap%
\pgfsetroundjoin%
\definecolor{currentfill}{rgb}{0.000000,0.000000,0.000000}%
\pgfsetfillcolor{currentfill}%
\pgfsetlinewidth{0.803000pt}%
\definecolor{currentstroke}{rgb}{0.000000,0.000000,0.000000}%
\pgfsetstrokecolor{currentstroke}%
\pgfsetdash{}{0pt}%
\pgfsys@defobject{currentmarker}{\pgfqpoint{0.000000in}{-0.048611in}}{\pgfqpoint{0.000000in}{0.000000in}}{%
\pgfpathmoveto{\pgfqpoint{0.000000in}{0.000000in}}%
\pgfpathlineto{\pgfqpoint{0.000000in}{-0.048611in}}%
\pgfusepath{stroke,fill}%
}%
\begin{pgfscope}%
\pgfsys@transformshift{0.671772in}{0.417642in}%
\pgfsys@useobject{currentmarker}{}%
\end{pgfscope}%
\end{pgfscope}%
\begin{pgfscope}%
\definecolor{textcolor}{rgb}{0.000000,0.000000,0.000000}%
\pgfsetstrokecolor{textcolor}%
\pgfsetfillcolor{textcolor}%
\pgftext[x=0.671772in,y=0.320420in,,top]{\color{textcolor}\rmfamily\fontsize{8.000000}{9.600000}\selectfont \(\displaystyle {10^{0}}\)}%
\end{pgfscope}%
\begin{pgfscope}%
\pgfpathrectangle{\pgfqpoint{0.589510in}{0.417642in}}{\pgfqpoint{1.809765in}{1.371397in}}%
\pgfusepath{clip}%
\pgfsetrectcap%
\pgfsetroundjoin%
\pgfsetlinewidth{0.803000pt}%
\definecolor{currentstroke}{rgb}{0.450000,0.450000,0.450000}%
\pgfsetstrokecolor{currentstroke}%
\pgfsetdash{}{0pt}%
\pgfpathmoveto{\pgfqpoint{1.128522in}{0.417642in}}%
\pgfpathlineto{\pgfqpoint{1.128522in}{1.789039in}}%
\pgfusepath{stroke}%
\end{pgfscope}%
\begin{pgfscope}%
\pgfsetbuttcap%
\pgfsetroundjoin%
\definecolor{currentfill}{rgb}{0.000000,0.000000,0.000000}%
\pgfsetfillcolor{currentfill}%
\pgfsetlinewidth{0.803000pt}%
\definecolor{currentstroke}{rgb}{0.000000,0.000000,0.000000}%
\pgfsetstrokecolor{currentstroke}%
\pgfsetdash{}{0pt}%
\pgfsys@defobject{currentmarker}{\pgfqpoint{0.000000in}{-0.048611in}}{\pgfqpoint{0.000000in}{0.000000in}}{%
\pgfpathmoveto{\pgfqpoint{0.000000in}{0.000000in}}%
\pgfpathlineto{\pgfqpoint{0.000000in}{-0.048611in}}%
\pgfusepath{stroke,fill}%
}%
\begin{pgfscope}%
\pgfsys@transformshift{1.128522in}{0.417642in}%
\pgfsys@useobject{currentmarker}{}%
\end{pgfscope}%
\end{pgfscope}%
\begin{pgfscope}%
\definecolor{textcolor}{rgb}{0.000000,0.000000,0.000000}%
\pgfsetstrokecolor{textcolor}%
\pgfsetfillcolor{textcolor}%
\pgftext[x=1.128522in,y=0.320420in,,top]{\color{textcolor}\rmfamily\fontsize{8.000000}{9.600000}\selectfont \(\displaystyle {10^{1}}\)}%
\end{pgfscope}%
\begin{pgfscope}%
\pgfpathrectangle{\pgfqpoint{0.589510in}{0.417642in}}{\pgfqpoint{1.809765in}{1.371397in}}%
\pgfusepath{clip}%
\pgfsetrectcap%
\pgfsetroundjoin%
\pgfsetlinewidth{0.803000pt}%
\definecolor{currentstroke}{rgb}{0.450000,0.450000,0.450000}%
\pgfsetstrokecolor{currentstroke}%
\pgfsetdash{}{0pt}%
\pgfpathmoveto{\pgfqpoint{1.585272in}{0.417642in}}%
\pgfpathlineto{\pgfqpoint{1.585272in}{1.789039in}}%
\pgfusepath{stroke}%
\end{pgfscope}%
\begin{pgfscope}%
\pgfsetbuttcap%
\pgfsetroundjoin%
\definecolor{currentfill}{rgb}{0.000000,0.000000,0.000000}%
\pgfsetfillcolor{currentfill}%
\pgfsetlinewidth{0.803000pt}%
\definecolor{currentstroke}{rgb}{0.000000,0.000000,0.000000}%
\pgfsetstrokecolor{currentstroke}%
\pgfsetdash{}{0pt}%
\pgfsys@defobject{currentmarker}{\pgfqpoint{0.000000in}{-0.048611in}}{\pgfqpoint{0.000000in}{0.000000in}}{%
\pgfpathmoveto{\pgfqpoint{0.000000in}{0.000000in}}%
\pgfpathlineto{\pgfqpoint{0.000000in}{-0.048611in}}%
\pgfusepath{stroke,fill}%
}%
\begin{pgfscope}%
\pgfsys@transformshift{1.585272in}{0.417642in}%
\pgfsys@useobject{currentmarker}{}%
\end{pgfscope}%
\end{pgfscope}%
\begin{pgfscope}%
\definecolor{textcolor}{rgb}{0.000000,0.000000,0.000000}%
\pgfsetstrokecolor{textcolor}%
\pgfsetfillcolor{textcolor}%
\pgftext[x=1.585272in,y=0.320420in,,top]{\color{textcolor}\rmfamily\fontsize{8.000000}{9.600000}\selectfont \(\displaystyle {10^{2}}\)}%
\end{pgfscope}%
\begin{pgfscope}%
\pgfpathrectangle{\pgfqpoint{0.589510in}{0.417642in}}{\pgfqpoint{1.809765in}{1.371397in}}%
\pgfusepath{clip}%
\pgfsetrectcap%
\pgfsetroundjoin%
\pgfsetlinewidth{0.803000pt}%
\definecolor{currentstroke}{rgb}{0.450000,0.450000,0.450000}%
\pgfsetstrokecolor{currentstroke}%
\pgfsetdash{}{0pt}%
\pgfpathmoveto{\pgfqpoint{2.042022in}{0.417642in}}%
\pgfpathlineto{\pgfqpoint{2.042022in}{1.789039in}}%
\pgfusepath{stroke}%
\end{pgfscope}%
\begin{pgfscope}%
\pgfsetbuttcap%
\pgfsetroundjoin%
\definecolor{currentfill}{rgb}{0.000000,0.000000,0.000000}%
\pgfsetfillcolor{currentfill}%
\pgfsetlinewidth{0.803000pt}%
\definecolor{currentstroke}{rgb}{0.000000,0.000000,0.000000}%
\pgfsetstrokecolor{currentstroke}%
\pgfsetdash{}{0pt}%
\pgfsys@defobject{currentmarker}{\pgfqpoint{0.000000in}{-0.048611in}}{\pgfqpoint{0.000000in}{0.000000in}}{%
\pgfpathmoveto{\pgfqpoint{0.000000in}{0.000000in}}%
\pgfpathlineto{\pgfqpoint{0.000000in}{-0.048611in}}%
\pgfusepath{stroke,fill}%
}%
\begin{pgfscope}%
\pgfsys@transformshift{2.042022in}{0.417642in}%
\pgfsys@useobject{currentmarker}{}%
\end{pgfscope}%
\end{pgfscope}%
\begin{pgfscope}%
\definecolor{textcolor}{rgb}{0.000000,0.000000,0.000000}%
\pgfsetstrokecolor{textcolor}%
\pgfsetfillcolor{textcolor}%
\pgftext[x=2.042022in,y=0.320420in,,top]{\color{textcolor}\rmfamily\fontsize{8.000000}{9.600000}\selectfont \(\displaystyle {10^{3}}\)}%
\end{pgfscope}%
\begin{pgfscope}%
\pgfpathrectangle{\pgfqpoint{0.589510in}{0.417642in}}{\pgfqpoint{1.809765in}{1.371397in}}%
\pgfusepath{clip}%
\pgfsetrectcap%
\pgfsetroundjoin%
\pgfsetlinewidth{0.803000pt}%
\definecolor{currentstroke}{rgb}{0.850000,0.850000,0.850000}%
\pgfsetstrokecolor{currentstroke}%
\pgfsetdash{}{0pt}%
\pgfpathmoveto{\pgfqpoint{0.601020in}{0.417642in}}%
\pgfpathlineto{\pgfqpoint{0.601020in}{1.789039in}}%
\pgfusepath{stroke}%
\end{pgfscope}%
\begin{pgfscope}%
\pgfsetbuttcap%
\pgfsetroundjoin%
\definecolor{currentfill}{rgb}{0.000000,0.000000,0.000000}%
\pgfsetfillcolor{currentfill}%
\pgfsetlinewidth{0.602250pt}%
\definecolor{currentstroke}{rgb}{0.000000,0.000000,0.000000}%
\pgfsetstrokecolor{currentstroke}%
\pgfsetdash{}{0pt}%
\pgfsys@defobject{currentmarker}{\pgfqpoint{0.000000in}{-0.027778in}}{\pgfqpoint{0.000000in}{0.000000in}}{%
\pgfpathmoveto{\pgfqpoint{0.000000in}{0.000000in}}%
\pgfpathlineto{\pgfqpoint{0.000000in}{-0.027778in}}%
\pgfusepath{stroke,fill}%
}%
\begin{pgfscope}%
\pgfsys@transformshift{0.601020in}{0.417642in}%
\pgfsys@useobject{currentmarker}{}%
\end{pgfscope}%
\end{pgfscope}%
\begin{pgfscope}%
\pgfpathrectangle{\pgfqpoint{0.589510in}{0.417642in}}{\pgfqpoint{1.809765in}{1.371397in}}%
\pgfusepath{clip}%
\pgfsetrectcap%
\pgfsetroundjoin%
\pgfsetlinewidth{0.803000pt}%
\definecolor{currentstroke}{rgb}{0.850000,0.850000,0.850000}%
\pgfsetstrokecolor{currentstroke}%
\pgfsetdash{}{0pt}%
\pgfpathmoveto{\pgfqpoint{0.627508in}{0.417642in}}%
\pgfpathlineto{\pgfqpoint{0.627508in}{1.789039in}}%
\pgfusepath{stroke}%
\end{pgfscope}%
\begin{pgfscope}%
\pgfsetbuttcap%
\pgfsetroundjoin%
\definecolor{currentfill}{rgb}{0.000000,0.000000,0.000000}%
\pgfsetfillcolor{currentfill}%
\pgfsetlinewidth{0.602250pt}%
\definecolor{currentstroke}{rgb}{0.000000,0.000000,0.000000}%
\pgfsetstrokecolor{currentstroke}%
\pgfsetdash{}{0pt}%
\pgfsys@defobject{currentmarker}{\pgfqpoint{0.000000in}{-0.027778in}}{\pgfqpoint{0.000000in}{0.000000in}}{%
\pgfpathmoveto{\pgfqpoint{0.000000in}{0.000000in}}%
\pgfpathlineto{\pgfqpoint{0.000000in}{-0.027778in}}%
\pgfusepath{stroke,fill}%
}%
\begin{pgfscope}%
\pgfsys@transformshift{0.627508in}{0.417642in}%
\pgfsys@useobject{currentmarker}{}%
\end{pgfscope}%
\end{pgfscope}%
\begin{pgfscope}%
\pgfpathrectangle{\pgfqpoint{0.589510in}{0.417642in}}{\pgfqpoint{1.809765in}{1.371397in}}%
\pgfusepath{clip}%
\pgfsetrectcap%
\pgfsetroundjoin%
\pgfsetlinewidth{0.803000pt}%
\definecolor{currentstroke}{rgb}{0.850000,0.850000,0.850000}%
\pgfsetstrokecolor{currentstroke}%
\pgfsetdash{}{0pt}%
\pgfpathmoveto{\pgfqpoint{0.650872in}{0.417642in}}%
\pgfpathlineto{\pgfqpoint{0.650872in}{1.789039in}}%
\pgfusepath{stroke}%
\end{pgfscope}%
\begin{pgfscope}%
\pgfsetbuttcap%
\pgfsetroundjoin%
\definecolor{currentfill}{rgb}{0.000000,0.000000,0.000000}%
\pgfsetfillcolor{currentfill}%
\pgfsetlinewidth{0.602250pt}%
\definecolor{currentstroke}{rgb}{0.000000,0.000000,0.000000}%
\pgfsetstrokecolor{currentstroke}%
\pgfsetdash{}{0pt}%
\pgfsys@defobject{currentmarker}{\pgfqpoint{0.000000in}{-0.027778in}}{\pgfqpoint{0.000000in}{0.000000in}}{%
\pgfpathmoveto{\pgfqpoint{0.000000in}{0.000000in}}%
\pgfpathlineto{\pgfqpoint{0.000000in}{-0.027778in}}%
\pgfusepath{stroke,fill}%
}%
\begin{pgfscope}%
\pgfsys@transformshift{0.650872in}{0.417642in}%
\pgfsys@useobject{currentmarker}{}%
\end{pgfscope}%
\end{pgfscope}%
\begin{pgfscope}%
\pgfpathrectangle{\pgfqpoint{0.589510in}{0.417642in}}{\pgfqpoint{1.809765in}{1.371397in}}%
\pgfusepath{clip}%
\pgfsetrectcap%
\pgfsetroundjoin%
\pgfsetlinewidth{0.803000pt}%
\definecolor{currentstroke}{rgb}{0.850000,0.850000,0.850000}%
\pgfsetstrokecolor{currentstroke}%
\pgfsetdash{}{0pt}%
\pgfpathmoveto{\pgfqpoint{0.809267in}{0.417642in}}%
\pgfpathlineto{\pgfqpoint{0.809267in}{1.789039in}}%
\pgfusepath{stroke}%
\end{pgfscope}%
\begin{pgfscope}%
\pgfsetbuttcap%
\pgfsetroundjoin%
\definecolor{currentfill}{rgb}{0.000000,0.000000,0.000000}%
\pgfsetfillcolor{currentfill}%
\pgfsetlinewidth{0.602250pt}%
\definecolor{currentstroke}{rgb}{0.000000,0.000000,0.000000}%
\pgfsetstrokecolor{currentstroke}%
\pgfsetdash{}{0pt}%
\pgfsys@defobject{currentmarker}{\pgfqpoint{0.000000in}{-0.027778in}}{\pgfqpoint{0.000000in}{0.000000in}}{%
\pgfpathmoveto{\pgfqpoint{0.000000in}{0.000000in}}%
\pgfpathlineto{\pgfqpoint{0.000000in}{-0.027778in}}%
\pgfusepath{stroke,fill}%
}%
\begin{pgfscope}%
\pgfsys@transformshift{0.809267in}{0.417642in}%
\pgfsys@useobject{currentmarker}{}%
\end{pgfscope}%
\end{pgfscope}%
\begin{pgfscope}%
\pgfpathrectangle{\pgfqpoint{0.589510in}{0.417642in}}{\pgfqpoint{1.809765in}{1.371397in}}%
\pgfusepath{clip}%
\pgfsetrectcap%
\pgfsetroundjoin%
\pgfsetlinewidth{0.803000pt}%
\definecolor{currentstroke}{rgb}{0.850000,0.850000,0.850000}%
\pgfsetstrokecolor{currentstroke}%
\pgfsetdash{}{0pt}%
\pgfpathmoveto{\pgfqpoint{0.889697in}{0.417642in}}%
\pgfpathlineto{\pgfqpoint{0.889697in}{1.789039in}}%
\pgfusepath{stroke}%
\end{pgfscope}%
\begin{pgfscope}%
\pgfsetbuttcap%
\pgfsetroundjoin%
\definecolor{currentfill}{rgb}{0.000000,0.000000,0.000000}%
\pgfsetfillcolor{currentfill}%
\pgfsetlinewidth{0.602250pt}%
\definecolor{currentstroke}{rgb}{0.000000,0.000000,0.000000}%
\pgfsetstrokecolor{currentstroke}%
\pgfsetdash{}{0pt}%
\pgfsys@defobject{currentmarker}{\pgfqpoint{0.000000in}{-0.027778in}}{\pgfqpoint{0.000000in}{0.000000in}}{%
\pgfpathmoveto{\pgfqpoint{0.000000in}{0.000000in}}%
\pgfpathlineto{\pgfqpoint{0.000000in}{-0.027778in}}%
\pgfusepath{stroke,fill}%
}%
\begin{pgfscope}%
\pgfsys@transformshift{0.889697in}{0.417642in}%
\pgfsys@useobject{currentmarker}{}%
\end{pgfscope}%
\end{pgfscope}%
\begin{pgfscope}%
\pgfpathrectangle{\pgfqpoint{0.589510in}{0.417642in}}{\pgfqpoint{1.809765in}{1.371397in}}%
\pgfusepath{clip}%
\pgfsetrectcap%
\pgfsetroundjoin%
\pgfsetlinewidth{0.803000pt}%
\definecolor{currentstroke}{rgb}{0.850000,0.850000,0.850000}%
\pgfsetstrokecolor{currentstroke}%
\pgfsetdash{}{0pt}%
\pgfpathmoveto{\pgfqpoint{0.946763in}{0.417642in}}%
\pgfpathlineto{\pgfqpoint{0.946763in}{1.789039in}}%
\pgfusepath{stroke}%
\end{pgfscope}%
\begin{pgfscope}%
\pgfsetbuttcap%
\pgfsetroundjoin%
\definecolor{currentfill}{rgb}{0.000000,0.000000,0.000000}%
\pgfsetfillcolor{currentfill}%
\pgfsetlinewidth{0.602250pt}%
\definecolor{currentstroke}{rgb}{0.000000,0.000000,0.000000}%
\pgfsetstrokecolor{currentstroke}%
\pgfsetdash{}{0pt}%
\pgfsys@defobject{currentmarker}{\pgfqpoint{0.000000in}{-0.027778in}}{\pgfqpoint{0.000000in}{0.000000in}}{%
\pgfpathmoveto{\pgfqpoint{0.000000in}{0.000000in}}%
\pgfpathlineto{\pgfqpoint{0.000000in}{-0.027778in}}%
\pgfusepath{stroke,fill}%
}%
\begin{pgfscope}%
\pgfsys@transformshift{0.946763in}{0.417642in}%
\pgfsys@useobject{currentmarker}{}%
\end{pgfscope}%
\end{pgfscope}%
\begin{pgfscope}%
\pgfpathrectangle{\pgfqpoint{0.589510in}{0.417642in}}{\pgfqpoint{1.809765in}{1.371397in}}%
\pgfusepath{clip}%
\pgfsetrectcap%
\pgfsetroundjoin%
\pgfsetlinewidth{0.803000pt}%
\definecolor{currentstroke}{rgb}{0.850000,0.850000,0.850000}%
\pgfsetstrokecolor{currentstroke}%
\pgfsetdash{}{0pt}%
\pgfpathmoveto{\pgfqpoint{0.991026in}{0.417642in}}%
\pgfpathlineto{\pgfqpoint{0.991026in}{1.789039in}}%
\pgfusepath{stroke}%
\end{pgfscope}%
\begin{pgfscope}%
\pgfsetbuttcap%
\pgfsetroundjoin%
\definecolor{currentfill}{rgb}{0.000000,0.000000,0.000000}%
\pgfsetfillcolor{currentfill}%
\pgfsetlinewidth{0.602250pt}%
\definecolor{currentstroke}{rgb}{0.000000,0.000000,0.000000}%
\pgfsetstrokecolor{currentstroke}%
\pgfsetdash{}{0pt}%
\pgfsys@defobject{currentmarker}{\pgfqpoint{0.000000in}{-0.027778in}}{\pgfqpoint{0.000000in}{0.000000in}}{%
\pgfpathmoveto{\pgfqpoint{0.000000in}{0.000000in}}%
\pgfpathlineto{\pgfqpoint{0.000000in}{-0.027778in}}%
\pgfusepath{stroke,fill}%
}%
\begin{pgfscope}%
\pgfsys@transformshift{0.991026in}{0.417642in}%
\pgfsys@useobject{currentmarker}{}%
\end{pgfscope}%
\end{pgfscope}%
\begin{pgfscope}%
\pgfpathrectangle{\pgfqpoint{0.589510in}{0.417642in}}{\pgfqpoint{1.809765in}{1.371397in}}%
\pgfusepath{clip}%
\pgfsetrectcap%
\pgfsetroundjoin%
\pgfsetlinewidth{0.803000pt}%
\definecolor{currentstroke}{rgb}{0.850000,0.850000,0.850000}%
\pgfsetstrokecolor{currentstroke}%
\pgfsetdash{}{0pt}%
\pgfpathmoveto{\pgfqpoint{1.027192in}{0.417642in}}%
\pgfpathlineto{\pgfqpoint{1.027192in}{1.789039in}}%
\pgfusepath{stroke}%
\end{pgfscope}%
\begin{pgfscope}%
\pgfsetbuttcap%
\pgfsetroundjoin%
\definecolor{currentfill}{rgb}{0.000000,0.000000,0.000000}%
\pgfsetfillcolor{currentfill}%
\pgfsetlinewidth{0.602250pt}%
\definecolor{currentstroke}{rgb}{0.000000,0.000000,0.000000}%
\pgfsetstrokecolor{currentstroke}%
\pgfsetdash{}{0pt}%
\pgfsys@defobject{currentmarker}{\pgfqpoint{0.000000in}{-0.027778in}}{\pgfqpoint{0.000000in}{0.000000in}}{%
\pgfpathmoveto{\pgfqpoint{0.000000in}{0.000000in}}%
\pgfpathlineto{\pgfqpoint{0.000000in}{-0.027778in}}%
\pgfusepath{stroke,fill}%
}%
\begin{pgfscope}%
\pgfsys@transformshift{1.027192in}{0.417642in}%
\pgfsys@useobject{currentmarker}{}%
\end{pgfscope}%
\end{pgfscope}%
\begin{pgfscope}%
\pgfpathrectangle{\pgfqpoint{0.589510in}{0.417642in}}{\pgfqpoint{1.809765in}{1.371397in}}%
\pgfusepath{clip}%
\pgfsetrectcap%
\pgfsetroundjoin%
\pgfsetlinewidth{0.803000pt}%
\definecolor{currentstroke}{rgb}{0.850000,0.850000,0.850000}%
\pgfsetstrokecolor{currentstroke}%
\pgfsetdash{}{0pt}%
\pgfpathmoveto{\pgfqpoint{1.057770in}{0.417642in}}%
\pgfpathlineto{\pgfqpoint{1.057770in}{1.789039in}}%
\pgfusepath{stroke}%
\end{pgfscope}%
\begin{pgfscope}%
\pgfsetbuttcap%
\pgfsetroundjoin%
\definecolor{currentfill}{rgb}{0.000000,0.000000,0.000000}%
\pgfsetfillcolor{currentfill}%
\pgfsetlinewidth{0.602250pt}%
\definecolor{currentstroke}{rgb}{0.000000,0.000000,0.000000}%
\pgfsetstrokecolor{currentstroke}%
\pgfsetdash{}{0pt}%
\pgfsys@defobject{currentmarker}{\pgfqpoint{0.000000in}{-0.027778in}}{\pgfqpoint{0.000000in}{0.000000in}}{%
\pgfpathmoveto{\pgfqpoint{0.000000in}{0.000000in}}%
\pgfpathlineto{\pgfqpoint{0.000000in}{-0.027778in}}%
\pgfusepath{stroke,fill}%
}%
\begin{pgfscope}%
\pgfsys@transformshift{1.057770in}{0.417642in}%
\pgfsys@useobject{currentmarker}{}%
\end{pgfscope}%
\end{pgfscope}%
\begin{pgfscope}%
\pgfpathrectangle{\pgfqpoint{0.589510in}{0.417642in}}{\pgfqpoint{1.809765in}{1.371397in}}%
\pgfusepath{clip}%
\pgfsetrectcap%
\pgfsetroundjoin%
\pgfsetlinewidth{0.803000pt}%
\definecolor{currentstroke}{rgb}{0.850000,0.850000,0.850000}%
\pgfsetstrokecolor{currentstroke}%
\pgfsetdash{}{0pt}%
\pgfpathmoveto{\pgfqpoint{1.084258in}{0.417642in}}%
\pgfpathlineto{\pgfqpoint{1.084258in}{1.789039in}}%
\pgfusepath{stroke}%
\end{pgfscope}%
\begin{pgfscope}%
\pgfsetbuttcap%
\pgfsetroundjoin%
\definecolor{currentfill}{rgb}{0.000000,0.000000,0.000000}%
\pgfsetfillcolor{currentfill}%
\pgfsetlinewidth{0.602250pt}%
\definecolor{currentstroke}{rgb}{0.000000,0.000000,0.000000}%
\pgfsetstrokecolor{currentstroke}%
\pgfsetdash{}{0pt}%
\pgfsys@defobject{currentmarker}{\pgfqpoint{0.000000in}{-0.027778in}}{\pgfqpoint{0.000000in}{0.000000in}}{%
\pgfpathmoveto{\pgfqpoint{0.000000in}{0.000000in}}%
\pgfpathlineto{\pgfqpoint{0.000000in}{-0.027778in}}%
\pgfusepath{stroke,fill}%
}%
\begin{pgfscope}%
\pgfsys@transformshift{1.084258in}{0.417642in}%
\pgfsys@useobject{currentmarker}{}%
\end{pgfscope}%
\end{pgfscope}%
\begin{pgfscope}%
\pgfpathrectangle{\pgfqpoint{0.589510in}{0.417642in}}{\pgfqpoint{1.809765in}{1.371397in}}%
\pgfusepath{clip}%
\pgfsetrectcap%
\pgfsetroundjoin%
\pgfsetlinewidth{0.803000pt}%
\definecolor{currentstroke}{rgb}{0.850000,0.850000,0.850000}%
\pgfsetstrokecolor{currentstroke}%
\pgfsetdash{}{0pt}%
\pgfpathmoveto{\pgfqpoint{1.107622in}{0.417642in}}%
\pgfpathlineto{\pgfqpoint{1.107622in}{1.789039in}}%
\pgfusepath{stroke}%
\end{pgfscope}%
\begin{pgfscope}%
\pgfsetbuttcap%
\pgfsetroundjoin%
\definecolor{currentfill}{rgb}{0.000000,0.000000,0.000000}%
\pgfsetfillcolor{currentfill}%
\pgfsetlinewidth{0.602250pt}%
\definecolor{currentstroke}{rgb}{0.000000,0.000000,0.000000}%
\pgfsetstrokecolor{currentstroke}%
\pgfsetdash{}{0pt}%
\pgfsys@defobject{currentmarker}{\pgfqpoint{0.000000in}{-0.027778in}}{\pgfqpoint{0.000000in}{0.000000in}}{%
\pgfpathmoveto{\pgfqpoint{0.000000in}{0.000000in}}%
\pgfpathlineto{\pgfqpoint{0.000000in}{-0.027778in}}%
\pgfusepath{stroke,fill}%
}%
\begin{pgfscope}%
\pgfsys@transformshift{1.107622in}{0.417642in}%
\pgfsys@useobject{currentmarker}{}%
\end{pgfscope}%
\end{pgfscope}%
\begin{pgfscope}%
\pgfpathrectangle{\pgfqpoint{0.589510in}{0.417642in}}{\pgfqpoint{1.809765in}{1.371397in}}%
\pgfusepath{clip}%
\pgfsetrectcap%
\pgfsetroundjoin%
\pgfsetlinewidth{0.803000pt}%
\definecolor{currentstroke}{rgb}{0.850000,0.850000,0.850000}%
\pgfsetstrokecolor{currentstroke}%
\pgfsetdash{}{0pt}%
\pgfpathmoveto{\pgfqpoint{1.266017in}{0.417642in}}%
\pgfpathlineto{\pgfqpoint{1.266017in}{1.789039in}}%
\pgfusepath{stroke}%
\end{pgfscope}%
\begin{pgfscope}%
\pgfsetbuttcap%
\pgfsetroundjoin%
\definecolor{currentfill}{rgb}{0.000000,0.000000,0.000000}%
\pgfsetfillcolor{currentfill}%
\pgfsetlinewidth{0.602250pt}%
\definecolor{currentstroke}{rgb}{0.000000,0.000000,0.000000}%
\pgfsetstrokecolor{currentstroke}%
\pgfsetdash{}{0pt}%
\pgfsys@defobject{currentmarker}{\pgfqpoint{0.000000in}{-0.027778in}}{\pgfqpoint{0.000000in}{0.000000in}}{%
\pgfpathmoveto{\pgfqpoint{0.000000in}{0.000000in}}%
\pgfpathlineto{\pgfqpoint{0.000000in}{-0.027778in}}%
\pgfusepath{stroke,fill}%
}%
\begin{pgfscope}%
\pgfsys@transformshift{1.266017in}{0.417642in}%
\pgfsys@useobject{currentmarker}{}%
\end{pgfscope}%
\end{pgfscope}%
\begin{pgfscope}%
\pgfpathrectangle{\pgfqpoint{0.589510in}{0.417642in}}{\pgfqpoint{1.809765in}{1.371397in}}%
\pgfusepath{clip}%
\pgfsetrectcap%
\pgfsetroundjoin%
\pgfsetlinewidth{0.803000pt}%
\definecolor{currentstroke}{rgb}{0.850000,0.850000,0.850000}%
\pgfsetstrokecolor{currentstroke}%
\pgfsetdash{}{0pt}%
\pgfpathmoveto{\pgfqpoint{1.346447in}{0.417642in}}%
\pgfpathlineto{\pgfqpoint{1.346447in}{1.789039in}}%
\pgfusepath{stroke}%
\end{pgfscope}%
\begin{pgfscope}%
\pgfsetbuttcap%
\pgfsetroundjoin%
\definecolor{currentfill}{rgb}{0.000000,0.000000,0.000000}%
\pgfsetfillcolor{currentfill}%
\pgfsetlinewidth{0.602250pt}%
\definecolor{currentstroke}{rgb}{0.000000,0.000000,0.000000}%
\pgfsetstrokecolor{currentstroke}%
\pgfsetdash{}{0pt}%
\pgfsys@defobject{currentmarker}{\pgfqpoint{0.000000in}{-0.027778in}}{\pgfqpoint{0.000000in}{0.000000in}}{%
\pgfpathmoveto{\pgfqpoint{0.000000in}{0.000000in}}%
\pgfpathlineto{\pgfqpoint{0.000000in}{-0.027778in}}%
\pgfusepath{stroke,fill}%
}%
\begin{pgfscope}%
\pgfsys@transformshift{1.346447in}{0.417642in}%
\pgfsys@useobject{currentmarker}{}%
\end{pgfscope}%
\end{pgfscope}%
\begin{pgfscope}%
\pgfpathrectangle{\pgfqpoint{0.589510in}{0.417642in}}{\pgfqpoint{1.809765in}{1.371397in}}%
\pgfusepath{clip}%
\pgfsetrectcap%
\pgfsetroundjoin%
\pgfsetlinewidth{0.803000pt}%
\definecolor{currentstroke}{rgb}{0.850000,0.850000,0.850000}%
\pgfsetstrokecolor{currentstroke}%
\pgfsetdash{}{0pt}%
\pgfpathmoveto{\pgfqpoint{1.403513in}{0.417642in}}%
\pgfpathlineto{\pgfqpoint{1.403513in}{1.789039in}}%
\pgfusepath{stroke}%
\end{pgfscope}%
\begin{pgfscope}%
\pgfsetbuttcap%
\pgfsetroundjoin%
\definecolor{currentfill}{rgb}{0.000000,0.000000,0.000000}%
\pgfsetfillcolor{currentfill}%
\pgfsetlinewidth{0.602250pt}%
\definecolor{currentstroke}{rgb}{0.000000,0.000000,0.000000}%
\pgfsetstrokecolor{currentstroke}%
\pgfsetdash{}{0pt}%
\pgfsys@defobject{currentmarker}{\pgfqpoint{0.000000in}{-0.027778in}}{\pgfqpoint{0.000000in}{0.000000in}}{%
\pgfpathmoveto{\pgfqpoint{0.000000in}{0.000000in}}%
\pgfpathlineto{\pgfqpoint{0.000000in}{-0.027778in}}%
\pgfusepath{stroke,fill}%
}%
\begin{pgfscope}%
\pgfsys@transformshift{1.403513in}{0.417642in}%
\pgfsys@useobject{currentmarker}{}%
\end{pgfscope}%
\end{pgfscope}%
\begin{pgfscope}%
\pgfpathrectangle{\pgfqpoint{0.589510in}{0.417642in}}{\pgfqpoint{1.809765in}{1.371397in}}%
\pgfusepath{clip}%
\pgfsetrectcap%
\pgfsetroundjoin%
\pgfsetlinewidth{0.803000pt}%
\definecolor{currentstroke}{rgb}{0.850000,0.850000,0.850000}%
\pgfsetstrokecolor{currentstroke}%
\pgfsetdash{}{0pt}%
\pgfpathmoveto{\pgfqpoint{1.447776in}{0.417642in}}%
\pgfpathlineto{\pgfqpoint{1.447776in}{1.789039in}}%
\pgfusepath{stroke}%
\end{pgfscope}%
\begin{pgfscope}%
\pgfsetbuttcap%
\pgfsetroundjoin%
\definecolor{currentfill}{rgb}{0.000000,0.000000,0.000000}%
\pgfsetfillcolor{currentfill}%
\pgfsetlinewidth{0.602250pt}%
\definecolor{currentstroke}{rgb}{0.000000,0.000000,0.000000}%
\pgfsetstrokecolor{currentstroke}%
\pgfsetdash{}{0pt}%
\pgfsys@defobject{currentmarker}{\pgfqpoint{0.000000in}{-0.027778in}}{\pgfqpoint{0.000000in}{0.000000in}}{%
\pgfpathmoveto{\pgfqpoint{0.000000in}{0.000000in}}%
\pgfpathlineto{\pgfqpoint{0.000000in}{-0.027778in}}%
\pgfusepath{stroke,fill}%
}%
\begin{pgfscope}%
\pgfsys@transformshift{1.447776in}{0.417642in}%
\pgfsys@useobject{currentmarker}{}%
\end{pgfscope}%
\end{pgfscope}%
\begin{pgfscope}%
\pgfpathrectangle{\pgfqpoint{0.589510in}{0.417642in}}{\pgfqpoint{1.809765in}{1.371397in}}%
\pgfusepath{clip}%
\pgfsetrectcap%
\pgfsetroundjoin%
\pgfsetlinewidth{0.803000pt}%
\definecolor{currentstroke}{rgb}{0.850000,0.850000,0.850000}%
\pgfsetstrokecolor{currentstroke}%
\pgfsetdash{}{0pt}%
\pgfpathmoveto{\pgfqpoint{1.483942in}{0.417642in}}%
\pgfpathlineto{\pgfqpoint{1.483942in}{1.789039in}}%
\pgfusepath{stroke}%
\end{pgfscope}%
\begin{pgfscope}%
\pgfsetbuttcap%
\pgfsetroundjoin%
\definecolor{currentfill}{rgb}{0.000000,0.000000,0.000000}%
\pgfsetfillcolor{currentfill}%
\pgfsetlinewidth{0.602250pt}%
\definecolor{currentstroke}{rgb}{0.000000,0.000000,0.000000}%
\pgfsetstrokecolor{currentstroke}%
\pgfsetdash{}{0pt}%
\pgfsys@defobject{currentmarker}{\pgfqpoint{0.000000in}{-0.027778in}}{\pgfqpoint{0.000000in}{0.000000in}}{%
\pgfpathmoveto{\pgfqpoint{0.000000in}{0.000000in}}%
\pgfpathlineto{\pgfqpoint{0.000000in}{-0.027778in}}%
\pgfusepath{stroke,fill}%
}%
\begin{pgfscope}%
\pgfsys@transformshift{1.483942in}{0.417642in}%
\pgfsys@useobject{currentmarker}{}%
\end{pgfscope}%
\end{pgfscope}%
\begin{pgfscope}%
\pgfpathrectangle{\pgfqpoint{0.589510in}{0.417642in}}{\pgfqpoint{1.809765in}{1.371397in}}%
\pgfusepath{clip}%
\pgfsetrectcap%
\pgfsetroundjoin%
\pgfsetlinewidth{0.803000pt}%
\definecolor{currentstroke}{rgb}{0.850000,0.850000,0.850000}%
\pgfsetstrokecolor{currentstroke}%
\pgfsetdash{}{0pt}%
\pgfpathmoveto{\pgfqpoint{1.514520in}{0.417642in}}%
\pgfpathlineto{\pgfqpoint{1.514520in}{1.789039in}}%
\pgfusepath{stroke}%
\end{pgfscope}%
\begin{pgfscope}%
\pgfsetbuttcap%
\pgfsetroundjoin%
\definecolor{currentfill}{rgb}{0.000000,0.000000,0.000000}%
\pgfsetfillcolor{currentfill}%
\pgfsetlinewidth{0.602250pt}%
\definecolor{currentstroke}{rgb}{0.000000,0.000000,0.000000}%
\pgfsetstrokecolor{currentstroke}%
\pgfsetdash{}{0pt}%
\pgfsys@defobject{currentmarker}{\pgfqpoint{0.000000in}{-0.027778in}}{\pgfqpoint{0.000000in}{0.000000in}}{%
\pgfpathmoveto{\pgfqpoint{0.000000in}{0.000000in}}%
\pgfpathlineto{\pgfqpoint{0.000000in}{-0.027778in}}%
\pgfusepath{stroke,fill}%
}%
\begin{pgfscope}%
\pgfsys@transformshift{1.514520in}{0.417642in}%
\pgfsys@useobject{currentmarker}{}%
\end{pgfscope}%
\end{pgfscope}%
\begin{pgfscope}%
\pgfpathrectangle{\pgfqpoint{0.589510in}{0.417642in}}{\pgfqpoint{1.809765in}{1.371397in}}%
\pgfusepath{clip}%
\pgfsetrectcap%
\pgfsetroundjoin%
\pgfsetlinewidth{0.803000pt}%
\definecolor{currentstroke}{rgb}{0.850000,0.850000,0.850000}%
\pgfsetstrokecolor{currentstroke}%
\pgfsetdash{}{0pt}%
\pgfpathmoveto{\pgfqpoint{1.541008in}{0.417642in}}%
\pgfpathlineto{\pgfqpoint{1.541008in}{1.789039in}}%
\pgfusepath{stroke}%
\end{pgfscope}%
\begin{pgfscope}%
\pgfsetbuttcap%
\pgfsetroundjoin%
\definecolor{currentfill}{rgb}{0.000000,0.000000,0.000000}%
\pgfsetfillcolor{currentfill}%
\pgfsetlinewidth{0.602250pt}%
\definecolor{currentstroke}{rgb}{0.000000,0.000000,0.000000}%
\pgfsetstrokecolor{currentstroke}%
\pgfsetdash{}{0pt}%
\pgfsys@defobject{currentmarker}{\pgfqpoint{0.000000in}{-0.027778in}}{\pgfqpoint{0.000000in}{0.000000in}}{%
\pgfpathmoveto{\pgfqpoint{0.000000in}{0.000000in}}%
\pgfpathlineto{\pgfqpoint{0.000000in}{-0.027778in}}%
\pgfusepath{stroke,fill}%
}%
\begin{pgfscope}%
\pgfsys@transformshift{1.541008in}{0.417642in}%
\pgfsys@useobject{currentmarker}{}%
\end{pgfscope}%
\end{pgfscope}%
\begin{pgfscope}%
\pgfpathrectangle{\pgfqpoint{0.589510in}{0.417642in}}{\pgfqpoint{1.809765in}{1.371397in}}%
\pgfusepath{clip}%
\pgfsetrectcap%
\pgfsetroundjoin%
\pgfsetlinewidth{0.803000pt}%
\definecolor{currentstroke}{rgb}{0.850000,0.850000,0.850000}%
\pgfsetstrokecolor{currentstroke}%
\pgfsetdash{}{0pt}%
\pgfpathmoveto{\pgfqpoint{1.564372in}{0.417642in}}%
\pgfpathlineto{\pgfqpoint{1.564372in}{1.789039in}}%
\pgfusepath{stroke}%
\end{pgfscope}%
\begin{pgfscope}%
\pgfsetbuttcap%
\pgfsetroundjoin%
\definecolor{currentfill}{rgb}{0.000000,0.000000,0.000000}%
\pgfsetfillcolor{currentfill}%
\pgfsetlinewidth{0.602250pt}%
\definecolor{currentstroke}{rgb}{0.000000,0.000000,0.000000}%
\pgfsetstrokecolor{currentstroke}%
\pgfsetdash{}{0pt}%
\pgfsys@defobject{currentmarker}{\pgfqpoint{0.000000in}{-0.027778in}}{\pgfqpoint{0.000000in}{0.000000in}}{%
\pgfpathmoveto{\pgfqpoint{0.000000in}{0.000000in}}%
\pgfpathlineto{\pgfqpoint{0.000000in}{-0.027778in}}%
\pgfusepath{stroke,fill}%
}%
\begin{pgfscope}%
\pgfsys@transformshift{1.564372in}{0.417642in}%
\pgfsys@useobject{currentmarker}{}%
\end{pgfscope}%
\end{pgfscope}%
\begin{pgfscope}%
\pgfpathrectangle{\pgfqpoint{0.589510in}{0.417642in}}{\pgfqpoint{1.809765in}{1.371397in}}%
\pgfusepath{clip}%
\pgfsetrectcap%
\pgfsetroundjoin%
\pgfsetlinewidth{0.803000pt}%
\definecolor{currentstroke}{rgb}{0.850000,0.850000,0.850000}%
\pgfsetstrokecolor{currentstroke}%
\pgfsetdash{}{0pt}%
\pgfpathmoveto{\pgfqpoint{1.722767in}{0.417642in}}%
\pgfpathlineto{\pgfqpoint{1.722767in}{1.789039in}}%
\pgfusepath{stroke}%
\end{pgfscope}%
\begin{pgfscope}%
\pgfsetbuttcap%
\pgfsetroundjoin%
\definecolor{currentfill}{rgb}{0.000000,0.000000,0.000000}%
\pgfsetfillcolor{currentfill}%
\pgfsetlinewidth{0.602250pt}%
\definecolor{currentstroke}{rgb}{0.000000,0.000000,0.000000}%
\pgfsetstrokecolor{currentstroke}%
\pgfsetdash{}{0pt}%
\pgfsys@defobject{currentmarker}{\pgfqpoint{0.000000in}{-0.027778in}}{\pgfqpoint{0.000000in}{0.000000in}}{%
\pgfpathmoveto{\pgfqpoint{0.000000in}{0.000000in}}%
\pgfpathlineto{\pgfqpoint{0.000000in}{-0.027778in}}%
\pgfusepath{stroke,fill}%
}%
\begin{pgfscope}%
\pgfsys@transformshift{1.722767in}{0.417642in}%
\pgfsys@useobject{currentmarker}{}%
\end{pgfscope}%
\end{pgfscope}%
\begin{pgfscope}%
\pgfpathrectangle{\pgfqpoint{0.589510in}{0.417642in}}{\pgfqpoint{1.809765in}{1.371397in}}%
\pgfusepath{clip}%
\pgfsetrectcap%
\pgfsetroundjoin%
\pgfsetlinewidth{0.803000pt}%
\definecolor{currentstroke}{rgb}{0.850000,0.850000,0.850000}%
\pgfsetstrokecolor{currentstroke}%
\pgfsetdash{}{0pt}%
\pgfpathmoveto{\pgfqpoint{1.803197in}{0.417642in}}%
\pgfpathlineto{\pgfqpoint{1.803197in}{1.789039in}}%
\pgfusepath{stroke}%
\end{pgfscope}%
\begin{pgfscope}%
\pgfsetbuttcap%
\pgfsetroundjoin%
\definecolor{currentfill}{rgb}{0.000000,0.000000,0.000000}%
\pgfsetfillcolor{currentfill}%
\pgfsetlinewidth{0.602250pt}%
\definecolor{currentstroke}{rgb}{0.000000,0.000000,0.000000}%
\pgfsetstrokecolor{currentstroke}%
\pgfsetdash{}{0pt}%
\pgfsys@defobject{currentmarker}{\pgfqpoint{0.000000in}{-0.027778in}}{\pgfqpoint{0.000000in}{0.000000in}}{%
\pgfpathmoveto{\pgfqpoint{0.000000in}{0.000000in}}%
\pgfpathlineto{\pgfqpoint{0.000000in}{-0.027778in}}%
\pgfusepath{stroke,fill}%
}%
\begin{pgfscope}%
\pgfsys@transformshift{1.803197in}{0.417642in}%
\pgfsys@useobject{currentmarker}{}%
\end{pgfscope}%
\end{pgfscope}%
\begin{pgfscope}%
\pgfpathrectangle{\pgfqpoint{0.589510in}{0.417642in}}{\pgfqpoint{1.809765in}{1.371397in}}%
\pgfusepath{clip}%
\pgfsetrectcap%
\pgfsetroundjoin%
\pgfsetlinewidth{0.803000pt}%
\definecolor{currentstroke}{rgb}{0.850000,0.850000,0.850000}%
\pgfsetstrokecolor{currentstroke}%
\pgfsetdash{}{0pt}%
\pgfpathmoveto{\pgfqpoint{1.860263in}{0.417642in}}%
\pgfpathlineto{\pgfqpoint{1.860263in}{1.789039in}}%
\pgfusepath{stroke}%
\end{pgfscope}%
\begin{pgfscope}%
\pgfsetbuttcap%
\pgfsetroundjoin%
\definecolor{currentfill}{rgb}{0.000000,0.000000,0.000000}%
\pgfsetfillcolor{currentfill}%
\pgfsetlinewidth{0.602250pt}%
\definecolor{currentstroke}{rgb}{0.000000,0.000000,0.000000}%
\pgfsetstrokecolor{currentstroke}%
\pgfsetdash{}{0pt}%
\pgfsys@defobject{currentmarker}{\pgfqpoint{0.000000in}{-0.027778in}}{\pgfqpoint{0.000000in}{0.000000in}}{%
\pgfpathmoveto{\pgfqpoint{0.000000in}{0.000000in}}%
\pgfpathlineto{\pgfqpoint{0.000000in}{-0.027778in}}%
\pgfusepath{stroke,fill}%
}%
\begin{pgfscope}%
\pgfsys@transformshift{1.860263in}{0.417642in}%
\pgfsys@useobject{currentmarker}{}%
\end{pgfscope}%
\end{pgfscope}%
\begin{pgfscope}%
\pgfpathrectangle{\pgfqpoint{0.589510in}{0.417642in}}{\pgfqpoint{1.809765in}{1.371397in}}%
\pgfusepath{clip}%
\pgfsetrectcap%
\pgfsetroundjoin%
\pgfsetlinewidth{0.803000pt}%
\definecolor{currentstroke}{rgb}{0.850000,0.850000,0.850000}%
\pgfsetstrokecolor{currentstroke}%
\pgfsetdash{}{0pt}%
\pgfpathmoveto{\pgfqpoint{1.904526in}{0.417642in}}%
\pgfpathlineto{\pgfqpoint{1.904526in}{1.789039in}}%
\pgfusepath{stroke}%
\end{pgfscope}%
\begin{pgfscope}%
\pgfsetbuttcap%
\pgfsetroundjoin%
\definecolor{currentfill}{rgb}{0.000000,0.000000,0.000000}%
\pgfsetfillcolor{currentfill}%
\pgfsetlinewidth{0.602250pt}%
\definecolor{currentstroke}{rgb}{0.000000,0.000000,0.000000}%
\pgfsetstrokecolor{currentstroke}%
\pgfsetdash{}{0pt}%
\pgfsys@defobject{currentmarker}{\pgfqpoint{0.000000in}{-0.027778in}}{\pgfqpoint{0.000000in}{0.000000in}}{%
\pgfpathmoveto{\pgfqpoint{0.000000in}{0.000000in}}%
\pgfpathlineto{\pgfqpoint{0.000000in}{-0.027778in}}%
\pgfusepath{stroke,fill}%
}%
\begin{pgfscope}%
\pgfsys@transformshift{1.904526in}{0.417642in}%
\pgfsys@useobject{currentmarker}{}%
\end{pgfscope}%
\end{pgfscope}%
\begin{pgfscope}%
\pgfpathrectangle{\pgfqpoint{0.589510in}{0.417642in}}{\pgfqpoint{1.809765in}{1.371397in}}%
\pgfusepath{clip}%
\pgfsetrectcap%
\pgfsetroundjoin%
\pgfsetlinewidth{0.803000pt}%
\definecolor{currentstroke}{rgb}{0.850000,0.850000,0.850000}%
\pgfsetstrokecolor{currentstroke}%
\pgfsetdash{}{0pt}%
\pgfpathmoveto{\pgfqpoint{1.940693in}{0.417642in}}%
\pgfpathlineto{\pgfqpoint{1.940693in}{1.789039in}}%
\pgfusepath{stroke}%
\end{pgfscope}%
\begin{pgfscope}%
\pgfsetbuttcap%
\pgfsetroundjoin%
\definecolor{currentfill}{rgb}{0.000000,0.000000,0.000000}%
\pgfsetfillcolor{currentfill}%
\pgfsetlinewidth{0.602250pt}%
\definecolor{currentstroke}{rgb}{0.000000,0.000000,0.000000}%
\pgfsetstrokecolor{currentstroke}%
\pgfsetdash{}{0pt}%
\pgfsys@defobject{currentmarker}{\pgfqpoint{0.000000in}{-0.027778in}}{\pgfqpoint{0.000000in}{0.000000in}}{%
\pgfpathmoveto{\pgfqpoint{0.000000in}{0.000000in}}%
\pgfpathlineto{\pgfqpoint{0.000000in}{-0.027778in}}%
\pgfusepath{stroke,fill}%
}%
\begin{pgfscope}%
\pgfsys@transformshift{1.940693in}{0.417642in}%
\pgfsys@useobject{currentmarker}{}%
\end{pgfscope}%
\end{pgfscope}%
\begin{pgfscope}%
\pgfpathrectangle{\pgfqpoint{0.589510in}{0.417642in}}{\pgfqpoint{1.809765in}{1.371397in}}%
\pgfusepath{clip}%
\pgfsetrectcap%
\pgfsetroundjoin%
\pgfsetlinewidth{0.803000pt}%
\definecolor{currentstroke}{rgb}{0.850000,0.850000,0.850000}%
\pgfsetstrokecolor{currentstroke}%
\pgfsetdash{}{0pt}%
\pgfpathmoveto{\pgfqpoint{1.971270in}{0.417642in}}%
\pgfpathlineto{\pgfqpoint{1.971270in}{1.789039in}}%
\pgfusepath{stroke}%
\end{pgfscope}%
\begin{pgfscope}%
\pgfsetbuttcap%
\pgfsetroundjoin%
\definecolor{currentfill}{rgb}{0.000000,0.000000,0.000000}%
\pgfsetfillcolor{currentfill}%
\pgfsetlinewidth{0.602250pt}%
\definecolor{currentstroke}{rgb}{0.000000,0.000000,0.000000}%
\pgfsetstrokecolor{currentstroke}%
\pgfsetdash{}{0pt}%
\pgfsys@defobject{currentmarker}{\pgfqpoint{0.000000in}{-0.027778in}}{\pgfqpoint{0.000000in}{0.000000in}}{%
\pgfpathmoveto{\pgfqpoint{0.000000in}{0.000000in}}%
\pgfpathlineto{\pgfqpoint{0.000000in}{-0.027778in}}%
\pgfusepath{stroke,fill}%
}%
\begin{pgfscope}%
\pgfsys@transformshift{1.971270in}{0.417642in}%
\pgfsys@useobject{currentmarker}{}%
\end{pgfscope}%
\end{pgfscope}%
\begin{pgfscope}%
\pgfpathrectangle{\pgfqpoint{0.589510in}{0.417642in}}{\pgfqpoint{1.809765in}{1.371397in}}%
\pgfusepath{clip}%
\pgfsetrectcap%
\pgfsetroundjoin%
\pgfsetlinewidth{0.803000pt}%
\definecolor{currentstroke}{rgb}{0.850000,0.850000,0.850000}%
\pgfsetstrokecolor{currentstroke}%
\pgfsetdash{}{0pt}%
\pgfpathmoveto{\pgfqpoint{1.997758in}{0.417642in}}%
\pgfpathlineto{\pgfqpoint{1.997758in}{1.789039in}}%
\pgfusepath{stroke}%
\end{pgfscope}%
\begin{pgfscope}%
\pgfsetbuttcap%
\pgfsetroundjoin%
\definecolor{currentfill}{rgb}{0.000000,0.000000,0.000000}%
\pgfsetfillcolor{currentfill}%
\pgfsetlinewidth{0.602250pt}%
\definecolor{currentstroke}{rgb}{0.000000,0.000000,0.000000}%
\pgfsetstrokecolor{currentstroke}%
\pgfsetdash{}{0pt}%
\pgfsys@defobject{currentmarker}{\pgfqpoint{0.000000in}{-0.027778in}}{\pgfqpoint{0.000000in}{0.000000in}}{%
\pgfpathmoveto{\pgfqpoint{0.000000in}{0.000000in}}%
\pgfpathlineto{\pgfqpoint{0.000000in}{-0.027778in}}%
\pgfusepath{stroke,fill}%
}%
\begin{pgfscope}%
\pgfsys@transformshift{1.997758in}{0.417642in}%
\pgfsys@useobject{currentmarker}{}%
\end{pgfscope}%
\end{pgfscope}%
\begin{pgfscope}%
\pgfpathrectangle{\pgfqpoint{0.589510in}{0.417642in}}{\pgfqpoint{1.809765in}{1.371397in}}%
\pgfusepath{clip}%
\pgfsetrectcap%
\pgfsetroundjoin%
\pgfsetlinewidth{0.803000pt}%
\definecolor{currentstroke}{rgb}{0.850000,0.850000,0.850000}%
\pgfsetstrokecolor{currentstroke}%
\pgfsetdash{}{0pt}%
\pgfpathmoveto{\pgfqpoint{2.021122in}{0.417642in}}%
\pgfpathlineto{\pgfqpoint{2.021122in}{1.789039in}}%
\pgfusepath{stroke}%
\end{pgfscope}%
\begin{pgfscope}%
\pgfsetbuttcap%
\pgfsetroundjoin%
\definecolor{currentfill}{rgb}{0.000000,0.000000,0.000000}%
\pgfsetfillcolor{currentfill}%
\pgfsetlinewidth{0.602250pt}%
\definecolor{currentstroke}{rgb}{0.000000,0.000000,0.000000}%
\pgfsetstrokecolor{currentstroke}%
\pgfsetdash{}{0pt}%
\pgfsys@defobject{currentmarker}{\pgfqpoint{0.000000in}{-0.027778in}}{\pgfqpoint{0.000000in}{0.000000in}}{%
\pgfpathmoveto{\pgfqpoint{0.000000in}{0.000000in}}%
\pgfpathlineto{\pgfqpoint{0.000000in}{-0.027778in}}%
\pgfusepath{stroke,fill}%
}%
\begin{pgfscope}%
\pgfsys@transformshift{2.021122in}{0.417642in}%
\pgfsys@useobject{currentmarker}{}%
\end{pgfscope}%
\end{pgfscope}%
\begin{pgfscope}%
\pgfpathrectangle{\pgfqpoint{0.589510in}{0.417642in}}{\pgfqpoint{1.809765in}{1.371397in}}%
\pgfusepath{clip}%
\pgfsetrectcap%
\pgfsetroundjoin%
\pgfsetlinewidth{0.803000pt}%
\definecolor{currentstroke}{rgb}{0.850000,0.850000,0.850000}%
\pgfsetstrokecolor{currentstroke}%
\pgfsetdash{}{0pt}%
\pgfpathmoveto{\pgfqpoint{2.179517in}{0.417642in}}%
\pgfpathlineto{\pgfqpoint{2.179517in}{1.789039in}}%
\pgfusepath{stroke}%
\end{pgfscope}%
\begin{pgfscope}%
\pgfsetbuttcap%
\pgfsetroundjoin%
\definecolor{currentfill}{rgb}{0.000000,0.000000,0.000000}%
\pgfsetfillcolor{currentfill}%
\pgfsetlinewidth{0.602250pt}%
\definecolor{currentstroke}{rgb}{0.000000,0.000000,0.000000}%
\pgfsetstrokecolor{currentstroke}%
\pgfsetdash{}{0pt}%
\pgfsys@defobject{currentmarker}{\pgfqpoint{0.000000in}{-0.027778in}}{\pgfqpoint{0.000000in}{0.000000in}}{%
\pgfpathmoveto{\pgfqpoint{0.000000in}{0.000000in}}%
\pgfpathlineto{\pgfqpoint{0.000000in}{-0.027778in}}%
\pgfusepath{stroke,fill}%
}%
\begin{pgfscope}%
\pgfsys@transformshift{2.179517in}{0.417642in}%
\pgfsys@useobject{currentmarker}{}%
\end{pgfscope}%
\end{pgfscope}%
\begin{pgfscope}%
\pgfpathrectangle{\pgfqpoint{0.589510in}{0.417642in}}{\pgfqpoint{1.809765in}{1.371397in}}%
\pgfusepath{clip}%
\pgfsetrectcap%
\pgfsetroundjoin%
\pgfsetlinewidth{0.803000pt}%
\definecolor{currentstroke}{rgb}{0.850000,0.850000,0.850000}%
\pgfsetstrokecolor{currentstroke}%
\pgfsetdash{}{0pt}%
\pgfpathmoveto{\pgfqpoint{2.259947in}{0.417642in}}%
\pgfpathlineto{\pgfqpoint{2.259947in}{1.789039in}}%
\pgfusepath{stroke}%
\end{pgfscope}%
\begin{pgfscope}%
\pgfsetbuttcap%
\pgfsetroundjoin%
\definecolor{currentfill}{rgb}{0.000000,0.000000,0.000000}%
\pgfsetfillcolor{currentfill}%
\pgfsetlinewidth{0.602250pt}%
\definecolor{currentstroke}{rgb}{0.000000,0.000000,0.000000}%
\pgfsetstrokecolor{currentstroke}%
\pgfsetdash{}{0pt}%
\pgfsys@defobject{currentmarker}{\pgfqpoint{0.000000in}{-0.027778in}}{\pgfqpoint{0.000000in}{0.000000in}}{%
\pgfpathmoveto{\pgfqpoint{0.000000in}{0.000000in}}%
\pgfpathlineto{\pgfqpoint{0.000000in}{-0.027778in}}%
\pgfusepath{stroke,fill}%
}%
\begin{pgfscope}%
\pgfsys@transformshift{2.259947in}{0.417642in}%
\pgfsys@useobject{currentmarker}{}%
\end{pgfscope}%
\end{pgfscope}%
\begin{pgfscope}%
\pgfpathrectangle{\pgfqpoint{0.589510in}{0.417642in}}{\pgfqpoint{1.809765in}{1.371397in}}%
\pgfusepath{clip}%
\pgfsetrectcap%
\pgfsetroundjoin%
\pgfsetlinewidth{0.803000pt}%
\definecolor{currentstroke}{rgb}{0.850000,0.850000,0.850000}%
\pgfsetstrokecolor{currentstroke}%
\pgfsetdash{}{0pt}%
\pgfpathmoveto{\pgfqpoint{2.317013in}{0.417642in}}%
\pgfpathlineto{\pgfqpoint{2.317013in}{1.789039in}}%
\pgfusepath{stroke}%
\end{pgfscope}%
\begin{pgfscope}%
\pgfsetbuttcap%
\pgfsetroundjoin%
\definecolor{currentfill}{rgb}{0.000000,0.000000,0.000000}%
\pgfsetfillcolor{currentfill}%
\pgfsetlinewidth{0.602250pt}%
\definecolor{currentstroke}{rgb}{0.000000,0.000000,0.000000}%
\pgfsetstrokecolor{currentstroke}%
\pgfsetdash{}{0pt}%
\pgfsys@defobject{currentmarker}{\pgfqpoint{0.000000in}{-0.027778in}}{\pgfqpoint{0.000000in}{0.000000in}}{%
\pgfpathmoveto{\pgfqpoint{0.000000in}{0.000000in}}%
\pgfpathlineto{\pgfqpoint{0.000000in}{-0.027778in}}%
\pgfusepath{stroke,fill}%
}%
\begin{pgfscope}%
\pgfsys@transformshift{2.317013in}{0.417642in}%
\pgfsys@useobject{currentmarker}{}%
\end{pgfscope}%
\end{pgfscope}%
\begin{pgfscope}%
\pgfpathrectangle{\pgfqpoint{0.589510in}{0.417642in}}{\pgfqpoint{1.809765in}{1.371397in}}%
\pgfusepath{clip}%
\pgfsetrectcap%
\pgfsetroundjoin%
\pgfsetlinewidth{0.803000pt}%
\definecolor{currentstroke}{rgb}{0.850000,0.850000,0.850000}%
\pgfsetstrokecolor{currentstroke}%
\pgfsetdash{}{0pt}%
\pgfpathmoveto{\pgfqpoint{2.361277in}{0.417642in}}%
\pgfpathlineto{\pgfqpoint{2.361277in}{1.789039in}}%
\pgfusepath{stroke}%
\end{pgfscope}%
\begin{pgfscope}%
\pgfsetbuttcap%
\pgfsetroundjoin%
\definecolor{currentfill}{rgb}{0.000000,0.000000,0.000000}%
\pgfsetfillcolor{currentfill}%
\pgfsetlinewidth{0.602250pt}%
\definecolor{currentstroke}{rgb}{0.000000,0.000000,0.000000}%
\pgfsetstrokecolor{currentstroke}%
\pgfsetdash{}{0pt}%
\pgfsys@defobject{currentmarker}{\pgfqpoint{0.000000in}{-0.027778in}}{\pgfqpoint{0.000000in}{0.000000in}}{%
\pgfpathmoveto{\pgfqpoint{0.000000in}{0.000000in}}%
\pgfpathlineto{\pgfqpoint{0.000000in}{-0.027778in}}%
\pgfusepath{stroke,fill}%
}%
\begin{pgfscope}%
\pgfsys@transformshift{2.361277in}{0.417642in}%
\pgfsys@useobject{currentmarker}{}%
\end{pgfscope}%
\end{pgfscope}%
\begin{pgfscope}%
\pgfpathrectangle{\pgfqpoint{0.589510in}{0.417642in}}{\pgfqpoint{1.809765in}{1.371397in}}%
\pgfusepath{clip}%
\pgfsetrectcap%
\pgfsetroundjoin%
\pgfsetlinewidth{0.803000pt}%
\definecolor{currentstroke}{rgb}{0.850000,0.850000,0.850000}%
\pgfsetstrokecolor{currentstroke}%
\pgfsetdash{}{0pt}%
\pgfpathmoveto{\pgfqpoint{2.397443in}{0.417642in}}%
\pgfpathlineto{\pgfqpoint{2.397443in}{1.789039in}}%
\pgfusepath{stroke}%
\end{pgfscope}%
\begin{pgfscope}%
\pgfsetbuttcap%
\pgfsetroundjoin%
\definecolor{currentfill}{rgb}{0.000000,0.000000,0.000000}%
\pgfsetfillcolor{currentfill}%
\pgfsetlinewidth{0.602250pt}%
\definecolor{currentstroke}{rgb}{0.000000,0.000000,0.000000}%
\pgfsetstrokecolor{currentstroke}%
\pgfsetdash{}{0pt}%
\pgfsys@defobject{currentmarker}{\pgfqpoint{0.000000in}{-0.027778in}}{\pgfqpoint{0.000000in}{0.000000in}}{%
\pgfpathmoveto{\pgfqpoint{0.000000in}{0.000000in}}%
\pgfpathlineto{\pgfqpoint{0.000000in}{-0.027778in}}%
\pgfusepath{stroke,fill}%
}%
\begin{pgfscope}%
\pgfsys@transformshift{2.397443in}{0.417642in}%
\pgfsys@useobject{currentmarker}{}%
\end{pgfscope}%
\end{pgfscope}%
\begin{pgfscope}%
\definecolor{textcolor}{rgb}{0.000000,0.000000,0.000000}%
\pgfsetstrokecolor{textcolor}%
\pgfsetfillcolor{textcolor}%
\pgftext[x=1.494392in,y=0.165003in,,top]{\color{textcolor}\rmfamily\fontsize{10.000000}{12.000000}\selectfont \(\displaystyle \tau\) in \unit{\second}}%
\end{pgfscope}%
\begin{pgfscope}%
\pgfpathrectangle{\pgfqpoint{0.589510in}{0.417642in}}{\pgfqpoint{1.809765in}{1.371397in}}%
\pgfusepath{clip}%
\pgfsetrectcap%
\pgfsetroundjoin%
\pgfsetlinewidth{0.803000pt}%
\definecolor{currentstroke}{rgb}{0.450000,0.450000,0.450000}%
\pgfsetstrokecolor{currentstroke}%
\pgfsetdash{}{0pt}%
\pgfpathmoveto{\pgfqpoint{0.589510in}{0.417642in}}%
\pgfpathlineto{\pgfqpoint{2.399275in}{0.417642in}}%
\pgfusepath{stroke}%
\end{pgfscope}%
\begin{pgfscope}%
\pgfsetbuttcap%
\pgfsetroundjoin%
\definecolor{currentfill}{rgb}{0.000000,0.000000,0.000000}%
\pgfsetfillcolor{currentfill}%
\pgfsetlinewidth{0.803000pt}%
\definecolor{currentstroke}{rgb}{0.000000,0.000000,0.000000}%
\pgfsetstrokecolor{currentstroke}%
\pgfsetdash{}{0pt}%
\pgfsys@defobject{currentmarker}{\pgfqpoint{-0.048611in}{0.000000in}}{\pgfqpoint{-0.000000in}{0.000000in}}{%
\pgfpathmoveto{\pgfqpoint{-0.000000in}{0.000000in}}%
\pgfpathlineto{\pgfqpoint{-0.048611in}{0.000000in}}%
\pgfusepath{stroke,fill}%
}%
\begin{pgfscope}%
\pgfsys@transformshift{0.589510in}{0.417642in}%
\pgfsys@useobject{currentmarker}{}%
\end{pgfscope}%
\end{pgfscope}%
\begin{pgfscope}%
\definecolor{textcolor}{rgb}{0.000000,0.000000,0.000000}%
\pgfsetstrokecolor{textcolor}%
\pgfsetfillcolor{textcolor}%
\pgftext[x=0.236114in, y=0.378489in, left, base]{\color{textcolor}\rmfamily\fontsize{8.000000}{9.600000}\selectfont \(\displaystyle {10^{-2}}\)}%
\end{pgfscope}%
\begin{pgfscope}%
\pgfpathrectangle{\pgfqpoint{0.589510in}{0.417642in}}{\pgfqpoint{1.809765in}{1.371397in}}%
\pgfusepath{clip}%
\pgfsetrectcap%
\pgfsetroundjoin%
\pgfsetlinewidth{0.803000pt}%
\definecolor{currentstroke}{rgb}{0.450000,0.450000,0.450000}%
\pgfsetstrokecolor{currentstroke}%
\pgfsetdash{}{0pt}%
\pgfpathmoveto{\pgfqpoint{0.589510in}{0.827077in}}%
\pgfpathlineto{\pgfqpoint{2.399275in}{0.827077in}}%
\pgfusepath{stroke}%
\end{pgfscope}%
\begin{pgfscope}%
\pgfsetbuttcap%
\pgfsetroundjoin%
\definecolor{currentfill}{rgb}{0.000000,0.000000,0.000000}%
\pgfsetfillcolor{currentfill}%
\pgfsetlinewidth{0.803000pt}%
\definecolor{currentstroke}{rgb}{0.000000,0.000000,0.000000}%
\pgfsetstrokecolor{currentstroke}%
\pgfsetdash{}{0pt}%
\pgfsys@defobject{currentmarker}{\pgfqpoint{-0.048611in}{0.000000in}}{\pgfqpoint{-0.000000in}{0.000000in}}{%
\pgfpathmoveto{\pgfqpoint{-0.000000in}{0.000000in}}%
\pgfpathlineto{\pgfqpoint{-0.048611in}{0.000000in}}%
\pgfusepath{stroke,fill}%
}%
\begin{pgfscope}%
\pgfsys@transformshift{0.589510in}{0.827077in}%
\pgfsys@useobject{currentmarker}{}%
\end{pgfscope}%
\end{pgfscope}%
\begin{pgfscope}%
\definecolor{textcolor}{rgb}{0.000000,0.000000,0.000000}%
\pgfsetstrokecolor{textcolor}%
\pgfsetfillcolor{textcolor}%
\pgftext[x=0.316361in, y=0.787924in, left, base]{\color{textcolor}\rmfamily\fontsize{8.000000}{9.600000}\selectfont \(\displaystyle {10^{0}}\)}%
\end{pgfscope}%
\begin{pgfscope}%
\pgfpathrectangle{\pgfqpoint{0.589510in}{0.417642in}}{\pgfqpoint{1.809765in}{1.371397in}}%
\pgfusepath{clip}%
\pgfsetrectcap%
\pgfsetroundjoin%
\pgfsetlinewidth{0.803000pt}%
\definecolor{currentstroke}{rgb}{0.450000,0.450000,0.450000}%
\pgfsetstrokecolor{currentstroke}%
\pgfsetdash{}{0pt}%
\pgfpathmoveto{\pgfqpoint{0.589510in}{1.236512in}}%
\pgfpathlineto{\pgfqpoint{2.399275in}{1.236512in}}%
\pgfusepath{stroke}%
\end{pgfscope}%
\begin{pgfscope}%
\pgfsetbuttcap%
\pgfsetroundjoin%
\definecolor{currentfill}{rgb}{0.000000,0.000000,0.000000}%
\pgfsetfillcolor{currentfill}%
\pgfsetlinewidth{0.803000pt}%
\definecolor{currentstroke}{rgb}{0.000000,0.000000,0.000000}%
\pgfsetstrokecolor{currentstroke}%
\pgfsetdash{}{0pt}%
\pgfsys@defobject{currentmarker}{\pgfqpoint{-0.048611in}{0.000000in}}{\pgfqpoint{-0.000000in}{0.000000in}}{%
\pgfpathmoveto{\pgfqpoint{-0.000000in}{0.000000in}}%
\pgfpathlineto{\pgfqpoint{-0.048611in}{0.000000in}}%
\pgfusepath{stroke,fill}%
}%
\begin{pgfscope}%
\pgfsys@transformshift{0.589510in}{1.236512in}%
\pgfsys@useobject{currentmarker}{}%
\end{pgfscope}%
\end{pgfscope}%
\begin{pgfscope}%
\definecolor{textcolor}{rgb}{0.000000,0.000000,0.000000}%
\pgfsetstrokecolor{textcolor}%
\pgfsetfillcolor{textcolor}%
\pgftext[x=0.316361in, y=1.197359in, left, base]{\color{textcolor}\rmfamily\fontsize{8.000000}{9.600000}\selectfont \(\displaystyle {10^{2}}\)}%
\end{pgfscope}%
\begin{pgfscope}%
\pgfpathrectangle{\pgfqpoint{0.589510in}{0.417642in}}{\pgfqpoint{1.809765in}{1.371397in}}%
\pgfusepath{clip}%
\pgfsetrectcap%
\pgfsetroundjoin%
\pgfsetlinewidth{0.803000pt}%
\definecolor{currentstroke}{rgb}{0.450000,0.450000,0.450000}%
\pgfsetstrokecolor{currentstroke}%
\pgfsetdash{}{0pt}%
\pgfpathmoveto{\pgfqpoint{0.589510in}{1.645947in}}%
\pgfpathlineto{\pgfqpoint{2.399275in}{1.645947in}}%
\pgfusepath{stroke}%
\end{pgfscope}%
\begin{pgfscope}%
\pgfsetbuttcap%
\pgfsetroundjoin%
\definecolor{currentfill}{rgb}{0.000000,0.000000,0.000000}%
\pgfsetfillcolor{currentfill}%
\pgfsetlinewidth{0.803000pt}%
\definecolor{currentstroke}{rgb}{0.000000,0.000000,0.000000}%
\pgfsetstrokecolor{currentstroke}%
\pgfsetdash{}{0pt}%
\pgfsys@defobject{currentmarker}{\pgfqpoint{-0.048611in}{0.000000in}}{\pgfqpoint{-0.000000in}{0.000000in}}{%
\pgfpathmoveto{\pgfqpoint{-0.000000in}{0.000000in}}%
\pgfpathlineto{\pgfqpoint{-0.048611in}{0.000000in}}%
\pgfusepath{stroke,fill}%
}%
\begin{pgfscope}%
\pgfsys@transformshift{0.589510in}{1.645947in}%
\pgfsys@useobject{currentmarker}{}%
\end{pgfscope}%
\end{pgfscope}%
\begin{pgfscope}%
\definecolor{textcolor}{rgb}{0.000000,0.000000,0.000000}%
\pgfsetstrokecolor{textcolor}%
\pgfsetfillcolor{textcolor}%
\pgftext[x=0.316361in, y=1.606795in, left, base]{\color{textcolor}\rmfamily\fontsize{8.000000}{9.600000}\selectfont \(\displaystyle {10^{4}}\)}%
\end{pgfscope}%
\begin{pgfscope}%
\definecolor{textcolor}{rgb}{0.000000,0.000000,0.000000}%
\pgfsetstrokecolor{textcolor}%
\pgfsetfillcolor{textcolor}%
\pgftext[x=0.180559in,y=1.103340in,,bottom,rotate=90.000000]{\color{textcolor}\rmfamily\fontsize{10.000000}{12.000000}\selectfont ADEV \(\displaystyle \sigma_A(\tau)\)}%
\end{pgfscope}%
\begin{pgfscope}%
\pgfpathrectangle{\pgfqpoint{0.589510in}{0.417642in}}{\pgfqpoint{1.809765in}{1.371397in}}%
\pgfusepath{clip}%
\pgfsetbuttcap%
\pgfsetroundjoin%
\pgfsetlinewidth{1.505625pt}%
\definecolor{currentstroke}{rgb}{0.835294,0.368627,0.000000}%
\pgfsetstrokecolor{currentstroke}%
\pgfsetdash{{5.550000pt}{2.400000pt}}{0.000000pt}%
\pgfpathmoveto{\pgfqpoint{0.671772in}{0.827077in}}%
\pgfpathlineto{\pgfqpoint{0.809267in}{0.857890in}}%
\pgfpathlineto{\pgfqpoint{0.946763in}{0.888703in}}%
\pgfpathlineto{\pgfqpoint{1.128522in}{0.929436in}}%
\pgfpathlineto{\pgfqpoint{1.266017in}{0.960249in}}%
\pgfpathlineto{\pgfqpoint{1.403513in}{0.991062in}}%
\pgfpathlineto{\pgfqpoint{1.585272in}{1.031795in}}%
\pgfpathlineto{\pgfqpoint{1.722767in}{1.062608in}}%
\pgfpathlineto{\pgfqpoint{1.860263in}{1.093421in}}%
\pgfpathlineto{\pgfqpoint{2.042022in}{1.134153in}}%
\pgfpathlineto{\pgfqpoint{2.179517in}{1.164966in}}%
\pgfpathlineto{\pgfqpoint{2.317013in}{1.195780in}}%
\pgfusepath{stroke}%
\end{pgfscope}%
\begin{pgfscope}%
\pgfpathrectangle{\pgfqpoint{0.589510in}{0.417642in}}{\pgfqpoint{1.809765in}{1.371397in}}%
\pgfusepath{clip}%
\pgfsetbuttcap%
\pgfsetroundjoin%
\definecolor{currentfill}{rgb}{0.835294,0.368627,0.000000}%
\pgfsetfillcolor{currentfill}%
\pgfsetlinewidth{1.003750pt}%
\definecolor{currentstroke}{rgb}{0.835294,0.368627,0.000000}%
\pgfsetstrokecolor{currentstroke}%
\pgfsetdash{}{0pt}%
\pgfsys@defobject{currentmarker}{\pgfqpoint{-0.020833in}{-0.020833in}}{\pgfqpoint{0.020833in}{0.020833in}}{%
\pgfpathmoveto{\pgfqpoint{0.000000in}{-0.020833in}}%
\pgfpathcurveto{\pgfqpoint{0.005525in}{-0.020833in}}{\pgfqpoint{0.010825in}{-0.018638in}}{\pgfqpoint{0.014731in}{-0.014731in}}%
\pgfpathcurveto{\pgfqpoint{0.018638in}{-0.010825in}}{\pgfqpoint{0.020833in}{-0.005525in}}{\pgfqpoint{0.020833in}{0.000000in}}%
\pgfpathcurveto{\pgfqpoint{0.020833in}{0.005525in}}{\pgfqpoint{0.018638in}{0.010825in}}{\pgfqpoint{0.014731in}{0.014731in}}%
\pgfpathcurveto{\pgfqpoint{0.010825in}{0.018638in}}{\pgfqpoint{0.005525in}{0.020833in}}{\pgfqpoint{0.000000in}{0.020833in}}%
\pgfpathcurveto{\pgfqpoint{-0.005525in}{0.020833in}}{\pgfqpoint{-0.010825in}{0.018638in}}{\pgfqpoint{-0.014731in}{0.014731in}}%
\pgfpathcurveto{\pgfqpoint{-0.018638in}{0.010825in}}{\pgfqpoint{-0.020833in}{0.005525in}}{\pgfqpoint{-0.020833in}{0.000000in}}%
\pgfpathcurveto{\pgfqpoint{-0.020833in}{-0.005525in}}{\pgfqpoint{-0.018638in}{-0.010825in}}{\pgfqpoint{-0.014731in}{-0.014731in}}%
\pgfpathcurveto{\pgfqpoint{-0.010825in}{-0.018638in}}{\pgfqpoint{-0.005525in}{-0.020833in}}{\pgfqpoint{0.000000in}{-0.020833in}}%
\pgfpathlineto{\pgfqpoint{0.000000in}{-0.020833in}}%
\pgfpathclose%
\pgfusepath{stroke,fill}%
}%
\begin{pgfscope}%
\pgfsys@transformshift{0.671772in}{0.845380in}%
\pgfsys@useobject{currentmarker}{}%
\end{pgfscope}%
\begin{pgfscope}%
\pgfsys@transformshift{0.809267in}{0.862796in}%
\pgfsys@useobject{currentmarker}{}%
\end{pgfscope}%
\begin{pgfscope}%
\pgfsys@transformshift{0.946763in}{0.888667in}%
\pgfsys@useobject{currentmarker}{}%
\end{pgfscope}%
\begin{pgfscope}%
\pgfsys@transformshift{1.128522in}{0.926846in}%
\pgfsys@useobject{currentmarker}{}%
\end{pgfscope}%
\begin{pgfscope}%
\pgfsys@transformshift{1.266017in}{0.958050in}%
\pgfsys@useobject{currentmarker}{}%
\end{pgfscope}%
\begin{pgfscope}%
\pgfsys@transformshift{1.403513in}{0.990530in}%
\pgfsys@useobject{currentmarker}{}%
\end{pgfscope}%
\begin{pgfscope}%
\pgfsys@transformshift{1.585272in}{1.029061in}%
\pgfsys@useobject{currentmarker}{}%
\end{pgfscope}%
\begin{pgfscope}%
\pgfsys@transformshift{1.722767in}{1.056186in}%
\pgfsys@useobject{currentmarker}{}%
\end{pgfscope}%
\begin{pgfscope}%
\pgfsys@transformshift{1.860263in}{1.087384in}%
\pgfsys@useobject{currentmarker}{}%
\end{pgfscope}%
\begin{pgfscope}%
\pgfsys@transformshift{2.042022in}{1.143452in}%
\pgfsys@useobject{currentmarker}{}%
\end{pgfscope}%
\begin{pgfscope}%
\pgfsys@transformshift{2.179517in}{1.178795in}%
\pgfsys@useobject{currentmarker}{}%
\end{pgfscope}%
\begin{pgfscope}%
\pgfsys@transformshift{2.317013in}{1.197055in}%
\pgfsys@useobject{currentmarker}{}%
\end{pgfscope}%
\end{pgfscope}%
\begin{pgfscope}%
\pgfsetrectcap%
\pgfsetmiterjoin%
\pgfsetlinewidth{0.803000pt}%
\definecolor{currentstroke}{rgb}{0.000000,0.000000,0.000000}%
\pgfsetstrokecolor{currentstroke}%
\pgfsetdash{}{0pt}%
\pgfpathmoveto{\pgfqpoint{0.589510in}{0.417642in}}%
\pgfpathlineto{\pgfqpoint{0.589510in}{1.789039in}}%
\pgfusepath{stroke}%
\end{pgfscope}%
\begin{pgfscope}%
\pgfsetrectcap%
\pgfsetmiterjoin%
\pgfsetlinewidth{0.803000pt}%
\definecolor{currentstroke}{rgb}{0.000000,0.000000,0.000000}%
\pgfsetstrokecolor{currentstroke}%
\pgfsetdash{}{0pt}%
\pgfpathmoveto{\pgfqpoint{2.399275in}{0.417642in}}%
\pgfpathlineto{\pgfqpoint{2.399275in}{1.789039in}}%
\pgfusepath{stroke}%
\end{pgfscope}%
\begin{pgfscope}%
\pgfsetrectcap%
\pgfsetmiterjoin%
\pgfsetlinewidth{0.803000pt}%
\definecolor{currentstroke}{rgb}{0.000000,0.000000,0.000000}%
\pgfsetstrokecolor{currentstroke}%
\pgfsetdash{}{0pt}%
\pgfpathmoveto{\pgfqpoint{0.589510in}{0.417642in}}%
\pgfpathlineto{\pgfqpoint{2.399275in}{0.417642in}}%
\pgfusepath{stroke}%
\end{pgfscope}%
\begin{pgfscope}%
\pgfsetrectcap%
\pgfsetmiterjoin%
\pgfsetlinewidth{0.803000pt}%
\definecolor{currentstroke}{rgb}{0.000000,0.000000,0.000000}%
\pgfsetstrokecolor{currentstroke}%
\pgfsetdash{}{0pt}%
\pgfpathmoveto{\pgfqpoint{0.589510in}{1.789039in}}%
\pgfpathlineto{\pgfqpoint{2.399275in}{1.789039in}}%
\pgfusepath{stroke}%
\end{pgfscope}%
\begin{pgfscope}%
\pgfsetbuttcap%
\pgfsetmiterjoin%
\definecolor{currentfill}{rgb}{1.000000,1.000000,1.000000}%
\pgfsetfillcolor{currentfill}%
\pgfsetfillopacity{0.800000}%
\pgfsetlinewidth{1.003750pt}%
\definecolor{currentstroke}{rgb}{0.800000,0.800000,0.800000}%
\pgfsetstrokecolor{currentstroke}%
\pgfsetstrokeopacity{0.800000}%
\pgfsetdash{}{0pt}%
\pgfpathmoveto{\pgfqpoint{1.212708in}{1.472371in}}%
\pgfpathlineto{\pgfqpoint{2.321497in}{1.472371in}}%
\pgfpathquadraticcurveto{\pgfqpoint{2.343719in}{1.472371in}}{\pgfqpoint{2.343719in}{1.494593in}}%
\pgfpathlineto{\pgfqpoint{2.343719in}{1.711261in}}%
\pgfpathquadraticcurveto{\pgfqpoint{2.343719in}{1.733483in}}{\pgfqpoint{2.321497in}{1.733483in}}%
\pgfpathlineto{\pgfqpoint{1.212708in}{1.733483in}}%
\pgfpathquadraticcurveto{\pgfqpoint{1.190486in}{1.733483in}}{\pgfqpoint{1.190486in}{1.711261in}}%
\pgfpathlineto{\pgfqpoint{1.190486in}{1.494593in}}%
\pgfpathquadraticcurveto{\pgfqpoint{1.190486in}{1.472371in}}{\pgfqpoint{1.212708in}{1.472371in}}%
\pgfpathlineto{\pgfqpoint{1.212708in}{1.472371in}}%
\pgfpathclose%
\pgfusepath{stroke,fill}%
\end{pgfscope}%
\begin{pgfscope}%
\pgfsetbuttcap%
\pgfsetroundjoin%
\pgfsetlinewidth{1.505625pt}%
\definecolor{currentstroke}{rgb}{0.835294,0.368627,0.000000}%
\pgfsetstrokecolor{currentstroke}%
\pgfsetdash{{5.550000pt}{2.400000pt}}{0.000000pt}%
\pgfpathmoveto{\pgfqpoint{1.234930in}{1.602426in}}%
\pgfpathlineto{\pgfqpoint{1.346041in}{1.602426in}}%
\pgfpathlineto{\pgfqpoint{1.457152in}{1.602426in}}%
\pgfusepath{stroke}%
\end{pgfscope}%
\begin{pgfscope}%
\definecolor{textcolor}{rgb}{0.000000,0.000000,0.000000}%
\pgfsetstrokecolor{textcolor}%
\pgfsetfillcolor{textcolor}%
\pgftext[x=1.546041in,y=1.563537in,left,base]{\color{textcolor}\rmfamily\fontsize{8.000000}{9.600000}\selectfont \(\displaystyle \propto\sqrt{h_{-2}}\tau^{+0.5}\)}%
\end{pgfscope}%
\end{pgfpicture}%
\makeatother%
\endgroup%
% data/simulations/sim_allan_variance.py
        } % scalebox
        \caption{Allan deviation}
        \label{fig:random_walk_adev}
    \end{subfigure}
    \caption{Different representations of random walk noise.}
    \label{fig:random_walk_noise_simulated}
\end{figure}


\clearpage
\subsubsection{Drift}
Finally, the last feature of the Allan deviation plot that needs to be discussed is drift. Drift happens at very long time scales and describes a linear dependence of the measurand on time. This is also part of the ageing effect. \citeauthor{adev_drift} discussed the effect of drift \cite{adev_drift} on the Allan variance and found the following relationship:
\begin{align}
    \sigma_A^2(\tau) = \frac{D^2}{2} \tau^2
\end{align}
with slope of the drift $D$.
\begin{figure}[ht]
    \centering
    \begin{subfigure}{0.32\linewidth}
        \centering
        \scalebox{0.75}{%
            %% Creator: Matplotlib, PGF backend
%%
%% To include the figure in your LaTeX document, write
%%   \input{<filename>.pgf}
%%
%% Make sure the required packages are loaded in your preamble
%%   \usepackage{pgf}
%%
%% Also ensure that all the required font packages are loaded; for instance,
%% the lmodern package is sometimes necessary when using math font.
%%   \usepackage{lmodern}
%%
%% Figures using additional raster images can only be included by \input if
%% they are in the same directory as the main LaTeX file. For loading figures
%% from other directories you can use the `import` package
%%   \usepackage{import}
%%
%% and then include the figures with
%%   \import{<path to file>}{<filename>.pgf}
%%
%% Matplotlib used the following preamble
%%   \usepackage{siunitx}
%%   \usepackage{fontspec}
%%   \makeatletter\@ifpackageloaded{underscore}{}{\usepackage[strings]{underscore}}\makeatother
%%
\begingroup%
\makeatletter%
\begin{pgfpicture}%
\pgfpathrectangle{\pgfpointorigin}{\pgfqpoint{2.440000in}{1.830000in}}%
\pgfusepath{use as bounding box, clip}%
\begin{pgfscope}%
\pgfsetbuttcap%
\pgfsetmiterjoin%
\definecolor{currentfill}{rgb}{1.000000,1.000000,1.000000}%
\pgfsetfillcolor{currentfill}%
\pgfsetlinewidth{0.000000pt}%
\definecolor{currentstroke}{rgb}{1.000000,1.000000,1.000000}%
\pgfsetstrokecolor{currentstroke}%
\pgfsetdash{}{0pt}%
\pgfpathmoveto{\pgfqpoint{0.000000in}{0.000000in}}%
\pgfpathlineto{\pgfqpoint{2.440000in}{0.000000in}}%
\pgfpathlineto{\pgfqpoint{2.440000in}{1.830000in}}%
\pgfpathlineto{\pgfqpoint{0.000000in}{1.830000in}}%
\pgfpathlineto{\pgfqpoint{0.000000in}{0.000000in}}%
\pgfpathclose%
\pgfusepath{fill}%
\end{pgfscope}%
\begin{pgfscope}%
\pgfsetbuttcap%
\pgfsetmiterjoin%
\definecolor{currentfill}{rgb}{1.000000,1.000000,1.000000}%
\pgfsetfillcolor{currentfill}%
\pgfsetlinewidth{0.000000pt}%
\definecolor{currentstroke}{rgb}{0.000000,0.000000,0.000000}%
\pgfsetstrokecolor{currentstroke}%
\pgfsetstrokeopacity{0.000000}%
\pgfsetdash{}{0pt}%
\pgfpathmoveto{\pgfqpoint{0.615980in}{0.416447in}}%
\pgfpathlineto{\pgfqpoint{2.398330in}{0.416447in}}%
\pgfpathlineto{\pgfqpoint{2.398330in}{1.788330in}}%
\pgfpathlineto{\pgfqpoint{0.615980in}{1.788330in}}%
\pgfpathlineto{\pgfqpoint{0.615980in}{0.416447in}}%
\pgfpathclose%
\pgfusepath{fill}%
\end{pgfscope}%
\begin{pgfscope}%
\pgfpathrectangle{\pgfqpoint{0.615980in}{0.416447in}}{\pgfqpoint{1.782350in}{1.371883in}}%
\pgfusepath{clip}%
\pgfsetrectcap%
\pgfsetroundjoin%
\pgfsetlinewidth{0.803000pt}%
\definecolor{currentstroke}{rgb}{0.450000,0.450000,0.450000}%
\pgfsetstrokecolor{currentstroke}%
\pgfsetdash{}{0pt}%
\pgfpathmoveto{\pgfqpoint{0.696996in}{0.416447in}}%
\pgfpathlineto{\pgfqpoint{0.696996in}{1.788330in}}%
\pgfusepath{stroke}%
\end{pgfscope}%
\begin{pgfscope}%
\pgfsetbuttcap%
\pgfsetroundjoin%
\definecolor{currentfill}{rgb}{0.000000,0.000000,0.000000}%
\pgfsetfillcolor{currentfill}%
\pgfsetlinewidth{0.803000pt}%
\definecolor{currentstroke}{rgb}{0.000000,0.000000,0.000000}%
\pgfsetstrokecolor{currentstroke}%
\pgfsetdash{}{0pt}%
\pgfsys@defobject{currentmarker}{\pgfqpoint{0.000000in}{-0.048611in}}{\pgfqpoint{0.000000in}{0.000000in}}{%
\pgfpathmoveto{\pgfqpoint{0.000000in}{0.000000in}}%
\pgfpathlineto{\pgfqpoint{0.000000in}{-0.048611in}}%
\pgfusepath{stroke,fill}%
}%
\begin{pgfscope}%
\pgfsys@transformshift{0.696996in}{0.416447in}%
\pgfsys@useobject{currentmarker}{}%
\end{pgfscope}%
\end{pgfscope}%
\begin{pgfscope}%
\definecolor{textcolor}{rgb}{0.000000,0.000000,0.000000}%
\pgfsetstrokecolor{textcolor}%
\pgfsetfillcolor{textcolor}%
\pgftext[x=0.696996in,y=0.319225in,,top]{\color{textcolor}\rmfamily\fontsize{8.000000}{9.600000}\selectfont \(\displaystyle {0}\)}%
\end{pgfscope}%
\begin{pgfscope}%
\pgfpathrectangle{\pgfqpoint{0.615980in}{0.416447in}}{\pgfqpoint{1.782350in}{1.371883in}}%
\pgfusepath{clip}%
\pgfsetrectcap%
\pgfsetroundjoin%
\pgfsetlinewidth{0.803000pt}%
\definecolor{currentstroke}{rgb}{0.450000,0.450000,0.450000}%
\pgfsetstrokecolor{currentstroke}%
\pgfsetdash{}{0pt}%
\pgfpathmoveto{\pgfqpoint{1.191538in}{0.416447in}}%
\pgfpathlineto{\pgfqpoint{1.191538in}{1.788330in}}%
\pgfusepath{stroke}%
\end{pgfscope}%
\begin{pgfscope}%
\pgfsetbuttcap%
\pgfsetroundjoin%
\definecolor{currentfill}{rgb}{0.000000,0.000000,0.000000}%
\pgfsetfillcolor{currentfill}%
\pgfsetlinewidth{0.803000pt}%
\definecolor{currentstroke}{rgb}{0.000000,0.000000,0.000000}%
\pgfsetstrokecolor{currentstroke}%
\pgfsetdash{}{0pt}%
\pgfsys@defobject{currentmarker}{\pgfqpoint{0.000000in}{-0.048611in}}{\pgfqpoint{0.000000in}{0.000000in}}{%
\pgfpathmoveto{\pgfqpoint{0.000000in}{0.000000in}}%
\pgfpathlineto{\pgfqpoint{0.000000in}{-0.048611in}}%
\pgfusepath{stroke,fill}%
}%
\begin{pgfscope}%
\pgfsys@transformshift{1.191538in}{0.416447in}%
\pgfsys@useobject{currentmarker}{}%
\end{pgfscope}%
\end{pgfscope}%
\begin{pgfscope}%
\definecolor{textcolor}{rgb}{0.000000,0.000000,0.000000}%
\pgfsetstrokecolor{textcolor}%
\pgfsetfillcolor{textcolor}%
\pgftext[x=1.191538in,y=0.319225in,,top]{\color{textcolor}\rmfamily\fontsize{8.000000}{9.600000}\selectfont \(\displaystyle {5000}\)}%
\end{pgfscope}%
\begin{pgfscope}%
\pgfpathrectangle{\pgfqpoint{0.615980in}{0.416447in}}{\pgfqpoint{1.782350in}{1.371883in}}%
\pgfusepath{clip}%
\pgfsetrectcap%
\pgfsetroundjoin%
\pgfsetlinewidth{0.803000pt}%
\definecolor{currentstroke}{rgb}{0.450000,0.450000,0.450000}%
\pgfsetstrokecolor{currentstroke}%
\pgfsetdash{}{0pt}%
\pgfpathmoveto{\pgfqpoint{1.686080in}{0.416447in}}%
\pgfpathlineto{\pgfqpoint{1.686080in}{1.788330in}}%
\pgfusepath{stroke}%
\end{pgfscope}%
\begin{pgfscope}%
\pgfsetbuttcap%
\pgfsetroundjoin%
\definecolor{currentfill}{rgb}{0.000000,0.000000,0.000000}%
\pgfsetfillcolor{currentfill}%
\pgfsetlinewidth{0.803000pt}%
\definecolor{currentstroke}{rgb}{0.000000,0.000000,0.000000}%
\pgfsetstrokecolor{currentstroke}%
\pgfsetdash{}{0pt}%
\pgfsys@defobject{currentmarker}{\pgfqpoint{0.000000in}{-0.048611in}}{\pgfqpoint{0.000000in}{0.000000in}}{%
\pgfpathmoveto{\pgfqpoint{0.000000in}{0.000000in}}%
\pgfpathlineto{\pgfqpoint{0.000000in}{-0.048611in}}%
\pgfusepath{stroke,fill}%
}%
\begin{pgfscope}%
\pgfsys@transformshift{1.686080in}{0.416447in}%
\pgfsys@useobject{currentmarker}{}%
\end{pgfscope}%
\end{pgfscope}%
\begin{pgfscope}%
\definecolor{textcolor}{rgb}{0.000000,0.000000,0.000000}%
\pgfsetstrokecolor{textcolor}%
\pgfsetfillcolor{textcolor}%
\pgftext[x=1.686080in,y=0.319225in,,top]{\color{textcolor}\rmfamily\fontsize{8.000000}{9.600000}\selectfont \(\displaystyle {10000}\)}%
\end{pgfscope}%
\begin{pgfscope}%
\pgfpathrectangle{\pgfqpoint{0.615980in}{0.416447in}}{\pgfqpoint{1.782350in}{1.371883in}}%
\pgfusepath{clip}%
\pgfsetrectcap%
\pgfsetroundjoin%
\pgfsetlinewidth{0.803000pt}%
\definecolor{currentstroke}{rgb}{0.450000,0.450000,0.450000}%
\pgfsetstrokecolor{currentstroke}%
\pgfsetdash{}{0pt}%
\pgfpathmoveto{\pgfqpoint{2.180623in}{0.416447in}}%
\pgfpathlineto{\pgfqpoint{2.180623in}{1.788330in}}%
\pgfusepath{stroke}%
\end{pgfscope}%
\begin{pgfscope}%
\pgfsetbuttcap%
\pgfsetroundjoin%
\definecolor{currentfill}{rgb}{0.000000,0.000000,0.000000}%
\pgfsetfillcolor{currentfill}%
\pgfsetlinewidth{0.803000pt}%
\definecolor{currentstroke}{rgb}{0.000000,0.000000,0.000000}%
\pgfsetstrokecolor{currentstroke}%
\pgfsetdash{}{0pt}%
\pgfsys@defobject{currentmarker}{\pgfqpoint{0.000000in}{-0.048611in}}{\pgfqpoint{0.000000in}{0.000000in}}{%
\pgfpathmoveto{\pgfqpoint{0.000000in}{0.000000in}}%
\pgfpathlineto{\pgfqpoint{0.000000in}{-0.048611in}}%
\pgfusepath{stroke,fill}%
}%
\begin{pgfscope}%
\pgfsys@transformshift{2.180623in}{0.416447in}%
\pgfsys@useobject{currentmarker}{}%
\end{pgfscope}%
\end{pgfscope}%
\begin{pgfscope}%
\definecolor{textcolor}{rgb}{0.000000,0.000000,0.000000}%
\pgfsetstrokecolor{textcolor}%
\pgfsetfillcolor{textcolor}%
\pgftext[x=2.180623in,y=0.319225in,,top]{\color{textcolor}\rmfamily\fontsize{8.000000}{9.600000}\selectfont \(\displaystyle {15000}\)}%
\end{pgfscope}%
\begin{pgfscope}%
\definecolor{textcolor}{rgb}{0.000000,0.000000,0.000000}%
\pgfsetstrokecolor{textcolor}%
\pgfsetfillcolor{textcolor}%
\pgftext[x=1.507155in,y=0.165003in,,top]{\color{textcolor}\rmfamily\fontsize{10.000000}{12.000000}\selectfont Time in \(\displaystyle \unit{\second}\)}%
\end{pgfscope}%
\begin{pgfscope}%
\pgfpathrectangle{\pgfqpoint{0.615980in}{0.416447in}}{\pgfqpoint{1.782350in}{1.371883in}}%
\pgfusepath{clip}%
\pgfsetrectcap%
\pgfsetroundjoin%
\pgfsetlinewidth{0.803000pt}%
\definecolor{currentstroke}{rgb}{0.450000,0.450000,0.450000}%
\pgfsetstrokecolor{currentstroke}%
\pgfsetdash{}{0pt}%
\pgfpathmoveto{\pgfqpoint{0.615980in}{0.478806in}}%
\pgfpathlineto{\pgfqpoint{2.398330in}{0.478806in}}%
\pgfusepath{stroke}%
\end{pgfscope}%
\begin{pgfscope}%
\pgfsetbuttcap%
\pgfsetroundjoin%
\definecolor{currentfill}{rgb}{0.000000,0.000000,0.000000}%
\pgfsetfillcolor{currentfill}%
\pgfsetlinewidth{0.803000pt}%
\definecolor{currentstroke}{rgb}{0.000000,0.000000,0.000000}%
\pgfsetstrokecolor{currentstroke}%
\pgfsetdash{}{0pt}%
\pgfsys@defobject{currentmarker}{\pgfqpoint{-0.048611in}{0.000000in}}{\pgfqpoint{-0.000000in}{0.000000in}}{%
\pgfpathmoveto{\pgfqpoint{-0.000000in}{0.000000in}}%
\pgfpathlineto{\pgfqpoint{-0.048611in}{0.000000in}}%
\pgfusepath{stroke,fill}%
}%
\begin{pgfscope}%
\pgfsys@transformshift{0.615980in}{0.478806in}%
\pgfsys@useobject{currentmarker}{}%
\end{pgfscope}%
\end{pgfscope}%
\begin{pgfscope}%
\definecolor{textcolor}{rgb}{0.000000,0.000000,0.000000}%
\pgfsetstrokecolor{textcolor}%
\pgfsetfillcolor{textcolor}%
\pgftext[x=0.459729in, y=0.440250in, left, base]{\color{textcolor}\rmfamily\fontsize{8.000000}{9.600000}\selectfont \(\displaystyle {0}\)}%
\end{pgfscope}%
\begin{pgfscope}%
\pgfpathrectangle{\pgfqpoint{0.615980in}{0.416447in}}{\pgfqpoint{1.782350in}{1.371883in}}%
\pgfusepath{clip}%
\pgfsetrectcap%
\pgfsetroundjoin%
\pgfsetlinewidth{0.803000pt}%
\definecolor{currentstroke}{rgb}{0.450000,0.450000,0.450000}%
\pgfsetstrokecolor{currentstroke}%
\pgfsetdash{}{0pt}%
\pgfpathmoveto{\pgfqpoint{0.615980in}{0.747967in}}%
\pgfpathlineto{\pgfqpoint{2.398330in}{0.747967in}}%
\pgfusepath{stroke}%
\end{pgfscope}%
\begin{pgfscope}%
\pgfsetbuttcap%
\pgfsetroundjoin%
\definecolor{currentfill}{rgb}{0.000000,0.000000,0.000000}%
\pgfsetfillcolor{currentfill}%
\pgfsetlinewidth{0.803000pt}%
\definecolor{currentstroke}{rgb}{0.000000,0.000000,0.000000}%
\pgfsetstrokecolor{currentstroke}%
\pgfsetdash{}{0pt}%
\pgfsys@defobject{currentmarker}{\pgfqpoint{-0.048611in}{0.000000in}}{\pgfqpoint{-0.000000in}{0.000000in}}{%
\pgfpathmoveto{\pgfqpoint{-0.000000in}{0.000000in}}%
\pgfpathlineto{\pgfqpoint{-0.048611in}{0.000000in}}%
\pgfusepath{stroke,fill}%
}%
\begin{pgfscope}%
\pgfsys@transformshift{0.615980in}{0.747967in}%
\pgfsys@useobject{currentmarker}{}%
\end{pgfscope}%
\end{pgfscope}%
\begin{pgfscope}%
\definecolor{textcolor}{rgb}{0.000000,0.000000,0.000000}%
\pgfsetstrokecolor{textcolor}%
\pgfsetfillcolor{textcolor}%
\pgftext[x=0.282643in, y=0.709411in, left, base]{\color{textcolor}\rmfamily\fontsize{8.000000}{9.600000}\selectfont \(\displaystyle {5000}\)}%
\end{pgfscope}%
\begin{pgfscope}%
\pgfpathrectangle{\pgfqpoint{0.615980in}{0.416447in}}{\pgfqpoint{1.782350in}{1.371883in}}%
\pgfusepath{clip}%
\pgfsetrectcap%
\pgfsetroundjoin%
\pgfsetlinewidth{0.803000pt}%
\definecolor{currentstroke}{rgb}{0.450000,0.450000,0.450000}%
\pgfsetstrokecolor{currentstroke}%
\pgfsetdash{}{0pt}%
\pgfpathmoveto{\pgfqpoint{0.615980in}{1.017128in}}%
\pgfpathlineto{\pgfqpoint{2.398330in}{1.017128in}}%
\pgfusepath{stroke}%
\end{pgfscope}%
\begin{pgfscope}%
\pgfsetbuttcap%
\pgfsetroundjoin%
\definecolor{currentfill}{rgb}{0.000000,0.000000,0.000000}%
\pgfsetfillcolor{currentfill}%
\pgfsetlinewidth{0.803000pt}%
\definecolor{currentstroke}{rgb}{0.000000,0.000000,0.000000}%
\pgfsetstrokecolor{currentstroke}%
\pgfsetdash{}{0pt}%
\pgfsys@defobject{currentmarker}{\pgfqpoint{-0.048611in}{0.000000in}}{\pgfqpoint{-0.000000in}{0.000000in}}{%
\pgfpathmoveto{\pgfqpoint{-0.000000in}{0.000000in}}%
\pgfpathlineto{\pgfqpoint{-0.048611in}{0.000000in}}%
\pgfusepath{stroke,fill}%
}%
\begin{pgfscope}%
\pgfsys@transformshift{0.615980in}{1.017128in}%
\pgfsys@useobject{currentmarker}{}%
\end{pgfscope}%
\end{pgfscope}%
\begin{pgfscope}%
\definecolor{textcolor}{rgb}{0.000000,0.000000,0.000000}%
\pgfsetstrokecolor{textcolor}%
\pgfsetfillcolor{textcolor}%
\pgftext[x=0.223614in, y=0.978572in, left, base]{\color{textcolor}\rmfamily\fontsize{8.000000}{9.600000}\selectfont \(\displaystyle {10000}\)}%
\end{pgfscope}%
\begin{pgfscope}%
\pgfpathrectangle{\pgfqpoint{0.615980in}{0.416447in}}{\pgfqpoint{1.782350in}{1.371883in}}%
\pgfusepath{clip}%
\pgfsetrectcap%
\pgfsetroundjoin%
\pgfsetlinewidth{0.803000pt}%
\definecolor{currentstroke}{rgb}{0.450000,0.450000,0.450000}%
\pgfsetstrokecolor{currentstroke}%
\pgfsetdash{}{0pt}%
\pgfpathmoveto{\pgfqpoint{0.615980in}{1.286289in}}%
\pgfpathlineto{\pgfqpoint{2.398330in}{1.286289in}}%
\pgfusepath{stroke}%
\end{pgfscope}%
\begin{pgfscope}%
\pgfsetbuttcap%
\pgfsetroundjoin%
\definecolor{currentfill}{rgb}{0.000000,0.000000,0.000000}%
\pgfsetfillcolor{currentfill}%
\pgfsetlinewidth{0.803000pt}%
\definecolor{currentstroke}{rgb}{0.000000,0.000000,0.000000}%
\pgfsetstrokecolor{currentstroke}%
\pgfsetdash{}{0pt}%
\pgfsys@defobject{currentmarker}{\pgfqpoint{-0.048611in}{0.000000in}}{\pgfqpoint{-0.000000in}{0.000000in}}{%
\pgfpathmoveto{\pgfqpoint{-0.000000in}{0.000000in}}%
\pgfpathlineto{\pgfqpoint{-0.048611in}{0.000000in}}%
\pgfusepath{stroke,fill}%
}%
\begin{pgfscope}%
\pgfsys@transformshift{0.615980in}{1.286289in}%
\pgfsys@useobject{currentmarker}{}%
\end{pgfscope}%
\end{pgfscope}%
\begin{pgfscope}%
\definecolor{textcolor}{rgb}{0.000000,0.000000,0.000000}%
\pgfsetstrokecolor{textcolor}%
\pgfsetfillcolor{textcolor}%
\pgftext[x=0.223614in, y=1.247734in, left, base]{\color{textcolor}\rmfamily\fontsize{8.000000}{9.600000}\selectfont \(\displaystyle {15000}\)}%
\end{pgfscope}%
\begin{pgfscope}%
\pgfpathrectangle{\pgfqpoint{0.615980in}{0.416447in}}{\pgfqpoint{1.782350in}{1.371883in}}%
\pgfusepath{clip}%
\pgfsetrectcap%
\pgfsetroundjoin%
\pgfsetlinewidth{0.803000pt}%
\definecolor{currentstroke}{rgb}{0.450000,0.450000,0.450000}%
\pgfsetstrokecolor{currentstroke}%
\pgfsetdash{}{0pt}%
\pgfpathmoveto{\pgfqpoint{0.615980in}{1.555450in}}%
\pgfpathlineto{\pgfqpoint{2.398330in}{1.555450in}}%
\pgfusepath{stroke}%
\end{pgfscope}%
\begin{pgfscope}%
\pgfsetbuttcap%
\pgfsetroundjoin%
\definecolor{currentfill}{rgb}{0.000000,0.000000,0.000000}%
\pgfsetfillcolor{currentfill}%
\pgfsetlinewidth{0.803000pt}%
\definecolor{currentstroke}{rgb}{0.000000,0.000000,0.000000}%
\pgfsetstrokecolor{currentstroke}%
\pgfsetdash{}{0pt}%
\pgfsys@defobject{currentmarker}{\pgfqpoint{-0.048611in}{0.000000in}}{\pgfqpoint{-0.000000in}{0.000000in}}{%
\pgfpathmoveto{\pgfqpoint{-0.000000in}{0.000000in}}%
\pgfpathlineto{\pgfqpoint{-0.048611in}{0.000000in}}%
\pgfusepath{stroke,fill}%
}%
\begin{pgfscope}%
\pgfsys@transformshift{0.615980in}{1.555450in}%
\pgfsys@useobject{currentmarker}{}%
\end{pgfscope}%
\end{pgfscope}%
\begin{pgfscope}%
\definecolor{textcolor}{rgb}{0.000000,0.000000,0.000000}%
\pgfsetstrokecolor{textcolor}%
\pgfsetfillcolor{textcolor}%
\pgftext[x=0.223614in, y=1.516895in, left, base]{\color{textcolor}\rmfamily\fontsize{8.000000}{9.600000}\selectfont \(\displaystyle {20000}\)}%
\end{pgfscope}%
\begin{pgfscope}%
\definecolor{textcolor}{rgb}{0.000000,0.000000,0.000000}%
\pgfsetstrokecolor{textcolor}%
\pgfsetfillcolor{textcolor}%
\pgftext[x=0.168059in,y=1.102389in,,bottom,rotate=90.000000]{\color{textcolor}\rmfamily\fontsize{10.000000}{12.000000}\selectfont Ampl. in arb. unit}%
\end{pgfscope}%
\begin{pgfscope}%
\pgfpathrectangle{\pgfqpoint{0.615980in}{0.416447in}}{\pgfqpoint{1.782350in}{1.371883in}}%
\pgfusepath{clip}%
\pgfsetrectcap%
\pgfsetroundjoin%
\pgfsetlinewidth{1.505625pt}%
\definecolor{currentstroke}{rgb}{0.800000,0.474510,0.654902}%
\pgfsetstrokecolor{currentstroke}%
\pgfsetdash{}{0pt}%
\pgfpathmoveto{\pgfqpoint{0.696996in}{0.478806in}}%
\pgfpathlineto{\pgfqpoint{2.317314in}{1.725972in}}%
\pgfpathlineto{\pgfqpoint{2.317314in}{1.725972in}}%
\pgfusepath{stroke}%
\end{pgfscope}%
\begin{pgfscope}%
\pgfsetrectcap%
\pgfsetmiterjoin%
\pgfsetlinewidth{0.803000pt}%
\definecolor{currentstroke}{rgb}{0.000000,0.000000,0.000000}%
\pgfsetstrokecolor{currentstroke}%
\pgfsetdash{}{0pt}%
\pgfpathmoveto{\pgfqpoint{0.615980in}{0.416447in}}%
\pgfpathlineto{\pgfqpoint{0.615980in}{1.788330in}}%
\pgfusepath{stroke}%
\end{pgfscope}%
\begin{pgfscope}%
\pgfsetrectcap%
\pgfsetmiterjoin%
\pgfsetlinewidth{0.803000pt}%
\definecolor{currentstroke}{rgb}{0.000000,0.000000,0.000000}%
\pgfsetstrokecolor{currentstroke}%
\pgfsetdash{}{0pt}%
\pgfpathmoveto{\pgfqpoint{2.398330in}{0.416447in}}%
\pgfpathlineto{\pgfqpoint{2.398330in}{1.788330in}}%
\pgfusepath{stroke}%
\end{pgfscope}%
\begin{pgfscope}%
\pgfsetrectcap%
\pgfsetmiterjoin%
\pgfsetlinewidth{0.803000pt}%
\definecolor{currentstroke}{rgb}{0.000000,0.000000,0.000000}%
\pgfsetstrokecolor{currentstroke}%
\pgfsetdash{}{0pt}%
\pgfpathmoveto{\pgfqpoint{0.615980in}{0.416447in}}%
\pgfpathlineto{\pgfqpoint{2.398330in}{0.416447in}}%
\pgfusepath{stroke}%
\end{pgfscope}%
\begin{pgfscope}%
\pgfsetrectcap%
\pgfsetmiterjoin%
\pgfsetlinewidth{0.803000pt}%
\definecolor{currentstroke}{rgb}{0.000000,0.000000,0.000000}%
\pgfsetstrokecolor{currentstroke}%
\pgfsetdash{}{0pt}%
\pgfpathmoveto{\pgfqpoint{0.615980in}{1.788330in}}%
\pgfpathlineto{\pgfqpoint{2.398330in}{1.788330in}}%
\pgfusepath{stroke}%
\end{pgfscope}%
\begin{pgfscope}%
\pgfsetbuttcap%
\pgfsetmiterjoin%
\definecolor{currentfill}{rgb}{1.000000,1.000000,1.000000}%
\pgfsetfillcolor{currentfill}%
\pgfsetfillopacity{0.800000}%
\pgfsetlinewidth{1.003750pt}%
\definecolor{currentstroke}{rgb}{0.800000,0.800000,0.800000}%
\pgfsetstrokecolor{currentstroke}%
\pgfsetstrokeopacity{0.800000}%
\pgfsetdash{}{0pt}%
\pgfpathmoveto{\pgfqpoint{0.693757in}{1.544552in}}%
\pgfpathlineto{\pgfqpoint{1.644535in}{1.544552in}}%
\pgfpathquadraticcurveto{\pgfqpoint{1.666757in}{1.544552in}}{\pgfqpoint{1.666757in}{1.566775in}}%
\pgfpathlineto{\pgfqpoint{1.666757in}{1.710552in}}%
\pgfpathquadraticcurveto{\pgfqpoint{1.666757in}{1.732774in}}{\pgfqpoint{1.644535in}{1.732774in}}%
\pgfpathlineto{\pgfqpoint{0.693757in}{1.732774in}}%
\pgfpathquadraticcurveto{\pgfqpoint{0.671535in}{1.732774in}}{\pgfqpoint{0.671535in}{1.710552in}}%
\pgfpathlineto{\pgfqpoint{0.671535in}{1.566775in}}%
\pgfpathquadraticcurveto{\pgfqpoint{0.671535in}{1.544552in}}{\pgfqpoint{0.693757in}{1.544552in}}%
\pgfpathlineto{\pgfqpoint{0.693757in}{1.544552in}}%
\pgfpathclose%
\pgfusepath{stroke,fill}%
\end{pgfscope}%
\begin{pgfscope}%
\pgfsetrectcap%
\pgfsetroundjoin%
\pgfsetlinewidth{1.505625pt}%
\definecolor{currentstroke}{rgb}{0.800000,0.474510,0.654902}%
\pgfsetstrokecolor{currentstroke}%
\pgfsetdash{}{0pt}%
\pgfpathmoveto{\pgfqpoint{0.715980in}{1.649441in}}%
\pgfpathlineto{\pgfqpoint{0.827091in}{1.649441in}}%
\pgfpathlineto{\pgfqpoint{0.938202in}{1.649441in}}%
\pgfusepath{stroke}%
\end{pgfscope}%
\begin{pgfscope}%
\definecolor{textcolor}{rgb}{0.000000,0.000000,0.000000}%
\pgfsetstrokecolor{textcolor}%
\pgfsetfillcolor{textcolor}%
\pgftext[x=1.027091in,y=1.610552in,left,base]{\color{textcolor}\rmfamily\fontsize{8.000000}{9.600000}\selectfont Linear drift}%
\end{pgfscope}%
\end{pgfpicture}%
\makeatother%
\endgroup%
% data/simulations/sim_allan_variance.py
        } % scalebox
        \caption{Time domain}
        \label{fig:drift_time}
    \end{subfigure}
    \begin{subfigure}{0.32\linewidth}
        \centering
        \scalebox{0.75}{%
            %% Creator: Matplotlib, PGF backend
%%
%% To include the figure in your LaTeX document, write
%%   \input{<filename>.pgf}
%%
%% Make sure the required packages are loaded in your preamble
%%   \usepackage{pgf}
%%
%% Also ensure that all the required font packages are loaded; for instance,
%% the lmodern package is sometimes necessary when using math font.
%%   \usepackage{lmodern}
%%
%% Figures using additional raster images can only be included by \input if
%% they are in the same directory as the main LaTeX file. For loading figures
%% from other directories you can use the `import` package
%%   \usepackage{import}
%%
%% and then include the figures with
%%   \import{<path to file>}{<filename>.pgf}
%%
%% Matplotlib used the following preamble
%%   \usepackage{siunitx}
%%   \sisetup{per-mode = symbol}%
%%   \usepackage{fontspec}
%%   \makeatletter\@ifpackageloaded{underscore}{}{\usepackage[strings]{underscore}}\makeatother
%%
\begingroup%
\makeatletter%
\begin{pgfpicture}%
\pgfpathrectangle{\pgfpointorigin}{\pgfqpoint{2.440945in}{1.830709in}}%
\pgfusepath{use as bounding box, clip}%
\begin{pgfscope}%
\pgfsetbuttcap%
\pgfsetmiterjoin%
\definecolor{currentfill}{rgb}{1.000000,1.000000,1.000000}%
\pgfsetfillcolor{currentfill}%
\pgfsetlinewidth{0.000000pt}%
\definecolor{currentstroke}{rgb}{1.000000,1.000000,1.000000}%
\pgfsetstrokecolor{currentstroke}%
\pgfsetdash{}{0pt}%
\pgfpathmoveto{\pgfqpoint{0.000000in}{0.000000in}}%
\pgfpathlineto{\pgfqpoint{2.440945in}{0.000000in}}%
\pgfpathlineto{\pgfqpoint{2.440945in}{1.830709in}}%
\pgfpathlineto{\pgfqpoint{0.000000in}{1.830709in}}%
\pgfpathlineto{\pgfqpoint{0.000000in}{0.000000in}}%
\pgfpathclose%
\pgfusepath{fill}%
\end{pgfscope}%
\begin{pgfscope}%
\pgfsetbuttcap%
\pgfsetmiterjoin%
\definecolor{currentfill}{rgb}{1.000000,1.000000,1.000000}%
\pgfsetfillcolor{currentfill}%
\pgfsetlinewidth{0.000000pt}%
\definecolor{currentstroke}{rgb}{0.000000,0.000000,0.000000}%
\pgfsetstrokecolor{currentstroke}%
\pgfsetstrokeopacity{0.000000}%
\pgfsetdash{}{0pt}%
\pgfpathmoveto{\pgfqpoint{0.589510in}{0.417642in}}%
\pgfpathlineto{\pgfqpoint{2.399275in}{0.417642in}}%
\pgfpathlineto{\pgfqpoint{2.399275in}{1.789039in}}%
\pgfpathlineto{\pgfqpoint{0.589510in}{1.789039in}}%
\pgfpathlineto{\pgfqpoint{0.589510in}{0.417642in}}%
\pgfpathclose%
\pgfusepath{fill}%
\end{pgfscope}%
\begin{pgfscope}%
\pgfpathrectangle{\pgfqpoint{0.589510in}{0.417642in}}{\pgfqpoint{1.809765in}{1.371397in}}%
\pgfusepath{clip}%
\pgfsetrectcap%
\pgfsetroundjoin%
\pgfsetlinewidth{0.803000pt}%
\definecolor{currentstroke}{rgb}{0.450000,0.450000,0.450000}%
\pgfsetstrokecolor{currentstroke}%
\pgfsetdash{}{0pt}%
\pgfpathmoveto{\pgfqpoint{0.671772in}{0.417642in}}%
\pgfpathlineto{\pgfqpoint{0.671772in}{1.789039in}}%
\pgfusepath{stroke}%
\end{pgfscope}%
\begin{pgfscope}%
\pgfsetbuttcap%
\pgfsetroundjoin%
\definecolor{currentfill}{rgb}{0.000000,0.000000,0.000000}%
\pgfsetfillcolor{currentfill}%
\pgfsetlinewidth{0.803000pt}%
\definecolor{currentstroke}{rgb}{0.000000,0.000000,0.000000}%
\pgfsetstrokecolor{currentstroke}%
\pgfsetdash{}{0pt}%
\pgfsys@defobject{currentmarker}{\pgfqpoint{0.000000in}{-0.048611in}}{\pgfqpoint{0.000000in}{0.000000in}}{%
\pgfpathmoveto{\pgfqpoint{0.000000in}{0.000000in}}%
\pgfpathlineto{\pgfqpoint{0.000000in}{-0.048611in}}%
\pgfusepath{stroke,fill}%
}%
\begin{pgfscope}%
\pgfsys@transformshift{0.671772in}{0.417642in}%
\pgfsys@useobject{currentmarker}{}%
\end{pgfscope}%
\end{pgfscope}%
\begin{pgfscope}%
\definecolor{textcolor}{rgb}{0.000000,0.000000,0.000000}%
\pgfsetstrokecolor{textcolor}%
\pgfsetfillcolor{textcolor}%
\pgftext[x=0.671772in,y=0.320420in,,top]{\color{textcolor}\rmfamily\fontsize{8.000000}{9.600000}\selectfont \(\displaystyle {10^{0}}\)}%
\end{pgfscope}%
\begin{pgfscope}%
\pgfpathrectangle{\pgfqpoint{0.589510in}{0.417642in}}{\pgfqpoint{1.809765in}{1.371397in}}%
\pgfusepath{clip}%
\pgfsetrectcap%
\pgfsetroundjoin%
\pgfsetlinewidth{0.803000pt}%
\definecolor{currentstroke}{rgb}{0.450000,0.450000,0.450000}%
\pgfsetstrokecolor{currentstroke}%
\pgfsetdash{}{0pt}%
\pgfpathmoveto{\pgfqpoint{1.128522in}{0.417642in}}%
\pgfpathlineto{\pgfqpoint{1.128522in}{1.789039in}}%
\pgfusepath{stroke}%
\end{pgfscope}%
\begin{pgfscope}%
\pgfsetbuttcap%
\pgfsetroundjoin%
\definecolor{currentfill}{rgb}{0.000000,0.000000,0.000000}%
\pgfsetfillcolor{currentfill}%
\pgfsetlinewidth{0.803000pt}%
\definecolor{currentstroke}{rgb}{0.000000,0.000000,0.000000}%
\pgfsetstrokecolor{currentstroke}%
\pgfsetdash{}{0pt}%
\pgfsys@defobject{currentmarker}{\pgfqpoint{0.000000in}{-0.048611in}}{\pgfqpoint{0.000000in}{0.000000in}}{%
\pgfpathmoveto{\pgfqpoint{0.000000in}{0.000000in}}%
\pgfpathlineto{\pgfqpoint{0.000000in}{-0.048611in}}%
\pgfusepath{stroke,fill}%
}%
\begin{pgfscope}%
\pgfsys@transformshift{1.128522in}{0.417642in}%
\pgfsys@useobject{currentmarker}{}%
\end{pgfscope}%
\end{pgfscope}%
\begin{pgfscope}%
\definecolor{textcolor}{rgb}{0.000000,0.000000,0.000000}%
\pgfsetstrokecolor{textcolor}%
\pgfsetfillcolor{textcolor}%
\pgftext[x=1.128522in,y=0.320420in,,top]{\color{textcolor}\rmfamily\fontsize{8.000000}{9.600000}\selectfont \(\displaystyle {10^{1}}\)}%
\end{pgfscope}%
\begin{pgfscope}%
\pgfpathrectangle{\pgfqpoint{0.589510in}{0.417642in}}{\pgfqpoint{1.809765in}{1.371397in}}%
\pgfusepath{clip}%
\pgfsetrectcap%
\pgfsetroundjoin%
\pgfsetlinewidth{0.803000pt}%
\definecolor{currentstroke}{rgb}{0.450000,0.450000,0.450000}%
\pgfsetstrokecolor{currentstroke}%
\pgfsetdash{}{0pt}%
\pgfpathmoveto{\pgfqpoint{1.585272in}{0.417642in}}%
\pgfpathlineto{\pgfqpoint{1.585272in}{1.789039in}}%
\pgfusepath{stroke}%
\end{pgfscope}%
\begin{pgfscope}%
\pgfsetbuttcap%
\pgfsetroundjoin%
\definecolor{currentfill}{rgb}{0.000000,0.000000,0.000000}%
\pgfsetfillcolor{currentfill}%
\pgfsetlinewidth{0.803000pt}%
\definecolor{currentstroke}{rgb}{0.000000,0.000000,0.000000}%
\pgfsetstrokecolor{currentstroke}%
\pgfsetdash{}{0pt}%
\pgfsys@defobject{currentmarker}{\pgfqpoint{0.000000in}{-0.048611in}}{\pgfqpoint{0.000000in}{0.000000in}}{%
\pgfpathmoveto{\pgfqpoint{0.000000in}{0.000000in}}%
\pgfpathlineto{\pgfqpoint{0.000000in}{-0.048611in}}%
\pgfusepath{stroke,fill}%
}%
\begin{pgfscope}%
\pgfsys@transformshift{1.585272in}{0.417642in}%
\pgfsys@useobject{currentmarker}{}%
\end{pgfscope}%
\end{pgfscope}%
\begin{pgfscope}%
\definecolor{textcolor}{rgb}{0.000000,0.000000,0.000000}%
\pgfsetstrokecolor{textcolor}%
\pgfsetfillcolor{textcolor}%
\pgftext[x=1.585272in,y=0.320420in,,top]{\color{textcolor}\rmfamily\fontsize{8.000000}{9.600000}\selectfont \(\displaystyle {10^{2}}\)}%
\end{pgfscope}%
\begin{pgfscope}%
\pgfpathrectangle{\pgfqpoint{0.589510in}{0.417642in}}{\pgfqpoint{1.809765in}{1.371397in}}%
\pgfusepath{clip}%
\pgfsetrectcap%
\pgfsetroundjoin%
\pgfsetlinewidth{0.803000pt}%
\definecolor{currentstroke}{rgb}{0.450000,0.450000,0.450000}%
\pgfsetstrokecolor{currentstroke}%
\pgfsetdash{}{0pt}%
\pgfpathmoveto{\pgfqpoint{2.042022in}{0.417642in}}%
\pgfpathlineto{\pgfqpoint{2.042022in}{1.789039in}}%
\pgfusepath{stroke}%
\end{pgfscope}%
\begin{pgfscope}%
\pgfsetbuttcap%
\pgfsetroundjoin%
\definecolor{currentfill}{rgb}{0.000000,0.000000,0.000000}%
\pgfsetfillcolor{currentfill}%
\pgfsetlinewidth{0.803000pt}%
\definecolor{currentstroke}{rgb}{0.000000,0.000000,0.000000}%
\pgfsetstrokecolor{currentstroke}%
\pgfsetdash{}{0pt}%
\pgfsys@defobject{currentmarker}{\pgfqpoint{0.000000in}{-0.048611in}}{\pgfqpoint{0.000000in}{0.000000in}}{%
\pgfpathmoveto{\pgfqpoint{0.000000in}{0.000000in}}%
\pgfpathlineto{\pgfqpoint{0.000000in}{-0.048611in}}%
\pgfusepath{stroke,fill}%
}%
\begin{pgfscope}%
\pgfsys@transformshift{2.042022in}{0.417642in}%
\pgfsys@useobject{currentmarker}{}%
\end{pgfscope}%
\end{pgfscope}%
\begin{pgfscope}%
\definecolor{textcolor}{rgb}{0.000000,0.000000,0.000000}%
\pgfsetstrokecolor{textcolor}%
\pgfsetfillcolor{textcolor}%
\pgftext[x=2.042022in,y=0.320420in,,top]{\color{textcolor}\rmfamily\fontsize{8.000000}{9.600000}\selectfont \(\displaystyle {10^{3}}\)}%
\end{pgfscope}%
\begin{pgfscope}%
\pgfpathrectangle{\pgfqpoint{0.589510in}{0.417642in}}{\pgfqpoint{1.809765in}{1.371397in}}%
\pgfusepath{clip}%
\pgfsetrectcap%
\pgfsetroundjoin%
\pgfsetlinewidth{0.803000pt}%
\definecolor{currentstroke}{rgb}{0.850000,0.850000,0.850000}%
\pgfsetstrokecolor{currentstroke}%
\pgfsetdash{}{0pt}%
\pgfpathmoveto{\pgfqpoint{0.601020in}{0.417642in}}%
\pgfpathlineto{\pgfqpoint{0.601020in}{1.789039in}}%
\pgfusepath{stroke}%
\end{pgfscope}%
\begin{pgfscope}%
\pgfsetbuttcap%
\pgfsetroundjoin%
\definecolor{currentfill}{rgb}{0.000000,0.000000,0.000000}%
\pgfsetfillcolor{currentfill}%
\pgfsetlinewidth{0.602250pt}%
\definecolor{currentstroke}{rgb}{0.000000,0.000000,0.000000}%
\pgfsetstrokecolor{currentstroke}%
\pgfsetdash{}{0pt}%
\pgfsys@defobject{currentmarker}{\pgfqpoint{0.000000in}{-0.027778in}}{\pgfqpoint{0.000000in}{0.000000in}}{%
\pgfpathmoveto{\pgfqpoint{0.000000in}{0.000000in}}%
\pgfpathlineto{\pgfqpoint{0.000000in}{-0.027778in}}%
\pgfusepath{stroke,fill}%
}%
\begin{pgfscope}%
\pgfsys@transformshift{0.601020in}{0.417642in}%
\pgfsys@useobject{currentmarker}{}%
\end{pgfscope}%
\end{pgfscope}%
\begin{pgfscope}%
\pgfpathrectangle{\pgfqpoint{0.589510in}{0.417642in}}{\pgfqpoint{1.809765in}{1.371397in}}%
\pgfusepath{clip}%
\pgfsetrectcap%
\pgfsetroundjoin%
\pgfsetlinewidth{0.803000pt}%
\definecolor{currentstroke}{rgb}{0.850000,0.850000,0.850000}%
\pgfsetstrokecolor{currentstroke}%
\pgfsetdash{}{0pt}%
\pgfpathmoveto{\pgfqpoint{0.627508in}{0.417642in}}%
\pgfpathlineto{\pgfqpoint{0.627508in}{1.789039in}}%
\pgfusepath{stroke}%
\end{pgfscope}%
\begin{pgfscope}%
\pgfsetbuttcap%
\pgfsetroundjoin%
\definecolor{currentfill}{rgb}{0.000000,0.000000,0.000000}%
\pgfsetfillcolor{currentfill}%
\pgfsetlinewidth{0.602250pt}%
\definecolor{currentstroke}{rgb}{0.000000,0.000000,0.000000}%
\pgfsetstrokecolor{currentstroke}%
\pgfsetdash{}{0pt}%
\pgfsys@defobject{currentmarker}{\pgfqpoint{0.000000in}{-0.027778in}}{\pgfqpoint{0.000000in}{0.000000in}}{%
\pgfpathmoveto{\pgfqpoint{0.000000in}{0.000000in}}%
\pgfpathlineto{\pgfqpoint{0.000000in}{-0.027778in}}%
\pgfusepath{stroke,fill}%
}%
\begin{pgfscope}%
\pgfsys@transformshift{0.627508in}{0.417642in}%
\pgfsys@useobject{currentmarker}{}%
\end{pgfscope}%
\end{pgfscope}%
\begin{pgfscope}%
\pgfpathrectangle{\pgfqpoint{0.589510in}{0.417642in}}{\pgfqpoint{1.809765in}{1.371397in}}%
\pgfusepath{clip}%
\pgfsetrectcap%
\pgfsetroundjoin%
\pgfsetlinewidth{0.803000pt}%
\definecolor{currentstroke}{rgb}{0.850000,0.850000,0.850000}%
\pgfsetstrokecolor{currentstroke}%
\pgfsetdash{}{0pt}%
\pgfpathmoveto{\pgfqpoint{0.650872in}{0.417642in}}%
\pgfpathlineto{\pgfqpoint{0.650872in}{1.789039in}}%
\pgfusepath{stroke}%
\end{pgfscope}%
\begin{pgfscope}%
\pgfsetbuttcap%
\pgfsetroundjoin%
\definecolor{currentfill}{rgb}{0.000000,0.000000,0.000000}%
\pgfsetfillcolor{currentfill}%
\pgfsetlinewidth{0.602250pt}%
\definecolor{currentstroke}{rgb}{0.000000,0.000000,0.000000}%
\pgfsetstrokecolor{currentstroke}%
\pgfsetdash{}{0pt}%
\pgfsys@defobject{currentmarker}{\pgfqpoint{0.000000in}{-0.027778in}}{\pgfqpoint{0.000000in}{0.000000in}}{%
\pgfpathmoveto{\pgfqpoint{0.000000in}{0.000000in}}%
\pgfpathlineto{\pgfqpoint{0.000000in}{-0.027778in}}%
\pgfusepath{stroke,fill}%
}%
\begin{pgfscope}%
\pgfsys@transformshift{0.650872in}{0.417642in}%
\pgfsys@useobject{currentmarker}{}%
\end{pgfscope}%
\end{pgfscope}%
\begin{pgfscope}%
\pgfpathrectangle{\pgfqpoint{0.589510in}{0.417642in}}{\pgfqpoint{1.809765in}{1.371397in}}%
\pgfusepath{clip}%
\pgfsetrectcap%
\pgfsetroundjoin%
\pgfsetlinewidth{0.803000pt}%
\definecolor{currentstroke}{rgb}{0.850000,0.850000,0.850000}%
\pgfsetstrokecolor{currentstroke}%
\pgfsetdash{}{0pt}%
\pgfpathmoveto{\pgfqpoint{0.809267in}{0.417642in}}%
\pgfpathlineto{\pgfqpoint{0.809267in}{1.789039in}}%
\pgfusepath{stroke}%
\end{pgfscope}%
\begin{pgfscope}%
\pgfsetbuttcap%
\pgfsetroundjoin%
\definecolor{currentfill}{rgb}{0.000000,0.000000,0.000000}%
\pgfsetfillcolor{currentfill}%
\pgfsetlinewidth{0.602250pt}%
\definecolor{currentstroke}{rgb}{0.000000,0.000000,0.000000}%
\pgfsetstrokecolor{currentstroke}%
\pgfsetdash{}{0pt}%
\pgfsys@defobject{currentmarker}{\pgfqpoint{0.000000in}{-0.027778in}}{\pgfqpoint{0.000000in}{0.000000in}}{%
\pgfpathmoveto{\pgfqpoint{0.000000in}{0.000000in}}%
\pgfpathlineto{\pgfqpoint{0.000000in}{-0.027778in}}%
\pgfusepath{stroke,fill}%
}%
\begin{pgfscope}%
\pgfsys@transformshift{0.809267in}{0.417642in}%
\pgfsys@useobject{currentmarker}{}%
\end{pgfscope}%
\end{pgfscope}%
\begin{pgfscope}%
\pgfpathrectangle{\pgfqpoint{0.589510in}{0.417642in}}{\pgfqpoint{1.809765in}{1.371397in}}%
\pgfusepath{clip}%
\pgfsetrectcap%
\pgfsetroundjoin%
\pgfsetlinewidth{0.803000pt}%
\definecolor{currentstroke}{rgb}{0.850000,0.850000,0.850000}%
\pgfsetstrokecolor{currentstroke}%
\pgfsetdash{}{0pt}%
\pgfpathmoveto{\pgfqpoint{0.889697in}{0.417642in}}%
\pgfpathlineto{\pgfqpoint{0.889697in}{1.789039in}}%
\pgfusepath{stroke}%
\end{pgfscope}%
\begin{pgfscope}%
\pgfsetbuttcap%
\pgfsetroundjoin%
\definecolor{currentfill}{rgb}{0.000000,0.000000,0.000000}%
\pgfsetfillcolor{currentfill}%
\pgfsetlinewidth{0.602250pt}%
\definecolor{currentstroke}{rgb}{0.000000,0.000000,0.000000}%
\pgfsetstrokecolor{currentstroke}%
\pgfsetdash{}{0pt}%
\pgfsys@defobject{currentmarker}{\pgfqpoint{0.000000in}{-0.027778in}}{\pgfqpoint{0.000000in}{0.000000in}}{%
\pgfpathmoveto{\pgfqpoint{0.000000in}{0.000000in}}%
\pgfpathlineto{\pgfqpoint{0.000000in}{-0.027778in}}%
\pgfusepath{stroke,fill}%
}%
\begin{pgfscope}%
\pgfsys@transformshift{0.889697in}{0.417642in}%
\pgfsys@useobject{currentmarker}{}%
\end{pgfscope}%
\end{pgfscope}%
\begin{pgfscope}%
\pgfpathrectangle{\pgfqpoint{0.589510in}{0.417642in}}{\pgfqpoint{1.809765in}{1.371397in}}%
\pgfusepath{clip}%
\pgfsetrectcap%
\pgfsetroundjoin%
\pgfsetlinewidth{0.803000pt}%
\definecolor{currentstroke}{rgb}{0.850000,0.850000,0.850000}%
\pgfsetstrokecolor{currentstroke}%
\pgfsetdash{}{0pt}%
\pgfpathmoveto{\pgfqpoint{0.946763in}{0.417642in}}%
\pgfpathlineto{\pgfqpoint{0.946763in}{1.789039in}}%
\pgfusepath{stroke}%
\end{pgfscope}%
\begin{pgfscope}%
\pgfsetbuttcap%
\pgfsetroundjoin%
\definecolor{currentfill}{rgb}{0.000000,0.000000,0.000000}%
\pgfsetfillcolor{currentfill}%
\pgfsetlinewidth{0.602250pt}%
\definecolor{currentstroke}{rgb}{0.000000,0.000000,0.000000}%
\pgfsetstrokecolor{currentstroke}%
\pgfsetdash{}{0pt}%
\pgfsys@defobject{currentmarker}{\pgfqpoint{0.000000in}{-0.027778in}}{\pgfqpoint{0.000000in}{0.000000in}}{%
\pgfpathmoveto{\pgfqpoint{0.000000in}{0.000000in}}%
\pgfpathlineto{\pgfqpoint{0.000000in}{-0.027778in}}%
\pgfusepath{stroke,fill}%
}%
\begin{pgfscope}%
\pgfsys@transformshift{0.946763in}{0.417642in}%
\pgfsys@useobject{currentmarker}{}%
\end{pgfscope}%
\end{pgfscope}%
\begin{pgfscope}%
\pgfpathrectangle{\pgfqpoint{0.589510in}{0.417642in}}{\pgfqpoint{1.809765in}{1.371397in}}%
\pgfusepath{clip}%
\pgfsetrectcap%
\pgfsetroundjoin%
\pgfsetlinewidth{0.803000pt}%
\definecolor{currentstroke}{rgb}{0.850000,0.850000,0.850000}%
\pgfsetstrokecolor{currentstroke}%
\pgfsetdash{}{0pt}%
\pgfpathmoveto{\pgfqpoint{0.991026in}{0.417642in}}%
\pgfpathlineto{\pgfqpoint{0.991026in}{1.789039in}}%
\pgfusepath{stroke}%
\end{pgfscope}%
\begin{pgfscope}%
\pgfsetbuttcap%
\pgfsetroundjoin%
\definecolor{currentfill}{rgb}{0.000000,0.000000,0.000000}%
\pgfsetfillcolor{currentfill}%
\pgfsetlinewidth{0.602250pt}%
\definecolor{currentstroke}{rgb}{0.000000,0.000000,0.000000}%
\pgfsetstrokecolor{currentstroke}%
\pgfsetdash{}{0pt}%
\pgfsys@defobject{currentmarker}{\pgfqpoint{0.000000in}{-0.027778in}}{\pgfqpoint{0.000000in}{0.000000in}}{%
\pgfpathmoveto{\pgfqpoint{0.000000in}{0.000000in}}%
\pgfpathlineto{\pgfqpoint{0.000000in}{-0.027778in}}%
\pgfusepath{stroke,fill}%
}%
\begin{pgfscope}%
\pgfsys@transformshift{0.991026in}{0.417642in}%
\pgfsys@useobject{currentmarker}{}%
\end{pgfscope}%
\end{pgfscope}%
\begin{pgfscope}%
\pgfpathrectangle{\pgfqpoint{0.589510in}{0.417642in}}{\pgfqpoint{1.809765in}{1.371397in}}%
\pgfusepath{clip}%
\pgfsetrectcap%
\pgfsetroundjoin%
\pgfsetlinewidth{0.803000pt}%
\definecolor{currentstroke}{rgb}{0.850000,0.850000,0.850000}%
\pgfsetstrokecolor{currentstroke}%
\pgfsetdash{}{0pt}%
\pgfpathmoveto{\pgfqpoint{1.027192in}{0.417642in}}%
\pgfpathlineto{\pgfqpoint{1.027192in}{1.789039in}}%
\pgfusepath{stroke}%
\end{pgfscope}%
\begin{pgfscope}%
\pgfsetbuttcap%
\pgfsetroundjoin%
\definecolor{currentfill}{rgb}{0.000000,0.000000,0.000000}%
\pgfsetfillcolor{currentfill}%
\pgfsetlinewidth{0.602250pt}%
\definecolor{currentstroke}{rgb}{0.000000,0.000000,0.000000}%
\pgfsetstrokecolor{currentstroke}%
\pgfsetdash{}{0pt}%
\pgfsys@defobject{currentmarker}{\pgfqpoint{0.000000in}{-0.027778in}}{\pgfqpoint{0.000000in}{0.000000in}}{%
\pgfpathmoveto{\pgfqpoint{0.000000in}{0.000000in}}%
\pgfpathlineto{\pgfqpoint{0.000000in}{-0.027778in}}%
\pgfusepath{stroke,fill}%
}%
\begin{pgfscope}%
\pgfsys@transformshift{1.027192in}{0.417642in}%
\pgfsys@useobject{currentmarker}{}%
\end{pgfscope}%
\end{pgfscope}%
\begin{pgfscope}%
\pgfpathrectangle{\pgfqpoint{0.589510in}{0.417642in}}{\pgfqpoint{1.809765in}{1.371397in}}%
\pgfusepath{clip}%
\pgfsetrectcap%
\pgfsetroundjoin%
\pgfsetlinewidth{0.803000pt}%
\definecolor{currentstroke}{rgb}{0.850000,0.850000,0.850000}%
\pgfsetstrokecolor{currentstroke}%
\pgfsetdash{}{0pt}%
\pgfpathmoveto{\pgfqpoint{1.057770in}{0.417642in}}%
\pgfpathlineto{\pgfqpoint{1.057770in}{1.789039in}}%
\pgfusepath{stroke}%
\end{pgfscope}%
\begin{pgfscope}%
\pgfsetbuttcap%
\pgfsetroundjoin%
\definecolor{currentfill}{rgb}{0.000000,0.000000,0.000000}%
\pgfsetfillcolor{currentfill}%
\pgfsetlinewidth{0.602250pt}%
\definecolor{currentstroke}{rgb}{0.000000,0.000000,0.000000}%
\pgfsetstrokecolor{currentstroke}%
\pgfsetdash{}{0pt}%
\pgfsys@defobject{currentmarker}{\pgfqpoint{0.000000in}{-0.027778in}}{\pgfqpoint{0.000000in}{0.000000in}}{%
\pgfpathmoveto{\pgfqpoint{0.000000in}{0.000000in}}%
\pgfpathlineto{\pgfqpoint{0.000000in}{-0.027778in}}%
\pgfusepath{stroke,fill}%
}%
\begin{pgfscope}%
\pgfsys@transformshift{1.057770in}{0.417642in}%
\pgfsys@useobject{currentmarker}{}%
\end{pgfscope}%
\end{pgfscope}%
\begin{pgfscope}%
\pgfpathrectangle{\pgfqpoint{0.589510in}{0.417642in}}{\pgfqpoint{1.809765in}{1.371397in}}%
\pgfusepath{clip}%
\pgfsetrectcap%
\pgfsetroundjoin%
\pgfsetlinewidth{0.803000pt}%
\definecolor{currentstroke}{rgb}{0.850000,0.850000,0.850000}%
\pgfsetstrokecolor{currentstroke}%
\pgfsetdash{}{0pt}%
\pgfpathmoveto{\pgfqpoint{1.084258in}{0.417642in}}%
\pgfpathlineto{\pgfqpoint{1.084258in}{1.789039in}}%
\pgfusepath{stroke}%
\end{pgfscope}%
\begin{pgfscope}%
\pgfsetbuttcap%
\pgfsetroundjoin%
\definecolor{currentfill}{rgb}{0.000000,0.000000,0.000000}%
\pgfsetfillcolor{currentfill}%
\pgfsetlinewidth{0.602250pt}%
\definecolor{currentstroke}{rgb}{0.000000,0.000000,0.000000}%
\pgfsetstrokecolor{currentstroke}%
\pgfsetdash{}{0pt}%
\pgfsys@defobject{currentmarker}{\pgfqpoint{0.000000in}{-0.027778in}}{\pgfqpoint{0.000000in}{0.000000in}}{%
\pgfpathmoveto{\pgfqpoint{0.000000in}{0.000000in}}%
\pgfpathlineto{\pgfqpoint{0.000000in}{-0.027778in}}%
\pgfusepath{stroke,fill}%
}%
\begin{pgfscope}%
\pgfsys@transformshift{1.084258in}{0.417642in}%
\pgfsys@useobject{currentmarker}{}%
\end{pgfscope}%
\end{pgfscope}%
\begin{pgfscope}%
\pgfpathrectangle{\pgfqpoint{0.589510in}{0.417642in}}{\pgfqpoint{1.809765in}{1.371397in}}%
\pgfusepath{clip}%
\pgfsetrectcap%
\pgfsetroundjoin%
\pgfsetlinewidth{0.803000pt}%
\definecolor{currentstroke}{rgb}{0.850000,0.850000,0.850000}%
\pgfsetstrokecolor{currentstroke}%
\pgfsetdash{}{0pt}%
\pgfpathmoveto{\pgfqpoint{1.107622in}{0.417642in}}%
\pgfpathlineto{\pgfqpoint{1.107622in}{1.789039in}}%
\pgfusepath{stroke}%
\end{pgfscope}%
\begin{pgfscope}%
\pgfsetbuttcap%
\pgfsetroundjoin%
\definecolor{currentfill}{rgb}{0.000000,0.000000,0.000000}%
\pgfsetfillcolor{currentfill}%
\pgfsetlinewidth{0.602250pt}%
\definecolor{currentstroke}{rgb}{0.000000,0.000000,0.000000}%
\pgfsetstrokecolor{currentstroke}%
\pgfsetdash{}{0pt}%
\pgfsys@defobject{currentmarker}{\pgfqpoint{0.000000in}{-0.027778in}}{\pgfqpoint{0.000000in}{0.000000in}}{%
\pgfpathmoveto{\pgfqpoint{0.000000in}{0.000000in}}%
\pgfpathlineto{\pgfqpoint{0.000000in}{-0.027778in}}%
\pgfusepath{stroke,fill}%
}%
\begin{pgfscope}%
\pgfsys@transformshift{1.107622in}{0.417642in}%
\pgfsys@useobject{currentmarker}{}%
\end{pgfscope}%
\end{pgfscope}%
\begin{pgfscope}%
\pgfpathrectangle{\pgfqpoint{0.589510in}{0.417642in}}{\pgfqpoint{1.809765in}{1.371397in}}%
\pgfusepath{clip}%
\pgfsetrectcap%
\pgfsetroundjoin%
\pgfsetlinewidth{0.803000pt}%
\definecolor{currentstroke}{rgb}{0.850000,0.850000,0.850000}%
\pgfsetstrokecolor{currentstroke}%
\pgfsetdash{}{0pt}%
\pgfpathmoveto{\pgfqpoint{1.266017in}{0.417642in}}%
\pgfpathlineto{\pgfqpoint{1.266017in}{1.789039in}}%
\pgfusepath{stroke}%
\end{pgfscope}%
\begin{pgfscope}%
\pgfsetbuttcap%
\pgfsetroundjoin%
\definecolor{currentfill}{rgb}{0.000000,0.000000,0.000000}%
\pgfsetfillcolor{currentfill}%
\pgfsetlinewidth{0.602250pt}%
\definecolor{currentstroke}{rgb}{0.000000,0.000000,0.000000}%
\pgfsetstrokecolor{currentstroke}%
\pgfsetdash{}{0pt}%
\pgfsys@defobject{currentmarker}{\pgfqpoint{0.000000in}{-0.027778in}}{\pgfqpoint{0.000000in}{0.000000in}}{%
\pgfpathmoveto{\pgfqpoint{0.000000in}{0.000000in}}%
\pgfpathlineto{\pgfqpoint{0.000000in}{-0.027778in}}%
\pgfusepath{stroke,fill}%
}%
\begin{pgfscope}%
\pgfsys@transformshift{1.266017in}{0.417642in}%
\pgfsys@useobject{currentmarker}{}%
\end{pgfscope}%
\end{pgfscope}%
\begin{pgfscope}%
\pgfpathrectangle{\pgfqpoint{0.589510in}{0.417642in}}{\pgfqpoint{1.809765in}{1.371397in}}%
\pgfusepath{clip}%
\pgfsetrectcap%
\pgfsetroundjoin%
\pgfsetlinewidth{0.803000pt}%
\definecolor{currentstroke}{rgb}{0.850000,0.850000,0.850000}%
\pgfsetstrokecolor{currentstroke}%
\pgfsetdash{}{0pt}%
\pgfpathmoveto{\pgfqpoint{1.346447in}{0.417642in}}%
\pgfpathlineto{\pgfqpoint{1.346447in}{1.789039in}}%
\pgfusepath{stroke}%
\end{pgfscope}%
\begin{pgfscope}%
\pgfsetbuttcap%
\pgfsetroundjoin%
\definecolor{currentfill}{rgb}{0.000000,0.000000,0.000000}%
\pgfsetfillcolor{currentfill}%
\pgfsetlinewidth{0.602250pt}%
\definecolor{currentstroke}{rgb}{0.000000,0.000000,0.000000}%
\pgfsetstrokecolor{currentstroke}%
\pgfsetdash{}{0pt}%
\pgfsys@defobject{currentmarker}{\pgfqpoint{0.000000in}{-0.027778in}}{\pgfqpoint{0.000000in}{0.000000in}}{%
\pgfpathmoveto{\pgfqpoint{0.000000in}{0.000000in}}%
\pgfpathlineto{\pgfqpoint{0.000000in}{-0.027778in}}%
\pgfusepath{stroke,fill}%
}%
\begin{pgfscope}%
\pgfsys@transformshift{1.346447in}{0.417642in}%
\pgfsys@useobject{currentmarker}{}%
\end{pgfscope}%
\end{pgfscope}%
\begin{pgfscope}%
\pgfpathrectangle{\pgfqpoint{0.589510in}{0.417642in}}{\pgfqpoint{1.809765in}{1.371397in}}%
\pgfusepath{clip}%
\pgfsetrectcap%
\pgfsetroundjoin%
\pgfsetlinewidth{0.803000pt}%
\definecolor{currentstroke}{rgb}{0.850000,0.850000,0.850000}%
\pgfsetstrokecolor{currentstroke}%
\pgfsetdash{}{0pt}%
\pgfpathmoveto{\pgfqpoint{1.403513in}{0.417642in}}%
\pgfpathlineto{\pgfqpoint{1.403513in}{1.789039in}}%
\pgfusepath{stroke}%
\end{pgfscope}%
\begin{pgfscope}%
\pgfsetbuttcap%
\pgfsetroundjoin%
\definecolor{currentfill}{rgb}{0.000000,0.000000,0.000000}%
\pgfsetfillcolor{currentfill}%
\pgfsetlinewidth{0.602250pt}%
\definecolor{currentstroke}{rgb}{0.000000,0.000000,0.000000}%
\pgfsetstrokecolor{currentstroke}%
\pgfsetdash{}{0pt}%
\pgfsys@defobject{currentmarker}{\pgfqpoint{0.000000in}{-0.027778in}}{\pgfqpoint{0.000000in}{0.000000in}}{%
\pgfpathmoveto{\pgfqpoint{0.000000in}{0.000000in}}%
\pgfpathlineto{\pgfqpoint{0.000000in}{-0.027778in}}%
\pgfusepath{stroke,fill}%
}%
\begin{pgfscope}%
\pgfsys@transformshift{1.403513in}{0.417642in}%
\pgfsys@useobject{currentmarker}{}%
\end{pgfscope}%
\end{pgfscope}%
\begin{pgfscope}%
\pgfpathrectangle{\pgfqpoint{0.589510in}{0.417642in}}{\pgfqpoint{1.809765in}{1.371397in}}%
\pgfusepath{clip}%
\pgfsetrectcap%
\pgfsetroundjoin%
\pgfsetlinewidth{0.803000pt}%
\definecolor{currentstroke}{rgb}{0.850000,0.850000,0.850000}%
\pgfsetstrokecolor{currentstroke}%
\pgfsetdash{}{0pt}%
\pgfpathmoveto{\pgfqpoint{1.447776in}{0.417642in}}%
\pgfpathlineto{\pgfqpoint{1.447776in}{1.789039in}}%
\pgfusepath{stroke}%
\end{pgfscope}%
\begin{pgfscope}%
\pgfsetbuttcap%
\pgfsetroundjoin%
\definecolor{currentfill}{rgb}{0.000000,0.000000,0.000000}%
\pgfsetfillcolor{currentfill}%
\pgfsetlinewidth{0.602250pt}%
\definecolor{currentstroke}{rgb}{0.000000,0.000000,0.000000}%
\pgfsetstrokecolor{currentstroke}%
\pgfsetdash{}{0pt}%
\pgfsys@defobject{currentmarker}{\pgfqpoint{0.000000in}{-0.027778in}}{\pgfqpoint{0.000000in}{0.000000in}}{%
\pgfpathmoveto{\pgfqpoint{0.000000in}{0.000000in}}%
\pgfpathlineto{\pgfqpoint{0.000000in}{-0.027778in}}%
\pgfusepath{stroke,fill}%
}%
\begin{pgfscope}%
\pgfsys@transformshift{1.447776in}{0.417642in}%
\pgfsys@useobject{currentmarker}{}%
\end{pgfscope}%
\end{pgfscope}%
\begin{pgfscope}%
\pgfpathrectangle{\pgfqpoint{0.589510in}{0.417642in}}{\pgfqpoint{1.809765in}{1.371397in}}%
\pgfusepath{clip}%
\pgfsetrectcap%
\pgfsetroundjoin%
\pgfsetlinewidth{0.803000pt}%
\definecolor{currentstroke}{rgb}{0.850000,0.850000,0.850000}%
\pgfsetstrokecolor{currentstroke}%
\pgfsetdash{}{0pt}%
\pgfpathmoveto{\pgfqpoint{1.483942in}{0.417642in}}%
\pgfpathlineto{\pgfqpoint{1.483942in}{1.789039in}}%
\pgfusepath{stroke}%
\end{pgfscope}%
\begin{pgfscope}%
\pgfsetbuttcap%
\pgfsetroundjoin%
\definecolor{currentfill}{rgb}{0.000000,0.000000,0.000000}%
\pgfsetfillcolor{currentfill}%
\pgfsetlinewidth{0.602250pt}%
\definecolor{currentstroke}{rgb}{0.000000,0.000000,0.000000}%
\pgfsetstrokecolor{currentstroke}%
\pgfsetdash{}{0pt}%
\pgfsys@defobject{currentmarker}{\pgfqpoint{0.000000in}{-0.027778in}}{\pgfqpoint{0.000000in}{0.000000in}}{%
\pgfpathmoveto{\pgfqpoint{0.000000in}{0.000000in}}%
\pgfpathlineto{\pgfqpoint{0.000000in}{-0.027778in}}%
\pgfusepath{stroke,fill}%
}%
\begin{pgfscope}%
\pgfsys@transformshift{1.483942in}{0.417642in}%
\pgfsys@useobject{currentmarker}{}%
\end{pgfscope}%
\end{pgfscope}%
\begin{pgfscope}%
\pgfpathrectangle{\pgfqpoint{0.589510in}{0.417642in}}{\pgfqpoint{1.809765in}{1.371397in}}%
\pgfusepath{clip}%
\pgfsetrectcap%
\pgfsetroundjoin%
\pgfsetlinewidth{0.803000pt}%
\definecolor{currentstroke}{rgb}{0.850000,0.850000,0.850000}%
\pgfsetstrokecolor{currentstroke}%
\pgfsetdash{}{0pt}%
\pgfpathmoveto{\pgfqpoint{1.514520in}{0.417642in}}%
\pgfpathlineto{\pgfqpoint{1.514520in}{1.789039in}}%
\pgfusepath{stroke}%
\end{pgfscope}%
\begin{pgfscope}%
\pgfsetbuttcap%
\pgfsetroundjoin%
\definecolor{currentfill}{rgb}{0.000000,0.000000,0.000000}%
\pgfsetfillcolor{currentfill}%
\pgfsetlinewidth{0.602250pt}%
\definecolor{currentstroke}{rgb}{0.000000,0.000000,0.000000}%
\pgfsetstrokecolor{currentstroke}%
\pgfsetdash{}{0pt}%
\pgfsys@defobject{currentmarker}{\pgfqpoint{0.000000in}{-0.027778in}}{\pgfqpoint{0.000000in}{0.000000in}}{%
\pgfpathmoveto{\pgfqpoint{0.000000in}{0.000000in}}%
\pgfpathlineto{\pgfqpoint{0.000000in}{-0.027778in}}%
\pgfusepath{stroke,fill}%
}%
\begin{pgfscope}%
\pgfsys@transformshift{1.514520in}{0.417642in}%
\pgfsys@useobject{currentmarker}{}%
\end{pgfscope}%
\end{pgfscope}%
\begin{pgfscope}%
\pgfpathrectangle{\pgfqpoint{0.589510in}{0.417642in}}{\pgfqpoint{1.809765in}{1.371397in}}%
\pgfusepath{clip}%
\pgfsetrectcap%
\pgfsetroundjoin%
\pgfsetlinewidth{0.803000pt}%
\definecolor{currentstroke}{rgb}{0.850000,0.850000,0.850000}%
\pgfsetstrokecolor{currentstroke}%
\pgfsetdash{}{0pt}%
\pgfpathmoveto{\pgfqpoint{1.541008in}{0.417642in}}%
\pgfpathlineto{\pgfqpoint{1.541008in}{1.789039in}}%
\pgfusepath{stroke}%
\end{pgfscope}%
\begin{pgfscope}%
\pgfsetbuttcap%
\pgfsetroundjoin%
\definecolor{currentfill}{rgb}{0.000000,0.000000,0.000000}%
\pgfsetfillcolor{currentfill}%
\pgfsetlinewidth{0.602250pt}%
\definecolor{currentstroke}{rgb}{0.000000,0.000000,0.000000}%
\pgfsetstrokecolor{currentstroke}%
\pgfsetdash{}{0pt}%
\pgfsys@defobject{currentmarker}{\pgfqpoint{0.000000in}{-0.027778in}}{\pgfqpoint{0.000000in}{0.000000in}}{%
\pgfpathmoveto{\pgfqpoint{0.000000in}{0.000000in}}%
\pgfpathlineto{\pgfqpoint{0.000000in}{-0.027778in}}%
\pgfusepath{stroke,fill}%
}%
\begin{pgfscope}%
\pgfsys@transformshift{1.541008in}{0.417642in}%
\pgfsys@useobject{currentmarker}{}%
\end{pgfscope}%
\end{pgfscope}%
\begin{pgfscope}%
\pgfpathrectangle{\pgfqpoint{0.589510in}{0.417642in}}{\pgfqpoint{1.809765in}{1.371397in}}%
\pgfusepath{clip}%
\pgfsetrectcap%
\pgfsetroundjoin%
\pgfsetlinewidth{0.803000pt}%
\definecolor{currentstroke}{rgb}{0.850000,0.850000,0.850000}%
\pgfsetstrokecolor{currentstroke}%
\pgfsetdash{}{0pt}%
\pgfpathmoveto{\pgfqpoint{1.564372in}{0.417642in}}%
\pgfpathlineto{\pgfqpoint{1.564372in}{1.789039in}}%
\pgfusepath{stroke}%
\end{pgfscope}%
\begin{pgfscope}%
\pgfsetbuttcap%
\pgfsetroundjoin%
\definecolor{currentfill}{rgb}{0.000000,0.000000,0.000000}%
\pgfsetfillcolor{currentfill}%
\pgfsetlinewidth{0.602250pt}%
\definecolor{currentstroke}{rgb}{0.000000,0.000000,0.000000}%
\pgfsetstrokecolor{currentstroke}%
\pgfsetdash{}{0pt}%
\pgfsys@defobject{currentmarker}{\pgfqpoint{0.000000in}{-0.027778in}}{\pgfqpoint{0.000000in}{0.000000in}}{%
\pgfpathmoveto{\pgfqpoint{0.000000in}{0.000000in}}%
\pgfpathlineto{\pgfqpoint{0.000000in}{-0.027778in}}%
\pgfusepath{stroke,fill}%
}%
\begin{pgfscope}%
\pgfsys@transformshift{1.564372in}{0.417642in}%
\pgfsys@useobject{currentmarker}{}%
\end{pgfscope}%
\end{pgfscope}%
\begin{pgfscope}%
\pgfpathrectangle{\pgfqpoint{0.589510in}{0.417642in}}{\pgfqpoint{1.809765in}{1.371397in}}%
\pgfusepath{clip}%
\pgfsetrectcap%
\pgfsetroundjoin%
\pgfsetlinewidth{0.803000pt}%
\definecolor{currentstroke}{rgb}{0.850000,0.850000,0.850000}%
\pgfsetstrokecolor{currentstroke}%
\pgfsetdash{}{0pt}%
\pgfpathmoveto{\pgfqpoint{1.722767in}{0.417642in}}%
\pgfpathlineto{\pgfqpoint{1.722767in}{1.789039in}}%
\pgfusepath{stroke}%
\end{pgfscope}%
\begin{pgfscope}%
\pgfsetbuttcap%
\pgfsetroundjoin%
\definecolor{currentfill}{rgb}{0.000000,0.000000,0.000000}%
\pgfsetfillcolor{currentfill}%
\pgfsetlinewidth{0.602250pt}%
\definecolor{currentstroke}{rgb}{0.000000,0.000000,0.000000}%
\pgfsetstrokecolor{currentstroke}%
\pgfsetdash{}{0pt}%
\pgfsys@defobject{currentmarker}{\pgfqpoint{0.000000in}{-0.027778in}}{\pgfqpoint{0.000000in}{0.000000in}}{%
\pgfpathmoveto{\pgfqpoint{0.000000in}{0.000000in}}%
\pgfpathlineto{\pgfqpoint{0.000000in}{-0.027778in}}%
\pgfusepath{stroke,fill}%
}%
\begin{pgfscope}%
\pgfsys@transformshift{1.722767in}{0.417642in}%
\pgfsys@useobject{currentmarker}{}%
\end{pgfscope}%
\end{pgfscope}%
\begin{pgfscope}%
\pgfpathrectangle{\pgfqpoint{0.589510in}{0.417642in}}{\pgfqpoint{1.809765in}{1.371397in}}%
\pgfusepath{clip}%
\pgfsetrectcap%
\pgfsetroundjoin%
\pgfsetlinewidth{0.803000pt}%
\definecolor{currentstroke}{rgb}{0.850000,0.850000,0.850000}%
\pgfsetstrokecolor{currentstroke}%
\pgfsetdash{}{0pt}%
\pgfpathmoveto{\pgfqpoint{1.803197in}{0.417642in}}%
\pgfpathlineto{\pgfqpoint{1.803197in}{1.789039in}}%
\pgfusepath{stroke}%
\end{pgfscope}%
\begin{pgfscope}%
\pgfsetbuttcap%
\pgfsetroundjoin%
\definecolor{currentfill}{rgb}{0.000000,0.000000,0.000000}%
\pgfsetfillcolor{currentfill}%
\pgfsetlinewidth{0.602250pt}%
\definecolor{currentstroke}{rgb}{0.000000,0.000000,0.000000}%
\pgfsetstrokecolor{currentstroke}%
\pgfsetdash{}{0pt}%
\pgfsys@defobject{currentmarker}{\pgfqpoint{0.000000in}{-0.027778in}}{\pgfqpoint{0.000000in}{0.000000in}}{%
\pgfpathmoveto{\pgfqpoint{0.000000in}{0.000000in}}%
\pgfpathlineto{\pgfqpoint{0.000000in}{-0.027778in}}%
\pgfusepath{stroke,fill}%
}%
\begin{pgfscope}%
\pgfsys@transformshift{1.803197in}{0.417642in}%
\pgfsys@useobject{currentmarker}{}%
\end{pgfscope}%
\end{pgfscope}%
\begin{pgfscope}%
\pgfpathrectangle{\pgfqpoint{0.589510in}{0.417642in}}{\pgfqpoint{1.809765in}{1.371397in}}%
\pgfusepath{clip}%
\pgfsetrectcap%
\pgfsetroundjoin%
\pgfsetlinewidth{0.803000pt}%
\definecolor{currentstroke}{rgb}{0.850000,0.850000,0.850000}%
\pgfsetstrokecolor{currentstroke}%
\pgfsetdash{}{0pt}%
\pgfpathmoveto{\pgfqpoint{1.860263in}{0.417642in}}%
\pgfpathlineto{\pgfqpoint{1.860263in}{1.789039in}}%
\pgfusepath{stroke}%
\end{pgfscope}%
\begin{pgfscope}%
\pgfsetbuttcap%
\pgfsetroundjoin%
\definecolor{currentfill}{rgb}{0.000000,0.000000,0.000000}%
\pgfsetfillcolor{currentfill}%
\pgfsetlinewidth{0.602250pt}%
\definecolor{currentstroke}{rgb}{0.000000,0.000000,0.000000}%
\pgfsetstrokecolor{currentstroke}%
\pgfsetdash{}{0pt}%
\pgfsys@defobject{currentmarker}{\pgfqpoint{0.000000in}{-0.027778in}}{\pgfqpoint{0.000000in}{0.000000in}}{%
\pgfpathmoveto{\pgfqpoint{0.000000in}{0.000000in}}%
\pgfpathlineto{\pgfqpoint{0.000000in}{-0.027778in}}%
\pgfusepath{stroke,fill}%
}%
\begin{pgfscope}%
\pgfsys@transformshift{1.860263in}{0.417642in}%
\pgfsys@useobject{currentmarker}{}%
\end{pgfscope}%
\end{pgfscope}%
\begin{pgfscope}%
\pgfpathrectangle{\pgfqpoint{0.589510in}{0.417642in}}{\pgfqpoint{1.809765in}{1.371397in}}%
\pgfusepath{clip}%
\pgfsetrectcap%
\pgfsetroundjoin%
\pgfsetlinewidth{0.803000pt}%
\definecolor{currentstroke}{rgb}{0.850000,0.850000,0.850000}%
\pgfsetstrokecolor{currentstroke}%
\pgfsetdash{}{0pt}%
\pgfpathmoveto{\pgfqpoint{1.904526in}{0.417642in}}%
\pgfpathlineto{\pgfqpoint{1.904526in}{1.789039in}}%
\pgfusepath{stroke}%
\end{pgfscope}%
\begin{pgfscope}%
\pgfsetbuttcap%
\pgfsetroundjoin%
\definecolor{currentfill}{rgb}{0.000000,0.000000,0.000000}%
\pgfsetfillcolor{currentfill}%
\pgfsetlinewidth{0.602250pt}%
\definecolor{currentstroke}{rgb}{0.000000,0.000000,0.000000}%
\pgfsetstrokecolor{currentstroke}%
\pgfsetdash{}{0pt}%
\pgfsys@defobject{currentmarker}{\pgfqpoint{0.000000in}{-0.027778in}}{\pgfqpoint{0.000000in}{0.000000in}}{%
\pgfpathmoveto{\pgfqpoint{0.000000in}{0.000000in}}%
\pgfpathlineto{\pgfqpoint{0.000000in}{-0.027778in}}%
\pgfusepath{stroke,fill}%
}%
\begin{pgfscope}%
\pgfsys@transformshift{1.904526in}{0.417642in}%
\pgfsys@useobject{currentmarker}{}%
\end{pgfscope}%
\end{pgfscope}%
\begin{pgfscope}%
\pgfpathrectangle{\pgfqpoint{0.589510in}{0.417642in}}{\pgfqpoint{1.809765in}{1.371397in}}%
\pgfusepath{clip}%
\pgfsetrectcap%
\pgfsetroundjoin%
\pgfsetlinewidth{0.803000pt}%
\definecolor{currentstroke}{rgb}{0.850000,0.850000,0.850000}%
\pgfsetstrokecolor{currentstroke}%
\pgfsetdash{}{0pt}%
\pgfpathmoveto{\pgfqpoint{1.940693in}{0.417642in}}%
\pgfpathlineto{\pgfqpoint{1.940693in}{1.789039in}}%
\pgfusepath{stroke}%
\end{pgfscope}%
\begin{pgfscope}%
\pgfsetbuttcap%
\pgfsetroundjoin%
\definecolor{currentfill}{rgb}{0.000000,0.000000,0.000000}%
\pgfsetfillcolor{currentfill}%
\pgfsetlinewidth{0.602250pt}%
\definecolor{currentstroke}{rgb}{0.000000,0.000000,0.000000}%
\pgfsetstrokecolor{currentstroke}%
\pgfsetdash{}{0pt}%
\pgfsys@defobject{currentmarker}{\pgfqpoint{0.000000in}{-0.027778in}}{\pgfqpoint{0.000000in}{0.000000in}}{%
\pgfpathmoveto{\pgfqpoint{0.000000in}{0.000000in}}%
\pgfpathlineto{\pgfqpoint{0.000000in}{-0.027778in}}%
\pgfusepath{stroke,fill}%
}%
\begin{pgfscope}%
\pgfsys@transformshift{1.940693in}{0.417642in}%
\pgfsys@useobject{currentmarker}{}%
\end{pgfscope}%
\end{pgfscope}%
\begin{pgfscope}%
\pgfpathrectangle{\pgfqpoint{0.589510in}{0.417642in}}{\pgfqpoint{1.809765in}{1.371397in}}%
\pgfusepath{clip}%
\pgfsetrectcap%
\pgfsetroundjoin%
\pgfsetlinewidth{0.803000pt}%
\definecolor{currentstroke}{rgb}{0.850000,0.850000,0.850000}%
\pgfsetstrokecolor{currentstroke}%
\pgfsetdash{}{0pt}%
\pgfpathmoveto{\pgfqpoint{1.971270in}{0.417642in}}%
\pgfpathlineto{\pgfqpoint{1.971270in}{1.789039in}}%
\pgfusepath{stroke}%
\end{pgfscope}%
\begin{pgfscope}%
\pgfsetbuttcap%
\pgfsetroundjoin%
\definecolor{currentfill}{rgb}{0.000000,0.000000,0.000000}%
\pgfsetfillcolor{currentfill}%
\pgfsetlinewidth{0.602250pt}%
\definecolor{currentstroke}{rgb}{0.000000,0.000000,0.000000}%
\pgfsetstrokecolor{currentstroke}%
\pgfsetdash{}{0pt}%
\pgfsys@defobject{currentmarker}{\pgfqpoint{0.000000in}{-0.027778in}}{\pgfqpoint{0.000000in}{0.000000in}}{%
\pgfpathmoveto{\pgfqpoint{0.000000in}{0.000000in}}%
\pgfpathlineto{\pgfqpoint{0.000000in}{-0.027778in}}%
\pgfusepath{stroke,fill}%
}%
\begin{pgfscope}%
\pgfsys@transformshift{1.971270in}{0.417642in}%
\pgfsys@useobject{currentmarker}{}%
\end{pgfscope}%
\end{pgfscope}%
\begin{pgfscope}%
\pgfpathrectangle{\pgfqpoint{0.589510in}{0.417642in}}{\pgfqpoint{1.809765in}{1.371397in}}%
\pgfusepath{clip}%
\pgfsetrectcap%
\pgfsetroundjoin%
\pgfsetlinewidth{0.803000pt}%
\definecolor{currentstroke}{rgb}{0.850000,0.850000,0.850000}%
\pgfsetstrokecolor{currentstroke}%
\pgfsetdash{}{0pt}%
\pgfpathmoveto{\pgfqpoint{1.997758in}{0.417642in}}%
\pgfpathlineto{\pgfqpoint{1.997758in}{1.789039in}}%
\pgfusepath{stroke}%
\end{pgfscope}%
\begin{pgfscope}%
\pgfsetbuttcap%
\pgfsetroundjoin%
\definecolor{currentfill}{rgb}{0.000000,0.000000,0.000000}%
\pgfsetfillcolor{currentfill}%
\pgfsetlinewidth{0.602250pt}%
\definecolor{currentstroke}{rgb}{0.000000,0.000000,0.000000}%
\pgfsetstrokecolor{currentstroke}%
\pgfsetdash{}{0pt}%
\pgfsys@defobject{currentmarker}{\pgfqpoint{0.000000in}{-0.027778in}}{\pgfqpoint{0.000000in}{0.000000in}}{%
\pgfpathmoveto{\pgfqpoint{0.000000in}{0.000000in}}%
\pgfpathlineto{\pgfqpoint{0.000000in}{-0.027778in}}%
\pgfusepath{stroke,fill}%
}%
\begin{pgfscope}%
\pgfsys@transformshift{1.997758in}{0.417642in}%
\pgfsys@useobject{currentmarker}{}%
\end{pgfscope}%
\end{pgfscope}%
\begin{pgfscope}%
\pgfpathrectangle{\pgfqpoint{0.589510in}{0.417642in}}{\pgfqpoint{1.809765in}{1.371397in}}%
\pgfusepath{clip}%
\pgfsetrectcap%
\pgfsetroundjoin%
\pgfsetlinewidth{0.803000pt}%
\definecolor{currentstroke}{rgb}{0.850000,0.850000,0.850000}%
\pgfsetstrokecolor{currentstroke}%
\pgfsetdash{}{0pt}%
\pgfpathmoveto{\pgfqpoint{2.021122in}{0.417642in}}%
\pgfpathlineto{\pgfqpoint{2.021122in}{1.789039in}}%
\pgfusepath{stroke}%
\end{pgfscope}%
\begin{pgfscope}%
\pgfsetbuttcap%
\pgfsetroundjoin%
\definecolor{currentfill}{rgb}{0.000000,0.000000,0.000000}%
\pgfsetfillcolor{currentfill}%
\pgfsetlinewidth{0.602250pt}%
\definecolor{currentstroke}{rgb}{0.000000,0.000000,0.000000}%
\pgfsetstrokecolor{currentstroke}%
\pgfsetdash{}{0pt}%
\pgfsys@defobject{currentmarker}{\pgfqpoint{0.000000in}{-0.027778in}}{\pgfqpoint{0.000000in}{0.000000in}}{%
\pgfpathmoveto{\pgfqpoint{0.000000in}{0.000000in}}%
\pgfpathlineto{\pgfqpoint{0.000000in}{-0.027778in}}%
\pgfusepath{stroke,fill}%
}%
\begin{pgfscope}%
\pgfsys@transformshift{2.021122in}{0.417642in}%
\pgfsys@useobject{currentmarker}{}%
\end{pgfscope}%
\end{pgfscope}%
\begin{pgfscope}%
\pgfpathrectangle{\pgfqpoint{0.589510in}{0.417642in}}{\pgfqpoint{1.809765in}{1.371397in}}%
\pgfusepath{clip}%
\pgfsetrectcap%
\pgfsetroundjoin%
\pgfsetlinewidth{0.803000pt}%
\definecolor{currentstroke}{rgb}{0.850000,0.850000,0.850000}%
\pgfsetstrokecolor{currentstroke}%
\pgfsetdash{}{0pt}%
\pgfpathmoveto{\pgfqpoint{2.179517in}{0.417642in}}%
\pgfpathlineto{\pgfqpoint{2.179517in}{1.789039in}}%
\pgfusepath{stroke}%
\end{pgfscope}%
\begin{pgfscope}%
\pgfsetbuttcap%
\pgfsetroundjoin%
\definecolor{currentfill}{rgb}{0.000000,0.000000,0.000000}%
\pgfsetfillcolor{currentfill}%
\pgfsetlinewidth{0.602250pt}%
\definecolor{currentstroke}{rgb}{0.000000,0.000000,0.000000}%
\pgfsetstrokecolor{currentstroke}%
\pgfsetdash{}{0pt}%
\pgfsys@defobject{currentmarker}{\pgfqpoint{0.000000in}{-0.027778in}}{\pgfqpoint{0.000000in}{0.000000in}}{%
\pgfpathmoveto{\pgfqpoint{0.000000in}{0.000000in}}%
\pgfpathlineto{\pgfqpoint{0.000000in}{-0.027778in}}%
\pgfusepath{stroke,fill}%
}%
\begin{pgfscope}%
\pgfsys@transformshift{2.179517in}{0.417642in}%
\pgfsys@useobject{currentmarker}{}%
\end{pgfscope}%
\end{pgfscope}%
\begin{pgfscope}%
\pgfpathrectangle{\pgfqpoint{0.589510in}{0.417642in}}{\pgfqpoint{1.809765in}{1.371397in}}%
\pgfusepath{clip}%
\pgfsetrectcap%
\pgfsetroundjoin%
\pgfsetlinewidth{0.803000pt}%
\definecolor{currentstroke}{rgb}{0.850000,0.850000,0.850000}%
\pgfsetstrokecolor{currentstroke}%
\pgfsetdash{}{0pt}%
\pgfpathmoveto{\pgfqpoint{2.259947in}{0.417642in}}%
\pgfpathlineto{\pgfqpoint{2.259947in}{1.789039in}}%
\pgfusepath{stroke}%
\end{pgfscope}%
\begin{pgfscope}%
\pgfsetbuttcap%
\pgfsetroundjoin%
\definecolor{currentfill}{rgb}{0.000000,0.000000,0.000000}%
\pgfsetfillcolor{currentfill}%
\pgfsetlinewidth{0.602250pt}%
\definecolor{currentstroke}{rgb}{0.000000,0.000000,0.000000}%
\pgfsetstrokecolor{currentstroke}%
\pgfsetdash{}{0pt}%
\pgfsys@defobject{currentmarker}{\pgfqpoint{0.000000in}{-0.027778in}}{\pgfqpoint{0.000000in}{0.000000in}}{%
\pgfpathmoveto{\pgfqpoint{0.000000in}{0.000000in}}%
\pgfpathlineto{\pgfqpoint{0.000000in}{-0.027778in}}%
\pgfusepath{stroke,fill}%
}%
\begin{pgfscope}%
\pgfsys@transformshift{2.259947in}{0.417642in}%
\pgfsys@useobject{currentmarker}{}%
\end{pgfscope}%
\end{pgfscope}%
\begin{pgfscope}%
\pgfpathrectangle{\pgfqpoint{0.589510in}{0.417642in}}{\pgfqpoint{1.809765in}{1.371397in}}%
\pgfusepath{clip}%
\pgfsetrectcap%
\pgfsetroundjoin%
\pgfsetlinewidth{0.803000pt}%
\definecolor{currentstroke}{rgb}{0.850000,0.850000,0.850000}%
\pgfsetstrokecolor{currentstroke}%
\pgfsetdash{}{0pt}%
\pgfpathmoveto{\pgfqpoint{2.317013in}{0.417642in}}%
\pgfpathlineto{\pgfqpoint{2.317013in}{1.789039in}}%
\pgfusepath{stroke}%
\end{pgfscope}%
\begin{pgfscope}%
\pgfsetbuttcap%
\pgfsetroundjoin%
\definecolor{currentfill}{rgb}{0.000000,0.000000,0.000000}%
\pgfsetfillcolor{currentfill}%
\pgfsetlinewidth{0.602250pt}%
\definecolor{currentstroke}{rgb}{0.000000,0.000000,0.000000}%
\pgfsetstrokecolor{currentstroke}%
\pgfsetdash{}{0pt}%
\pgfsys@defobject{currentmarker}{\pgfqpoint{0.000000in}{-0.027778in}}{\pgfqpoint{0.000000in}{0.000000in}}{%
\pgfpathmoveto{\pgfqpoint{0.000000in}{0.000000in}}%
\pgfpathlineto{\pgfqpoint{0.000000in}{-0.027778in}}%
\pgfusepath{stroke,fill}%
}%
\begin{pgfscope}%
\pgfsys@transformshift{2.317013in}{0.417642in}%
\pgfsys@useobject{currentmarker}{}%
\end{pgfscope}%
\end{pgfscope}%
\begin{pgfscope}%
\pgfpathrectangle{\pgfqpoint{0.589510in}{0.417642in}}{\pgfqpoint{1.809765in}{1.371397in}}%
\pgfusepath{clip}%
\pgfsetrectcap%
\pgfsetroundjoin%
\pgfsetlinewidth{0.803000pt}%
\definecolor{currentstroke}{rgb}{0.850000,0.850000,0.850000}%
\pgfsetstrokecolor{currentstroke}%
\pgfsetdash{}{0pt}%
\pgfpathmoveto{\pgfqpoint{2.361277in}{0.417642in}}%
\pgfpathlineto{\pgfqpoint{2.361277in}{1.789039in}}%
\pgfusepath{stroke}%
\end{pgfscope}%
\begin{pgfscope}%
\pgfsetbuttcap%
\pgfsetroundjoin%
\definecolor{currentfill}{rgb}{0.000000,0.000000,0.000000}%
\pgfsetfillcolor{currentfill}%
\pgfsetlinewidth{0.602250pt}%
\definecolor{currentstroke}{rgb}{0.000000,0.000000,0.000000}%
\pgfsetstrokecolor{currentstroke}%
\pgfsetdash{}{0pt}%
\pgfsys@defobject{currentmarker}{\pgfqpoint{0.000000in}{-0.027778in}}{\pgfqpoint{0.000000in}{0.000000in}}{%
\pgfpathmoveto{\pgfqpoint{0.000000in}{0.000000in}}%
\pgfpathlineto{\pgfqpoint{0.000000in}{-0.027778in}}%
\pgfusepath{stroke,fill}%
}%
\begin{pgfscope}%
\pgfsys@transformshift{2.361277in}{0.417642in}%
\pgfsys@useobject{currentmarker}{}%
\end{pgfscope}%
\end{pgfscope}%
\begin{pgfscope}%
\pgfpathrectangle{\pgfqpoint{0.589510in}{0.417642in}}{\pgfqpoint{1.809765in}{1.371397in}}%
\pgfusepath{clip}%
\pgfsetrectcap%
\pgfsetroundjoin%
\pgfsetlinewidth{0.803000pt}%
\definecolor{currentstroke}{rgb}{0.850000,0.850000,0.850000}%
\pgfsetstrokecolor{currentstroke}%
\pgfsetdash{}{0pt}%
\pgfpathmoveto{\pgfqpoint{2.397443in}{0.417642in}}%
\pgfpathlineto{\pgfqpoint{2.397443in}{1.789039in}}%
\pgfusepath{stroke}%
\end{pgfscope}%
\begin{pgfscope}%
\pgfsetbuttcap%
\pgfsetroundjoin%
\definecolor{currentfill}{rgb}{0.000000,0.000000,0.000000}%
\pgfsetfillcolor{currentfill}%
\pgfsetlinewidth{0.602250pt}%
\definecolor{currentstroke}{rgb}{0.000000,0.000000,0.000000}%
\pgfsetstrokecolor{currentstroke}%
\pgfsetdash{}{0pt}%
\pgfsys@defobject{currentmarker}{\pgfqpoint{0.000000in}{-0.027778in}}{\pgfqpoint{0.000000in}{0.000000in}}{%
\pgfpathmoveto{\pgfqpoint{0.000000in}{0.000000in}}%
\pgfpathlineto{\pgfqpoint{0.000000in}{-0.027778in}}%
\pgfusepath{stroke,fill}%
}%
\begin{pgfscope}%
\pgfsys@transformshift{2.397443in}{0.417642in}%
\pgfsys@useobject{currentmarker}{}%
\end{pgfscope}%
\end{pgfscope}%
\begin{pgfscope}%
\definecolor{textcolor}{rgb}{0.000000,0.000000,0.000000}%
\pgfsetstrokecolor{textcolor}%
\pgfsetfillcolor{textcolor}%
\pgftext[x=1.494392in,y=0.165003in,,top]{\color{textcolor}\rmfamily\fontsize{10.000000}{12.000000}\selectfont \(\displaystyle \tau\) in \unit{\second}}%
\end{pgfscope}%
\begin{pgfscope}%
\pgfpathrectangle{\pgfqpoint{0.589510in}{0.417642in}}{\pgfqpoint{1.809765in}{1.371397in}}%
\pgfusepath{clip}%
\pgfsetrectcap%
\pgfsetroundjoin%
\pgfsetlinewidth{0.803000pt}%
\definecolor{currentstroke}{rgb}{0.450000,0.450000,0.450000}%
\pgfsetstrokecolor{currentstroke}%
\pgfsetdash{}{0pt}%
\pgfpathmoveto{\pgfqpoint{0.589510in}{0.417642in}}%
\pgfpathlineto{\pgfqpoint{2.399275in}{0.417642in}}%
\pgfusepath{stroke}%
\end{pgfscope}%
\begin{pgfscope}%
\pgfsetbuttcap%
\pgfsetroundjoin%
\definecolor{currentfill}{rgb}{0.000000,0.000000,0.000000}%
\pgfsetfillcolor{currentfill}%
\pgfsetlinewidth{0.803000pt}%
\definecolor{currentstroke}{rgb}{0.000000,0.000000,0.000000}%
\pgfsetstrokecolor{currentstroke}%
\pgfsetdash{}{0pt}%
\pgfsys@defobject{currentmarker}{\pgfqpoint{-0.048611in}{0.000000in}}{\pgfqpoint{-0.000000in}{0.000000in}}{%
\pgfpathmoveto{\pgfqpoint{-0.000000in}{0.000000in}}%
\pgfpathlineto{\pgfqpoint{-0.048611in}{0.000000in}}%
\pgfusepath{stroke,fill}%
}%
\begin{pgfscope}%
\pgfsys@transformshift{0.589510in}{0.417642in}%
\pgfsys@useobject{currentmarker}{}%
\end{pgfscope}%
\end{pgfscope}%
\begin{pgfscope}%
\definecolor{textcolor}{rgb}{0.000000,0.000000,0.000000}%
\pgfsetstrokecolor{textcolor}%
\pgfsetfillcolor{textcolor}%
\pgftext[x=0.236114in, y=0.378489in, left, base]{\color{textcolor}\rmfamily\fontsize{8.000000}{9.600000}\selectfont \(\displaystyle {10^{-2}}\)}%
\end{pgfscope}%
\begin{pgfscope}%
\pgfpathrectangle{\pgfqpoint{0.589510in}{0.417642in}}{\pgfqpoint{1.809765in}{1.371397in}}%
\pgfusepath{clip}%
\pgfsetrectcap%
\pgfsetroundjoin%
\pgfsetlinewidth{0.803000pt}%
\definecolor{currentstroke}{rgb}{0.450000,0.450000,0.450000}%
\pgfsetstrokecolor{currentstroke}%
\pgfsetdash{}{0pt}%
\pgfpathmoveto{\pgfqpoint{0.589510in}{0.827077in}}%
\pgfpathlineto{\pgfqpoint{2.399275in}{0.827077in}}%
\pgfusepath{stroke}%
\end{pgfscope}%
\begin{pgfscope}%
\pgfsetbuttcap%
\pgfsetroundjoin%
\definecolor{currentfill}{rgb}{0.000000,0.000000,0.000000}%
\pgfsetfillcolor{currentfill}%
\pgfsetlinewidth{0.803000pt}%
\definecolor{currentstroke}{rgb}{0.000000,0.000000,0.000000}%
\pgfsetstrokecolor{currentstroke}%
\pgfsetdash{}{0pt}%
\pgfsys@defobject{currentmarker}{\pgfqpoint{-0.048611in}{0.000000in}}{\pgfqpoint{-0.000000in}{0.000000in}}{%
\pgfpathmoveto{\pgfqpoint{-0.000000in}{0.000000in}}%
\pgfpathlineto{\pgfqpoint{-0.048611in}{0.000000in}}%
\pgfusepath{stroke,fill}%
}%
\begin{pgfscope}%
\pgfsys@transformshift{0.589510in}{0.827077in}%
\pgfsys@useobject{currentmarker}{}%
\end{pgfscope}%
\end{pgfscope}%
\begin{pgfscope}%
\definecolor{textcolor}{rgb}{0.000000,0.000000,0.000000}%
\pgfsetstrokecolor{textcolor}%
\pgfsetfillcolor{textcolor}%
\pgftext[x=0.316361in, y=0.787924in, left, base]{\color{textcolor}\rmfamily\fontsize{8.000000}{9.600000}\selectfont \(\displaystyle {10^{0}}\)}%
\end{pgfscope}%
\begin{pgfscope}%
\pgfpathrectangle{\pgfqpoint{0.589510in}{0.417642in}}{\pgfqpoint{1.809765in}{1.371397in}}%
\pgfusepath{clip}%
\pgfsetrectcap%
\pgfsetroundjoin%
\pgfsetlinewidth{0.803000pt}%
\definecolor{currentstroke}{rgb}{0.450000,0.450000,0.450000}%
\pgfsetstrokecolor{currentstroke}%
\pgfsetdash{}{0pt}%
\pgfpathmoveto{\pgfqpoint{0.589510in}{1.236512in}}%
\pgfpathlineto{\pgfqpoint{2.399275in}{1.236512in}}%
\pgfusepath{stroke}%
\end{pgfscope}%
\begin{pgfscope}%
\pgfsetbuttcap%
\pgfsetroundjoin%
\definecolor{currentfill}{rgb}{0.000000,0.000000,0.000000}%
\pgfsetfillcolor{currentfill}%
\pgfsetlinewidth{0.803000pt}%
\definecolor{currentstroke}{rgb}{0.000000,0.000000,0.000000}%
\pgfsetstrokecolor{currentstroke}%
\pgfsetdash{}{0pt}%
\pgfsys@defobject{currentmarker}{\pgfqpoint{-0.048611in}{0.000000in}}{\pgfqpoint{-0.000000in}{0.000000in}}{%
\pgfpathmoveto{\pgfqpoint{-0.000000in}{0.000000in}}%
\pgfpathlineto{\pgfqpoint{-0.048611in}{0.000000in}}%
\pgfusepath{stroke,fill}%
}%
\begin{pgfscope}%
\pgfsys@transformshift{0.589510in}{1.236512in}%
\pgfsys@useobject{currentmarker}{}%
\end{pgfscope}%
\end{pgfscope}%
\begin{pgfscope}%
\definecolor{textcolor}{rgb}{0.000000,0.000000,0.000000}%
\pgfsetstrokecolor{textcolor}%
\pgfsetfillcolor{textcolor}%
\pgftext[x=0.316361in, y=1.197359in, left, base]{\color{textcolor}\rmfamily\fontsize{8.000000}{9.600000}\selectfont \(\displaystyle {10^{2}}\)}%
\end{pgfscope}%
\begin{pgfscope}%
\pgfpathrectangle{\pgfqpoint{0.589510in}{0.417642in}}{\pgfqpoint{1.809765in}{1.371397in}}%
\pgfusepath{clip}%
\pgfsetrectcap%
\pgfsetroundjoin%
\pgfsetlinewidth{0.803000pt}%
\definecolor{currentstroke}{rgb}{0.450000,0.450000,0.450000}%
\pgfsetstrokecolor{currentstroke}%
\pgfsetdash{}{0pt}%
\pgfpathmoveto{\pgfqpoint{0.589510in}{1.645947in}}%
\pgfpathlineto{\pgfqpoint{2.399275in}{1.645947in}}%
\pgfusepath{stroke}%
\end{pgfscope}%
\begin{pgfscope}%
\pgfsetbuttcap%
\pgfsetroundjoin%
\definecolor{currentfill}{rgb}{0.000000,0.000000,0.000000}%
\pgfsetfillcolor{currentfill}%
\pgfsetlinewidth{0.803000pt}%
\definecolor{currentstroke}{rgb}{0.000000,0.000000,0.000000}%
\pgfsetstrokecolor{currentstroke}%
\pgfsetdash{}{0pt}%
\pgfsys@defobject{currentmarker}{\pgfqpoint{-0.048611in}{0.000000in}}{\pgfqpoint{-0.000000in}{0.000000in}}{%
\pgfpathmoveto{\pgfqpoint{-0.000000in}{0.000000in}}%
\pgfpathlineto{\pgfqpoint{-0.048611in}{0.000000in}}%
\pgfusepath{stroke,fill}%
}%
\begin{pgfscope}%
\pgfsys@transformshift{0.589510in}{1.645947in}%
\pgfsys@useobject{currentmarker}{}%
\end{pgfscope}%
\end{pgfscope}%
\begin{pgfscope}%
\definecolor{textcolor}{rgb}{0.000000,0.000000,0.000000}%
\pgfsetstrokecolor{textcolor}%
\pgfsetfillcolor{textcolor}%
\pgftext[x=0.316361in, y=1.606795in, left, base]{\color{textcolor}\rmfamily\fontsize{8.000000}{9.600000}\selectfont \(\displaystyle {10^{4}}\)}%
\end{pgfscope}%
\begin{pgfscope}%
\definecolor{textcolor}{rgb}{0.000000,0.000000,0.000000}%
\pgfsetstrokecolor{textcolor}%
\pgfsetfillcolor{textcolor}%
\pgftext[x=0.180559in,y=1.103340in,,bottom,rotate=90.000000]{\color{textcolor}\rmfamily\fontsize{10.000000}{12.000000}\selectfont ADEV \(\displaystyle \sigma_A(\tau)\)}%
\end{pgfscope}%
\begin{pgfscope}%
\pgfpathrectangle{\pgfqpoint{0.589510in}{0.417642in}}{\pgfqpoint{1.809765in}{1.371397in}}%
\pgfusepath{clip}%
\pgfsetbuttcap%
\pgfsetroundjoin%
\pgfsetlinewidth{1.505625pt}%
\definecolor{currentstroke}{rgb}{0.800000,0.470588,0.737255}%
\pgfsetstrokecolor{currentstroke}%
\pgfsetdash{{5.550000pt}{2.400000pt}}{0.000000pt}%
\pgfpathmoveto{\pgfqpoint{0.671772in}{0.827077in}}%
\pgfpathlineto{\pgfqpoint{0.809267in}{0.888703in}}%
\pgfpathlineto{\pgfqpoint{0.946763in}{0.950329in}}%
\pgfpathlineto{\pgfqpoint{1.128522in}{1.031795in}}%
\pgfpathlineto{\pgfqpoint{1.266017in}{1.093421in}}%
\pgfpathlineto{\pgfqpoint{1.403513in}{1.155047in}}%
\pgfpathlineto{\pgfqpoint{1.585272in}{1.236512in}}%
\pgfpathlineto{\pgfqpoint{1.722767in}{1.298138in}}%
\pgfpathlineto{\pgfqpoint{1.860263in}{1.359764in}}%
\pgfpathlineto{\pgfqpoint{2.042022in}{1.441230in}}%
\pgfpathlineto{\pgfqpoint{2.179517in}{1.502856in}}%
\pgfpathlineto{\pgfqpoint{2.317013in}{1.564482in}}%
\pgfusepath{stroke}%
\end{pgfscope}%
\begin{pgfscope}%
\pgfpathrectangle{\pgfqpoint{0.589510in}{0.417642in}}{\pgfqpoint{1.809765in}{1.371397in}}%
\pgfusepath{clip}%
\pgfsetbuttcap%
\pgfsetroundjoin%
\definecolor{currentfill}{rgb}{0.800000,0.470588,0.737255}%
\pgfsetfillcolor{currentfill}%
\pgfsetlinewidth{1.003750pt}%
\definecolor{currentstroke}{rgb}{0.800000,0.470588,0.737255}%
\pgfsetstrokecolor{currentstroke}%
\pgfsetdash{}{0pt}%
\pgfsys@defobject{currentmarker}{\pgfqpoint{-0.020833in}{-0.020833in}}{\pgfqpoint{0.020833in}{0.020833in}}{%
\pgfpathmoveto{\pgfqpoint{0.000000in}{-0.020833in}}%
\pgfpathcurveto{\pgfqpoint{0.005525in}{-0.020833in}}{\pgfqpoint{0.010825in}{-0.018638in}}{\pgfqpoint{0.014731in}{-0.014731in}}%
\pgfpathcurveto{\pgfqpoint{0.018638in}{-0.010825in}}{\pgfqpoint{0.020833in}{-0.005525in}}{\pgfqpoint{0.020833in}{0.000000in}}%
\pgfpathcurveto{\pgfqpoint{0.020833in}{0.005525in}}{\pgfqpoint{0.018638in}{0.010825in}}{\pgfqpoint{0.014731in}{0.014731in}}%
\pgfpathcurveto{\pgfqpoint{0.010825in}{0.018638in}}{\pgfqpoint{0.005525in}{0.020833in}}{\pgfqpoint{0.000000in}{0.020833in}}%
\pgfpathcurveto{\pgfqpoint{-0.005525in}{0.020833in}}{\pgfqpoint{-0.010825in}{0.018638in}}{\pgfqpoint{-0.014731in}{0.014731in}}%
\pgfpathcurveto{\pgfqpoint{-0.018638in}{0.010825in}}{\pgfqpoint{-0.020833in}{0.005525in}}{\pgfqpoint{-0.020833in}{0.000000in}}%
\pgfpathcurveto{\pgfqpoint{-0.020833in}{-0.005525in}}{\pgfqpoint{-0.018638in}{-0.010825in}}{\pgfqpoint{-0.014731in}{-0.014731in}}%
\pgfpathcurveto{\pgfqpoint{-0.010825in}{-0.018638in}}{\pgfqpoint{-0.005525in}{-0.020833in}}{\pgfqpoint{0.000000in}{-0.020833in}}%
\pgfpathlineto{\pgfqpoint{0.000000in}{-0.020833in}}%
\pgfpathclose%
\pgfusepath{stroke,fill}%
}%
\begin{pgfscope}%
\pgfsys@transformshift{0.671772in}{0.827077in}%
\pgfsys@useobject{currentmarker}{}%
\end{pgfscope}%
\begin{pgfscope}%
\pgfsys@transformshift{0.809267in}{0.888703in}%
\pgfsys@useobject{currentmarker}{}%
\end{pgfscope}%
\begin{pgfscope}%
\pgfsys@transformshift{0.946763in}{0.950329in}%
\pgfsys@useobject{currentmarker}{}%
\end{pgfscope}%
\begin{pgfscope}%
\pgfsys@transformshift{1.128522in}{1.031795in}%
\pgfsys@useobject{currentmarker}{}%
\end{pgfscope}%
\begin{pgfscope}%
\pgfsys@transformshift{1.266017in}{1.093421in}%
\pgfsys@useobject{currentmarker}{}%
\end{pgfscope}%
\begin{pgfscope}%
\pgfsys@transformshift{1.403513in}{1.155047in}%
\pgfsys@useobject{currentmarker}{}%
\end{pgfscope}%
\begin{pgfscope}%
\pgfsys@transformshift{1.585272in}{1.236512in}%
\pgfsys@useobject{currentmarker}{}%
\end{pgfscope}%
\begin{pgfscope}%
\pgfsys@transformshift{1.722767in}{1.298138in}%
\pgfsys@useobject{currentmarker}{}%
\end{pgfscope}%
\begin{pgfscope}%
\pgfsys@transformshift{1.860263in}{1.359764in}%
\pgfsys@useobject{currentmarker}{}%
\end{pgfscope}%
\begin{pgfscope}%
\pgfsys@transformshift{2.042022in}{1.441230in}%
\pgfsys@useobject{currentmarker}{}%
\end{pgfscope}%
\begin{pgfscope}%
\pgfsys@transformshift{2.179517in}{1.502856in}%
\pgfsys@useobject{currentmarker}{}%
\end{pgfscope}%
\begin{pgfscope}%
\pgfsys@transformshift{2.317013in}{1.564482in}%
\pgfsys@useobject{currentmarker}{}%
\end{pgfscope}%
\end{pgfscope}%
\begin{pgfscope}%
\pgfsetrectcap%
\pgfsetmiterjoin%
\pgfsetlinewidth{0.803000pt}%
\definecolor{currentstroke}{rgb}{0.000000,0.000000,0.000000}%
\pgfsetstrokecolor{currentstroke}%
\pgfsetdash{}{0pt}%
\pgfpathmoveto{\pgfqpoint{0.589510in}{0.417642in}}%
\pgfpathlineto{\pgfqpoint{0.589510in}{1.789039in}}%
\pgfusepath{stroke}%
\end{pgfscope}%
\begin{pgfscope}%
\pgfsetrectcap%
\pgfsetmiterjoin%
\pgfsetlinewidth{0.803000pt}%
\definecolor{currentstroke}{rgb}{0.000000,0.000000,0.000000}%
\pgfsetstrokecolor{currentstroke}%
\pgfsetdash{}{0pt}%
\pgfpathmoveto{\pgfqpoint{2.399275in}{0.417642in}}%
\pgfpathlineto{\pgfqpoint{2.399275in}{1.789039in}}%
\pgfusepath{stroke}%
\end{pgfscope}%
\begin{pgfscope}%
\pgfsetrectcap%
\pgfsetmiterjoin%
\pgfsetlinewidth{0.803000pt}%
\definecolor{currentstroke}{rgb}{0.000000,0.000000,0.000000}%
\pgfsetstrokecolor{currentstroke}%
\pgfsetdash{}{0pt}%
\pgfpathmoveto{\pgfqpoint{0.589510in}{0.417642in}}%
\pgfpathlineto{\pgfqpoint{2.399275in}{0.417642in}}%
\pgfusepath{stroke}%
\end{pgfscope}%
\begin{pgfscope}%
\pgfsetrectcap%
\pgfsetmiterjoin%
\pgfsetlinewidth{0.803000pt}%
\definecolor{currentstroke}{rgb}{0.000000,0.000000,0.000000}%
\pgfsetstrokecolor{currentstroke}%
\pgfsetdash{}{0pt}%
\pgfpathmoveto{\pgfqpoint{0.589510in}{1.789039in}}%
\pgfpathlineto{\pgfqpoint{2.399275in}{1.789039in}}%
\pgfusepath{stroke}%
\end{pgfscope}%
\begin{pgfscope}%
\pgfsetbuttcap%
\pgfsetmiterjoin%
\definecolor{currentfill}{rgb}{1.000000,1.000000,1.000000}%
\pgfsetfillcolor{currentfill}%
\pgfsetfillopacity{0.800000}%
\pgfsetlinewidth{1.003750pt}%
\definecolor{currentstroke}{rgb}{0.800000,0.800000,0.800000}%
\pgfsetstrokecolor{currentstroke}%
\pgfsetstrokeopacity{0.800000}%
\pgfsetdash{}{0pt}%
\pgfpathmoveto{\pgfqpoint{0.667288in}{1.544733in}}%
\pgfpathlineto{\pgfqpoint{1.448613in}{1.544733in}}%
\pgfpathquadraticcurveto{\pgfqpoint{1.470835in}{1.544733in}}{\pgfqpoint{1.470835in}{1.566956in}}%
\pgfpathlineto{\pgfqpoint{1.470835in}{1.711261in}}%
\pgfpathquadraticcurveto{\pgfqpoint{1.470835in}{1.733483in}}{\pgfqpoint{1.448613in}{1.733483in}}%
\pgfpathlineto{\pgfqpoint{0.667288in}{1.733483in}}%
\pgfpathquadraticcurveto{\pgfqpoint{0.645065in}{1.733483in}}{\pgfqpoint{0.645065in}{1.711261in}}%
\pgfpathlineto{\pgfqpoint{0.645065in}{1.566956in}}%
\pgfpathquadraticcurveto{\pgfqpoint{0.645065in}{1.544733in}}{\pgfqpoint{0.667288in}{1.544733in}}%
\pgfpathlineto{\pgfqpoint{0.667288in}{1.544733in}}%
\pgfpathclose%
\pgfusepath{stroke,fill}%
\end{pgfscope}%
\begin{pgfscope}%
\pgfsetbuttcap%
\pgfsetroundjoin%
\pgfsetlinewidth{1.505625pt}%
\definecolor{currentstroke}{rgb}{0.800000,0.470588,0.737255}%
\pgfsetstrokecolor{currentstroke}%
\pgfsetdash{{5.550000pt}{2.400000pt}}{0.000000pt}%
\pgfpathmoveto{\pgfqpoint{0.689510in}{1.649622in}}%
\pgfpathlineto{\pgfqpoint{0.800621in}{1.649622in}}%
\pgfpathlineto{\pgfqpoint{0.911732in}{1.649622in}}%
\pgfusepath{stroke}%
\end{pgfscope}%
\begin{pgfscope}%
\definecolor{textcolor}{rgb}{0.000000,0.000000,0.000000}%
\pgfsetstrokecolor{textcolor}%
\pgfsetfillcolor{textcolor}%
\pgftext[x=1.000621in,y=1.610733in,left,base]{\color{textcolor}\rmfamily\fontsize{8.000000}{9.600000}\selectfont \(\displaystyle \propto D\tau^{+1}\)}%
\end{pgfscope}%
\end{pgfpicture}%
\makeatother%
\endgroup%
% data/simulations/sim_allan_variance.py
        } % scalebox
        \caption{Allan deviation}
        \label{fig:drift_adev}
    \end{subfigure}
    \caption{Different representations of linear drift.}
    \label{fig:drift_noise_simulated}
\end{figure}

\subsubsection{Dead Time}%
\label{sec:dead_time}
The coefficients given in the previous examples were derived using the assumption that all samples in a measurement are continuous with a dead time $\theta = 0$. Unfortunately, measurements sometimes have a dead time that is non negligible. This problem was extensively discussed by \citeauthor{psd_to_adev} \cite{psd_to_adev}. \citeauthor{adev_frequency_counter} even developed special models to account for the algorithms of modern frequency counters \cite{adev_frequency_counter}. While some frequency counters support gapless measurements, the situation is entirely different for digitizers and digital multimeters. Several settings commonly used affect the dead time, which can be considerable. It is therefore important to discuss typical measurement settings for voltmeters to estimate the errors that arise from those settings. The focus of this discussion lies on the dead time introduced by digital multimeters, but the application is not limited to this field.

The most commonly used settings that affect the dead time of a voltmeter are autozeroing and line synchronization. Autozeroing is done by adding additional measurements to the normal input integration cycle. To correct for the zero offset drift a zero measurement is added where the ADC is switched to the low terminal. Additionally, some devices add a reading of the reference voltage to correct for gain errors. The implementation details and type of measurements are manufacturer dependent.

The other setting, which can be enabled in voltmeters, is the line synchronization to increase the noise rejection of the instrument. This setting synchronizes the start of a measurement to the zero crossing of the power line. Depending on the instrument, this might cause a delay of one power line cycle (PLC) after each measurement if the instrument is not capable of processing the previous measurement while at the same time recording another one.

A simple measurement with dead time is shown in figure \ref{fig:allan_variance_definitions} on page \pageref{fig:allan_variance_definitions}. That model assumes that the dead time is constant and is always added after the actual integration time $\tau$. This is rarely true for real measurement data as many devices and even ADCs use internal averaging and autozeroing to produce a measurement. The actual dead time is therefore spread over the whole measurement and not limited to the end of the measurement. An example is the Keysight \device{3458A} DMM that automatically switches to averaging when selecting integration times greater than \qty{10}{\plc}. The reason is simple as for longer integration times, more and more flicker noise starts contributing to the measurement. The measurement is therefore split into single measurements of \qty{10}{\plc} and, using autozeroing, the flicker noise is suppressed. This is discussed in more detail as an example in section \ref{sec:autozero}. The mathematical problem of a distributed dead time was already noted by \citeauthor{adev_noise_types} \cite{adev_noise_types} and it is distinctively different from the calculations made by \citeauthor{psd_to_adev} \cite{psd_to_adev} for a single dead time at the end of the measurement. The exact mathematical treatment is complex and is beyond the scope of this work, especially considering that autozeroing does a lot more than just adding dead time at the end of the measurement. Fortunately, using a few assumptions, the problem can be greatly simplified.

An interesting observation can be made for white noise. Since it is uncorrelated, it makes no difference whether it is sampled in full, or only partially, and therefore the Allan deviation for a white noise process with or without dead time is the same:
\begin{equation}
    \sigma^2(N,T, \tau) = \sigma^2(N=2,T=\tau, \tau) = \sigma_A^2(\tau) \frac 1 2 h_0 \tau^{-1}
\end{equation}

Consequently, if the dead time is added at a frequency high enough, so that the input amplifier output is dominated by white noise, the dead time will have no influence on the Allan variance.

Finally, \citeauthor{psd_to_adev} \cite{psd_to_adev} notes that for measurement durations or averaging times $T \gg T_0$, the Allan variance with respect to $T$ shows an asymptotic behaviour of $\sigma_A^2(T) \to \sigma_A^2(\tau)$.

\subsection{Example}%
\label{sec:noise_example}
Using the results from the previous sections, it is possible to simulate a typical measurement sample containing white noise, flicker noise and random walk behaviour. The simulation was written in Python using the \textit{AllanTools} library \cite{allantools} to generate the time domain data, which was then converted to a power spectrum using the algorithm developed by \citeauthor{welch} \cite{welch} to estimate the power spectral density. The Allan deviation was calculated using the \textit{AllanTools} library. The full Python source is available at \cite{supplemental_material} and found in \external{data/simulations/sim\_allan\_variance\_example.py}. The time domain data shown here were downsampled from $2^{25}$ data points to \num{2000} points for faster plotting using the Largest-Triangle-Three-Buckets (LTTB) algorithm created by \citeauthor{lttb} \cite{lttb} and also available as a Python library. The downsampling algorithm chosen is optimal for this application because it aims to visually keep the result the same by favouring parts of the data where there is more dynamics. The only difference noticeable to the author is that the edges of the white noise plot are a slightly rougher. The full data set can be obtained using the source code given above if one desires. The power spectrum and the Allan deviation were always calculated from the full dataset. The data of the power spectrum were additionally binned to be evenly spaced on a logarithmic scale. This considerably reduced the high frequency noise and made the plot easier to read while not negatively impacting its shape.
\begin{figure}[ht]
    \centering
    \begin{subfigure}{0.32\linewidth}
        \centering
        \scalebox{0.75}{%
            %% Creator: Matplotlib, PGF backend
%%
%% To include the figure in your LaTeX document, write
%%   \input{<filename>.pgf}
%%
%% Make sure the required packages are loaded in your preamble
%%   \usepackage{pgf}
%%
%% Also ensure that all the required font packages are loaded; for instance,
%% the lmodern package is sometimes necessary when using math font.
%%   \usepackage{lmodern}
%%
%% Figures using additional raster images can only be included by \input if
%% they are in the same directory as the main LaTeX file. For loading figures
%% from other directories you can use the `import` package
%%   \usepackage{import}
%%
%% and then include the figures with
%%   \import{<path to file>}{<filename>.pgf}
%%
%% Matplotlib used the following preamble
%%   \usepackage{siunitx}
%%   \usepackage{fontspec}
%%   \makeatletter\@ifpackageloaded{underscore}{}{\usepackage[strings]{underscore}}\makeatother
%%
\begingroup%
\makeatletter%
\begin{pgfpicture}%
\pgfpathrectangle{\pgfpointorigin}{\pgfqpoint{2.440945in}{1.830709in}}%
\pgfusepath{use as bounding box, clip}%
\begin{pgfscope}%
\pgfsetbuttcap%
\pgfsetmiterjoin%
\definecolor{currentfill}{rgb}{1.000000,1.000000,1.000000}%
\pgfsetfillcolor{currentfill}%
\pgfsetlinewidth{0.000000pt}%
\definecolor{currentstroke}{rgb}{1.000000,1.000000,1.000000}%
\pgfsetstrokecolor{currentstroke}%
\pgfsetdash{}{0pt}%
\pgfpathmoveto{\pgfqpoint{0.000000in}{0.000000in}}%
\pgfpathlineto{\pgfqpoint{2.440945in}{0.000000in}}%
\pgfpathlineto{\pgfqpoint{2.440945in}{1.830709in}}%
\pgfpathlineto{\pgfqpoint{0.000000in}{1.830709in}}%
\pgfpathlineto{\pgfqpoint{0.000000in}{0.000000in}}%
\pgfpathclose%
\pgfusepath{fill}%
\end{pgfscope}%
\begin{pgfscope}%
\pgfsetbuttcap%
\pgfsetmiterjoin%
\definecolor{currentfill}{rgb}{1.000000,1.000000,1.000000}%
\pgfsetfillcolor{currentfill}%
\pgfsetlinewidth{0.000000pt}%
\definecolor{currentstroke}{rgb}{0.000000,0.000000,0.000000}%
\pgfsetstrokecolor{currentstroke}%
\pgfsetstrokeopacity{0.000000}%
\pgfsetdash{}{0pt}%
\pgfpathmoveto{\pgfqpoint{0.589745in}{0.416447in}}%
\pgfpathlineto{\pgfqpoint{2.399275in}{0.416447in}}%
\pgfpathlineto{\pgfqpoint{2.399275in}{1.789039in}}%
\pgfpathlineto{\pgfqpoint{0.589745in}{1.789039in}}%
\pgfpathlineto{\pgfqpoint{0.589745in}{0.416447in}}%
\pgfpathclose%
\pgfusepath{fill}%
\end{pgfscope}%
\begin{pgfscope}%
\pgfpathrectangle{\pgfqpoint{0.589745in}{0.416447in}}{\pgfqpoint{1.809530in}{1.372591in}}%
\pgfusepath{clip}%
\pgfsetrectcap%
\pgfsetroundjoin%
\pgfsetlinewidth{0.803000pt}%
\definecolor{currentstroke}{rgb}{0.450000,0.450000,0.450000}%
\pgfsetstrokecolor{currentstroke}%
\pgfsetdash{}{0pt}%
\pgfpathmoveto{\pgfqpoint{0.671996in}{0.416447in}}%
\pgfpathlineto{\pgfqpoint{0.671996in}{1.789039in}}%
\pgfusepath{stroke}%
\end{pgfscope}%
\begin{pgfscope}%
\pgfsetbuttcap%
\pgfsetroundjoin%
\definecolor{currentfill}{rgb}{0.000000,0.000000,0.000000}%
\pgfsetfillcolor{currentfill}%
\pgfsetlinewidth{0.803000pt}%
\definecolor{currentstroke}{rgb}{0.000000,0.000000,0.000000}%
\pgfsetstrokecolor{currentstroke}%
\pgfsetdash{}{0pt}%
\pgfsys@defobject{currentmarker}{\pgfqpoint{0.000000in}{-0.048611in}}{\pgfqpoint{0.000000in}{0.000000in}}{%
\pgfpathmoveto{\pgfqpoint{0.000000in}{0.000000in}}%
\pgfpathlineto{\pgfqpoint{0.000000in}{-0.048611in}}%
\pgfusepath{stroke,fill}%
}%
\begin{pgfscope}%
\pgfsys@transformshift{0.671996in}{0.416447in}%
\pgfsys@useobject{currentmarker}{}%
\end{pgfscope}%
\end{pgfscope}%
\begin{pgfscope}%
\definecolor{textcolor}{rgb}{0.000000,0.000000,0.000000}%
\pgfsetstrokecolor{textcolor}%
\pgfsetfillcolor{textcolor}%
\pgftext[x=0.671996in,y=0.319225in,,top]{\color{textcolor}\rmfamily\fontsize{8.000000}{9.600000}\selectfont \(\displaystyle {0}\)}%
\end{pgfscope}%
\begin{pgfscope}%
\pgfpathrectangle{\pgfqpoint{0.589745in}{0.416447in}}{\pgfqpoint{1.809530in}{1.372591in}}%
\pgfusepath{clip}%
\pgfsetrectcap%
\pgfsetroundjoin%
\pgfsetlinewidth{0.803000pt}%
\definecolor{currentstroke}{rgb}{0.450000,0.450000,0.450000}%
\pgfsetstrokecolor{currentstroke}%
\pgfsetdash{}{0pt}%
\pgfpathmoveto{\pgfqpoint{1.162253in}{0.416447in}}%
\pgfpathlineto{\pgfqpoint{1.162253in}{1.789039in}}%
\pgfusepath{stroke}%
\end{pgfscope}%
\begin{pgfscope}%
\pgfsetbuttcap%
\pgfsetroundjoin%
\definecolor{currentfill}{rgb}{0.000000,0.000000,0.000000}%
\pgfsetfillcolor{currentfill}%
\pgfsetlinewidth{0.803000pt}%
\definecolor{currentstroke}{rgb}{0.000000,0.000000,0.000000}%
\pgfsetstrokecolor{currentstroke}%
\pgfsetdash{}{0pt}%
\pgfsys@defobject{currentmarker}{\pgfqpoint{0.000000in}{-0.048611in}}{\pgfqpoint{0.000000in}{0.000000in}}{%
\pgfpathmoveto{\pgfqpoint{0.000000in}{0.000000in}}%
\pgfpathlineto{\pgfqpoint{0.000000in}{-0.048611in}}%
\pgfusepath{stroke,fill}%
}%
\begin{pgfscope}%
\pgfsys@transformshift{1.162253in}{0.416447in}%
\pgfsys@useobject{currentmarker}{}%
\end{pgfscope}%
\end{pgfscope}%
\begin{pgfscope}%
\definecolor{textcolor}{rgb}{0.000000,0.000000,0.000000}%
\pgfsetstrokecolor{textcolor}%
\pgfsetfillcolor{textcolor}%
\pgftext[x=1.162253in,y=0.319225in,,top]{\color{textcolor}\rmfamily\fontsize{8.000000}{9.600000}\selectfont \(\displaystyle {10}\)}%
\end{pgfscope}%
\begin{pgfscope}%
\pgfpathrectangle{\pgfqpoint{0.589745in}{0.416447in}}{\pgfqpoint{1.809530in}{1.372591in}}%
\pgfusepath{clip}%
\pgfsetrectcap%
\pgfsetroundjoin%
\pgfsetlinewidth{0.803000pt}%
\definecolor{currentstroke}{rgb}{0.450000,0.450000,0.450000}%
\pgfsetstrokecolor{currentstroke}%
\pgfsetdash{}{0pt}%
\pgfpathmoveto{\pgfqpoint{1.652509in}{0.416447in}}%
\pgfpathlineto{\pgfqpoint{1.652509in}{1.789039in}}%
\pgfusepath{stroke}%
\end{pgfscope}%
\begin{pgfscope}%
\pgfsetbuttcap%
\pgfsetroundjoin%
\definecolor{currentfill}{rgb}{0.000000,0.000000,0.000000}%
\pgfsetfillcolor{currentfill}%
\pgfsetlinewidth{0.803000pt}%
\definecolor{currentstroke}{rgb}{0.000000,0.000000,0.000000}%
\pgfsetstrokecolor{currentstroke}%
\pgfsetdash{}{0pt}%
\pgfsys@defobject{currentmarker}{\pgfqpoint{0.000000in}{-0.048611in}}{\pgfqpoint{0.000000in}{0.000000in}}{%
\pgfpathmoveto{\pgfqpoint{0.000000in}{0.000000in}}%
\pgfpathlineto{\pgfqpoint{0.000000in}{-0.048611in}}%
\pgfusepath{stroke,fill}%
}%
\begin{pgfscope}%
\pgfsys@transformshift{1.652509in}{0.416447in}%
\pgfsys@useobject{currentmarker}{}%
\end{pgfscope}%
\end{pgfscope}%
\begin{pgfscope}%
\definecolor{textcolor}{rgb}{0.000000,0.000000,0.000000}%
\pgfsetstrokecolor{textcolor}%
\pgfsetfillcolor{textcolor}%
\pgftext[x=1.652509in,y=0.319225in,,top]{\color{textcolor}\rmfamily\fontsize{8.000000}{9.600000}\selectfont \(\displaystyle {20}\)}%
\end{pgfscope}%
\begin{pgfscope}%
\pgfpathrectangle{\pgfqpoint{0.589745in}{0.416447in}}{\pgfqpoint{1.809530in}{1.372591in}}%
\pgfusepath{clip}%
\pgfsetrectcap%
\pgfsetroundjoin%
\pgfsetlinewidth{0.803000pt}%
\definecolor{currentstroke}{rgb}{0.450000,0.450000,0.450000}%
\pgfsetstrokecolor{currentstroke}%
\pgfsetdash{}{0pt}%
\pgfpathmoveto{\pgfqpoint{2.142765in}{0.416447in}}%
\pgfpathlineto{\pgfqpoint{2.142765in}{1.789039in}}%
\pgfusepath{stroke}%
\end{pgfscope}%
\begin{pgfscope}%
\pgfsetbuttcap%
\pgfsetroundjoin%
\definecolor{currentfill}{rgb}{0.000000,0.000000,0.000000}%
\pgfsetfillcolor{currentfill}%
\pgfsetlinewidth{0.803000pt}%
\definecolor{currentstroke}{rgb}{0.000000,0.000000,0.000000}%
\pgfsetstrokecolor{currentstroke}%
\pgfsetdash{}{0pt}%
\pgfsys@defobject{currentmarker}{\pgfqpoint{0.000000in}{-0.048611in}}{\pgfqpoint{0.000000in}{0.000000in}}{%
\pgfpathmoveto{\pgfqpoint{0.000000in}{0.000000in}}%
\pgfpathlineto{\pgfqpoint{0.000000in}{-0.048611in}}%
\pgfusepath{stroke,fill}%
}%
\begin{pgfscope}%
\pgfsys@transformshift{2.142765in}{0.416447in}%
\pgfsys@useobject{currentmarker}{}%
\end{pgfscope}%
\end{pgfscope}%
\begin{pgfscope}%
\definecolor{textcolor}{rgb}{0.000000,0.000000,0.000000}%
\pgfsetstrokecolor{textcolor}%
\pgfsetfillcolor{textcolor}%
\pgftext[x=2.142765in,y=0.319225in,,top]{\color{textcolor}\rmfamily\fontsize{8.000000}{9.600000}\selectfont \(\displaystyle {30}\)}%
\end{pgfscope}%
\begin{pgfscope}%
\definecolor{textcolor}{rgb}{0.000000,0.000000,0.000000}%
\pgfsetstrokecolor{textcolor}%
\pgfsetfillcolor{textcolor}%
\pgftext[x=1.494510in,y=0.165003in,,top]{\color{textcolor}\rmfamily\fontsize{10.000000}{12.000000}\selectfont Time in \(\displaystyle \unit{\second}\)}%
\end{pgfscope}%
\begin{pgfscope}%
\pgfpathrectangle{\pgfqpoint{0.589745in}{0.416447in}}{\pgfqpoint{1.809530in}{1.372591in}}%
\pgfusepath{clip}%
\pgfsetrectcap%
\pgfsetroundjoin%
\pgfsetlinewidth{0.803000pt}%
\definecolor{currentstroke}{rgb}{0.450000,0.450000,0.450000}%
\pgfsetstrokecolor{currentstroke}%
\pgfsetdash{}{0pt}%
\pgfpathmoveto{\pgfqpoint{0.589745in}{0.553707in}}%
\pgfpathlineto{\pgfqpoint{2.399275in}{0.553707in}}%
\pgfusepath{stroke}%
\end{pgfscope}%
\begin{pgfscope}%
\pgfsetbuttcap%
\pgfsetroundjoin%
\definecolor{currentfill}{rgb}{0.000000,0.000000,0.000000}%
\pgfsetfillcolor{currentfill}%
\pgfsetlinewidth{0.803000pt}%
\definecolor{currentstroke}{rgb}{0.000000,0.000000,0.000000}%
\pgfsetstrokecolor{currentstroke}%
\pgfsetdash{}{0pt}%
\pgfsys@defobject{currentmarker}{\pgfqpoint{-0.048611in}{0.000000in}}{\pgfqpoint{-0.000000in}{0.000000in}}{%
\pgfpathmoveto{\pgfqpoint{-0.000000in}{0.000000in}}%
\pgfpathlineto{\pgfqpoint{-0.048611in}{0.000000in}}%
\pgfusepath{stroke,fill}%
}%
\begin{pgfscope}%
\pgfsys@transformshift{0.589745in}{0.553707in}%
\pgfsys@useobject{currentmarker}{}%
\end{pgfscope}%
\end{pgfscope}%
\begin{pgfscope}%
\definecolor{textcolor}{rgb}{0.000000,0.000000,0.000000}%
\pgfsetstrokecolor{textcolor}%
\pgfsetfillcolor{textcolor}%
\pgftext[x=0.223614in, y=0.515151in, left, base]{\color{textcolor}\rmfamily\fontsize{8.000000}{9.600000}\selectfont \(\displaystyle {\ensuremath{-}200}\)}%
\end{pgfscope}%
\begin{pgfscope}%
\pgfpathrectangle{\pgfqpoint{0.589745in}{0.416447in}}{\pgfqpoint{1.809530in}{1.372591in}}%
\pgfusepath{clip}%
\pgfsetrectcap%
\pgfsetroundjoin%
\pgfsetlinewidth{0.803000pt}%
\definecolor{currentstroke}{rgb}{0.450000,0.450000,0.450000}%
\pgfsetstrokecolor{currentstroke}%
\pgfsetdash{}{0pt}%
\pgfpathmoveto{\pgfqpoint{0.589745in}{0.828225in}}%
\pgfpathlineto{\pgfqpoint{2.399275in}{0.828225in}}%
\pgfusepath{stroke}%
\end{pgfscope}%
\begin{pgfscope}%
\pgfsetbuttcap%
\pgfsetroundjoin%
\definecolor{currentfill}{rgb}{0.000000,0.000000,0.000000}%
\pgfsetfillcolor{currentfill}%
\pgfsetlinewidth{0.803000pt}%
\definecolor{currentstroke}{rgb}{0.000000,0.000000,0.000000}%
\pgfsetstrokecolor{currentstroke}%
\pgfsetdash{}{0pt}%
\pgfsys@defobject{currentmarker}{\pgfqpoint{-0.048611in}{0.000000in}}{\pgfqpoint{-0.000000in}{0.000000in}}{%
\pgfpathmoveto{\pgfqpoint{-0.000000in}{0.000000in}}%
\pgfpathlineto{\pgfqpoint{-0.048611in}{0.000000in}}%
\pgfusepath{stroke,fill}%
}%
\begin{pgfscope}%
\pgfsys@transformshift{0.589745in}{0.828225in}%
\pgfsys@useobject{currentmarker}{}%
\end{pgfscope}%
\end{pgfscope}%
\begin{pgfscope}%
\definecolor{textcolor}{rgb}{0.000000,0.000000,0.000000}%
\pgfsetstrokecolor{textcolor}%
\pgfsetfillcolor{textcolor}%
\pgftext[x=0.223614in, y=0.789669in, left, base]{\color{textcolor}\rmfamily\fontsize{8.000000}{9.600000}\selectfont \(\displaystyle {\ensuremath{-}100}\)}%
\end{pgfscope}%
\begin{pgfscope}%
\pgfpathrectangle{\pgfqpoint{0.589745in}{0.416447in}}{\pgfqpoint{1.809530in}{1.372591in}}%
\pgfusepath{clip}%
\pgfsetrectcap%
\pgfsetroundjoin%
\pgfsetlinewidth{0.803000pt}%
\definecolor{currentstroke}{rgb}{0.450000,0.450000,0.450000}%
\pgfsetstrokecolor{currentstroke}%
\pgfsetdash{}{0pt}%
\pgfpathmoveto{\pgfqpoint{0.589745in}{1.102743in}}%
\pgfpathlineto{\pgfqpoint{2.399275in}{1.102743in}}%
\pgfusepath{stroke}%
\end{pgfscope}%
\begin{pgfscope}%
\pgfsetbuttcap%
\pgfsetroundjoin%
\definecolor{currentfill}{rgb}{0.000000,0.000000,0.000000}%
\pgfsetfillcolor{currentfill}%
\pgfsetlinewidth{0.803000pt}%
\definecolor{currentstroke}{rgb}{0.000000,0.000000,0.000000}%
\pgfsetstrokecolor{currentstroke}%
\pgfsetdash{}{0pt}%
\pgfsys@defobject{currentmarker}{\pgfqpoint{-0.048611in}{0.000000in}}{\pgfqpoint{-0.000000in}{0.000000in}}{%
\pgfpathmoveto{\pgfqpoint{-0.000000in}{0.000000in}}%
\pgfpathlineto{\pgfqpoint{-0.048611in}{0.000000in}}%
\pgfusepath{stroke,fill}%
}%
\begin{pgfscope}%
\pgfsys@transformshift{0.589745in}{1.102743in}%
\pgfsys@useobject{currentmarker}{}%
\end{pgfscope}%
\end{pgfscope}%
\begin{pgfscope}%
\definecolor{textcolor}{rgb}{0.000000,0.000000,0.000000}%
\pgfsetstrokecolor{textcolor}%
\pgfsetfillcolor{textcolor}%
\pgftext[x=0.433494in, y=1.064188in, left, base]{\color{textcolor}\rmfamily\fontsize{8.000000}{9.600000}\selectfont \(\displaystyle {0}\)}%
\end{pgfscope}%
\begin{pgfscope}%
\pgfpathrectangle{\pgfqpoint{0.589745in}{0.416447in}}{\pgfqpoint{1.809530in}{1.372591in}}%
\pgfusepath{clip}%
\pgfsetrectcap%
\pgfsetroundjoin%
\pgfsetlinewidth{0.803000pt}%
\definecolor{currentstroke}{rgb}{0.450000,0.450000,0.450000}%
\pgfsetstrokecolor{currentstroke}%
\pgfsetdash{}{0pt}%
\pgfpathmoveto{\pgfqpoint{0.589745in}{1.377261in}}%
\pgfpathlineto{\pgfqpoint{2.399275in}{1.377261in}}%
\pgfusepath{stroke}%
\end{pgfscope}%
\begin{pgfscope}%
\pgfsetbuttcap%
\pgfsetroundjoin%
\definecolor{currentfill}{rgb}{0.000000,0.000000,0.000000}%
\pgfsetfillcolor{currentfill}%
\pgfsetlinewidth{0.803000pt}%
\definecolor{currentstroke}{rgb}{0.000000,0.000000,0.000000}%
\pgfsetstrokecolor{currentstroke}%
\pgfsetdash{}{0pt}%
\pgfsys@defobject{currentmarker}{\pgfqpoint{-0.048611in}{0.000000in}}{\pgfqpoint{-0.000000in}{0.000000in}}{%
\pgfpathmoveto{\pgfqpoint{-0.000000in}{0.000000in}}%
\pgfpathlineto{\pgfqpoint{-0.048611in}{0.000000in}}%
\pgfusepath{stroke,fill}%
}%
\begin{pgfscope}%
\pgfsys@transformshift{0.589745in}{1.377261in}%
\pgfsys@useobject{currentmarker}{}%
\end{pgfscope}%
\end{pgfscope}%
\begin{pgfscope}%
\definecolor{textcolor}{rgb}{0.000000,0.000000,0.000000}%
\pgfsetstrokecolor{textcolor}%
\pgfsetfillcolor{textcolor}%
\pgftext[x=0.315437in, y=1.338706in, left, base]{\color{textcolor}\rmfamily\fontsize{8.000000}{9.600000}\selectfont \(\displaystyle {100}\)}%
\end{pgfscope}%
\begin{pgfscope}%
\pgfpathrectangle{\pgfqpoint{0.589745in}{0.416447in}}{\pgfqpoint{1.809530in}{1.372591in}}%
\pgfusepath{clip}%
\pgfsetrectcap%
\pgfsetroundjoin%
\pgfsetlinewidth{0.803000pt}%
\definecolor{currentstroke}{rgb}{0.450000,0.450000,0.450000}%
\pgfsetstrokecolor{currentstroke}%
\pgfsetdash{}{0pt}%
\pgfpathmoveto{\pgfqpoint{0.589745in}{1.651780in}}%
\pgfpathlineto{\pgfqpoint{2.399275in}{1.651780in}}%
\pgfusepath{stroke}%
\end{pgfscope}%
\begin{pgfscope}%
\pgfsetbuttcap%
\pgfsetroundjoin%
\definecolor{currentfill}{rgb}{0.000000,0.000000,0.000000}%
\pgfsetfillcolor{currentfill}%
\pgfsetlinewidth{0.803000pt}%
\definecolor{currentstroke}{rgb}{0.000000,0.000000,0.000000}%
\pgfsetstrokecolor{currentstroke}%
\pgfsetdash{}{0pt}%
\pgfsys@defobject{currentmarker}{\pgfqpoint{-0.048611in}{0.000000in}}{\pgfqpoint{-0.000000in}{0.000000in}}{%
\pgfpathmoveto{\pgfqpoint{-0.000000in}{0.000000in}}%
\pgfpathlineto{\pgfqpoint{-0.048611in}{0.000000in}}%
\pgfusepath{stroke,fill}%
}%
\begin{pgfscope}%
\pgfsys@transformshift{0.589745in}{1.651780in}%
\pgfsys@useobject{currentmarker}{}%
\end{pgfscope}%
\end{pgfscope}%
\begin{pgfscope}%
\definecolor{textcolor}{rgb}{0.000000,0.000000,0.000000}%
\pgfsetstrokecolor{textcolor}%
\pgfsetfillcolor{textcolor}%
\pgftext[x=0.315437in, y=1.613224in, left, base]{\color{textcolor}\rmfamily\fontsize{8.000000}{9.600000}\selectfont \(\displaystyle {200}\)}%
\end{pgfscope}%
\begin{pgfscope}%
\definecolor{textcolor}{rgb}{0.000000,0.000000,0.000000}%
\pgfsetstrokecolor{textcolor}%
\pgfsetfillcolor{textcolor}%
\pgftext[x=0.168059in,y=1.102743in,,bottom,rotate=90.000000]{\color{textcolor}\rmfamily\fontsize{10.000000}{12.000000}\selectfont Ampl. in arb. unit}%
\end{pgfscope}%
\begin{pgfscope}%
\pgfpathrectangle{\pgfqpoint{0.589745in}{0.416447in}}{\pgfqpoint{1.809530in}{1.372591in}}%
\pgfusepath{clip}%
\pgfsetrectcap%
\pgfsetroundjoin%
\pgfsetlinewidth{1.505625pt}%
\definecolor{currentstroke}{rgb}{0.003922,0.450980,0.698039}%
\pgfsetstrokecolor{currentstroke}%
\pgfsetdash{}{0pt}%
\pgfpathmoveto{\pgfqpoint{0.671996in}{1.090740in}}%
\pgfpathlineto{\pgfqpoint{0.672773in}{1.491574in}}%
\pgfpathlineto{\pgfqpoint{0.674136in}{0.729859in}}%
\pgfpathlineto{\pgfqpoint{0.675559in}{1.429163in}}%
\pgfpathlineto{\pgfqpoint{0.677371in}{0.774373in}}%
\pgfpathlineto{\pgfqpoint{0.678945in}{1.498782in}}%
\pgfpathlineto{\pgfqpoint{0.680948in}{0.723031in}}%
\pgfpathlineto{\pgfqpoint{0.681923in}{1.419106in}}%
\pgfpathlineto{\pgfqpoint{0.684197in}{0.720412in}}%
\pgfpathlineto{\pgfqpoint{0.685248in}{1.415086in}}%
\pgfpathlineto{\pgfqpoint{0.687242in}{0.733681in}}%
\pgfpathlineto{\pgfqpoint{0.688541in}{1.442882in}}%
\pgfpathlineto{\pgfqpoint{0.690888in}{0.728333in}}%
\pgfpathlineto{\pgfqpoint{0.692239in}{1.476188in}}%
\pgfpathlineto{\pgfqpoint{0.694117in}{0.683498in}}%
\pgfpathlineto{\pgfqpoint{0.695495in}{1.451670in}}%
\pgfpathlineto{\pgfqpoint{0.697207in}{0.735776in}}%
\pgfpathlineto{\pgfqpoint{0.698911in}{1.463289in}}%
\pgfpathlineto{\pgfqpoint{0.700391in}{0.744937in}}%
\pgfpathlineto{\pgfqpoint{0.701675in}{1.441656in}}%
\pgfpathlineto{\pgfqpoint{0.704063in}{0.727954in}}%
\pgfpathlineto{\pgfqpoint{0.705067in}{1.426096in}}%
\pgfpathlineto{\pgfqpoint{0.707098in}{0.743138in}}%
\pgfpathlineto{\pgfqpoint{0.708441in}{1.397964in}}%
\pgfpathlineto{\pgfqpoint{0.710049in}{0.799480in}}%
\pgfpathlineto{\pgfqpoint{0.711723in}{1.388621in}}%
\pgfpathlineto{\pgfqpoint{0.713247in}{0.801299in}}%
\pgfpathlineto{\pgfqpoint{0.715560in}{1.447160in}}%
\pgfpathlineto{\pgfqpoint{0.716758in}{0.703681in}}%
\pgfpathlineto{\pgfqpoint{0.718524in}{1.448192in}}%
\pgfpathlineto{\pgfqpoint{0.719855in}{0.772556in}}%
\pgfpathlineto{\pgfqpoint{0.721524in}{1.444516in}}%
\pgfpathlineto{\pgfqpoint{0.723555in}{0.730409in}}%
\pgfpathlineto{\pgfqpoint{0.724753in}{1.421263in}}%
\pgfpathlineto{\pgfqpoint{0.726626in}{0.750148in}}%
\pgfpathlineto{\pgfqpoint{0.728164in}{1.404535in}}%
\pgfpathlineto{\pgfqpoint{0.729705in}{0.802950in}}%
\pgfpathlineto{\pgfqpoint{0.731637in}{1.451430in}}%
\pgfpathlineto{\pgfqpoint{0.733391in}{0.772664in}}%
\pgfpathlineto{\pgfqpoint{0.734637in}{1.447355in}}%
\pgfpathlineto{\pgfqpoint{0.736338in}{0.788776in}}%
\pgfpathlineto{\pgfqpoint{0.738103in}{1.443705in}}%
\pgfpathlineto{\pgfqpoint{0.739886in}{0.729444in}}%
\pgfpathlineto{\pgfqpoint{0.741382in}{1.431065in}}%
\pgfpathlineto{\pgfqpoint{0.743021in}{0.782882in}}%
\pgfpathlineto{\pgfqpoint{0.744868in}{1.420151in}}%
\pgfpathlineto{\pgfqpoint{0.746251in}{0.727718in}}%
\pgfpathlineto{\pgfqpoint{0.748238in}{1.433801in}}%
\pgfpathlineto{\pgfqpoint{0.749515in}{0.780686in}}%
\pgfpathlineto{\pgfqpoint{0.751576in}{1.480571in}}%
\pgfpathlineto{\pgfqpoint{0.753008in}{0.717734in}}%
\pgfpathlineto{\pgfqpoint{0.754582in}{1.424817in}}%
\pgfpathlineto{\pgfqpoint{0.756073in}{0.796932in}}%
\pgfpathlineto{\pgfqpoint{0.757883in}{1.442620in}}%
\pgfpathlineto{\pgfqpoint{0.759689in}{0.750762in}}%
\pgfpathlineto{\pgfqpoint{0.762055in}{1.479276in}}%
\pgfpathlineto{\pgfqpoint{0.762753in}{0.796516in}}%
\pgfpathlineto{\pgfqpoint{0.764516in}{1.395296in}}%
\pgfpathlineto{\pgfqpoint{0.765974in}{0.788424in}}%
\pgfpathlineto{\pgfqpoint{0.767871in}{1.470798in}}%
\pgfpathlineto{\pgfqpoint{0.769471in}{0.713479in}}%
\pgfpathlineto{\pgfqpoint{0.771138in}{1.436364in}}%
\pgfpathlineto{\pgfqpoint{0.772690in}{0.775600in}}%
\pgfpathlineto{\pgfqpoint{0.774561in}{1.475613in}}%
\pgfpathlineto{\pgfqpoint{0.776733in}{0.710283in}}%
\pgfpathlineto{\pgfqpoint{0.777495in}{1.396885in}}%
\pgfpathlineto{\pgfqpoint{0.779297in}{0.777462in}}%
\pgfpathlineto{\pgfqpoint{0.780978in}{1.438155in}}%
\pgfpathlineto{\pgfqpoint{0.782584in}{0.769700in}}%
\pgfpathlineto{\pgfqpoint{0.784093in}{1.502985in}}%
\pgfpathlineto{\pgfqpoint{0.786097in}{0.739969in}}%
\pgfpathlineto{\pgfqpoint{0.787749in}{1.450767in}}%
\pgfpathlineto{\pgfqpoint{0.789148in}{0.769555in}}%
\pgfpathlineto{\pgfqpoint{0.791337in}{1.487102in}}%
\pgfpathlineto{\pgfqpoint{0.792709in}{0.765376in}}%
\pgfpathlineto{\pgfqpoint{0.794320in}{1.456617in}}%
\pgfpathlineto{\pgfqpoint{0.795697in}{0.806621in}}%
\pgfpathlineto{\pgfqpoint{0.797278in}{1.463589in}}%
\pgfpathlineto{\pgfqpoint{0.799190in}{0.725901in}}%
\pgfpathlineto{\pgfqpoint{0.800745in}{1.456930in}}%
\pgfpathlineto{\pgfqpoint{0.802543in}{0.772460in}}%
\pgfpathlineto{\pgfqpoint{0.804163in}{1.420557in}}%
\pgfpathlineto{\pgfqpoint{0.805614in}{0.716585in}}%
\pgfpathlineto{\pgfqpoint{0.807247in}{1.451408in}}%
\pgfpathlineto{\pgfqpoint{0.809018in}{0.755786in}}%
\pgfpathlineto{\pgfqpoint{0.810745in}{1.480827in}}%
\pgfpathlineto{\pgfqpoint{0.812399in}{0.787471in}}%
\pgfpathlineto{\pgfqpoint{0.813754in}{1.423003in}}%
\pgfpathlineto{\pgfqpoint{0.815746in}{0.791918in}}%
\pgfpathlineto{\pgfqpoint{0.817284in}{1.437768in}}%
\pgfpathlineto{\pgfqpoint{0.818927in}{0.798232in}}%
\pgfpathlineto{\pgfqpoint{0.820363in}{1.449995in}}%
\pgfpathlineto{\pgfqpoint{0.822195in}{0.705712in}}%
\pgfpathlineto{\pgfqpoint{0.823729in}{1.392620in}}%
\pgfpathlineto{\pgfqpoint{0.825311in}{0.794080in}}%
\pgfpathlineto{\pgfqpoint{0.826948in}{1.428238in}}%
\pgfpathlineto{\pgfqpoint{0.828620in}{0.771295in}}%
\pgfpathlineto{\pgfqpoint{0.830618in}{1.422850in}}%
\pgfpathlineto{\pgfqpoint{0.831904in}{0.794237in}}%
\pgfpathlineto{\pgfqpoint{0.833554in}{1.407878in}}%
\pgfpathlineto{\pgfqpoint{0.835883in}{0.773698in}}%
\pgfpathlineto{\pgfqpoint{0.837169in}{1.483124in}}%
\pgfpathlineto{\pgfqpoint{0.838764in}{0.728389in}}%
\pgfpathlineto{\pgfqpoint{0.840689in}{1.539729in}}%
\pgfpathlineto{\pgfqpoint{0.842029in}{0.750178in}}%
\pgfpathlineto{\pgfqpoint{0.843601in}{1.427875in}}%
\pgfpathlineto{\pgfqpoint{0.845643in}{0.766464in}}%
\pgfpathlineto{\pgfqpoint{0.847461in}{1.508897in}}%
\pgfpathlineto{\pgfqpoint{0.848451in}{0.787708in}}%
\pgfpathlineto{\pgfqpoint{0.850120in}{1.427949in}}%
\pgfpathlineto{\pgfqpoint{0.851867in}{0.761128in}}%
\pgfpathlineto{\pgfqpoint{0.853784in}{1.470326in}}%
\pgfpathlineto{\pgfqpoint{0.855190in}{0.775099in}}%
\pgfpathlineto{\pgfqpoint{0.856646in}{1.398171in}}%
\pgfpathlineto{\pgfqpoint{0.858480in}{0.782136in}}%
\pgfpathlineto{\pgfqpoint{0.860012in}{1.416830in}}%
\pgfpathlineto{\pgfqpoint{0.861699in}{0.753798in}}%
\pgfpathlineto{\pgfqpoint{0.863423in}{1.422984in}}%
\pgfpathlineto{\pgfqpoint{0.864867in}{0.763181in}}%
\pgfpathlineto{\pgfqpoint{0.866791in}{1.529877in}}%
\pgfpathlineto{\pgfqpoint{0.868283in}{0.801982in}}%
\pgfpathlineto{\pgfqpoint{0.869853in}{1.406314in}}%
\pgfpathlineto{\pgfqpoint{0.871525in}{0.785712in}}%
\pgfpathlineto{\pgfqpoint{0.873424in}{1.449325in}}%
\pgfpathlineto{\pgfqpoint{0.874891in}{0.795985in}}%
\pgfpathlineto{\pgfqpoint{0.876426in}{1.389736in}}%
\pgfpathlineto{\pgfqpoint{0.878138in}{0.802793in}}%
\pgfpathlineto{\pgfqpoint{0.879714in}{1.439143in}}%
\pgfpathlineto{\pgfqpoint{0.882022in}{0.707381in}}%
\pgfpathlineto{\pgfqpoint{0.883038in}{1.384697in}}%
\pgfpathlineto{\pgfqpoint{0.885174in}{0.738322in}}%
\pgfpathlineto{\pgfqpoint{0.886365in}{1.405495in}}%
\pgfpathlineto{\pgfqpoint{0.887946in}{0.798013in}}%
\pgfpathlineto{\pgfqpoint{0.889612in}{1.409957in}}%
\pgfpathlineto{\pgfqpoint{0.891250in}{0.756406in}}%
\pgfpathlineto{\pgfqpoint{0.893020in}{1.423582in}}%
\pgfpathlineto{\pgfqpoint{0.894878in}{0.763119in}}%
\pgfpathlineto{\pgfqpoint{0.896271in}{1.449545in}}%
\pgfpathlineto{\pgfqpoint{0.898098in}{0.772040in}}%
\pgfpathlineto{\pgfqpoint{0.899513in}{1.425447in}}%
\pgfpathlineto{\pgfqpoint{0.901504in}{0.791211in}}%
\pgfpathlineto{\pgfqpoint{0.903089in}{1.423649in}}%
\pgfpathlineto{\pgfqpoint{0.904587in}{0.747761in}}%
\pgfpathlineto{\pgfqpoint{0.906130in}{1.425987in}}%
\pgfpathlineto{\pgfqpoint{0.907800in}{0.788707in}}%
\pgfpathlineto{\pgfqpoint{0.909753in}{1.448490in}}%
\pgfpathlineto{\pgfqpoint{0.911172in}{0.781183in}}%
\pgfpathlineto{\pgfqpoint{0.913348in}{1.470276in}}%
\pgfpathlineto{\pgfqpoint{0.914453in}{0.770430in}}%
\pgfpathlineto{\pgfqpoint{0.916166in}{1.423496in}}%
\pgfpathlineto{\pgfqpoint{0.917707in}{0.783755in}}%
\pgfpathlineto{\pgfqpoint{0.920511in}{1.555897in}}%
\pgfpathlineto{\pgfqpoint{0.921001in}{0.768227in}}%
\pgfpathlineto{\pgfqpoint{0.922916in}{1.437372in}}%
\pgfpathlineto{\pgfqpoint{0.924486in}{0.760548in}}%
\pgfpathlineto{\pgfqpoint{0.925871in}{1.495752in}}%
\pgfpathlineto{\pgfqpoint{0.927750in}{0.803860in}}%
\pgfpathlineto{\pgfqpoint{0.929150in}{1.425024in}}%
\pgfpathlineto{\pgfqpoint{0.930921in}{0.751156in}}%
\pgfpathlineto{\pgfqpoint{0.932492in}{1.454377in}}%
\pgfpathlineto{\pgfqpoint{0.934372in}{0.736057in}}%
\pgfpathlineto{\pgfqpoint{0.935734in}{1.423394in}}%
\pgfpathlineto{\pgfqpoint{0.937436in}{0.782227in}}%
\pgfpathlineto{\pgfqpoint{0.939050in}{1.422322in}}%
\pgfpathlineto{\pgfqpoint{0.941042in}{0.786353in}}%
\pgfpathlineto{\pgfqpoint{0.942390in}{1.478144in}}%
\pgfpathlineto{\pgfqpoint{0.943986in}{0.792665in}}%
\pgfpathlineto{\pgfqpoint{0.945680in}{1.395631in}}%
\pgfpathlineto{\pgfqpoint{0.947418in}{0.764381in}}%
\pgfpathlineto{\pgfqpoint{0.948937in}{1.454863in}}%
\pgfpathlineto{\pgfqpoint{0.950711in}{0.796077in}}%
\pgfpathlineto{\pgfqpoint{0.952329in}{1.410284in}}%
\pgfpathlineto{\pgfqpoint{0.953884in}{0.758163in}}%
\pgfpathlineto{\pgfqpoint{0.955739in}{1.470179in}}%
\pgfpathlineto{\pgfqpoint{0.957386in}{0.787869in}}%
\pgfpathlineto{\pgfqpoint{0.958938in}{1.419337in}}%
\pgfpathlineto{\pgfqpoint{0.960520in}{0.822435in}}%
\pgfpathlineto{\pgfqpoint{0.962899in}{1.462282in}}%
\pgfpathlineto{\pgfqpoint{0.963980in}{0.771753in}}%
\pgfpathlineto{\pgfqpoint{0.965824in}{1.476277in}}%
\pgfpathlineto{\pgfqpoint{0.967236in}{0.785250in}}%
\pgfpathlineto{\pgfqpoint{0.968782in}{1.472012in}}%
\pgfpathlineto{\pgfqpoint{0.970481in}{0.773513in}}%
\pgfpathlineto{\pgfqpoint{0.971998in}{1.474921in}}%
\pgfpathlineto{\pgfqpoint{0.973892in}{0.767309in}}%
\pgfpathlineto{\pgfqpoint{0.975499in}{1.445214in}}%
\pgfpathlineto{\pgfqpoint{0.976944in}{0.767976in}}%
\pgfpathlineto{\pgfqpoint{0.979137in}{1.477348in}}%
\pgfpathlineto{\pgfqpoint{0.980419in}{0.764387in}}%
\pgfpathlineto{\pgfqpoint{0.982030in}{1.449620in}}%
\pgfpathlineto{\pgfqpoint{0.983702in}{0.773321in}}%
\pgfpathlineto{\pgfqpoint{0.985370in}{1.440132in}}%
\pgfpathlineto{\pgfqpoint{0.987499in}{0.706459in}}%
\pgfpathlineto{\pgfqpoint{0.988535in}{1.461796in}}%
\pgfpathlineto{\pgfqpoint{0.990131in}{0.774889in}}%
\pgfpathlineto{\pgfqpoint{0.991872in}{1.498342in}}%
\pgfpathlineto{\pgfqpoint{0.993623in}{0.787602in}}%
\pgfpathlineto{\pgfqpoint{0.995400in}{1.438399in}}%
\pgfpathlineto{\pgfqpoint{0.996906in}{0.719652in}}%
\pgfpathlineto{\pgfqpoint{0.998449in}{1.429685in}}%
\pgfpathlineto{\pgfqpoint{1.000262in}{0.727884in}}%
\pgfpathlineto{\pgfqpoint{1.001710in}{1.418715in}}%
\pgfpathlineto{\pgfqpoint{1.003447in}{0.707877in}}%
\pgfpathlineto{\pgfqpoint{1.005003in}{1.412436in}}%
\pgfpathlineto{\pgfqpoint{1.006669in}{0.778905in}}%
\pgfpathlineto{\pgfqpoint{1.009070in}{1.452301in}}%
\pgfpathlineto{\pgfqpoint{1.009932in}{0.785520in}}%
\pgfpathlineto{\pgfqpoint{1.012155in}{1.523100in}}%
\pgfpathlineto{\pgfqpoint{1.013205in}{0.768000in}}%
\pgfpathlineto{\pgfqpoint{1.015893in}{1.480180in}}%
\pgfpathlineto{\pgfqpoint{1.016583in}{0.792817in}}%
\pgfpathlineto{\pgfqpoint{1.018773in}{1.471792in}}%
\pgfpathlineto{\pgfqpoint{1.019995in}{0.761049in}}%
\pgfpathlineto{\pgfqpoint{1.021522in}{1.444662in}}%
\pgfpathlineto{\pgfqpoint{1.023112in}{0.798432in}}%
\pgfpathlineto{\pgfqpoint{1.024863in}{1.442024in}}%
\pgfpathlineto{\pgfqpoint{1.026572in}{0.776775in}}%
\pgfpathlineto{\pgfqpoint{1.028839in}{1.523567in}}%
\pgfpathlineto{\pgfqpoint{1.029706in}{0.798383in}}%
\pgfpathlineto{\pgfqpoint{1.031850in}{1.436670in}}%
\pgfpathlineto{\pgfqpoint{1.033078in}{0.745242in}}%
\pgfpathlineto{\pgfqpoint{1.034681in}{1.392443in}}%
\pgfpathlineto{\pgfqpoint{1.036312in}{0.760387in}}%
\pgfpathlineto{\pgfqpoint{1.038236in}{1.457618in}}%
\pgfpathlineto{\pgfqpoint{1.039665in}{0.740580in}}%
\pgfpathlineto{\pgfqpoint{1.041383in}{1.406661in}}%
\pgfpathlineto{\pgfqpoint{1.043045in}{0.795002in}}%
\pgfpathlineto{\pgfqpoint{1.044767in}{1.478945in}}%
\pgfpathlineto{\pgfqpoint{1.046656in}{0.769843in}}%
\pgfpathlineto{\pgfqpoint{1.047818in}{1.408366in}}%
\pgfpathlineto{\pgfqpoint{1.049647in}{0.775869in}}%
\pgfpathlineto{\pgfqpoint{1.051159in}{1.433596in}}%
\pgfpathlineto{\pgfqpoint{1.052968in}{0.761823in}}%
\pgfpathlineto{\pgfqpoint{1.054433in}{1.430195in}}%
\pgfpathlineto{\pgfqpoint{1.056499in}{0.755279in}}%
\pgfpathlineto{\pgfqpoint{1.058048in}{1.447089in}}%
\pgfpathlineto{\pgfqpoint{1.059360in}{0.770370in}}%
\pgfpathlineto{\pgfqpoint{1.061423in}{1.464398in}}%
\pgfpathlineto{\pgfqpoint{1.062986in}{0.790776in}}%
\pgfpathlineto{\pgfqpoint{1.064302in}{1.470748in}}%
\pgfpathlineto{\pgfqpoint{1.066355in}{0.784207in}}%
\pgfpathlineto{\pgfqpoint{1.067636in}{1.431705in}}%
\pgfpathlineto{\pgfqpoint{1.069590in}{0.790204in}}%
\pgfpathlineto{\pgfqpoint{1.071207in}{1.463142in}}%
\pgfpathlineto{\pgfqpoint{1.072778in}{0.733590in}}%
\pgfpathlineto{\pgfqpoint{1.074233in}{1.410029in}}%
\pgfpathlineto{\pgfqpoint{1.076710in}{0.727242in}}%
\pgfpathlineto{\pgfqpoint{1.077829in}{1.461360in}}%
\pgfpathlineto{\pgfqpoint{1.079140in}{0.786247in}}%
\pgfpathlineto{\pgfqpoint{1.081639in}{1.502784in}}%
\pgfpathlineto{\pgfqpoint{1.082583in}{0.785700in}}%
\pgfpathlineto{\pgfqpoint{1.084166in}{1.456162in}}%
\pgfpathlineto{\pgfqpoint{1.086135in}{0.754740in}}%
\pgfpathlineto{\pgfqpoint{1.087412in}{1.415705in}}%
\pgfpathlineto{\pgfqpoint{1.089272in}{0.797026in}}%
\pgfpathlineto{\pgfqpoint{1.090865in}{1.440234in}}%
\pgfpathlineto{\pgfqpoint{1.092547in}{0.756789in}}%
\pgfpathlineto{\pgfqpoint{1.093987in}{1.410579in}}%
\pgfpathlineto{\pgfqpoint{1.095658in}{0.766355in}}%
\pgfpathlineto{\pgfqpoint{1.097331in}{1.377450in}}%
\pgfpathlineto{\pgfqpoint{1.098974in}{0.759108in}}%
\pgfpathlineto{\pgfqpoint{1.101081in}{1.441557in}}%
\pgfpathlineto{\pgfqpoint{1.102497in}{0.746575in}}%
\pgfpathlineto{\pgfqpoint{1.104131in}{1.444769in}}%
\pgfpathlineto{\pgfqpoint{1.105745in}{0.748928in}}%
\pgfpathlineto{\pgfqpoint{1.107234in}{1.451518in}}%
\pgfpathlineto{\pgfqpoint{1.108943in}{0.765184in}}%
\pgfpathlineto{\pgfqpoint{1.110504in}{1.421998in}}%
\pgfpathlineto{\pgfqpoint{1.112299in}{0.757907in}}%
\pgfpathlineto{\pgfqpoint{1.113775in}{1.409574in}}%
\pgfpathlineto{\pgfqpoint{1.115400in}{0.799914in}}%
\pgfpathlineto{\pgfqpoint{1.117288in}{1.492464in}}%
\pgfpathlineto{\pgfqpoint{1.118780in}{0.795325in}}%
\pgfpathlineto{\pgfqpoint{1.120494in}{1.459195in}}%
\pgfpathlineto{\pgfqpoint{1.122402in}{0.719250in}}%
\pgfpathlineto{\pgfqpoint{1.124243in}{1.477820in}}%
\pgfpathlineto{\pgfqpoint{1.125366in}{0.785611in}}%
\pgfpathlineto{\pgfqpoint{1.127659in}{1.456350in}}%
\pgfpathlineto{\pgfqpoint{1.129067in}{0.729287in}}%
\pgfpathlineto{\pgfqpoint{1.130323in}{1.462526in}}%
\pgfpathlineto{\pgfqpoint{1.132070in}{0.779338in}}%
\pgfpathlineto{\pgfqpoint{1.133675in}{1.412016in}}%
\pgfpathlineto{\pgfqpoint{1.135409in}{0.735296in}}%
\pgfpathlineto{\pgfqpoint{1.136909in}{1.422102in}}%
\pgfpathlineto{\pgfqpoint{1.139175in}{0.727233in}}%
\pgfpathlineto{\pgfqpoint{1.140148in}{1.424301in}}%
\pgfpathlineto{\pgfqpoint{1.142301in}{0.752515in}}%
\pgfpathlineto{\pgfqpoint{1.144086in}{1.469140in}}%
\pgfpathlineto{\pgfqpoint{1.145175in}{0.796777in}}%
\pgfpathlineto{\pgfqpoint{1.146865in}{1.426803in}}%
\pgfpathlineto{\pgfqpoint{1.148971in}{0.758819in}}%
\pgfpathlineto{\pgfqpoint{1.150130in}{1.417840in}}%
\pgfpathlineto{\pgfqpoint{1.151831in}{0.795215in}}%
\pgfpathlineto{\pgfqpoint{1.153955in}{1.449743in}}%
\pgfpathlineto{\pgfqpoint{1.155596in}{0.727954in}}%
\pgfpathlineto{\pgfqpoint{1.156625in}{1.416781in}}%
\pgfpathlineto{\pgfqpoint{1.158311in}{0.756894in}}%
\pgfpathlineto{\pgfqpoint{1.160277in}{1.465614in}}%
\pgfpathlineto{\pgfqpoint{1.162017in}{0.722637in}}%
\pgfpathlineto{\pgfqpoint{1.163332in}{1.434524in}}%
\pgfpathlineto{\pgfqpoint{1.164997in}{0.753390in}}%
\pgfpathlineto{\pgfqpoint{1.166574in}{1.404989in}}%
\pgfpathlineto{\pgfqpoint{1.168889in}{0.713813in}}%
\pgfpathlineto{\pgfqpoint{1.170552in}{1.514028in}}%
\pgfpathlineto{\pgfqpoint{1.171464in}{0.817611in}}%
\pgfpathlineto{\pgfqpoint{1.173151in}{1.460276in}}%
\pgfpathlineto{\pgfqpoint{1.175095in}{0.796302in}}%
\pgfpathlineto{\pgfqpoint{1.176881in}{1.500637in}}%
\pgfpathlineto{\pgfqpoint{1.178324in}{0.783964in}}%
\pgfpathlineto{\pgfqpoint{1.179920in}{1.399304in}}%
\pgfpathlineto{\pgfqpoint{1.181591in}{0.731610in}}%
\pgfpathlineto{\pgfqpoint{1.183055in}{1.406660in}}%
\pgfpathlineto{\pgfqpoint{1.185369in}{0.725710in}}%
\pgfpathlineto{\pgfqpoint{1.186316in}{1.433478in}}%
\pgfpathlineto{\pgfqpoint{1.188048in}{0.766937in}}%
\pgfpathlineto{\pgfqpoint{1.189693in}{1.441401in}}%
\pgfpathlineto{\pgfqpoint{1.191305in}{0.771462in}}%
\pgfpathlineto{\pgfqpoint{1.193019in}{1.458152in}}%
\pgfpathlineto{\pgfqpoint{1.194565in}{0.755484in}}%
\pgfpathlineto{\pgfqpoint{1.196253in}{1.500347in}}%
\pgfpathlineto{\pgfqpoint{1.197830in}{0.788275in}}%
\pgfpathlineto{\pgfqpoint{1.200143in}{1.462709in}}%
\pgfpathlineto{\pgfqpoint{1.201269in}{0.781629in}}%
\pgfpathlineto{\pgfqpoint{1.203183in}{1.484940in}}%
\pgfpathlineto{\pgfqpoint{1.204429in}{0.810314in}}%
\pgfpathlineto{\pgfqpoint{1.206430in}{1.469078in}}%
\pgfpathlineto{\pgfqpoint{1.208110in}{0.770876in}}%
\pgfpathlineto{\pgfqpoint{1.209443in}{1.415212in}}%
\pgfpathlineto{\pgfqpoint{1.211231in}{0.766954in}}%
\pgfpathlineto{\pgfqpoint{1.213145in}{1.442370in}}%
\pgfpathlineto{\pgfqpoint{1.214614in}{0.802995in}}%
\pgfpathlineto{\pgfqpoint{1.216125in}{1.417697in}}%
\pgfpathlineto{\pgfqpoint{1.218077in}{0.725846in}}%
\pgfpathlineto{\pgfqpoint{1.219408in}{1.442198in}}%
\pgfpathlineto{\pgfqpoint{1.221099in}{0.767550in}}%
\pgfpathlineto{\pgfqpoint{1.223142in}{1.459001in}}%
\pgfpathlineto{\pgfqpoint{1.224221in}{0.788859in}}%
\pgfpathlineto{\pgfqpoint{1.226003in}{1.437356in}}%
\pgfpathlineto{\pgfqpoint{1.227490in}{0.799324in}}%
\pgfpathlineto{\pgfqpoint{1.229144in}{1.405901in}}%
\pgfpathlineto{\pgfqpoint{1.230815in}{0.787264in}}%
\pgfpathlineto{\pgfqpoint{1.232827in}{1.454379in}}%
\pgfpathlineto{\pgfqpoint{1.234103in}{0.804824in}}%
\pgfpathlineto{\pgfqpoint{1.235859in}{1.436800in}}%
\pgfpathlineto{\pgfqpoint{1.237592in}{0.769728in}}%
\pgfpathlineto{\pgfqpoint{1.239114in}{1.452884in}}%
\pgfpathlineto{\pgfqpoint{1.240710in}{0.820069in}}%
\pgfpathlineto{\pgfqpoint{1.242424in}{1.452067in}}%
\pgfpathlineto{\pgfqpoint{1.244058in}{0.818653in}}%
\pgfpathlineto{\pgfqpoint{1.245682in}{1.442247in}}%
\pgfpathlineto{\pgfqpoint{1.247851in}{0.699750in}}%
\pgfpathlineto{\pgfqpoint{1.249112in}{1.424056in}}%
\pgfpathlineto{\pgfqpoint{1.250917in}{0.778846in}}%
\pgfpathlineto{\pgfqpoint{1.252636in}{1.463019in}}%
\pgfpathlineto{\pgfqpoint{1.254010in}{0.778122in}}%
\pgfpathlineto{\pgfqpoint{1.256297in}{1.467844in}}%
\pgfpathlineto{\pgfqpoint{1.257158in}{0.786812in}}%
\pgfpathlineto{\pgfqpoint{1.258927in}{1.407619in}}%
\pgfpathlineto{\pgfqpoint{1.260660in}{0.735103in}}%
\pgfpathlineto{\pgfqpoint{1.262261in}{1.440610in}}%
\pgfpathlineto{\pgfqpoint{1.264216in}{0.718090in}}%
\pgfpathlineto{\pgfqpoint{1.265440in}{1.422321in}}%
\pgfpathlineto{\pgfqpoint{1.267307in}{0.756455in}}%
\pgfpathlineto{\pgfqpoint{1.269465in}{1.495594in}}%
\pgfpathlineto{\pgfqpoint{1.270423in}{0.800384in}}%
\pgfpathlineto{\pgfqpoint{1.272133in}{1.460177in}}%
\pgfpathlineto{\pgfqpoint{1.273717in}{0.798870in}}%
\pgfpathlineto{\pgfqpoint{1.275544in}{1.422151in}}%
\pgfpathlineto{\pgfqpoint{1.277176in}{0.771558in}}%
\pgfpathlineto{\pgfqpoint{1.278672in}{1.442166in}}%
\pgfpathlineto{\pgfqpoint{1.280259in}{0.771038in}}%
\pgfpathlineto{\pgfqpoint{1.282086in}{1.445378in}}%
\pgfpathlineto{\pgfqpoint{1.283543in}{0.791376in}}%
\pgfpathlineto{\pgfqpoint{1.285422in}{1.465461in}}%
\pgfpathlineto{\pgfqpoint{1.286925in}{0.795275in}}%
\pgfpathlineto{\pgfqpoint{1.288502in}{1.432064in}}%
\pgfpathlineto{\pgfqpoint{1.290279in}{0.801088in}}%
\pgfpathlineto{\pgfqpoint{1.291970in}{1.427538in}}%
\pgfpathlineto{\pgfqpoint{1.293549in}{0.775525in}}%
\pgfpathlineto{\pgfqpoint{1.295190in}{1.459144in}}%
\pgfpathlineto{\pgfqpoint{1.296968in}{0.723430in}}%
\pgfpathlineto{\pgfqpoint{1.298572in}{1.455916in}}%
\pgfpathlineto{\pgfqpoint{1.300041in}{0.801368in}}%
\pgfpathlineto{\pgfqpoint{1.302639in}{1.478340in}}%
\pgfpathlineto{\pgfqpoint{1.303875in}{0.701501in}}%
\pgfpathlineto{\pgfqpoint{1.305097in}{1.448997in}}%
\pgfpathlineto{\pgfqpoint{1.306785in}{0.766170in}}%
\pgfpathlineto{\pgfqpoint{1.308869in}{1.454211in}}%
\pgfpathlineto{\pgfqpoint{1.310216in}{0.737558in}}%
\pgfpathlineto{\pgfqpoint{1.311809in}{1.432649in}}%
\pgfpathlineto{\pgfqpoint{1.313278in}{0.811686in}}%
\pgfpathlineto{\pgfqpoint{1.314906in}{1.435532in}}%
\pgfpathlineto{\pgfqpoint{1.316569in}{0.780512in}}%
\pgfpathlineto{\pgfqpoint{1.318349in}{1.485703in}}%
\pgfpathlineto{\pgfqpoint{1.319916in}{0.773420in}}%
\pgfpathlineto{\pgfqpoint{1.321555in}{1.403569in}}%
\pgfpathlineto{\pgfqpoint{1.323518in}{0.774758in}}%
\pgfpathlineto{\pgfqpoint{1.325068in}{1.460182in}}%
\pgfpathlineto{\pgfqpoint{1.326563in}{0.781693in}}%
\pgfpathlineto{\pgfqpoint{1.328100in}{1.419533in}}%
\pgfpathlineto{\pgfqpoint{1.329786in}{0.719791in}}%
\pgfpathlineto{\pgfqpoint{1.331734in}{1.438183in}}%
\pgfpathlineto{\pgfqpoint{1.333498in}{0.739726in}}%
\pgfpathlineto{\pgfqpoint{1.334698in}{1.423133in}}%
\pgfpathlineto{\pgfqpoint{1.336565in}{0.768067in}}%
\pgfpathlineto{\pgfqpoint{1.338839in}{1.481727in}}%
\pgfpathlineto{\pgfqpoint{1.339673in}{0.796108in}}%
\pgfpathlineto{\pgfqpoint{1.341636in}{1.461394in}}%
\pgfpathlineto{\pgfqpoint{1.343095in}{0.766249in}}%
\pgfpathlineto{\pgfqpoint{1.344560in}{1.403982in}}%
\pgfpathlineto{\pgfqpoint{1.346239in}{0.777759in}}%
\pgfpathlineto{\pgfqpoint{1.348340in}{1.450172in}}%
\pgfpathlineto{\pgfqpoint{1.349591in}{0.658364in}}%
\pgfpathlineto{\pgfqpoint{1.351227in}{1.418446in}}%
\pgfpathlineto{\pgfqpoint{1.352810in}{0.696715in}}%
\pgfpathlineto{\pgfqpoint{1.354767in}{1.413161in}}%
\pgfpathlineto{\pgfqpoint{1.356372in}{0.753112in}}%
\pgfpathlineto{\pgfqpoint{1.357960in}{1.470073in}}%
\pgfpathlineto{\pgfqpoint{1.359742in}{0.765170in}}%
\pgfpathlineto{\pgfqpoint{1.361143in}{1.410469in}}%
\pgfpathlineto{\pgfqpoint{1.363000in}{0.745077in}}%
\pgfpathlineto{\pgfqpoint{1.364519in}{1.425243in}}%
\pgfpathlineto{\pgfqpoint{1.366113in}{0.771425in}}%
\pgfpathlineto{\pgfqpoint{1.367849in}{1.543434in}}%
\pgfpathlineto{\pgfqpoint{1.369521in}{0.782312in}}%
\pgfpathlineto{\pgfqpoint{1.370938in}{1.431858in}}%
\pgfpathlineto{\pgfqpoint{1.373125in}{0.721331in}}%
\pgfpathlineto{\pgfqpoint{1.374259in}{1.452393in}}%
\pgfpathlineto{\pgfqpoint{1.375921in}{0.792517in}}%
\pgfpathlineto{\pgfqpoint{1.377957in}{1.464700in}}%
\pgfpathlineto{\pgfqpoint{1.379225in}{0.810859in}}%
\pgfpathlineto{\pgfqpoint{1.380788in}{1.382294in}}%
\pgfpathlineto{\pgfqpoint{1.382913in}{0.760582in}}%
\pgfpathlineto{\pgfqpoint{1.384861in}{1.513405in}}%
\pgfpathlineto{\pgfqpoint{1.386080in}{0.751196in}}%
\pgfpathlineto{\pgfqpoint{1.387791in}{1.444299in}}%
\pgfpathlineto{\pgfqpoint{1.389365in}{0.764746in}}%
\pgfpathlineto{\pgfqpoint{1.390689in}{1.446677in}}%
\pgfpathlineto{\pgfqpoint{1.392531in}{0.746974in}}%
\pgfpathlineto{\pgfqpoint{1.394237in}{1.485485in}}%
\pgfpathlineto{\pgfqpoint{1.395873in}{0.734516in}}%
\pgfpathlineto{\pgfqpoint{1.397332in}{1.436880in}}%
\pgfpathlineto{\pgfqpoint{1.399008in}{0.757294in}}%
\pgfpathlineto{\pgfqpoint{1.400568in}{1.413136in}}%
\pgfpathlineto{\pgfqpoint{1.402959in}{0.733680in}}%
\pgfpathlineto{\pgfqpoint{1.403910in}{1.421674in}}%
\pgfpathlineto{\pgfqpoint{1.405622in}{0.778784in}}%
\pgfpathlineto{\pgfqpoint{1.407787in}{1.445623in}}%
\pgfpathlineto{\pgfqpoint{1.409875in}{0.705731in}}%
\pgfpathlineto{\pgfqpoint{1.410553in}{1.432716in}}%
\pgfpathlineto{\pgfqpoint{1.412218in}{0.747619in}}%
\pgfpathlineto{\pgfqpoint{1.413828in}{1.420338in}}%
\pgfpathlineto{\pgfqpoint{1.416150in}{0.714483in}}%
\pgfpathlineto{\pgfqpoint{1.417149in}{1.467297in}}%
\pgfpathlineto{\pgfqpoint{1.418970in}{0.756661in}}%
\pgfpathlineto{\pgfqpoint{1.420744in}{1.547964in}}%
\pgfpathlineto{\pgfqpoint{1.422040in}{0.768278in}}%
\pgfpathlineto{\pgfqpoint{1.424105in}{1.432667in}}%
\pgfpathlineto{\pgfqpoint{1.426099in}{0.642129in}}%
\pgfpathlineto{\pgfqpoint{1.427106in}{1.481658in}}%
\pgfpathlineto{\pgfqpoint{1.428795in}{0.774526in}}%
\pgfpathlineto{\pgfqpoint{1.430327in}{1.411980in}}%
\pgfpathlineto{\pgfqpoint{1.432215in}{0.731207in}}%
\pgfpathlineto{\pgfqpoint{1.434270in}{1.500319in}}%
\pgfpathlineto{\pgfqpoint{1.435178in}{0.789421in}}%
\pgfpathlineto{\pgfqpoint{1.437224in}{1.476982in}}%
\pgfpathlineto{\pgfqpoint{1.438857in}{0.716381in}}%
\pgfpathlineto{\pgfqpoint{1.440245in}{1.400470in}}%
\pgfpathlineto{\pgfqpoint{1.442070in}{0.783661in}}%
\pgfpathlineto{\pgfqpoint{1.443778in}{1.442492in}}%
\pgfpathlineto{\pgfqpoint{1.445082in}{0.772921in}}%
\pgfpathlineto{\pgfqpoint{1.446841in}{1.402478in}}%
\pgfpathlineto{\pgfqpoint{1.448729in}{0.749696in}}%
\pgfpathlineto{\pgfqpoint{1.450148in}{1.435651in}}%
\pgfpathlineto{\pgfqpoint{1.451703in}{0.784772in}}%
\pgfpathlineto{\pgfqpoint{1.454060in}{1.476815in}}%
\pgfpathlineto{\pgfqpoint{1.455710in}{0.733760in}}%
\pgfpathlineto{\pgfqpoint{1.456731in}{1.437782in}}%
\pgfpathlineto{\pgfqpoint{1.458305in}{0.780965in}}%
\pgfpathlineto{\pgfqpoint{1.461262in}{1.519287in}}%
\pgfpathlineto{\pgfqpoint{1.462043in}{0.717259in}}%
\pgfpathlineto{\pgfqpoint{1.463504in}{1.417596in}}%
\pgfpathlineto{\pgfqpoint{1.464868in}{0.816810in}}%
\pgfpathlineto{\pgfqpoint{1.466612in}{1.458764in}}%
\pgfpathlineto{\pgfqpoint{1.468236in}{0.742207in}}%
\pgfpathlineto{\pgfqpoint{1.470208in}{1.480716in}}%
\pgfpathlineto{\pgfqpoint{1.471832in}{0.722593in}}%
\pgfpathlineto{\pgfqpoint{1.473131in}{1.422598in}}%
\pgfpathlineto{\pgfqpoint{1.474911in}{0.773398in}}%
\pgfpathlineto{\pgfqpoint{1.476509in}{1.416204in}}%
\pgfpathlineto{\pgfqpoint{1.478159in}{0.768287in}}%
\pgfpathlineto{\pgfqpoint{1.480011in}{1.478802in}}%
\pgfpathlineto{\pgfqpoint{1.481361in}{0.729130in}}%
\pgfpathlineto{\pgfqpoint{1.483059in}{1.433048in}}%
\pgfpathlineto{\pgfqpoint{1.484668in}{0.756853in}}%
\pgfpathlineto{\pgfqpoint{1.486451in}{1.405894in}}%
\pgfpathlineto{\pgfqpoint{1.488021in}{0.784138in}}%
\pgfpathlineto{\pgfqpoint{1.489678in}{1.429267in}}%
\pgfpathlineto{\pgfqpoint{1.491815in}{0.759892in}}%
\pgfpathlineto{\pgfqpoint{1.492962in}{1.467270in}}%
\pgfpathlineto{\pgfqpoint{1.495145in}{0.715841in}}%
\pgfpathlineto{\pgfqpoint{1.496179in}{1.391166in}}%
\pgfpathlineto{\pgfqpoint{1.498327in}{0.774926in}}%
\pgfpathlineto{\pgfqpoint{1.499857in}{1.448166in}}%
\pgfpathlineto{\pgfqpoint{1.501176in}{0.818882in}}%
\pgfpathlineto{\pgfqpoint{1.503106in}{1.449141in}}%
\pgfpathlineto{\pgfqpoint{1.504547in}{0.738439in}}%
\pgfpathlineto{\pgfqpoint{1.506285in}{1.518518in}}%
\pgfpathlineto{\pgfqpoint{1.508106in}{0.738979in}}%
\pgfpathlineto{\pgfqpoint{1.509367in}{1.397014in}}%
\pgfpathlineto{\pgfqpoint{1.511193in}{0.739437in}}%
\pgfpathlineto{\pgfqpoint{1.512820in}{1.446165in}}%
\pgfpathlineto{\pgfqpoint{1.514510in}{0.762249in}}%
\pgfpathlineto{\pgfqpoint{1.516272in}{1.472156in}}%
\pgfpathlineto{\pgfqpoint{1.517666in}{0.810738in}}%
\pgfpathlineto{\pgfqpoint{1.519557in}{1.481097in}}%
\pgfpathlineto{\pgfqpoint{1.521121in}{0.743950in}}%
\pgfpathlineto{\pgfqpoint{1.522715in}{1.431995in}}%
\pgfpathlineto{\pgfqpoint{1.524231in}{0.828610in}}%
\pgfpathlineto{\pgfqpoint{1.526307in}{1.445119in}}%
\pgfpathlineto{\pgfqpoint{1.528044in}{0.774767in}}%
\pgfpathlineto{\pgfqpoint{1.529226in}{1.404959in}}%
\pgfpathlineto{\pgfqpoint{1.530845in}{0.787751in}}%
\pgfpathlineto{\pgfqpoint{1.532564in}{1.414129in}}%
\pgfpathlineto{\pgfqpoint{1.534097in}{0.816126in}}%
\pgfpathlineto{\pgfqpoint{1.535915in}{1.454213in}}%
\pgfpathlineto{\pgfqpoint{1.537856in}{0.724747in}}%
\pgfpathlineto{\pgfqpoint{1.539196in}{1.449105in}}%
\pgfpathlineto{\pgfqpoint{1.540860in}{0.792475in}}%
\pgfpathlineto{\pgfqpoint{1.542326in}{1.372604in}}%
\pgfpathlineto{\pgfqpoint{1.544128in}{0.775506in}}%
\pgfpathlineto{\pgfqpoint{1.545890in}{1.423477in}}%
\pgfpathlineto{\pgfqpoint{1.547457in}{0.774094in}}%
\pgfpathlineto{\pgfqpoint{1.548962in}{1.446465in}}%
\pgfpathlineto{\pgfqpoint{1.550803in}{0.785424in}}%
\pgfpathlineto{\pgfqpoint{1.552256in}{1.446132in}}%
\pgfpathlineto{\pgfqpoint{1.554100in}{0.790015in}}%
\pgfpathlineto{\pgfqpoint{1.555823in}{1.461446in}}%
\pgfpathlineto{\pgfqpoint{1.557295in}{0.792195in}}%
\pgfpathlineto{\pgfqpoint{1.559195in}{1.452343in}}%
\pgfpathlineto{\pgfqpoint{1.560546in}{0.795039in}}%
\pgfpathlineto{\pgfqpoint{1.562552in}{1.442629in}}%
\pgfpathlineto{\pgfqpoint{1.563833in}{0.760646in}}%
\pgfpathlineto{\pgfqpoint{1.565582in}{1.434882in}}%
\pgfpathlineto{\pgfqpoint{1.567334in}{0.711003in}}%
\pgfpathlineto{\pgfqpoint{1.568731in}{1.391833in}}%
\pgfpathlineto{\pgfqpoint{1.570374in}{0.817471in}}%
\pgfpathlineto{\pgfqpoint{1.572317in}{1.512442in}}%
\pgfpathlineto{\pgfqpoint{1.573783in}{0.770269in}}%
\pgfpathlineto{\pgfqpoint{1.575746in}{1.457601in}}%
\pgfpathlineto{\pgfqpoint{1.576937in}{0.796510in}}%
\pgfpathlineto{\pgfqpoint{1.578664in}{1.457673in}}%
\pgfpathlineto{\pgfqpoint{1.580430in}{0.806658in}}%
\pgfpathlineto{\pgfqpoint{1.581937in}{1.408784in}}%
\pgfpathlineto{\pgfqpoint{1.583588in}{0.803521in}}%
\pgfpathlineto{\pgfqpoint{1.585338in}{1.402948in}}%
\pgfpathlineto{\pgfqpoint{1.587315in}{0.747842in}}%
\pgfpathlineto{\pgfqpoint{1.588552in}{1.423317in}}%
\pgfpathlineto{\pgfqpoint{1.590224in}{0.776402in}}%
\pgfpathlineto{\pgfqpoint{1.592202in}{1.472500in}}%
\pgfpathlineto{\pgfqpoint{1.593593in}{0.790982in}}%
\pgfpathlineto{\pgfqpoint{1.595530in}{1.451755in}}%
\pgfpathlineto{\pgfqpoint{1.596905in}{0.745753in}}%
\pgfpathlineto{\pgfqpoint{1.598755in}{1.470417in}}%
\pgfpathlineto{\pgfqpoint{1.600194in}{0.780502in}}%
\pgfpathlineto{\pgfqpoint{1.601701in}{1.437850in}}%
\pgfpathlineto{\pgfqpoint{1.603390in}{0.780438in}}%
\pgfpathlineto{\pgfqpoint{1.605073in}{1.393177in}}%
\pgfpathlineto{\pgfqpoint{1.606840in}{0.776692in}}%
\pgfpathlineto{\pgfqpoint{1.608294in}{1.458038in}}%
\pgfpathlineto{\pgfqpoint{1.610291in}{0.739008in}}%
\pgfpathlineto{\pgfqpoint{1.611982in}{1.472096in}}%
\pgfpathlineto{\pgfqpoint{1.613433in}{0.789844in}}%
\pgfpathlineto{\pgfqpoint{1.615018in}{1.415550in}}%
\pgfpathlineto{\pgfqpoint{1.616901in}{0.758632in}}%
\pgfpathlineto{\pgfqpoint{1.618264in}{1.448853in}}%
\pgfpathlineto{\pgfqpoint{1.619853in}{0.795341in}}%
\pgfpathlineto{\pgfqpoint{1.621965in}{1.427934in}}%
\pgfpathlineto{\pgfqpoint{1.623188in}{0.782732in}}%
\pgfpathlineto{\pgfqpoint{1.625351in}{1.467226in}}%
\pgfpathlineto{\pgfqpoint{1.626613in}{0.788028in}}%
\pgfpathlineto{\pgfqpoint{1.628826in}{1.470665in}}%
\pgfpathlineto{\pgfqpoint{1.629949in}{0.790777in}}%
\pgfpathlineto{\pgfqpoint{1.631871in}{1.462962in}}%
\pgfpathlineto{\pgfqpoint{1.633215in}{0.739959in}}%
\pgfpathlineto{\pgfqpoint{1.634697in}{1.415208in}}%
\pgfpathlineto{\pgfqpoint{1.636448in}{0.778506in}}%
\pgfpathlineto{\pgfqpoint{1.638007in}{1.435022in}}%
\pgfpathlineto{\pgfqpoint{1.640481in}{0.719558in}}%
\pgfpathlineto{\pgfqpoint{1.641404in}{1.430385in}}%
\pgfpathlineto{\pgfqpoint{1.643545in}{0.707980in}}%
\pgfpathlineto{\pgfqpoint{1.644894in}{1.448282in}}%
\pgfpathlineto{\pgfqpoint{1.646457in}{0.700047in}}%
\pgfpathlineto{\pgfqpoint{1.648134in}{1.419490in}}%
\pgfpathlineto{\pgfqpoint{1.649674in}{0.763617in}}%
\pgfpathlineto{\pgfqpoint{1.651391in}{1.494688in}}%
\pgfpathlineto{\pgfqpoint{1.653161in}{0.765913in}}%
\pgfpathlineto{\pgfqpoint{1.654463in}{1.413020in}}%
\pgfpathlineto{\pgfqpoint{1.656778in}{0.713466in}}%
\pgfpathlineto{\pgfqpoint{1.657775in}{1.399826in}}%
\pgfpathlineto{\pgfqpoint{1.659557in}{0.703410in}}%
\pgfpathlineto{\pgfqpoint{1.661445in}{1.433977in}}%
\pgfpathlineto{\pgfqpoint{1.662976in}{0.760593in}}%
\pgfpathlineto{\pgfqpoint{1.664310in}{1.442479in}}%
\pgfpathlineto{\pgfqpoint{1.666302in}{0.721066in}}%
\pgfpathlineto{\pgfqpoint{1.667726in}{1.414356in}}%
\pgfpathlineto{\pgfqpoint{1.669267in}{0.791809in}}%
\pgfpathlineto{\pgfqpoint{1.671280in}{1.442447in}}%
\pgfpathlineto{\pgfqpoint{1.672589in}{0.791152in}}%
\pgfpathlineto{\pgfqpoint{1.674389in}{1.434517in}}%
\pgfpathlineto{\pgfqpoint{1.676509in}{0.692775in}}%
\pgfpathlineto{\pgfqpoint{1.677487in}{1.437494in}}%
\pgfpathlineto{\pgfqpoint{1.679195in}{0.799025in}}%
\pgfpathlineto{\pgfqpoint{1.681492in}{1.472151in}}%
\pgfpathlineto{\pgfqpoint{1.682615in}{0.746473in}}%
\pgfpathlineto{\pgfqpoint{1.684162in}{1.424299in}}%
\pgfpathlineto{\pgfqpoint{1.686212in}{0.701839in}}%
\pgfpathlineto{\pgfqpoint{1.687511in}{1.406016in}}%
\pgfpathlineto{\pgfqpoint{1.689104in}{0.809204in}}%
\pgfpathlineto{\pgfqpoint{1.691244in}{1.500407in}}%
\pgfpathlineto{\pgfqpoint{1.692695in}{0.777740in}}%
\pgfpathlineto{\pgfqpoint{1.694067in}{1.397803in}}%
\pgfpathlineto{\pgfqpoint{1.695790in}{0.791246in}}%
\pgfpathlineto{\pgfqpoint{1.697945in}{1.493517in}}%
\pgfpathlineto{\pgfqpoint{1.698961in}{0.749052in}}%
\pgfpathlineto{\pgfqpoint{1.700828in}{1.448331in}}%
\pgfpathlineto{\pgfqpoint{1.702251in}{0.739588in}}%
\pgfpathlineto{\pgfqpoint{1.704548in}{1.454350in}}%
\pgfpathlineto{\pgfqpoint{1.705645in}{0.772105in}}%
\pgfpathlineto{\pgfqpoint{1.707171in}{1.457787in}}%
\pgfpathlineto{\pgfqpoint{1.708920in}{0.788070in}}%
\pgfpathlineto{\pgfqpoint{1.710600in}{1.421975in}}%
\pgfpathlineto{\pgfqpoint{1.712165in}{0.732268in}}%
\pgfpathlineto{\pgfqpoint{1.714583in}{1.470509in}}%
\pgfpathlineto{\pgfqpoint{1.715657in}{0.734119in}}%
\pgfpathlineto{\pgfqpoint{1.717567in}{1.463053in}}%
\pgfpathlineto{\pgfqpoint{1.718820in}{0.796966in}}%
\pgfpathlineto{\pgfqpoint{1.720503in}{1.435775in}}%
\pgfpathlineto{\pgfqpoint{1.722110in}{0.773186in}}%
\pgfpathlineto{\pgfqpoint{1.723808in}{1.418054in}}%
\pgfpathlineto{\pgfqpoint{1.725799in}{0.737462in}}%
\pgfpathlineto{\pgfqpoint{1.727039in}{1.453983in}}%
\pgfpathlineto{\pgfqpoint{1.729402in}{0.685264in}}%
\pgfpathlineto{\pgfqpoint{1.730286in}{1.435711in}}%
\pgfpathlineto{\pgfqpoint{1.732121in}{0.777766in}}%
\pgfpathlineto{\pgfqpoint{1.733966in}{1.417092in}}%
\pgfpathlineto{\pgfqpoint{1.735282in}{0.783162in}}%
\pgfpathlineto{\pgfqpoint{1.736874in}{1.407694in}}%
\pgfpathlineto{\pgfqpoint{1.738699in}{0.744198in}}%
\pgfpathlineto{\pgfqpoint{1.740290in}{1.467387in}}%
\pgfpathlineto{\pgfqpoint{1.742438in}{0.710297in}}%
\pgfpathlineto{\pgfqpoint{1.743690in}{1.430654in}}%
\pgfpathlineto{\pgfqpoint{1.745095in}{0.792752in}}%
\pgfpathlineto{\pgfqpoint{1.747339in}{1.459986in}}%
\pgfpathlineto{\pgfqpoint{1.748586in}{0.749342in}}%
\pgfpathlineto{\pgfqpoint{1.750550in}{1.439848in}}%
\pgfpathlineto{\pgfqpoint{1.751670in}{0.766656in}}%
\pgfpathlineto{\pgfqpoint{1.753685in}{1.506426in}}%
\pgfpathlineto{\pgfqpoint{1.755106in}{0.773563in}}%
\pgfpathlineto{\pgfqpoint{1.757034in}{1.427105in}}%
\pgfpathlineto{\pgfqpoint{1.758856in}{0.776581in}}%
\pgfpathlineto{\pgfqpoint{1.760197in}{1.446462in}}%
\pgfpathlineto{\pgfqpoint{1.761607in}{0.788627in}}%
\pgfpathlineto{\pgfqpoint{1.763490in}{1.490227in}}%
\pgfpathlineto{\pgfqpoint{1.764954in}{0.771370in}}%
\pgfpathlineto{\pgfqpoint{1.766942in}{1.542403in}}%
\pgfpathlineto{\pgfqpoint{1.768150in}{0.778393in}}%
\pgfpathlineto{\pgfqpoint{1.770315in}{1.430022in}}%
\pgfpathlineto{\pgfqpoint{1.771532in}{0.783931in}}%
\pgfpathlineto{\pgfqpoint{1.773202in}{1.422059in}}%
\pgfpathlineto{\pgfqpoint{1.774786in}{0.797621in}}%
\pgfpathlineto{\pgfqpoint{1.776405in}{1.437903in}}%
\pgfpathlineto{\pgfqpoint{1.778162in}{0.773637in}}%
\pgfpathlineto{\pgfqpoint{1.780149in}{1.462998in}}%
\pgfpathlineto{\pgfqpoint{1.781342in}{0.772987in}}%
\pgfpathlineto{\pgfqpoint{1.783060in}{1.433015in}}%
\pgfpathlineto{\pgfqpoint{1.784719in}{0.798327in}}%
\pgfpathlineto{\pgfqpoint{1.787122in}{1.518675in}}%
\pgfpathlineto{\pgfqpoint{1.787924in}{0.821485in}}%
\pgfpathlineto{\pgfqpoint{1.789613in}{1.396914in}}%
\pgfpathlineto{\pgfqpoint{1.791246in}{0.763887in}}%
\pgfpathlineto{\pgfqpoint{1.792967in}{1.430301in}}%
\pgfpathlineto{\pgfqpoint{1.794702in}{0.799536in}}%
\pgfpathlineto{\pgfqpoint{1.796402in}{1.412416in}}%
\pgfpathlineto{\pgfqpoint{1.798005in}{0.692726in}}%
\pgfpathlineto{\pgfqpoint{1.799498in}{1.391077in}}%
\pgfpathlineto{\pgfqpoint{1.801684in}{0.721671in}}%
\pgfpathlineto{\pgfqpoint{1.802888in}{1.446701in}}%
\pgfpathlineto{\pgfqpoint{1.804604in}{0.716378in}}%
\pgfpathlineto{\pgfqpoint{1.806133in}{1.418625in}}%
\pgfpathlineto{\pgfqpoint{1.808795in}{0.643737in}}%
\pgfpathlineto{\pgfqpoint{1.809392in}{1.396063in}}%
\pgfpathlineto{\pgfqpoint{1.811001in}{0.744066in}}%
\pgfpathlineto{\pgfqpoint{1.812725in}{1.384085in}}%
\pgfpathlineto{\pgfqpoint{1.814813in}{0.750853in}}%
\pgfpathlineto{\pgfqpoint{1.815982in}{1.421428in}}%
\pgfpathlineto{\pgfqpoint{1.817896in}{0.729720in}}%
\pgfpathlineto{\pgfqpoint{1.819306in}{1.449267in}}%
\pgfpathlineto{\pgfqpoint{1.820990in}{0.773537in}}%
\pgfpathlineto{\pgfqpoint{1.822608in}{1.414711in}}%
\pgfpathlineto{\pgfqpoint{1.824785in}{0.750199in}}%
\pgfpathlineto{\pgfqpoint{1.826065in}{1.424323in}}%
\pgfpathlineto{\pgfqpoint{1.827737in}{0.780264in}}%
\pgfpathlineto{\pgfqpoint{1.829381in}{1.433162in}}%
\pgfpathlineto{\pgfqpoint{1.831092in}{0.722086in}}%
\pgfpathlineto{\pgfqpoint{1.832501in}{1.429528in}}%
\pgfpathlineto{\pgfqpoint{1.834274in}{0.770918in}}%
\pgfpathlineto{\pgfqpoint{1.836101in}{1.431013in}}%
\pgfpathlineto{\pgfqpoint{1.837444in}{0.780822in}}%
\pgfpathlineto{\pgfqpoint{1.839203in}{1.445707in}}%
\pgfpathlineto{\pgfqpoint{1.840888in}{0.787250in}}%
\pgfpathlineto{\pgfqpoint{1.842321in}{1.405202in}}%
\pgfpathlineto{\pgfqpoint{1.844038in}{0.807966in}}%
\pgfpathlineto{\pgfqpoint{1.845913in}{1.493333in}}%
\pgfpathlineto{\pgfqpoint{1.847718in}{0.759511in}}%
\pgfpathlineto{\pgfqpoint{1.848985in}{1.415877in}}%
\pgfpathlineto{\pgfqpoint{1.851204in}{0.758976in}}%
\pgfpathlineto{\pgfqpoint{1.852715in}{1.484143in}}%
\pgfpathlineto{\pgfqpoint{1.853866in}{0.793491in}}%
\pgfpathlineto{\pgfqpoint{1.855813in}{1.405388in}}%
\pgfpathlineto{\pgfqpoint{1.857307in}{0.750157in}}%
\pgfpathlineto{\pgfqpoint{1.858818in}{1.423364in}}%
\pgfpathlineto{\pgfqpoint{1.861134in}{0.686044in}}%
\pgfpathlineto{\pgfqpoint{1.862439in}{1.454337in}}%
\pgfpathlineto{\pgfqpoint{1.863795in}{0.791744in}}%
\pgfpathlineto{\pgfqpoint{1.865590in}{1.459903in}}%
\pgfpathlineto{\pgfqpoint{1.867077in}{0.779192in}}%
\pgfpathlineto{\pgfqpoint{1.868849in}{1.476732in}}%
\pgfpathlineto{\pgfqpoint{1.870383in}{0.810577in}}%
\pgfpathlineto{\pgfqpoint{1.872161in}{1.488318in}}%
\pgfpathlineto{\pgfqpoint{1.873678in}{0.765374in}}%
\pgfpathlineto{\pgfqpoint{1.875566in}{1.434806in}}%
\pgfpathlineto{\pgfqpoint{1.876953in}{0.796449in}}%
\pgfpathlineto{\pgfqpoint{1.878792in}{1.449607in}}%
\pgfpathlineto{\pgfqpoint{1.880255in}{0.824964in}}%
\pgfpathlineto{\pgfqpoint{1.882185in}{1.445030in}}%
\pgfpathlineto{\pgfqpoint{1.883721in}{0.752515in}}%
\pgfpathlineto{\pgfqpoint{1.885198in}{1.392062in}}%
\pgfpathlineto{\pgfqpoint{1.887054in}{0.770899in}}%
\pgfpathlineto{\pgfqpoint{1.888741in}{1.482634in}}%
\pgfpathlineto{\pgfqpoint{1.890354in}{0.746214in}}%
\pgfpathlineto{\pgfqpoint{1.891901in}{1.443245in}}%
\pgfpathlineto{\pgfqpoint{1.894006in}{0.769012in}}%
\pgfpathlineto{\pgfqpoint{1.895188in}{1.444800in}}%
\pgfpathlineto{\pgfqpoint{1.896928in}{0.767952in}}%
\pgfpathlineto{\pgfqpoint{1.898662in}{1.454926in}}%
\pgfpathlineto{\pgfqpoint{1.900082in}{0.781707in}}%
\pgfpathlineto{\pgfqpoint{1.902417in}{1.483975in}}%
\pgfpathlineto{\pgfqpoint{1.903575in}{0.737038in}}%
\pgfpathlineto{\pgfqpoint{1.905227in}{1.438063in}}%
\pgfpathlineto{\pgfqpoint{1.906897in}{0.773701in}}%
\pgfpathlineto{\pgfqpoint{1.908974in}{1.463251in}}%
\pgfpathlineto{\pgfqpoint{1.909906in}{0.797067in}}%
\pgfpathlineto{\pgfqpoint{1.911864in}{1.427734in}}%
\pgfpathlineto{\pgfqpoint{1.913237in}{0.795460in}}%
\pgfpathlineto{\pgfqpoint{1.915249in}{1.414886in}}%
\pgfpathlineto{\pgfqpoint{1.916708in}{0.767320in}}%
\pgfpathlineto{\pgfqpoint{1.918380in}{1.495082in}}%
\pgfpathlineto{\pgfqpoint{1.919815in}{0.776104in}}%
\pgfpathlineto{\pgfqpoint{1.921685in}{1.439300in}}%
\pgfpathlineto{\pgfqpoint{1.923717in}{0.669844in}}%
\pgfpathlineto{\pgfqpoint{1.924866in}{1.401513in}}%
\pgfpathlineto{\pgfqpoint{1.926493in}{0.801710in}}%
\pgfpathlineto{\pgfqpoint{1.928220in}{1.456241in}}%
\pgfpathlineto{\pgfqpoint{1.929958in}{0.706015in}}%
\pgfpathlineto{\pgfqpoint{1.931927in}{1.468016in}}%
\pgfpathlineto{\pgfqpoint{1.932999in}{0.762295in}}%
\pgfpathlineto{\pgfqpoint{1.934721in}{1.450723in}}%
\pgfpathlineto{\pgfqpoint{1.937024in}{0.717418in}}%
\pgfpathlineto{\pgfqpoint{1.937994in}{1.418088in}}%
\pgfpathlineto{\pgfqpoint{1.939973in}{0.768325in}}%
\pgfpathlineto{\pgfqpoint{1.941252in}{1.482382in}}%
\pgfpathlineto{\pgfqpoint{1.943476in}{0.725005in}}%
\pgfpathlineto{\pgfqpoint{1.944981in}{1.476355in}}%
\pgfpathlineto{\pgfqpoint{1.946182in}{0.762822in}}%
\pgfpathlineto{\pgfqpoint{1.948001in}{1.426717in}}%
\pgfpathlineto{\pgfqpoint{1.949580in}{0.761042in}}%
\pgfpathlineto{\pgfqpoint{1.951361in}{1.438849in}}%
\pgfpathlineto{\pgfqpoint{1.953404in}{0.718518in}}%
\pgfpathlineto{\pgfqpoint{1.954418in}{1.402776in}}%
\pgfpathlineto{\pgfqpoint{1.956970in}{0.733031in}}%
\pgfpathlineto{\pgfqpoint{1.957702in}{1.386877in}}%
\pgfpathlineto{\pgfqpoint{1.959663in}{0.781392in}}%
\pgfpathlineto{\pgfqpoint{1.961071in}{1.447359in}}%
\pgfpathlineto{\pgfqpoint{1.962775in}{0.774544in}}%
\pgfpathlineto{\pgfqpoint{1.964598in}{1.437187in}}%
\pgfpathlineto{\pgfqpoint{1.966014in}{0.777095in}}%
\pgfpathlineto{\pgfqpoint{1.967766in}{1.419255in}}%
\pgfpathlineto{\pgfqpoint{1.969236in}{0.751959in}}%
\pgfpathlineto{\pgfqpoint{1.971138in}{1.426533in}}%
\pgfpathlineto{\pgfqpoint{1.973453in}{0.708462in}}%
\pgfpathlineto{\pgfqpoint{1.974808in}{1.514223in}}%
\pgfpathlineto{\pgfqpoint{1.975862in}{0.791250in}}%
\pgfpathlineto{\pgfqpoint{1.977871in}{1.441415in}}%
\pgfpathlineto{\pgfqpoint{1.979240in}{0.769429in}}%
\pgfpathlineto{\pgfqpoint{1.981037in}{1.432360in}}%
\pgfpathlineto{\pgfqpoint{1.983708in}{0.673788in}}%
\pgfpathlineto{\pgfqpoint{1.984196in}{1.488455in}}%
\pgfpathlineto{\pgfqpoint{1.986182in}{0.730610in}}%
\pgfpathlineto{\pgfqpoint{1.987916in}{1.453850in}}%
\pgfpathlineto{\pgfqpoint{1.989248in}{0.797037in}}%
\pgfpathlineto{\pgfqpoint{1.991025in}{1.427300in}}%
\pgfpathlineto{\pgfqpoint{1.992956in}{0.748313in}}%
\pgfpathlineto{\pgfqpoint{1.993961in}{1.466476in}}%
\pgfpathlineto{\pgfqpoint{1.995982in}{0.736905in}}%
\pgfpathlineto{\pgfqpoint{1.997460in}{1.423189in}}%
\pgfpathlineto{\pgfqpoint{1.998935in}{0.688578in}}%
\pgfpathlineto{\pgfqpoint{2.000718in}{1.425469in}}%
\pgfpathlineto{\pgfqpoint{2.002344in}{0.785024in}}%
\pgfpathlineto{\pgfqpoint{2.004462in}{1.456718in}}%
\pgfpathlineto{\pgfqpoint{2.005684in}{0.774356in}}%
\pgfpathlineto{\pgfqpoint{2.007315in}{1.467386in}}%
\pgfpathlineto{\pgfqpoint{2.008916in}{0.751238in}}%
\pgfpathlineto{\pgfqpoint{2.010866in}{1.490942in}}%
\pgfpathlineto{\pgfqpoint{2.012172in}{0.749012in}}%
\pgfpathlineto{\pgfqpoint{2.014014in}{1.452237in}}%
\pgfpathlineto{\pgfqpoint{2.015797in}{0.747839in}}%
\pgfpathlineto{\pgfqpoint{2.017256in}{1.435993in}}%
\pgfpathlineto{\pgfqpoint{2.018715in}{0.696320in}}%
\pgfpathlineto{\pgfqpoint{2.020502in}{1.429795in}}%
\pgfpathlineto{\pgfqpoint{2.022118in}{0.759717in}}%
\pgfpathlineto{\pgfqpoint{2.023869in}{1.475756in}}%
\pgfpathlineto{\pgfqpoint{2.025472in}{0.801731in}}%
\pgfpathlineto{\pgfqpoint{2.027082in}{1.454776in}}%
\pgfpathlineto{\pgfqpoint{2.028919in}{0.782775in}}%
\pgfpathlineto{\pgfqpoint{2.030260in}{1.399729in}}%
\pgfpathlineto{\pgfqpoint{2.032247in}{0.731766in}}%
\pgfpathlineto{\pgfqpoint{2.033644in}{1.473795in}}%
\pgfpathlineto{\pgfqpoint{2.035259in}{0.806137in}}%
\pgfpathlineto{\pgfqpoint{2.036848in}{1.461335in}}%
\pgfpathlineto{\pgfqpoint{2.038683in}{0.775732in}}%
\pgfpathlineto{\pgfqpoint{2.040176in}{1.427363in}}%
\pgfpathlineto{\pgfqpoint{2.041840in}{0.790642in}}%
\pgfpathlineto{\pgfqpoint{2.044090in}{1.454527in}}%
\pgfpathlineto{\pgfqpoint{2.045252in}{0.747242in}}%
\pgfpathlineto{\pgfqpoint{2.046740in}{1.422405in}}%
\pgfpathlineto{\pgfqpoint{2.048475in}{0.740809in}}%
\pgfpathlineto{\pgfqpoint{2.050221in}{1.445732in}}%
\pgfpathlineto{\pgfqpoint{2.051891in}{0.773105in}}%
\pgfpathlineto{\pgfqpoint{2.053466in}{1.456963in}}%
\pgfpathlineto{\pgfqpoint{2.054980in}{0.766324in}}%
\pgfpathlineto{\pgfqpoint{2.057289in}{1.443768in}}%
\pgfpathlineto{\pgfqpoint{2.058407in}{0.759790in}}%
\pgfpathlineto{\pgfqpoint{2.059992in}{1.408946in}}%
\pgfpathlineto{\pgfqpoint{2.061767in}{0.797822in}}%
\pgfpathlineto{\pgfqpoint{2.063306in}{1.458095in}}%
\pgfpathlineto{\pgfqpoint{2.064885in}{0.800037in}}%
\pgfpathlineto{\pgfqpoint{2.066552in}{1.403308in}}%
\pgfpathlineto{\pgfqpoint{2.068292in}{0.719306in}}%
\pgfpathlineto{\pgfqpoint{2.069832in}{1.428689in}}%
\pgfpathlineto{\pgfqpoint{2.071662in}{0.755832in}}%
\pgfpathlineto{\pgfqpoint{2.073137in}{1.429217in}}%
\pgfpathlineto{\pgfqpoint{2.074844in}{0.778014in}}%
\pgfpathlineto{\pgfqpoint{2.077292in}{1.477326in}}%
\pgfpathlineto{\pgfqpoint{2.078045in}{0.780817in}}%
\pgfpathlineto{\pgfqpoint{2.080012in}{1.429672in}}%
\pgfpathlineto{\pgfqpoint{2.081794in}{0.754208in}}%
\pgfpathlineto{\pgfqpoint{2.083085in}{1.426730in}}%
\pgfpathlineto{\pgfqpoint{2.084962in}{0.778213in}}%
\pgfpathlineto{\pgfqpoint{2.086391in}{1.433786in}}%
\pgfpathlineto{\pgfqpoint{2.087935in}{0.818912in}}%
\pgfpathlineto{\pgfqpoint{2.089597in}{1.407030in}}%
\pgfpathlineto{\pgfqpoint{2.091782in}{0.674995in}}%
\pgfpathlineto{\pgfqpoint{2.092858in}{1.407836in}}%
\pgfpathlineto{\pgfqpoint{2.094584in}{0.804176in}}%
\pgfpathlineto{\pgfqpoint{2.096434in}{1.452709in}}%
\pgfpathlineto{\pgfqpoint{2.097969in}{0.790189in}}%
\pgfpathlineto{\pgfqpoint{2.099565in}{1.437145in}}%
\pgfpathlineto{\pgfqpoint{2.101209in}{0.757598in}}%
\pgfpathlineto{\pgfqpoint{2.102965in}{1.424414in}}%
\pgfpathlineto{\pgfqpoint{2.104587in}{0.795107in}}%
\pgfpathlineto{\pgfqpoint{2.106273in}{1.417449in}}%
\pgfpathlineto{\pgfqpoint{2.108216in}{0.761456in}}%
\pgfpathlineto{\pgfqpoint{2.109348in}{1.447478in}}%
\pgfpathlineto{\pgfqpoint{2.111358in}{0.808808in}}%
\pgfpathlineto{\pgfqpoint{2.113382in}{1.486056in}}%
\pgfpathlineto{\pgfqpoint{2.114385in}{0.817801in}}%
\pgfpathlineto{\pgfqpoint{2.116240in}{1.415425in}}%
\pgfpathlineto{\pgfqpoint{2.117885in}{0.781813in}}%
\pgfpathlineto{\pgfqpoint{2.119343in}{1.413961in}}%
\pgfpathlineto{\pgfqpoint{2.121138in}{0.739544in}}%
\pgfpathlineto{\pgfqpoint{2.122545in}{1.440055in}}%
\pgfpathlineto{\pgfqpoint{2.124324in}{0.754818in}}%
\pgfpathlineto{\pgfqpoint{2.126126in}{1.434268in}}%
\pgfpathlineto{\pgfqpoint{2.127955in}{0.777619in}}%
\pgfpathlineto{\pgfqpoint{2.129364in}{1.395562in}}%
\pgfpathlineto{\pgfqpoint{2.130982in}{0.805156in}}%
\pgfpathlineto{\pgfqpoint{2.132857in}{1.463664in}}%
\pgfpathlineto{\pgfqpoint{2.134222in}{0.778022in}}%
\pgfpathlineto{\pgfqpoint{2.136276in}{1.479148in}}%
\pgfpathlineto{\pgfqpoint{2.137578in}{0.787975in}}%
\pgfpathlineto{\pgfqpoint{2.139119in}{1.424901in}}%
\pgfpathlineto{\pgfqpoint{2.140994in}{0.770413in}}%
\pgfpathlineto{\pgfqpoint{2.142375in}{1.393051in}}%
\pgfpathlineto{\pgfqpoint{2.144758in}{0.752153in}}%
\pgfpathlineto{\pgfqpoint{2.145785in}{1.441344in}}%
\pgfpathlineto{\pgfqpoint{2.147478in}{0.793257in}}%
\pgfpathlineto{\pgfqpoint{2.148948in}{1.423369in}}%
\pgfpathlineto{\pgfqpoint{2.150680in}{0.770411in}}%
\pgfpathlineto{\pgfqpoint{2.152237in}{1.421096in}}%
\pgfpathlineto{\pgfqpoint{2.153859in}{0.762600in}}%
\pgfpathlineto{\pgfqpoint{2.155697in}{1.418123in}}%
\pgfpathlineto{\pgfqpoint{2.157253in}{0.759607in}}%
\pgfpathlineto{\pgfqpoint{2.159243in}{1.444543in}}%
\pgfpathlineto{\pgfqpoint{2.160931in}{0.755786in}}%
\pgfpathlineto{\pgfqpoint{2.162160in}{1.436981in}}%
\pgfpathlineto{\pgfqpoint{2.164490in}{0.749492in}}%
\pgfpathlineto{\pgfqpoint{2.165562in}{1.437545in}}%
\pgfpathlineto{\pgfqpoint{2.167147in}{0.748241in}}%
\pgfpathlineto{\pgfqpoint{2.169006in}{1.493648in}}%
\pgfpathlineto{\pgfqpoint{2.170533in}{0.743094in}}%
\pgfpathlineto{\pgfqpoint{2.172058in}{1.477129in}}%
\pgfpathlineto{\pgfqpoint{2.173686in}{0.797033in}}%
\pgfpathlineto{\pgfqpoint{2.175366in}{1.419695in}}%
\pgfpathlineto{\pgfqpoint{2.177185in}{0.773192in}}%
\pgfpathlineto{\pgfqpoint{2.179034in}{1.459331in}}%
\pgfpathlineto{\pgfqpoint{2.180230in}{0.814414in}}%
\pgfpathlineto{\pgfqpoint{2.181936in}{1.444745in}}%
\pgfpathlineto{\pgfqpoint{2.183919in}{0.739948in}}%
\pgfpathlineto{\pgfqpoint{2.185162in}{1.423079in}}%
\pgfpathlineto{\pgfqpoint{2.186929in}{0.766985in}}%
\pgfpathlineto{\pgfqpoint{2.188649in}{1.430016in}}%
\pgfpathlineto{\pgfqpoint{2.190248in}{0.774836in}}%
\pgfpathlineto{\pgfqpoint{2.191864in}{1.436336in}}%
\pgfpathlineto{\pgfqpoint{2.193433in}{0.802848in}}%
\pgfpathlineto{\pgfqpoint{2.195089in}{1.422480in}}%
\pgfpathlineto{\pgfqpoint{2.197082in}{0.754660in}}%
\pgfpathlineto{\pgfqpoint{2.198486in}{1.505013in}}%
\pgfpathlineto{\pgfqpoint{2.200163in}{0.744945in}}%
\pgfpathlineto{\pgfqpoint{2.201944in}{1.429312in}}%
\pgfpathlineto{\pgfqpoint{2.203564in}{0.747190in}}%
\pgfpathlineto{\pgfqpoint{2.205404in}{1.493079in}}%
\pgfpathlineto{\pgfqpoint{2.206980in}{0.780124in}}%
\pgfpathlineto{\pgfqpoint{2.208314in}{1.426623in}}%
\pgfpathlineto{\pgfqpoint{2.210054in}{0.757516in}}%
\pgfpathlineto{\pgfqpoint{2.211923in}{1.451452in}}%
\pgfpathlineto{\pgfqpoint{2.213204in}{0.805509in}}%
\pgfpathlineto{\pgfqpoint{2.215990in}{1.509328in}}%
\pgfpathlineto{\pgfqpoint{2.216600in}{0.803232in}}%
\pgfpathlineto{\pgfqpoint{2.218261in}{1.452407in}}%
\pgfpathlineto{\pgfqpoint{2.220481in}{0.709214in}}%
\pgfpathlineto{\pgfqpoint{2.221486in}{1.395582in}}%
\pgfpathlineto{\pgfqpoint{2.223686in}{0.753460in}}%
\pgfpathlineto{\pgfqpoint{2.225059in}{1.454685in}}%
\pgfpathlineto{\pgfqpoint{2.226713in}{0.754155in}}%
\pgfpathlineto{\pgfqpoint{2.229089in}{1.532662in}}%
\pgfpathlineto{\pgfqpoint{2.229710in}{0.740624in}}%
\pgfpathlineto{\pgfqpoint{2.231381in}{1.469389in}}%
\pgfpathlineto{\pgfqpoint{2.233151in}{0.778179in}}%
\pgfpathlineto{\pgfqpoint{2.235419in}{1.438688in}}%
\pgfpathlineto{\pgfqpoint{2.236873in}{0.746066in}}%
\pgfpathlineto{\pgfqpoint{2.238045in}{1.445635in}}%
\pgfpathlineto{\pgfqpoint{2.239694in}{0.735004in}}%
\pgfpathlineto{\pgfqpoint{2.241571in}{1.438349in}}%
\pgfpathlineto{\pgfqpoint{2.243276in}{0.766639in}}%
\pgfpathlineto{\pgfqpoint{2.244500in}{1.418631in}}%
\pgfpathlineto{\pgfqpoint{2.246171in}{0.770484in}}%
\pgfpathlineto{\pgfqpoint{2.248409in}{1.460323in}}%
\pgfpathlineto{\pgfqpoint{2.249606in}{0.772939in}}%
\pgfpathlineto{\pgfqpoint{2.252076in}{1.480633in}}%
\pgfpathlineto{\pgfqpoint{2.252792in}{0.784427in}}%
\pgfpathlineto{\pgfqpoint{2.254476in}{1.436324in}}%
\pgfpathlineto{\pgfqpoint{2.256548in}{0.753036in}}%
\pgfpathlineto{\pgfqpoint{2.258025in}{1.446347in}}%
\pgfpathlineto{\pgfqpoint{2.259722in}{0.763153in}}%
\pgfpathlineto{\pgfqpoint{2.261399in}{1.452539in}}%
\pgfpathlineto{\pgfqpoint{2.263408in}{0.742267in}}%
\pgfpathlineto{\pgfqpoint{2.264557in}{1.471002in}}%
\pgfpathlineto{\pgfqpoint{2.266152in}{0.768211in}}%
\pgfpathlineto{\pgfqpoint{2.268255in}{1.493663in}}%
\pgfpathlineto{\pgfqpoint{2.269446in}{0.804264in}}%
\pgfpathlineto{\pgfqpoint{2.271021in}{1.403982in}}%
\pgfpathlineto{\pgfqpoint{2.272617in}{0.814627in}}%
\pgfpathlineto{\pgfqpoint{2.275441in}{1.503917in}}%
\pgfpathlineto{\pgfqpoint{2.275880in}{0.775779in}}%
\pgfpathlineto{\pgfqpoint{2.277622in}{1.461559in}}%
\pgfpathlineto{\pgfqpoint{2.279189in}{0.774266in}}%
\pgfpathlineto{\pgfqpoint{2.280793in}{1.394477in}}%
\pgfpathlineto{\pgfqpoint{2.282683in}{0.749331in}}%
\pgfpathlineto{\pgfqpoint{2.284649in}{1.429715in}}%
\pgfpathlineto{\pgfqpoint{2.285760in}{0.795178in}}%
\pgfpathlineto{\pgfqpoint{2.288067in}{1.458540in}}%
\pgfpathlineto{\pgfqpoint{2.289139in}{0.771741in}}%
\pgfpathlineto{\pgfqpoint{2.291137in}{1.462263in}}%
\pgfpathlineto{\pgfqpoint{2.293431in}{0.699744in}}%
\pgfpathlineto{\pgfqpoint{2.293957in}{1.377076in}}%
\pgfpathlineto{\pgfqpoint{2.296124in}{0.737562in}}%
\pgfpathlineto{\pgfqpoint{2.297373in}{1.426785in}}%
\pgfpathlineto{\pgfqpoint{2.299114in}{0.791480in}}%
\pgfpathlineto{\pgfqpoint{2.300542in}{1.536740in}}%
\pgfpathlineto{\pgfqpoint{2.302278in}{0.780863in}}%
\pgfpathlineto{\pgfqpoint{2.304544in}{1.474448in}}%
\pgfpathlineto{\pgfqpoint{2.305686in}{0.798730in}}%
\pgfpathlineto{\pgfqpoint{2.307644in}{1.443996in}}%
\pgfpathlineto{\pgfqpoint{2.309250in}{0.771776in}}%
\pgfpathlineto{\pgfqpoint{2.310802in}{1.481117in}}%
\pgfpathlineto{\pgfqpoint{2.312169in}{0.764968in}}%
\pgfpathlineto{\pgfqpoint{2.313831in}{1.420318in}}%
\pgfpathlineto{\pgfqpoint{2.316913in}{1.402308in}}%
\pgfpathlineto{\pgfqpoint{2.317024in}{0.975695in}}%
\pgfpathlineto{\pgfqpoint{2.317024in}{0.975695in}}%
\pgfusepath{stroke}%
\end{pgfscope}%
\begin{pgfscope}%
\pgfsetrectcap%
\pgfsetmiterjoin%
\pgfsetlinewidth{0.803000pt}%
\definecolor{currentstroke}{rgb}{0.000000,0.000000,0.000000}%
\pgfsetstrokecolor{currentstroke}%
\pgfsetdash{}{0pt}%
\pgfpathmoveto{\pgfqpoint{0.589745in}{0.416447in}}%
\pgfpathlineto{\pgfqpoint{0.589745in}{1.789039in}}%
\pgfusepath{stroke}%
\end{pgfscope}%
\begin{pgfscope}%
\pgfsetrectcap%
\pgfsetmiterjoin%
\pgfsetlinewidth{0.803000pt}%
\definecolor{currentstroke}{rgb}{0.000000,0.000000,0.000000}%
\pgfsetstrokecolor{currentstroke}%
\pgfsetdash{}{0pt}%
\pgfpathmoveto{\pgfqpoint{2.399275in}{0.416447in}}%
\pgfpathlineto{\pgfqpoint{2.399275in}{1.789039in}}%
\pgfusepath{stroke}%
\end{pgfscope}%
\begin{pgfscope}%
\pgfsetrectcap%
\pgfsetmiterjoin%
\pgfsetlinewidth{0.803000pt}%
\definecolor{currentstroke}{rgb}{0.000000,0.000000,0.000000}%
\pgfsetstrokecolor{currentstroke}%
\pgfsetdash{}{0pt}%
\pgfpathmoveto{\pgfqpoint{0.589745in}{0.416447in}}%
\pgfpathlineto{\pgfqpoint{2.399275in}{0.416447in}}%
\pgfusepath{stroke}%
\end{pgfscope}%
\begin{pgfscope}%
\pgfsetrectcap%
\pgfsetmiterjoin%
\pgfsetlinewidth{0.803000pt}%
\definecolor{currentstroke}{rgb}{0.000000,0.000000,0.000000}%
\pgfsetstrokecolor{currentstroke}%
\pgfsetdash{}{0pt}%
\pgfpathmoveto{\pgfqpoint{0.589745in}{1.789039in}}%
\pgfpathlineto{\pgfqpoint{2.399275in}{1.789039in}}%
\pgfusepath{stroke}%
\end{pgfscope}%
\begin{pgfscope}%
\pgfsetbuttcap%
\pgfsetmiterjoin%
\definecolor{currentfill}{rgb}{1.000000,1.000000,1.000000}%
\pgfsetfillcolor{currentfill}%
\pgfsetfillopacity{0.800000}%
\pgfsetlinewidth{1.003750pt}%
\definecolor{currentstroke}{rgb}{0.800000,0.800000,0.800000}%
\pgfsetstrokecolor{currentstroke}%
\pgfsetstrokeopacity{0.800000}%
\pgfsetdash{}{0pt}%
\pgfpathmoveto{\pgfqpoint{0.667523in}{1.545261in}}%
\pgfpathlineto{\pgfqpoint{1.636634in}{1.545261in}}%
\pgfpathquadraticcurveto{\pgfqpoint{1.658856in}{1.545261in}}{\pgfqpoint{1.658856in}{1.567483in}}%
\pgfpathlineto{\pgfqpoint{1.658856in}{1.711261in}}%
\pgfpathquadraticcurveto{\pgfqpoint{1.658856in}{1.733483in}}{\pgfqpoint{1.636634in}{1.733483in}}%
\pgfpathlineto{\pgfqpoint{0.667523in}{1.733483in}}%
\pgfpathquadraticcurveto{\pgfqpoint{0.645300in}{1.733483in}}{\pgfqpoint{0.645300in}{1.711261in}}%
\pgfpathlineto{\pgfqpoint{0.645300in}{1.567483in}}%
\pgfpathquadraticcurveto{\pgfqpoint{0.645300in}{1.545261in}}{\pgfqpoint{0.667523in}{1.545261in}}%
\pgfpathlineto{\pgfqpoint{0.667523in}{1.545261in}}%
\pgfpathclose%
\pgfusepath{stroke,fill}%
\end{pgfscope}%
\begin{pgfscope}%
\pgfsetrectcap%
\pgfsetroundjoin%
\pgfsetlinewidth{1.505625pt}%
\definecolor{currentstroke}{rgb}{0.003922,0.450980,0.698039}%
\pgfsetstrokecolor{currentstroke}%
\pgfsetdash{}{0pt}%
\pgfpathmoveto{\pgfqpoint{0.689745in}{1.650150in}}%
\pgfpathlineto{\pgfqpoint{0.800856in}{1.650150in}}%
\pgfpathlineto{\pgfqpoint{0.911967in}{1.650150in}}%
\pgfusepath{stroke}%
\end{pgfscope}%
\begin{pgfscope}%
\definecolor{textcolor}{rgb}{0.000000,0.000000,0.000000}%
\pgfsetstrokecolor{textcolor}%
\pgfsetfillcolor{textcolor}%
\pgftext[x=1.000856in,y=1.611261in,left,base]{\color{textcolor}\rmfamily\fontsize{8.000000}{9.600000}\selectfont White noise}%
\end{pgfscope}%
\end{pgfpicture}%
\makeatother%
\endgroup%
% data/simulations/sim_allan_variance_example.py
        } % scalebox
        \caption{White noise}
    \end{subfigure}
    \begin{subfigure}{0.32\linewidth}
        \centering
        \scalebox{0.75}{%
            %% Creator: Matplotlib, PGF backend
%%
%% To include the figure in your LaTeX document, write
%%   \input{<filename>.pgf}
%%
%% Make sure the required packages are loaded in your preamble
%%   \usepackage{pgf}
%%
%% Also ensure that all the required font packages are loaded; for instance,
%% the lmodern package is sometimes necessary when using math font.
%%   \usepackage{lmodern}
%%
%% Figures using additional raster images can only be included by \input if
%% they are in the same directory as the main LaTeX file. For loading figures
%% from other directories you can use the `import` package
%%   \usepackage{import}
%%
%% and then include the figures with
%%   \import{<path to file>}{<filename>.pgf}
%%
%% Matplotlib used the following preamble
%%   \usepackage{siunitx}
%%   \sisetup{per-mode = symbol}%
%%   \usepackage{fontspec}
%%   \makeatletter\@ifpackageloaded{underscore}{}{\usepackage[strings]{underscore}}\makeatother
%%
\begingroup%
\makeatletter%
\begin{pgfpicture}%
\pgfpathrectangle{\pgfpointorigin}{\pgfqpoint{2.440945in}{1.830709in}}%
\pgfusepath{use as bounding box, clip}%
\begin{pgfscope}%
\pgfsetbuttcap%
\pgfsetmiterjoin%
\definecolor{currentfill}{rgb}{1.000000,1.000000,1.000000}%
\pgfsetfillcolor{currentfill}%
\pgfsetlinewidth{0.000000pt}%
\definecolor{currentstroke}{rgb}{1.000000,1.000000,1.000000}%
\pgfsetstrokecolor{currentstroke}%
\pgfsetdash{}{0pt}%
\pgfpathmoveto{\pgfqpoint{0.000000in}{0.000000in}}%
\pgfpathlineto{\pgfqpoint{2.440945in}{0.000000in}}%
\pgfpathlineto{\pgfqpoint{2.440945in}{1.830709in}}%
\pgfpathlineto{\pgfqpoint{0.000000in}{1.830709in}}%
\pgfpathlineto{\pgfqpoint{0.000000in}{0.000000in}}%
\pgfpathclose%
\pgfusepath{fill}%
\end{pgfscope}%
\begin{pgfscope}%
\pgfsetbuttcap%
\pgfsetmiterjoin%
\definecolor{currentfill}{rgb}{1.000000,1.000000,1.000000}%
\pgfsetfillcolor{currentfill}%
\pgfsetlinewidth{0.000000pt}%
\definecolor{currentstroke}{rgb}{0.000000,0.000000,0.000000}%
\pgfsetstrokecolor{currentstroke}%
\pgfsetstrokeopacity{0.000000}%
\pgfsetdash{}{0pt}%
\pgfpathmoveto{\pgfqpoint{0.530716in}{0.416447in}}%
\pgfpathlineto{\pgfqpoint{2.399275in}{0.416447in}}%
\pgfpathlineto{\pgfqpoint{2.399275in}{1.789039in}}%
\pgfpathlineto{\pgfqpoint{0.530716in}{1.789039in}}%
\pgfpathlineto{\pgfqpoint{0.530716in}{0.416447in}}%
\pgfpathclose%
\pgfusepath{fill}%
\end{pgfscope}%
\begin{pgfscope}%
\pgfpathrectangle{\pgfqpoint{0.530716in}{0.416447in}}{\pgfqpoint{1.868559in}{1.372591in}}%
\pgfusepath{clip}%
\pgfsetrectcap%
\pgfsetroundjoin%
\pgfsetlinewidth{0.803000pt}%
\definecolor{currentstroke}{rgb}{0.450000,0.450000,0.450000}%
\pgfsetstrokecolor{currentstroke}%
\pgfsetdash{}{0pt}%
\pgfpathmoveto{\pgfqpoint{0.615651in}{0.416447in}}%
\pgfpathlineto{\pgfqpoint{0.615651in}{1.789039in}}%
\pgfusepath{stroke}%
\end{pgfscope}%
\begin{pgfscope}%
\pgfsetbuttcap%
\pgfsetroundjoin%
\definecolor{currentfill}{rgb}{0.000000,0.000000,0.000000}%
\pgfsetfillcolor{currentfill}%
\pgfsetlinewidth{0.803000pt}%
\definecolor{currentstroke}{rgb}{0.000000,0.000000,0.000000}%
\pgfsetstrokecolor{currentstroke}%
\pgfsetdash{}{0pt}%
\pgfsys@defobject{currentmarker}{\pgfqpoint{0.000000in}{-0.048611in}}{\pgfqpoint{0.000000in}{0.000000in}}{%
\pgfpathmoveto{\pgfqpoint{0.000000in}{0.000000in}}%
\pgfpathlineto{\pgfqpoint{0.000000in}{-0.048611in}}%
\pgfusepath{stroke,fill}%
}%
\begin{pgfscope}%
\pgfsys@transformshift{0.615651in}{0.416447in}%
\pgfsys@useobject{currentmarker}{}%
\end{pgfscope}%
\end{pgfscope}%
\begin{pgfscope}%
\definecolor{textcolor}{rgb}{0.000000,0.000000,0.000000}%
\pgfsetstrokecolor{textcolor}%
\pgfsetfillcolor{textcolor}%
\pgftext[x=0.615651in,y=0.319225in,,top]{\color{textcolor}\rmfamily\fontsize{8.000000}{9.600000}\selectfont \(\displaystyle {0}\)}%
\end{pgfscope}%
\begin{pgfscope}%
\pgfpathrectangle{\pgfqpoint{0.530716in}{0.416447in}}{\pgfqpoint{1.868559in}{1.372591in}}%
\pgfusepath{clip}%
\pgfsetrectcap%
\pgfsetroundjoin%
\pgfsetlinewidth{0.803000pt}%
\definecolor{currentstroke}{rgb}{0.450000,0.450000,0.450000}%
\pgfsetstrokecolor{currentstroke}%
\pgfsetdash{}{0pt}%
\pgfpathmoveto{\pgfqpoint{1.121900in}{0.416447in}}%
\pgfpathlineto{\pgfqpoint{1.121900in}{1.789039in}}%
\pgfusepath{stroke}%
\end{pgfscope}%
\begin{pgfscope}%
\pgfsetbuttcap%
\pgfsetroundjoin%
\definecolor{currentfill}{rgb}{0.000000,0.000000,0.000000}%
\pgfsetfillcolor{currentfill}%
\pgfsetlinewidth{0.803000pt}%
\definecolor{currentstroke}{rgb}{0.000000,0.000000,0.000000}%
\pgfsetstrokecolor{currentstroke}%
\pgfsetdash{}{0pt}%
\pgfsys@defobject{currentmarker}{\pgfqpoint{0.000000in}{-0.048611in}}{\pgfqpoint{0.000000in}{0.000000in}}{%
\pgfpathmoveto{\pgfqpoint{0.000000in}{0.000000in}}%
\pgfpathlineto{\pgfqpoint{0.000000in}{-0.048611in}}%
\pgfusepath{stroke,fill}%
}%
\begin{pgfscope}%
\pgfsys@transformshift{1.121900in}{0.416447in}%
\pgfsys@useobject{currentmarker}{}%
\end{pgfscope}%
\end{pgfscope}%
\begin{pgfscope}%
\definecolor{textcolor}{rgb}{0.000000,0.000000,0.000000}%
\pgfsetstrokecolor{textcolor}%
\pgfsetfillcolor{textcolor}%
\pgftext[x=1.121900in,y=0.319225in,,top]{\color{textcolor}\rmfamily\fontsize{8.000000}{9.600000}\selectfont \(\displaystyle {10}\)}%
\end{pgfscope}%
\begin{pgfscope}%
\pgfpathrectangle{\pgfqpoint{0.530716in}{0.416447in}}{\pgfqpoint{1.868559in}{1.372591in}}%
\pgfusepath{clip}%
\pgfsetrectcap%
\pgfsetroundjoin%
\pgfsetlinewidth{0.803000pt}%
\definecolor{currentstroke}{rgb}{0.450000,0.450000,0.450000}%
\pgfsetstrokecolor{currentstroke}%
\pgfsetdash{}{0pt}%
\pgfpathmoveto{\pgfqpoint{1.628149in}{0.416447in}}%
\pgfpathlineto{\pgfqpoint{1.628149in}{1.789039in}}%
\pgfusepath{stroke}%
\end{pgfscope}%
\begin{pgfscope}%
\pgfsetbuttcap%
\pgfsetroundjoin%
\definecolor{currentfill}{rgb}{0.000000,0.000000,0.000000}%
\pgfsetfillcolor{currentfill}%
\pgfsetlinewidth{0.803000pt}%
\definecolor{currentstroke}{rgb}{0.000000,0.000000,0.000000}%
\pgfsetstrokecolor{currentstroke}%
\pgfsetdash{}{0pt}%
\pgfsys@defobject{currentmarker}{\pgfqpoint{0.000000in}{-0.048611in}}{\pgfqpoint{0.000000in}{0.000000in}}{%
\pgfpathmoveto{\pgfqpoint{0.000000in}{0.000000in}}%
\pgfpathlineto{\pgfqpoint{0.000000in}{-0.048611in}}%
\pgfusepath{stroke,fill}%
}%
\begin{pgfscope}%
\pgfsys@transformshift{1.628149in}{0.416447in}%
\pgfsys@useobject{currentmarker}{}%
\end{pgfscope}%
\end{pgfscope}%
\begin{pgfscope}%
\definecolor{textcolor}{rgb}{0.000000,0.000000,0.000000}%
\pgfsetstrokecolor{textcolor}%
\pgfsetfillcolor{textcolor}%
\pgftext[x=1.628149in,y=0.319225in,,top]{\color{textcolor}\rmfamily\fontsize{8.000000}{9.600000}\selectfont \(\displaystyle {20}\)}%
\end{pgfscope}%
\begin{pgfscope}%
\pgfpathrectangle{\pgfqpoint{0.530716in}{0.416447in}}{\pgfqpoint{1.868559in}{1.372591in}}%
\pgfusepath{clip}%
\pgfsetrectcap%
\pgfsetroundjoin%
\pgfsetlinewidth{0.803000pt}%
\definecolor{currentstroke}{rgb}{0.450000,0.450000,0.450000}%
\pgfsetstrokecolor{currentstroke}%
\pgfsetdash{}{0pt}%
\pgfpathmoveto{\pgfqpoint{2.134398in}{0.416447in}}%
\pgfpathlineto{\pgfqpoint{2.134398in}{1.789039in}}%
\pgfusepath{stroke}%
\end{pgfscope}%
\begin{pgfscope}%
\pgfsetbuttcap%
\pgfsetroundjoin%
\definecolor{currentfill}{rgb}{0.000000,0.000000,0.000000}%
\pgfsetfillcolor{currentfill}%
\pgfsetlinewidth{0.803000pt}%
\definecolor{currentstroke}{rgb}{0.000000,0.000000,0.000000}%
\pgfsetstrokecolor{currentstroke}%
\pgfsetdash{}{0pt}%
\pgfsys@defobject{currentmarker}{\pgfqpoint{0.000000in}{-0.048611in}}{\pgfqpoint{0.000000in}{0.000000in}}{%
\pgfpathmoveto{\pgfqpoint{0.000000in}{0.000000in}}%
\pgfpathlineto{\pgfqpoint{0.000000in}{-0.048611in}}%
\pgfusepath{stroke,fill}%
}%
\begin{pgfscope}%
\pgfsys@transformshift{2.134398in}{0.416447in}%
\pgfsys@useobject{currentmarker}{}%
\end{pgfscope}%
\end{pgfscope}%
\begin{pgfscope}%
\definecolor{textcolor}{rgb}{0.000000,0.000000,0.000000}%
\pgfsetstrokecolor{textcolor}%
\pgfsetfillcolor{textcolor}%
\pgftext[x=2.134398in,y=0.319225in,,top]{\color{textcolor}\rmfamily\fontsize{8.000000}{9.600000}\selectfont \(\displaystyle {30}\)}%
\end{pgfscope}%
\begin{pgfscope}%
\definecolor{textcolor}{rgb}{0.000000,0.000000,0.000000}%
\pgfsetstrokecolor{textcolor}%
\pgfsetfillcolor{textcolor}%
\pgftext[x=1.464996in,y=0.165003in,,top]{\color{textcolor}\rmfamily\fontsize{10.000000}{12.000000}\selectfont Time in \(\displaystyle \unit{\second}\)}%
\end{pgfscope}%
\begin{pgfscope}%
\pgfpathrectangle{\pgfqpoint{0.530716in}{0.416447in}}{\pgfqpoint{1.868559in}{1.372591in}}%
\pgfusepath{clip}%
\pgfsetrectcap%
\pgfsetroundjoin%
\pgfsetlinewidth{0.803000pt}%
\definecolor{currentstroke}{rgb}{0.450000,0.450000,0.450000}%
\pgfsetstrokecolor{currentstroke}%
\pgfsetdash{}{0pt}%
\pgfpathmoveto{\pgfqpoint{0.530716in}{0.416448in}}%
\pgfpathlineto{\pgfqpoint{2.399275in}{0.416448in}}%
\pgfusepath{stroke}%
\end{pgfscope}%
\begin{pgfscope}%
\pgfsetbuttcap%
\pgfsetroundjoin%
\definecolor{currentfill}{rgb}{0.000000,0.000000,0.000000}%
\pgfsetfillcolor{currentfill}%
\pgfsetlinewidth{0.803000pt}%
\definecolor{currentstroke}{rgb}{0.000000,0.000000,0.000000}%
\pgfsetstrokecolor{currentstroke}%
\pgfsetdash{}{0pt}%
\pgfsys@defobject{currentmarker}{\pgfqpoint{-0.048611in}{0.000000in}}{\pgfqpoint{-0.000000in}{0.000000in}}{%
\pgfpathmoveto{\pgfqpoint{-0.000000in}{0.000000in}}%
\pgfpathlineto{\pgfqpoint{-0.048611in}{0.000000in}}%
\pgfusepath{stroke,fill}%
}%
\begin{pgfscope}%
\pgfsys@transformshift{0.530716in}{0.416448in}%
\pgfsys@useobject{currentmarker}{}%
\end{pgfscope}%
\end{pgfscope}%
\begin{pgfscope}%
\definecolor{textcolor}{rgb}{0.000000,0.000000,0.000000}%
\pgfsetstrokecolor{textcolor}%
\pgfsetfillcolor{textcolor}%
\pgftext[x=0.223614in, y=0.377892in, left, base]{\color{textcolor}\rmfamily\fontsize{8.000000}{9.600000}\selectfont \(\displaystyle {\ensuremath{-}60}\)}%
\end{pgfscope}%
\begin{pgfscope}%
\pgfpathrectangle{\pgfqpoint{0.530716in}{0.416447in}}{\pgfqpoint{1.868559in}{1.372591in}}%
\pgfusepath{clip}%
\pgfsetrectcap%
\pgfsetroundjoin%
\pgfsetlinewidth{0.803000pt}%
\definecolor{currentstroke}{rgb}{0.450000,0.450000,0.450000}%
\pgfsetstrokecolor{currentstroke}%
\pgfsetdash{}{0pt}%
\pgfpathmoveto{\pgfqpoint{0.530716in}{0.666010in}}%
\pgfpathlineto{\pgfqpoint{2.399275in}{0.666010in}}%
\pgfusepath{stroke}%
\end{pgfscope}%
\begin{pgfscope}%
\pgfsetbuttcap%
\pgfsetroundjoin%
\definecolor{currentfill}{rgb}{0.000000,0.000000,0.000000}%
\pgfsetfillcolor{currentfill}%
\pgfsetlinewidth{0.803000pt}%
\definecolor{currentstroke}{rgb}{0.000000,0.000000,0.000000}%
\pgfsetstrokecolor{currentstroke}%
\pgfsetdash{}{0pt}%
\pgfsys@defobject{currentmarker}{\pgfqpoint{-0.048611in}{0.000000in}}{\pgfqpoint{-0.000000in}{0.000000in}}{%
\pgfpathmoveto{\pgfqpoint{-0.000000in}{0.000000in}}%
\pgfpathlineto{\pgfqpoint{-0.048611in}{0.000000in}}%
\pgfusepath{stroke,fill}%
}%
\begin{pgfscope}%
\pgfsys@transformshift{0.530716in}{0.666010in}%
\pgfsys@useobject{currentmarker}{}%
\end{pgfscope}%
\end{pgfscope}%
\begin{pgfscope}%
\definecolor{textcolor}{rgb}{0.000000,0.000000,0.000000}%
\pgfsetstrokecolor{textcolor}%
\pgfsetfillcolor{textcolor}%
\pgftext[x=0.223614in, y=0.627454in, left, base]{\color{textcolor}\rmfamily\fontsize{8.000000}{9.600000}\selectfont \(\displaystyle {\ensuremath{-}40}\)}%
\end{pgfscope}%
\begin{pgfscope}%
\pgfpathrectangle{\pgfqpoint{0.530716in}{0.416447in}}{\pgfqpoint{1.868559in}{1.372591in}}%
\pgfusepath{clip}%
\pgfsetrectcap%
\pgfsetroundjoin%
\pgfsetlinewidth{0.803000pt}%
\definecolor{currentstroke}{rgb}{0.450000,0.450000,0.450000}%
\pgfsetstrokecolor{currentstroke}%
\pgfsetdash{}{0pt}%
\pgfpathmoveto{\pgfqpoint{0.530716in}{0.915572in}}%
\pgfpathlineto{\pgfqpoint{2.399275in}{0.915572in}}%
\pgfusepath{stroke}%
\end{pgfscope}%
\begin{pgfscope}%
\pgfsetbuttcap%
\pgfsetroundjoin%
\definecolor{currentfill}{rgb}{0.000000,0.000000,0.000000}%
\pgfsetfillcolor{currentfill}%
\pgfsetlinewidth{0.803000pt}%
\definecolor{currentstroke}{rgb}{0.000000,0.000000,0.000000}%
\pgfsetstrokecolor{currentstroke}%
\pgfsetdash{}{0pt}%
\pgfsys@defobject{currentmarker}{\pgfqpoint{-0.048611in}{0.000000in}}{\pgfqpoint{-0.000000in}{0.000000in}}{%
\pgfpathmoveto{\pgfqpoint{-0.000000in}{0.000000in}}%
\pgfpathlineto{\pgfqpoint{-0.048611in}{0.000000in}}%
\pgfusepath{stroke,fill}%
}%
\begin{pgfscope}%
\pgfsys@transformshift{0.530716in}{0.915572in}%
\pgfsys@useobject{currentmarker}{}%
\end{pgfscope}%
\end{pgfscope}%
\begin{pgfscope}%
\definecolor{textcolor}{rgb}{0.000000,0.000000,0.000000}%
\pgfsetstrokecolor{textcolor}%
\pgfsetfillcolor{textcolor}%
\pgftext[x=0.223614in, y=0.877016in, left, base]{\color{textcolor}\rmfamily\fontsize{8.000000}{9.600000}\selectfont \(\displaystyle {\ensuremath{-}20}\)}%
\end{pgfscope}%
\begin{pgfscope}%
\pgfpathrectangle{\pgfqpoint{0.530716in}{0.416447in}}{\pgfqpoint{1.868559in}{1.372591in}}%
\pgfusepath{clip}%
\pgfsetrectcap%
\pgfsetroundjoin%
\pgfsetlinewidth{0.803000pt}%
\definecolor{currentstroke}{rgb}{0.450000,0.450000,0.450000}%
\pgfsetstrokecolor{currentstroke}%
\pgfsetdash{}{0pt}%
\pgfpathmoveto{\pgfqpoint{0.530716in}{1.165134in}}%
\pgfpathlineto{\pgfqpoint{2.399275in}{1.165134in}}%
\pgfusepath{stroke}%
\end{pgfscope}%
\begin{pgfscope}%
\pgfsetbuttcap%
\pgfsetroundjoin%
\definecolor{currentfill}{rgb}{0.000000,0.000000,0.000000}%
\pgfsetfillcolor{currentfill}%
\pgfsetlinewidth{0.803000pt}%
\definecolor{currentstroke}{rgb}{0.000000,0.000000,0.000000}%
\pgfsetstrokecolor{currentstroke}%
\pgfsetdash{}{0pt}%
\pgfsys@defobject{currentmarker}{\pgfqpoint{-0.048611in}{0.000000in}}{\pgfqpoint{-0.000000in}{0.000000in}}{%
\pgfpathmoveto{\pgfqpoint{-0.000000in}{0.000000in}}%
\pgfpathlineto{\pgfqpoint{-0.048611in}{0.000000in}}%
\pgfusepath{stroke,fill}%
}%
\begin{pgfscope}%
\pgfsys@transformshift{0.530716in}{1.165134in}%
\pgfsys@useobject{currentmarker}{}%
\end{pgfscope}%
\end{pgfscope}%
\begin{pgfscope}%
\definecolor{textcolor}{rgb}{0.000000,0.000000,0.000000}%
\pgfsetstrokecolor{textcolor}%
\pgfsetfillcolor{textcolor}%
\pgftext[x=0.374465in, y=1.126578in, left, base]{\color{textcolor}\rmfamily\fontsize{8.000000}{9.600000}\selectfont \(\displaystyle {0}\)}%
\end{pgfscope}%
\begin{pgfscope}%
\pgfpathrectangle{\pgfqpoint{0.530716in}{0.416447in}}{\pgfqpoint{1.868559in}{1.372591in}}%
\pgfusepath{clip}%
\pgfsetrectcap%
\pgfsetroundjoin%
\pgfsetlinewidth{0.803000pt}%
\definecolor{currentstroke}{rgb}{0.450000,0.450000,0.450000}%
\pgfsetstrokecolor{currentstroke}%
\pgfsetdash{}{0pt}%
\pgfpathmoveto{\pgfqpoint{0.530716in}{1.414696in}}%
\pgfpathlineto{\pgfqpoint{2.399275in}{1.414696in}}%
\pgfusepath{stroke}%
\end{pgfscope}%
\begin{pgfscope}%
\pgfsetbuttcap%
\pgfsetroundjoin%
\definecolor{currentfill}{rgb}{0.000000,0.000000,0.000000}%
\pgfsetfillcolor{currentfill}%
\pgfsetlinewidth{0.803000pt}%
\definecolor{currentstroke}{rgb}{0.000000,0.000000,0.000000}%
\pgfsetstrokecolor{currentstroke}%
\pgfsetdash{}{0pt}%
\pgfsys@defobject{currentmarker}{\pgfqpoint{-0.048611in}{0.000000in}}{\pgfqpoint{-0.000000in}{0.000000in}}{%
\pgfpathmoveto{\pgfqpoint{-0.000000in}{0.000000in}}%
\pgfpathlineto{\pgfqpoint{-0.048611in}{0.000000in}}%
\pgfusepath{stroke,fill}%
}%
\begin{pgfscope}%
\pgfsys@transformshift{0.530716in}{1.414696in}%
\pgfsys@useobject{currentmarker}{}%
\end{pgfscope}%
\end{pgfscope}%
\begin{pgfscope}%
\definecolor{textcolor}{rgb}{0.000000,0.000000,0.000000}%
\pgfsetstrokecolor{textcolor}%
\pgfsetfillcolor{textcolor}%
\pgftext[x=0.315437in, y=1.376140in, left, base]{\color{textcolor}\rmfamily\fontsize{8.000000}{9.600000}\selectfont \(\displaystyle {20}\)}%
\end{pgfscope}%
\begin{pgfscope}%
\pgfpathrectangle{\pgfqpoint{0.530716in}{0.416447in}}{\pgfqpoint{1.868559in}{1.372591in}}%
\pgfusepath{clip}%
\pgfsetrectcap%
\pgfsetroundjoin%
\pgfsetlinewidth{0.803000pt}%
\definecolor{currentstroke}{rgb}{0.450000,0.450000,0.450000}%
\pgfsetstrokecolor{currentstroke}%
\pgfsetdash{}{0pt}%
\pgfpathmoveto{\pgfqpoint{0.530716in}{1.664258in}}%
\pgfpathlineto{\pgfqpoint{2.399275in}{1.664258in}}%
\pgfusepath{stroke}%
\end{pgfscope}%
\begin{pgfscope}%
\pgfsetbuttcap%
\pgfsetroundjoin%
\definecolor{currentfill}{rgb}{0.000000,0.000000,0.000000}%
\pgfsetfillcolor{currentfill}%
\pgfsetlinewidth{0.803000pt}%
\definecolor{currentstroke}{rgb}{0.000000,0.000000,0.000000}%
\pgfsetstrokecolor{currentstroke}%
\pgfsetdash{}{0pt}%
\pgfsys@defobject{currentmarker}{\pgfqpoint{-0.048611in}{0.000000in}}{\pgfqpoint{-0.000000in}{0.000000in}}{%
\pgfpathmoveto{\pgfqpoint{-0.000000in}{0.000000in}}%
\pgfpathlineto{\pgfqpoint{-0.048611in}{0.000000in}}%
\pgfusepath{stroke,fill}%
}%
\begin{pgfscope}%
\pgfsys@transformshift{0.530716in}{1.664258in}%
\pgfsys@useobject{currentmarker}{}%
\end{pgfscope}%
\end{pgfscope}%
\begin{pgfscope}%
\definecolor{textcolor}{rgb}{0.000000,0.000000,0.000000}%
\pgfsetstrokecolor{textcolor}%
\pgfsetfillcolor{textcolor}%
\pgftext[x=0.315437in, y=1.625702in, left, base]{\color{textcolor}\rmfamily\fontsize{8.000000}{9.600000}\selectfont \(\displaystyle {40}\)}%
\end{pgfscope}%
\begin{pgfscope}%
\definecolor{textcolor}{rgb}{0.000000,0.000000,0.000000}%
\pgfsetstrokecolor{textcolor}%
\pgfsetfillcolor{textcolor}%
\pgftext[x=0.168059in,y=1.102743in,,bottom,rotate=90.000000]{\color{textcolor}\rmfamily\fontsize{10.000000}{12.000000}\selectfont Ampl. in arb. unit}%
\end{pgfscope}%
\begin{pgfscope}%
\pgfpathrectangle{\pgfqpoint{0.530716in}{0.416447in}}{\pgfqpoint{1.868559in}{1.372591in}}%
\pgfusepath{clip}%
\pgfsetrectcap%
\pgfsetroundjoin%
\pgfsetlinewidth{1.505625pt}%
\definecolor{currentstroke}{rgb}{0.007843,0.619608,0.450980}%
\pgfsetstrokecolor{currentstroke}%
\pgfsetdash{}{0pt}%
\pgfpathmoveto{\pgfqpoint{0.615651in}{1.156832in}}%
\pgfpathlineto{\pgfqpoint{0.616432in}{1.450241in}}%
\pgfpathlineto{\pgfqpoint{0.617394in}{0.742294in}}%
\pgfpathlineto{\pgfqpoint{0.619847in}{1.470900in}}%
\pgfpathlineto{\pgfqpoint{0.621196in}{0.846692in}}%
\pgfpathlineto{\pgfqpoint{0.623562in}{1.448838in}}%
\pgfpathlineto{\pgfqpoint{0.624212in}{0.880680in}}%
\pgfpathlineto{\pgfqpoint{0.626020in}{1.339614in}}%
\pgfpathlineto{\pgfqpoint{0.627910in}{0.849642in}}%
\pgfpathlineto{\pgfqpoint{0.629455in}{1.445279in}}%
\pgfpathlineto{\pgfqpoint{0.631007in}{0.889192in}}%
\pgfpathlineto{\pgfqpoint{0.633428in}{1.522295in}}%
\pgfpathlineto{\pgfqpoint{0.634660in}{0.926602in}}%
\pgfpathlineto{\pgfqpoint{0.636078in}{1.445081in}}%
\pgfpathlineto{\pgfqpoint{0.637847in}{0.799528in}}%
\pgfpathlineto{\pgfqpoint{0.640295in}{1.492273in}}%
\pgfpathlineto{\pgfqpoint{0.641409in}{0.882508in}}%
\pgfpathlineto{\pgfqpoint{0.643412in}{1.474472in}}%
\pgfpathlineto{\pgfqpoint{0.644828in}{0.887842in}}%
\pgfpathlineto{\pgfqpoint{0.647694in}{1.471292in}}%
\pgfpathlineto{\pgfqpoint{0.648053in}{0.854082in}}%
\pgfpathlineto{\pgfqpoint{0.650413in}{1.378737in}}%
\pgfpathlineto{\pgfqpoint{0.652891in}{0.845128in}}%
\pgfpathlineto{\pgfqpoint{0.653666in}{1.509141in}}%
\pgfpathlineto{\pgfqpoint{0.655043in}{0.891763in}}%
\pgfpathlineto{\pgfqpoint{0.656740in}{1.548211in}}%
\pgfpathlineto{\pgfqpoint{0.658334in}{0.862738in}}%
\pgfpathlineto{\pgfqpoint{0.659944in}{1.340439in}}%
\pgfpathlineto{\pgfqpoint{0.661879in}{0.874195in}}%
\pgfpathlineto{\pgfqpoint{0.663500in}{1.427279in}}%
\pgfpathlineto{\pgfqpoint{0.665042in}{0.809515in}}%
\pgfpathlineto{\pgfqpoint{0.666917in}{1.434759in}}%
\pgfpathlineto{\pgfqpoint{0.668462in}{0.865674in}}%
\pgfpathlineto{\pgfqpoint{0.670259in}{1.429510in}}%
\pgfpathlineto{\pgfqpoint{0.671893in}{0.887183in}}%
\pgfpathlineto{\pgfqpoint{0.674092in}{1.373681in}}%
\pgfpathlineto{\pgfqpoint{0.675315in}{0.835882in}}%
\pgfpathlineto{\pgfqpoint{0.677054in}{1.398491in}}%
\pgfpathlineto{\pgfqpoint{0.678765in}{0.860939in}}%
\pgfpathlineto{\pgfqpoint{0.680350in}{1.360509in}}%
\pgfpathlineto{\pgfqpoint{0.682879in}{0.802777in}}%
\pgfpathlineto{\pgfqpoint{0.683746in}{1.376189in}}%
\pgfpathlineto{\pgfqpoint{0.685565in}{0.840080in}}%
\pgfpathlineto{\pgfqpoint{0.687257in}{1.509437in}}%
\pgfpathlineto{\pgfqpoint{0.689232in}{0.912886in}}%
\pgfpathlineto{\pgfqpoint{0.690933in}{1.481339in}}%
\pgfpathlineto{\pgfqpoint{0.692269in}{0.940442in}}%
\pgfpathlineto{\pgfqpoint{0.694000in}{1.344554in}}%
\pgfpathlineto{\pgfqpoint{0.696145in}{0.853257in}}%
\pgfpathlineto{\pgfqpoint{0.697467in}{1.420182in}}%
\pgfpathlineto{\pgfqpoint{0.699859in}{0.768371in}}%
\pgfpathlineto{\pgfqpoint{0.700871in}{1.363784in}}%
\pgfpathlineto{\pgfqpoint{0.703067in}{0.790118in}}%
\pgfpathlineto{\pgfqpoint{0.704577in}{1.413570in}}%
\pgfpathlineto{\pgfqpoint{0.705917in}{0.810737in}}%
\pgfpathlineto{\pgfqpoint{0.707619in}{1.363241in}}%
\pgfpathlineto{\pgfqpoint{0.709660in}{0.873179in}}%
\pgfpathlineto{\pgfqpoint{0.711115in}{1.419525in}}%
\pgfpathlineto{\pgfqpoint{0.712672in}{0.831020in}}%
\pgfpathlineto{\pgfqpoint{0.714396in}{1.336336in}}%
\pgfpathlineto{\pgfqpoint{0.716334in}{0.884156in}}%
\pgfpathlineto{\pgfqpoint{0.718054in}{1.471042in}}%
\pgfpathlineto{\pgfqpoint{0.720738in}{0.810377in}}%
\pgfpathlineto{\pgfqpoint{0.721404in}{1.440602in}}%
\pgfpathlineto{\pgfqpoint{0.723770in}{0.880509in}}%
\pgfpathlineto{\pgfqpoint{0.725396in}{1.476523in}}%
\pgfpathlineto{\pgfqpoint{0.726859in}{0.916658in}}%
\pgfpathlineto{\pgfqpoint{0.728750in}{1.452108in}}%
\pgfpathlineto{\pgfqpoint{0.730886in}{0.780471in}}%
\pgfpathlineto{\pgfqpoint{0.731426in}{1.389823in}}%
\pgfpathlineto{\pgfqpoint{0.734539in}{0.674626in}}%
\pgfpathlineto{\pgfqpoint{0.734852in}{1.326221in}}%
\pgfpathlineto{\pgfqpoint{0.736509in}{0.790222in}}%
\pgfpathlineto{\pgfqpoint{0.739483in}{1.464705in}}%
\pgfpathlineto{\pgfqpoint{0.740051in}{0.758433in}}%
\pgfpathlineto{\pgfqpoint{0.741870in}{1.369339in}}%
\pgfpathlineto{\pgfqpoint{0.743599in}{0.779355in}}%
\pgfpathlineto{\pgfqpoint{0.745294in}{1.420401in}}%
\pgfpathlineto{\pgfqpoint{0.746938in}{0.748726in}}%
\pgfpathlineto{\pgfqpoint{0.748437in}{1.339746in}}%
\pgfpathlineto{\pgfqpoint{0.751150in}{0.800355in}}%
\pgfpathlineto{\pgfqpoint{0.752183in}{1.331399in}}%
\pgfpathlineto{\pgfqpoint{0.753725in}{0.804265in}}%
\pgfpathlineto{\pgfqpoint{0.755707in}{1.416797in}}%
\pgfpathlineto{\pgfqpoint{0.756933in}{0.938837in}}%
\pgfpathlineto{\pgfqpoint{0.759451in}{1.494951in}}%
\pgfpathlineto{\pgfqpoint{0.760625in}{0.913474in}}%
\pgfpathlineto{\pgfqpoint{0.762116in}{1.492028in}}%
\pgfpathlineto{\pgfqpoint{0.763847in}{0.905385in}}%
\pgfpathlineto{\pgfqpoint{0.766257in}{1.471850in}}%
\pgfpathlineto{\pgfqpoint{0.767154in}{0.993562in}}%
\pgfpathlineto{\pgfqpoint{0.769702in}{1.526977in}}%
\pgfpathlineto{\pgfqpoint{0.770765in}{0.928119in}}%
\pgfpathlineto{\pgfqpoint{0.772363in}{1.446906in}}%
\pgfpathlineto{\pgfqpoint{0.774116in}{0.908456in}}%
\pgfpathlineto{\pgfqpoint{0.776531in}{1.488005in}}%
\pgfpathlineto{\pgfqpoint{0.777425in}{0.859194in}}%
\pgfpathlineto{\pgfqpoint{0.779714in}{1.443640in}}%
\pgfpathlineto{\pgfqpoint{0.780763in}{0.892654in}}%
\pgfpathlineto{\pgfqpoint{0.783792in}{1.514644in}}%
\pgfpathlineto{\pgfqpoint{0.784485in}{0.883196in}}%
\pgfpathlineto{\pgfqpoint{0.785936in}{1.426414in}}%
\pgfpathlineto{\pgfqpoint{0.788173in}{0.947512in}}%
\pgfpathlineto{\pgfqpoint{0.789375in}{1.462589in}}%
\pgfpathlineto{\pgfqpoint{0.791114in}{0.868316in}}%
\pgfpathlineto{\pgfqpoint{0.793260in}{1.497555in}}%
\pgfpathlineto{\pgfqpoint{0.794420in}{0.866289in}}%
\pgfpathlineto{\pgfqpoint{0.796244in}{1.457019in}}%
\pgfpathlineto{\pgfqpoint{0.798330in}{0.715368in}}%
\pgfpathlineto{\pgfqpoint{0.799733in}{1.351676in}}%
\pgfpathlineto{\pgfqpoint{0.801536in}{0.734302in}}%
\pgfpathlineto{\pgfqpoint{0.802976in}{1.426365in}}%
\pgfpathlineto{\pgfqpoint{0.804754in}{0.757707in}}%
\pgfpathlineto{\pgfqpoint{0.806494in}{1.350369in}}%
\pgfpathlineto{\pgfqpoint{0.808318in}{0.771184in}}%
\pgfpathlineto{\pgfqpoint{0.809724in}{1.394718in}}%
\pgfpathlineto{\pgfqpoint{0.811461in}{0.767236in}}%
\pgfpathlineto{\pgfqpoint{0.813314in}{1.365001in}}%
\pgfpathlineto{\pgfqpoint{0.815998in}{0.821741in}}%
\pgfpathlineto{\pgfqpoint{0.817083in}{1.393698in}}%
\pgfpathlineto{\pgfqpoint{0.818327in}{0.814170in}}%
\pgfpathlineto{\pgfqpoint{0.820246in}{1.381786in}}%
\pgfpathlineto{\pgfqpoint{0.822011in}{0.791705in}}%
\pgfpathlineto{\pgfqpoint{0.823701in}{1.358932in}}%
\pgfpathlineto{\pgfqpoint{0.825275in}{0.744595in}}%
\pgfpathlineto{\pgfqpoint{0.826727in}{1.287101in}}%
\pgfpathlineto{\pgfqpoint{0.828598in}{0.798519in}}%
\pgfpathlineto{\pgfqpoint{0.831166in}{1.472620in}}%
\pgfpathlineto{\pgfqpoint{0.831859in}{0.924038in}}%
\pgfpathlineto{\pgfqpoint{0.833543in}{1.507278in}}%
\pgfpathlineto{\pgfqpoint{0.835314in}{0.863482in}}%
\pgfpathlineto{\pgfqpoint{0.837332in}{1.367222in}}%
\pgfpathlineto{\pgfqpoint{0.838819in}{0.847159in}}%
\pgfpathlineto{\pgfqpoint{0.840538in}{1.396769in}}%
\pgfpathlineto{\pgfqpoint{0.842870in}{0.820381in}}%
\pgfpathlineto{\pgfqpoint{0.843867in}{1.392623in}}%
\pgfpathlineto{\pgfqpoint{0.845450in}{0.872364in}}%
\pgfpathlineto{\pgfqpoint{0.847713in}{1.373807in}}%
\pgfpathlineto{\pgfqpoint{0.848896in}{0.868391in}}%
\pgfpathlineto{\pgfqpoint{0.850560in}{1.393068in}}%
\pgfpathlineto{\pgfqpoint{0.852259in}{0.814313in}}%
\pgfpathlineto{\pgfqpoint{0.853989in}{1.400467in}}%
\pgfpathlineto{\pgfqpoint{0.855719in}{0.900362in}}%
\pgfpathlineto{\pgfqpoint{0.858306in}{1.498931in}}%
\pgfpathlineto{\pgfqpoint{0.859172in}{0.917509in}}%
\pgfpathlineto{\pgfqpoint{0.860801in}{1.389155in}}%
\pgfpathlineto{\pgfqpoint{0.862613in}{0.932774in}}%
\pgfpathlineto{\pgfqpoint{0.864241in}{1.533849in}}%
\pgfpathlineto{\pgfqpoint{0.866052in}{0.777558in}}%
\pgfpathlineto{\pgfqpoint{0.867613in}{1.424244in}}%
\pgfpathlineto{\pgfqpoint{0.869308in}{0.815237in}}%
\pgfpathlineto{\pgfqpoint{0.871396in}{1.376424in}}%
\pgfpathlineto{\pgfqpoint{0.874245in}{0.803422in}}%
\pgfpathlineto{\pgfqpoint{0.874927in}{1.457847in}}%
\pgfpathlineto{\pgfqpoint{0.876083in}{0.871994in}}%
\pgfpathlineto{\pgfqpoint{0.878175in}{1.456820in}}%
\pgfpathlineto{\pgfqpoint{0.880370in}{0.921554in}}%
\pgfpathlineto{\pgfqpoint{0.881309in}{1.498580in}}%
\pgfpathlineto{\pgfqpoint{0.883277in}{0.888172in}}%
\pgfpathlineto{\pgfqpoint{0.884593in}{1.344105in}}%
\pgfpathlineto{\pgfqpoint{0.886300in}{0.841362in}}%
\pgfpathlineto{\pgfqpoint{0.888005in}{1.387512in}}%
\pgfpathlineto{\pgfqpoint{0.889768in}{0.921628in}}%
\pgfpathlineto{\pgfqpoint{0.891801in}{1.482694in}}%
\pgfpathlineto{\pgfqpoint{0.893265in}{0.948950in}}%
\pgfpathlineto{\pgfqpoint{0.895636in}{1.513324in}}%
\pgfpathlineto{\pgfqpoint{0.896917in}{0.916931in}}%
\pgfpathlineto{\pgfqpoint{0.898489in}{1.539453in}}%
\pgfpathlineto{\pgfqpoint{0.900311in}{0.908455in}}%
\pgfpathlineto{\pgfqpoint{0.901675in}{1.513349in}}%
\pgfpathlineto{\pgfqpoint{0.903597in}{0.908168in}}%
\pgfpathlineto{\pgfqpoint{0.905988in}{1.502479in}}%
\pgfpathlineto{\pgfqpoint{0.906804in}{0.907485in}}%
\pgfpathlineto{\pgfqpoint{0.908650in}{1.587525in}}%
\pgfpathlineto{\pgfqpoint{0.910822in}{0.920678in}}%
\pgfpathlineto{\pgfqpoint{0.911961in}{1.554807in}}%
\pgfpathlineto{\pgfqpoint{0.914371in}{0.884871in}}%
\pgfpathlineto{\pgfqpoint{0.915300in}{1.408608in}}%
\pgfpathlineto{\pgfqpoint{0.917517in}{0.854593in}}%
\pgfpathlineto{\pgfqpoint{0.919492in}{1.477070in}}%
\pgfpathlineto{\pgfqpoint{0.920795in}{0.902038in}}%
\pgfpathlineto{\pgfqpoint{0.922083in}{1.474279in}}%
\pgfpathlineto{\pgfqpoint{0.923814in}{0.957287in}}%
\pgfpathlineto{\pgfqpoint{0.925778in}{1.525466in}}%
\pgfpathlineto{\pgfqpoint{0.927138in}{0.931785in}}%
\pgfpathlineto{\pgfqpoint{0.929534in}{1.441948in}}%
\pgfpathlineto{\pgfqpoint{0.930554in}{0.955560in}}%
\pgfpathlineto{\pgfqpoint{0.932775in}{1.430819in}}%
\pgfpathlineto{\pgfqpoint{0.935065in}{0.921305in}}%
\pgfpathlineto{\pgfqpoint{0.935648in}{1.531959in}}%
\pgfpathlineto{\pgfqpoint{0.937535in}{0.906132in}}%
\pgfpathlineto{\pgfqpoint{0.939169in}{1.471098in}}%
\pgfpathlineto{\pgfqpoint{0.940791in}{0.951167in}}%
\pgfpathlineto{\pgfqpoint{0.943665in}{1.544942in}}%
\pgfpathlineto{\pgfqpoint{0.944586in}{0.898931in}}%
\pgfpathlineto{\pgfqpoint{0.945999in}{1.433094in}}%
\pgfpathlineto{\pgfqpoint{0.947625in}{0.879003in}}%
\pgfpathlineto{\pgfqpoint{0.949769in}{1.378891in}}%
\pgfpathlineto{\pgfqpoint{0.950997in}{0.912752in}}%
\pgfpathlineto{\pgfqpoint{0.952829in}{1.379585in}}%
\pgfpathlineto{\pgfqpoint{0.954691in}{0.889452in}}%
\pgfpathlineto{\pgfqpoint{0.956452in}{1.486021in}}%
\pgfpathlineto{\pgfqpoint{0.958162in}{0.869724in}}%
\pgfpathlineto{\pgfqpoint{0.959617in}{1.401100in}}%
\pgfpathlineto{\pgfqpoint{0.961209in}{0.734108in}}%
\pgfpathlineto{\pgfqpoint{0.963746in}{1.492955in}}%
\pgfpathlineto{\pgfqpoint{0.965285in}{0.872325in}}%
\pgfpathlineto{\pgfqpoint{0.966377in}{1.507863in}}%
\pgfpathlineto{\pgfqpoint{0.968374in}{0.884213in}}%
\pgfpathlineto{\pgfqpoint{0.970271in}{1.429187in}}%
\pgfpathlineto{\pgfqpoint{0.971411in}{0.886002in}}%
\pgfpathlineto{\pgfqpoint{0.974017in}{1.406059in}}%
\pgfpathlineto{\pgfqpoint{0.974897in}{0.781412in}}%
\pgfpathlineto{\pgfqpoint{0.976871in}{1.351814in}}%
\pgfpathlineto{\pgfqpoint{0.978721in}{0.757191in}}%
\pgfpathlineto{\pgfqpoint{0.979918in}{1.256919in}}%
\pgfpathlineto{\pgfqpoint{0.981876in}{0.873097in}}%
\pgfpathlineto{\pgfqpoint{0.983693in}{1.373339in}}%
\pgfpathlineto{\pgfqpoint{0.985011in}{0.791196in}}%
\pgfpathlineto{\pgfqpoint{0.987274in}{1.363373in}}%
\pgfpathlineto{\pgfqpoint{0.988550in}{0.784469in}}%
\pgfpathlineto{\pgfqpoint{0.991257in}{1.444367in}}%
\pgfpathlineto{\pgfqpoint{0.991833in}{0.851289in}}%
\pgfpathlineto{\pgfqpoint{0.993894in}{1.359522in}}%
\pgfpathlineto{\pgfqpoint{0.995738in}{0.778259in}}%
\pgfpathlineto{\pgfqpoint{0.997025in}{1.386589in}}%
\pgfpathlineto{\pgfqpoint{0.999181in}{0.645774in}}%
\pgfpathlineto{\pgfqpoint{1.000623in}{1.306094in}}%
\pgfpathlineto{\pgfqpoint{1.002236in}{0.846441in}}%
\pgfpathlineto{\pgfqpoint{1.004042in}{1.375647in}}%
\pgfpathlineto{\pgfqpoint{1.005653in}{0.810537in}}%
\pgfpathlineto{\pgfqpoint{1.007578in}{1.513156in}}%
\pgfpathlineto{\pgfqpoint{1.009042in}{0.864344in}}%
\pgfpathlineto{\pgfqpoint{1.011978in}{1.454333in}}%
\pgfpathlineto{\pgfqpoint{1.012266in}{0.909838in}}%
\pgfpathlineto{\pgfqpoint{1.014123in}{1.475503in}}%
\pgfpathlineto{\pgfqpoint{1.015740in}{0.884043in}}%
\pgfpathlineto{\pgfqpoint{1.017699in}{1.413894in}}%
\pgfpathlineto{\pgfqpoint{1.019148in}{0.751592in}}%
\pgfpathlineto{\pgfqpoint{1.020900in}{1.401789in}}%
\pgfpathlineto{\pgfqpoint{1.022491in}{0.823158in}}%
\pgfpathlineto{\pgfqpoint{1.024394in}{1.480543in}}%
\pgfpathlineto{\pgfqpoint{1.025922in}{0.947012in}}%
\pgfpathlineto{\pgfqpoint{1.028113in}{1.487877in}}%
\pgfpathlineto{\pgfqpoint{1.029689in}{0.802973in}}%
\pgfpathlineto{\pgfqpoint{1.030980in}{1.371157in}}%
\pgfpathlineto{\pgfqpoint{1.033650in}{0.830375in}}%
\pgfpathlineto{\pgfqpoint{1.034445in}{1.405521in}}%
\pgfpathlineto{\pgfqpoint{1.037207in}{0.828401in}}%
\pgfpathlineto{\pgfqpoint{1.038127in}{1.487782in}}%
\pgfpathlineto{\pgfqpoint{1.040351in}{0.819619in}}%
\pgfpathlineto{\pgfqpoint{1.042793in}{1.513692in}}%
\pgfpathlineto{\pgfqpoint{1.043488in}{0.861332in}}%
\pgfpathlineto{\pgfqpoint{1.044593in}{1.427431in}}%
\pgfpathlineto{\pgfqpoint{1.046944in}{0.883225in}}%
\pgfpathlineto{\pgfqpoint{1.048469in}{1.447412in}}%
\pgfpathlineto{\pgfqpoint{1.049817in}{0.883119in}}%
\pgfpathlineto{\pgfqpoint{1.051992in}{1.411912in}}%
\pgfpathlineto{\pgfqpoint{1.053201in}{0.853558in}}%
\pgfpathlineto{\pgfqpoint{1.055340in}{1.405780in}}%
\pgfpathlineto{\pgfqpoint{1.056983in}{0.743926in}}%
\pgfpathlineto{\pgfqpoint{1.058367in}{1.440112in}}%
\pgfpathlineto{\pgfqpoint{1.059943in}{0.827433in}}%
\pgfpathlineto{\pgfqpoint{1.061648in}{1.357110in}}%
\pgfpathlineto{\pgfqpoint{1.064338in}{0.774082in}}%
\pgfpathlineto{\pgfqpoint{1.065233in}{1.441909in}}%
\pgfpathlineto{\pgfqpoint{1.066810in}{0.840347in}}%
\pgfpathlineto{\pgfqpoint{1.068655in}{1.360599in}}%
\pgfpathlineto{\pgfqpoint{1.070136in}{0.785215in}}%
\pgfpathlineto{\pgfqpoint{1.071897in}{1.300760in}}%
\pgfpathlineto{\pgfqpoint{1.073714in}{0.742543in}}%
\pgfpathlineto{\pgfqpoint{1.075307in}{1.360294in}}%
\pgfpathlineto{\pgfqpoint{1.077400in}{0.780633in}}%
\pgfpathlineto{\pgfqpoint{1.079139in}{1.351105in}}%
\pgfpathlineto{\pgfqpoint{1.081307in}{0.740457in}}%
\pgfpathlineto{\pgfqpoint{1.082170in}{1.327336in}}%
\pgfpathlineto{\pgfqpoint{1.084820in}{0.725161in}}%
\pgfpathlineto{\pgfqpoint{1.086254in}{1.557564in}}%
\pgfpathlineto{\pgfqpoint{1.087417in}{0.790382in}}%
\pgfpathlineto{\pgfqpoint{1.088972in}{1.414836in}}%
\pgfpathlineto{\pgfqpoint{1.090698in}{0.810500in}}%
\pgfpathlineto{\pgfqpoint{1.092663in}{1.452575in}}%
\pgfpathlineto{\pgfqpoint{1.094159in}{0.919488in}}%
\pgfpathlineto{\pgfqpoint{1.095806in}{1.516601in}}%
\pgfpathlineto{\pgfqpoint{1.097896in}{0.846377in}}%
\pgfpathlineto{\pgfqpoint{1.099480in}{1.478080in}}%
\pgfpathlineto{\pgfqpoint{1.100835in}{0.853790in}}%
\pgfpathlineto{\pgfqpoint{1.102691in}{1.408532in}}%
\pgfpathlineto{\pgfqpoint{1.104446in}{0.885906in}}%
\pgfpathlineto{\pgfqpoint{1.106034in}{1.490774in}}%
\pgfpathlineto{\pgfqpoint{1.108618in}{0.819548in}}%
\pgfpathlineto{\pgfqpoint{1.109363in}{1.315869in}}%
\pgfpathlineto{\pgfqpoint{1.111479in}{0.729454in}}%
\pgfpathlineto{\pgfqpoint{1.113803in}{1.439193in}}%
\pgfpathlineto{\pgfqpoint{1.114608in}{0.868177in}}%
\pgfpathlineto{\pgfqpoint{1.116377in}{1.451482in}}%
\pgfpathlineto{\pgfqpoint{1.118260in}{0.841156in}}%
\pgfpathlineto{\pgfqpoint{1.119482in}{1.309352in}}%
\pgfpathlineto{\pgfqpoint{1.121304in}{0.811562in}}%
\pgfpathlineto{\pgfqpoint{1.122915in}{1.423087in}}%
\pgfpathlineto{\pgfqpoint{1.125101in}{0.807245in}}%
\pgfpathlineto{\pgfqpoint{1.126422in}{1.432524in}}%
\pgfpathlineto{\pgfqpoint{1.128029in}{0.844427in}}%
\pgfpathlineto{\pgfqpoint{1.130834in}{1.398182in}}%
\pgfpathlineto{\pgfqpoint{1.131544in}{0.826145in}}%
\pgfpathlineto{\pgfqpoint{1.133515in}{1.385237in}}%
\pgfpathlineto{\pgfqpoint{1.134806in}{0.830879in}}%
\pgfpathlineto{\pgfqpoint{1.136644in}{1.403136in}}%
\pgfpathlineto{\pgfqpoint{1.138671in}{0.787966in}}%
\pgfpathlineto{\pgfqpoint{1.140016in}{1.457258in}}%
\pgfpathlineto{\pgfqpoint{1.143294in}{0.740488in}}%
\pgfpathlineto{\pgfqpoint{1.143485in}{1.385574in}}%
\pgfpathlineto{\pgfqpoint{1.145837in}{0.850793in}}%
\pgfpathlineto{\pgfqpoint{1.146766in}{1.456760in}}%
\pgfpathlineto{\pgfqpoint{1.149835in}{0.884616in}}%
\pgfpathlineto{\pgfqpoint{1.150478in}{1.512453in}}%
\pgfpathlineto{\pgfqpoint{1.151870in}{0.994495in}}%
\pgfpathlineto{\pgfqpoint{1.153569in}{1.505026in}}%
\pgfpathlineto{\pgfqpoint{1.155787in}{0.960517in}}%
\pgfpathlineto{\pgfqpoint{1.157125in}{1.503530in}}%
\pgfpathlineto{\pgfqpoint{1.158812in}{0.963353in}}%
\pgfpathlineto{\pgfqpoint{1.160410in}{1.474675in}}%
\pgfpathlineto{\pgfqpoint{1.162414in}{0.909654in}}%
\pgfpathlineto{\pgfqpoint{1.164794in}{1.538489in}}%
\pgfpathlineto{\pgfqpoint{1.165838in}{0.873208in}}%
\pgfpathlineto{\pgfqpoint{1.167311in}{1.442612in}}%
\pgfpathlineto{\pgfqpoint{1.169259in}{0.925143in}}%
\pgfpathlineto{\pgfqpoint{1.170667in}{1.477915in}}%
\pgfpathlineto{\pgfqpoint{1.172360in}{0.922297in}}%
\pgfpathlineto{\pgfqpoint{1.174642in}{1.422264in}}%
\pgfpathlineto{\pgfqpoint{1.175744in}{0.806239in}}%
\pgfpathlineto{\pgfqpoint{1.178218in}{1.428860in}}%
\pgfpathlineto{\pgfqpoint{1.180047in}{0.796716in}}%
\pgfpathlineto{\pgfqpoint{1.181443in}{1.415498in}}%
\pgfpathlineto{\pgfqpoint{1.183010in}{0.796452in}}%
\pgfpathlineto{\pgfqpoint{1.184277in}{1.406728in}}%
\pgfpathlineto{\pgfqpoint{1.186113in}{0.763731in}}%
\pgfpathlineto{\pgfqpoint{1.187681in}{1.347763in}}%
\pgfpathlineto{\pgfqpoint{1.190363in}{0.798024in}}%
\pgfpathlineto{\pgfqpoint{1.192151in}{1.460023in}}%
\pgfpathlineto{\pgfqpoint{1.192806in}{0.901358in}}%
\pgfpathlineto{\pgfqpoint{1.194491in}{1.387740in}}%
\pgfpathlineto{\pgfqpoint{1.196252in}{0.839179in}}%
\pgfpathlineto{\pgfqpoint{1.198080in}{1.376560in}}%
\pgfpathlineto{\pgfqpoint{1.199817in}{0.736201in}}%
\pgfpathlineto{\pgfqpoint{1.201632in}{1.301999in}}%
\pgfpathlineto{\pgfqpoint{1.203646in}{0.937425in}}%
\pgfpathlineto{\pgfqpoint{1.204586in}{1.478389in}}%
\pgfpathlineto{\pgfqpoint{1.206302in}{0.937890in}}%
\pgfpathlineto{\pgfqpoint{1.208190in}{1.491280in}}%
\pgfpathlineto{\pgfqpoint{1.209835in}{0.976344in}}%
\pgfpathlineto{\pgfqpoint{1.211687in}{1.472132in}}%
\pgfpathlineto{\pgfqpoint{1.213376in}{0.813102in}}%
\pgfpathlineto{\pgfqpoint{1.214828in}{1.470046in}}%
\pgfpathlineto{\pgfqpoint{1.216622in}{0.856156in}}%
\pgfpathlineto{\pgfqpoint{1.219029in}{1.401248in}}%
\pgfpathlineto{\pgfqpoint{1.219963in}{0.931914in}}%
\pgfpathlineto{\pgfqpoint{1.221789in}{1.551178in}}%
\pgfpathlineto{\pgfqpoint{1.223697in}{0.847978in}}%
\pgfpathlineto{\pgfqpoint{1.225386in}{1.478049in}}%
\pgfpathlineto{\pgfqpoint{1.227753in}{0.909462in}}%
\pgfpathlineto{\pgfqpoint{1.228640in}{1.487709in}}%
\pgfpathlineto{\pgfqpoint{1.230676in}{0.829875in}}%
\pgfpathlineto{\pgfqpoint{1.232387in}{1.482571in}}%
\pgfpathlineto{\pgfqpoint{1.233692in}{0.921937in}}%
\pgfpathlineto{\pgfqpoint{1.235309in}{1.413658in}}%
\pgfpathlineto{\pgfqpoint{1.236932in}{0.976124in}}%
\pgfpathlineto{\pgfqpoint{1.238883in}{1.554555in}}%
\pgfpathlineto{\pgfqpoint{1.241222in}{0.912960in}}%
\pgfpathlineto{\pgfqpoint{1.242066in}{1.480549in}}%
\pgfpathlineto{\pgfqpoint{1.243837in}{0.978919in}}%
\pgfpathlineto{\pgfqpoint{1.246399in}{1.555454in}}%
\pgfpathlineto{\pgfqpoint{1.247979in}{0.935981in}}%
\pgfpathlineto{\pgfqpoint{1.249569in}{1.475871in}}%
\pgfpathlineto{\pgfqpoint{1.250883in}{0.875638in}}%
\pgfpathlineto{\pgfqpoint{1.252480in}{1.509636in}}%
\pgfpathlineto{\pgfqpoint{1.253965in}{0.996461in}}%
\pgfpathlineto{\pgfqpoint{1.257123in}{1.564500in}}%
\pgfpathlineto{\pgfqpoint{1.257381in}{0.892660in}}%
\pgfpathlineto{\pgfqpoint{1.259051in}{1.488080in}}%
\pgfpathlineto{\pgfqpoint{1.261112in}{0.942865in}}%
\pgfpathlineto{\pgfqpoint{1.263111in}{1.458405in}}%
\pgfpathlineto{\pgfqpoint{1.264509in}{0.816240in}}%
\pgfpathlineto{\pgfqpoint{1.266242in}{1.382471in}}%
\pgfpathlineto{\pgfqpoint{1.267639in}{0.834252in}}%
\pgfpathlineto{\pgfqpoint{1.270356in}{1.436403in}}%
\pgfpathlineto{\pgfqpoint{1.271010in}{0.833792in}}%
\pgfpathlineto{\pgfqpoint{1.273173in}{1.467076in}}%
\pgfpathlineto{\pgfqpoint{1.274432in}{0.756330in}}%
\pgfpathlineto{\pgfqpoint{1.276331in}{1.350065in}}%
\pgfpathlineto{\pgfqpoint{1.277997in}{0.729097in}}%
\pgfpathlineto{\pgfqpoint{1.279517in}{1.353769in}}%
\pgfpathlineto{\pgfqpoint{1.282686in}{0.720429in}}%
\pgfpathlineto{\pgfqpoint{1.283491in}{1.395651in}}%
\pgfpathlineto{\pgfqpoint{1.284844in}{0.775899in}}%
\pgfpathlineto{\pgfqpoint{1.286789in}{1.436842in}}%
\pgfpathlineto{\pgfqpoint{1.288168in}{0.947540in}}%
\pgfpathlineto{\pgfqpoint{1.289718in}{1.377053in}}%
\pgfpathlineto{\pgfqpoint{1.291704in}{0.805452in}}%
\pgfpathlineto{\pgfqpoint{1.293355in}{1.454858in}}%
\pgfpathlineto{\pgfqpoint{1.294804in}{0.824903in}}%
\pgfpathlineto{\pgfqpoint{1.297801in}{1.487828in}}%
\pgfpathlineto{\pgfqpoint{1.298601in}{0.772792in}}%
\pgfpathlineto{\pgfqpoint{1.299909in}{1.259036in}}%
\pgfpathlineto{\pgfqpoint{1.301702in}{0.805113in}}%
\pgfpathlineto{\pgfqpoint{1.303427in}{1.383189in}}%
\pgfpathlineto{\pgfqpoint{1.305118in}{0.800841in}}%
\pgfpathlineto{\pgfqpoint{1.306815in}{1.427833in}}%
\pgfpathlineto{\pgfqpoint{1.308892in}{0.845396in}}%
\pgfpathlineto{\pgfqpoint{1.310313in}{1.438262in}}%
\pgfpathlineto{\pgfqpoint{1.311889in}{0.928050in}}%
\pgfpathlineto{\pgfqpoint{1.313978in}{1.459769in}}%
\pgfpathlineto{\pgfqpoint{1.315309in}{0.862627in}}%
\pgfpathlineto{\pgfqpoint{1.317485in}{1.444277in}}%
\pgfpathlineto{\pgfqpoint{1.318891in}{0.940091in}}%
\pgfpathlineto{\pgfqpoint{1.320543in}{1.483701in}}%
\pgfpathlineto{\pgfqpoint{1.322252in}{0.824375in}}%
\pgfpathlineto{\pgfqpoint{1.323844in}{1.457470in}}%
\pgfpathlineto{\pgfqpoint{1.326089in}{0.724520in}}%
\pgfpathlineto{\pgfqpoint{1.327397in}{1.384039in}}%
\pgfpathlineto{\pgfqpoint{1.329012in}{0.898998in}}%
\pgfpathlineto{\pgfqpoint{1.330972in}{1.393238in}}%
\pgfpathlineto{\pgfqpoint{1.332943in}{0.786785in}}%
\pgfpathlineto{\pgfqpoint{1.334235in}{1.392056in}}%
\pgfpathlineto{\pgfqpoint{1.335852in}{0.844582in}}%
\pgfpathlineto{\pgfqpoint{1.337904in}{1.452388in}}%
\pgfpathlineto{\pgfqpoint{1.339465in}{0.819427in}}%
\pgfpathlineto{\pgfqpoint{1.340908in}{1.385303in}}%
\pgfpathlineto{\pgfqpoint{1.342740in}{0.883747in}}%
\pgfpathlineto{\pgfqpoint{1.344544in}{1.476847in}}%
\pgfpathlineto{\pgfqpoint{1.346471in}{0.843830in}}%
\pgfpathlineto{\pgfqpoint{1.347731in}{1.344154in}}%
\pgfpathlineto{\pgfqpoint{1.349546in}{0.829216in}}%
\pgfpathlineto{\pgfqpoint{1.350967in}{1.414576in}}%
\pgfpathlineto{\pgfqpoint{1.353106in}{0.792047in}}%
\pgfpathlineto{\pgfqpoint{1.355123in}{1.427216in}}%
\pgfpathlineto{\pgfqpoint{1.356098in}{0.895095in}}%
\pgfpathlineto{\pgfqpoint{1.357846in}{1.432999in}}%
\pgfpathlineto{\pgfqpoint{1.359720in}{0.831141in}}%
\pgfpathlineto{\pgfqpoint{1.361279in}{1.439374in}}%
\pgfpathlineto{\pgfqpoint{1.363066in}{0.801246in}}%
\pgfpathlineto{\pgfqpoint{1.365199in}{1.429601in}}%
\pgfpathlineto{\pgfqpoint{1.366358in}{0.898265in}}%
\pgfpathlineto{\pgfqpoint{1.368608in}{1.402404in}}%
\pgfpathlineto{\pgfqpoint{1.370141in}{0.809457in}}%
\pgfpathlineto{\pgfqpoint{1.371688in}{1.365213in}}%
\pgfpathlineto{\pgfqpoint{1.373600in}{0.681113in}}%
\pgfpathlineto{\pgfqpoint{1.375496in}{1.438752in}}%
\pgfpathlineto{\pgfqpoint{1.376607in}{0.775223in}}%
\pgfpathlineto{\pgfqpoint{1.379034in}{1.271338in}}%
\pgfpathlineto{\pgfqpoint{1.380095in}{0.698125in}}%
\pgfpathlineto{\pgfqpoint{1.382067in}{1.355799in}}%
\pgfpathlineto{\pgfqpoint{1.383550in}{0.793057in}}%
\pgfpathlineto{\pgfqpoint{1.385749in}{1.383307in}}%
\pgfpathlineto{\pgfqpoint{1.386732in}{0.702633in}}%
\pgfpathlineto{\pgfqpoint{1.388706in}{1.362018in}}%
\pgfpathlineto{\pgfqpoint{1.390478in}{0.789974in}}%
\pgfpathlineto{\pgfqpoint{1.391894in}{1.405644in}}%
\pgfpathlineto{\pgfqpoint{1.393768in}{0.784924in}}%
\pgfpathlineto{\pgfqpoint{1.395647in}{1.417129in}}%
\pgfpathlineto{\pgfqpoint{1.397624in}{0.741581in}}%
\pgfpathlineto{\pgfqpoint{1.398860in}{1.392680in}}%
\pgfpathlineto{\pgfqpoint{1.400481in}{0.921537in}}%
\pgfpathlineto{\pgfqpoint{1.402876in}{1.424858in}}%
\pgfpathlineto{\pgfqpoint{1.403796in}{0.890553in}}%
\pgfpathlineto{\pgfqpoint{1.405432in}{1.382607in}}%
\pgfpathlineto{\pgfqpoint{1.407194in}{0.833907in}}%
\pgfpathlineto{\pgfqpoint{1.409212in}{1.435953in}}%
\pgfpathlineto{\pgfqpoint{1.410587in}{0.877449in}}%
\pgfpathlineto{\pgfqpoint{1.412686in}{1.417560in}}%
\pgfpathlineto{\pgfqpoint{1.414522in}{0.893745in}}%
\pgfpathlineto{\pgfqpoint{1.416015in}{1.605753in}}%
\pgfpathlineto{\pgfqpoint{1.417632in}{0.858890in}}%
\pgfpathlineto{\pgfqpoint{1.419625in}{1.440061in}}%
\pgfpathlineto{\pgfqpoint{1.420909in}{0.807002in}}%
\pgfpathlineto{\pgfqpoint{1.422822in}{1.391606in}}%
\pgfpathlineto{\pgfqpoint{1.424821in}{0.809985in}}%
\pgfpathlineto{\pgfqpoint{1.426026in}{1.449228in}}%
\pgfpathlineto{\pgfqpoint{1.428148in}{0.778746in}}%
\pgfpathlineto{\pgfqpoint{1.429295in}{1.323713in}}%
\pgfpathlineto{\pgfqpoint{1.431723in}{0.780216in}}%
\pgfpathlineto{\pgfqpoint{1.432668in}{1.325402in}}%
\pgfpathlineto{\pgfqpoint{1.435050in}{0.728165in}}%
\pgfpathlineto{\pgfqpoint{1.436559in}{1.364553in}}%
\pgfpathlineto{\pgfqpoint{1.437818in}{0.745708in}}%
\pgfpathlineto{\pgfqpoint{1.440797in}{1.453012in}}%
\pgfpathlineto{\pgfqpoint{1.441377in}{0.883427in}}%
\pgfpathlineto{\pgfqpoint{1.443985in}{1.447941in}}%
\pgfpathlineto{\pgfqpoint{1.444987in}{0.742887in}}%
\pgfpathlineto{\pgfqpoint{1.446416in}{1.362179in}}%
\pgfpathlineto{\pgfqpoint{1.448269in}{0.748897in}}%
\pgfpathlineto{\pgfqpoint{1.450520in}{1.327036in}}%
\pgfpathlineto{\pgfqpoint{1.451641in}{0.693523in}}%
\pgfpathlineto{\pgfqpoint{1.453436in}{1.340798in}}%
\pgfpathlineto{\pgfqpoint{1.454991in}{0.791454in}}%
\pgfpathlineto{\pgfqpoint{1.457038in}{1.407657in}}%
\pgfpathlineto{\pgfqpoint{1.458219in}{0.776878in}}%
\pgfpathlineto{\pgfqpoint{1.459900in}{1.292688in}}%
\pgfpathlineto{\pgfqpoint{1.462532in}{0.730894in}}%
\pgfpathlineto{\pgfqpoint{1.463360in}{1.288663in}}%
\pgfpathlineto{\pgfqpoint{1.465155in}{0.773732in}}%
\pgfpathlineto{\pgfqpoint{1.466908in}{1.405804in}}%
\pgfpathlineto{\pgfqpoint{1.468531in}{0.823142in}}%
\pgfpathlineto{\pgfqpoint{1.470352in}{1.331305in}}%
\pgfpathlineto{\pgfqpoint{1.472642in}{0.683926in}}%
\pgfpathlineto{\pgfqpoint{1.473574in}{1.352305in}}%
\pgfpathlineto{\pgfqpoint{1.476372in}{0.723550in}}%
\pgfpathlineto{\pgfqpoint{1.477176in}{1.362540in}}%
\pgfpathlineto{\pgfqpoint{1.478747in}{0.731520in}}%
\pgfpathlineto{\pgfqpoint{1.481095in}{1.322030in}}%
\pgfpathlineto{\pgfqpoint{1.482433in}{0.702077in}}%
\pgfpathlineto{\pgfqpoint{1.483774in}{1.290359in}}%
\pgfpathlineto{\pgfqpoint{1.485518in}{0.747890in}}%
\pgfpathlineto{\pgfqpoint{1.488008in}{1.334599in}}%
\pgfpathlineto{\pgfqpoint{1.489086in}{0.812789in}}%
\pgfpathlineto{\pgfqpoint{1.490636in}{1.307850in}}%
\pgfpathlineto{\pgfqpoint{1.492416in}{0.787280in}}%
\pgfpathlineto{\pgfqpoint{1.494078in}{1.339495in}}%
\pgfpathlineto{\pgfqpoint{1.495804in}{0.849248in}}%
\pgfpathlineto{\pgfqpoint{1.497613in}{1.320831in}}%
\pgfpathlineto{\pgfqpoint{1.499835in}{0.728479in}}%
\pgfpathlineto{\pgfqpoint{1.500901in}{1.377530in}}%
\pgfpathlineto{\pgfqpoint{1.502514in}{0.794818in}}%
\pgfpathlineto{\pgfqpoint{1.504564in}{1.382294in}}%
\pgfpathlineto{\pgfqpoint{1.506372in}{0.711039in}}%
\pgfpathlineto{\pgfqpoint{1.507756in}{1.308129in}}%
\pgfpathlineto{\pgfqpoint{1.509335in}{0.833888in}}%
\pgfpathlineto{\pgfqpoint{1.511918in}{1.517844in}}%
\pgfpathlineto{\pgfqpoint{1.512678in}{0.840924in}}%
\pgfpathlineto{\pgfqpoint{1.516034in}{1.389029in}}%
\pgfpathlineto{\pgfqpoint{1.516175in}{0.851433in}}%
\pgfpathlineto{\pgfqpoint{1.518249in}{1.475349in}}%
\pgfpathlineto{\pgfqpoint{1.519516in}{0.804473in}}%
\pgfpathlineto{\pgfqpoint{1.522077in}{1.432680in}}%
\pgfpathlineto{\pgfqpoint{1.522980in}{0.960765in}}%
\pgfpathlineto{\pgfqpoint{1.525630in}{1.443056in}}%
\pgfpathlineto{\pgfqpoint{1.526710in}{0.744728in}}%
\pgfpathlineto{\pgfqpoint{1.528797in}{1.368300in}}%
\pgfpathlineto{\pgfqpoint{1.529830in}{0.731176in}}%
\pgfpathlineto{\pgfqpoint{1.531555in}{1.324231in}}%
\pgfpathlineto{\pgfqpoint{1.533144in}{0.746752in}}%
\pgfpathlineto{\pgfqpoint{1.535782in}{1.437870in}}%
\pgfpathlineto{\pgfqpoint{1.536929in}{0.820126in}}%
\pgfpathlineto{\pgfqpoint{1.538690in}{1.345098in}}%
\pgfpathlineto{\pgfqpoint{1.540383in}{0.677946in}}%
\pgfpathlineto{\pgfqpoint{1.541768in}{1.355984in}}%
\pgfpathlineto{\pgfqpoint{1.543736in}{0.774827in}}%
\pgfpathlineto{\pgfqpoint{1.545243in}{1.354026in}}%
\pgfpathlineto{\pgfqpoint{1.547524in}{0.811900in}}%
\pgfpathlineto{\pgfqpoint{1.549398in}{1.510101in}}%
\pgfpathlineto{\pgfqpoint{1.550703in}{0.876951in}}%
\pgfpathlineto{\pgfqpoint{1.552267in}{1.413884in}}%
\pgfpathlineto{\pgfqpoint{1.553881in}{0.781112in}}%
\pgfpathlineto{\pgfqpoint{1.556634in}{1.324688in}}%
\pgfpathlineto{\pgfqpoint{1.556972in}{0.819462in}}%
\pgfpathlineto{\pgfqpoint{1.558629in}{1.268034in}}%
\pgfpathlineto{\pgfqpoint{1.560365in}{0.707305in}}%
\pgfpathlineto{\pgfqpoint{1.562320in}{1.247627in}}%
\pgfpathlineto{\pgfqpoint{1.564044in}{0.645955in}}%
\pgfpathlineto{\pgfqpoint{1.565529in}{1.300454in}}%
\pgfpathlineto{\pgfqpoint{1.567253in}{0.821222in}}%
\pgfpathlineto{\pgfqpoint{1.568854in}{1.325741in}}%
\pgfpathlineto{\pgfqpoint{1.570693in}{0.771242in}}%
\pgfpathlineto{\pgfqpoint{1.572838in}{1.359858in}}%
\pgfpathlineto{\pgfqpoint{1.574100in}{0.785646in}}%
\pgfpathlineto{\pgfqpoint{1.576795in}{1.339458in}}%
\pgfpathlineto{\pgfqpoint{1.577506in}{0.734619in}}%
\pgfpathlineto{\pgfqpoint{1.579369in}{1.422965in}}%
\pgfpathlineto{\pgfqpoint{1.581055in}{0.838006in}}%
\pgfpathlineto{\pgfqpoint{1.582470in}{1.364695in}}%
\pgfpathlineto{\pgfqpoint{1.584494in}{0.756866in}}%
\pgfpathlineto{\pgfqpoint{1.586448in}{1.331644in}}%
\pgfpathlineto{\pgfqpoint{1.587655in}{0.777953in}}%
\pgfpathlineto{\pgfqpoint{1.589310in}{1.264234in}}%
\pgfpathlineto{\pgfqpoint{1.591268in}{0.726712in}}%
\pgfpathlineto{\pgfqpoint{1.592715in}{1.294303in}}%
\pgfpathlineto{\pgfqpoint{1.594691in}{0.667326in}}%
\pgfpathlineto{\pgfqpoint{1.596276in}{1.282776in}}%
\pgfpathlineto{\pgfqpoint{1.598458in}{0.686788in}}%
\pgfpathlineto{\pgfqpoint{1.599711in}{1.353833in}}%
\pgfpathlineto{\pgfqpoint{1.601275in}{0.693250in}}%
\pgfpathlineto{\pgfqpoint{1.603205in}{1.197850in}}%
\pgfpathlineto{\pgfqpoint{1.605245in}{0.662766in}}%
\pgfpathlineto{\pgfqpoint{1.606436in}{1.239326in}}%
\pgfpathlineto{\pgfqpoint{1.609253in}{0.632740in}}%
\pgfpathlineto{\pgfqpoint{1.610223in}{1.291390in}}%
\pgfpathlineto{\pgfqpoint{1.611960in}{0.583104in}}%
\pgfpathlineto{\pgfqpoint{1.613398in}{1.281041in}}%
\pgfpathlineto{\pgfqpoint{1.615205in}{0.710858in}}%
\pgfpathlineto{\pgfqpoint{1.616637in}{1.305816in}}%
\pgfpathlineto{\pgfqpoint{1.619116in}{0.648970in}}%
\pgfpathlineto{\pgfqpoint{1.619910in}{1.206795in}}%
\pgfpathlineto{\pgfqpoint{1.621859in}{0.666028in}}%
\pgfpathlineto{\pgfqpoint{1.624367in}{1.213621in}}%
\pgfpathlineto{\pgfqpoint{1.625043in}{0.692451in}}%
\pgfpathlineto{\pgfqpoint{1.627956in}{1.332501in}}%
\pgfpathlineto{\pgfqpoint{1.628489in}{0.786789in}}%
\pgfpathlineto{\pgfqpoint{1.631311in}{1.449064in}}%
\pgfpathlineto{\pgfqpoint{1.631860in}{0.705608in}}%
\pgfpathlineto{\pgfqpoint{1.633690in}{1.301321in}}%
\pgfpathlineto{\pgfqpoint{1.635828in}{0.589584in}}%
\pgfpathlineto{\pgfqpoint{1.637029in}{1.366969in}}%
\pgfpathlineto{\pgfqpoint{1.639349in}{0.746610in}}%
\pgfpathlineto{\pgfqpoint{1.640747in}{1.236294in}}%
\pgfpathlineto{\pgfqpoint{1.642057in}{0.683055in}}%
\pgfpathlineto{\pgfqpoint{1.645030in}{1.319193in}}%
\pgfpathlineto{\pgfqpoint{1.645440in}{0.643841in}}%
\pgfpathlineto{\pgfqpoint{1.647496in}{1.275788in}}%
\pgfpathlineto{\pgfqpoint{1.649671in}{0.630137in}}%
\pgfpathlineto{\pgfqpoint{1.650816in}{1.303238in}}%
\pgfpathlineto{\pgfqpoint{1.652277in}{0.690283in}}%
\pgfpathlineto{\pgfqpoint{1.654805in}{1.318124in}}%
\pgfpathlineto{\pgfqpoint{1.655816in}{0.722921in}}%
\pgfpathlineto{\pgfqpoint{1.658901in}{1.412767in}}%
\pgfpathlineto{\pgfqpoint{1.659129in}{0.822622in}}%
\pgfpathlineto{\pgfqpoint{1.660960in}{1.358618in}}%
\pgfpathlineto{\pgfqpoint{1.663186in}{0.776963in}}%
\pgfpathlineto{\pgfqpoint{1.664451in}{1.383123in}}%
\pgfpathlineto{\pgfqpoint{1.665967in}{0.824785in}}%
\pgfpathlineto{\pgfqpoint{1.668043in}{1.296710in}}%
\pgfpathlineto{\pgfqpoint{1.669270in}{0.790281in}}%
\pgfpathlineto{\pgfqpoint{1.671500in}{1.278818in}}%
\pgfpathlineto{\pgfqpoint{1.673005in}{0.757550in}}%
\pgfpathlineto{\pgfqpoint{1.674738in}{1.296486in}}%
\pgfpathlineto{\pgfqpoint{1.676162in}{0.712748in}}%
\pgfpathlineto{\pgfqpoint{1.677874in}{1.292808in}}%
\pgfpathlineto{\pgfqpoint{1.680154in}{0.718748in}}%
\pgfpathlineto{\pgfqpoint{1.681669in}{1.291293in}}%
\pgfpathlineto{\pgfqpoint{1.683641in}{0.745297in}}%
\pgfpathlineto{\pgfqpoint{1.684928in}{1.276445in}}%
\pgfpathlineto{\pgfqpoint{1.686305in}{0.806973in}}%
\pgfpathlineto{\pgfqpoint{1.687993in}{1.330609in}}%
\pgfpathlineto{\pgfqpoint{1.689983in}{0.792975in}}%
\pgfpathlineto{\pgfqpoint{1.692477in}{1.366835in}}%
\pgfpathlineto{\pgfqpoint{1.693284in}{0.724133in}}%
\pgfpathlineto{\pgfqpoint{1.694870in}{1.227512in}}%
\pgfpathlineto{\pgfqpoint{1.697583in}{0.728683in}}%
\pgfpathlineto{\pgfqpoint{1.698487in}{1.334367in}}%
\pgfpathlineto{\pgfqpoint{1.700116in}{0.775164in}}%
\pgfpathlineto{\pgfqpoint{1.702215in}{1.224535in}}%
\pgfpathlineto{\pgfqpoint{1.703538in}{0.787977in}}%
\pgfpathlineto{\pgfqpoint{1.705129in}{1.366220in}}%
\pgfpathlineto{\pgfqpoint{1.706773in}{0.761050in}}%
\pgfpathlineto{\pgfqpoint{1.708997in}{1.440312in}}%
\pgfpathlineto{\pgfqpoint{1.710466in}{0.722146in}}%
\pgfpathlineto{\pgfqpoint{1.711853in}{1.356220in}}%
\pgfpathlineto{\pgfqpoint{1.713617in}{0.729204in}}%
\pgfpathlineto{\pgfqpoint{1.715228in}{1.356904in}}%
\pgfpathlineto{\pgfqpoint{1.717491in}{0.687959in}}%
\pgfpathlineto{\pgfqpoint{1.718639in}{1.277564in}}%
\pgfpathlineto{\pgfqpoint{1.721375in}{0.681350in}}%
\pgfpathlineto{\pgfqpoint{1.722129in}{1.285130in}}%
\pgfpathlineto{\pgfqpoint{1.724581in}{0.754990in}}%
\pgfpathlineto{\pgfqpoint{1.725472in}{1.302543in}}%
\pgfpathlineto{\pgfqpoint{1.727394in}{0.730605in}}%
\pgfpathlineto{\pgfqpoint{1.729066in}{1.356801in}}%
\pgfpathlineto{\pgfqpoint{1.730775in}{0.638822in}}%
\pgfpathlineto{\pgfqpoint{1.732435in}{1.242282in}}%
\pgfpathlineto{\pgfqpoint{1.734598in}{0.677654in}}%
\pgfpathlineto{\pgfqpoint{1.735841in}{1.306973in}}%
\pgfpathlineto{\pgfqpoint{1.737576in}{0.638336in}}%
\pgfpathlineto{\pgfqpoint{1.740034in}{1.259378in}}%
\pgfpathlineto{\pgfqpoint{1.741393in}{0.704275in}}%
\pgfpathlineto{\pgfqpoint{1.742618in}{1.283443in}}%
\pgfpathlineto{\pgfqpoint{1.745394in}{0.700208in}}%
\pgfpathlineto{\pgfqpoint{1.745877in}{1.209760in}}%
\pgfpathlineto{\pgfqpoint{1.748210in}{0.681589in}}%
\pgfpathlineto{\pgfqpoint{1.749377in}{1.324719in}}%
\pgfpathlineto{\pgfqpoint{1.751511in}{0.679240in}}%
\pgfpathlineto{\pgfqpoint{1.752780in}{1.260954in}}%
\pgfpathlineto{\pgfqpoint{1.754779in}{0.652040in}}%
\pgfpathlineto{\pgfqpoint{1.756093in}{1.216319in}}%
\pgfpathlineto{\pgfqpoint{1.759121in}{0.624902in}}%
\pgfpathlineto{\pgfqpoint{1.759591in}{1.163564in}}%
\pgfpathlineto{\pgfqpoint{1.762788in}{0.602769in}}%
\pgfpathlineto{\pgfqpoint{1.762901in}{1.094409in}}%
\pgfpathlineto{\pgfqpoint{1.765059in}{0.662404in}}%
\pgfpathlineto{\pgfqpoint{1.766614in}{1.267047in}}%
\pgfpathlineto{\pgfqpoint{1.767988in}{0.724971in}}%
\pgfpathlineto{\pgfqpoint{1.769945in}{1.224729in}}%
\pgfpathlineto{\pgfqpoint{1.771463in}{0.761820in}}%
\pgfpathlineto{\pgfqpoint{1.773683in}{1.277339in}}%
\pgfpathlineto{\pgfqpoint{1.775107in}{0.691911in}}%
\pgfpathlineto{\pgfqpoint{1.777790in}{1.269663in}}%
\pgfpathlineto{\pgfqpoint{1.778402in}{0.676729in}}%
\pgfpathlineto{\pgfqpoint{1.779956in}{1.187738in}}%
\pgfpathlineto{\pgfqpoint{1.781911in}{0.690366in}}%
\pgfpathlineto{\pgfqpoint{1.783429in}{1.235012in}}%
\pgfpathlineto{\pgfqpoint{1.785080in}{0.687769in}}%
\pgfpathlineto{\pgfqpoint{1.787109in}{1.204632in}}%
\pgfpathlineto{\pgfqpoint{1.788498in}{0.649327in}}%
\pgfpathlineto{\pgfqpoint{1.790458in}{1.208000in}}%
\pgfpathlineto{\pgfqpoint{1.791819in}{0.732657in}}%
\pgfpathlineto{\pgfqpoint{1.794039in}{1.319874in}}%
\pgfpathlineto{\pgfqpoint{1.795230in}{0.679355in}}%
\pgfpathlineto{\pgfqpoint{1.797368in}{1.238907in}}%
\pgfpathlineto{\pgfqpoint{1.798647in}{0.687532in}}%
\pgfpathlineto{\pgfqpoint{1.800613in}{1.109656in}}%
\pgfpathlineto{\pgfqpoint{1.802037in}{0.656053in}}%
\pgfpathlineto{\pgfqpoint{1.803922in}{1.222545in}}%
\pgfpathlineto{\pgfqpoint{1.805697in}{0.646565in}}%
\pgfpathlineto{\pgfqpoint{1.807327in}{1.243172in}}%
\pgfpathlineto{\pgfqpoint{1.809327in}{0.586155in}}%
\pgfpathlineto{\pgfqpoint{1.810825in}{1.135401in}}%
\pgfpathlineto{\pgfqpoint{1.813881in}{0.568605in}}%
\pgfpathlineto{\pgfqpoint{1.814443in}{1.195736in}}%
\pgfpathlineto{\pgfqpoint{1.816241in}{0.670774in}}%
\pgfpathlineto{\pgfqpoint{1.818296in}{1.296535in}}%
\pgfpathlineto{\pgfqpoint{1.819042in}{0.770876in}}%
\pgfpathlineto{\pgfqpoint{1.820906in}{1.257411in}}%
\pgfpathlineto{\pgfqpoint{1.822462in}{0.642678in}}%
\pgfpathlineto{\pgfqpoint{1.824270in}{1.245649in}}%
\pgfpathlineto{\pgfqpoint{1.825973in}{0.768510in}}%
\pgfpathlineto{\pgfqpoint{1.828264in}{1.320876in}}%
\pgfpathlineto{\pgfqpoint{1.829745in}{0.656979in}}%
\pgfpathlineto{\pgfqpoint{1.832358in}{1.308453in}}%
\pgfpathlineto{\pgfqpoint{1.833463in}{0.694266in}}%
\pgfpathlineto{\pgfqpoint{1.835141in}{1.284739in}}%
\pgfpathlineto{\pgfqpoint{1.836373in}{0.688377in}}%
\pgfpathlineto{\pgfqpoint{1.838087in}{1.331821in}}%
\pgfpathlineto{\pgfqpoint{1.839512in}{0.775289in}}%
\pgfpathlineto{\pgfqpoint{1.841723in}{1.296099in}}%
\pgfpathlineto{\pgfqpoint{1.842892in}{0.762506in}}%
\pgfpathlineto{\pgfqpoint{1.844588in}{1.353181in}}%
\pgfpathlineto{\pgfqpoint{1.846408in}{0.761065in}}%
\pgfpathlineto{\pgfqpoint{1.848030in}{1.389034in}}%
\pgfpathlineto{\pgfqpoint{1.849902in}{0.807017in}}%
\pgfpathlineto{\pgfqpoint{1.852507in}{1.360991in}}%
\pgfpathlineto{\pgfqpoint{1.853186in}{0.758436in}}%
\pgfpathlineto{\pgfqpoint{1.854841in}{1.269639in}}%
\pgfpathlineto{\pgfqpoint{1.856610in}{0.813305in}}%
\pgfpathlineto{\pgfqpoint{1.859600in}{1.374758in}}%
\pgfpathlineto{\pgfqpoint{1.860144in}{0.839233in}}%
\pgfpathlineto{\pgfqpoint{1.861595in}{1.247082in}}%
\pgfpathlineto{\pgfqpoint{1.864550in}{0.738659in}}%
\pgfpathlineto{\pgfqpoint{1.865546in}{1.382926in}}%
\pgfpathlineto{\pgfqpoint{1.867263in}{0.697124in}}%
\pgfpathlineto{\pgfqpoint{1.868912in}{1.331739in}}%
\pgfpathlineto{\pgfqpoint{1.870172in}{0.736037in}}%
\pgfpathlineto{\pgfqpoint{1.871854in}{1.281899in}}%
\pgfpathlineto{\pgfqpoint{1.874775in}{0.631686in}}%
\pgfpathlineto{\pgfqpoint{1.875335in}{1.330094in}}%
\pgfpathlineto{\pgfqpoint{1.877479in}{0.688239in}}%
\pgfpathlineto{\pgfqpoint{1.879390in}{1.325885in}}%
\pgfpathlineto{\pgfqpoint{1.880326in}{0.677089in}}%
\pgfpathlineto{\pgfqpoint{1.882389in}{1.208665in}}%
\pgfpathlineto{\pgfqpoint{1.884342in}{0.751361in}}%
\pgfpathlineto{\pgfqpoint{1.885706in}{1.336232in}}%
\pgfpathlineto{\pgfqpoint{1.887227in}{0.701271in}}%
\pgfpathlineto{\pgfqpoint{1.889943in}{1.348121in}}%
\pgfpathlineto{\pgfqpoint{1.891255in}{0.770234in}}%
\pgfpathlineto{\pgfqpoint{1.892382in}{1.339690in}}%
\pgfpathlineto{\pgfqpoint{1.894182in}{0.743708in}}%
\pgfpathlineto{\pgfqpoint{1.895927in}{1.282798in}}%
\pgfpathlineto{\pgfqpoint{1.897354in}{0.848876in}}%
\pgfpathlineto{\pgfqpoint{1.899364in}{1.392256in}}%
\pgfpathlineto{\pgfqpoint{1.901243in}{0.841281in}}%
\pgfpathlineto{\pgfqpoint{1.902610in}{1.446298in}}%
\pgfpathlineto{\pgfqpoint{1.904635in}{0.816625in}}%
\pgfpathlineto{\pgfqpoint{1.906066in}{1.392024in}}%
\pgfpathlineto{\pgfqpoint{1.907644in}{0.801019in}}%
\pgfpathlineto{\pgfqpoint{1.909469in}{1.356366in}}%
\pgfpathlineto{\pgfqpoint{1.911619in}{0.789843in}}%
\pgfpathlineto{\pgfqpoint{1.912839in}{1.289666in}}%
\pgfpathlineto{\pgfqpoint{1.914405in}{0.766897in}}%
\pgfpathlineto{\pgfqpoint{1.916170in}{1.379025in}}%
\pgfpathlineto{\pgfqpoint{1.918721in}{0.771562in}}%
\pgfpathlineto{\pgfqpoint{1.920238in}{1.489897in}}%
\pgfpathlineto{\pgfqpoint{1.921262in}{0.849064in}}%
\pgfpathlineto{\pgfqpoint{1.923053in}{1.370189in}}%
\pgfpathlineto{\pgfqpoint{1.924764in}{0.802876in}}%
\pgfpathlineto{\pgfqpoint{1.926293in}{1.287388in}}%
\pgfpathlineto{\pgfqpoint{1.928055in}{0.724102in}}%
\pgfpathlineto{\pgfqpoint{1.929742in}{1.307003in}}%
\pgfpathlineto{\pgfqpoint{1.931449in}{0.768843in}}%
\pgfpathlineto{\pgfqpoint{1.933128in}{1.228483in}}%
\pgfpathlineto{\pgfqpoint{1.935546in}{0.726924in}}%
\pgfpathlineto{\pgfqpoint{1.936533in}{1.289955in}}%
\pgfpathlineto{\pgfqpoint{1.938858in}{0.807543in}}%
\pgfpathlineto{\pgfqpoint{1.940048in}{1.393451in}}%
\pgfpathlineto{\pgfqpoint{1.941752in}{0.725593in}}%
\pgfpathlineto{\pgfqpoint{1.944168in}{1.258516in}}%
\pgfpathlineto{\pgfqpoint{1.945008in}{0.719286in}}%
\pgfpathlineto{\pgfqpoint{1.946939in}{1.243148in}}%
\pgfpathlineto{\pgfqpoint{1.948621in}{0.774137in}}%
\pgfpathlineto{\pgfqpoint{1.950400in}{1.336121in}}%
\pgfpathlineto{\pgfqpoint{1.952030in}{0.875227in}}%
\pgfpathlineto{\pgfqpoint{1.953781in}{1.385602in}}%
\pgfpathlineto{\pgfqpoint{1.955275in}{0.849499in}}%
\pgfpathlineto{\pgfqpoint{1.956939in}{1.368551in}}%
\pgfpathlineto{\pgfqpoint{1.959027in}{0.835782in}}%
\pgfpathlineto{\pgfqpoint{1.960659in}{1.405829in}}%
\pgfpathlineto{\pgfqpoint{1.962043in}{0.791578in}}%
\pgfpathlineto{\pgfqpoint{1.964893in}{1.387296in}}%
\pgfpathlineto{\pgfqpoint{1.965948in}{0.755649in}}%
\pgfpathlineto{\pgfqpoint{1.967467in}{1.428316in}}%
\pgfpathlineto{\pgfqpoint{1.968884in}{0.901330in}}%
\pgfpathlineto{\pgfqpoint{1.970525in}{1.418947in}}%
\pgfpathlineto{\pgfqpoint{1.972612in}{0.802457in}}%
\pgfpathlineto{\pgfqpoint{1.974221in}{1.407615in}}%
\pgfpathlineto{\pgfqpoint{1.975655in}{0.871172in}}%
\pgfpathlineto{\pgfqpoint{1.978797in}{1.381327in}}%
\pgfpathlineto{\pgfqpoint{1.979062in}{0.843243in}}%
\pgfpathlineto{\pgfqpoint{1.981756in}{1.383788in}}%
\pgfpathlineto{\pgfqpoint{1.983263in}{0.724219in}}%
\pgfpathlineto{\pgfqpoint{1.984658in}{1.371937in}}%
\pgfpathlineto{\pgfqpoint{1.986621in}{0.735783in}}%
\pgfpathlineto{\pgfqpoint{1.988389in}{1.267887in}}%
\pgfpathlineto{\pgfqpoint{1.989332in}{0.737303in}}%
\pgfpathlineto{\pgfqpoint{1.991853in}{1.318219in}}%
\pgfpathlineto{\pgfqpoint{1.992688in}{0.696063in}}%
\pgfpathlineto{\pgfqpoint{1.994465in}{1.282202in}}%
\pgfpathlineto{\pgfqpoint{1.996101in}{0.757504in}}%
\pgfpathlineto{\pgfqpoint{1.998785in}{1.332810in}}%
\pgfpathlineto{\pgfqpoint{1.999504in}{0.755043in}}%
\pgfpathlineto{\pgfqpoint{2.001915in}{1.285334in}}%
\pgfpathlineto{\pgfqpoint{2.002888in}{0.773758in}}%
\pgfpathlineto{\pgfqpoint{2.004761in}{1.311260in}}%
\pgfpathlineto{\pgfqpoint{2.006312in}{0.762342in}}%
\pgfpathlineto{\pgfqpoint{2.007998in}{1.330232in}}%
\pgfpathlineto{\pgfqpoint{2.010727in}{0.700027in}}%
\pgfpathlineto{\pgfqpoint{2.012245in}{1.393634in}}%
\pgfpathlineto{\pgfqpoint{2.014573in}{0.726918in}}%
\pgfpathlineto{\pgfqpoint{2.015077in}{1.370609in}}%
\pgfpathlineto{\pgfqpoint{2.017079in}{0.669752in}}%
\pgfpathlineto{\pgfqpoint{2.018654in}{1.261565in}}%
\pgfpathlineto{\pgfqpoint{2.021373in}{0.718187in}}%
\pgfpathlineto{\pgfqpoint{2.022059in}{1.276379in}}%
\pgfpathlineto{\pgfqpoint{2.023439in}{0.816797in}}%
\pgfpathlineto{\pgfqpoint{2.025632in}{1.370554in}}%
\pgfpathlineto{\pgfqpoint{2.026762in}{0.821245in}}%
\pgfpathlineto{\pgfqpoint{2.028415in}{1.337700in}}%
\pgfpathlineto{\pgfqpoint{2.030213in}{0.777981in}}%
\pgfpathlineto{\pgfqpoint{2.032129in}{1.318711in}}%
\pgfpathlineto{\pgfqpoint{2.033858in}{0.762672in}}%
\pgfpathlineto{\pgfqpoint{2.035360in}{1.351547in}}%
\pgfpathlineto{\pgfqpoint{2.036995in}{0.719755in}}%
\pgfpathlineto{\pgfqpoint{2.038697in}{1.322210in}}%
\pgfpathlineto{\pgfqpoint{2.040512in}{0.801419in}}%
\pgfpathlineto{\pgfqpoint{2.042122in}{1.288763in}}%
\pgfpathlineto{\pgfqpoint{2.043759in}{0.644304in}}%
\pgfpathlineto{\pgfqpoint{2.045649in}{1.300121in}}%
\pgfpathlineto{\pgfqpoint{2.048584in}{0.710206in}}%
\pgfpathlineto{\pgfqpoint{2.049340in}{1.365397in}}%
\pgfpathlineto{\pgfqpoint{2.050712in}{0.820707in}}%
\pgfpathlineto{\pgfqpoint{2.052749in}{1.438704in}}%
\pgfpathlineto{\pgfqpoint{2.054313in}{0.755420in}}%
\pgfpathlineto{\pgfqpoint{2.055759in}{1.297957in}}%
\pgfpathlineto{\pgfqpoint{2.057959in}{0.762061in}}%
\pgfpathlineto{\pgfqpoint{2.059397in}{1.363895in}}%
\pgfpathlineto{\pgfqpoint{2.060767in}{0.766136in}}%
\pgfpathlineto{\pgfqpoint{2.063145in}{1.387825in}}%
\pgfpathlineto{\pgfqpoint{2.064372in}{0.730001in}}%
\pgfpathlineto{\pgfqpoint{2.066375in}{1.332504in}}%
\pgfpathlineto{\pgfqpoint{2.068549in}{0.678490in}}%
\pgfpathlineto{\pgfqpoint{2.069548in}{1.349359in}}%
\pgfpathlineto{\pgfqpoint{2.071960in}{0.781061in}}%
\pgfpathlineto{\pgfqpoint{2.072663in}{1.287488in}}%
\pgfpathlineto{\pgfqpoint{2.075548in}{0.604348in}}%
\pgfpathlineto{\pgfqpoint{2.076205in}{1.265442in}}%
\pgfpathlineto{\pgfqpoint{2.078292in}{0.722096in}}%
\pgfpathlineto{\pgfqpoint{2.079555in}{1.433502in}}%
\pgfpathlineto{\pgfqpoint{2.081556in}{0.795688in}}%
\pgfpathlineto{\pgfqpoint{2.082974in}{1.392659in}}%
\pgfpathlineto{\pgfqpoint{2.084963in}{0.798368in}}%
\pgfpathlineto{\pgfqpoint{2.086283in}{1.271955in}}%
\pgfpathlineto{\pgfqpoint{2.088007in}{0.785934in}}%
\pgfpathlineto{\pgfqpoint{2.089819in}{1.371507in}}%
\pgfpathlineto{\pgfqpoint{2.091610in}{0.749031in}}%
\pgfpathlineto{\pgfqpoint{2.093185in}{1.328304in}}%
\pgfpathlineto{\pgfqpoint{2.095250in}{0.655051in}}%
\pgfpathlineto{\pgfqpoint{2.096628in}{1.238474in}}%
\pgfpathlineto{\pgfqpoint{2.098715in}{0.742517in}}%
\pgfpathlineto{\pgfqpoint{2.099911in}{1.350552in}}%
\pgfpathlineto{\pgfqpoint{2.102030in}{0.696227in}}%
\pgfpathlineto{\pgfqpoint{2.103313in}{1.204882in}}%
\pgfpathlineto{\pgfqpoint{2.105632in}{0.760906in}}%
\pgfpathlineto{\pgfqpoint{2.106721in}{1.235577in}}%
\pgfpathlineto{\pgfqpoint{2.108662in}{0.767119in}}%
\pgfpathlineto{\pgfqpoint{2.110352in}{1.288692in}}%
\pgfpathlineto{\pgfqpoint{2.111977in}{0.704537in}}%
\pgfpathlineto{\pgfqpoint{2.113521in}{1.272820in}}%
\pgfpathlineto{\pgfqpoint{2.115210in}{0.750090in}}%
\pgfpathlineto{\pgfqpoint{2.117056in}{1.312807in}}%
\pgfpathlineto{\pgfqpoint{2.119650in}{0.661173in}}%
\pgfpathlineto{\pgfqpoint{2.120484in}{1.309459in}}%
\pgfpathlineto{\pgfqpoint{2.122250in}{0.681584in}}%
\pgfpathlineto{\pgfqpoint{2.124001in}{1.201402in}}%
\pgfpathlineto{\pgfqpoint{2.125422in}{0.621531in}}%
\pgfpathlineto{\pgfqpoint{2.127306in}{1.207185in}}%
\pgfpathlineto{\pgfqpoint{2.129146in}{0.671907in}}%
\pgfpathlineto{\pgfqpoint{2.131306in}{1.333852in}}%
\pgfpathlineto{\pgfqpoint{2.132276in}{0.710345in}}%
\pgfpathlineto{\pgfqpoint{2.134302in}{1.271419in}}%
\pgfpathlineto{\pgfqpoint{2.136038in}{0.617762in}}%
\pgfpathlineto{\pgfqpoint{2.137441in}{1.228183in}}%
\pgfpathlineto{\pgfqpoint{2.139132in}{0.691364in}}%
\pgfpathlineto{\pgfqpoint{2.141070in}{1.259590in}}%
\pgfpathlineto{\pgfqpoint{2.142728in}{0.717700in}}%
\pgfpathlineto{\pgfqpoint{2.144430in}{1.345134in}}%
\pgfpathlineto{\pgfqpoint{2.146443in}{0.759467in}}%
\pgfpathlineto{\pgfqpoint{2.148658in}{1.442893in}}%
\pgfpathlineto{\pgfqpoint{2.149444in}{0.867222in}}%
\pgfpathlineto{\pgfqpoint{2.151264in}{1.273997in}}%
\pgfpathlineto{\pgfqpoint{2.152685in}{0.693500in}}%
\pgfpathlineto{\pgfqpoint{2.154440in}{1.238090in}}%
\pgfpathlineto{\pgfqpoint{2.157142in}{0.681407in}}%
\pgfpathlineto{\pgfqpoint{2.158138in}{1.321444in}}%
\pgfpathlineto{\pgfqpoint{2.159638in}{0.674064in}}%
\pgfpathlineto{\pgfqpoint{2.161676in}{1.367567in}}%
\pgfpathlineto{\pgfqpoint{2.163140in}{0.705708in}}%
\pgfpathlineto{\pgfqpoint{2.164777in}{1.325284in}}%
\pgfpathlineto{\pgfqpoint{2.166825in}{0.655595in}}%
\pgfpathlineto{\pgfqpoint{2.169269in}{1.346361in}}%
\pgfpathlineto{\pgfqpoint{2.170218in}{0.666006in}}%
\pgfpathlineto{\pgfqpoint{2.171658in}{1.198628in}}%
\pgfpathlineto{\pgfqpoint{2.174109in}{0.660202in}}%
\pgfpathlineto{\pgfqpoint{2.175177in}{1.227546in}}%
\pgfpathlineto{\pgfqpoint{2.177975in}{0.586562in}}%
\pgfpathlineto{\pgfqpoint{2.178225in}{1.192217in}}%
\pgfpathlineto{\pgfqpoint{2.180250in}{0.711431in}}%
\pgfpathlineto{\pgfqpoint{2.182056in}{1.226069in}}%
\pgfpathlineto{\pgfqpoint{2.183401in}{0.677389in}}%
\pgfpathlineto{\pgfqpoint{2.185725in}{1.303376in}}%
\pgfpathlineto{\pgfqpoint{2.187361in}{0.651653in}}%
\pgfpathlineto{\pgfqpoint{2.188934in}{1.229175in}}%
\pgfpathlineto{\pgfqpoint{2.190527in}{0.718273in}}%
\pgfpathlineto{\pgfqpoint{2.192086in}{1.340500in}}%
\pgfpathlineto{\pgfqpoint{2.193769in}{0.694033in}}%
\pgfpathlineto{\pgfqpoint{2.195217in}{1.280654in}}%
\pgfpathlineto{\pgfqpoint{2.197203in}{0.614986in}}%
\pgfpathlineto{\pgfqpoint{2.199420in}{1.284042in}}%
\pgfpathlineto{\pgfqpoint{2.200747in}{0.705579in}}%
\pgfpathlineto{\pgfqpoint{2.202251in}{1.265048in}}%
\pgfpathlineto{\pgfqpoint{2.203737in}{0.753481in}}%
\pgfpathlineto{\pgfqpoint{2.206277in}{1.313404in}}%
\pgfpathlineto{\pgfqpoint{2.207417in}{0.718382in}}%
\pgfpathlineto{\pgfqpoint{2.209057in}{1.230245in}}%
\pgfpathlineto{\pgfqpoint{2.210702in}{0.737619in}}%
\pgfpathlineto{\pgfqpoint{2.212636in}{1.206943in}}%
\pgfpathlineto{\pgfqpoint{2.214229in}{0.702648in}}%
\pgfpathlineto{\pgfqpoint{2.216068in}{1.294745in}}%
\pgfpathlineto{\pgfqpoint{2.217326in}{0.696674in}}%
\pgfpathlineto{\pgfqpoint{2.219853in}{1.270147in}}%
\pgfpathlineto{\pgfqpoint{2.220758in}{0.601357in}}%
\pgfpathlineto{\pgfqpoint{2.222667in}{1.193771in}}%
\pgfpathlineto{\pgfqpoint{2.225085in}{0.582317in}}%
\pgfpathlineto{\pgfqpoint{2.225873in}{1.352392in}}%
\pgfpathlineto{\pgfqpoint{2.227577in}{0.747667in}}%
\pgfpathlineto{\pgfqpoint{2.230259in}{1.325258in}}%
\pgfpathlineto{\pgfqpoint{2.231043in}{0.708761in}}%
\pgfpathlineto{\pgfqpoint{2.233074in}{1.246824in}}%
\pgfpathlineto{\pgfqpoint{2.234575in}{0.654397in}}%
\pgfpathlineto{\pgfqpoint{2.236108in}{1.223755in}}%
\pgfpathlineto{\pgfqpoint{2.238507in}{0.679366in}}%
\pgfpathlineto{\pgfqpoint{2.239454in}{1.227458in}}%
\pgfpathlineto{\pgfqpoint{2.242188in}{0.652758in}}%
\pgfpathlineto{\pgfqpoint{2.243310in}{1.322080in}}%
\pgfpathlineto{\pgfqpoint{2.244639in}{0.778534in}}%
\pgfpathlineto{\pgfqpoint{2.246543in}{1.273052in}}%
\pgfpathlineto{\pgfqpoint{2.248436in}{0.667046in}}%
\pgfpathlineto{\pgfqpoint{2.249692in}{1.181938in}}%
\pgfpathlineto{\pgfqpoint{2.251528in}{0.663577in}}%
\pgfpathlineto{\pgfqpoint{2.253116in}{1.195142in}}%
\pgfpathlineto{\pgfqpoint{2.254945in}{0.608045in}}%
\pgfpathlineto{\pgfqpoint{2.257197in}{1.172677in}}%
\pgfpathlineto{\pgfqpoint{2.258464in}{0.539229in}}%
\pgfpathlineto{\pgfqpoint{2.260185in}{1.177050in}}%
\pgfpathlineto{\pgfqpoint{2.262390in}{0.687503in}}%
\pgfpathlineto{\pgfqpoint{2.263297in}{1.214281in}}%
\pgfpathlineto{\pgfqpoint{2.265059in}{0.643453in}}%
\pgfpathlineto{\pgfqpoint{2.267086in}{1.222817in}}%
\pgfpathlineto{\pgfqpoint{2.269317in}{0.651448in}}%
\pgfpathlineto{\pgfqpoint{2.270246in}{1.354059in}}%
\pgfpathlineto{\pgfqpoint{2.272281in}{0.563005in}}%
\pgfpathlineto{\pgfqpoint{2.273756in}{1.230409in}}%
\pgfpathlineto{\pgfqpoint{2.276196in}{0.693813in}}%
\pgfpathlineto{\pgfqpoint{2.277241in}{1.224158in}}%
\pgfpathlineto{\pgfqpoint{2.278653in}{0.683640in}}%
\pgfpathlineto{\pgfqpoint{2.281234in}{1.199927in}}%
\pgfpathlineto{\pgfqpoint{2.282046in}{0.691565in}}%
\pgfpathlineto{\pgfqpoint{2.283726in}{1.192277in}}%
\pgfpathlineto{\pgfqpoint{2.285637in}{0.647033in}}%
\pgfpathlineto{\pgfqpoint{2.287738in}{1.382732in}}%
\pgfpathlineto{\pgfqpoint{2.289213in}{0.593513in}}%
\pgfpathlineto{\pgfqpoint{2.291148in}{1.164005in}}%
\pgfpathlineto{\pgfqpoint{2.292545in}{0.670537in}}%
\pgfpathlineto{\pgfqpoint{2.294082in}{1.190349in}}%
\pgfpathlineto{\pgfqpoint{2.295728in}{0.636291in}}%
\pgfpathlineto{\pgfqpoint{2.297332in}{1.166970in}}%
\pgfpathlineto{\pgfqpoint{2.299965in}{0.642534in}}%
\pgfpathlineto{\pgfqpoint{2.300866in}{1.234500in}}%
\pgfpathlineto{\pgfqpoint{2.303211in}{0.602479in}}%
\pgfpathlineto{\pgfqpoint{2.304495in}{1.155288in}}%
\pgfpathlineto{\pgfqpoint{2.306702in}{0.576008in}}%
\pgfpathlineto{\pgfqpoint{2.307576in}{1.209793in}}%
\pgfpathlineto{\pgfqpoint{2.309728in}{0.601725in}}%
\pgfpathlineto{\pgfqpoint{2.311120in}{1.219229in}}%
\pgfpathlineto{\pgfqpoint{2.313145in}{0.647880in}}%
\pgfpathlineto{\pgfqpoint{2.314340in}{0.993918in}}%
\pgfpathlineto{\pgfqpoint{2.314340in}{0.993918in}}%
\pgfusepath{stroke}%
\end{pgfscope}%
\begin{pgfscope}%
\pgfsetrectcap%
\pgfsetmiterjoin%
\pgfsetlinewidth{0.803000pt}%
\definecolor{currentstroke}{rgb}{0.000000,0.000000,0.000000}%
\pgfsetstrokecolor{currentstroke}%
\pgfsetdash{}{0pt}%
\pgfpathmoveto{\pgfqpoint{0.530716in}{0.416448in}}%
\pgfpathlineto{\pgfqpoint{0.530716in}{1.789039in}}%
\pgfusepath{stroke}%
\end{pgfscope}%
\begin{pgfscope}%
\pgfsetrectcap%
\pgfsetmiterjoin%
\pgfsetlinewidth{0.803000pt}%
\definecolor{currentstroke}{rgb}{0.000000,0.000000,0.000000}%
\pgfsetstrokecolor{currentstroke}%
\pgfsetdash{}{0pt}%
\pgfpathmoveto{\pgfqpoint{2.399275in}{0.416448in}}%
\pgfpathlineto{\pgfqpoint{2.399275in}{1.789039in}}%
\pgfusepath{stroke}%
\end{pgfscope}%
\begin{pgfscope}%
\pgfsetrectcap%
\pgfsetmiterjoin%
\pgfsetlinewidth{0.803000pt}%
\definecolor{currentstroke}{rgb}{0.000000,0.000000,0.000000}%
\pgfsetstrokecolor{currentstroke}%
\pgfsetdash{}{0pt}%
\pgfpathmoveto{\pgfqpoint{0.530716in}{0.416447in}}%
\pgfpathlineto{\pgfqpoint{2.399275in}{0.416447in}}%
\pgfusepath{stroke}%
\end{pgfscope}%
\begin{pgfscope}%
\pgfsetrectcap%
\pgfsetmiterjoin%
\pgfsetlinewidth{0.803000pt}%
\definecolor{currentstroke}{rgb}{0.000000,0.000000,0.000000}%
\pgfsetstrokecolor{currentstroke}%
\pgfsetdash{}{0pt}%
\pgfpathmoveto{\pgfqpoint{0.530716in}{1.789039in}}%
\pgfpathlineto{\pgfqpoint{2.399275in}{1.789039in}}%
\pgfusepath{stroke}%
\end{pgfscope}%
\begin{pgfscope}%
\pgfsetbuttcap%
\pgfsetmiterjoin%
\definecolor{currentfill}{rgb}{1.000000,1.000000,1.000000}%
\pgfsetfillcolor{currentfill}%
\pgfsetfillopacity{0.800000}%
\pgfsetlinewidth{1.003750pt}%
\definecolor{currentstroke}{rgb}{0.800000,0.800000,0.800000}%
\pgfsetstrokecolor{currentstroke}%
\pgfsetstrokeopacity{0.800000}%
\pgfsetdash{}{0pt}%
\pgfpathmoveto{\pgfqpoint{0.608494in}{1.545261in}}%
\pgfpathlineto{\pgfqpoint{1.608827in}{1.545261in}}%
\pgfpathquadraticcurveto{\pgfqpoint{1.631049in}{1.545261in}}{\pgfqpoint{1.631049in}{1.567483in}}%
\pgfpathlineto{\pgfqpoint{1.631049in}{1.711261in}}%
\pgfpathquadraticcurveto{\pgfqpoint{1.631049in}{1.733483in}}{\pgfqpoint{1.608827in}{1.733483in}}%
\pgfpathlineto{\pgfqpoint{0.608494in}{1.733483in}}%
\pgfpathquadraticcurveto{\pgfqpoint{0.586272in}{1.733483in}}{\pgfqpoint{0.586272in}{1.711261in}}%
\pgfpathlineto{\pgfqpoint{0.586272in}{1.567483in}}%
\pgfpathquadraticcurveto{\pgfqpoint{0.586272in}{1.545261in}}{\pgfqpoint{0.608494in}{1.545261in}}%
\pgfpathlineto{\pgfqpoint{0.608494in}{1.545261in}}%
\pgfpathclose%
\pgfusepath{stroke,fill}%
\end{pgfscope}%
\begin{pgfscope}%
\pgfsetrectcap%
\pgfsetroundjoin%
\pgfsetlinewidth{1.505625pt}%
\definecolor{currentstroke}{rgb}{0.007843,0.619608,0.450980}%
\pgfsetstrokecolor{currentstroke}%
\pgfsetdash{}{0pt}%
\pgfpathmoveto{\pgfqpoint{0.630716in}{1.650150in}}%
\pgfpathlineto{\pgfqpoint{0.741827in}{1.650150in}}%
\pgfpathlineto{\pgfqpoint{0.852938in}{1.650150in}}%
\pgfusepath{stroke}%
\end{pgfscope}%
\begin{pgfscope}%
\definecolor{textcolor}{rgb}{0.000000,0.000000,0.000000}%
\pgfsetstrokecolor{textcolor}%
\pgfsetfillcolor{textcolor}%
\pgftext[x=0.941827in,y=1.611261in,left,base]{\color{textcolor}\rmfamily\fontsize{8.000000}{9.600000}\selectfont Flicker noise}%
\end{pgfscope}%
\end{pgfpicture}%
\makeatother%
\endgroup%
% data/simulations/sim_allan_variance_example.py
        } % scalebox
        \caption{Flicker noise}
    \end{subfigure}
    \begin{subfigure}{0.32\linewidth}
        \centering
        \scalebox{0.75}{%
            %% Creator: Matplotlib, PGF backend
%%
%% To include the figure in your LaTeX document, write
%%   \input{<filename>.pgf}
%%
%% Make sure the required packages are loaded in your preamble
%%   \usepackage{pgf}
%%
%% Also ensure that all the required font packages are loaded; for instance,
%% the lmodern package is sometimes necessary when using math font.
%%   \usepackage{lmodern}
%%
%% Figures using additional raster images can only be included by \input if
%% they are in the same directory as the main LaTeX file. For loading figures
%% from other directories you can use the `import` package
%%   \usepackage{import}
%%
%% and then include the figures with
%%   \import{<path to file>}{<filename>.pgf}
%%
%% Matplotlib used the following preamble
%%   \def\mathdefault#1{#1}
%%   \everymath=\expandafter{\the\everymath\displaystyle}
%%   \usepackage{siunitx}
%%   \sisetup{per-mode = symbol}%
%%   \ifdefined\pdftexversion\else  % non-pdftex case.
%%     \usepackage{fontspec}
%%   \fi
%%   \makeatletter\@ifpackageloaded{underscore}{}{\usepackage[strings]{underscore}}\makeatother
%%
\begingroup%
\makeatletter%
\begin{pgfpicture}%
\pgfpathrectangle{\pgfpointorigin}{\pgfqpoint{2.440945in}{1.830709in}}%
\pgfusepath{use as bounding box, clip}%
\begin{pgfscope}%
\pgfsetbuttcap%
\pgfsetmiterjoin%
\definecolor{currentfill}{rgb}{1.000000,1.000000,1.000000}%
\pgfsetfillcolor{currentfill}%
\pgfsetlinewidth{0.000000pt}%
\definecolor{currentstroke}{rgb}{1.000000,1.000000,1.000000}%
\pgfsetstrokecolor{currentstroke}%
\pgfsetdash{}{0pt}%
\pgfpathmoveto{\pgfqpoint{0.000000in}{0.000000in}}%
\pgfpathlineto{\pgfqpoint{2.440945in}{0.000000in}}%
\pgfpathlineto{\pgfqpoint{2.440945in}{1.830709in}}%
\pgfpathlineto{\pgfqpoint{0.000000in}{1.830709in}}%
\pgfpathlineto{\pgfqpoint{0.000000in}{0.000000in}}%
\pgfpathclose%
\pgfusepath{fill}%
\end{pgfscope}%
\begin{pgfscope}%
\pgfsetbuttcap%
\pgfsetmiterjoin%
\definecolor{currentfill}{rgb}{1.000000,1.000000,1.000000}%
\pgfsetfillcolor{currentfill}%
\pgfsetlinewidth{0.000000pt}%
\definecolor{currentstroke}{rgb}{0.000000,0.000000,0.000000}%
\pgfsetstrokecolor{currentstroke}%
\pgfsetstrokeopacity{0.000000}%
\pgfsetdash{}{0pt}%
\pgfpathmoveto{\pgfqpoint{0.530716in}{0.416447in}}%
\pgfpathlineto{\pgfqpoint{2.399275in}{0.416447in}}%
\pgfpathlineto{\pgfqpoint{2.399275in}{1.789039in}}%
\pgfpathlineto{\pgfqpoint{0.530716in}{1.789039in}}%
\pgfpathlineto{\pgfqpoint{0.530716in}{0.416447in}}%
\pgfpathclose%
\pgfusepath{fill}%
\end{pgfscope}%
\begin{pgfscope}%
\pgfpathrectangle{\pgfqpoint{0.530716in}{0.416447in}}{\pgfqpoint{1.868559in}{1.372591in}}%
\pgfusepath{clip}%
\pgfsetrectcap%
\pgfsetroundjoin%
\pgfsetlinewidth{0.803000pt}%
\definecolor{currentstroke}{rgb}{0.450000,0.450000,0.450000}%
\pgfsetstrokecolor{currentstroke}%
\pgfsetdash{}{0pt}%
\pgfpathmoveto{\pgfqpoint{0.615651in}{0.416447in}}%
\pgfpathlineto{\pgfqpoint{0.615651in}{1.789039in}}%
\pgfusepath{stroke}%
\end{pgfscope}%
\begin{pgfscope}%
\pgfsetbuttcap%
\pgfsetroundjoin%
\definecolor{currentfill}{rgb}{0.000000,0.000000,0.000000}%
\pgfsetfillcolor{currentfill}%
\pgfsetlinewidth{0.803000pt}%
\definecolor{currentstroke}{rgb}{0.000000,0.000000,0.000000}%
\pgfsetstrokecolor{currentstroke}%
\pgfsetdash{}{0pt}%
\pgfsys@defobject{currentmarker}{\pgfqpoint{0.000000in}{-0.048611in}}{\pgfqpoint{0.000000in}{0.000000in}}{%
\pgfpathmoveto{\pgfqpoint{0.000000in}{0.000000in}}%
\pgfpathlineto{\pgfqpoint{0.000000in}{-0.048611in}}%
\pgfusepath{stroke,fill}%
}%
\begin{pgfscope}%
\pgfsys@transformshift{0.615651in}{0.416447in}%
\pgfsys@useobject{currentmarker}{}%
\end{pgfscope}%
\end{pgfscope}%
\begin{pgfscope}%
\definecolor{textcolor}{rgb}{0.000000,0.000000,0.000000}%
\pgfsetstrokecolor{textcolor}%
\pgfsetfillcolor{textcolor}%
\pgftext[x=0.615651in,y=0.319225in,,top]{\color{textcolor}{\rmfamily\fontsize{8.000000}{9.600000}\selectfont\catcode`\^=\active\def^{\ifmmode\sp\else\^{}\fi}\catcode`\%=\active\def%{\%}$\mathdefault{0}$}}%
\end{pgfscope}%
\begin{pgfscope}%
\pgfpathrectangle{\pgfqpoint{0.530716in}{0.416447in}}{\pgfqpoint{1.868559in}{1.372591in}}%
\pgfusepath{clip}%
\pgfsetrectcap%
\pgfsetroundjoin%
\pgfsetlinewidth{0.803000pt}%
\definecolor{currentstroke}{rgb}{0.450000,0.450000,0.450000}%
\pgfsetstrokecolor{currentstroke}%
\pgfsetdash{}{0pt}%
\pgfpathmoveto{\pgfqpoint{1.121900in}{0.416447in}}%
\pgfpathlineto{\pgfqpoint{1.121900in}{1.789039in}}%
\pgfusepath{stroke}%
\end{pgfscope}%
\begin{pgfscope}%
\pgfsetbuttcap%
\pgfsetroundjoin%
\definecolor{currentfill}{rgb}{0.000000,0.000000,0.000000}%
\pgfsetfillcolor{currentfill}%
\pgfsetlinewidth{0.803000pt}%
\definecolor{currentstroke}{rgb}{0.000000,0.000000,0.000000}%
\pgfsetstrokecolor{currentstroke}%
\pgfsetdash{}{0pt}%
\pgfsys@defobject{currentmarker}{\pgfqpoint{0.000000in}{-0.048611in}}{\pgfqpoint{0.000000in}{0.000000in}}{%
\pgfpathmoveto{\pgfqpoint{0.000000in}{0.000000in}}%
\pgfpathlineto{\pgfqpoint{0.000000in}{-0.048611in}}%
\pgfusepath{stroke,fill}%
}%
\begin{pgfscope}%
\pgfsys@transformshift{1.121900in}{0.416447in}%
\pgfsys@useobject{currentmarker}{}%
\end{pgfscope}%
\end{pgfscope}%
\begin{pgfscope}%
\definecolor{textcolor}{rgb}{0.000000,0.000000,0.000000}%
\pgfsetstrokecolor{textcolor}%
\pgfsetfillcolor{textcolor}%
\pgftext[x=1.121900in,y=0.319225in,,top]{\color{textcolor}{\rmfamily\fontsize{8.000000}{9.600000}\selectfont\catcode`\^=\active\def^{\ifmmode\sp\else\^{}\fi}\catcode`\%=\active\def%{\%}$\mathdefault{10}$}}%
\end{pgfscope}%
\begin{pgfscope}%
\pgfpathrectangle{\pgfqpoint{0.530716in}{0.416447in}}{\pgfqpoint{1.868559in}{1.372591in}}%
\pgfusepath{clip}%
\pgfsetrectcap%
\pgfsetroundjoin%
\pgfsetlinewidth{0.803000pt}%
\definecolor{currentstroke}{rgb}{0.450000,0.450000,0.450000}%
\pgfsetstrokecolor{currentstroke}%
\pgfsetdash{}{0pt}%
\pgfpathmoveto{\pgfqpoint{1.628149in}{0.416447in}}%
\pgfpathlineto{\pgfqpoint{1.628149in}{1.789039in}}%
\pgfusepath{stroke}%
\end{pgfscope}%
\begin{pgfscope}%
\pgfsetbuttcap%
\pgfsetroundjoin%
\definecolor{currentfill}{rgb}{0.000000,0.000000,0.000000}%
\pgfsetfillcolor{currentfill}%
\pgfsetlinewidth{0.803000pt}%
\definecolor{currentstroke}{rgb}{0.000000,0.000000,0.000000}%
\pgfsetstrokecolor{currentstroke}%
\pgfsetdash{}{0pt}%
\pgfsys@defobject{currentmarker}{\pgfqpoint{0.000000in}{-0.048611in}}{\pgfqpoint{0.000000in}{0.000000in}}{%
\pgfpathmoveto{\pgfqpoint{0.000000in}{0.000000in}}%
\pgfpathlineto{\pgfqpoint{0.000000in}{-0.048611in}}%
\pgfusepath{stroke,fill}%
}%
\begin{pgfscope}%
\pgfsys@transformshift{1.628149in}{0.416447in}%
\pgfsys@useobject{currentmarker}{}%
\end{pgfscope}%
\end{pgfscope}%
\begin{pgfscope}%
\definecolor{textcolor}{rgb}{0.000000,0.000000,0.000000}%
\pgfsetstrokecolor{textcolor}%
\pgfsetfillcolor{textcolor}%
\pgftext[x=1.628149in,y=0.319225in,,top]{\color{textcolor}{\rmfamily\fontsize{8.000000}{9.600000}\selectfont\catcode`\^=\active\def^{\ifmmode\sp\else\^{}\fi}\catcode`\%=\active\def%{\%}$\mathdefault{20}$}}%
\end{pgfscope}%
\begin{pgfscope}%
\pgfpathrectangle{\pgfqpoint{0.530716in}{0.416447in}}{\pgfqpoint{1.868559in}{1.372591in}}%
\pgfusepath{clip}%
\pgfsetrectcap%
\pgfsetroundjoin%
\pgfsetlinewidth{0.803000pt}%
\definecolor{currentstroke}{rgb}{0.450000,0.450000,0.450000}%
\pgfsetstrokecolor{currentstroke}%
\pgfsetdash{}{0pt}%
\pgfpathmoveto{\pgfqpoint{2.134398in}{0.416447in}}%
\pgfpathlineto{\pgfqpoint{2.134398in}{1.789039in}}%
\pgfusepath{stroke}%
\end{pgfscope}%
\begin{pgfscope}%
\pgfsetbuttcap%
\pgfsetroundjoin%
\definecolor{currentfill}{rgb}{0.000000,0.000000,0.000000}%
\pgfsetfillcolor{currentfill}%
\pgfsetlinewidth{0.803000pt}%
\definecolor{currentstroke}{rgb}{0.000000,0.000000,0.000000}%
\pgfsetstrokecolor{currentstroke}%
\pgfsetdash{}{0pt}%
\pgfsys@defobject{currentmarker}{\pgfqpoint{0.000000in}{-0.048611in}}{\pgfqpoint{0.000000in}{0.000000in}}{%
\pgfpathmoveto{\pgfqpoint{0.000000in}{0.000000in}}%
\pgfpathlineto{\pgfqpoint{0.000000in}{-0.048611in}}%
\pgfusepath{stroke,fill}%
}%
\begin{pgfscope}%
\pgfsys@transformshift{2.134398in}{0.416447in}%
\pgfsys@useobject{currentmarker}{}%
\end{pgfscope}%
\end{pgfscope}%
\begin{pgfscope}%
\definecolor{textcolor}{rgb}{0.000000,0.000000,0.000000}%
\pgfsetstrokecolor{textcolor}%
\pgfsetfillcolor{textcolor}%
\pgftext[x=2.134398in,y=0.319225in,,top]{\color{textcolor}{\rmfamily\fontsize{8.000000}{9.600000}\selectfont\catcode`\^=\active\def^{\ifmmode\sp\else\^{}\fi}\catcode`\%=\active\def%{\%}$\mathdefault{30}$}}%
\end{pgfscope}%
\begin{pgfscope}%
\definecolor{textcolor}{rgb}{0.000000,0.000000,0.000000}%
\pgfsetstrokecolor{textcolor}%
\pgfsetfillcolor{textcolor}%
\pgftext[x=1.464996in,y=0.165003in,,top]{\color{textcolor}{\rmfamily\fontsize{10.000000}{12.000000}\selectfont\catcode`\^=\active\def^{\ifmmode\sp\else\^{}\fi}\catcode`\%=\active\def%{\%}Time in $\unit{\second}$}}%
\end{pgfscope}%
\begin{pgfscope}%
\pgfpathrectangle{\pgfqpoint{0.530716in}{0.416447in}}{\pgfqpoint{1.868559in}{1.372591in}}%
\pgfusepath{clip}%
\pgfsetrectcap%
\pgfsetroundjoin%
\pgfsetlinewidth{0.803000pt}%
\definecolor{currentstroke}{rgb}{0.450000,0.450000,0.450000}%
\pgfsetstrokecolor{currentstroke}%
\pgfsetdash{}{0pt}%
\pgfpathmoveto{\pgfqpoint{0.530716in}{0.416448in}}%
\pgfpathlineto{\pgfqpoint{2.399275in}{0.416448in}}%
\pgfusepath{stroke}%
\end{pgfscope}%
\begin{pgfscope}%
\pgfsetbuttcap%
\pgfsetroundjoin%
\definecolor{currentfill}{rgb}{0.000000,0.000000,0.000000}%
\pgfsetfillcolor{currentfill}%
\pgfsetlinewidth{0.803000pt}%
\definecolor{currentstroke}{rgb}{0.000000,0.000000,0.000000}%
\pgfsetstrokecolor{currentstroke}%
\pgfsetdash{}{0pt}%
\pgfsys@defobject{currentmarker}{\pgfqpoint{-0.048611in}{0.000000in}}{\pgfqpoint{-0.000000in}{0.000000in}}{%
\pgfpathmoveto{\pgfqpoint{-0.000000in}{0.000000in}}%
\pgfpathlineto{\pgfqpoint{-0.048611in}{0.000000in}}%
\pgfusepath{stroke,fill}%
}%
\begin{pgfscope}%
\pgfsys@transformshift{0.530716in}{0.416448in}%
\pgfsys@useobject{currentmarker}{}%
\end{pgfscope}%
\end{pgfscope}%
\begin{pgfscope}%
\definecolor{textcolor}{rgb}{0.000000,0.000000,0.000000}%
\pgfsetstrokecolor{textcolor}%
\pgfsetfillcolor{textcolor}%
\pgftext[x=0.223614in, y=0.377892in, left, base]{\color{textcolor}{\rmfamily\fontsize{8.000000}{9.600000}\selectfont\catcode`\^=\active\def^{\ifmmode\sp\else\^{}\fi}\catcode`\%=\active\def%{\%}$\mathdefault{\ensuremath{-}50}$}}%
\end{pgfscope}%
\begin{pgfscope}%
\pgfpathrectangle{\pgfqpoint{0.530716in}{0.416447in}}{\pgfqpoint{1.868559in}{1.372591in}}%
\pgfusepath{clip}%
\pgfsetrectcap%
\pgfsetroundjoin%
\pgfsetlinewidth{0.803000pt}%
\definecolor{currentstroke}{rgb}{0.450000,0.450000,0.450000}%
\pgfsetstrokecolor{currentstroke}%
\pgfsetdash{}{0pt}%
\pgfpathmoveto{\pgfqpoint{0.530716in}{0.714837in}}%
\pgfpathlineto{\pgfqpoint{2.399275in}{0.714837in}}%
\pgfusepath{stroke}%
\end{pgfscope}%
\begin{pgfscope}%
\pgfsetbuttcap%
\pgfsetroundjoin%
\definecolor{currentfill}{rgb}{0.000000,0.000000,0.000000}%
\pgfsetfillcolor{currentfill}%
\pgfsetlinewidth{0.803000pt}%
\definecolor{currentstroke}{rgb}{0.000000,0.000000,0.000000}%
\pgfsetstrokecolor{currentstroke}%
\pgfsetdash{}{0pt}%
\pgfsys@defobject{currentmarker}{\pgfqpoint{-0.048611in}{0.000000in}}{\pgfqpoint{-0.000000in}{0.000000in}}{%
\pgfpathmoveto{\pgfqpoint{-0.000000in}{0.000000in}}%
\pgfpathlineto{\pgfqpoint{-0.048611in}{0.000000in}}%
\pgfusepath{stroke,fill}%
}%
\begin{pgfscope}%
\pgfsys@transformshift{0.530716in}{0.714837in}%
\pgfsys@useobject{currentmarker}{}%
\end{pgfscope}%
\end{pgfscope}%
\begin{pgfscope}%
\definecolor{textcolor}{rgb}{0.000000,0.000000,0.000000}%
\pgfsetstrokecolor{textcolor}%
\pgfsetfillcolor{textcolor}%
\pgftext[x=0.374465in, y=0.676281in, left, base]{\color{textcolor}{\rmfamily\fontsize{8.000000}{9.600000}\selectfont\catcode`\^=\active\def^{\ifmmode\sp\else\^{}\fi}\catcode`\%=\active\def%{\%}$\mathdefault{0}$}}%
\end{pgfscope}%
\begin{pgfscope}%
\pgfpathrectangle{\pgfqpoint{0.530716in}{0.416447in}}{\pgfqpoint{1.868559in}{1.372591in}}%
\pgfusepath{clip}%
\pgfsetrectcap%
\pgfsetroundjoin%
\pgfsetlinewidth{0.803000pt}%
\definecolor{currentstroke}{rgb}{0.450000,0.450000,0.450000}%
\pgfsetstrokecolor{currentstroke}%
\pgfsetdash{}{0pt}%
\pgfpathmoveto{\pgfqpoint{0.530716in}{1.013226in}}%
\pgfpathlineto{\pgfqpoint{2.399275in}{1.013226in}}%
\pgfusepath{stroke}%
\end{pgfscope}%
\begin{pgfscope}%
\pgfsetbuttcap%
\pgfsetroundjoin%
\definecolor{currentfill}{rgb}{0.000000,0.000000,0.000000}%
\pgfsetfillcolor{currentfill}%
\pgfsetlinewidth{0.803000pt}%
\definecolor{currentstroke}{rgb}{0.000000,0.000000,0.000000}%
\pgfsetstrokecolor{currentstroke}%
\pgfsetdash{}{0pt}%
\pgfsys@defobject{currentmarker}{\pgfqpoint{-0.048611in}{0.000000in}}{\pgfqpoint{-0.000000in}{0.000000in}}{%
\pgfpathmoveto{\pgfqpoint{-0.000000in}{0.000000in}}%
\pgfpathlineto{\pgfqpoint{-0.048611in}{0.000000in}}%
\pgfusepath{stroke,fill}%
}%
\begin{pgfscope}%
\pgfsys@transformshift{0.530716in}{1.013226in}%
\pgfsys@useobject{currentmarker}{}%
\end{pgfscope}%
\end{pgfscope}%
\begin{pgfscope}%
\definecolor{textcolor}{rgb}{0.000000,0.000000,0.000000}%
\pgfsetstrokecolor{textcolor}%
\pgfsetfillcolor{textcolor}%
\pgftext[x=0.315437in, y=0.974671in, left, base]{\color{textcolor}{\rmfamily\fontsize{8.000000}{9.600000}\selectfont\catcode`\^=\active\def^{\ifmmode\sp\else\^{}\fi}\catcode`\%=\active\def%{\%}$\mathdefault{50}$}}%
\end{pgfscope}%
\begin{pgfscope}%
\pgfpathrectangle{\pgfqpoint{0.530716in}{0.416447in}}{\pgfqpoint{1.868559in}{1.372591in}}%
\pgfusepath{clip}%
\pgfsetrectcap%
\pgfsetroundjoin%
\pgfsetlinewidth{0.803000pt}%
\definecolor{currentstroke}{rgb}{0.450000,0.450000,0.450000}%
\pgfsetstrokecolor{currentstroke}%
\pgfsetdash{}{0pt}%
\pgfpathmoveto{\pgfqpoint{0.530716in}{1.311616in}}%
\pgfpathlineto{\pgfqpoint{2.399275in}{1.311616in}}%
\pgfusepath{stroke}%
\end{pgfscope}%
\begin{pgfscope}%
\pgfsetbuttcap%
\pgfsetroundjoin%
\definecolor{currentfill}{rgb}{0.000000,0.000000,0.000000}%
\pgfsetfillcolor{currentfill}%
\pgfsetlinewidth{0.803000pt}%
\definecolor{currentstroke}{rgb}{0.000000,0.000000,0.000000}%
\pgfsetstrokecolor{currentstroke}%
\pgfsetdash{}{0pt}%
\pgfsys@defobject{currentmarker}{\pgfqpoint{-0.048611in}{0.000000in}}{\pgfqpoint{-0.000000in}{0.000000in}}{%
\pgfpathmoveto{\pgfqpoint{-0.000000in}{0.000000in}}%
\pgfpathlineto{\pgfqpoint{-0.048611in}{0.000000in}}%
\pgfusepath{stroke,fill}%
}%
\begin{pgfscope}%
\pgfsys@transformshift{0.530716in}{1.311616in}%
\pgfsys@useobject{currentmarker}{}%
\end{pgfscope}%
\end{pgfscope}%
\begin{pgfscope}%
\definecolor{textcolor}{rgb}{0.000000,0.000000,0.000000}%
\pgfsetstrokecolor{textcolor}%
\pgfsetfillcolor{textcolor}%
\pgftext[x=0.256408in, y=1.273060in, left, base]{\color{textcolor}{\rmfamily\fontsize{8.000000}{9.600000}\selectfont\catcode`\^=\active\def^{\ifmmode\sp\else\^{}\fi}\catcode`\%=\active\def%{\%}$\mathdefault{100}$}}%
\end{pgfscope}%
\begin{pgfscope}%
\pgfpathrectangle{\pgfqpoint{0.530716in}{0.416447in}}{\pgfqpoint{1.868559in}{1.372591in}}%
\pgfusepath{clip}%
\pgfsetrectcap%
\pgfsetroundjoin%
\pgfsetlinewidth{0.803000pt}%
\definecolor{currentstroke}{rgb}{0.450000,0.450000,0.450000}%
\pgfsetstrokecolor{currentstroke}%
\pgfsetdash{}{0pt}%
\pgfpathmoveto{\pgfqpoint{0.530716in}{1.610005in}}%
\pgfpathlineto{\pgfqpoint{2.399275in}{1.610005in}}%
\pgfusepath{stroke}%
\end{pgfscope}%
\begin{pgfscope}%
\pgfsetbuttcap%
\pgfsetroundjoin%
\definecolor{currentfill}{rgb}{0.000000,0.000000,0.000000}%
\pgfsetfillcolor{currentfill}%
\pgfsetlinewidth{0.803000pt}%
\definecolor{currentstroke}{rgb}{0.000000,0.000000,0.000000}%
\pgfsetstrokecolor{currentstroke}%
\pgfsetdash{}{0pt}%
\pgfsys@defobject{currentmarker}{\pgfqpoint{-0.048611in}{0.000000in}}{\pgfqpoint{-0.000000in}{0.000000in}}{%
\pgfpathmoveto{\pgfqpoint{-0.000000in}{0.000000in}}%
\pgfpathlineto{\pgfqpoint{-0.048611in}{0.000000in}}%
\pgfusepath{stroke,fill}%
}%
\begin{pgfscope}%
\pgfsys@transformshift{0.530716in}{1.610005in}%
\pgfsys@useobject{currentmarker}{}%
\end{pgfscope}%
\end{pgfscope}%
\begin{pgfscope}%
\definecolor{textcolor}{rgb}{0.000000,0.000000,0.000000}%
\pgfsetstrokecolor{textcolor}%
\pgfsetfillcolor{textcolor}%
\pgftext[x=0.256408in, y=1.571450in, left, base]{\color{textcolor}{\rmfamily\fontsize{8.000000}{9.600000}\selectfont\catcode`\^=\active\def^{\ifmmode\sp\else\^{}\fi}\catcode`\%=\active\def%{\%}$\mathdefault{150}$}}%
\end{pgfscope}%
\begin{pgfscope}%
\definecolor{textcolor}{rgb}{0.000000,0.000000,0.000000}%
\pgfsetstrokecolor{textcolor}%
\pgfsetfillcolor{textcolor}%
\pgftext[x=0.168059in,y=1.102743in,,bottom,rotate=90.000000]{\color{textcolor}{\rmfamily\fontsize{10.000000}{12.000000}\selectfont\catcode`\^=\active\def^{\ifmmode\sp\else\^{}\fi}\catcode`\%=\active\def%{\%}Ampl. in arb. unit}}%
\end{pgfscope}%
\begin{pgfscope}%
\pgfpathrectangle{\pgfqpoint{0.530716in}{0.416447in}}{\pgfqpoint{1.868559in}{1.372591in}}%
\pgfusepath{clip}%
\pgfsetrectcap%
\pgfsetroundjoin%
\pgfsetlinewidth{1.505625pt}%
\definecolor{currentstroke}{rgb}{0.835294,0.368627,0.000000}%
\pgfsetstrokecolor{currentstroke}%
\pgfsetdash{}{0pt}%
\pgfpathmoveto{\pgfqpoint{0.615651in}{0.714768in}}%
\pgfpathlineto{\pgfqpoint{0.616760in}{0.759646in}}%
\pgfpathlineto{\pgfqpoint{0.617637in}{0.722366in}}%
\pgfpathlineto{\pgfqpoint{0.619375in}{0.738805in}}%
\pgfpathlineto{\pgfqpoint{0.621286in}{0.670021in}}%
\pgfpathlineto{\pgfqpoint{0.622772in}{0.684024in}}%
\pgfpathlineto{\pgfqpoint{0.625137in}{0.665872in}}%
\pgfpathlineto{\pgfqpoint{0.625945in}{0.701541in}}%
\pgfpathlineto{\pgfqpoint{0.627660in}{0.666111in}}%
\pgfpathlineto{\pgfqpoint{0.630083in}{0.670111in}}%
\pgfpathlineto{\pgfqpoint{0.631444in}{0.712729in}}%
\pgfpathlineto{\pgfqpoint{0.634298in}{0.732817in}}%
\pgfpathlineto{\pgfqpoint{0.635545in}{0.703507in}}%
\pgfpathlineto{\pgfqpoint{0.636601in}{0.719256in}}%
\pgfpathlineto{\pgfqpoint{0.638617in}{0.695973in}}%
\pgfpathlineto{\pgfqpoint{0.639882in}{0.721209in}}%
\pgfpathlineto{\pgfqpoint{0.642583in}{0.693544in}}%
\pgfpathlineto{\pgfqpoint{0.643204in}{0.723081in}}%
\pgfpathlineto{\pgfqpoint{0.645561in}{0.749616in}}%
\pgfpathlineto{\pgfqpoint{0.646860in}{0.724070in}}%
\pgfpathlineto{\pgfqpoint{0.648018in}{0.757032in}}%
\pgfpathlineto{\pgfqpoint{0.649891in}{0.757006in}}%
\pgfpathlineto{\pgfqpoint{0.652173in}{0.794665in}}%
\pgfpathlineto{\pgfqpoint{0.654482in}{0.761753in}}%
\pgfpathlineto{\pgfqpoint{0.655876in}{0.830587in}}%
\pgfpathlineto{\pgfqpoint{0.657303in}{0.800804in}}%
\pgfpathlineto{\pgfqpoint{0.659813in}{0.784956in}}%
\pgfpathlineto{\pgfqpoint{0.660444in}{0.814907in}}%
\pgfpathlineto{\pgfqpoint{0.662459in}{0.810995in}}%
\pgfpathlineto{\pgfqpoint{0.664305in}{0.873863in}}%
\pgfpathlineto{\pgfqpoint{0.665995in}{0.810302in}}%
\pgfpathlineto{\pgfqpoint{0.668139in}{0.836233in}}%
\pgfpathlineto{\pgfqpoint{0.671145in}{0.757937in}}%
\pgfpathlineto{\pgfqpoint{0.672058in}{0.786107in}}%
\pgfpathlineto{\pgfqpoint{0.673532in}{0.760356in}}%
\pgfpathlineto{\pgfqpoint{0.676799in}{0.750668in}}%
\pgfpathlineto{\pgfqpoint{0.677970in}{0.797881in}}%
\pgfpathlineto{\pgfqpoint{0.680174in}{0.809832in}}%
\pgfpathlineto{\pgfqpoint{0.682190in}{0.754316in}}%
\pgfpathlineto{\pgfqpoint{0.685094in}{0.751530in}}%
\pgfpathlineto{\pgfqpoint{0.685504in}{0.715003in}}%
\pgfpathlineto{\pgfqpoint{0.687611in}{0.707380in}}%
\pgfpathlineto{\pgfqpoint{0.688959in}{0.730060in}}%
\pgfpathlineto{\pgfqpoint{0.692199in}{0.711210in}}%
\pgfpathlineto{\pgfqpoint{0.692946in}{0.678432in}}%
\pgfpathlineto{\pgfqpoint{0.694978in}{0.691527in}}%
\pgfpathlineto{\pgfqpoint{0.697014in}{0.683958in}}%
\pgfpathlineto{\pgfqpoint{0.698380in}{0.641612in}}%
\pgfpathlineto{\pgfqpoint{0.699054in}{0.650310in}}%
\pgfpathlineto{\pgfqpoint{0.702291in}{0.580804in}}%
\pgfpathlineto{\pgfqpoint{0.702606in}{0.606973in}}%
\pgfpathlineto{\pgfqpoint{0.705237in}{0.582407in}}%
\pgfpathlineto{\pgfqpoint{0.706294in}{0.622672in}}%
\pgfpathlineto{\pgfqpoint{0.708239in}{0.580633in}}%
\pgfpathlineto{\pgfqpoint{0.709316in}{0.624555in}}%
\pgfpathlineto{\pgfqpoint{0.713277in}{0.556473in}}%
\pgfpathlineto{\pgfqpoint{0.714466in}{0.585825in}}%
\pgfpathlineto{\pgfqpoint{0.717672in}{0.617458in}}%
\pgfpathlineto{\pgfqpoint{0.718615in}{0.595732in}}%
\pgfpathlineto{\pgfqpoint{0.720504in}{0.607228in}}%
\pgfpathlineto{\pgfqpoint{0.721976in}{0.664894in}}%
\pgfpathlineto{\pgfqpoint{0.723755in}{0.626429in}}%
\pgfpathlineto{\pgfqpoint{0.724720in}{0.657175in}}%
\pgfpathlineto{\pgfqpoint{0.726338in}{0.643517in}}%
\pgfpathlineto{\pgfqpoint{0.728443in}{0.704427in}}%
\pgfpathlineto{\pgfqpoint{0.729975in}{0.670129in}}%
\pgfpathlineto{\pgfqpoint{0.733619in}{0.736035in}}%
\pgfpathlineto{\pgfqpoint{0.735228in}{0.687219in}}%
\pgfpathlineto{\pgfqpoint{0.736531in}{0.737355in}}%
\pgfpathlineto{\pgfqpoint{0.738905in}{0.757439in}}%
\pgfpathlineto{\pgfqpoint{0.741516in}{0.756243in}}%
\pgfpathlineto{\pgfqpoint{0.743123in}{0.701454in}}%
\pgfpathlineto{\pgfqpoint{0.744236in}{0.711579in}}%
\pgfpathlineto{\pgfqpoint{0.745211in}{0.686920in}}%
\pgfpathlineto{\pgfqpoint{0.747006in}{0.681972in}}%
\pgfpathlineto{\pgfqpoint{0.749595in}{0.711839in}}%
\pgfpathlineto{\pgfqpoint{0.750230in}{0.686017in}}%
\pgfpathlineto{\pgfqpoint{0.752807in}{0.727709in}}%
\pgfpathlineto{\pgfqpoint{0.754031in}{0.714510in}}%
\pgfpathlineto{\pgfqpoint{0.755904in}{0.749429in}}%
\pgfpathlineto{\pgfqpoint{0.756979in}{0.724529in}}%
\pgfpathlineto{\pgfqpoint{0.760113in}{0.769085in}}%
\pgfpathlineto{\pgfqpoint{0.760830in}{0.734850in}}%
\pgfpathlineto{\pgfqpoint{0.762200in}{0.760837in}}%
\pgfpathlineto{\pgfqpoint{0.765198in}{0.730254in}}%
\pgfpathlineto{\pgfqpoint{0.766253in}{0.777271in}}%
\pgfpathlineto{\pgfqpoint{0.767682in}{0.755229in}}%
\pgfpathlineto{\pgfqpoint{0.768844in}{0.787104in}}%
\pgfpathlineto{\pgfqpoint{0.771117in}{0.731141in}}%
\pgfpathlineto{\pgfqpoint{0.773806in}{0.707053in}}%
\pgfpathlineto{\pgfqpoint{0.774517in}{0.745435in}}%
\pgfpathlineto{\pgfqpoint{0.776935in}{0.712170in}}%
\pgfpathlineto{\pgfqpoint{0.778785in}{0.769058in}}%
\pgfpathlineto{\pgfqpoint{0.780227in}{0.769447in}}%
\pgfpathlineto{\pgfqpoint{0.781695in}{0.819497in}}%
\pgfpathlineto{\pgfqpoint{0.783047in}{0.800564in}}%
\pgfpathlineto{\pgfqpoint{0.784462in}{0.823768in}}%
\pgfpathlineto{\pgfqpoint{0.786245in}{0.795248in}}%
\pgfpathlineto{\pgfqpoint{0.789214in}{0.795199in}}%
\pgfpathlineto{\pgfqpoint{0.790178in}{0.826523in}}%
\pgfpathlineto{\pgfqpoint{0.791526in}{0.803932in}}%
\pgfpathlineto{\pgfqpoint{0.793450in}{0.848201in}}%
\pgfpathlineto{\pgfqpoint{0.794386in}{0.830774in}}%
\pgfpathlineto{\pgfqpoint{0.797350in}{0.835627in}}%
\pgfpathlineto{\pgfqpoint{0.798290in}{0.868283in}}%
\pgfpathlineto{\pgfqpoint{0.800002in}{0.859451in}}%
\pgfpathlineto{\pgfqpoint{0.802401in}{0.929234in}}%
\pgfpathlineto{\pgfqpoint{0.804512in}{0.886093in}}%
\pgfpathlineto{\pgfqpoint{0.806043in}{0.938940in}}%
\pgfpathlineto{\pgfqpoint{0.806368in}{0.918164in}}%
\pgfpathlineto{\pgfqpoint{0.808772in}{0.879933in}}%
\pgfpathlineto{\pgfqpoint{0.810147in}{0.914228in}}%
\pgfpathlineto{\pgfqpoint{0.812363in}{0.893561in}}%
\pgfpathlineto{\pgfqpoint{0.813148in}{0.918549in}}%
\pgfpathlineto{\pgfqpoint{0.815524in}{0.942989in}}%
\pgfpathlineto{\pgfqpoint{0.817052in}{0.916522in}}%
\pgfpathlineto{\pgfqpoint{0.818458in}{0.956583in}}%
\pgfpathlineto{\pgfqpoint{0.819904in}{0.909834in}}%
\pgfpathlineto{\pgfqpoint{0.822585in}{0.932810in}}%
\pgfpathlineto{\pgfqpoint{0.823864in}{0.900535in}}%
\pgfpathlineto{\pgfqpoint{0.825717in}{0.943501in}}%
\pgfpathlineto{\pgfqpoint{0.827312in}{0.898991in}}%
\pgfpathlineto{\pgfqpoint{0.828720in}{0.926536in}}%
\pgfpathlineto{\pgfqpoint{0.830511in}{0.919002in}}%
\pgfpathlineto{\pgfqpoint{0.833087in}{0.944058in}}%
\pgfpathlineto{\pgfqpoint{0.835222in}{0.887645in}}%
\pgfpathlineto{\pgfqpoint{0.835860in}{0.901697in}}%
\pgfpathlineto{\pgfqpoint{0.836970in}{0.873511in}}%
\pgfpathlineto{\pgfqpoint{0.840974in}{0.916766in}}%
\pgfpathlineto{\pgfqpoint{0.843625in}{0.850018in}}%
\pgfpathlineto{\pgfqpoint{0.844450in}{0.862951in}}%
\pgfpathlineto{\pgfqpoint{0.846118in}{0.808398in}}%
\pgfpathlineto{\pgfqpoint{0.847341in}{0.808368in}}%
\pgfpathlineto{\pgfqpoint{0.848929in}{0.846770in}}%
\pgfpathlineto{\pgfqpoint{0.851812in}{0.874935in}}%
\pgfpathlineto{\pgfqpoint{0.853807in}{0.848002in}}%
\pgfpathlineto{\pgfqpoint{0.854872in}{0.871894in}}%
\pgfpathlineto{\pgfqpoint{0.856065in}{0.841368in}}%
\pgfpathlineto{\pgfqpoint{0.857675in}{0.867342in}}%
\pgfpathlineto{\pgfqpoint{0.859288in}{0.830686in}}%
\pgfpathlineto{\pgfqpoint{0.862162in}{0.859630in}}%
\pgfpathlineto{\pgfqpoint{0.863210in}{0.826784in}}%
\pgfpathlineto{\pgfqpoint{0.865305in}{0.833943in}}%
\pgfpathlineto{\pgfqpoint{0.867163in}{0.783267in}}%
\pgfpathlineto{\pgfqpoint{0.868674in}{0.788206in}}%
\pgfpathlineto{\pgfqpoint{0.869596in}{0.745071in}}%
\pgfpathlineto{\pgfqpoint{0.871277in}{0.734448in}}%
\pgfpathlineto{\pgfqpoint{0.874135in}{0.774762in}}%
\pgfpathlineto{\pgfqpoint{0.875471in}{0.714077in}}%
\pgfpathlineto{\pgfqpoint{0.877410in}{0.755167in}}%
\pgfpathlineto{\pgfqpoint{0.878115in}{0.712341in}}%
\pgfpathlineto{\pgfqpoint{0.880138in}{0.735567in}}%
\pgfpathlineto{\pgfqpoint{0.882519in}{0.747547in}}%
\pgfpathlineto{\pgfqpoint{0.883168in}{0.711402in}}%
\pgfpathlineto{\pgfqpoint{0.886283in}{0.732692in}}%
\pgfpathlineto{\pgfqpoint{0.889069in}{0.687886in}}%
\pgfpathlineto{\pgfqpoint{0.891199in}{0.729547in}}%
\pgfpathlineto{\pgfqpoint{0.891502in}{0.710445in}}%
\pgfpathlineto{\pgfqpoint{0.893936in}{0.682490in}}%
\pgfpathlineto{\pgfqpoint{0.896000in}{0.693381in}}%
\pgfpathlineto{\pgfqpoint{0.897808in}{0.657787in}}%
\pgfpathlineto{\pgfqpoint{0.900333in}{0.717339in}}%
\pgfpathlineto{\pgfqpoint{0.901611in}{0.673853in}}%
\pgfpathlineto{\pgfqpoint{0.903948in}{0.667876in}}%
\pgfpathlineto{\pgfqpoint{0.905010in}{0.630440in}}%
\pgfpathlineto{\pgfqpoint{0.907435in}{0.594312in}}%
\pgfpathlineto{\pgfqpoint{0.909977in}{0.604732in}}%
\pgfpathlineto{\pgfqpoint{0.910530in}{0.632557in}}%
\pgfpathlineto{\pgfqpoint{0.912365in}{0.653018in}}%
\pgfpathlineto{\pgfqpoint{0.914194in}{0.621593in}}%
\pgfpathlineto{\pgfqpoint{0.916198in}{0.620586in}}%
\pgfpathlineto{\pgfqpoint{0.918302in}{0.655147in}}%
\pgfpathlineto{\pgfqpoint{0.919432in}{0.639108in}}%
\pgfpathlineto{\pgfqpoint{0.921147in}{0.675466in}}%
\pgfpathlineto{\pgfqpoint{0.923026in}{0.689758in}}%
\pgfpathlineto{\pgfqpoint{0.924957in}{0.689632in}}%
\pgfpathlineto{\pgfqpoint{0.925964in}{0.637461in}}%
\pgfpathlineto{\pgfqpoint{0.928098in}{0.672054in}}%
\pgfpathlineto{\pgfqpoint{0.930067in}{0.648908in}}%
\pgfpathlineto{\pgfqpoint{0.930751in}{0.686032in}}%
\pgfpathlineto{\pgfqpoint{0.932347in}{0.651879in}}%
\pgfpathlineto{\pgfqpoint{0.934428in}{0.697617in}}%
\pgfpathlineto{\pgfqpoint{0.937177in}{0.704102in}}%
\pgfpathlineto{\pgfqpoint{0.938401in}{0.669120in}}%
\pgfpathlineto{\pgfqpoint{0.940328in}{0.723774in}}%
\pgfpathlineto{\pgfqpoint{0.940808in}{0.699720in}}%
\pgfpathlineto{\pgfqpoint{0.942622in}{0.720100in}}%
\pgfpathlineto{\pgfqpoint{0.944443in}{0.673813in}}%
\pgfpathlineto{\pgfqpoint{0.946027in}{0.706462in}}%
\pgfpathlineto{\pgfqpoint{0.948939in}{0.703601in}}%
\pgfpathlineto{\pgfqpoint{0.949540in}{0.682230in}}%
\pgfpathlineto{\pgfqpoint{0.951843in}{0.696192in}}%
\pgfpathlineto{\pgfqpoint{0.953171in}{0.693801in}}%
\pgfpathlineto{\pgfqpoint{0.956617in}{0.614819in}}%
\pgfpathlineto{\pgfqpoint{0.958351in}{0.650749in}}%
\pgfpathlineto{\pgfqpoint{0.960475in}{0.654667in}}%
\pgfpathlineto{\pgfqpoint{0.961574in}{0.613024in}}%
\pgfpathlineto{\pgfqpoint{0.963931in}{0.590206in}}%
\pgfpathlineto{\pgfqpoint{0.965853in}{0.615072in}}%
\pgfpathlineto{\pgfqpoint{0.966501in}{0.589948in}}%
\pgfpathlineto{\pgfqpoint{0.969122in}{0.610731in}}%
\pgfpathlineto{\pgfqpoint{0.970302in}{0.573703in}}%
\pgfpathlineto{\pgfqpoint{0.971526in}{0.597567in}}%
\pgfpathlineto{\pgfqpoint{0.973844in}{0.562668in}}%
\pgfpathlineto{\pgfqpoint{0.974849in}{0.587077in}}%
\pgfpathlineto{\pgfqpoint{0.976498in}{0.566316in}}%
\pgfpathlineto{\pgfqpoint{0.978739in}{0.600276in}}%
\pgfpathlineto{\pgfqpoint{0.979905in}{0.566749in}}%
\pgfpathlineto{\pgfqpoint{0.981607in}{0.599269in}}%
\pgfpathlineto{\pgfqpoint{0.984747in}{0.581305in}}%
\pgfpathlineto{\pgfqpoint{0.985164in}{0.603665in}}%
\pgfpathlineto{\pgfqpoint{0.988397in}{0.605619in}}%
\pgfpathlineto{\pgfqpoint{0.989579in}{0.559810in}}%
\pgfpathlineto{\pgfqpoint{0.990858in}{0.587169in}}%
\pgfpathlineto{\pgfqpoint{0.991821in}{0.550516in}}%
\pgfpathlineto{\pgfqpoint{0.994132in}{0.519379in}}%
\pgfpathlineto{\pgfqpoint{0.996219in}{0.517584in}}%
\pgfpathlineto{\pgfqpoint{0.997494in}{0.577822in}}%
\pgfpathlineto{\pgfqpoint{0.999406in}{0.568143in}}%
\pgfpathlineto{\pgfqpoint{1.000888in}{0.597974in}}%
\pgfpathlineto{\pgfqpoint{1.002170in}{0.591148in}}%
\pgfpathlineto{\pgfqpoint{1.003877in}{0.625394in}}%
\pgfpathlineto{\pgfqpoint{1.006954in}{0.585816in}}%
\pgfpathlineto{\pgfqpoint{1.007693in}{0.614211in}}%
\pgfpathlineto{\pgfqpoint{1.009389in}{0.606782in}}%
\pgfpathlineto{\pgfqpoint{1.010604in}{0.662895in}}%
\pgfpathlineto{\pgfqpoint{1.013279in}{0.641891in}}%
\pgfpathlineto{\pgfqpoint{1.014675in}{0.642676in}}%
\pgfpathlineto{\pgfqpoint{1.016024in}{0.723194in}}%
\pgfpathlineto{\pgfqpoint{1.017548in}{0.687416in}}%
\pgfpathlineto{\pgfqpoint{1.020512in}{0.659481in}}%
\pgfpathlineto{\pgfqpoint{1.022401in}{0.684331in}}%
\pgfpathlineto{\pgfqpoint{1.023038in}{0.636829in}}%
\pgfpathlineto{\pgfqpoint{1.024984in}{0.659162in}}%
\pgfpathlineto{\pgfqpoint{1.027760in}{0.593883in}}%
\pgfpathlineto{\pgfqpoint{1.030004in}{0.639873in}}%
\pgfpathlineto{\pgfqpoint{1.031815in}{0.627431in}}%
\pgfpathlineto{\pgfqpoint{1.033475in}{0.632977in}}%
\pgfpathlineto{\pgfqpoint{1.034696in}{0.674641in}}%
\pgfpathlineto{\pgfqpoint{1.036228in}{0.684946in}}%
\pgfpathlineto{\pgfqpoint{1.038018in}{0.634671in}}%
\pgfpathlineto{\pgfqpoint{1.040642in}{0.606813in}}%
\pgfpathlineto{\pgfqpoint{1.042634in}{0.672317in}}%
\pgfpathlineto{\pgfqpoint{1.043484in}{0.657351in}}%
\pgfpathlineto{\pgfqpoint{1.045305in}{0.694346in}}%
\pgfpathlineto{\pgfqpoint{1.047123in}{0.655765in}}%
\pgfpathlineto{\pgfqpoint{1.048226in}{0.684082in}}%
\pgfpathlineto{\pgfqpoint{1.049704in}{0.641332in}}%
\pgfpathlineto{\pgfqpoint{1.051937in}{0.659171in}}%
\pgfpathlineto{\pgfqpoint{1.055026in}{0.755235in}}%
\pgfpathlineto{\pgfqpoint{1.058108in}{0.785157in}}%
\pgfpathlineto{\pgfqpoint{1.058905in}{0.754419in}}%
\pgfpathlineto{\pgfqpoint{1.061280in}{0.710866in}}%
\pgfpathlineto{\pgfqpoint{1.063223in}{0.753896in}}%
\pgfpathlineto{\pgfqpoint{1.064702in}{0.705323in}}%
\pgfpathlineto{\pgfqpoint{1.065224in}{0.726397in}}%
\pgfpathlineto{\pgfqpoint{1.067784in}{0.733669in}}%
\pgfpathlineto{\pgfqpoint{1.069933in}{0.707696in}}%
\pgfpathlineto{\pgfqpoint{1.070524in}{0.736360in}}%
\pgfpathlineto{\pgfqpoint{1.072508in}{0.697302in}}%
\pgfpathlineto{\pgfqpoint{1.074244in}{0.722877in}}%
\pgfpathlineto{\pgfqpoint{1.075284in}{0.692126in}}%
\pgfpathlineto{\pgfqpoint{1.078440in}{0.688961in}}%
\pgfpathlineto{\pgfqpoint{1.080040in}{0.728163in}}%
\pgfpathlineto{\pgfqpoint{1.080896in}{0.693617in}}%
\pgfpathlineto{\pgfqpoint{1.082842in}{0.695904in}}%
\pgfpathlineto{\pgfqpoint{1.085085in}{0.633477in}}%
\pgfpathlineto{\pgfqpoint{1.087079in}{0.683900in}}%
\pgfpathlineto{\pgfqpoint{1.088190in}{0.659208in}}%
\pgfpathlineto{\pgfqpoint{1.090392in}{0.664812in}}%
\pgfpathlineto{\pgfqpoint{1.091326in}{0.637507in}}%
\pgfpathlineto{\pgfqpoint{1.092616in}{0.667198in}}%
\pgfpathlineto{\pgfqpoint{1.093944in}{0.631277in}}%
\pgfpathlineto{\pgfqpoint{1.096871in}{0.627413in}}%
\pgfpathlineto{\pgfqpoint{1.097828in}{0.653693in}}%
\pgfpathlineto{\pgfqpoint{1.099079in}{0.626810in}}%
\pgfpathlineto{\pgfqpoint{1.102307in}{0.684830in}}%
\pgfpathlineto{\pgfqpoint{1.103590in}{0.653698in}}%
\pgfpathlineto{\pgfqpoint{1.104966in}{0.676174in}}%
\pgfpathlineto{\pgfqpoint{1.106860in}{0.680639in}}%
\pgfpathlineto{\pgfqpoint{1.107849in}{0.624350in}}%
\pgfpathlineto{\pgfqpoint{1.110183in}{0.633219in}}%
\pgfpathlineto{\pgfqpoint{1.111708in}{0.605231in}}%
\pgfpathlineto{\pgfqpoint{1.112882in}{0.625207in}}%
\pgfpathlineto{\pgfqpoint{1.115257in}{0.588914in}}%
\pgfpathlineto{\pgfqpoint{1.116400in}{0.626249in}}%
\pgfpathlineto{\pgfqpoint{1.117786in}{0.603582in}}%
\pgfpathlineto{\pgfqpoint{1.120212in}{0.651861in}}%
\pgfpathlineto{\pgfqpoint{1.122562in}{0.666610in}}%
\pgfpathlineto{\pgfqpoint{1.124413in}{0.666008in}}%
\pgfpathlineto{\pgfqpoint{1.126061in}{0.614839in}}%
\pgfpathlineto{\pgfqpoint{1.126549in}{0.640227in}}%
\pgfpathlineto{\pgfqpoint{1.129255in}{0.647541in}}%
\pgfpathlineto{\pgfqpoint{1.130006in}{0.615458in}}%
\pgfpathlineto{\pgfqpoint{1.131423in}{0.638269in}}%
\pgfpathlineto{\pgfqpoint{1.133379in}{0.647033in}}%
\pgfpathlineto{\pgfqpoint{1.134795in}{0.619400in}}%
\pgfpathlineto{\pgfqpoint{1.137471in}{0.651207in}}%
\pgfpathlineto{\pgfqpoint{1.138803in}{0.634171in}}%
\pgfpathlineto{\pgfqpoint{1.140035in}{0.664849in}}%
\pgfpathlineto{\pgfqpoint{1.142097in}{0.666053in}}%
\pgfpathlineto{\pgfqpoint{1.143324in}{0.698741in}}%
\pgfpathlineto{\pgfqpoint{1.145201in}{0.678231in}}%
\pgfpathlineto{\pgfqpoint{1.146913in}{0.714250in}}%
\pgfpathlineto{\pgfqpoint{1.149645in}{0.737644in}}%
\pgfpathlineto{\pgfqpoint{1.150515in}{0.707401in}}%
\pgfpathlineto{\pgfqpoint{1.152738in}{0.756432in}}%
\pgfpathlineto{\pgfqpoint{1.154053in}{0.710664in}}%
\pgfpathlineto{\pgfqpoint{1.155901in}{0.725131in}}%
\pgfpathlineto{\pgfqpoint{1.157664in}{0.688485in}}%
\pgfpathlineto{\pgfqpoint{1.159351in}{0.712631in}}%
\pgfpathlineto{\pgfqpoint{1.160349in}{0.682445in}}%
\pgfpathlineto{\pgfqpoint{1.163614in}{0.684211in}}%
\pgfpathlineto{\pgfqpoint{1.164575in}{0.730344in}}%
\pgfpathlineto{\pgfqpoint{1.165433in}{0.685403in}}%
\pgfpathlineto{\pgfqpoint{1.168428in}{0.694021in}}%
\pgfpathlineto{\pgfqpoint{1.169642in}{0.725012in}}%
\pgfpathlineto{\pgfqpoint{1.170660in}{0.708602in}}%
\pgfpathlineto{\pgfqpoint{1.173935in}{0.734800in}}%
\pgfpathlineto{\pgfqpoint{1.175585in}{0.682144in}}%
\pgfpathlineto{\pgfqpoint{1.178296in}{0.768362in}}%
\pgfpathlineto{\pgfqpoint{1.179850in}{0.797946in}}%
\pgfpathlineto{\pgfqpoint{1.180871in}{0.767921in}}%
\pgfpathlineto{\pgfqpoint{1.183473in}{0.753734in}}%
\pgfpathlineto{\pgfqpoint{1.184875in}{0.785729in}}%
\pgfpathlineto{\pgfqpoint{1.186989in}{0.762077in}}%
\pgfpathlineto{\pgfqpoint{1.187584in}{0.779833in}}%
\pgfpathlineto{\pgfqpoint{1.190848in}{0.787151in}}%
\pgfpathlineto{\pgfqpoint{1.191508in}{0.748507in}}%
\pgfpathlineto{\pgfqpoint{1.192932in}{0.739296in}}%
\pgfpathlineto{\pgfqpoint{1.195720in}{0.798524in}}%
\pgfpathlineto{\pgfqpoint{1.197634in}{0.794886in}}%
\pgfpathlineto{\pgfqpoint{1.199159in}{0.822478in}}%
\pgfpathlineto{\pgfqpoint{1.199645in}{0.799999in}}%
\pgfpathlineto{\pgfqpoint{1.202210in}{0.807171in}}%
\pgfpathlineto{\pgfqpoint{1.203895in}{0.766604in}}%
\pgfpathlineto{\pgfqpoint{1.206023in}{0.805327in}}%
\pgfpathlineto{\pgfqpoint{1.207148in}{0.789175in}}%
\pgfpathlineto{\pgfqpoint{1.208788in}{0.823331in}}%
\pgfpathlineto{\pgfqpoint{1.211252in}{0.808488in}}%
\pgfpathlineto{\pgfqpoint{1.211884in}{0.839159in}}%
\pgfpathlineto{\pgfqpoint{1.214146in}{0.795660in}}%
\pgfpathlineto{\pgfqpoint{1.215898in}{0.829529in}}%
\pgfpathlineto{\pgfqpoint{1.216560in}{0.805243in}}%
\pgfpathlineto{\pgfqpoint{1.219548in}{0.757148in}}%
\pgfpathlineto{\pgfqpoint{1.219979in}{0.787249in}}%
\pgfpathlineto{\pgfqpoint{1.223303in}{0.726062in}}%
\pgfpathlineto{\pgfqpoint{1.224028in}{0.752407in}}%
\pgfpathlineto{\pgfqpoint{1.225986in}{0.745623in}}%
\pgfpathlineto{\pgfqpoint{1.227999in}{0.775653in}}%
\pgfpathlineto{\pgfqpoint{1.230587in}{0.701565in}}%
\pgfpathlineto{\pgfqpoint{1.232527in}{0.684863in}}%
\pgfpathlineto{\pgfqpoint{1.234792in}{0.711427in}}%
\pgfpathlineto{\pgfqpoint{1.235435in}{0.674531in}}%
\pgfpathlineto{\pgfqpoint{1.238084in}{0.645968in}}%
\pgfpathlineto{\pgfqpoint{1.239014in}{0.665979in}}%
\pgfpathlineto{\pgfqpoint{1.241752in}{0.650045in}}%
\pgfpathlineto{\pgfqpoint{1.243066in}{0.710871in}}%
\pgfpathlineto{\pgfqpoint{1.245173in}{0.696623in}}%
\pgfpathlineto{\pgfqpoint{1.245686in}{0.714329in}}%
\pgfpathlineto{\pgfqpoint{1.247502in}{0.694909in}}%
\pgfpathlineto{\pgfqpoint{1.249992in}{0.770039in}}%
\pgfpathlineto{\pgfqpoint{1.251163in}{0.760743in}}%
\pgfpathlineto{\pgfqpoint{1.252489in}{0.791459in}}%
\pgfpathlineto{\pgfqpoint{1.255264in}{0.790321in}}%
\pgfpathlineto{\pgfqpoint{1.257271in}{0.741174in}}%
\pgfpathlineto{\pgfqpoint{1.258282in}{0.767704in}}%
\pgfpathlineto{\pgfqpoint{1.261000in}{0.725134in}}%
\pgfpathlineto{\pgfqpoint{1.264151in}{0.817351in}}%
\pgfpathlineto{\pgfqpoint{1.265851in}{0.762570in}}%
\pgfpathlineto{\pgfqpoint{1.266593in}{0.773510in}}%
\pgfpathlineto{\pgfqpoint{1.268275in}{0.738525in}}%
\pgfpathlineto{\pgfqpoint{1.270642in}{0.768170in}}%
\pgfpathlineto{\pgfqpoint{1.271657in}{0.743049in}}%
\pgfpathlineto{\pgfqpoint{1.272738in}{0.760822in}}%
\pgfpathlineto{\pgfqpoint{1.275177in}{0.717810in}}%
\pgfpathlineto{\pgfqpoint{1.276076in}{0.748095in}}%
\pgfpathlineto{\pgfqpoint{1.279469in}{0.715865in}}%
\pgfpathlineto{\pgfqpoint{1.281159in}{0.741540in}}%
\pgfpathlineto{\pgfqpoint{1.281439in}{0.721048in}}%
\pgfpathlineto{\pgfqpoint{1.284284in}{0.684558in}}%
\pgfpathlineto{\pgfqpoint{1.285968in}{0.706821in}}%
\pgfpathlineto{\pgfqpoint{1.287893in}{0.650952in}}%
\pgfpathlineto{\pgfqpoint{1.288365in}{0.669715in}}%
\pgfpathlineto{\pgfqpoint{1.292578in}{0.735006in}}%
\pgfpathlineto{\pgfqpoint{1.293228in}{0.697508in}}%
\pgfpathlineto{\pgfqpoint{1.295092in}{0.710918in}}%
\pgfpathlineto{\pgfqpoint{1.296760in}{0.674738in}}%
\pgfpathlineto{\pgfqpoint{1.299892in}{0.656375in}}%
\pgfpathlineto{\pgfqpoint{1.301318in}{0.695319in}}%
\pgfpathlineto{\pgfqpoint{1.301673in}{0.664115in}}%
\pgfpathlineto{\pgfqpoint{1.304335in}{0.651325in}}%
\pgfpathlineto{\pgfqpoint{1.306524in}{0.708179in}}%
\pgfpathlineto{\pgfqpoint{1.307607in}{0.688869in}}%
\pgfpathlineto{\pgfqpoint{1.309903in}{0.747184in}}%
\pgfpathlineto{\pgfqpoint{1.311284in}{0.707391in}}%
\pgfpathlineto{\pgfqpoint{1.313134in}{0.757073in}}%
\pgfpathlineto{\pgfqpoint{1.315024in}{0.727606in}}%
\pgfpathlineto{\pgfqpoint{1.316612in}{0.754588in}}%
\pgfpathlineto{\pgfqpoint{1.317547in}{0.721940in}}%
\pgfpathlineto{\pgfqpoint{1.319865in}{0.729020in}}%
\pgfpathlineto{\pgfqpoint{1.320780in}{0.712394in}}%
\pgfpathlineto{\pgfqpoint{1.325249in}{0.741597in}}%
\pgfpathlineto{\pgfqpoint{1.326225in}{0.703531in}}%
\pgfpathlineto{\pgfqpoint{1.328123in}{0.685505in}}%
\pgfpathlineto{\pgfqpoint{1.329604in}{0.731423in}}%
\pgfpathlineto{\pgfqpoint{1.331524in}{0.730448in}}%
\pgfpathlineto{\pgfqpoint{1.332640in}{0.697699in}}%
\pgfpathlineto{\pgfqpoint{1.335195in}{0.731683in}}%
\pgfpathlineto{\pgfqpoint{1.335902in}{0.707373in}}%
\pgfpathlineto{\pgfqpoint{1.337569in}{0.744593in}}%
\pgfpathlineto{\pgfqpoint{1.340294in}{0.706527in}}%
\pgfpathlineto{\pgfqpoint{1.342151in}{0.753476in}}%
\pgfpathlineto{\pgfqpoint{1.343919in}{0.759467in}}%
\pgfpathlineto{\pgfqpoint{1.345185in}{0.718518in}}%
\pgfpathlineto{\pgfqpoint{1.346033in}{0.745199in}}%
\pgfpathlineto{\pgfqpoint{1.347705in}{0.705804in}}%
\pgfpathlineto{\pgfqpoint{1.349624in}{0.724893in}}%
\pgfpathlineto{\pgfqpoint{1.352062in}{0.700712in}}%
\pgfpathlineto{\pgfqpoint{1.352771in}{0.719008in}}%
\pgfpathlineto{\pgfqpoint{1.355558in}{0.666928in}}%
\pgfpathlineto{\pgfqpoint{1.357130in}{0.715994in}}%
\pgfpathlineto{\pgfqpoint{1.358831in}{0.669561in}}%
\pgfpathlineto{\pgfqpoint{1.359527in}{0.694768in}}%
\pgfpathlineto{\pgfqpoint{1.361375in}{0.628405in}}%
\pgfpathlineto{\pgfqpoint{1.363780in}{0.688793in}}%
\pgfpathlineto{\pgfqpoint{1.365132in}{0.687545in}}%
\pgfpathlineto{\pgfqpoint{1.366329in}{0.647662in}}%
\pgfpathlineto{\pgfqpoint{1.369190in}{0.645504in}}%
\pgfpathlineto{\pgfqpoint{1.369896in}{0.685948in}}%
\pgfpathlineto{\pgfqpoint{1.371548in}{0.686309in}}%
\pgfpathlineto{\pgfqpoint{1.374672in}{0.626459in}}%
\pgfpathlineto{\pgfqpoint{1.376005in}{0.637721in}}%
\pgfpathlineto{\pgfqpoint{1.377268in}{0.598720in}}%
\pgfpathlineto{\pgfqpoint{1.380830in}{0.659802in}}%
\pgfpathlineto{\pgfqpoint{1.382254in}{0.622761in}}%
\pgfpathlineto{\pgfqpoint{1.384849in}{0.625634in}}%
\pgfpathlineto{\pgfqpoint{1.385468in}{0.666739in}}%
\pgfpathlineto{\pgfqpoint{1.386906in}{0.649271in}}%
\pgfpathlineto{\pgfqpoint{1.389253in}{0.697939in}}%
\pgfpathlineto{\pgfqpoint{1.390117in}{0.652779in}}%
\pgfpathlineto{\pgfqpoint{1.393500in}{0.657915in}}%
\pgfpathlineto{\pgfqpoint{1.394124in}{0.633490in}}%
\pgfpathlineto{\pgfqpoint{1.395223in}{0.634960in}}%
\pgfpathlineto{\pgfqpoint{1.398179in}{0.582710in}}%
\pgfpathlineto{\pgfqpoint{1.400028in}{0.629027in}}%
\pgfpathlineto{\pgfqpoint{1.400325in}{0.610253in}}%
\pgfpathlineto{\pgfqpoint{1.402075in}{0.618142in}}%
\pgfpathlineto{\pgfqpoint{1.404200in}{0.573031in}}%
\pgfpathlineto{\pgfqpoint{1.406618in}{0.580360in}}%
\pgfpathlineto{\pgfqpoint{1.407943in}{0.627850in}}%
\pgfpathlineto{\pgfqpoint{1.410269in}{0.667149in}}%
\pgfpathlineto{\pgfqpoint{1.410824in}{0.635191in}}%
\pgfpathlineto{\pgfqpoint{1.413183in}{0.613054in}}%
\pgfpathlineto{\pgfqpoint{1.414854in}{0.640096in}}%
\pgfpathlineto{\pgfqpoint{1.417271in}{0.605360in}}%
\pgfpathlineto{\pgfqpoint{1.417918in}{0.605902in}}%
\pgfpathlineto{\pgfqpoint{1.420655in}{0.691425in}}%
\pgfpathlineto{\pgfqpoint{1.420982in}{0.677860in}}%
\pgfpathlineto{\pgfqpoint{1.422996in}{0.701672in}}%
\pgfpathlineto{\pgfqpoint{1.426078in}{0.683479in}}%
\pgfpathlineto{\pgfqpoint{1.428178in}{0.616032in}}%
\pgfpathlineto{\pgfqpoint{1.430640in}{0.595720in}}%
\pgfpathlineto{\pgfqpoint{1.431095in}{0.617021in}}%
\pgfpathlineto{\pgfqpoint{1.433354in}{0.583698in}}%
\pgfpathlineto{\pgfqpoint{1.434922in}{0.604217in}}%
\pgfpathlineto{\pgfqpoint{1.436078in}{0.577535in}}%
\pgfpathlineto{\pgfqpoint{1.437849in}{0.586105in}}%
\pgfpathlineto{\pgfqpoint{1.439991in}{0.530698in}}%
\pgfpathlineto{\pgfqpoint{1.441627in}{0.563231in}}%
\pgfpathlineto{\pgfqpoint{1.443063in}{0.517793in}}%
\pgfpathlineto{\pgfqpoint{1.444755in}{0.546375in}}%
\pgfpathlineto{\pgfqpoint{1.447617in}{0.513896in}}%
\pgfpathlineto{\pgfqpoint{1.448547in}{0.553253in}}%
\pgfpathlineto{\pgfqpoint{1.450397in}{0.569554in}}%
\pgfpathlineto{\pgfqpoint{1.451548in}{0.532630in}}%
\pgfpathlineto{\pgfqpoint{1.453963in}{0.535189in}}%
\pgfpathlineto{\pgfqpoint{1.456481in}{0.620750in}}%
\pgfpathlineto{\pgfqpoint{1.456933in}{0.600401in}}%
\pgfpathlineto{\pgfqpoint{1.460784in}{0.690142in}}%
\pgfpathlineto{\pgfqpoint{1.462687in}{0.654001in}}%
\pgfpathlineto{\pgfqpoint{1.463879in}{0.692971in}}%
\pgfpathlineto{\pgfqpoint{1.466630in}{0.660371in}}%
\pgfpathlineto{\pgfqpoint{1.467954in}{0.676397in}}%
\pgfpathlineto{\pgfqpoint{1.469429in}{0.654981in}}%
\pgfpathlineto{\pgfqpoint{1.470864in}{0.679632in}}%
\pgfpathlineto{\pgfqpoint{1.473039in}{0.684446in}}%
\pgfpathlineto{\pgfqpoint{1.473756in}{0.660083in}}%
\pgfpathlineto{\pgfqpoint{1.476659in}{0.667921in}}%
\pgfpathlineto{\pgfqpoint{1.477857in}{0.705399in}}%
\pgfpathlineto{\pgfqpoint{1.478812in}{0.692009in}}%
\pgfpathlineto{\pgfqpoint{1.480923in}{0.729783in}}%
\pgfpathlineto{\pgfqpoint{1.482348in}{0.685815in}}%
\pgfpathlineto{\pgfqpoint{1.483736in}{0.707612in}}%
\pgfpathlineto{\pgfqpoint{1.486025in}{0.699459in}}%
\pgfpathlineto{\pgfqpoint{1.488715in}{0.732452in}}%
\pgfpathlineto{\pgfqpoint{1.490231in}{0.700164in}}%
\pgfpathlineto{\pgfqpoint{1.491149in}{0.723796in}}%
\pgfpathlineto{\pgfqpoint{1.493296in}{0.686341in}}%
\pgfpathlineto{\pgfqpoint{1.495839in}{0.753068in}}%
\pgfpathlineto{\pgfqpoint{1.497790in}{0.709479in}}%
\pgfpathlineto{\pgfqpoint{1.500092in}{0.693861in}}%
\pgfpathlineto{\pgfqpoint{1.501089in}{0.718111in}}%
\pgfpathlineto{\pgfqpoint{1.503703in}{0.719192in}}%
\pgfpathlineto{\pgfqpoint{1.505582in}{0.660462in}}%
\pgfpathlineto{\pgfqpoint{1.506797in}{0.683861in}}%
\pgfpathlineto{\pgfqpoint{1.510140in}{0.600254in}}%
\pgfpathlineto{\pgfqpoint{1.511922in}{0.620477in}}%
\pgfpathlineto{\pgfqpoint{1.514093in}{0.679733in}}%
\pgfpathlineto{\pgfqpoint{1.514839in}{0.650332in}}%
\pgfpathlineto{\pgfqpoint{1.516919in}{0.694994in}}%
\pgfpathlineto{\pgfqpoint{1.519152in}{0.712965in}}%
\pgfpathlineto{\pgfqpoint{1.520468in}{0.672586in}}%
\pgfpathlineto{\pgfqpoint{1.522075in}{0.664904in}}%
\pgfpathlineto{\pgfqpoint{1.523710in}{0.712096in}}%
\pgfpathlineto{\pgfqpoint{1.524694in}{0.701525in}}%
\pgfpathlineto{\pgfqpoint{1.527794in}{0.721375in}}%
\pgfpathlineto{\pgfqpoint{1.528652in}{0.752726in}}%
\pgfpathlineto{\pgfqpoint{1.530880in}{0.749460in}}%
\pgfpathlineto{\pgfqpoint{1.531963in}{0.786414in}}%
\pgfpathlineto{\pgfqpoint{1.533315in}{0.768243in}}%
\pgfpathlineto{\pgfqpoint{1.535395in}{0.839889in}}%
\pgfpathlineto{\pgfqpoint{1.537529in}{0.807682in}}%
\pgfpathlineto{\pgfqpoint{1.539817in}{0.835769in}}%
\pgfpathlineto{\pgfqpoint{1.541404in}{0.832593in}}%
\pgfpathlineto{\pgfqpoint{1.541858in}{0.805452in}}%
\pgfpathlineto{\pgfqpoint{1.543357in}{0.806232in}}%
\pgfpathlineto{\pgfqpoint{1.545340in}{0.766222in}}%
\pgfpathlineto{\pgfqpoint{1.547151in}{0.778431in}}%
\pgfpathlineto{\pgfqpoint{1.548929in}{0.743851in}}%
\pgfpathlineto{\pgfqpoint{1.551410in}{0.773606in}}%
\pgfpathlineto{\pgfqpoint{1.551966in}{0.736777in}}%
\pgfpathlineto{\pgfqpoint{1.554988in}{0.787123in}}%
\pgfpathlineto{\pgfqpoint{1.556044in}{0.767511in}}%
\pgfpathlineto{\pgfqpoint{1.558104in}{0.813385in}}%
\pgfpathlineto{\pgfqpoint{1.559194in}{0.782732in}}%
\pgfpathlineto{\pgfqpoint{1.560354in}{0.828948in}}%
\pgfpathlineto{\pgfqpoint{1.563193in}{0.813142in}}%
\pgfpathlineto{\pgfqpoint{1.564863in}{0.843139in}}%
\pgfpathlineto{\pgfqpoint{1.565485in}{0.821590in}}%
\pgfpathlineto{\pgfqpoint{1.568937in}{0.776862in}}%
\pgfpathlineto{\pgfqpoint{1.570888in}{0.814693in}}%
\pgfpathlineto{\pgfqpoint{1.573044in}{0.785886in}}%
\pgfpathlineto{\pgfqpoint{1.575018in}{0.830879in}}%
\pgfpathlineto{\pgfqpoint{1.576070in}{0.793297in}}%
\pgfpathlineto{\pgfqpoint{1.578143in}{0.766963in}}%
\pgfpathlineto{\pgfqpoint{1.579780in}{0.795096in}}%
\pgfpathlineto{\pgfqpoint{1.580795in}{0.769690in}}%
\pgfpathlineto{\pgfqpoint{1.582589in}{0.787746in}}%
\pgfpathlineto{\pgfqpoint{1.584253in}{0.753824in}}%
\pgfpathlineto{\pgfqpoint{1.586061in}{0.788872in}}%
\pgfpathlineto{\pgfqpoint{1.588619in}{0.752263in}}%
\pgfpathlineto{\pgfqpoint{1.589310in}{0.770636in}}%
\pgfpathlineto{\pgfqpoint{1.591242in}{0.711850in}}%
\pgfpathlineto{\pgfqpoint{1.593067in}{0.726660in}}%
\pgfpathlineto{\pgfqpoint{1.595863in}{0.687988in}}%
\pgfpathlineto{\pgfqpoint{1.596905in}{0.724658in}}%
\pgfpathlineto{\pgfqpoint{1.598161in}{0.711526in}}%
\pgfpathlineto{\pgfqpoint{1.600755in}{0.754487in}}%
\pgfpathlineto{\pgfqpoint{1.601185in}{0.732213in}}%
\pgfpathlineto{\pgfqpoint{1.603945in}{0.731385in}}%
\pgfpathlineto{\pgfqpoint{1.607475in}{0.658119in}}%
\pgfpathlineto{\pgfqpoint{1.610283in}{0.723903in}}%
\pgfpathlineto{\pgfqpoint{1.612423in}{0.676684in}}%
\pgfpathlineto{\pgfqpoint{1.613877in}{0.722712in}}%
\pgfpathlineto{\pgfqpoint{1.616192in}{0.739974in}}%
\pgfpathlineto{\pgfqpoint{1.617037in}{0.712384in}}%
\pgfpathlineto{\pgfqpoint{1.618661in}{0.740898in}}%
\pgfpathlineto{\pgfqpoint{1.621387in}{0.684623in}}%
\pgfpathlineto{\pgfqpoint{1.622467in}{0.697329in}}%
\pgfpathlineto{\pgfqpoint{1.624160in}{0.663465in}}%
\pgfpathlineto{\pgfqpoint{1.625977in}{0.696251in}}%
\pgfpathlineto{\pgfqpoint{1.628211in}{0.637434in}}%
\pgfpathlineto{\pgfqpoint{1.629430in}{0.687004in}}%
\pgfpathlineto{\pgfqpoint{1.631020in}{0.698848in}}%
\pgfpathlineto{\pgfqpoint{1.632439in}{0.660721in}}%
\pgfpathlineto{\pgfqpoint{1.634047in}{0.698407in}}%
\pgfpathlineto{\pgfqpoint{1.635477in}{0.663037in}}%
\pgfpathlineto{\pgfqpoint{1.640320in}{0.757784in}}%
\pgfpathlineto{\pgfqpoint{1.642944in}{0.696333in}}%
\pgfpathlineto{\pgfqpoint{1.645209in}{0.668449in}}%
\pgfpathlineto{\pgfqpoint{1.646901in}{0.700061in}}%
\pgfpathlineto{\pgfqpoint{1.648378in}{0.654814in}}%
\pgfpathlineto{\pgfqpoint{1.650296in}{0.680818in}}%
\pgfpathlineto{\pgfqpoint{1.652472in}{0.619995in}}%
\pgfpathlineto{\pgfqpoint{1.654090in}{0.647976in}}%
\pgfpathlineto{\pgfqpoint{1.655659in}{0.615411in}}%
\pgfpathlineto{\pgfqpoint{1.658935in}{0.581732in}}%
\pgfpathlineto{\pgfqpoint{1.659550in}{0.603223in}}%
\pgfpathlineto{\pgfqpoint{1.663201in}{0.526056in}}%
\pgfpathlineto{\pgfqpoint{1.665648in}{0.533524in}}%
\pgfpathlineto{\pgfqpoint{1.666637in}{0.563545in}}%
\pgfpathlineto{\pgfqpoint{1.667558in}{0.546641in}}%
\pgfpathlineto{\pgfqpoint{1.669823in}{0.584491in}}%
\pgfpathlineto{\pgfqpoint{1.671034in}{0.552413in}}%
\pgfpathlineto{\pgfqpoint{1.672749in}{0.574170in}}%
\pgfpathlineto{\pgfqpoint{1.674755in}{0.557626in}}%
\pgfpathlineto{\pgfqpoint{1.677382in}{0.550342in}}%
\pgfpathlineto{\pgfqpoint{1.679585in}{0.625620in}}%
\pgfpathlineto{\pgfqpoint{1.681468in}{0.610383in}}%
\pgfpathlineto{\pgfqpoint{1.684315in}{0.538237in}}%
\pgfpathlineto{\pgfqpoint{1.685806in}{0.582356in}}%
\pgfpathlineto{\pgfqpoint{1.687213in}{0.558520in}}%
\pgfpathlineto{\pgfqpoint{1.688736in}{0.580644in}}%
\pgfpathlineto{\pgfqpoint{1.689778in}{0.564209in}}%
\pgfpathlineto{\pgfqpoint{1.691539in}{0.578125in}}%
\pgfpathlineto{\pgfqpoint{1.694236in}{0.551493in}}%
\pgfpathlineto{\pgfqpoint{1.696307in}{0.504104in}}%
\pgfpathlineto{\pgfqpoint{1.697068in}{0.541212in}}%
\pgfpathlineto{\pgfqpoint{1.699102in}{0.562592in}}%
\pgfpathlineto{\pgfqpoint{1.700039in}{0.534016in}}%
\pgfpathlineto{\pgfqpoint{1.701698in}{0.550023in}}%
\pgfpathlineto{\pgfqpoint{1.703656in}{0.525298in}}%
\pgfpathlineto{\pgfqpoint{1.705064in}{0.544623in}}%
\pgfpathlineto{\pgfqpoint{1.707160in}{0.518859in}}%
\pgfpathlineto{\pgfqpoint{1.709749in}{0.563526in}}%
\pgfpathlineto{\pgfqpoint{1.710551in}{0.541270in}}%
\pgfpathlineto{\pgfqpoint{1.712547in}{0.537055in}}%
\pgfpathlineto{\pgfqpoint{1.713843in}{0.568871in}}%
\pgfpathlineto{\pgfqpoint{1.715314in}{0.535068in}}%
\pgfpathlineto{\pgfqpoint{1.717328in}{0.567025in}}%
\pgfpathlineto{\pgfqpoint{1.719643in}{0.539662in}}%
\pgfpathlineto{\pgfqpoint{1.721627in}{0.543063in}}%
\pgfpathlineto{\pgfqpoint{1.725125in}{0.618537in}}%
\pgfpathlineto{\pgfqpoint{1.726910in}{0.558184in}}%
\pgfpathlineto{\pgfqpoint{1.727760in}{0.580013in}}%
\pgfpathlineto{\pgfqpoint{1.729432in}{0.549483in}}%
\pgfpathlineto{\pgfqpoint{1.732505in}{0.610596in}}%
\pgfpathlineto{\pgfqpoint{1.735315in}{0.624209in}}%
\pgfpathlineto{\pgfqpoint{1.735694in}{0.601863in}}%
\pgfpathlineto{\pgfqpoint{1.737450in}{0.589901in}}%
\pgfpathlineto{\pgfqpoint{1.739333in}{0.618429in}}%
\pgfpathlineto{\pgfqpoint{1.741238in}{0.597987in}}%
\pgfpathlineto{\pgfqpoint{1.742992in}{0.636040in}}%
\pgfpathlineto{\pgfqpoint{1.744181in}{0.622865in}}%
\pgfpathlineto{\pgfqpoint{1.747410in}{0.660399in}}%
\pgfpathlineto{\pgfqpoint{1.748334in}{0.627342in}}%
\pgfpathlineto{\pgfqpoint{1.749453in}{0.663722in}}%
\pgfpathlineto{\pgfqpoint{1.751585in}{0.657967in}}%
\pgfpathlineto{\pgfqpoint{1.753149in}{0.705065in}}%
\pgfpathlineto{\pgfqpoint{1.754591in}{0.685353in}}%
\pgfpathlineto{\pgfqpoint{1.756469in}{0.710433in}}%
\pgfpathlineto{\pgfqpoint{1.759354in}{0.673849in}}%
\pgfpathlineto{\pgfqpoint{1.760061in}{0.700852in}}%
\pgfpathlineto{\pgfqpoint{1.761174in}{0.665289in}}%
\pgfpathlineto{\pgfqpoint{1.764542in}{0.707430in}}%
\pgfpathlineto{\pgfqpoint{1.765002in}{0.684607in}}%
\pgfpathlineto{\pgfqpoint{1.767925in}{0.766034in}}%
\pgfpathlineto{\pgfqpoint{1.769510in}{0.767400in}}%
\pgfpathlineto{\pgfqpoint{1.770674in}{0.726689in}}%
\pgfpathlineto{\pgfqpoint{1.771478in}{0.760093in}}%
\pgfpathlineto{\pgfqpoint{1.773178in}{0.763159in}}%
\pgfpathlineto{\pgfqpoint{1.775718in}{0.705836in}}%
\pgfpathlineto{\pgfqpoint{1.776670in}{0.742733in}}%
\pgfpathlineto{\pgfqpoint{1.778772in}{0.771966in}}%
\pgfpathlineto{\pgfqpoint{1.780372in}{0.747097in}}%
\pgfpathlineto{\pgfqpoint{1.782322in}{0.783515in}}%
\pgfpathlineto{\pgfqpoint{1.784164in}{0.729734in}}%
\pgfpathlineto{\pgfqpoint{1.785863in}{0.768178in}}%
\pgfpathlineto{\pgfqpoint{1.787088in}{0.720612in}}%
\pgfpathlineto{\pgfqpoint{1.789708in}{0.775267in}}%
\pgfpathlineto{\pgfqpoint{1.790846in}{0.753089in}}%
\pgfpathlineto{\pgfqpoint{1.793004in}{0.764778in}}%
\pgfpathlineto{\pgfqpoint{1.794240in}{0.816091in}}%
\pgfpathlineto{\pgfqpoint{1.795738in}{0.789980in}}%
\pgfpathlineto{\pgfqpoint{1.797209in}{0.817563in}}%
\pgfpathlineto{\pgfqpoint{1.800083in}{0.837649in}}%
\pgfpathlineto{\pgfqpoint{1.801285in}{0.822621in}}%
\pgfpathlineto{\pgfqpoint{1.803712in}{0.815400in}}%
\pgfpathlineto{\pgfqpoint{1.805353in}{0.871753in}}%
\pgfpathlineto{\pgfqpoint{1.807074in}{0.853799in}}%
\pgfpathlineto{\pgfqpoint{1.807417in}{0.875315in}}%
\pgfpathlineto{\pgfqpoint{1.808876in}{0.864819in}}%
\pgfpathlineto{\pgfqpoint{1.810765in}{0.904444in}}%
\pgfpathlineto{\pgfqpoint{1.813167in}{0.930669in}}%
\pgfpathlineto{\pgfqpoint{1.814564in}{0.890592in}}%
\pgfpathlineto{\pgfqpoint{1.815966in}{0.918035in}}%
\pgfpathlineto{\pgfqpoint{1.818861in}{0.931787in}}%
\pgfpathlineto{\pgfqpoint{1.819803in}{0.901499in}}%
\pgfpathlineto{\pgfqpoint{1.821022in}{0.920378in}}%
\pgfpathlineto{\pgfqpoint{1.822864in}{0.922447in}}%
\pgfpathlineto{\pgfqpoint{1.824218in}{0.874668in}}%
\pgfpathlineto{\pgfqpoint{1.828126in}{0.839313in}}%
\pgfpathlineto{\pgfqpoint{1.829900in}{0.882636in}}%
\pgfpathlineto{\pgfqpoint{1.831378in}{0.877757in}}%
\pgfpathlineto{\pgfqpoint{1.833109in}{0.928163in}}%
\pgfpathlineto{\pgfqpoint{1.834641in}{0.921104in}}%
\pgfpathlineto{\pgfqpoint{1.837549in}{1.003612in}}%
\pgfpathlineto{\pgfqpoint{1.839376in}{0.987587in}}%
\pgfpathlineto{\pgfqpoint{1.840858in}{1.061421in}}%
\pgfpathlineto{\pgfqpoint{1.842246in}{1.040634in}}%
\pgfpathlineto{\pgfqpoint{1.843132in}{1.082594in}}%
\pgfpathlineto{\pgfqpoint{1.845293in}{1.115094in}}%
\pgfpathlineto{\pgfqpoint{1.846379in}{1.092092in}}%
\pgfpathlineto{\pgfqpoint{1.848073in}{1.132622in}}%
\pgfpathlineto{\pgfqpoint{1.850171in}{1.152410in}}%
\pgfpathlineto{\pgfqpoint{1.852157in}{1.115298in}}%
\pgfpathlineto{\pgfqpoint{1.854010in}{1.154286in}}%
\pgfpathlineto{\pgfqpoint{1.855846in}{1.123594in}}%
\pgfpathlineto{\pgfqpoint{1.858077in}{1.118545in}}%
\pgfpathlineto{\pgfqpoint{1.858597in}{1.155763in}}%
\pgfpathlineto{\pgfqpoint{1.861109in}{1.176188in}}%
\pgfpathlineto{\pgfqpoint{1.864215in}{1.117625in}}%
\pgfpathlineto{\pgfqpoint{1.865050in}{1.147981in}}%
\pgfpathlineto{\pgfqpoint{1.867869in}{1.155607in}}%
\pgfpathlineto{\pgfqpoint{1.870397in}{1.221887in}}%
\pgfpathlineto{\pgfqpoint{1.873430in}{1.230366in}}%
\pgfpathlineto{\pgfqpoint{1.873919in}{1.205976in}}%
\pgfpathlineto{\pgfqpoint{1.876806in}{1.277455in}}%
\pgfpathlineto{\pgfqpoint{1.877397in}{1.260672in}}%
\pgfpathlineto{\pgfqpoint{1.878629in}{1.288888in}}%
\pgfpathlineto{\pgfqpoint{1.880931in}{1.299199in}}%
\pgfpathlineto{\pgfqpoint{1.882037in}{1.268406in}}%
\pgfpathlineto{\pgfqpoint{1.884874in}{1.253539in}}%
\pgfpathlineto{\pgfqpoint{1.886401in}{1.279096in}}%
\pgfpathlineto{\pgfqpoint{1.888819in}{1.232043in}}%
\pgfpathlineto{\pgfqpoint{1.889996in}{1.260146in}}%
\pgfpathlineto{\pgfqpoint{1.890758in}{1.238838in}}%
\pgfpathlineto{\pgfqpoint{1.892238in}{1.264835in}}%
\pgfpathlineto{\pgfqpoint{1.895393in}{1.241695in}}%
\pgfpathlineto{\pgfqpoint{1.896811in}{1.277346in}}%
\pgfpathlineto{\pgfqpoint{1.898032in}{1.230684in}}%
\pgfpathlineto{\pgfqpoint{1.899815in}{1.224783in}}%
\pgfpathlineto{\pgfqpoint{1.901347in}{1.256567in}}%
\pgfpathlineto{\pgfqpoint{1.903364in}{1.230468in}}%
\pgfpathlineto{\pgfqpoint{1.904986in}{1.225329in}}%
\pgfpathlineto{\pgfqpoint{1.907182in}{1.281842in}}%
\pgfpathlineto{\pgfqpoint{1.907951in}{1.253974in}}%
\pgfpathlineto{\pgfqpoint{1.910286in}{1.279105in}}%
\pgfpathlineto{\pgfqpoint{1.912425in}{1.246201in}}%
\pgfpathlineto{\pgfqpoint{1.912920in}{1.261650in}}%
\pgfpathlineto{\pgfqpoint{1.915851in}{1.239233in}}%
\pgfpathlineto{\pgfqpoint{1.917054in}{1.272982in}}%
\pgfpathlineto{\pgfqpoint{1.918387in}{1.253918in}}%
\pgfpathlineto{\pgfqpoint{1.919828in}{1.297764in}}%
\pgfpathlineto{\pgfqpoint{1.921978in}{1.309438in}}%
\pgfpathlineto{\pgfqpoint{1.924238in}{1.286263in}}%
\pgfpathlineto{\pgfqpoint{1.925345in}{1.357383in}}%
\pgfpathlineto{\pgfqpoint{1.927174in}{1.366835in}}%
\pgfpathlineto{\pgfqpoint{1.930943in}{1.282511in}}%
\pgfpathlineto{\pgfqpoint{1.931515in}{1.300602in}}%
\pgfpathlineto{\pgfqpoint{1.933980in}{1.325781in}}%
\pgfpathlineto{\pgfqpoint{1.936386in}{1.257802in}}%
\pgfpathlineto{\pgfqpoint{1.937187in}{1.273406in}}%
\pgfpathlineto{\pgfqpoint{1.938199in}{1.253965in}}%
\pgfpathlineto{\pgfqpoint{1.941300in}{1.262881in}}%
\pgfpathlineto{\pgfqpoint{1.942807in}{1.275623in}}%
\pgfpathlineto{\pgfqpoint{1.944767in}{1.335160in}}%
\pgfpathlineto{\pgfqpoint{1.946249in}{1.322193in}}%
\pgfpathlineto{\pgfqpoint{1.947750in}{1.362652in}}%
\pgfpathlineto{\pgfqpoint{1.948429in}{1.329659in}}%
\pgfpathlineto{\pgfqpoint{1.950304in}{1.356568in}}%
\pgfpathlineto{\pgfqpoint{1.951969in}{1.319557in}}%
\pgfpathlineto{\pgfqpoint{1.954914in}{1.297683in}}%
\pgfpathlineto{\pgfqpoint{1.956307in}{1.331562in}}%
\pgfpathlineto{\pgfqpoint{1.957763in}{1.304506in}}%
\pgfpathlineto{\pgfqpoint{1.959042in}{1.341274in}}%
\pgfpathlineto{\pgfqpoint{1.961775in}{1.358576in}}%
\pgfpathlineto{\pgfqpoint{1.962540in}{1.323150in}}%
\pgfpathlineto{\pgfqpoint{1.964755in}{1.346787in}}%
\pgfpathlineto{\pgfqpoint{1.965553in}{1.328654in}}%
\pgfpathlineto{\pgfqpoint{1.967851in}{1.309846in}}%
\pgfpathlineto{\pgfqpoint{1.969897in}{1.354318in}}%
\pgfpathlineto{\pgfqpoint{1.970968in}{1.322076in}}%
\pgfpathlineto{\pgfqpoint{1.973103in}{1.370082in}}%
\pgfpathlineto{\pgfqpoint{1.976601in}{1.294047in}}%
\pgfpathlineto{\pgfqpoint{1.978618in}{1.285652in}}%
\pgfpathlineto{\pgfqpoint{1.979841in}{1.319337in}}%
\pgfpathlineto{\pgfqpoint{1.981239in}{1.297756in}}%
\pgfpathlineto{\pgfqpoint{1.983779in}{1.339278in}}%
\pgfpathlineto{\pgfqpoint{1.984836in}{1.317646in}}%
\pgfpathlineto{\pgfqpoint{1.986249in}{1.342348in}}%
\pgfpathlineto{\pgfqpoint{1.988078in}{1.301967in}}%
\pgfpathlineto{\pgfqpoint{1.989722in}{1.348466in}}%
\pgfpathlineto{\pgfqpoint{1.991469in}{1.327263in}}%
\pgfpathlineto{\pgfqpoint{1.992892in}{1.391004in}}%
\pgfpathlineto{\pgfqpoint{1.995618in}{1.418869in}}%
\pgfpathlineto{\pgfqpoint{1.996172in}{1.382504in}}%
\pgfpathlineto{\pgfqpoint{1.998114in}{1.353118in}}%
\pgfpathlineto{\pgfqpoint{1.999493in}{1.383373in}}%
\pgfpathlineto{\pgfqpoint{2.002009in}{1.346229in}}%
\pgfpathlineto{\pgfqpoint{2.003671in}{1.391992in}}%
\pgfpathlineto{\pgfqpoint{2.004985in}{1.346456in}}%
\pgfpathlineto{\pgfqpoint{2.006854in}{1.367831in}}%
\pgfpathlineto{\pgfqpoint{2.009036in}{1.302187in}}%
\pgfpathlineto{\pgfqpoint{2.010523in}{1.292579in}}%
\pgfpathlineto{\pgfqpoint{2.012316in}{1.324311in}}%
\pgfpathlineto{\pgfqpoint{2.013454in}{1.302941in}}%
\pgfpathlineto{\pgfqpoint{2.015243in}{1.336310in}}%
\pgfpathlineto{\pgfqpoint{2.016633in}{1.290414in}}%
\pgfpathlineto{\pgfqpoint{2.018187in}{1.315149in}}%
\pgfpathlineto{\pgfqpoint{2.020586in}{1.322344in}}%
\pgfpathlineto{\pgfqpoint{2.023100in}{1.295194in}}%
\pgfpathlineto{\pgfqpoint{2.025746in}{1.330054in}}%
\pgfpathlineto{\pgfqpoint{2.028153in}{1.275535in}}%
\pgfpathlineto{\pgfqpoint{2.029518in}{1.326316in}}%
\pgfpathlineto{\pgfqpoint{2.030505in}{1.295752in}}%
\pgfpathlineto{\pgfqpoint{2.033193in}{1.292835in}}%
\pgfpathlineto{\pgfqpoint{2.034251in}{1.322022in}}%
\pgfpathlineto{\pgfqpoint{2.035716in}{1.298656in}}%
\pgfpathlineto{\pgfqpoint{2.038181in}{1.345598in}}%
\pgfpathlineto{\pgfqpoint{2.038755in}{1.325229in}}%
\pgfpathlineto{\pgfqpoint{2.041201in}{1.306657in}}%
\pgfpathlineto{\pgfqpoint{2.042274in}{1.340146in}}%
\pgfpathlineto{\pgfqpoint{2.044668in}{1.312077in}}%
\pgfpathlineto{\pgfqpoint{2.046187in}{1.353974in}}%
\pgfpathlineto{\pgfqpoint{2.048439in}{1.308956in}}%
\pgfpathlineto{\pgfqpoint{2.049142in}{1.326687in}}%
\pgfpathlineto{\pgfqpoint{2.051989in}{1.304022in}}%
\pgfpathlineto{\pgfqpoint{2.053123in}{1.336690in}}%
\pgfpathlineto{\pgfqpoint{2.054220in}{1.329087in}}%
\pgfpathlineto{\pgfqpoint{2.056465in}{1.400828in}}%
\pgfpathlineto{\pgfqpoint{2.058933in}{1.422932in}}%
\pgfpathlineto{\pgfqpoint{2.059718in}{1.388561in}}%
\pgfpathlineto{\pgfqpoint{2.061091in}{1.407526in}}%
\pgfpathlineto{\pgfqpoint{2.062749in}{1.389941in}}%
\pgfpathlineto{\pgfqpoint{2.065285in}{1.411153in}}%
\pgfpathlineto{\pgfqpoint{2.066131in}{1.386689in}}%
\pgfpathlineto{\pgfqpoint{2.067874in}{1.421093in}}%
\pgfpathlineto{\pgfqpoint{2.070029in}{1.435500in}}%
\pgfpathlineto{\pgfqpoint{2.071298in}{1.407104in}}%
\pgfpathlineto{\pgfqpoint{2.074349in}{1.373365in}}%
\pgfpathlineto{\pgfqpoint{2.075179in}{1.395369in}}%
\pgfpathlineto{\pgfqpoint{2.076168in}{1.362701in}}%
\pgfpathlineto{\pgfqpoint{2.078579in}{1.354624in}}%
\pgfpathlineto{\pgfqpoint{2.079990in}{1.355396in}}%
\pgfpathlineto{\pgfqpoint{2.082859in}{1.403935in}}%
\pgfpathlineto{\pgfqpoint{2.084439in}{1.389282in}}%
\pgfpathlineto{\pgfqpoint{2.085561in}{1.426718in}}%
\pgfpathlineto{\pgfqpoint{2.087364in}{1.391296in}}%
\pgfpathlineto{\pgfqpoint{2.088803in}{1.414696in}}%
\pgfpathlineto{\pgfqpoint{2.090658in}{1.385634in}}%
\pgfpathlineto{\pgfqpoint{2.093043in}{1.411850in}}%
\pgfpathlineto{\pgfqpoint{2.094237in}{1.373949in}}%
\pgfpathlineto{\pgfqpoint{2.095977in}{1.405727in}}%
\pgfpathlineto{\pgfqpoint{2.096664in}{1.377310in}}%
\pgfpathlineto{\pgfqpoint{2.098964in}{1.433451in}}%
\pgfpathlineto{\pgfqpoint{2.100097in}{1.389238in}}%
\pgfpathlineto{\pgfqpoint{2.102538in}{1.372650in}}%
\pgfpathlineto{\pgfqpoint{2.104647in}{1.400971in}}%
\pgfpathlineto{\pgfqpoint{2.105791in}{1.355171in}}%
\pgfpathlineto{\pgfqpoint{2.107074in}{1.374414in}}%
\pgfpathlineto{\pgfqpoint{2.109654in}{1.358892in}}%
\pgfpathlineto{\pgfqpoint{2.111469in}{1.403451in}}%
\pgfpathlineto{\pgfqpoint{2.113269in}{1.366944in}}%
\pgfpathlineto{\pgfqpoint{2.114185in}{1.386087in}}%
\pgfpathlineto{\pgfqpoint{2.116638in}{1.369492in}}%
\pgfpathlineto{\pgfqpoint{2.117740in}{1.422134in}}%
\pgfpathlineto{\pgfqpoint{2.118904in}{1.385643in}}%
\pgfpathlineto{\pgfqpoint{2.121223in}{1.387049in}}%
\pgfpathlineto{\pgfqpoint{2.122640in}{1.344680in}}%
\pgfpathlineto{\pgfqpoint{2.124717in}{1.344134in}}%
\pgfpathlineto{\pgfqpoint{2.125685in}{1.386786in}}%
\pgfpathlineto{\pgfqpoint{2.128158in}{1.341892in}}%
\pgfpathlineto{\pgfqpoint{2.128819in}{1.368180in}}%
\pgfpathlineto{\pgfqpoint{2.130735in}{1.320827in}}%
\pgfpathlineto{\pgfqpoint{2.132831in}{1.356144in}}%
\pgfpathlineto{\pgfqpoint{2.134246in}{1.307192in}}%
\pgfpathlineto{\pgfqpoint{2.137235in}{1.337983in}}%
\pgfpathlineto{\pgfqpoint{2.138127in}{1.312331in}}%
\pgfpathlineto{\pgfqpoint{2.139238in}{1.341614in}}%
\pgfpathlineto{\pgfqpoint{2.142012in}{1.385139in}}%
\pgfpathlineto{\pgfqpoint{2.144116in}{1.368340in}}%
\pgfpathlineto{\pgfqpoint{2.145700in}{1.313731in}}%
\pgfpathlineto{\pgfqpoint{2.146129in}{1.330861in}}%
\pgfpathlineto{\pgfqpoint{2.148895in}{1.292464in}}%
\pgfpathlineto{\pgfqpoint{2.150943in}{1.334233in}}%
\pgfpathlineto{\pgfqpoint{2.151582in}{1.317754in}}%
\pgfpathlineto{\pgfqpoint{2.154309in}{1.307852in}}%
\pgfpathlineto{\pgfqpoint{2.155296in}{1.343052in}}%
\pgfpathlineto{\pgfqpoint{2.156677in}{1.293037in}}%
\pgfpathlineto{\pgfqpoint{2.158991in}{1.272354in}}%
\pgfpathlineto{\pgfqpoint{2.159622in}{1.294401in}}%
\pgfpathlineto{\pgfqpoint{2.162581in}{1.261694in}}%
\pgfpathlineto{\pgfqpoint{2.164430in}{1.263139in}}%
\pgfpathlineto{\pgfqpoint{2.165409in}{1.305683in}}%
\pgfpathlineto{\pgfqpoint{2.166401in}{1.269830in}}%
\pgfpathlineto{\pgfqpoint{2.168442in}{1.340085in}}%
\pgfpathlineto{\pgfqpoint{2.170473in}{1.308034in}}%
\pgfpathlineto{\pgfqpoint{2.172322in}{1.356887in}}%
\pgfpathlineto{\pgfqpoint{2.174625in}{1.368715in}}%
\pgfpathlineto{\pgfqpoint{2.175476in}{1.329455in}}%
\pgfpathlineto{\pgfqpoint{2.178158in}{1.315981in}}%
\pgfpathlineto{\pgfqpoint{2.179726in}{1.266485in}}%
\pgfpathlineto{\pgfqpoint{2.180318in}{1.293199in}}%
\pgfpathlineto{\pgfqpoint{2.182234in}{1.249240in}}%
\pgfpathlineto{\pgfqpoint{2.183651in}{1.279157in}}%
\pgfpathlineto{\pgfqpoint{2.185120in}{1.280009in}}%
\pgfpathlineto{\pgfqpoint{2.186816in}{1.292751in}}%
\pgfpathlineto{\pgfqpoint{2.189341in}{1.380961in}}%
\pgfpathlineto{\pgfqpoint{2.191717in}{1.313705in}}%
\pgfpathlineto{\pgfqpoint{2.194318in}{1.376247in}}%
\pgfpathlineto{\pgfqpoint{2.196759in}{1.334053in}}%
\pgfpathlineto{\pgfqpoint{2.198453in}{1.335145in}}%
\pgfpathlineto{\pgfqpoint{2.198881in}{1.363301in}}%
\pgfpathlineto{\pgfqpoint{2.201479in}{1.409215in}}%
\pgfpathlineto{\pgfqpoint{2.203029in}{1.378729in}}%
\pgfpathlineto{\pgfqpoint{2.205308in}{1.402684in}}%
\pgfpathlineto{\pgfqpoint{2.206874in}{1.375633in}}%
\pgfpathlineto{\pgfqpoint{2.207625in}{1.412215in}}%
\pgfpathlineto{\pgfqpoint{2.209322in}{1.407505in}}%
\pgfpathlineto{\pgfqpoint{2.210523in}{1.438197in}}%
\pgfpathlineto{\pgfqpoint{2.212271in}{1.446250in}}%
\pgfpathlineto{\pgfqpoint{2.215026in}{1.409558in}}%
\pgfpathlineto{\pgfqpoint{2.216382in}{1.453665in}}%
\pgfpathlineto{\pgfqpoint{2.217347in}{1.423479in}}%
\pgfpathlineto{\pgfqpoint{2.219836in}{1.461847in}}%
\pgfpathlineto{\pgfqpoint{2.221140in}{1.432432in}}%
\pgfpathlineto{\pgfqpoint{2.223714in}{1.426784in}}%
\pgfpathlineto{\pgfqpoint{2.224410in}{1.463821in}}%
\pgfpathlineto{\pgfqpoint{2.227515in}{1.438948in}}%
\pgfpathlineto{\pgfqpoint{2.228910in}{1.451434in}}%
\pgfpathlineto{\pgfqpoint{2.230015in}{1.498950in}}%
\pgfpathlineto{\pgfqpoint{2.232509in}{1.523226in}}%
\pgfpathlineto{\pgfqpoint{2.233320in}{1.489273in}}%
\pgfpathlineto{\pgfqpoint{2.236044in}{1.506297in}}%
\pgfpathlineto{\pgfqpoint{2.237092in}{1.544090in}}%
\pgfpathlineto{\pgfqpoint{2.238439in}{1.521596in}}%
\pgfpathlineto{\pgfqpoint{2.239541in}{1.553504in}}%
\pgfpathlineto{\pgfqpoint{2.241398in}{1.542005in}}%
\pgfpathlineto{\pgfqpoint{2.244537in}{1.581287in}}%
\pgfpathlineto{\pgfqpoint{2.246137in}{1.627601in}}%
\pgfpathlineto{\pgfqpoint{2.247105in}{1.597336in}}%
\pgfpathlineto{\pgfqpoint{2.249437in}{1.641670in}}%
\pgfpathlineto{\pgfqpoint{2.250623in}{1.611708in}}%
\pgfpathlineto{\pgfqpoint{2.252332in}{1.603914in}}%
\pgfpathlineto{\pgfqpoint{2.254086in}{1.632587in}}%
\pgfpathlineto{\pgfqpoint{2.256093in}{1.589054in}}%
\pgfpathlineto{\pgfqpoint{2.256509in}{1.603672in}}%
\pgfpathlineto{\pgfqpoint{2.258388in}{1.607020in}}%
\pgfpathlineto{\pgfqpoint{2.261295in}{1.565167in}}%
\pgfpathlineto{\pgfqpoint{2.262536in}{1.622890in}}%
\pgfpathlineto{\pgfqpoint{2.263782in}{1.589854in}}%
\pgfpathlineto{\pgfqpoint{2.266511in}{1.627454in}}%
\pgfpathlineto{\pgfqpoint{2.268233in}{1.582528in}}%
\pgfpathlineto{\pgfqpoint{2.268734in}{1.612760in}}%
\pgfpathlineto{\pgfqpoint{2.270087in}{1.579165in}}%
\pgfpathlineto{\pgfqpoint{2.273292in}{1.567806in}}%
\pgfpathlineto{\pgfqpoint{2.274644in}{1.600939in}}%
\pgfpathlineto{\pgfqpoint{2.276473in}{1.585603in}}%
\pgfpathlineto{\pgfqpoint{2.277047in}{1.608259in}}%
\pgfpathlineto{\pgfqpoint{2.279708in}{1.660183in}}%
\pgfpathlineto{\pgfqpoint{2.281716in}{1.639445in}}%
\pgfpathlineto{\pgfqpoint{2.282309in}{1.601120in}}%
\pgfpathlineto{\pgfqpoint{2.286341in}{1.578189in}}%
\pgfpathlineto{\pgfqpoint{2.287785in}{1.525409in}}%
\pgfpathlineto{\pgfqpoint{2.290314in}{1.525557in}}%
\pgfpathlineto{\pgfqpoint{2.291380in}{1.487439in}}%
\pgfpathlineto{\pgfqpoint{2.292679in}{1.499690in}}%
\pgfpathlineto{\pgfqpoint{2.295231in}{1.481887in}}%
\pgfpathlineto{\pgfqpoint{2.297478in}{1.433279in}}%
\pgfpathlineto{\pgfqpoint{2.299321in}{1.484312in}}%
\pgfpathlineto{\pgfqpoint{2.300738in}{1.470086in}}%
\pgfpathlineto{\pgfqpoint{2.303129in}{1.528180in}}%
\pgfpathlineto{\pgfqpoint{2.304260in}{1.499080in}}%
\pgfpathlineto{\pgfqpoint{2.307082in}{1.551165in}}%
\pgfpathlineto{\pgfqpoint{2.308111in}{1.514450in}}%
\pgfpathlineto{\pgfqpoint{2.310105in}{1.532408in}}%
\pgfpathlineto{\pgfqpoint{2.312502in}{1.488322in}}%
\pgfpathlineto{\pgfqpoint{2.313568in}{1.511890in}}%
\pgfpathlineto{\pgfqpoint{2.314340in}{1.487796in}}%
\pgfpathlineto{\pgfqpoint{2.314340in}{1.487796in}}%
\pgfusepath{stroke}%
\end{pgfscope}%
\begin{pgfscope}%
\pgfsetrectcap%
\pgfsetmiterjoin%
\pgfsetlinewidth{0.803000pt}%
\definecolor{currentstroke}{rgb}{0.000000,0.000000,0.000000}%
\pgfsetstrokecolor{currentstroke}%
\pgfsetdash{}{0pt}%
\pgfpathmoveto{\pgfqpoint{0.530716in}{0.416448in}}%
\pgfpathlineto{\pgfqpoint{0.530716in}{1.789039in}}%
\pgfusepath{stroke}%
\end{pgfscope}%
\begin{pgfscope}%
\pgfsetrectcap%
\pgfsetmiterjoin%
\pgfsetlinewidth{0.803000pt}%
\definecolor{currentstroke}{rgb}{0.000000,0.000000,0.000000}%
\pgfsetstrokecolor{currentstroke}%
\pgfsetdash{}{0pt}%
\pgfpathmoveto{\pgfqpoint{2.399275in}{0.416448in}}%
\pgfpathlineto{\pgfqpoint{2.399275in}{1.789039in}}%
\pgfusepath{stroke}%
\end{pgfscope}%
\begin{pgfscope}%
\pgfsetrectcap%
\pgfsetmiterjoin%
\pgfsetlinewidth{0.803000pt}%
\definecolor{currentstroke}{rgb}{0.000000,0.000000,0.000000}%
\pgfsetstrokecolor{currentstroke}%
\pgfsetdash{}{0pt}%
\pgfpathmoveto{\pgfqpoint{0.530716in}{0.416447in}}%
\pgfpathlineto{\pgfqpoint{2.399275in}{0.416447in}}%
\pgfusepath{stroke}%
\end{pgfscope}%
\begin{pgfscope}%
\pgfsetrectcap%
\pgfsetmiterjoin%
\pgfsetlinewidth{0.803000pt}%
\definecolor{currentstroke}{rgb}{0.000000,0.000000,0.000000}%
\pgfsetstrokecolor{currentstroke}%
\pgfsetdash{}{0pt}%
\pgfpathmoveto{\pgfqpoint{0.530716in}{1.789039in}}%
\pgfpathlineto{\pgfqpoint{2.399275in}{1.789039in}}%
\pgfusepath{stroke}%
\end{pgfscope}%
\begin{pgfscope}%
\pgfsetbuttcap%
\pgfsetmiterjoin%
\definecolor{currentfill}{rgb}{1.000000,1.000000,1.000000}%
\pgfsetfillcolor{currentfill}%
\pgfsetfillopacity{0.800000}%
\pgfsetlinewidth{1.003750pt}%
\definecolor{currentstroke}{rgb}{0.800000,0.800000,0.800000}%
\pgfsetstrokecolor{currentstroke}%
\pgfsetstrokeopacity{0.800000}%
\pgfsetdash{}{0pt}%
\pgfpathmoveto{\pgfqpoint{0.608494in}{1.545261in}}%
\pgfpathlineto{\pgfqpoint{1.673716in}{1.545261in}}%
\pgfpathquadraticcurveto{\pgfqpoint{1.695938in}{1.545261in}}{\pgfqpoint{1.695938in}{1.567483in}}%
\pgfpathlineto{\pgfqpoint{1.695938in}{1.711261in}}%
\pgfpathquadraticcurveto{\pgfqpoint{1.695938in}{1.733483in}}{\pgfqpoint{1.673716in}{1.733483in}}%
\pgfpathlineto{\pgfqpoint{0.608494in}{1.733483in}}%
\pgfpathquadraticcurveto{\pgfqpoint{0.586272in}{1.733483in}}{\pgfqpoint{0.586272in}{1.711261in}}%
\pgfpathlineto{\pgfqpoint{0.586272in}{1.567483in}}%
\pgfpathquadraticcurveto{\pgfqpoint{0.586272in}{1.545261in}}{\pgfqpoint{0.608494in}{1.545261in}}%
\pgfpathlineto{\pgfqpoint{0.608494in}{1.545261in}}%
\pgfpathclose%
\pgfusepath{stroke,fill}%
\end{pgfscope}%
\begin{pgfscope}%
\pgfsetrectcap%
\pgfsetroundjoin%
\pgfsetlinewidth{1.505625pt}%
\definecolor{currentstroke}{rgb}{0.835294,0.368627,0.000000}%
\pgfsetstrokecolor{currentstroke}%
\pgfsetdash{}{0pt}%
\pgfpathmoveto{\pgfqpoint{0.630716in}{1.650150in}}%
\pgfpathlineto{\pgfqpoint{0.741827in}{1.650150in}}%
\pgfpathlineto{\pgfqpoint{0.852938in}{1.650150in}}%
\pgfusepath{stroke}%
\end{pgfscope}%
\begin{pgfscope}%
\definecolor{textcolor}{rgb}{0.000000,0.000000,0.000000}%
\pgfsetstrokecolor{textcolor}%
\pgfsetfillcolor{textcolor}%
\pgftext[x=0.941827in,y=1.611261in,left,base]{\color{textcolor}{\rmfamily\fontsize{8.000000}{9.600000}\selectfont\catcode`\^=\active\def^{\ifmmode\sp\else\^{}\fi}\catcode`\%=\active\def%{\%}Random walk}}%
\end{pgfscope}%
\end{pgfpicture}%
\makeatother%
\endgroup%
% data/simulations/sim_allan_variance_example.py
        } % scalebox
        \caption{Random walk}
    \end{subfigure}
    \caption{Three separate noise components that were summed together to simulate a typical noise source.}
    \label{fig:adev_example_noise_types}
\end{figure}

The three time series shown in figure \ref{fig:adev_example_noise_types} were sequentially generated using a fixed seed for the random number generator to ensure repeatability as long as the order of creation is kept the same. For generating the noise, the algorithm presented by \citeauthor{noise_generation} \cite{noise_generation}, implemented in the \textit{AllanTools} Python library was used. The noise strength parameters were deliberately chosen in such a way that both the white noise and the random walk part have more noise power than the flicker noise. This allows to distinguish them in the following plots at both extremes of the frequency scale and time scale. Finally, the three types of noise data were summed together to give the combined signal, which is shown in figure \ref{fig:adev_example_time}, again downsampled using LTTB. The summed series clearly shows the white noise content and it is possible to deduce either flicker or random walk noise, but it is highly obscured due to the amount of white noise. Using only the time domain plot makes it very hard to clearly distinguish the type of noise present, let alone estimate the individual noise power of the three components. Therefore, a different analysis tool is called for.
\begin{figure}[ht]
    \centering
    %% Creator: Matplotlib, PGF backend
%%
%% To include the figure in your LaTeX document, write
%%   \input{<filename>.pgf}
%%
%% Make sure the required packages are loaded in your preamble
%%   \usepackage{pgf}
%%
%% Also ensure that all the required font packages are loaded; for instance,
%% the lmodern package is sometimes necessary when using math font.
%%   \usepackage{lmodern}
%%
%% Figures using additional raster images can only be included by \input if
%% they are in the same directory as the main LaTeX file. For loading figures
%% from other directories you can use the `import` package
%%   \usepackage{import}
%%
%% and then include the figures with
%%   \import{<path to file>}{<filename>.pgf}
%%
%% Matplotlib used the following preamble
%%   \def\mathdefault#1{#1}
%%   \everymath=\expandafter{\the\everymath\displaystyle}
%%   \usepackage{siunitx}
%%   \sisetup{per-mode = symbol}%
%%   \ifdefined\pdftexversion\else  % non-pdftex case.
%%     \usepackage{fontspec}
%%   \fi
%%   \makeatletter\@ifpackageloaded{underscore}{}{\usepackage[strings]{underscore}}\makeatother
%%
\begingroup%
\makeatletter%
\begin{pgfpicture}%
\pgfpathrectangle{\pgfpointorigin}{\pgfqpoint{4.068242in}{2.514312in}}%
\pgfusepath{use as bounding box, clip}%
\begin{pgfscope}%
\pgfsetbuttcap%
\pgfsetmiterjoin%
\definecolor{currentfill}{rgb}{1.000000,1.000000,1.000000}%
\pgfsetfillcolor{currentfill}%
\pgfsetlinewidth{0.000000pt}%
\definecolor{currentstroke}{rgb}{1.000000,1.000000,1.000000}%
\pgfsetstrokecolor{currentstroke}%
\pgfsetdash{}{0pt}%
\pgfpathmoveto{\pgfqpoint{0.000000in}{0.000000in}}%
\pgfpathlineto{\pgfqpoint{4.068242in}{0.000000in}}%
\pgfpathlineto{\pgfqpoint{4.068242in}{2.514312in}}%
\pgfpathlineto{\pgfqpoint{0.000000in}{2.514312in}}%
\pgfpathlineto{\pgfqpoint{0.000000in}{0.000000in}}%
\pgfpathclose%
\pgfusepath{fill}%
\end{pgfscope}%
\begin{pgfscope}%
\pgfsetbuttcap%
\pgfsetmiterjoin%
\definecolor{currentfill}{rgb}{1.000000,1.000000,1.000000}%
\pgfsetfillcolor{currentfill}%
\pgfsetlinewidth{0.000000pt}%
\definecolor{currentstroke}{rgb}{0.000000,0.000000,0.000000}%
\pgfsetstrokecolor{currentstroke}%
\pgfsetstrokeopacity{0.000000}%
\pgfsetdash{}{0pt}%
\pgfpathmoveto{\pgfqpoint{0.589745in}{0.416447in}}%
\pgfpathlineto{\pgfqpoint{3.988913in}{0.416447in}}%
\pgfpathlineto{\pgfqpoint{3.988913in}{2.472642in}}%
\pgfpathlineto{\pgfqpoint{0.589745in}{2.472642in}}%
\pgfpathlineto{\pgfqpoint{0.589745in}{0.416447in}}%
\pgfpathclose%
\pgfusepath{fill}%
\end{pgfscope}%
\begin{pgfscope}%
\pgfpathrectangle{\pgfqpoint{0.589745in}{0.416447in}}{\pgfqpoint{3.399168in}{2.056194in}}%
\pgfusepath{clip}%
\pgfsetrectcap%
\pgfsetroundjoin%
\pgfsetlinewidth{0.803000pt}%
\definecolor{currentstroke}{rgb}{0.450000,0.450000,0.450000}%
\pgfsetstrokecolor{currentstroke}%
\pgfsetdash{}{0pt}%
\pgfpathmoveto{\pgfqpoint{0.744252in}{0.416447in}}%
\pgfpathlineto{\pgfqpoint{0.744252in}{2.472642in}}%
\pgfusepath{stroke}%
\end{pgfscope}%
\begin{pgfscope}%
\pgfsetbuttcap%
\pgfsetroundjoin%
\definecolor{currentfill}{rgb}{0.000000,0.000000,0.000000}%
\pgfsetfillcolor{currentfill}%
\pgfsetlinewidth{0.803000pt}%
\definecolor{currentstroke}{rgb}{0.000000,0.000000,0.000000}%
\pgfsetstrokecolor{currentstroke}%
\pgfsetdash{}{0pt}%
\pgfsys@defobject{currentmarker}{\pgfqpoint{0.000000in}{-0.048611in}}{\pgfqpoint{0.000000in}{0.000000in}}{%
\pgfpathmoveto{\pgfqpoint{0.000000in}{0.000000in}}%
\pgfpathlineto{\pgfqpoint{0.000000in}{-0.048611in}}%
\pgfusepath{stroke,fill}%
}%
\begin{pgfscope}%
\pgfsys@transformshift{0.744252in}{0.416447in}%
\pgfsys@useobject{currentmarker}{}%
\end{pgfscope}%
\end{pgfscope}%
\begin{pgfscope}%
\definecolor{textcolor}{rgb}{0.000000,0.000000,0.000000}%
\pgfsetstrokecolor{textcolor}%
\pgfsetfillcolor{textcolor}%
\pgftext[x=0.744252in,y=0.319225in,,top]{\color{textcolor}{\rmfamily\fontsize{8.000000}{9.600000}\selectfont\catcode`\^=\active\def^{\ifmmode\sp\else\^{}\fi}\catcode`\%=\active\def%{\%}$\mathdefault{0}$}}%
\end{pgfscope}%
\begin{pgfscope}%
\pgfpathrectangle{\pgfqpoint{0.589745in}{0.416447in}}{\pgfqpoint{3.399168in}{2.056194in}}%
\pgfusepath{clip}%
\pgfsetrectcap%
\pgfsetroundjoin%
\pgfsetlinewidth{0.803000pt}%
\definecolor{currentstroke}{rgb}{0.450000,0.450000,0.450000}%
\pgfsetstrokecolor{currentstroke}%
\pgfsetdash{}{0pt}%
\pgfpathmoveto{\pgfqpoint{1.204721in}{0.416447in}}%
\pgfpathlineto{\pgfqpoint{1.204721in}{2.472642in}}%
\pgfusepath{stroke}%
\end{pgfscope}%
\begin{pgfscope}%
\pgfsetbuttcap%
\pgfsetroundjoin%
\definecolor{currentfill}{rgb}{0.000000,0.000000,0.000000}%
\pgfsetfillcolor{currentfill}%
\pgfsetlinewidth{0.803000pt}%
\definecolor{currentstroke}{rgb}{0.000000,0.000000,0.000000}%
\pgfsetstrokecolor{currentstroke}%
\pgfsetdash{}{0pt}%
\pgfsys@defobject{currentmarker}{\pgfqpoint{0.000000in}{-0.048611in}}{\pgfqpoint{0.000000in}{0.000000in}}{%
\pgfpathmoveto{\pgfqpoint{0.000000in}{0.000000in}}%
\pgfpathlineto{\pgfqpoint{0.000000in}{-0.048611in}}%
\pgfusepath{stroke,fill}%
}%
\begin{pgfscope}%
\pgfsys@transformshift{1.204721in}{0.416447in}%
\pgfsys@useobject{currentmarker}{}%
\end{pgfscope}%
\end{pgfscope}%
\begin{pgfscope}%
\definecolor{textcolor}{rgb}{0.000000,0.000000,0.000000}%
\pgfsetstrokecolor{textcolor}%
\pgfsetfillcolor{textcolor}%
\pgftext[x=1.204721in,y=0.319225in,,top]{\color{textcolor}{\rmfamily\fontsize{8.000000}{9.600000}\selectfont\catcode`\^=\active\def^{\ifmmode\sp\else\^{}\fi}\catcode`\%=\active\def%{\%}$\mathdefault{5}$}}%
\end{pgfscope}%
\begin{pgfscope}%
\pgfpathrectangle{\pgfqpoint{0.589745in}{0.416447in}}{\pgfqpoint{3.399168in}{2.056194in}}%
\pgfusepath{clip}%
\pgfsetrectcap%
\pgfsetroundjoin%
\pgfsetlinewidth{0.803000pt}%
\definecolor{currentstroke}{rgb}{0.450000,0.450000,0.450000}%
\pgfsetstrokecolor{currentstroke}%
\pgfsetdash{}{0pt}%
\pgfpathmoveto{\pgfqpoint{1.665190in}{0.416447in}}%
\pgfpathlineto{\pgfqpoint{1.665190in}{2.472642in}}%
\pgfusepath{stroke}%
\end{pgfscope}%
\begin{pgfscope}%
\pgfsetbuttcap%
\pgfsetroundjoin%
\definecolor{currentfill}{rgb}{0.000000,0.000000,0.000000}%
\pgfsetfillcolor{currentfill}%
\pgfsetlinewidth{0.803000pt}%
\definecolor{currentstroke}{rgb}{0.000000,0.000000,0.000000}%
\pgfsetstrokecolor{currentstroke}%
\pgfsetdash{}{0pt}%
\pgfsys@defobject{currentmarker}{\pgfqpoint{0.000000in}{-0.048611in}}{\pgfqpoint{0.000000in}{0.000000in}}{%
\pgfpathmoveto{\pgfqpoint{0.000000in}{0.000000in}}%
\pgfpathlineto{\pgfqpoint{0.000000in}{-0.048611in}}%
\pgfusepath{stroke,fill}%
}%
\begin{pgfscope}%
\pgfsys@transformshift{1.665190in}{0.416447in}%
\pgfsys@useobject{currentmarker}{}%
\end{pgfscope}%
\end{pgfscope}%
\begin{pgfscope}%
\definecolor{textcolor}{rgb}{0.000000,0.000000,0.000000}%
\pgfsetstrokecolor{textcolor}%
\pgfsetfillcolor{textcolor}%
\pgftext[x=1.665190in,y=0.319225in,,top]{\color{textcolor}{\rmfamily\fontsize{8.000000}{9.600000}\selectfont\catcode`\^=\active\def^{\ifmmode\sp\else\^{}\fi}\catcode`\%=\active\def%{\%}$\mathdefault{10}$}}%
\end{pgfscope}%
\begin{pgfscope}%
\pgfpathrectangle{\pgfqpoint{0.589745in}{0.416447in}}{\pgfqpoint{3.399168in}{2.056194in}}%
\pgfusepath{clip}%
\pgfsetrectcap%
\pgfsetroundjoin%
\pgfsetlinewidth{0.803000pt}%
\definecolor{currentstroke}{rgb}{0.450000,0.450000,0.450000}%
\pgfsetstrokecolor{currentstroke}%
\pgfsetdash{}{0pt}%
\pgfpathmoveto{\pgfqpoint{2.125658in}{0.416447in}}%
\pgfpathlineto{\pgfqpoint{2.125658in}{2.472642in}}%
\pgfusepath{stroke}%
\end{pgfscope}%
\begin{pgfscope}%
\pgfsetbuttcap%
\pgfsetroundjoin%
\definecolor{currentfill}{rgb}{0.000000,0.000000,0.000000}%
\pgfsetfillcolor{currentfill}%
\pgfsetlinewidth{0.803000pt}%
\definecolor{currentstroke}{rgb}{0.000000,0.000000,0.000000}%
\pgfsetstrokecolor{currentstroke}%
\pgfsetdash{}{0pt}%
\pgfsys@defobject{currentmarker}{\pgfqpoint{0.000000in}{-0.048611in}}{\pgfqpoint{0.000000in}{0.000000in}}{%
\pgfpathmoveto{\pgfqpoint{0.000000in}{0.000000in}}%
\pgfpathlineto{\pgfqpoint{0.000000in}{-0.048611in}}%
\pgfusepath{stroke,fill}%
}%
\begin{pgfscope}%
\pgfsys@transformshift{2.125658in}{0.416447in}%
\pgfsys@useobject{currentmarker}{}%
\end{pgfscope}%
\end{pgfscope}%
\begin{pgfscope}%
\definecolor{textcolor}{rgb}{0.000000,0.000000,0.000000}%
\pgfsetstrokecolor{textcolor}%
\pgfsetfillcolor{textcolor}%
\pgftext[x=2.125658in,y=0.319225in,,top]{\color{textcolor}{\rmfamily\fontsize{8.000000}{9.600000}\selectfont\catcode`\^=\active\def^{\ifmmode\sp\else\^{}\fi}\catcode`\%=\active\def%{\%}$\mathdefault{15}$}}%
\end{pgfscope}%
\begin{pgfscope}%
\pgfpathrectangle{\pgfqpoint{0.589745in}{0.416447in}}{\pgfqpoint{3.399168in}{2.056194in}}%
\pgfusepath{clip}%
\pgfsetrectcap%
\pgfsetroundjoin%
\pgfsetlinewidth{0.803000pt}%
\definecolor{currentstroke}{rgb}{0.450000,0.450000,0.450000}%
\pgfsetstrokecolor{currentstroke}%
\pgfsetdash{}{0pt}%
\pgfpathmoveto{\pgfqpoint{2.586127in}{0.416447in}}%
\pgfpathlineto{\pgfqpoint{2.586127in}{2.472642in}}%
\pgfusepath{stroke}%
\end{pgfscope}%
\begin{pgfscope}%
\pgfsetbuttcap%
\pgfsetroundjoin%
\definecolor{currentfill}{rgb}{0.000000,0.000000,0.000000}%
\pgfsetfillcolor{currentfill}%
\pgfsetlinewidth{0.803000pt}%
\definecolor{currentstroke}{rgb}{0.000000,0.000000,0.000000}%
\pgfsetstrokecolor{currentstroke}%
\pgfsetdash{}{0pt}%
\pgfsys@defobject{currentmarker}{\pgfqpoint{0.000000in}{-0.048611in}}{\pgfqpoint{0.000000in}{0.000000in}}{%
\pgfpathmoveto{\pgfqpoint{0.000000in}{0.000000in}}%
\pgfpathlineto{\pgfqpoint{0.000000in}{-0.048611in}}%
\pgfusepath{stroke,fill}%
}%
\begin{pgfscope}%
\pgfsys@transformshift{2.586127in}{0.416447in}%
\pgfsys@useobject{currentmarker}{}%
\end{pgfscope}%
\end{pgfscope}%
\begin{pgfscope}%
\definecolor{textcolor}{rgb}{0.000000,0.000000,0.000000}%
\pgfsetstrokecolor{textcolor}%
\pgfsetfillcolor{textcolor}%
\pgftext[x=2.586127in,y=0.319225in,,top]{\color{textcolor}{\rmfamily\fontsize{8.000000}{9.600000}\selectfont\catcode`\^=\active\def^{\ifmmode\sp\else\^{}\fi}\catcode`\%=\active\def%{\%}$\mathdefault{20}$}}%
\end{pgfscope}%
\begin{pgfscope}%
\pgfpathrectangle{\pgfqpoint{0.589745in}{0.416447in}}{\pgfqpoint{3.399168in}{2.056194in}}%
\pgfusepath{clip}%
\pgfsetrectcap%
\pgfsetroundjoin%
\pgfsetlinewidth{0.803000pt}%
\definecolor{currentstroke}{rgb}{0.450000,0.450000,0.450000}%
\pgfsetstrokecolor{currentstroke}%
\pgfsetdash{}{0pt}%
\pgfpathmoveto{\pgfqpoint{3.046596in}{0.416447in}}%
\pgfpathlineto{\pgfqpoint{3.046596in}{2.472642in}}%
\pgfusepath{stroke}%
\end{pgfscope}%
\begin{pgfscope}%
\pgfsetbuttcap%
\pgfsetroundjoin%
\definecolor{currentfill}{rgb}{0.000000,0.000000,0.000000}%
\pgfsetfillcolor{currentfill}%
\pgfsetlinewidth{0.803000pt}%
\definecolor{currentstroke}{rgb}{0.000000,0.000000,0.000000}%
\pgfsetstrokecolor{currentstroke}%
\pgfsetdash{}{0pt}%
\pgfsys@defobject{currentmarker}{\pgfqpoint{0.000000in}{-0.048611in}}{\pgfqpoint{0.000000in}{0.000000in}}{%
\pgfpathmoveto{\pgfqpoint{0.000000in}{0.000000in}}%
\pgfpathlineto{\pgfqpoint{0.000000in}{-0.048611in}}%
\pgfusepath{stroke,fill}%
}%
\begin{pgfscope}%
\pgfsys@transformshift{3.046596in}{0.416447in}%
\pgfsys@useobject{currentmarker}{}%
\end{pgfscope}%
\end{pgfscope}%
\begin{pgfscope}%
\definecolor{textcolor}{rgb}{0.000000,0.000000,0.000000}%
\pgfsetstrokecolor{textcolor}%
\pgfsetfillcolor{textcolor}%
\pgftext[x=3.046596in,y=0.319225in,,top]{\color{textcolor}{\rmfamily\fontsize{8.000000}{9.600000}\selectfont\catcode`\^=\active\def^{\ifmmode\sp\else\^{}\fi}\catcode`\%=\active\def%{\%}$\mathdefault{25}$}}%
\end{pgfscope}%
\begin{pgfscope}%
\pgfpathrectangle{\pgfqpoint{0.589745in}{0.416447in}}{\pgfqpoint{3.399168in}{2.056194in}}%
\pgfusepath{clip}%
\pgfsetrectcap%
\pgfsetroundjoin%
\pgfsetlinewidth{0.803000pt}%
\definecolor{currentstroke}{rgb}{0.450000,0.450000,0.450000}%
\pgfsetstrokecolor{currentstroke}%
\pgfsetdash{}{0pt}%
\pgfpathmoveto{\pgfqpoint{3.507065in}{0.416447in}}%
\pgfpathlineto{\pgfqpoint{3.507065in}{2.472642in}}%
\pgfusepath{stroke}%
\end{pgfscope}%
\begin{pgfscope}%
\pgfsetbuttcap%
\pgfsetroundjoin%
\definecolor{currentfill}{rgb}{0.000000,0.000000,0.000000}%
\pgfsetfillcolor{currentfill}%
\pgfsetlinewidth{0.803000pt}%
\definecolor{currentstroke}{rgb}{0.000000,0.000000,0.000000}%
\pgfsetstrokecolor{currentstroke}%
\pgfsetdash{}{0pt}%
\pgfsys@defobject{currentmarker}{\pgfqpoint{0.000000in}{-0.048611in}}{\pgfqpoint{0.000000in}{0.000000in}}{%
\pgfpathmoveto{\pgfqpoint{0.000000in}{0.000000in}}%
\pgfpathlineto{\pgfqpoint{0.000000in}{-0.048611in}}%
\pgfusepath{stroke,fill}%
}%
\begin{pgfscope}%
\pgfsys@transformshift{3.507065in}{0.416447in}%
\pgfsys@useobject{currentmarker}{}%
\end{pgfscope}%
\end{pgfscope}%
\begin{pgfscope}%
\definecolor{textcolor}{rgb}{0.000000,0.000000,0.000000}%
\pgfsetstrokecolor{textcolor}%
\pgfsetfillcolor{textcolor}%
\pgftext[x=3.507065in,y=0.319225in,,top]{\color{textcolor}{\rmfamily\fontsize{8.000000}{9.600000}\selectfont\catcode`\^=\active\def^{\ifmmode\sp\else\^{}\fi}\catcode`\%=\active\def%{\%}$\mathdefault{30}$}}%
\end{pgfscope}%
\begin{pgfscope}%
\pgfpathrectangle{\pgfqpoint{0.589745in}{0.416447in}}{\pgfqpoint{3.399168in}{2.056194in}}%
\pgfusepath{clip}%
\pgfsetrectcap%
\pgfsetroundjoin%
\pgfsetlinewidth{0.803000pt}%
\definecolor{currentstroke}{rgb}{0.450000,0.450000,0.450000}%
\pgfsetstrokecolor{currentstroke}%
\pgfsetdash{}{0pt}%
\pgfpathmoveto{\pgfqpoint{3.967533in}{0.416447in}}%
\pgfpathlineto{\pgfqpoint{3.967533in}{2.472642in}}%
\pgfusepath{stroke}%
\end{pgfscope}%
\begin{pgfscope}%
\pgfsetbuttcap%
\pgfsetroundjoin%
\definecolor{currentfill}{rgb}{0.000000,0.000000,0.000000}%
\pgfsetfillcolor{currentfill}%
\pgfsetlinewidth{0.803000pt}%
\definecolor{currentstroke}{rgb}{0.000000,0.000000,0.000000}%
\pgfsetstrokecolor{currentstroke}%
\pgfsetdash{}{0pt}%
\pgfsys@defobject{currentmarker}{\pgfqpoint{0.000000in}{-0.048611in}}{\pgfqpoint{0.000000in}{0.000000in}}{%
\pgfpathmoveto{\pgfqpoint{0.000000in}{0.000000in}}%
\pgfpathlineto{\pgfqpoint{0.000000in}{-0.048611in}}%
\pgfusepath{stroke,fill}%
}%
\begin{pgfscope}%
\pgfsys@transformshift{3.967533in}{0.416447in}%
\pgfsys@useobject{currentmarker}{}%
\end{pgfscope}%
\end{pgfscope}%
\begin{pgfscope}%
\definecolor{textcolor}{rgb}{0.000000,0.000000,0.000000}%
\pgfsetstrokecolor{textcolor}%
\pgfsetfillcolor{textcolor}%
\pgftext[x=3.967533in,y=0.319225in,,top]{\color{textcolor}{\rmfamily\fontsize{8.000000}{9.600000}\selectfont\catcode`\^=\active\def^{\ifmmode\sp\else\^{}\fi}\catcode`\%=\active\def%{\%}$\mathdefault{35}$}}%
\end{pgfscope}%
\begin{pgfscope}%
\definecolor{textcolor}{rgb}{0.000000,0.000000,0.000000}%
\pgfsetstrokecolor{textcolor}%
\pgfsetfillcolor{textcolor}%
\pgftext[x=2.289329in,y=0.165003in,,top]{\color{textcolor}{\rmfamily\fontsize{10.000000}{12.000000}\selectfont\catcode`\^=\active\def^{\ifmmode\sp\else\^{}\fi}\catcode`\%=\active\def%{\%}Time in $\unit{\second}$}}%
\end{pgfscope}%
\begin{pgfscope}%
\pgfpathrectangle{\pgfqpoint{0.589745in}{0.416447in}}{\pgfqpoint{3.399168in}{2.056194in}}%
\pgfusepath{clip}%
\pgfsetrectcap%
\pgfsetroundjoin%
\pgfsetlinewidth{0.803000pt}%
\definecolor{currentstroke}{rgb}{0.450000,0.450000,0.450000}%
\pgfsetstrokecolor{currentstroke}%
\pgfsetdash{}{0pt}%
\pgfpathmoveto{\pgfqpoint{0.589745in}{0.484425in}}%
\pgfpathlineto{\pgfqpoint{3.988913in}{0.484425in}}%
\pgfusepath{stroke}%
\end{pgfscope}%
\begin{pgfscope}%
\pgfsetbuttcap%
\pgfsetroundjoin%
\definecolor{currentfill}{rgb}{0.000000,0.000000,0.000000}%
\pgfsetfillcolor{currentfill}%
\pgfsetlinewidth{0.803000pt}%
\definecolor{currentstroke}{rgb}{0.000000,0.000000,0.000000}%
\pgfsetstrokecolor{currentstroke}%
\pgfsetdash{}{0pt}%
\pgfsys@defobject{currentmarker}{\pgfqpoint{-0.048611in}{0.000000in}}{\pgfqpoint{-0.000000in}{0.000000in}}{%
\pgfpathmoveto{\pgfqpoint{-0.000000in}{0.000000in}}%
\pgfpathlineto{\pgfqpoint{-0.048611in}{0.000000in}}%
\pgfusepath{stroke,fill}%
}%
\begin{pgfscope}%
\pgfsys@transformshift{0.589745in}{0.484425in}%
\pgfsys@useobject{currentmarker}{}%
\end{pgfscope}%
\end{pgfscope}%
\begin{pgfscope}%
\definecolor{textcolor}{rgb}{0.000000,0.000000,0.000000}%
\pgfsetstrokecolor{textcolor}%
\pgfsetfillcolor{textcolor}%
\pgftext[x=0.223614in, y=0.445869in, left, base]{\color{textcolor}{\rmfamily\fontsize{8.000000}{9.600000}\selectfont\catcode`\^=\active\def^{\ifmmode\sp\else\^{}\fi}\catcode`\%=\active\def%{\%}$\mathdefault{\ensuremath{-}200}$}}%
\end{pgfscope}%
\begin{pgfscope}%
\pgfpathrectangle{\pgfqpoint{0.589745in}{0.416447in}}{\pgfqpoint{3.399168in}{2.056194in}}%
\pgfusepath{clip}%
\pgfsetrectcap%
\pgfsetroundjoin%
\pgfsetlinewidth{0.803000pt}%
\definecolor{currentstroke}{rgb}{0.450000,0.450000,0.450000}%
\pgfsetstrokecolor{currentstroke}%
\pgfsetdash{}{0pt}%
\pgfpathmoveto{\pgfqpoint{0.589745in}{0.888753in}}%
\pgfpathlineto{\pgfqpoint{3.988913in}{0.888753in}}%
\pgfusepath{stroke}%
\end{pgfscope}%
\begin{pgfscope}%
\pgfsetbuttcap%
\pgfsetroundjoin%
\definecolor{currentfill}{rgb}{0.000000,0.000000,0.000000}%
\pgfsetfillcolor{currentfill}%
\pgfsetlinewidth{0.803000pt}%
\definecolor{currentstroke}{rgb}{0.000000,0.000000,0.000000}%
\pgfsetstrokecolor{currentstroke}%
\pgfsetdash{}{0pt}%
\pgfsys@defobject{currentmarker}{\pgfqpoint{-0.048611in}{0.000000in}}{\pgfqpoint{-0.000000in}{0.000000in}}{%
\pgfpathmoveto{\pgfqpoint{-0.000000in}{0.000000in}}%
\pgfpathlineto{\pgfqpoint{-0.048611in}{0.000000in}}%
\pgfusepath{stroke,fill}%
}%
\begin{pgfscope}%
\pgfsys@transformshift{0.589745in}{0.888753in}%
\pgfsys@useobject{currentmarker}{}%
\end{pgfscope}%
\end{pgfscope}%
\begin{pgfscope}%
\definecolor{textcolor}{rgb}{0.000000,0.000000,0.000000}%
\pgfsetstrokecolor{textcolor}%
\pgfsetfillcolor{textcolor}%
\pgftext[x=0.223614in, y=0.850197in, left, base]{\color{textcolor}{\rmfamily\fontsize{8.000000}{9.600000}\selectfont\catcode`\^=\active\def^{\ifmmode\sp\else\^{}\fi}\catcode`\%=\active\def%{\%}$\mathdefault{\ensuremath{-}100}$}}%
\end{pgfscope}%
\begin{pgfscope}%
\pgfpathrectangle{\pgfqpoint{0.589745in}{0.416447in}}{\pgfqpoint{3.399168in}{2.056194in}}%
\pgfusepath{clip}%
\pgfsetrectcap%
\pgfsetroundjoin%
\pgfsetlinewidth{0.803000pt}%
\definecolor{currentstroke}{rgb}{0.450000,0.450000,0.450000}%
\pgfsetstrokecolor{currentstroke}%
\pgfsetdash{}{0pt}%
\pgfpathmoveto{\pgfqpoint{0.589745in}{1.293081in}}%
\pgfpathlineto{\pgfqpoint{3.988913in}{1.293081in}}%
\pgfusepath{stroke}%
\end{pgfscope}%
\begin{pgfscope}%
\pgfsetbuttcap%
\pgfsetroundjoin%
\definecolor{currentfill}{rgb}{0.000000,0.000000,0.000000}%
\pgfsetfillcolor{currentfill}%
\pgfsetlinewidth{0.803000pt}%
\definecolor{currentstroke}{rgb}{0.000000,0.000000,0.000000}%
\pgfsetstrokecolor{currentstroke}%
\pgfsetdash{}{0pt}%
\pgfsys@defobject{currentmarker}{\pgfqpoint{-0.048611in}{0.000000in}}{\pgfqpoint{-0.000000in}{0.000000in}}{%
\pgfpathmoveto{\pgfqpoint{-0.000000in}{0.000000in}}%
\pgfpathlineto{\pgfqpoint{-0.048611in}{0.000000in}}%
\pgfusepath{stroke,fill}%
}%
\begin{pgfscope}%
\pgfsys@transformshift{0.589745in}{1.293081in}%
\pgfsys@useobject{currentmarker}{}%
\end{pgfscope}%
\end{pgfscope}%
\begin{pgfscope}%
\definecolor{textcolor}{rgb}{0.000000,0.000000,0.000000}%
\pgfsetstrokecolor{textcolor}%
\pgfsetfillcolor{textcolor}%
\pgftext[x=0.433494in, y=1.254525in, left, base]{\color{textcolor}{\rmfamily\fontsize{8.000000}{9.600000}\selectfont\catcode`\^=\active\def^{\ifmmode\sp\else\^{}\fi}\catcode`\%=\active\def%{\%}$\mathdefault{0}$}}%
\end{pgfscope}%
\begin{pgfscope}%
\pgfpathrectangle{\pgfqpoint{0.589745in}{0.416447in}}{\pgfqpoint{3.399168in}{2.056194in}}%
\pgfusepath{clip}%
\pgfsetrectcap%
\pgfsetroundjoin%
\pgfsetlinewidth{0.803000pt}%
\definecolor{currentstroke}{rgb}{0.450000,0.450000,0.450000}%
\pgfsetstrokecolor{currentstroke}%
\pgfsetdash{}{0pt}%
\pgfpathmoveto{\pgfqpoint{0.589745in}{1.697409in}}%
\pgfpathlineto{\pgfqpoint{3.988913in}{1.697409in}}%
\pgfusepath{stroke}%
\end{pgfscope}%
\begin{pgfscope}%
\pgfsetbuttcap%
\pgfsetroundjoin%
\definecolor{currentfill}{rgb}{0.000000,0.000000,0.000000}%
\pgfsetfillcolor{currentfill}%
\pgfsetlinewidth{0.803000pt}%
\definecolor{currentstroke}{rgb}{0.000000,0.000000,0.000000}%
\pgfsetstrokecolor{currentstroke}%
\pgfsetdash{}{0pt}%
\pgfsys@defobject{currentmarker}{\pgfqpoint{-0.048611in}{0.000000in}}{\pgfqpoint{-0.000000in}{0.000000in}}{%
\pgfpathmoveto{\pgfqpoint{-0.000000in}{0.000000in}}%
\pgfpathlineto{\pgfqpoint{-0.048611in}{0.000000in}}%
\pgfusepath{stroke,fill}%
}%
\begin{pgfscope}%
\pgfsys@transformshift{0.589745in}{1.697409in}%
\pgfsys@useobject{currentmarker}{}%
\end{pgfscope}%
\end{pgfscope}%
\begin{pgfscope}%
\definecolor{textcolor}{rgb}{0.000000,0.000000,0.000000}%
\pgfsetstrokecolor{textcolor}%
\pgfsetfillcolor{textcolor}%
\pgftext[x=0.315437in, y=1.658853in, left, base]{\color{textcolor}{\rmfamily\fontsize{8.000000}{9.600000}\selectfont\catcode`\^=\active\def^{\ifmmode\sp\else\^{}\fi}\catcode`\%=\active\def%{\%}$\mathdefault{100}$}}%
\end{pgfscope}%
\begin{pgfscope}%
\pgfpathrectangle{\pgfqpoint{0.589745in}{0.416447in}}{\pgfqpoint{3.399168in}{2.056194in}}%
\pgfusepath{clip}%
\pgfsetrectcap%
\pgfsetroundjoin%
\pgfsetlinewidth{0.803000pt}%
\definecolor{currentstroke}{rgb}{0.450000,0.450000,0.450000}%
\pgfsetstrokecolor{currentstroke}%
\pgfsetdash{}{0pt}%
\pgfpathmoveto{\pgfqpoint{0.589745in}{2.101736in}}%
\pgfpathlineto{\pgfqpoint{3.988913in}{2.101736in}}%
\pgfusepath{stroke}%
\end{pgfscope}%
\begin{pgfscope}%
\pgfsetbuttcap%
\pgfsetroundjoin%
\definecolor{currentfill}{rgb}{0.000000,0.000000,0.000000}%
\pgfsetfillcolor{currentfill}%
\pgfsetlinewidth{0.803000pt}%
\definecolor{currentstroke}{rgb}{0.000000,0.000000,0.000000}%
\pgfsetstrokecolor{currentstroke}%
\pgfsetdash{}{0pt}%
\pgfsys@defobject{currentmarker}{\pgfqpoint{-0.048611in}{0.000000in}}{\pgfqpoint{-0.000000in}{0.000000in}}{%
\pgfpathmoveto{\pgfqpoint{-0.000000in}{0.000000in}}%
\pgfpathlineto{\pgfqpoint{-0.048611in}{0.000000in}}%
\pgfusepath{stroke,fill}%
}%
\begin{pgfscope}%
\pgfsys@transformshift{0.589745in}{2.101736in}%
\pgfsys@useobject{currentmarker}{}%
\end{pgfscope}%
\end{pgfscope}%
\begin{pgfscope}%
\definecolor{textcolor}{rgb}{0.000000,0.000000,0.000000}%
\pgfsetstrokecolor{textcolor}%
\pgfsetfillcolor{textcolor}%
\pgftext[x=0.315437in, y=2.063181in, left, base]{\color{textcolor}{\rmfamily\fontsize{8.000000}{9.600000}\selectfont\catcode`\^=\active\def^{\ifmmode\sp\else\^{}\fi}\catcode`\%=\active\def%{\%}$\mathdefault{200}$}}%
\end{pgfscope}%
\begin{pgfscope}%
\definecolor{textcolor}{rgb}{0.000000,0.000000,0.000000}%
\pgfsetstrokecolor{textcolor}%
\pgfsetfillcolor{textcolor}%
\pgftext[x=0.168059in,y=1.444545in,,bottom,rotate=90.000000]{\color{textcolor}{\rmfamily\fontsize{10.000000}{12.000000}\selectfont\catcode`\^=\active\def^{\ifmmode\sp\else\^{}\fi}\catcode`\%=\active\def%{\%}Ampl. in arb. unit}}%
\end{pgfscope}%
\begin{pgfscope}%
\pgfpathrectangle{\pgfqpoint{0.589745in}{0.416447in}}{\pgfqpoint{3.399168in}{2.056194in}}%
\pgfusepath{clip}%
\pgfsetrectcap%
\pgfsetroundjoin%
\pgfsetlinewidth{1.505625pt}%
\definecolor{currentstroke}{rgb}{0.337255,0.705882,0.913725}%
\pgfsetstrokecolor{currentstroke}%
\pgfsetdash{}{0pt}%
\pgfpathmoveto{\pgfqpoint{0.744252in}{1.272666in}}%
\pgfpathlineto{\pgfqpoint{0.745711in}{1.875251in}}%
\pgfpathlineto{\pgfqpoint{0.747412in}{0.788929in}}%
\pgfpathlineto{\pgfqpoint{0.750946in}{1.797723in}}%
\pgfpathlineto{\pgfqpoint{0.754011in}{0.799839in}}%
\pgfpathlineto{\pgfqpoint{0.757305in}{1.839454in}}%
\pgfpathlineto{\pgfqpoint{0.761069in}{0.654724in}}%
\pgfpathlineto{\pgfqpoint{0.762900in}{1.758460in}}%
\pgfpathlineto{\pgfqpoint{0.767171in}{0.655461in}}%
\pgfpathlineto{\pgfqpoint{0.769145in}{1.717794in}}%
\pgfpathlineto{\pgfqpoint{0.772172in}{0.773027in}}%
\pgfpathlineto{\pgfqpoint{0.775332in}{1.809775in}}%
\pgfpathlineto{\pgfqpoint{0.779740in}{0.766812in}}%
\pgfpathlineto{\pgfqpoint{0.781486in}{1.788505in}}%
\pgfpathlineto{\pgfqpoint{0.785806in}{0.582074in}}%
\pgfpathlineto{\pgfqpoint{0.787816in}{1.738982in}}%
\pgfpathlineto{\pgfqpoint{0.791611in}{0.760855in}}%
\pgfpathlineto{\pgfqpoint{0.794009in}{1.771789in}}%
\pgfpathlineto{\pgfqpoint{0.797592in}{0.760554in}}%
\pgfpathlineto{\pgfqpoint{0.800003in}{1.854767in}}%
\pgfpathlineto{\pgfqpoint{0.804489in}{0.745829in}}%
\pgfpathlineto{\pgfqpoint{0.806376in}{1.771757in}}%
\pgfpathlineto{\pgfqpoint{0.810190in}{0.754865in}}%
\pgfpathlineto{\pgfqpoint{0.812673in}{1.793594in}}%
\pgfpathlineto{\pgfqpoint{0.815735in}{0.894099in}}%
\pgfpathlineto{\pgfqpoint{0.818878in}{1.842742in}}%
\pgfpathlineto{\pgfqpoint{0.821706in}{0.919454in}}%
\pgfpathlineto{\pgfqpoint{0.826087in}{1.844562in}}%
\pgfpathlineto{\pgfqpoint{0.828337in}{0.745771in}}%
\pgfpathlineto{\pgfqpoint{0.831034in}{1.832079in}}%
\pgfpathlineto{\pgfqpoint{0.834154in}{0.849886in}}%
\pgfpathlineto{\pgfqpoint{0.837289in}{1.877521in}}%
\pgfpathlineto{\pgfqpoint{0.840325in}{0.847068in}}%
\pgfpathlineto{\pgfqpoint{0.843638in}{1.873066in}}%
\pgfpathlineto{\pgfqpoint{0.846627in}{0.838027in}}%
\pgfpathlineto{\pgfqpoint{0.850291in}{1.790477in}}%
\pgfpathlineto{\pgfqpoint{0.854113in}{0.756615in}}%
\pgfpathlineto{\pgfqpoint{0.856286in}{1.842955in}}%
\pgfpathlineto{\pgfqpoint{0.859065in}{0.874280in}}%
\pgfpathlineto{\pgfqpoint{0.861922in}{1.865573in}}%
\pgfpathlineto{\pgfqpoint{0.865117in}{0.839271in}}%
\pgfpathlineto{\pgfqpoint{0.868432in}{1.794452in}}%
\pgfpathlineto{\pgfqpoint{0.871783in}{0.756225in}}%
\pgfpathlineto{\pgfqpoint{0.874592in}{1.806603in}}%
\pgfpathlineto{\pgfqpoint{0.877749in}{0.799668in}}%
\pgfpathlineto{\pgfqpoint{0.881425in}{1.796014in}}%
\pgfpathlineto{\pgfqpoint{0.883738in}{0.752236in}}%
\pgfpathlineto{\pgfqpoint{0.887471in}{1.768290in}}%
\pgfpathlineto{\pgfqpoint{0.889870in}{0.791829in}}%
\pgfpathlineto{\pgfqpoint{0.893742in}{1.825907in}}%
\pgfpathlineto{\pgfqpoint{0.896432in}{0.687044in}}%
\pgfpathlineto{\pgfqpoint{0.899388in}{1.702669in}}%
\pgfpathlineto{\pgfqpoint{0.902188in}{0.752077in}}%
\pgfpathlineto{\pgfqpoint{0.906161in}{1.750280in}}%
\pgfpathlineto{\pgfqpoint{0.908983in}{0.671533in}}%
\pgfpathlineto{\pgfqpoint{0.913427in}{1.757597in}}%
\pgfpathlineto{\pgfqpoint{0.915024in}{0.743541in}}%
\pgfpathlineto{\pgfqpoint{0.917921in}{1.674567in}}%
\pgfpathlineto{\pgfqpoint{0.921679in}{0.635810in}}%
\pgfpathlineto{\pgfqpoint{0.924351in}{1.733782in}}%
\pgfpathlineto{\pgfqpoint{0.927357in}{0.637482in}}%
\pgfpathlineto{\pgfqpoint{0.930489in}{1.747196in}}%
\pgfpathlineto{\pgfqpoint{0.933403in}{0.733669in}}%
\pgfpathlineto{\pgfqpoint{0.936919in}{1.799536in}}%
\pgfpathlineto{\pgfqpoint{0.940998in}{0.653739in}}%
\pgfpathlineto{\pgfqpoint{0.942430in}{1.695509in}}%
\pgfpathlineto{\pgfqpoint{0.945815in}{0.798105in}}%
\pgfpathlineto{\pgfqpoint{0.948973in}{1.785973in}}%
\pgfpathlineto{\pgfqpoint{0.951990in}{0.778300in}}%
\pgfpathlineto{\pgfqpoint{0.954824in}{1.875265in}}%
\pgfpathlineto{\pgfqpoint{0.957989in}{0.779834in}}%
\pgfpathlineto{\pgfqpoint{0.961691in}{1.773552in}}%
\pgfpathlineto{\pgfqpoint{0.964319in}{0.781419in}}%
\pgfpathlineto{\pgfqpoint{0.968433in}{1.884725in}}%
\pgfpathlineto{\pgfqpoint{0.970903in}{0.790897in}}%
\pgfpathlineto{\pgfqpoint{0.974036in}{1.793347in}}%
\pgfpathlineto{\pgfqpoint{0.976539in}{0.806743in}}%
\pgfpathlineto{\pgfqpoint{0.979593in}{1.782721in}}%
\pgfpathlineto{\pgfqpoint{0.983183in}{0.657203in}}%
\pgfpathlineto{\pgfqpoint{0.986104in}{1.758697in}}%
\pgfpathlineto{\pgfqpoint{0.989182in}{0.783543in}}%
\pgfpathlineto{\pgfqpoint{0.992526in}{1.747306in}}%
\pgfpathlineto{\pgfqpoint{0.995251in}{0.709991in}}%
\pgfpathlineto{\pgfqpoint{0.998213in}{1.787122in}}%
\pgfpathlineto{\pgfqpoint{1.001646in}{0.784070in}}%
\pgfpathlineto{\pgfqpoint{1.004889in}{1.812247in}}%
\pgfpathlineto{\pgfqpoint{1.007869in}{0.828398in}}%
\pgfpathlineto{\pgfqpoint{1.010541in}{1.787751in}}%
\pgfpathlineto{\pgfqpoint{1.014284in}{0.890366in}}%
\pgfpathlineto{\pgfqpoint{1.018561in}{1.882517in}}%
\pgfpathlineto{\pgfqpoint{1.020049in}{0.900634in}}%
\pgfpathlineto{\pgfqpoint{1.023157in}{1.896021in}}%
\pgfpathlineto{\pgfqpoint{1.026398in}{0.705586in}}%
\pgfpathlineto{\pgfqpoint{1.029121in}{1.742947in}}%
\pgfpathlineto{\pgfqpoint{1.032405in}{0.819158in}}%
\pgfpathlineto{\pgfqpoint{1.035326in}{1.760331in}}%
\pgfpathlineto{\pgfqpoint{1.038467in}{0.749065in}}%
\pgfpathlineto{\pgfqpoint{1.041674in}{1.773020in}}%
\pgfpathlineto{\pgfqpoint{1.044636in}{0.853180in}}%
\pgfpathlineto{\pgfqpoint{1.047736in}{1.828438in}}%
\pgfpathlineto{\pgfqpoint{1.051170in}{0.933448in}}%
\pgfpathlineto{\pgfqpoint{1.054526in}{1.917491in}}%
\pgfpathlineto{\pgfqpoint{1.057522in}{0.813936in}}%
\pgfpathlineto{\pgfqpoint{1.061138in}{2.021819in}}%
\pgfpathlineto{\pgfqpoint{1.063655in}{0.849829in}}%
\pgfpathlineto{\pgfqpoint{1.067905in}{1.944301in}}%
\pgfpathlineto{\pgfqpoint{1.069836in}{0.897685in}}%
\pgfpathlineto{\pgfqpoint{1.073859in}{1.960757in}}%
\pgfpathlineto{\pgfqpoint{1.075720in}{0.881474in}}%
\pgfpathlineto{\pgfqpoint{1.078854in}{1.869749in}}%
\pgfpathlineto{\pgfqpoint{1.082136in}{0.857459in}}%
\pgfpathlineto{\pgfqpoint{1.085738in}{1.949972in}}%
\pgfpathlineto{\pgfqpoint{1.088187in}{0.894068in}}%
\pgfpathlineto{\pgfqpoint{1.091115in}{1.876864in}}%
\pgfpathlineto{\pgfqpoint{1.094558in}{0.861883in}}%
\pgfpathlineto{\pgfqpoint{1.097297in}{1.880402in}}%
\pgfpathlineto{\pgfqpoint{1.100631in}{0.861277in}}%
\pgfpathlineto{\pgfqpoint{1.104283in}{1.958873in}}%
\pgfpathlineto{\pgfqpoint{1.107181in}{0.888489in}}%
\pgfpathlineto{\pgfqpoint{1.110171in}{2.037097in}}%
\pgfpathlineto{\pgfqpoint{1.114131in}{0.870639in}}%
\pgfpathlineto{\pgfqpoint{1.116915in}{1.940087in}}%
\pgfpathlineto{\pgfqpoint{1.119363in}{0.933338in}}%
\pgfpathlineto{\pgfqpoint{1.122631in}{1.921139in}}%
\pgfpathlineto{\pgfqpoint{1.125387in}{0.933229in}}%
\pgfpathlineto{\pgfqpoint{1.128270in}{1.830239in}}%
\pgfpathlineto{\pgfqpoint{1.131486in}{0.934417in}}%
\pgfpathlineto{\pgfqpoint{1.134446in}{1.932943in}}%
\pgfpathlineto{\pgfqpoint{1.138782in}{0.816581in}}%
\pgfpathlineto{\pgfqpoint{1.140691in}{1.921445in}}%
\pgfpathlineto{\pgfqpoint{1.144703in}{0.858471in}}%
\pgfpathlineto{\pgfqpoint{1.146940in}{1.859721in}}%
\pgfpathlineto{\pgfqpoint{1.150060in}{0.943005in}}%
\pgfpathlineto{\pgfqpoint{1.153039in}{1.847201in}}%
\pgfpathlineto{\pgfqpoint{1.156116in}{0.901068in}}%
\pgfpathlineto{\pgfqpoint{1.159442in}{1.873934in}}%
\pgfpathlineto{\pgfqpoint{1.162931in}{0.880185in}}%
\pgfpathlineto{\pgfqpoint{1.165548in}{1.836940in}}%
\pgfpathlineto{\pgfqpoint{1.168980in}{0.882796in}}%
\pgfpathlineto{\pgfqpoint{1.172229in}{1.904247in}}%
\pgfpathlineto{\pgfqpoint{1.174947in}{0.955687in}}%
\pgfpathlineto{\pgfqpoint{1.179691in}{1.945910in}}%
\pgfpathlineto{\pgfqpoint{1.181170in}{0.900570in}}%
\pgfpathlineto{\pgfqpoint{1.185900in}{1.989200in}}%
\pgfpathlineto{\pgfqpoint{1.187205in}{0.911706in}}%
\pgfpathlineto{\pgfqpoint{1.190182in}{1.853504in}}%
\pgfpathlineto{\pgfqpoint{1.193541in}{0.907903in}}%
\pgfpathlineto{\pgfqpoint{1.196553in}{1.887582in}}%
\pgfpathlineto{\pgfqpoint{1.199703in}{0.800499in}}%
\pgfpathlineto{\pgfqpoint{1.202592in}{1.791498in}}%
\pgfpathlineto{\pgfqpoint{1.205816in}{0.785048in}}%
\pgfpathlineto{\pgfqpoint{1.209259in}{1.849530in}}%
\pgfpathlineto{\pgfqpoint{1.212003in}{0.788980in}}%
\pgfpathlineto{\pgfqpoint{1.215602in}{1.849558in}}%
\pgfpathlineto{\pgfqpoint{1.218550in}{0.758430in}}%
\pgfpathlineto{\pgfqpoint{1.221152in}{1.879197in}}%
\pgfpathlineto{\pgfqpoint{1.224839in}{0.811526in}}%
\pgfpathlineto{\pgfqpoint{1.227311in}{1.784505in}}%
\pgfpathlineto{\pgfqpoint{1.230637in}{0.787732in}}%
\pgfpathlineto{\pgfqpoint{1.233484in}{1.831833in}}%
\pgfpathlineto{\pgfqpoint{1.237121in}{0.714125in}}%
\pgfpathlineto{\pgfqpoint{1.239679in}{1.776204in}}%
\pgfpathlineto{\pgfqpoint{1.242875in}{0.817908in}}%
\pgfpathlineto{\pgfqpoint{1.245972in}{1.805106in}}%
\pgfpathlineto{\pgfqpoint{1.249804in}{0.791882in}}%
\pgfpathlineto{\pgfqpoint{1.252183in}{1.835041in}}%
\pgfpathlineto{\pgfqpoint{1.255180in}{0.828514in}}%
\pgfpathlineto{\pgfqpoint{1.258362in}{1.741431in}}%
\pgfpathlineto{\pgfqpoint{1.262099in}{0.763366in}}%
\pgfpathlineto{\pgfqpoint{1.264946in}{1.827356in}}%
\pgfpathlineto{\pgfqpoint{1.267972in}{0.812642in}}%
\pgfpathlineto{\pgfqpoint{1.270812in}{1.712594in}}%
\pgfpathlineto{\pgfqpoint{1.273773in}{0.720523in}}%
\pgfpathlineto{\pgfqpoint{1.277258in}{1.869055in}}%
\pgfpathlineto{\pgfqpoint{1.280352in}{0.758030in}}%
\pgfpathlineto{\pgfqpoint{1.283267in}{1.780685in}}%
\pgfpathlineto{\pgfqpoint{1.288166in}{0.701773in}}%
\pgfpathlineto{\pgfqpoint{1.289306in}{1.672484in}}%
\pgfpathlineto{\pgfqpoint{1.292860in}{0.717072in}}%
\pgfpathlineto{\pgfqpoint{1.296202in}{1.801444in}}%
\pgfpathlineto{\pgfqpoint{1.298769in}{0.796244in}}%
\pgfpathlineto{\pgfqpoint{1.301662in}{1.810004in}}%
\pgfpathlineto{\pgfqpoint{1.304950in}{0.808069in}}%
\pgfpathlineto{\pgfqpoint{1.307800in}{1.805414in}}%
\pgfpathlineto{\pgfqpoint{1.312337in}{0.668443in}}%
\pgfpathlineto{\pgfqpoint{1.314376in}{1.762781in}}%
\pgfpathlineto{\pgfqpoint{1.317092in}{0.806950in}}%
\pgfpathlineto{\pgfqpoint{1.321211in}{1.810564in}}%
\pgfpathlineto{\pgfqpoint{1.323619in}{0.777713in}}%
\pgfpathlineto{\pgfqpoint{1.326851in}{1.830400in}}%
\pgfpathlineto{\pgfqpoint{1.329785in}{0.759683in}}%
\pgfpathlineto{\pgfqpoint{1.332919in}{1.772641in}}%
\pgfpathlineto{\pgfqpoint{1.336239in}{0.766174in}}%
\pgfpathlineto{\pgfqpoint{1.338981in}{1.840950in}}%
\pgfpathlineto{\pgfqpoint{1.342724in}{0.704000in}}%
\pgfpathlineto{\pgfqpoint{1.345133in}{1.883719in}}%
\pgfpathlineto{\pgfqpoint{1.348161in}{0.827035in}}%
\pgfpathlineto{\pgfqpoint{1.351761in}{1.782630in}}%
\pgfpathlineto{\pgfqpoint{1.354307in}{0.713297in}}%
\pgfpathlineto{\pgfqpoint{1.357948in}{1.761032in}}%
\pgfpathlineto{\pgfqpoint{1.360893in}{0.682206in}}%
\pgfpathlineto{\pgfqpoint{1.363615in}{1.706372in}}%
\pgfpathlineto{\pgfqpoint{1.366877in}{0.627517in}}%
\pgfpathlineto{\pgfqpoint{1.369799in}{1.699172in}}%
\pgfpathlineto{\pgfqpoint{1.372930in}{0.724833in}}%
\pgfpathlineto{\pgfqpoint{1.376093in}{1.672543in}}%
\pgfpathlineto{\pgfqpoint{1.379059in}{0.763009in}}%
\pgfpathlineto{\pgfqpoint{1.383234in}{1.823456in}}%
\pgfpathlineto{\pgfqpoint{1.385355in}{0.680200in}}%
\pgfpathlineto{\pgfqpoint{1.388560in}{1.663225in}}%
\pgfpathlineto{\pgfqpoint{1.391451in}{0.754997in}}%
\pgfpathlineto{\pgfqpoint{1.394850in}{1.692936in}}%
\pgfpathlineto{\pgfqpoint{1.398163in}{0.626624in}}%
\pgfpathlineto{\pgfqpoint{1.400831in}{1.648729in}}%
\pgfpathlineto{\pgfqpoint{1.403817in}{0.692658in}}%
\pgfpathlineto{\pgfqpoint{1.406900in}{1.629772in}}%
\pgfpathlineto{\pgfqpoint{1.410317in}{0.709975in}}%
\pgfpathlineto{\pgfqpoint{1.414575in}{1.806362in}}%
\pgfpathlineto{\pgfqpoint{1.416204in}{0.743546in}}%
\pgfpathlineto{\pgfqpoint{1.419597in}{1.632974in}}%
\pgfpathlineto{\pgfqpoint{1.422539in}{0.653928in}}%
\pgfpathlineto{\pgfqpoint{1.425550in}{1.622315in}}%
\pgfpathlineto{\pgfqpoint{1.428612in}{0.656066in}}%
\pgfpathlineto{\pgfqpoint{1.432226in}{1.613892in}}%
\pgfpathlineto{\pgfqpoint{1.434912in}{0.619693in}}%
\pgfpathlineto{\pgfqpoint{1.438138in}{1.616072in}}%
\pgfpathlineto{\pgfqpoint{1.441956in}{0.627953in}}%
\pgfpathlineto{\pgfqpoint{1.444495in}{1.731409in}}%
\pgfpathlineto{\pgfqpoint{1.447194in}{0.752983in}}%
\pgfpathlineto{\pgfqpoint{1.450728in}{1.701562in}}%
\pgfpathlineto{\pgfqpoint{1.453793in}{0.758791in}}%
\pgfpathlineto{\pgfqpoint{1.456504in}{1.694472in}}%
\pgfpathlineto{\pgfqpoint{1.459901in}{0.667854in}}%
\pgfpathlineto{\pgfqpoint{1.462653in}{1.690091in}}%
\pgfpathlineto{\pgfqpoint{1.466035in}{0.767817in}}%
\pgfpathlineto{\pgfqpoint{1.469133in}{1.755375in}}%
\pgfpathlineto{\pgfqpoint{1.471909in}{0.759791in}}%
\pgfpathlineto{\pgfqpoint{1.475783in}{1.823114in}}%
\pgfpathlineto{\pgfqpoint{1.478268in}{0.809084in}}%
\pgfpathlineto{\pgfqpoint{1.481192in}{1.814053in}}%
\pgfpathlineto{\pgfqpoint{1.484365in}{0.775563in}}%
\pgfpathlineto{\pgfqpoint{1.487455in}{1.757353in}}%
\pgfpathlineto{\pgfqpoint{1.490596in}{0.780977in}}%
\pgfpathlineto{\pgfqpoint{1.494163in}{1.759413in}}%
\pgfpathlineto{\pgfqpoint{1.496720in}{0.697410in}}%
\pgfpathlineto{\pgfqpoint{1.499941in}{1.725637in}}%
\pgfpathlineto{\pgfqpoint{1.504500in}{0.655705in}}%
\pgfpathlineto{\pgfqpoint{1.506601in}{1.784577in}}%
\pgfpathlineto{\pgfqpoint{1.510411in}{0.683565in}}%
\pgfpathlineto{\pgfqpoint{1.513758in}{1.883006in}}%
\pgfpathlineto{\pgfqpoint{1.515367in}{0.811881in}}%
\pgfpathlineto{\pgfqpoint{1.518506in}{1.744334in}}%
\pgfpathlineto{\pgfqpoint{1.522205in}{0.702022in}}%
\pgfpathlineto{\pgfqpoint{1.524604in}{1.725265in}}%
\pgfpathlineto{\pgfqpoint{1.529121in}{0.768523in}}%
\pgfpathlineto{\pgfqpoint{1.531328in}{1.791105in}}%
\pgfpathlineto{\pgfqpoint{1.533871in}{0.766251in}}%
\pgfpathlineto{\pgfqpoint{1.537100in}{1.754815in}}%
\pgfpathlineto{\pgfqpoint{1.540092in}{0.759105in}}%
\pgfpathlineto{\pgfqpoint{1.543214in}{1.714110in}}%
\pgfpathlineto{\pgfqpoint{1.546322in}{0.780844in}}%
\pgfpathlineto{\pgfqpoint{1.550280in}{1.830361in}}%
\pgfpathlineto{\pgfqpoint{1.552940in}{0.765106in}}%
\pgfpathlineto{\pgfqpoint{1.556010in}{1.792444in}}%
\pgfpathlineto{\pgfqpoint{1.559042in}{0.775919in}}%
\pgfpathlineto{\pgfqpoint{1.561838in}{1.845564in}}%
\pgfpathlineto{\pgfqpoint{1.565049in}{0.781856in}}%
\pgfpathlineto{\pgfqpoint{1.567981in}{1.728569in}}%
\pgfpathlineto{\pgfqpoint{1.571060in}{0.748649in}}%
\pgfpathlineto{\pgfqpoint{1.574126in}{1.712308in}}%
\pgfpathlineto{\pgfqpoint{1.577350in}{0.781822in}}%
\pgfpathlineto{\pgfqpoint{1.580725in}{1.814139in}}%
\pgfpathlineto{\pgfqpoint{1.584134in}{0.762176in}}%
\pgfpathlineto{\pgfqpoint{1.586747in}{1.771037in}}%
\pgfpathlineto{\pgfqpoint{1.590331in}{0.650256in}}%
\pgfpathlineto{\pgfqpoint{1.593790in}{1.797491in}}%
\pgfpathlineto{\pgfqpoint{1.595899in}{0.760245in}}%
\pgfpathlineto{\pgfqpoint{1.600207in}{1.820602in}}%
\pgfpathlineto{\pgfqpoint{1.602852in}{0.689420in}}%
\pgfpathlineto{\pgfqpoint{1.605211in}{1.788276in}}%
\pgfpathlineto{\pgfqpoint{1.608508in}{0.753748in}}%
\pgfpathlineto{\pgfqpoint{1.612525in}{1.769270in}}%
\pgfpathlineto{\pgfqpoint{1.614764in}{0.725972in}}%
\pgfpathlineto{\pgfqpoint{1.617583in}{1.734620in}}%
\pgfpathlineto{\pgfqpoint{1.621839in}{0.656438in}}%
\pgfpathlineto{\pgfqpoint{1.623666in}{1.687226in}}%
\pgfpathlineto{\pgfqpoint{1.627710in}{0.735331in}}%
\pgfpathlineto{\pgfqpoint{1.631064in}{1.848249in}}%
\pgfpathlineto{\pgfqpoint{1.633109in}{0.768241in}}%
\pgfpathlineto{\pgfqpoint{1.636284in}{1.762197in}}%
\pgfpathlineto{\pgfqpoint{1.640241in}{0.741127in}}%
\pgfpathlineto{\pgfqpoint{1.642756in}{1.704209in}}%
\pgfpathlineto{\pgfqpoint{1.645614in}{0.694191in}}%
\pgfpathlineto{\pgfqpoint{1.649604in}{1.736065in}}%
\pgfpathlineto{\pgfqpoint{1.652685in}{0.623419in}}%
\pgfpathlineto{\pgfqpoint{1.654618in}{1.677126in}}%
\pgfpathlineto{\pgfqpoint{1.657785in}{0.728659in}}%
\pgfpathlineto{\pgfqpoint{1.661480in}{1.723193in}}%
\pgfpathlineto{\pgfqpoint{1.664748in}{0.668600in}}%
\pgfpathlineto{\pgfqpoint{1.667218in}{1.762296in}}%
\pgfpathlineto{\pgfqpoint{1.670346in}{0.668618in}}%
\pgfpathlineto{\pgfqpoint{1.675039in}{1.807834in}}%
\pgfpathlineto{\pgfqpoint{1.677657in}{0.649103in}}%
\pgfpathlineto{\pgfqpoint{1.680780in}{1.817799in}}%
\pgfpathlineto{\pgfqpoint{1.682611in}{0.762457in}}%
\pgfpathlineto{\pgfqpoint{1.685661in}{1.775403in}}%
\pgfpathlineto{\pgfqpoint{1.688952in}{0.767811in}}%
\pgfpathlineto{\pgfqpoint{1.692668in}{1.790942in}}%
\pgfpathlineto{\pgfqpoint{1.695214in}{0.779141in}}%
\pgfpathlineto{\pgfqpoint{1.698377in}{1.706182in}}%
\pgfpathlineto{\pgfqpoint{1.701517in}{0.699021in}}%
\pgfpathlineto{\pgfqpoint{1.704266in}{1.722299in}}%
\pgfpathlineto{\pgfqpoint{1.708614in}{0.711589in}}%
\pgfpathlineto{\pgfqpoint{1.710393in}{1.818130in}}%
\pgfpathlineto{\pgfqpoint{1.713645in}{0.822125in}}%
\pgfpathlineto{\pgfqpoint{1.716736in}{1.809079in}}%
\pgfpathlineto{\pgfqpoint{1.719836in}{0.868995in}}%
\pgfpathlineto{\pgfqpoint{1.722983in}{1.855804in}}%
\pgfpathlineto{\pgfqpoint{1.725888in}{0.792681in}}%
\pgfpathlineto{\pgfqpoint{1.729058in}{1.909552in}}%
\pgfpathlineto{\pgfqpoint{1.732344in}{0.789285in}}%
\pgfpathlineto{\pgfqpoint{1.736367in}{1.822959in}}%
\pgfpathlineto{\pgfqpoint{1.738293in}{0.811625in}}%
\pgfpathlineto{\pgfqpoint{1.742076in}{1.859733in}}%
\pgfpathlineto{\pgfqpoint{1.744414in}{0.855586in}}%
\pgfpathlineto{\pgfqpoint{1.748176in}{1.792105in}}%
\pgfpathlineto{\pgfqpoint{1.751333in}{0.818330in}}%
\pgfpathlineto{\pgfqpoint{1.753844in}{1.775692in}}%
\pgfpathlineto{\pgfqpoint{1.757195in}{0.787008in}}%
\pgfpathlineto{\pgfqpoint{1.760791in}{1.807181in}}%
\pgfpathlineto{\pgfqpoint{1.763469in}{0.822860in}}%
\pgfpathlineto{\pgfqpoint{1.766389in}{1.771249in}}%
\pgfpathlineto{\pgfqpoint{1.770046in}{0.756212in}}%
\pgfpathlineto{\pgfqpoint{1.773206in}{1.908806in}}%
\pgfpathlineto{\pgfqpoint{1.775732in}{0.837359in}}%
\pgfpathlineto{\pgfqpoint{1.780122in}{1.896044in}}%
\pgfpathlineto{\pgfqpoint{1.781597in}{0.837252in}}%
\pgfpathlineto{\pgfqpoint{1.785945in}{1.848350in}}%
\pgfpathlineto{\pgfqpoint{1.787758in}{0.837417in}}%
\pgfpathlineto{\pgfqpoint{1.790844in}{1.754574in}}%
\pgfpathlineto{\pgfqpoint{1.793982in}{0.885962in}}%
\pgfpathlineto{\pgfqpoint{1.797220in}{1.766572in}}%
\pgfpathlineto{\pgfqpoint{1.800160in}{0.870475in}}%
\pgfpathlineto{\pgfqpoint{1.803457in}{1.828689in}}%
\pgfpathlineto{\pgfqpoint{1.806458in}{0.829286in}}%
\pgfpathlineto{\pgfqpoint{1.809573in}{1.837726in}}%
\pgfpathlineto{\pgfqpoint{1.814210in}{0.823131in}}%
\pgfpathlineto{\pgfqpoint{1.815789in}{1.855899in}}%
\pgfpathlineto{\pgfqpoint{1.818859in}{0.921078in}}%
\pgfpathlineto{\pgfqpoint{1.821910in}{1.864757in}}%
\pgfpathlineto{\pgfqpoint{1.825985in}{0.717947in}}%
\pgfpathlineto{\pgfqpoint{1.828354in}{1.852107in}}%
\pgfpathlineto{\pgfqpoint{1.831199in}{0.930040in}}%
\pgfpathlineto{\pgfqpoint{1.836359in}{1.976016in}}%
\pgfpathlineto{\pgfqpoint{1.837500in}{0.876369in}}%
\pgfpathlineto{\pgfqpoint{1.841423in}{1.824897in}}%
\pgfpathlineto{\pgfqpoint{1.843468in}{0.848320in}}%
\pgfpathlineto{\pgfqpoint{1.846840in}{1.817932in}}%
\pgfpathlineto{\pgfqpoint{1.850046in}{0.781454in}}%
\pgfpathlineto{\pgfqpoint{1.853053in}{1.821227in}}%
\pgfpathlineto{\pgfqpoint{1.857552in}{0.730388in}}%
\pgfpathlineto{\pgfqpoint{1.859025in}{1.831197in}}%
\pgfpathlineto{\pgfqpoint{1.862532in}{0.771736in}}%
\pgfpathlineto{\pgfqpoint{1.866586in}{1.800193in}}%
\pgfpathlineto{\pgfqpoint{1.868387in}{0.835156in}}%
\pgfpathlineto{\pgfqpoint{1.871598in}{1.790015in}}%
\pgfpathlineto{\pgfqpoint{1.874708in}{0.808105in}}%
\pgfpathlineto{\pgfqpoint{1.878007in}{1.805655in}}%
\pgfpathlineto{\pgfqpoint{1.881071in}{0.778077in}}%
\pgfpathlineto{\pgfqpoint{1.884239in}{1.841130in}}%
\pgfpathlineto{\pgfqpoint{1.886863in}{0.789600in}}%
\pgfpathlineto{\pgfqpoint{1.890295in}{1.804150in}}%
\pgfpathlineto{\pgfqpoint{1.893229in}{0.872953in}}%
\pgfpathlineto{\pgfqpoint{1.896561in}{1.860306in}}%
\pgfpathlineto{\pgfqpoint{1.899549in}{0.869503in}}%
\pgfpathlineto{\pgfqpoint{1.902708in}{1.832992in}}%
\pgfpathlineto{\pgfqpoint{1.905901in}{0.909490in}}%
\pgfpathlineto{\pgfqpoint{1.908861in}{1.864693in}}%
\pgfpathlineto{\pgfqpoint{1.911693in}{0.870398in}}%
\pgfpathlineto{\pgfqpoint{1.914910in}{1.861527in}}%
\pgfpathlineto{\pgfqpoint{1.918250in}{0.720602in}}%
\pgfpathlineto{\pgfqpoint{1.921263in}{1.842582in}}%
\pgfpathlineto{\pgfqpoint{1.925075in}{0.790526in}}%
\pgfpathlineto{\pgfqpoint{1.927426in}{1.799417in}}%
\pgfpathlineto{\pgfqpoint{1.931225in}{0.745070in}}%
\pgfpathlineto{\pgfqpoint{1.933520in}{1.790387in}}%
\pgfpathlineto{\pgfqpoint{1.936386in}{0.837948in}}%
\pgfpathlineto{\pgfqpoint{1.940606in}{1.864980in}}%
\pgfpathlineto{\pgfqpoint{1.943136in}{0.694803in}}%
\pgfpathlineto{\pgfqpoint{1.946128in}{1.786850in}}%
\pgfpathlineto{\pgfqpoint{1.949253in}{0.803246in}}%
\pgfpathlineto{\pgfqpoint{1.951946in}{1.791034in}}%
\pgfpathlineto{\pgfqpoint{1.955070in}{0.806601in}}%
\pgfpathlineto{\pgfqpoint{1.958903in}{1.845159in}}%
\pgfpathlineto{\pgfqpoint{1.961358in}{0.726972in}}%
\pgfpathlineto{\pgfqpoint{1.964437in}{1.694758in}}%
\pgfpathlineto{\pgfqpoint{1.969332in}{0.719675in}}%
\pgfpathlineto{\pgfqpoint{1.970478in}{1.735404in}}%
\pgfpathlineto{\pgfqpoint{1.973844in}{0.819223in}}%
\pgfpathlineto{\pgfqpoint{1.976730in}{1.774053in}}%
\pgfpathlineto{\pgfqpoint{1.979898in}{0.690637in}}%
\pgfpathlineto{\pgfqpoint{1.983558in}{1.725359in}}%
\pgfpathlineto{\pgfqpoint{1.986872in}{0.703221in}}%
\pgfpathlineto{\pgfqpoint{1.989048in}{1.668585in}}%
\pgfpathlineto{\pgfqpoint{1.992633in}{0.752613in}}%
\pgfpathlineto{\pgfqpoint{1.996904in}{1.849444in}}%
\pgfpathlineto{\pgfqpoint{1.998475in}{0.809865in}}%
\pgfpathlineto{\pgfqpoint{2.002158in}{1.804198in}}%
\pgfpathlineto{\pgfqpoint{2.004899in}{0.787515in}}%
\pgfpathlineto{\pgfqpoint{2.007602in}{1.738236in}}%
\pgfpathlineto{\pgfqpoint{2.010805in}{0.781285in}}%
\pgfpathlineto{\pgfqpoint{2.014752in}{1.853602in}}%
\pgfpathlineto{\pgfqpoint{2.017102in}{0.635134in}}%
\pgfpathlineto{\pgfqpoint{2.020174in}{1.763770in}}%
\pgfpathlineto{\pgfqpoint{2.023148in}{0.741186in}}%
\pgfpathlineto{\pgfqpoint{2.027629in}{1.822423in}}%
\pgfpathlineto{\pgfqpoint{2.029839in}{0.766948in}}%
\pgfpathlineto{\pgfqpoint{2.032822in}{1.876907in}}%
\pgfpathlineto{\pgfqpoint{2.036170in}{0.764070in}}%
\pgfpathlineto{\pgfqpoint{2.039686in}{1.800503in}}%
\pgfpathlineto{\pgfqpoint{2.042291in}{0.715393in}}%
\pgfpathlineto{\pgfqpoint{2.045144in}{1.777578in}}%
\pgfpathlineto{\pgfqpoint{2.048138in}{0.779942in}}%
\pgfpathlineto{\pgfqpoint{2.051399in}{1.943766in}}%
\pgfpathlineto{\pgfqpoint{2.054539in}{0.798398in}}%
\pgfpathlineto{\pgfqpoint{2.057202in}{1.770430in}}%
\pgfpathlineto{\pgfqpoint{2.061309in}{0.738286in}}%
\pgfpathlineto{\pgfqpoint{2.063490in}{1.799309in}}%
\pgfpathlineto{\pgfqpoint{2.066493in}{0.889998in}}%
\pgfpathlineto{\pgfqpoint{2.070387in}{1.871953in}}%
\pgfpathlineto{\pgfqpoint{2.072769in}{0.868350in}}%
\pgfpathlineto{\pgfqpoint{2.075957in}{1.733426in}}%
\pgfpathlineto{\pgfqpoint{2.079697in}{0.736388in}}%
\pgfpathlineto{\pgfqpoint{2.083355in}{1.895202in}}%
\pgfpathlineto{\pgfqpoint{2.085646in}{0.708493in}}%
\pgfpathlineto{\pgfqpoint{2.088859in}{1.773593in}}%
\pgfpathlineto{\pgfqpoint{2.091816in}{0.758955in}}%
\pgfpathlineto{\pgfqpoint{2.094303in}{1.790745in}}%
\pgfpathlineto{\pgfqpoint{2.097763in}{0.724848in}}%
\pgfpathlineto{\pgfqpoint{2.100967in}{1.798581in}}%
\pgfpathlineto{\pgfqpoint{2.104042in}{0.659050in}}%
\pgfpathlineto{\pgfqpoint{2.106783in}{1.737604in}}%
\pgfpathlineto{\pgfqpoint{2.109930in}{0.755907in}}%
\pgfpathlineto{\pgfqpoint{2.112860in}{1.680340in}}%
\pgfpathlineto{\pgfqpoint{2.117352in}{0.695983in}}%
\pgfpathlineto{\pgfqpoint{2.119138in}{1.729551in}}%
\pgfpathlineto{\pgfqpoint{2.122355in}{0.725559in}}%
\pgfpathlineto{\pgfqpoint{2.125318in}{1.687704in}}%
\pgfpathlineto{\pgfqpoint{2.130345in}{0.526279in}}%
\pgfpathlineto{\pgfqpoint{2.131498in}{1.628819in}}%
\pgfpathlineto{\pgfqpoint{2.134746in}{0.679193in}}%
\pgfpathlineto{\pgfqpoint{2.137770in}{1.702910in}}%
\pgfpathlineto{\pgfqpoint{2.142132in}{0.600600in}}%
\pgfpathlineto{\pgfqpoint{2.143901in}{1.742906in}}%
\pgfpathlineto{\pgfqpoint{2.147428in}{0.696343in}}%
\pgfpathlineto{\pgfqpoint{2.150762in}{1.893783in}}%
\pgfpathlineto{\pgfqpoint{2.153336in}{0.731811in}}%
\pgfpathlineto{\pgfqpoint{2.156385in}{1.673946in}}%
\pgfpathlineto{\pgfqpoint{2.160821in}{0.568414in}}%
\pgfpathlineto{\pgfqpoint{2.162712in}{1.777740in}}%
\pgfpathlineto{\pgfqpoint{2.165885in}{0.731896in}}%
\pgfpathlineto{\pgfqpoint{2.170425in}{1.762483in}}%
\pgfpathlineto{\pgfqpoint{2.171725in}{0.741743in}}%
\pgfpathlineto{\pgfqpoint{2.176170in}{1.807107in}}%
\pgfpathlineto{\pgfqpoint{2.177876in}{0.734220in}}%
\pgfpathlineto{\pgfqpoint{2.181719in}{1.742879in}}%
\pgfpathlineto{\pgfqpoint{2.184786in}{0.637132in}}%
\pgfpathlineto{\pgfqpoint{2.187804in}{1.711301in}}%
\pgfpathlineto{\pgfqpoint{2.190822in}{0.796994in}}%
\pgfpathlineto{\pgfqpoint{2.194030in}{1.765018in}}%
\pgfpathlineto{\pgfqpoint{2.196480in}{0.734134in}}%
\pgfpathlineto{\pgfqpoint{2.199971in}{1.708051in}}%
\pgfpathlineto{\pgfqpoint{2.202793in}{0.700391in}}%
\pgfpathlineto{\pgfqpoint{2.205995in}{1.729830in}}%
\pgfpathlineto{\pgfqpoint{2.209430in}{0.708870in}}%
\pgfpathlineto{\pgfqpoint{2.212091in}{1.708539in}}%
\pgfpathlineto{\pgfqpoint{2.215124in}{0.802817in}}%
\pgfpathlineto{\pgfqpoint{2.218453in}{1.770323in}}%
\pgfpathlineto{\pgfqpoint{2.222588in}{0.633481in}}%
\pgfpathlineto{\pgfqpoint{2.224842in}{1.669216in}}%
\pgfpathlineto{\pgfqpoint{2.228340in}{0.616188in}}%
\pgfpathlineto{\pgfqpoint{2.231086in}{1.649206in}}%
\pgfpathlineto{\pgfqpoint{2.234532in}{0.693697in}}%
\pgfpathlineto{\pgfqpoint{2.236924in}{1.721962in}}%
\pgfpathlineto{\pgfqpoint{2.239974in}{0.639059in}}%
\pgfpathlineto{\pgfqpoint{2.243678in}{1.666486in}}%
\pgfpathlineto{\pgfqpoint{2.246729in}{0.610227in}}%
\pgfpathlineto{\pgfqpoint{2.249168in}{1.672323in}}%
\pgfpathlineto{\pgfqpoint{2.252513in}{0.635741in}}%
\pgfpathlineto{\pgfqpoint{2.255514in}{1.627386in}}%
\pgfpathlineto{\pgfqpoint{2.258765in}{0.649197in}}%
\pgfpathlineto{\pgfqpoint{2.261627in}{1.661287in}}%
\pgfpathlineto{\pgfqpoint{2.264629in}{0.544615in}}%
\pgfpathlineto{\pgfqpoint{2.267819in}{1.653934in}}%
\pgfpathlineto{\pgfqpoint{2.270841in}{0.690601in}}%
\pgfpathlineto{\pgfqpoint{2.274191in}{1.666476in}}%
\pgfpathlineto{\pgfqpoint{2.277140in}{0.714166in}}%
\pgfpathlineto{\pgfqpoint{2.280253in}{1.694370in}}%
\pgfpathlineto{\pgfqpoint{2.284268in}{0.724308in}}%
\pgfpathlineto{\pgfqpoint{2.286421in}{1.752524in}}%
\pgfpathlineto{\pgfqpoint{2.290522in}{0.661274in}}%
\pgfpathlineto{\pgfqpoint{2.293292in}{1.753204in}}%
\pgfpathlineto{\pgfqpoint{2.296500in}{0.732384in}}%
\pgfpathlineto{\pgfqpoint{2.298788in}{1.717465in}}%
\pgfpathlineto{\pgfqpoint{2.301851in}{0.797528in}}%
\pgfpathlineto{\pgfqpoint{2.304852in}{1.711060in}}%
\pgfpathlineto{\pgfqpoint{2.308183in}{0.706012in}}%
\pgfpathlineto{\pgfqpoint{2.311448in}{1.897168in}}%
\pgfpathlineto{\pgfqpoint{2.314869in}{0.678070in}}%
\pgfpathlineto{\pgfqpoint{2.317238in}{1.683229in}}%
\pgfpathlineto{\pgfqpoint{2.320667in}{0.664946in}}%
\pgfpathlineto{\pgfqpoint{2.323745in}{1.738181in}}%
\pgfpathlineto{\pgfqpoint{2.326656in}{0.751406in}}%
\pgfpathlineto{\pgfqpoint{2.330209in}{1.805635in}}%
\pgfpathlineto{\pgfqpoint{2.333297in}{0.810478in}}%
\pgfpathlineto{\pgfqpoint{2.336182in}{1.772065in}}%
\pgfpathlineto{\pgfqpoint{2.339317in}{0.710491in}}%
\pgfpathlineto{\pgfqpoint{2.342089in}{1.737886in}}%
\pgfpathlineto{\pgfqpoint{2.345341in}{0.860003in}}%
\pgfpathlineto{\pgfqpoint{2.348724in}{1.745804in}}%
\pgfpathlineto{\pgfqpoint{2.352132in}{0.771360in}}%
\pgfpathlineto{\pgfqpoint{2.354543in}{1.740617in}}%
\pgfpathlineto{\pgfqpoint{2.357584in}{0.797746in}}%
\pgfpathlineto{\pgfqpoint{2.360813in}{1.721577in}}%
\pgfpathlineto{\pgfqpoint{2.364399in}{0.742717in}}%
\pgfpathlineto{\pgfqpoint{2.367107in}{1.771192in}}%
\pgfpathlineto{\pgfqpoint{2.370754in}{0.657142in}}%
\pgfpathlineto{\pgfqpoint{2.373271in}{1.680846in}}%
\pgfpathlineto{\pgfqpoint{2.376070in}{0.763110in}}%
\pgfpathlineto{\pgfqpoint{2.379314in}{1.639492in}}%
\pgfpathlineto{\pgfqpoint{2.382536in}{0.713978in}}%
\pgfpathlineto{\pgfqpoint{2.385845in}{1.740200in}}%
\pgfpathlineto{\pgfqpoint{2.388790in}{0.735711in}}%
\pgfpathlineto{\pgfqpoint{2.391617in}{1.792473in}}%
\pgfpathlineto{\pgfqpoint{2.394794in}{0.818250in}}%
\pgfpathlineto{\pgfqpoint{2.397804in}{1.806310in}}%
\pgfpathlineto{\pgfqpoint{2.401268in}{0.782967in}}%
\pgfpathlineto{\pgfqpoint{2.404505in}{1.826103in}}%
\pgfpathlineto{\pgfqpoint{2.407269in}{0.792043in}}%
\pgfpathlineto{\pgfqpoint{2.410839in}{1.789675in}}%
\pgfpathlineto{\pgfqpoint{2.413376in}{0.811548in}}%
\pgfpathlineto{\pgfqpoint{2.417145in}{1.854965in}}%
\pgfpathlineto{\pgfqpoint{2.419551in}{0.848813in}}%
\pgfpathlineto{\pgfqpoint{2.422643in}{1.810693in}}%
\pgfpathlineto{\pgfqpoint{2.426128in}{0.737669in}}%
\pgfpathlineto{\pgfqpoint{2.428949in}{1.798494in}}%
\pgfpathlineto{\pgfqpoint{2.432561in}{0.844391in}}%
\pgfpathlineto{\pgfqpoint{2.435488in}{1.863327in}}%
\pgfpathlineto{\pgfqpoint{2.438242in}{0.838785in}}%
\pgfpathlineto{\pgfqpoint{2.441369in}{1.809195in}}%
\pgfpathlineto{\pgfqpoint{2.444166in}{0.851380in}}%
\pgfpathlineto{\pgfqpoint{2.447411in}{1.830919in}}%
\pgfpathlineto{\pgfqpoint{2.450601in}{0.859841in}}%
\pgfpathlineto{\pgfqpoint{2.454202in}{1.775117in}}%
\pgfpathlineto{\pgfqpoint{2.456594in}{0.888417in}}%
\pgfpathlineto{\pgfqpoint{2.461245in}{1.846452in}}%
\pgfpathlineto{\pgfqpoint{2.463662in}{0.760348in}}%
\pgfpathlineto{\pgfqpoint{2.466233in}{1.804570in}}%
\pgfpathlineto{\pgfqpoint{2.469126in}{0.822068in}}%
\pgfpathlineto{\pgfqpoint{2.472841in}{1.914343in}}%
\pgfpathlineto{\pgfqpoint{2.475739in}{0.820671in}}%
\pgfpathlineto{\pgfqpoint{2.479094in}{1.832475in}}%
\pgfpathlineto{\pgfqpoint{2.481677in}{0.742389in}}%
\pgfpathlineto{\pgfqpoint{2.485152in}{1.873299in}}%
\pgfpathlineto{\pgfqpoint{2.487855in}{0.820139in}}%
\pgfpathlineto{\pgfqpoint{2.490685in}{1.811316in}}%
\pgfpathlineto{\pgfqpoint{2.493859in}{0.825142in}}%
\pgfpathlineto{\pgfqpoint{2.497769in}{1.805439in}}%
\pgfpathlineto{\pgfqpoint{2.500203in}{0.876453in}}%
\pgfpathlineto{\pgfqpoint{2.503070in}{1.875550in}}%
\pgfpathlineto{\pgfqpoint{2.506821in}{0.701511in}}%
\pgfpathlineto{\pgfqpoint{2.509998in}{1.820910in}}%
\pgfpathlineto{\pgfqpoint{2.512723in}{0.857361in}}%
\pgfpathlineto{\pgfqpoint{2.515462in}{1.735102in}}%
\pgfpathlineto{\pgfqpoint{2.519238in}{0.723539in}}%
\pgfpathlineto{\pgfqpoint{2.521633in}{1.748376in}}%
\pgfpathlineto{\pgfqpoint{2.525298in}{0.726221in}}%
\pgfpathlineto{\pgfqpoint{2.527983in}{1.688118in}}%
\pgfpathlineto{\pgfqpoint{2.531049in}{0.763099in}}%
\pgfpathlineto{\pgfqpoint{2.535111in}{1.816027in}}%
\pgfpathlineto{\pgfqpoint{2.537483in}{0.771449in}}%
\pgfpathlineto{\pgfqpoint{2.541639in}{1.777512in}}%
\pgfpathlineto{\pgfqpoint{2.544403in}{0.671171in}}%
\pgfpathlineto{\pgfqpoint{2.546631in}{1.729558in}}%
\pgfpathlineto{\pgfqpoint{2.549883in}{0.701169in}}%
\pgfpathlineto{\pgfqpoint{2.552669in}{1.707264in}}%
\pgfpathlineto{\pgfqpoint{2.557078in}{0.722381in}}%
\pgfpathlineto{\pgfqpoint{2.558885in}{1.711162in}}%
\pgfpathlineto{\pgfqpoint{2.563533in}{0.699032in}}%
\pgfpathlineto{\pgfqpoint{2.565267in}{1.736939in}}%
\pgfpathlineto{\pgfqpoint{2.569288in}{0.645910in}}%
\pgfpathlineto{\pgfqpoint{2.571823in}{1.737940in}}%
\pgfpathlineto{\pgfqpoint{2.574759in}{0.607009in}}%
\pgfpathlineto{\pgfqpoint{2.577909in}{1.691843in}}%
\pgfpathlineto{\pgfqpoint{2.580802in}{0.702745in}}%
\pgfpathlineto{\pgfqpoint{2.584026in}{1.747936in}}%
\pgfpathlineto{\pgfqpoint{2.586703in}{0.759217in}}%
\pgfpathlineto{\pgfqpoint{2.589798in}{1.688682in}}%
\pgfpathlineto{\pgfqpoint{2.593201in}{0.677019in}}%
\pgfpathlineto{\pgfqpoint{2.596864in}{1.761554in}}%
\pgfpathlineto{\pgfqpoint{2.599367in}{0.606977in}}%
\pgfpathlineto{\pgfqpoint{2.602281in}{1.713036in}}%
\pgfpathlineto{\pgfqpoint{2.605789in}{0.749696in}}%
\pgfpathlineto{\pgfqpoint{2.608295in}{1.768177in}}%
\pgfpathlineto{\pgfqpoint{2.612038in}{0.674966in}}%
\pgfpathlineto{\pgfqpoint{2.614692in}{1.692534in}}%
\pgfpathlineto{\pgfqpoint{2.617606in}{0.715983in}}%
\pgfpathlineto{\pgfqpoint{2.621021in}{1.670428in}}%
\pgfpathlineto{\pgfqpoint{2.623847in}{0.687749in}}%
\pgfpathlineto{\pgfqpoint{2.627229in}{1.668518in}}%
\pgfpathlineto{\pgfqpoint{2.630029in}{0.593505in}}%
\pgfpathlineto{\pgfqpoint{2.633049in}{1.696896in}}%
\pgfpathlineto{\pgfqpoint{2.636438in}{0.714486in}}%
\pgfpathlineto{\pgfqpoint{2.640571in}{1.683428in}}%
\pgfpathlineto{\pgfqpoint{2.642682in}{0.671030in}}%
\pgfpathlineto{\pgfqpoint{2.645588in}{1.665352in}}%
\pgfpathlineto{\pgfqpoint{2.649438in}{0.544461in}}%
\pgfpathlineto{\pgfqpoint{2.651878in}{1.572166in}}%
\pgfpathlineto{\pgfqpoint{2.654870in}{0.708496in}}%
\pgfpathlineto{\pgfqpoint{2.658890in}{1.732946in}}%
\pgfpathlineto{\pgfqpoint{2.661616in}{0.669466in}}%
\pgfpathlineto{\pgfqpoint{2.664193in}{1.576577in}}%
\pgfpathlineto{\pgfqpoint{2.667181in}{0.681773in}}%
\pgfpathlineto{\pgfqpoint{2.671479in}{1.699470in}}%
\pgfpathlineto{\pgfqpoint{2.673387in}{0.600414in}}%
\pgfpathlineto{\pgfqpoint{2.676893in}{1.715129in}}%
\pgfpathlineto{\pgfqpoint{2.680440in}{0.592936in}}%
\pgfpathlineto{\pgfqpoint{2.683182in}{1.644231in}}%
\pgfpathlineto{\pgfqpoint{2.685942in}{0.646836in}}%
\pgfpathlineto{\pgfqpoint{2.688809in}{1.609968in}}%
\pgfpathlineto{\pgfqpoint{2.692120in}{0.718033in}}%
\pgfpathlineto{\pgfqpoint{2.695250in}{1.680986in}}%
\pgfpathlineto{\pgfqpoint{2.698602in}{0.581125in}}%
\pgfpathlineto{\pgfqpoint{2.701207in}{1.621357in}}%
\pgfpathlineto{\pgfqpoint{2.704750in}{0.612587in}}%
\pgfpathlineto{\pgfqpoint{2.708338in}{1.621774in}}%
\pgfpathlineto{\pgfqpoint{2.710691in}{0.684440in}}%
\pgfpathlineto{\pgfqpoint{2.715369in}{1.730314in}}%
\pgfpathlineto{\pgfqpoint{2.716789in}{0.646017in}}%
\pgfpathlineto{\pgfqpoint{2.720061in}{1.609926in}}%
\pgfpathlineto{\pgfqpoint{2.723801in}{0.588567in}}%
\pgfpathlineto{\pgfqpoint{2.726130in}{1.657554in}}%
\pgfpathlineto{\pgfqpoint{2.730569in}{0.509911in}}%
\pgfpathlineto{\pgfqpoint{2.732229in}{1.663564in}}%
\pgfpathlineto{\pgfqpoint{2.736020in}{0.644975in}}%
\pgfpathlineto{\pgfqpoint{2.739143in}{1.647333in}}%
\pgfpathlineto{\pgfqpoint{2.741614in}{0.616764in}}%
\pgfpathlineto{\pgfqpoint{2.744606in}{1.624702in}}%
\pgfpathlineto{\pgfqpoint{2.748034in}{0.635493in}}%
\pgfpathlineto{\pgfqpoint{2.751022in}{1.696958in}}%
\pgfpathlineto{\pgfqpoint{2.754005in}{0.634936in}}%
\pgfpathlineto{\pgfqpoint{2.757409in}{1.653377in}}%
\pgfpathlineto{\pgfqpoint{2.760257in}{0.690786in}}%
\pgfpathlineto{\pgfqpoint{2.764263in}{1.732830in}}%
\pgfpathlineto{\pgfqpoint{2.766606in}{0.647608in}}%
\pgfpathlineto{\pgfqpoint{2.770220in}{1.652768in}}%
\pgfpathlineto{\pgfqpoint{2.772399in}{0.654965in}}%
\pgfpathlineto{\pgfqpoint{2.776185in}{1.796048in}}%
\pgfpathlineto{\pgfqpoint{2.778575in}{0.702343in}}%
\pgfpathlineto{\pgfqpoint{2.782476in}{1.648178in}}%
\pgfpathlineto{\pgfqpoint{2.785327in}{0.661832in}}%
\pgfpathlineto{\pgfqpoint{2.788927in}{1.672497in}}%
\pgfpathlineto{\pgfqpoint{2.791066in}{0.716321in}}%
\pgfpathlineto{\pgfqpoint{2.794603in}{1.746679in}}%
\pgfpathlineto{\pgfqpoint{2.797166in}{0.713447in}}%
\pgfpathlineto{\pgfqpoint{2.801087in}{1.829264in}}%
\pgfpathlineto{\pgfqpoint{2.803357in}{0.702472in}}%
\pgfpathlineto{\pgfqpoint{2.807424in}{1.677585in}}%
\pgfpathlineto{\pgfqpoint{2.810269in}{0.740576in}}%
\pgfpathlineto{\pgfqpoint{2.812867in}{1.709116in}}%
\pgfpathlineto{\pgfqpoint{2.816582in}{0.676331in}}%
\pgfpathlineto{\pgfqpoint{2.818878in}{1.754862in}}%
\pgfpathlineto{\pgfqpoint{2.822165in}{0.695715in}}%
\pgfpathlineto{\pgfqpoint{2.825897in}{1.741469in}}%
\pgfpathlineto{\pgfqpoint{2.828138in}{0.719447in}}%
\pgfpathlineto{\pgfqpoint{2.831366in}{1.655480in}}%
\pgfpathlineto{\pgfqpoint{2.834482in}{0.724208in}}%
\pgfpathlineto{\pgfqpoint{2.838995in}{1.867958in}}%
\pgfpathlineto{\pgfqpoint{2.840503in}{0.867602in}}%
\pgfpathlineto{\pgfqpoint{2.844451in}{1.774901in}}%
\pgfpathlineto{\pgfqpoint{2.846741in}{0.745769in}}%
\pgfpathlineto{\pgfqpoint{2.850173in}{1.744235in}}%
\pgfpathlineto{\pgfqpoint{2.853461in}{0.768411in}}%
\pgfpathlineto{\pgfqpoint{2.856198in}{1.685818in}}%
\pgfpathlineto{\pgfqpoint{2.859439in}{0.630326in}}%
\pgfpathlineto{\pgfqpoint{2.862243in}{1.715714in}}%
\pgfpathlineto{\pgfqpoint{2.865358in}{0.754921in}}%
\pgfpathlineto{\pgfqpoint{2.868611in}{1.760313in}}%
\pgfpathlineto{\pgfqpoint{2.871835in}{0.644867in}}%
\pgfpathlineto{\pgfqpoint{2.874708in}{1.704982in}}%
\pgfpathlineto{\pgfqpoint{2.879707in}{0.551777in}}%
\pgfpathlineto{\pgfqpoint{2.880830in}{1.667501in}}%
\pgfpathlineto{\pgfqpoint{2.883852in}{0.743325in}}%
\pgfpathlineto{\pgfqpoint{2.887091in}{1.730275in}}%
\pgfpathlineto{\pgfqpoint{2.890779in}{0.762310in}}%
\pgfpathlineto{\pgfqpoint{2.893916in}{1.777451in}}%
\pgfpathlineto{\pgfqpoint{2.896803in}{0.749282in}}%
\pgfpathlineto{\pgfqpoint{2.899452in}{1.767511in}}%
\pgfpathlineto{\pgfqpoint{2.902822in}{0.783301in}}%
\pgfpathlineto{\pgfqpoint{2.905654in}{1.785450in}}%
\pgfpathlineto{\pgfqpoint{2.908755in}{0.834315in}}%
\pgfpathlineto{\pgfqpoint{2.912148in}{1.830131in}}%
\pgfpathlineto{\pgfqpoint{2.915208in}{0.815165in}}%
\pgfpathlineto{\pgfqpoint{2.918379in}{1.826702in}}%
\pgfpathlineto{\pgfqpoint{2.921592in}{0.813746in}}%
\pgfpathlineto{\pgfqpoint{2.924240in}{1.808747in}}%
\pgfpathlineto{\pgfqpoint{2.928232in}{0.835190in}}%
\pgfpathlineto{\pgfqpoint{2.930824in}{1.822982in}}%
\pgfpathlineto{\pgfqpoint{2.934305in}{0.812979in}}%
\pgfpathlineto{\pgfqpoint{2.936828in}{1.889217in}}%
\pgfpathlineto{\pgfqpoint{2.940253in}{0.830487in}}%
\pgfpathlineto{\pgfqpoint{2.942812in}{1.830945in}}%
\pgfpathlineto{\pgfqpoint{2.945912in}{0.861586in}}%
\pgfpathlineto{\pgfqpoint{2.949433in}{1.884236in}}%
\pgfpathlineto{\pgfqpoint{2.952824in}{0.804246in}}%
\pgfpathlineto{\pgfqpoint{2.955205in}{1.778426in}}%
\pgfpathlineto{\pgfqpoint{2.959372in}{0.853959in}}%
\pgfpathlineto{\pgfqpoint{2.961445in}{1.854930in}}%
\pgfpathlineto{\pgfqpoint{2.964373in}{0.942953in}}%
\pgfpathlineto{\pgfqpoint{2.968029in}{1.909100in}}%
\pgfpathlineto{\pgfqpoint{2.970682in}{0.963874in}}%
\pgfpathlineto{\pgfqpoint{2.974669in}{2.009049in}}%
\pgfpathlineto{\pgfqpoint{2.978025in}{0.914436in}}%
\pgfpathlineto{\pgfqpoint{2.980134in}{2.010916in}}%
\pgfpathlineto{\pgfqpoint{2.983024in}{1.066199in}}%
\pgfpathlineto{\pgfqpoint{2.986396in}{2.089498in}}%
\pgfpathlineto{\pgfqpoint{2.989273in}{1.074609in}}%
\pgfpathlineto{\pgfqpoint{2.992518in}{2.057651in}}%
\pgfpathlineto{\pgfqpoint{2.996153in}{1.058603in}}%
\pgfpathlineto{\pgfqpoint{2.998740in}{2.116376in}}%
\pgfpathlineto{\pgfqpoint{3.001548in}{1.054571in}}%
\pgfpathlineto{\pgfqpoint{3.005136in}{2.051913in}}%
\pgfpathlineto{\pgfqpoint{3.007742in}{1.104159in}}%
\pgfpathlineto{\pgfqpoint{3.011195in}{2.060062in}}%
\pgfpathlineto{\pgfqpoint{3.015703in}{0.987518in}}%
\pgfpathlineto{\pgfqpoint{3.017213in}{2.026072in}}%
\pgfpathlineto{\pgfqpoint{3.020235in}{1.091615in}}%
\pgfpathlineto{\pgfqpoint{3.024119in}{2.060192in}}%
\pgfpathlineto{\pgfqpoint{3.026715in}{1.040189in}}%
\pgfpathlineto{\pgfqpoint{3.029885in}{2.140157in}}%
\pgfpathlineto{\pgfqpoint{3.032612in}{1.090748in}}%
\pgfpathlineto{\pgfqpoint{3.035821in}{2.149254in}}%
\pgfpathlineto{\pgfqpoint{3.039775in}{1.095650in}}%
\pgfpathlineto{\pgfqpoint{3.041995in}{2.115382in}}%
\pgfpathlineto{\pgfqpoint{3.045263in}{1.108646in}}%
\pgfpathlineto{\pgfqpoint{3.048520in}{2.108196in}}%
\pgfpathlineto{\pgfqpoint{3.051588in}{1.108352in}}%
\pgfpathlineto{\pgfqpoint{3.055575in}{2.156870in}}%
\pgfpathlineto{\pgfqpoint{3.057777in}{1.103719in}}%
\pgfpathlineto{\pgfqpoint{3.060854in}{2.129942in}}%
\pgfpathlineto{\pgfqpoint{3.063991in}{1.130636in}}%
\pgfpathlineto{\pgfqpoint{3.067076in}{2.125996in}}%
\pgfpathlineto{\pgfqpoint{3.069643in}{1.143259in}}%
\pgfpathlineto{\pgfqpoint{3.073320in}{2.086989in}}%
\pgfpathlineto{\pgfqpoint{3.075899in}{1.188212in}}%
\pgfpathlineto{\pgfqpoint{3.079680in}{2.115543in}}%
\pgfpathlineto{\pgfqpoint{3.082507in}{1.134137in}}%
\pgfpathlineto{\pgfqpoint{3.085562in}{2.242691in}}%
\pgfpathlineto{\pgfqpoint{3.088257in}{1.142268in}}%
\pgfpathlineto{\pgfqpoint{3.091768in}{2.132090in}}%
\pgfpathlineto{\pgfqpoint{3.095586in}{1.013926in}}%
\pgfpathlineto{\pgfqpoint{3.098047in}{2.109079in}}%
\pgfpathlineto{\pgfqpoint{3.100801in}{1.193945in}}%
\pgfpathlineto{\pgfqpoint{3.104049in}{2.152720in}}%
\pgfpathlineto{\pgfqpoint{3.107309in}{1.020879in}}%
\pgfpathlineto{\pgfqpoint{3.111009in}{2.190652in}}%
\pgfpathlineto{\pgfqpoint{3.113023in}{1.148344in}}%
\pgfpathlineto{\pgfqpoint{3.117524in}{2.231626in}}%
\pgfpathlineto{\pgfqpoint{3.119174in}{1.206354in}}%
\pgfpathlineto{\pgfqpoint{3.122599in}{2.151536in}}%
\pgfpathlineto{\pgfqpoint{3.126123in}{1.164058in}}%
\pgfpathlineto{\pgfqpoint{3.128525in}{2.266264in}}%
\pgfpathlineto{\pgfqpoint{3.132170in}{1.159680in}}%
\pgfpathlineto{\pgfqpoint{3.134652in}{2.163097in}}%
\pgfpathlineto{\pgfqpoint{3.137787in}{1.143762in}}%
\pgfpathlineto{\pgfqpoint{3.141903in}{2.147242in}}%
\pgfpathlineto{\pgfqpoint{3.145315in}{1.053187in}}%
\pgfpathlineto{\pgfqpoint{3.147516in}{2.150224in}}%
\pgfpathlineto{\pgfqpoint{3.151352in}{1.041085in}}%
\pgfpathlineto{\pgfqpoint{3.153543in}{2.133475in}}%
\pgfpathlineto{\pgfqpoint{3.156764in}{1.154491in}}%
\pgfpathlineto{\pgfqpoint{3.161393in}{2.217569in}}%
\pgfpathlineto{\pgfqpoint{3.162668in}{1.221786in}}%
\pgfpathlineto{\pgfqpoint{3.165754in}{2.140652in}}%
\pgfpathlineto{\pgfqpoint{3.168956in}{1.191094in}}%
\pgfpathlineto{\pgfqpoint{3.172278in}{2.190552in}}%
\pgfpathlineto{\pgfqpoint{3.175040in}{1.211029in}}%
\pgfpathlineto{\pgfqpoint{3.178331in}{2.141626in}}%
\pgfpathlineto{\pgfqpoint{3.181093in}{1.152387in}}%
\pgfpathlineto{\pgfqpoint{3.184666in}{2.187640in}}%
\pgfpathlineto{\pgfqpoint{3.189015in}{1.104555in}}%
\pgfpathlineto{\pgfqpoint{3.191560in}{2.287249in}}%
\pgfpathlineto{\pgfqpoint{3.193540in}{1.200854in}}%
\pgfpathlineto{\pgfqpoint{3.196931in}{2.116186in}}%
\pgfpathlineto{\pgfqpoint{3.199885in}{1.172236in}}%
\pgfpathlineto{\pgfqpoint{3.203260in}{2.197795in}}%
\pgfpathlineto{\pgfqpoint{3.208278in}{1.045347in}}%
\pgfpathlineto{\pgfqpoint{3.209195in}{2.277182in}}%
\pgfpathlineto{\pgfqpoint{3.212925in}{1.121802in}}%
\pgfpathlineto{\pgfqpoint{3.216182in}{2.259146in}}%
\pgfpathlineto{\pgfqpoint{3.218438in}{1.236711in}}%
\pgfpathlineto{\pgfqpoint{3.222023in}{2.132436in}}%
\pgfpathlineto{\pgfqpoint{3.225650in}{1.166121in}}%
\pgfpathlineto{\pgfqpoint{3.227538in}{2.215892in}}%
\pgfpathlineto{\pgfqpoint{3.231335in}{1.148757in}}%
\pgfpathlineto{\pgfqpoint{3.234019in}{2.116748in}}%
\pgfpathlineto{\pgfqpoint{3.236882in}{1.087955in}}%
\pgfpathlineto{\pgfqpoint{3.240232in}{2.105718in}}%
\pgfpathlineto{\pgfqpoint{3.245247in}{1.052815in}}%
\pgfpathlineto{\pgfqpoint{3.246668in}{2.142523in}}%
\pgfpathlineto{\pgfqpoint{3.249817in}{1.226781in}}%
\pgfpathlineto{\pgfqpoint{3.253023in}{2.236447in}}%
\pgfpathlineto{\pgfqpoint{3.256078in}{1.153415in}}%
\pgfpathlineto{\pgfqpoint{3.259516in}{2.284358in}}%
\pgfpathlineto{\pgfqpoint{3.261748in}{1.141085in}}%
\pgfpathlineto{\pgfqpoint{3.265207in}{2.190654in}}%
\pgfpathlineto{\pgfqpoint{3.267926in}{1.227591in}}%
\pgfpathlineto{\pgfqpoint{3.271298in}{2.187721in}}%
\pgfpathlineto{\pgfqpoint{3.274039in}{1.058130in}}%
\pgfpathlineto{\pgfqpoint{3.277396in}{2.173345in}}%
\pgfpathlineto{\pgfqpoint{3.280431in}{1.103605in}}%
\pgfpathlineto{\pgfqpoint{3.283720in}{2.216093in}}%
\pgfpathlineto{\pgfqpoint{3.286732in}{1.233706in}}%
\pgfpathlineto{\pgfqpoint{3.289755in}{2.224814in}}%
\pgfpathlineto{\pgfqpoint{3.292966in}{1.163428in}}%
\pgfpathlineto{\pgfqpoint{3.295724in}{2.086722in}}%
\pgfpathlineto{\pgfqpoint{3.298767in}{1.172788in}}%
\pgfpathlineto{\pgfqpoint{3.302366in}{2.245857in}}%
\pgfpathlineto{\pgfqpoint{3.305115in}{1.204705in}}%
\pgfpathlineto{\pgfqpoint{3.308100in}{2.221973in}}%
\pgfpathlineto{\pgfqpoint{3.311512in}{1.152524in}}%
\pgfpathlineto{\pgfqpoint{3.314352in}{2.174983in}}%
\pgfpathlineto{\pgfqpoint{3.317366in}{1.177280in}}%
\pgfpathlineto{\pgfqpoint{3.321704in}{2.175747in}}%
\pgfpathlineto{\pgfqpoint{3.323676in}{1.111170in}}%
\pgfpathlineto{\pgfqpoint{3.327084in}{2.147805in}}%
\pgfpathlineto{\pgfqpoint{3.329990in}{1.108999in}}%
\pgfpathlineto{\pgfqpoint{3.332849in}{2.165495in}}%
\pgfpathlineto{\pgfqpoint{3.336702in}{1.182633in}}%
\pgfpathlineto{\pgfqpoint{3.339317in}{2.205944in}}%
\pgfpathlineto{\pgfqpoint{3.342161in}{1.108517in}}%
\pgfpathlineto{\pgfqpoint{3.345235in}{2.077278in}}%
\pgfpathlineto{\pgfqpoint{3.349861in}{1.072344in}}%
\pgfpathlineto{\pgfqpoint{3.351670in}{2.115844in}}%
\pgfpathlineto{\pgfqpoint{3.355841in}{1.144066in}}%
\pgfpathlineto{\pgfqpoint{3.357801in}{2.202958in}}%
\pgfpathlineto{\pgfqpoint{3.360767in}{1.260308in}}%
\pgfpathlineto{\pgfqpoint{3.363898in}{2.140886in}}%
\pgfpathlineto{\pgfqpoint{3.367168in}{1.172115in}}%
\pgfpathlineto{\pgfqpoint{3.370061in}{2.257481in}}%
\pgfpathlineto{\pgfqpoint{3.373498in}{1.220842in}}%
\pgfpathlineto{\pgfqpoint{3.376269in}{2.165500in}}%
\pgfpathlineto{\pgfqpoint{3.379641in}{1.244985in}}%
\pgfpathlineto{\pgfqpoint{3.383412in}{2.232983in}}%
\pgfpathlineto{\pgfqpoint{3.385490in}{1.255587in}}%
\pgfpathlineto{\pgfqpoint{3.389074in}{2.232475in}}%
\pgfpathlineto{\pgfqpoint{3.392530in}{1.202951in}}%
\pgfpathlineto{\pgfqpoint{3.394956in}{2.226570in}}%
\pgfpathlineto{\pgfqpoint{3.397871in}{1.233427in}}%
\pgfpathlineto{\pgfqpoint{3.401166in}{2.189840in}}%
\pgfpathlineto{\pgfqpoint{3.405027in}{1.159462in}}%
\pgfpathlineto{\pgfqpoint{3.407206in}{2.182023in}}%
\pgfpathlineto{\pgfqpoint{3.411294in}{1.086386in}}%
\pgfpathlineto{\pgfqpoint{3.413766in}{2.246122in}}%
\pgfpathlineto{\pgfqpoint{3.416557in}{1.266528in}}%
\pgfpathlineto{\pgfqpoint{3.419901in}{2.261986in}}%
\pgfpathlineto{\pgfqpoint{3.422690in}{1.242741in}}%
\pgfpathlineto{\pgfqpoint{3.425913in}{2.244927in}}%
\pgfpathlineto{\pgfqpoint{3.429001in}{1.218648in}}%
\pgfpathlineto{\pgfqpoint{3.432301in}{2.222558in}}%
\pgfpathlineto{\pgfqpoint{3.435348in}{1.201857in}}%
\pgfpathlineto{\pgfqpoint{3.438514in}{2.207047in}}%
\pgfpathlineto{\pgfqpoint{3.442165in}{1.209792in}}%
\pgfpathlineto{\pgfqpoint{3.444290in}{2.282323in}}%
\pgfpathlineto{\pgfqpoint{3.448067in}{1.248726in}}%
\pgfpathlineto{\pgfqpoint{3.451869in}{2.259948in}}%
\pgfpathlineto{\pgfqpoint{3.454666in}{1.184053in}}%
\pgfpathlineto{\pgfqpoint{3.456934in}{2.111278in}}%
\pgfpathlineto{\pgfqpoint{3.460024in}{1.229675in}}%
\pgfpathlineto{\pgfqpoint{3.463550in}{2.187228in}}%
\pgfpathlineto{\pgfqpoint{3.466438in}{1.135976in}}%
\pgfpathlineto{\pgfqpoint{3.469081in}{2.210116in}}%
\pgfpathlineto{\pgfqpoint{3.472184in}{1.213009in}}%
\pgfpathlineto{\pgfqpoint{3.475807in}{2.197324in}}%
\pgfpathlineto{\pgfqpoint{3.479243in}{1.207641in}}%
\pgfpathlineto{\pgfqpoint{3.481889in}{2.145370in}}%
\pgfpathlineto{\pgfqpoint{3.485159in}{1.173837in}}%
\pgfpathlineto{\pgfqpoint{3.488928in}{2.184268in}}%
\pgfpathlineto{\pgfqpoint{3.490888in}{1.180901in}}%
\pgfpathlineto{\pgfqpoint{3.494874in}{2.275832in}}%
\pgfpathlineto{\pgfqpoint{3.497320in}{1.179099in}}%
\pgfpathlineto{\pgfqpoint{3.501508in}{2.172309in}}%
\pgfpathlineto{\pgfqpoint{3.503206in}{1.178215in}}%
\pgfpathlineto{\pgfqpoint{3.506331in}{2.101553in}}%
\pgfpathlineto{\pgfqpoint{3.509456in}{1.152182in}}%
\pgfpathlineto{\pgfqpoint{3.512738in}{2.156626in}}%
\pgfpathlineto{\pgfqpoint{3.515853in}{1.182796in}}%
\pgfpathlineto{\pgfqpoint{3.518679in}{2.127870in}}%
\pgfpathlineto{\pgfqpoint{3.522807in}{1.129659in}}%
\pgfpathlineto{\pgfqpoint{3.524858in}{2.159841in}}%
\pgfpathlineto{\pgfqpoint{3.527904in}{1.176611in}}%
\pgfpathlineto{\pgfqpoint{3.531286in}{2.171293in}}%
\pgfpathlineto{\pgfqpoint{3.534279in}{1.205229in}}%
\pgfpathlineto{\pgfqpoint{3.538018in}{2.156584in}}%
\pgfpathlineto{\pgfqpoint{3.541189in}{1.115504in}}%
\pgfpathlineto{\pgfqpoint{3.543498in}{2.142552in}}%
\pgfpathlineto{\pgfqpoint{3.546575in}{1.225501in}}%
\pgfpathlineto{\pgfqpoint{3.550091in}{2.130148in}}%
\pgfpathlineto{\pgfqpoint{3.552865in}{1.125731in}}%
\pgfpathlineto{\pgfqpoint{3.556356in}{2.232123in}}%
\pgfpathlineto{\pgfqpoint{3.559226in}{1.057722in}}%
\pgfpathlineto{\pgfqpoint{3.562090in}{2.184102in}}%
\pgfpathlineto{\pgfqpoint{3.566357in}{1.078651in}}%
\pgfpathlineto{\pgfqpoint{3.568304in}{2.115827in}}%
\pgfpathlineto{\pgfqpoint{3.571721in}{1.174131in}}%
\pgfpathlineto{\pgfqpoint{3.575195in}{2.175014in}}%
\pgfpathlineto{\pgfqpoint{3.577893in}{1.222409in}}%
\pgfpathlineto{\pgfqpoint{3.580646in}{2.167908in}}%
\pgfpathlineto{\pgfqpoint{3.584371in}{1.081170in}}%
\pgfpathlineto{\pgfqpoint{3.587409in}{2.180374in}}%
\pgfpathlineto{\pgfqpoint{3.590025in}{1.095128in}}%
\pgfpathlineto{\pgfqpoint{3.593695in}{2.095097in}}%
\pgfpathlineto{\pgfqpoint{3.596260in}{1.148190in}}%
\pgfpathlineto{\pgfqpoint{3.599296in}{2.098725in}}%
\pgfpathlineto{\pgfqpoint{3.603757in}{1.069476in}}%
\pgfpathlineto{\pgfqpoint{3.605354in}{2.132930in}}%
\pgfpathlineto{\pgfqpoint{3.609098in}{1.117512in}}%
\pgfpathlineto{\pgfqpoint{3.611734in}{2.241785in}}%
\pgfpathlineto{\pgfqpoint{3.614885in}{1.143460in}}%
\pgfpathlineto{\pgfqpoint{3.617753in}{2.166962in}}%
\pgfpathlineto{\pgfqpoint{3.621275in}{1.114458in}}%
\pgfpathlineto{\pgfqpoint{3.626045in}{2.327946in}}%
\pgfpathlineto{\pgfqpoint{3.627716in}{1.198076in}}%
\pgfpathlineto{\pgfqpoint{3.630776in}{2.233241in}}%
\pgfpathlineto{\pgfqpoint{3.633465in}{1.162894in}}%
\pgfpathlineto{\pgfqpoint{3.636977in}{2.194581in}}%
\pgfpathlineto{\pgfqpoint{3.639382in}{1.245709in}}%
\pgfpathlineto{\pgfqpoint{3.644617in}{2.304238in}}%
\pgfpathlineto{\pgfqpoint{3.645762in}{1.272723in}}%
\pgfpathlineto{\pgfqpoint{3.648883in}{2.200823in}}%
\pgfpathlineto{\pgfqpoint{3.653051in}{1.123675in}}%
\pgfpathlineto{\pgfqpoint{3.654941in}{2.183745in}}%
\pgfpathlineto{\pgfqpoint{3.658342in}{1.196663in}}%
\pgfpathlineto{\pgfqpoint{3.661651in}{2.222683in}}%
\pgfpathlineto{\pgfqpoint{3.664759in}{1.154717in}}%
\pgfpathlineto{\pgfqpoint{3.668146in}{2.255017in}}%
\pgfpathlineto{\pgfqpoint{3.670388in}{1.194941in}}%
\pgfpathlineto{\pgfqpoint{3.673527in}{2.322298in}}%
\pgfpathlineto{\pgfqpoint{3.676882in}{1.260639in}}%
\pgfpathlineto{\pgfqpoint{3.681611in}{2.307266in}}%
\pgfpathlineto{\pgfqpoint{3.683248in}{1.278412in}}%
\pgfpathlineto{\pgfqpoint{3.686046in}{2.279383in}}%
\pgfpathlineto{\pgfqpoint{3.689143in}{1.265001in}}%
\pgfpathlineto{\pgfqpoint{3.692668in}{2.261825in}}%
\pgfpathlineto{\pgfqpoint{3.695872in}{1.291000in}}%
\pgfpathlineto{\pgfqpoint{3.698172in}{2.289700in}}%
\pgfpathlineto{\pgfqpoint{3.701311in}{1.334439in}}%
\pgfpathlineto{\pgfqpoint{3.704506in}{2.279686in}}%
\pgfpathlineto{\pgfqpoint{3.707602in}{1.332628in}}%
\pgfpathlineto{\pgfqpoint{3.710804in}{2.336382in}}%
\pgfpathlineto{\pgfqpoint{3.713748in}{1.361871in}}%
\pgfpathlineto{\pgfqpoint{3.716809in}{2.340835in}}%
\pgfpathlineto{\pgfqpoint{3.720142in}{1.388516in}}%
\pgfpathlineto{\pgfqpoint{3.723578in}{2.324893in}}%
\pgfpathlineto{\pgfqpoint{3.726766in}{1.263075in}}%
\pgfpathlineto{\pgfqpoint{3.729917in}{2.330861in}}%
\pgfpathlineto{\pgfqpoint{3.732702in}{1.325851in}}%
\pgfpathlineto{\pgfqpoint{3.735848in}{2.350469in}}%
\pgfpathlineto{\pgfqpoint{3.738843in}{1.352877in}}%
\pgfpathlineto{\pgfqpoint{3.741655in}{2.347272in}}%
\pgfpathlineto{\pgfqpoint{3.744940in}{1.371811in}}%
\pgfpathlineto{\pgfqpoint{3.748203in}{2.282375in}}%
\pgfpathlineto{\pgfqpoint{3.750866in}{1.390164in}}%
\pgfpathlineto{\pgfqpoint{3.754119in}{2.282886in}}%
\pgfpathlineto{\pgfqpoint{3.757119in}{1.288935in}}%
\pgfpathlineto{\pgfqpoint{3.760390in}{2.340751in}}%
\pgfpathlineto{\pgfqpoint{3.763334in}{1.364268in}}%
\pgfpathlineto{\pgfqpoint{3.766348in}{2.263976in}}%
\pgfpathlineto{\pgfqpoint{3.769504in}{1.362879in}}%
\pgfpathlineto{\pgfqpoint{3.773591in}{2.324090in}}%
\pgfpathlineto{\pgfqpoint{3.775677in}{1.355020in}}%
\pgfpathlineto{\pgfqpoint{3.778736in}{2.257003in}}%
\pgfpathlineto{\pgfqpoint{3.782024in}{1.307885in}}%
\pgfpathlineto{\pgfqpoint{3.785779in}{2.352014in}}%
\pgfpathlineto{\pgfqpoint{3.790087in}{1.138459in}}%
\pgfpathlineto{\pgfqpoint{3.791076in}{2.159669in}}%
\pgfpathlineto{\pgfqpoint{3.795146in}{1.246066in}}%
\pgfpathlineto{\pgfqpoint{3.797317in}{2.174220in}}%
\pgfpathlineto{\pgfqpoint{3.800619in}{1.222292in}}%
\pgfpathlineto{\pgfqpoint{3.803446in}{2.379178in}}%
\pgfpathlineto{\pgfqpoint{3.806707in}{1.273600in}}%
\pgfpathlineto{\pgfqpoint{3.810962in}{2.301196in}}%
\pgfpathlineto{\pgfqpoint{3.813047in}{1.301876in}}%
\pgfpathlineto{\pgfqpoint{3.816785in}{2.257857in}}%
\pgfpathlineto{\pgfqpoint{3.819803in}{1.277645in}}%
\pgfpathlineto{\pgfqpoint{3.822719in}{2.362293in}}%
\pgfpathlineto{\pgfqpoint{3.826051in}{1.214922in}}%
\pgfpathlineto{\pgfqpoint{3.828409in}{2.239411in}}%
\pgfpathlineto{\pgfqpoint{3.831643in}{1.297881in}}%
\pgfpathlineto{\pgfqpoint{3.834405in}{1.574171in}}%
\pgfpathlineto{\pgfqpoint{3.834405in}{1.574171in}}%
\pgfusepath{stroke}%
\end{pgfscope}%
\begin{pgfscope}%
\pgfsetrectcap%
\pgfsetmiterjoin%
\pgfsetlinewidth{0.803000pt}%
\definecolor{currentstroke}{rgb}{0.000000,0.000000,0.000000}%
\pgfsetstrokecolor{currentstroke}%
\pgfsetdash{}{0pt}%
\pgfpathmoveto{\pgfqpoint{0.589745in}{0.416447in}}%
\pgfpathlineto{\pgfqpoint{0.589745in}{2.472642in}}%
\pgfusepath{stroke}%
\end{pgfscope}%
\begin{pgfscope}%
\pgfsetrectcap%
\pgfsetmiterjoin%
\pgfsetlinewidth{0.803000pt}%
\definecolor{currentstroke}{rgb}{0.000000,0.000000,0.000000}%
\pgfsetstrokecolor{currentstroke}%
\pgfsetdash{}{0pt}%
\pgfpathmoveto{\pgfqpoint{3.988913in}{0.416447in}}%
\pgfpathlineto{\pgfqpoint{3.988913in}{2.472642in}}%
\pgfusepath{stroke}%
\end{pgfscope}%
\begin{pgfscope}%
\pgfsetrectcap%
\pgfsetmiterjoin%
\pgfsetlinewidth{0.803000pt}%
\definecolor{currentstroke}{rgb}{0.000000,0.000000,0.000000}%
\pgfsetstrokecolor{currentstroke}%
\pgfsetdash{}{0pt}%
\pgfpathmoveto{\pgfqpoint{0.589745in}{0.416447in}}%
\pgfpathlineto{\pgfqpoint{3.988913in}{0.416447in}}%
\pgfusepath{stroke}%
\end{pgfscope}%
\begin{pgfscope}%
\pgfsetrectcap%
\pgfsetmiterjoin%
\pgfsetlinewidth{0.803000pt}%
\definecolor{currentstroke}{rgb}{0.000000,0.000000,0.000000}%
\pgfsetstrokecolor{currentstroke}%
\pgfsetdash{}{0pt}%
\pgfpathmoveto{\pgfqpoint{0.589745in}{2.472642in}}%
\pgfpathlineto{\pgfqpoint{3.988913in}{2.472642in}}%
\pgfusepath{stroke}%
\end{pgfscope}%
\end{pgfpicture}%
\makeatother%
\endgroup%
% data/simulations/sim_allan_variance_example.py
    \caption{A simulated time series containing white noise, flicker noise and random walk behaviour.}
    \label{fig:adev_example_time}
\end{figure}

A common approach to identify noise sources is the power spectrum. It is easily accessible, even in real-time using spectrum analysers and, utilizing the computational power of modern computers, large time-domain data sets can be converted using the fast Fourier transform, making this the method of choice in the lab. The power spectral density  of figure \ref{fig:adev_example_time} is shown in figure \ref{fig:adev_example_psd}. It allows to clearly separate the white noise part from the other $f^{\alpha}$ components. The dashed lines representing the individual components were plotted using the $h_\alpha$ values calculated from the input parameters of the simulation. The noise spectral density $h_0$ of the white noise signal can be easily extracted even by hand without resorting to a fit. This yields $h_{0} = \qty{2e-3}{\per \Hz} $. $h_{-1}$ and $h_{-2}$ can be extracted as well using a fit to
\begin{equation}
    S(f) = \sum_{\alpha = -2}^0 h_\alpha f^\alpha \, .
\end{equation}
The noise corner frequency $f_c$ can either be calculated from $h_0$ and $h_{-1}$ using equation \ref{eqn:corner_frequency} or determined graphically by constructing a tangent with a slope of $-1$ to the spectral density. From the intersection of the blue $h_0$ line and the green $h_{-1}$ line the corner frequency is found to be $f_c \approx \qty{1.8}{\kHz}$.
\begin{figure}[hb]
    \centering
    %% Creator: Matplotlib, PGF backend
%%
%% To include the figure in your LaTeX document, write
%%   \input{<filename>.pgf}
%%
%% Make sure the required packages are loaded in your preamble
%%   \usepackage{pgf}
%%
%% Also ensure that all the required font packages are loaded; for instance,
%% the lmodern package is sometimes necessary when using math font.
%%   \usepackage{lmodern}
%%
%% Figures using additional raster images can only be included by \input if
%% they are in the same directory as the main LaTeX file. For loading figures
%% from other directories you can use the `import` package
%%   \usepackage{import}
%%
%% and then include the figures with
%%   \import{<path to file>}{<filename>.pgf}
%%
%% Matplotlib used the following preamble
%%   \def\mathdefault#1{#1}
%%   \everymath=\expandafter{\the\everymath\displaystyle}
%%   \usepackage{siunitx}
%%   \sisetup{per-mode = symbol}%
%%   \ifdefined\pdftexversion\else  % non-pdftex case.
%%     \usepackage{fontspec}
%%   \fi
%%   \makeatletter\@ifpackageloaded{underscore}{}{\usepackage[strings]{underscore}}\makeatother
%%
\begingroup%
\makeatletter%
\begin{pgfpicture}%
\pgfpathrectangle{\pgfpointorigin}{\pgfqpoint{4.068242in}{2.514312in}}%
\pgfusepath{use as bounding box, clip}%
\begin{pgfscope}%
\pgfsetbuttcap%
\pgfsetmiterjoin%
\definecolor{currentfill}{rgb}{1.000000,1.000000,1.000000}%
\pgfsetfillcolor{currentfill}%
\pgfsetlinewidth{0.000000pt}%
\definecolor{currentstroke}{rgb}{1.000000,1.000000,1.000000}%
\pgfsetstrokecolor{currentstroke}%
\pgfsetdash{}{0pt}%
\pgfpathmoveto{\pgfqpoint{0.000000in}{0.000000in}}%
\pgfpathlineto{\pgfqpoint{4.068242in}{0.000000in}}%
\pgfpathlineto{\pgfqpoint{4.068242in}{2.514312in}}%
\pgfpathlineto{\pgfqpoint{0.000000in}{2.514312in}}%
\pgfpathlineto{\pgfqpoint{0.000000in}{0.000000in}}%
\pgfpathclose%
\pgfusepath{fill}%
\end{pgfscope}%
\begin{pgfscope}%
\pgfsetbuttcap%
\pgfsetmiterjoin%
\definecolor{currentfill}{rgb}{1.000000,1.000000,1.000000}%
\pgfsetfillcolor{currentfill}%
\pgfsetlinewidth{0.000000pt}%
\definecolor{currentstroke}{rgb}{0.000000,0.000000,0.000000}%
\pgfsetstrokecolor{currentstroke}%
\pgfsetstrokeopacity{0.000000}%
\pgfsetdash{}{0pt}%
\pgfpathmoveto{\pgfqpoint{0.594525in}{0.417642in}}%
\pgfpathlineto{\pgfqpoint{3.948753in}{0.417642in}}%
\pgfpathlineto{\pgfqpoint{3.948753in}{2.472642in}}%
\pgfpathlineto{\pgfqpoint{0.594525in}{2.472642in}}%
\pgfpathlineto{\pgfqpoint{0.594525in}{0.417642in}}%
\pgfpathclose%
\pgfusepath{fill}%
\end{pgfscope}%
\begin{pgfscope}%
\pgfpathrectangle{\pgfqpoint{0.594525in}{0.417642in}}{\pgfqpoint{3.354228in}{2.055000in}}%
\pgfusepath{clip}%
\pgfsetrectcap%
\pgfsetroundjoin%
\pgfsetlinewidth{0.803000pt}%
\definecolor{currentstroke}{rgb}{0.450000,0.450000,0.450000}%
\pgfsetstrokecolor{currentstroke}%
\pgfsetdash{}{0pt}%
\pgfpathmoveto{\pgfqpoint{0.711813in}{0.417642in}}%
\pgfpathlineto{\pgfqpoint{0.711813in}{2.472642in}}%
\pgfusepath{stroke}%
\end{pgfscope}%
\begin{pgfscope}%
\pgfsetbuttcap%
\pgfsetroundjoin%
\definecolor{currentfill}{rgb}{0.000000,0.000000,0.000000}%
\pgfsetfillcolor{currentfill}%
\pgfsetlinewidth{0.803000pt}%
\definecolor{currentstroke}{rgb}{0.000000,0.000000,0.000000}%
\pgfsetstrokecolor{currentstroke}%
\pgfsetdash{}{0pt}%
\pgfsys@defobject{currentmarker}{\pgfqpoint{0.000000in}{-0.048611in}}{\pgfqpoint{0.000000in}{0.000000in}}{%
\pgfpathmoveto{\pgfqpoint{0.000000in}{0.000000in}}%
\pgfpathlineto{\pgfqpoint{0.000000in}{-0.048611in}}%
\pgfusepath{stroke,fill}%
}%
\begin{pgfscope}%
\pgfsys@transformshift{0.711813in}{0.417642in}%
\pgfsys@useobject{currentmarker}{}%
\end{pgfscope}%
\end{pgfscope}%
\begin{pgfscope}%
\definecolor{textcolor}{rgb}{0.000000,0.000000,0.000000}%
\pgfsetstrokecolor{textcolor}%
\pgfsetfillcolor{textcolor}%
\pgftext[x=0.711813in,y=0.320420in,,top]{\color{textcolor}{\rmfamily\fontsize{8.000000}{9.600000}\selectfont\catcode`\^=\active\def^{\ifmmode\sp\else\^{}\fi}\catcode`\%=\active\def%{\%}$\mathdefault{10^{-1}}$}}%
\end{pgfscope}%
\begin{pgfscope}%
\pgfpathrectangle{\pgfqpoint{0.594525in}{0.417642in}}{\pgfqpoint{3.354228in}{2.055000in}}%
\pgfusepath{clip}%
\pgfsetrectcap%
\pgfsetroundjoin%
\pgfsetlinewidth{0.803000pt}%
\definecolor{currentstroke}{rgb}{0.450000,0.450000,0.450000}%
\pgfsetstrokecolor{currentstroke}%
\pgfsetdash{}{0pt}%
\pgfpathmoveto{\pgfqpoint{1.172783in}{0.417642in}}%
\pgfpathlineto{\pgfqpoint{1.172783in}{2.472642in}}%
\pgfusepath{stroke}%
\end{pgfscope}%
\begin{pgfscope}%
\pgfsetbuttcap%
\pgfsetroundjoin%
\definecolor{currentfill}{rgb}{0.000000,0.000000,0.000000}%
\pgfsetfillcolor{currentfill}%
\pgfsetlinewidth{0.803000pt}%
\definecolor{currentstroke}{rgb}{0.000000,0.000000,0.000000}%
\pgfsetstrokecolor{currentstroke}%
\pgfsetdash{}{0pt}%
\pgfsys@defobject{currentmarker}{\pgfqpoint{0.000000in}{-0.048611in}}{\pgfqpoint{0.000000in}{0.000000in}}{%
\pgfpathmoveto{\pgfqpoint{0.000000in}{0.000000in}}%
\pgfpathlineto{\pgfqpoint{0.000000in}{-0.048611in}}%
\pgfusepath{stroke,fill}%
}%
\begin{pgfscope}%
\pgfsys@transformshift{1.172783in}{0.417642in}%
\pgfsys@useobject{currentmarker}{}%
\end{pgfscope}%
\end{pgfscope}%
\begin{pgfscope}%
\definecolor{textcolor}{rgb}{0.000000,0.000000,0.000000}%
\pgfsetstrokecolor{textcolor}%
\pgfsetfillcolor{textcolor}%
\pgftext[x=1.172783in,y=0.320420in,,top]{\color{textcolor}{\rmfamily\fontsize{8.000000}{9.600000}\selectfont\catcode`\^=\active\def^{\ifmmode\sp\else\^{}\fi}\catcode`\%=\active\def%{\%}$\mathdefault{10^{0}}$}}%
\end{pgfscope}%
\begin{pgfscope}%
\pgfpathrectangle{\pgfqpoint{0.594525in}{0.417642in}}{\pgfqpoint{3.354228in}{2.055000in}}%
\pgfusepath{clip}%
\pgfsetrectcap%
\pgfsetroundjoin%
\pgfsetlinewidth{0.803000pt}%
\definecolor{currentstroke}{rgb}{0.450000,0.450000,0.450000}%
\pgfsetstrokecolor{currentstroke}%
\pgfsetdash{}{0pt}%
\pgfpathmoveto{\pgfqpoint{1.633754in}{0.417642in}}%
\pgfpathlineto{\pgfqpoint{1.633754in}{2.472642in}}%
\pgfusepath{stroke}%
\end{pgfscope}%
\begin{pgfscope}%
\pgfsetbuttcap%
\pgfsetroundjoin%
\definecolor{currentfill}{rgb}{0.000000,0.000000,0.000000}%
\pgfsetfillcolor{currentfill}%
\pgfsetlinewidth{0.803000pt}%
\definecolor{currentstroke}{rgb}{0.000000,0.000000,0.000000}%
\pgfsetstrokecolor{currentstroke}%
\pgfsetdash{}{0pt}%
\pgfsys@defobject{currentmarker}{\pgfqpoint{0.000000in}{-0.048611in}}{\pgfqpoint{0.000000in}{0.000000in}}{%
\pgfpathmoveto{\pgfqpoint{0.000000in}{0.000000in}}%
\pgfpathlineto{\pgfqpoint{0.000000in}{-0.048611in}}%
\pgfusepath{stroke,fill}%
}%
\begin{pgfscope}%
\pgfsys@transformshift{1.633754in}{0.417642in}%
\pgfsys@useobject{currentmarker}{}%
\end{pgfscope}%
\end{pgfscope}%
\begin{pgfscope}%
\definecolor{textcolor}{rgb}{0.000000,0.000000,0.000000}%
\pgfsetstrokecolor{textcolor}%
\pgfsetfillcolor{textcolor}%
\pgftext[x=1.633754in,y=0.320420in,,top]{\color{textcolor}{\rmfamily\fontsize{8.000000}{9.600000}\selectfont\catcode`\^=\active\def^{\ifmmode\sp\else\^{}\fi}\catcode`\%=\active\def%{\%}$\mathdefault{10^{1}}$}}%
\end{pgfscope}%
\begin{pgfscope}%
\pgfpathrectangle{\pgfqpoint{0.594525in}{0.417642in}}{\pgfqpoint{3.354228in}{2.055000in}}%
\pgfusepath{clip}%
\pgfsetrectcap%
\pgfsetroundjoin%
\pgfsetlinewidth{0.803000pt}%
\definecolor{currentstroke}{rgb}{0.450000,0.450000,0.450000}%
\pgfsetstrokecolor{currentstroke}%
\pgfsetdash{}{0pt}%
\pgfpathmoveto{\pgfqpoint{2.094724in}{0.417642in}}%
\pgfpathlineto{\pgfqpoint{2.094724in}{2.472642in}}%
\pgfusepath{stroke}%
\end{pgfscope}%
\begin{pgfscope}%
\pgfsetbuttcap%
\pgfsetroundjoin%
\definecolor{currentfill}{rgb}{0.000000,0.000000,0.000000}%
\pgfsetfillcolor{currentfill}%
\pgfsetlinewidth{0.803000pt}%
\definecolor{currentstroke}{rgb}{0.000000,0.000000,0.000000}%
\pgfsetstrokecolor{currentstroke}%
\pgfsetdash{}{0pt}%
\pgfsys@defobject{currentmarker}{\pgfqpoint{0.000000in}{-0.048611in}}{\pgfqpoint{0.000000in}{0.000000in}}{%
\pgfpathmoveto{\pgfqpoint{0.000000in}{0.000000in}}%
\pgfpathlineto{\pgfqpoint{0.000000in}{-0.048611in}}%
\pgfusepath{stroke,fill}%
}%
\begin{pgfscope}%
\pgfsys@transformshift{2.094724in}{0.417642in}%
\pgfsys@useobject{currentmarker}{}%
\end{pgfscope}%
\end{pgfscope}%
\begin{pgfscope}%
\definecolor{textcolor}{rgb}{0.000000,0.000000,0.000000}%
\pgfsetstrokecolor{textcolor}%
\pgfsetfillcolor{textcolor}%
\pgftext[x=2.094724in,y=0.320420in,,top]{\color{textcolor}{\rmfamily\fontsize{8.000000}{9.600000}\selectfont\catcode`\^=\active\def^{\ifmmode\sp\else\^{}\fi}\catcode`\%=\active\def%{\%}$\mathdefault{10^{2}}$}}%
\end{pgfscope}%
\begin{pgfscope}%
\pgfpathrectangle{\pgfqpoint{0.594525in}{0.417642in}}{\pgfqpoint{3.354228in}{2.055000in}}%
\pgfusepath{clip}%
\pgfsetrectcap%
\pgfsetroundjoin%
\pgfsetlinewidth{0.803000pt}%
\definecolor{currentstroke}{rgb}{0.450000,0.450000,0.450000}%
\pgfsetstrokecolor{currentstroke}%
\pgfsetdash{}{0pt}%
\pgfpathmoveto{\pgfqpoint{2.555695in}{0.417642in}}%
\pgfpathlineto{\pgfqpoint{2.555695in}{2.472642in}}%
\pgfusepath{stroke}%
\end{pgfscope}%
\begin{pgfscope}%
\pgfsetbuttcap%
\pgfsetroundjoin%
\definecolor{currentfill}{rgb}{0.000000,0.000000,0.000000}%
\pgfsetfillcolor{currentfill}%
\pgfsetlinewidth{0.803000pt}%
\definecolor{currentstroke}{rgb}{0.000000,0.000000,0.000000}%
\pgfsetstrokecolor{currentstroke}%
\pgfsetdash{}{0pt}%
\pgfsys@defobject{currentmarker}{\pgfqpoint{0.000000in}{-0.048611in}}{\pgfqpoint{0.000000in}{0.000000in}}{%
\pgfpathmoveto{\pgfqpoint{0.000000in}{0.000000in}}%
\pgfpathlineto{\pgfqpoint{0.000000in}{-0.048611in}}%
\pgfusepath{stroke,fill}%
}%
\begin{pgfscope}%
\pgfsys@transformshift{2.555695in}{0.417642in}%
\pgfsys@useobject{currentmarker}{}%
\end{pgfscope}%
\end{pgfscope}%
\begin{pgfscope}%
\definecolor{textcolor}{rgb}{0.000000,0.000000,0.000000}%
\pgfsetstrokecolor{textcolor}%
\pgfsetfillcolor{textcolor}%
\pgftext[x=2.555695in,y=0.320420in,,top]{\color{textcolor}{\rmfamily\fontsize{8.000000}{9.600000}\selectfont\catcode`\^=\active\def^{\ifmmode\sp\else\^{}\fi}\catcode`\%=\active\def%{\%}$\mathdefault{10^{3}}$}}%
\end{pgfscope}%
\begin{pgfscope}%
\pgfpathrectangle{\pgfqpoint{0.594525in}{0.417642in}}{\pgfqpoint{3.354228in}{2.055000in}}%
\pgfusepath{clip}%
\pgfsetrectcap%
\pgfsetroundjoin%
\pgfsetlinewidth{0.803000pt}%
\definecolor{currentstroke}{rgb}{0.450000,0.450000,0.450000}%
\pgfsetstrokecolor{currentstroke}%
\pgfsetdash{}{0pt}%
\pgfpathmoveto{\pgfqpoint{3.016665in}{0.417642in}}%
\pgfpathlineto{\pgfqpoint{3.016665in}{2.472642in}}%
\pgfusepath{stroke}%
\end{pgfscope}%
\begin{pgfscope}%
\pgfsetbuttcap%
\pgfsetroundjoin%
\definecolor{currentfill}{rgb}{0.000000,0.000000,0.000000}%
\pgfsetfillcolor{currentfill}%
\pgfsetlinewidth{0.803000pt}%
\definecolor{currentstroke}{rgb}{0.000000,0.000000,0.000000}%
\pgfsetstrokecolor{currentstroke}%
\pgfsetdash{}{0pt}%
\pgfsys@defobject{currentmarker}{\pgfqpoint{0.000000in}{-0.048611in}}{\pgfqpoint{0.000000in}{0.000000in}}{%
\pgfpathmoveto{\pgfqpoint{0.000000in}{0.000000in}}%
\pgfpathlineto{\pgfqpoint{0.000000in}{-0.048611in}}%
\pgfusepath{stroke,fill}%
}%
\begin{pgfscope}%
\pgfsys@transformshift{3.016665in}{0.417642in}%
\pgfsys@useobject{currentmarker}{}%
\end{pgfscope}%
\end{pgfscope}%
\begin{pgfscope}%
\definecolor{textcolor}{rgb}{0.000000,0.000000,0.000000}%
\pgfsetstrokecolor{textcolor}%
\pgfsetfillcolor{textcolor}%
\pgftext[x=3.016665in,y=0.320420in,,top]{\color{textcolor}{\rmfamily\fontsize{8.000000}{9.600000}\selectfont\catcode`\^=\active\def^{\ifmmode\sp\else\^{}\fi}\catcode`\%=\active\def%{\%}$\mathdefault{10^{4}}$}}%
\end{pgfscope}%
\begin{pgfscope}%
\pgfpathrectangle{\pgfqpoint{0.594525in}{0.417642in}}{\pgfqpoint{3.354228in}{2.055000in}}%
\pgfusepath{clip}%
\pgfsetrectcap%
\pgfsetroundjoin%
\pgfsetlinewidth{0.803000pt}%
\definecolor{currentstroke}{rgb}{0.450000,0.450000,0.450000}%
\pgfsetstrokecolor{currentstroke}%
\pgfsetdash{}{0pt}%
\pgfpathmoveto{\pgfqpoint{3.477636in}{0.417642in}}%
\pgfpathlineto{\pgfqpoint{3.477636in}{2.472642in}}%
\pgfusepath{stroke}%
\end{pgfscope}%
\begin{pgfscope}%
\pgfsetbuttcap%
\pgfsetroundjoin%
\definecolor{currentfill}{rgb}{0.000000,0.000000,0.000000}%
\pgfsetfillcolor{currentfill}%
\pgfsetlinewidth{0.803000pt}%
\definecolor{currentstroke}{rgb}{0.000000,0.000000,0.000000}%
\pgfsetstrokecolor{currentstroke}%
\pgfsetdash{}{0pt}%
\pgfsys@defobject{currentmarker}{\pgfqpoint{0.000000in}{-0.048611in}}{\pgfqpoint{0.000000in}{0.000000in}}{%
\pgfpathmoveto{\pgfqpoint{0.000000in}{0.000000in}}%
\pgfpathlineto{\pgfqpoint{0.000000in}{-0.048611in}}%
\pgfusepath{stroke,fill}%
}%
\begin{pgfscope}%
\pgfsys@transformshift{3.477636in}{0.417642in}%
\pgfsys@useobject{currentmarker}{}%
\end{pgfscope}%
\end{pgfscope}%
\begin{pgfscope}%
\definecolor{textcolor}{rgb}{0.000000,0.000000,0.000000}%
\pgfsetstrokecolor{textcolor}%
\pgfsetfillcolor{textcolor}%
\pgftext[x=3.477636in,y=0.320420in,,top]{\color{textcolor}{\rmfamily\fontsize{8.000000}{9.600000}\selectfont\catcode`\^=\active\def^{\ifmmode\sp\else\^{}\fi}\catcode`\%=\active\def%{\%}$\mathdefault{10^{5}}$}}%
\end{pgfscope}%
\begin{pgfscope}%
\pgfpathrectangle{\pgfqpoint{0.594525in}{0.417642in}}{\pgfqpoint{3.354228in}{2.055000in}}%
\pgfusepath{clip}%
\pgfsetrectcap%
\pgfsetroundjoin%
\pgfsetlinewidth{0.803000pt}%
\definecolor{currentstroke}{rgb}{0.450000,0.450000,0.450000}%
\pgfsetstrokecolor{currentstroke}%
\pgfsetdash{}{0pt}%
\pgfpathmoveto{\pgfqpoint{3.938606in}{0.417642in}}%
\pgfpathlineto{\pgfqpoint{3.938606in}{2.472642in}}%
\pgfusepath{stroke}%
\end{pgfscope}%
\begin{pgfscope}%
\pgfsetbuttcap%
\pgfsetroundjoin%
\definecolor{currentfill}{rgb}{0.000000,0.000000,0.000000}%
\pgfsetfillcolor{currentfill}%
\pgfsetlinewidth{0.803000pt}%
\definecolor{currentstroke}{rgb}{0.000000,0.000000,0.000000}%
\pgfsetstrokecolor{currentstroke}%
\pgfsetdash{}{0pt}%
\pgfsys@defobject{currentmarker}{\pgfqpoint{0.000000in}{-0.048611in}}{\pgfqpoint{0.000000in}{0.000000in}}{%
\pgfpathmoveto{\pgfqpoint{0.000000in}{0.000000in}}%
\pgfpathlineto{\pgfqpoint{0.000000in}{-0.048611in}}%
\pgfusepath{stroke,fill}%
}%
\begin{pgfscope}%
\pgfsys@transformshift{3.938606in}{0.417642in}%
\pgfsys@useobject{currentmarker}{}%
\end{pgfscope}%
\end{pgfscope}%
\begin{pgfscope}%
\definecolor{textcolor}{rgb}{0.000000,0.000000,0.000000}%
\pgfsetstrokecolor{textcolor}%
\pgfsetfillcolor{textcolor}%
\pgftext[x=3.938606in,y=0.320420in,,top]{\color{textcolor}{\rmfamily\fontsize{8.000000}{9.600000}\selectfont\catcode`\^=\active\def^{\ifmmode\sp\else\^{}\fi}\catcode`\%=\active\def%{\%}$\mathdefault{10^{6}}$}}%
\end{pgfscope}%
\begin{pgfscope}%
\pgfpathrectangle{\pgfqpoint{0.594525in}{0.417642in}}{\pgfqpoint{3.354228in}{2.055000in}}%
\pgfusepath{clip}%
\pgfsetrectcap%
\pgfsetroundjoin%
\pgfsetlinewidth{0.803000pt}%
\definecolor{currentstroke}{rgb}{0.850000,0.850000,0.850000}%
\pgfsetstrokecolor{currentstroke}%
\pgfsetdash{}{0pt}%
\pgfpathmoveto{\pgfqpoint{0.609547in}{0.417642in}}%
\pgfpathlineto{\pgfqpoint{0.609547in}{2.472642in}}%
\pgfusepath{stroke}%
\end{pgfscope}%
\begin{pgfscope}%
\pgfsetbuttcap%
\pgfsetroundjoin%
\definecolor{currentfill}{rgb}{0.000000,0.000000,0.000000}%
\pgfsetfillcolor{currentfill}%
\pgfsetlinewidth{0.602250pt}%
\definecolor{currentstroke}{rgb}{0.000000,0.000000,0.000000}%
\pgfsetstrokecolor{currentstroke}%
\pgfsetdash{}{0pt}%
\pgfsys@defobject{currentmarker}{\pgfqpoint{0.000000in}{-0.027778in}}{\pgfqpoint{0.000000in}{0.000000in}}{%
\pgfpathmoveto{\pgfqpoint{0.000000in}{0.000000in}}%
\pgfpathlineto{\pgfqpoint{0.000000in}{-0.027778in}}%
\pgfusepath{stroke,fill}%
}%
\begin{pgfscope}%
\pgfsys@transformshift{0.609547in}{0.417642in}%
\pgfsys@useobject{currentmarker}{}%
\end{pgfscope}%
\end{pgfscope}%
\begin{pgfscope}%
\pgfpathrectangle{\pgfqpoint{0.594525in}{0.417642in}}{\pgfqpoint{3.354228in}{2.055000in}}%
\pgfusepath{clip}%
\pgfsetrectcap%
\pgfsetroundjoin%
\pgfsetlinewidth{0.803000pt}%
\definecolor{currentstroke}{rgb}{0.850000,0.850000,0.850000}%
\pgfsetstrokecolor{currentstroke}%
\pgfsetdash{}{0pt}%
\pgfpathmoveto{\pgfqpoint{0.640408in}{0.417642in}}%
\pgfpathlineto{\pgfqpoint{0.640408in}{2.472642in}}%
\pgfusepath{stroke}%
\end{pgfscope}%
\begin{pgfscope}%
\pgfsetbuttcap%
\pgfsetroundjoin%
\definecolor{currentfill}{rgb}{0.000000,0.000000,0.000000}%
\pgfsetfillcolor{currentfill}%
\pgfsetlinewidth{0.602250pt}%
\definecolor{currentstroke}{rgb}{0.000000,0.000000,0.000000}%
\pgfsetstrokecolor{currentstroke}%
\pgfsetdash{}{0pt}%
\pgfsys@defobject{currentmarker}{\pgfqpoint{0.000000in}{-0.027778in}}{\pgfqpoint{0.000000in}{0.000000in}}{%
\pgfpathmoveto{\pgfqpoint{0.000000in}{0.000000in}}%
\pgfpathlineto{\pgfqpoint{0.000000in}{-0.027778in}}%
\pgfusepath{stroke,fill}%
}%
\begin{pgfscope}%
\pgfsys@transformshift{0.640408in}{0.417642in}%
\pgfsys@useobject{currentmarker}{}%
\end{pgfscope}%
\end{pgfscope}%
\begin{pgfscope}%
\pgfpathrectangle{\pgfqpoint{0.594525in}{0.417642in}}{\pgfqpoint{3.354228in}{2.055000in}}%
\pgfusepath{clip}%
\pgfsetrectcap%
\pgfsetroundjoin%
\pgfsetlinewidth{0.803000pt}%
\definecolor{currentstroke}{rgb}{0.850000,0.850000,0.850000}%
\pgfsetstrokecolor{currentstroke}%
\pgfsetdash{}{0pt}%
\pgfpathmoveto{\pgfqpoint{0.667140in}{0.417642in}}%
\pgfpathlineto{\pgfqpoint{0.667140in}{2.472642in}}%
\pgfusepath{stroke}%
\end{pgfscope}%
\begin{pgfscope}%
\pgfsetbuttcap%
\pgfsetroundjoin%
\definecolor{currentfill}{rgb}{0.000000,0.000000,0.000000}%
\pgfsetfillcolor{currentfill}%
\pgfsetlinewidth{0.602250pt}%
\definecolor{currentstroke}{rgb}{0.000000,0.000000,0.000000}%
\pgfsetstrokecolor{currentstroke}%
\pgfsetdash{}{0pt}%
\pgfsys@defobject{currentmarker}{\pgfqpoint{0.000000in}{-0.027778in}}{\pgfqpoint{0.000000in}{0.000000in}}{%
\pgfpathmoveto{\pgfqpoint{0.000000in}{0.000000in}}%
\pgfpathlineto{\pgfqpoint{0.000000in}{-0.027778in}}%
\pgfusepath{stroke,fill}%
}%
\begin{pgfscope}%
\pgfsys@transformshift{0.667140in}{0.417642in}%
\pgfsys@useobject{currentmarker}{}%
\end{pgfscope}%
\end{pgfscope}%
\begin{pgfscope}%
\pgfpathrectangle{\pgfqpoint{0.594525in}{0.417642in}}{\pgfqpoint{3.354228in}{2.055000in}}%
\pgfusepath{clip}%
\pgfsetrectcap%
\pgfsetroundjoin%
\pgfsetlinewidth{0.803000pt}%
\definecolor{currentstroke}{rgb}{0.850000,0.850000,0.850000}%
\pgfsetstrokecolor{currentstroke}%
\pgfsetdash{}{0pt}%
\pgfpathmoveto{\pgfqpoint{0.690720in}{0.417642in}}%
\pgfpathlineto{\pgfqpoint{0.690720in}{2.472642in}}%
\pgfusepath{stroke}%
\end{pgfscope}%
\begin{pgfscope}%
\pgfsetbuttcap%
\pgfsetroundjoin%
\definecolor{currentfill}{rgb}{0.000000,0.000000,0.000000}%
\pgfsetfillcolor{currentfill}%
\pgfsetlinewidth{0.602250pt}%
\definecolor{currentstroke}{rgb}{0.000000,0.000000,0.000000}%
\pgfsetstrokecolor{currentstroke}%
\pgfsetdash{}{0pt}%
\pgfsys@defobject{currentmarker}{\pgfqpoint{0.000000in}{-0.027778in}}{\pgfqpoint{0.000000in}{0.000000in}}{%
\pgfpathmoveto{\pgfqpoint{0.000000in}{0.000000in}}%
\pgfpathlineto{\pgfqpoint{0.000000in}{-0.027778in}}%
\pgfusepath{stroke,fill}%
}%
\begin{pgfscope}%
\pgfsys@transformshift{0.690720in}{0.417642in}%
\pgfsys@useobject{currentmarker}{}%
\end{pgfscope}%
\end{pgfscope}%
\begin{pgfscope}%
\pgfpathrectangle{\pgfqpoint{0.594525in}{0.417642in}}{\pgfqpoint{3.354228in}{2.055000in}}%
\pgfusepath{clip}%
\pgfsetrectcap%
\pgfsetroundjoin%
\pgfsetlinewidth{0.803000pt}%
\definecolor{currentstroke}{rgb}{0.850000,0.850000,0.850000}%
\pgfsetstrokecolor{currentstroke}%
\pgfsetdash{}{0pt}%
\pgfpathmoveto{\pgfqpoint{0.850579in}{0.417642in}}%
\pgfpathlineto{\pgfqpoint{0.850579in}{2.472642in}}%
\pgfusepath{stroke}%
\end{pgfscope}%
\begin{pgfscope}%
\pgfsetbuttcap%
\pgfsetroundjoin%
\definecolor{currentfill}{rgb}{0.000000,0.000000,0.000000}%
\pgfsetfillcolor{currentfill}%
\pgfsetlinewidth{0.602250pt}%
\definecolor{currentstroke}{rgb}{0.000000,0.000000,0.000000}%
\pgfsetstrokecolor{currentstroke}%
\pgfsetdash{}{0pt}%
\pgfsys@defobject{currentmarker}{\pgfqpoint{0.000000in}{-0.027778in}}{\pgfqpoint{0.000000in}{0.000000in}}{%
\pgfpathmoveto{\pgfqpoint{0.000000in}{0.000000in}}%
\pgfpathlineto{\pgfqpoint{0.000000in}{-0.027778in}}%
\pgfusepath{stroke,fill}%
}%
\begin{pgfscope}%
\pgfsys@transformshift{0.850579in}{0.417642in}%
\pgfsys@useobject{currentmarker}{}%
\end{pgfscope}%
\end{pgfscope}%
\begin{pgfscope}%
\pgfpathrectangle{\pgfqpoint{0.594525in}{0.417642in}}{\pgfqpoint{3.354228in}{2.055000in}}%
\pgfusepath{clip}%
\pgfsetrectcap%
\pgfsetroundjoin%
\pgfsetlinewidth{0.803000pt}%
\definecolor{currentstroke}{rgb}{0.850000,0.850000,0.850000}%
\pgfsetstrokecolor{currentstroke}%
\pgfsetdash{}{0pt}%
\pgfpathmoveto{\pgfqpoint{0.931752in}{0.417642in}}%
\pgfpathlineto{\pgfqpoint{0.931752in}{2.472642in}}%
\pgfusepath{stroke}%
\end{pgfscope}%
\begin{pgfscope}%
\pgfsetbuttcap%
\pgfsetroundjoin%
\definecolor{currentfill}{rgb}{0.000000,0.000000,0.000000}%
\pgfsetfillcolor{currentfill}%
\pgfsetlinewidth{0.602250pt}%
\definecolor{currentstroke}{rgb}{0.000000,0.000000,0.000000}%
\pgfsetstrokecolor{currentstroke}%
\pgfsetdash{}{0pt}%
\pgfsys@defobject{currentmarker}{\pgfqpoint{0.000000in}{-0.027778in}}{\pgfqpoint{0.000000in}{0.000000in}}{%
\pgfpathmoveto{\pgfqpoint{0.000000in}{0.000000in}}%
\pgfpathlineto{\pgfqpoint{0.000000in}{-0.027778in}}%
\pgfusepath{stroke,fill}%
}%
\begin{pgfscope}%
\pgfsys@transformshift{0.931752in}{0.417642in}%
\pgfsys@useobject{currentmarker}{}%
\end{pgfscope}%
\end{pgfscope}%
\begin{pgfscope}%
\pgfpathrectangle{\pgfqpoint{0.594525in}{0.417642in}}{\pgfqpoint{3.354228in}{2.055000in}}%
\pgfusepath{clip}%
\pgfsetrectcap%
\pgfsetroundjoin%
\pgfsetlinewidth{0.803000pt}%
\definecolor{currentstroke}{rgb}{0.850000,0.850000,0.850000}%
\pgfsetstrokecolor{currentstroke}%
\pgfsetdash{}{0pt}%
\pgfpathmoveto{\pgfqpoint{0.989345in}{0.417642in}}%
\pgfpathlineto{\pgfqpoint{0.989345in}{2.472642in}}%
\pgfusepath{stroke}%
\end{pgfscope}%
\begin{pgfscope}%
\pgfsetbuttcap%
\pgfsetroundjoin%
\definecolor{currentfill}{rgb}{0.000000,0.000000,0.000000}%
\pgfsetfillcolor{currentfill}%
\pgfsetlinewidth{0.602250pt}%
\definecolor{currentstroke}{rgb}{0.000000,0.000000,0.000000}%
\pgfsetstrokecolor{currentstroke}%
\pgfsetdash{}{0pt}%
\pgfsys@defobject{currentmarker}{\pgfqpoint{0.000000in}{-0.027778in}}{\pgfqpoint{0.000000in}{0.000000in}}{%
\pgfpathmoveto{\pgfqpoint{0.000000in}{0.000000in}}%
\pgfpathlineto{\pgfqpoint{0.000000in}{-0.027778in}}%
\pgfusepath{stroke,fill}%
}%
\begin{pgfscope}%
\pgfsys@transformshift{0.989345in}{0.417642in}%
\pgfsys@useobject{currentmarker}{}%
\end{pgfscope}%
\end{pgfscope}%
\begin{pgfscope}%
\pgfpathrectangle{\pgfqpoint{0.594525in}{0.417642in}}{\pgfqpoint{3.354228in}{2.055000in}}%
\pgfusepath{clip}%
\pgfsetrectcap%
\pgfsetroundjoin%
\pgfsetlinewidth{0.803000pt}%
\definecolor{currentstroke}{rgb}{0.850000,0.850000,0.850000}%
\pgfsetstrokecolor{currentstroke}%
\pgfsetdash{}{0pt}%
\pgfpathmoveto{\pgfqpoint{1.034017in}{0.417642in}}%
\pgfpathlineto{\pgfqpoint{1.034017in}{2.472642in}}%
\pgfusepath{stroke}%
\end{pgfscope}%
\begin{pgfscope}%
\pgfsetbuttcap%
\pgfsetroundjoin%
\definecolor{currentfill}{rgb}{0.000000,0.000000,0.000000}%
\pgfsetfillcolor{currentfill}%
\pgfsetlinewidth{0.602250pt}%
\definecolor{currentstroke}{rgb}{0.000000,0.000000,0.000000}%
\pgfsetstrokecolor{currentstroke}%
\pgfsetdash{}{0pt}%
\pgfsys@defobject{currentmarker}{\pgfqpoint{0.000000in}{-0.027778in}}{\pgfqpoint{0.000000in}{0.000000in}}{%
\pgfpathmoveto{\pgfqpoint{0.000000in}{0.000000in}}%
\pgfpathlineto{\pgfqpoint{0.000000in}{-0.027778in}}%
\pgfusepath{stroke,fill}%
}%
\begin{pgfscope}%
\pgfsys@transformshift{1.034017in}{0.417642in}%
\pgfsys@useobject{currentmarker}{}%
\end{pgfscope}%
\end{pgfscope}%
\begin{pgfscope}%
\pgfpathrectangle{\pgfqpoint{0.594525in}{0.417642in}}{\pgfqpoint{3.354228in}{2.055000in}}%
\pgfusepath{clip}%
\pgfsetrectcap%
\pgfsetroundjoin%
\pgfsetlinewidth{0.803000pt}%
\definecolor{currentstroke}{rgb}{0.850000,0.850000,0.850000}%
\pgfsetstrokecolor{currentstroke}%
\pgfsetdash{}{0pt}%
\pgfpathmoveto{\pgfqpoint{1.070518in}{0.417642in}}%
\pgfpathlineto{\pgfqpoint{1.070518in}{2.472642in}}%
\pgfusepath{stroke}%
\end{pgfscope}%
\begin{pgfscope}%
\pgfsetbuttcap%
\pgfsetroundjoin%
\definecolor{currentfill}{rgb}{0.000000,0.000000,0.000000}%
\pgfsetfillcolor{currentfill}%
\pgfsetlinewidth{0.602250pt}%
\definecolor{currentstroke}{rgb}{0.000000,0.000000,0.000000}%
\pgfsetstrokecolor{currentstroke}%
\pgfsetdash{}{0pt}%
\pgfsys@defobject{currentmarker}{\pgfqpoint{0.000000in}{-0.027778in}}{\pgfqpoint{0.000000in}{0.000000in}}{%
\pgfpathmoveto{\pgfqpoint{0.000000in}{0.000000in}}%
\pgfpathlineto{\pgfqpoint{0.000000in}{-0.027778in}}%
\pgfusepath{stroke,fill}%
}%
\begin{pgfscope}%
\pgfsys@transformshift{1.070518in}{0.417642in}%
\pgfsys@useobject{currentmarker}{}%
\end{pgfscope}%
\end{pgfscope}%
\begin{pgfscope}%
\pgfpathrectangle{\pgfqpoint{0.594525in}{0.417642in}}{\pgfqpoint{3.354228in}{2.055000in}}%
\pgfusepath{clip}%
\pgfsetrectcap%
\pgfsetroundjoin%
\pgfsetlinewidth{0.803000pt}%
\definecolor{currentstroke}{rgb}{0.850000,0.850000,0.850000}%
\pgfsetstrokecolor{currentstroke}%
\pgfsetdash{}{0pt}%
\pgfpathmoveto{\pgfqpoint{1.101378in}{0.417642in}}%
\pgfpathlineto{\pgfqpoint{1.101378in}{2.472642in}}%
\pgfusepath{stroke}%
\end{pgfscope}%
\begin{pgfscope}%
\pgfsetbuttcap%
\pgfsetroundjoin%
\definecolor{currentfill}{rgb}{0.000000,0.000000,0.000000}%
\pgfsetfillcolor{currentfill}%
\pgfsetlinewidth{0.602250pt}%
\definecolor{currentstroke}{rgb}{0.000000,0.000000,0.000000}%
\pgfsetstrokecolor{currentstroke}%
\pgfsetdash{}{0pt}%
\pgfsys@defobject{currentmarker}{\pgfqpoint{0.000000in}{-0.027778in}}{\pgfqpoint{0.000000in}{0.000000in}}{%
\pgfpathmoveto{\pgfqpoint{0.000000in}{0.000000in}}%
\pgfpathlineto{\pgfqpoint{0.000000in}{-0.027778in}}%
\pgfusepath{stroke,fill}%
}%
\begin{pgfscope}%
\pgfsys@transformshift{1.101378in}{0.417642in}%
\pgfsys@useobject{currentmarker}{}%
\end{pgfscope}%
\end{pgfscope}%
\begin{pgfscope}%
\pgfpathrectangle{\pgfqpoint{0.594525in}{0.417642in}}{\pgfqpoint{3.354228in}{2.055000in}}%
\pgfusepath{clip}%
\pgfsetrectcap%
\pgfsetroundjoin%
\pgfsetlinewidth{0.803000pt}%
\definecolor{currentstroke}{rgb}{0.850000,0.850000,0.850000}%
\pgfsetstrokecolor{currentstroke}%
\pgfsetdash{}{0pt}%
\pgfpathmoveto{\pgfqpoint{1.128111in}{0.417642in}}%
\pgfpathlineto{\pgfqpoint{1.128111in}{2.472642in}}%
\pgfusepath{stroke}%
\end{pgfscope}%
\begin{pgfscope}%
\pgfsetbuttcap%
\pgfsetroundjoin%
\definecolor{currentfill}{rgb}{0.000000,0.000000,0.000000}%
\pgfsetfillcolor{currentfill}%
\pgfsetlinewidth{0.602250pt}%
\definecolor{currentstroke}{rgb}{0.000000,0.000000,0.000000}%
\pgfsetstrokecolor{currentstroke}%
\pgfsetdash{}{0pt}%
\pgfsys@defobject{currentmarker}{\pgfqpoint{0.000000in}{-0.027778in}}{\pgfqpoint{0.000000in}{0.000000in}}{%
\pgfpathmoveto{\pgfqpoint{0.000000in}{0.000000in}}%
\pgfpathlineto{\pgfqpoint{0.000000in}{-0.027778in}}%
\pgfusepath{stroke,fill}%
}%
\begin{pgfscope}%
\pgfsys@transformshift{1.128111in}{0.417642in}%
\pgfsys@useobject{currentmarker}{}%
\end{pgfscope}%
\end{pgfscope}%
\begin{pgfscope}%
\pgfpathrectangle{\pgfqpoint{0.594525in}{0.417642in}}{\pgfqpoint{3.354228in}{2.055000in}}%
\pgfusepath{clip}%
\pgfsetrectcap%
\pgfsetroundjoin%
\pgfsetlinewidth{0.803000pt}%
\definecolor{currentstroke}{rgb}{0.850000,0.850000,0.850000}%
\pgfsetstrokecolor{currentstroke}%
\pgfsetdash{}{0pt}%
\pgfpathmoveto{\pgfqpoint{1.151691in}{0.417642in}}%
\pgfpathlineto{\pgfqpoint{1.151691in}{2.472642in}}%
\pgfusepath{stroke}%
\end{pgfscope}%
\begin{pgfscope}%
\pgfsetbuttcap%
\pgfsetroundjoin%
\definecolor{currentfill}{rgb}{0.000000,0.000000,0.000000}%
\pgfsetfillcolor{currentfill}%
\pgfsetlinewidth{0.602250pt}%
\definecolor{currentstroke}{rgb}{0.000000,0.000000,0.000000}%
\pgfsetstrokecolor{currentstroke}%
\pgfsetdash{}{0pt}%
\pgfsys@defobject{currentmarker}{\pgfqpoint{0.000000in}{-0.027778in}}{\pgfqpoint{0.000000in}{0.000000in}}{%
\pgfpathmoveto{\pgfqpoint{0.000000in}{0.000000in}}%
\pgfpathlineto{\pgfqpoint{0.000000in}{-0.027778in}}%
\pgfusepath{stroke,fill}%
}%
\begin{pgfscope}%
\pgfsys@transformshift{1.151691in}{0.417642in}%
\pgfsys@useobject{currentmarker}{}%
\end{pgfscope}%
\end{pgfscope}%
\begin{pgfscope}%
\pgfpathrectangle{\pgfqpoint{0.594525in}{0.417642in}}{\pgfqpoint{3.354228in}{2.055000in}}%
\pgfusepath{clip}%
\pgfsetrectcap%
\pgfsetroundjoin%
\pgfsetlinewidth{0.803000pt}%
\definecolor{currentstroke}{rgb}{0.850000,0.850000,0.850000}%
\pgfsetstrokecolor{currentstroke}%
\pgfsetdash{}{0pt}%
\pgfpathmoveto{\pgfqpoint{1.311549in}{0.417642in}}%
\pgfpathlineto{\pgfqpoint{1.311549in}{2.472642in}}%
\pgfusepath{stroke}%
\end{pgfscope}%
\begin{pgfscope}%
\pgfsetbuttcap%
\pgfsetroundjoin%
\definecolor{currentfill}{rgb}{0.000000,0.000000,0.000000}%
\pgfsetfillcolor{currentfill}%
\pgfsetlinewidth{0.602250pt}%
\definecolor{currentstroke}{rgb}{0.000000,0.000000,0.000000}%
\pgfsetstrokecolor{currentstroke}%
\pgfsetdash{}{0pt}%
\pgfsys@defobject{currentmarker}{\pgfqpoint{0.000000in}{-0.027778in}}{\pgfqpoint{0.000000in}{0.000000in}}{%
\pgfpathmoveto{\pgfqpoint{0.000000in}{0.000000in}}%
\pgfpathlineto{\pgfqpoint{0.000000in}{-0.027778in}}%
\pgfusepath{stroke,fill}%
}%
\begin{pgfscope}%
\pgfsys@transformshift{1.311549in}{0.417642in}%
\pgfsys@useobject{currentmarker}{}%
\end{pgfscope}%
\end{pgfscope}%
\begin{pgfscope}%
\pgfpathrectangle{\pgfqpoint{0.594525in}{0.417642in}}{\pgfqpoint{3.354228in}{2.055000in}}%
\pgfusepath{clip}%
\pgfsetrectcap%
\pgfsetroundjoin%
\pgfsetlinewidth{0.803000pt}%
\definecolor{currentstroke}{rgb}{0.850000,0.850000,0.850000}%
\pgfsetstrokecolor{currentstroke}%
\pgfsetdash{}{0pt}%
\pgfpathmoveto{\pgfqpoint{1.392722in}{0.417642in}}%
\pgfpathlineto{\pgfqpoint{1.392722in}{2.472642in}}%
\pgfusepath{stroke}%
\end{pgfscope}%
\begin{pgfscope}%
\pgfsetbuttcap%
\pgfsetroundjoin%
\definecolor{currentfill}{rgb}{0.000000,0.000000,0.000000}%
\pgfsetfillcolor{currentfill}%
\pgfsetlinewidth{0.602250pt}%
\definecolor{currentstroke}{rgb}{0.000000,0.000000,0.000000}%
\pgfsetstrokecolor{currentstroke}%
\pgfsetdash{}{0pt}%
\pgfsys@defobject{currentmarker}{\pgfqpoint{0.000000in}{-0.027778in}}{\pgfqpoint{0.000000in}{0.000000in}}{%
\pgfpathmoveto{\pgfqpoint{0.000000in}{0.000000in}}%
\pgfpathlineto{\pgfqpoint{0.000000in}{-0.027778in}}%
\pgfusepath{stroke,fill}%
}%
\begin{pgfscope}%
\pgfsys@transformshift{1.392722in}{0.417642in}%
\pgfsys@useobject{currentmarker}{}%
\end{pgfscope}%
\end{pgfscope}%
\begin{pgfscope}%
\pgfpathrectangle{\pgfqpoint{0.594525in}{0.417642in}}{\pgfqpoint{3.354228in}{2.055000in}}%
\pgfusepath{clip}%
\pgfsetrectcap%
\pgfsetroundjoin%
\pgfsetlinewidth{0.803000pt}%
\definecolor{currentstroke}{rgb}{0.850000,0.850000,0.850000}%
\pgfsetstrokecolor{currentstroke}%
\pgfsetdash{}{0pt}%
\pgfpathmoveto{\pgfqpoint{1.450315in}{0.417642in}}%
\pgfpathlineto{\pgfqpoint{1.450315in}{2.472642in}}%
\pgfusepath{stroke}%
\end{pgfscope}%
\begin{pgfscope}%
\pgfsetbuttcap%
\pgfsetroundjoin%
\definecolor{currentfill}{rgb}{0.000000,0.000000,0.000000}%
\pgfsetfillcolor{currentfill}%
\pgfsetlinewidth{0.602250pt}%
\definecolor{currentstroke}{rgb}{0.000000,0.000000,0.000000}%
\pgfsetstrokecolor{currentstroke}%
\pgfsetdash{}{0pt}%
\pgfsys@defobject{currentmarker}{\pgfqpoint{0.000000in}{-0.027778in}}{\pgfqpoint{0.000000in}{0.000000in}}{%
\pgfpathmoveto{\pgfqpoint{0.000000in}{0.000000in}}%
\pgfpathlineto{\pgfqpoint{0.000000in}{-0.027778in}}%
\pgfusepath{stroke,fill}%
}%
\begin{pgfscope}%
\pgfsys@transformshift{1.450315in}{0.417642in}%
\pgfsys@useobject{currentmarker}{}%
\end{pgfscope}%
\end{pgfscope}%
\begin{pgfscope}%
\pgfpathrectangle{\pgfqpoint{0.594525in}{0.417642in}}{\pgfqpoint{3.354228in}{2.055000in}}%
\pgfusepath{clip}%
\pgfsetrectcap%
\pgfsetroundjoin%
\pgfsetlinewidth{0.803000pt}%
\definecolor{currentstroke}{rgb}{0.850000,0.850000,0.850000}%
\pgfsetstrokecolor{currentstroke}%
\pgfsetdash{}{0pt}%
\pgfpathmoveto{\pgfqpoint{1.494988in}{0.417642in}}%
\pgfpathlineto{\pgfqpoint{1.494988in}{2.472642in}}%
\pgfusepath{stroke}%
\end{pgfscope}%
\begin{pgfscope}%
\pgfsetbuttcap%
\pgfsetroundjoin%
\definecolor{currentfill}{rgb}{0.000000,0.000000,0.000000}%
\pgfsetfillcolor{currentfill}%
\pgfsetlinewidth{0.602250pt}%
\definecolor{currentstroke}{rgb}{0.000000,0.000000,0.000000}%
\pgfsetstrokecolor{currentstroke}%
\pgfsetdash{}{0pt}%
\pgfsys@defobject{currentmarker}{\pgfqpoint{0.000000in}{-0.027778in}}{\pgfqpoint{0.000000in}{0.000000in}}{%
\pgfpathmoveto{\pgfqpoint{0.000000in}{0.000000in}}%
\pgfpathlineto{\pgfqpoint{0.000000in}{-0.027778in}}%
\pgfusepath{stroke,fill}%
}%
\begin{pgfscope}%
\pgfsys@transformshift{1.494988in}{0.417642in}%
\pgfsys@useobject{currentmarker}{}%
\end{pgfscope}%
\end{pgfscope}%
\begin{pgfscope}%
\pgfpathrectangle{\pgfqpoint{0.594525in}{0.417642in}}{\pgfqpoint{3.354228in}{2.055000in}}%
\pgfusepath{clip}%
\pgfsetrectcap%
\pgfsetroundjoin%
\pgfsetlinewidth{0.803000pt}%
\definecolor{currentstroke}{rgb}{0.850000,0.850000,0.850000}%
\pgfsetstrokecolor{currentstroke}%
\pgfsetdash{}{0pt}%
\pgfpathmoveto{\pgfqpoint{1.531488in}{0.417642in}}%
\pgfpathlineto{\pgfqpoint{1.531488in}{2.472642in}}%
\pgfusepath{stroke}%
\end{pgfscope}%
\begin{pgfscope}%
\pgfsetbuttcap%
\pgfsetroundjoin%
\definecolor{currentfill}{rgb}{0.000000,0.000000,0.000000}%
\pgfsetfillcolor{currentfill}%
\pgfsetlinewidth{0.602250pt}%
\definecolor{currentstroke}{rgb}{0.000000,0.000000,0.000000}%
\pgfsetstrokecolor{currentstroke}%
\pgfsetdash{}{0pt}%
\pgfsys@defobject{currentmarker}{\pgfqpoint{0.000000in}{-0.027778in}}{\pgfqpoint{0.000000in}{0.000000in}}{%
\pgfpathmoveto{\pgfqpoint{0.000000in}{0.000000in}}%
\pgfpathlineto{\pgfqpoint{0.000000in}{-0.027778in}}%
\pgfusepath{stroke,fill}%
}%
\begin{pgfscope}%
\pgfsys@transformshift{1.531488in}{0.417642in}%
\pgfsys@useobject{currentmarker}{}%
\end{pgfscope}%
\end{pgfscope}%
\begin{pgfscope}%
\pgfpathrectangle{\pgfqpoint{0.594525in}{0.417642in}}{\pgfqpoint{3.354228in}{2.055000in}}%
\pgfusepath{clip}%
\pgfsetrectcap%
\pgfsetroundjoin%
\pgfsetlinewidth{0.803000pt}%
\definecolor{currentstroke}{rgb}{0.850000,0.850000,0.850000}%
\pgfsetstrokecolor{currentstroke}%
\pgfsetdash{}{0pt}%
\pgfpathmoveto{\pgfqpoint{1.562349in}{0.417642in}}%
\pgfpathlineto{\pgfqpoint{1.562349in}{2.472642in}}%
\pgfusepath{stroke}%
\end{pgfscope}%
\begin{pgfscope}%
\pgfsetbuttcap%
\pgfsetroundjoin%
\definecolor{currentfill}{rgb}{0.000000,0.000000,0.000000}%
\pgfsetfillcolor{currentfill}%
\pgfsetlinewidth{0.602250pt}%
\definecolor{currentstroke}{rgb}{0.000000,0.000000,0.000000}%
\pgfsetstrokecolor{currentstroke}%
\pgfsetdash{}{0pt}%
\pgfsys@defobject{currentmarker}{\pgfqpoint{0.000000in}{-0.027778in}}{\pgfqpoint{0.000000in}{0.000000in}}{%
\pgfpathmoveto{\pgfqpoint{0.000000in}{0.000000in}}%
\pgfpathlineto{\pgfqpoint{0.000000in}{-0.027778in}}%
\pgfusepath{stroke,fill}%
}%
\begin{pgfscope}%
\pgfsys@transformshift{1.562349in}{0.417642in}%
\pgfsys@useobject{currentmarker}{}%
\end{pgfscope}%
\end{pgfscope}%
\begin{pgfscope}%
\pgfpathrectangle{\pgfqpoint{0.594525in}{0.417642in}}{\pgfqpoint{3.354228in}{2.055000in}}%
\pgfusepath{clip}%
\pgfsetrectcap%
\pgfsetroundjoin%
\pgfsetlinewidth{0.803000pt}%
\definecolor{currentstroke}{rgb}{0.850000,0.850000,0.850000}%
\pgfsetstrokecolor{currentstroke}%
\pgfsetdash{}{0pt}%
\pgfpathmoveto{\pgfqpoint{1.589081in}{0.417642in}}%
\pgfpathlineto{\pgfqpoint{1.589081in}{2.472642in}}%
\pgfusepath{stroke}%
\end{pgfscope}%
\begin{pgfscope}%
\pgfsetbuttcap%
\pgfsetroundjoin%
\definecolor{currentfill}{rgb}{0.000000,0.000000,0.000000}%
\pgfsetfillcolor{currentfill}%
\pgfsetlinewidth{0.602250pt}%
\definecolor{currentstroke}{rgb}{0.000000,0.000000,0.000000}%
\pgfsetstrokecolor{currentstroke}%
\pgfsetdash{}{0pt}%
\pgfsys@defobject{currentmarker}{\pgfqpoint{0.000000in}{-0.027778in}}{\pgfqpoint{0.000000in}{0.000000in}}{%
\pgfpathmoveto{\pgfqpoint{0.000000in}{0.000000in}}%
\pgfpathlineto{\pgfqpoint{0.000000in}{-0.027778in}}%
\pgfusepath{stroke,fill}%
}%
\begin{pgfscope}%
\pgfsys@transformshift{1.589081in}{0.417642in}%
\pgfsys@useobject{currentmarker}{}%
\end{pgfscope}%
\end{pgfscope}%
\begin{pgfscope}%
\pgfpathrectangle{\pgfqpoint{0.594525in}{0.417642in}}{\pgfqpoint{3.354228in}{2.055000in}}%
\pgfusepath{clip}%
\pgfsetrectcap%
\pgfsetroundjoin%
\pgfsetlinewidth{0.803000pt}%
\definecolor{currentstroke}{rgb}{0.850000,0.850000,0.850000}%
\pgfsetstrokecolor{currentstroke}%
\pgfsetdash{}{0pt}%
\pgfpathmoveto{\pgfqpoint{1.612661in}{0.417642in}}%
\pgfpathlineto{\pgfqpoint{1.612661in}{2.472642in}}%
\pgfusepath{stroke}%
\end{pgfscope}%
\begin{pgfscope}%
\pgfsetbuttcap%
\pgfsetroundjoin%
\definecolor{currentfill}{rgb}{0.000000,0.000000,0.000000}%
\pgfsetfillcolor{currentfill}%
\pgfsetlinewidth{0.602250pt}%
\definecolor{currentstroke}{rgb}{0.000000,0.000000,0.000000}%
\pgfsetstrokecolor{currentstroke}%
\pgfsetdash{}{0pt}%
\pgfsys@defobject{currentmarker}{\pgfqpoint{0.000000in}{-0.027778in}}{\pgfqpoint{0.000000in}{0.000000in}}{%
\pgfpathmoveto{\pgfqpoint{0.000000in}{0.000000in}}%
\pgfpathlineto{\pgfqpoint{0.000000in}{-0.027778in}}%
\pgfusepath{stroke,fill}%
}%
\begin{pgfscope}%
\pgfsys@transformshift{1.612661in}{0.417642in}%
\pgfsys@useobject{currentmarker}{}%
\end{pgfscope}%
\end{pgfscope}%
\begin{pgfscope}%
\pgfpathrectangle{\pgfqpoint{0.594525in}{0.417642in}}{\pgfqpoint{3.354228in}{2.055000in}}%
\pgfusepath{clip}%
\pgfsetrectcap%
\pgfsetroundjoin%
\pgfsetlinewidth{0.803000pt}%
\definecolor{currentstroke}{rgb}{0.850000,0.850000,0.850000}%
\pgfsetstrokecolor{currentstroke}%
\pgfsetdash{}{0pt}%
\pgfpathmoveto{\pgfqpoint{1.772520in}{0.417642in}}%
\pgfpathlineto{\pgfqpoint{1.772520in}{2.472642in}}%
\pgfusepath{stroke}%
\end{pgfscope}%
\begin{pgfscope}%
\pgfsetbuttcap%
\pgfsetroundjoin%
\definecolor{currentfill}{rgb}{0.000000,0.000000,0.000000}%
\pgfsetfillcolor{currentfill}%
\pgfsetlinewidth{0.602250pt}%
\definecolor{currentstroke}{rgb}{0.000000,0.000000,0.000000}%
\pgfsetstrokecolor{currentstroke}%
\pgfsetdash{}{0pt}%
\pgfsys@defobject{currentmarker}{\pgfqpoint{0.000000in}{-0.027778in}}{\pgfqpoint{0.000000in}{0.000000in}}{%
\pgfpathmoveto{\pgfqpoint{0.000000in}{0.000000in}}%
\pgfpathlineto{\pgfqpoint{0.000000in}{-0.027778in}}%
\pgfusepath{stroke,fill}%
}%
\begin{pgfscope}%
\pgfsys@transformshift{1.772520in}{0.417642in}%
\pgfsys@useobject{currentmarker}{}%
\end{pgfscope}%
\end{pgfscope}%
\begin{pgfscope}%
\pgfpathrectangle{\pgfqpoint{0.594525in}{0.417642in}}{\pgfqpoint{3.354228in}{2.055000in}}%
\pgfusepath{clip}%
\pgfsetrectcap%
\pgfsetroundjoin%
\pgfsetlinewidth{0.803000pt}%
\definecolor{currentstroke}{rgb}{0.850000,0.850000,0.850000}%
\pgfsetstrokecolor{currentstroke}%
\pgfsetdash{}{0pt}%
\pgfpathmoveto{\pgfqpoint{1.853693in}{0.417642in}}%
\pgfpathlineto{\pgfqpoint{1.853693in}{2.472642in}}%
\pgfusepath{stroke}%
\end{pgfscope}%
\begin{pgfscope}%
\pgfsetbuttcap%
\pgfsetroundjoin%
\definecolor{currentfill}{rgb}{0.000000,0.000000,0.000000}%
\pgfsetfillcolor{currentfill}%
\pgfsetlinewidth{0.602250pt}%
\definecolor{currentstroke}{rgb}{0.000000,0.000000,0.000000}%
\pgfsetstrokecolor{currentstroke}%
\pgfsetdash{}{0pt}%
\pgfsys@defobject{currentmarker}{\pgfqpoint{0.000000in}{-0.027778in}}{\pgfqpoint{0.000000in}{0.000000in}}{%
\pgfpathmoveto{\pgfqpoint{0.000000in}{0.000000in}}%
\pgfpathlineto{\pgfqpoint{0.000000in}{-0.027778in}}%
\pgfusepath{stroke,fill}%
}%
\begin{pgfscope}%
\pgfsys@transformshift{1.853693in}{0.417642in}%
\pgfsys@useobject{currentmarker}{}%
\end{pgfscope}%
\end{pgfscope}%
\begin{pgfscope}%
\pgfpathrectangle{\pgfqpoint{0.594525in}{0.417642in}}{\pgfqpoint{3.354228in}{2.055000in}}%
\pgfusepath{clip}%
\pgfsetrectcap%
\pgfsetroundjoin%
\pgfsetlinewidth{0.803000pt}%
\definecolor{currentstroke}{rgb}{0.850000,0.850000,0.850000}%
\pgfsetstrokecolor{currentstroke}%
\pgfsetdash{}{0pt}%
\pgfpathmoveto{\pgfqpoint{1.911286in}{0.417642in}}%
\pgfpathlineto{\pgfqpoint{1.911286in}{2.472642in}}%
\pgfusepath{stroke}%
\end{pgfscope}%
\begin{pgfscope}%
\pgfsetbuttcap%
\pgfsetroundjoin%
\definecolor{currentfill}{rgb}{0.000000,0.000000,0.000000}%
\pgfsetfillcolor{currentfill}%
\pgfsetlinewidth{0.602250pt}%
\definecolor{currentstroke}{rgb}{0.000000,0.000000,0.000000}%
\pgfsetstrokecolor{currentstroke}%
\pgfsetdash{}{0pt}%
\pgfsys@defobject{currentmarker}{\pgfqpoint{0.000000in}{-0.027778in}}{\pgfqpoint{0.000000in}{0.000000in}}{%
\pgfpathmoveto{\pgfqpoint{0.000000in}{0.000000in}}%
\pgfpathlineto{\pgfqpoint{0.000000in}{-0.027778in}}%
\pgfusepath{stroke,fill}%
}%
\begin{pgfscope}%
\pgfsys@transformshift{1.911286in}{0.417642in}%
\pgfsys@useobject{currentmarker}{}%
\end{pgfscope}%
\end{pgfscope}%
\begin{pgfscope}%
\pgfpathrectangle{\pgfqpoint{0.594525in}{0.417642in}}{\pgfqpoint{3.354228in}{2.055000in}}%
\pgfusepath{clip}%
\pgfsetrectcap%
\pgfsetroundjoin%
\pgfsetlinewidth{0.803000pt}%
\definecolor{currentstroke}{rgb}{0.850000,0.850000,0.850000}%
\pgfsetstrokecolor{currentstroke}%
\pgfsetdash{}{0pt}%
\pgfpathmoveto{\pgfqpoint{1.955958in}{0.417642in}}%
\pgfpathlineto{\pgfqpoint{1.955958in}{2.472642in}}%
\pgfusepath{stroke}%
\end{pgfscope}%
\begin{pgfscope}%
\pgfsetbuttcap%
\pgfsetroundjoin%
\definecolor{currentfill}{rgb}{0.000000,0.000000,0.000000}%
\pgfsetfillcolor{currentfill}%
\pgfsetlinewidth{0.602250pt}%
\definecolor{currentstroke}{rgb}{0.000000,0.000000,0.000000}%
\pgfsetstrokecolor{currentstroke}%
\pgfsetdash{}{0pt}%
\pgfsys@defobject{currentmarker}{\pgfqpoint{0.000000in}{-0.027778in}}{\pgfqpoint{0.000000in}{0.000000in}}{%
\pgfpathmoveto{\pgfqpoint{0.000000in}{0.000000in}}%
\pgfpathlineto{\pgfqpoint{0.000000in}{-0.027778in}}%
\pgfusepath{stroke,fill}%
}%
\begin{pgfscope}%
\pgfsys@transformshift{1.955958in}{0.417642in}%
\pgfsys@useobject{currentmarker}{}%
\end{pgfscope}%
\end{pgfscope}%
\begin{pgfscope}%
\pgfpathrectangle{\pgfqpoint{0.594525in}{0.417642in}}{\pgfqpoint{3.354228in}{2.055000in}}%
\pgfusepath{clip}%
\pgfsetrectcap%
\pgfsetroundjoin%
\pgfsetlinewidth{0.803000pt}%
\definecolor{currentstroke}{rgb}{0.850000,0.850000,0.850000}%
\pgfsetstrokecolor{currentstroke}%
\pgfsetdash{}{0pt}%
\pgfpathmoveto{\pgfqpoint{1.992459in}{0.417642in}}%
\pgfpathlineto{\pgfqpoint{1.992459in}{2.472642in}}%
\pgfusepath{stroke}%
\end{pgfscope}%
\begin{pgfscope}%
\pgfsetbuttcap%
\pgfsetroundjoin%
\definecolor{currentfill}{rgb}{0.000000,0.000000,0.000000}%
\pgfsetfillcolor{currentfill}%
\pgfsetlinewidth{0.602250pt}%
\definecolor{currentstroke}{rgb}{0.000000,0.000000,0.000000}%
\pgfsetstrokecolor{currentstroke}%
\pgfsetdash{}{0pt}%
\pgfsys@defobject{currentmarker}{\pgfqpoint{0.000000in}{-0.027778in}}{\pgfqpoint{0.000000in}{0.000000in}}{%
\pgfpathmoveto{\pgfqpoint{0.000000in}{0.000000in}}%
\pgfpathlineto{\pgfqpoint{0.000000in}{-0.027778in}}%
\pgfusepath{stroke,fill}%
}%
\begin{pgfscope}%
\pgfsys@transformshift{1.992459in}{0.417642in}%
\pgfsys@useobject{currentmarker}{}%
\end{pgfscope}%
\end{pgfscope}%
\begin{pgfscope}%
\pgfpathrectangle{\pgfqpoint{0.594525in}{0.417642in}}{\pgfqpoint{3.354228in}{2.055000in}}%
\pgfusepath{clip}%
\pgfsetrectcap%
\pgfsetroundjoin%
\pgfsetlinewidth{0.803000pt}%
\definecolor{currentstroke}{rgb}{0.850000,0.850000,0.850000}%
\pgfsetstrokecolor{currentstroke}%
\pgfsetdash{}{0pt}%
\pgfpathmoveto{\pgfqpoint{2.023319in}{0.417642in}}%
\pgfpathlineto{\pgfqpoint{2.023319in}{2.472642in}}%
\pgfusepath{stroke}%
\end{pgfscope}%
\begin{pgfscope}%
\pgfsetbuttcap%
\pgfsetroundjoin%
\definecolor{currentfill}{rgb}{0.000000,0.000000,0.000000}%
\pgfsetfillcolor{currentfill}%
\pgfsetlinewidth{0.602250pt}%
\definecolor{currentstroke}{rgb}{0.000000,0.000000,0.000000}%
\pgfsetstrokecolor{currentstroke}%
\pgfsetdash{}{0pt}%
\pgfsys@defobject{currentmarker}{\pgfqpoint{0.000000in}{-0.027778in}}{\pgfqpoint{0.000000in}{0.000000in}}{%
\pgfpathmoveto{\pgfqpoint{0.000000in}{0.000000in}}%
\pgfpathlineto{\pgfqpoint{0.000000in}{-0.027778in}}%
\pgfusepath{stroke,fill}%
}%
\begin{pgfscope}%
\pgfsys@transformshift{2.023319in}{0.417642in}%
\pgfsys@useobject{currentmarker}{}%
\end{pgfscope}%
\end{pgfscope}%
\begin{pgfscope}%
\pgfpathrectangle{\pgfqpoint{0.594525in}{0.417642in}}{\pgfqpoint{3.354228in}{2.055000in}}%
\pgfusepath{clip}%
\pgfsetrectcap%
\pgfsetroundjoin%
\pgfsetlinewidth{0.803000pt}%
\definecolor{currentstroke}{rgb}{0.850000,0.850000,0.850000}%
\pgfsetstrokecolor{currentstroke}%
\pgfsetdash{}{0pt}%
\pgfpathmoveto{\pgfqpoint{2.050052in}{0.417642in}}%
\pgfpathlineto{\pgfqpoint{2.050052in}{2.472642in}}%
\pgfusepath{stroke}%
\end{pgfscope}%
\begin{pgfscope}%
\pgfsetbuttcap%
\pgfsetroundjoin%
\definecolor{currentfill}{rgb}{0.000000,0.000000,0.000000}%
\pgfsetfillcolor{currentfill}%
\pgfsetlinewidth{0.602250pt}%
\definecolor{currentstroke}{rgb}{0.000000,0.000000,0.000000}%
\pgfsetstrokecolor{currentstroke}%
\pgfsetdash{}{0pt}%
\pgfsys@defobject{currentmarker}{\pgfqpoint{0.000000in}{-0.027778in}}{\pgfqpoint{0.000000in}{0.000000in}}{%
\pgfpathmoveto{\pgfqpoint{0.000000in}{0.000000in}}%
\pgfpathlineto{\pgfqpoint{0.000000in}{-0.027778in}}%
\pgfusepath{stroke,fill}%
}%
\begin{pgfscope}%
\pgfsys@transformshift{2.050052in}{0.417642in}%
\pgfsys@useobject{currentmarker}{}%
\end{pgfscope}%
\end{pgfscope}%
\begin{pgfscope}%
\pgfpathrectangle{\pgfqpoint{0.594525in}{0.417642in}}{\pgfqpoint{3.354228in}{2.055000in}}%
\pgfusepath{clip}%
\pgfsetrectcap%
\pgfsetroundjoin%
\pgfsetlinewidth{0.803000pt}%
\definecolor{currentstroke}{rgb}{0.850000,0.850000,0.850000}%
\pgfsetstrokecolor{currentstroke}%
\pgfsetdash{}{0pt}%
\pgfpathmoveto{\pgfqpoint{2.073632in}{0.417642in}}%
\pgfpathlineto{\pgfqpoint{2.073632in}{2.472642in}}%
\pgfusepath{stroke}%
\end{pgfscope}%
\begin{pgfscope}%
\pgfsetbuttcap%
\pgfsetroundjoin%
\definecolor{currentfill}{rgb}{0.000000,0.000000,0.000000}%
\pgfsetfillcolor{currentfill}%
\pgfsetlinewidth{0.602250pt}%
\definecolor{currentstroke}{rgb}{0.000000,0.000000,0.000000}%
\pgfsetstrokecolor{currentstroke}%
\pgfsetdash{}{0pt}%
\pgfsys@defobject{currentmarker}{\pgfqpoint{0.000000in}{-0.027778in}}{\pgfqpoint{0.000000in}{0.000000in}}{%
\pgfpathmoveto{\pgfqpoint{0.000000in}{0.000000in}}%
\pgfpathlineto{\pgfqpoint{0.000000in}{-0.027778in}}%
\pgfusepath{stroke,fill}%
}%
\begin{pgfscope}%
\pgfsys@transformshift{2.073632in}{0.417642in}%
\pgfsys@useobject{currentmarker}{}%
\end{pgfscope}%
\end{pgfscope}%
\begin{pgfscope}%
\pgfpathrectangle{\pgfqpoint{0.594525in}{0.417642in}}{\pgfqpoint{3.354228in}{2.055000in}}%
\pgfusepath{clip}%
\pgfsetrectcap%
\pgfsetroundjoin%
\pgfsetlinewidth{0.803000pt}%
\definecolor{currentstroke}{rgb}{0.850000,0.850000,0.850000}%
\pgfsetstrokecolor{currentstroke}%
\pgfsetdash{}{0pt}%
\pgfpathmoveto{\pgfqpoint{2.233490in}{0.417642in}}%
\pgfpathlineto{\pgfqpoint{2.233490in}{2.472642in}}%
\pgfusepath{stroke}%
\end{pgfscope}%
\begin{pgfscope}%
\pgfsetbuttcap%
\pgfsetroundjoin%
\definecolor{currentfill}{rgb}{0.000000,0.000000,0.000000}%
\pgfsetfillcolor{currentfill}%
\pgfsetlinewidth{0.602250pt}%
\definecolor{currentstroke}{rgb}{0.000000,0.000000,0.000000}%
\pgfsetstrokecolor{currentstroke}%
\pgfsetdash{}{0pt}%
\pgfsys@defobject{currentmarker}{\pgfqpoint{0.000000in}{-0.027778in}}{\pgfqpoint{0.000000in}{0.000000in}}{%
\pgfpathmoveto{\pgfqpoint{0.000000in}{0.000000in}}%
\pgfpathlineto{\pgfqpoint{0.000000in}{-0.027778in}}%
\pgfusepath{stroke,fill}%
}%
\begin{pgfscope}%
\pgfsys@transformshift{2.233490in}{0.417642in}%
\pgfsys@useobject{currentmarker}{}%
\end{pgfscope}%
\end{pgfscope}%
\begin{pgfscope}%
\pgfpathrectangle{\pgfqpoint{0.594525in}{0.417642in}}{\pgfqpoint{3.354228in}{2.055000in}}%
\pgfusepath{clip}%
\pgfsetrectcap%
\pgfsetroundjoin%
\pgfsetlinewidth{0.803000pt}%
\definecolor{currentstroke}{rgb}{0.850000,0.850000,0.850000}%
\pgfsetstrokecolor{currentstroke}%
\pgfsetdash{}{0pt}%
\pgfpathmoveto{\pgfqpoint{2.314663in}{0.417642in}}%
\pgfpathlineto{\pgfqpoint{2.314663in}{2.472642in}}%
\pgfusepath{stroke}%
\end{pgfscope}%
\begin{pgfscope}%
\pgfsetbuttcap%
\pgfsetroundjoin%
\definecolor{currentfill}{rgb}{0.000000,0.000000,0.000000}%
\pgfsetfillcolor{currentfill}%
\pgfsetlinewidth{0.602250pt}%
\definecolor{currentstroke}{rgb}{0.000000,0.000000,0.000000}%
\pgfsetstrokecolor{currentstroke}%
\pgfsetdash{}{0pt}%
\pgfsys@defobject{currentmarker}{\pgfqpoint{0.000000in}{-0.027778in}}{\pgfqpoint{0.000000in}{0.000000in}}{%
\pgfpathmoveto{\pgfqpoint{0.000000in}{0.000000in}}%
\pgfpathlineto{\pgfqpoint{0.000000in}{-0.027778in}}%
\pgfusepath{stroke,fill}%
}%
\begin{pgfscope}%
\pgfsys@transformshift{2.314663in}{0.417642in}%
\pgfsys@useobject{currentmarker}{}%
\end{pgfscope}%
\end{pgfscope}%
\begin{pgfscope}%
\pgfpathrectangle{\pgfqpoint{0.594525in}{0.417642in}}{\pgfqpoint{3.354228in}{2.055000in}}%
\pgfusepath{clip}%
\pgfsetrectcap%
\pgfsetroundjoin%
\pgfsetlinewidth{0.803000pt}%
\definecolor{currentstroke}{rgb}{0.850000,0.850000,0.850000}%
\pgfsetstrokecolor{currentstroke}%
\pgfsetdash{}{0pt}%
\pgfpathmoveto{\pgfqpoint{2.372256in}{0.417642in}}%
\pgfpathlineto{\pgfqpoint{2.372256in}{2.472642in}}%
\pgfusepath{stroke}%
\end{pgfscope}%
\begin{pgfscope}%
\pgfsetbuttcap%
\pgfsetroundjoin%
\definecolor{currentfill}{rgb}{0.000000,0.000000,0.000000}%
\pgfsetfillcolor{currentfill}%
\pgfsetlinewidth{0.602250pt}%
\definecolor{currentstroke}{rgb}{0.000000,0.000000,0.000000}%
\pgfsetstrokecolor{currentstroke}%
\pgfsetdash{}{0pt}%
\pgfsys@defobject{currentmarker}{\pgfqpoint{0.000000in}{-0.027778in}}{\pgfqpoint{0.000000in}{0.000000in}}{%
\pgfpathmoveto{\pgfqpoint{0.000000in}{0.000000in}}%
\pgfpathlineto{\pgfqpoint{0.000000in}{-0.027778in}}%
\pgfusepath{stroke,fill}%
}%
\begin{pgfscope}%
\pgfsys@transformshift{2.372256in}{0.417642in}%
\pgfsys@useobject{currentmarker}{}%
\end{pgfscope}%
\end{pgfscope}%
\begin{pgfscope}%
\pgfpathrectangle{\pgfqpoint{0.594525in}{0.417642in}}{\pgfqpoint{3.354228in}{2.055000in}}%
\pgfusepath{clip}%
\pgfsetrectcap%
\pgfsetroundjoin%
\pgfsetlinewidth{0.803000pt}%
\definecolor{currentstroke}{rgb}{0.850000,0.850000,0.850000}%
\pgfsetstrokecolor{currentstroke}%
\pgfsetdash{}{0pt}%
\pgfpathmoveto{\pgfqpoint{2.416929in}{0.417642in}}%
\pgfpathlineto{\pgfqpoint{2.416929in}{2.472642in}}%
\pgfusepath{stroke}%
\end{pgfscope}%
\begin{pgfscope}%
\pgfsetbuttcap%
\pgfsetroundjoin%
\definecolor{currentfill}{rgb}{0.000000,0.000000,0.000000}%
\pgfsetfillcolor{currentfill}%
\pgfsetlinewidth{0.602250pt}%
\definecolor{currentstroke}{rgb}{0.000000,0.000000,0.000000}%
\pgfsetstrokecolor{currentstroke}%
\pgfsetdash{}{0pt}%
\pgfsys@defobject{currentmarker}{\pgfqpoint{0.000000in}{-0.027778in}}{\pgfqpoint{0.000000in}{0.000000in}}{%
\pgfpathmoveto{\pgfqpoint{0.000000in}{0.000000in}}%
\pgfpathlineto{\pgfqpoint{0.000000in}{-0.027778in}}%
\pgfusepath{stroke,fill}%
}%
\begin{pgfscope}%
\pgfsys@transformshift{2.416929in}{0.417642in}%
\pgfsys@useobject{currentmarker}{}%
\end{pgfscope}%
\end{pgfscope}%
\begin{pgfscope}%
\pgfpathrectangle{\pgfqpoint{0.594525in}{0.417642in}}{\pgfqpoint{3.354228in}{2.055000in}}%
\pgfusepath{clip}%
\pgfsetrectcap%
\pgfsetroundjoin%
\pgfsetlinewidth{0.803000pt}%
\definecolor{currentstroke}{rgb}{0.850000,0.850000,0.850000}%
\pgfsetstrokecolor{currentstroke}%
\pgfsetdash{}{0pt}%
\pgfpathmoveto{\pgfqpoint{2.453429in}{0.417642in}}%
\pgfpathlineto{\pgfqpoint{2.453429in}{2.472642in}}%
\pgfusepath{stroke}%
\end{pgfscope}%
\begin{pgfscope}%
\pgfsetbuttcap%
\pgfsetroundjoin%
\definecolor{currentfill}{rgb}{0.000000,0.000000,0.000000}%
\pgfsetfillcolor{currentfill}%
\pgfsetlinewidth{0.602250pt}%
\definecolor{currentstroke}{rgb}{0.000000,0.000000,0.000000}%
\pgfsetstrokecolor{currentstroke}%
\pgfsetdash{}{0pt}%
\pgfsys@defobject{currentmarker}{\pgfqpoint{0.000000in}{-0.027778in}}{\pgfqpoint{0.000000in}{0.000000in}}{%
\pgfpathmoveto{\pgfqpoint{0.000000in}{0.000000in}}%
\pgfpathlineto{\pgfqpoint{0.000000in}{-0.027778in}}%
\pgfusepath{stroke,fill}%
}%
\begin{pgfscope}%
\pgfsys@transformshift{2.453429in}{0.417642in}%
\pgfsys@useobject{currentmarker}{}%
\end{pgfscope}%
\end{pgfscope}%
\begin{pgfscope}%
\pgfpathrectangle{\pgfqpoint{0.594525in}{0.417642in}}{\pgfqpoint{3.354228in}{2.055000in}}%
\pgfusepath{clip}%
\pgfsetrectcap%
\pgfsetroundjoin%
\pgfsetlinewidth{0.803000pt}%
\definecolor{currentstroke}{rgb}{0.850000,0.850000,0.850000}%
\pgfsetstrokecolor{currentstroke}%
\pgfsetdash{}{0pt}%
\pgfpathmoveto{\pgfqpoint{2.484290in}{0.417642in}}%
\pgfpathlineto{\pgfqpoint{2.484290in}{2.472642in}}%
\pgfusepath{stroke}%
\end{pgfscope}%
\begin{pgfscope}%
\pgfsetbuttcap%
\pgfsetroundjoin%
\definecolor{currentfill}{rgb}{0.000000,0.000000,0.000000}%
\pgfsetfillcolor{currentfill}%
\pgfsetlinewidth{0.602250pt}%
\definecolor{currentstroke}{rgb}{0.000000,0.000000,0.000000}%
\pgfsetstrokecolor{currentstroke}%
\pgfsetdash{}{0pt}%
\pgfsys@defobject{currentmarker}{\pgfqpoint{0.000000in}{-0.027778in}}{\pgfqpoint{0.000000in}{0.000000in}}{%
\pgfpathmoveto{\pgfqpoint{0.000000in}{0.000000in}}%
\pgfpathlineto{\pgfqpoint{0.000000in}{-0.027778in}}%
\pgfusepath{stroke,fill}%
}%
\begin{pgfscope}%
\pgfsys@transformshift{2.484290in}{0.417642in}%
\pgfsys@useobject{currentmarker}{}%
\end{pgfscope}%
\end{pgfscope}%
\begin{pgfscope}%
\pgfpathrectangle{\pgfqpoint{0.594525in}{0.417642in}}{\pgfqpoint{3.354228in}{2.055000in}}%
\pgfusepath{clip}%
\pgfsetrectcap%
\pgfsetroundjoin%
\pgfsetlinewidth{0.803000pt}%
\definecolor{currentstroke}{rgb}{0.850000,0.850000,0.850000}%
\pgfsetstrokecolor{currentstroke}%
\pgfsetdash{}{0pt}%
\pgfpathmoveto{\pgfqpoint{2.511022in}{0.417642in}}%
\pgfpathlineto{\pgfqpoint{2.511022in}{2.472642in}}%
\pgfusepath{stroke}%
\end{pgfscope}%
\begin{pgfscope}%
\pgfsetbuttcap%
\pgfsetroundjoin%
\definecolor{currentfill}{rgb}{0.000000,0.000000,0.000000}%
\pgfsetfillcolor{currentfill}%
\pgfsetlinewidth{0.602250pt}%
\definecolor{currentstroke}{rgb}{0.000000,0.000000,0.000000}%
\pgfsetstrokecolor{currentstroke}%
\pgfsetdash{}{0pt}%
\pgfsys@defobject{currentmarker}{\pgfqpoint{0.000000in}{-0.027778in}}{\pgfqpoint{0.000000in}{0.000000in}}{%
\pgfpathmoveto{\pgfqpoint{0.000000in}{0.000000in}}%
\pgfpathlineto{\pgfqpoint{0.000000in}{-0.027778in}}%
\pgfusepath{stroke,fill}%
}%
\begin{pgfscope}%
\pgfsys@transformshift{2.511022in}{0.417642in}%
\pgfsys@useobject{currentmarker}{}%
\end{pgfscope}%
\end{pgfscope}%
\begin{pgfscope}%
\pgfpathrectangle{\pgfqpoint{0.594525in}{0.417642in}}{\pgfqpoint{3.354228in}{2.055000in}}%
\pgfusepath{clip}%
\pgfsetrectcap%
\pgfsetroundjoin%
\pgfsetlinewidth{0.803000pt}%
\definecolor{currentstroke}{rgb}{0.850000,0.850000,0.850000}%
\pgfsetstrokecolor{currentstroke}%
\pgfsetdash{}{0pt}%
\pgfpathmoveto{\pgfqpoint{2.534602in}{0.417642in}}%
\pgfpathlineto{\pgfqpoint{2.534602in}{2.472642in}}%
\pgfusepath{stroke}%
\end{pgfscope}%
\begin{pgfscope}%
\pgfsetbuttcap%
\pgfsetroundjoin%
\definecolor{currentfill}{rgb}{0.000000,0.000000,0.000000}%
\pgfsetfillcolor{currentfill}%
\pgfsetlinewidth{0.602250pt}%
\definecolor{currentstroke}{rgb}{0.000000,0.000000,0.000000}%
\pgfsetstrokecolor{currentstroke}%
\pgfsetdash{}{0pt}%
\pgfsys@defobject{currentmarker}{\pgfqpoint{0.000000in}{-0.027778in}}{\pgfqpoint{0.000000in}{0.000000in}}{%
\pgfpathmoveto{\pgfqpoint{0.000000in}{0.000000in}}%
\pgfpathlineto{\pgfqpoint{0.000000in}{-0.027778in}}%
\pgfusepath{stroke,fill}%
}%
\begin{pgfscope}%
\pgfsys@transformshift{2.534602in}{0.417642in}%
\pgfsys@useobject{currentmarker}{}%
\end{pgfscope}%
\end{pgfscope}%
\begin{pgfscope}%
\pgfpathrectangle{\pgfqpoint{0.594525in}{0.417642in}}{\pgfqpoint{3.354228in}{2.055000in}}%
\pgfusepath{clip}%
\pgfsetrectcap%
\pgfsetroundjoin%
\pgfsetlinewidth{0.803000pt}%
\definecolor{currentstroke}{rgb}{0.850000,0.850000,0.850000}%
\pgfsetstrokecolor{currentstroke}%
\pgfsetdash{}{0pt}%
\pgfpathmoveto{\pgfqpoint{2.694461in}{0.417642in}}%
\pgfpathlineto{\pgfqpoint{2.694461in}{2.472642in}}%
\pgfusepath{stroke}%
\end{pgfscope}%
\begin{pgfscope}%
\pgfsetbuttcap%
\pgfsetroundjoin%
\definecolor{currentfill}{rgb}{0.000000,0.000000,0.000000}%
\pgfsetfillcolor{currentfill}%
\pgfsetlinewidth{0.602250pt}%
\definecolor{currentstroke}{rgb}{0.000000,0.000000,0.000000}%
\pgfsetstrokecolor{currentstroke}%
\pgfsetdash{}{0pt}%
\pgfsys@defobject{currentmarker}{\pgfqpoint{0.000000in}{-0.027778in}}{\pgfqpoint{0.000000in}{0.000000in}}{%
\pgfpathmoveto{\pgfqpoint{0.000000in}{0.000000in}}%
\pgfpathlineto{\pgfqpoint{0.000000in}{-0.027778in}}%
\pgfusepath{stroke,fill}%
}%
\begin{pgfscope}%
\pgfsys@transformshift{2.694461in}{0.417642in}%
\pgfsys@useobject{currentmarker}{}%
\end{pgfscope}%
\end{pgfscope}%
\begin{pgfscope}%
\pgfpathrectangle{\pgfqpoint{0.594525in}{0.417642in}}{\pgfqpoint{3.354228in}{2.055000in}}%
\pgfusepath{clip}%
\pgfsetrectcap%
\pgfsetroundjoin%
\pgfsetlinewidth{0.803000pt}%
\definecolor{currentstroke}{rgb}{0.850000,0.850000,0.850000}%
\pgfsetstrokecolor{currentstroke}%
\pgfsetdash{}{0pt}%
\pgfpathmoveto{\pgfqpoint{2.775634in}{0.417642in}}%
\pgfpathlineto{\pgfqpoint{2.775634in}{2.472642in}}%
\pgfusepath{stroke}%
\end{pgfscope}%
\begin{pgfscope}%
\pgfsetbuttcap%
\pgfsetroundjoin%
\definecolor{currentfill}{rgb}{0.000000,0.000000,0.000000}%
\pgfsetfillcolor{currentfill}%
\pgfsetlinewidth{0.602250pt}%
\definecolor{currentstroke}{rgb}{0.000000,0.000000,0.000000}%
\pgfsetstrokecolor{currentstroke}%
\pgfsetdash{}{0pt}%
\pgfsys@defobject{currentmarker}{\pgfqpoint{0.000000in}{-0.027778in}}{\pgfqpoint{0.000000in}{0.000000in}}{%
\pgfpathmoveto{\pgfqpoint{0.000000in}{0.000000in}}%
\pgfpathlineto{\pgfqpoint{0.000000in}{-0.027778in}}%
\pgfusepath{stroke,fill}%
}%
\begin{pgfscope}%
\pgfsys@transformshift{2.775634in}{0.417642in}%
\pgfsys@useobject{currentmarker}{}%
\end{pgfscope}%
\end{pgfscope}%
\begin{pgfscope}%
\pgfpathrectangle{\pgfqpoint{0.594525in}{0.417642in}}{\pgfqpoint{3.354228in}{2.055000in}}%
\pgfusepath{clip}%
\pgfsetrectcap%
\pgfsetroundjoin%
\pgfsetlinewidth{0.803000pt}%
\definecolor{currentstroke}{rgb}{0.850000,0.850000,0.850000}%
\pgfsetstrokecolor{currentstroke}%
\pgfsetdash{}{0pt}%
\pgfpathmoveto{\pgfqpoint{2.833227in}{0.417642in}}%
\pgfpathlineto{\pgfqpoint{2.833227in}{2.472642in}}%
\pgfusepath{stroke}%
\end{pgfscope}%
\begin{pgfscope}%
\pgfsetbuttcap%
\pgfsetroundjoin%
\definecolor{currentfill}{rgb}{0.000000,0.000000,0.000000}%
\pgfsetfillcolor{currentfill}%
\pgfsetlinewidth{0.602250pt}%
\definecolor{currentstroke}{rgb}{0.000000,0.000000,0.000000}%
\pgfsetstrokecolor{currentstroke}%
\pgfsetdash{}{0pt}%
\pgfsys@defobject{currentmarker}{\pgfqpoint{0.000000in}{-0.027778in}}{\pgfqpoint{0.000000in}{0.000000in}}{%
\pgfpathmoveto{\pgfqpoint{0.000000in}{0.000000in}}%
\pgfpathlineto{\pgfqpoint{0.000000in}{-0.027778in}}%
\pgfusepath{stroke,fill}%
}%
\begin{pgfscope}%
\pgfsys@transformshift{2.833227in}{0.417642in}%
\pgfsys@useobject{currentmarker}{}%
\end{pgfscope}%
\end{pgfscope}%
\begin{pgfscope}%
\pgfpathrectangle{\pgfqpoint{0.594525in}{0.417642in}}{\pgfqpoint{3.354228in}{2.055000in}}%
\pgfusepath{clip}%
\pgfsetrectcap%
\pgfsetroundjoin%
\pgfsetlinewidth{0.803000pt}%
\definecolor{currentstroke}{rgb}{0.850000,0.850000,0.850000}%
\pgfsetstrokecolor{currentstroke}%
\pgfsetdash{}{0pt}%
\pgfpathmoveto{\pgfqpoint{2.877899in}{0.417642in}}%
\pgfpathlineto{\pgfqpoint{2.877899in}{2.472642in}}%
\pgfusepath{stroke}%
\end{pgfscope}%
\begin{pgfscope}%
\pgfsetbuttcap%
\pgfsetroundjoin%
\definecolor{currentfill}{rgb}{0.000000,0.000000,0.000000}%
\pgfsetfillcolor{currentfill}%
\pgfsetlinewidth{0.602250pt}%
\definecolor{currentstroke}{rgb}{0.000000,0.000000,0.000000}%
\pgfsetstrokecolor{currentstroke}%
\pgfsetdash{}{0pt}%
\pgfsys@defobject{currentmarker}{\pgfqpoint{0.000000in}{-0.027778in}}{\pgfqpoint{0.000000in}{0.000000in}}{%
\pgfpathmoveto{\pgfqpoint{0.000000in}{0.000000in}}%
\pgfpathlineto{\pgfqpoint{0.000000in}{-0.027778in}}%
\pgfusepath{stroke,fill}%
}%
\begin{pgfscope}%
\pgfsys@transformshift{2.877899in}{0.417642in}%
\pgfsys@useobject{currentmarker}{}%
\end{pgfscope}%
\end{pgfscope}%
\begin{pgfscope}%
\pgfpathrectangle{\pgfqpoint{0.594525in}{0.417642in}}{\pgfqpoint{3.354228in}{2.055000in}}%
\pgfusepath{clip}%
\pgfsetrectcap%
\pgfsetroundjoin%
\pgfsetlinewidth{0.803000pt}%
\definecolor{currentstroke}{rgb}{0.850000,0.850000,0.850000}%
\pgfsetstrokecolor{currentstroke}%
\pgfsetdash{}{0pt}%
\pgfpathmoveto{\pgfqpoint{2.914400in}{0.417642in}}%
\pgfpathlineto{\pgfqpoint{2.914400in}{2.472642in}}%
\pgfusepath{stroke}%
\end{pgfscope}%
\begin{pgfscope}%
\pgfsetbuttcap%
\pgfsetroundjoin%
\definecolor{currentfill}{rgb}{0.000000,0.000000,0.000000}%
\pgfsetfillcolor{currentfill}%
\pgfsetlinewidth{0.602250pt}%
\definecolor{currentstroke}{rgb}{0.000000,0.000000,0.000000}%
\pgfsetstrokecolor{currentstroke}%
\pgfsetdash{}{0pt}%
\pgfsys@defobject{currentmarker}{\pgfqpoint{0.000000in}{-0.027778in}}{\pgfqpoint{0.000000in}{0.000000in}}{%
\pgfpathmoveto{\pgfqpoint{0.000000in}{0.000000in}}%
\pgfpathlineto{\pgfqpoint{0.000000in}{-0.027778in}}%
\pgfusepath{stroke,fill}%
}%
\begin{pgfscope}%
\pgfsys@transformshift{2.914400in}{0.417642in}%
\pgfsys@useobject{currentmarker}{}%
\end{pgfscope}%
\end{pgfscope}%
\begin{pgfscope}%
\pgfpathrectangle{\pgfqpoint{0.594525in}{0.417642in}}{\pgfqpoint{3.354228in}{2.055000in}}%
\pgfusepath{clip}%
\pgfsetrectcap%
\pgfsetroundjoin%
\pgfsetlinewidth{0.803000pt}%
\definecolor{currentstroke}{rgb}{0.850000,0.850000,0.850000}%
\pgfsetstrokecolor{currentstroke}%
\pgfsetdash{}{0pt}%
\pgfpathmoveto{\pgfqpoint{2.945260in}{0.417642in}}%
\pgfpathlineto{\pgfqpoint{2.945260in}{2.472642in}}%
\pgfusepath{stroke}%
\end{pgfscope}%
\begin{pgfscope}%
\pgfsetbuttcap%
\pgfsetroundjoin%
\definecolor{currentfill}{rgb}{0.000000,0.000000,0.000000}%
\pgfsetfillcolor{currentfill}%
\pgfsetlinewidth{0.602250pt}%
\definecolor{currentstroke}{rgb}{0.000000,0.000000,0.000000}%
\pgfsetstrokecolor{currentstroke}%
\pgfsetdash{}{0pt}%
\pgfsys@defobject{currentmarker}{\pgfqpoint{0.000000in}{-0.027778in}}{\pgfqpoint{0.000000in}{0.000000in}}{%
\pgfpathmoveto{\pgfqpoint{0.000000in}{0.000000in}}%
\pgfpathlineto{\pgfqpoint{0.000000in}{-0.027778in}}%
\pgfusepath{stroke,fill}%
}%
\begin{pgfscope}%
\pgfsys@transformshift{2.945260in}{0.417642in}%
\pgfsys@useobject{currentmarker}{}%
\end{pgfscope}%
\end{pgfscope}%
\begin{pgfscope}%
\pgfpathrectangle{\pgfqpoint{0.594525in}{0.417642in}}{\pgfqpoint{3.354228in}{2.055000in}}%
\pgfusepath{clip}%
\pgfsetrectcap%
\pgfsetroundjoin%
\pgfsetlinewidth{0.803000pt}%
\definecolor{currentstroke}{rgb}{0.850000,0.850000,0.850000}%
\pgfsetstrokecolor{currentstroke}%
\pgfsetdash{}{0pt}%
\pgfpathmoveto{\pgfqpoint{2.971993in}{0.417642in}}%
\pgfpathlineto{\pgfqpoint{2.971993in}{2.472642in}}%
\pgfusepath{stroke}%
\end{pgfscope}%
\begin{pgfscope}%
\pgfsetbuttcap%
\pgfsetroundjoin%
\definecolor{currentfill}{rgb}{0.000000,0.000000,0.000000}%
\pgfsetfillcolor{currentfill}%
\pgfsetlinewidth{0.602250pt}%
\definecolor{currentstroke}{rgb}{0.000000,0.000000,0.000000}%
\pgfsetstrokecolor{currentstroke}%
\pgfsetdash{}{0pt}%
\pgfsys@defobject{currentmarker}{\pgfqpoint{0.000000in}{-0.027778in}}{\pgfqpoint{0.000000in}{0.000000in}}{%
\pgfpathmoveto{\pgfqpoint{0.000000in}{0.000000in}}%
\pgfpathlineto{\pgfqpoint{0.000000in}{-0.027778in}}%
\pgfusepath{stroke,fill}%
}%
\begin{pgfscope}%
\pgfsys@transformshift{2.971993in}{0.417642in}%
\pgfsys@useobject{currentmarker}{}%
\end{pgfscope}%
\end{pgfscope}%
\begin{pgfscope}%
\pgfpathrectangle{\pgfqpoint{0.594525in}{0.417642in}}{\pgfqpoint{3.354228in}{2.055000in}}%
\pgfusepath{clip}%
\pgfsetrectcap%
\pgfsetroundjoin%
\pgfsetlinewidth{0.803000pt}%
\definecolor{currentstroke}{rgb}{0.850000,0.850000,0.850000}%
\pgfsetstrokecolor{currentstroke}%
\pgfsetdash{}{0pt}%
\pgfpathmoveto{\pgfqpoint{2.995573in}{0.417642in}}%
\pgfpathlineto{\pgfqpoint{2.995573in}{2.472642in}}%
\pgfusepath{stroke}%
\end{pgfscope}%
\begin{pgfscope}%
\pgfsetbuttcap%
\pgfsetroundjoin%
\definecolor{currentfill}{rgb}{0.000000,0.000000,0.000000}%
\pgfsetfillcolor{currentfill}%
\pgfsetlinewidth{0.602250pt}%
\definecolor{currentstroke}{rgb}{0.000000,0.000000,0.000000}%
\pgfsetstrokecolor{currentstroke}%
\pgfsetdash{}{0pt}%
\pgfsys@defobject{currentmarker}{\pgfqpoint{0.000000in}{-0.027778in}}{\pgfqpoint{0.000000in}{0.000000in}}{%
\pgfpathmoveto{\pgfqpoint{0.000000in}{0.000000in}}%
\pgfpathlineto{\pgfqpoint{0.000000in}{-0.027778in}}%
\pgfusepath{stroke,fill}%
}%
\begin{pgfscope}%
\pgfsys@transformshift{2.995573in}{0.417642in}%
\pgfsys@useobject{currentmarker}{}%
\end{pgfscope}%
\end{pgfscope}%
\begin{pgfscope}%
\pgfpathrectangle{\pgfqpoint{0.594525in}{0.417642in}}{\pgfqpoint{3.354228in}{2.055000in}}%
\pgfusepath{clip}%
\pgfsetrectcap%
\pgfsetroundjoin%
\pgfsetlinewidth{0.803000pt}%
\definecolor{currentstroke}{rgb}{0.850000,0.850000,0.850000}%
\pgfsetstrokecolor{currentstroke}%
\pgfsetdash{}{0pt}%
\pgfpathmoveto{\pgfqpoint{3.155431in}{0.417642in}}%
\pgfpathlineto{\pgfqpoint{3.155431in}{2.472642in}}%
\pgfusepath{stroke}%
\end{pgfscope}%
\begin{pgfscope}%
\pgfsetbuttcap%
\pgfsetroundjoin%
\definecolor{currentfill}{rgb}{0.000000,0.000000,0.000000}%
\pgfsetfillcolor{currentfill}%
\pgfsetlinewidth{0.602250pt}%
\definecolor{currentstroke}{rgb}{0.000000,0.000000,0.000000}%
\pgfsetstrokecolor{currentstroke}%
\pgfsetdash{}{0pt}%
\pgfsys@defobject{currentmarker}{\pgfqpoint{0.000000in}{-0.027778in}}{\pgfqpoint{0.000000in}{0.000000in}}{%
\pgfpathmoveto{\pgfqpoint{0.000000in}{0.000000in}}%
\pgfpathlineto{\pgfqpoint{0.000000in}{-0.027778in}}%
\pgfusepath{stroke,fill}%
}%
\begin{pgfscope}%
\pgfsys@transformshift{3.155431in}{0.417642in}%
\pgfsys@useobject{currentmarker}{}%
\end{pgfscope}%
\end{pgfscope}%
\begin{pgfscope}%
\pgfpathrectangle{\pgfqpoint{0.594525in}{0.417642in}}{\pgfqpoint{3.354228in}{2.055000in}}%
\pgfusepath{clip}%
\pgfsetrectcap%
\pgfsetroundjoin%
\pgfsetlinewidth{0.803000pt}%
\definecolor{currentstroke}{rgb}{0.850000,0.850000,0.850000}%
\pgfsetstrokecolor{currentstroke}%
\pgfsetdash{}{0pt}%
\pgfpathmoveto{\pgfqpoint{3.236604in}{0.417642in}}%
\pgfpathlineto{\pgfqpoint{3.236604in}{2.472642in}}%
\pgfusepath{stroke}%
\end{pgfscope}%
\begin{pgfscope}%
\pgfsetbuttcap%
\pgfsetroundjoin%
\definecolor{currentfill}{rgb}{0.000000,0.000000,0.000000}%
\pgfsetfillcolor{currentfill}%
\pgfsetlinewidth{0.602250pt}%
\definecolor{currentstroke}{rgb}{0.000000,0.000000,0.000000}%
\pgfsetstrokecolor{currentstroke}%
\pgfsetdash{}{0pt}%
\pgfsys@defobject{currentmarker}{\pgfqpoint{0.000000in}{-0.027778in}}{\pgfqpoint{0.000000in}{0.000000in}}{%
\pgfpathmoveto{\pgfqpoint{0.000000in}{0.000000in}}%
\pgfpathlineto{\pgfqpoint{0.000000in}{-0.027778in}}%
\pgfusepath{stroke,fill}%
}%
\begin{pgfscope}%
\pgfsys@transformshift{3.236604in}{0.417642in}%
\pgfsys@useobject{currentmarker}{}%
\end{pgfscope}%
\end{pgfscope}%
\begin{pgfscope}%
\pgfpathrectangle{\pgfqpoint{0.594525in}{0.417642in}}{\pgfqpoint{3.354228in}{2.055000in}}%
\pgfusepath{clip}%
\pgfsetrectcap%
\pgfsetroundjoin%
\pgfsetlinewidth{0.803000pt}%
\definecolor{currentstroke}{rgb}{0.850000,0.850000,0.850000}%
\pgfsetstrokecolor{currentstroke}%
\pgfsetdash{}{0pt}%
\pgfpathmoveto{\pgfqpoint{3.294197in}{0.417642in}}%
\pgfpathlineto{\pgfqpoint{3.294197in}{2.472642in}}%
\pgfusepath{stroke}%
\end{pgfscope}%
\begin{pgfscope}%
\pgfsetbuttcap%
\pgfsetroundjoin%
\definecolor{currentfill}{rgb}{0.000000,0.000000,0.000000}%
\pgfsetfillcolor{currentfill}%
\pgfsetlinewidth{0.602250pt}%
\definecolor{currentstroke}{rgb}{0.000000,0.000000,0.000000}%
\pgfsetstrokecolor{currentstroke}%
\pgfsetdash{}{0pt}%
\pgfsys@defobject{currentmarker}{\pgfqpoint{0.000000in}{-0.027778in}}{\pgfqpoint{0.000000in}{0.000000in}}{%
\pgfpathmoveto{\pgfqpoint{0.000000in}{0.000000in}}%
\pgfpathlineto{\pgfqpoint{0.000000in}{-0.027778in}}%
\pgfusepath{stroke,fill}%
}%
\begin{pgfscope}%
\pgfsys@transformshift{3.294197in}{0.417642in}%
\pgfsys@useobject{currentmarker}{}%
\end{pgfscope}%
\end{pgfscope}%
\begin{pgfscope}%
\pgfpathrectangle{\pgfqpoint{0.594525in}{0.417642in}}{\pgfqpoint{3.354228in}{2.055000in}}%
\pgfusepath{clip}%
\pgfsetrectcap%
\pgfsetroundjoin%
\pgfsetlinewidth{0.803000pt}%
\definecolor{currentstroke}{rgb}{0.850000,0.850000,0.850000}%
\pgfsetstrokecolor{currentstroke}%
\pgfsetdash{}{0pt}%
\pgfpathmoveto{\pgfqpoint{3.338870in}{0.417642in}}%
\pgfpathlineto{\pgfqpoint{3.338870in}{2.472642in}}%
\pgfusepath{stroke}%
\end{pgfscope}%
\begin{pgfscope}%
\pgfsetbuttcap%
\pgfsetroundjoin%
\definecolor{currentfill}{rgb}{0.000000,0.000000,0.000000}%
\pgfsetfillcolor{currentfill}%
\pgfsetlinewidth{0.602250pt}%
\definecolor{currentstroke}{rgb}{0.000000,0.000000,0.000000}%
\pgfsetstrokecolor{currentstroke}%
\pgfsetdash{}{0pt}%
\pgfsys@defobject{currentmarker}{\pgfqpoint{0.000000in}{-0.027778in}}{\pgfqpoint{0.000000in}{0.000000in}}{%
\pgfpathmoveto{\pgfqpoint{0.000000in}{0.000000in}}%
\pgfpathlineto{\pgfqpoint{0.000000in}{-0.027778in}}%
\pgfusepath{stroke,fill}%
}%
\begin{pgfscope}%
\pgfsys@transformshift{3.338870in}{0.417642in}%
\pgfsys@useobject{currentmarker}{}%
\end{pgfscope}%
\end{pgfscope}%
\begin{pgfscope}%
\pgfpathrectangle{\pgfqpoint{0.594525in}{0.417642in}}{\pgfqpoint{3.354228in}{2.055000in}}%
\pgfusepath{clip}%
\pgfsetrectcap%
\pgfsetroundjoin%
\pgfsetlinewidth{0.803000pt}%
\definecolor{currentstroke}{rgb}{0.850000,0.850000,0.850000}%
\pgfsetstrokecolor{currentstroke}%
\pgfsetdash{}{0pt}%
\pgfpathmoveto{\pgfqpoint{3.375370in}{0.417642in}}%
\pgfpathlineto{\pgfqpoint{3.375370in}{2.472642in}}%
\pgfusepath{stroke}%
\end{pgfscope}%
\begin{pgfscope}%
\pgfsetbuttcap%
\pgfsetroundjoin%
\definecolor{currentfill}{rgb}{0.000000,0.000000,0.000000}%
\pgfsetfillcolor{currentfill}%
\pgfsetlinewidth{0.602250pt}%
\definecolor{currentstroke}{rgb}{0.000000,0.000000,0.000000}%
\pgfsetstrokecolor{currentstroke}%
\pgfsetdash{}{0pt}%
\pgfsys@defobject{currentmarker}{\pgfqpoint{0.000000in}{-0.027778in}}{\pgfqpoint{0.000000in}{0.000000in}}{%
\pgfpathmoveto{\pgfqpoint{0.000000in}{0.000000in}}%
\pgfpathlineto{\pgfqpoint{0.000000in}{-0.027778in}}%
\pgfusepath{stroke,fill}%
}%
\begin{pgfscope}%
\pgfsys@transformshift{3.375370in}{0.417642in}%
\pgfsys@useobject{currentmarker}{}%
\end{pgfscope}%
\end{pgfscope}%
\begin{pgfscope}%
\pgfpathrectangle{\pgfqpoint{0.594525in}{0.417642in}}{\pgfqpoint{3.354228in}{2.055000in}}%
\pgfusepath{clip}%
\pgfsetrectcap%
\pgfsetroundjoin%
\pgfsetlinewidth{0.803000pt}%
\definecolor{currentstroke}{rgb}{0.850000,0.850000,0.850000}%
\pgfsetstrokecolor{currentstroke}%
\pgfsetdash{}{0pt}%
\pgfpathmoveto{\pgfqpoint{3.406231in}{0.417642in}}%
\pgfpathlineto{\pgfqpoint{3.406231in}{2.472642in}}%
\pgfusepath{stroke}%
\end{pgfscope}%
\begin{pgfscope}%
\pgfsetbuttcap%
\pgfsetroundjoin%
\definecolor{currentfill}{rgb}{0.000000,0.000000,0.000000}%
\pgfsetfillcolor{currentfill}%
\pgfsetlinewidth{0.602250pt}%
\definecolor{currentstroke}{rgb}{0.000000,0.000000,0.000000}%
\pgfsetstrokecolor{currentstroke}%
\pgfsetdash{}{0pt}%
\pgfsys@defobject{currentmarker}{\pgfqpoint{0.000000in}{-0.027778in}}{\pgfqpoint{0.000000in}{0.000000in}}{%
\pgfpathmoveto{\pgfqpoint{0.000000in}{0.000000in}}%
\pgfpathlineto{\pgfqpoint{0.000000in}{-0.027778in}}%
\pgfusepath{stroke,fill}%
}%
\begin{pgfscope}%
\pgfsys@transformshift{3.406231in}{0.417642in}%
\pgfsys@useobject{currentmarker}{}%
\end{pgfscope}%
\end{pgfscope}%
\begin{pgfscope}%
\pgfpathrectangle{\pgfqpoint{0.594525in}{0.417642in}}{\pgfqpoint{3.354228in}{2.055000in}}%
\pgfusepath{clip}%
\pgfsetrectcap%
\pgfsetroundjoin%
\pgfsetlinewidth{0.803000pt}%
\definecolor{currentstroke}{rgb}{0.850000,0.850000,0.850000}%
\pgfsetstrokecolor{currentstroke}%
\pgfsetdash{}{0pt}%
\pgfpathmoveto{\pgfqpoint{3.432963in}{0.417642in}}%
\pgfpathlineto{\pgfqpoint{3.432963in}{2.472642in}}%
\pgfusepath{stroke}%
\end{pgfscope}%
\begin{pgfscope}%
\pgfsetbuttcap%
\pgfsetroundjoin%
\definecolor{currentfill}{rgb}{0.000000,0.000000,0.000000}%
\pgfsetfillcolor{currentfill}%
\pgfsetlinewidth{0.602250pt}%
\definecolor{currentstroke}{rgb}{0.000000,0.000000,0.000000}%
\pgfsetstrokecolor{currentstroke}%
\pgfsetdash{}{0pt}%
\pgfsys@defobject{currentmarker}{\pgfqpoint{0.000000in}{-0.027778in}}{\pgfqpoint{0.000000in}{0.000000in}}{%
\pgfpathmoveto{\pgfqpoint{0.000000in}{0.000000in}}%
\pgfpathlineto{\pgfqpoint{0.000000in}{-0.027778in}}%
\pgfusepath{stroke,fill}%
}%
\begin{pgfscope}%
\pgfsys@transformshift{3.432963in}{0.417642in}%
\pgfsys@useobject{currentmarker}{}%
\end{pgfscope}%
\end{pgfscope}%
\begin{pgfscope}%
\pgfpathrectangle{\pgfqpoint{0.594525in}{0.417642in}}{\pgfqpoint{3.354228in}{2.055000in}}%
\pgfusepath{clip}%
\pgfsetrectcap%
\pgfsetroundjoin%
\pgfsetlinewidth{0.803000pt}%
\definecolor{currentstroke}{rgb}{0.850000,0.850000,0.850000}%
\pgfsetstrokecolor{currentstroke}%
\pgfsetdash{}{0pt}%
\pgfpathmoveto{\pgfqpoint{3.456543in}{0.417642in}}%
\pgfpathlineto{\pgfqpoint{3.456543in}{2.472642in}}%
\pgfusepath{stroke}%
\end{pgfscope}%
\begin{pgfscope}%
\pgfsetbuttcap%
\pgfsetroundjoin%
\definecolor{currentfill}{rgb}{0.000000,0.000000,0.000000}%
\pgfsetfillcolor{currentfill}%
\pgfsetlinewidth{0.602250pt}%
\definecolor{currentstroke}{rgb}{0.000000,0.000000,0.000000}%
\pgfsetstrokecolor{currentstroke}%
\pgfsetdash{}{0pt}%
\pgfsys@defobject{currentmarker}{\pgfqpoint{0.000000in}{-0.027778in}}{\pgfqpoint{0.000000in}{0.000000in}}{%
\pgfpathmoveto{\pgfqpoint{0.000000in}{0.000000in}}%
\pgfpathlineto{\pgfqpoint{0.000000in}{-0.027778in}}%
\pgfusepath{stroke,fill}%
}%
\begin{pgfscope}%
\pgfsys@transformshift{3.456543in}{0.417642in}%
\pgfsys@useobject{currentmarker}{}%
\end{pgfscope}%
\end{pgfscope}%
\begin{pgfscope}%
\pgfpathrectangle{\pgfqpoint{0.594525in}{0.417642in}}{\pgfqpoint{3.354228in}{2.055000in}}%
\pgfusepath{clip}%
\pgfsetrectcap%
\pgfsetroundjoin%
\pgfsetlinewidth{0.803000pt}%
\definecolor{currentstroke}{rgb}{0.850000,0.850000,0.850000}%
\pgfsetstrokecolor{currentstroke}%
\pgfsetdash{}{0pt}%
\pgfpathmoveto{\pgfqpoint{3.616402in}{0.417642in}}%
\pgfpathlineto{\pgfqpoint{3.616402in}{2.472642in}}%
\pgfusepath{stroke}%
\end{pgfscope}%
\begin{pgfscope}%
\pgfsetbuttcap%
\pgfsetroundjoin%
\definecolor{currentfill}{rgb}{0.000000,0.000000,0.000000}%
\pgfsetfillcolor{currentfill}%
\pgfsetlinewidth{0.602250pt}%
\definecolor{currentstroke}{rgb}{0.000000,0.000000,0.000000}%
\pgfsetstrokecolor{currentstroke}%
\pgfsetdash{}{0pt}%
\pgfsys@defobject{currentmarker}{\pgfqpoint{0.000000in}{-0.027778in}}{\pgfqpoint{0.000000in}{0.000000in}}{%
\pgfpathmoveto{\pgfqpoint{0.000000in}{0.000000in}}%
\pgfpathlineto{\pgfqpoint{0.000000in}{-0.027778in}}%
\pgfusepath{stroke,fill}%
}%
\begin{pgfscope}%
\pgfsys@transformshift{3.616402in}{0.417642in}%
\pgfsys@useobject{currentmarker}{}%
\end{pgfscope}%
\end{pgfscope}%
\begin{pgfscope}%
\pgfpathrectangle{\pgfqpoint{0.594525in}{0.417642in}}{\pgfqpoint{3.354228in}{2.055000in}}%
\pgfusepath{clip}%
\pgfsetrectcap%
\pgfsetroundjoin%
\pgfsetlinewidth{0.803000pt}%
\definecolor{currentstroke}{rgb}{0.850000,0.850000,0.850000}%
\pgfsetstrokecolor{currentstroke}%
\pgfsetdash{}{0pt}%
\pgfpathmoveto{\pgfqpoint{3.697575in}{0.417642in}}%
\pgfpathlineto{\pgfqpoint{3.697575in}{2.472642in}}%
\pgfusepath{stroke}%
\end{pgfscope}%
\begin{pgfscope}%
\pgfsetbuttcap%
\pgfsetroundjoin%
\definecolor{currentfill}{rgb}{0.000000,0.000000,0.000000}%
\pgfsetfillcolor{currentfill}%
\pgfsetlinewidth{0.602250pt}%
\definecolor{currentstroke}{rgb}{0.000000,0.000000,0.000000}%
\pgfsetstrokecolor{currentstroke}%
\pgfsetdash{}{0pt}%
\pgfsys@defobject{currentmarker}{\pgfqpoint{0.000000in}{-0.027778in}}{\pgfqpoint{0.000000in}{0.000000in}}{%
\pgfpathmoveto{\pgfqpoint{0.000000in}{0.000000in}}%
\pgfpathlineto{\pgfqpoint{0.000000in}{-0.027778in}}%
\pgfusepath{stroke,fill}%
}%
\begin{pgfscope}%
\pgfsys@transformshift{3.697575in}{0.417642in}%
\pgfsys@useobject{currentmarker}{}%
\end{pgfscope}%
\end{pgfscope}%
\begin{pgfscope}%
\pgfpathrectangle{\pgfqpoint{0.594525in}{0.417642in}}{\pgfqpoint{3.354228in}{2.055000in}}%
\pgfusepath{clip}%
\pgfsetrectcap%
\pgfsetroundjoin%
\pgfsetlinewidth{0.803000pt}%
\definecolor{currentstroke}{rgb}{0.850000,0.850000,0.850000}%
\pgfsetstrokecolor{currentstroke}%
\pgfsetdash{}{0pt}%
\pgfpathmoveto{\pgfqpoint{3.755168in}{0.417642in}}%
\pgfpathlineto{\pgfqpoint{3.755168in}{2.472642in}}%
\pgfusepath{stroke}%
\end{pgfscope}%
\begin{pgfscope}%
\pgfsetbuttcap%
\pgfsetroundjoin%
\definecolor{currentfill}{rgb}{0.000000,0.000000,0.000000}%
\pgfsetfillcolor{currentfill}%
\pgfsetlinewidth{0.602250pt}%
\definecolor{currentstroke}{rgb}{0.000000,0.000000,0.000000}%
\pgfsetstrokecolor{currentstroke}%
\pgfsetdash{}{0pt}%
\pgfsys@defobject{currentmarker}{\pgfqpoint{0.000000in}{-0.027778in}}{\pgfqpoint{0.000000in}{0.000000in}}{%
\pgfpathmoveto{\pgfqpoint{0.000000in}{0.000000in}}%
\pgfpathlineto{\pgfqpoint{0.000000in}{-0.027778in}}%
\pgfusepath{stroke,fill}%
}%
\begin{pgfscope}%
\pgfsys@transformshift{3.755168in}{0.417642in}%
\pgfsys@useobject{currentmarker}{}%
\end{pgfscope}%
\end{pgfscope}%
\begin{pgfscope}%
\pgfpathrectangle{\pgfqpoint{0.594525in}{0.417642in}}{\pgfqpoint{3.354228in}{2.055000in}}%
\pgfusepath{clip}%
\pgfsetrectcap%
\pgfsetroundjoin%
\pgfsetlinewidth{0.803000pt}%
\definecolor{currentstroke}{rgb}{0.850000,0.850000,0.850000}%
\pgfsetstrokecolor{currentstroke}%
\pgfsetdash{}{0pt}%
\pgfpathmoveto{\pgfqpoint{3.799840in}{0.417642in}}%
\pgfpathlineto{\pgfqpoint{3.799840in}{2.472642in}}%
\pgfusepath{stroke}%
\end{pgfscope}%
\begin{pgfscope}%
\pgfsetbuttcap%
\pgfsetroundjoin%
\definecolor{currentfill}{rgb}{0.000000,0.000000,0.000000}%
\pgfsetfillcolor{currentfill}%
\pgfsetlinewidth{0.602250pt}%
\definecolor{currentstroke}{rgb}{0.000000,0.000000,0.000000}%
\pgfsetstrokecolor{currentstroke}%
\pgfsetdash{}{0pt}%
\pgfsys@defobject{currentmarker}{\pgfqpoint{0.000000in}{-0.027778in}}{\pgfqpoint{0.000000in}{0.000000in}}{%
\pgfpathmoveto{\pgfqpoint{0.000000in}{0.000000in}}%
\pgfpathlineto{\pgfqpoint{0.000000in}{-0.027778in}}%
\pgfusepath{stroke,fill}%
}%
\begin{pgfscope}%
\pgfsys@transformshift{3.799840in}{0.417642in}%
\pgfsys@useobject{currentmarker}{}%
\end{pgfscope}%
\end{pgfscope}%
\begin{pgfscope}%
\pgfpathrectangle{\pgfqpoint{0.594525in}{0.417642in}}{\pgfqpoint{3.354228in}{2.055000in}}%
\pgfusepath{clip}%
\pgfsetrectcap%
\pgfsetroundjoin%
\pgfsetlinewidth{0.803000pt}%
\definecolor{currentstroke}{rgb}{0.850000,0.850000,0.850000}%
\pgfsetstrokecolor{currentstroke}%
\pgfsetdash{}{0pt}%
\pgfpathmoveto{\pgfqpoint{3.836341in}{0.417642in}}%
\pgfpathlineto{\pgfqpoint{3.836341in}{2.472642in}}%
\pgfusepath{stroke}%
\end{pgfscope}%
\begin{pgfscope}%
\pgfsetbuttcap%
\pgfsetroundjoin%
\definecolor{currentfill}{rgb}{0.000000,0.000000,0.000000}%
\pgfsetfillcolor{currentfill}%
\pgfsetlinewidth{0.602250pt}%
\definecolor{currentstroke}{rgb}{0.000000,0.000000,0.000000}%
\pgfsetstrokecolor{currentstroke}%
\pgfsetdash{}{0pt}%
\pgfsys@defobject{currentmarker}{\pgfqpoint{0.000000in}{-0.027778in}}{\pgfqpoint{0.000000in}{0.000000in}}{%
\pgfpathmoveto{\pgfqpoint{0.000000in}{0.000000in}}%
\pgfpathlineto{\pgfqpoint{0.000000in}{-0.027778in}}%
\pgfusepath{stroke,fill}%
}%
\begin{pgfscope}%
\pgfsys@transformshift{3.836341in}{0.417642in}%
\pgfsys@useobject{currentmarker}{}%
\end{pgfscope}%
\end{pgfscope}%
\begin{pgfscope}%
\pgfpathrectangle{\pgfqpoint{0.594525in}{0.417642in}}{\pgfqpoint{3.354228in}{2.055000in}}%
\pgfusepath{clip}%
\pgfsetrectcap%
\pgfsetroundjoin%
\pgfsetlinewidth{0.803000pt}%
\definecolor{currentstroke}{rgb}{0.850000,0.850000,0.850000}%
\pgfsetstrokecolor{currentstroke}%
\pgfsetdash{}{0pt}%
\pgfpathmoveto{\pgfqpoint{3.867201in}{0.417642in}}%
\pgfpathlineto{\pgfqpoint{3.867201in}{2.472642in}}%
\pgfusepath{stroke}%
\end{pgfscope}%
\begin{pgfscope}%
\pgfsetbuttcap%
\pgfsetroundjoin%
\definecolor{currentfill}{rgb}{0.000000,0.000000,0.000000}%
\pgfsetfillcolor{currentfill}%
\pgfsetlinewidth{0.602250pt}%
\definecolor{currentstroke}{rgb}{0.000000,0.000000,0.000000}%
\pgfsetstrokecolor{currentstroke}%
\pgfsetdash{}{0pt}%
\pgfsys@defobject{currentmarker}{\pgfqpoint{0.000000in}{-0.027778in}}{\pgfqpoint{0.000000in}{0.000000in}}{%
\pgfpathmoveto{\pgfqpoint{0.000000in}{0.000000in}}%
\pgfpathlineto{\pgfqpoint{0.000000in}{-0.027778in}}%
\pgfusepath{stroke,fill}%
}%
\begin{pgfscope}%
\pgfsys@transformshift{3.867201in}{0.417642in}%
\pgfsys@useobject{currentmarker}{}%
\end{pgfscope}%
\end{pgfscope}%
\begin{pgfscope}%
\pgfpathrectangle{\pgfqpoint{0.594525in}{0.417642in}}{\pgfqpoint{3.354228in}{2.055000in}}%
\pgfusepath{clip}%
\pgfsetrectcap%
\pgfsetroundjoin%
\pgfsetlinewidth{0.803000pt}%
\definecolor{currentstroke}{rgb}{0.850000,0.850000,0.850000}%
\pgfsetstrokecolor{currentstroke}%
\pgfsetdash{}{0pt}%
\pgfpathmoveto{\pgfqpoint{3.893934in}{0.417642in}}%
\pgfpathlineto{\pgfqpoint{3.893934in}{2.472642in}}%
\pgfusepath{stroke}%
\end{pgfscope}%
\begin{pgfscope}%
\pgfsetbuttcap%
\pgfsetroundjoin%
\definecolor{currentfill}{rgb}{0.000000,0.000000,0.000000}%
\pgfsetfillcolor{currentfill}%
\pgfsetlinewidth{0.602250pt}%
\definecolor{currentstroke}{rgb}{0.000000,0.000000,0.000000}%
\pgfsetstrokecolor{currentstroke}%
\pgfsetdash{}{0pt}%
\pgfsys@defobject{currentmarker}{\pgfqpoint{0.000000in}{-0.027778in}}{\pgfqpoint{0.000000in}{0.000000in}}{%
\pgfpathmoveto{\pgfqpoint{0.000000in}{0.000000in}}%
\pgfpathlineto{\pgfqpoint{0.000000in}{-0.027778in}}%
\pgfusepath{stroke,fill}%
}%
\begin{pgfscope}%
\pgfsys@transformshift{3.893934in}{0.417642in}%
\pgfsys@useobject{currentmarker}{}%
\end{pgfscope}%
\end{pgfscope}%
\begin{pgfscope}%
\pgfpathrectangle{\pgfqpoint{0.594525in}{0.417642in}}{\pgfqpoint{3.354228in}{2.055000in}}%
\pgfusepath{clip}%
\pgfsetrectcap%
\pgfsetroundjoin%
\pgfsetlinewidth{0.803000pt}%
\definecolor{currentstroke}{rgb}{0.850000,0.850000,0.850000}%
\pgfsetstrokecolor{currentstroke}%
\pgfsetdash{}{0pt}%
\pgfpathmoveto{\pgfqpoint{3.917514in}{0.417642in}}%
\pgfpathlineto{\pgfqpoint{3.917514in}{2.472642in}}%
\pgfusepath{stroke}%
\end{pgfscope}%
\begin{pgfscope}%
\pgfsetbuttcap%
\pgfsetroundjoin%
\definecolor{currentfill}{rgb}{0.000000,0.000000,0.000000}%
\pgfsetfillcolor{currentfill}%
\pgfsetlinewidth{0.602250pt}%
\definecolor{currentstroke}{rgb}{0.000000,0.000000,0.000000}%
\pgfsetstrokecolor{currentstroke}%
\pgfsetdash{}{0pt}%
\pgfsys@defobject{currentmarker}{\pgfqpoint{0.000000in}{-0.027778in}}{\pgfqpoint{0.000000in}{0.000000in}}{%
\pgfpathmoveto{\pgfqpoint{0.000000in}{0.000000in}}%
\pgfpathlineto{\pgfqpoint{0.000000in}{-0.027778in}}%
\pgfusepath{stroke,fill}%
}%
\begin{pgfscope}%
\pgfsys@transformshift{3.917514in}{0.417642in}%
\pgfsys@useobject{currentmarker}{}%
\end{pgfscope}%
\end{pgfscope}%
\begin{pgfscope}%
\definecolor{textcolor}{rgb}{0.000000,0.000000,0.000000}%
\pgfsetstrokecolor{textcolor}%
\pgfsetfillcolor{textcolor}%
\pgftext[x=2.271639in,y=0.165003in,,top]{\color{textcolor}{\rmfamily\fontsize{10.000000}{12.000000}\selectfont\catcode`\^=\active\def^{\ifmmode\sp\else\^{}\fi}\catcode`\%=\active\def%{\%}Frequency in $\unit{\Hz}$}}%
\end{pgfscope}%
\begin{pgfscope}%
\pgfpathrectangle{\pgfqpoint{0.594525in}{0.417642in}}{\pgfqpoint{3.354228in}{2.055000in}}%
\pgfusepath{clip}%
\pgfsetrectcap%
\pgfsetroundjoin%
\pgfsetlinewidth{0.803000pt}%
\definecolor{currentstroke}{rgb}{0.450000,0.450000,0.450000}%
\pgfsetstrokecolor{currentstroke}%
\pgfsetdash{}{0pt}%
\pgfpathmoveto{\pgfqpoint{0.594525in}{0.627861in}}%
\pgfpathlineto{\pgfqpoint{3.948753in}{0.627861in}}%
\pgfusepath{stroke}%
\end{pgfscope}%
\begin{pgfscope}%
\pgfsetbuttcap%
\pgfsetroundjoin%
\definecolor{currentfill}{rgb}{0.000000,0.000000,0.000000}%
\pgfsetfillcolor{currentfill}%
\pgfsetlinewidth{0.803000pt}%
\definecolor{currentstroke}{rgb}{0.000000,0.000000,0.000000}%
\pgfsetstrokecolor{currentstroke}%
\pgfsetdash{}{0pt}%
\pgfsys@defobject{currentmarker}{\pgfqpoint{-0.048611in}{0.000000in}}{\pgfqpoint{-0.000000in}{0.000000in}}{%
\pgfpathmoveto{\pgfqpoint{-0.000000in}{0.000000in}}%
\pgfpathlineto{\pgfqpoint{-0.048611in}{0.000000in}}%
\pgfusepath{stroke,fill}%
}%
\begin{pgfscope}%
\pgfsys@transformshift{0.594525in}{0.627861in}%
\pgfsys@useobject{currentmarker}{}%
\end{pgfscope}%
\end{pgfscope}%
\begin{pgfscope}%
\definecolor{textcolor}{rgb}{0.000000,0.000000,0.000000}%
\pgfsetstrokecolor{textcolor}%
\pgfsetfillcolor{textcolor}%
\pgftext[x=0.241129in, y=0.588708in, left, base]{\color{textcolor}{\rmfamily\fontsize{8.000000}{9.600000}\selectfont\catcode`\^=\active\def^{\ifmmode\sp\else\^{}\fi}\catcode`\%=\active\def%{\%}$\mathdefault{10^{-3}}$}}%
\end{pgfscope}%
\begin{pgfscope}%
\pgfpathrectangle{\pgfqpoint{0.594525in}{0.417642in}}{\pgfqpoint{3.354228in}{2.055000in}}%
\pgfusepath{clip}%
\pgfsetrectcap%
\pgfsetroundjoin%
\pgfsetlinewidth{0.803000pt}%
\definecolor{currentstroke}{rgb}{0.450000,0.450000,0.450000}%
\pgfsetstrokecolor{currentstroke}%
\pgfsetdash{}{0pt}%
\pgfpathmoveto{\pgfqpoint{0.594525in}{0.905564in}}%
\pgfpathlineto{\pgfqpoint{3.948753in}{0.905564in}}%
\pgfusepath{stroke}%
\end{pgfscope}%
\begin{pgfscope}%
\pgfsetbuttcap%
\pgfsetroundjoin%
\definecolor{currentfill}{rgb}{0.000000,0.000000,0.000000}%
\pgfsetfillcolor{currentfill}%
\pgfsetlinewidth{0.803000pt}%
\definecolor{currentstroke}{rgb}{0.000000,0.000000,0.000000}%
\pgfsetstrokecolor{currentstroke}%
\pgfsetdash{}{0pt}%
\pgfsys@defobject{currentmarker}{\pgfqpoint{-0.048611in}{0.000000in}}{\pgfqpoint{-0.000000in}{0.000000in}}{%
\pgfpathmoveto{\pgfqpoint{-0.000000in}{0.000000in}}%
\pgfpathlineto{\pgfqpoint{-0.048611in}{0.000000in}}%
\pgfusepath{stroke,fill}%
}%
\begin{pgfscope}%
\pgfsys@transformshift{0.594525in}{0.905564in}%
\pgfsys@useobject{currentmarker}{}%
\end{pgfscope}%
\end{pgfscope}%
\begin{pgfscope}%
\definecolor{textcolor}{rgb}{0.000000,0.000000,0.000000}%
\pgfsetstrokecolor{textcolor}%
\pgfsetfillcolor{textcolor}%
\pgftext[x=0.241129in, y=0.866411in, left, base]{\color{textcolor}{\rmfamily\fontsize{8.000000}{9.600000}\selectfont\catcode`\^=\active\def^{\ifmmode\sp\else\^{}\fi}\catcode`\%=\active\def%{\%}$\mathdefault{10^{-2}}$}}%
\end{pgfscope}%
\begin{pgfscope}%
\pgfpathrectangle{\pgfqpoint{0.594525in}{0.417642in}}{\pgfqpoint{3.354228in}{2.055000in}}%
\pgfusepath{clip}%
\pgfsetrectcap%
\pgfsetroundjoin%
\pgfsetlinewidth{0.803000pt}%
\definecolor{currentstroke}{rgb}{0.450000,0.450000,0.450000}%
\pgfsetstrokecolor{currentstroke}%
\pgfsetdash{}{0pt}%
\pgfpathmoveto{\pgfqpoint{0.594525in}{1.183267in}}%
\pgfpathlineto{\pgfqpoint{3.948753in}{1.183267in}}%
\pgfusepath{stroke}%
\end{pgfscope}%
\begin{pgfscope}%
\pgfsetbuttcap%
\pgfsetroundjoin%
\definecolor{currentfill}{rgb}{0.000000,0.000000,0.000000}%
\pgfsetfillcolor{currentfill}%
\pgfsetlinewidth{0.803000pt}%
\definecolor{currentstroke}{rgb}{0.000000,0.000000,0.000000}%
\pgfsetstrokecolor{currentstroke}%
\pgfsetdash{}{0pt}%
\pgfsys@defobject{currentmarker}{\pgfqpoint{-0.048611in}{0.000000in}}{\pgfqpoint{-0.000000in}{0.000000in}}{%
\pgfpathmoveto{\pgfqpoint{-0.000000in}{0.000000in}}%
\pgfpathlineto{\pgfqpoint{-0.048611in}{0.000000in}}%
\pgfusepath{stroke,fill}%
}%
\begin{pgfscope}%
\pgfsys@transformshift{0.594525in}{1.183267in}%
\pgfsys@useobject{currentmarker}{}%
\end{pgfscope}%
\end{pgfscope}%
\begin{pgfscope}%
\definecolor{textcolor}{rgb}{0.000000,0.000000,0.000000}%
\pgfsetstrokecolor{textcolor}%
\pgfsetfillcolor{textcolor}%
\pgftext[x=0.241129in, y=1.144114in, left, base]{\color{textcolor}{\rmfamily\fontsize{8.000000}{9.600000}\selectfont\catcode`\^=\active\def^{\ifmmode\sp\else\^{}\fi}\catcode`\%=\active\def%{\%}$\mathdefault{10^{-1}}$}}%
\end{pgfscope}%
\begin{pgfscope}%
\pgfpathrectangle{\pgfqpoint{0.594525in}{0.417642in}}{\pgfqpoint{3.354228in}{2.055000in}}%
\pgfusepath{clip}%
\pgfsetrectcap%
\pgfsetroundjoin%
\pgfsetlinewidth{0.803000pt}%
\definecolor{currentstroke}{rgb}{0.450000,0.450000,0.450000}%
\pgfsetstrokecolor{currentstroke}%
\pgfsetdash{}{0pt}%
\pgfpathmoveto{\pgfqpoint{0.594525in}{1.460970in}}%
\pgfpathlineto{\pgfqpoint{3.948753in}{1.460970in}}%
\pgfusepath{stroke}%
\end{pgfscope}%
\begin{pgfscope}%
\pgfsetbuttcap%
\pgfsetroundjoin%
\definecolor{currentfill}{rgb}{0.000000,0.000000,0.000000}%
\pgfsetfillcolor{currentfill}%
\pgfsetlinewidth{0.803000pt}%
\definecolor{currentstroke}{rgb}{0.000000,0.000000,0.000000}%
\pgfsetstrokecolor{currentstroke}%
\pgfsetdash{}{0pt}%
\pgfsys@defobject{currentmarker}{\pgfqpoint{-0.048611in}{0.000000in}}{\pgfqpoint{-0.000000in}{0.000000in}}{%
\pgfpathmoveto{\pgfqpoint{-0.000000in}{0.000000in}}%
\pgfpathlineto{\pgfqpoint{-0.048611in}{0.000000in}}%
\pgfusepath{stroke,fill}%
}%
\begin{pgfscope}%
\pgfsys@transformshift{0.594525in}{1.460970in}%
\pgfsys@useobject{currentmarker}{}%
\end{pgfscope}%
\end{pgfscope}%
\begin{pgfscope}%
\definecolor{textcolor}{rgb}{0.000000,0.000000,0.000000}%
\pgfsetstrokecolor{textcolor}%
\pgfsetfillcolor{textcolor}%
\pgftext[x=0.321376in, y=1.421817in, left, base]{\color{textcolor}{\rmfamily\fontsize{8.000000}{9.600000}\selectfont\catcode`\^=\active\def^{\ifmmode\sp\else\^{}\fi}\catcode`\%=\active\def%{\%}$\mathdefault{10^{0}}$}}%
\end{pgfscope}%
\begin{pgfscope}%
\pgfpathrectangle{\pgfqpoint{0.594525in}{0.417642in}}{\pgfqpoint{3.354228in}{2.055000in}}%
\pgfusepath{clip}%
\pgfsetrectcap%
\pgfsetroundjoin%
\pgfsetlinewidth{0.803000pt}%
\definecolor{currentstroke}{rgb}{0.450000,0.450000,0.450000}%
\pgfsetstrokecolor{currentstroke}%
\pgfsetdash{}{0pt}%
\pgfpathmoveto{\pgfqpoint{0.594525in}{1.738673in}}%
\pgfpathlineto{\pgfqpoint{3.948753in}{1.738673in}}%
\pgfusepath{stroke}%
\end{pgfscope}%
\begin{pgfscope}%
\pgfsetbuttcap%
\pgfsetroundjoin%
\definecolor{currentfill}{rgb}{0.000000,0.000000,0.000000}%
\pgfsetfillcolor{currentfill}%
\pgfsetlinewidth{0.803000pt}%
\definecolor{currentstroke}{rgb}{0.000000,0.000000,0.000000}%
\pgfsetstrokecolor{currentstroke}%
\pgfsetdash{}{0pt}%
\pgfsys@defobject{currentmarker}{\pgfqpoint{-0.048611in}{0.000000in}}{\pgfqpoint{-0.000000in}{0.000000in}}{%
\pgfpathmoveto{\pgfqpoint{-0.000000in}{0.000000in}}%
\pgfpathlineto{\pgfqpoint{-0.048611in}{0.000000in}}%
\pgfusepath{stroke,fill}%
}%
\begin{pgfscope}%
\pgfsys@transformshift{0.594525in}{1.738673in}%
\pgfsys@useobject{currentmarker}{}%
\end{pgfscope}%
\end{pgfscope}%
\begin{pgfscope}%
\definecolor{textcolor}{rgb}{0.000000,0.000000,0.000000}%
\pgfsetstrokecolor{textcolor}%
\pgfsetfillcolor{textcolor}%
\pgftext[x=0.321376in, y=1.699520in, left, base]{\color{textcolor}{\rmfamily\fontsize{8.000000}{9.600000}\selectfont\catcode`\^=\active\def^{\ifmmode\sp\else\^{}\fi}\catcode`\%=\active\def%{\%}$\mathdefault{10^{1}}$}}%
\end{pgfscope}%
\begin{pgfscope}%
\pgfpathrectangle{\pgfqpoint{0.594525in}{0.417642in}}{\pgfqpoint{3.354228in}{2.055000in}}%
\pgfusepath{clip}%
\pgfsetrectcap%
\pgfsetroundjoin%
\pgfsetlinewidth{0.803000pt}%
\definecolor{currentstroke}{rgb}{0.450000,0.450000,0.450000}%
\pgfsetstrokecolor{currentstroke}%
\pgfsetdash{}{0pt}%
\pgfpathmoveto{\pgfqpoint{0.594525in}{2.016376in}}%
\pgfpathlineto{\pgfqpoint{3.948753in}{2.016376in}}%
\pgfusepath{stroke}%
\end{pgfscope}%
\begin{pgfscope}%
\pgfsetbuttcap%
\pgfsetroundjoin%
\definecolor{currentfill}{rgb}{0.000000,0.000000,0.000000}%
\pgfsetfillcolor{currentfill}%
\pgfsetlinewidth{0.803000pt}%
\definecolor{currentstroke}{rgb}{0.000000,0.000000,0.000000}%
\pgfsetstrokecolor{currentstroke}%
\pgfsetdash{}{0pt}%
\pgfsys@defobject{currentmarker}{\pgfqpoint{-0.048611in}{0.000000in}}{\pgfqpoint{-0.000000in}{0.000000in}}{%
\pgfpathmoveto{\pgfqpoint{-0.000000in}{0.000000in}}%
\pgfpathlineto{\pgfqpoint{-0.048611in}{0.000000in}}%
\pgfusepath{stroke,fill}%
}%
\begin{pgfscope}%
\pgfsys@transformshift{0.594525in}{2.016376in}%
\pgfsys@useobject{currentmarker}{}%
\end{pgfscope}%
\end{pgfscope}%
\begin{pgfscope}%
\definecolor{textcolor}{rgb}{0.000000,0.000000,0.000000}%
\pgfsetstrokecolor{textcolor}%
\pgfsetfillcolor{textcolor}%
\pgftext[x=0.321376in, y=1.977223in, left, base]{\color{textcolor}{\rmfamily\fontsize{8.000000}{9.600000}\selectfont\catcode`\^=\active\def^{\ifmmode\sp\else\^{}\fi}\catcode`\%=\active\def%{\%}$\mathdefault{10^{2}}$}}%
\end{pgfscope}%
\begin{pgfscope}%
\pgfpathrectangle{\pgfqpoint{0.594525in}{0.417642in}}{\pgfqpoint{3.354228in}{2.055000in}}%
\pgfusepath{clip}%
\pgfsetrectcap%
\pgfsetroundjoin%
\pgfsetlinewidth{0.803000pt}%
\definecolor{currentstroke}{rgb}{0.450000,0.450000,0.450000}%
\pgfsetstrokecolor{currentstroke}%
\pgfsetdash{}{0pt}%
\pgfpathmoveto{\pgfqpoint{0.594525in}{2.294079in}}%
\pgfpathlineto{\pgfqpoint{3.948753in}{2.294079in}}%
\pgfusepath{stroke}%
\end{pgfscope}%
\begin{pgfscope}%
\pgfsetbuttcap%
\pgfsetroundjoin%
\definecolor{currentfill}{rgb}{0.000000,0.000000,0.000000}%
\pgfsetfillcolor{currentfill}%
\pgfsetlinewidth{0.803000pt}%
\definecolor{currentstroke}{rgb}{0.000000,0.000000,0.000000}%
\pgfsetstrokecolor{currentstroke}%
\pgfsetdash{}{0pt}%
\pgfsys@defobject{currentmarker}{\pgfqpoint{-0.048611in}{0.000000in}}{\pgfqpoint{-0.000000in}{0.000000in}}{%
\pgfpathmoveto{\pgfqpoint{-0.000000in}{0.000000in}}%
\pgfpathlineto{\pgfqpoint{-0.048611in}{0.000000in}}%
\pgfusepath{stroke,fill}%
}%
\begin{pgfscope}%
\pgfsys@transformshift{0.594525in}{2.294079in}%
\pgfsys@useobject{currentmarker}{}%
\end{pgfscope}%
\end{pgfscope}%
\begin{pgfscope}%
\definecolor{textcolor}{rgb}{0.000000,0.000000,0.000000}%
\pgfsetstrokecolor{textcolor}%
\pgfsetfillcolor{textcolor}%
\pgftext[x=0.321376in, y=2.254926in, left, base]{\color{textcolor}{\rmfamily\fontsize{8.000000}{9.600000}\selectfont\catcode`\^=\active\def^{\ifmmode\sp\else\^{}\fi}\catcode`\%=\active\def%{\%}$\mathdefault{10^{3}}$}}%
\end{pgfscope}%
\begin{pgfscope}%
\pgfpathrectangle{\pgfqpoint{0.594525in}{0.417642in}}{\pgfqpoint{3.354228in}{2.055000in}}%
\pgfusepath{clip}%
\pgfsetrectcap%
\pgfsetroundjoin%
\pgfsetlinewidth{0.803000pt}%
\definecolor{currentstroke}{rgb}{0.850000,0.850000,0.850000}%
\pgfsetstrokecolor{currentstroke}%
\pgfsetdash{}{0pt}%
\pgfpathmoveto{\pgfqpoint{0.594525in}{0.433755in}}%
\pgfpathlineto{\pgfqpoint{3.948753in}{0.433755in}}%
\pgfusepath{stroke}%
\end{pgfscope}%
\begin{pgfscope}%
\pgfsetbuttcap%
\pgfsetroundjoin%
\definecolor{currentfill}{rgb}{0.000000,0.000000,0.000000}%
\pgfsetfillcolor{currentfill}%
\pgfsetlinewidth{0.602250pt}%
\definecolor{currentstroke}{rgb}{0.000000,0.000000,0.000000}%
\pgfsetstrokecolor{currentstroke}%
\pgfsetdash{}{0pt}%
\pgfsys@defobject{currentmarker}{\pgfqpoint{-0.027778in}{0.000000in}}{\pgfqpoint{-0.000000in}{0.000000in}}{%
\pgfpathmoveto{\pgfqpoint{-0.000000in}{0.000000in}}%
\pgfpathlineto{\pgfqpoint{-0.027778in}{0.000000in}}%
\pgfusepath{stroke,fill}%
}%
\begin{pgfscope}%
\pgfsys@transformshift{0.594525in}{0.433755in}%
\pgfsys@useobject{currentmarker}{}%
\end{pgfscope}%
\end{pgfscope}%
\begin{pgfscope}%
\pgfpathrectangle{\pgfqpoint{0.594525in}{0.417642in}}{\pgfqpoint{3.354228in}{2.055000in}}%
\pgfusepath{clip}%
\pgfsetrectcap%
\pgfsetroundjoin%
\pgfsetlinewidth{0.803000pt}%
\definecolor{currentstroke}{rgb}{0.850000,0.850000,0.850000}%
\pgfsetstrokecolor{currentstroke}%
\pgfsetdash{}{0pt}%
\pgfpathmoveto{\pgfqpoint{0.594525in}{0.482656in}}%
\pgfpathlineto{\pgfqpoint{3.948753in}{0.482656in}}%
\pgfusepath{stroke}%
\end{pgfscope}%
\begin{pgfscope}%
\pgfsetbuttcap%
\pgfsetroundjoin%
\definecolor{currentfill}{rgb}{0.000000,0.000000,0.000000}%
\pgfsetfillcolor{currentfill}%
\pgfsetlinewidth{0.602250pt}%
\definecolor{currentstroke}{rgb}{0.000000,0.000000,0.000000}%
\pgfsetstrokecolor{currentstroke}%
\pgfsetdash{}{0pt}%
\pgfsys@defobject{currentmarker}{\pgfqpoint{-0.027778in}{0.000000in}}{\pgfqpoint{-0.000000in}{0.000000in}}{%
\pgfpathmoveto{\pgfqpoint{-0.000000in}{0.000000in}}%
\pgfpathlineto{\pgfqpoint{-0.027778in}{0.000000in}}%
\pgfusepath{stroke,fill}%
}%
\begin{pgfscope}%
\pgfsys@transformshift{0.594525in}{0.482656in}%
\pgfsys@useobject{currentmarker}{}%
\end{pgfscope}%
\end{pgfscope}%
\begin{pgfscope}%
\pgfpathrectangle{\pgfqpoint{0.594525in}{0.417642in}}{\pgfqpoint{3.354228in}{2.055000in}}%
\pgfusepath{clip}%
\pgfsetrectcap%
\pgfsetroundjoin%
\pgfsetlinewidth{0.803000pt}%
\definecolor{currentstroke}{rgb}{0.850000,0.850000,0.850000}%
\pgfsetstrokecolor{currentstroke}%
\pgfsetdash{}{0pt}%
\pgfpathmoveto{\pgfqpoint{0.594525in}{0.517352in}}%
\pgfpathlineto{\pgfqpoint{3.948753in}{0.517352in}}%
\pgfusepath{stroke}%
\end{pgfscope}%
\begin{pgfscope}%
\pgfsetbuttcap%
\pgfsetroundjoin%
\definecolor{currentfill}{rgb}{0.000000,0.000000,0.000000}%
\pgfsetfillcolor{currentfill}%
\pgfsetlinewidth{0.602250pt}%
\definecolor{currentstroke}{rgb}{0.000000,0.000000,0.000000}%
\pgfsetstrokecolor{currentstroke}%
\pgfsetdash{}{0pt}%
\pgfsys@defobject{currentmarker}{\pgfqpoint{-0.027778in}{0.000000in}}{\pgfqpoint{-0.000000in}{0.000000in}}{%
\pgfpathmoveto{\pgfqpoint{-0.000000in}{0.000000in}}%
\pgfpathlineto{\pgfqpoint{-0.027778in}{0.000000in}}%
\pgfusepath{stroke,fill}%
}%
\begin{pgfscope}%
\pgfsys@transformshift{0.594525in}{0.517352in}%
\pgfsys@useobject{currentmarker}{}%
\end{pgfscope}%
\end{pgfscope}%
\begin{pgfscope}%
\pgfpathrectangle{\pgfqpoint{0.594525in}{0.417642in}}{\pgfqpoint{3.354228in}{2.055000in}}%
\pgfusepath{clip}%
\pgfsetrectcap%
\pgfsetroundjoin%
\pgfsetlinewidth{0.803000pt}%
\definecolor{currentstroke}{rgb}{0.850000,0.850000,0.850000}%
\pgfsetstrokecolor{currentstroke}%
\pgfsetdash{}{0pt}%
\pgfpathmoveto{\pgfqpoint{0.594525in}{0.544264in}}%
\pgfpathlineto{\pgfqpoint{3.948753in}{0.544264in}}%
\pgfusepath{stroke}%
\end{pgfscope}%
\begin{pgfscope}%
\pgfsetbuttcap%
\pgfsetroundjoin%
\definecolor{currentfill}{rgb}{0.000000,0.000000,0.000000}%
\pgfsetfillcolor{currentfill}%
\pgfsetlinewidth{0.602250pt}%
\definecolor{currentstroke}{rgb}{0.000000,0.000000,0.000000}%
\pgfsetstrokecolor{currentstroke}%
\pgfsetdash{}{0pt}%
\pgfsys@defobject{currentmarker}{\pgfqpoint{-0.027778in}{0.000000in}}{\pgfqpoint{-0.000000in}{0.000000in}}{%
\pgfpathmoveto{\pgfqpoint{-0.000000in}{0.000000in}}%
\pgfpathlineto{\pgfqpoint{-0.027778in}{0.000000in}}%
\pgfusepath{stroke,fill}%
}%
\begin{pgfscope}%
\pgfsys@transformshift{0.594525in}{0.544264in}%
\pgfsys@useobject{currentmarker}{}%
\end{pgfscope}%
\end{pgfscope}%
\begin{pgfscope}%
\pgfpathrectangle{\pgfqpoint{0.594525in}{0.417642in}}{\pgfqpoint{3.354228in}{2.055000in}}%
\pgfusepath{clip}%
\pgfsetrectcap%
\pgfsetroundjoin%
\pgfsetlinewidth{0.803000pt}%
\definecolor{currentstroke}{rgb}{0.850000,0.850000,0.850000}%
\pgfsetstrokecolor{currentstroke}%
\pgfsetdash{}{0pt}%
\pgfpathmoveto{\pgfqpoint{0.594525in}{0.566253in}}%
\pgfpathlineto{\pgfqpoint{3.948753in}{0.566253in}}%
\pgfusepath{stroke}%
\end{pgfscope}%
\begin{pgfscope}%
\pgfsetbuttcap%
\pgfsetroundjoin%
\definecolor{currentfill}{rgb}{0.000000,0.000000,0.000000}%
\pgfsetfillcolor{currentfill}%
\pgfsetlinewidth{0.602250pt}%
\definecolor{currentstroke}{rgb}{0.000000,0.000000,0.000000}%
\pgfsetstrokecolor{currentstroke}%
\pgfsetdash{}{0pt}%
\pgfsys@defobject{currentmarker}{\pgfqpoint{-0.027778in}{0.000000in}}{\pgfqpoint{-0.000000in}{0.000000in}}{%
\pgfpathmoveto{\pgfqpoint{-0.000000in}{0.000000in}}%
\pgfpathlineto{\pgfqpoint{-0.027778in}{0.000000in}}%
\pgfusepath{stroke,fill}%
}%
\begin{pgfscope}%
\pgfsys@transformshift{0.594525in}{0.566253in}%
\pgfsys@useobject{currentmarker}{}%
\end{pgfscope}%
\end{pgfscope}%
\begin{pgfscope}%
\pgfpathrectangle{\pgfqpoint{0.594525in}{0.417642in}}{\pgfqpoint{3.354228in}{2.055000in}}%
\pgfusepath{clip}%
\pgfsetrectcap%
\pgfsetroundjoin%
\pgfsetlinewidth{0.803000pt}%
\definecolor{currentstroke}{rgb}{0.850000,0.850000,0.850000}%
\pgfsetstrokecolor{currentstroke}%
\pgfsetdash{}{0pt}%
\pgfpathmoveto{\pgfqpoint{0.594525in}{0.584844in}}%
\pgfpathlineto{\pgfqpoint{3.948753in}{0.584844in}}%
\pgfusepath{stroke}%
\end{pgfscope}%
\begin{pgfscope}%
\pgfsetbuttcap%
\pgfsetroundjoin%
\definecolor{currentfill}{rgb}{0.000000,0.000000,0.000000}%
\pgfsetfillcolor{currentfill}%
\pgfsetlinewidth{0.602250pt}%
\definecolor{currentstroke}{rgb}{0.000000,0.000000,0.000000}%
\pgfsetstrokecolor{currentstroke}%
\pgfsetdash{}{0pt}%
\pgfsys@defobject{currentmarker}{\pgfqpoint{-0.027778in}{0.000000in}}{\pgfqpoint{-0.000000in}{0.000000in}}{%
\pgfpathmoveto{\pgfqpoint{-0.000000in}{0.000000in}}%
\pgfpathlineto{\pgfqpoint{-0.027778in}{0.000000in}}%
\pgfusepath{stroke,fill}%
}%
\begin{pgfscope}%
\pgfsys@transformshift{0.594525in}{0.584844in}%
\pgfsys@useobject{currentmarker}{}%
\end{pgfscope}%
\end{pgfscope}%
\begin{pgfscope}%
\pgfpathrectangle{\pgfqpoint{0.594525in}{0.417642in}}{\pgfqpoint{3.354228in}{2.055000in}}%
\pgfusepath{clip}%
\pgfsetrectcap%
\pgfsetroundjoin%
\pgfsetlinewidth{0.803000pt}%
\definecolor{currentstroke}{rgb}{0.850000,0.850000,0.850000}%
\pgfsetstrokecolor{currentstroke}%
\pgfsetdash{}{0pt}%
\pgfpathmoveto{\pgfqpoint{0.594525in}{0.600949in}}%
\pgfpathlineto{\pgfqpoint{3.948753in}{0.600949in}}%
\pgfusepath{stroke}%
\end{pgfscope}%
\begin{pgfscope}%
\pgfsetbuttcap%
\pgfsetroundjoin%
\definecolor{currentfill}{rgb}{0.000000,0.000000,0.000000}%
\pgfsetfillcolor{currentfill}%
\pgfsetlinewidth{0.602250pt}%
\definecolor{currentstroke}{rgb}{0.000000,0.000000,0.000000}%
\pgfsetstrokecolor{currentstroke}%
\pgfsetdash{}{0pt}%
\pgfsys@defobject{currentmarker}{\pgfqpoint{-0.027778in}{0.000000in}}{\pgfqpoint{-0.000000in}{0.000000in}}{%
\pgfpathmoveto{\pgfqpoint{-0.000000in}{0.000000in}}%
\pgfpathlineto{\pgfqpoint{-0.027778in}{0.000000in}}%
\pgfusepath{stroke,fill}%
}%
\begin{pgfscope}%
\pgfsys@transformshift{0.594525in}{0.600949in}%
\pgfsys@useobject{currentmarker}{}%
\end{pgfscope}%
\end{pgfscope}%
\begin{pgfscope}%
\pgfpathrectangle{\pgfqpoint{0.594525in}{0.417642in}}{\pgfqpoint{3.354228in}{2.055000in}}%
\pgfusepath{clip}%
\pgfsetrectcap%
\pgfsetroundjoin%
\pgfsetlinewidth{0.803000pt}%
\definecolor{currentstroke}{rgb}{0.850000,0.850000,0.850000}%
\pgfsetstrokecolor{currentstroke}%
\pgfsetdash{}{0pt}%
\pgfpathmoveto{\pgfqpoint{0.594525in}{0.615154in}}%
\pgfpathlineto{\pgfqpoint{3.948753in}{0.615154in}}%
\pgfusepath{stroke}%
\end{pgfscope}%
\begin{pgfscope}%
\pgfsetbuttcap%
\pgfsetroundjoin%
\definecolor{currentfill}{rgb}{0.000000,0.000000,0.000000}%
\pgfsetfillcolor{currentfill}%
\pgfsetlinewidth{0.602250pt}%
\definecolor{currentstroke}{rgb}{0.000000,0.000000,0.000000}%
\pgfsetstrokecolor{currentstroke}%
\pgfsetdash{}{0pt}%
\pgfsys@defobject{currentmarker}{\pgfqpoint{-0.027778in}{0.000000in}}{\pgfqpoint{-0.000000in}{0.000000in}}{%
\pgfpathmoveto{\pgfqpoint{-0.000000in}{0.000000in}}%
\pgfpathlineto{\pgfqpoint{-0.027778in}{0.000000in}}%
\pgfusepath{stroke,fill}%
}%
\begin{pgfscope}%
\pgfsys@transformshift{0.594525in}{0.615154in}%
\pgfsys@useobject{currentmarker}{}%
\end{pgfscope}%
\end{pgfscope}%
\begin{pgfscope}%
\pgfpathrectangle{\pgfqpoint{0.594525in}{0.417642in}}{\pgfqpoint{3.354228in}{2.055000in}}%
\pgfusepath{clip}%
\pgfsetrectcap%
\pgfsetroundjoin%
\pgfsetlinewidth{0.803000pt}%
\definecolor{currentstroke}{rgb}{0.850000,0.850000,0.850000}%
\pgfsetstrokecolor{currentstroke}%
\pgfsetdash{}{0pt}%
\pgfpathmoveto{\pgfqpoint{0.594525in}{0.711458in}}%
\pgfpathlineto{\pgfqpoint{3.948753in}{0.711458in}}%
\pgfusepath{stroke}%
\end{pgfscope}%
\begin{pgfscope}%
\pgfsetbuttcap%
\pgfsetroundjoin%
\definecolor{currentfill}{rgb}{0.000000,0.000000,0.000000}%
\pgfsetfillcolor{currentfill}%
\pgfsetlinewidth{0.602250pt}%
\definecolor{currentstroke}{rgb}{0.000000,0.000000,0.000000}%
\pgfsetstrokecolor{currentstroke}%
\pgfsetdash{}{0pt}%
\pgfsys@defobject{currentmarker}{\pgfqpoint{-0.027778in}{0.000000in}}{\pgfqpoint{-0.000000in}{0.000000in}}{%
\pgfpathmoveto{\pgfqpoint{-0.000000in}{0.000000in}}%
\pgfpathlineto{\pgfqpoint{-0.027778in}{0.000000in}}%
\pgfusepath{stroke,fill}%
}%
\begin{pgfscope}%
\pgfsys@transformshift{0.594525in}{0.711458in}%
\pgfsys@useobject{currentmarker}{}%
\end{pgfscope}%
\end{pgfscope}%
\begin{pgfscope}%
\pgfpathrectangle{\pgfqpoint{0.594525in}{0.417642in}}{\pgfqpoint{3.354228in}{2.055000in}}%
\pgfusepath{clip}%
\pgfsetrectcap%
\pgfsetroundjoin%
\pgfsetlinewidth{0.803000pt}%
\definecolor{currentstroke}{rgb}{0.850000,0.850000,0.850000}%
\pgfsetstrokecolor{currentstroke}%
\pgfsetdash{}{0pt}%
\pgfpathmoveto{\pgfqpoint{0.594525in}{0.760359in}}%
\pgfpathlineto{\pgfqpoint{3.948753in}{0.760359in}}%
\pgfusepath{stroke}%
\end{pgfscope}%
\begin{pgfscope}%
\pgfsetbuttcap%
\pgfsetroundjoin%
\definecolor{currentfill}{rgb}{0.000000,0.000000,0.000000}%
\pgfsetfillcolor{currentfill}%
\pgfsetlinewidth{0.602250pt}%
\definecolor{currentstroke}{rgb}{0.000000,0.000000,0.000000}%
\pgfsetstrokecolor{currentstroke}%
\pgfsetdash{}{0pt}%
\pgfsys@defobject{currentmarker}{\pgfqpoint{-0.027778in}{0.000000in}}{\pgfqpoint{-0.000000in}{0.000000in}}{%
\pgfpathmoveto{\pgfqpoint{-0.000000in}{0.000000in}}%
\pgfpathlineto{\pgfqpoint{-0.027778in}{0.000000in}}%
\pgfusepath{stroke,fill}%
}%
\begin{pgfscope}%
\pgfsys@transformshift{0.594525in}{0.760359in}%
\pgfsys@useobject{currentmarker}{}%
\end{pgfscope}%
\end{pgfscope}%
\begin{pgfscope}%
\pgfpathrectangle{\pgfqpoint{0.594525in}{0.417642in}}{\pgfqpoint{3.354228in}{2.055000in}}%
\pgfusepath{clip}%
\pgfsetrectcap%
\pgfsetroundjoin%
\pgfsetlinewidth{0.803000pt}%
\definecolor{currentstroke}{rgb}{0.850000,0.850000,0.850000}%
\pgfsetstrokecolor{currentstroke}%
\pgfsetdash{}{0pt}%
\pgfpathmoveto{\pgfqpoint{0.594525in}{0.795055in}}%
\pgfpathlineto{\pgfqpoint{3.948753in}{0.795055in}}%
\pgfusepath{stroke}%
\end{pgfscope}%
\begin{pgfscope}%
\pgfsetbuttcap%
\pgfsetroundjoin%
\definecolor{currentfill}{rgb}{0.000000,0.000000,0.000000}%
\pgfsetfillcolor{currentfill}%
\pgfsetlinewidth{0.602250pt}%
\definecolor{currentstroke}{rgb}{0.000000,0.000000,0.000000}%
\pgfsetstrokecolor{currentstroke}%
\pgfsetdash{}{0pt}%
\pgfsys@defobject{currentmarker}{\pgfqpoint{-0.027778in}{0.000000in}}{\pgfqpoint{-0.000000in}{0.000000in}}{%
\pgfpathmoveto{\pgfqpoint{-0.000000in}{0.000000in}}%
\pgfpathlineto{\pgfqpoint{-0.027778in}{0.000000in}}%
\pgfusepath{stroke,fill}%
}%
\begin{pgfscope}%
\pgfsys@transformshift{0.594525in}{0.795055in}%
\pgfsys@useobject{currentmarker}{}%
\end{pgfscope}%
\end{pgfscope}%
\begin{pgfscope}%
\pgfpathrectangle{\pgfqpoint{0.594525in}{0.417642in}}{\pgfqpoint{3.354228in}{2.055000in}}%
\pgfusepath{clip}%
\pgfsetrectcap%
\pgfsetroundjoin%
\pgfsetlinewidth{0.803000pt}%
\definecolor{currentstroke}{rgb}{0.850000,0.850000,0.850000}%
\pgfsetstrokecolor{currentstroke}%
\pgfsetdash{}{0pt}%
\pgfpathmoveto{\pgfqpoint{0.594525in}{0.821967in}}%
\pgfpathlineto{\pgfqpoint{3.948753in}{0.821967in}}%
\pgfusepath{stroke}%
\end{pgfscope}%
\begin{pgfscope}%
\pgfsetbuttcap%
\pgfsetroundjoin%
\definecolor{currentfill}{rgb}{0.000000,0.000000,0.000000}%
\pgfsetfillcolor{currentfill}%
\pgfsetlinewidth{0.602250pt}%
\definecolor{currentstroke}{rgb}{0.000000,0.000000,0.000000}%
\pgfsetstrokecolor{currentstroke}%
\pgfsetdash{}{0pt}%
\pgfsys@defobject{currentmarker}{\pgfqpoint{-0.027778in}{0.000000in}}{\pgfqpoint{-0.000000in}{0.000000in}}{%
\pgfpathmoveto{\pgfqpoint{-0.000000in}{0.000000in}}%
\pgfpathlineto{\pgfqpoint{-0.027778in}{0.000000in}}%
\pgfusepath{stroke,fill}%
}%
\begin{pgfscope}%
\pgfsys@transformshift{0.594525in}{0.821967in}%
\pgfsys@useobject{currentmarker}{}%
\end{pgfscope}%
\end{pgfscope}%
\begin{pgfscope}%
\pgfpathrectangle{\pgfqpoint{0.594525in}{0.417642in}}{\pgfqpoint{3.354228in}{2.055000in}}%
\pgfusepath{clip}%
\pgfsetrectcap%
\pgfsetroundjoin%
\pgfsetlinewidth{0.803000pt}%
\definecolor{currentstroke}{rgb}{0.850000,0.850000,0.850000}%
\pgfsetstrokecolor{currentstroke}%
\pgfsetdash{}{0pt}%
\pgfpathmoveto{\pgfqpoint{0.594525in}{0.843956in}}%
\pgfpathlineto{\pgfqpoint{3.948753in}{0.843956in}}%
\pgfusepath{stroke}%
\end{pgfscope}%
\begin{pgfscope}%
\pgfsetbuttcap%
\pgfsetroundjoin%
\definecolor{currentfill}{rgb}{0.000000,0.000000,0.000000}%
\pgfsetfillcolor{currentfill}%
\pgfsetlinewidth{0.602250pt}%
\definecolor{currentstroke}{rgb}{0.000000,0.000000,0.000000}%
\pgfsetstrokecolor{currentstroke}%
\pgfsetdash{}{0pt}%
\pgfsys@defobject{currentmarker}{\pgfqpoint{-0.027778in}{0.000000in}}{\pgfqpoint{-0.000000in}{0.000000in}}{%
\pgfpathmoveto{\pgfqpoint{-0.000000in}{0.000000in}}%
\pgfpathlineto{\pgfqpoint{-0.027778in}{0.000000in}}%
\pgfusepath{stroke,fill}%
}%
\begin{pgfscope}%
\pgfsys@transformshift{0.594525in}{0.843956in}%
\pgfsys@useobject{currentmarker}{}%
\end{pgfscope}%
\end{pgfscope}%
\begin{pgfscope}%
\pgfpathrectangle{\pgfqpoint{0.594525in}{0.417642in}}{\pgfqpoint{3.354228in}{2.055000in}}%
\pgfusepath{clip}%
\pgfsetrectcap%
\pgfsetroundjoin%
\pgfsetlinewidth{0.803000pt}%
\definecolor{currentstroke}{rgb}{0.850000,0.850000,0.850000}%
\pgfsetstrokecolor{currentstroke}%
\pgfsetdash{}{0pt}%
\pgfpathmoveto{\pgfqpoint{0.594525in}{0.862547in}}%
\pgfpathlineto{\pgfqpoint{3.948753in}{0.862547in}}%
\pgfusepath{stroke}%
\end{pgfscope}%
\begin{pgfscope}%
\pgfsetbuttcap%
\pgfsetroundjoin%
\definecolor{currentfill}{rgb}{0.000000,0.000000,0.000000}%
\pgfsetfillcolor{currentfill}%
\pgfsetlinewidth{0.602250pt}%
\definecolor{currentstroke}{rgb}{0.000000,0.000000,0.000000}%
\pgfsetstrokecolor{currentstroke}%
\pgfsetdash{}{0pt}%
\pgfsys@defobject{currentmarker}{\pgfqpoint{-0.027778in}{0.000000in}}{\pgfqpoint{-0.000000in}{0.000000in}}{%
\pgfpathmoveto{\pgfqpoint{-0.000000in}{0.000000in}}%
\pgfpathlineto{\pgfqpoint{-0.027778in}{0.000000in}}%
\pgfusepath{stroke,fill}%
}%
\begin{pgfscope}%
\pgfsys@transformshift{0.594525in}{0.862547in}%
\pgfsys@useobject{currentmarker}{}%
\end{pgfscope}%
\end{pgfscope}%
\begin{pgfscope}%
\pgfpathrectangle{\pgfqpoint{0.594525in}{0.417642in}}{\pgfqpoint{3.354228in}{2.055000in}}%
\pgfusepath{clip}%
\pgfsetrectcap%
\pgfsetroundjoin%
\pgfsetlinewidth{0.803000pt}%
\definecolor{currentstroke}{rgb}{0.850000,0.850000,0.850000}%
\pgfsetstrokecolor{currentstroke}%
\pgfsetdash{}{0pt}%
\pgfpathmoveto{\pgfqpoint{0.594525in}{0.878652in}}%
\pgfpathlineto{\pgfqpoint{3.948753in}{0.878652in}}%
\pgfusepath{stroke}%
\end{pgfscope}%
\begin{pgfscope}%
\pgfsetbuttcap%
\pgfsetroundjoin%
\definecolor{currentfill}{rgb}{0.000000,0.000000,0.000000}%
\pgfsetfillcolor{currentfill}%
\pgfsetlinewidth{0.602250pt}%
\definecolor{currentstroke}{rgb}{0.000000,0.000000,0.000000}%
\pgfsetstrokecolor{currentstroke}%
\pgfsetdash{}{0pt}%
\pgfsys@defobject{currentmarker}{\pgfqpoint{-0.027778in}{0.000000in}}{\pgfqpoint{-0.000000in}{0.000000in}}{%
\pgfpathmoveto{\pgfqpoint{-0.000000in}{0.000000in}}%
\pgfpathlineto{\pgfqpoint{-0.027778in}{0.000000in}}%
\pgfusepath{stroke,fill}%
}%
\begin{pgfscope}%
\pgfsys@transformshift{0.594525in}{0.878652in}%
\pgfsys@useobject{currentmarker}{}%
\end{pgfscope}%
\end{pgfscope}%
\begin{pgfscope}%
\pgfpathrectangle{\pgfqpoint{0.594525in}{0.417642in}}{\pgfqpoint{3.354228in}{2.055000in}}%
\pgfusepath{clip}%
\pgfsetrectcap%
\pgfsetroundjoin%
\pgfsetlinewidth{0.803000pt}%
\definecolor{currentstroke}{rgb}{0.850000,0.850000,0.850000}%
\pgfsetstrokecolor{currentstroke}%
\pgfsetdash{}{0pt}%
\pgfpathmoveto{\pgfqpoint{0.594525in}{0.892857in}}%
\pgfpathlineto{\pgfqpoint{3.948753in}{0.892857in}}%
\pgfusepath{stroke}%
\end{pgfscope}%
\begin{pgfscope}%
\pgfsetbuttcap%
\pgfsetroundjoin%
\definecolor{currentfill}{rgb}{0.000000,0.000000,0.000000}%
\pgfsetfillcolor{currentfill}%
\pgfsetlinewidth{0.602250pt}%
\definecolor{currentstroke}{rgb}{0.000000,0.000000,0.000000}%
\pgfsetstrokecolor{currentstroke}%
\pgfsetdash{}{0pt}%
\pgfsys@defobject{currentmarker}{\pgfqpoint{-0.027778in}{0.000000in}}{\pgfqpoint{-0.000000in}{0.000000in}}{%
\pgfpathmoveto{\pgfqpoint{-0.000000in}{0.000000in}}%
\pgfpathlineto{\pgfqpoint{-0.027778in}{0.000000in}}%
\pgfusepath{stroke,fill}%
}%
\begin{pgfscope}%
\pgfsys@transformshift{0.594525in}{0.892857in}%
\pgfsys@useobject{currentmarker}{}%
\end{pgfscope}%
\end{pgfscope}%
\begin{pgfscope}%
\pgfpathrectangle{\pgfqpoint{0.594525in}{0.417642in}}{\pgfqpoint{3.354228in}{2.055000in}}%
\pgfusepath{clip}%
\pgfsetrectcap%
\pgfsetroundjoin%
\pgfsetlinewidth{0.803000pt}%
\definecolor{currentstroke}{rgb}{0.850000,0.850000,0.850000}%
\pgfsetstrokecolor{currentstroke}%
\pgfsetdash{}{0pt}%
\pgfpathmoveto{\pgfqpoint{0.594525in}{0.989161in}}%
\pgfpathlineto{\pgfqpoint{3.948753in}{0.989161in}}%
\pgfusepath{stroke}%
\end{pgfscope}%
\begin{pgfscope}%
\pgfsetbuttcap%
\pgfsetroundjoin%
\definecolor{currentfill}{rgb}{0.000000,0.000000,0.000000}%
\pgfsetfillcolor{currentfill}%
\pgfsetlinewidth{0.602250pt}%
\definecolor{currentstroke}{rgb}{0.000000,0.000000,0.000000}%
\pgfsetstrokecolor{currentstroke}%
\pgfsetdash{}{0pt}%
\pgfsys@defobject{currentmarker}{\pgfqpoint{-0.027778in}{0.000000in}}{\pgfqpoint{-0.000000in}{0.000000in}}{%
\pgfpathmoveto{\pgfqpoint{-0.000000in}{0.000000in}}%
\pgfpathlineto{\pgfqpoint{-0.027778in}{0.000000in}}%
\pgfusepath{stroke,fill}%
}%
\begin{pgfscope}%
\pgfsys@transformshift{0.594525in}{0.989161in}%
\pgfsys@useobject{currentmarker}{}%
\end{pgfscope}%
\end{pgfscope}%
\begin{pgfscope}%
\pgfpathrectangle{\pgfqpoint{0.594525in}{0.417642in}}{\pgfqpoint{3.354228in}{2.055000in}}%
\pgfusepath{clip}%
\pgfsetrectcap%
\pgfsetroundjoin%
\pgfsetlinewidth{0.803000pt}%
\definecolor{currentstroke}{rgb}{0.850000,0.850000,0.850000}%
\pgfsetstrokecolor{currentstroke}%
\pgfsetdash{}{0pt}%
\pgfpathmoveto{\pgfqpoint{0.594525in}{1.038062in}}%
\pgfpathlineto{\pgfqpoint{3.948753in}{1.038062in}}%
\pgfusepath{stroke}%
\end{pgfscope}%
\begin{pgfscope}%
\pgfsetbuttcap%
\pgfsetroundjoin%
\definecolor{currentfill}{rgb}{0.000000,0.000000,0.000000}%
\pgfsetfillcolor{currentfill}%
\pgfsetlinewidth{0.602250pt}%
\definecolor{currentstroke}{rgb}{0.000000,0.000000,0.000000}%
\pgfsetstrokecolor{currentstroke}%
\pgfsetdash{}{0pt}%
\pgfsys@defobject{currentmarker}{\pgfqpoint{-0.027778in}{0.000000in}}{\pgfqpoint{-0.000000in}{0.000000in}}{%
\pgfpathmoveto{\pgfqpoint{-0.000000in}{0.000000in}}%
\pgfpathlineto{\pgfqpoint{-0.027778in}{0.000000in}}%
\pgfusepath{stroke,fill}%
}%
\begin{pgfscope}%
\pgfsys@transformshift{0.594525in}{1.038062in}%
\pgfsys@useobject{currentmarker}{}%
\end{pgfscope}%
\end{pgfscope}%
\begin{pgfscope}%
\pgfpathrectangle{\pgfqpoint{0.594525in}{0.417642in}}{\pgfqpoint{3.354228in}{2.055000in}}%
\pgfusepath{clip}%
\pgfsetrectcap%
\pgfsetroundjoin%
\pgfsetlinewidth{0.803000pt}%
\definecolor{currentstroke}{rgb}{0.850000,0.850000,0.850000}%
\pgfsetstrokecolor{currentstroke}%
\pgfsetdash{}{0pt}%
\pgfpathmoveto{\pgfqpoint{0.594525in}{1.072758in}}%
\pgfpathlineto{\pgfqpoint{3.948753in}{1.072758in}}%
\pgfusepath{stroke}%
\end{pgfscope}%
\begin{pgfscope}%
\pgfsetbuttcap%
\pgfsetroundjoin%
\definecolor{currentfill}{rgb}{0.000000,0.000000,0.000000}%
\pgfsetfillcolor{currentfill}%
\pgfsetlinewidth{0.602250pt}%
\definecolor{currentstroke}{rgb}{0.000000,0.000000,0.000000}%
\pgfsetstrokecolor{currentstroke}%
\pgfsetdash{}{0pt}%
\pgfsys@defobject{currentmarker}{\pgfqpoint{-0.027778in}{0.000000in}}{\pgfqpoint{-0.000000in}{0.000000in}}{%
\pgfpathmoveto{\pgfqpoint{-0.000000in}{0.000000in}}%
\pgfpathlineto{\pgfqpoint{-0.027778in}{0.000000in}}%
\pgfusepath{stroke,fill}%
}%
\begin{pgfscope}%
\pgfsys@transformshift{0.594525in}{1.072758in}%
\pgfsys@useobject{currentmarker}{}%
\end{pgfscope}%
\end{pgfscope}%
\begin{pgfscope}%
\pgfpathrectangle{\pgfqpoint{0.594525in}{0.417642in}}{\pgfqpoint{3.354228in}{2.055000in}}%
\pgfusepath{clip}%
\pgfsetrectcap%
\pgfsetroundjoin%
\pgfsetlinewidth{0.803000pt}%
\definecolor{currentstroke}{rgb}{0.850000,0.850000,0.850000}%
\pgfsetstrokecolor{currentstroke}%
\pgfsetdash{}{0pt}%
\pgfpathmoveto{\pgfqpoint{0.594525in}{1.099670in}}%
\pgfpathlineto{\pgfqpoint{3.948753in}{1.099670in}}%
\pgfusepath{stroke}%
\end{pgfscope}%
\begin{pgfscope}%
\pgfsetbuttcap%
\pgfsetroundjoin%
\definecolor{currentfill}{rgb}{0.000000,0.000000,0.000000}%
\pgfsetfillcolor{currentfill}%
\pgfsetlinewidth{0.602250pt}%
\definecolor{currentstroke}{rgb}{0.000000,0.000000,0.000000}%
\pgfsetstrokecolor{currentstroke}%
\pgfsetdash{}{0pt}%
\pgfsys@defobject{currentmarker}{\pgfqpoint{-0.027778in}{0.000000in}}{\pgfqpoint{-0.000000in}{0.000000in}}{%
\pgfpathmoveto{\pgfqpoint{-0.000000in}{0.000000in}}%
\pgfpathlineto{\pgfqpoint{-0.027778in}{0.000000in}}%
\pgfusepath{stroke,fill}%
}%
\begin{pgfscope}%
\pgfsys@transformshift{0.594525in}{1.099670in}%
\pgfsys@useobject{currentmarker}{}%
\end{pgfscope}%
\end{pgfscope}%
\begin{pgfscope}%
\pgfpathrectangle{\pgfqpoint{0.594525in}{0.417642in}}{\pgfqpoint{3.354228in}{2.055000in}}%
\pgfusepath{clip}%
\pgfsetrectcap%
\pgfsetroundjoin%
\pgfsetlinewidth{0.803000pt}%
\definecolor{currentstroke}{rgb}{0.850000,0.850000,0.850000}%
\pgfsetstrokecolor{currentstroke}%
\pgfsetdash{}{0pt}%
\pgfpathmoveto{\pgfqpoint{0.594525in}{1.121659in}}%
\pgfpathlineto{\pgfqpoint{3.948753in}{1.121659in}}%
\pgfusepath{stroke}%
\end{pgfscope}%
\begin{pgfscope}%
\pgfsetbuttcap%
\pgfsetroundjoin%
\definecolor{currentfill}{rgb}{0.000000,0.000000,0.000000}%
\pgfsetfillcolor{currentfill}%
\pgfsetlinewidth{0.602250pt}%
\definecolor{currentstroke}{rgb}{0.000000,0.000000,0.000000}%
\pgfsetstrokecolor{currentstroke}%
\pgfsetdash{}{0pt}%
\pgfsys@defobject{currentmarker}{\pgfqpoint{-0.027778in}{0.000000in}}{\pgfqpoint{-0.000000in}{0.000000in}}{%
\pgfpathmoveto{\pgfqpoint{-0.000000in}{0.000000in}}%
\pgfpathlineto{\pgfqpoint{-0.027778in}{0.000000in}}%
\pgfusepath{stroke,fill}%
}%
\begin{pgfscope}%
\pgfsys@transformshift{0.594525in}{1.121659in}%
\pgfsys@useobject{currentmarker}{}%
\end{pgfscope}%
\end{pgfscope}%
\begin{pgfscope}%
\pgfpathrectangle{\pgfqpoint{0.594525in}{0.417642in}}{\pgfqpoint{3.354228in}{2.055000in}}%
\pgfusepath{clip}%
\pgfsetrectcap%
\pgfsetroundjoin%
\pgfsetlinewidth{0.803000pt}%
\definecolor{currentstroke}{rgb}{0.850000,0.850000,0.850000}%
\pgfsetstrokecolor{currentstroke}%
\pgfsetdash{}{0pt}%
\pgfpathmoveto{\pgfqpoint{0.594525in}{1.140250in}}%
\pgfpathlineto{\pgfqpoint{3.948753in}{1.140250in}}%
\pgfusepath{stroke}%
\end{pgfscope}%
\begin{pgfscope}%
\pgfsetbuttcap%
\pgfsetroundjoin%
\definecolor{currentfill}{rgb}{0.000000,0.000000,0.000000}%
\pgfsetfillcolor{currentfill}%
\pgfsetlinewidth{0.602250pt}%
\definecolor{currentstroke}{rgb}{0.000000,0.000000,0.000000}%
\pgfsetstrokecolor{currentstroke}%
\pgfsetdash{}{0pt}%
\pgfsys@defobject{currentmarker}{\pgfqpoint{-0.027778in}{0.000000in}}{\pgfqpoint{-0.000000in}{0.000000in}}{%
\pgfpathmoveto{\pgfqpoint{-0.000000in}{0.000000in}}%
\pgfpathlineto{\pgfqpoint{-0.027778in}{0.000000in}}%
\pgfusepath{stroke,fill}%
}%
\begin{pgfscope}%
\pgfsys@transformshift{0.594525in}{1.140250in}%
\pgfsys@useobject{currentmarker}{}%
\end{pgfscope}%
\end{pgfscope}%
\begin{pgfscope}%
\pgfpathrectangle{\pgfqpoint{0.594525in}{0.417642in}}{\pgfqpoint{3.354228in}{2.055000in}}%
\pgfusepath{clip}%
\pgfsetrectcap%
\pgfsetroundjoin%
\pgfsetlinewidth{0.803000pt}%
\definecolor{currentstroke}{rgb}{0.850000,0.850000,0.850000}%
\pgfsetstrokecolor{currentstroke}%
\pgfsetdash{}{0pt}%
\pgfpathmoveto{\pgfqpoint{0.594525in}{1.156355in}}%
\pgfpathlineto{\pgfqpoint{3.948753in}{1.156355in}}%
\pgfusepath{stroke}%
\end{pgfscope}%
\begin{pgfscope}%
\pgfsetbuttcap%
\pgfsetroundjoin%
\definecolor{currentfill}{rgb}{0.000000,0.000000,0.000000}%
\pgfsetfillcolor{currentfill}%
\pgfsetlinewidth{0.602250pt}%
\definecolor{currentstroke}{rgb}{0.000000,0.000000,0.000000}%
\pgfsetstrokecolor{currentstroke}%
\pgfsetdash{}{0pt}%
\pgfsys@defobject{currentmarker}{\pgfqpoint{-0.027778in}{0.000000in}}{\pgfqpoint{-0.000000in}{0.000000in}}{%
\pgfpathmoveto{\pgfqpoint{-0.000000in}{0.000000in}}%
\pgfpathlineto{\pgfqpoint{-0.027778in}{0.000000in}}%
\pgfusepath{stroke,fill}%
}%
\begin{pgfscope}%
\pgfsys@transformshift{0.594525in}{1.156355in}%
\pgfsys@useobject{currentmarker}{}%
\end{pgfscope}%
\end{pgfscope}%
\begin{pgfscope}%
\pgfpathrectangle{\pgfqpoint{0.594525in}{0.417642in}}{\pgfqpoint{3.354228in}{2.055000in}}%
\pgfusepath{clip}%
\pgfsetrectcap%
\pgfsetroundjoin%
\pgfsetlinewidth{0.803000pt}%
\definecolor{currentstroke}{rgb}{0.850000,0.850000,0.850000}%
\pgfsetstrokecolor{currentstroke}%
\pgfsetdash{}{0pt}%
\pgfpathmoveto{\pgfqpoint{0.594525in}{1.170560in}}%
\pgfpathlineto{\pgfqpoint{3.948753in}{1.170560in}}%
\pgfusepath{stroke}%
\end{pgfscope}%
\begin{pgfscope}%
\pgfsetbuttcap%
\pgfsetroundjoin%
\definecolor{currentfill}{rgb}{0.000000,0.000000,0.000000}%
\pgfsetfillcolor{currentfill}%
\pgfsetlinewidth{0.602250pt}%
\definecolor{currentstroke}{rgb}{0.000000,0.000000,0.000000}%
\pgfsetstrokecolor{currentstroke}%
\pgfsetdash{}{0pt}%
\pgfsys@defobject{currentmarker}{\pgfqpoint{-0.027778in}{0.000000in}}{\pgfqpoint{-0.000000in}{0.000000in}}{%
\pgfpathmoveto{\pgfqpoint{-0.000000in}{0.000000in}}%
\pgfpathlineto{\pgfqpoint{-0.027778in}{0.000000in}}%
\pgfusepath{stroke,fill}%
}%
\begin{pgfscope}%
\pgfsys@transformshift{0.594525in}{1.170560in}%
\pgfsys@useobject{currentmarker}{}%
\end{pgfscope}%
\end{pgfscope}%
\begin{pgfscope}%
\pgfpathrectangle{\pgfqpoint{0.594525in}{0.417642in}}{\pgfqpoint{3.354228in}{2.055000in}}%
\pgfusepath{clip}%
\pgfsetrectcap%
\pgfsetroundjoin%
\pgfsetlinewidth{0.803000pt}%
\definecolor{currentstroke}{rgb}{0.850000,0.850000,0.850000}%
\pgfsetstrokecolor{currentstroke}%
\pgfsetdash{}{0pt}%
\pgfpathmoveto{\pgfqpoint{0.594525in}{1.266864in}}%
\pgfpathlineto{\pgfqpoint{3.948753in}{1.266864in}}%
\pgfusepath{stroke}%
\end{pgfscope}%
\begin{pgfscope}%
\pgfsetbuttcap%
\pgfsetroundjoin%
\definecolor{currentfill}{rgb}{0.000000,0.000000,0.000000}%
\pgfsetfillcolor{currentfill}%
\pgfsetlinewidth{0.602250pt}%
\definecolor{currentstroke}{rgb}{0.000000,0.000000,0.000000}%
\pgfsetstrokecolor{currentstroke}%
\pgfsetdash{}{0pt}%
\pgfsys@defobject{currentmarker}{\pgfqpoint{-0.027778in}{0.000000in}}{\pgfqpoint{-0.000000in}{0.000000in}}{%
\pgfpathmoveto{\pgfqpoint{-0.000000in}{0.000000in}}%
\pgfpathlineto{\pgfqpoint{-0.027778in}{0.000000in}}%
\pgfusepath{stroke,fill}%
}%
\begin{pgfscope}%
\pgfsys@transformshift{0.594525in}{1.266864in}%
\pgfsys@useobject{currentmarker}{}%
\end{pgfscope}%
\end{pgfscope}%
\begin{pgfscope}%
\pgfpathrectangle{\pgfqpoint{0.594525in}{0.417642in}}{\pgfqpoint{3.354228in}{2.055000in}}%
\pgfusepath{clip}%
\pgfsetrectcap%
\pgfsetroundjoin%
\pgfsetlinewidth{0.803000pt}%
\definecolor{currentstroke}{rgb}{0.850000,0.850000,0.850000}%
\pgfsetstrokecolor{currentstroke}%
\pgfsetdash{}{0pt}%
\pgfpathmoveto{\pgfqpoint{0.594525in}{1.315765in}}%
\pgfpathlineto{\pgfqpoint{3.948753in}{1.315765in}}%
\pgfusepath{stroke}%
\end{pgfscope}%
\begin{pgfscope}%
\pgfsetbuttcap%
\pgfsetroundjoin%
\definecolor{currentfill}{rgb}{0.000000,0.000000,0.000000}%
\pgfsetfillcolor{currentfill}%
\pgfsetlinewidth{0.602250pt}%
\definecolor{currentstroke}{rgb}{0.000000,0.000000,0.000000}%
\pgfsetstrokecolor{currentstroke}%
\pgfsetdash{}{0pt}%
\pgfsys@defobject{currentmarker}{\pgfqpoint{-0.027778in}{0.000000in}}{\pgfqpoint{-0.000000in}{0.000000in}}{%
\pgfpathmoveto{\pgfqpoint{-0.000000in}{0.000000in}}%
\pgfpathlineto{\pgfqpoint{-0.027778in}{0.000000in}}%
\pgfusepath{stroke,fill}%
}%
\begin{pgfscope}%
\pgfsys@transformshift{0.594525in}{1.315765in}%
\pgfsys@useobject{currentmarker}{}%
\end{pgfscope}%
\end{pgfscope}%
\begin{pgfscope}%
\pgfpathrectangle{\pgfqpoint{0.594525in}{0.417642in}}{\pgfqpoint{3.354228in}{2.055000in}}%
\pgfusepath{clip}%
\pgfsetrectcap%
\pgfsetroundjoin%
\pgfsetlinewidth{0.803000pt}%
\definecolor{currentstroke}{rgb}{0.850000,0.850000,0.850000}%
\pgfsetstrokecolor{currentstroke}%
\pgfsetdash{}{0pt}%
\pgfpathmoveto{\pgfqpoint{0.594525in}{1.350461in}}%
\pgfpathlineto{\pgfqpoint{3.948753in}{1.350461in}}%
\pgfusepath{stroke}%
\end{pgfscope}%
\begin{pgfscope}%
\pgfsetbuttcap%
\pgfsetroundjoin%
\definecolor{currentfill}{rgb}{0.000000,0.000000,0.000000}%
\pgfsetfillcolor{currentfill}%
\pgfsetlinewidth{0.602250pt}%
\definecolor{currentstroke}{rgb}{0.000000,0.000000,0.000000}%
\pgfsetstrokecolor{currentstroke}%
\pgfsetdash{}{0pt}%
\pgfsys@defobject{currentmarker}{\pgfqpoint{-0.027778in}{0.000000in}}{\pgfqpoint{-0.000000in}{0.000000in}}{%
\pgfpathmoveto{\pgfqpoint{-0.000000in}{0.000000in}}%
\pgfpathlineto{\pgfqpoint{-0.027778in}{0.000000in}}%
\pgfusepath{stroke,fill}%
}%
\begin{pgfscope}%
\pgfsys@transformshift{0.594525in}{1.350461in}%
\pgfsys@useobject{currentmarker}{}%
\end{pgfscope}%
\end{pgfscope}%
\begin{pgfscope}%
\pgfpathrectangle{\pgfqpoint{0.594525in}{0.417642in}}{\pgfqpoint{3.354228in}{2.055000in}}%
\pgfusepath{clip}%
\pgfsetrectcap%
\pgfsetroundjoin%
\pgfsetlinewidth{0.803000pt}%
\definecolor{currentstroke}{rgb}{0.850000,0.850000,0.850000}%
\pgfsetstrokecolor{currentstroke}%
\pgfsetdash{}{0pt}%
\pgfpathmoveto{\pgfqpoint{0.594525in}{1.377373in}}%
\pgfpathlineto{\pgfqpoint{3.948753in}{1.377373in}}%
\pgfusepath{stroke}%
\end{pgfscope}%
\begin{pgfscope}%
\pgfsetbuttcap%
\pgfsetroundjoin%
\definecolor{currentfill}{rgb}{0.000000,0.000000,0.000000}%
\pgfsetfillcolor{currentfill}%
\pgfsetlinewidth{0.602250pt}%
\definecolor{currentstroke}{rgb}{0.000000,0.000000,0.000000}%
\pgfsetstrokecolor{currentstroke}%
\pgfsetdash{}{0pt}%
\pgfsys@defobject{currentmarker}{\pgfqpoint{-0.027778in}{0.000000in}}{\pgfqpoint{-0.000000in}{0.000000in}}{%
\pgfpathmoveto{\pgfqpoint{-0.000000in}{0.000000in}}%
\pgfpathlineto{\pgfqpoint{-0.027778in}{0.000000in}}%
\pgfusepath{stroke,fill}%
}%
\begin{pgfscope}%
\pgfsys@transformshift{0.594525in}{1.377373in}%
\pgfsys@useobject{currentmarker}{}%
\end{pgfscope}%
\end{pgfscope}%
\begin{pgfscope}%
\pgfpathrectangle{\pgfqpoint{0.594525in}{0.417642in}}{\pgfqpoint{3.354228in}{2.055000in}}%
\pgfusepath{clip}%
\pgfsetrectcap%
\pgfsetroundjoin%
\pgfsetlinewidth{0.803000pt}%
\definecolor{currentstroke}{rgb}{0.850000,0.850000,0.850000}%
\pgfsetstrokecolor{currentstroke}%
\pgfsetdash{}{0pt}%
\pgfpathmoveto{\pgfqpoint{0.594525in}{1.399362in}}%
\pgfpathlineto{\pgfqpoint{3.948753in}{1.399362in}}%
\pgfusepath{stroke}%
\end{pgfscope}%
\begin{pgfscope}%
\pgfsetbuttcap%
\pgfsetroundjoin%
\definecolor{currentfill}{rgb}{0.000000,0.000000,0.000000}%
\pgfsetfillcolor{currentfill}%
\pgfsetlinewidth{0.602250pt}%
\definecolor{currentstroke}{rgb}{0.000000,0.000000,0.000000}%
\pgfsetstrokecolor{currentstroke}%
\pgfsetdash{}{0pt}%
\pgfsys@defobject{currentmarker}{\pgfqpoint{-0.027778in}{0.000000in}}{\pgfqpoint{-0.000000in}{0.000000in}}{%
\pgfpathmoveto{\pgfqpoint{-0.000000in}{0.000000in}}%
\pgfpathlineto{\pgfqpoint{-0.027778in}{0.000000in}}%
\pgfusepath{stroke,fill}%
}%
\begin{pgfscope}%
\pgfsys@transformshift{0.594525in}{1.399362in}%
\pgfsys@useobject{currentmarker}{}%
\end{pgfscope}%
\end{pgfscope}%
\begin{pgfscope}%
\pgfpathrectangle{\pgfqpoint{0.594525in}{0.417642in}}{\pgfqpoint{3.354228in}{2.055000in}}%
\pgfusepath{clip}%
\pgfsetrectcap%
\pgfsetroundjoin%
\pgfsetlinewidth{0.803000pt}%
\definecolor{currentstroke}{rgb}{0.850000,0.850000,0.850000}%
\pgfsetstrokecolor{currentstroke}%
\pgfsetdash{}{0pt}%
\pgfpathmoveto{\pgfqpoint{0.594525in}{1.417953in}}%
\pgfpathlineto{\pgfqpoint{3.948753in}{1.417953in}}%
\pgfusepath{stroke}%
\end{pgfscope}%
\begin{pgfscope}%
\pgfsetbuttcap%
\pgfsetroundjoin%
\definecolor{currentfill}{rgb}{0.000000,0.000000,0.000000}%
\pgfsetfillcolor{currentfill}%
\pgfsetlinewidth{0.602250pt}%
\definecolor{currentstroke}{rgb}{0.000000,0.000000,0.000000}%
\pgfsetstrokecolor{currentstroke}%
\pgfsetdash{}{0pt}%
\pgfsys@defobject{currentmarker}{\pgfqpoint{-0.027778in}{0.000000in}}{\pgfqpoint{-0.000000in}{0.000000in}}{%
\pgfpathmoveto{\pgfqpoint{-0.000000in}{0.000000in}}%
\pgfpathlineto{\pgfqpoint{-0.027778in}{0.000000in}}%
\pgfusepath{stroke,fill}%
}%
\begin{pgfscope}%
\pgfsys@transformshift{0.594525in}{1.417953in}%
\pgfsys@useobject{currentmarker}{}%
\end{pgfscope}%
\end{pgfscope}%
\begin{pgfscope}%
\pgfpathrectangle{\pgfqpoint{0.594525in}{0.417642in}}{\pgfqpoint{3.354228in}{2.055000in}}%
\pgfusepath{clip}%
\pgfsetrectcap%
\pgfsetroundjoin%
\pgfsetlinewidth{0.803000pt}%
\definecolor{currentstroke}{rgb}{0.850000,0.850000,0.850000}%
\pgfsetstrokecolor{currentstroke}%
\pgfsetdash{}{0pt}%
\pgfpathmoveto{\pgfqpoint{0.594525in}{1.434058in}}%
\pgfpathlineto{\pgfqpoint{3.948753in}{1.434058in}}%
\pgfusepath{stroke}%
\end{pgfscope}%
\begin{pgfscope}%
\pgfsetbuttcap%
\pgfsetroundjoin%
\definecolor{currentfill}{rgb}{0.000000,0.000000,0.000000}%
\pgfsetfillcolor{currentfill}%
\pgfsetlinewidth{0.602250pt}%
\definecolor{currentstroke}{rgb}{0.000000,0.000000,0.000000}%
\pgfsetstrokecolor{currentstroke}%
\pgfsetdash{}{0pt}%
\pgfsys@defobject{currentmarker}{\pgfqpoint{-0.027778in}{0.000000in}}{\pgfqpoint{-0.000000in}{0.000000in}}{%
\pgfpathmoveto{\pgfqpoint{-0.000000in}{0.000000in}}%
\pgfpathlineto{\pgfqpoint{-0.027778in}{0.000000in}}%
\pgfusepath{stroke,fill}%
}%
\begin{pgfscope}%
\pgfsys@transformshift{0.594525in}{1.434058in}%
\pgfsys@useobject{currentmarker}{}%
\end{pgfscope}%
\end{pgfscope}%
\begin{pgfscope}%
\pgfpathrectangle{\pgfqpoint{0.594525in}{0.417642in}}{\pgfqpoint{3.354228in}{2.055000in}}%
\pgfusepath{clip}%
\pgfsetrectcap%
\pgfsetroundjoin%
\pgfsetlinewidth{0.803000pt}%
\definecolor{currentstroke}{rgb}{0.850000,0.850000,0.850000}%
\pgfsetstrokecolor{currentstroke}%
\pgfsetdash{}{0pt}%
\pgfpathmoveto{\pgfqpoint{0.594525in}{1.448263in}}%
\pgfpathlineto{\pgfqpoint{3.948753in}{1.448263in}}%
\pgfusepath{stroke}%
\end{pgfscope}%
\begin{pgfscope}%
\pgfsetbuttcap%
\pgfsetroundjoin%
\definecolor{currentfill}{rgb}{0.000000,0.000000,0.000000}%
\pgfsetfillcolor{currentfill}%
\pgfsetlinewidth{0.602250pt}%
\definecolor{currentstroke}{rgb}{0.000000,0.000000,0.000000}%
\pgfsetstrokecolor{currentstroke}%
\pgfsetdash{}{0pt}%
\pgfsys@defobject{currentmarker}{\pgfqpoint{-0.027778in}{0.000000in}}{\pgfqpoint{-0.000000in}{0.000000in}}{%
\pgfpathmoveto{\pgfqpoint{-0.000000in}{0.000000in}}%
\pgfpathlineto{\pgfqpoint{-0.027778in}{0.000000in}}%
\pgfusepath{stroke,fill}%
}%
\begin{pgfscope}%
\pgfsys@transformshift{0.594525in}{1.448263in}%
\pgfsys@useobject{currentmarker}{}%
\end{pgfscope}%
\end{pgfscope}%
\begin{pgfscope}%
\pgfpathrectangle{\pgfqpoint{0.594525in}{0.417642in}}{\pgfqpoint{3.354228in}{2.055000in}}%
\pgfusepath{clip}%
\pgfsetrectcap%
\pgfsetroundjoin%
\pgfsetlinewidth{0.803000pt}%
\definecolor{currentstroke}{rgb}{0.850000,0.850000,0.850000}%
\pgfsetstrokecolor{currentstroke}%
\pgfsetdash{}{0pt}%
\pgfpathmoveto{\pgfqpoint{0.594525in}{1.544567in}}%
\pgfpathlineto{\pgfqpoint{3.948753in}{1.544567in}}%
\pgfusepath{stroke}%
\end{pgfscope}%
\begin{pgfscope}%
\pgfsetbuttcap%
\pgfsetroundjoin%
\definecolor{currentfill}{rgb}{0.000000,0.000000,0.000000}%
\pgfsetfillcolor{currentfill}%
\pgfsetlinewidth{0.602250pt}%
\definecolor{currentstroke}{rgb}{0.000000,0.000000,0.000000}%
\pgfsetstrokecolor{currentstroke}%
\pgfsetdash{}{0pt}%
\pgfsys@defobject{currentmarker}{\pgfqpoint{-0.027778in}{0.000000in}}{\pgfqpoint{-0.000000in}{0.000000in}}{%
\pgfpathmoveto{\pgfqpoint{-0.000000in}{0.000000in}}%
\pgfpathlineto{\pgfqpoint{-0.027778in}{0.000000in}}%
\pgfusepath{stroke,fill}%
}%
\begin{pgfscope}%
\pgfsys@transformshift{0.594525in}{1.544567in}%
\pgfsys@useobject{currentmarker}{}%
\end{pgfscope}%
\end{pgfscope}%
\begin{pgfscope}%
\pgfpathrectangle{\pgfqpoint{0.594525in}{0.417642in}}{\pgfqpoint{3.354228in}{2.055000in}}%
\pgfusepath{clip}%
\pgfsetrectcap%
\pgfsetroundjoin%
\pgfsetlinewidth{0.803000pt}%
\definecolor{currentstroke}{rgb}{0.850000,0.850000,0.850000}%
\pgfsetstrokecolor{currentstroke}%
\pgfsetdash{}{0pt}%
\pgfpathmoveto{\pgfqpoint{0.594525in}{1.593468in}}%
\pgfpathlineto{\pgfqpoint{3.948753in}{1.593468in}}%
\pgfusepath{stroke}%
\end{pgfscope}%
\begin{pgfscope}%
\pgfsetbuttcap%
\pgfsetroundjoin%
\definecolor{currentfill}{rgb}{0.000000,0.000000,0.000000}%
\pgfsetfillcolor{currentfill}%
\pgfsetlinewidth{0.602250pt}%
\definecolor{currentstroke}{rgb}{0.000000,0.000000,0.000000}%
\pgfsetstrokecolor{currentstroke}%
\pgfsetdash{}{0pt}%
\pgfsys@defobject{currentmarker}{\pgfqpoint{-0.027778in}{0.000000in}}{\pgfqpoint{-0.000000in}{0.000000in}}{%
\pgfpathmoveto{\pgfqpoint{-0.000000in}{0.000000in}}%
\pgfpathlineto{\pgfqpoint{-0.027778in}{0.000000in}}%
\pgfusepath{stroke,fill}%
}%
\begin{pgfscope}%
\pgfsys@transformshift{0.594525in}{1.593468in}%
\pgfsys@useobject{currentmarker}{}%
\end{pgfscope}%
\end{pgfscope}%
\begin{pgfscope}%
\pgfpathrectangle{\pgfqpoint{0.594525in}{0.417642in}}{\pgfqpoint{3.354228in}{2.055000in}}%
\pgfusepath{clip}%
\pgfsetrectcap%
\pgfsetroundjoin%
\pgfsetlinewidth{0.803000pt}%
\definecolor{currentstroke}{rgb}{0.850000,0.850000,0.850000}%
\pgfsetstrokecolor{currentstroke}%
\pgfsetdash{}{0pt}%
\pgfpathmoveto{\pgfqpoint{0.594525in}{1.628164in}}%
\pgfpathlineto{\pgfqpoint{3.948753in}{1.628164in}}%
\pgfusepath{stroke}%
\end{pgfscope}%
\begin{pgfscope}%
\pgfsetbuttcap%
\pgfsetroundjoin%
\definecolor{currentfill}{rgb}{0.000000,0.000000,0.000000}%
\pgfsetfillcolor{currentfill}%
\pgfsetlinewidth{0.602250pt}%
\definecolor{currentstroke}{rgb}{0.000000,0.000000,0.000000}%
\pgfsetstrokecolor{currentstroke}%
\pgfsetdash{}{0pt}%
\pgfsys@defobject{currentmarker}{\pgfqpoint{-0.027778in}{0.000000in}}{\pgfqpoint{-0.000000in}{0.000000in}}{%
\pgfpathmoveto{\pgfqpoint{-0.000000in}{0.000000in}}%
\pgfpathlineto{\pgfqpoint{-0.027778in}{0.000000in}}%
\pgfusepath{stroke,fill}%
}%
\begin{pgfscope}%
\pgfsys@transformshift{0.594525in}{1.628164in}%
\pgfsys@useobject{currentmarker}{}%
\end{pgfscope}%
\end{pgfscope}%
\begin{pgfscope}%
\pgfpathrectangle{\pgfqpoint{0.594525in}{0.417642in}}{\pgfqpoint{3.354228in}{2.055000in}}%
\pgfusepath{clip}%
\pgfsetrectcap%
\pgfsetroundjoin%
\pgfsetlinewidth{0.803000pt}%
\definecolor{currentstroke}{rgb}{0.850000,0.850000,0.850000}%
\pgfsetstrokecolor{currentstroke}%
\pgfsetdash{}{0pt}%
\pgfpathmoveto{\pgfqpoint{0.594525in}{1.655076in}}%
\pgfpathlineto{\pgfqpoint{3.948753in}{1.655076in}}%
\pgfusepath{stroke}%
\end{pgfscope}%
\begin{pgfscope}%
\pgfsetbuttcap%
\pgfsetroundjoin%
\definecolor{currentfill}{rgb}{0.000000,0.000000,0.000000}%
\pgfsetfillcolor{currentfill}%
\pgfsetlinewidth{0.602250pt}%
\definecolor{currentstroke}{rgb}{0.000000,0.000000,0.000000}%
\pgfsetstrokecolor{currentstroke}%
\pgfsetdash{}{0pt}%
\pgfsys@defobject{currentmarker}{\pgfqpoint{-0.027778in}{0.000000in}}{\pgfqpoint{-0.000000in}{0.000000in}}{%
\pgfpathmoveto{\pgfqpoint{-0.000000in}{0.000000in}}%
\pgfpathlineto{\pgfqpoint{-0.027778in}{0.000000in}}%
\pgfusepath{stroke,fill}%
}%
\begin{pgfscope}%
\pgfsys@transformshift{0.594525in}{1.655076in}%
\pgfsys@useobject{currentmarker}{}%
\end{pgfscope}%
\end{pgfscope}%
\begin{pgfscope}%
\pgfpathrectangle{\pgfqpoint{0.594525in}{0.417642in}}{\pgfqpoint{3.354228in}{2.055000in}}%
\pgfusepath{clip}%
\pgfsetrectcap%
\pgfsetroundjoin%
\pgfsetlinewidth{0.803000pt}%
\definecolor{currentstroke}{rgb}{0.850000,0.850000,0.850000}%
\pgfsetstrokecolor{currentstroke}%
\pgfsetdash{}{0pt}%
\pgfpathmoveto{\pgfqpoint{0.594525in}{1.677065in}}%
\pgfpathlineto{\pgfqpoint{3.948753in}{1.677065in}}%
\pgfusepath{stroke}%
\end{pgfscope}%
\begin{pgfscope}%
\pgfsetbuttcap%
\pgfsetroundjoin%
\definecolor{currentfill}{rgb}{0.000000,0.000000,0.000000}%
\pgfsetfillcolor{currentfill}%
\pgfsetlinewidth{0.602250pt}%
\definecolor{currentstroke}{rgb}{0.000000,0.000000,0.000000}%
\pgfsetstrokecolor{currentstroke}%
\pgfsetdash{}{0pt}%
\pgfsys@defobject{currentmarker}{\pgfqpoint{-0.027778in}{0.000000in}}{\pgfqpoint{-0.000000in}{0.000000in}}{%
\pgfpathmoveto{\pgfqpoint{-0.000000in}{0.000000in}}%
\pgfpathlineto{\pgfqpoint{-0.027778in}{0.000000in}}%
\pgfusepath{stroke,fill}%
}%
\begin{pgfscope}%
\pgfsys@transformshift{0.594525in}{1.677065in}%
\pgfsys@useobject{currentmarker}{}%
\end{pgfscope}%
\end{pgfscope}%
\begin{pgfscope}%
\pgfpathrectangle{\pgfqpoint{0.594525in}{0.417642in}}{\pgfqpoint{3.354228in}{2.055000in}}%
\pgfusepath{clip}%
\pgfsetrectcap%
\pgfsetroundjoin%
\pgfsetlinewidth{0.803000pt}%
\definecolor{currentstroke}{rgb}{0.850000,0.850000,0.850000}%
\pgfsetstrokecolor{currentstroke}%
\pgfsetdash{}{0pt}%
\pgfpathmoveto{\pgfqpoint{0.594525in}{1.695656in}}%
\pgfpathlineto{\pgfqpoint{3.948753in}{1.695656in}}%
\pgfusepath{stroke}%
\end{pgfscope}%
\begin{pgfscope}%
\pgfsetbuttcap%
\pgfsetroundjoin%
\definecolor{currentfill}{rgb}{0.000000,0.000000,0.000000}%
\pgfsetfillcolor{currentfill}%
\pgfsetlinewidth{0.602250pt}%
\definecolor{currentstroke}{rgb}{0.000000,0.000000,0.000000}%
\pgfsetstrokecolor{currentstroke}%
\pgfsetdash{}{0pt}%
\pgfsys@defobject{currentmarker}{\pgfqpoint{-0.027778in}{0.000000in}}{\pgfqpoint{-0.000000in}{0.000000in}}{%
\pgfpathmoveto{\pgfqpoint{-0.000000in}{0.000000in}}%
\pgfpathlineto{\pgfqpoint{-0.027778in}{0.000000in}}%
\pgfusepath{stroke,fill}%
}%
\begin{pgfscope}%
\pgfsys@transformshift{0.594525in}{1.695656in}%
\pgfsys@useobject{currentmarker}{}%
\end{pgfscope}%
\end{pgfscope}%
\begin{pgfscope}%
\pgfpathrectangle{\pgfqpoint{0.594525in}{0.417642in}}{\pgfqpoint{3.354228in}{2.055000in}}%
\pgfusepath{clip}%
\pgfsetrectcap%
\pgfsetroundjoin%
\pgfsetlinewidth{0.803000pt}%
\definecolor{currentstroke}{rgb}{0.850000,0.850000,0.850000}%
\pgfsetstrokecolor{currentstroke}%
\pgfsetdash{}{0pt}%
\pgfpathmoveto{\pgfqpoint{0.594525in}{1.711761in}}%
\pgfpathlineto{\pgfqpoint{3.948753in}{1.711761in}}%
\pgfusepath{stroke}%
\end{pgfscope}%
\begin{pgfscope}%
\pgfsetbuttcap%
\pgfsetroundjoin%
\definecolor{currentfill}{rgb}{0.000000,0.000000,0.000000}%
\pgfsetfillcolor{currentfill}%
\pgfsetlinewidth{0.602250pt}%
\definecolor{currentstroke}{rgb}{0.000000,0.000000,0.000000}%
\pgfsetstrokecolor{currentstroke}%
\pgfsetdash{}{0pt}%
\pgfsys@defobject{currentmarker}{\pgfqpoint{-0.027778in}{0.000000in}}{\pgfqpoint{-0.000000in}{0.000000in}}{%
\pgfpathmoveto{\pgfqpoint{-0.000000in}{0.000000in}}%
\pgfpathlineto{\pgfqpoint{-0.027778in}{0.000000in}}%
\pgfusepath{stroke,fill}%
}%
\begin{pgfscope}%
\pgfsys@transformshift{0.594525in}{1.711761in}%
\pgfsys@useobject{currentmarker}{}%
\end{pgfscope}%
\end{pgfscope}%
\begin{pgfscope}%
\pgfpathrectangle{\pgfqpoint{0.594525in}{0.417642in}}{\pgfqpoint{3.354228in}{2.055000in}}%
\pgfusepath{clip}%
\pgfsetrectcap%
\pgfsetroundjoin%
\pgfsetlinewidth{0.803000pt}%
\definecolor{currentstroke}{rgb}{0.850000,0.850000,0.850000}%
\pgfsetstrokecolor{currentstroke}%
\pgfsetdash{}{0pt}%
\pgfpathmoveto{\pgfqpoint{0.594525in}{1.725966in}}%
\pgfpathlineto{\pgfqpoint{3.948753in}{1.725966in}}%
\pgfusepath{stroke}%
\end{pgfscope}%
\begin{pgfscope}%
\pgfsetbuttcap%
\pgfsetroundjoin%
\definecolor{currentfill}{rgb}{0.000000,0.000000,0.000000}%
\pgfsetfillcolor{currentfill}%
\pgfsetlinewidth{0.602250pt}%
\definecolor{currentstroke}{rgb}{0.000000,0.000000,0.000000}%
\pgfsetstrokecolor{currentstroke}%
\pgfsetdash{}{0pt}%
\pgfsys@defobject{currentmarker}{\pgfqpoint{-0.027778in}{0.000000in}}{\pgfqpoint{-0.000000in}{0.000000in}}{%
\pgfpathmoveto{\pgfqpoint{-0.000000in}{0.000000in}}%
\pgfpathlineto{\pgfqpoint{-0.027778in}{0.000000in}}%
\pgfusepath{stroke,fill}%
}%
\begin{pgfscope}%
\pgfsys@transformshift{0.594525in}{1.725966in}%
\pgfsys@useobject{currentmarker}{}%
\end{pgfscope}%
\end{pgfscope}%
\begin{pgfscope}%
\pgfpathrectangle{\pgfqpoint{0.594525in}{0.417642in}}{\pgfqpoint{3.354228in}{2.055000in}}%
\pgfusepath{clip}%
\pgfsetrectcap%
\pgfsetroundjoin%
\pgfsetlinewidth{0.803000pt}%
\definecolor{currentstroke}{rgb}{0.850000,0.850000,0.850000}%
\pgfsetstrokecolor{currentstroke}%
\pgfsetdash{}{0pt}%
\pgfpathmoveto{\pgfqpoint{0.594525in}{1.822270in}}%
\pgfpathlineto{\pgfqpoint{3.948753in}{1.822270in}}%
\pgfusepath{stroke}%
\end{pgfscope}%
\begin{pgfscope}%
\pgfsetbuttcap%
\pgfsetroundjoin%
\definecolor{currentfill}{rgb}{0.000000,0.000000,0.000000}%
\pgfsetfillcolor{currentfill}%
\pgfsetlinewidth{0.602250pt}%
\definecolor{currentstroke}{rgb}{0.000000,0.000000,0.000000}%
\pgfsetstrokecolor{currentstroke}%
\pgfsetdash{}{0pt}%
\pgfsys@defobject{currentmarker}{\pgfqpoint{-0.027778in}{0.000000in}}{\pgfqpoint{-0.000000in}{0.000000in}}{%
\pgfpathmoveto{\pgfqpoint{-0.000000in}{0.000000in}}%
\pgfpathlineto{\pgfqpoint{-0.027778in}{0.000000in}}%
\pgfusepath{stroke,fill}%
}%
\begin{pgfscope}%
\pgfsys@transformshift{0.594525in}{1.822270in}%
\pgfsys@useobject{currentmarker}{}%
\end{pgfscope}%
\end{pgfscope}%
\begin{pgfscope}%
\pgfpathrectangle{\pgfqpoint{0.594525in}{0.417642in}}{\pgfqpoint{3.354228in}{2.055000in}}%
\pgfusepath{clip}%
\pgfsetrectcap%
\pgfsetroundjoin%
\pgfsetlinewidth{0.803000pt}%
\definecolor{currentstroke}{rgb}{0.850000,0.850000,0.850000}%
\pgfsetstrokecolor{currentstroke}%
\pgfsetdash{}{0pt}%
\pgfpathmoveto{\pgfqpoint{0.594525in}{1.871171in}}%
\pgfpathlineto{\pgfqpoint{3.948753in}{1.871171in}}%
\pgfusepath{stroke}%
\end{pgfscope}%
\begin{pgfscope}%
\pgfsetbuttcap%
\pgfsetroundjoin%
\definecolor{currentfill}{rgb}{0.000000,0.000000,0.000000}%
\pgfsetfillcolor{currentfill}%
\pgfsetlinewidth{0.602250pt}%
\definecolor{currentstroke}{rgb}{0.000000,0.000000,0.000000}%
\pgfsetstrokecolor{currentstroke}%
\pgfsetdash{}{0pt}%
\pgfsys@defobject{currentmarker}{\pgfqpoint{-0.027778in}{0.000000in}}{\pgfqpoint{-0.000000in}{0.000000in}}{%
\pgfpathmoveto{\pgfqpoint{-0.000000in}{0.000000in}}%
\pgfpathlineto{\pgfqpoint{-0.027778in}{0.000000in}}%
\pgfusepath{stroke,fill}%
}%
\begin{pgfscope}%
\pgfsys@transformshift{0.594525in}{1.871171in}%
\pgfsys@useobject{currentmarker}{}%
\end{pgfscope}%
\end{pgfscope}%
\begin{pgfscope}%
\pgfpathrectangle{\pgfqpoint{0.594525in}{0.417642in}}{\pgfqpoint{3.354228in}{2.055000in}}%
\pgfusepath{clip}%
\pgfsetrectcap%
\pgfsetroundjoin%
\pgfsetlinewidth{0.803000pt}%
\definecolor{currentstroke}{rgb}{0.850000,0.850000,0.850000}%
\pgfsetstrokecolor{currentstroke}%
\pgfsetdash{}{0pt}%
\pgfpathmoveto{\pgfqpoint{0.594525in}{1.905867in}}%
\pgfpathlineto{\pgfqpoint{3.948753in}{1.905867in}}%
\pgfusepath{stroke}%
\end{pgfscope}%
\begin{pgfscope}%
\pgfsetbuttcap%
\pgfsetroundjoin%
\definecolor{currentfill}{rgb}{0.000000,0.000000,0.000000}%
\pgfsetfillcolor{currentfill}%
\pgfsetlinewidth{0.602250pt}%
\definecolor{currentstroke}{rgb}{0.000000,0.000000,0.000000}%
\pgfsetstrokecolor{currentstroke}%
\pgfsetdash{}{0pt}%
\pgfsys@defobject{currentmarker}{\pgfqpoint{-0.027778in}{0.000000in}}{\pgfqpoint{-0.000000in}{0.000000in}}{%
\pgfpathmoveto{\pgfqpoint{-0.000000in}{0.000000in}}%
\pgfpathlineto{\pgfqpoint{-0.027778in}{0.000000in}}%
\pgfusepath{stroke,fill}%
}%
\begin{pgfscope}%
\pgfsys@transformshift{0.594525in}{1.905867in}%
\pgfsys@useobject{currentmarker}{}%
\end{pgfscope}%
\end{pgfscope}%
\begin{pgfscope}%
\pgfpathrectangle{\pgfqpoint{0.594525in}{0.417642in}}{\pgfqpoint{3.354228in}{2.055000in}}%
\pgfusepath{clip}%
\pgfsetrectcap%
\pgfsetroundjoin%
\pgfsetlinewidth{0.803000pt}%
\definecolor{currentstroke}{rgb}{0.850000,0.850000,0.850000}%
\pgfsetstrokecolor{currentstroke}%
\pgfsetdash{}{0pt}%
\pgfpathmoveto{\pgfqpoint{0.594525in}{1.932779in}}%
\pgfpathlineto{\pgfqpoint{3.948753in}{1.932779in}}%
\pgfusepath{stroke}%
\end{pgfscope}%
\begin{pgfscope}%
\pgfsetbuttcap%
\pgfsetroundjoin%
\definecolor{currentfill}{rgb}{0.000000,0.000000,0.000000}%
\pgfsetfillcolor{currentfill}%
\pgfsetlinewidth{0.602250pt}%
\definecolor{currentstroke}{rgb}{0.000000,0.000000,0.000000}%
\pgfsetstrokecolor{currentstroke}%
\pgfsetdash{}{0pt}%
\pgfsys@defobject{currentmarker}{\pgfqpoint{-0.027778in}{0.000000in}}{\pgfqpoint{-0.000000in}{0.000000in}}{%
\pgfpathmoveto{\pgfqpoint{-0.000000in}{0.000000in}}%
\pgfpathlineto{\pgfqpoint{-0.027778in}{0.000000in}}%
\pgfusepath{stroke,fill}%
}%
\begin{pgfscope}%
\pgfsys@transformshift{0.594525in}{1.932779in}%
\pgfsys@useobject{currentmarker}{}%
\end{pgfscope}%
\end{pgfscope}%
\begin{pgfscope}%
\pgfpathrectangle{\pgfqpoint{0.594525in}{0.417642in}}{\pgfqpoint{3.354228in}{2.055000in}}%
\pgfusepath{clip}%
\pgfsetrectcap%
\pgfsetroundjoin%
\pgfsetlinewidth{0.803000pt}%
\definecolor{currentstroke}{rgb}{0.850000,0.850000,0.850000}%
\pgfsetstrokecolor{currentstroke}%
\pgfsetdash{}{0pt}%
\pgfpathmoveto{\pgfqpoint{0.594525in}{1.954768in}}%
\pgfpathlineto{\pgfqpoint{3.948753in}{1.954768in}}%
\pgfusepath{stroke}%
\end{pgfscope}%
\begin{pgfscope}%
\pgfsetbuttcap%
\pgfsetroundjoin%
\definecolor{currentfill}{rgb}{0.000000,0.000000,0.000000}%
\pgfsetfillcolor{currentfill}%
\pgfsetlinewidth{0.602250pt}%
\definecolor{currentstroke}{rgb}{0.000000,0.000000,0.000000}%
\pgfsetstrokecolor{currentstroke}%
\pgfsetdash{}{0pt}%
\pgfsys@defobject{currentmarker}{\pgfqpoint{-0.027778in}{0.000000in}}{\pgfqpoint{-0.000000in}{0.000000in}}{%
\pgfpathmoveto{\pgfqpoint{-0.000000in}{0.000000in}}%
\pgfpathlineto{\pgfqpoint{-0.027778in}{0.000000in}}%
\pgfusepath{stroke,fill}%
}%
\begin{pgfscope}%
\pgfsys@transformshift{0.594525in}{1.954768in}%
\pgfsys@useobject{currentmarker}{}%
\end{pgfscope}%
\end{pgfscope}%
\begin{pgfscope}%
\pgfpathrectangle{\pgfqpoint{0.594525in}{0.417642in}}{\pgfqpoint{3.354228in}{2.055000in}}%
\pgfusepath{clip}%
\pgfsetrectcap%
\pgfsetroundjoin%
\pgfsetlinewidth{0.803000pt}%
\definecolor{currentstroke}{rgb}{0.850000,0.850000,0.850000}%
\pgfsetstrokecolor{currentstroke}%
\pgfsetdash{}{0pt}%
\pgfpathmoveto{\pgfqpoint{0.594525in}{1.973359in}}%
\pgfpathlineto{\pgfqpoint{3.948753in}{1.973359in}}%
\pgfusepath{stroke}%
\end{pgfscope}%
\begin{pgfscope}%
\pgfsetbuttcap%
\pgfsetroundjoin%
\definecolor{currentfill}{rgb}{0.000000,0.000000,0.000000}%
\pgfsetfillcolor{currentfill}%
\pgfsetlinewidth{0.602250pt}%
\definecolor{currentstroke}{rgb}{0.000000,0.000000,0.000000}%
\pgfsetstrokecolor{currentstroke}%
\pgfsetdash{}{0pt}%
\pgfsys@defobject{currentmarker}{\pgfqpoint{-0.027778in}{0.000000in}}{\pgfqpoint{-0.000000in}{0.000000in}}{%
\pgfpathmoveto{\pgfqpoint{-0.000000in}{0.000000in}}%
\pgfpathlineto{\pgfqpoint{-0.027778in}{0.000000in}}%
\pgfusepath{stroke,fill}%
}%
\begin{pgfscope}%
\pgfsys@transformshift{0.594525in}{1.973359in}%
\pgfsys@useobject{currentmarker}{}%
\end{pgfscope}%
\end{pgfscope}%
\begin{pgfscope}%
\pgfpathrectangle{\pgfqpoint{0.594525in}{0.417642in}}{\pgfqpoint{3.354228in}{2.055000in}}%
\pgfusepath{clip}%
\pgfsetrectcap%
\pgfsetroundjoin%
\pgfsetlinewidth{0.803000pt}%
\definecolor{currentstroke}{rgb}{0.850000,0.850000,0.850000}%
\pgfsetstrokecolor{currentstroke}%
\pgfsetdash{}{0pt}%
\pgfpathmoveto{\pgfqpoint{0.594525in}{1.989464in}}%
\pgfpathlineto{\pgfqpoint{3.948753in}{1.989464in}}%
\pgfusepath{stroke}%
\end{pgfscope}%
\begin{pgfscope}%
\pgfsetbuttcap%
\pgfsetroundjoin%
\definecolor{currentfill}{rgb}{0.000000,0.000000,0.000000}%
\pgfsetfillcolor{currentfill}%
\pgfsetlinewidth{0.602250pt}%
\definecolor{currentstroke}{rgb}{0.000000,0.000000,0.000000}%
\pgfsetstrokecolor{currentstroke}%
\pgfsetdash{}{0pt}%
\pgfsys@defobject{currentmarker}{\pgfqpoint{-0.027778in}{0.000000in}}{\pgfqpoint{-0.000000in}{0.000000in}}{%
\pgfpathmoveto{\pgfqpoint{-0.000000in}{0.000000in}}%
\pgfpathlineto{\pgfqpoint{-0.027778in}{0.000000in}}%
\pgfusepath{stroke,fill}%
}%
\begin{pgfscope}%
\pgfsys@transformshift{0.594525in}{1.989464in}%
\pgfsys@useobject{currentmarker}{}%
\end{pgfscope}%
\end{pgfscope}%
\begin{pgfscope}%
\pgfpathrectangle{\pgfqpoint{0.594525in}{0.417642in}}{\pgfqpoint{3.354228in}{2.055000in}}%
\pgfusepath{clip}%
\pgfsetrectcap%
\pgfsetroundjoin%
\pgfsetlinewidth{0.803000pt}%
\definecolor{currentstroke}{rgb}{0.850000,0.850000,0.850000}%
\pgfsetstrokecolor{currentstroke}%
\pgfsetdash{}{0pt}%
\pgfpathmoveto{\pgfqpoint{0.594525in}{2.003669in}}%
\pgfpathlineto{\pgfqpoint{3.948753in}{2.003669in}}%
\pgfusepath{stroke}%
\end{pgfscope}%
\begin{pgfscope}%
\pgfsetbuttcap%
\pgfsetroundjoin%
\definecolor{currentfill}{rgb}{0.000000,0.000000,0.000000}%
\pgfsetfillcolor{currentfill}%
\pgfsetlinewidth{0.602250pt}%
\definecolor{currentstroke}{rgb}{0.000000,0.000000,0.000000}%
\pgfsetstrokecolor{currentstroke}%
\pgfsetdash{}{0pt}%
\pgfsys@defobject{currentmarker}{\pgfqpoint{-0.027778in}{0.000000in}}{\pgfqpoint{-0.000000in}{0.000000in}}{%
\pgfpathmoveto{\pgfqpoint{-0.000000in}{0.000000in}}%
\pgfpathlineto{\pgfqpoint{-0.027778in}{0.000000in}}%
\pgfusepath{stroke,fill}%
}%
\begin{pgfscope}%
\pgfsys@transformshift{0.594525in}{2.003669in}%
\pgfsys@useobject{currentmarker}{}%
\end{pgfscope}%
\end{pgfscope}%
\begin{pgfscope}%
\pgfpathrectangle{\pgfqpoint{0.594525in}{0.417642in}}{\pgfqpoint{3.354228in}{2.055000in}}%
\pgfusepath{clip}%
\pgfsetrectcap%
\pgfsetroundjoin%
\pgfsetlinewidth{0.803000pt}%
\definecolor{currentstroke}{rgb}{0.850000,0.850000,0.850000}%
\pgfsetstrokecolor{currentstroke}%
\pgfsetdash{}{0pt}%
\pgfpathmoveto{\pgfqpoint{0.594525in}{2.099973in}}%
\pgfpathlineto{\pgfqpoint{3.948753in}{2.099973in}}%
\pgfusepath{stroke}%
\end{pgfscope}%
\begin{pgfscope}%
\pgfsetbuttcap%
\pgfsetroundjoin%
\definecolor{currentfill}{rgb}{0.000000,0.000000,0.000000}%
\pgfsetfillcolor{currentfill}%
\pgfsetlinewidth{0.602250pt}%
\definecolor{currentstroke}{rgb}{0.000000,0.000000,0.000000}%
\pgfsetstrokecolor{currentstroke}%
\pgfsetdash{}{0pt}%
\pgfsys@defobject{currentmarker}{\pgfqpoint{-0.027778in}{0.000000in}}{\pgfqpoint{-0.000000in}{0.000000in}}{%
\pgfpathmoveto{\pgfqpoint{-0.000000in}{0.000000in}}%
\pgfpathlineto{\pgfqpoint{-0.027778in}{0.000000in}}%
\pgfusepath{stroke,fill}%
}%
\begin{pgfscope}%
\pgfsys@transformshift{0.594525in}{2.099973in}%
\pgfsys@useobject{currentmarker}{}%
\end{pgfscope}%
\end{pgfscope}%
\begin{pgfscope}%
\pgfpathrectangle{\pgfqpoint{0.594525in}{0.417642in}}{\pgfqpoint{3.354228in}{2.055000in}}%
\pgfusepath{clip}%
\pgfsetrectcap%
\pgfsetroundjoin%
\pgfsetlinewidth{0.803000pt}%
\definecolor{currentstroke}{rgb}{0.850000,0.850000,0.850000}%
\pgfsetstrokecolor{currentstroke}%
\pgfsetdash{}{0pt}%
\pgfpathmoveto{\pgfqpoint{0.594525in}{2.148874in}}%
\pgfpathlineto{\pgfqpoint{3.948753in}{2.148874in}}%
\pgfusepath{stroke}%
\end{pgfscope}%
\begin{pgfscope}%
\pgfsetbuttcap%
\pgfsetroundjoin%
\definecolor{currentfill}{rgb}{0.000000,0.000000,0.000000}%
\pgfsetfillcolor{currentfill}%
\pgfsetlinewidth{0.602250pt}%
\definecolor{currentstroke}{rgb}{0.000000,0.000000,0.000000}%
\pgfsetstrokecolor{currentstroke}%
\pgfsetdash{}{0pt}%
\pgfsys@defobject{currentmarker}{\pgfqpoint{-0.027778in}{0.000000in}}{\pgfqpoint{-0.000000in}{0.000000in}}{%
\pgfpathmoveto{\pgfqpoint{-0.000000in}{0.000000in}}%
\pgfpathlineto{\pgfqpoint{-0.027778in}{0.000000in}}%
\pgfusepath{stroke,fill}%
}%
\begin{pgfscope}%
\pgfsys@transformshift{0.594525in}{2.148874in}%
\pgfsys@useobject{currentmarker}{}%
\end{pgfscope}%
\end{pgfscope}%
\begin{pgfscope}%
\pgfpathrectangle{\pgfqpoint{0.594525in}{0.417642in}}{\pgfqpoint{3.354228in}{2.055000in}}%
\pgfusepath{clip}%
\pgfsetrectcap%
\pgfsetroundjoin%
\pgfsetlinewidth{0.803000pt}%
\definecolor{currentstroke}{rgb}{0.850000,0.850000,0.850000}%
\pgfsetstrokecolor{currentstroke}%
\pgfsetdash{}{0pt}%
\pgfpathmoveto{\pgfqpoint{0.594525in}{2.183570in}}%
\pgfpathlineto{\pgfqpoint{3.948753in}{2.183570in}}%
\pgfusepath{stroke}%
\end{pgfscope}%
\begin{pgfscope}%
\pgfsetbuttcap%
\pgfsetroundjoin%
\definecolor{currentfill}{rgb}{0.000000,0.000000,0.000000}%
\pgfsetfillcolor{currentfill}%
\pgfsetlinewidth{0.602250pt}%
\definecolor{currentstroke}{rgb}{0.000000,0.000000,0.000000}%
\pgfsetstrokecolor{currentstroke}%
\pgfsetdash{}{0pt}%
\pgfsys@defobject{currentmarker}{\pgfqpoint{-0.027778in}{0.000000in}}{\pgfqpoint{-0.000000in}{0.000000in}}{%
\pgfpathmoveto{\pgfqpoint{-0.000000in}{0.000000in}}%
\pgfpathlineto{\pgfqpoint{-0.027778in}{0.000000in}}%
\pgfusepath{stroke,fill}%
}%
\begin{pgfscope}%
\pgfsys@transformshift{0.594525in}{2.183570in}%
\pgfsys@useobject{currentmarker}{}%
\end{pgfscope}%
\end{pgfscope}%
\begin{pgfscope}%
\pgfpathrectangle{\pgfqpoint{0.594525in}{0.417642in}}{\pgfqpoint{3.354228in}{2.055000in}}%
\pgfusepath{clip}%
\pgfsetrectcap%
\pgfsetroundjoin%
\pgfsetlinewidth{0.803000pt}%
\definecolor{currentstroke}{rgb}{0.850000,0.850000,0.850000}%
\pgfsetstrokecolor{currentstroke}%
\pgfsetdash{}{0pt}%
\pgfpathmoveto{\pgfqpoint{0.594525in}{2.210482in}}%
\pgfpathlineto{\pgfqpoint{3.948753in}{2.210482in}}%
\pgfusepath{stroke}%
\end{pgfscope}%
\begin{pgfscope}%
\pgfsetbuttcap%
\pgfsetroundjoin%
\definecolor{currentfill}{rgb}{0.000000,0.000000,0.000000}%
\pgfsetfillcolor{currentfill}%
\pgfsetlinewidth{0.602250pt}%
\definecolor{currentstroke}{rgb}{0.000000,0.000000,0.000000}%
\pgfsetstrokecolor{currentstroke}%
\pgfsetdash{}{0pt}%
\pgfsys@defobject{currentmarker}{\pgfqpoint{-0.027778in}{0.000000in}}{\pgfqpoint{-0.000000in}{0.000000in}}{%
\pgfpathmoveto{\pgfqpoint{-0.000000in}{0.000000in}}%
\pgfpathlineto{\pgfqpoint{-0.027778in}{0.000000in}}%
\pgfusepath{stroke,fill}%
}%
\begin{pgfscope}%
\pgfsys@transformshift{0.594525in}{2.210482in}%
\pgfsys@useobject{currentmarker}{}%
\end{pgfscope}%
\end{pgfscope}%
\begin{pgfscope}%
\pgfpathrectangle{\pgfqpoint{0.594525in}{0.417642in}}{\pgfqpoint{3.354228in}{2.055000in}}%
\pgfusepath{clip}%
\pgfsetrectcap%
\pgfsetroundjoin%
\pgfsetlinewidth{0.803000pt}%
\definecolor{currentstroke}{rgb}{0.850000,0.850000,0.850000}%
\pgfsetstrokecolor{currentstroke}%
\pgfsetdash{}{0pt}%
\pgfpathmoveto{\pgfqpoint{0.594525in}{2.232471in}}%
\pgfpathlineto{\pgfqpoint{3.948753in}{2.232471in}}%
\pgfusepath{stroke}%
\end{pgfscope}%
\begin{pgfscope}%
\pgfsetbuttcap%
\pgfsetroundjoin%
\definecolor{currentfill}{rgb}{0.000000,0.000000,0.000000}%
\pgfsetfillcolor{currentfill}%
\pgfsetlinewidth{0.602250pt}%
\definecolor{currentstroke}{rgb}{0.000000,0.000000,0.000000}%
\pgfsetstrokecolor{currentstroke}%
\pgfsetdash{}{0pt}%
\pgfsys@defobject{currentmarker}{\pgfqpoint{-0.027778in}{0.000000in}}{\pgfqpoint{-0.000000in}{0.000000in}}{%
\pgfpathmoveto{\pgfqpoint{-0.000000in}{0.000000in}}%
\pgfpathlineto{\pgfqpoint{-0.027778in}{0.000000in}}%
\pgfusepath{stroke,fill}%
}%
\begin{pgfscope}%
\pgfsys@transformshift{0.594525in}{2.232471in}%
\pgfsys@useobject{currentmarker}{}%
\end{pgfscope}%
\end{pgfscope}%
\begin{pgfscope}%
\pgfpathrectangle{\pgfqpoint{0.594525in}{0.417642in}}{\pgfqpoint{3.354228in}{2.055000in}}%
\pgfusepath{clip}%
\pgfsetrectcap%
\pgfsetroundjoin%
\pgfsetlinewidth{0.803000pt}%
\definecolor{currentstroke}{rgb}{0.850000,0.850000,0.850000}%
\pgfsetstrokecolor{currentstroke}%
\pgfsetdash{}{0pt}%
\pgfpathmoveto{\pgfqpoint{0.594525in}{2.251062in}}%
\pgfpathlineto{\pgfqpoint{3.948753in}{2.251062in}}%
\pgfusepath{stroke}%
\end{pgfscope}%
\begin{pgfscope}%
\pgfsetbuttcap%
\pgfsetroundjoin%
\definecolor{currentfill}{rgb}{0.000000,0.000000,0.000000}%
\pgfsetfillcolor{currentfill}%
\pgfsetlinewidth{0.602250pt}%
\definecolor{currentstroke}{rgb}{0.000000,0.000000,0.000000}%
\pgfsetstrokecolor{currentstroke}%
\pgfsetdash{}{0pt}%
\pgfsys@defobject{currentmarker}{\pgfqpoint{-0.027778in}{0.000000in}}{\pgfqpoint{-0.000000in}{0.000000in}}{%
\pgfpathmoveto{\pgfqpoint{-0.000000in}{0.000000in}}%
\pgfpathlineto{\pgfqpoint{-0.027778in}{0.000000in}}%
\pgfusepath{stroke,fill}%
}%
\begin{pgfscope}%
\pgfsys@transformshift{0.594525in}{2.251062in}%
\pgfsys@useobject{currentmarker}{}%
\end{pgfscope}%
\end{pgfscope}%
\begin{pgfscope}%
\pgfpathrectangle{\pgfqpoint{0.594525in}{0.417642in}}{\pgfqpoint{3.354228in}{2.055000in}}%
\pgfusepath{clip}%
\pgfsetrectcap%
\pgfsetroundjoin%
\pgfsetlinewidth{0.803000pt}%
\definecolor{currentstroke}{rgb}{0.850000,0.850000,0.850000}%
\pgfsetstrokecolor{currentstroke}%
\pgfsetdash{}{0pt}%
\pgfpathmoveto{\pgfqpoint{0.594525in}{2.267167in}}%
\pgfpathlineto{\pgfqpoint{3.948753in}{2.267167in}}%
\pgfusepath{stroke}%
\end{pgfscope}%
\begin{pgfscope}%
\pgfsetbuttcap%
\pgfsetroundjoin%
\definecolor{currentfill}{rgb}{0.000000,0.000000,0.000000}%
\pgfsetfillcolor{currentfill}%
\pgfsetlinewidth{0.602250pt}%
\definecolor{currentstroke}{rgb}{0.000000,0.000000,0.000000}%
\pgfsetstrokecolor{currentstroke}%
\pgfsetdash{}{0pt}%
\pgfsys@defobject{currentmarker}{\pgfqpoint{-0.027778in}{0.000000in}}{\pgfqpoint{-0.000000in}{0.000000in}}{%
\pgfpathmoveto{\pgfqpoint{-0.000000in}{0.000000in}}%
\pgfpathlineto{\pgfqpoint{-0.027778in}{0.000000in}}%
\pgfusepath{stroke,fill}%
}%
\begin{pgfscope}%
\pgfsys@transformshift{0.594525in}{2.267167in}%
\pgfsys@useobject{currentmarker}{}%
\end{pgfscope}%
\end{pgfscope}%
\begin{pgfscope}%
\pgfpathrectangle{\pgfqpoint{0.594525in}{0.417642in}}{\pgfqpoint{3.354228in}{2.055000in}}%
\pgfusepath{clip}%
\pgfsetrectcap%
\pgfsetroundjoin%
\pgfsetlinewidth{0.803000pt}%
\definecolor{currentstroke}{rgb}{0.850000,0.850000,0.850000}%
\pgfsetstrokecolor{currentstroke}%
\pgfsetdash{}{0pt}%
\pgfpathmoveto{\pgfqpoint{0.594525in}{2.281372in}}%
\pgfpathlineto{\pgfqpoint{3.948753in}{2.281372in}}%
\pgfusepath{stroke}%
\end{pgfscope}%
\begin{pgfscope}%
\pgfsetbuttcap%
\pgfsetroundjoin%
\definecolor{currentfill}{rgb}{0.000000,0.000000,0.000000}%
\pgfsetfillcolor{currentfill}%
\pgfsetlinewidth{0.602250pt}%
\definecolor{currentstroke}{rgb}{0.000000,0.000000,0.000000}%
\pgfsetstrokecolor{currentstroke}%
\pgfsetdash{}{0pt}%
\pgfsys@defobject{currentmarker}{\pgfqpoint{-0.027778in}{0.000000in}}{\pgfqpoint{-0.000000in}{0.000000in}}{%
\pgfpathmoveto{\pgfqpoint{-0.000000in}{0.000000in}}%
\pgfpathlineto{\pgfqpoint{-0.027778in}{0.000000in}}%
\pgfusepath{stroke,fill}%
}%
\begin{pgfscope}%
\pgfsys@transformshift{0.594525in}{2.281372in}%
\pgfsys@useobject{currentmarker}{}%
\end{pgfscope}%
\end{pgfscope}%
\begin{pgfscope}%
\pgfpathrectangle{\pgfqpoint{0.594525in}{0.417642in}}{\pgfqpoint{3.354228in}{2.055000in}}%
\pgfusepath{clip}%
\pgfsetrectcap%
\pgfsetroundjoin%
\pgfsetlinewidth{0.803000pt}%
\definecolor{currentstroke}{rgb}{0.850000,0.850000,0.850000}%
\pgfsetstrokecolor{currentstroke}%
\pgfsetdash{}{0pt}%
\pgfpathmoveto{\pgfqpoint{0.594525in}{2.377676in}}%
\pgfpathlineto{\pgfqpoint{3.948753in}{2.377676in}}%
\pgfusepath{stroke}%
\end{pgfscope}%
\begin{pgfscope}%
\pgfsetbuttcap%
\pgfsetroundjoin%
\definecolor{currentfill}{rgb}{0.000000,0.000000,0.000000}%
\pgfsetfillcolor{currentfill}%
\pgfsetlinewidth{0.602250pt}%
\definecolor{currentstroke}{rgb}{0.000000,0.000000,0.000000}%
\pgfsetstrokecolor{currentstroke}%
\pgfsetdash{}{0pt}%
\pgfsys@defobject{currentmarker}{\pgfqpoint{-0.027778in}{0.000000in}}{\pgfqpoint{-0.000000in}{0.000000in}}{%
\pgfpathmoveto{\pgfqpoint{-0.000000in}{0.000000in}}%
\pgfpathlineto{\pgfqpoint{-0.027778in}{0.000000in}}%
\pgfusepath{stroke,fill}%
}%
\begin{pgfscope}%
\pgfsys@transformshift{0.594525in}{2.377676in}%
\pgfsys@useobject{currentmarker}{}%
\end{pgfscope}%
\end{pgfscope}%
\begin{pgfscope}%
\pgfpathrectangle{\pgfqpoint{0.594525in}{0.417642in}}{\pgfqpoint{3.354228in}{2.055000in}}%
\pgfusepath{clip}%
\pgfsetrectcap%
\pgfsetroundjoin%
\pgfsetlinewidth{0.803000pt}%
\definecolor{currentstroke}{rgb}{0.850000,0.850000,0.850000}%
\pgfsetstrokecolor{currentstroke}%
\pgfsetdash{}{0pt}%
\pgfpathmoveto{\pgfqpoint{0.594525in}{2.426577in}}%
\pgfpathlineto{\pgfqpoint{3.948753in}{2.426577in}}%
\pgfusepath{stroke}%
\end{pgfscope}%
\begin{pgfscope}%
\pgfsetbuttcap%
\pgfsetroundjoin%
\definecolor{currentfill}{rgb}{0.000000,0.000000,0.000000}%
\pgfsetfillcolor{currentfill}%
\pgfsetlinewidth{0.602250pt}%
\definecolor{currentstroke}{rgb}{0.000000,0.000000,0.000000}%
\pgfsetstrokecolor{currentstroke}%
\pgfsetdash{}{0pt}%
\pgfsys@defobject{currentmarker}{\pgfqpoint{-0.027778in}{0.000000in}}{\pgfqpoint{-0.000000in}{0.000000in}}{%
\pgfpathmoveto{\pgfqpoint{-0.000000in}{0.000000in}}%
\pgfpathlineto{\pgfqpoint{-0.027778in}{0.000000in}}%
\pgfusepath{stroke,fill}%
}%
\begin{pgfscope}%
\pgfsys@transformshift{0.594525in}{2.426577in}%
\pgfsys@useobject{currentmarker}{}%
\end{pgfscope}%
\end{pgfscope}%
\begin{pgfscope}%
\pgfpathrectangle{\pgfqpoint{0.594525in}{0.417642in}}{\pgfqpoint{3.354228in}{2.055000in}}%
\pgfusepath{clip}%
\pgfsetrectcap%
\pgfsetroundjoin%
\pgfsetlinewidth{0.803000pt}%
\definecolor{currentstroke}{rgb}{0.850000,0.850000,0.850000}%
\pgfsetstrokecolor{currentstroke}%
\pgfsetdash{}{0pt}%
\pgfpathmoveto{\pgfqpoint{0.594525in}{2.461273in}}%
\pgfpathlineto{\pgfqpoint{3.948753in}{2.461273in}}%
\pgfusepath{stroke}%
\end{pgfscope}%
\begin{pgfscope}%
\pgfsetbuttcap%
\pgfsetroundjoin%
\definecolor{currentfill}{rgb}{0.000000,0.000000,0.000000}%
\pgfsetfillcolor{currentfill}%
\pgfsetlinewidth{0.602250pt}%
\definecolor{currentstroke}{rgb}{0.000000,0.000000,0.000000}%
\pgfsetstrokecolor{currentstroke}%
\pgfsetdash{}{0pt}%
\pgfsys@defobject{currentmarker}{\pgfqpoint{-0.027778in}{0.000000in}}{\pgfqpoint{-0.000000in}{0.000000in}}{%
\pgfpathmoveto{\pgfqpoint{-0.000000in}{0.000000in}}%
\pgfpathlineto{\pgfqpoint{-0.027778in}{0.000000in}}%
\pgfusepath{stroke,fill}%
}%
\begin{pgfscope}%
\pgfsys@transformshift{0.594525in}{2.461273in}%
\pgfsys@useobject{currentmarker}{}%
\end{pgfscope}%
\end{pgfscope}%
\begin{pgfscope}%
\definecolor{textcolor}{rgb}{0.000000,0.000000,0.000000}%
\pgfsetstrokecolor{textcolor}%
\pgfsetfillcolor{textcolor}%
\pgftext[x=0.185574in,y=1.445142in,,bottom,rotate=90.000000]{\color{textcolor}{\rmfamily\fontsize{10.000000}{12.000000}\selectfont\catcode`\^=\active\def^{\ifmmode\sp\else\^{}\fi}\catcode`\%=\active\def%{\%}$S_y(f)$ in $\unit{1 \per \Hz}$}}%
\end{pgfscope}%
\begin{pgfscope}%
\pgfpathrectangle{\pgfqpoint{0.594525in}{0.417642in}}{\pgfqpoint{3.354228in}{2.055000in}}%
\pgfusepath{clip}%
\pgfsetbuttcap%
\pgfsetroundjoin%
\definecolor{currentfill}{rgb}{0.337255,0.705882,0.913725}%
\pgfsetfillcolor{currentfill}%
\pgfsetlinewidth{1.003750pt}%
\definecolor{currentstroke}{rgb}{0.337255,0.705882,0.913725}%
\pgfsetstrokecolor{currentstroke}%
\pgfsetdash{}{0pt}%
\pgfsys@defobject{currentmarker}{\pgfqpoint{-0.013889in}{-0.013889in}}{\pgfqpoint{0.013889in}{0.013889in}}{%
\pgfpathmoveto{\pgfqpoint{0.000000in}{-0.013889in}}%
\pgfpathcurveto{\pgfqpoint{0.003683in}{-0.013889in}}{\pgfqpoint{0.007216in}{-0.012425in}}{\pgfqpoint{0.009821in}{-0.009821in}}%
\pgfpathcurveto{\pgfqpoint{0.012425in}{-0.007216in}}{\pgfqpoint{0.013889in}{-0.003683in}}{\pgfqpoint{0.013889in}{0.000000in}}%
\pgfpathcurveto{\pgfqpoint{0.013889in}{0.003683in}}{\pgfqpoint{0.012425in}{0.007216in}}{\pgfqpoint{0.009821in}{0.009821in}}%
\pgfpathcurveto{\pgfqpoint{0.007216in}{0.012425in}}{\pgfqpoint{0.003683in}{0.013889in}}{\pgfqpoint{0.000000in}{0.013889in}}%
\pgfpathcurveto{\pgfqpoint{-0.003683in}{0.013889in}}{\pgfqpoint{-0.007216in}{0.012425in}}{\pgfqpoint{-0.009821in}{0.009821in}}%
\pgfpathcurveto{\pgfqpoint{-0.012425in}{0.007216in}}{\pgfqpoint{-0.013889in}{0.003683in}}{\pgfqpoint{-0.013889in}{0.000000in}}%
\pgfpathcurveto{\pgfqpoint{-0.013889in}{-0.003683in}}{\pgfqpoint{-0.012425in}{-0.007216in}}{\pgfqpoint{-0.009821in}{-0.009821in}}%
\pgfpathcurveto{\pgfqpoint{-0.007216in}{-0.012425in}}{\pgfqpoint{-0.003683in}{-0.013889in}}{\pgfqpoint{0.000000in}{-0.013889in}}%
\pgfpathlineto{\pgfqpoint{0.000000in}{-0.013889in}}%
\pgfpathclose%
\pgfusepath{stroke,fill}%
}%
\begin{pgfscope}%
\pgfsys@transformshift{0.746990in}{2.379233in}%
\pgfsys@useobject{currentmarker}{}%
\end{pgfscope}%
\begin{pgfscope}%
\pgfsys@transformshift{0.885756in}{2.243556in}%
\pgfsys@useobject{currentmarker}{}%
\end{pgfscope}%
\begin{pgfscope}%
\pgfsys@transformshift{0.966928in}{2.056793in}%
\pgfsys@useobject{currentmarker}{}%
\end{pgfscope}%
\begin{pgfscope}%
\pgfsys@transformshift{1.024522in}{1.898870in}%
\pgfsys@useobject{currentmarker}{}%
\end{pgfscope}%
\begin{pgfscope}%
\pgfsys@transformshift{1.069194in}{1.856385in}%
\pgfsys@useobject{currentmarker}{}%
\end{pgfscope}%
\begin{pgfscope}%
\pgfsys@transformshift{1.105694in}{1.864235in}%
\pgfsys@useobject{currentmarker}{}%
\end{pgfscope}%
\begin{pgfscope}%
\pgfsys@transformshift{1.136555in}{1.827026in}%
\pgfsys@useobject{currentmarker}{}%
\end{pgfscope}%
\begin{pgfscope}%
\pgfsys@transformshift{1.163287in}{1.826804in}%
\pgfsys@useobject{currentmarker}{}%
\end{pgfscope}%
\begin{pgfscope}%
\pgfsys@transformshift{1.186867in}{1.765936in}%
\pgfsys@useobject{currentmarker}{}%
\end{pgfscope}%
\begin{pgfscope}%
\pgfsys@transformshift{1.207960in}{1.746792in}%
\pgfsys@useobject{currentmarker}{}%
\end{pgfscope}%
\begin{pgfscope}%
\pgfsys@transformshift{1.227041in}{1.790462in}%
\pgfsys@useobject{currentmarker}{}%
\end{pgfscope}%
\begin{pgfscope}%
\pgfsys@transformshift{1.244460in}{1.783902in}%
\pgfsys@useobject{currentmarker}{}%
\end{pgfscope}%
\begin{pgfscope}%
\pgfsys@transformshift{1.260485in}{1.709114in}%
\pgfsys@useobject{currentmarker}{}%
\end{pgfscope}%
\begin{pgfscope}%
\pgfsys@transformshift{1.275321in}{1.606967in}%
\pgfsys@useobject{currentmarker}{}%
\end{pgfscope}%
\begin{pgfscope}%
\pgfsys@transformshift{1.289133in}{1.662951in}%
\pgfsys@useobject{currentmarker}{}%
\end{pgfscope}%
\begin{pgfscope}%
\pgfsys@transformshift{1.302053in}{1.705387in}%
\pgfsys@useobject{currentmarker}{}%
\end{pgfscope}%
\begin{pgfscope}%
\pgfsys@transformshift{1.319993in}{1.683763in}%
\pgfsys@useobject{currentmarker}{}%
\end{pgfscope}%
\begin{pgfscope}%
\pgfsys@transformshift{1.336457in}{1.671174in}%
\pgfsys@useobject{currentmarker}{}%
\end{pgfscope}%
\begin{pgfscope}%
\pgfsys@transformshift{1.351669in}{1.601513in}%
\pgfsys@useobject{currentmarker}{}%
\end{pgfscope}%
\begin{pgfscope}%
\pgfsys@transformshift{1.370306in}{1.543332in}%
\pgfsys@useobject{currentmarker}{}%
\end{pgfscope}%
\begin{pgfscope}%
\pgfsys@transformshift{1.387354in}{1.488294in}%
\pgfsys@useobject{currentmarker}{}%
\end{pgfscope}%
\begin{pgfscope}%
\pgfsys@transformshift{1.403064in}{1.549546in}%
\pgfsys@useobject{currentmarker}{}%
\end{pgfscope}%
\begin{pgfscope}%
\pgfsys@transformshift{1.417630in}{1.606127in}%
\pgfsys@useobject{currentmarker}{}%
\end{pgfscope}%
\begin{pgfscope}%
\pgfsys@transformshift{1.434463in}{1.450020in}%
\pgfsys@useobject{currentmarker}{}%
\end{pgfscope}%
\begin{pgfscope}%
\pgfsys@transformshift{1.449990in}{1.471065in}%
\pgfsys@useobject{currentmarker}{}%
\end{pgfscope}%
\begin{pgfscope}%
\pgfsys@transformshift{1.464399in}{1.531919in}%
\pgfsys@useobject{currentmarker}{}%
\end{pgfscope}%
\begin{pgfscope}%
\pgfsys@transformshift{1.480423in}{1.502359in}%
\pgfsys@useobject{currentmarker}{}%
\end{pgfscope}%
\begin{pgfscope}%
\pgfsys@transformshift{1.497629in}{1.563649in}%
\pgfsys@useobject{currentmarker}{}%
\end{pgfscope}%
\begin{pgfscope}%
\pgfsys@transformshift{1.515636in}{1.502029in}%
\pgfsys@useobject{currentmarker}{}%
\end{pgfscope}%
\begin{pgfscope}%
\pgfsys@transformshift{1.532157in}{1.432523in}%
\pgfsys@useobject{currentmarker}{}%
\end{pgfscope}%
\begin{pgfscope}%
\pgfsys@transformshift{1.547417in}{1.395633in}%
\pgfsys@useobject{currentmarker}{}%
\end{pgfscope}%
\begin{pgfscope}%
\pgfsys@transformshift{1.563300in}{1.416128in}%
\pgfsys@useobject{currentmarker}{}%
\end{pgfscope}%
\begin{pgfscope}%
\pgfsys@transformshift{1.579585in}{1.445891in}%
\pgfsys@useobject{currentmarker}{}%
\end{pgfscope}%
\begin{pgfscope}%
\pgfsys@transformshift{1.596090in}{1.388475in}%
\pgfsys@useobject{currentmarker}{}%
\end{pgfscope}%
\begin{pgfscope}%
\pgfsys@transformshift{1.612668in}{1.422640in}%
\pgfsys@useobject{currentmarker}{}%
\end{pgfscope}%
\begin{pgfscope}%
\pgfsys@transformshift{1.627977in}{1.374793in}%
\pgfsys@useobject{currentmarker}{}%
\end{pgfscope}%
\begin{pgfscope}%
\pgfsys@transformshift{1.644473in}{1.335631in}%
\pgfsys@useobject{currentmarker}{}%
\end{pgfscope}%
\begin{pgfscope}%
\pgfsys@transformshift{1.660758in}{1.377480in}%
\pgfsys@useobject{currentmarker}{}%
\end{pgfscope}%
\begin{pgfscope}%
\pgfsys@transformshift{1.676782in}{1.340806in}%
\pgfsys@useobject{currentmarker}{}%
\end{pgfscope}%
\begin{pgfscope}%
\pgfsys@transformshift{1.693398in}{1.369059in}%
\pgfsys@useobject{currentmarker}{}%
\end{pgfscope}%
\begin{pgfscope}%
\pgfsys@transformshift{1.709559in}{1.325074in}%
\pgfsys@useobject{currentmarker}{}%
\end{pgfscope}%
\begin{pgfscope}%
\pgfsys@transformshift{1.725269in}{1.327570in}%
\pgfsys@useobject{currentmarker}{}%
\end{pgfscope}%
\begin{pgfscope}%
\pgfsys@transformshift{1.741235in}{1.281837in}%
\pgfsys@useobject{currentmarker}{}%
\end{pgfscope}%
\begin{pgfscope}%
\pgfsys@transformshift{1.757955in}{1.302941in}%
\pgfsys@useobject{currentmarker}{}%
\end{pgfscope}%
\begin{pgfscope}%
\pgfsys@transformshift{1.773980in}{1.277292in}%
\pgfsys@useobject{currentmarker}{}%
\end{pgfscope}%
\begin{pgfscope}%
\pgfsys@transformshift{1.789913in}{1.286649in}%
\pgfsys@useobject{currentmarker}{}%
\end{pgfscope}%
\begin{pgfscope}%
\pgfsys@transformshift{1.806189in}{1.243027in}%
\pgfsys@useobject{currentmarker}{}%
\end{pgfscope}%
\begin{pgfscope}%
\pgfsys@transformshift{1.822640in}{1.239388in}%
\pgfsys@useobject{currentmarker}{}%
\end{pgfscope}%
\begin{pgfscope}%
\pgfsys@transformshift{1.839128in}{1.235830in}%
\pgfsys@useobject{currentmarker}{}%
\end{pgfscope}%
\begin{pgfscope}%
\pgfsys@transformshift{1.855153in}{1.222210in}%
\pgfsys@useobject{currentmarker}{}%
\end{pgfscope}%
\begin{pgfscope}%
\pgfsys@transformshift{1.871086in}{1.215613in}%
\pgfsys@useobject{currentmarker}{}%
\end{pgfscope}%
\begin{pgfscope}%
\pgfsys@transformshift{1.887530in}{1.214554in}%
\pgfsys@useobject{currentmarker}{}%
\end{pgfscope}%
\begin{pgfscope}%
\pgfsys@transformshift{1.903968in}{1.206013in}%
\pgfsys@useobject{currentmarker}{}%
\end{pgfscope}%
\begin{pgfscope}%
\pgfsys@transformshift{1.920016in}{1.173425in}%
\pgfsys@useobject{currentmarker}{}%
\end{pgfscope}%
\begin{pgfscope}%
\pgfsys@transformshift{1.936194in}{1.168756in}%
\pgfsys@useobject{currentmarker}{}%
\end{pgfscope}%
\begin{pgfscope}%
\pgfsys@transformshift{1.952380in}{1.169913in}%
\pgfsys@useobject{currentmarker}{}%
\end{pgfscope}%
\begin{pgfscope}%
\pgfsys@transformshift{1.968479in}{1.176104in}%
\pgfsys@useobject{currentmarker}{}%
\end{pgfscope}%
\begin{pgfscope}%
\pgfsys@transformshift{1.984831in}{1.135554in}%
\pgfsys@useobject{currentmarker}{}%
\end{pgfscope}%
\begin{pgfscope}%
\pgfsys@transformshift{2.001093in}{1.116627in}%
\pgfsys@useobject{currentmarker}{}%
\end{pgfscope}%
\begin{pgfscope}%
\pgfsys@transformshift{2.017191in}{1.124437in}%
\pgfsys@useobject{currentmarker}{}%
\end{pgfscope}%
\begin{pgfscope}%
\pgfsys@transformshift{2.033391in}{1.092334in}%
\pgfsys@useobject{currentmarker}{}%
\end{pgfscope}%
\begin{pgfscope}%
\pgfsys@transformshift{2.049727in}{1.111404in}%
\pgfsys@useobject{currentmarker}{}%
\end{pgfscope}%
\begin{pgfscope}%
\pgfsys@transformshift{2.065935in}{1.072411in}%
\pgfsys@useobject{currentmarker}{}%
\end{pgfscope}%
\begin{pgfscope}%
\pgfsys@transformshift{2.082076in}{1.078977in}%
\pgfsys@useobject{currentmarker}{}%
\end{pgfscope}%
\begin{pgfscope}%
\pgfsys@transformshift{2.098305in}{1.064510in}%
\pgfsys@useobject{currentmarker}{}%
\end{pgfscope}%
\begin{pgfscope}%
\pgfsys@transformshift{2.114510in}{1.063636in}%
\pgfsys@useobject{currentmarker}{}%
\end{pgfscope}%
\begin{pgfscope}%
\pgfsys@transformshift{2.130800in}{1.042743in}%
\pgfsys@useobject{currentmarker}{}%
\end{pgfscope}%
\begin{pgfscope}%
\pgfsys@transformshift{2.147062in}{1.052546in}%
\pgfsys@useobject{currentmarker}{}%
\end{pgfscope}%
\begin{pgfscope}%
\pgfsys@transformshift{2.163206in}{1.025750in}%
\pgfsys@useobject{currentmarker}{}%
\end{pgfscope}%
\begin{pgfscope}%
\pgfsys@transformshift{2.179400in}{1.019931in}%
\pgfsys@useobject{currentmarker}{}%
\end{pgfscope}%
\begin{pgfscope}%
\pgfsys@transformshift{2.195683in}{1.013549in}%
\pgfsys@useobject{currentmarker}{}%
\end{pgfscope}%
\begin{pgfscope}%
\pgfsys@transformshift{2.211873in}{1.007561in}%
\pgfsys@useobject{currentmarker}{}%
\end{pgfscope}%
\begin{pgfscope}%
\pgfsys@transformshift{2.228081in}{0.992004in}%
\pgfsys@useobject{currentmarker}{}%
\end{pgfscope}%
\begin{pgfscope}%
\pgfsys@transformshift{2.244323in}{0.978230in}%
\pgfsys@useobject{currentmarker}{}%
\end{pgfscope}%
\begin{pgfscope}%
\pgfsys@transformshift{2.260495in}{0.979707in}%
\pgfsys@useobject{currentmarker}{}%
\end{pgfscope}%
\begin{pgfscope}%
\pgfsys@transformshift{2.276711in}{0.971229in}%
\pgfsys@useobject{currentmarker}{}%
\end{pgfscope}%
\begin{pgfscope}%
\pgfsys@transformshift{2.292957in}{0.958435in}%
\pgfsys@useobject{currentmarker}{}%
\end{pgfscope}%
\begin{pgfscope}%
\pgfsys@transformshift{2.309173in}{0.950074in}%
\pgfsys@useobject{currentmarker}{}%
\end{pgfscope}%
\begin{pgfscope}%
\pgfsys@transformshift{2.325382in}{0.944091in}%
\pgfsys@useobject{currentmarker}{}%
\end{pgfscope}%
\begin{pgfscope}%
\pgfsys@transformshift{2.341598in}{0.930700in}%
\pgfsys@useobject{currentmarker}{}%
\end{pgfscope}%
\begin{pgfscope}%
\pgfsys@transformshift{2.357820in}{0.929578in}%
\pgfsys@useobject{currentmarker}{}%
\end{pgfscope}%
\begin{pgfscope}%
\pgfsys@transformshift{2.374042in}{0.918616in}%
\pgfsys@useobject{currentmarker}{}%
\end{pgfscope}%
\begin{pgfscope}%
\pgfsys@transformshift{2.390250in}{0.917534in}%
\pgfsys@useobject{currentmarker}{}%
\end{pgfscope}%
\begin{pgfscope}%
\pgfsys@transformshift{2.406455in}{0.903755in}%
\pgfsys@useobject{currentmarker}{}%
\end{pgfscope}%
\begin{pgfscope}%
\pgfsys@transformshift{2.422655in}{0.891084in}%
\pgfsys@useobject{currentmarker}{}%
\end{pgfscope}%
\begin{pgfscope}%
\pgfsys@transformshift{2.438865in}{0.889678in}%
\pgfsys@useobject{currentmarker}{}%
\end{pgfscope}%
\begin{pgfscope}%
\pgfsys@transformshift{2.455086in}{0.878679in}%
\pgfsys@useobject{currentmarker}{}%
\end{pgfscope}%
\begin{pgfscope}%
\pgfsys@transformshift{2.471314in}{0.868727in}%
\pgfsys@useobject{currentmarker}{}%
\end{pgfscope}%
\begin{pgfscope}%
\pgfsys@transformshift{2.487535in}{0.865610in}%
\pgfsys@useobject{currentmarker}{}%
\end{pgfscope}%
\begin{pgfscope}%
\pgfsys@transformshift{2.503750in}{0.858415in}%
\pgfsys@useobject{currentmarker}{}%
\end{pgfscope}%
\begin{pgfscope}%
\pgfsys@transformshift{2.519966in}{0.854501in}%
\pgfsys@useobject{currentmarker}{}%
\end{pgfscope}%
\begin{pgfscope}%
\pgfsys@transformshift{2.536180in}{0.842774in}%
\pgfsys@useobject{currentmarker}{}%
\end{pgfscope}%
\begin{pgfscope}%
\pgfsys@transformshift{2.552384in}{0.837419in}%
\pgfsys@useobject{currentmarker}{}%
\end{pgfscope}%
\begin{pgfscope}%
\pgfsys@transformshift{2.568596in}{0.831753in}%
\pgfsys@useobject{currentmarker}{}%
\end{pgfscope}%
\begin{pgfscope}%
\pgfsys@transformshift{2.584815in}{0.823528in}%
\pgfsys@useobject{currentmarker}{}%
\end{pgfscope}%
\begin{pgfscope}%
\pgfsys@transformshift{2.601030in}{0.820128in}%
\pgfsys@useobject{currentmarker}{}%
\end{pgfscope}%
\begin{pgfscope}%
\pgfsys@transformshift{2.617244in}{0.814941in}%
\pgfsys@useobject{currentmarker}{}%
\end{pgfscope}%
\begin{pgfscope}%
\pgfsys@transformshift{2.633459in}{0.807903in}%
\pgfsys@useobject{currentmarker}{}%
\end{pgfscope}%
\begin{pgfscope}%
\pgfsys@transformshift{2.649680in}{0.804096in}%
\pgfsys@useobject{currentmarker}{}%
\end{pgfscope}%
\begin{pgfscope}%
\pgfsys@transformshift{2.665892in}{0.797992in}%
\pgfsys@useobject{currentmarker}{}%
\end{pgfscope}%
\begin{pgfscope}%
\pgfsys@transformshift{2.682104in}{0.793483in}%
\pgfsys@useobject{currentmarker}{}%
\end{pgfscope}%
\begin{pgfscope}%
\pgfsys@transformshift{2.698323in}{0.787330in}%
\pgfsys@useobject{currentmarker}{}%
\end{pgfscope}%
\begin{pgfscope}%
\pgfsys@transformshift{2.714538in}{0.781159in}%
\pgfsys@useobject{currentmarker}{}%
\end{pgfscope}%
\begin{pgfscope}%
\pgfsys@transformshift{2.730751in}{0.781768in}%
\pgfsys@useobject{currentmarker}{}%
\end{pgfscope}%
\begin{pgfscope}%
\pgfsys@transformshift{2.746964in}{0.774018in}%
\pgfsys@useobject{currentmarker}{}%
\end{pgfscope}%
\begin{pgfscope}%
\pgfsys@transformshift{2.763180in}{0.770046in}%
\pgfsys@useobject{currentmarker}{}%
\end{pgfscope}%
\begin{pgfscope}%
\pgfsys@transformshift{2.779398in}{0.765747in}%
\pgfsys@useobject{currentmarker}{}%
\end{pgfscope}%
\begin{pgfscope}%
\pgfsys@transformshift{2.795610in}{0.763610in}%
\pgfsys@useobject{currentmarker}{}%
\end{pgfscope}%
\begin{pgfscope}%
\pgfsys@transformshift{2.811822in}{0.757480in}%
\pgfsys@useobject{currentmarker}{}%
\end{pgfscope}%
\begin{pgfscope}%
\pgfsys@transformshift{2.828038in}{0.754600in}%
\pgfsys@useobject{currentmarker}{}%
\end{pgfscope}%
\begin{pgfscope}%
\pgfsys@transformshift{2.844254in}{0.754444in}%
\pgfsys@useobject{currentmarker}{}%
\end{pgfscope}%
\begin{pgfscope}%
\pgfsys@transformshift{2.860471in}{0.750129in}%
\pgfsys@useobject{currentmarker}{}%
\end{pgfscope}%
\begin{pgfscope}%
\pgfsys@transformshift{2.876686in}{0.748200in}%
\pgfsys@useobject{currentmarker}{}%
\end{pgfscope}%
\begin{pgfscope}%
\pgfsys@transformshift{2.892900in}{0.748109in}%
\pgfsys@useobject{currentmarker}{}%
\end{pgfscope}%
\begin{pgfscope}%
\pgfsys@transformshift{2.909115in}{0.744904in}%
\pgfsys@useobject{currentmarker}{}%
\end{pgfscope}%
\begin{pgfscope}%
\pgfsys@transformshift{2.925330in}{0.740187in}%
\pgfsys@useobject{currentmarker}{}%
\end{pgfscope}%
\begin{pgfscope}%
\pgfsys@transformshift{2.941545in}{0.739881in}%
\pgfsys@useobject{currentmarker}{}%
\end{pgfscope}%
\begin{pgfscope}%
\pgfsys@transformshift{2.957760in}{0.738337in}%
\pgfsys@useobject{currentmarker}{}%
\end{pgfscope}%
\begin{pgfscope}%
\pgfsys@transformshift{2.973976in}{0.735816in}%
\pgfsys@useobject{currentmarker}{}%
\end{pgfscope}%
\begin{pgfscope}%
\pgfsys@transformshift{2.990191in}{0.732653in}%
\pgfsys@useobject{currentmarker}{}%
\end{pgfscope}%
\begin{pgfscope}%
\pgfsys@transformshift{3.006406in}{0.732006in}%
\pgfsys@useobject{currentmarker}{}%
\end{pgfscope}%
\begin{pgfscope}%
\pgfsys@transformshift{3.022621in}{0.730071in}%
\pgfsys@useobject{currentmarker}{}%
\end{pgfscope}%
\begin{pgfscope}%
\pgfsys@transformshift{3.038835in}{0.729087in}%
\pgfsys@useobject{currentmarker}{}%
\end{pgfscope}%
\begin{pgfscope}%
\pgfsys@transformshift{3.055050in}{0.728537in}%
\pgfsys@useobject{currentmarker}{}%
\end{pgfscope}%
\begin{pgfscope}%
\pgfsys@transformshift{3.071266in}{0.726789in}%
\pgfsys@useobject{currentmarker}{}%
\end{pgfscope}%
\begin{pgfscope}%
\pgfsys@transformshift{3.087480in}{0.725607in}%
\pgfsys@useobject{currentmarker}{}%
\end{pgfscope}%
\begin{pgfscope}%
\pgfsys@transformshift{3.103696in}{0.725214in}%
\pgfsys@useobject{currentmarker}{}%
\end{pgfscope}%
\begin{pgfscope}%
\pgfsys@transformshift{3.119911in}{0.723874in}%
\pgfsys@useobject{currentmarker}{}%
\end{pgfscope}%
\begin{pgfscope}%
\pgfsys@transformshift{3.136126in}{0.723253in}%
\pgfsys@useobject{currentmarker}{}%
\end{pgfscope}%
\begin{pgfscope}%
\pgfsys@transformshift{3.152341in}{0.721667in}%
\pgfsys@useobject{currentmarker}{}%
\end{pgfscope}%
\begin{pgfscope}%
\pgfsys@transformshift{3.168556in}{0.721789in}%
\pgfsys@useobject{currentmarker}{}%
\end{pgfscope}%
\begin{pgfscope}%
\pgfsys@transformshift{3.184771in}{0.719817in}%
\pgfsys@useobject{currentmarker}{}%
\end{pgfscope}%
\begin{pgfscope}%
\pgfsys@transformshift{3.200986in}{0.719872in}%
\pgfsys@useobject{currentmarker}{}%
\end{pgfscope}%
\begin{pgfscope}%
\pgfsys@transformshift{3.217201in}{0.719614in}%
\pgfsys@useobject{currentmarker}{}%
\end{pgfscope}%
\begin{pgfscope}%
\pgfsys@transformshift{3.233416in}{0.718734in}%
\pgfsys@useobject{currentmarker}{}%
\end{pgfscope}%
\begin{pgfscope}%
\pgfsys@transformshift{3.249631in}{0.717738in}%
\pgfsys@useobject{currentmarker}{}%
\end{pgfscope}%
\begin{pgfscope}%
\pgfsys@transformshift{3.265846in}{0.717815in}%
\pgfsys@useobject{currentmarker}{}%
\end{pgfscope}%
\begin{pgfscope}%
\pgfsys@transformshift{3.282061in}{0.716725in}%
\pgfsys@useobject{currentmarker}{}%
\end{pgfscope}%
\begin{pgfscope}%
\pgfsys@transformshift{3.298276in}{0.716712in}%
\pgfsys@useobject{currentmarker}{}%
\end{pgfscope}%
\begin{pgfscope}%
\pgfsys@transformshift{3.314491in}{0.716562in}%
\pgfsys@useobject{currentmarker}{}%
\end{pgfscope}%
\begin{pgfscope}%
\pgfsys@transformshift{3.330706in}{0.716467in}%
\pgfsys@useobject{currentmarker}{}%
\end{pgfscope}%
\begin{pgfscope}%
\pgfsys@transformshift{3.346922in}{0.716088in}%
\pgfsys@useobject{currentmarker}{}%
\end{pgfscope}%
\begin{pgfscope}%
\pgfsys@transformshift{3.363136in}{0.714802in}%
\pgfsys@useobject{currentmarker}{}%
\end{pgfscope}%
\begin{pgfscope}%
\pgfsys@transformshift{3.379351in}{0.714805in}%
\pgfsys@useobject{currentmarker}{}%
\end{pgfscope}%
\begin{pgfscope}%
\pgfsys@transformshift{3.395567in}{0.714392in}%
\pgfsys@useobject{currentmarker}{}%
\end{pgfscope}%
\begin{pgfscope}%
\pgfsys@transformshift{3.411782in}{0.714454in}%
\pgfsys@useobject{currentmarker}{}%
\end{pgfscope}%
\begin{pgfscope}%
\pgfsys@transformshift{3.427997in}{0.713648in}%
\pgfsys@useobject{currentmarker}{}%
\end{pgfscope}%
\begin{pgfscope}%
\pgfsys@transformshift{3.444212in}{0.713892in}%
\pgfsys@useobject{currentmarker}{}%
\end{pgfscope}%
\begin{pgfscope}%
\pgfsys@transformshift{3.460427in}{0.713522in}%
\pgfsys@useobject{currentmarker}{}%
\end{pgfscope}%
\begin{pgfscope}%
\pgfsys@transformshift{3.476642in}{0.713406in}%
\pgfsys@useobject{currentmarker}{}%
\end{pgfscope}%
\begin{pgfscope}%
\pgfsys@transformshift{3.492857in}{0.713361in}%
\pgfsys@useobject{currentmarker}{}%
\end{pgfscope}%
\begin{pgfscope}%
\pgfsys@transformshift{3.509072in}{0.713372in}%
\pgfsys@useobject{currentmarker}{}%
\end{pgfscope}%
\begin{pgfscope}%
\pgfsys@transformshift{3.525287in}{0.713442in}%
\pgfsys@useobject{currentmarker}{}%
\end{pgfscope}%
\begin{pgfscope}%
\pgfsys@transformshift{3.541502in}{0.713218in}%
\pgfsys@useobject{currentmarker}{}%
\end{pgfscope}%
\begin{pgfscope}%
\pgfsys@transformshift{3.557717in}{0.713438in}%
\pgfsys@useobject{currentmarker}{}%
\end{pgfscope}%
\begin{pgfscope}%
\pgfsys@transformshift{3.573932in}{0.712779in}%
\pgfsys@useobject{currentmarker}{}%
\end{pgfscope}%
\begin{pgfscope}%
\pgfsys@transformshift{3.590147in}{0.712848in}%
\pgfsys@useobject{currentmarker}{}%
\end{pgfscope}%
\begin{pgfscope}%
\pgfsys@transformshift{3.606362in}{0.712807in}%
\pgfsys@useobject{currentmarker}{}%
\end{pgfscope}%
\begin{pgfscope}%
\pgfsys@transformshift{3.622577in}{0.712727in}%
\pgfsys@useobject{currentmarker}{}%
\end{pgfscope}%
\begin{pgfscope}%
\pgfsys@transformshift{3.638792in}{0.712649in}%
\pgfsys@useobject{currentmarker}{}%
\end{pgfscope}%
\begin{pgfscope}%
\pgfsys@transformshift{3.655007in}{0.712589in}%
\pgfsys@useobject{currentmarker}{}%
\end{pgfscope}%
\begin{pgfscope}%
\pgfsys@transformshift{3.671222in}{0.712303in}%
\pgfsys@useobject{currentmarker}{}%
\end{pgfscope}%
\begin{pgfscope}%
\pgfsys@transformshift{3.687437in}{0.712575in}%
\pgfsys@useobject{currentmarker}{}%
\end{pgfscope}%
\begin{pgfscope}%
\pgfsys@transformshift{3.703652in}{0.712411in}%
\pgfsys@useobject{currentmarker}{}%
\end{pgfscope}%
\begin{pgfscope}%
\pgfsys@transformshift{3.719867in}{0.712464in}%
\pgfsys@useobject{currentmarker}{}%
\end{pgfscope}%
\begin{pgfscope}%
\pgfsys@transformshift{3.736082in}{0.712341in}%
\pgfsys@useobject{currentmarker}{}%
\end{pgfscope}%
\begin{pgfscope}%
\pgfsys@transformshift{3.752297in}{0.712363in}%
\pgfsys@useobject{currentmarker}{}%
\end{pgfscope}%
\begin{pgfscope}%
\pgfsys@transformshift{3.768513in}{0.712220in}%
\pgfsys@useobject{currentmarker}{}%
\end{pgfscope}%
\begin{pgfscope}%
\pgfsys@transformshift{3.784728in}{0.712177in}%
\pgfsys@useobject{currentmarker}{}%
\end{pgfscope}%
\begin{pgfscope}%
\pgfsys@transformshift{3.796288in}{0.711977in}%
\pgfsys@useobject{currentmarker}{}%
\end{pgfscope}%
\end{pgfscope}%
\begin{pgfscope}%
\pgfpathrectangle{\pgfqpoint{0.594525in}{0.417642in}}{\pgfqpoint{3.354228in}{2.055000in}}%
\pgfusepath{clip}%
\pgfsetbuttcap%
\pgfsetroundjoin%
\pgfsetlinewidth{1.505625pt}%
\definecolor{currentstroke}{rgb}{0.003922,0.450980,0.698039}%
\pgfsetstrokecolor{currentstroke}%
\pgfsetdash{{5.550000pt}{2.400000pt}}{0.000000pt}%
\pgfpathmoveto{\pgfqpoint{0.746990in}{0.711458in}}%
\pgfpathlineto{\pgfqpoint{3.796288in}{0.711458in}}%
\pgfpathlineto{\pgfqpoint{3.796288in}{0.711458in}}%
\pgfusepath{stroke}%
\end{pgfscope}%
\begin{pgfscope}%
\pgfpathrectangle{\pgfqpoint{0.594525in}{0.417642in}}{\pgfqpoint{3.354228in}{2.055000in}}%
\pgfusepath{clip}%
\pgfsetbuttcap%
\pgfsetroundjoin%
\pgfsetlinewidth{1.505625pt}%
\definecolor{currentstroke}{rgb}{0.007843,0.619608,0.450980}%
\pgfsetstrokecolor{currentstroke}%
\pgfsetdash{{5.550000pt}{2.400000pt}}{0.000000pt}%
\pgfpathmoveto{\pgfqpoint{0.746990in}{1.872194in}}%
\pgfpathlineto{\pgfqpoint{0.885756in}{1.788597in}}%
\pgfpathlineto{\pgfqpoint{0.966928in}{1.739696in}}%
\pgfpathlineto{\pgfqpoint{1.024522in}{1.705000in}}%
\pgfpathlineto{\pgfqpoint{1.069194in}{1.678088in}}%
\pgfpathlineto{\pgfqpoint{1.105694in}{1.656099in}}%
\pgfpathlineto{\pgfqpoint{1.136555in}{1.637508in}}%
\pgfpathlineto{\pgfqpoint{1.163287in}{1.621403in}}%
\pgfpathlineto{\pgfqpoint{1.186867in}{1.607198in}}%
\pgfpathlineto{\pgfqpoint{1.207960in}{1.594491in}}%
\pgfpathlineto{\pgfqpoint{1.227041in}{1.582996in}}%
\pgfpathlineto{\pgfqpoint{1.244460in}{1.572502in}}%
\pgfpathlineto{\pgfqpoint{1.260485in}{1.562849in}}%
\pgfpathlineto{\pgfqpoint{1.275321in}{1.553911in}}%
\pgfpathlineto{\pgfqpoint{1.289133in}{1.545590in}}%
\pgfpathlineto{\pgfqpoint{1.302053in}{1.537806in}}%
\pgfpathlineto{\pgfqpoint{1.319993in}{1.526999in}}%
\pgfpathlineto{\pgfqpoint{1.336457in}{1.517080in}}%
\pgfpathlineto{\pgfqpoint{1.351669in}{1.507916in}}%
\pgfpathlineto{\pgfqpoint{1.370306in}{1.496689in}}%
\pgfpathlineto{\pgfqpoint{1.387354in}{1.486418in}}%
\pgfpathlineto{\pgfqpoint{1.403064in}{1.476954in}}%
\pgfpathlineto{\pgfqpoint{1.417630in}{1.468179in}}%
\pgfpathlineto{\pgfqpoint{1.434463in}{1.458038in}}%
\pgfpathlineto{\pgfqpoint{1.449990in}{1.448684in}}%
\pgfpathlineto{\pgfqpoint{1.464399in}{1.440004in}}%
\pgfpathlineto{\pgfqpoint{1.480423in}{1.430351in}}%
\pgfpathlineto{\pgfqpoint{1.497629in}{1.419985in}}%
\pgfpathlineto{\pgfqpoint{1.515636in}{1.409137in}}%
\pgfpathlineto{\pgfqpoint{1.532157in}{1.399185in}}%
\pgfpathlineto{\pgfqpoint{1.547417in}{1.389991in}}%
\pgfpathlineto{\pgfqpoint{1.563300in}{1.380423in}}%
\pgfpathlineto{\pgfqpoint{1.579585in}{1.370612in}}%
\pgfpathlineto{\pgfqpoint{1.596090in}{1.360669in}}%
\pgfpathlineto{\pgfqpoint{1.612668in}{1.350682in}}%
\pgfpathlineto{\pgfqpoint{1.627977in}{1.341460in}}%
\pgfpathlineto{\pgfqpoint{1.644473in}{1.331522in}}%
\pgfpathlineto{\pgfqpoint{1.660758in}{1.321711in}}%
\pgfpathlineto{\pgfqpoint{1.676782in}{1.312058in}}%
\pgfpathlineto{\pgfqpoint{1.693398in}{1.302048in}}%
\pgfpathlineto{\pgfqpoint{1.709559in}{1.292312in}}%
\pgfpathlineto{\pgfqpoint{1.725269in}{1.282848in}}%
\pgfpathlineto{\pgfqpoint{1.741235in}{1.273230in}}%
\pgfpathlineto{\pgfqpoint{1.757955in}{1.263157in}}%
\pgfpathlineto{\pgfqpoint{1.773980in}{1.253503in}}%
\pgfpathlineto{\pgfqpoint{1.789913in}{1.243904in}}%
\pgfpathlineto{\pgfqpoint{1.806189in}{1.234099in}}%
\pgfpathlineto{\pgfqpoint{1.822640in}{1.224189in}}%
\pgfpathlineto{\pgfqpoint{1.839128in}{1.214256in}}%
\pgfpathlineto{\pgfqpoint{1.855153in}{1.204602in}}%
\pgfpathlineto{\pgfqpoint{1.871086in}{1.195003in}}%
\pgfpathlineto{\pgfqpoint{1.887530in}{1.185097in}}%
\pgfpathlineto{\pgfqpoint{1.903968in}{1.175194in}}%
\pgfpathlineto{\pgfqpoint{1.920016in}{1.165526in}}%
\pgfpathlineto{\pgfqpoint{1.936194in}{1.155780in}}%
\pgfpathlineto{\pgfqpoint{1.952380in}{1.146029in}}%
\pgfpathlineto{\pgfqpoint{1.968479in}{1.136331in}}%
\pgfpathlineto{\pgfqpoint{1.984831in}{1.126480in}}%
\pgfpathlineto{\pgfqpoint{2.001093in}{1.116683in}}%
\pgfpathlineto{\pgfqpoint{2.017191in}{1.106985in}}%
\pgfpathlineto{\pgfqpoint{2.033391in}{1.097226in}}%
\pgfpathlineto{\pgfqpoint{2.049727in}{1.087384in}}%
\pgfpathlineto{\pgfqpoint{2.065935in}{1.077620in}}%
\pgfpathlineto{\pgfqpoint{2.082076in}{1.067896in}}%
\pgfpathlineto{\pgfqpoint{2.098305in}{1.058119in}}%
\pgfpathlineto{\pgfqpoint{2.114510in}{1.048357in}}%
\pgfpathlineto{\pgfqpoint{2.130800in}{1.038543in}}%
\pgfpathlineto{\pgfqpoint{2.147062in}{1.028747in}}%
\pgfpathlineto{\pgfqpoint{2.163206in}{1.019021in}}%
\pgfpathlineto{\pgfqpoint{2.179400in}{1.009265in}}%
\pgfpathlineto{\pgfqpoint{2.195683in}{0.999456in}}%
\pgfpathlineto{\pgfqpoint{2.211873in}{0.989702in}}%
\pgfpathlineto{\pgfqpoint{2.228081in}{0.979938in}}%
\pgfpathlineto{\pgfqpoint{2.244323in}{0.970154in}}%
\pgfpathlineto{\pgfqpoint{2.260495in}{0.960411in}}%
\pgfpathlineto{\pgfqpoint{2.276711in}{0.950642in}}%
\pgfpathlineto{\pgfqpoint{2.292957in}{0.940855in}}%
\pgfpathlineto{\pgfqpoint{2.309173in}{0.931086in}}%
\pgfpathlineto{\pgfqpoint{2.325382in}{0.921321in}}%
\pgfpathlineto{\pgfqpoint{2.341598in}{0.911552in}}%
\pgfpathlineto{\pgfqpoint{2.357820in}{0.901779in}}%
\pgfpathlineto{\pgfqpoint{2.374042in}{0.892007in}}%
\pgfpathlineto{\pgfqpoint{2.390250in}{0.882243in}}%
\pgfpathlineto{\pgfqpoint{2.406455in}{0.872480in}}%
\pgfpathlineto{\pgfqpoint{2.422655in}{0.862721in}}%
\pgfpathlineto{\pgfqpoint{2.438865in}{0.852956in}}%
\pgfpathlineto{\pgfqpoint{2.455086in}{0.843183in}}%
\pgfpathlineto{\pgfqpoint{2.471314in}{0.833407in}}%
\pgfpathlineto{\pgfqpoint{2.487535in}{0.823635in}}%
\pgfpathlineto{\pgfqpoint{2.503750in}{0.813866in}}%
\pgfpathlineto{\pgfqpoint{2.519966in}{0.804098in}}%
\pgfpathlineto{\pgfqpoint{2.536180in}{0.794330in}}%
\pgfpathlineto{\pgfqpoint{2.552384in}{0.784568in}}%
\pgfpathlineto{\pgfqpoint{2.568596in}{0.774801in}}%
\pgfpathlineto{\pgfqpoint{2.584815in}{0.765031in}}%
\pgfpathlineto{\pgfqpoint{2.601030in}{0.755262in}}%
\pgfpathlineto{\pgfqpoint{2.617244in}{0.745495in}}%
\pgfpathlineto{\pgfqpoint{2.633459in}{0.735726in}}%
\pgfpathlineto{\pgfqpoint{2.649680in}{0.725954in}}%
\pgfpathlineto{\pgfqpoint{2.665892in}{0.716188in}}%
\pgfpathlineto{\pgfqpoint{2.682104in}{0.706421in}}%
\pgfpathlineto{\pgfqpoint{2.698323in}{0.696650in}}%
\pgfpathlineto{\pgfqpoint{2.714538in}{0.686881in}}%
\pgfpathlineto{\pgfqpoint{2.730751in}{0.677114in}}%
\pgfpathlineto{\pgfqpoint{2.746964in}{0.667347in}}%
\pgfpathlineto{\pgfqpoint{2.763180in}{0.657578in}}%
\pgfpathlineto{\pgfqpoint{2.779398in}{0.647808in}}%
\pgfpathlineto{\pgfqpoint{2.795610in}{0.638041in}}%
\pgfpathlineto{\pgfqpoint{2.811822in}{0.628274in}}%
\pgfpathlineto{\pgfqpoint{2.828038in}{0.618505in}}%
\pgfpathlineto{\pgfqpoint{2.844254in}{0.608737in}}%
\pgfpathlineto{\pgfqpoint{2.860471in}{0.598967in}}%
\pgfpathlineto{\pgfqpoint{2.876686in}{0.589199in}}%
\pgfpathlineto{\pgfqpoint{2.892900in}{0.579431in}}%
\pgfpathlineto{\pgfqpoint{2.909115in}{0.569662in}}%
\pgfpathlineto{\pgfqpoint{2.925330in}{0.559894in}}%
\pgfpathlineto{\pgfqpoint{2.941545in}{0.550126in}}%
\pgfpathlineto{\pgfqpoint{2.957760in}{0.540357in}}%
\pgfpathlineto{\pgfqpoint{2.973976in}{0.530588in}}%
\pgfpathlineto{\pgfqpoint{2.990191in}{0.520820in}}%
\pgfpathlineto{\pgfqpoint{3.006406in}{0.511051in}}%
\pgfusepath{stroke}%
\end{pgfscope}%
\begin{pgfscope}%
\pgfpathrectangle{\pgfqpoint{0.594525in}{0.417642in}}{\pgfqpoint{3.354228in}{2.055000in}}%
\pgfusepath{clip}%
\pgfsetbuttcap%
\pgfsetroundjoin%
\pgfsetlinewidth{1.505625pt}%
\definecolor{currentstroke}{rgb}{0.835294,0.368627,0.000000}%
\pgfsetstrokecolor{currentstroke}%
\pgfsetdash{{5.550000pt}{2.400000pt}}{0.000000pt}%
\pgfpathmoveto{\pgfqpoint{0.746990in}{2.302180in}}%
\pgfpathlineto{\pgfqpoint{0.885756in}{2.134986in}}%
\pgfpathlineto{\pgfqpoint{0.966928in}{2.037184in}}%
\pgfpathlineto{\pgfqpoint{1.024522in}{1.967792in}}%
\pgfpathlineto{\pgfqpoint{1.069194in}{1.913968in}}%
\pgfpathlineto{\pgfqpoint{1.105694in}{1.869990in}}%
\pgfpathlineto{\pgfqpoint{1.136555in}{1.832808in}}%
\pgfpathlineto{\pgfqpoint{1.163287in}{1.800598in}}%
\pgfpathlineto{\pgfqpoint{1.186867in}{1.772188in}}%
\pgfpathlineto{\pgfqpoint{1.207960in}{1.746774in}}%
\pgfpathlineto{\pgfqpoint{1.227041in}{1.723784in}}%
\pgfpathlineto{\pgfqpoint{1.244460in}{1.702796in}}%
\pgfpathlineto{\pgfqpoint{1.260485in}{1.683489in}}%
\pgfpathlineto{\pgfqpoint{1.275321in}{1.665614in}}%
\pgfpathlineto{\pgfqpoint{1.289133in}{1.648972in}}%
\pgfpathlineto{\pgfqpoint{1.302053in}{1.633405in}}%
\pgfpathlineto{\pgfqpoint{1.319993in}{1.611789in}}%
\pgfpathlineto{\pgfqpoint{1.336457in}{1.591953in}}%
\pgfpathlineto{\pgfqpoint{1.351669in}{1.573624in}}%
\pgfpathlineto{\pgfqpoint{1.370306in}{1.551170in}}%
\pgfpathlineto{\pgfqpoint{1.387354in}{1.530629in}}%
\pgfpathlineto{\pgfqpoint{1.403064in}{1.511701in}}%
\pgfpathlineto{\pgfqpoint{1.417630in}{1.494150in}}%
\pgfpathlineto{\pgfqpoint{1.434463in}{1.473869in}}%
\pgfpathlineto{\pgfqpoint{1.449990in}{1.455161in}}%
\pgfpathlineto{\pgfqpoint{1.464399in}{1.437800in}}%
\pgfpathlineto{\pgfqpoint{1.480423in}{1.418493in}}%
\pgfpathlineto{\pgfqpoint{1.497629in}{1.397763in}}%
\pgfpathlineto{\pgfqpoint{1.515636in}{1.376067in}}%
\pgfpathlineto{\pgfqpoint{1.532157in}{1.356162in}}%
\pgfpathlineto{\pgfqpoint{1.547417in}{1.337775in}}%
\pgfpathlineto{\pgfqpoint{1.563300in}{1.318638in}}%
\pgfpathlineto{\pgfqpoint{1.579585in}{1.299017in}}%
\pgfpathlineto{\pgfqpoint{1.596090in}{1.279131in}}%
\pgfpathlineto{\pgfqpoint{1.612668in}{1.259157in}}%
\pgfpathlineto{\pgfqpoint{1.627977in}{1.240712in}}%
\pgfpathlineto{\pgfqpoint{1.644473in}{1.220836in}}%
\pgfpathlineto{\pgfqpoint{1.660758in}{1.201215in}}%
\pgfpathlineto{\pgfqpoint{1.676782in}{1.181908in}}%
\pgfpathlineto{\pgfqpoint{1.693398in}{1.161888in}}%
\pgfpathlineto{\pgfqpoint{1.709559in}{1.142417in}}%
\pgfpathlineto{\pgfqpoint{1.725269in}{1.123489in}}%
\pgfpathlineto{\pgfqpoint{1.741235in}{1.104252in}}%
\pgfpathlineto{\pgfqpoint{1.757955in}{1.084105in}}%
\pgfpathlineto{\pgfqpoint{1.773980in}{1.064798in}}%
\pgfpathlineto{\pgfqpoint{1.789913in}{1.045601in}}%
\pgfpathlineto{\pgfqpoint{1.806189in}{1.025990in}}%
\pgfpathlineto{\pgfqpoint{1.822640in}{1.006169in}}%
\pgfpathlineto{\pgfqpoint{1.839128in}{0.986303in}}%
\pgfpathlineto{\pgfqpoint{1.855153in}{0.966996in}}%
\pgfpathlineto{\pgfqpoint{1.871086in}{0.947799in}}%
\pgfpathlineto{\pgfqpoint{1.887530in}{0.927985in}}%
\pgfpathlineto{\pgfqpoint{1.903968in}{0.908181in}}%
\pgfpathlineto{\pgfqpoint{1.920016in}{0.888845in}}%
\pgfpathlineto{\pgfqpoint{1.936194in}{0.869353in}}%
\pgfpathlineto{\pgfqpoint{1.952380in}{0.849850in}}%
\pgfpathlineto{\pgfqpoint{1.968479in}{0.830453in}}%
\pgfpathlineto{\pgfqpoint{1.984831in}{0.810752in}}%
\pgfpathlineto{\pgfqpoint{2.001093in}{0.791158in}}%
\pgfpathlineto{\pgfqpoint{2.017191in}{0.771762in}}%
\pgfpathlineto{\pgfqpoint{2.033391in}{0.752243in}}%
\pgfpathlineto{\pgfqpoint{2.049727in}{0.732561in}}%
\pgfpathlineto{\pgfqpoint{2.065935in}{0.713033in}}%
\pgfpathlineto{\pgfqpoint{2.082076in}{0.693585in}}%
\pgfusepath{stroke}%
\end{pgfscope}%
\begin{pgfscope}%
\pgfsetrectcap%
\pgfsetmiterjoin%
\pgfsetlinewidth{0.803000pt}%
\definecolor{currentstroke}{rgb}{0.000000,0.000000,0.000000}%
\pgfsetstrokecolor{currentstroke}%
\pgfsetdash{}{0pt}%
\pgfpathmoveto{\pgfqpoint{0.594525in}{0.417642in}}%
\pgfpathlineto{\pgfqpoint{0.594525in}{2.472642in}}%
\pgfusepath{stroke}%
\end{pgfscope}%
\begin{pgfscope}%
\pgfsetrectcap%
\pgfsetmiterjoin%
\pgfsetlinewidth{0.803000pt}%
\definecolor{currentstroke}{rgb}{0.000000,0.000000,0.000000}%
\pgfsetstrokecolor{currentstroke}%
\pgfsetdash{}{0pt}%
\pgfpathmoveto{\pgfqpoint{3.948753in}{0.417642in}}%
\pgfpathlineto{\pgfqpoint{3.948753in}{2.472642in}}%
\pgfusepath{stroke}%
\end{pgfscope}%
\begin{pgfscope}%
\pgfsetrectcap%
\pgfsetmiterjoin%
\pgfsetlinewidth{0.803000pt}%
\definecolor{currentstroke}{rgb}{0.000000,0.000000,0.000000}%
\pgfsetstrokecolor{currentstroke}%
\pgfsetdash{}{0pt}%
\pgfpathmoveto{\pgfqpoint{0.594525in}{0.417642in}}%
\pgfpathlineto{\pgfqpoint{3.948753in}{0.417642in}}%
\pgfusepath{stroke}%
\end{pgfscope}%
\begin{pgfscope}%
\pgfsetrectcap%
\pgfsetmiterjoin%
\pgfsetlinewidth{0.803000pt}%
\definecolor{currentstroke}{rgb}{0.000000,0.000000,0.000000}%
\pgfsetstrokecolor{currentstroke}%
\pgfsetdash{}{0pt}%
\pgfpathmoveto{\pgfqpoint{0.594525in}{2.472642in}}%
\pgfpathlineto{\pgfqpoint{3.948753in}{2.472642in}}%
\pgfusepath{stroke}%
\end{pgfscope}%
\begin{pgfscope}%
\pgfsetbuttcap%
\pgfsetmiterjoin%
\definecolor{currentfill}{rgb}{1.000000,1.000000,1.000000}%
\pgfsetfillcolor{currentfill}%
\pgfsetfillopacity{0.800000}%
\pgfsetlinewidth{1.003750pt}%
\definecolor{currentstroke}{rgb}{0.800000,0.800000,0.800000}%
\pgfsetstrokecolor{currentstroke}%
\pgfsetstrokeopacity{0.800000}%
\pgfsetdash{}{0pt}%
\pgfpathmoveto{\pgfqpoint{2.353194in}{1.841887in}}%
\pgfpathlineto{\pgfqpoint{3.870975in}{1.841887in}}%
\pgfpathquadraticcurveto{\pgfqpoint{3.893197in}{1.841887in}}{\pgfqpoint{3.893197in}{1.864109in}}%
\pgfpathlineto{\pgfqpoint{3.893197in}{2.394864in}}%
\pgfpathquadraticcurveto{\pgfqpoint{3.893197in}{2.417086in}}{\pgfqpoint{3.870975in}{2.417086in}}%
\pgfpathlineto{\pgfqpoint{2.353194in}{2.417086in}}%
\pgfpathquadraticcurveto{\pgfqpoint{2.330972in}{2.417086in}}{\pgfqpoint{2.330972in}{2.394864in}}%
\pgfpathlineto{\pgfqpoint{2.330972in}{1.864109in}}%
\pgfpathquadraticcurveto{\pgfqpoint{2.330972in}{1.841887in}}{\pgfqpoint{2.353194in}{1.841887in}}%
\pgfpathlineto{\pgfqpoint{2.353194in}{1.841887in}}%
\pgfpathclose%
\pgfusepath{stroke,fill}%
\end{pgfscope}%
\begin{pgfscope}%
\pgfsetbuttcap%
\pgfsetroundjoin%
\pgfsetlinewidth{1.505625pt}%
\definecolor{currentstroke}{rgb}{0.003922,0.450980,0.698039}%
\pgfsetstrokecolor{currentstroke}%
\pgfsetdash{{5.550000pt}{2.400000pt}}{0.000000pt}%
\pgfpathmoveto{\pgfqpoint{2.375417in}{2.311669in}}%
\pgfpathlineto{\pgfqpoint{2.486528in}{2.311669in}}%
\pgfpathlineto{\pgfqpoint{2.597639in}{2.311669in}}%
\pgfusepath{stroke}%
\end{pgfscope}%
\begin{pgfscope}%
\definecolor{textcolor}{rgb}{0.000000,0.000000,0.000000}%
\pgfsetstrokecolor{textcolor}%
\pgfsetfillcolor{textcolor}%
\pgftext[x=2.686528in,y=2.272781in,left,base]{\color{textcolor}{\rmfamily\fontsize{8.000000}{9.600000}\selectfont\catcode`\^=\active\def^{\ifmmode\sp\else\^{}\fi}\catcode`\%=\active\def%{\%}White noise $h_{0}f^{0}$}}%
\end{pgfscope}%
\begin{pgfscope}%
\pgfsetbuttcap%
\pgfsetroundjoin%
\pgfsetlinewidth{1.505625pt}%
\definecolor{currentstroke}{rgb}{0.007843,0.619608,0.450980}%
\pgfsetstrokecolor{currentstroke}%
\pgfsetdash{{5.550000pt}{2.400000pt}}{0.000000pt}%
\pgfpathmoveto{\pgfqpoint{2.375417in}{2.134648in}}%
\pgfpathlineto{\pgfqpoint{2.486528in}{2.134648in}}%
\pgfpathlineto{\pgfqpoint{2.597639in}{2.134648in}}%
\pgfusepath{stroke}%
\end{pgfscope}%
\begin{pgfscope}%
\definecolor{textcolor}{rgb}{0.000000,0.000000,0.000000}%
\pgfsetstrokecolor{textcolor}%
\pgfsetfillcolor{textcolor}%
\pgftext[x=2.686528in,y=2.095759in,left,base]{\color{textcolor}{\rmfamily\fontsize{8.000000}{9.600000}\selectfont\catcode`\^=\active\def^{\ifmmode\sp\else\^{}\fi}\catcode`\%=\active\def%{\%}Flicker noise $h_{-1}f^{-1}$}}%
\end{pgfscope}%
\begin{pgfscope}%
\pgfsetbuttcap%
\pgfsetroundjoin%
\pgfsetlinewidth{1.505625pt}%
\definecolor{currentstroke}{rgb}{0.835294,0.368627,0.000000}%
\pgfsetstrokecolor{currentstroke}%
\pgfsetdash{{5.550000pt}{2.400000pt}}{0.000000pt}%
\pgfpathmoveto{\pgfqpoint{2.375417in}{1.952226in}}%
\pgfpathlineto{\pgfqpoint{2.486528in}{1.952226in}}%
\pgfpathlineto{\pgfqpoint{2.597639in}{1.952226in}}%
\pgfusepath{stroke}%
\end{pgfscope}%
\begin{pgfscope}%
\definecolor{textcolor}{rgb}{0.000000,0.000000,0.000000}%
\pgfsetstrokecolor{textcolor}%
\pgfsetfillcolor{textcolor}%
\pgftext[x=2.686528in,y=1.913337in,left,base]{\color{textcolor}{\rmfamily\fontsize{8.000000}{9.600000}\selectfont\catcode`\^=\active\def^{\ifmmode\sp\else\^{}\fi}\catcode`\%=\active\def%{\%}Random walk $h_{-2}f^{-2}$}}%
\end{pgfscope}%
\end{pgfpicture}%
\makeatother%
\endgroup%
% data/simulations/sim_allan_variance_example.py
    \caption{A simulated power spectrum containing white noise, flicker noise and random walk behaviour.}
    \label{fig:adev_example_psd}
\end{figure}

To get an even better representation of the individual noise contributions, the Allan variance or Allan deviation can be used. The Allan deviation plot shown in figure \ref{fig:adev_example_adev} gives very clean results and all noise components can be clearly identified. The individual components were plotted using dashed lines as well.
\begin{figure}[ht]
    \centering
    %% Creator: Matplotlib, PGF backend
%%
%% To include the figure in your LaTeX document, write
%%   \input{<filename>.pgf}
%%
%% Make sure the required packages are loaded in your preamble
%%   \usepackage{pgf}
%%
%% Also ensure that all the required font packages are loaded; for instance,
%% the lmodern package is sometimes necessary when using math font.
%%   \usepackage{lmodern}
%%
%% Figures using additional raster images can only be included by \input if
%% they are in the same directory as the main LaTeX file. For loading figures
%% from other directories you can use the `import` package
%%   \usepackage{import}
%%
%% and then include the figures with
%%   \import{<path to file>}{<filename>.pgf}
%%
%% Matplotlib used the following preamble
%%   \usepackage{siunitx}
%%   \usepackage{fontspec}
%%   \makeatletter\@ifpackageloaded{underscore}{}{\usepackage[strings]{underscore}}\makeatother
%%
\begingroup%
\makeatletter%
\begin{pgfpicture}%
\pgfpathrectangle{\pgfpointorigin}{\pgfqpoint{4.060000in}{2.510000in}}%
\pgfusepath{use as bounding box, clip}%
\begin{pgfscope}%
\pgfsetbuttcap%
\pgfsetmiterjoin%
\definecolor{currentfill}{rgb}{1.000000,1.000000,1.000000}%
\pgfsetfillcolor{currentfill}%
\pgfsetlinewidth{0.000000pt}%
\definecolor{currentstroke}{rgb}{1.000000,1.000000,1.000000}%
\pgfsetstrokecolor{currentstroke}%
\pgfsetdash{}{0pt}%
\pgfpathmoveto{\pgfqpoint{0.000000in}{0.000000in}}%
\pgfpathlineto{\pgfqpoint{4.060000in}{0.000000in}}%
\pgfpathlineto{\pgfqpoint{4.060000in}{2.510000in}}%
\pgfpathlineto{\pgfqpoint{0.000000in}{2.510000in}}%
\pgfpathlineto{\pgfqpoint{0.000000in}{0.000000in}}%
\pgfpathclose%
\pgfusepath{fill}%
\end{pgfscope}%
\begin{pgfscope}%
\pgfsetbuttcap%
\pgfsetmiterjoin%
\definecolor{currentfill}{rgb}{1.000000,1.000000,1.000000}%
\pgfsetfillcolor{currentfill}%
\pgfsetlinewidth{0.000000pt}%
\definecolor{currentstroke}{rgb}{0.000000,0.000000,0.000000}%
\pgfsetstrokecolor{currentstroke}%
\pgfsetstrokeopacity{0.000000}%
\pgfsetdash{}{0pt}%
\pgfpathmoveto{\pgfqpoint{0.509263in}{0.417642in}}%
\pgfpathlineto{\pgfqpoint{4.018330in}{0.417642in}}%
\pgfpathlineto{\pgfqpoint{4.018330in}{2.468330in}}%
\pgfpathlineto{\pgfqpoint{0.509263in}{2.468330in}}%
\pgfpathlineto{\pgfqpoint{0.509263in}{0.417642in}}%
\pgfpathclose%
\pgfusepath{fill}%
\end{pgfscope}%
\begin{pgfscope}%
\pgfpathrectangle{\pgfqpoint{0.509263in}{0.417642in}}{\pgfqpoint{3.509067in}{2.050688in}}%
\pgfusepath{clip}%
\pgfsetrectcap%
\pgfsetroundjoin%
\pgfsetlinewidth{0.803000pt}%
\definecolor{currentstroke}{rgb}{0.450000,0.450000,0.450000}%
\pgfsetstrokecolor{currentstroke}%
\pgfsetdash{}{0pt}%
\pgfpathmoveto{\pgfqpoint{0.668766in}{0.417642in}}%
\pgfpathlineto{\pgfqpoint{0.668766in}{2.468330in}}%
\pgfusepath{stroke}%
\end{pgfscope}%
\begin{pgfscope}%
\pgfsetbuttcap%
\pgfsetroundjoin%
\definecolor{currentfill}{rgb}{0.000000,0.000000,0.000000}%
\pgfsetfillcolor{currentfill}%
\pgfsetlinewidth{0.803000pt}%
\definecolor{currentstroke}{rgb}{0.000000,0.000000,0.000000}%
\pgfsetstrokecolor{currentstroke}%
\pgfsetdash{}{0pt}%
\pgfsys@defobject{currentmarker}{\pgfqpoint{0.000000in}{-0.048611in}}{\pgfqpoint{0.000000in}{0.000000in}}{%
\pgfpathmoveto{\pgfqpoint{0.000000in}{0.000000in}}%
\pgfpathlineto{\pgfqpoint{0.000000in}{-0.048611in}}%
\pgfusepath{stroke,fill}%
}%
\begin{pgfscope}%
\pgfsys@transformshift{0.668766in}{0.417642in}%
\pgfsys@useobject{currentmarker}{}%
\end{pgfscope}%
\end{pgfscope}%
\begin{pgfscope}%
\definecolor{textcolor}{rgb}{0.000000,0.000000,0.000000}%
\pgfsetstrokecolor{textcolor}%
\pgfsetfillcolor{textcolor}%
\pgftext[x=0.668766in,y=0.320420in,,top]{\color{textcolor}\rmfamily\fontsize{8.000000}{9.600000}\selectfont \(\displaystyle {10^{-6}}\)}%
\end{pgfscope}%
\begin{pgfscope}%
\pgfpathrectangle{\pgfqpoint{0.509263in}{0.417642in}}{\pgfqpoint{3.509067in}{2.050688in}}%
\pgfusepath{clip}%
\pgfsetrectcap%
\pgfsetroundjoin%
\pgfsetlinewidth{0.803000pt}%
\definecolor{currentstroke}{rgb}{0.450000,0.450000,0.450000}%
\pgfsetstrokecolor{currentstroke}%
\pgfsetdash{}{0pt}%
\pgfpathmoveto{\pgfqpoint{1.124489in}{0.417642in}}%
\pgfpathlineto{\pgfqpoint{1.124489in}{2.468330in}}%
\pgfusepath{stroke}%
\end{pgfscope}%
\begin{pgfscope}%
\pgfsetbuttcap%
\pgfsetroundjoin%
\definecolor{currentfill}{rgb}{0.000000,0.000000,0.000000}%
\pgfsetfillcolor{currentfill}%
\pgfsetlinewidth{0.803000pt}%
\definecolor{currentstroke}{rgb}{0.000000,0.000000,0.000000}%
\pgfsetstrokecolor{currentstroke}%
\pgfsetdash{}{0pt}%
\pgfsys@defobject{currentmarker}{\pgfqpoint{0.000000in}{-0.048611in}}{\pgfqpoint{0.000000in}{0.000000in}}{%
\pgfpathmoveto{\pgfqpoint{0.000000in}{0.000000in}}%
\pgfpathlineto{\pgfqpoint{0.000000in}{-0.048611in}}%
\pgfusepath{stroke,fill}%
}%
\begin{pgfscope}%
\pgfsys@transformshift{1.124489in}{0.417642in}%
\pgfsys@useobject{currentmarker}{}%
\end{pgfscope}%
\end{pgfscope}%
\begin{pgfscope}%
\definecolor{textcolor}{rgb}{0.000000,0.000000,0.000000}%
\pgfsetstrokecolor{textcolor}%
\pgfsetfillcolor{textcolor}%
\pgftext[x=1.124489in,y=0.320420in,,top]{\color{textcolor}\rmfamily\fontsize{8.000000}{9.600000}\selectfont \(\displaystyle {10^{-5}}\)}%
\end{pgfscope}%
\begin{pgfscope}%
\pgfpathrectangle{\pgfqpoint{0.509263in}{0.417642in}}{\pgfqpoint{3.509067in}{2.050688in}}%
\pgfusepath{clip}%
\pgfsetrectcap%
\pgfsetroundjoin%
\pgfsetlinewidth{0.803000pt}%
\definecolor{currentstroke}{rgb}{0.450000,0.450000,0.450000}%
\pgfsetstrokecolor{currentstroke}%
\pgfsetdash{}{0pt}%
\pgfpathmoveto{\pgfqpoint{1.580212in}{0.417642in}}%
\pgfpathlineto{\pgfqpoint{1.580212in}{2.468330in}}%
\pgfusepath{stroke}%
\end{pgfscope}%
\begin{pgfscope}%
\pgfsetbuttcap%
\pgfsetroundjoin%
\definecolor{currentfill}{rgb}{0.000000,0.000000,0.000000}%
\pgfsetfillcolor{currentfill}%
\pgfsetlinewidth{0.803000pt}%
\definecolor{currentstroke}{rgb}{0.000000,0.000000,0.000000}%
\pgfsetstrokecolor{currentstroke}%
\pgfsetdash{}{0pt}%
\pgfsys@defobject{currentmarker}{\pgfqpoint{0.000000in}{-0.048611in}}{\pgfqpoint{0.000000in}{0.000000in}}{%
\pgfpathmoveto{\pgfqpoint{0.000000in}{0.000000in}}%
\pgfpathlineto{\pgfqpoint{0.000000in}{-0.048611in}}%
\pgfusepath{stroke,fill}%
}%
\begin{pgfscope}%
\pgfsys@transformshift{1.580212in}{0.417642in}%
\pgfsys@useobject{currentmarker}{}%
\end{pgfscope}%
\end{pgfscope}%
\begin{pgfscope}%
\definecolor{textcolor}{rgb}{0.000000,0.000000,0.000000}%
\pgfsetstrokecolor{textcolor}%
\pgfsetfillcolor{textcolor}%
\pgftext[x=1.580212in,y=0.320420in,,top]{\color{textcolor}\rmfamily\fontsize{8.000000}{9.600000}\selectfont \(\displaystyle {10^{-4}}\)}%
\end{pgfscope}%
\begin{pgfscope}%
\pgfpathrectangle{\pgfqpoint{0.509263in}{0.417642in}}{\pgfqpoint{3.509067in}{2.050688in}}%
\pgfusepath{clip}%
\pgfsetrectcap%
\pgfsetroundjoin%
\pgfsetlinewidth{0.803000pt}%
\definecolor{currentstroke}{rgb}{0.450000,0.450000,0.450000}%
\pgfsetstrokecolor{currentstroke}%
\pgfsetdash{}{0pt}%
\pgfpathmoveto{\pgfqpoint{2.035935in}{0.417642in}}%
\pgfpathlineto{\pgfqpoint{2.035935in}{2.468330in}}%
\pgfusepath{stroke}%
\end{pgfscope}%
\begin{pgfscope}%
\pgfsetbuttcap%
\pgfsetroundjoin%
\definecolor{currentfill}{rgb}{0.000000,0.000000,0.000000}%
\pgfsetfillcolor{currentfill}%
\pgfsetlinewidth{0.803000pt}%
\definecolor{currentstroke}{rgb}{0.000000,0.000000,0.000000}%
\pgfsetstrokecolor{currentstroke}%
\pgfsetdash{}{0pt}%
\pgfsys@defobject{currentmarker}{\pgfqpoint{0.000000in}{-0.048611in}}{\pgfqpoint{0.000000in}{0.000000in}}{%
\pgfpathmoveto{\pgfqpoint{0.000000in}{0.000000in}}%
\pgfpathlineto{\pgfqpoint{0.000000in}{-0.048611in}}%
\pgfusepath{stroke,fill}%
}%
\begin{pgfscope}%
\pgfsys@transformshift{2.035935in}{0.417642in}%
\pgfsys@useobject{currentmarker}{}%
\end{pgfscope}%
\end{pgfscope}%
\begin{pgfscope}%
\definecolor{textcolor}{rgb}{0.000000,0.000000,0.000000}%
\pgfsetstrokecolor{textcolor}%
\pgfsetfillcolor{textcolor}%
\pgftext[x=2.035935in,y=0.320420in,,top]{\color{textcolor}\rmfamily\fontsize{8.000000}{9.600000}\selectfont \(\displaystyle {10^{-3}}\)}%
\end{pgfscope}%
\begin{pgfscope}%
\pgfpathrectangle{\pgfqpoint{0.509263in}{0.417642in}}{\pgfqpoint{3.509067in}{2.050688in}}%
\pgfusepath{clip}%
\pgfsetrectcap%
\pgfsetroundjoin%
\pgfsetlinewidth{0.803000pt}%
\definecolor{currentstroke}{rgb}{0.450000,0.450000,0.450000}%
\pgfsetstrokecolor{currentstroke}%
\pgfsetdash{}{0pt}%
\pgfpathmoveto{\pgfqpoint{2.491658in}{0.417642in}}%
\pgfpathlineto{\pgfqpoint{2.491658in}{2.468330in}}%
\pgfusepath{stroke}%
\end{pgfscope}%
\begin{pgfscope}%
\pgfsetbuttcap%
\pgfsetroundjoin%
\definecolor{currentfill}{rgb}{0.000000,0.000000,0.000000}%
\pgfsetfillcolor{currentfill}%
\pgfsetlinewidth{0.803000pt}%
\definecolor{currentstroke}{rgb}{0.000000,0.000000,0.000000}%
\pgfsetstrokecolor{currentstroke}%
\pgfsetdash{}{0pt}%
\pgfsys@defobject{currentmarker}{\pgfqpoint{0.000000in}{-0.048611in}}{\pgfqpoint{0.000000in}{0.000000in}}{%
\pgfpathmoveto{\pgfqpoint{0.000000in}{0.000000in}}%
\pgfpathlineto{\pgfqpoint{0.000000in}{-0.048611in}}%
\pgfusepath{stroke,fill}%
}%
\begin{pgfscope}%
\pgfsys@transformshift{2.491658in}{0.417642in}%
\pgfsys@useobject{currentmarker}{}%
\end{pgfscope}%
\end{pgfscope}%
\begin{pgfscope}%
\definecolor{textcolor}{rgb}{0.000000,0.000000,0.000000}%
\pgfsetstrokecolor{textcolor}%
\pgfsetfillcolor{textcolor}%
\pgftext[x=2.491658in,y=0.320420in,,top]{\color{textcolor}\rmfamily\fontsize{8.000000}{9.600000}\selectfont \(\displaystyle {10^{-2}}\)}%
\end{pgfscope}%
\begin{pgfscope}%
\pgfpathrectangle{\pgfqpoint{0.509263in}{0.417642in}}{\pgfqpoint{3.509067in}{2.050688in}}%
\pgfusepath{clip}%
\pgfsetrectcap%
\pgfsetroundjoin%
\pgfsetlinewidth{0.803000pt}%
\definecolor{currentstroke}{rgb}{0.450000,0.450000,0.450000}%
\pgfsetstrokecolor{currentstroke}%
\pgfsetdash{}{0pt}%
\pgfpathmoveto{\pgfqpoint{2.947381in}{0.417642in}}%
\pgfpathlineto{\pgfqpoint{2.947381in}{2.468330in}}%
\pgfusepath{stroke}%
\end{pgfscope}%
\begin{pgfscope}%
\pgfsetbuttcap%
\pgfsetroundjoin%
\definecolor{currentfill}{rgb}{0.000000,0.000000,0.000000}%
\pgfsetfillcolor{currentfill}%
\pgfsetlinewidth{0.803000pt}%
\definecolor{currentstroke}{rgb}{0.000000,0.000000,0.000000}%
\pgfsetstrokecolor{currentstroke}%
\pgfsetdash{}{0pt}%
\pgfsys@defobject{currentmarker}{\pgfqpoint{0.000000in}{-0.048611in}}{\pgfqpoint{0.000000in}{0.000000in}}{%
\pgfpathmoveto{\pgfqpoint{0.000000in}{0.000000in}}%
\pgfpathlineto{\pgfqpoint{0.000000in}{-0.048611in}}%
\pgfusepath{stroke,fill}%
}%
\begin{pgfscope}%
\pgfsys@transformshift{2.947381in}{0.417642in}%
\pgfsys@useobject{currentmarker}{}%
\end{pgfscope}%
\end{pgfscope}%
\begin{pgfscope}%
\definecolor{textcolor}{rgb}{0.000000,0.000000,0.000000}%
\pgfsetstrokecolor{textcolor}%
\pgfsetfillcolor{textcolor}%
\pgftext[x=2.947381in,y=0.320420in,,top]{\color{textcolor}\rmfamily\fontsize{8.000000}{9.600000}\selectfont \(\displaystyle {10^{-1}}\)}%
\end{pgfscope}%
\begin{pgfscope}%
\pgfpathrectangle{\pgfqpoint{0.509263in}{0.417642in}}{\pgfqpoint{3.509067in}{2.050688in}}%
\pgfusepath{clip}%
\pgfsetrectcap%
\pgfsetroundjoin%
\pgfsetlinewidth{0.803000pt}%
\definecolor{currentstroke}{rgb}{0.450000,0.450000,0.450000}%
\pgfsetstrokecolor{currentstroke}%
\pgfsetdash{}{0pt}%
\pgfpathmoveto{\pgfqpoint{3.403104in}{0.417642in}}%
\pgfpathlineto{\pgfqpoint{3.403104in}{2.468330in}}%
\pgfusepath{stroke}%
\end{pgfscope}%
\begin{pgfscope}%
\pgfsetbuttcap%
\pgfsetroundjoin%
\definecolor{currentfill}{rgb}{0.000000,0.000000,0.000000}%
\pgfsetfillcolor{currentfill}%
\pgfsetlinewidth{0.803000pt}%
\definecolor{currentstroke}{rgb}{0.000000,0.000000,0.000000}%
\pgfsetstrokecolor{currentstroke}%
\pgfsetdash{}{0pt}%
\pgfsys@defobject{currentmarker}{\pgfqpoint{0.000000in}{-0.048611in}}{\pgfqpoint{0.000000in}{0.000000in}}{%
\pgfpathmoveto{\pgfqpoint{0.000000in}{0.000000in}}%
\pgfpathlineto{\pgfqpoint{0.000000in}{-0.048611in}}%
\pgfusepath{stroke,fill}%
}%
\begin{pgfscope}%
\pgfsys@transformshift{3.403104in}{0.417642in}%
\pgfsys@useobject{currentmarker}{}%
\end{pgfscope}%
\end{pgfscope}%
\begin{pgfscope}%
\definecolor{textcolor}{rgb}{0.000000,0.000000,0.000000}%
\pgfsetstrokecolor{textcolor}%
\pgfsetfillcolor{textcolor}%
\pgftext[x=3.403104in,y=0.320420in,,top]{\color{textcolor}\rmfamily\fontsize{8.000000}{9.600000}\selectfont \(\displaystyle {10^{0}}\)}%
\end{pgfscope}%
\begin{pgfscope}%
\pgfpathrectangle{\pgfqpoint{0.509263in}{0.417642in}}{\pgfqpoint{3.509067in}{2.050688in}}%
\pgfusepath{clip}%
\pgfsetrectcap%
\pgfsetroundjoin%
\pgfsetlinewidth{0.803000pt}%
\definecolor{currentstroke}{rgb}{0.450000,0.450000,0.450000}%
\pgfsetstrokecolor{currentstroke}%
\pgfsetdash{}{0pt}%
\pgfpathmoveto{\pgfqpoint{3.858827in}{0.417642in}}%
\pgfpathlineto{\pgfqpoint{3.858827in}{2.468330in}}%
\pgfusepath{stroke}%
\end{pgfscope}%
\begin{pgfscope}%
\pgfsetbuttcap%
\pgfsetroundjoin%
\definecolor{currentfill}{rgb}{0.000000,0.000000,0.000000}%
\pgfsetfillcolor{currentfill}%
\pgfsetlinewidth{0.803000pt}%
\definecolor{currentstroke}{rgb}{0.000000,0.000000,0.000000}%
\pgfsetstrokecolor{currentstroke}%
\pgfsetdash{}{0pt}%
\pgfsys@defobject{currentmarker}{\pgfqpoint{0.000000in}{-0.048611in}}{\pgfqpoint{0.000000in}{0.000000in}}{%
\pgfpathmoveto{\pgfqpoint{0.000000in}{0.000000in}}%
\pgfpathlineto{\pgfqpoint{0.000000in}{-0.048611in}}%
\pgfusepath{stroke,fill}%
}%
\begin{pgfscope}%
\pgfsys@transformshift{3.858827in}{0.417642in}%
\pgfsys@useobject{currentmarker}{}%
\end{pgfscope}%
\end{pgfscope}%
\begin{pgfscope}%
\definecolor{textcolor}{rgb}{0.000000,0.000000,0.000000}%
\pgfsetstrokecolor{textcolor}%
\pgfsetfillcolor{textcolor}%
\pgftext[x=3.858827in,y=0.320420in,,top]{\color{textcolor}\rmfamily\fontsize{8.000000}{9.600000}\selectfont \(\displaystyle {10^{1}}\)}%
\end{pgfscope}%
\begin{pgfscope}%
\pgfpathrectangle{\pgfqpoint{0.509263in}{0.417642in}}{\pgfqpoint{3.509067in}{2.050688in}}%
\pgfusepath{clip}%
\pgfsetrectcap%
\pgfsetroundjoin%
\pgfsetlinewidth{0.803000pt}%
\definecolor{currentstroke}{rgb}{0.850000,0.850000,0.850000}%
\pgfsetstrokecolor{currentstroke}%
\pgfsetdash{}{0pt}%
\pgfpathmoveto{\pgfqpoint{0.531580in}{0.417642in}}%
\pgfpathlineto{\pgfqpoint{0.531580in}{2.468330in}}%
\pgfusepath{stroke}%
\end{pgfscope}%
\begin{pgfscope}%
\pgfsetbuttcap%
\pgfsetroundjoin%
\definecolor{currentfill}{rgb}{0.000000,0.000000,0.000000}%
\pgfsetfillcolor{currentfill}%
\pgfsetlinewidth{0.602250pt}%
\definecolor{currentstroke}{rgb}{0.000000,0.000000,0.000000}%
\pgfsetstrokecolor{currentstroke}%
\pgfsetdash{}{0pt}%
\pgfsys@defobject{currentmarker}{\pgfqpoint{0.000000in}{-0.027778in}}{\pgfqpoint{0.000000in}{0.000000in}}{%
\pgfpathmoveto{\pgfqpoint{0.000000in}{0.000000in}}%
\pgfpathlineto{\pgfqpoint{0.000000in}{-0.027778in}}%
\pgfusepath{stroke,fill}%
}%
\begin{pgfscope}%
\pgfsys@transformshift{0.531580in}{0.417642in}%
\pgfsys@useobject{currentmarker}{}%
\end{pgfscope}%
\end{pgfscope}%
\begin{pgfscope}%
\pgfpathrectangle{\pgfqpoint{0.509263in}{0.417642in}}{\pgfqpoint{3.509067in}{2.050688in}}%
\pgfusepath{clip}%
\pgfsetrectcap%
\pgfsetroundjoin%
\pgfsetlinewidth{0.803000pt}%
\definecolor{currentstroke}{rgb}{0.850000,0.850000,0.850000}%
\pgfsetstrokecolor{currentstroke}%
\pgfsetdash{}{0pt}%
\pgfpathmoveto{\pgfqpoint{0.567665in}{0.417642in}}%
\pgfpathlineto{\pgfqpoint{0.567665in}{2.468330in}}%
\pgfusepath{stroke}%
\end{pgfscope}%
\begin{pgfscope}%
\pgfsetbuttcap%
\pgfsetroundjoin%
\definecolor{currentfill}{rgb}{0.000000,0.000000,0.000000}%
\pgfsetfillcolor{currentfill}%
\pgfsetlinewidth{0.602250pt}%
\definecolor{currentstroke}{rgb}{0.000000,0.000000,0.000000}%
\pgfsetstrokecolor{currentstroke}%
\pgfsetdash{}{0pt}%
\pgfsys@defobject{currentmarker}{\pgfqpoint{0.000000in}{-0.027778in}}{\pgfqpoint{0.000000in}{0.000000in}}{%
\pgfpathmoveto{\pgfqpoint{0.000000in}{0.000000in}}%
\pgfpathlineto{\pgfqpoint{0.000000in}{-0.027778in}}%
\pgfusepath{stroke,fill}%
}%
\begin{pgfscope}%
\pgfsys@transformshift{0.567665in}{0.417642in}%
\pgfsys@useobject{currentmarker}{}%
\end{pgfscope}%
\end{pgfscope}%
\begin{pgfscope}%
\pgfpathrectangle{\pgfqpoint{0.509263in}{0.417642in}}{\pgfqpoint{3.509067in}{2.050688in}}%
\pgfusepath{clip}%
\pgfsetrectcap%
\pgfsetroundjoin%
\pgfsetlinewidth{0.803000pt}%
\definecolor{currentstroke}{rgb}{0.850000,0.850000,0.850000}%
\pgfsetstrokecolor{currentstroke}%
\pgfsetdash{}{0pt}%
\pgfpathmoveto{\pgfqpoint{0.598174in}{0.417642in}}%
\pgfpathlineto{\pgfqpoint{0.598174in}{2.468330in}}%
\pgfusepath{stroke}%
\end{pgfscope}%
\begin{pgfscope}%
\pgfsetbuttcap%
\pgfsetroundjoin%
\definecolor{currentfill}{rgb}{0.000000,0.000000,0.000000}%
\pgfsetfillcolor{currentfill}%
\pgfsetlinewidth{0.602250pt}%
\definecolor{currentstroke}{rgb}{0.000000,0.000000,0.000000}%
\pgfsetstrokecolor{currentstroke}%
\pgfsetdash{}{0pt}%
\pgfsys@defobject{currentmarker}{\pgfqpoint{0.000000in}{-0.027778in}}{\pgfqpoint{0.000000in}{0.000000in}}{%
\pgfpathmoveto{\pgfqpoint{0.000000in}{0.000000in}}%
\pgfpathlineto{\pgfqpoint{0.000000in}{-0.027778in}}%
\pgfusepath{stroke,fill}%
}%
\begin{pgfscope}%
\pgfsys@transformshift{0.598174in}{0.417642in}%
\pgfsys@useobject{currentmarker}{}%
\end{pgfscope}%
\end{pgfscope}%
\begin{pgfscope}%
\pgfpathrectangle{\pgfqpoint{0.509263in}{0.417642in}}{\pgfqpoint{3.509067in}{2.050688in}}%
\pgfusepath{clip}%
\pgfsetrectcap%
\pgfsetroundjoin%
\pgfsetlinewidth{0.803000pt}%
\definecolor{currentstroke}{rgb}{0.850000,0.850000,0.850000}%
\pgfsetstrokecolor{currentstroke}%
\pgfsetdash{}{0pt}%
\pgfpathmoveto{\pgfqpoint{0.624602in}{0.417642in}}%
\pgfpathlineto{\pgfqpoint{0.624602in}{2.468330in}}%
\pgfusepath{stroke}%
\end{pgfscope}%
\begin{pgfscope}%
\pgfsetbuttcap%
\pgfsetroundjoin%
\definecolor{currentfill}{rgb}{0.000000,0.000000,0.000000}%
\pgfsetfillcolor{currentfill}%
\pgfsetlinewidth{0.602250pt}%
\definecolor{currentstroke}{rgb}{0.000000,0.000000,0.000000}%
\pgfsetstrokecolor{currentstroke}%
\pgfsetdash{}{0pt}%
\pgfsys@defobject{currentmarker}{\pgfqpoint{0.000000in}{-0.027778in}}{\pgfqpoint{0.000000in}{0.000000in}}{%
\pgfpathmoveto{\pgfqpoint{0.000000in}{0.000000in}}%
\pgfpathlineto{\pgfqpoint{0.000000in}{-0.027778in}}%
\pgfusepath{stroke,fill}%
}%
\begin{pgfscope}%
\pgfsys@transformshift{0.624602in}{0.417642in}%
\pgfsys@useobject{currentmarker}{}%
\end{pgfscope}%
\end{pgfscope}%
\begin{pgfscope}%
\pgfpathrectangle{\pgfqpoint{0.509263in}{0.417642in}}{\pgfqpoint{3.509067in}{2.050688in}}%
\pgfusepath{clip}%
\pgfsetrectcap%
\pgfsetroundjoin%
\pgfsetlinewidth{0.803000pt}%
\definecolor{currentstroke}{rgb}{0.850000,0.850000,0.850000}%
\pgfsetstrokecolor{currentstroke}%
\pgfsetdash{}{0pt}%
\pgfpathmoveto{\pgfqpoint{0.647914in}{0.417642in}}%
\pgfpathlineto{\pgfqpoint{0.647914in}{2.468330in}}%
\pgfusepath{stroke}%
\end{pgfscope}%
\begin{pgfscope}%
\pgfsetbuttcap%
\pgfsetroundjoin%
\definecolor{currentfill}{rgb}{0.000000,0.000000,0.000000}%
\pgfsetfillcolor{currentfill}%
\pgfsetlinewidth{0.602250pt}%
\definecolor{currentstroke}{rgb}{0.000000,0.000000,0.000000}%
\pgfsetstrokecolor{currentstroke}%
\pgfsetdash{}{0pt}%
\pgfsys@defobject{currentmarker}{\pgfqpoint{0.000000in}{-0.027778in}}{\pgfqpoint{0.000000in}{0.000000in}}{%
\pgfpathmoveto{\pgfqpoint{0.000000in}{0.000000in}}%
\pgfpathlineto{\pgfqpoint{0.000000in}{-0.027778in}}%
\pgfusepath{stroke,fill}%
}%
\begin{pgfscope}%
\pgfsys@transformshift{0.647914in}{0.417642in}%
\pgfsys@useobject{currentmarker}{}%
\end{pgfscope}%
\end{pgfscope}%
\begin{pgfscope}%
\pgfpathrectangle{\pgfqpoint{0.509263in}{0.417642in}}{\pgfqpoint{3.509067in}{2.050688in}}%
\pgfusepath{clip}%
\pgfsetrectcap%
\pgfsetroundjoin%
\pgfsetlinewidth{0.803000pt}%
\definecolor{currentstroke}{rgb}{0.850000,0.850000,0.850000}%
\pgfsetstrokecolor{currentstroke}%
\pgfsetdash{}{0pt}%
\pgfpathmoveto{\pgfqpoint{0.805953in}{0.417642in}}%
\pgfpathlineto{\pgfqpoint{0.805953in}{2.468330in}}%
\pgfusepath{stroke}%
\end{pgfscope}%
\begin{pgfscope}%
\pgfsetbuttcap%
\pgfsetroundjoin%
\definecolor{currentfill}{rgb}{0.000000,0.000000,0.000000}%
\pgfsetfillcolor{currentfill}%
\pgfsetlinewidth{0.602250pt}%
\definecolor{currentstroke}{rgb}{0.000000,0.000000,0.000000}%
\pgfsetstrokecolor{currentstroke}%
\pgfsetdash{}{0pt}%
\pgfsys@defobject{currentmarker}{\pgfqpoint{0.000000in}{-0.027778in}}{\pgfqpoint{0.000000in}{0.000000in}}{%
\pgfpathmoveto{\pgfqpoint{0.000000in}{0.000000in}}%
\pgfpathlineto{\pgfqpoint{0.000000in}{-0.027778in}}%
\pgfusepath{stroke,fill}%
}%
\begin{pgfscope}%
\pgfsys@transformshift{0.805953in}{0.417642in}%
\pgfsys@useobject{currentmarker}{}%
\end{pgfscope}%
\end{pgfscope}%
\begin{pgfscope}%
\pgfpathrectangle{\pgfqpoint{0.509263in}{0.417642in}}{\pgfqpoint{3.509067in}{2.050688in}}%
\pgfusepath{clip}%
\pgfsetrectcap%
\pgfsetroundjoin%
\pgfsetlinewidth{0.803000pt}%
\definecolor{currentstroke}{rgb}{0.850000,0.850000,0.850000}%
\pgfsetstrokecolor{currentstroke}%
\pgfsetdash{}{0pt}%
\pgfpathmoveto{\pgfqpoint{0.886202in}{0.417642in}}%
\pgfpathlineto{\pgfqpoint{0.886202in}{2.468330in}}%
\pgfusepath{stroke}%
\end{pgfscope}%
\begin{pgfscope}%
\pgfsetbuttcap%
\pgfsetroundjoin%
\definecolor{currentfill}{rgb}{0.000000,0.000000,0.000000}%
\pgfsetfillcolor{currentfill}%
\pgfsetlinewidth{0.602250pt}%
\definecolor{currentstroke}{rgb}{0.000000,0.000000,0.000000}%
\pgfsetstrokecolor{currentstroke}%
\pgfsetdash{}{0pt}%
\pgfsys@defobject{currentmarker}{\pgfqpoint{0.000000in}{-0.027778in}}{\pgfqpoint{0.000000in}{0.000000in}}{%
\pgfpathmoveto{\pgfqpoint{0.000000in}{0.000000in}}%
\pgfpathlineto{\pgfqpoint{0.000000in}{-0.027778in}}%
\pgfusepath{stroke,fill}%
}%
\begin{pgfscope}%
\pgfsys@transformshift{0.886202in}{0.417642in}%
\pgfsys@useobject{currentmarker}{}%
\end{pgfscope}%
\end{pgfscope}%
\begin{pgfscope}%
\pgfpathrectangle{\pgfqpoint{0.509263in}{0.417642in}}{\pgfqpoint{3.509067in}{2.050688in}}%
\pgfusepath{clip}%
\pgfsetrectcap%
\pgfsetroundjoin%
\pgfsetlinewidth{0.803000pt}%
\definecolor{currentstroke}{rgb}{0.850000,0.850000,0.850000}%
\pgfsetstrokecolor{currentstroke}%
\pgfsetdash{}{0pt}%
\pgfpathmoveto{\pgfqpoint{0.943139in}{0.417642in}}%
\pgfpathlineto{\pgfqpoint{0.943139in}{2.468330in}}%
\pgfusepath{stroke}%
\end{pgfscope}%
\begin{pgfscope}%
\pgfsetbuttcap%
\pgfsetroundjoin%
\definecolor{currentfill}{rgb}{0.000000,0.000000,0.000000}%
\pgfsetfillcolor{currentfill}%
\pgfsetlinewidth{0.602250pt}%
\definecolor{currentstroke}{rgb}{0.000000,0.000000,0.000000}%
\pgfsetstrokecolor{currentstroke}%
\pgfsetdash{}{0pt}%
\pgfsys@defobject{currentmarker}{\pgfqpoint{0.000000in}{-0.027778in}}{\pgfqpoint{0.000000in}{0.000000in}}{%
\pgfpathmoveto{\pgfqpoint{0.000000in}{0.000000in}}%
\pgfpathlineto{\pgfqpoint{0.000000in}{-0.027778in}}%
\pgfusepath{stroke,fill}%
}%
\begin{pgfscope}%
\pgfsys@transformshift{0.943139in}{0.417642in}%
\pgfsys@useobject{currentmarker}{}%
\end{pgfscope}%
\end{pgfscope}%
\begin{pgfscope}%
\pgfpathrectangle{\pgfqpoint{0.509263in}{0.417642in}}{\pgfqpoint{3.509067in}{2.050688in}}%
\pgfusepath{clip}%
\pgfsetrectcap%
\pgfsetroundjoin%
\pgfsetlinewidth{0.803000pt}%
\definecolor{currentstroke}{rgb}{0.850000,0.850000,0.850000}%
\pgfsetstrokecolor{currentstroke}%
\pgfsetdash{}{0pt}%
\pgfpathmoveto{\pgfqpoint{0.987303in}{0.417642in}}%
\pgfpathlineto{\pgfqpoint{0.987303in}{2.468330in}}%
\pgfusepath{stroke}%
\end{pgfscope}%
\begin{pgfscope}%
\pgfsetbuttcap%
\pgfsetroundjoin%
\definecolor{currentfill}{rgb}{0.000000,0.000000,0.000000}%
\pgfsetfillcolor{currentfill}%
\pgfsetlinewidth{0.602250pt}%
\definecolor{currentstroke}{rgb}{0.000000,0.000000,0.000000}%
\pgfsetstrokecolor{currentstroke}%
\pgfsetdash{}{0pt}%
\pgfsys@defobject{currentmarker}{\pgfqpoint{0.000000in}{-0.027778in}}{\pgfqpoint{0.000000in}{0.000000in}}{%
\pgfpathmoveto{\pgfqpoint{0.000000in}{0.000000in}}%
\pgfpathlineto{\pgfqpoint{0.000000in}{-0.027778in}}%
\pgfusepath{stroke,fill}%
}%
\begin{pgfscope}%
\pgfsys@transformshift{0.987303in}{0.417642in}%
\pgfsys@useobject{currentmarker}{}%
\end{pgfscope}%
\end{pgfscope}%
\begin{pgfscope}%
\pgfpathrectangle{\pgfqpoint{0.509263in}{0.417642in}}{\pgfqpoint{3.509067in}{2.050688in}}%
\pgfusepath{clip}%
\pgfsetrectcap%
\pgfsetroundjoin%
\pgfsetlinewidth{0.803000pt}%
\definecolor{currentstroke}{rgb}{0.850000,0.850000,0.850000}%
\pgfsetstrokecolor{currentstroke}%
\pgfsetdash{}{0pt}%
\pgfpathmoveto{\pgfqpoint{1.023388in}{0.417642in}}%
\pgfpathlineto{\pgfqpoint{1.023388in}{2.468330in}}%
\pgfusepath{stroke}%
\end{pgfscope}%
\begin{pgfscope}%
\pgfsetbuttcap%
\pgfsetroundjoin%
\definecolor{currentfill}{rgb}{0.000000,0.000000,0.000000}%
\pgfsetfillcolor{currentfill}%
\pgfsetlinewidth{0.602250pt}%
\definecolor{currentstroke}{rgb}{0.000000,0.000000,0.000000}%
\pgfsetstrokecolor{currentstroke}%
\pgfsetdash{}{0pt}%
\pgfsys@defobject{currentmarker}{\pgfqpoint{0.000000in}{-0.027778in}}{\pgfqpoint{0.000000in}{0.000000in}}{%
\pgfpathmoveto{\pgfqpoint{0.000000in}{0.000000in}}%
\pgfpathlineto{\pgfqpoint{0.000000in}{-0.027778in}}%
\pgfusepath{stroke,fill}%
}%
\begin{pgfscope}%
\pgfsys@transformshift{1.023388in}{0.417642in}%
\pgfsys@useobject{currentmarker}{}%
\end{pgfscope}%
\end{pgfscope}%
\begin{pgfscope}%
\pgfpathrectangle{\pgfqpoint{0.509263in}{0.417642in}}{\pgfqpoint{3.509067in}{2.050688in}}%
\pgfusepath{clip}%
\pgfsetrectcap%
\pgfsetroundjoin%
\pgfsetlinewidth{0.803000pt}%
\definecolor{currentstroke}{rgb}{0.850000,0.850000,0.850000}%
\pgfsetstrokecolor{currentstroke}%
\pgfsetdash{}{0pt}%
\pgfpathmoveto{\pgfqpoint{1.053897in}{0.417642in}}%
\pgfpathlineto{\pgfqpoint{1.053897in}{2.468330in}}%
\pgfusepath{stroke}%
\end{pgfscope}%
\begin{pgfscope}%
\pgfsetbuttcap%
\pgfsetroundjoin%
\definecolor{currentfill}{rgb}{0.000000,0.000000,0.000000}%
\pgfsetfillcolor{currentfill}%
\pgfsetlinewidth{0.602250pt}%
\definecolor{currentstroke}{rgb}{0.000000,0.000000,0.000000}%
\pgfsetstrokecolor{currentstroke}%
\pgfsetdash{}{0pt}%
\pgfsys@defobject{currentmarker}{\pgfqpoint{0.000000in}{-0.027778in}}{\pgfqpoint{0.000000in}{0.000000in}}{%
\pgfpathmoveto{\pgfqpoint{0.000000in}{0.000000in}}%
\pgfpathlineto{\pgfqpoint{0.000000in}{-0.027778in}}%
\pgfusepath{stroke,fill}%
}%
\begin{pgfscope}%
\pgfsys@transformshift{1.053897in}{0.417642in}%
\pgfsys@useobject{currentmarker}{}%
\end{pgfscope}%
\end{pgfscope}%
\begin{pgfscope}%
\pgfpathrectangle{\pgfqpoint{0.509263in}{0.417642in}}{\pgfqpoint{3.509067in}{2.050688in}}%
\pgfusepath{clip}%
\pgfsetrectcap%
\pgfsetroundjoin%
\pgfsetlinewidth{0.803000pt}%
\definecolor{currentstroke}{rgb}{0.850000,0.850000,0.850000}%
\pgfsetstrokecolor{currentstroke}%
\pgfsetdash{}{0pt}%
\pgfpathmoveto{\pgfqpoint{1.080325in}{0.417642in}}%
\pgfpathlineto{\pgfqpoint{1.080325in}{2.468330in}}%
\pgfusepath{stroke}%
\end{pgfscope}%
\begin{pgfscope}%
\pgfsetbuttcap%
\pgfsetroundjoin%
\definecolor{currentfill}{rgb}{0.000000,0.000000,0.000000}%
\pgfsetfillcolor{currentfill}%
\pgfsetlinewidth{0.602250pt}%
\definecolor{currentstroke}{rgb}{0.000000,0.000000,0.000000}%
\pgfsetstrokecolor{currentstroke}%
\pgfsetdash{}{0pt}%
\pgfsys@defobject{currentmarker}{\pgfqpoint{0.000000in}{-0.027778in}}{\pgfqpoint{0.000000in}{0.000000in}}{%
\pgfpathmoveto{\pgfqpoint{0.000000in}{0.000000in}}%
\pgfpathlineto{\pgfqpoint{0.000000in}{-0.027778in}}%
\pgfusepath{stroke,fill}%
}%
\begin{pgfscope}%
\pgfsys@transformshift{1.080325in}{0.417642in}%
\pgfsys@useobject{currentmarker}{}%
\end{pgfscope}%
\end{pgfscope}%
\begin{pgfscope}%
\pgfpathrectangle{\pgfqpoint{0.509263in}{0.417642in}}{\pgfqpoint{3.509067in}{2.050688in}}%
\pgfusepath{clip}%
\pgfsetrectcap%
\pgfsetroundjoin%
\pgfsetlinewidth{0.803000pt}%
\definecolor{currentstroke}{rgb}{0.850000,0.850000,0.850000}%
\pgfsetstrokecolor{currentstroke}%
\pgfsetdash{}{0pt}%
\pgfpathmoveto{\pgfqpoint{1.103637in}{0.417642in}}%
\pgfpathlineto{\pgfqpoint{1.103637in}{2.468330in}}%
\pgfusepath{stroke}%
\end{pgfscope}%
\begin{pgfscope}%
\pgfsetbuttcap%
\pgfsetroundjoin%
\definecolor{currentfill}{rgb}{0.000000,0.000000,0.000000}%
\pgfsetfillcolor{currentfill}%
\pgfsetlinewidth{0.602250pt}%
\definecolor{currentstroke}{rgb}{0.000000,0.000000,0.000000}%
\pgfsetstrokecolor{currentstroke}%
\pgfsetdash{}{0pt}%
\pgfsys@defobject{currentmarker}{\pgfqpoint{0.000000in}{-0.027778in}}{\pgfqpoint{0.000000in}{0.000000in}}{%
\pgfpathmoveto{\pgfqpoint{0.000000in}{0.000000in}}%
\pgfpathlineto{\pgfqpoint{0.000000in}{-0.027778in}}%
\pgfusepath{stroke,fill}%
}%
\begin{pgfscope}%
\pgfsys@transformshift{1.103637in}{0.417642in}%
\pgfsys@useobject{currentmarker}{}%
\end{pgfscope}%
\end{pgfscope}%
\begin{pgfscope}%
\pgfpathrectangle{\pgfqpoint{0.509263in}{0.417642in}}{\pgfqpoint{3.509067in}{2.050688in}}%
\pgfusepath{clip}%
\pgfsetrectcap%
\pgfsetroundjoin%
\pgfsetlinewidth{0.803000pt}%
\definecolor{currentstroke}{rgb}{0.850000,0.850000,0.850000}%
\pgfsetstrokecolor{currentstroke}%
\pgfsetdash{}{0pt}%
\pgfpathmoveto{\pgfqpoint{1.261676in}{0.417642in}}%
\pgfpathlineto{\pgfqpoint{1.261676in}{2.468330in}}%
\pgfusepath{stroke}%
\end{pgfscope}%
\begin{pgfscope}%
\pgfsetbuttcap%
\pgfsetroundjoin%
\definecolor{currentfill}{rgb}{0.000000,0.000000,0.000000}%
\pgfsetfillcolor{currentfill}%
\pgfsetlinewidth{0.602250pt}%
\definecolor{currentstroke}{rgb}{0.000000,0.000000,0.000000}%
\pgfsetstrokecolor{currentstroke}%
\pgfsetdash{}{0pt}%
\pgfsys@defobject{currentmarker}{\pgfqpoint{0.000000in}{-0.027778in}}{\pgfqpoint{0.000000in}{0.000000in}}{%
\pgfpathmoveto{\pgfqpoint{0.000000in}{0.000000in}}%
\pgfpathlineto{\pgfqpoint{0.000000in}{-0.027778in}}%
\pgfusepath{stroke,fill}%
}%
\begin{pgfscope}%
\pgfsys@transformshift{1.261676in}{0.417642in}%
\pgfsys@useobject{currentmarker}{}%
\end{pgfscope}%
\end{pgfscope}%
\begin{pgfscope}%
\pgfpathrectangle{\pgfqpoint{0.509263in}{0.417642in}}{\pgfqpoint{3.509067in}{2.050688in}}%
\pgfusepath{clip}%
\pgfsetrectcap%
\pgfsetroundjoin%
\pgfsetlinewidth{0.803000pt}%
\definecolor{currentstroke}{rgb}{0.850000,0.850000,0.850000}%
\pgfsetstrokecolor{currentstroke}%
\pgfsetdash{}{0pt}%
\pgfpathmoveto{\pgfqpoint{1.341925in}{0.417642in}}%
\pgfpathlineto{\pgfqpoint{1.341925in}{2.468330in}}%
\pgfusepath{stroke}%
\end{pgfscope}%
\begin{pgfscope}%
\pgfsetbuttcap%
\pgfsetroundjoin%
\definecolor{currentfill}{rgb}{0.000000,0.000000,0.000000}%
\pgfsetfillcolor{currentfill}%
\pgfsetlinewidth{0.602250pt}%
\definecolor{currentstroke}{rgb}{0.000000,0.000000,0.000000}%
\pgfsetstrokecolor{currentstroke}%
\pgfsetdash{}{0pt}%
\pgfsys@defobject{currentmarker}{\pgfqpoint{0.000000in}{-0.027778in}}{\pgfqpoint{0.000000in}{0.000000in}}{%
\pgfpathmoveto{\pgfqpoint{0.000000in}{0.000000in}}%
\pgfpathlineto{\pgfqpoint{0.000000in}{-0.027778in}}%
\pgfusepath{stroke,fill}%
}%
\begin{pgfscope}%
\pgfsys@transformshift{1.341925in}{0.417642in}%
\pgfsys@useobject{currentmarker}{}%
\end{pgfscope}%
\end{pgfscope}%
\begin{pgfscope}%
\pgfpathrectangle{\pgfqpoint{0.509263in}{0.417642in}}{\pgfqpoint{3.509067in}{2.050688in}}%
\pgfusepath{clip}%
\pgfsetrectcap%
\pgfsetroundjoin%
\pgfsetlinewidth{0.803000pt}%
\definecolor{currentstroke}{rgb}{0.850000,0.850000,0.850000}%
\pgfsetstrokecolor{currentstroke}%
\pgfsetdash{}{0pt}%
\pgfpathmoveto{\pgfqpoint{1.398862in}{0.417642in}}%
\pgfpathlineto{\pgfqpoint{1.398862in}{2.468330in}}%
\pgfusepath{stroke}%
\end{pgfscope}%
\begin{pgfscope}%
\pgfsetbuttcap%
\pgfsetroundjoin%
\definecolor{currentfill}{rgb}{0.000000,0.000000,0.000000}%
\pgfsetfillcolor{currentfill}%
\pgfsetlinewidth{0.602250pt}%
\definecolor{currentstroke}{rgb}{0.000000,0.000000,0.000000}%
\pgfsetstrokecolor{currentstroke}%
\pgfsetdash{}{0pt}%
\pgfsys@defobject{currentmarker}{\pgfqpoint{0.000000in}{-0.027778in}}{\pgfqpoint{0.000000in}{0.000000in}}{%
\pgfpathmoveto{\pgfqpoint{0.000000in}{0.000000in}}%
\pgfpathlineto{\pgfqpoint{0.000000in}{-0.027778in}}%
\pgfusepath{stroke,fill}%
}%
\begin{pgfscope}%
\pgfsys@transformshift{1.398862in}{0.417642in}%
\pgfsys@useobject{currentmarker}{}%
\end{pgfscope}%
\end{pgfscope}%
\begin{pgfscope}%
\pgfpathrectangle{\pgfqpoint{0.509263in}{0.417642in}}{\pgfqpoint{3.509067in}{2.050688in}}%
\pgfusepath{clip}%
\pgfsetrectcap%
\pgfsetroundjoin%
\pgfsetlinewidth{0.803000pt}%
\definecolor{currentstroke}{rgb}{0.850000,0.850000,0.850000}%
\pgfsetstrokecolor{currentstroke}%
\pgfsetdash{}{0pt}%
\pgfpathmoveto{\pgfqpoint{1.443026in}{0.417642in}}%
\pgfpathlineto{\pgfqpoint{1.443026in}{2.468330in}}%
\pgfusepath{stroke}%
\end{pgfscope}%
\begin{pgfscope}%
\pgfsetbuttcap%
\pgfsetroundjoin%
\definecolor{currentfill}{rgb}{0.000000,0.000000,0.000000}%
\pgfsetfillcolor{currentfill}%
\pgfsetlinewidth{0.602250pt}%
\definecolor{currentstroke}{rgb}{0.000000,0.000000,0.000000}%
\pgfsetstrokecolor{currentstroke}%
\pgfsetdash{}{0pt}%
\pgfsys@defobject{currentmarker}{\pgfqpoint{0.000000in}{-0.027778in}}{\pgfqpoint{0.000000in}{0.000000in}}{%
\pgfpathmoveto{\pgfqpoint{0.000000in}{0.000000in}}%
\pgfpathlineto{\pgfqpoint{0.000000in}{-0.027778in}}%
\pgfusepath{stroke,fill}%
}%
\begin{pgfscope}%
\pgfsys@transformshift{1.443026in}{0.417642in}%
\pgfsys@useobject{currentmarker}{}%
\end{pgfscope}%
\end{pgfscope}%
\begin{pgfscope}%
\pgfpathrectangle{\pgfqpoint{0.509263in}{0.417642in}}{\pgfqpoint{3.509067in}{2.050688in}}%
\pgfusepath{clip}%
\pgfsetrectcap%
\pgfsetroundjoin%
\pgfsetlinewidth{0.803000pt}%
\definecolor{currentstroke}{rgb}{0.850000,0.850000,0.850000}%
\pgfsetstrokecolor{currentstroke}%
\pgfsetdash{}{0pt}%
\pgfpathmoveto{\pgfqpoint{1.479111in}{0.417642in}}%
\pgfpathlineto{\pgfqpoint{1.479111in}{2.468330in}}%
\pgfusepath{stroke}%
\end{pgfscope}%
\begin{pgfscope}%
\pgfsetbuttcap%
\pgfsetroundjoin%
\definecolor{currentfill}{rgb}{0.000000,0.000000,0.000000}%
\pgfsetfillcolor{currentfill}%
\pgfsetlinewidth{0.602250pt}%
\definecolor{currentstroke}{rgb}{0.000000,0.000000,0.000000}%
\pgfsetstrokecolor{currentstroke}%
\pgfsetdash{}{0pt}%
\pgfsys@defobject{currentmarker}{\pgfqpoint{0.000000in}{-0.027778in}}{\pgfqpoint{0.000000in}{0.000000in}}{%
\pgfpathmoveto{\pgfqpoint{0.000000in}{0.000000in}}%
\pgfpathlineto{\pgfqpoint{0.000000in}{-0.027778in}}%
\pgfusepath{stroke,fill}%
}%
\begin{pgfscope}%
\pgfsys@transformshift{1.479111in}{0.417642in}%
\pgfsys@useobject{currentmarker}{}%
\end{pgfscope}%
\end{pgfscope}%
\begin{pgfscope}%
\pgfpathrectangle{\pgfqpoint{0.509263in}{0.417642in}}{\pgfqpoint{3.509067in}{2.050688in}}%
\pgfusepath{clip}%
\pgfsetrectcap%
\pgfsetroundjoin%
\pgfsetlinewidth{0.803000pt}%
\definecolor{currentstroke}{rgb}{0.850000,0.850000,0.850000}%
\pgfsetstrokecolor{currentstroke}%
\pgfsetdash{}{0pt}%
\pgfpathmoveto{\pgfqpoint{1.509620in}{0.417642in}}%
\pgfpathlineto{\pgfqpoint{1.509620in}{2.468330in}}%
\pgfusepath{stroke}%
\end{pgfscope}%
\begin{pgfscope}%
\pgfsetbuttcap%
\pgfsetroundjoin%
\definecolor{currentfill}{rgb}{0.000000,0.000000,0.000000}%
\pgfsetfillcolor{currentfill}%
\pgfsetlinewidth{0.602250pt}%
\definecolor{currentstroke}{rgb}{0.000000,0.000000,0.000000}%
\pgfsetstrokecolor{currentstroke}%
\pgfsetdash{}{0pt}%
\pgfsys@defobject{currentmarker}{\pgfqpoint{0.000000in}{-0.027778in}}{\pgfqpoint{0.000000in}{0.000000in}}{%
\pgfpathmoveto{\pgfqpoint{0.000000in}{0.000000in}}%
\pgfpathlineto{\pgfqpoint{0.000000in}{-0.027778in}}%
\pgfusepath{stroke,fill}%
}%
\begin{pgfscope}%
\pgfsys@transformshift{1.509620in}{0.417642in}%
\pgfsys@useobject{currentmarker}{}%
\end{pgfscope}%
\end{pgfscope}%
\begin{pgfscope}%
\pgfpathrectangle{\pgfqpoint{0.509263in}{0.417642in}}{\pgfqpoint{3.509067in}{2.050688in}}%
\pgfusepath{clip}%
\pgfsetrectcap%
\pgfsetroundjoin%
\pgfsetlinewidth{0.803000pt}%
\definecolor{currentstroke}{rgb}{0.850000,0.850000,0.850000}%
\pgfsetstrokecolor{currentstroke}%
\pgfsetdash{}{0pt}%
\pgfpathmoveto{\pgfqpoint{1.536048in}{0.417642in}}%
\pgfpathlineto{\pgfqpoint{1.536048in}{2.468330in}}%
\pgfusepath{stroke}%
\end{pgfscope}%
\begin{pgfscope}%
\pgfsetbuttcap%
\pgfsetroundjoin%
\definecolor{currentfill}{rgb}{0.000000,0.000000,0.000000}%
\pgfsetfillcolor{currentfill}%
\pgfsetlinewidth{0.602250pt}%
\definecolor{currentstroke}{rgb}{0.000000,0.000000,0.000000}%
\pgfsetstrokecolor{currentstroke}%
\pgfsetdash{}{0pt}%
\pgfsys@defobject{currentmarker}{\pgfqpoint{0.000000in}{-0.027778in}}{\pgfqpoint{0.000000in}{0.000000in}}{%
\pgfpathmoveto{\pgfqpoint{0.000000in}{0.000000in}}%
\pgfpathlineto{\pgfqpoint{0.000000in}{-0.027778in}}%
\pgfusepath{stroke,fill}%
}%
\begin{pgfscope}%
\pgfsys@transformshift{1.536048in}{0.417642in}%
\pgfsys@useobject{currentmarker}{}%
\end{pgfscope}%
\end{pgfscope}%
\begin{pgfscope}%
\pgfpathrectangle{\pgfqpoint{0.509263in}{0.417642in}}{\pgfqpoint{3.509067in}{2.050688in}}%
\pgfusepath{clip}%
\pgfsetrectcap%
\pgfsetroundjoin%
\pgfsetlinewidth{0.803000pt}%
\definecolor{currentstroke}{rgb}{0.850000,0.850000,0.850000}%
\pgfsetstrokecolor{currentstroke}%
\pgfsetdash{}{0pt}%
\pgfpathmoveto{\pgfqpoint{1.559360in}{0.417642in}}%
\pgfpathlineto{\pgfqpoint{1.559360in}{2.468330in}}%
\pgfusepath{stroke}%
\end{pgfscope}%
\begin{pgfscope}%
\pgfsetbuttcap%
\pgfsetroundjoin%
\definecolor{currentfill}{rgb}{0.000000,0.000000,0.000000}%
\pgfsetfillcolor{currentfill}%
\pgfsetlinewidth{0.602250pt}%
\definecolor{currentstroke}{rgb}{0.000000,0.000000,0.000000}%
\pgfsetstrokecolor{currentstroke}%
\pgfsetdash{}{0pt}%
\pgfsys@defobject{currentmarker}{\pgfqpoint{0.000000in}{-0.027778in}}{\pgfqpoint{0.000000in}{0.000000in}}{%
\pgfpathmoveto{\pgfqpoint{0.000000in}{0.000000in}}%
\pgfpathlineto{\pgfqpoint{0.000000in}{-0.027778in}}%
\pgfusepath{stroke,fill}%
}%
\begin{pgfscope}%
\pgfsys@transformshift{1.559360in}{0.417642in}%
\pgfsys@useobject{currentmarker}{}%
\end{pgfscope}%
\end{pgfscope}%
\begin{pgfscope}%
\pgfpathrectangle{\pgfqpoint{0.509263in}{0.417642in}}{\pgfqpoint{3.509067in}{2.050688in}}%
\pgfusepath{clip}%
\pgfsetrectcap%
\pgfsetroundjoin%
\pgfsetlinewidth{0.803000pt}%
\definecolor{currentstroke}{rgb}{0.850000,0.850000,0.850000}%
\pgfsetstrokecolor{currentstroke}%
\pgfsetdash{}{0pt}%
\pgfpathmoveto{\pgfqpoint{1.717399in}{0.417642in}}%
\pgfpathlineto{\pgfqpoint{1.717399in}{2.468330in}}%
\pgfusepath{stroke}%
\end{pgfscope}%
\begin{pgfscope}%
\pgfsetbuttcap%
\pgfsetroundjoin%
\definecolor{currentfill}{rgb}{0.000000,0.000000,0.000000}%
\pgfsetfillcolor{currentfill}%
\pgfsetlinewidth{0.602250pt}%
\definecolor{currentstroke}{rgb}{0.000000,0.000000,0.000000}%
\pgfsetstrokecolor{currentstroke}%
\pgfsetdash{}{0pt}%
\pgfsys@defobject{currentmarker}{\pgfqpoint{0.000000in}{-0.027778in}}{\pgfqpoint{0.000000in}{0.000000in}}{%
\pgfpathmoveto{\pgfqpoint{0.000000in}{0.000000in}}%
\pgfpathlineto{\pgfqpoint{0.000000in}{-0.027778in}}%
\pgfusepath{stroke,fill}%
}%
\begin{pgfscope}%
\pgfsys@transformshift{1.717399in}{0.417642in}%
\pgfsys@useobject{currentmarker}{}%
\end{pgfscope}%
\end{pgfscope}%
\begin{pgfscope}%
\pgfpathrectangle{\pgfqpoint{0.509263in}{0.417642in}}{\pgfqpoint{3.509067in}{2.050688in}}%
\pgfusepath{clip}%
\pgfsetrectcap%
\pgfsetroundjoin%
\pgfsetlinewidth{0.803000pt}%
\definecolor{currentstroke}{rgb}{0.850000,0.850000,0.850000}%
\pgfsetstrokecolor{currentstroke}%
\pgfsetdash{}{0pt}%
\pgfpathmoveto{\pgfqpoint{1.797647in}{0.417642in}}%
\pgfpathlineto{\pgfqpoint{1.797647in}{2.468330in}}%
\pgfusepath{stroke}%
\end{pgfscope}%
\begin{pgfscope}%
\pgfsetbuttcap%
\pgfsetroundjoin%
\definecolor{currentfill}{rgb}{0.000000,0.000000,0.000000}%
\pgfsetfillcolor{currentfill}%
\pgfsetlinewidth{0.602250pt}%
\definecolor{currentstroke}{rgb}{0.000000,0.000000,0.000000}%
\pgfsetstrokecolor{currentstroke}%
\pgfsetdash{}{0pt}%
\pgfsys@defobject{currentmarker}{\pgfqpoint{0.000000in}{-0.027778in}}{\pgfqpoint{0.000000in}{0.000000in}}{%
\pgfpathmoveto{\pgfqpoint{0.000000in}{0.000000in}}%
\pgfpathlineto{\pgfqpoint{0.000000in}{-0.027778in}}%
\pgfusepath{stroke,fill}%
}%
\begin{pgfscope}%
\pgfsys@transformshift{1.797647in}{0.417642in}%
\pgfsys@useobject{currentmarker}{}%
\end{pgfscope}%
\end{pgfscope}%
\begin{pgfscope}%
\pgfpathrectangle{\pgfqpoint{0.509263in}{0.417642in}}{\pgfqpoint{3.509067in}{2.050688in}}%
\pgfusepath{clip}%
\pgfsetrectcap%
\pgfsetroundjoin%
\pgfsetlinewidth{0.803000pt}%
\definecolor{currentstroke}{rgb}{0.850000,0.850000,0.850000}%
\pgfsetstrokecolor{currentstroke}%
\pgfsetdash{}{0pt}%
\pgfpathmoveto{\pgfqpoint{1.854585in}{0.417642in}}%
\pgfpathlineto{\pgfqpoint{1.854585in}{2.468330in}}%
\pgfusepath{stroke}%
\end{pgfscope}%
\begin{pgfscope}%
\pgfsetbuttcap%
\pgfsetroundjoin%
\definecolor{currentfill}{rgb}{0.000000,0.000000,0.000000}%
\pgfsetfillcolor{currentfill}%
\pgfsetlinewidth{0.602250pt}%
\definecolor{currentstroke}{rgb}{0.000000,0.000000,0.000000}%
\pgfsetstrokecolor{currentstroke}%
\pgfsetdash{}{0pt}%
\pgfsys@defobject{currentmarker}{\pgfqpoint{0.000000in}{-0.027778in}}{\pgfqpoint{0.000000in}{0.000000in}}{%
\pgfpathmoveto{\pgfqpoint{0.000000in}{0.000000in}}%
\pgfpathlineto{\pgfqpoint{0.000000in}{-0.027778in}}%
\pgfusepath{stroke,fill}%
}%
\begin{pgfscope}%
\pgfsys@transformshift{1.854585in}{0.417642in}%
\pgfsys@useobject{currentmarker}{}%
\end{pgfscope}%
\end{pgfscope}%
\begin{pgfscope}%
\pgfpathrectangle{\pgfqpoint{0.509263in}{0.417642in}}{\pgfqpoint{3.509067in}{2.050688in}}%
\pgfusepath{clip}%
\pgfsetrectcap%
\pgfsetroundjoin%
\pgfsetlinewidth{0.803000pt}%
\definecolor{currentstroke}{rgb}{0.850000,0.850000,0.850000}%
\pgfsetstrokecolor{currentstroke}%
\pgfsetdash{}{0pt}%
\pgfpathmoveto{\pgfqpoint{1.898749in}{0.417642in}}%
\pgfpathlineto{\pgfqpoint{1.898749in}{2.468330in}}%
\pgfusepath{stroke}%
\end{pgfscope}%
\begin{pgfscope}%
\pgfsetbuttcap%
\pgfsetroundjoin%
\definecolor{currentfill}{rgb}{0.000000,0.000000,0.000000}%
\pgfsetfillcolor{currentfill}%
\pgfsetlinewidth{0.602250pt}%
\definecolor{currentstroke}{rgb}{0.000000,0.000000,0.000000}%
\pgfsetstrokecolor{currentstroke}%
\pgfsetdash{}{0pt}%
\pgfsys@defobject{currentmarker}{\pgfqpoint{0.000000in}{-0.027778in}}{\pgfqpoint{0.000000in}{0.000000in}}{%
\pgfpathmoveto{\pgfqpoint{0.000000in}{0.000000in}}%
\pgfpathlineto{\pgfqpoint{0.000000in}{-0.027778in}}%
\pgfusepath{stroke,fill}%
}%
\begin{pgfscope}%
\pgfsys@transformshift{1.898749in}{0.417642in}%
\pgfsys@useobject{currentmarker}{}%
\end{pgfscope}%
\end{pgfscope}%
\begin{pgfscope}%
\pgfpathrectangle{\pgfqpoint{0.509263in}{0.417642in}}{\pgfqpoint{3.509067in}{2.050688in}}%
\pgfusepath{clip}%
\pgfsetrectcap%
\pgfsetroundjoin%
\pgfsetlinewidth{0.803000pt}%
\definecolor{currentstroke}{rgb}{0.850000,0.850000,0.850000}%
\pgfsetstrokecolor{currentstroke}%
\pgfsetdash{}{0pt}%
\pgfpathmoveto{\pgfqpoint{1.934834in}{0.417642in}}%
\pgfpathlineto{\pgfqpoint{1.934834in}{2.468330in}}%
\pgfusepath{stroke}%
\end{pgfscope}%
\begin{pgfscope}%
\pgfsetbuttcap%
\pgfsetroundjoin%
\definecolor{currentfill}{rgb}{0.000000,0.000000,0.000000}%
\pgfsetfillcolor{currentfill}%
\pgfsetlinewidth{0.602250pt}%
\definecolor{currentstroke}{rgb}{0.000000,0.000000,0.000000}%
\pgfsetstrokecolor{currentstroke}%
\pgfsetdash{}{0pt}%
\pgfsys@defobject{currentmarker}{\pgfqpoint{0.000000in}{-0.027778in}}{\pgfqpoint{0.000000in}{0.000000in}}{%
\pgfpathmoveto{\pgfqpoint{0.000000in}{0.000000in}}%
\pgfpathlineto{\pgfqpoint{0.000000in}{-0.027778in}}%
\pgfusepath{stroke,fill}%
}%
\begin{pgfscope}%
\pgfsys@transformshift{1.934834in}{0.417642in}%
\pgfsys@useobject{currentmarker}{}%
\end{pgfscope}%
\end{pgfscope}%
\begin{pgfscope}%
\pgfpathrectangle{\pgfqpoint{0.509263in}{0.417642in}}{\pgfqpoint{3.509067in}{2.050688in}}%
\pgfusepath{clip}%
\pgfsetrectcap%
\pgfsetroundjoin%
\pgfsetlinewidth{0.803000pt}%
\definecolor{currentstroke}{rgb}{0.850000,0.850000,0.850000}%
\pgfsetstrokecolor{currentstroke}%
\pgfsetdash{}{0pt}%
\pgfpathmoveto{\pgfqpoint{1.965343in}{0.417642in}}%
\pgfpathlineto{\pgfqpoint{1.965343in}{2.468330in}}%
\pgfusepath{stroke}%
\end{pgfscope}%
\begin{pgfscope}%
\pgfsetbuttcap%
\pgfsetroundjoin%
\definecolor{currentfill}{rgb}{0.000000,0.000000,0.000000}%
\pgfsetfillcolor{currentfill}%
\pgfsetlinewidth{0.602250pt}%
\definecolor{currentstroke}{rgb}{0.000000,0.000000,0.000000}%
\pgfsetstrokecolor{currentstroke}%
\pgfsetdash{}{0pt}%
\pgfsys@defobject{currentmarker}{\pgfqpoint{0.000000in}{-0.027778in}}{\pgfqpoint{0.000000in}{0.000000in}}{%
\pgfpathmoveto{\pgfqpoint{0.000000in}{0.000000in}}%
\pgfpathlineto{\pgfqpoint{0.000000in}{-0.027778in}}%
\pgfusepath{stroke,fill}%
}%
\begin{pgfscope}%
\pgfsys@transformshift{1.965343in}{0.417642in}%
\pgfsys@useobject{currentmarker}{}%
\end{pgfscope}%
\end{pgfscope}%
\begin{pgfscope}%
\pgfpathrectangle{\pgfqpoint{0.509263in}{0.417642in}}{\pgfqpoint{3.509067in}{2.050688in}}%
\pgfusepath{clip}%
\pgfsetrectcap%
\pgfsetroundjoin%
\pgfsetlinewidth{0.803000pt}%
\definecolor{currentstroke}{rgb}{0.850000,0.850000,0.850000}%
\pgfsetstrokecolor{currentstroke}%
\pgfsetdash{}{0pt}%
\pgfpathmoveto{\pgfqpoint{1.991771in}{0.417642in}}%
\pgfpathlineto{\pgfqpoint{1.991771in}{2.468330in}}%
\pgfusepath{stroke}%
\end{pgfscope}%
\begin{pgfscope}%
\pgfsetbuttcap%
\pgfsetroundjoin%
\definecolor{currentfill}{rgb}{0.000000,0.000000,0.000000}%
\pgfsetfillcolor{currentfill}%
\pgfsetlinewidth{0.602250pt}%
\definecolor{currentstroke}{rgb}{0.000000,0.000000,0.000000}%
\pgfsetstrokecolor{currentstroke}%
\pgfsetdash{}{0pt}%
\pgfsys@defobject{currentmarker}{\pgfqpoint{0.000000in}{-0.027778in}}{\pgfqpoint{0.000000in}{0.000000in}}{%
\pgfpathmoveto{\pgfqpoint{0.000000in}{0.000000in}}%
\pgfpathlineto{\pgfqpoint{0.000000in}{-0.027778in}}%
\pgfusepath{stroke,fill}%
}%
\begin{pgfscope}%
\pgfsys@transformshift{1.991771in}{0.417642in}%
\pgfsys@useobject{currentmarker}{}%
\end{pgfscope}%
\end{pgfscope}%
\begin{pgfscope}%
\pgfpathrectangle{\pgfqpoint{0.509263in}{0.417642in}}{\pgfqpoint{3.509067in}{2.050688in}}%
\pgfusepath{clip}%
\pgfsetrectcap%
\pgfsetroundjoin%
\pgfsetlinewidth{0.803000pt}%
\definecolor{currentstroke}{rgb}{0.850000,0.850000,0.850000}%
\pgfsetstrokecolor{currentstroke}%
\pgfsetdash{}{0pt}%
\pgfpathmoveto{\pgfqpoint{2.015083in}{0.417642in}}%
\pgfpathlineto{\pgfqpoint{2.015083in}{2.468330in}}%
\pgfusepath{stroke}%
\end{pgfscope}%
\begin{pgfscope}%
\pgfsetbuttcap%
\pgfsetroundjoin%
\definecolor{currentfill}{rgb}{0.000000,0.000000,0.000000}%
\pgfsetfillcolor{currentfill}%
\pgfsetlinewidth{0.602250pt}%
\definecolor{currentstroke}{rgb}{0.000000,0.000000,0.000000}%
\pgfsetstrokecolor{currentstroke}%
\pgfsetdash{}{0pt}%
\pgfsys@defobject{currentmarker}{\pgfqpoint{0.000000in}{-0.027778in}}{\pgfqpoint{0.000000in}{0.000000in}}{%
\pgfpathmoveto{\pgfqpoint{0.000000in}{0.000000in}}%
\pgfpathlineto{\pgfqpoint{0.000000in}{-0.027778in}}%
\pgfusepath{stroke,fill}%
}%
\begin{pgfscope}%
\pgfsys@transformshift{2.015083in}{0.417642in}%
\pgfsys@useobject{currentmarker}{}%
\end{pgfscope}%
\end{pgfscope}%
\begin{pgfscope}%
\pgfpathrectangle{\pgfqpoint{0.509263in}{0.417642in}}{\pgfqpoint{3.509067in}{2.050688in}}%
\pgfusepath{clip}%
\pgfsetrectcap%
\pgfsetroundjoin%
\pgfsetlinewidth{0.803000pt}%
\definecolor{currentstroke}{rgb}{0.850000,0.850000,0.850000}%
\pgfsetstrokecolor{currentstroke}%
\pgfsetdash{}{0pt}%
\pgfpathmoveto{\pgfqpoint{2.173122in}{0.417642in}}%
\pgfpathlineto{\pgfqpoint{2.173122in}{2.468330in}}%
\pgfusepath{stroke}%
\end{pgfscope}%
\begin{pgfscope}%
\pgfsetbuttcap%
\pgfsetroundjoin%
\definecolor{currentfill}{rgb}{0.000000,0.000000,0.000000}%
\pgfsetfillcolor{currentfill}%
\pgfsetlinewidth{0.602250pt}%
\definecolor{currentstroke}{rgb}{0.000000,0.000000,0.000000}%
\pgfsetstrokecolor{currentstroke}%
\pgfsetdash{}{0pt}%
\pgfsys@defobject{currentmarker}{\pgfqpoint{0.000000in}{-0.027778in}}{\pgfqpoint{0.000000in}{0.000000in}}{%
\pgfpathmoveto{\pgfqpoint{0.000000in}{0.000000in}}%
\pgfpathlineto{\pgfqpoint{0.000000in}{-0.027778in}}%
\pgfusepath{stroke,fill}%
}%
\begin{pgfscope}%
\pgfsys@transformshift{2.173122in}{0.417642in}%
\pgfsys@useobject{currentmarker}{}%
\end{pgfscope}%
\end{pgfscope}%
\begin{pgfscope}%
\pgfpathrectangle{\pgfqpoint{0.509263in}{0.417642in}}{\pgfqpoint{3.509067in}{2.050688in}}%
\pgfusepath{clip}%
\pgfsetrectcap%
\pgfsetroundjoin%
\pgfsetlinewidth{0.803000pt}%
\definecolor{currentstroke}{rgb}{0.850000,0.850000,0.850000}%
\pgfsetstrokecolor{currentstroke}%
\pgfsetdash{}{0pt}%
\pgfpathmoveto{\pgfqpoint{2.253370in}{0.417642in}}%
\pgfpathlineto{\pgfqpoint{2.253370in}{2.468330in}}%
\pgfusepath{stroke}%
\end{pgfscope}%
\begin{pgfscope}%
\pgfsetbuttcap%
\pgfsetroundjoin%
\definecolor{currentfill}{rgb}{0.000000,0.000000,0.000000}%
\pgfsetfillcolor{currentfill}%
\pgfsetlinewidth{0.602250pt}%
\definecolor{currentstroke}{rgb}{0.000000,0.000000,0.000000}%
\pgfsetstrokecolor{currentstroke}%
\pgfsetdash{}{0pt}%
\pgfsys@defobject{currentmarker}{\pgfqpoint{0.000000in}{-0.027778in}}{\pgfqpoint{0.000000in}{0.000000in}}{%
\pgfpathmoveto{\pgfqpoint{0.000000in}{0.000000in}}%
\pgfpathlineto{\pgfqpoint{0.000000in}{-0.027778in}}%
\pgfusepath{stroke,fill}%
}%
\begin{pgfscope}%
\pgfsys@transformshift{2.253370in}{0.417642in}%
\pgfsys@useobject{currentmarker}{}%
\end{pgfscope}%
\end{pgfscope}%
\begin{pgfscope}%
\pgfpathrectangle{\pgfqpoint{0.509263in}{0.417642in}}{\pgfqpoint{3.509067in}{2.050688in}}%
\pgfusepath{clip}%
\pgfsetrectcap%
\pgfsetroundjoin%
\pgfsetlinewidth{0.803000pt}%
\definecolor{currentstroke}{rgb}{0.850000,0.850000,0.850000}%
\pgfsetstrokecolor{currentstroke}%
\pgfsetdash{}{0pt}%
\pgfpathmoveto{\pgfqpoint{2.310308in}{0.417642in}}%
\pgfpathlineto{\pgfqpoint{2.310308in}{2.468330in}}%
\pgfusepath{stroke}%
\end{pgfscope}%
\begin{pgfscope}%
\pgfsetbuttcap%
\pgfsetroundjoin%
\definecolor{currentfill}{rgb}{0.000000,0.000000,0.000000}%
\pgfsetfillcolor{currentfill}%
\pgfsetlinewidth{0.602250pt}%
\definecolor{currentstroke}{rgb}{0.000000,0.000000,0.000000}%
\pgfsetstrokecolor{currentstroke}%
\pgfsetdash{}{0pt}%
\pgfsys@defobject{currentmarker}{\pgfqpoint{0.000000in}{-0.027778in}}{\pgfqpoint{0.000000in}{0.000000in}}{%
\pgfpathmoveto{\pgfqpoint{0.000000in}{0.000000in}}%
\pgfpathlineto{\pgfqpoint{0.000000in}{-0.027778in}}%
\pgfusepath{stroke,fill}%
}%
\begin{pgfscope}%
\pgfsys@transformshift{2.310308in}{0.417642in}%
\pgfsys@useobject{currentmarker}{}%
\end{pgfscope}%
\end{pgfscope}%
\begin{pgfscope}%
\pgfpathrectangle{\pgfqpoint{0.509263in}{0.417642in}}{\pgfqpoint{3.509067in}{2.050688in}}%
\pgfusepath{clip}%
\pgfsetrectcap%
\pgfsetroundjoin%
\pgfsetlinewidth{0.803000pt}%
\definecolor{currentstroke}{rgb}{0.850000,0.850000,0.850000}%
\pgfsetstrokecolor{currentstroke}%
\pgfsetdash{}{0pt}%
\pgfpathmoveto{\pgfqpoint{2.354472in}{0.417642in}}%
\pgfpathlineto{\pgfqpoint{2.354472in}{2.468330in}}%
\pgfusepath{stroke}%
\end{pgfscope}%
\begin{pgfscope}%
\pgfsetbuttcap%
\pgfsetroundjoin%
\definecolor{currentfill}{rgb}{0.000000,0.000000,0.000000}%
\pgfsetfillcolor{currentfill}%
\pgfsetlinewidth{0.602250pt}%
\definecolor{currentstroke}{rgb}{0.000000,0.000000,0.000000}%
\pgfsetstrokecolor{currentstroke}%
\pgfsetdash{}{0pt}%
\pgfsys@defobject{currentmarker}{\pgfqpoint{0.000000in}{-0.027778in}}{\pgfqpoint{0.000000in}{0.000000in}}{%
\pgfpathmoveto{\pgfqpoint{0.000000in}{0.000000in}}%
\pgfpathlineto{\pgfqpoint{0.000000in}{-0.027778in}}%
\pgfusepath{stroke,fill}%
}%
\begin{pgfscope}%
\pgfsys@transformshift{2.354472in}{0.417642in}%
\pgfsys@useobject{currentmarker}{}%
\end{pgfscope}%
\end{pgfscope}%
\begin{pgfscope}%
\pgfpathrectangle{\pgfqpoint{0.509263in}{0.417642in}}{\pgfqpoint{3.509067in}{2.050688in}}%
\pgfusepath{clip}%
\pgfsetrectcap%
\pgfsetroundjoin%
\pgfsetlinewidth{0.803000pt}%
\definecolor{currentstroke}{rgb}{0.850000,0.850000,0.850000}%
\pgfsetstrokecolor{currentstroke}%
\pgfsetdash{}{0pt}%
\pgfpathmoveto{\pgfqpoint{2.390557in}{0.417642in}}%
\pgfpathlineto{\pgfqpoint{2.390557in}{2.468330in}}%
\pgfusepath{stroke}%
\end{pgfscope}%
\begin{pgfscope}%
\pgfsetbuttcap%
\pgfsetroundjoin%
\definecolor{currentfill}{rgb}{0.000000,0.000000,0.000000}%
\pgfsetfillcolor{currentfill}%
\pgfsetlinewidth{0.602250pt}%
\definecolor{currentstroke}{rgb}{0.000000,0.000000,0.000000}%
\pgfsetstrokecolor{currentstroke}%
\pgfsetdash{}{0pt}%
\pgfsys@defobject{currentmarker}{\pgfqpoint{0.000000in}{-0.027778in}}{\pgfqpoint{0.000000in}{0.000000in}}{%
\pgfpathmoveto{\pgfqpoint{0.000000in}{0.000000in}}%
\pgfpathlineto{\pgfqpoint{0.000000in}{-0.027778in}}%
\pgfusepath{stroke,fill}%
}%
\begin{pgfscope}%
\pgfsys@transformshift{2.390557in}{0.417642in}%
\pgfsys@useobject{currentmarker}{}%
\end{pgfscope}%
\end{pgfscope}%
\begin{pgfscope}%
\pgfpathrectangle{\pgfqpoint{0.509263in}{0.417642in}}{\pgfqpoint{3.509067in}{2.050688in}}%
\pgfusepath{clip}%
\pgfsetrectcap%
\pgfsetroundjoin%
\pgfsetlinewidth{0.803000pt}%
\definecolor{currentstroke}{rgb}{0.850000,0.850000,0.850000}%
\pgfsetstrokecolor{currentstroke}%
\pgfsetdash{}{0pt}%
\pgfpathmoveto{\pgfqpoint{2.421066in}{0.417642in}}%
\pgfpathlineto{\pgfqpoint{2.421066in}{2.468330in}}%
\pgfusepath{stroke}%
\end{pgfscope}%
\begin{pgfscope}%
\pgfsetbuttcap%
\pgfsetroundjoin%
\definecolor{currentfill}{rgb}{0.000000,0.000000,0.000000}%
\pgfsetfillcolor{currentfill}%
\pgfsetlinewidth{0.602250pt}%
\definecolor{currentstroke}{rgb}{0.000000,0.000000,0.000000}%
\pgfsetstrokecolor{currentstroke}%
\pgfsetdash{}{0pt}%
\pgfsys@defobject{currentmarker}{\pgfqpoint{0.000000in}{-0.027778in}}{\pgfqpoint{0.000000in}{0.000000in}}{%
\pgfpathmoveto{\pgfqpoint{0.000000in}{0.000000in}}%
\pgfpathlineto{\pgfqpoint{0.000000in}{-0.027778in}}%
\pgfusepath{stroke,fill}%
}%
\begin{pgfscope}%
\pgfsys@transformshift{2.421066in}{0.417642in}%
\pgfsys@useobject{currentmarker}{}%
\end{pgfscope}%
\end{pgfscope}%
\begin{pgfscope}%
\pgfpathrectangle{\pgfqpoint{0.509263in}{0.417642in}}{\pgfqpoint{3.509067in}{2.050688in}}%
\pgfusepath{clip}%
\pgfsetrectcap%
\pgfsetroundjoin%
\pgfsetlinewidth{0.803000pt}%
\definecolor{currentstroke}{rgb}{0.850000,0.850000,0.850000}%
\pgfsetstrokecolor{currentstroke}%
\pgfsetdash{}{0pt}%
\pgfpathmoveto{\pgfqpoint{2.447494in}{0.417642in}}%
\pgfpathlineto{\pgfqpoint{2.447494in}{2.468330in}}%
\pgfusepath{stroke}%
\end{pgfscope}%
\begin{pgfscope}%
\pgfsetbuttcap%
\pgfsetroundjoin%
\definecolor{currentfill}{rgb}{0.000000,0.000000,0.000000}%
\pgfsetfillcolor{currentfill}%
\pgfsetlinewidth{0.602250pt}%
\definecolor{currentstroke}{rgb}{0.000000,0.000000,0.000000}%
\pgfsetstrokecolor{currentstroke}%
\pgfsetdash{}{0pt}%
\pgfsys@defobject{currentmarker}{\pgfqpoint{0.000000in}{-0.027778in}}{\pgfqpoint{0.000000in}{0.000000in}}{%
\pgfpathmoveto{\pgfqpoint{0.000000in}{0.000000in}}%
\pgfpathlineto{\pgfqpoint{0.000000in}{-0.027778in}}%
\pgfusepath{stroke,fill}%
}%
\begin{pgfscope}%
\pgfsys@transformshift{2.447494in}{0.417642in}%
\pgfsys@useobject{currentmarker}{}%
\end{pgfscope}%
\end{pgfscope}%
\begin{pgfscope}%
\pgfpathrectangle{\pgfqpoint{0.509263in}{0.417642in}}{\pgfqpoint{3.509067in}{2.050688in}}%
\pgfusepath{clip}%
\pgfsetrectcap%
\pgfsetroundjoin%
\pgfsetlinewidth{0.803000pt}%
\definecolor{currentstroke}{rgb}{0.850000,0.850000,0.850000}%
\pgfsetstrokecolor{currentstroke}%
\pgfsetdash{}{0pt}%
\pgfpathmoveto{\pgfqpoint{2.470805in}{0.417642in}}%
\pgfpathlineto{\pgfqpoint{2.470805in}{2.468330in}}%
\pgfusepath{stroke}%
\end{pgfscope}%
\begin{pgfscope}%
\pgfsetbuttcap%
\pgfsetroundjoin%
\definecolor{currentfill}{rgb}{0.000000,0.000000,0.000000}%
\pgfsetfillcolor{currentfill}%
\pgfsetlinewidth{0.602250pt}%
\definecolor{currentstroke}{rgb}{0.000000,0.000000,0.000000}%
\pgfsetstrokecolor{currentstroke}%
\pgfsetdash{}{0pt}%
\pgfsys@defobject{currentmarker}{\pgfqpoint{0.000000in}{-0.027778in}}{\pgfqpoint{0.000000in}{0.000000in}}{%
\pgfpathmoveto{\pgfqpoint{0.000000in}{0.000000in}}%
\pgfpathlineto{\pgfqpoint{0.000000in}{-0.027778in}}%
\pgfusepath{stroke,fill}%
}%
\begin{pgfscope}%
\pgfsys@transformshift{2.470805in}{0.417642in}%
\pgfsys@useobject{currentmarker}{}%
\end{pgfscope}%
\end{pgfscope}%
\begin{pgfscope}%
\pgfpathrectangle{\pgfqpoint{0.509263in}{0.417642in}}{\pgfqpoint{3.509067in}{2.050688in}}%
\pgfusepath{clip}%
\pgfsetrectcap%
\pgfsetroundjoin%
\pgfsetlinewidth{0.803000pt}%
\definecolor{currentstroke}{rgb}{0.850000,0.850000,0.850000}%
\pgfsetstrokecolor{currentstroke}%
\pgfsetdash{}{0pt}%
\pgfpathmoveto{\pgfqpoint{2.628844in}{0.417642in}}%
\pgfpathlineto{\pgfqpoint{2.628844in}{2.468330in}}%
\pgfusepath{stroke}%
\end{pgfscope}%
\begin{pgfscope}%
\pgfsetbuttcap%
\pgfsetroundjoin%
\definecolor{currentfill}{rgb}{0.000000,0.000000,0.000000}%
\pgfsetfillcolor{currentfill}%
\pgfsetlinewidth{0.602250pt}%
\definecolor{currentstroke}{rgb}{0.000000,0.000000,0.000000}%
\pgfsetstrokecolor{currentstroke}%
\pgfsetdash{}{0pt}%
\pgfsys@defobject{currentmarker}{\pgfqpoint{0.000000in}{-0.027778in}}{\pgfqpoint{0.000000in}{0.000000in}}{%
\pgfpathmoveto{\pgfqpoint{0.000000in}{0.000000in}}%
\pgfpathlineto{\pgfqpoint{0.000000in}{-0.027778in}}%
\pgfusepath{stroke,fill}%
}%
\begin{pgfscope}%
\pgfsys@transformshift{2.628844in}{0.417642in}%
\pgfsys@useobject{currentmarker}{}%
\end{pgfscope}%
\end{pgfscope}%
\begin{pgfscope}%
\pgfpathrectangle{\pgfqpoint{0.509263in}{0.417642in}}{\pgfqpoint{3.509067in}{2.050688in}}%
\pgfusepath{clip}%
\pgfsetrectcap%
\pgfsetroundjoin%
\pgfsetlinewidth{0.803000pt}%
\definecolor{currentstroke}{rgb}{0.850000,0.850000,0.850000}%
\pgfsetstrokecolor{currentstroke}%
\pgfsetdash{}{0pt}%
\pgfpathmoveto{\pgfqpoint{2.709093in}{0.417642in}}%
\pgfpathlineto{\pgfqpoint{2.709093in}{2.468330in}}%
\pgfusepath{stroke}%
\end{pgfscope}%
\begin{pgfscope}%
\pgfsetbuttcap%
\pgfsetroundjoin%
\definecolor{currentfill}{rgb}{0.000000,0.000000,0.000000}%
\pgfsetfillcolor{currentfill}%
\pgfsetlinewidth{0.602250pt}%
\definecolor{currentstroke}{rgb}{0.000000,0.000000,0.000000}%
\pgfsetstrokecolor{currentstroke}%
\pgfsetdash{}{0pt}%
\pgfsys@defobject{currentmarker}{\pgfqpoint{0.000000in}{-0.027778in}}{\pgfqpoint{0.000000in}{0.000000in}}{%
\pgfpathmoveto{\pgfqpoint{0.000000in}{0.000000in}}%
\pgfpathlineto{\pgfqpoint{0.000000in}{-0.027778in}}%
\pgfusepath{stroke,fill}%
}%
\begin{pgfscope}%
\pgfsys@transformshift{2.709093in}{0.417642in}%
\pgfsys@useobject{currentmarker}{}%
\end{pgfscope}%
\end{pgfscope}%
\begin{pgfscope}%
\pgfpathrectangle{\pgfqpoint{0.509263in}{0.417642in}}{\pgfqpoint{3.509067in}{2.050688in}}%
\pgfusepath{clip}%
\pgfsetrectcap%
\pgfsetroundjoin%
\pgfsetlinewidth{0.803000pt}%
\definecolor{currentstroke}{rgb}{0.850000,0.850000,0.850000}%
\pgfsetstrokecolor{currentstroke}%
\pgfsetdash{}{0pt}%
\pgfpathmoveto{\pgfqpoint{2.766031in}{0.417642in}}%
\pgfpathlineto{\pgfqpoint{2.766031in}{2.468330in}}%
\pgfusepath{stroke}%
\end{pgfscope}%
\begin{pgfscope}%
\pgfsetbuttcap%
\pgfsetroundjoin%
\definecolor{currentfill}{rgb}{0.000000,0.000000,0.000000}%
\pgfsetfillcolor{currentfill}%
\pgfsetlinewidth{0.602250pt}%
\definecolor{currentstroke}{rgb}{0.000000,0.000000,0.000000}%
\pgfsetstrokecolor{currentstroke}%
\pgfsetdash{}{0pt}%
\pgfsys@defobject{currentmarker}{\pgfqpoint{0.000000in}{-0.027778in}}{\pgfqpoint{0.000000in}{0.000000in}}{%
\pgfpathmoveto{\pgfqpoint{0.000000in}{0.000000in}}%
\pgfpathlineto{\pgfqpoint{0.000000in}{-0.027778in}}%
\pgfusepath{stroke,fill}%
}%
\begin{pgfscope}%
\pgfsys@transformshift{2.766031in}{0.417642in}%
\pgfsys@useobject{currentmarker}{}%
\end{pgfscope}%
\end{pgfscope}%
\begin{pgfscope}%
\pgfpathrectangle{\pgfqpoint{0.509263in}{0.417642in}}{\pgfqpoint{3.509067in}{2.050688in}}%
\pgfusepath{clip}%
\pgfsetrectcap%
\pgfsetroundjoin%
\pgfsetlinewidth{0.803000pt}%
\definecolor{currentstroke}{rgb}{0.850000,0.850000,0.850000}%
\pgfsetstrokecolor{currentstroke}%
\pgfsetdash{}{0pt}%
\pgfpathmoveto{\pgfqpoint{2.810195in}{0.417642in}}%
\pgfpathlineto{\pgfqpoint{2.810195in}{2.468330in}}%
\pgfusepath{stroke}%
\end{pgfscope}%
\begin{pgfscope}%
\pgfsetbuttcap%
\pgfsetroundjoin%
\definecolor{currentfill}{rgb}{0.000000,0.000000,0.000000}%
\pgfsetfillcolor{currentfill}%
\pgfsetlinewidth{0.602250pt}%
\definecolor{currentstroke}{rgb}{0.000000,0.000000,0.000000}%
\pgfsetstrokecolor{currentstroke}%
\pgfsetdash{}{0pt}%
\pgfsys@defobject{currentmarker}{\pgfqpoint{0.000000in}{-0.027778in}}{\pgfqpoint{0.000000in}{0.000000in}}{%
\pgfpathmoveto{\pgfqpoint{0.000000in}{0.000000in}}%
\pgfpathlineto{\pgfqpoint{0.000000in}{-0.027778in}}%
\pgfusepath{stroke,fill}%
}%
\begin{pgfscope}%
\pgfsys@transformshift{2.810195in}{0.417642in}%
\pgfsys@useobject{currentmarker}{}%
\end{pgfscope}%
\end{pgfscope}%
\begin{pgfscope}%
\pgfpathrectangle{\pgfqpoint{0.509263in}{0.417642in}}{\pgfqpoint{3.509067in}{2.050688in}}%
\pgfusepath{clip}%
\pgfsetrectcap%
\pgfsetroundjoin%
\pgfsetlinewidth{0.803000pt}%
\definecolor{currentstroke}{rgb}{0.850000,0.850000,0.850000}%
\pgfsetstrokecolor{currentstroke}%
\pgfsetdash{}{0pt}%
\pgfpathmoveto{\pgfqpoint{2.846280in}{0.417642in}}%
\pgfpathlineto{\pgfqpoint{2.846280in}{2.468330in}}%
\pgfusepath{stroke}%
\end{pgfscope}%
\begin{pgfscope}%
\pgfsetbuttcap%
\pgfsetroundjoin%
\definecolor{currentfill}{rgb}{0.000000,0.000000,0.000000}%
\pgfsetfillcolor{currentfill}%
\pgfsetlinewidth{0.602250pt}%
\definecolor{currentstroke}{rgb}{0.000000,0.000000,0.000000}%
\pgfsetstrokecolor{currentstroke}%
\pgfsetdash{}{0pt}%
\pgfsys@defobject{currentmarker}{\pgfqpoint{0.000000in}{-0.027778in}}{\pgfqpoint{0.000000in}{0.000000in}}{%
\pgfpathmoveto{\pgfqpoint{0.000000in}{0.000000in}}%
\pgfpathlineto{\pgfqpoint{0.000000in}{-0.027778in}}%
\pgfusepath{stroke,fill}%
}%
\begin{pgfscope}%
\pgfsys@transformshift{2.846280in}{0.417642in}%
\pgfsys@useobject{currentmarker}{}%
\end{pgfscope}%
\end{pgfscope}%
\begin{pgfscope}%
\pgfpathrectangle{\pgfqpoint{0.509263in}{0.417642in}}{\pgfqpoint{3.509067in}{2.050688in}}%
\pgfusepath{clip}%
\pgfsetrectcap%
\pgfsetroundjoin%
\pgfsetlinewidth{0.803000pt}%
\definecolor{currentstroke}{rgb}{0.850000,0.850000,0.850000}%
\pgfsetstrokecolor{currentstroke}%
\pgfsetdash{}{0pt}%
\pgfpathmoveto{\pgfqpoint{2.876789in}{0.417642in}}%
\pgfpathlineto{\pgfqpoint{2.876789in}{2.468330in}}%
\pgfusepath{stroke}%
\end{pgfscope}%
\begin{pgfscope}%
\pgfsetbuttcap%
\pgfsetroundjoin%
\definecolor{currentfill}{rgb}{0.000000,0.000000,0.000000}%
\pgfsetfillcolor{currentfill}%
\pgfsetlinewidth{0.602250pt}%
\definecolor{currentstroke}{rgb}{0.000000,0.000000,0.000000}%
\pgfsetstrokecolor{currentstroke}%
\pgfsetdash{}{0pt}%
\pgfsys@defobject{currentmarker}{\pgfqpoint{0.000000in}{-0.027778in}}{\pgfqpoint{0.000000in}{0.000000in}}{%
\pgfpathmoveto{\pgfqpoint{0.000000in}{0.000000in}}%
\pgfpathlineto{\pgfqpoint{0.000000in}{-0.027778in}}%
\pgfusepath{stroke,fill}%
}%
\begin{pgfscope}%
\pgfsys@transformshift{2.876789in}{0.417642in}%
\pgfsys@useobject{currentmarker}{}%
\end{pgfscope}%
\end{pgfscope}%
\begin{pgfscope}%
\pgfpathrectangle{\pgfqpoint{0.509263in}{0.417642in}}{\pgfqpoint{3.509067in}{2.050688in}}%
\pgfusepath{clip}%
\pgfsetrectcap%
\pgfsetroundjoin%
\pgfsetlinewidth{0.803000pt}%
\definecolor{currentstroke}{rgb}{0.850000,0.850000,0.850000}%
\pgfsetstrokecolor{currentstroke}%
\pgfsetdash{}{0pt}%
\pgfpathmoveto{\pgfqpoint{2.903217in}{0.417642in}}%
\pgfpathlineto{\pgfqpoint{2.903217in}{2.468330in}}%
\pgfusepath{stroke}%
\end{pgfscope}%
\begin{pgfscope}%
\pgfsetbuttcap%
\pgfsetroundjoin%
\definecolor{currentfill}{rgb}{0.000000,0.000000,0.000000}%
\pgfsetfillcolor{currentfill}%
\pgfsetlinewidth{0.602250pt}%
\definecolor{currentstroke}{rgb}{0.000000,0.000000,0.000000}%
\pgfsetstrokecolor{currentstroke}%
\pgfsetdash{}{0pt}%
\pgfsys@defobject{currentmarker}{\pgfqpoint{0.000000in}{-0.027778in}}{\pgfqpoint{0.000000in}{0.000000in}}{%
\pgfpathmoveto{\pgfqpoint{0.000000in}{0.000000in}}%
\pgfpathlineto{\pgfqpoint{0.000000in}{-0.027778in}}%
\pgfusepath{stroke,fill}%
}%
\begin{pgfscope}%
\pgfsys@transformshift{2.903217in}{0.417642in}%
\pgfsys@useobject{currentmarker}{}%
\end{pgfscope}%
\end{pgfscope}%
\begin{pgfscope}%
\pgfpathrectangle{\pgfqpoint{0.509263in}{0.417642in}}{\pgfqpoint{3.509067in}{2.050688in}}%
\pgfusepath{clip}%
\pgfsetrectcap%
\pgfsetroundjoin%
\pgfsetlinewidth{0.803000pt}%
\definecolor{currentstroke}{rgb}{0.850000,0.850000,0.850000}%
\pgfsetstrokecolor{currentstroke}%
\pgfsetdash{}{0pt}%
\pgfpathmoveto{\pgfqpoint{2.926528in}{0.417642in}}%
\pgfpathlineto{\pgfqpoint{2.926528in}{2.468330in}}%
\pgfusepath{stroke}%
\end{pgfscope}%
\begin{pgfscope}%
\pgfsetbuttcap%
\pgfsetroundjoin%
\definecolor{currentfill}{rgb}{0.000000,0.000000,0.000000}%
\pgfsetfillcolor{currentfill}%
\pgfsetlinewidth{0.602250pt}%
\definecolor{currentstroke}{rgb}{0.000000,0.000000,0.000000}%
\pgfsetstrokecolor{currentstroke}%
\pgfsetdash{}{0pt}%
\pgfsys@defobject{currentmarker}{\pgfqpoint{0.000000in}{-0.027778in}}{\pgfqpoint{0.000000in}{0.000000in}}{%
\pgfpathmoveto{\pgfqpoint{0.000000in}{0.000000in}}%
\pgfpathlineto{\pgfqpoint{0.000000in}{-0.027778in}}%
\pgfusepath{stroke,fill}%
}%
\begin{pgfscope}%
\pgfsys@transformshift{2.926528in}{0.417642in}%
\pgfsys@useobject{currentmarker}{}%
\end{pgfscope}%
\end{pgfscope}%
\begin{pgfscope}%
\pgfpathrectangle{\pgfqpoint{0.509263in}{0.417642in}}{\pgfqpoint{3.509067in}{2.050688in}}%
\pgfusepath{clip}%
\pgfsetrectcap%
\pgfsetroundjoin%
\pgfsetlinewidth{0.803000pt}%
\definecolor{currentstroke}{rgb}{0.850000,0.850000,0.850000}%
\pgfsetstrokecolor{currentstroke}%
\pgfsetdash{}{0pt}%
\pgfpathmoveto{\pgfqpoint{3.084567in}{0.417642in}}%
\pgfpathlineto{\pgfqpoint{3.084567in}{2.468330in}}%
\pgfusepath{stroke}%
\end{pgfscope}%
\begin{pgfscope}%
\pgfsetbuttcap%
\pgfsetroundjoin%
\definecolor{currentfill}{rgb}{0.000000,0.000000,0.000000}%
\pgfsetfillcolor{currentfill}%
\pgfsetlinewidth{0.602250pt}%
\definecolor{currentstroke}{rgb}{0.000000,0.000000,0.000000}%
\pgfsetstrokecolor{currentstroke}%
\pgfsetdash{}{0pt}%
\pgfsys@defobject{currentmarker}{\pgfqpoint{0.000000in}{-0.027778in}}{\pgfqpoint{0.000000in}{0.000000in}}{%
\pgfpathmoveto{\pgfqpoint{0.000000in}{0.000000in}}%
\pgfpathlineto{\pgfqpoint{0.000000in}{-0.027778in}}%
\pgfusepath{stroke,fill}%
}%
\begin{pgfscope}%
\pgfsys@transformshift{3.084567in}{0.417642in}%
\pgfsys@useobject{currentmarker}{}%
\end{pgfscope}%
\end{pgfscope}%
\begin{pgfscope}%
\pgfpathrectangle{\pgfqpoint{0.509263in}{0.417642in}}{\pgfqpoint{3.509067in}{2.050688in}}%
\pgfusepath{clip}%
\pgfsetrectcap%
\pgfsetroundjoin%
\pgfsetlinewidth{0.803000pt}%
\definecolor{currentstroke}{rgb}{0.850000,0.850000,0.850000}%
\pgfsetstrokecolor{currentstroke}%
\pgfsetdash{}{0pt}%
\pgfpathmoveto{\pgfqpoint{3.164816in}{0.417642in}}%
\pgfpathlineto{\pgfqpoint{3.164816in}{2.468330in}}%
\pgfusepath{stroke}%
\end{pgfscope}%
\begin{pgfscope}%
\pgfsetbuttcap%
\pgfsetroundjoin%
\definecolor{currentfill}{rgb}{0.000000,0.000000,0.000000}%
\pgfsetfillcolor{currentfill}%
\pgfsetlinewidth{0.602250pt}%
\definecolor{currentstroke}{rgb}{0.000000,0.000000,0.000000}%
\pgfsetstrokecolor{currentstroke}%
\pgfsetdash{}{0pt}%
\pgfsys@defobject{currentmarker}{\pgfqpoint{0.000000in}{-0.027778in}}{\pgfqpoint{0.000000in}{0.000000in}}{%
\pgfpathmoveto{\pgfqpoint{0.000000in}{0.000000in}}%
\pgfpathlineto{\pgfqpoint{0.000000in}{-0.027778in}}%
\pgfusepath{stroke,fill}%
}%
\begin{pgfscope}%
\pgfsys@transformshift{3.164816in}{0.417642in}%
\pgfsys@useobject{currentmarker}{}%
\end{pgfscope}%
\end{pgfscope}%
\begin{pgfscope}%
\pgfpathrectangle{\pgfqpoint{0.509263in}{0.417642in}}{\pgfqpoint{3.509067in}{2.050688in}}%
\pgfusepath{clip}%
\pgfsetrectcap%
\pgfsetroundjoin%
\pgfsetlinewidth{0.803000pt}%
\definecolor{currentstroke}{rgb}{0.850000,0.850000,0.850000}%
\pgfsetstrokecolor{currentstroke}%
\pgfsetdash{}{0pt}%
\pgfpathmoveto{\pgfqpoint{3.221754in}{0.417642in}}%
\pgfpathlineto{\pgfqpoint{3.221754in}{2.468330in}}%
\pgfusepath{stroke}%
\end{pgfscope}%
\begin{pgfscope}%
\pgfsetbuttcap%
\pgfsetroundjoin%
\definecolor{currentfill}{rgb}{0.000000,0.000000,0.000000}%
\pgfsetfillcolor{currentfill}%
\pgfsetlinewidth{0.602250pt}%
\definecolor{currentstroke}{rgb}{0.000000,0.000000,0.000000}%
\pgfsetstrokecolor{currentstroke}%
\pgfsetdash{}{0pt}%
\pgfsys@defobject{currentmarker}{\pgfqpoint{0.000000in}{-0.027778in}}{\pgfqpoint{0.000000in}{0.000000in}}{%
\pgfpathmoveto{\pgfqpoint{0.000000in}{0.000000in}}%
\pgfpathlineto{\pgfqpoint{0.000000in}{-0.027778in}}%
\pgfusepath{stroke,fill}%
}%
\begin{pgfscope}%
\pgfsys@transformshift{3.221754in}{0.417642in}%
\pgfsys@useobject{currentmarker}{}%
\end{pgfscope}%
\end{pgfscope}%
\begin{pgfscope}%
\pgfpathrectangle{\pgfqpoint{0.509263in}{0.417642in}}{\pgfqpoint{3.509067in}{2.050688in}}%
\pgfusepath{clip}%
\pgfsetrectcap%
\pgfsetroundjoin%
\pgfsetlinewidth{0.803000pt}%
\definecolor{currentstroke}{rgb}{0.850000,0.850000,0.850000}%
\pgfsetstrokecolor{currentstroke}%
\pgfsetdash{}{0pt}%
\pgfpathmoveto{\pgfqpoint{3.265918in}{0.417642in}}%
\pgfpathlineto{\pgfqpoint{3.265918in}{2.468330in}}%
\pgfusepath{stroke}%
\end{pgfscope}%
\begin{pgfscope}%
\pgfsetbuttcap%
\pgfsetroundjoin%
\definecolor{currentfill}{rgb}{0.000000,0.000000,0.000000}%
\pgfsetfillcolor{currentfill}%
\pgfsetlinewidth{0.602250pt}%
\definecolor{currentstroke}{rgb}{0.000000,0.000000,0.000000}%
\pgfsetstrokecolor{currentstroke}%
\pgfsetdash{}{0pt}%
\pgfsys@defobject{currentmarker}{\pgfqpoint{0.000000in}{-0.027778in}}{\pgfqpoint{0.000000in}{0.000000in}}{%
\pgfpathmoveto{\pgfqpoint{0.000000in}{0.000000in}}%
\pgfpathlineto{\pgfqpoint{0.000000in}{-0.027778in}}%
\pgfusepath{stroke,fill}%
}%
\begin{pgfscope}%
\pgfsys@transformshift{3.265918in}{0.417642in}%
\pgfsys@useobject{currentmarker}{}%
\end{pgfscope}%
\end{pgfscope}%
\begin{pgfscope}%
\pgfpathrectangle{\pgfqpoint{0.509263in}{0.417642in}}{\pgfqpoint{3.509067in}{2.050688in}}%
\pgfusepath{clip}%
\pgfsetrectcap%
\pgfsetroundjoin%
\pgfsetlinewidth{0.803000pt}%
\definecolor{currentstroke}{rgb}{0.850000,0.850000,0.850000}%
\pgfsetstrokecolor{currentstroke}%
\pgfsetdash{}{0pt}%
\pgfpathmoveto{\pgfqpoint{3.302002in}{0.417642in}}%
\pgfpathlineto{\pgfqpoint{3.302002in}{2.468330in}}%
\pgfusepath{stroke}%
\end{pgfscope}%
\begin{pgfscope}%
\pgfsetbuttcap%
\pgfsetroundjoin%
\definecolor{currentfill}{rgb}{0.000000,0.000000,0.000000}%
\pgfsetfillcolor{currentfill}%
\pgfsetlinewidth{0.602250pt}%
\definecolor{currentstroke}{rgb}{0.000000,0.000000,0.000000}%
\pgfsetstrokecolor{currentstroke}%
\pgfsetdash{}{0pt}%
\pgfsys@defobject{currentmarker}{\pgfqpoint{0.000000in}{-0.027778in}}{\pgfqpoint{0.000000in}{0.000000in}}{%
\pgfpathmoveto{\pgfqpoint{0.000000in}{0.000000in}}%
\pgfpathlineto{\pgfqpoint{0.000000in}{-0.027778in}}%
\pgfusepath{stroke,fill}%
}%
\begin{pgfscope}%
\pgfsys@transformshift{3.302002in}{0.417642in}%
\pgfsys@useobject{currentmarker}{}%
\end{pgfscope}%
\end{pgfscope}%
\begin{pgfscope}%
\pgfpathrectangle{\pgfqpoint{0.509263in}{0.417642in}}{\pgfqpoint{3.509067in}{2.050688in}}%
\pgfusepath{clip}%
\pgfsetrectcap%
\pgfsetroundjoin%
\pgfsetlinewidth{0.803000pt}%
\definecolor{currentstroke}{rgb}{0.850000,0.850000,0.850000}%
\pgfsetstrokecolor{currentstroke}%
\pgfsetdash{}{0pt}%
\pgfpathmoveto{\pgfqpoint{3.332512in}{0.417642in}}%
\pgfpathlineto{\pgfqpoint{3.332512in}{2.468330in}}%
\pgfusepath{stroke}%
\end{pgfscope}%
\begin{pgfscope}%
\pgfsetbuttcap%
\pgfsetroundjoin%
\definecolor{currentfill}{rgb}{0.000000,0.000000,0.000000}%
\pgfsetfillcolor{currentfill}%
\pgfsetlinewidth{0.602250pt}%
\definecolor{currentstroke}{rgb}{0.000000,0.000000,0.000000}%
\pgfsetstrokecolor{currentstroke}%
\pgfsetdash{}{0pt}%
\pgfsys@defobject{currentmarker}{\pgfqpoint{0.000000in}{-0.027778in}}{\pgfqpoint{0.000000in}{0.000000in}}{%
\pgfpathmoveto{\pgfqpoint{0.000000in}{0.000000in}}%
\pgfpathlineto{\pgfqpoint{0.000000in}{-0.027778in}}%
\pgfusepath{stroke,fill}%
}%
\begin{pgfscope}%
\pgfsys@transformshift{3.332512in}{0.417642in}%
\pgfsys@useobject{currentmarker}{}%
\end{pgfscope}%
\end{pgfscope}%
\begin{pgfscope}%
\pgfpathrectangle{\pgfqpoint{0.509263in}{0.417642in}}{\pgfqpoint{3.509067in}{2.050688in}}%
\pgfusepath{clip}%
\pgfsetrectcap%
\pgfsetroundjoin%
\pgfsetlinewidth{0.803000pt}%
\definecolor{currentstroke}{rgb}{0.850000,0.850000,0.850000}%
\pgfsetstrokecolor{currentstroke}%
\pgfsetdash{}{0pt}%
\pgfpathmoveto{\pgfqpoint{3.358940in}{0.417642in}}%
\pgfpathlineto{\pgfqpoint{3.358940in}{2.468330in}}%
\pgfusepath{stroke}%
\end{pgfscope}%
\begin{pgfscope}%
\pgfsetbuttcap%
\pgfsetroundjoin%
\definecolor{currentfill}{rgb}{0.000000,0.000000,0.000000}%
\pgfsetfillcolor{currentfill}%
\pgfsetlinewidth{0.602250pt}%
\definecolor{currentstroke}{rgb}{0.000000,0.000000,0.000000}%
\pgfsetstrokecolor{currentstroke}%
\pgfsetdash{}{0pt}%
\pgfsys@defobject{currentmarker}{\pgfqpoint{0.000000in}{-0.027778in}}{\pgfqpoint{0.000000in}{0.000000in}}{%
\pgfpathmoveto{\pgfqpoint{0.000000in}{0.000000in}}%
\pgfpathlineto{\pgfqpoint{0.000000in}{-0.027778in}}%
\pgfusepath{stroke,fill}%
}%
\begin{pgfscope}%
\pgfsys@transformshift{3.358940in}{0.417642in}%
\pgfsys@useobject{currentmarker}{}%
\end{pgfscope}%
\end{pgfscope}%
\begin{pgfscope}%
\pgfpathrectangle{\pgfqpoint{0.509263in}{0.417642in}}{\pgfqpoint{3.509067in}{2.050688in}}%
\pgfusepath{clip}%
\pgfsetrectcap%
\pgfsetroundjoin%
\pgfsetlinewidth{0.803000pt}%
\definecolor{currentstroke}{rgb}{0.850000,0.850000,0.850000}%
\pgfsetstrokecolor{currentstroke}%
\pgfsetdash{}{0pt}%
\pgfpathmoveto{\pgfqpoint{3.382251in}{0.417642in}}%
\pgfpathlineto{\pgfqpoint{3.382251in}{2.468330in}}%
\pgfusepath{stroke}%
\end{pgfscope}%
\begin{pgfscope}%
\pgfsetbuttcap%
\pgfsetroundjoin%
\definecolor{currentfill}{rgb}{0.000000,0.000000,0.000000}%
\pgfsetfillcolor{currentfill}%
\pgfsetlinewidth{0.602250pt}%
\definecolor{currentstroke}{rgb}{0.000000,0.000000,0.000000}%
\pgfsetstrokecolor{currentstroke}%
\pgfsetdash{}{0pt}%
\pgfsys@defobject{currentmarker}{\pgfqpoint{0.000000in}{-0.027778in}}{\pgfqpoint{0.000000in}{0.000000in}}{%
\pgfpathmoveto{\pgfqpoint{0.000000in}{0.000000in}}%
\pgfpathlineto{\pgfqpoint{0.000000in}{-0.027778in}}%
\pgfusepath{stroke,fill}%
}%
\begin{pgfscope}%
\pgfsys@transformshift{3.382251in}{0.417642in}%
\pgfsys@useobject{currentmarker}{}%
\end{pgfscope}%
\end{pgfscope}%
\begin{pgfscope}%
\pgfpathrectangle{\pgfqpoint{0.509263in}{0.417642in}}{\pgfqpoint{3.509067in}{2.050688in}}%
\pgfusepath{clip}%
\pgfsetrectcap%
\pgfsetroundjoin%
\pgfsetlinewidth{0.803000pt}%
\definecolor{currentstroke}{rgb}{0.850000,0.850000,0.850000}%
\pgfsetstrokecolor{currentstroke}%
\pgfsetdash{}{0pt}%
\pgfpathmoveto{\pgfqpoint{3.540290in}{0.417642in}}%
\pgfpathlineto{\pgfqpoint{3.540290in}{2.468330in}}%
\pgfusepath{stroke}%
\end{pgfscope}%
\begin{pgfscope}%
\pgfsetbuttcap%
\pgfsetroundjoin%
\definecolor{currentfill}{rgb}{0.000000,0.000000,0.000000}%
\pgfsetfillcolor{currentfill}%
\pgfsetlinewidth{0.602250pt}%
\definecolor{currentstroke}{rgb}{0.000000,0.000000,0.000000}%
\pgfsetstrokecolor{currentstroke}%
\pgfsetdash{}{0pt}%
\pgfsys@defobject{currentmarker}{\pgfqpoint{0.000000in}{-0.027778in}}{\pgfqpoint{0.000000in}{0.000000in}}{%
\pgfpathmoveto{\pgfqpoint{0.000000in}{0.000000in}}%
\pgfpathlineto{\pgfqpoint{0.000000in}{-0.027778in}}%
\pgfusepath{stroke,fill}%
}%
\begin{pgfscope}%
\pgfsys@transformshift{3.540290in}{0.417642in}%
\pgfsys@useobject{currentmarker}{}%
\end{pgfscope}%
\end{pgfscope}%
\begin{pgfscope}%
\pgfpathrectangle{\pgfqpoint{0.509263in}{0.417642in}}{\pgfqpoint{3.509067in}{2.050688in}}%
\pgfusepath{clip}%
\pgfsetrectcap%
\pgfsetroundjoin%
\pgfsetlinewidth{0.803000pt}%
\definecolor{currentstroke}{rgb}{0.850000,0.850000,0.850000}%
\pgfsetstrokecolor{currentstroke}%
\pgfsetdash{}{0pt}%
\pgfpathmoveto{\pgfqpoint{3.620539in}{0.417642in}}%
\pgfpathlineto{\pgfqpoint{3.620539in}{2.468330in}}%
\pgfusepath{stroke}%
\end{pgfscope}%
\begin{pgfscope}%
\pgfsetbuttcap%
\pgfsetroundjoin%
\definecolor{currentfill}{rgb}{0.000000,0.000000,0.000000}%
\pgfsetfillcolor{currentfill}%
\pgfsetlinewidth{0.602250pt}%
\definecolor{currentstroke}{rgb}{0.000000,0.000000,0.000000}%
\pgfsetstrokecolor{currentstroke}%
\pgfsetdash{}{0pt}%
\pgfsys@defobject{currentmarker}{\pgfqpoint{0.000000in}{-0.027778in}}{\pgfqpoint{0.000000in}{0.000000in}}{%
\pgfpathmoveto{\pgfqpoint{0.000000in}{0.000000in}}%
\pgfpathlineto{\pgfqpoint{0.000000in}{-0.027778in}}%
\pgfusepath{stroke,fill}%
}%
\begin{pgfscope}%
\pgfsys@transformshift{3.620539in}{0.417642in}%
\pgfsys@useobject{currentmarker}{}%
\end{pgfscope}%
\end{pgfscope}%
\begin{pgfscope}%
\pgfpathrectangle{\pgfqpoint{0.509263in}{0.417642in}}{\pgfqpoint{3.509067in}{2.050688in}}%
\pgfusepath{clip}%
\pgfsetrectcap%
\pgfsetroundjoin%
\pgfsetlinewidth{0.803000pt}%
\definecolor{currentstroke}{rgb}{0.850000,0.850000,0.850000}%
\pgfsetstrokecolor{currentstroke}%
\pgfsetdash{}{0pt}%
\pgfpathmoveto{\pgfqpoint{3.677477in}{0.417642in}}%
\pgfpathlineto{\pgfqpoint{3.677477in}{2.468330in}}%
\pgfusepath{stroke}%
\end{pgfscope}%
\begin{pgfscope}%
\pgfsetbuttcap%
\pgfsetroundjoin%
\definecolor{currentfill}{rgb}{0.000000,0.000000,0.000000}%
\pgfsetfillcolor{currentfill}%
\pgfsetlinewidth{0.602250pt}%
\definecolor{currentstroke}{rgb}{0.000000,0.000000,0.000000}%
\pgfsetstrokecolor{currentstroke}%
\pgfsetdash{}{0pt}%
\pgfsys@defobject{currentmarker}{\pgfqpoint{0.000000in}{-0.027778in}}{\pgfqpoint{0.000000in}{0.000000in}}{%
\pgfpathmoveto{\pgfqpoint{0.000000in}{0.000000in}}%
\pgfpathlineto{\pgfqpoint{0.000000in}{-0.027778in}}%
\pgfusepath{stroke,fill}%
}%
\begin{pgfscope}%
\pgfsys@transformshift{3.677477in}{0.417642in}%
\pgfsys@useobject{currentmarker}{}%
\end{pgfscope}%
\end{pgfscope}%
\begin{pgfscope}%
\pgfpathrectangle{\pgfqpoint{0.509263in}{0.417642in}}{\pgfqpoint{3.509067in}{2.050688in}}%
\pgfusepath{clip}%
\pgfsetrectcap%
\pgfsetroundjoin%
\pgfsetlinewidth{0.803000pt}%
\definecolor{currentstroke}{rgb}{0.850000,0.850000,0.850000}%
\pgfsetstrokecolor{currentstroke}%
\pgfsetdash{}{0pt}%
\pgfpathmoveto{\pgfqpoint{3.721641in}{0.417642in}}%
\pgfpathlineto{\pgfqpoint{3.721641in}{2.468330in}}%
\pgfusepath{stroke}%
\end{pgfscope}%
\begin{pgfscope}%
\pgfsetbuttcap%
\pgfsetroundjoin%
\definecolor{currentfill}{rgb}{0.000000,0.000000,0.000000}%
\pgfsetfillcolor{currentfill}%
\pgfsetlinewidth{0.602250pt}%
\definecolor{currentstroke}{rgb}{0.000000,0.000000,0.000000}%
\pgfsetstrokecolor{currentstroke}%
\pgfsetdash{}{0pt}%
\pgfsys@defobject{currentmarker}{\pgfqpoint{0.000000in}{-0.027778in}}{\pgfqpoint{0.000000in}{0.000000in}}{%
\pgfpathmoveto{\pgfqpoint{0.000000in}{0.000000in}}%
\pgfpathlineto{\pgfqpoint{0.000000in}{-0.027778in}}%
\pgfusepath{stroke,fill}%
}%
\begin{pgfscope}%
\pgfsys@transformshift{3.721641in}{0.417642in}%
\pgfsys@useobject{currentmarker}{}%
\end{pgfscope}%
\end{pgfscope}%
\begin{pgfscope}%
\pgfpathrectangle{\pgfqpoint{0.509263in}{0.417642in}}{\pgfqpoint{3.509067in}{2.050688in}}%
\pgfusepath{clip}%
\pgfsetrectcap%
\pgfsetroundjoin%
\pgfsetlinewidth{0.803000pt}%
\definecolor{currentstroke}{rgb}{0.850000,0.850000,0.850000}%
\pgfsetstrokecolor{currentstroke}%
\pgfsetdash{}{0pt}%
\pgfpathmoveto{\pgfqpoint{3.757725in}{0.417642in}}%
\pgfpathlineto{\pgfqpoint{3.757725in}{2.468330in}}%
\pgfusepath{stroke}%
\end{pgfscope}%
\begin{pgfscope}%
\pgfsetbuttcap%
\pgfsetroundjoin%
\definecolor{currentfill}{rgb}{0.000000,0.000000,0.000000}%
\pgfsetfillcolor{currentfill}%
\pgfsetlinewidth{0.602250pt}%
\definecolor{currentstroke}{rgb}{0.000000,0.000000,0.000000}%
\pgfsetstrokecolor{currentstroke}%
\pgfsetdash{}{0pt}%
\pgfsys@defobject{currentmarker}{\pgfqpoint{0.000000in}{-0.027778in}}{\pgfqpoint{0.000000in}{0.000000in}}{%
\pgfpathmoveto{\pgfqpoint{0.000000in}{0.000000in}}%
\pgfpathlineto{\pgfqpoint{0.000000in}{-0.027778in}}%
\pgfusepath{stroke,fill}%
}%
\begin{pgfscope}%
\pgfsys@transformshift{3.757725in}{0.417642in}%
\pgfsys@useobject{currentmarker}{}%
\end{pgfscope}%
\end{pgfscope}%
\begin{pgfscope}%
\pgfpathrectangle{\pgfqpoint{0.509263in}{0.417642in}}{\pgfqpoint{3.509067in}{2.050688in}}%
\pgfusepath{clip}%
\pgfsetrectcap%
\pgfsetroundjoin%
\pgfsetlinewidth{0.803000pt}%
\definecolor{currentstroke}{rgb}{0.850000,0.850000,0.850000}%
\pgfsetstrokecolor{currentstroke}%
\pgfsetdash{}{0pt}%
\pgfpathmoveto{\pgfqpoint{3.788235in}{0.417642in}}%
\pgfpathlineto{\pgfqpoint{3.788235in}{2.468330in}}%
\pgfusepath{stroke}%
\end{pgfscope}%
\begin{pgfscope}%
\pgfsetbuttcap%
\pgfsetroundjoin%
\definecolor{currentfill}{rgb}{0.000000,0.000000,0.000000}%
\pgfsetfillcolor{currentfill}%
\pgfsetlinewidth{0.602250pt}%
\definecolor{currentstroke}{rgb}{0.000000,0.000000,0.000000}%
\pgfsetstrokecolor{currentstroke}%
\pgfsetdash{}{0pt}%
\pgfsys@defobject{currentmarker}{\pgfqpoint{0.000000in}{-0.027778in}}{\pgfqpoint{0.000000in}{0.000000in}}{%
\pgfpathmoveto{\pgfqpoint{0.000000in}{0.000000in}}%
\pgfpathlineto{\pgfqpoint{0.000000in}{-0.027778in}}%
\pgfusepath{stroke,fill}%
}%
\begin{pgfscope}%
\pgfsys@transformshift{3.788235in}{0.417642in}%
\pgfsys@useobject{currentmarker}{}%
\end{pgfscope}%
\end{pgfscope}%
\begin{pgfscope}%
\pgfpathrectangle{\pgfqpoint{0.509263in}{0.417642in}}{\pgfqpoint{3.509067in}{2.050688in}}%
\pgfusepath{clip}%
\pgfsetrectcap%
\pgfsetroundjoin%
\pgfsetlinewidth{0.803000pt}%
\definecolor{currentstroke}{rgb}{0.850000,0.850000,0.850000}%
\pgfsetstrokecolor{currentstroke}%
\pgfsetdash{}{0pt}%
\pgfpathmoveto{\pgfqpoint{3.814663in}{0.417642in}}%
\pgfpathlineto{\pgfqpoint{3.814663in}{2.468330in}}%
\pgfusepath{stroke}%
\end{pgfscope}%
\begin{pgfscope}%
\pgfsetbuttcap%
\pgfsetroundjoin%
\definecolor{currentfill}{rgb}{0.000000,0.000000,0.000000}%
\pgfsetfillcolor{currentfill}%
\pgfsetlinewidth{0.602250pt}%
\definecolor{currentstroke}{rgb}{0.000000,0.000000,0.000000}%
\pgfsetstrokecolor{currentstroke}%
\pgfsetdash{}{0pt}%
\pgfsys@defobject{currentmarker}{\pgfqpoint{0.000000in}{-0.027778in}}{\pgfqpoint{0.000000in}{0.000000in}}{%
\pgfpathmoveto{\pgfqpoint{0.000000in}{0.000000in}}%
\pgfpathlineto{\pgfqpoint{0.000000in}{-0.027778in}}%
\pgfusepath{stroke,fill}%
}%
\begin{pgfscope}%
\pgfsys@transformshift{3.814663in}{0.417642in}%
\pgfsys@useobject{currentmarker}{}%
\end{pgfscope}%
\end{pgfscope}%
\begin{pgfscope}%
\pgfpathrectangle{\pgfqpoint{0.509263in}{0.417642in}}{\pgfqpoint{3.509067in}{2.050688in}}%
\pgfusepath{clip}%
\pgfsetrectcap%
\pgfsetroundjoin%
\pgfsetlinewidth{0.803000pt}%
\definecolor{currentstroke}{rgb}{0.850000,0.850000,0.850000}%
\pgfsetstrokecolor{currentstroke}%
\pgfsetdash{}{0pt}%
\pgfpathmoveto{\pgfqpoint{3.837974in}{0.417642in}}%
\pgfpathlineto{\pgfqpoint{3.837974in}{2.468330in}}%
\pgfusepath{stroke}%
\end{pgfscope}%
\begin{pgfscope}%
\pgfsetbuttcap%
\pgfsetroundjoin%
\definecolor{currentfill}{rgb}{0.000000,0.000000,0.000000}%
\pgfsetfillcolor{currentfill}%
\pgfsetlinewidth{0.602250pt}%
\definecolor{currentstroke}{rgb}{0.000000,0.000000,0.000000}%
\pgfsetstrokecolor{currentstroke}%
\pgfsetdash{}{0pt}%
\pgfsys@defobject{currentmarker}{\pgfqpoint{0.000000in}{-0.027778in}}{\pgfqpoint{0.000000in}{0.000000in}}{%
\pgfpathmoveto{\pgfqpoint{0.000000in}{0.000000in}}%
\pgfpathlineto{\pgfqpoint{0.000000in}{-0.027778in}}%
\pgfusepath{stroke,fill}%
}%
\begin{pgfscope}%
\pgfsys@transformshift{3.837974in}{0.417642in}%
\pgfsys@useobject{currentmarker}{}%
\end{pgfscope}%
\end{pgfscope}%
\begin{pgfscope}%
\pgfpathrectangle{\pgfqpoint{0.509263in}{0.417642in}}{\pgfqpoint{3.509067in}{2.050688in}}%
\pgfusepath{clip}%
\pgfsetrectcap%
\pgfsetroundjoin%
\pgfsetlinewidth{0.803000pt}%
\definecolor{currentstroke}{rgb}{0.850000,0.850000,0.850000}%
\pgfsetstrokecolor{currentstroke}%
\pgfsetdash{}{0pt}%
\pgfpathmoveto{\pgfqpoint{3.996013in}{0.417642in}}%
\pgfpathlineto{\pgfqpoint{3.996013in}{2.468330in}}%
\pgfusepath{stroke}%
\end{pgfscope}%
\begin{pgfscope}%
\pgfsetbuttcap%
\pgfsetroundjoin%
\definecolor{currentfill}{rgb}{0.000000,0.000000,0.000000}%
\pgfsetfillcolor{currentfill}%
\pgfsetlinewidth{0.602250pt}%
\definecolor{currentstroke}{rgb}{0.000000,0.000000,0.000000}%
\pgfsetstrokecolor{currentstroke}%
\pgfsetdash{}{0pt}%
\pgfsys@defobject{currentmarker}{\pgfqpoint{0.000000in}{-0.027778in}}{\pgfqpoint{0.000000in}{0.000000in}}{%
\pgfpathmoveto{\pgfqpoint{0.000000in}{0.000000in}}%
\pgfpathlineto{\pgfqpoint{0.000000in}{-0.027778in}}%
\pgfusepath{stroke,fill}%
}%
\begin{pgfscope}%
\pgfsys@transformshift{3.996013in}{0.417642in}%
\pgfsys@useobject{currentmarker}{}%
\end{pgfscope}%
\end{pgfscope}%
\begin{pgfscope}%
\definecolor{textcolor}{rgb}{0.000000,0.000000,0.000000}%
\pgfsetstrokecolor{textcolor}%
\pgfsetfillcolor{textcolor}%
\pgftext[x=2.263797in,y=0.165003in,,top]{\color{textcolor}\rmfamily\fontsize{10.000000}{12.000000}\selectfont \(\displaystyle \tau\) in \unit{\second}}%
\end{pgfscope}%
\begin{pgfscope}%
\pgfpathrectangle{\pgfqpoint{0.509263in}{0.417642in}}{\pgfqpoint{3.509067in}{2.050688in}}%
\pgfusepath{clip}%
\pgfsetrectcap%
\pgfsetroundjoin%
\pgfsetlinewidth{0.803000pt}%
\definecolor{currentstroke}{rgb}{0.450000,0.450000,0.450000}%
\pgfsetstrokecolor{currentstroke}%
\pgfsetdash{}{0pt}%
\pgfpathmoveto{\pgfqpoint{0.509263in}{0.510855in}}%
\pgfpathlineto{\pgfqpoint{4.018330in}{0.510855in}}%
\pgfusepath{stroke}%
\end{pgfscope}%
\begin{pgfscope}%
\pgfsetbuttcap%
\pgfsetroundjoin%
\definecolor{currentfill}{rgb}{0.000000,0.000000,0.000000}%
\pgfsetfillcolor{currentfill}%
\pgfsetlinewidth{0.803000pt}%
\definecolor{currentstroke}{rgb}{0.000000,0.000000,0.000000}%
\pgfsetstrokecolor{currentstroke}%
\pgfsetdash{}{0pt}%
\pgfsys@defobject{currentmarker}{\pgfqpoint{-0.048611in}{0.000000in}}{\pgfqpoint{-0.000000in}{0.000000in}}{%
\pgfpathmoveto{\pgfqpoint{-0.000000in}{0.000000in}}%
\pgfpathlineto{\pgfqpoint{-0.048611in}{0.000000in}}%
\pgfusepath{stroke,fill}%
}%
\begin{pgfscope}%
\pgfsys@transformshift{0.509263in}{0.510855in}%
\pgfsys@useobject{currentmarker}{}%
\end{pgfscope}%
\end{pgfscope}%
\begin{pgfscope}%
\definecolor{textcolor}{rgb}{0.000000,0.000000,0.000000}%
\pgfsetstrokecolor{textcolor}%
\pgfsetfillcolor{textcolor}%
\pgftext[x=0.236114in, y=0.471702in, left, base]{\color{textcolor}\rmfamily\fontsize{8.000000}{9.600000}\selectfont \(\displaystyle {10^{0}}\)}%
\end{pgfscope}%
\begin{pgfscope}%
\pgfpathrectangle{\pgfqpoint{0.509263in}{0.417642in}}{\pgfqpoint{3.509067in}{2.050688in}}%
\pgfusepath{clip}%
\pgfsetrectcap%
\pgfsetroundjoin%
\pgfsetlinewidth{0.803000pt}%
\definecolor{currentstroke}{rgb}{0.450000,0.450000,0.450000}%
\pgfsetstrokecolor{currentstroke}%
\pgfsetdash{}{0pt}%
\pgfpathmoveto{\pgfqpoint{0.509263in}{1.748798in}}%
\pgfpathlineto{\pgfqpoint{4.018330in}{1.748798in}}%
\pgfusepath{stroke}%
\end{pgfscope}%
\begin{pgfscope}%
\pgfsetbuttcap%
\pgfsetroundjoin%
\definecolor{currentfill}{rgb}{0.000000,0.000000,0.000000}%
\pgfsetfillcolor{currentfill}%
\pgfsetlinewidth{0.803000pt}%
\definecolor{currentstroke}{rgb}{0.000000,0.000000,0.000000}%
\pgfsetstrokecolor{currentstroke}%
\pgfsetdash{}{0pt}%
\pgfsys@defobject{currentmarker}{\pgfqpoint{-0.048611in}{0.000000in}}{\pgfqpoint{-0.000000in}{0.000000in}}{%
\pgfpathmoveto{\pgfqpoint{-0.000000in}{0.000000in}}%
\pgfpathlineto{\pgfqpoint{-0.048611in}{0.000000in}}%
\pgfusepath{stroke,fill}%
}%
\begin{pgfscope}%
\pgfsys@transformshift{0.509263in}{1.748798in}%
\pgfsys@useobject{currentmarker}{}%
\end{pgfscope}%
\end{pgfscope}%
\begin{pgfscope}%
\definecolor{textcolor}{rgb}{0.000000,0.000000,0.000000}%
\pgfsetstrokecolor{textcolor}%
\pgfsetfillcolor{textcolor}%
\pgftext[x=0.236114in, y=1.709645in, left, base]{\color{textcolor}\rmfamily\fontsize{8.000000}{9.600000}\selectfont \(\displaystyle {10^{1}}\)}%
\end{pgfscope}%
\begin{pgfscope}%
\pgfpathrectangle{\pgfqpoint{0.509263in}{0.417642in}}{\pgfqpoint{3.509067in}{2.050688in}}%
\pgfusepath{clip}%
\pgfsetrectcap%
\pgfsetroundjoin%
\pgfsetlinewidth{0.803000pt}%
\definecolor{currentstroke}{rgb}{0.850000,0.850000,0.850000}%
\pgfsetstrokecolor{currentstroke}%
\pgfsetdash{}{0pt}%
\pgfpathmoveto{\pgfqpoint{0.509263in}{0.454210in}}%
\pgfpathlineto{\pgfqpoint{4.018330in}{0.454210in}}%
\pgfusepath{stroke}%
\end{pgfscope}%
\begin{pgfscope}%
\pgfsetbuttcap%
\pgfsetroundjoin%
\definecolor{currentfill}{rgb}{0.000000,0.000000,0.000000}%
\pgfsetfillcolor{currentfill}%
\pgfsetlinewidth{0.602250pt}%
\definecolor{currentstroke}{rgb}{0.000000,0.000000,0.000000}%
\pgfsetstrokecolor{currentstroke}%
\pgfsetdash{}{0pt}%
\pgfsys@defobject{currentmarker}{\pgfqpoint{-0.027778in}{0.000000in}}{\pgfqpoint{-0.000000in}{0.000000in}}{%
\pgfpathmoveto{\pgfqpoint{-0.000000in}{0.000000in}}%
\pgfpathlineto{\pgfqpoint{-0.027778in}{0.000000in}}%
\pgfusepath{stroke,fill}%
}%
\begin{pgfscope}%
\pgfsys@transformshift{0.509263in}{0.454210in}%
\pgfsys@useobject{currentmarker}{}%
\end{pgfscope}%
\end{pgfscope}%
\begin{pgfscope}%
\pgfpathrectangle{\pgfqpoint{0.509263in}{0.417642in}}{\pgfqpoint{3.509067in}{2.050688in}}%
\pgfusepath{clip}%
\pgfsetrectcap%
\pgfsetroundjoin%
\pgfsetlinewidth{0.803000pt}%
\definecolor{currentstroke}{rgb}{0.850000,0.850000,0.850000}%
\pgfsetstrokecolor{currentstroke}%
\pgfsetdash{}{0pt}%
\pgfpathmoveto{\pgfqpoint{0.509263in}{0.883513in}}%
\pgfpathlineto{\pgfqpoint{4.018330in}{0.883513in}}%
\pgfusepath{stroke}%
\end{pgfscope}%
\begin{pgfscope}%
\pgfsetbuttcap%
\pgfsetroundjoin%
\definecolor{currentfill}{rgb}{0.000000,0.000000,0.000000}%
\pgfsetfillcolor{currentfill}%
\pgfsetlinewidth{0.602250pt}%
\definecolor{currentstroke}{rgb}{0.000000,0.000000,0.000000}%
\pgfsetstrokecolor{currentstroke}%
\pgfsetdash{}{0pt}%
\pgfsys@defobject{currentmarker}{\pgfqpoint{-0.027778in}{0.000000in}}{\pgfqpoint{-0.000000in}{0.000000in}}{%
\pgfpathmoveto{\pgfqpoint{-0.000000in}{0.000000in}}%
\pgfpathlineto{\pgfqpoint{-0.027778in}{0.000000in}}%
\pgfusepath{stroke,fill}%
}%
\begin{pgfscope}%
\pgfsys@transformshift{0.509263in}{0.883513in}%
\pgfsys@useobject{currentmarker}{}%
\end{pgfscope}%
\end{pgfscope}%
\begin{pgfscope}%
\pgfpathrectangle{\pgfqpoint{0.509263in}{0.417642in}}{\pgfqpoint{3.509067in}{2.050688in}}%
\pgfusepath{clip}%
\pgfsetrectcap%
\pgfsetroundjoin%
\pgfsetlinewidth{0.803000pt}%
\definecolor{currentstroke}{rgb}{0.850000,0.850000,0.850000}%
\pgfsetstrokecolor{currentstroke}%
\pgfsetdash{}{0pt}%
\pgfpathmoveto{\pgfqpoint{0.509263in}{1.101504in}}%
\pgfpathlineto{\pgfqpoint{4.018330in}{1.101504in}}%
\pgfusepath{stroke}%
\end{pgfscope}%
\begin{pgfscope}%
\pgfsetbuttcap%
\pgfsetroundjoin%
\definecolor{currentfill}{rgb}{0.000000,0.000000,0.000000}%
\pgfsetfillcolor{currentfill}%
\pgfsetlinewidth{0.602250pt}%
\definecolor{currentstroke}{rgb}{0.000000,0.000000,0.000000}%
\pgfsetstrokecolor{currentstroke}%
\pgfsetdash{}{0pt}%
\pgfsys@defobject{currentmarker}{\pgfqpoint{-0.027778in}{0.000000in}}{\pgfqpoint{-0.000000in}{0.000000in}}{%
\pgfpathmoveto{\pgfqpoint{-0.000000in}{0.000000in}}%
\pgfpathlineto{\pgfqpoint{-0.027778in}{0.000000in}}%
\pgfusepath{stroke,fill}%
}%
\begin{pgfscope}%
\pgfsys@transformshift{0.509263in}{1.101504in}%
\pgfsys@useobject{currentmarker}{}%
\end{pgfscope}%
\end{pgfscope}%
\begin{pgfscope}%
\pgfpathrectangle{\pgfqpoint{0.509263in}{0.417642in}}{\pgfqpoint{3.509067in}{2.050688in}}%
\pgfusepath{clip}%
\pgfsetrectcap%
\pgfsetroundjoin%
\pgfsetlinewidth{0.803000pt}%
\definecolor{currentstroke}{rgb}{0.850000,0.850000,0.850000}%
\pgfsetstrokecolor{currentstroke}%
\pgfsetdash{}{0pt}%
\pgfpathmoveto{\pgfqpoint{0.509263in}{1.256171in}}%
\pgfpathlineto{\pgfqpoint{4.018330in}{1.256171in}}%
\pgfusepath{stroke}%
\end{pgfscope}%
\begin{pgfscope}%
\pgfsetbuttcap%
\pgfsetroundjoin%
\definecolor{currentfill}{rgb}{0.000000,0.000000,0.000000}%
\pgfsetfillcolor{currentfill}%
\pgfsetlinewidth{0.602250pt}%
\definecolor{currentstroke}{rgb}{0.000000,0.000000,0.000000}%
\pgfsetstrokecolor{currentstroke}%
\pgfsetdash{}{0pt}%
\pgfsys@defobject{currentmarker}{\pgfqpoint{-0.027778in}{0.000000in}}{\pgfqpoint{-0.000000in}{0.000000in}}{%
\pgfpathmoveto{\pgfqpoint{-0.000000in}{0.000000in}}%
\pgfpathlineto{\pgfqpoint{-0.027778in}{0.000000in}}%
\pgfusepath{stroke,fill}%
}%
\begin{pgfscope}%
\pgfsys@transformshift{0.509263in}{1.256171in}%
\pgfsys@useobject{currentmarker}{}%
\end{pgfscope}%
\end{pgfscope}%
\begin{pgfscope}%
\pgfpathrectangle{\pgfqpoint{0.509263in}{0.417642in}}{\pgfqpoint{3.509067in}{2.050688in}}%
\pgfusepath{clip}%
\pgfsetrectcap%
\pgfsetroundjoin%
\pgfsetlinewidth{0.803000pt}%
\definecolor{currentstroke}{rgb}{0.850000,0.850000,0.850000}%
\pgfsetstrokecolor{currentstroke}%
\pgfsetdash{}{0pt}%
\pgfpathmoveto{\pgfqpoint{0.509263in}{1.376140in}}%
\pgfpathlineto{\pgfqpoint{4.018330in}{1.376140in}}%
\pgfusepath{stroke}%
\end{pgfscope}%
\begin{pgfscope}%
\pgfsetbuttcap%
\pgfsetroundjoin%
\definecolor{currentfill}{rgb}{0.000000,0.000000,0.000000}%
\pgfsetfillcolor{currentfill}%
\pgfsetlinewidth{0.602250pt}%
\definecolor{currentstroke}{rgb}{0.000000,0.000000,0.000000}%
\pgfsetstrokecolor{currentstroke}%
\pgfsetdash{}{0pt}%
\pgfsys@defobject{currentmarker}{\pgfqpoint{-0.027778in}{0.000000in}}{\pgfqpoint{-0.000000in}{0.000000in}}{%
\pgfpathmoveto{\pgfqpoint{-0.000000in}{0.000000in}}%
\pgfpathlineto{\pgfqpoint{-0.027778in}{0.000000in}}%
\pgfusepath{stroke,fill}%
}%
\begin{pgfscope}%
\pgfsys@transformshift{0.509263in}{1.376140in}%
\pgfsys@useobject{currentmarker}{}%
\end{pgfscope}%
\end{pgfscope}%
\begin{pgfscope}%
\pgfpathrectangle{\pgfqpoint{0.509263in}{0.417642in}}{\pgfqpoint{3.509067in}{2.050688in}}%
\pgfusepath{clip}%
\pgfsetrectcap%
\pgfsetroundjoin%
\pgfsetlinewidth{0.803000pt}%
\definecolor{currentstroke}{rgb}{0.850000,0.850000,0.850000}%
\pgfsetstrokecolor{currentstroke}%
\pgfsetdash{}{0pt}%
\pgfpathmoveto{\pgfqpoint{0.509263in}{1.474162in}}%
\pgfpathlineto{\pgfqpoint{4.018330in}{1.474162in}}%
\pgfusepath{stroke}%
\end{pgfscope}%
\begin{pgfscope}%
\pgfsetbuttcap%
\pgfsetroundjoin%
\definecolor{currentfill}{rgb}{0.000000,0.000000,0.000000}%
\pgfsetfillcolor{currentfill}%
\pgfsetlinewidth{0.602250pt}%
\definecolor{currentstroke}{rgb}{0.000000,0.000000,0.000000}%
\pgfsetstrokecolor{currentstroke}%
\pgfsetdash{}{0pt}%
\pgfsys@defobject{currentmarker}{\pgfqpoint{-0.027778in}{0.000000in}}{\pgfqpoint{-0.000000in}{0.000000in}}{%
\pgfpathmoveto{\pgfqpoint{-0.000000in}{0.000000in}}%
\pgfpathlineto{\pgfqpoint{-0.027778in}{0.000000in}}%
\pgfusepath{stroke,fill}%
}%
\begin{pgfscope}%
\pgfsys@transformshift{0.509263in}{1.474162in}%
\pgfsys@useobject{currentmarker}{}%
\end{pgfscope}%
\end{pgfscope}%
\begin{pgfscope}%
\pgfpathrectangle{\pgfqpoint{0.509263in}{0.417642in}}{\pgfqpoint{3.509067in}{2.050688in}}%
\pgfusepath{clip}%
\pgfsetrectcap%
\pgfsetroundjoin%
\pgfsetlinewidth{0.803000pt}%
\definecolor{currentstroke}{rgb}{0.850000,0.850000,0.850000}%
\pgfsetstrokecolor{currentstroke}%
\pgfsetdash{}{0pt}%
\pgfpathmoveto{\pgfqpoint{0.509263in}{1.557038in}}%
\pgfpathlineto{\pgfqpoint{4.018330in}{1.557038in}}%
\pgfusepath{stroke}%
\end{pgfscope}%
\begin{pgfscope}%
\pgfsetbuttcap%
\pgfsetroundjoin%
\definecolor{currentfill}{rgb}{0.000000,0.000000,0.000000}%
\pgfsetfillcolor{currentfill}%
\pgfsetlinewidth{0.602250pt}%
\definecolor{currentstroke}{rgb}{0.000000,0.000000,0.000000}%
\pgfsetstrokecolor{currentstroke}%
\pgfsetdash{}{0pt}%
\pgfsys@defobject{currentmarker}{\pgfqpoint{-0.027778in}{0.000000in}}{\pgfqpoint{-0.000000in}{0.000000in}}{%
\pgfpathmoveto{\pgfqpoint{-0.000000in}{0.000000in}}%
\pgfpathlineto{\pgfqpoint{-0.027778in}{0.000000in}}%
\pgfusepath{stroke,fill}%
}%
\begin{pgfscope}%
\pgfsys@transformshift{0.509263in}{1.557038in}%
\pgfsys@useobject{currentmarker}{}%
\end{pgfscope}%
\end{pgfscope}%
\begin{pgfscope}%
\pgfpathrectangle{\pgfqpoint{0.509263in}{0.417642in}}{\pgfqpoint{3.509067in}{2.050688in}}%
\pgfusepath{clip}%
\pgfsetrectcap%
\pgfsetroundjoin%
\pgfsetlinewidth{0.803000pt}%
\definecolor{currentstroke}{rgb}{0.850000,0.850000,0.850000}%
\pgfsetstrokecolor{currentstroke}%
\pgfsetdash{}{0pt}%
\pgfpathmoveto{\pgfqpoint{0.509263in}{1.628829in}}%
\pgfpathlineto{\pgfqpoint{4.018330in}{1.628829in}}%
\pgfusepath{stroke}%
\end{pgfscope}%
\begin{pgfscope}%
\pgfsetbuttcap%
\pgfsetroundjoin%
\definecolor{currentfill}{rgb}{0.000000,0.000000,0.000000}%
\pgfsetfillcolor{currentfill}%
\pgfsetlinewidth{0.602250pt}%
\definecolor{currentstroke}{rgb}{0.000000,0.000000,0.000000}%
\pgfsetstrokecolor{currentstroke}%
\pgfsetdash{}{0pt}%
\pgfsys@defobject{currentmarker}{\pgfqpoint{-0.027778in}{0.000000in}}{\pgfqpoint{-0.000000in}{0.000000in}}{%
\pgfpathmoveto{\pgfqpoint{-0.000000in}{0.000000in}}%
\pgfpathlineto{\pgfqpoint{-0.027778in}{0.000000in}}%
\pgfusepath{stroke,fill}%
}%
\begin{pgfscope}%
\pgfsys@transformshift{0.509263in}{1.628829in}%
\pgfsys@useobject{currentmarker}{}%
\end{pgfscope}%
\end{pgfscope}%
\begin{pgfscope}%
\pgfpathrectangle{\pgfqpoint{0.509263in}{0.417642in}}{\pgfqpoint{3.509067in}{2.050688in}}%
\pgfusepath{clip}%
\pgfsetrectcap%
\pgfsetroundjoin%
\pgfsetlinewidth{0.803000pt}%
\definecolor{currentstroke}{rgb}{0.850000,0.850000,0.850000}%
\pgfsetstrokecolor{currentstroke}%
\pgfsetdash{}{0pt}%
\pgfpathmoveto{\pgfqpoint{0.509263in}{1.692153in}}%
\pgfpathlineto{\pgfqpoint{4.018330in}{1.692153in}}%
\pgfusepath{stroke}%
\end{pgfscope}%
\begin{pgfscope}%
\pgfsetbuttcap%
\pgfsetroundjoin%
\definecolor{currentfill}{rgb}{0.000000,0.000000,0.000000}%
\pgfsetfillcolor{currentfill}%
\pgfsetlinewidth{0.602250pt}%
\definecolor{currentstroke}{rgb}{0.000000,0.000000,0.000000}%
\pgfsetstrokecolor{currentstroke}%
\pgfsetdash{}{0pt}%
\pgfsys@defobject{currentmarker}{\pgfqpoint{-0.027778in}{0.000000in}}{\pgfqpoint{-0.000000in}{0.000000in}}{%
\pgfpathmoveto{\pgfqpoint{-0.000000in}{0.000000in}}%
\pgfpathlineto{\pgfqpoint{-0.027778in}{0.000000in}}%
\pgfusepath{stroke,fill}%
}%
\begin{pgfscope}%
\pgfsys@transformshift{0.509263in}{1.692153in}%
\pgfsys@useobject{currentmarker}{}%
\end{pgfscope}%
\end{pgfscope}%
\begin{pgfscope}%
\pgfpathrectangle{\pgfqpoint{0.509263in}{0.417642in}}{\pgfqpoint{3.509067in}{2.050688in}}%
\pgfusepath{clip}%
\pgfsetrectcap%
\pgfsetroundjoin%
\pgfsetlinewidth{0.803000pt}%
\definecolor{currentstroke}{rgb}{0.850000,0.850000,0.850000}%
\pgfsetstrokecolor{currentstroke}%
\pgfsetdash{}{0pt}%
\pgfpathmoveto{\pgfqpoint{0.509263in}{2.121456in}}%
\pgfpathlineto{\pgfqpoint{4.018330in}{2.121456in}}%
\pgfusepath{stroke}%
\end{pgfscope}%
\begin{pgfscope}%
\pgfsetbuttcap%
\pgfsetroundjoin%
\definecolor{currentfill}{rgb}{0.000000,0.000000,0.000000}%
\pgfsetfillcolor{currentfill}%
\pgfsetlinewidth{0.602250pt}%
\definecolor{currentstroke}{rgb}{0.000000,0.000000,0.000000}%
\pgfsetstrokecolor{currentstroke}%
\pgfsetdash{}{0pt}%
\pgfsys@defobject{currentmarker}{\pgfqpoint{-0.027778in}{0.000000in}}{\pgfqpoint{-0.000000in}{0.000000in}}{%
\pgfpathmoveto{\pgfqpoint{-0.000000in}{0.000000in}}%
\pgfpathlineto{\pgfqpoint{-0.027778in}{0.000000in}}%
\pgfusepath{stroke,fill}%
}%
\begin{pgfscope}%
\pgfsys@transformshift{0.509263in}{2.121456in}%
\pgfsys@useobject{currentmarker}{}%
\end{pgfscope}%
\end{pgfscope}%
\begin{pgfscope}%
\pgfpathrectangle{\pgfqpoint{0.509263in}{0.417642in}}{\pgfqpoint{3.509067in}{2.050688in}}%
\pgfusepath{clip}%
\pgfsetrectcap%
\pgfsetroundjoin%
\pgfsetlinewidth{0.803000pt}%
\definecolor{currentstroke}{rgb}{0.850000,0.850000,0.850000}%
\pgfsetstrokecolor{currentstroke}%
\pgfsetdash{}{0pt}%
\pgfpathmoveto{\pgfqpoint{0.509263in}{2.339447in}}%
\pgfpathlineto{\pgfqpoint{4.018330in}{2.339447in}}%
\pgfusepath{stroke}%
\end{pgfscope}%
\begin{pgfscope}%
\pgfsetbuttcap%
\pgfsetroundjoin%
\definecolor{currentfill}{rgb}{0.000000,0.000000,0.000000}%
\pgfsetfillcolor{currentfill}%
\pgfsetlinewidth{0.602250pt}%
\definecolor{currentstroke}{rgb}{0.000000,0.000000,0.000000}%
\pgfsetstrokecolor{currentstroke}%
\pgfsetdash{}{0pt}%
\pgfsys@defobject{currentmarker}{\pgfqpoint{-0.027778in}{0.000000in}}{\pgfqpoint{-0.000000in}{0.000000in}}{%
\pgfpathmoveto{\pgfqpoint{-0.000000in}{0.000000in}}%
\pgfpathlineto{\pgfqpoint{-0.027778in}{0.000000in}}%
\pgfusepath{stroke,fill}%
}%
\begin{pgfscope}%
\pgfsys@transformshift{0.509263in}{2.339447in}%
\pgfsys@useobject{currentmarker}{}%
\end{pgfscope}%
\end{pgfscope}%
\begin{pgfscope}%
\definecolor{textcolor}{rgb}{0.000000,0.000000,0.000000}%
\pgfsetstrokecolor{textcolor}%
\pgfsetfillcolor{textcolor}%
\pgftext[x=0.180559in,y=1.442986in,,bottom,rotate=90.000000]{\color{textcolor}\rmfamily\fontsize{10.000000}{12.000000}\selectfont ADEV \(\displaystyle \sigma_A(\tau)\)}%
\end{pgfscope}%
\begin{pgfscope}%
\pgfpathrectangle{\pgfqpoint{0.509263in}{0.417642in}}{\pgfqpoint{3.509067in}{2.050688in}}%
\pgfusepath{clip}%
\pgfsetbuttcap%
\pgfsetroundjoin%
\definecolor{currentfill}{rgb}{0.337255,0.705882,0.913725}%
\pgfsetfillcolor{currentfill}%
\pgfsetlinewidth{1.003750pt}%
\definecolor{currentstroke}{rgb}{0.337255,0.705882,0.913725}%
\pgfsetstrokecolor{currentstroke}%
\pgfsetdash{}{0pt}%
\pgfsys@defobject{currentmarker}{\pgfqpoint{-0.020833in}{-0.020833in}}{\pgfqpoint{0.020833in}{0.020833in}}{%
\pgfpathmoveto{\pgfqpoint{0.000000in}{-0.020833in}}%
\pgfpathcurveto{\pgfqpoint{0.005525in}{-0.020833in}}{\pgfqpoint{0.010825in}{-0.018638in}}{\pgfqpoint{0.014731in}{-0.014731in}}%
\pgfpathcurveto{\pgfqpoint{0.018638in}{-0.010825in}}{\pgfqpoint{0.020833in}{-0.005525in}}{\pgfqpoint{0.020833in}{0.000000in}}%
\pgfpathcurveto{\pgfqpoint{0.020833in}{0.005525in}}{\pgfqpoint{0.018638in}{0.010825in}}{\pgfqpoint{0.014731in}{0.014731in}}%
\pgfpathcurveto{\pgfqpoint{0.010825in}{0.018638in}}{\pgfqpoint{0.005525in}{0.020833in}}{\pgfqpoint{0.000000in}{0.020833in}}%
\pgfpathcurveto{\pgfqpoint{-0.005525in}{0.020833in}}{\pgfqpoint{-0.010825in}{0.018638in}}{\pgfqpoint{-0.014731in}{0.014731in}}%
\pgfpathcurveto{\pgfqpoint{-0.018638in}{0.010825in}}{\pgfqpoint{-0.020833in}{0.005525in}}{\pgfqpoint{-0.020833in}{0.000000in}}%
\pgfpathcurveto{\pgfqpoint{-0.020833in}{-0.005525in}}{\pgfqpoint{-0.018638in}{-0.010825in}}{\pgfqpoint{-0.014731in}{-0.014731in}}%
\pgfpathcurveto{\pgfqpoint{-0.010825in}{-0.018638in}}{\pgfqpoint{-0.005525in}{-0.020833in}}{\pgfqpoint{0.000000in}{-0.020833in}}%
\pgfpathlineto{\pgfqpoint{0.000000in}{-0.020833in}}%
\pgfpathclose%
\pgfusepath{stroke,fill}%
}%
\begin{pgfscope}%
\pgfsys@transformshift{0.668766in}{2.369899in}%
\pgfsys@useobject{currentmarker}{}%
\end{pgfscope}%
\begin{pgfscope}%
\pgfsys@transformshift{0.805953in}{2.184671in}%
\pgfsys@useobject{currentmarker}{}%
\end{pgfscope}%
\begin{pgfscope}%
\pgfsys@transformshift{0.943139in}{2.000581in}%
\pgfsys@useobject{currentmarker}{}%
\end{pgfscope}%
\begin{pgfscope}%
\pgfsys@transformshift{1.124489in}{1.762046in}%
\pgfsys@useobject{currentmarker}{}%
\end{pgfscope}%
\begin{pgfscope}%
\pgfsys@transformshift{1.261676in}{1.588084in}%
\pgfsys@useobject{currentmarker}{}%
\end{pgfscope}%
\begin{pgfscope}%
\pgfsys@transformshift{1.398862in}{1.424762in}%
\pgfsys@useobject{currentmarker}{}%
\end{pgfscope}%
\begin{pgfscope}%
\pgfsys@transformshift{1.578223in}{1.239876in}%
\pgfsys@useobject{currentmarker}{}%
\end{pgfscope}%
\begin{pgfscope}%
\pgfsys@transformshift{1.716407in}{1.130909in}%
\pgfsys@useobject{currentmarker}{}%
\end{pgfscope}%
\begin{pgfscope}%
\pgfsys@transformshift{1.854089in}{1.054751in}%
\pgfsys@useobject{currentmarker}{}%
\end{pgfscope}%
\begin{pgfscope}%
\pgfsys@transformshift{2.035935in}{0.997981in}%
\pgfsys@useobject{currentmarker}{}%
\end{pgfscope}%
\begin{pgfscope}%
\pgfsys@transformshift{2.173122in}{0.978226in}%
\pgfsys@useobject{currentmarker}{}%
\end{pgfscope}%
\begin{pgfscope}%
\pgfsys@transformshift{2.310308in}{0.977475in}%
\pgfsys@useobject{currentmarker}{}%
\end{pgfscope}%
\begin{pgfscope}%
\pgfsys@transformshift{2.491658in}{1.000768in}%
\pgfsys@useobject{currentmarker}{}%
\end{pgfscope}%
\begin{pgfscope}%
\pgfsys@transformshift{2.628844in}{1.035824in}%
\pgfsys@useobject{currentmarker}{}%
\end{pgfscope}%
\begin{pgfscope}%
\pgfsys@transformshift{2.766031in}{1.098173in}%
\pgfsys@useobject{currentmarker}{}%
\end{pgfscope}%
\begin{pgfscope}%
\pgfsys@transformshift{2.947379in}{1.232832in}%
\pgfsys@useobject{currentmarker}{}%
\end{pgfscope}%
\begin{pgfscope}%
\pgfsys@transformshift{3.084566in}{1.383681in}%
\pgfsys@useobject{currentmarker}{}%
\end{pgfscope}%
\begin{pgfscope}%
\pgfsys@transformshift{3.221753in}{1.561532in}%
\pgfsys@useobject{currentmarker}{}%
\end{pgfscope}%
\begin{pgfscope}%
\pgfsys@transformshift{3.403104in}{1.855187in}%
\pgfsys@useobject{currentmarker}{}%
\end{pgfscope}%
\begin{pgfscope}%
\pgfsys@transformshift{3.540290in}{2.082638in}%
\pgfsys@useobject{currentmarker}{}%
\end{pgfscope}%
\begin{pgfscope}%
\pgfsys@transformshift{3.677477in}{2.284021in}%
\pgfsys@useobject{currentmarker}{}%
\end{pgfscope}%
\begin{pgfscope}%
\pgfsys@transformshift{3.858827in}{2.375117in}%
\pgfsys@useobject{currentmarker}{}%
\end{pgfscope}%
\end{pgfscope}%
\begin{pgfscope}%
\pgfpathrectangle{\pgfqpoint{0.509263in}{0.417642in}}{\pgfqpoint{3.509067in}{2.050688in}}%
\pgfusepath{clip}%
\pgfsetbuttcap%
\pgfsetroundjoin%
\pgfsetlinewidth{1.505625pt}%
\definecolor{currentstroke}{rgb}{0.003922,0.450980,0.698039}%
\pgfsetstrokecolor{currentstroke}%
\pgfsetdash{{5.550000pt}{2.400000pt}}{0.000000pt}%
\pgfpathmoveto{\pgfqpoint{0.668766in}{2.367769in}}%
\pgfpathlineto{\pgfqpoint{0.805953in}{2.181440in}}%
\pgfpathlineto{\pgfqpoint{0.943139in}{1.995111in}}%
\pgfpathlineto{\pgfqpoint{1.124489in}{1.748798in}}%
\pgfpathlineto{\pgfqpoint{1.261676in}{1.562469in}}%
\pgfpathlineto{\pgfqpoint{1.398862in}{1.376140in}}%
\pgfpathlineto{\pgfqpoint{1.578223in}{1.132528in}}%
\pgfpathlineto{\pgfqpoint{1.716407in}{0.944845in}}%
\pgfpathlineto{\pgfqpoint{1.854089in}{0.757841in}}%
\pgfpathlineto{\pgfqpoint{2.035935in}{0.510855in}}%
\pgfusepath{stroke}%
\end{pgfscope}%
\begin{pgfscope}%
\pgfpathrectangle{\pgfqpoint{0.509263in}{0.417642in}}{\pgfqpoint{3.509067in}{2.050688in}}%
\pgfusepath{clip}%
\pgfsetbuttcap%
\pgfsetroundjoin%
\pgfsetlinewidth{1.505625pt}%
\definecolor{currentstroke}{rgb}{0.007843,0.619608,0.450980}%
\pgfsetstrokecolor{currentstroke}%
\pgfsetdash{{5.550000pt}{2.400000pt}}{0.000000pt}%
\pgfpathmoveto{\pgfqpoint{0.668766in}{0.943498in}}%
\pgfpathlineto{\pgfqpoint{0.805953in}{0.943498in}}%
\pgfpathlineto{\pgfqpoint{0.943139in}{0.943498in}}%
\pgfpathlineto{\pgfqpoint{1.124489in}{0.943498in}}%
\pgfpathlineto{\pgfqpoint{1.261676in}{0.943498in}}%
\pgfpathlineto{\pgfqpoint{1.398862in}{0.943498in}}%
\pgfpathlineto{\pgfqpoint{1.578223in}{0.943498in}}%
\pgfpathlineto{\pgfqpoint{1.716407in}{0.943498in}}%
\pgfpathlineto{\pgfqpoint{1.854089in}{0.943498in}}%
\pgfpathlineto{\pgfqpoint{2.035935in}{0.943498in}}%
\pgfpathlineto{\pgfqpoint{2.173122in}{0.943498in}}%
\pgfpathlineto{\pgfqpoint{2.310308in}{0.943498in}}%
\pgfpathlineto{\pgfqpoint{2.491658in}{0.943498in}}%
\pgfpathlineto{\pgfqpoint{2.628844in}{0.943498in}}%
\pgfpathlineto{\pgfqpoint{2.766031in}{0.943498in}}%
\pgfpathlineto{\pgfqpoint{2.947379in}{0.943498in}}%
\pgfpathlineto{\pgfqpoint{3.084566in}{0.943498in}}%
\pgfpathlineto{\pgfqpoint{3.221753in}{0.943498in}}%
\pgfpathlineto{\pgfqpoint{3.403104in}{0.943498in}}%
\pgfpathlineto{\pgfqpoint{3.540290in}{0.943498in}}%
\pgfpathlineto{\pgfqpoint{3.677477in}{0.943498in}}%
\pgfpathlineto{\pgfqpoint{3.858827in}{0.943498in}}%
\pgfusepath{stroke}%
\end{pgfscope}%
\begin{pgfscope}%
\pgfpathrectangle{\pgfqpoint{0.509263in}{0.417642in}}{\pgfqpoint{3.509067in}{2.050688in}}%
\pgfusepath{clip}%
\pgfsetbuttcap%
\pgfsetroundjoin%
\pgfsetlinewidth{1.505625pt}%
\definecolor{currentstroke}{rgb}{0.835294,0.368627,0.000000}%
\pgfsetstrokecolor{currentstroke}%
\pgfsetdash{{5.550000pt}{2.400000pt}}{0.000000pt}%
\pgfpathmoveto{\pgfqpoint{2.491658in}{0.510855in}}%
\pgfpathlineto{\pgfqpoint{2.628844in}{0.697184in}}%
\pgfpathlineto{\pgfqpoint{2.766031in}{0.883513in}}%
\pgfpathlineto{\pgfqpoint{2.947379in}{1.129824in}}%
\pgfpathlineto{\pgfqpoint{3.084566in}{1.316154in}}%
\pgfpathlineto{\pgfqpoint{3.221753in}{1.502484in}}%
\pgfpathlineto{\pgfqpoint{3.403104in}{1.748798in}}%
\pgfpathlineto{\pgfqpoint{3.540290in}{1.935127in}}%
\pgfpathlineto{\pgfqpoint{3.677477in}{2.121456in}}%
\pgfpathlineto{\pgfqpoint{3.858827in}{2.367769in}}%
\pgfusepath{stroke}%
\end{pgfscope}%
\begin{pgfscope}%
\pgfsetrectcap%
\pgfsetmiterjoin%
\pgfsetlinewidth{0.803000pt}%
\definecolor{currentstroke}{rgb}{0.000000,0.000000,0.000000}%
\pgfsetstrokecolor{currentstroke}%
\pgfsetdash{}{0pt}%
\pgfpathmoveto{\pgfqpoint{0.509263in}{0.417642in}}%
\pgfpathlineto{\pgfqpoint{0.509263in}{2.468330in}}%
\pgfusepath{stroke}%
\end{pgfscope}%
\begin{pgfscope}%
\pgfsetrectcap%
\pgfsetmiterjoin%
\pgfsetlinewidth{0.803000pt}%
\definecolor{currentstroke}{rgb}{0.000000,0.000000,0.000000}%
\pgfsetstrokecolor{currentstroke}%
\pgfsetdash{}{0pt}%
\pgfpathmoveto{\pgfqpoint{4.018330in}{0.417642in}}%
\pgfpathlineto{\pgfqpoint{4.018330in}{2.468330in}}%
\pgfusepath{stroke}%
\end{pgfscope}%
\begin{pgfscope}%
\pgfsetrectcap%
\pgfsetmiterjoin%
\pgfsetlinewidth{0.803000pt}%
\definecolor{currentstroke}{rgb}{0.000000,0.000000,0.000000}%
\pgfsetstrokecolor{currentstroke}%
\pgfsetdash{}{0pt}%
\pgfpathmoveto{\pgfqpoint{0.509263in}{0.417642in}}%
\pgfpathlineto{\pgfqpoint{4.018330in}{0.417642in}}%
\pgfusepath{stroke}%
\end{pgfscope}%
\begin{pgfscope}%
\pgfsetrectcap%
\pgfsetmiterjoin%
\pgfsetlinewidth{0.803000pt}%
\definecolor{currentstroke}{rgb}{0.000000,0.000000,0.000000}%
\pgfsetstrokecolor{currentstroke}%
\pgfsetdash{}{0pt}%
\pgfpathmoveto{\pgfqpoint{0.509263in}{2.468330in}}%
\pgfpathlineto{\pgfqpoint{4.018330in}{2.468330in}}%
\pgfusepath{stroke}%
\end{pgfscope}%
\begin{pgfscope}%
\pgfsetbuttcap%
\pgfsetmiterjoin%
\definecolor{currentfill}{rgb}{1.000000,1.000000,1.000000}%
\pgfsetfillcolor{currentfill}%
\pgfsetfillopacity{0.800000}%
\pgfsetlinewidth{1.003750pt}%
\definecolor{currentstroke}{rgb}{0.800000,0.800000,0.800000}%
\pgfsetstrokecolor{currentstroke}%
\pgfsetstrokeopacity{0.800000}%
\pgfsetdash{}{0pt}%
\pgfpathmoveto{\pgfqpoint{1.334902in}{1.696104in}}%
\pgfpathlineto{\pgfqpoint{3.192691in}{1.696104in}}%
\pgfpathquadraticcurveto{\pgfqpoint{3.214914in}{1.696104in}}{\pgfqpoint{3.214914in}{1.718327in}}%
\pgfpathlineto{\pgfqpoint{3.214914in}{2.390552in}}%
\pgfpathquadraticcurveto{\pgfqpoint{3.214914in}{2.412774in}}{\pgfqpoint{3.192691in}{2.412774in}}%
\pgfpathlineto{\pgfqpoint{1.334902in}{2.412774in}}%
\pgfpathquadraticcurveto{\pgfqpoint{1.312680in}{2.412774in}}{\pgfqpoint{1.312680in}{2.390552in}}%
\pgfpathlineto{\pgfqpoint{1.312680in}{1.718327in}}%
\pgfpathquadraticcurveto{\pgfqpoint{1.312680in}{1.696104in}}{\pgfqpoint{1.334902in}{1.696104in}}%
\pgfpathlineto{\pgfqpoint{1.334902in}{1.696104in}}%
\pgfpathclose%
\pgfusepath{stroke,fill}%
\end{pgfscope}%
\begin{pgfscope}%
\pgfsetbuttcap%
\pgfsetroundjoin%
\pgfsetlinewidth{1.505625pt}%
\definecolor{currentstroke}{rgb}{0.003922,0.450980,0.698039}%
\pgfsetstrokecolor{currentstroke}%
\pgfsetdash{{5.550000pt}{2.400000pt}}{0.000000pt}%
\pgfpathmoveto{\pgfqpoint{1.357124in}{2.275930in}}%
\pgfpathlineto{\pgfqpoint{1.468235in}{2.275930in}}%
\pgfpathlineto{\pgfqpoint{1.579346in}{2.275930in}}%
\pgfusepath{stroke}%
\end{pgfscope}%
\begin{pgfscope}%
\definecolor{textcolor}{rgb}{0.000000,0.000000,0.000000}%
\pgfsetstrokecolor{textcolor}%
\pgfsetfillcolor{textcolor}%
\pgftext[x=1.668235in,y=2.237041in,left,base]{\color{textcolor}\rmfamily\fontsize{8.000000}{9.600000}\selectfont White noise \(\displaystyle \propto \sqrt{h_{0}}\tau^{-0.5}\)}%
\end{pgfscope}%
\begin{pgfscope}%
\pgfsetbuttcap%
\pgfsetroundjoin%
\pgfsetlinewidth{1.505625pt}%
\definecolor{currentstroke}{rgb}{0.007843,0.619608,0.450980}%
\pgfsetstrokecolor{currentstroke}%
\pgfsetdash{{5.550000pt}{2.400000pt}}{0.000000pt}%
\pgfpathmoveto{\pgfqpoint{1.357124in}{2.053938in}}%
\pgfpathlineto{\pgfqpoint{1.468235in}{2.053938in}}%
\pgfpathlineto{\pgfqpoint{1.579346in}{2.053938in}}%
\pgfusepath{stroke}%
\end{pgfscope}%
\begin{pgfscope}%
\definecolor{textcolor}{rgb}{0.000000,0.000000,0.000000}%
\pgfsetstrokecolor{textcolor}%
\pgfsetfillcolor{textcolor}%
\pgftext[x=1.668235in,y=2.015049in,left,base]{\color{textcolor}\rmfamily\fontsize{8.000000}{9.600000}\selectfont Flicker noise \(\displaystyle \propto \sqrt{h_{-1}}\tau^{+0.0}\)}%
\end{pgfscope}%
\begin{pgfscope}%
\pgfsetbuttcap%
\pgfsetroundjoin%
\pgfsetlinewidth{1.505625pt}%
\definecolor{currentstroke}{rgb}{0.835294,0.368627,0.000000}%
\pgfsetstrokecolor{currentstroke}%
\pgfsetdash{{5.550000pt}{2.400000pt}}{0.000000pt}%
\pgfpathmoveto{\pgfqpoint{1.357124in}{1.826159in}}%
\pgfpathlineto{\pgfqpoint{1.468235in}{1.826159in}}%
\pgfpathlineto{\pgfqpoint{1.579346in}{1.826159in}}%
\pgfusepath{stroke}%
\end{pgfscope}%
\begin{pgfscope}%
\definecolor{textcolor}{rgb}{0.000000,0.000000,0.000000}%
\pgfsetstrokecolor{textcolor}%
\pgfsetfillcolor{textcolor}%
\pgftext[x=1.668235in,y=1.787270in,left,base]{\color{textcolor}\rmfamily\fontsize{8.000000}{9.600000}\selectfont Random walk \(\displaystyle \propto \sqrt{h_{-2}}\tau^{+0.5}\)}%
\end{pgfscope}%
\end{pgfpicture}%
\makeatother%
\endgroup%
% data/simulations/sim_allan_variance_example.py
    \caption{A simulated Allan deviation containing white noise, flicker noise and random walk behaviour.}
    \label{fig:adev_example_adev}
\end{figure}

The Allan variance was calculated using the overlapping Allan variance algorithm \cite{oadev_definition} and only Allan deviation values for frequency values of $(1, 2, 4)$ per decade were plotted. The overlapping Allan variance gives a better confidence at longer intervals or lower frequencies, allowing to identify very low frequency noise like the random walk shown here. Reference \cite{oadev_definition} also gives a very good comparison of other algorithms to identify even more noise types in data sets like phase noise. Plotting only three values per decade improves the clarity of the plot, because at longer $\tau$s, even though the overlapping Allan variance is used, some oscillations inevitably show up. Using fewer values of $\tau$ causes less distractions in this case.
From the figure \ref{fig:adev_example_adev}, the Allan deviation of the flicker noise can be estimated from the flat minimum to be around $2.3$ or $\sqrt{5}$. Using table \ref{tab:adev_alpha} the Allan variance can be converted to
\begin{equation*}
    h_{-1} = \frac{5}{2 \ln 2} \approx 3.6
\end{equation*}

Using the previously found $h_0$, this corner frequency is calculated using equation \ref{eqn:corner_frequency} to be:
\begin{equation*}
    f_c = \frac{5}{\qty{2e-3}{\per \Hz} \cdot 2 \ln 2} \approx \qty{1.8}{\kHz}
\end{equation*}

This is obviously the same result as the one from the geometric approach above.

This concludes the examples section for different noise types. The reader should now be able to identify different types of noise in measurement data and have learnt to appreciate the value that the Allen variance brings to the table. An example was presented that applied all techniques shown in this section to extract information about the noise sources in a dataset. Additionally, Python source code is provided to further explore the topic.

\clearpage
\section{Autozeroing}%
\label{sec:autozero}
Autozeroing (AZ), sometimes called zero-drift or dynamic offset compensation, is such an important concept that it must be discussed in its own right. The need for autozeroing comes from the typical behaviour of amplifiers. Every amplifier has some offset, be it small or large, and especially at high gains, this offset becomes a problem for high precision measurements. To make matters worse, this offset is not stable over time and drifts with both time and temperature. It can therefore not be calibrated out once, it must be permanently adjusted during operation, depending on environmental conditions. This procedure is called autozeroing.

There are many different ways to implement autozeroing and regarding operational amplifiers a good overview can be found in \cite{horowitz1989}. As an example, the autozero cycle for the Keithley \device{Model 2002} and the Keysight \device{3458A} Multimeter is shown in figure \ref{fig:dmm_autozero_comparison}. Keithley uses a more complex and slower algorithm, while HP implemented a simpler but faster algorithm. The most simple (digital) approach is to regularly switch the input from the signal to zero, take a reading, then subtract this reading from all subsequent readings until a new zero reading is taken. An alternative approach adds another measurement of the reference voltage to apply a gain correction as well. This is done by the Keithley \device{Model 2002} and works very well to suppress gain drift in the input amplifier due to temperature changes but increases the time between samples by another \qty{50}{\percent}. The Keysight/HP \device{3458A} in the other hand calculates those gain corrections only during the manual auto-calibration (ACAL) routine to maintain a higher throughput.
\begin{figure}[hb]
    \centering
    %\resizebox {0.8\textwidth} {!} {
        \import{figures/}{dmm_autozero.tex}
    %} % resizebox
    \caption{Auto-zero phases of the Keysight \device{3458A} and Keithley \device{Model 2002}.}
    \label{fig:dmm_autozero_comparison}
\end{figure}

\subsection{Offset Nulling}%
\label{sec:autozero_offset_nulling}
Offset nulling is the most basic approach to autozeroing. It aims to remove the offset drift of an amplifier. Especially at high gains, the offset, which is multiplied by the gain, can be substantial. In order to explain how offset nulling works and how it shapes the spectrum, it is best to discuss it based on an example. While this technique can also be found in many integrated circuits, it is more noticeable in DMMs, because it is a switchable option. Therefore, the example data set simulated is based on the parameters of the aforementioned Keysight \device{3458A} multimeter. The corner frequency and the white noise floor is modeled after the \qty{10}{\V} range of the \device{3458A} \cite{3458A_noise_floor, sampling_with_3458A} with the values given below. Do note that both references \cite{3458A_noise_floor, sampling_with_3458A} contain a typographical error. The corner frequency of the noise floor is erroneously given as \qty{0.5}{\Hz}, but should be \qty{1.5}{\Hz}. This can be seen in figure 2.35 in \citep[p. 116]{sampling_with_3458A}, where the noise spectral density is plotted and it was also confirmed with the author \cite{lapuh_email_corner_frequency}. The data used in this section is generated using the Python \textit{AllanTools} library \cite{allantools} and the simulation source code can be found in \external{data/simulations/sim\_autozero.py} as part of the online supplemental material \cite{supplemental_material}.
\begin{figure}[ht]
    \centering
    \scalebox{1}{%
        \import{figures/}{offset_nulling_definitions.tex}
    } % scalebox
    \caption{Integration sequences of the offset nulling algorithm. Solid lines denote sampled data. Red is the input signal, green is the zero reading and blue is the dead time required for switching inputs.}
    \label{fig:dmm_autozer_offset_nulling}
\end{figure}

For this simulation, a noise-free and arbitrarily chosen \qty{10}{\V} input is assumed to be sampled by the device at a sampling rate of \qty{10}{\plc} at \qty{50}{\Hz}, the same rate discussed previously on page \pageref{sec:dead_time}. As it will be shown, the actual mean value of the input signal has no bearing on the outcome of the calculation when considering offset nulling, but its value must be considered for other types of autozeroing as discussed in section \ref{sec:autozero_gain} and is included here only for the sake of completeness.

Figure \ref{fig:dmm_autozer_offset_nulling} shows the individual sequences of the offset nulling algorithm. First, the source is sampled for $\tau_s = \qty{10}{\plc}$, then the input is switched to the LO terminal. While this operation is very fast and takes less than \qty{1}{\ms} \cite{article_3458A_input_impedance}, if the instrument is synchronized to the line frequency the zero measurement will nonetheless be delayed until the next zero crossing, hence the dead-time $\theta = \qty{1}{\plc}$. Finally, the zero reference is measured for another $\tau_r = \qty{10}{\plc}$ and then the instrument switches back to the HI terminal.

The data is simulated in the following way: First, two sets of noise data are generated, a white noise spectrum with a noise spectral density of \qty[power-half-as-sqrt, per-mode=symbol]{165}{\nV \Hz\tothe{-0.5}} and a flicker noise spectrum with an intensity scaled to result in a final spectrum with a corner frequency of \qty{1.5}{\Hz}. The required flicker noise intensity is calculated using equation \ref{eqn:corner_frequency}. To get a good low frequency estimate, $2^{20} \approx 10^{6}$ values were generated. Finally, the two noise data sets are summed with the noise-free input source to give the final result. Other effects, such as power-line hum are neglected in this simple simulation because it would needlessly overcomplicate the example and limit the educational value. The same goes for higher order random-walk $f^{-2}$ noise components, which can be introduced by temperature fluctuations and other environmental effects and would be present in a real measurement.
\begin{figure}[ht]
    \centering
    %% Creator: Matplotlib, PGF backend
%%
%% To include the figure in your LaTeX document, write
%%   \input{<filename>.pgf}
%%
%% Make sure the required packages are loaded in your preamble
%%   \usepackage{pgf}
%%
%% Also ensure that all the required font packages are loaded; for instance,
%% the lmodern package is sometimes necessary when using math font.
%%   \usepackage{lmodern}
%%
%% Figures using additional raster images can only be included by \input if
%% they are in the same directory as the main LaTeX file. For loading figures
%% from other directories you can use the `import` package
%%   \usepackage{import}
%%
%% and then include the figures with
%%   \import{<path to file>}{<filename>.pgf}
%%
%% Matplotlib used the following preamble
%%   \def\mathdefault#1{#1}
%%   \everymath=\expandafter{\the\everymath\displaystyle}
%%   \usepackage{siunitx}
%%   \sisetup{per-mode = symbol}%
%%   \ifdefined\pdftexversion\else  % non-pdftex case.
%%     \usepackage{fontspec}
%%   \fi
%%   \makeatletter\@ifpackageloaded{underscore}{}{\usepackage[strings]{underscore}}\makeatother
%%
\begingroup%
\makeatletter%
\begin{pgfpicture}%
\pgfpathrectangle{\pgfpointorigin}{\pgfqpoint{4.068242in}{2.514312in}}%
\pgfusepath{use as bounding box, clip}%
\begin{pgfscope}%
\pgfsetbuttcap%
\pgfsetmiterjoin%
\definecolor{currentfill}{rgb}{1.000000,1.000000,1.000000}%
\pgfsetfillcolor{currentfill}%
\pgfsetlinewidth{0.000000pt}%
\definecolor{currentstroke}{rgb}{1.000000,1.000000,1.000000}%
\pgfsetstrokecolor{currentstroke}%
\pgfsetdash{}{0pt}%
\pgfpathmoveto{\pgfqpoint{0.000000in}{0.000000in}}%
\pgfpathlineto{\pgfqpoint{4.068242in}{0.000000in}}%
\pgfpathlineto{\pgfqpoint{4.068242in}{2.514312in}}%
\pgfpathlineto{\pgfqpoint{0.000000in}{2.514312in}}%
\pgfpathlineto{\pgfqpoint{0.000000in}{0.000000in}}%
\pgfpathclose%
\pgfusepath{fill}%
\end{pgfscope}%
\begin{pgfscope}%
\pgfsetbuttcap%
\pgfsetmiterjoin%
\definecolor{currentfill}{rgb}{1.000000,1.000000,1.000000}%
\pgfsetfillcolor{currentfill}%
\pgfsetlinewidth{0.000000pt}%
\definecolor{currentstroke}{rgb}{0.000000,0.000000,0.000000}%
\pgfsetstrokecolor{currentstroke}%
\pgfsetstrokeopacity{0.000000}%
\pgfsetdash{}{0pt}%
\pgfpathmoveto{\pgfqpoint{0.471687in}{0.416447in}}%
\pgfpathlineto{\pgfqpoint{4.009533in}{0.416447in}}%
\pgfpathlineto{\pgfqpoint{4.009533in}{2.341095in}}%
\pgfpathlineto{\pgfqpoint{0.471687in}{2.341095in}}%
\pgfpathlineto{\pgfqpoint{0.471687in}{0.416447in}}%
\pgfpathclose%
\pgfusepath{fill}%
\end{pgfscope}%
\begin{pgfscope}%
\pgfpathrectangle{\pgfqpoint{0.471687in}{0.416447in}}{\pgfqpoint{3.537845in}{1.924647in}}%
\pgfusepath{clip}%
\pgfsetrectcap%
\pgfsetroundjoin%
\pgfsetlinewidth{0.803000pt}%
\definecolor{currentstroke}{rgb}{0.450000,0.450000,0.450000}%
\pgfsetstrokecolor{currentstroke}%
\pgfsetdash{}{0pt}%
\pgfpathmoveto{\pgfqpoint{0.632499in}{0.416447in}}%
\pgfpathlineto{\pgfqpoint{0.632499in}{2.341095in}}%
\pgfusepath{stroke}%
\end{pgfscope}%
\begin{pgfscope}%
\pgfsetbuttcap%
\pgfsetroundjoin%
\definecolor{currentfill}{rgb}{0.000000,0.000000,0.000000}%
\pgfsetfillcolor{currentfill}%
\pgfsetlinewidth{0.803000pt}%
\definecolor{currentstroke}{rgb}{0.000000,0.000000,0.000000}%
\pgfsetstrokecolor{currentstroke}%
\pgfsetdash{}{0pt}%
\pgfsys@defobject{currentmarker}{\pgfqpoint{0.000000in}{-0.048611in}}{\pgfqpoint{0.000000in}{0.000000in}}{%
\pgfpathmoveto{\pgfqpoint{0.000000in}{0.000000in}}%
\pgfpathlineto{\pgfqpoint{0.000000in}{-0.048611in}}%
\pgfusepath{stroke,fill}%
}%
\begin{pgfscope}%
\pgfsys@transformshift{0.632499in}{0.416447in}%
\pgfsys@useobject{currentmarker}{}%
\end{pgfscope}%
\end{pgfscope}%
\begin{pgfscope}%
\definecolor{textcolor}{rgb}{0.000000,0.000000,0.000000}%
\pgfsetstrokecolor{textcolor}%
\pgfsetfillcolor{textcolor}%
\pgftext[x=0.632499in,y=0.319225in,,top]{\color{textcolor}{\rmfamily\fontsize{8.000000}{9.600000}\selectfont\catcode`\^=\active\def^{\ifmmode\sp\else\^{}\fi}\catcode`\%=\active\def%{\%}$\mathdefault{0}$}}%
\end{pgfscope}%
\begin{pgfscope}%
\pgfpathrectangle{\pgfqpoint{0.471687in}{0.416447in}}{\pgfqpoint{3.537845in}{1.924647in}}%
\pgfusepath{clip}%
\pgfsetrectcap%
\pgfsetroundjoin%
\pgfsetlinewidth{0.803000pt}%
\definecolor{currentstroke}{rgb}{0.450000,0.450000,0.450000}%
\pgfsetstrokecolor{currentstroke}%
\pgfsetdash{}{0pt}%
\pgfpathmoveto{\pgfqpoint{1.034527in}{0.416447in}}%
\pgfpathlineto{\pgfqpoint{1.034527in}{2.341095in}}%
\pgfusepath{stroke}%
\end{pgfscope}%
\begin{pgfscope}%
\pgfsetbuttcap%
\pgfsetroundjoin%
\definecolor{currentfill}{rgb}{0.000000,0.000000,0.000000}%
\pgfsetfillcolor{currentfill}%
\pgfsetlinewidth{0.803000pt}%
\definecolor{currentstroke}{rgb}{0.000000,0.000000,0.000000}%
\pgfsetstrokecolor{currentstroke}%
\pgfsetdash{}{0pt}%
\pgfsys@defobject{currentmarker}{\pgfqpoint{0.000000in}{-0.048611in}}{\pgfqpoint{0.000000in}{0.000000in}}{%
\pgfpathmoveto{\pgfqpoint{0.000000in}{0.000000in}}%
\pgfpathlineto{\pgfqpoint{0.000000in}{-0.048611in}}%
\pgfusepath{stroke,fill}%
}%
\begin{pgfscope}%
\pgfsys@transformshift{1.034527in}{0.416447in}%
\pgfsys@useobject{currentmarker}{}%
\end{pgfscope}%
\end{pgfscope}%
\begin{pgfscope}%
\definecolor{textcolor}{rgb}{0.000000,0.000000,0.000000}%
\pgfsetstrokecolor{textcolor}%
\pgfsetfillcolor{textcolor}%
\pgftext[x=1.034527in,y=0.319225in,,top]{\color{textcolor}{\rmfamily\fontsize{8.000000}{9.600000}\selectfont\catcode`\^=\active\def^{\ifmmode\sp\else\^{}\fi}\catcode`\%=\active\def%{\%}$\mathdefault{25000}$}}%
\end{pgfscope}%
\begin{pgfscope}%
\pgfpathrectangle{\pgfqpoint{0.471687in}{0.416447in}}{\pgfqpoint{3.537845in}{1.924647in}}%
\pgfusepath{clip}%
\pgfsetrectcap%
\pgfsetroundjoin%
\pgfsetlinewidth{0.803000pt}%
\definecolor{currentstroke}{rgb}{0.450000,0.450000,0.450000}%
\pgfsetstrokecolor{currentstroke}%
\pgfsetdash{}{0pt}%
\pgfpathmoveto{\pgfqpoint{1.436555in}{0.416447in}}%
\pgfpathlineto{\pgfqpoint{1.436555in}{2.341095in}}%
\pgfusepath{stroke}%
\end{pgfscope}%
\begin{pgfscope}%
\pgfsetbuttcap%
\pgfsetroundjoin%
\definecolor{currentfill}{rgb}{0.000000,0.000000,0.000000}%
\pgfsetfillcolor{currentfill}%
\pgfsetlinewidth{0.803000pt}%
\definecolor{currentstroke}{rgb}{0.000000,0.000000,0.000000}%
\pgfsetstrokecolor{currentstroke}%
\pgfsetdash{}{0pt}%
\pgfsys@defobject{currentmarker}{\pgfqpoint{0.000000in}{-0.048611in}}{\pgfqpoint{0.000000in}{0.000000in}}{%
\pgfpathmoveto{\pgfqpoint{0.000000in}{0.000000in}}%
\pgfpathlineto{\pgfqpoint{0.000000in}{-0.048611in}}%
\pgfusepath{stroke,fill}%
}%
\begin{pgfscope}%
\pgfsys@transformshift{1.436555in}{0.416447in}%
\pgfsys@useobject{currentmarker}{}%
\end{pgfscope}%
\end{pgfscope}%
\begin{pgfscope}%
\definecolor{textcolor}{rgb}{0.000000,0.000000,0.000000}%
\pgfsetstrokecolor{textcolor}%
\pgfsetfillcolor{textcolor}%
\pgftext[x=1.436555in,y=0.319225in,,top]{\color{textcolor}{\rmfamily\fontsize{8.000000}{9.600000}\selectfont\catcode`\^=\active\def^{\ifmmode\sp\else\^{}\fi}\catcode`\%=\active\def%{\%}$\mathdefault{50000}$}}%
\end{pgfscope}%
\begin{pgfscope}%
\pgfpathrectangle{\pgfqpoint{0.471687in}{0.416447in}}{\pgfqpoint{3.537845in}{1.924647in}}%
\pgfusepath{clip}%
\pgfsetrectcap%
\pgfsetroundjoin%
\pgfsetlinewidth{0.803000pt}%
\definecolor{currentstroke}{rgb}{0.450000,0.450000,0.450000}%
\pgfsetstrokecolor{currentstroke}%
\pgfsetdash{}{0pt}%
\pgfpathmoveto{\pgfqpoint{1.838584in}{0.416447in}}%
\pgfpathlineto{\pgfqpoint{1.838584in}{2.341095in}}%
\pgfusepath{stroke}%
\end{pgfscope}%
\begin{pgfscope}%
\pgfsetbuttcap%
\pgfsetroundjoin%
\definecolor{currentfill}{rgb}{0.000000,0.000000,0.000000}%
\pgfsetfillcolor{currentfill}%
\pgfsetlinewidth{0.803000pt}%
\definecolor{currentstroke}{rgb}{0.000000,0.000000,0.000000}%
\pgfsetstrokecolor{currentstroke}%
\pgfsetdash{}{0pt}%
\pgfsys@defobject{currentmarker}{\pgfqpoint{0.000000in}{-0.048611in}}{\pgfqpoint{0.000000in}{0.000000in}}{%
\pgfpathmoveto{\pgfqpoint{0.000000in}{0.000000in}}%
\pgfpathlineto{\pgfqpoint{0.000000in}{-0.048611in}}%
\pgfusepath{stroke,fill}%
}%
\begin{pgfscope}%
\pgfsys@transformshift{1.838584in}{0.416447in}%
\pgfsys@useobject{currentmarker}{}%
\end{pgfscope}%
\end{pgfscope}%
\begin{pgfscope}%
\definecolor{textcolor}{rgb}{0.000000,0.000000,0.000000}%
\pgfsetstrokecolor{textcolor}%
\pgfsetfillcolor{textcolor}%
\pgftext[x=1.838584in,y=0.319225in,,top]{\color{textcolor}{\rmfamily\fontsize{8.000000}{9.600000}\selectfont\catcode`\^=\active\def^{\ifmmode\sp\else\^{}\fi}\catcode`\%=\active\def%{\%}$\mathdefault{75000}$}}%
\end{pgfscope}%
\begin{pgfscope}%
\pgfpathrectangle{\pgfqpoint{0.471687in}{0.416447in}}{\pgfqpoint{3.537845in}{1.924647in}}%
\pgfusepath{clip}%
\pgfsetrectcap%
\pgfsetroundjoin%
\pgfsetlinewidth{0.803000pt}%
\definecolor{currentstroke}{rgb}{0.450000,0.450000,0.450000}%
\pgfsetstrokecolor{currentstroke}%
\pgfsetdash{}{0pt}%
\pgfpathmoveto{\pgfqpoint{2.240612in}{0.416447in}}%
\pgfpathlineto{\pgfqpoint{2.240612in}{2.341095in}}%
\pgfusepath{stroke}%
\end{pgfscope}%
\begin{pgfscope}%
\pgfsetbuttcap%
\pgfsetroundjoin%
\definecolor{currentfill}{rgb}{0.000000,0.000000,0.000000}%
\pgfsetfillcolor{currentfill}%
\pgfsetlinewidth{0.803000pt}%
\definecolor{currentstroke}{rgb}{0.000000,0.000000,0.000000}%
\pgfsetstrokecolor{currentstroke}%
\pgfsetdash{}{0pt}%
\pgfsys@defobject{currentmarker}{\pgfqpoint{0.000000in}{-0.048611in}}{\pgfqpoint{0.000000in}{0.000000in}}{%
\pgfpathmoveto{\pgfqpoint{0.000000in}{0.000000in}}%
\pgfpathlineto{\pgfqpoint{0.000000in}{-0.048611in}}%
\pgfusepath{stroke,fill}%
}%
\begin{pgfscope}%
\pgfsys@transformshift{2.240612in}{0.416447in}%
\pgfsys@useobject{currentmarker}{}%
\end{pgfscope}%
\end{pgfscope}%
\begin{pgfscope}%
\definecolor{textcolor}{rgb}{0.000000,0.000000,0.000000}%
\pgfsetstrokecolor{textcolor}%
\pgfsetfillcolor{textcolor}%
\pgftext[x=2.240612in,y=0.319225in,,top]{\color{textcolor}{\rmfamily\fontsize{8.000000}{9.600000}\selectfont\catcode`\^=\active\def^{\ifmmode\sp\else\^{}\fi}\catcode`\%=\active\def%{\%}$\mathdefault{100000}$}}%
\end{pgfscope}%
\begin{pgfscope}%
\pgfpathrectangle{\pgfqpoint{0.471687in}{0.416447in}}{\pgfqpoint{3.537845in}{1.924647in}}%
\pgfusepath{clip}%
\pgfsetrectcap%
\pgfsetroundjoin%
\pgfsetlinewidth{0.803000pt}%
\definecolor{currentstroke}{rgb}{0.450000,0.450000,0.450000}%
\pgfsetstrokecolor{currentstroke}%
\pgfsetdash{}{0pt}%
\pgfpathmoveto{\pgfqpoint{2.642640in}{0.416447in}}%
\pgfpathlineto{\pgfqpoint{2.642640in}{2.341095in}}%
\pgfusepath{stroke}%
\end{pgfscope}%
\begin{pgfscope}%
\pgfsetbuttcap%
\pgfsetroundjoin%
\definecolor{currentfill}{rgb}{0.000000,0.000000,0.000000}%
\pgfsetfillcolor{currentfill}%
\pgfsetlinewidth{0.803000pt}%
\definecolor{currentstroke}{rgb}{0.000000,0.000000,0.000000}%
\pgfsetstrokecolor{currentstroke}%
\pgfsetdash{}{0pt}%
\pgfsys@defobject{currentmarker}{\pgfqpoint{0.000000in}{-0.048611in}}{\pgfqpoint{0.000000in}{0.000000in}}{%
\pgfpathmoveto{\pgfqpoint{0.000000in}{0.000000in}}%
\pgfpathlineto{\pgfqpoint{0.000000in}{-0.048611in}}%
\pgfusepath{stroke,fill}%
}%
\begin{pgfscope}%
\pgfsys@transformshift{2.642640in}{0.416447in}%
\pgfsys@useobject{currentmarker}{}%
\end{pgfscope}%
\end{pgfscope}%
\begin{pgfscope}%
\definecolor{textcolor}{rgb}{0.000000,0.000000,0.000000}%
\pgfsetstrokecolor{textcolor}%
\pgfsetfillcolor{textcolor}%
\pgftext[x=2.642640in,y=0.319225in,,top]{\color{textcolor}{\rmfamily\fontsize{8.000000}{9.600000}\selectfont\catcode`\^=\active\def^{\ifmmode\sp\else\^{}\fi}\catcode`\%=\active\def%{\%}$\mathdefault{125000}$}}%
\end{pgfscope}%
\begin{pgfscope}%
\pgfpathrectangle{\pgfqpoint{0.471687in}{0.416447in}}{\pgfqpoint{3.537845in}{1.924647in}}%
\pgfusepath{clip}%
\pgfsetrectcap%
\pgfsetroundjoin%
\pgfsetlinewidth{0.803000pt}%
\definecolor{currentstroke}{rgb}{0.450000,0.450000,0.450000}%
\pgfsetstrokecolor{currentstroke}%
\pgfsetdash{}{0pt}%
\pgfpathmoveto{\pgfqpoint{3.044668in}{0.416447in}}%
\pgfpathlineto{\pgfqpoint{3.044668in}{2.341095in}}%
\pgfusepath{stroke}%
\end{pgfscope}%
\begin{pgfscope}%
\pgfsetbuttcap%
\pgfsetroundjoin%
\definecolor{currentfill}{rgb}{0.000000,0.000000,0.000000}%
\pgfsetfillcolor{currentfill}%
\pgfsetlinewidth{0.803000pt}%
\definecolor{currentstroke}{rgb}{0.000000,0.000000,0.000000}%
\pgfsetstrokecolor{currentstroke}%
\pgfsetdash{}{0pt}%
\pgfsys@defobject{currentmarker}{\pgfqpoint{0.000000in}{-0.048611in}}{\pgfqpoint{0.000000in}{0.000000in}}{%
\pgfpathmoveto{\pgfqpoint{0.000000in}{0.000000in}}%
\pgfpathlineto{\pgfqpoint{0.000000in}{-0.048611in}}%
\pgfusepath{stroke,fill}%
}%
\begin{pgfscope}%
\pgfsys@transformshift{3.044668in}{0.416447in}%
\pgfsys@useobject{currentmarker}{}%
\end{pgfscope}%
\end{pgfscope}%
\begin{pgfscope}%
\definecolor{textcolor}{rgb}{0.000000,0.000000,0.000000}%
\pgfsetstrokecolor{textcolor}%
\pgfsetfillcolor{textcolor}%
\pgftext[x=3.044668in,y=0.319225in,,top]{\color{textcolor}{\rmfamily\fontsize{8.000000}{9.600000}\selectfont\catcode`\^=\active\def^{\ifmmode\sp\else\^{}\fi}\catcode`\%=\active\def%{\%}$\mathdefault{150000}$}}%
\end{pgfscope}%
\begin{pgfscope}%
\pgfpathrectangle{\pgfqpoint{0.471687in}{0.416447in}}{\pgfqpoint{3.537845in}{1.924647in}}%
\pgfusepath{clip}%
\pgfsetrectcap%
\pgfsetroundjoin%
\pgfsetlinewidth{0.803000pt}%
\definecolor{currentstroke}{rgb}{0.450000,0.450000,0.450000}%
\pgfsetstrokecolor{currentstroke}%
\pgfsetdash{}{0pt}%
\pgfpathmoveto{\pgfqpoint{3.446697in}{0.416447in}}%
\pgfpathlineto{\pgfqpoint{3.446697in}{2.341095in}}%
\pgfusepath{stroke}%
\end{pgfscope}%
\begin{pgfscope}%
\pgfsetbuttcap%
\pgfsetroundjoin%
\definecolor{currentfill}{rgb}{0.000000,0.000000,0.000000}%
\pgfsetfillcolor{currentfill}%
\pgfsetlinewidth{0.803000pt}%
\definecolor{currentstroke}{rgb}{0.000000,0.000000,0.000000}%
\pgfsetstrokecolor{currentstroke}%
\pgfsetdash{}{0pt}%
\pgfsys@defobject{currentmarker}{\pgfqpoint{0.000000in}{-0.048611in}}{\pgfqpoint{0.000000in}{0.000000in}}{%
\pgfpathmoveto{\pgfqpoint{0.000000in}{0.000000in}}%
\pgfpathlineto{\pgfqpoint{0.000000in}{-0.048611in}}%
\pgfusepath{stroke,fill}%
}%
\begin{pgfscope}%
\pgfsys@transformshift{3.446697in}{0.416447in}%
\pgfsys@useobject{currentmarker}{}%
\end{pgfscope}%
\end{pgfscope}%
\begin{pgfscope}%
\definecolor{textcolor}{rgb}{0.000000,0.000000,0.000000}%
\pgfsetstrokecolor{textcolor}%
\pgfsetfillcolor{textcolor}%
\pgftext[x=3.446697in,y=0.319225in,,top]{\color{textcolor}{\rmfamily\fontsize{8.000000}{9.600000}\selectfont\catcode`\^=\active\def^{\ifmmode\sp\else\^{}\fi}\catcode`\%=\active\def%{\%}$\mathdefault{175000}$}}%
\end{pgfscope}%
\begin{pgfscope}%
\pgfpathrectangle{\pgfqpoint{0.471687in}{0.416447in}}{\pgfqpoint{3.537845in}{1.924647in}}%
\pgfusepath{clip}%
\pgfsetrectcap%
\pgfsetroundjoin%
\pgfsetlinewidth{0.803000pt}%
\definecolor{currentstroke}{rgb}{0.450000,0.450000,0.450000}%
\pgfsetstrokecolor{currentstroke}%
\pgfsetdash{}{0pt}%
\pgfpathmoveto{\pgfqpoint{3.848725in}{0.416447in}}%
\pgfpathlineto{\pgfqpoint{3.848725in}{2.341095in}}%
\pgfusepath{stroke}%
\end{pgfscope}%
\begin{pgfscope}%
\pgfsetbuttcap%
\pgfsetroundjoin%
\definecolor{currentfill}{rgb}{0.000000,0.000000,0.000000}%
\pgfsetfillcolor{currentfill}%
\pgfsetlinewidth{0.803000pt}%
\definecolor{currentstroke}{rgb}{0.000000,0.000000,0.000000}%
\pgfsetstrokecolor{currentstroke}%
\pgfsetdash{}{0pt}%
\pgfsys@defobject{currentmarker}{\pgfqpoint{0.000000in}{-0.048611in}}{\pgfqpoint{0.000000in}{0.000000in}}{%
\pgfpathmoveto{\pgfqpoint{0.000000in}{0.000000in}}%
\pgfpathlineto{\pgfqpoint{0.000000in}{-0.048611in}}%
\pgfusepath{stroke,fill}%
}%
\begin{pgfscope}%
\pgfsys@transformshift{3.848725in}{0.416447in}%
\pgfsys@useobject{currentmarker}{}%
\end{pgfscope}%
\end{pgfscope}%
\begin{pgfscope}%
\definecolor{textcolor}{rgb}{0.000000,0.000000,0.000000}%
\pgfsetstrokecolor{textcolor}%
\pgfsetfillcolor{textcolor}%
\pgftext[x=3.848725in,y=0.319225in,,top]{\color{textcolor}{\rmfamily\fontsize{8.000000}{9.600000}\selectfont\catcode`\^=\active\def^{\ifmmode\sp\else\^{}\fi}\catcode`\%=\active\def%{\%}$\mathdefault{200000}$}}%
\end{pgfscope}%
\begin{pgfscope}%
\definecolor{textcolor}{rgb}{0.000000,0.000000,0.000000}%
\pgfsetstrokecolor{textcolor}%
\pgfsetfillcolor{textcolor}%
\pgftext[x=2.240610in,y=0.165003in,,top]{\color{textcolor}{\rmfamily\fontsize{10.000000}{12.000000}\selectfont\catcode`\^=\active\def^{\ifmmode\sp\else\^{}\fi}\catcode`\%=\active\def%{\%}Time in $\unit{\second}$}}%
\end{pgfscope}%
\begin{pgfscope}%
\pgfpathrectangle{\pgfqpoint{0.471687in}{0.416447in}}{\pgfqpoint{3.537845in}{1.924647in}}%
\pgfusepath{clip}%
\pgfsetrectcap%
\pgfsetroundjoin%
\pgfsetlinewidth{0.803000pt}%
\definecolor{currentstroke}{rgb}{0.450000,0.450000,0.450000}%
\pgfsetstrokecolor{currentstroke}%
\pgfsetdash{}{0pt}%
\pgfpathmoveto{\pgfqpoint{0.471687in}{0.523372in}}%
\pgfpathlineto{\pgfqpoint{4.009533in}{0.523372in}}%
\pgfusepath{stroke}%
\end{pgfscope}%
\begin{pgfscope}%
\pgfsetbuttcap%
\pgfsetroundjoin%
\definecolor{currentfill}{rgb}{0.000000,0.000000,0.000000}%
\pgfsetfillcolor{currentfill}%
\pgfsetlinewidth{0.803000pt}%
\definecolor{currentstroke}{rgb}{0.000000,0.000000,0.000000}%
\pgfsetstrokecolor{currentstroke}%
\pgfsetdash{}{0pt}%
\pgfsys@defobject{currentmarker}{\pgfqpoint{-0.048611in}{0.000000in}}{\pgfqpoint{-0.000000in}{0.000000in}}{%
\pgfpathmoveto{\pgfqpoint{-0.000000in}{0.000000in}}%
\pgfpathlineto{\pgfqpoint{-0.048611in}{0.000000in}}%
\pgfusepath{stroke,fill}%
}%
\begin{pgfscope}%
\pgfsys@transformshift{0.471687in}{0.523372in}%
\pgfsys@useobject{currentmarker}{}%
\end{pgfscope}%
\end{pgfscope}%
\begin{pgfscope}%
\definecolor{textcolor}{rgb}{0.000000,0.000000,0.000000}%
\pgfsetstrokecolor{textcolor}%
\pgfsetfillcolor{textcolor}%
\pgftext[x=0.223614in, y=0.484817in, left, base]{\color{textcolor}{\rmfamily\fontsize{8.000000}{9.600000}\selectfont\catcode`\^=\active\def^{\ifmmode\sp\else\^{}\fi}\catcode`\%=\active\def%{\%}$\mathdefault{\ensuremath{-}4}$}}%
\end{pgfscope}%
\begin{pgfscope}%
\pgfpathrectangle{\pgfqpoint{0.471687in}{0.416447in}}{\pgfqpoint{3.537845in}{1.924647in}}%
\pgfusepath{clip}%
\pgfsetrectcap%
\pgfsetroundjoin%
\pgfsetlinewidth{0.803000pt}%
\definecolor{currentstroke}{rgb}{0.450000,0.450000,0.450000}%
\pgfsetstrokecolor{currentstroke}%
\pgfsetdash{}{0pt}%
\pgfpathmoveto{\pgfqpoint{0.471687in}{0.951072in}}%
\pgfpathlineto{\pgfqpoint{4.009533in}{0.951072in}}%
\pgfusepath{stroke}%
\end{pgfscope}%
\begin{pgfscope}%
\pgfsetbuttcap%
\pgfsetroundjoin%
\definecolor{currentfill}{rgb}{0.000000,0.000000,0.000000}%
\pgfsetfillcolor{currentfill}%
\pgfsetlinewidth{0.803000pt}%
\definecolor{currentstroke}{rgb}{0.000000,0.000000,0.000000}%
\pgfsetstrokecolor{currentstroke}%
\pgfsetdash{}{0pt}%
\pgfsys@defobject{currentmarker}{\pgfqpoint{-0.048611in}{0.000000in}}{\pgfqpoint{-0.000000in}{0.000000in}}{%
\pgfpathmoveto{\pgfqpoint{-0.000000in}{0.000000in}}%
\pgfpathlineto{\pgfqpoint{-0.048611in}{0.000000in}}%
\pgfusepath{stroke,fill}%
}%
\begin{pgfscope}%
\pgfsys@transformshift{0.471687in}{0.951072in}%
\pgfsys@useobject{currentmarker}{}%
\end{pgfscope}%
\end{pgfscope}%
\begin{pgfscope}%
\definecolor{textcolor}{rgb}{0.000000,0.000000,0.000000}%
\pgfsetstrokecolor{textcolor}%
\pgfsetfillcolor{textcolor}%
\pgftext[x=0.223614in, y=0.912516in, left, base]{\color{textcolor}{\rmfamily\fontsize{8.000000}{9.600000}\selectfont\catcode`\^=\active\def^{\ifmmode\sp\else\^{}\fi}\catcode`\%=\active\def%{\%}$\mathdefault{\ensuremath{-}2}$}}%
\end{pgfscope}%
\begin{pgfscope}%
\pgfpathrectangle{\pgfqpoint{0.471687in}{0.416447in}}{\pgfqpoint{3.537845in}{1.924647in}}%
\pgfusepath{clip}%
\pgfsetrectcap%
\pgfsetroundjoin%
\pgfsetlinewidth{0.803000pt}%
\definecolor{currentstroke}{rgb}{0.450000,0.450000,0.450000}%
\pgfsetstrokecolor{currentstroke}%
\pgfsetdash{}{0pt}%
\pgfpathmoveto{\pgfqpoint{0.471687in}{1.378771in}}%
\pgfpathlineto{\pgfqpoint{4.009533in}{1.378771in}}%
\pgfusepath{stroke}%
\end{pgfscope}%
\begin{pgfscope}%
\pgfsetbuttcap%
\pgfsetroundjoin%
\definecolor{currentfill}{rgb}{0.000000,0.000000,0.000000}%
\pgfsetfillcolor{currentfill}%
\pgfsetlinewidth{0.803000pt}%
\definecolor{currentstroke}{rgb}{0.000000,0.000000,0.000000}%
\pgfsetstrokecolor{currentstroke}%
\pgfsetdash{}{0pt}%
\pgfsys@defobject{currentmarker}{\pgfqpoint{-0.048611in}{0.000000in}}{\pgfqpoint{-0.000000in}{0.000000in}}{%
\pgfpathmoveto{\pgfqpoint{-0.000000in}{0.000000in}}%
\pgfpathlineto{\pgfqpoint{-0.048611in}{0.000000in}}%
\pgfusepath{stroke,fill}%
}%
\begin{pgfscope}%
\pgfsys@transformshift{0.471687in}{1.378771in}%
\pgfsys@useobject{currentmarker}{}%
\end{pgfscope}%
\end{pgfscope}%
\begin{pgfscope}%
\definecolor{textcolor}{rgb}{0.000000,0.000000,0.000000}%
\pgfsetstrokecolor{textcolor}%
\pgfsetfillcolor{textcolor}%
\pgftext[x=0.315437in, y=1.340216in, left, base]{\color{textcolor}{\rmfamily\fontsize{8.000000}{9.600000}\selectfont\catcode`\^=\active\def^{\ifmmode\sp\else\^{}\fi}\catcode`\%=\active\def%{\%}$\mathdefault{0}$}}%
\end{pgfscope}%
\begin{pgfscope}%
\pgfpathrectangle{\pgfqpoint{0.471687in}{0.416447in}}{\pgfqpoint{3.537845in}{1.924647in}}%
\pgfusepath{clip}%
\pgfsetrectcap%
\pgfsetroundjoin%
\pgfsetlinewidth{0.803000pt}%
\definecolor{currentstroke}{rgb}{0.450000,0.450000,0.450000}%
\pgfsetstrokecolor{currentstroke}%
\pgfsetdash{}{0pt}%
\pgfpathmoveto{\pgfqpoint{0.471687in}{1.806471in}}%
\pgfpathlineto{\pgfqpoint{4.009533in}{1.806471in}}%
\pgfusepath{stroke}%
\end{pgfscope}%
\begin{pgfscope}%
\pgfsetbuttcap%
\pgfsetroundjoin%
\definecolor{currentfill}{rgb}{0.000000,0.000000,0.000000}%
\pgfsetfillcolor{currentfill}%
\pgfsetlinewidth{0.803000pt}%
\definecolor{currentstroke}{rgb}{0.000000,0.000000,0.000000}%
\pgfsetstrokecolor{currentstroke}%
\pgfsetdash{}{0pt}%
\pgfsys@defobject{currentmarker}{\pgfqpoint{-0.048611in}{0.000000in}}{\pgfqpoint{-0.000000in}{0.000000in}}{%
\pgfpathmoveto{\pgfqpoint{-0.000000in}{0.000000in}}%
\pgfpathlineto{\pgfqpoint{-0.048611in}{0.000000in}}%
\pgfusepath{stroke,fill}%
}%
\begin{pgfscope}%
\pgfsys@transformshift{0.471687in}{1.806471in}%
\pgfsys@useobject{currentmarker}{}%
\end{pgfscope}%
\end{pgfscope}%
\begin{pgfscope}%
\definecolor{textcolor}{rgb}{0.000000,0.000000,0.000000}%
\pgfsetstrokecolor{textcolor}%
\pgfsetfillcolor{textcolor}%
\pgftext[x=0.315437in, y=1.767915in, left, base]{\color{textcolor}{\rmfamily\fontsize{8.000000}{9.600000}\selectfont\catcode`\^=\active\def^{\ifmmode\sp\else\^{}\fi}\catcode`\%=\active\def%{\%}$\mathdefault{2}$}}%
\end{pgfscope}%
\begin{pgfscope}%
\pgfpathrectangle{\pgfqpoint{0.471687in}{0.416447in}}{\pgfqpoint{3.537845in}{1.924647in}}%
\pgfusepath{clip}%
\pgfsetrectcap%
\pgfsetroundjoin%
\pgfsetlinewidth{0.803000pt}%
\definecolor{currentstroke}{rgb}{0.450000,0.450000,0.450000}%
\pgfsetstrokecolor{currentstroke}%
\pgfsetdash{}{0pt}%
\pgfpathmoveto{\pgfqpoint{0.471687in}{2.234170in}}%
\pgfpathlineto{\pgfqpoint{4.009533in}{2.234170in}}%
\pgfusepath{stroke}%
\end{pgfscope}%
\begin{pgfscope}%
\pgfsetbuttcap%
\pgfsetroundjoin%
\definecolor{currentfill}{rgb}{0.000000,0.000000,0.000000}%
\pgfsetfillcolor{currentfill}%
\pgfsetlinewidth{0.803000pt}%
\definecolor{currentstroke}{rgb}{0.000000,0.000000,0.000000}%
\pgfsetstrokecolor{currentstroke}%
\pgfsetdash{}{0pt}%
\pgfsys@defobject{currentmarker}{\pgfqpoint{-0.048611in}{0.000000in}}{\pgfqpoint{-0.000000in}{0.000000in}}{%
\pgfpathmoveto{\pgfqpoint{-0.000000in}{0.000000in}}%
\pgfpathlineto{\pgfqpoint{-0.048611in}{0.000000in}}%
\pgfusepath{stroke,fill}%
}%
\begin{pgfscope}%
\pgfsys@transformshift{0.471687in}{2.234170in}%
\pgfsys@useobject{currentmarker}{}%
\end{pgfscope}%
\end{pgfscope}%
\begin{pgfscope}%
\definecolor{textcolor}{rgb}{0.000000,0.000000,0.000000}%
\pgfsetstrokecolor{textcolor}%
\pgfsetfillcolor{textcolor}%
\pgftext[x=0.315437in, y=2.195614in, left, base]{\color{textcolor}{\rmfamily\fontsize{8.000000}{9.600000}\selectfont\catcode`\^=\active\def^{\ifmmode\sp\else\^{}\fi}\catcode`\%=\active\def%{\%}$\mathdefault{4}$}}%
\end{pgfscope}%
\begin{pgfscope}%
\definecolor{textcolor}{rgb}{0.000000,0.000000,0.000000}%
\pgfsetstrokecolor{textcolor}%
\pgfsetfillcolor{textcolor}%
\pgftext[x=0.168059in,y=1.378771in,,bottom,rotate=90.000000]{\color{textcolor}{\rmfamily\fontsize{10.000000}{12.000000}\selectfont\catcode`\^=\active\def^{\ifmmode\sp\else\^{}\fi}\catcode`\%=\active\def%{\%}Amplitude in $\unit{\V}$}}%
\end{pgfscope}%
\begin{pgfscope}%
\definecolor{textcolor}{rgb}{0.000000,0.000000,0.000000}%
\pgfsetstrokecolor{textcolor}%
\pgfsetfillcolor{textcolor}%
\pgftext[x=0.471687in,y=2.382761in,left,base]{\color{textcolor}{\rmfamily\fontsize{8.000000}{9.600000}\selectfont\catcode`\^=\active\def^{\ifmmode\sp\else\^{}\fi}\catcode`\%=\active\def%{\%}$\times\mathdefault{10^{\ensuremath{-}6}}\mathdefault{+10^{1}}$}}%
\end{pgfscope}%
\begin{pgfscope}%
\pgfpathrectangle{\pgfqpoint{0.471687in}{0.416447in}}{\pgfqpoint{3.537845in}{1.924647in}}%
\pgfusepath{clip}%
\pgfsetrectcap%
\pgfsetroundjoin%
\pgfsetlinewidth{1.505625pt}%
\definecolor{currentstroke}{rgb}{0.835294,0.368627,0.000000}%
\pgfsetstrokecolor{currentstroke}%
\pgfsetdash{}{0pt}%
\pgfpathmoveto{\pgfqpoint{0.632499in}{1.430680in}}%
\pgfpathlineto{\pgfqpoint{0.634245in}{0.968052in}}%
\pgfpathlineto{\pgfqpoint{0.638500in}{1.985499in}}%
\pgfpathlineto{\pgfqpoint{0.639053in}{1.124216in}}%
\pgfpathlineto{\pgfqpoint{0.642318in}{1.931083in}}%
\pgfpathlineto{\pgfqpoint{0.647416in}{1.089168in}}%
\pgfpathlineto{\pgfqpoint{0.649297in}{1.969079in}}%
\pgfpathlineto{\pgfqpoint{0.654958in}{1.095363in}}%
\pgfpathlineto{\pgfqpoint{0.657016in}{2.190406in}}%
\pgfpathlineto{\pgfqpoint{0.658302in}{1.190770in}}%
\pgfpathlineto{\pgfqpoint{0.662310in}{1.850846in}}%
\pgfpathlineto{\pgfqpoint{0.664828in}{1.005117in}}%
\pgfpathlineto{\pgfqpoint{0.669119in}{1.873075in}}%
\pgfpathlineto{\pgfqpoint{0.672303in}{1.204864in}}%
\pgfpathlineto{\pgfqpoint{0.674798in}{1.827537in}}%
\pgfpathlineto{\pgfqpoint{0.677873in}{1.047833in}}%
\pgfpathlineto{\pgfqpoint{0.682936in}{1.931260in}}%
\pgfpathlineto{\pgfqpoint{0.684261in}{1.147985in}}%
\pgfpathlineto{\pgfqpoint{0.687860in}{1.774538in}}%
\pgfpathlineto{\pgfqpoint{0.692102in}{1.106976in}}%
\pgfpathlineto{\pgfqpoint{0.694369in}{2.009669in}}%
\pgfpathlineto{\pgfqpoint{0.698489in}{1.149501in}}%
\pgfpathlineto{\pgfqpoint{0.701464in}{1.887997in}}%
\pgfpathlineto{\pgfqpoint{0.704073in}{1.099874in}}%
\pgfpathlineto{\pgfqpoint{0.706948in}{1.746473in}}%
\pgfpathlineto{\pgfqpoint{0.711650in}{0.958670in}}%
\pgfpathlineto{\pgfqpoint{0.713962in}{1.900388in}}%
\pgfpathlineto{\pgfqpoint{0.716732in}{1.098145in}}%
\pgfpathlineto{\pgfqpoint{0.719723in}{1.696503in}}%
\pgfpathlineto{\pgfqpoint{0.725075in}{1.018407in}}%
\pgfpathlineto{\pgfqpoint{0.727670in}{1.732802in}}%
\pgfpathlineto{\pgfqpoint{0.729986in}{1.056747in}}%
\pgfpathlineto{\pgfqpoint{0.733179in}{1.867784in}}%
\pgfpathlineto{\pgfqpoint{0.735688in}{0.772095in}}%
\pgfpathlineto{\pgfqpoint{0.740242in}{1.578507in}}%
\pgfpathlineto{\pgfqpoint{0.744822in}{0.827652in}}%
\pgfpathlineto{\pgfqpoint{0.745407in}{1.512238in}}%
\pgfpathlineto{\pgfqpoint{0.748781in}{0.897421in}}%
\pgfpathlineto{\pgfqpoint{0.753043in}{1.830236in}}%
\pgfpathlineto{\pgfqpoint{0.755072in}{0.948349in}}%
\pgfpathlineto{\pgfqpoint{0.758951in}{1.826932in}}%
\pgfpathlineto{\pgfqpoint{0.761730in}{0.975509in}}%
\pgfpathlineto{\pgfqpoint{0.765158in}{1.787970in}}%
\pgfpathlineto{\pgfqpoint{0.769417in}{1.046219in}}%
\pgfpathlineto{\pgfqpoint{0.771292in}{1.655556in}}%
\pgfpathlineto{\pgfqpoint{0.774382in}{1.104537in}}%
\pgfpathlineto{\pgfqpoint{0.777586in}{1.783627in}}%
\pgfpathlineto{\pgfqpoint{0.781310in}{1.070233in}}%
\pgfpathlineto{\pgfqpoint{0.784533in}{1.793805in}}%
\pgfpathlineto{\pgfqpoint{0.788170in}{0.924067in}}%
\pgfpathlineto{\pgfqpoint{0.790583in}{1.671511in}}%
\pgfpathlineto{\pgfqpoint{0.796047in}{1.103346in}}%
\pgfpathlineto{\pgfqpoint{0.798022in}{1.932074in}}%
\pgfpathlineto{\pgfqpoint{0.802351in}{1.004872in}}%
\pgfpathlineto{\pgfqpoint{0.803840in}{1.762802in}}%
\pgfpathlineto{\pgfqpoint{0.808581in}{0.921355in}}%
\pgfpathlineto{\pgfqpoint{0.810169in}{1.694202in}}%
\pgfpathlineto{\pgfqpoint{0.815290in}{1.115274in}}%
\pgfpathlineto{\pgfqpoint{0.818493in}{2.029896in}}%
\pgfpathlineto{\pgfqpoint{0.819783in}{1.109167in}}%
\pgfpathlineto{\pgfqpoint{0.822947in}{1.733064in}}%
\pgfpathlineto{\pgfqpoint{0.825884in}{1.001701in}}%
\pgfpathlineto{\pgfqpoint{0.829171in}{1.623826in}}%
\pgfpathlineto{\pgfqpoint{0.834243in}{0.943107in}}%
\pgfpathlineto{\pgfqpoint{0.836623in}{1.762142in}}%
\pgfpathlineto{\pgfqpoint{0.839106in}{1.131069in}}%
\pgfpathlineto{\pgfqpoint{0.842090in}{1.848988in}}%
\pgfpathlineto{\pgfqpoint{0.845866in}{0.966611in}}%
\pgfpathlineto{\pgfqpoint{0.848484in}{1.856997in}}%
\pgfpathlineto{\pgfqpoint{0.851749in}{1.176952in}}%
\pgfpathlineto{\pgfqpoint{0.854962in}{1.795240in}}%
\pgfpathlineto{\pgfqpoint{0.860716in}{1.011878in}}%
\pgfpathlineto{\pgfqpoint{0.861886in}{1.905554in}}%
\pgfpathlineto{\pgfqpoint{0.867476in}{0.955354in}}%
\pgfpathlineto{\pgfqpoint{0.869062in}{1.870282in}}%
\pgfpathlineto{\pgfqpoint{0.873690in}{1.038392in}}%
\pgfpathlineto{\pgfqpoint{0.874468in}{1.821037in}}%
\pgfpathlineto{\pgfqpoint{0.878006in}{1.058417in}}%
\pgfpathlineto{\pgfqpoint{0.882290in}{1.749600in}}%
\pgfpathlineto{\pgfqpoint{0.883972in}{1.019161in}}%
\pgfpathlineto{\pgfqpoint{0.888160in}{1.697312in}}%
\pgfpathlineto{\pgfqpoint{0.890787in}{1.068473in}}%
\pgfpathlineto{\pgfqpoint{0.894248in}{1.626309in}}%
\pgfpathlineto{\pgfqpoint{0.897828in}{1.044816in}}%
\pgfpathlineto{\pgfqpoint{0.900594in}{1.826492in}}%
\pgfpathlineto{\pgfqpoint{0.903530in}{1.260540in}}%
\pgfpathlineto{\pgfqpoint{0.907560in}{1.937101in}}%
\pgfpathlineto{\pgfqpoint{0.911979in}{1.177181in}}%
\pgfpathlineto{\pgfqpoint{0.914542in}{1.820772in}}%
\pgfpathlineto{\pgfqpoint{0.916202in}{1.042015in}}%
\pgfpathlineto{\pgfqpoint{0.919341in}{1.843630in}}%
\pgfpathlineto{\pgfqpoint{0.922676in}{1.217567in}}%
\pgfpathlineto{\pgfqpoint{0.926134in}{1.825967in}}%
\pgfpathlineto{\pgfqpoint{0.929614in}{0.936009in}}%
\pgfpathlineto{\pgfqpoint{0.932499in}{1.716004in}}%
\pgfpathlineto{\pgfqpoint{0.936699in}{0.845262in}}%
\pgfpathlineto{\pgfqpoint{0.939799in}{1.618204in}}%
\pgfpathlineto{\pgfqpoint{0.942009in}{1.047300in}}%
\pgfpathlineto{\pgfqpoint{0.945119in}{1.649732in}}%
\pgfpathlineto{\pgfqpoint{0.948696in}{0.905935in}}%
\pgfpathlineto{\pgfqpoint{0.952253in}{1.629647in}}%
\pgfpathlineto{\pgfqpoint{0.956382in}{0.785481in}}%
\pgfpathlineto{\pgfqpoint{0.958122in}{1.811913in}}%
\pgfpathlineto{\pgfqpoint{0.961605in}{0.960533in}}%
\pgfpathlineto{\pgfqpoint{0.965442in}{1.734255in}}%
\pgfpathlineto{\pgfqpoint{0.968298in}{1.040423in}}%
\pgfpathlineto{\pgfqpoint{0.970923in}{1.629346in}}%
\pgfpathlineto{\pgfqpoint{0.974384in}{0.996550in}}%
\pgfpathlineto{\pgfqpoint{0.977867in}{1.839476in}}%
\pgfpathlineto{\pgfqpoint{0.982437in}{0.829037in}}%
\pgfpathlineto{\pgfqpoint{0.985270in}{1.769430in}}%
\pgfpathlineto{\pgfqpoint{0.987966in}{0.950728in}}%
\pgfpathlineto{\pgfqpoint{0.993028in}{1.875909in}}%
\pgfpathlineto{\pgfqpoint{0.995302in}{0.993866in}}%
\pgfpathlineto{\pgfqpoint{0.998036in}{1.847801in}}%
\pgfpathlineto{\pgfqpoint{1.001908in}{0.839141in}}%
\pgfpathlineto{\pgfqpoint{1.003172in}{1.619939in}}%
\pgfpathlineto{\pgfqpoint{1.007511in}{0.978198in}}%
\pgfpathlineto{\pgfqpoint{1.011541in}{1.661960in}}%
\pgfpathlineto{\pgfqpoint{1.014313in}{0.764828in}}%
\pgfpathlineto{\pgfqpoint{1.016297in}{1.613906in}}%
\pgfpathlineto{\pgfqpoint{1.021694in}{0.882179in}}%
\pgfpathlineto{\pgfqpoint{1.023862in}{1.777976in}}%
\pgfpathlineto{\pgfqpoint{1.025853in}{1.043519in}}%
\pgfpathlineto{\pgfqpoint{1.029699in}{1.884456in}}%
\pgfpathlineto{\pgfqpoint{1.032231in}{1.128484in}}%
\pgfpathlineto{\pgfqpoint{1.035711in}{1.773435in}}%
\pgfpathlineto{\pgfqpoint{1.039194in}{1.001486in}}%
\pgfpathlineto{\pgfqpoint{1.041902in}{1.691688in}}%
\pgfpathlineto{\pgfqpoint{1.045668in}{0.974242in}}%
\pgfpathlineto{\pgfqpoint{1.048489in}{1.697055in}}%
\pgfpathlineto{\pgfqpoint{1.051882in}{0.990361in}}%
\pgfpathlineto{\pgfqpoint{1.056864in}{1.696834in}}%
\pgfpathlineto{\pgfqpoint{1.058147in}{0.967570in}}%
\pgfpathlineto{\pgfqpoint{1.062962in}{1.686818in}}%
\pgfpathlineto{\pgfqpoint{1.065001in}{0.891553in}}%
\pgfpathlineto{\pgfqpoint{1.068921in}{1.696908in}}%
\pgfpathlineto{\pgfqpoint{1.071526in}{1.045720in}}%
\pgfpathlineto{\pgfqpoint{1.077119in}{1.832273in}}%
\pgfpathlineto{\pgfqpoint{1.077242in}{1.172340in}}%
\pgfpathlineto{\pgfqpoint{1.080940in}{1.692327in}}%
\pgfpathlineto{\pgfqpoint{1.084378in}{0.988176in}}%
\pgfpathlineto{\pgfqpoint{1.087273in}{1.745921in}}%
\pgfpathlineto{\pgfqpoint{1.092734in}{0.790603in}}%
\pgfpathlineto{\pgfqpoint{1.093487in}{1.463187in}}%
\pgfpathlineto{\pgfqpoint{1.097015in}{1.007820in}}%
\pgfpathlineto{\pgfqpoint{1.102048in}{1.889336in}}%
\pgfpathlineto{\pgfqpoint{1.103557in}{0.915372in}}%
\pgfpathlineto{\pgfqpoint{1.106989in}{1.668801in}}%
\pgfpathlineto{\pgfqpoint{1.109918in}{0.993739in}}%
\pgfpathlineto{\pgfqpoint{1.112823in}{1.691799in}}%
\pgfpathlineto{\pgfqpoint{1.116682in}{0.984569in}}%
\pgfpathlineto{\pgfqpoint{1.119159in}{1.807782in}}%
\pgfpathlineto{\pgfqpoint{1.125051in}{0.833871in}}%
\pgfpathlineto{\pgfqpoint{1.125836in}{1.637366in}}%
\pgfpathlineto{\pgfqpoint{1.129840in}{1.056872in}}%
\pgfpathlineto{\pgfqpoint{1.132062in}{1.711763in}}%
\pgfpathlineto{\pgfqpoint{1.137880in}{0.915740in}}%
\pgfpathlineto{\pgfqpoint{1.138479in}{1.652582in}}%
\pgfpathlineto{\pgfqpoint{1.143911in}{1.971625in}}%
\pgfpathlineto{\pgfqpoint{1.145689in}{1.079925in}}%
\pgfpathlineto{\pgfqpoint{1.149073in}{1.755255in}}%
\pgfpathlineto{\pgfqpoint{1.152218in}{0.979252in}}%
\pgfpathlineto{\pgfqpoint{1.154611in}{1.765493in}}%
\pgfpathlineto{\pgfqpoint{1.158426in}{1.184761in}}%
\pgfpathlineto{\pgfqpoint{1.163562in}{1.933343in}}%
\pgfpathlineto{\pgfqpoint{1.166360in}{0.910362in}}%
\pgfpathlineto{\pgfqpoint{1.169075in}{1.860061in}}%
\pgfpathlineto{\pgfqpoint{1.172590in}{0.931551in}}%
\pgfpathlineto{\pgfqpoint{1.174240in}{1.765055in}}%
\pgfpathlineto{\pgfqpoint{1.179219in}{0.977709in}}%
\pgfpathlineto{\pgfqpoint{1.180544in}{1.833828in}}%
\pgfpathlineto{\pgfqpoint{1.183972in}{1.133998in}}%
\pgfpathlineto{\pgfqpoint{1.187192in}{1.712011in}}%
\pgfpathlineto{\pgfqpoint{1.191167in}{2.072419in}}%
\pgfpathlineto{\pgfqpoint{1.193302in}{1.215512in}}%
\pgfpathlineto{\pgfqpoint{1.196573in}{1.822201in}}%
\pgfpathlineto{\pgfqpoint{1.200680in}{0.952849in}}%
\pgfpathlineto{\pgfqpoint{1.203655in}{1.705790in}}%
\pgfpathlineto{\pgfqpoint{1.208200in}{0.862706in}}%
\pgfpathlineto{\pgfqpoint{1.210017in}{1.665380in}}%
\pgfpathlineto{\pgfqpoint{1.212613in}{0.775256in}}%
\pgfpathlineto{\pgfqpoint{1.215948in}{1.607347in}}%
\pgfpathlineto{\pgfqpoint{1.221274in}{0.819193in}}%
\pgfpathlineto{\pgfqpoint{1.223081in}{1.684897in}}%
\pgfpathlineto{\pgfqpoint{1.228491in}{0.866797in}}%
\pgfpathlineto{\pgfqpoint{1.229314in}{1.711989in}}%
\pgfpathlineto{\pgfqpoint{1.232373in}{0.955984in}}%
\pgfpathlineto{\pgfqpoint{1.235194in}{1.578846in}}%
\pgfpathlineto{\pgfqpoint{1.239095in}{0.889421in}}%
\pgfpathlineto{\pgfqpoint{1.242816in}{1.710359in}}%
\pgfpathlineto{\pgfqpoint{1.245701in}{0.956659in}}%
\pgfpathlineto{\pgfqpoint{1.248190in}{1.604753in}}%
\pgfpathlineto{\pgfqpoint{1.251896in}{1.002331in}}%
\pgfpathlineto{\pgfqpoint{1.255485in}{1.860032in}}%
\pgfpathlineto{\pgfqpoint{1.259032in}{0.982308in}}%
\pgfpathlineto{\pgfqpoint{1.262448in}{1.937698in}}%
\pgfpathlineto{\pgfqpoint{1.265835in}{1.023291in}}%
\pgfpathlineto{\pgfqpoint{1.268308in}{1.948098in}}%
\pgfpathlineto{\pgfqpoint{1.271415in}{1.241715in}}%
\pgfpathlineto{\pgfqpoint{1.275027in}{2.074919in}}%
\pgfpathlineto{\pgfqpoint{1.277352in}{1.304992in}}%
\pgfpathlineto{\pgfqpoint{1.280867in}{1.978696in}}%
\pgfpathlineto{\pgfqpoint{1.284695in}{1.280667in}}%
\pgfpathlineto{\pgfqpoint{1.286811in}{1.938241in}}%
\pgfpathlineto{\pgfqpoint{1.290995in}{1.262571in}}%
\pgfpathlineto{\pgfqpoint{1.293742in}{1.885238in}}%
\pgfpathlineto{\pgfqpoint{1.297695in}{1.133605in}}%
\pgfpathlineto{\pgfqpoint{1.300966in}{1.848955in}}%
\pgfpathlineto{\pgfqpoint{1.305211in}{1.086484in}}%
\pgfpathlineto{\pgfqpoint{1.306610in}{1.904408in}}%
\pgfpathlineto{\pgfqpoint{1.310209in}{1.281812in}}%
\pgfpathlineto{\pgfqpoint{1.313210in}{1.862825in}}%
\pgfpathlineto{\pgfqpoint{1.316641in}{1.092195in}}%
\pgfpathlineto{\pgfqpoint{1.320829in}{1.779570in}}%
\pgfpathlineto{\pgfqpoint{1.322421in}{1.002353in}}%
\pgfpathlineto{\pgfqpoint{1.326014in}{1.814976in}}%
\pgfpathlineto{\pgfqpoint{1.330819in}{1.141472in}}%
\pgfpathlineto{\pgfqpoint{1.332748in}{2.003963in}}%
\pgfpathlineto{\pgfqpoint{1.338242in}{1.047933in}}%
\pgfpathlineto{\pgfqpoint{1.339300in}{1.831691in}}%
\pgfpathlineto{\pgfqpoint{1.343667in}{1.099431in}}%
\pgfpathlineto{\pgfqpoint{1.345327in}{1.782226in}}%
\pgfpathlineto{\pgfqpoint{1.348032in}{1.113959in}}%
\pgfpathlineto{\pgfqpoint{1.353294in}{0.911376in}}%
\pgfpathlineto{\pgfqpoint{1.354635in}{1.726347in}}%
\pgfpathlineto{\pgfqpoint{1.357635in}{1.177942in}}%
\pgfpathlineto{\pgfqpoint{1.362238in}{1.827962in}}%
\pgfpathlineto{\pgfqpoint{1.364637in}{1.124606in}}%
\pgfpathlineto{\pgfqpoint{1.367284in}{1.790054in}}%
\pgfpathlineto{\pgfqpoint{1.371369in}{0.894142in}}%
\pgfpathlineto{\pgfqpoint{1.375405in}{1.678959in}}%
\pgfpathlineto{\pgfqpoint{1.376981in}{0.954803in}}%
\pgfpathlineto{\pgfqpoint{1.380210in}{1.748285in}}%
\pgfpathlineto{\pgfqpoint{1.383568in}{0.848123in}}%
\pgfpathlineto{\pgfqpoint{1.386800in}{1.537972in}}%
\pgfpathlineto{\pgfqpoint{1.392438in}{1.749766in}}%
\pgfpathlineto{\pgfqpoint{1.393155in}{0.956062in}}%
\pgfpathlineto{\pgfqpoint{1.398147in}{1.583439in}}%
\pgfpathlineto{\pgfqpoint{1.399672in}{0.784676in}}%
\pgfpathlineto{\pgfqpoint{1.403673in}{1.456269in}}%
\pgfpathlineto{\pgfqpoint{1.405991in}{0.733755in}}%
\pgfpathlineto{\pgfqpoint{1.410279in}{1.722510in}}%
\pgfpathlineto{\pgfqpoint{1.412974in}{0.925227in}}%
\pgfpathlineto{\pgfqpoint{1.416801in}{1.679970in}}%
\pgfpathlineto{\pgfqpoint{1.419438in}{0.900515in}}%
\pgfpathlineto{\pgfqpoint{1.422397in}{1.593938in}}%
\pgfpathlineto{\pgfqpoint{1.425511in}{0.711278in}}%
\pgfpathlineto{\pgfqpoint{1.428868in}{1.527149in}}%
\pgfpathlineto{\pgfqpoint{1.433780in}{0.710430in}}%
\pgfpathlineto{\pgfqpoint{1.436359in}{1.562541in}}%
\pgfpathlineto{\pgfqpoint{1.438234in}{0.681817in}}%
\pgfpathlineto{\pgfqpoint{1.442965in}{1.634305in}}%
\pgfpathlineto{\pgfqpoint{1.444715in}{0.811441in}}%
\pgfpathlineto{\pgfqpoint{1.449198in}{1.622988in}}%
\pgfpathlineto{\pgfqpoint{1.451479in}{0.862779in}}%
\pgfpathlineto{\pgfqpoint{1.454756in}{1.489889in}}%
\pgfpathlineto{\pgfqpoint{1.457786in}{0.928367in}}%
\pgfpathlineto{\pgfqpoint{1.463507in}{1.766827in}}%
\pgfpathlineto{\pgfqpoint{1.464964in}{0.909889in}}%
\pgfpathlineto{\pgfqpoint{1.467637in}{1.652607in}}%
\pgfpathlineto{\pgfqpoint{1.471615in}{0.813558in}}%
\pgfpathlineto{\pgfqpoint{1.476279in}{1.791893in}}%
\pgfpathlineto{\pgfqpoint{1.477093in}{0.948800in}}%
\pgfpathlineto{\pgfqpoint{1.481496in}{1.723994in}}%
\pgfpathlineto{\pgfqpoint{1.483502in}{0.937984in}}%
\pgfpathlineto{\pgfqpoint{1.489407in}{1.852351in}}%
\pgfpathlineto{\pgfqpoint{1.490961in}{0.758506in}}%
\pgfpathlineto{\pgfqpoint{1.493913in}{1.679964in}}%
\pgfpathlineto{\pgfqpoint{1.498426in}{0.630443in}}%
\pgfpathlineto{\pgfqpoint{1.499680in}{1.645622in}}%
\pgfpathlineto{\pgfqpoint{1.504974in}{0.839854in}}%
\pgfpathlineto{\pgfqpoint{1.508348in}{1.688410in}}%
\pgfpathlineto{\pgfqpoint{1.509416in}{0.882838in}}%
\pgfpathlineto{\pgfqpoint{1.513600in}{1.749357in}}%
\pgfpathlineto{\pgfqpoint{1.517025in}{0.657644in}}%
\pgfpathlineto{\pgfqpoint{1.518968in}{1.489886in}}%
\pgfpathlineto{\pgfqpoint{1.524127in}{0.743989in}}%
\pgfpathlineto{\pgfqpoint{1.526259in}{1.512845in}}%
\pgfpathlineto{\pgfqpoint{1.531289in}{0.620111in}}%
\pgfpathlineto{\pgfqpoint{1.532354in}{1.537912in}}%
\pgfpathlineto{\pgfqpoint{1.537390in}{0.644664in}}%
\pgfpathlineto{\pgfqpoint{1.539249in}{1.471676in}}%
\pgfpathlineto{\pgfqpoint{1.541539in}{0.697669in}}%
\pgfpathlineto{\pgfqpoint{1.545051in}{1.589145in}}%
\pgfpathlineto{\pgfqpoint{1.547766in}{0.889231in}}%
\pgfpathlineto{\pgfqpoint{1.551124in}{1.627851in}}%
\pgfpathlineto{\pgfqpoint{1.557035in}{0.698691in}}%
\pgfpathlineto{\pgfqpoint{1.559891in}{1.750099in}}%
\pgfpathlineto{\pgfqpoint{1.561255in}{0.818276in}}%
\pgfpathlineto{\pgfqpoint{1.563966in}{1.660117in}}%
\pgfpathlineto{\pgfqpoint{1.569926in}{0.677395in}}%
\pgfpathlineto{\pgfqpoint{1.570408in}{1.508287in}}%
\pgfpathlineto{\pgfqpoint{1.574026in}{0.869650in}}%
\pgfpathlineto{\pgfqpoint{1.577481in}{1.498505in}}%
\pgfpathlineto{\pgfqpoint{1.581539in}{0.794492in}}%
\pgfpathlineto{\pgfqpoint{1.585019in}{1.636739in}}%
\pgfpathlineto{\pgfqpoint{1.586554in}{0.753191in}}%
\pgfpathlineto{\pgfqpoint{1.590230in}{1.554852in}}%
\pgfpathlineto{\pgfqpoint{1.594218in}{0.858379in}}%
\pgfpathlineto{\pgfqpoint{1.597312in}{1.528983in}}%
\pgfpathlineto{\pgfqpoint{1.601155in}{0.823657in}}%
\pgfpathlineto{\pgfqpoint{1.603246in}{1.601832in}}%
\pgfpathlineto{\pgfqpoint{1.607559in}{0.735530in}}%
\pgfpathlineto{\pgfqpoint{1.609189in}{1.454204in}}%
\pgfpathlineto{\pgfqpoint{1.612583in}{0.732526in}}%
\pgfpathlineto{\pgfqpoint{1.615750in}{1.331831in}}%
\pgfpathlineto{\pgfqpoint{1.619205in}{0.608129in}}%
\pgfpathlineto{\pgfqpoint{1.622148in}{1.439599in}}%
\pgfpathlineto{\pgfqpoint{1.625290in}{0.800061in}}%
\pgfpathlineto{\pgfqpoint{1.629667in}{1.436012in}}%
\pgfpathlineto{\pgfqpoint{1.632182in}{0.707304in}}%
\pgfpathlineto{\pgfqpoint{1.635636in}{1.316883in}}%
\pgfpathlineto{\pgfqpoint{1.638782in}{0.649311in}}%
\pgfpathlineto{\pgfqpoint{1.642702in}{1.441877in}}%
\pgfpathlineto{\pgfqpoint{1.644864in}{0.713940in}}%
\pgfpathlineto{\pgfqpoint{1.648588in}{1.441023in}}%
\pgfpathlineto{\pgfqpoint{1.651267in}{0.789182in}}%
\pgfpathlineto{\pgfqpoint{1.654522in}{1.553938in}}%
\pgfpathlineto{\pgfqpoint{1.658340in}{0.733858in}}%
\pgfpathlineto{\pgfqpoint{1.662003in}{1.451426in}}%
\pgfpathlineto{\pgfqpoint{1.664344in}{0.675083in}}%
\pgfpathlineto{\pgfqpoint{1.667168in}{1.416310in}}%
\pgfpathlineto{\pgfqpoint{1.670992in}{0.723429in}}%
\pgfpathlineto{\pgfqpoint{1.674845in}{1.418463in}}%
\pgfpathlineto{\pgfqpoint{1.678753in}{0.882980in}}%
\pgfpathlineto{\pgfqpoint{1.681448in}{1.690097in}}%
\pgfpathlineto{\pgfqpoint{1.683198in}{0.972405in}}%
\pgfpathlineto{\pgfqpoint{1.689251in}{1.683989in}}%
\pgfpathlineto{\pgfqpoint{1.690293in}{0.927260in}}%
\pgfpathlineto{\pgfqpoint{1.693532in}{1.598448in}}%
\pgfpathlineto{\pgfqpoint{1.696812in}{0.814951in}}%
\pgfpathlineto{\pgfqpoint{1.699955in}{1.506727in}}%
\pgfpathlineto{\pgfqpoint{1.704579in}{0.704741in}}%
\pgfpathlineto{\pgfqpoint{1.705789in}{1.498923in}}%
\pgfpathlineto{\pgfqpoint{1.710687in}{0.895190in}}%
\pgfpathlineto{\pgfqpoint{1.712167in}{1.789597in}}%
\pgfpathlineto{\pgfqpoint{1.716203in}{0.789263in}}%
\pgfpathlineto{\pgfqpoint{1.718831in}{1.481980in}}%
\pgfpathlineto{\pgfqpoint{1.721912in}{0.729672in}}%
\pgfpathlineto{\pgfqpoint{1.725031in}{1.651436in}}%
\pgfpathlineto{\pgfqpoint{1.728518in}{0.775804in}}%
\pgfpathlineto{\pgfqpoint{1.732056in}{1.549220in}}%
\pgfpathlineto{\pgfqpoint{1.735317in}{0.777410in}}%
\pgfpathlineto{\pgfqpoint{1.738054in}{1.526166in}}%
\pgfpathlineto{\pgfqpoint{1.743312in}{0.570350in}}%
\pgfpathlineto{\pgfqpoint{1.744715in}{1.400569in}}%
\pgfpathlineto{\pgfqpoint{1.747889in}{0.774991in}}%
\pgfpathlineto{\pgfqpoint{1.750835in}{1.563000in}}%
\pgfpathlineto{\pgfqpoint{1.757194in}{0.915183in}}%
\pgfpathlineto{\pgfqpoint{1.758085in}{1.607195in}}%
\pgfpathlineto{\pgfqpoint{1.760606in}{0.880013in}}%
\pgfpathlineto{\pgfqpoint{1.763903in}{1.574950in}}%
\pgfpathlineto{\pgfqpoint{1.768579in}{0.820946in}}%
\pgfpathlineto{\pgfqpoint{1.770290in}{1.484504in}}%
\pgfpathlineto{\pgfqpoint{1.774018in}{0.695659in}}%
\pgfpathlineto{\pgfqpoint{1.776719in}{1.615819in}}%
\pgfpathlineto{\pgfqpoint{1.779852in}{0.829560in}}%
\pgfpathlineto{\pgfqpoint{1.783046in}{1.547109in}}%
\pgfpathlineto{\pgfqpoint{1.787442in}{0.889841in}}%
\pgfpathlineto{\pgfqpoint{1.790697in}{1.684909in}}%
\pgfpathlineto{\pgfqpoint{1.792675in}{0.891776in}}%
\pgfpathlineto{\pgfqpoint{1.796194in}{1.512492in}}%
\pgfpathlineto{\pgfqpoint{1.801722in}{0.708732in}}%
\pgfpathlineto{\pgfqpoint{1.802700in}{1.667812in}}%
\pgfpathlineto{\pgfqpoint{1.807888in}{0.907500in}}%
\pgfpathlineto{\pgfqpoint{1.808962in}{1.578757in}}%
\pgfpathlineto{\pgfqpoint{1.812783in}{0.930459in}}%
\pgfpathlineto{\pgfqpoint{1.817128in}{1.612791in}}%
\pgfpathlineto{\pgfqpoint{1.819215in}{0.775045in}}%
\pgfpathlineto{\pgfqpoint{1.821705in}{1.566609in}}%
\pgfpathlineto{\pgfqpoint{1.826455in}{0.693014in}}%
\pgfpathlineto{\pgfqpoint{1.828517in}{1.435546in}}%
\pgfpathlineto{\pgfqpoint{1.833212in}{0.672636in}}%
\pgfpathlineto{\pgfqpoint{1.835580in}{1.619446in}}%
\pgfpathlineto{\pgfqpoint{1.838580in}{0.800532in}}%
\pgfpathlineto{\pgfqpoint{1.843003in}{1.648438in}}%
\pgfpathlineto{\pgfqpoint{1.844633in}{0.923943in}}%
\pgfpathlineto{\pgfqpoint{1.847866in}{1.740729in}}%
\pgfpathlineto{\pgfqpoint{1.850792in}{0.848203in}}%
\pgfpathlineto{\pgfqpoint{1.854324in}{1.633617in}}%
\pgfpathlineto{\pgfqpoint{1.857968in}{0.927248in}}%
\pgfpathlineto{\pgfqpoint{1.861367in}{1.727135in}}%
\pgfpathlineto{\pgfqpoint{1.864133in}{0.875163in}}%
\pgfpathlineto{\pgfqpoint{1.868819in}{1.702824in}}%
\pgfpathlineto{\pgfqpoint{1.870064in}{0.941125in}}%
\pgfpathlineto{\pgfqpoint{1.873718in}{1.733926in}}%
\pgfpathlineto{\pgfqpoint{1.877056in}{0.989882in}}%
\pgfpathlineto{\pgfqpoint{1.881543in}{1.702397in}}%
\pgfpathlineto{\pgfqpoint{1.882980in}{0.982549in}}%
\pgfpathlineto{\pgfqpoint{1.889191in}{1.848679in}}%
\pgfpathlineto{\pgfqpoint{1.890249in}{0.888276in}}%
\pgfpathlineto{\pgfqpoint{1.893333in}{1.707294in}}%
\pgfpathlineto{\pgfqpoint{1.897524in}{0.866552in}}%
\pgfpathlineto{\pgfqpoint{1.899078in}{1.640814in}}%
\pgfpathlineto{\pgfqpoint{1.902831in}{0.922727in}}%
\pgfpathlineto{\pgfqpoint{1.905529in}{1.542260in}}%
\pgfpathlineto{\pgfqpoint{1.908771in}{0.913409in}}%
\pgfpathlineto{\pgfqpoint{1.912081in}{1.750899in}}%
\pgfpathlineto{\pgfqpoint{1.916950in}{0.955370in}}%
\pgfpathlineto{\pgfqpoint{1.920835in}{1.752519in}}%
\pgfpathlineto{\pgfqpoint{1.923029in}{1.010378in}}%
\pgfpathlineto{\pgfqpoint{1.924936in}{1.822814in}}%
\pgfpathlineto{\pgfqpoint{1.928718in}{1.123528in}}%
\pgfpathlineto{\pgfqpoint{1.931243in}{1.761543in}}%
\pgfpathlineto{\pgfqpoint{1.936797in}{0.959837in}}%
\pgfpathlineto{\pgfqpoint{1.937762in}{1.674645in}}%
\pgfpathlineto{\pgfqpoint{1.943249in}{0.928949in}}%
\pgfpathlineto{\pgfqpoint{1.945066in}{1.708933in}}%
\pgfpathlineto{\pgfqpoint{1.947848in}{1.008886in}}%
\pgfpathlineto{\pgfqpoint{1.951634in}{1.659136in}}%
\pgfpathlineto{\pgfqpoint{1.954043in}{1.005793in}}%
\pgfpathlineto{\pgfqpoint{1.957118in}{1.670768in}}%
\pgfpathlineto{\pgfqpoint{1.960398in}{1.022496in}}%
\pgfpathlineto{\pgfqpoint{1.965515in}{1.714698in}}%
\pgfpathlineto{\pgfqpoint{1.967773in}{0.890662in}}%
\pgfpathlineto{\pgfqpoint{1.971243in}{1.805292in}}%
\pgfpathlineto{\pgfqpoint{1.973298in}{1.094597in}}%
\pgfpathlineto{\pgfqpoint{1.976418in}{1.772810in}}%
\pgfpathlineto{\pgfqpoint{1.979602in}{0.988959in}}%
\pgfpathlineto{\pgfqpoint{1.982937in}{1.686176in}}%
\pgfpathlineto{\pgfqpoint{1.986382in}{1.082700in}}%
\pgfpathlineto{\pgfqpoint{1.992445in}{1.789967in}}%
\pgfpathlineto{\pgfqpoint{1.994442in}{0.831298in}}%
\pgfpathlineto{\pgfqpoint{1.996565in}{1.697814in}}%
\pgfpathlineto{\pgfqpoint{2.000678in}{0.829397in}}%
\pgfpathlineto{\pgfqpoint{2.002228in}{1.658456in}}%
\pgfpathlineto{\pgfqpoint{2.005473in}{1.041266in}}%
\pgfpathlineto{\pgfqpoint{2.010031in}{1.809440in}}%
\pgfpathlineto{\pgfqpoint{2.012096in}{1.065412in}}%
\pgfpathlineto{\pgfqpoint{2.015418in}{1.739102in}}%
\pgfpathlineto{\pgfqpoint{2.018322in}{0.919453in}}%
\pgfpathlineto{\pgfqpoint{2.021834in}{1.651429in}}%
\pgfpathlineto{\pgfqpoint{2.026228in}{0.954601in}}%
\pgfpathlineto{\pgfqpoint{2.027939in}{1.606518in}}%
\pgfpathlineto{\pgfqpoint{2.032159in}{0.871254in}}%
\pgfpathlineto{\pgfqpoint{2.035941in}{1.558572in}}%
\pgfpathlineto{\pgfqpoint{2.040025in}{0.764328in}}%
\pgfpathlineto{\pgfqpoint{2.041019in}{1.856127in}}%
\pgfpathlineto{\pgfqpoint{2.044641in}{0.800301in}}%
\pgfpathlineto{\pgfqpoint{2.047535in}{1.592914in}}%
\pgfpathlineto{\pgfqpoint{2.051851in}{0.921010in}}%
\pgfpathlineto{\pgfqpoint{2.054492in}{1.868518in}}%
\pgfpathlineto{\pgfqpoint{2.058268in}{0.912542in}}%
\pgfpathlineto{\pgfqpoint{2.061706in}{1.839221in}}%
\pgfpathlineto{\pgfqpoint{2.065128in}{0.777430in}}%
\pgfpathlineto{\pgfqpoint{2.068019in}{1.637986in}}%
\pgfpathlineto{\pgfqpoint{2.071271in}{0.993362in}}%
\pgfpathlineto{\pgfqpoint{2.076218in}{1.662625in}}%
\pgfpathlineto{\pgfqpoint{2.076459in}{0.922983in}}%
\pgfpathlineto{\pgfqpoint{2.082036in}{1.552626in}}%
\pgfpathlineto{\pgfqpoint{2.083171in}{0.838074in}}%
\pgfpathlineto{\pgfqpoint{2.087317in}{1.664766in}}%
\pgfpathlineto{\pgfqpoint{2.091019in}{0.703718in}}%
\pgfpathlineto{\pgfqpoint{2.092723in}{1.448907in}}%
\pgfpathlineto{\pgfqpoint{2.096541in}{0.737818in}}%
\pgfpathlineto{\pgfqpoint{2.098979in}{1.581389in}}%
\pgfpathlineto{\pgfqpoint{2.102156in}{0.902918in}}%
\pgfpathlineto{\pgfqpoint{2.105910in}{1.535286in}}%
\pgfpathlineto{\pgfqpoint{2.110113in}{0.632613in}}%
\pgfpathlineto{\pgfqpoint{2.111889in}{1.472382in}}%
\pgfpathlineto{\pgfqpoint{2.115423in}{0.863679in}}%
\pgfpathlineto{\pgfqpoint{2.118418in}{1.585341in}}%
\pgfpathlineto{\pgfqpoint{2.123033in}{0.909726in}}%
\pgfpathlineto{\pgfqpoint{2.125056in}{1.550217in}}%
\pgfpathlineto{\pgfqpoint{2.129256in}{0.700195in}}%
\pgfpathlineto{\pgfqpoint{2.131205in}{1.491085in}}%
\pgfpathlineto{\pgfqpoint{2.135335in}{0.907127in}}%
\pgfpathlineto{\pgfqpoint{2.138693in}{1.801791in}}%
\pgfpathlineto{\pgfqpoint{2.141266in}{0.984061in}}%
\pgfpathlineto{\pgfqpoint{2.145180in}{1.608588in}}%
\pgfpathlineto{\pgfqpoint{2.149091in}{0.801331in}}%
\pgfpathlineto{\pgfqpoint{2.151870in}{1.743745in}}%
\pgfpathlineto{\pgfqpoint{2.154603in}{0.882261in}}%
\pgfpathlineto{\pgfqpoint{2.157334in}{1.668553in}}%
\pgfpathlineto{\pgfqpoint{2.160065in}{1.007819in}}%
\pgfpathlineto{\pgfqpoint{2.164593in}{1.700215in}}%
\pgfpathlineto{\pgfqpoint{2.167533in}{1.006985in}}%
\pgfpathlineto{\pgfqpoint{2.171212in}{1.771390in}}%
\pgfpathlineto{\pgfqpoint{2.173644in}{0.999435in}}%
\pgfpathlineto{\pgfqpoint{2.176506in}{1.818868in}}%
\pgfpathlineto{\pgfqpoint{2.180735in}{0.781030in}}%
\pgfpathlineto{\pgfqpoint{2.183077in}{1.621864in}}%
\pgfpathlineto{\pgfqpoint{2.185962in}{0.919920in}}%
\pgfpathlineto{\pgfqpoint{2.189326in}{1.654028in}}%
\pgfpathlineto{\pgfqpoint{2.194912in}{0.862084in}}%
\pgfpathlineto{\pgfqpoint{2.196160in}{1.580746in}}%
\pgfpathlineto{\pgfqpoint{2.198839in}{0.863942in}}%
\pgfpathlineto{\pgfqpoint{2.203599in}{1.814994in}}%
\pgfpathlineto{\pgfqpoint{2.205922in}{0.912235in}}%
\pgfpathlineto{\pgfqpoint{2.209299in}{1.691521in}}%
\pgfpathlineto{\pgfqpoint{2.212399in}{1.064011in}}%
\pgfpathlineto{\pgfqpoint{2.216230in}{1.853460in}}%
\pgfpathlineto{\pgfqpoint{2.218101in}{1.070464in}}%
\pgfpathlineto{\pgfqpoint{2.222466in}{1.951681in}}%
\pgfpathlineto{\pgfqpoint{2.224572in}{1.125465in}}%
\pgfpathlineto{\pgfqpoint{2.228458in}{1.779962in}}%
\pgfpathlineto{\pgfqpoint{2.231867in}{0.949718in}}%
\pgfpathlineto{\pgfqpoint{2.234273in}{1.742334in}}%
\pgfpathlineto{\pgfqpoint{2.238422in}{1.135801in}}%
\pgfpathlineto{\pgfqpoint{2.241570in}{1.935994in}}%
\pgfpathlineto{\pgfqpoint{2.245623in}{0.899194in}}%
\pgfpathlineto{\pgfqpoint{2.247453in}{1.771575in}}%
\pgfpathlineto{\pgfqpoint{2.251598in}{0.917478in}}%
\pgfpathlineto{\pgfqpoint{2.255323in}{1.825886in}}%
\pgfpathlineto{\pgfqpoint{2.256741in}{0.953512in}}%
\pgfpathlineto{\pgfqpoint{2.259967in}{1.537949in}}%
\pgfpathlineto{\pgfqpoint{2.265505in}{0.826583in}}%
\pgfpathlineto{\pgfqpoint{2.267573in}{1.668070in}}%
\pgfpathlineto{\pgfqpoint{2.270632in}{0.848511in}}%
\pgfpathlineto{\pgfqpoint{2.273665in}{1.611427in}}%
\pgfpathlineto{\pgfqpoint{2.277232in}{0.993389in}}%
\pgfpathlineto{\pgfqpoint{2.279354in}{1.798695in}}%
\pgfpathlineto{\pgfqpoint{2.284134in}{0.897368in}}%
\pgfpathlineto{\pgfqpoint{2.286536in}{1.901702in}}%
\pgfpathlineto{\pgfqpoint{2.290682in}{0.781422in}}%
\pgfpathlineto{\pgfqpoint{2.292551in}{1.601235in}}%
\pgfpathlineto{\pgfqpoint{2.297748in}{1.157007in}}%
\pgfpathlineto{\pgfqpoint{2.299337in}{2.021623in}}%
\pgfpathlineto{\pgfqpoint{2.302093in}{1.064222in}}%
\pgfpathlineto{\pgfqpoint{2.307554in}{1.763694in}}%
\pgfpathlineto{\pgfqpoint{2.309063in}{0.794197in}}%
\pgfpathlineto{\pgfqpoint{2.312025in}{1.745309in}}%
\pgfpathlineto{\pgfqpoint{2.314974in}{1.175289in}}%
\pgfpathlineto{\pgfqpoint{2.319419in}{1.853572in}}%
\pgfpathlineto{\pgfqpoint{2.321448in}{1.012163in}}%
\pgfpathlineto{\pgfqpoint{2.325703in}{1.647467in}}%
\pgfpathlineto{\pgfqpoint{2.327678in}{0.919689in}}%
\pgfpathlineto{\pgfqpoint{2.333104in}{1.630772in}}%
\pgfpathlineto{\pgfqpoint{2.334574in}{0.862044in}}%
\pgfpathlineto{\pgfqpoint{2.339794in}{1.826373in}}%
\pgfpathlineto{\pgfqpoint{2.341013in}{0.957295in}}%
\pgfpathlineto{\pgfqpoint{2.343972in}{1.753878in}}%
\pgfpathlineto{\pgfqpoint{2.348642in}{0.750344in}}%
\pgfpathlineto{\pgfqpoint{2.351034in}{1.718004in}}%
\pgfpathlineto{\pgfqpoint{2.353730in}{0.957911in}}%
\pgfpathlineto{\pgfqpoint{2.356734in}{1.632953in}}%
\pgfpathlineto{\pgfqpoint{2.359863in}{0.872386in}}%
\pgfpathlineto{\pgfqpoint{2.365456in}{1.809028in}}%
\pgfpathlineto{\pgfqpoint{2.366334in}{0.940722in}}%
\pgfpathlineto{\pgfqpoint{2.370168in}{1.700282in}}%
\pgfpathlineto{\pgfqpoint{2.372876in}{1.068076in}}%
\pgfpathlineto{\pgfqpoint{2.376578in}{1.701532in}}%
\pgfpathlineto{\pgfqpoint{2.379193in}{0.918621in}}%
\pgfpathlineto{\pgfqpoint{2.383579in}{1.737581in}}%
\pgfpathlineto{\pgfqpoint{2.386866in}{0.970222in}}%
\pgfpathlineto{\pgfqpoint{2.389472in}{1.893853in}}%
\pgfpathlineto{\pgfqpoint{2.392414in}{1.085803in}}%
\pgfpathlineto{\pgfqpoint{2.397824in}{1.805651in}}%
\pgfpathlineto{\pgfqpoint{2.398625in}{1.022182in}}%
\pgfpathlineto{\pgfqpoint{2.402263in}{1.841278in}}%
\pgfpathlineto{\pgfqpoint{2.406151in}{0.831320in}}%
\pgfpathlineto{\pgfqpoint{2.409084in}{1.746455in}}%
\pgfpathlineto{\pgfqpoint{2.411426in}{1.065093in}}%
\pgfpathlineto{\pgfqpoint{2.415105in}{1.814397in}}%
\pgfpathlineto{\pgfqpoint{2.418276in}{1.066317in}}%
\pgfpathlineto{\pgfqpoint{2.421528in}{1.934482in}}%
\pgfpathlineto{\pgfqpoint{2.424512in}{1.102070in}}%
\pgfpathlineto{\pgfqpoint{2.427619in}{1.700468in}}%
\pgfpathlineto{\pgfqpoint{2.433010in}{0.881610in}}%
\pgfpathlineto{\pgfqpoint{2.434824in}{1.806410in}}%
\pgfpathlineto{\pgfqpoint{2.438583in}{0.857670in}}%
\pgfpathlineto{\pgfqpoint{2.441706in}{1.758700in}}%
\pgfpathlineto{\pgfqpoint{2.444582in}{1.039149in}}%
\pgfpathlineto{\pgfqpoint{2.447145in}{1.776283in}}%
\pgfpathlineto{\pgfqpoint{2.451828in}{0.767886in}}%
\pgfpathlineto{\pgfqpoint{2.453423in}{1.613557in}}%
\pgfpathlineto{\pgfqpoint{2.457919in}{1.090369in}}%
\pgfpathlineto{\pgfqpoint{2.460769in}{1.912316in}}%
\pgfpathlineto{\pgfqpoint{2.463474in}{1.054680in}}%
\pgfpathlineto{\pgfqpoint{2.468234in}{1.948731in}}%
\pgfpathlineto{\pgfqpoint{2.469565in}{0.957542in}}%
\pgfpathlineto{\pgfqpoint{2.475139in}{1.901278in}}%
\pgfpathlineto{\pgfqpoint{2.477127in}{1.208308in}}%
\pgfpathlineto{\pgfqpoint{2.480005in}{1.842955in}}%
\pgfpathlineto{\pgfqpoint{2.483266in}{1.068096in}}%
\pgfpathlineto{\pgfqpoint{2.485929in}{1.712995in}}%
\pgfpathlineto{\pgfqpoint{2.489220in}{0.994582in}}%
\pgfpathlineto{\pgfqpoint{2.492803in}{1.849703in}}%
\pgfpathlineto{\pgfqpoint{2.495257in}{0.796309in}}%
\pgfpathlineto{\pgfqpoint{2.498884in}{1.781293in}}%
\pgfpathlineto{\pgfqpoint{2.503127in}{1.069759in}}%
\pgfpathlineto{\pgfqpoint{2.504934in}{1.749766in}}%
\pgfpathlineto{\pgfqpoint{2.509324in}{0.920635in}}%
\pgfpathlineto{\pgfqpoint{2.511724in}{1.532208in}}%
\pgfpathlineto{\pgfqpoint{2.516815in}{0.862008in}}%
\pgfpathlineto{\pgfqpoint{2.519266in}{1.718428in}}%
\pgfpathlineto{\pgfqpoint{2.521765in}{1.001900in}}%
\pgfpathlineto{\pgfqpoint{2.525663in}{1.925198in}}%
\pgfpathlineto{\pgfqpoint{2.527959in}{1.060882in}}%
\pgfpathlineto{\pgfqpoint{2.530879in}{1.859479in}}%
\pgfpathlineto{\pgfqpoint{2.533973in}{1.183806in}}%
\pgfpathlineto{\pgfqpoint{2.537653in}{1.942556in}}%
\pgfpathlineto{\pgfqpoint{2.540763in}{0.993621in}}%
\pgfpathlineto{\pgfqpoint{2.544648in}{1.650248in}}%
\pgfpathlineto{\pgfqpoint{2.548122in}{0.892002in}}%
\pgfpathlineto{\pgfqpoint{2.551232in}{1.770917in}}%
\pgfpathlineto{\pgfqpoint{2.553956in}{0.979987in}}%
\pgfpathlineto{\pgfqpoint{2.556809in}{1.760880in}}%
\pgfpathlineto{\pgfqpoint{2.559774in}{1.034205in}}%
\pgfpathlineto{\pgfqpoint{2.562932in}{1.655442in}}%
\pgfpathlineto{\pgfqpoint{2.566178in}{1.090134in}}%
\pgfpathlineto{\pgfqpoint{2.569397in}{1.889557in}}%
\pgfpathlineto{\pgfqpoint{2.572652in}{0.933266in}}%
\pgfpathlineto{\pgfqpoint{2.576431in}{1.899535in}}%
\pgfpathlineto{\pgfqpoint{2.579361in}{0.966440in}}%
\pgfpathlineto{\pgfqpoint{2.583445in}{1.924846in}}%
\pgfpathlineto{\pgfqpoint{2.585468in}{0.965682in}}%
\pgfpathlineto{\pgfqpoint{2.588730in}{1.588247in}}%
\pgfpathlineto{\pgfqpoint{2.592145in}{0.982159in}}%
\pgfpathlineto{\pgfqpoint{2.595201in}{1.822984in}}%
\pgfpathlineto{\pgfqpoint{2.598353in}{1.059243in}}%
\pgfpathlineto{\pgfqpoint{2.602447in}{1.820261in}}%
\pgfpathlineto{\pgfqpoint{2.605062in}{1.063182in}}%
\pgfpathlineto{\pgfqpoint{2.611073in}{1.826118in}}%
\pgfpathlineto{\pgfqpoint{2.611333in}{1.081784in}}%
\pgfpathlineto{\pgfqpoint{2.614910in}{1.615588in}}%
\pgfpathlineto{\pgfqpoint{2.617917in}{0.840549in}}%
\pgfpathlineto{\pgfqpoint{2.622609in}{1.694709in}}%
\pgfpathlineto{\pgfqpoint{2.624134in}{0.907606in}}%
\pgfpathlineto{\pgfqpoint{2.627881in}{1.762854in}}%
\pgfpathlineto{\pgfqpoint{2.631068in}{0.920814in}}%
\pgfpathlineto{\pgfqpoint{2.636073in}{1.772215in}}%
\pgfpathlineto{\pgfqpoint{2.637253in}{0.934094in}}%
\pgfpathlineto{\pgfqpoint{2.641592in}{1.693229in}}%
\pgfpathlineto{\pgfqpoint{2.643595in}{0.897469in}}%
\pgfpathlineto{\pgfqpoint{2.648191in}{1.647367in}}%
\pgfpathlineto{\pgfqpoint{2.650291in}{0.921891in}}%
\pgfpathlineto{\pgfqpoint{2.653694in}{1.714138in}}%
\pgfpathlineto{\pgfqpoint{2.656914in}{0.880736in}}%
\pgfpathlineto{\pgfqpoint{2.662536in}{1.768990in}}%
\pgfpathlineto{\pgfqpoint{2.664041in}{1.023964in}}%
\pgfpathlineto{\pgfqpoint{2.666691in}{1.625501in}}%
\pgfpathlineto{\pgfqpoint{2.669660in}{0.950246in}}%
\pgfpathlineto{\pgfqpoint{2.673381in}{1.866510in}}%
\pgfpathlineto{\pgfqpoint{2.676160in}{1.028218in}}%
\pgfpathlineto{\pgfqpoint{2.679212in}{1.678036in}}%
\pgfpathlineto{\pgfqpoint{2.685233in}{0.905068in}}%
\pgfpathlineto{\pgfqpoint{2.687223in}{1.886122in}}%
\pgfpathlineto{\pgfqpoint{2.689086in}{1.075029in}}%
\pgfpathlineto{\pgfqpoint{2.694663in}{1.684481in}}%
\pgfpathlineto{\pgfqpoint{2.695370in}{0.818686in}}%
\pgfpathlineto{\pgfqpoint{2.698583in}{1.606714in}}%
\pgfpathlineto{\pgfqpoint{2.702330in}{0.780271in}}%
\pgfpathlineto{\pgfqpoint{2.705167in}{1.578882in}}%
\pgfpathlineto{\pgfqpoint{2.708608in}{0.950186in}}%
\pgfpathlineto{\pgfqpoint{2.712223in}{1.806233in}}%
\pgfpathlineto{\pgfqpoint{2.714703in}{1.017754in}}%
\pgfpathlineto{\pgfqpoint{2.718427in}{1.663695in}}%
\pgfpathlineto{\pgfqpoint{2.720884in}{0.995834in}}%
\pgfpathlineto{\pgfqpoint{2.724068in}{1.550305in}}%
\pgfpathlineto{\pgfqpoint{2.729655in}{0.866057in}}%
\pgfpathlineto{\pgfqpoint{2.731418in}{1.828398in}}%
\pgfpathlineto{\pgfqpoint{2.734016in}{1.178173in}}%
\pgfpathlineto{\pgfqpoint{2.738747in}{1.790507in}}%
\pgfpathlineto{\pgfqpoint{2.740294in}{1.187983in}}%
\pgfpathlineto{\pgfqpoint{2.746299in}{1.977396in}}%
\pgfpathlineto{\pgfqpoint{2.746669in}{1.263331in}}%
\pgfpathlineto{\pgfqpoint{2.750937in}{1.946632in}}%
\pgfpathlineto{\pgfqpoint{2.753661in}{0.918863in}}%
\pgfpathlineto{\pgfqpoint{2.756585in}{2.006791in}}%
\pgfpathlineto{\pgfqpoint{2.760113in}{1.057601in}}%
\pgfpathlineto{\pgfqpoint{2.762917in}{1.841328in}}%
\pgfpathlineto{\pgfqpoint{2.767912in}{0.847416in}}%
\pgfpathlineto{\pgfqpoint{2.769414in}{1.826609in}}%
\pgfpathlineto{\pgfqpoint{2.773631in}{1.011507in}}%
\pgfpathlineto{\pgfqpoint{2.775994in}{1.866642in}}%
\pgfpathlineto{\pgfqpoint{2.779204in}{1.129066in}}%
\pgfpathlineto{\pgfqpoint{2.782044in}{1.780421in}}%
\pgfpathlineto{\pgfqpoint{2.786164in}{1.057129in}}%
\pgfpathlineto{\pgfqpoint{2.788998in}{1.876976in}}%
\pgfpathlineto{\pgfqpoint{2.792011in}{1.058078in}}%
\pgfpathlineto{\pgfqpoint{2.795868in}{1.869087in}}%
\pgfpathlineto{\pgfqpoint{2.798733in}{0.991840in}}%
\pgfpathlineto{\pgfqpoint{2.801937in}{1.703061in}}%
\pgfpathlineto{\pgfqpoint{2.805802in}{0.889525in}}%
\pgfpathlineto{\pgfqpoint{2.808083in}{1.749343in}}%
\pgfpathlineto{\pgfqpoint{2.811746in}{0.954567in}}%
\pgfpathlineto{\pgfqpoint{2.815326in}{1.855127in}}%
\pgfpathlineto{\pgfqpoint{2.817683in}{0.989288in}}%
\pgfpathlineto{\pgfqpoint{2.821359in}{1.582902in}}%
\pgfpathlineto{\pgfqpoint{2.823948in}{0.919988in}}%
\pgfpathlineto{\pgfqpoint{2.827329in}{1.735701in}}%
\pgfpathlineto{\pgfqpoint{2.830487in}{1.070805in}}%
\pgfpathlineto{\pgfqpoint{2.834932in}{1.794928in}}%
\pgfpathlineto{\pgfqpoint{2.838910in}{0.991202in}}%
\pgfpathlineto{\pgfqpoint{2.840351in}{1.814960in}}%
\pgfpathlineto{\pgfqpoint{2.843484in}{1.090184in}}%
\pgfpathlineto{\pgfqpoint{2.846494in}{1.644577in}}%
\pgfpathlineto{\pgfqpoint{2.850116in}{0.810149in}}%
\pgfpathlineto{\pgfqpoint{2.855397in}{1.595348in}}%
\pgfpathlineto{\pgfqpoint{2.857410in}{0.814524in}}%
\pgfpathlineto{\pgfqpoint{2.860893in}{1.789140in}}%
\pgfpathlineto{\pgfqpoint{2.864328in}{0.849448in}}%
\pgfpathlineto{\pgfqpoint{2.867197in}{1.782919in}}%
\pgfpathlineto{\pgfqpoint{2.869175in}{1.024388in}}%
\pgfpathlineto{\pgfqpoint{2.873793in}{1.802783in}}%
\pgfpathlineto{\pgfqpoint{2.877302in}{0.844922in}}%
\pgfpathlineto{\pgfqpoint{2.878930in}{1.755209in}}%
\pgfpathlineto{\pgfqpoint{2.882107in}{1.014119in}}%
\pgfpathlineto{\pgfqpoint{2.885224in}{1.943856in}}%
\pgfpathlineto{\pgfqpoint{2.890386in}{0.961232in}}%
\pgfpathlineto{\pgfqpoint{2.892181in}{1.734930in}}%
\pgfpathlineto{\pgfqpoint{2.895236in}{0.868594in}}%
\pgfpathlineto{\pgfqpoint{2.898118in}{1.665558in}}%
\pgfpathlineto{\pgfqpoint{2.901472in}{0.929498in}}%
\pgfpathlineto{\pgfqpoint{2.906049in}{1.816844in}}%
\pgfpathlineto{\pgfqpoint{2.909571in}{0.910442in}}%
\pgfpathlineto{\pgfqpoint{2.911353in}{1.707759in}}%
\pgfpathlineto{\pgfqpoint{2.915482in}{0.974453in}}%
\pgfpathlineto{\pgfqpoint{2.918862in}{1.919745in}}%
\pgfpathlineto{\pgfqpoint{2.921410in}{1.023127in}}%
\pgfpathlineto{\pgfqpoint{2.924060in}{1.624976in}}%
\pgfpathlineto{\pgfqpoint{2.927048in}{0.994623in}}%
\pgfpathlineto{\pgfqpoint{2.931686in}{1.747253in}}%
\pgfpathlineto{\pgfqpoint{2.933628in}{0.984770in}}%
\pgfpathlineto{\pgfqpoint{2.937082in}{1.857535in}}%
\pgfpathlineto{\pgfqpoint{2.940270in}{0.933868in}}%
\pgfpathlineto{\pgfqpoint{2.943196in}{1.811389in}}%
\pgfpathlineto{\pgfqpoint{2.947062in}{0.991173in}}%
\pgfpathlineto{\pgfqpoint{2.949912in}{1.682963in}}%
\pgfpathlineto{\pgfqpoint{2.953775in}{0.913413in}}%
\pgfpathlineto{\pgfqpoint{2.958345in}{1.825556in}}%
\pgfpathlineto{\pgfqpoint{2.959966in}{1.048662in}}%
\pgfpathlineto{\pgfqpoint{2.965411in}{1.776239in}}%
\pgfpathlineto{\pgfqpoint{2.966755in}{0.895614in}}%
\pgfpathlineto{\pgfqpoint{2.969840in}{1.695529in}}%
\pgfpathlineto{\pgfqpoint{2.972451in}{0.898239in}}%
\pgfpathlineto{\pgfqpoint{2.976854in}{1.662297in}}%
\pgfpathlineto{\pgfqpoint{2.980762in}{0.660032in}}%
\pgfpathlineto{\pgfqpoint{2.983036in}{1.466610in}}%
\pgfpathlineto{\pgfqpoint{2.985297in}{0.886352in}}%
\pgfpathlineto{\pgfqpoint{2.988529in}{1.703320in}}%
\pgfpathlineto{\pgfqpoint{2.991964in}{0.834251in}}%
\pgfpathlineto{\pgfqpoint{2.994746in}{1.622796in}}%
\pgfpathlineto{\pgfqpoint{2.998721in}{0.870662in}}%
\pgfpathlineto{\pgfqpoint{3.001970in}{1.546487in}}%
\pgfpathlineto{\pgfqpoint{3.006913in}{0.670157in}}%
\pgfpathlineto{\pgfqpoint{3.008052in}{1.568505in}}%
\pgfpathlineto{\pgfqpoint{3.011075in}{0.865064in}}%
\pgfpathlineto{\pgfqpoint{3.014622in}{1.584654in}}%
\pgfpathlineto{\pgfqpoint{3.020431in}{0.673748in}}%
\pgfpathlineto{\pgfqpoint{3.020878in}{1.512865in}}%
\pgfpathlineto{\pgfqpoint{3.023779in}{0.802605in}}%
\pgfpathlineto{\pgfqpoint{3.027700in}{1.703251in}}%
\pgfpathlineto{\pgfqpoint{3.030713in}{0.867606in}}%
\pgfpathlineto{\pgfqpoint{3.034447in}{1.745146in}}%
\pgfpathlineto{\pgfqpoint{3.038258in}{0.945518in}}%
\pgfpathlineto{\pgfqpoint{3.040609in}{1.717493in}}%
\pgfpathlineto{\pgfqpoint{3.045746in}{0.875349in}}%
\pgfpathlineto{\pgfqpoint{3.046894in}{1.828734in}}%
\pgfpathlineto{\pgfqpoint{3.049538in}{1.057518in}}%
\pgfpathlineto{\pgfqpoint{3.053333in}{1.750315in}}%
\pgfpathlineto{\pgfqpoint{3.058543in}{0.879115in}}%
\pgfpathlineto{\pgfqpoint{3.059974in}{1.672653in}}%
\pgfpathlineto{\pgfqpoint{3.063335in}{0.949207in}}%
\pgfpathlineto{\pgfqpoint{3.068751in}{1.606303in}}%
\pgfpathlineto{\pgfqpoint{3.069096in}{0.736073in}}%
\pgfpathlineto{\pgfqpoint{3.072273in}{1.527521in}}%
\pgfpathlineto{\pgfqpoint{3.077886in}{0.833526in}}%
\pgfpathlineto{\pgfqpoint{3.078538in}{1.605475in}}%
\pgfpathlineto{\pgfqpoint{3.082417in}{0.948962in}}%
\pgfpathlineto{\pgfqpoint{3.085418in}{1.683759in}}%
\pgfpathlineto{\pgfqpoint{3.088451in}{0.889638in}}%
\pgfpathlineto{\pgfqpoint{3.091497in}{1.669540in}}%
\pgfpathlineto{\pgfqpoint{3.094832in}{0.806651in}}%
\pgfpathlineto{\pgfqpoint{3.099518in}{1.630623in}}%
\pgfpathlineto{\pgfqpoint{3.101232in}{0.843335in}}%
\pgfpathlineto{\pgfqpoint{3.104378in}{1.563011in}}%
\pgfpathlineto{\pgfqpoint{3.107555in}{0.925189in}}%
\pgfpathlineto{\pgfqpoint{3.113184in}{1.747384in}}%
\pgfpathlineto{\pgfqpoint{3.115290in}{0.905623in}}%
\pgfpathlineto{\pgfqpoint{3.117812in}{1.761240in}}%
\pgfpathlineto{\pgfqpoint{3.120716in}{0.998986in}}%
\pgfpathlineto{\pgfqpoint{3.124431in}{1.867480in}}%
\pgfpathlineto{\pgfqpoint{3.128905in}{0.952165in}}%
\pgfpathlineto{\pgfqpoint{3.130140in}{1.804000in}}%
\pgfpathlineto{\pgfqpoint{3.133488in}{0.900209in}}%
\pgfpathlineto{\pgfqpoint{3.136623in}{1.748713in}}%
\pgfpathlineto{\pgfqpoint{3.140740in}{0.975184in}}%
\pgfpathlineto{\pgfqpoint{3.143915in}{1.743061in}}%
\pgfpathlineto{\pgfqpoint{3.146957in}{0.980998in}}%
\pgfpathlineto{\pgfqpoint{3.150218in}{1.686933in}}%
\pgfpathlineto{\pgfqpoint{3.153049in}{0.825020in}}%
\pgfpathlineto{\pgfqpoint{3.156950in}{1.657964in}}%
\pgfpathlineto{\pgfqpoint{3.159613in}{0.934679in}}%
\pgfpathlineto{\pgfqpoint{3.163122in}{1.718542in}}%
\pgfpathlineto{\pgfqpoint{3.166615in}{0.972485in}}%
\pgfpathlineto{\pgfqpoint{3.169069in}{1.814869in}}%
\pgfpathlineto{\pgfqpoint{3.172433in}{1.045628in}}%
\pgfpathlineto{\pgfqpoint{3.177952in}{1.878096in}}%
\pgfpathlineto{\pgfqpoint{3.180129in}{1.039185in}}%
\pgfpathlineto{\pgfqpoint{3.182921in}{1.843996in}}%
\pgfpathlineto{\pgfqpoint{3.186002in}{0.960242in}}%
\pgfpathlineto{\pgfqpoint{3.188434in}{1.671890in}}%
\pgfpathlineto{\pgfqpoint{3.192467in}{1.012299in}}%
\pgfpathlineto{\pgfqpoint{3.197394in}{1.842117in}}%
\pgfpathlineto{\pgfqpoint{3.197844in}{1.062739in}}%
\pgfpathlineto{\pgfqpoint{3.203051in}{0.755256in}}%
\pgfpathlineto{\pgfqpoint{3.204762in}{1.663935in}}%
\pgfpathlineto{\pgfqpoint{3.207779in}{1.042614in}}%
\pgfpathlineto{\pgfqpoint{3.211475in}{1.760342in}}%
\pgfpathlineto{\pgfqpoint{3.214035in}{0.989228in}}%
\pgfpathlineto{\pgfqpoint{3.218261in}{1.703246in}}%
\pgfpathlineto{\pgfqpoint{3.220779in}{0.943894in}}%
\pgfpathlineto{\pgfqpoint{3.224192in}{1.825514in}}%
\pgfpathlineto{\pgfqpoint{3.228289in}{0.976914in}}%
\pgfpathlineto{\pgfqpoint{3.230306in}{1.581785in}}%
\pgfpathlineto{\pgfqpoint{3.233242in}{0.926671in}}%
\pgfpathlineto{\pgfqpoint{3.238166in}{1.588980in}}%
\pgfpathlineto{\pgfqpoint{3.240585in}{0.962023in}}%
\pgfpathlineto{\pgfqpoint{3.244885in}{1.796312in}}%
\pgfpathlineto{\pgfqpoint{3.246998in}{1.028953in}}%
\pgfpathlineto{\pgfqpoint{3.249494in}{1.753299in}}%
\pgfpathlineto{\pgfqpoint{3.255749in}{0.803999in}}%
\pgfpathlineto{\pgfqpoint{3.255997in}{1.532283in}}%
\pgfpathlineto{\pgfqpoint{3.260757in}{0.900950in}}%
\pgfpathlineto{\pgfqpoint{3.262217in}{1.577741in}}%
\pgfpathlineto{\pgfqpoint{3.265855in}{0.947063in}}%
\pgfpathlineto{\pgfqpoint{3.269016in}{1.628055in}}%
\pgfpathlineto{\pgfqpoint{3.272840in}{0.708262in}}%
\pgfpathlineto{\pgfqpoint{3.275716in}{1.513746in}}%
\pgfpathlineto{\pgfqpoint{3.278987in}{0.846099in}}%
\pgfpathlineto{\pgfqpoint{3.282820in}{1.854609in}}%
\pgfpathlineto{\pgfqpoint{3.284885in}{1.061885in}}%
\pgfpathlineto{\pgfqpoint{3.290800in}{1.741944in}}%
\pgfpathlineto{\pgfqpoint{3.292202in}{0.883998in}}%
\pgfpathlineto{\pgfqpoint{3.296316in}{1.786122in}}%
\pgfpathlineto{\pgfqpoint{3.297911in}{0.989289in}}%
\pgfpathlineto{\pgfqpoint{3.301188in}{1.615408in}}%
\pgfpathlineto{\pgfqpoint{3.304346in}{0.742513in}}%
\pgfpathlineto{\pgfqpoint{3.308257in}{1.719095in}}%
\pgfpathlineto{\pgfqpoint{3.311010in}{1.097399in}}%
\pgfpathlineto{\pgfqpoint{3.315108in}{1.737897in}}%
\pgfpathlineto{\pgfqpoint{3.317263in}{1.090284in}}%
\pgfpathlineto{\pgfqpoint{3.320305in}{1.835993in}}%
\pgfpathlineto{\pgfqpoint{3.323515in}{0.990343in}}%
\pgfpathlineto{\pgfqpoint{3.327342in}{1.695577in}}%
\pgfpathlineto{\pgfqpoint{3.331466in}{0.835078in}}%
\pgfpathlineto{\pgfqpoint{3.333749in}{1.570309in}}%
\pgfpathlineto{\pgfqpoint{3.337236in}{0.983297in}}%
\pgfpathlineto{\pgfqpoint{3.339889in}{1.795916in}}%
\pgfpathlineto{\pgfqpoint{3.343256in}{1.020119in}}%
\pgfpathlineto{\pgfqpoint{3.347074in}{1.982290in}}%
\pgfpathlineto{\pgfqpoint{3.350033in}{1.036170in}}%
\pgfpathlineto{\pgfqpoint{3.352780in}{2.024448in}}%
\pgfpathlineto{\pgfqpoint{3.355658in}{1.100624in}}%
\pgfpathlineto{\pgfqpoint{3.359029in}{1.712666in}}%
\pgfpathlineto{\pgfqpoint{3.362991in}{0.950796in}}%
\pgfpathlineto{\pgfqpoint{3.365918in}{1.783922in}}%
\pgfpathlineto{\pgfqpoint{3.369067in}{1.098002in}}%
\pgfpathlineto{\pgfqpoint{3.372968in}{1.634746in}}%
\pgfpathlineto{\pgfqpoint{3.376493in}{0.969849in}}%
\pgfpathlineto{\pgfqpoint{3.378799in}{1.735630in}}%
\pgfpathlineto{\pgfqpoint{3.381574in}{1.092998in}}%
\pgfpathlineto{\pgfqpoint{3.387135in}{1.936830in}}%
\pgfpathlineto{\pgfqpoint{3.388187in}{1.146869in}}%
\pgfpathlineto{\pgfqpoint{3.391188in}{1.985900in}}%
\pgfpathlineto{\pgfqpoint{3.395883in}{0.948939in}}%
\pgfpathlineto{\pgfqpoint{3.397759in}{1.753979in}}%
\pgfpathlineto{\pgfqpoint{3.401017in}{1.235226in}}%
\pgfpathlineto{\pgfqpoint{3.404178in}{1.820138in}}%
\pgfpathlineto{\pgfqpoint{3.407922in}{0.916424in}}%
\pgfpathlineto{\pgfqpoint{3.411653in}{1.795697in}}%
\pgfpathlineto{\pgfqpoint{3.415416in}{1.113373in}}%
\pgfpathlineto{\pgfqpoint{3.417361in}{2.052008in}}%
\pgfpathlineto{\pgfqpoint{3.420410in}{0.883252in}}%
\pgfpathlineto{\pgfqpoint{3.423418in}{1.719792in}}%
\pgfpathlineto{\pgfqpoint{3.429506in}{0.969003in}}%
\pgfpathlineto{\pgfqpoint{3.429889in}{1.558972in}}%
\pgfpathlineto{\pgfqpoint{3.434440in}{0.925016in}}%
\pgfpathlineto{\pgfqpoint{3.437151in}{1.828995in}}%
\pgfpathlineto{\pgfqpoint{3.440415in}{1.191560in}}%
\pgfpathlineto{\pgfqpoint{3.442972in}{1.930541in}}%
\pgfpathlineto{\pgfqpoint{3.445986in}{1.075268in}}%
\pgfpathlineto{\pgfqpoint{3.449823in}{1.854954in}}%
\pgfpathlineto{\pgfqpoint{3.453579in}{1.060684in}}%
\pgfpathlineto{\pgfqpoint{3.456667in}{1.830860in}}%
\pgfpathlineto{\pgfqpoint{3.460028in}{0.957671in}}%
\pgfpathlineto{\pgfqpoint{3.462855in}{1.733337in}}%
\pgfpathlineto{\pgfqpoint{3.465499in}{1.063256in}}%
\pgfpathlineto{\pgfqpoint{3.468930in}{1.767766in}}%
\pgfpathlineto{\pgfqpoint{3.473848in}{1.030600in}}%
\pgfpathlineto{\pgfqpoint{3.475221in}{1.733718in}}%
\pgfpathlineto{\pgfqpoint{3.478730in}{1.023525in}}%
\pgfpathlineto{\pgfqpoint{3.482493in}{1.767455in}}%
\pgfpathlineto{\pgfqpoint{3.485375in}{0.956169in}}%
\pgfpathlineto{\pgfqpoint{3.489566in}{1.724870in}}%
\pgfpathlineto{\pgfqpoint{3.491997in}{1.066992in}}%
\pgfpathlineto{\pgfqpoint{3.496883in}{0.858850in}}%
\pgfpathlineto{\pgfqpoint{3.499385in}{1.918062in}}%
\pgfpathlineto{\pgfqpoint{3.501134in}{1.182107in}}%
\pgfpathlineto{\pgfqpoint{3.503958in}{1.850719in}}%
\pgfpathlineto{\pgfqpoint{3.507837in}{1.070640in}}%
\pgfpathlineto{\pgfqpoint{3.510413in}{1.774393in}}%
\pgfpathlineto{\pgfqpoint{3.513784in}{0.895734in}}%
\pgfpathlineto{\pgfqpoint{3.517219in}{1.725437in}}%
\pgfpathlineto{\pgfqpoint{3.520020in}{1.001599in}}%
\pgfpathlineto{\pgfqpoint{3.524179in}{1.727256in}}%
\pgfpathlineto{\pgfqpoint{3.527228in}{0.974435in}}%
\pgfpathlineto{\pgfqpoint{3.530016in}{1.571154in}}%
\pgfpathlineto{\pgfqpoint{3.533368in}{0.880700in}}%
\pgfpathlineto{\pgfqpoint{3.537484in}{1.837016in}}%
\pgfpathlineto{\pgfqpoint{3.539446in}{0.858323in}}%
\pgfpathlineto{\pgfqpoint{3.543077in}{1.587124in}}%
\pgfpathlineto{\pgfqpoint{3.546731in}{1.017936in}}%
\pgfpathlineto{\pgfqpoint{3.549053in}{1.730089in}}%
\pgfpathlineto{\pgfqpoint{3.552774in}{0.755856in}}%
\pgfpathlineto{\pgfqpoint{3.556495in}{1.772793in}}%
\pgfpathlineto{\pgfqpoint{3.561091in}{0.842894in}}%
\pgfpathlineto{\pgfqpoint{3.561934in}{1.663079in}}%
\pgfpathlineto{\pgfqpoint{3.566874in}{0.879296in}}%
\pgfpathlineto{\pgfqpoint{3.568698in}{1.560969in}}%
\pgfpathlineto{\pgfqpoint{3.571718in}{1.872271in}}%
\pgfpathlineto{\pgfqpoint{3.576159in}{0.903639in}}%
\pgfpathlineto{\pgfqpoint{3.579446in}{1.611368in}}%
\pgfpathlineto{\pgfqpoint{3.582515in}{0.805028in}}%
\pgfpathlineto{\pgfqpoint{3.585300in}{1.629637in}}%
\pgfpathlineto{\pgfqpoint{3.589471in}{1.047982in}}%
\pgfpathlineto{\pgfqpoint{3.591317in}{1.769397in}}%
\pgfpathlineto{\pgfqpoint{3.594392in}{0.988513in}}%
\pgfpathlineto{\pgfqpoint{3.598905in}{1.813553in}}%
\pgfpathlineto{\pgfqpoint{3.601600in}{1.008684in}}%
\pgfpathlineto{\pgfqpoint{3.605926in}{1.923267in}}%
\pgfpathlineto{\pgfqpoint{3.607048in}{1.256496in}}%
\pgfpathlineto{\pgfqpoint{3.610525in}{1.988509in}}%
\pgfpathlineto{\pgfqpoint{3.614326in}{1.085068in}}%
\pgfpathlineto{\pgfqpoint{3.618179in}{1.769092in}}%
\pgfpathlineto{\pgfqpoint{3.621595in}{0.876961in}}%
\pgfpathlineto{\pgfqpoint{3.624065in}{1.797899in}}%
\pgfpathlineto{\pgfqpoint{3.627362in}{1.021217in}}%
\pgfpathlineto{\pgfqpoint{3.630491in}{1.771748in}}%
\pgfpathlineto{\pgfqpoint{3.633331in}{1.036552in}}%
\pgfpathlineto{\pgfqpoint{3.636145in}{1.689656in}}%
\pgfpathlineto{\pgfqpoint{3.641163in}{1.930095in}}%
\pgfpathlineto{\pgfqpoint{3.642812in}{1.081034in}}%
\pgfpathlineto{\pgfqpoint{3.646122in}{1.732400in}}%
\pgfpathlineto{\pgfqpoint{3.649978in}{1.122963in}}%
\pgfpathlineto{\pgfqpoint{3.655041in}{1.859100in}}%
\pgfpathlineto{\pgfqpoint{3.655372in}{1.143464in}}%
\pgfpathlineto{\pgfqpoint{3.659656in}{1.900586in}}%
\pgfpathlineto{\pgfqpoint{3.661830in}{1.056045in}}%
\pgfpathlineto{\pgfqpoint{3.665905in}{1.707857in}}%
\pgfpathlineto{\pgfqpoint{3.668835in}{1.083309in}}%
\pgfpathlineto{\pgfqpoint{3.672013in}{1.878059in}}%
\pgfpathlineto{\pgfqpoint{3.675940in}{1.148154in}}%
\pgfpathlineto{\pgfqpoint{3.678262in}{2.018501in}}%
\pgfpathlineto{\pgfqpoint{3.681150in}{1.256754in}}%
\pgfpathlineto{\pgfqpoint{3.684823in}{1.865859in}}%
\pgfpathlineto{\pgfqpoint{3.688017in}{1.103308in}}%
\pgfpathlineto{\pgfqpoint{3.692741in}{1.968059in}}%
\pgfpathlineto{\pgfqpoint{3.694455in}{1.099359in}}%
\pgfpathlineto{\pgfqpoint{3.698508in}{1.940680in}}%
\pgfpathlineto{\pgfqpoint{3.701226in}{1.091639in}}%
\pgfpathlineto{\pgfqpoint{3.705928in}{1.803590in}}%
\pgfpathlineto{\pgfqpoint{3.707349in}{1.082754in}}%
\pgfpathlineto{\pgfqpoint{3.712100in}{1.892710in}}%
\pgfpathlineto{\pgfqpoint{3.713450in}{1.001728in}}%
\pgfpathlineto{\pgfqpoint{3.717085in}{1.748017in}}%
\pgfpathlineto{\pgfqpoint{3.719848in}{1.046413in}}%
\pgfpathlineto{\pgfqpoint{3.724926in}{1.967068in}}%
\pgfpathlineto{\pgfqpoint{3.726557in}{0.832223in}}%
\pgfpathlineto{\pgfqpoint{3.729686in}{1.667796in}}%
\pgfpathlineto{\pgfqpoint{3.734941in}{1.043111in}}%
\pgfpathlineto{\pgfqpoint{3.737115in}{1.995613in}}%
\pgfpathlineto{\pgfqpoint{3.739206in}{1.183050in}}%
\pgfpathlineto{\pgfqpoint{3.743303in}{1.874538in}}%
\pgfpathlineto{\pgfqpoint{3.748006in}{0.977201in}}%
\pgfpathlineto{\pgfqpoint{3.748977in}{1.631054in}}%
\pgfpathlineto{\pgfqpoint{3.752988in}{0.996064in}}%
\pgfpathlineto{\pgfqpoint{3.755274in}{1.809934in}}%
\pgfpathlineto{\pgfqpoint{3.759166in}{1.049967in}}%
\pgfpathlineto{\pgfqpoint{3.763501in}{1.790746in}}%
\pgfpathlineto{\pgfqpoint{3.765219in}{1.034307in}}%
\pgfpathlineto{\pgfqpoint{3.769722in}{1.641681in}}%
\pgfpathlineto{\pgfqpoint{3.771706in}{0.988269in}}%
\pgfpathlineto{\pgfqpoint{3.776160in}{1.987729in}}%
\pgfpathlineto{\pgfqpoint{3.778010in}{1.194180in}}%
\pgfpathlineto{\pgfqpoint{3.782017in}{1.742033in}}%
\pgfpathlineto{\pgfqpoint{3.784828in}{1.126087in}}%
\pgfpathlineto{\pgfqpoint{3.788337in}{1.842919in}}%
\pgfpathlineto{\pgfqpoint{3.790981in}{1.201041in}}%
\pgfpathlineto{\pgfqpoint{3.795586in}{1.929884in}}%
\pgfpathlineto{\pgfqpoint{3.798304in}{1.114313in}}%
\pgfpathlineto{\pgfqpoint{3.802948in}{1.893229in}}%
\pgfpathlineto{\pgfqpoint{3.803900in}{1.255506in}}%
\pgfpathlineto{\pgfqpoint{3.808574in}{2.011650in}}%
\pgfpathlineto{\pgfqpoint{3.810886in}{1.133143in}}%
\pgfpathlineto{\pgfqpoint{3.813771in}{1.797370in}}%
\pgfpathlineto{\pgfqpoint{3.817682in}{1.040072in}}%
\pgfpathlineto{\pgfqpoint{3.820084in}{1.751931in}}%
\pgfpathlineto{\pgfqpoint{3.824565in}{1.134498in}}%
\pgfpathlineto{\pgfqpoint{3.827343in}{1.764495in}}%
\pgfpathlineto{\pgfqpoint{3.829955in}{1.120362in}}%
\pgfpathlineto{\pgfqpoint{3.833303in}{1.802406in}}%
\pgfpathlineto{\pgfqpoint{3.836439in}{0.845209in}}%
\pgfpathlineto{\pgfqpoint{3.840797in}{1.821079in}}%
\pgfpathlineto{\pgfqpoint{3.843103in}{0.973193in}}%
\pgfpathlineto{\pgfqpoint{3.846284in}{1.704880in}}%
\pgfpathlineto{\pgfqpoint{3.848722in}{1.335684in}}%
\pgfpathlineto{\pgfqpoint{3.848722in}{1.335684in}}%
\pgfusepath{stroke}%
\end{pgfscope}%
\begin{pgfscope}%
\pgfsetrectcap%
\pgfsetmiterjoin%
\pgfsetlinewidth{0.803000pt}%
\definecolor{currentstroke}{rgb}{0.000000,0.000000,0.000000}%
\pgfsetstrokecolor{currentstroke}%
\pgfsetdash{}{0pt}%
\pgfpathmoveto{\pgfqpoint{0.471687in}{0.416447in}}%
\pgfpathlineto{\pgfqpoint{0.471687in}{2.341095in}}%
\pgfusepath{stroke}%
\end{pgfscope}%
\begin{pgfscope}%
\pgfsetrectcap%
\pgfsetmiterjoin%
\pgfsetlinewidth{0.803000pt}%
\definecolor{currentstroke}{rgb}{0.000000,0.000000,0.000000}%
\pgfsetstrokecolor{currentstroke}%
\pgfsetdash{}{0pt}%
\pgfpathmoveto{\pgfqpoint{4.009533in}{0.416447in}}%
\pgfpathlineto{\pgfqpoint{4.009533in}{2.341095in}}%
\pgfusepath{stroke}%
\end{pgfscope}%
\begin{pgfscope}%
\pgfsetrectcap%
\pgfsetmiterjoin%
\pgfsetlinewidth{0.803000pt}%
\definecolor{currentstroke}{rgb}{0.000000,0.000000,0.000000}%
\pgfsetstrokecolor{currentstroke}%
\pgfsetdash{}{0pt}%
\pgfpathmoveto{\pgfqpoint{0.471687in}{0.416447in}}%
\pgfpathlineto{\pgfqpoint{4.009533in}{0.416447in}}%
\pgfusepath{stroke}%
\end{pgfscope}%
\begin{pgfscope}%
\pgfsetrectcap%
\pgfsetmiterjoin%
\pgfsetlinewidth{0.803000pt}%
\definecolor{currentstroke}{rgb}{0.000000,0.000000,0.000000}%
\pgfsetstrokecolor{currentstroke}%
\pgfsetdash{}{0pt}%
\pgfpathmoveto{\pgfqpoint{0.471687in}{2.341095in}}%
\pgfpathlineto{\pgfqpoint{4.009533in}{2.341095in}}%
\pgfusepath{stroke}%
\end{pgfscope}%
\end{pgfpicture}%
\makeatother%
\endgroup%
% data/simulations/sim_autozero.py
    \caption{Time series data with white noise and flicker noise.}
    \label{fig:autozero_raw_time}
\end{figure}

The time domain plot of the simulation is shown in figure \ref{fig:autozero_raw_time}. The white noise component is clearly visible, while the $f^{-1}$ flicker noise can be recognized, but its strength can hardly be estimated. It was already shown in section \ref{sec:noise_example} that different types of noise have different frequency components and those can be distinguished in the frequency domain, which leads to the next approach.
\begin{figure}[hb]
    \centering
    %% Creator: Matplotlib, PGF backend
%%
%% To include the figure in your LaTeX document, write
%%   \input{<filename>.pgf}
%%
%% Make sure the required packages are loaded in your preamble
%%   \usepackage{pgf}
%%
%% Also ensure that all the required font packages are loaded; for instance,
%% the lmodern package is sometimes necessary when using math font.
%%   \usepackage{lmodern}
%%
%% Figures using additional raster images can only be included by \input if
%% they are in the same directory as the main LaTeX file. For loading figures
%% from other directories you can use the `import` package
%%   \usepackage{import}
%%
%% and then include the figures with
%%   \import{<path to file>}{<filename>.pgf}
%%
%% Matplotlib used the following preamble
%%   \usepackage{siunitx}
%%   \usepackage{fontspec}
%%   \makeatletter\@ifpackageloaded{underscore}{}{\usepackage[strings]{underscore}}\makeatother
%%
\begingroup%
\makeatletter%
\begin{pgfpicture}%
\pgfpathrectangle{\pgfpointorigin}{\pgfqpoint{4.068242in}{2.514312in}}%
\pgfusepath{use as bounding box, clip}%
\begin{pgfscope}%
\pgfsetbuttcap%
\pgfsetmiterjoin%
\definecolor{currentfill}{rgb}{1.000000,1.000000,1.000000}%
\pgfsetfillcolor{currentfill}%
\pgfsetlinewidth{0.000000pt}%
\definecolor{currentstroke}{rgb}{1.000000,1.000000,1.000000}%
\pgfsetstrokecolor{currentstroke}%
\pgfsetdash{}{0pt}%
\pgfpathmoveto{\pgfqpoint{0.000000in}{0.000000in}}%
\pgfpathlineto{\pgfqpoint{4.068242in}{0.000000in}}%
\pgfpathlineto{\pgfqpoint{4.068242in}{2.514312in}}%
\pgfpathlineto{\pgfqpoint{0.000000in}{2.514312in}}%
\pgfpathlineto{\pgfqpoint{0.000000in}{0.000000in}}%
\pgfpathclose%
\pgfusepath{fill}%
\end{pgfscope}%
\begin{pgfscope}%
\pgfsetbuttcap%
\pgfsetmiterjoin%
\definecolor{currentfill}{rgb}{1.000000,1.000000,1.000000}%
\pgfsetfillcolor{currentfill}%
\pgfsetlinewidth{0.000000pt}%
\definecolor{currentstroke}{rgb}{0.000000,0.000000,0.000000}%
\pgfsetstrokecolor{currentstroke}%
\pgfsetstrokeopacity{0.000000}%
\pgfsetdash{}{0pt}%
\pgfpathmoveto{\pgfqpoint{0.661284in}{0.417642in}}%
\pgfpathlineto{\pgfqpoint{4.026572in}{0.417642in}}%
\pgfpathlineto{\pgfqpoint{4.026572in}{2.472642in}}%
\pgfpathlineto{\pgfqpoint{0.661284in}{2.472642in}}%
\pgfpathlineto{\pgfqpoint{0.661284in}{0.417642in}}%
\pgfpathclose%
\pgfusepath{fill}%
\end{pgfscope}%
\begin{pgfscope}%
\pgfpathrectangle{\pgfqpoint{0.661284in}{0.417642in}}{\pgfqpoint{3.365288in}{2.055000in}}%
\pgfusepath{clip}%
\pgfsetrectcap%
\pgfsetroundjoin%
\pgfsetlinewidth{0.803000pt}%
\definecolor{currentstroke}{rgb}{0.450000,0.450000,0.450000}%
\pgfsetstrokecolor{currentstroke}%
\pgfsetdash{}{0pt}%
\pgfpathmoveto{\pgfqpoint{0.714553in}{0.417642in}}%
\pgfpathlineto{\pgfqpoint{0.714553in}{2.472642in}}%
\pgfusepath{stroke}%
\end{pgfscope}%
\begin{pgfscope}%
\pgfsetbuttcap%
\pgfsetroundjoin%
\definecolor{currentfill}{rgb}{0.000000,0.000000,0.000000}%
\pgfsetfillcolor{currentfill}%
\pgfsetlinewidth{0.803000pt}%
\definecolor{currentstroke}{rgb}{0.000000,0.000000,0.000000}%
\pgfsetstrokecolor{currentstroke}%
\pgfsetdash{}{0pt}%
\pgfsys@defobject{currentmarker}{\pgfqpoint{0.000000in}{-0.048611in}}{\pgfqpoint{0.000000in}{0.000000in}}{%
\pgfpathmoveto{\pgfqpoint{0.000000in}{0.000000in}}%
\pgfpathlineto{\pgfqpoint{0.000000in}{-0.048611in}}%
\pgfusepath{stroke,fill}%
}%
\begin{pgfscope}%
\pgfsys@transformshift{0.714553in}{0.417642in}%
\pgfsys@useobject{currentmarker}{}%
\end{pgfscope}%
\end{pgfscope}%
\begin{pgfscope}%
\definecolor{textcolor}{rgb}{0.000000,0.000000,0.000000}%
\pgfsetstrokecolor{textcolor}%
\pgfsetfillcolor{textcolor}%
\pgftext[x=0.714553in,y=0.320420in,,top]{\color{textcolor}\rmfamily\fontsize{8.000000}{9.600000}\selectfont \(\displaystyle {10^{-3}}\)}%
\end{pgfscope}%
\begin{pgfscope}%
\pgfpathrectangle{\pgfqpoint{0.661284in}{0.417642in}}{\pgfqpoint{3.365288in}{2.055000in}}%
\pgfusepath{clip}%
\pgfsetrectcap%
\pgfsetroundjoin%
\pgfsetlinewidth{0.803000pt}%
\definecolor{currentstroke}{rgb}{0.450000,0.450000,0.450000}%
\pgfsetstrokecolor{currentstroke}%
\pgfsetdash{}{0pt}%
\pgfpathmoveto{\pgfqpoint{1.644249in}{0.417642in}}%
\pgfpathlineto{\pgfqpoint{1.644249in}{2.472642in}}%
\pgfusepath{stroke}%
\end{pgfscope}%
\begin{pgfscope}%
\pgfsetbuttcap%
\pgfsetroundjoin%
\definecolor{currentfill}{rgb}{0.000000,0.000000,0.000000}%
\pgfsetfillcolor{currentfill}%
\pgfsetlinewidth{0.803000pt}%
\definecolor{currentstroke}{rgb}{0.000000,0.000000,0.000000}%
\pgfsetstrokecolor{currentstroke}%
\pgfsetdash{}{0pt}%
\pgfsys@defobject{currentmarker}{\pgfqpoint{0.000000in}{-0.048611in}}{\pgfqpoint{0.000000in}{0.000000in}}{%
\pgfpathmoveto{\pgfqpoint{0.000000in}{0.000000in}}%
\pgfpathlineto{\pgfqpoint{0.000000in}{-0.048611in}}%
\pgfusepath{stroke,fill}%
}%
\begin{pgfscope}%
\pgfsys@transformshift{1.644249in}{0.417642in}%
\pgfsys@useobject{currentmarker}{}%
\end{pgfscope}%
\end{pgfscope}%
\begin{pgfscope}%
\definecolor{textcolor}{rgb}{0.000000,0.000000,0.000000}%
\pgfsetstrokecolor{textcolor}%
\pgfsetfillcolor{textcolor}%
\pgftext[x=1.644249in,y=0.320420in,,top]{\color{textcolor}\rmfamily\fontsize{8.000000}{9.600000}\selectfont \(\displaystyle {10^{-2}}\)}%
\end{pgfscope}%
\begin{pgfscope}%
\pgfpathrectangle{\pgfqpoint{0.661284in}{0.417642in}}{\pgfqpoint{3.365288in}{2.055000in}}%
\pgfusepath{clip}%
\pgfsetrectcap%
\pgfsetroundjoin%
\pgfsetlinewidth{0.803000pt}%
\definecolor{currentstroke}{rgb}{0.450000,0.450000,0.450000}%
\pgfsetstrokecolor{currentstroke}%
\pgfsetdash{}{0pt}%
\pgfpathmoveto{\pgfqpoint{2.573945in}{0.417642in}}%
\pgfpathlineto{\pgfqpoint{2.573945in}{2.472642in}}%
\pgfusepath{stroke}%
\end{pgfscope}%
\begin{pgfscope}%
\pgfsetbuttcap%
\pgfsetroundjoin%
\definecolor{currentfill}{rgb}{0.000000,0.000000,0.000000}%
\pgfsetfillcolor{currentfill}%
\pgfsetlinewidth{0.803000pt}%
\definecolor{currentstroke}{rgb}{0.000000,0.000000,0.000000}%
\pgfsetstrokecolor{currentstroke}%
\pgfsetdash{}{0pt}%
\pgfsys@defobject{currentmarker}{\pgfqpoint{0.000000in}{-0.048611in}}{\pgfqpoint{0.000000in}{0.000000in}}{%
\pgfpathmoveto{\pgfqpoint{0.000000in}{0.000000in}}%
\pgfpathlineto{\pgfqpoint{0.000000in}{-0.048611in}}%
\pgfusepath{stroke,fill}%
}%
\begin{pgfscope}%
\pgfsys@transformshift{2.573945in}{0.417642in}%
\pgfsys@useobject{currentmarker}{}%
\end{pgfscope}%
\end{pgfscope}%
\begin{pgfscope}%
\definecolor{textcolor}{rgb}{0.000000,0.000000,0.000000}%
\pgfsetstrokecolor{textcolor}%
\pgfsetfillcolor{textcolor}%
\pgftext[x=2.573945in,y=0.320420in,,top]{\color{textcolor}\rmfamily\fontsize{8.000000}{9.600000}\selectfont \(\displaystyle {10^{-1}}\)}%
\end{pgfscope}%
\begin{pgfscope}%
\pgfpathrectangle{\pgfqpoint{0.661284in}{0.417642in}}{\pgfqpoint{3.365288in}{2.055000in}}%
\pgfusepath{clip}%
\pgfsetrectcap%
\pgfsetroundjoin%
\pgfsetlinewidth{0.803000pt}%
\definecolor{currentstroke}{rgb}{0.450000,0.450000,0.450000}%
\pgfsetstrokecolor{currentstroke}%
\pgfsetdash{}{0pt}%
\pgfpathmoveto{\pgfqpoint{3.503641in}{0.417642in}}%
\pgfpathlineto{\pgfqpoint{3.503641in}{2.472642in}}%
\pgfusepath{stroke}%
\end{pgfscope}%
\begin{pgfscope}%
\pgfsetbuttcap%
\pgfsetroundjoin%
\definecolor{currentfill}{rgb}{0.000000,0.000000,0.000000}%
\pgfsetfillcolor{currentfill}%
\pgfsetlinewidth{0.803000pt}%
\definecolor{currentstroke}{rgb}{0.000000,0.000000,0.000000}%
\pgfsetstrokecolor{currentstroke}%
\pgfsetdash{}{0pt}%
\pgfsys@defobject{currentmarker}{\pgfqpoint{0.000000in}{-0.048611in}}{\pgfqpoint{0.000000in}{0.000000in}}{%
\pgfpathmoveto{\pgfqpoint{0.000000in}{0.000000in}}%
\pgfpathlineto{\pgfqpoint{0.000000in}{-0.048611in}}%
\pgfusepath{stroke,fill}%
}%
\begin{pgfscope}%
\pgfsys@transformshift{3.503641in}{0.417642in}%
\pgfsys@useobject{currentmarker}{}%
\end{pgfscope}%
\end{pgfscope}%
\begin{pgfscope}%
\definecolor{textcolor}{rgb}{0.000000,0.000000,0.000000}%
\pgfsetstrokecolor{textcolor}%
\pgfsetfillcolor{textcolor}%
\pgftext[x=3.503641in,y=0.320420in,,top]{\color{textcolor}\rmfamily\fontsize{8.000000}{9.600000}\selectfont \(\displaystyle {10^{0}}\)}%
\end{pgfscope}%
\begin{pgfscope}%
\pgfpathrectangle{\pgfqpoint{0.661284in}{0.417642in}}{\pgfqpoint{3.365288in}{2.055000in}}%
\pgfusepath{clip}%
\pgfsetrectcap%
\pgfsetroundjoin%
\pgfsetlinewidth{0.803000pt}%
\definecolor{currentstroke}{rgb}{0.850000,0.850000,0.850000}%
\pgfsetstrokecolor{currentstroke}%
\pgfsetdash{}{0pt}%
\pgfpathmoveto{\pgfqpoint{0.672012in}{0.417642in}}%
\pgfpathlineto{\pgfqpoint{0.672012in}{2.472642in}}%
\pgfusepath{stroke}%
\end{pgfscope}%
\begin{pgfscope}%
\pgfsetbuttcap%
\pgfsetroundjoin%
\definecolor{currentfill}{rgb}{0.000000,0.000000,0.000000}%
\pgfsetfillcolor{currentfill}%
\pgfsetlinewidth{0.602250pt}%
\definecolor{currentstroke}{rgb}{0.000000,0.000000,0.000000}%
\pgfsetstrokecolor{currentstroke}%
\pgfsetdash{}{0pt}%
\pgfsys@defobject{currentmarker}{\pgfqpoint{0.000000in}{-0.027778in}}{\pgfqpoint{0.000000in}{0.000000in}}{%
\pgfpathmoveto{\pgfqpoint{0.000000in}{0.000000in}}%
\pgfpathlineto{\pgfqpoint{0.000000in}{-0.027778in}}%
\pgfusepath{stroke,fill}%
}%
\begin{pgfscope}%
\pgfsys@transformshift{0.672012in}{0.417642in}%
\pgfsys@useobject{currentmarker}{}%
\end{pgfscope}%
\end{pgfscope}%
\begin{pgfscope}%
\pgfpathrectangle{\pgfqpoint{0.661284in}{0.417642in}}{\pgfqpoint{3.365288in}{2.055000in}}%
\pgfusepath{clip}%
\pgfsetrectcap%
\pgfsetroundjoin%
\pgfsetlinewidth{0.803000pt}%
\definecolor{currentstroke}{rgb}{0.850000,0.850000,0.850000}%
\pgfsetstrokecolor{currentstroke}%
\pgfsetdash{}{0pt}%
\pgfpathmoveto{\pgfqpoint{0.994419in}{0.417642in}}%
\pgfpathlineto{\pgfqpoint{0.994419in}{2.472642in}}%
\pgfusepath{stroke}%
\end{pgfscope}%
\begin{pgfscope}%
\pgfsetbuttcap%
\pgfsetroundjoin%
\definecolor{currentfill}{rgb}{0.000000,0.000000,0.000000}%
\pgfsetfillcolor{currentfill}%
\pgfsetlinewidth{0.602250pt}%
\definecolor{currentstroke}{rgb}{0.000000,0.000000,0.000000}%
\pgfsetstrokecolor{currentstroke}%
\pgfsetdash{}{0pt}%
\pgfsys@defobject{currentmarker}{\pgfqpoint{0.000000in}{-0.027778in}}{\pgfqpoint{0.000000in}{0.000000in}}{%
\pgfpathmoveto{\pgfqpoint{0.000000in}{0.000000in}}%
\pgfpathlineto{\pgfqpoint{0.000000in}{-0.027778in}}%
\pgfusepath{stroke,fill}%
}%
\begin{pgfscope}%
\pgfsys@transformshift{0.994419in}{0.417642in}%
\pgfsys@useobject{currentmarker}{}%
\end{pgfscope}%
\end{pgfscope}%
\begin{pgfscope}%
\pgfpathrectangle{\pgfqpoint{0.661284in}{0.417642in}}{\pgfqpoint{3.365288in}{2.055000in}}%
\pgfusepath{clip}%
\pgfsetrectcap%
\pgfsetroundjoin%
\pgfsetlinewidth{0.803000pt}%
\definecolor{currentstroke}{rgb}{0.850000,0.850000,0.850000}%
\pgfsetstrokecolor{currentstroke}%
\pgfsetdash{}{0pt}%
\pgfpathmoveto{\pgfqpoint{1.158130in}{0.417642in}}%
\pgfpathlineto{\pgfqpoint{1.158130in}{2.472642in}}%
\pgfusepath{stroke}%
\end{pgfscope}%
\begin{pgfscope}%
\pgfsetbuttcap%
\pgfsetroundjoin%
\definecolor{currentfill}{rgb}{0.000000,0.000000,0.000000}%
\pgfsetfillcolor{currentfill}%
\pgfsetlinewidth{0.602250pt}%
\definecolor{currentstroke}{rgb}{0.000000,0.000000,0.000000}%
\pgfsetstrokecolor{currentstroke}%
\pgfsetdash{}{0pt}%
\pgfsys@defobject{currentmarker}{\pgfqpoint{0.000000in}{-0.027778in}}{\pgfqpoint{0.000000in}{0.000000in}}{%
\pgfpathmoveto{\pgfqpoint{0.000000in}{0.000000in}}%
\pgfpathlineto{\pgfqpoint{0.000000in}{-0.027778in}}%
\pgfusepath{stroke,fill}%
}%
\begin{pgfscope}%
\pgfsys@transformshift{1.158130in}{0.417642in}%
\pgfsys@useobject{currentmarker}{}%
\end{pgfscope}%
\end{pgfscope}%
\begin{pgfscope}%
\pgfpathrectangle{\pgfqpoint{0.661284in}{0.417642in}}{\pgfqpoint{3.365288in}{2.055000in}}%
\pgfusepath{clip}%
\pgfsetrectcap%
\pgfsetroundjoin%
\pgfsetlinewidth{0.803000pt}%
\definecolor{currentstroke}{rgb}{0.850000,0.850000,0.850000}%
\pgfsetstrokecolor{currentstroke}%
\pgfsetdash{}{0pt}%
\pgfpathmoveto{\pgfqpoint{1.274285in}{0.417642in}}%
\pgfpathlineto{\pgfqpoint{1.274285in}{2.472642in}}%
\pgfusepath{stroke}%
\end{pgfscope}%
\begin{pgfscope}%
\pgfsetbuttcap%
\pgfsetroundjoin%
\definecolor{currentfill}{rgb}{0.000000,0.000000,0.000000}%
\pgfsetfillcolor{currentfill}%
\pgfsetlinewidth{0.602250pt}%
\definecolor{currentstroke}{rgb}{0.000000,0.000000,0.000000}%
\pgfsetstrokecolor{currentstroke}%
\pgfsetdash{}{0pt}%
\pgfsys@defobject{currentmarker}{\pgfqpoint{0.000000in}{-0.027778in}}{\pgfqpoint{0.000000in}{0.000000in}}{%
\pgfpathmoveto{\pgfqpoint{0.000000in}{0.000000in}}%
\pgfpathlineto{\pgfqpoint{0.000000in}{-0.027778in}}%
\pgfusepath{stroke,fill}%
}%
\begin{pgfscope}%
\pgfsys@transformshift{1.274285in}{0.417642in}%
\pgfsys@useobject{currentmarker}{}%
\end{pgfscope}%
\end{pgfscope}%
\begin{pgfscope}%
\pgfpathrectangle{\pgfqpoint{0.661284in}{0.417642in}}{\pgfqpoint{3.365288in}{2.055000in}}%
\pgfusepath{clip}%
\pgfsetrectcap%
\pgfsetroundjoin%
\pgfsetlinewidth{0.803000pt}%
\definecolor{currentstroke}{rgb}{0.850000,0.850000,0.850000}%
\pgfsetstrokecolor{currentstroke}%
\pgfsetdash{}{0pt}%
\pgfpathmoveto{\pgfqpoint{1.364382in}{0.417642in}}%
\pgfpathlineto{\pgfqpoint{1.364382in}{2.472642in}}%
\pgfusepath{stroke}%
\end{pgfscope}%
\begin{pgfscope}%
\pgfsetbuttcap%
\pgfsetroundjoin%
\definecolor{currentfill}{rgb}{0.000000,0.000000,0.000000}%
\pgfsetfillcolor{currentfill}%
\pgfsetlinewidth{0.602250pt}%
\definecolor{currentstroke}{rgb}{0.000000,0.000000,0.000000}%
\pgfsetstrokecolor{currentstroke}%
\pgfsetdash{}{0pt}%
\pgfsys@defobject{currentmarker}{\pgfqpoint{0.000000in}{-0.027778in}}{\pgfqpoint{0.000000in}{0.000000in}}{%
\pgfpathmoveto{\pgfqpoint{0.000000in}{0.000000in}}%
\pgfpathlineto{\pgfqpoint{0.000000in}{-0.027778in}}%
\pgfusepath{stroke,fill}%
}%
\begin{pgfscope}%
\pgfsys@transformshift{1.364382in}{0.417642in}%
\pgfsys@useobject{currentmarker}{}%
\end{pgfscope}%
\end{pgfscope}%
\begin{pgfscope}%
\pgfpathrectangle{\pgfqpoint{0.661284in}{0.417642in}}{\pgfqpoint{3.365288in}{2.055000in}}%
\pgfusepath{clip}%
\pgfsetrectcap%
\pgfsetroundjoin%
\pgfsetlinewidth{0.803000pt}%
\definecolor{currentstroke}{rgb}{0.850000,0.850000,0.850000}%
\pgfsetstrokecolor{currentstroke}%
\pgfsetdash{}{0pt}%
\pgfpathmoveto{\pgfqpoint{1.437997in}{0.417642in}}%
\pgfpathlineto{\pgfqpoint{1.437997in}{2.472642in}}%
\pgfusepath{stroke}%
\end{pgfscope}%
\begin{pgfscope}%
\pgfsetbuttcap%
\pgfsetroundjoin%
\definecolor{currentfill}{rgb}{0.000000,0.000000,0.000000}%
\pgfsetfillcolor{currentfill}%
\pgfsetlinewidth{0.602250pt}%
\definecolor{currentstroke}{rgb}{0.000000,0.000000,0.000000}%
\pgfsetstrokecolor{currentstroke}%
\pgfsetdash{}{0pt}%
\pgfsys@defobject{currentmarker}{\pgfqpoint{0.000000in}{-0.027778in}}{\pgfqpoint{0.000000in}{0.000000in}}{%
\pgfpathmoveto{\pgfqpoint{0.000000in}{0.000000in}}%
\pgfpathlineto{\pgfqpoint{0.000000in}{-0.027778in}}%
\pgfusepath{stroke,fill}%
}%
\begin{pgfscope}%
\pgfsys@transformshift{1.437997in}{0.417642in}%
\pgfsys@useobject{currentmarker}{}%
\end{pgfscope}%
\end{pgfscope}%
\begin{pgfscope}%
\pgfpathrectangle{\pgfqpoint{0.661284in}{0.417642in}}{\pgfqpoint{3.365288in}{2.055000in}}%
\pgfusepath{clip}%
\pgfsetrectcap%
\pgfsetroundjoin%
\pgfsetlinewidth{0.803000pt}%
\definecolor{currentstroke}{rgb}{0.850000,0.850000,0.850000}%
\pgfsetstrokecolor{currentstroke}%
\pgfsetdash{}{0pt}%
\pgfpathmoveto{\pgfqpoint{1.500237in}{0.417642in}}%
\pgfpathlineto{\pgfqpoint{1.500237in}{2.472642in}}%
\pgfusepath{stroke}%
\end{pgfscope}%
\begin{pgfscope}%
\pgfsetbuttcap%
\pgfsetroundjoin%
\definecolor{currentfill}{rgb}{0.000000,0.000000,0.000000}%
\pgfsetfillcolor{currentfill}%
\pgfsetlinewidth{0.602250pt}%
\definecolor{currentstroke}{rgb}{0.000000,0.000000,0.000000}%
\pgfsetstrokecolor{currentstroke}%
\pgfsetdash{}{0pt}%
\pgfsys@defobject{currentmarker}{\pgfqpoint{0.000000in}{-0.027778in}}{\pgfqpoint{0.000000in}{0.000000in}}{%
\pgfpathmoveto{\pgfqpoint{0.000000in}{0.000000in}}%
\pgfpathlineto{\pgfqpoint{0.000000in}{-0.027778in}}%
\pgfusepath{stroke,fill}%
}%
\begin{pgfscope}%
\pgfsys@transformshift{1.500237in}{0.417642in}%
\pgfsys@useobject{currentmarker}{}%
\end{pgfscope}%
\end{pgfscope}%
\begin{pgfscope}%
\pgfpathrectangle{\pgfqpoint{0.661284in}{0.417642in}}{\pgfqpoint{3.365288in}{2.055000in}}%
\pgfusepath{clip}%
\pgfsetrectcap%
\pgfsetroundjoin%
\pgfsetlinewidth{0.803000pt}%
\definecolor{currentstroke}{rgb}{0.850000,0.850000,0.850000}%
\pgfsetstrokecolor{currentstroke}%
\pgfsetdash{}{0pt}%
\pgfpathmoveto{\pgfqpoint{1.554152in}{0.417642in}}%
\pgfpathlineto{\pgfqpoint{1.554152in}{2.472642in}}%
\pgfusepath{stroke}%
\end{pgfscope}%
\begin{pgfscope}%
\pgfsetbuttcap%
\pgfsetroundjoin%
\definecolor{currentfill}{rgb}{0.000000,0.000000,0.000000}%
\pgfsetfillcolor{currentfill}%
\pgfsetlinewidth{0.602250pt}%
\definecolor{currentstroke}{rgb}{0.000000,0.000000,0.000000}%
\pgfsetstrokecolor{currentstroke}%
\pgfsetdash{}{0pt}%
\pgfsys@defobject{currentmarker}{\pgfqpoint{0.000000in}{-0.027778in}}{\pgfqpoint{0.000000in}{0.000000in}}{%
\pgfpathmoveto{\pgfqpoint{0.000000in}{0.000000in}}%
\pgfpathlineto{\pgfqpoint{0.000000in}{-0.027778in}}%
\pgfusepath{stroke,fill}%
}%
\begin{pgfscope}%
\pgfsys@transformshift{1.554152in}{0.417642in}%
\pgfsys@useobject{currentmarker}{}%
\end{pgfscope}%
\end{pgfscope}%
\begin{pgfscope}%
\pgfpathrectangle{\pgfqpoint{0.661284in}{0.417642in}}{\pgfqpoint{3.365288in}{2.055000in}}%
\pgfusepath{clip}%
\pgfsetrectcap%
\pgfsetroundjoin%
\pgfsetlinewidth{0.803000pt}%
\definecolor{currentstroke}{rgb}{0.850000,0.850000,0.850000}%
\pgfsetstrokecolor{currentstroke}%
\pgfsetdash{}{0pt}%
\pgfpathmoveto{\pgfqpoint{1.601708in}{0.417642in}}%
\pgfpathlineto{\pgfqpoint{1.601708in}{2.472642in}}%
\pgfusepath{stroke}%
\end{pgfscope}%
\begin{pgfscope}%
\pgfsetbuttcap%
\pgfsetroundjoin%
\definecolor{currentfill}{rgb}{0.000000,0.000000,0.000000}%
\pgfsetfillcolor{currentfill}%
\pgfsetlinewidth{0.602250pt}%
\definecolor{currentstroke}{rgb}{0.000000,0.000000,0.000000}%
\pgfsetstrokecolor{currentstroke}%
\pgfsetdash{}{0pt}%
\pgfsys@defobject{currentmarker}{\pgfqpoint{0.000000in}{-0.027778in}}{\pgfqpoint{0.000000in}{0.000000in}}{%
\pgfpathmoveto{\pgfqpoint{0.000000in}{0.000000in}}%
\pgfpathlineto{\pgfqpoint{0.000000in}{-0.027778in}}%
\pgfusepath{stroke,fill}%
}%
\begin{pgfscope}%
\pgfsys@transformshift{1.601708in}{0.417642in}%
\pgfsys@useobject{currentmarker}{}%
\end{pgfscope}%
\end{pgfscope}%
\begin{pgfscope}%
\pgfpathrectangle{\pgfqpoint{0.661284in}{0.417642in}}{\pgfqpoint{3.365288in}{2.055000in}}%
\pgfusepath{clip}%
\pgfsetrectcap%
\pgfsetroundjoin%
\pgfsetlinewidth{0.803000pt}%
\definecolor{currentstroke}{rgb}{0.850000,0.850000,0.850000}%
\pgfsetstrokecolor{currentstroke}%
\pgfsetdash{}{0pt}%
\pgfpathmoveto{\pgfqpoint{1.924115in}{0.417642in}}%
\pgfpathlineto{\pgfqpoint{1.924115in}{2.472642in}}%
\pgfusepath{stroke}%
\end{pgfscope}%
\begin{pgfscope}%
\pgfsetbuttcap%
\pgfsetroundjoin%
\definecolor{currentfill}{rgb}{0.000000,0.000000,0.000000}%
\pgfsetfillcolor{currentfill}%
\pgfsetlinewidth{0.602250pt}%
\definecolor{currentstroke}{rgb}{0.000000,0.000000,0.000000}%
\pgfsetstrokecolor{currentstroke}%
\pgfsetdash{}{0pt}%
\pgfsys@defobject{currentmarker}{\pgfqpoint{0.000000in}{-0.027778in}}{\pgfqpoint{0.000000in}{0.000000in}}{%
\pgfpathmoveto{\pgfqpoint{0.000000in}{0.000000in}}%
\pgfpathlineto{\pgfqpoint{0.000000in}{-0.027778in}}%
\pgfusepath{stroke,fill}%
}%
\begin{pgfscope}%
\pgfsys@transformshift{1.924115in}{0.417642in}%
\pgfsys@useobject{currentmarker}{}%
\end{pgfscope}%
\end{pgfscope}%
\begin{pgfscope}%
\pgfpathrectangle{\pgfqpoint{0.661284in}{0.417642in}}{\pgfqpoint{3.365288in}{2.055000in}}%
\pgfusepath{clip}%
\pgfsetrectcap%
\pgfsetroundjoin%
\pgfsetlinewidth{0.803000pt}%
\definecolor{currentstroke}{rgb}{0.850000,0.850000,0.850000}%
\pgfsetstrokecolor{currentstroke}%
\pgfsetdash{}{0pt}%
\pgfpathmoveto{\pgfqpoint{2.087826in}{0.417642in}}%
\pgfpathlineto{\pgfqpoint{2.087826in}{2.472642in}}%
\pgfusepath{stroke}%
\end{pgfscope}%
\begin{pgfscope}%
\pgfsetbuttcap%
\pgfsetroundjoin%
\definecolor{currentfill}{rgb}{0.000000,0.000000,0.000000}%
\pgfsetfillcolor{currentfill}%
\pgfsetlinewidth{0.602250pt}%
\definecolor{currentstroke}{rgb}{0.000000,0.000000,0.000000}%
\pgfsetstrokecolor{currentstroke}%
\pgfsetdash{}{0pt}%
\pgfsys@defobject{currentmarker}{\pgfqpoint{0.000000in}{-0.027778in}}{\pgfqpoint{0.000000in}{0.000000in}}{%
\pgfpathmoveto{\pgfqpoint{0.000000in}{0.000000in}}%
\pgfpathlineto{\pgfqpoint{0.000000in}{-0.027778in}}%
\pgfusepath{stroke,fill}%
}%
\begin{pgfscope}%
\pgfsys@transformshift{2.087826in}{0.417642in}%
\pgfsys@useobject{currentmarker}{}%
\end{pgfscope}%
\end{pgfscope}%
\begin{pgfscope}%
\pgfpathrectangle{\pgfqpoint{0.661284in}{0.417642in}}{\pgfqpoint{3.365288in}{2.055000in}}%
\pgfusepath{clip}%
\pgfsetrectcap%
\pgfsetroundjoin%
\pgfsetlinewidth{0.803000pt}%
\definecolor{currentstroke}{rgb}{0.850000,0.850000,0.850000}%
\pgfsetstrokecolor{currentstroke}%
\pgfsetdash{}{0pt}%
\pgfpathmoveto{\pgfqpoint{2.203981in}{0.417642in}}%
\pgfpathlineto{\pgfqpoint{2.203981in}{2.472642in}}%
\pgfusepath{stroke}%
\end{pgfscope}%
\begin{pgfscope}%
\pgfsetbuttcap%
\pgfsetroundjoin%
\definecolor{currentfill}{rgb}{0.000000,0.000000,0.000000}%
\pgfsetfillcolor{currentfill}%
\pgfsetlinewidth{0.602250pt}%
\definecolor{currentstroke}{rgb}{0.000000,0.000000,0.000000}%
\pgfsetstrokecolor{currentstroke}%
\pgfsetdash{}{0pt}%
\pgfsys@defobject{currentmarker}{\pgfqpoint{0.000000in}{-0.027778in}}{\pgfqpoint{0.000000in}{0.000000in}}{%
\pgfpathmoveto{\pgfqpoint{0.000000in}{0.000000in}}%
\pgfpathlineto{\pgfqpoint{0.000000in}{-0.027778in}}%
\pgfusepath{stroke,fill}%
}%
\begin{pgfscope}%
\pgfsys@transformshift{2.203981in}{0.417642in}%
\pgfsys@useobject{currentmarker}{}%
\end{pgfscope}%
\end{pgfscope}%
\begin{pgfscope}%
\pgfpathrectangle{\pgfqpoint{0.661284in}{0.417642in}}{\pgfqpoint{3.365288in}{2.055000in}}%
\pgfusepath{clip}%
\pgfsetrectcap%
\pgfsetroundjoin%
\pgfsetlinewidth{0.803000pt}%
\definecolor{currentstroke}{rgb}{0.850000,0.850000,0.850000}%
\pgfsetstrokecolor{currentstroke}%
\pgfsetdash{}{0pt}%
\pgfpathmoveto{\pgfqpoint{2.294078in}{0.417642in}}%
\pgfpathlineto{\pgfqpoint{2.294078in}{2.472642in}}%
\pgfusepath{stroke}%
\end{pgfscope}%
\begin{pgfscope}%
\pgfsetbuttcap%
\pgfsetroundjoin%
\definecolor{currentfill}{rgb}{0.000000,0.000000,0.000000}%
\pgfsetfillcolor{currentfill}%
\pgfsetlinewidth{0.602250pt}%
\definecolor{currentstroke}{rgb}{0.000000,0.000000,0.000000}%
\pgfsetstrokecolor{currentstroke}%
\pgfsetdash{}{0pt}%
\pgfsys@defobject{currentmarker}{\pgfqpoint{0.000000in}{-0.027778in}}{\pgfqpoint{0.000000in}{0.000000in}}{%
\pgfpathmoveto{\pgfqpoint{0.000000in}{0.000000in}}%
\pgfpathlineto{\pgfqpoint{0.000000in}{-0.027778in}}%
\pgfusepath{stroke,fill}%
}%
\begin{pgfscope}%
\pgfsys@transformshift{2.294078in}{0.417642in}%
\pgfsys@useobject{currentmarker}{}%
\end{pgfscope}%
\end{pgfscope}%
\begin{pgfscope}%
\pgfpathrectangle{\pgfqpoint{0.661284in}{0.417642in}}{\pgfqpoint{3.365288in}{2.055000in}}%
\pgfusepath{clip}%
\pgfsetrectcap%
\pgfsetroundjoin%
\pgfsetlinewidth{0.803000pt}%
\definecolor{currentstroke}{rgb}{0.850000,0.850000,0.850000}%
\pgfsetstrokecolor{currentstroke}%
\pgfsetdash{}{0pt}%
\pgfpathmoveto{\pgfqpoint{2.367693in}{0.417642in}}%
\pgfpathlineto{\pgfqpoint{2.367693in}{2.472642in}}%
\pgfusepath{stroke}%
\end{pgfscope}%
\begin{pgfscope}%
\pgfsetbuttcap%
\pgfsetroundjoin%
\definecolor{currentfill}{rgb}{0.000000,0.000000,0.000000}%
\pgfsetfillcolor{currentfill}%
\pgfsetlinewidth{0.602250pt}%
\definecolor{currentstroke}{rgb}{0.000000,0.000000,0.000000}%
\pgfsetstrokecolor{currentstroke}%
\pgfsetdash{}{0pt}%
\pgfsys@defobject{currentmarker}{\pgfqpoint{0.000000in}{-0.027778in}}{\pgfqpoint{0.000000in}{0.000000in}}{%
\pgfpathmoveto{\pgfqpoint{0.000000in}{0.000000in}}%
\pgfpathlineto{\pgfqpoint{0.000000in}{-0.027778in}}%
\pgfusepath{stroke,fill}%
}%
\begin{pgfscope}%
\pgfsys@transformshift{2.367693in}{0.417642in}%
\pgfsys@useobject{currentmarker}{}%
\end{pgfscope}%
\end{pgfscope}%
\begin{pgfscope}%
\pgfpathrectangle{\pgfqpoint{0.661284in}{0.417642in}}{\pgfqpoint{3.365288in}{2.055000in}}%
\pgfusepath{clip}%
\pgfsetrectcap%
\pgfsetroundjoin%
\pgfsetlinewidth{0.803000pt}%
\definecolor{currentstroke}{rgb}{0.850000,0.850000,0.850000}%
\pgfsetstrokecolor{currentstroke}%
\pgfsetdash{}{0pt}%
\pgfpathmoveto{\pgfqpoint{2.429933in}{0.417642in}}%
\pgfpathlineto{\pgfqpoint{2.429933in}{2.472642in}}%
\pgfusepath{stroke}%
\end{pgfscope}%
\begin{pgfscope}%
\pgfsetbuttcap%
\pgfsetroundjoin%
\definecolor{currentfill}{rgb}{0.000000,0.000000,0.000000}%
\pgfsetfillcolor{currentfill}%
\pgfsetlinewidth{0.602250pt}%
\definecolor{currentstroke}{rgb}{0.000000,0.000000,0.000000}%
\pgfsetstrokecolor{currentstroke}%
\pgfsetdash{}{0pt}%
\pgfsys@defobject{currentmarker}{\pgfqpoint{0.000000in}{-0.027778in}}{\pgfqpoint{0.000000in}{0.000000in}}{%
\pgfpathmoveto{\pgfqpoint{0.000000in}{0.000000in}}%
\pgfpathlineto{\pgfqpoint{0.000000in}{-0.027778in}}%
\pgfusepath{stroke,fill}%
}%
\begin{pgfscope}%
\pgfsys@transformshift{2.429933in}{0.417642in}%
\pgfsys@useobject{currentmarker}{}%
\end{pgfscope}%
\end{pgfscope}%
\begin{pgfscope}%
\pgfpathrectangle{\pgfqpoint{0.661284in}{0.417642in}}{\pgfqpoint{3.365288in}{2.055000in}}%
\pgfusepath{clip}%
\pgfsetrectcap%
\pgfsetroundjoin%
\pgfsetlinewidth{0.803000pt}%
\definecolor{currentstroke}{rgb}{0.850000,0.850000,0.850000}%
\pgfsetstrokecolor{currentstroke}%
\pgfsetdash{}{0pt}%
\pgfpathmoveto{\pgfqpoint{2.483848in}{0.417642in}}%
\pgfpathlineto{\pgfqpoint{2.483848in}{2.472642in}}%
\pgfusepath{stroke}%
\end{pgfscope}%
\begin{pgfscope}%
\pgfsetbuttcap%
\pgfsetroundjoin%
\definecolor{currentfill}{rgb}{0.000000,0.000000,0.000000}%
\pgfsetfillcolor{currentfill}%
\pgfsetlinewidth{0.602250pt}%
\definecolor{currentstroke}{rgb}{0.000000,0.000000,0.000000}%
\pgfsetstrokecolor{currentstroke}%
\pgfsetdash{}{0pt}%
\pgfsys@defobject{currentmarker}{\pgfqpoint{0.000000in}{-0.027778in}}{\pgfqpoint{0.000000in}{0.000000in}}{%
\pgfpathmoveto{\pgfqpoint{0.000000in}{0.000000in}}%
\pgfpathlineto{\pgfqpoint{0.000000in}{-0.027778in}}%
\pgfusepath{stroke,fill}%
}%
\begin{pgfscope}%
\pgfsys@transformshift{2.483848in}{0.417642in}%
\pgfsys@useobject{currentmarker}{}%
\end{pgfscope}%
\end{pgfscope}%
\begin{pgfscope}%
\pgfpathrectangle{\pgfqpoint{0.661284in}{0.417642in}}{\pgfqpoint{3.365288in}{2.055000in}}%
\pgfusepath{clip}%
\pgfsetrectcap%
\pgfsetroundjoin%
\pgfsetlinewidth{0.803000pt}%
\definecolor{currentstroke}{rgb}{0.850000,0.850000,0.850000}%
\pgfsetstrokecolor{currentstroke}%
\pgfsetdash{}{0pt}%
\pgfpathmoveto{\pgfqpoint{2.531404in}{0.417642in}}%
\pgfpathlineto{\pgfqpoint{2.531404in}{2.472642in}}%
\pgfusepath{stroke}%
\end{pgfscope}%
\begin{pgfscope}%
\pgfsetbuttcap%
\pgfsetroundjoin%
\definecolor{currentfill}{rgb}{0.000000,0.000000,0.000000}%
\pgfsetfillcolor{currentfill}%
\pgfsetlinewidth{0.602250pt}%
\definecolor{currentstroke}{rgb}{0.000000,0.000000,0.000000}%
\pgfsetstrokecolor{currentstroke}%
\pgfsetdash{}{0pt}%
\pgfsys@defobject{currentmarker}{\pgfqpoint{0.000000in}{-0.027778in}}{\pgfqpoint{0.000000in}{0.000000in}}{%
\pgfpathmoveto{\pgfqpoint{0.000000in}{0.000000in}}%
\pgfpathlineto{\pgfqpoint{0.000000in}{-0.027778in}}%
\pgfusepath{stroke,fill}%
}%
\begin{pgfscope}%
\pgfsys@transformshift{2.531404in}{0.417642in}%
\pgfsys@useobject{currentmarker}{}%
\end{pgfscope}%
\end{pgfscope}%
\begin{pgfscope}%
\pgfpathrectangle{\pgfqpoint{0.661284in}{0.417642in}}{\pgfqpoint{3.365288in}{2.055000in}}%
\pgfusepath{clip}%
\pgfsetrectcap%
\pgfsetroundjoin%
\pgfsetlinewidth{0.803000pt}%
\definecolor{currentstroke}{rgb}{0.850000,0.850000,0.850000}%
\pgfsetstrokecolor{currentstroke}%
\pgfsetdash{}{0pt}%
\pgfpathmoveto{\pgfqpoint{2.853811in}{0.417642in}}%
\pgfpathlineto{\pgfqpoint{2.853811in}{2.472642in}}%
\pgfusepath{stroke}%
\end{pgfscope}%
\begin{pgfscope}%
\pgfsetbuttcap%
\pgfsetroundjoin%
\definecolor{currentfill}{rgb}{0.000000,0.000000,0.000000}%
\pgfsetfillcolor{currentfill}%
\pgfsetlinewidth{0.602250pt}%
\definecolor{currentstroke}{rgb}{0.000000,0.000000,0.000000}%
\pgfsetstrokecolor{currentstroke}%
\pgfsetdash{}{0pt}%
\pgfsys@defobject{currentmarker}{\pgfqpoint{0.000000in}{-0.027778in}}{\pgfqpoint{0.000000in}{0.000000in}}{%
\pgfpathmoveto{\pgfqpoint{0.000000in}{0.000000in}}%
\pgfpathlineto{\pgfqpoint{0.000000in}{-0.027778in}}%
\pgfusepath{stroke,fill}%
}%
\begin{pgfscope}%
\pgfsys@transformshift{2.853811in}{0.417642in}%
\pgfsys@useobject{currentmarker}{}%
\end{pgfscope}%
\end{pgfscope}%
\begin{pgfscope}%
\pgfpathrectangle{\pgfqpoint{0.661284in}{0.417642in}}{\pgfqpoint{3.365288in}{2.055000in}}%
\pgfusepath{clip}%
\pgfsetrectcap%
\pgfsetroundjoin%
\pgfsetlinewidth{0.803000pt}%
\definecolor{currentstroke}{rgb}{0.850000,0.850000,0.850000}%
\pgfsetstrokecolor{currentstroke}%
\pgfsetdash{}{0pt}%
\pgfpathmoveto{\pgfqpoint{3.017522in}{0.417642in}}%
\pgfpathlineto{\pgfqpoint{3.017522in}{2.472642in}}%
\pgfusepath{stroke}%
\end{pgfscope}%
\begin{pgfscope}%
\pgfsetbuttcap%
\pgfsetroundjoin%
\definecolor{currentfill}{rgb}{0.000000,0.000000,0.000000}%
\pgfsetfillcolor{currentfill}%
\pgfsetlinewidth{0.602250pt}%
\definecolor{currentstroke}{rgb}{0.000000,0.000000,0.000000}%
\pgfsetstrokecolor{currentstroke}%
\pgfsetdash{}{0pt}%
\pgfsys@defobject{currentmarker}{\pgfqpoint{0.000000in}{-0.027778in}}{\pgfqpoint{0.000000in}{0.000000in}}{%
\pgfpathmoveto{\pgfqpoint{0.000000in}{0.000000in}}%
\pgfpathlineto{\pgfqpoint{0.000000in}{-0.027778in}}%
\pgfusepath{stroke,fill}%
}%
\begin{pgfscope}%
\pgfsys@transformshift{3.017522in}{0.417642in}%
\pgfsys@useobject{currentmarker}{}%
\end{pgfscope}%
\end{pgfscope}%
\begin{pgfscope}%
\pgfpathrectangle{\pgfqpoint{0.661284in}{0.417642in}}{\pgfqpoint{3.365288in}{2.055000in}}%
\pgfusepath{clip}%
\pgfsetrectcap%
\pgfsetroundjoin%
\pgfsetlinewidth{0.803000pt}%
\definecolor{currentstroke}{rgb}{0.850000,0.850000,0.850000}%
\pgfsetstrokecolor{currentstroke}%
\pgfsetdash{}{0pt}%
\pgfpathmoveto{\pgfqpoint{3.133677in}{0.417642in}}%
\pgfpathlineto{\pgfqpoint{3.133677in}{2.472642in}}%
\pgfusepath{stroke}%
\end{pgfscope}%
\begin{pgfscope}%
\pgfsetbuttcap%
\pgfsetroundjoin%
\definecolor{currentfill}{rgb}{0.000000,0.000000,0.000000}%
\pgfsetfillcolor{currentfill}%
\pgfsetlinewidth{0.602250pt}%
\definecolor{currentstroke}{rgb}{0.000000,0.000000,0.000000}%
\pgfsetstrokecolor{currentstroke}%
\pgfsetdash{}{0pt}%
\pgfsys@defobject{currentmarker}{\pgfqpoint{0.000000in}{-0.027778in}}{\pgfqpoint{0.000000in}{0.000000in}}{%
\pgfpathmoveto{\pgfqpoint{0.000000in}{0.000000in}}%
\pgfpathlineto{\pgfqpoint{0.000000in}{-0.027778in}}%
\pgfusepath{stroke,fill}%
}%
\begin{pgfscope}%
\pgfsys@transformshift{3.133677in}{0.417642in}%
\pgfsys@useobject{currentmarker}{}%
\end{pgfscope}%
\end{pgfscope}%
\begin{pgfscope}%
\pgfpathrectangle{\pgfqpoint{0.661284in}{0.417642in}}{\pgfqpoint{3.365288in}{2.055000in}}%
\pgfusepath{clip}%
\pgfsetrectcap%
\pgfsetroundjoin%
\pgfsetlinewidth{0.803000pt}%
\definecolor{currentstroke}{rgb}{0.850000,0.850000,0.850000}%
\pgfsetstrokecolor{currentstroke}%
\pgfsetdash{}{0pt}%
\pgfpathmoveto{\pgfqpoint{3.223774in}{0.417642in}}%
\pgfpathlineto{\pgfqpoint{3.223774in}{2.472642in}}%
\pgfusepath{stroke}%
\end{pgfscope}%
\begin{pgfscope}%
\pgfsetbuttcap%
\pgfsetroundjoin%
\definecolor{currentfill}{rgb}{0.000000,0.000000,0.000000}%
\pgfsetfillcolor{currentfill}%
\pgfsetlinewidth{0.602250pt}%
\definecolor{currentstroke}{rgb}{0.000000,0.000000,0.000000}%
\pgfsetstrokecolor{currentstroke}%
\pgfsetdash{}{0pt}%
\pgfsys@defobject{currentmarker}{\pgfqpoint{0.000000in}{-0.027778in}}{\pgfqpoint{0.000000in}{0.000000in}}{%
\pgfpathmoveto{\pgfqpoint{0.000000in}{0.000000in}}%
\pgfpathlineto{\pgfqpoint{0.000000in}{-0.027778in}}%
\pgfusepath{stroke,fill}%
}%
\begin{pgfscope}%
\pgfsys@transformshift{3.223774in}{0.417642in}%
\pgfsys@useobject{currentmarker}{}%
\end{pgfscope}%
\end{pgfscope}%
\begin{pgfscope}%
\pgfpathrectangle{\pgfqpoint{0.661284in}{0.417642in}}{\pgfqpoint{3.365288in}{2.055000in}}%
\pgfusepath{clip}%
\pgfsetrectcap%
\pgfsetroundjoin%
\pgfsetlinewidth{0.803000pt}%
\definecolor{currentstroke}{rgb}{0.850000,0.850000,0.850000}%
\pgfsetstrokecolor{currentstroke}%
\pgfsetdash{}{0pt}%
\pgfpathmoveto{\pgfqpoint{3.297389in}{0.417642in}}%
\pgfpathlineto{\pgfqpoint{3.297389in}{2.472642in}}%
\pgfusepath{stroke}%
\end{pgfscope}%
\begin{pgfscope}%
\pgfsetbuttcap%
\pgfsetroundjoin%
\definecolor{currentfill}{rgb}{0.000000,0.000000,0.000000}%
\pgfsetfillcolor{currentfill}%
\pgfsetlinewidth{0.602250pt}%
\definecolor{currentstroke}{rgb}{0.000000,0.000000,0.000000}%
\pgfsetstrokecolor{currentstroke}%
\pgfsetdash{}{0pt}%
\pgfsys@defobject{currentmarker}{\pgfqpoint{0.000000in}{-0.027778in}}{\pgfqpoint{0.000000in}{0.000000in}}{%
\pgfpathmoveto{\pgfqpoint{0.000000in}{0.000000in}}%
\pgfpathlineto{\pgfqpoint{0.000000in}{-0.027778in}}%
\pgfusepath{stroke,fill}%
}%
\begin{pgfscope}%
\pgfsys@transformshift{3.297389in}{0.417642in}%
\pgfsys@useobject{currentmarker}{}%
\end{pgfscope}%
\end{pgfscope}%
\begin{pgfscope}%
\pgfpathrectangle{\pgfqpoint{0.661284in}{0.417642in}}{\pgfqpoint{3.365288in}{2.055000in}}%
\pgfusepath{clip}%
\pgfsetrectcap%
\pgfsetroundjoin%
\pgfsetlinewidth{0.803000pt}%
\definecolor{currentstroke}{rgb}{0.850000,0.850000,0.850000}%
\pgfsetstrokecolor{currentstroke}%
\pgfsetdash{}{0pt}%
\pgfpathmoveto{\pgfqpoint{3.359629in}{0.417642in}}%
\pgfpathlineto{\pgfqpoint{3.359629in}{2.472642in}}%
\pgfusepath{stroke}%
\end{pgfscope}%
\begin{pgfscope}%
\pgfsetbuttcap%
\pgfsetroundjoin%
\definecolor{currentfill}{rgb}{0.000000,0.000000,0.000000}%
\pgfsetfillcolor{currentfill}%
\pgfsetlinewidth{0.602250pt}%
\definecolor{currentstroke}{rgb}{0.000000,0.000000,0.000000}%
\pgfsetstrokecolor{currentstroke}%
\pgfsetdash{}{0pt}%
\pgfsys@defobject{currentmarker}{\pgfqpoint{0.000000in}{-0.027778in}}{\pgfqpoint{0.000000in}{0.000000in}}{%
\pgfpathmoveto{\pgfqpoint{0.000000in}{0.000000in}}%
\pgfpathlineto{\pgfqpoint{0.000000in}{-0.027778in}}%
\pgfusepath{stroke,fill}%
}%
\begin{pgfscope}%
\pgfsys@transformshift{3.359629in}{0.417642in}%
\pgfsys@useobject{currentmarker}{}%
\end{pgfscope}%
\end{pgfscope}%
\begin{pgfscope}%
\pgfpathrectangle{\pgfqpoint{0.661284in}{0.417642in}}{\pgfqpoint{3.365288in}{2.055000in}}%
\pgfusepath{clip}%
\pgfsetrectcap%
\pgfsetroundjoin%
\pgfsetlinewidth{0.803000pt}%
\definecolor{currentstroke}{rgb}{0.850000,0.850000,0.850000}%
\pgfsetstrokecolor{currentstroke}%
\pgfsetdash{}{0pt}%
\pgfpathmoveto{\pgfqpoint{3.413544in}{0.417642in}}%
\pgfpathlineto{\pgfqpoint{3.413544in}{2.472642in}}%
\pgfusepath{stroke}%
\end{pgfscope}%
\begin{pgfscope}%
\pgfsetbuttcap%
\pgfsetroundjoin%
\definecolor{currentfill}{rgb}{0.000000,0.000000,0.000000}%
\pgfsetfillcolor{currentfill}%
\pgfsetlinewidth{0.602250pt}%
\definecolor{currentstroke}{rgb}{0.000000,0.000000,0.000000}%
\pgfsetstrokecolor{currentstroke}%
\pgfsetdash{}{0pt}%
\pgfsys@defobject{currentmarker}{\pgfqpoint{0.000000in}{-0.027778in}}{\pgfqpoint{0.000000in}{0.000000in}}{%
\pgfpathmoveto{\pgfqpoint{0.000000in}{0.000000in}}%
\pgfpathlineto{\pgfqpoint{0.000000in}{-0.027778in}}%
\pgfusepath{stroke,fill}%
}%
\begin{pgfscope}%
\pgfsys@transformshift{3.413544in}{0.417642in}%
\pgfsys@useobject{currentmarker}{}%
\end{pgfscope}%
\end{pgfscope}%
\begin{pgfscope}%
\pgfpathrectangle{\pgfqpoint{0.661284in}{0.417642in}}{\pgfqpoint{3.365288in}{2.055000in}}%
\pgfusepath{clip}%
\pgfsetrectcap%
\pgfsetroundjoin%
\pgfsetlinewidth{0.803000pt}%
\definecolor{currentstroke}{rgb}{0.850000,0.850000,0.850000}%
\pgfsetstrokecolor{currentstroke}%
\pgfsetdash{}{0pt}%
\pgfpathmoveto{\pgfqpoint{3.461100in}{0.417642in}}%
\pgfpathlineto{\pgfqpoint{3.461100in}{2.472642in}}%
\pgfusepath{stroke}%
\end{pgfscope}%
\begin{pgfscope}%
\pgfsetbuttcap%
\pgfsetroundjoin%
\definecolor{currentfill}{rgb}{0.000000,0.000000,0.000000}%
\pgfsetfillcolor{currentfill}%
\pgfsetlinewidth{0.602250pt}%
\definecolor{currentstroke}{rgb}{0.000000,0.000000,0.000000}%
\pgfsetstrokecolor{currentstroke}%
\pgfsetdash{}{0pt}%
\pgfsys@defobject{currentmarker}{\pgfqpoint{0.000000in}{-0.027778in}}{\pgfqpoint{0.000000in}{0.000000in}}{%
\pgfpathmoveto{\pgfqpoint{0.000000in}{0.000000in}}%
\pgfpathlineto{\pgfqpoint{0.000000in}{-0.027778in}}%
\pgfusepath{stroke,fill}%
}%
\begin{pgfscope}%
\pgfsys@transformshift{3.461100in}{0.417642in}%
\pgfsys@useobject{currentmarker}{}%
\end{pgfscope}%
\end{pgfscope}%
\begin{pgfscope}%
\pgfpathrectangle{\pgfqpoint{0.661284in}{0.417642in}}{\pgfqpoint{3.365288in}{2.055000in}}%
\pgfusepath{clip}%
\pgfsetrectcap%
\pgfsetroundjoin%
\pgfsetlinewidth{0.803000pt}%
\definecolor{currentstroke}{rgb}{0.850000,0.850000,0.850000}%
\pgfsetstrokecolor{currentstroke}%
\pgfsetdash{}{0pt}%
\pgfpathmoveto{\pgfqpoint{3.783507in}{0.417642in}}%
\pgfpathlineto{\pgfqpoint{3.783507in}{2.472642in}}%
\pgfusepath{stroke}%
\end{pgfscope}%
\begin{pgfscope}%
\pgfsetbuttcap%
\pgfsetroundjoin%
\definecolor{currentfill}{rgb}{0.000000,0.000000,0.000000}%
\pgfsetfillcolor{currentfill}%
\pgfsetlinewidth{0.602250pt}%
\definecolor{currentstroke}{rgb}{0.000000,0.000000,0.000000}%
\pgfsetstrokecolor{currentstroke}%
\pgfsetdash{}{0pt}%
\pgfsys@defobject{currentmarker}{\pgfqpoint{0.000000in}{-0.027778in}}{\pgfqpoint{0.000000in}{0.000000in}}{%
\pgfpathmoveto{\pgfqpoint{0.000000in}{0.000000in}}%
\pgfpathlineto{\pgfqpoint{0.000000in}{-0.027778in}}%
\pgfusepath{stroke,fill}%
}%
\begin{pgfscope}%
\pgfsys@transformshift{3.783507in}{0.417642in}%
\pgfsys@useobject{currentmarker}{}%
\end{pgfscope}%
\end{pgfscope}%
\begin{pgfscope}%
\pgfpathrectangle{\pgfqpoint{0.661284in}{0.417642in}}{\pgfqpoint{3.365288in}{2.055000in}}%
\pgfusepath{clip}%
\pgfsetrectcap%
\pgfsetroundjoin%
\pgfsetlinewidth{0.803000pt}%
\definecolor{currentstroke}{rgb}{0.850000,0.850000,0.850000}%
\pgfsetstrokecolor{currentstroke}%
\pgfsetdash{}{0pt}%
\pgfpathmoveto{\pgfqpoint{3.947218in}{0.417642in}}%
\pgfpathlineto{\pgfqpoint{3.947218in}{2.472642in}}%
\pgfusepath{stroke}%
\end{pgfscope}%
\begin{pgfscope}%
\pgfsetbuttcap%
\pgfsetroundjoin%
\definecolor{currentfill}{rgb}{0.000000,0.000000,0.000000}%
\pgfsetfillcolor{currentfill}%
\pgfsetlinewidth{0.602250pt}%
\definecolor{currentstroke}{rgb}{0.000000,0.000000,0.000000}%
\pgfsetstrokecolor{currentstroke}%
\pgfsetdash{}{0pt}%
\pgfsys@defobject{currentmarker}{\pgfqpoint{0.000000in}{-0.027778in}}{\pgfqpoint{0.000000in}{0.000000in}}{%
\pgfpathmoveto{\pgfqpoint{0.000000in}{0.000000in}}%
\pgfpathlineto{\pgfqpoint{0.000000in}{-0.027778in}}%
\pgfusepath{stroke,fill}%
}%
\begin{pgfscope}%
\pgfsys@transformshift{3.947218in}{0.417642in}%
\pgfsys@useobject{currentmarker}{}%
\end{pgfscope}%
\end{pgfscope}%
\begin{pgfscope}%
\definecolor{textcolor}{rgb}{0.000000,0.000000,0.000000}%
\pgfsetstrokecolor{textcolor}%
\pgfsetfillcolor{textcolor}%
\pgftext[x=2.343928in,y=0.165003in,,top]{\color{textcolor}\rmfamily\fontsize{10.000000}{12.000000}\selectfont Frequency in \(\displaystyle \unit{\Hz}\)}%
\end{pgfscope}%
\begin{pgfscope}%
\pgfpathrectangle{\pgfqpoint{0.661284in}{0.417642in}}{\pgfqpoint{3.365288in}{2.055000in}}%
\pgfusepath{clip}%
\pgfsetrectcap%
\pgfsetroundjoin%
\pgfsetlinewidth{0.803000pt}%
\definecolor{currentstroke}{rgb}{0.450000,0.450000,0.450000}%
\pgfsetstrokecolor{currentstroke}%
\pgfsetdash{}{0pt}%
\pgfpathmoveto{\pgfqpoint{0.661284in}{0.957775in}}%
\pgfpathlineto{\pgfqpoint{4.026572in}{0.957775in}}%
\pgfusepath{stroke}%
\end{pgfscope}%
\begin{pgfscope}%
\pgfsetbuttcap%
\pgfsetroundjoin%
\definecolor{currentfill}{rgb}{0.000000,0.000000,0.000000}%
\pgfsetfillcolor{currentfill}%
\pgfsetlinewidth{0.803000pt}%
\definecolor{currentstroke}{rgb}{0.000000,0.000000,0.000000}%
\pgfsetstrokecolor{currentstroke}%
\pgfsetdash{}{0pt}%
\pgfsys@defobject{currentmarker}{\pgfqpoint{-0.048611in}{0.000000in}}{\pgfqpoint{-0.000000in}{0.000000in}}{%
\pgfpathmoveto{\pgfqpoint{-0.000000in}{0.000000in}}%
\pgfpathlineto{\pgfqpoint{-0.048611in}{0.000000in}}%
\pgfusepath{stroke,fill}%
}%
\begin{pgfscope}%
\pgfsys@transformshift{0.661284in}{0.957775in}%
\pgfsys@useobject{currentmarker}{}%
\end{pgfscope}%
\end{pgfscope}%
\begin{pgfscope}%
\definecolor{textcolor}{rgb}{0.000000,0.000000,0.000000}%
\pgfsetstrokecolor{textcolor}%
\pgfsetfillcolor{textcolor}%
\pgftext[x=0.256963in, y=0.918622in, left, base]{\color{textcolor}\rmfamily\fontsize{8.000000}{9.600000}\selectfont \(\displaystyle {10^{-13}}\)}%
\end{pgfscope}%
\begin{pgfscope}%
\pgfpathrectangle{\pgfqpoint{0.661284in}{0.417642in}}{\pgfqpoint{3.365288in}{2.055000in}}%
\pgfusepath{clip}%
\pgfsetrectcap%
\pgfsetroundjoin%
\pgfsetlinewidth{0.803000pt}%
\definecolor{currentstroke}{rgb}{0.450000,0.450000,0.450000}%
\pgfsetstrokecolor{currentstroke}%
\pgfsetdash{}{0pt}%
\pgfpathmoveto{\pgfqpoint{0.661284in}{1.525490in}}%
\pgfpathlineto{\pgfqpoint{4.026572in}{1.525490in}}%
\pgfusepath{stroke}%
\end{pgfscope}%
\begin{pgfscope}%
\pgfsetbuttcap%
\pgfsetroundjoin%
\definecolor{currentfill}{rgb}{0.000000,0.000000,0.000000}%
\pgfsetfillcolor{currentfill}%
\pgfsetlinewidth{0.803000pt}%
\definecolor{currentstroke}{rgb}{0.000000,0.000000,0.000000}%
\pgfsetstrokecolor{currentstroke}%
\pgfsetdash{}{0pt}%
\pgfsys@defobject{currentmarker}{\pgfqpoint{-0.048611in}{0.000000in}}{\pgfqpoint{-0.000000in}{0.000000in}}{%
\pgfpathmoveto{\pgfqpoint{-0.000000in}{0.000000in}}%
\pgfpathlineto{\pgfqpoint{-0.048611in}{0.000000in}}%
\pgfusepath{stroke,fill}%
}%
\begin{pgfscope}%
\pgfsys@transformshift{0.661284in}{1.525490in}%
\pgfsys@useobject{currentmarker}{}%
\end{pgfscope}%
\end{pgfscope}%
\begin{pgfscope}%
\definecolor{textcolor}{rgb}{0.000000,0.000000,0.000000}%
\pgfsetstrokecolor{textcolor}%
\pgfsetfillcolor{textcolor}%
\pgftext[x=0.256963in, y=1.486338in, left, base]{\color{textcolor}\rmfamily\fontsize{8.000000}{9.600000}\selectfont \(\displaystyle {10^{-12}}\)}%
\end{pgfscope}%
\begin{pgfscope}%
\pgfpathrectangle{\pgfqpoint{0.661284in}{0.417642in}}{\pgfqpoint{3.365288in}{2.055000in}}%
\pgfusepath{clip}%
\pgfsetrectcap%
\pgfsetroundjoin%
\pgfsetlinewidth{0.803000pt}%
\definecolor{currentstroke}{rgb}{0.450000,0.450000,0.450000}%
\pgfsetstrokecolor{currentstroke}%
\pgfsetdash{}{0pt}%
\pgfpathmoveto{\pgfqpoint{0.661284in}{2.093205in}}%
\pgfpathlineto{\pgfqpoint{4.026572in}{2.093205in}}%
\pgfusepath{stroke}%
\end{pgfscope}%
\begin{pgfscope}%
\pgfsetbuttcap%
\pgfsetroundjoin%
\definecolor{currentfill}{rgb}{0.000000,0.000000,0.000000}%
\pgfsetfillcolor{currentfill}%
\pgfsetlinewidth{0.803000pt}%
\definecolor{currentstroke}{rgb}{0.000000,0.000000,0.000000}%
\pgfsetstrokecolor{currentstroke}%
\pgfsetdash{}{0pt}%
\pgfsys@defobject{currentmarker}{\pgfqpoint{-0.048611in}{0.000000in}}{\pgfqpoint{-0.000000in}{0.000000in}}{%
\pgfpathmoveto{\pgfqpoint{-0.000000in}{0.000000in}}%
\pgfpathlineto{\pgfqpoint{-0.048611in}{0.000000in}}%
\pgfusepath{stroke,fill}%
}%
\begin{pgfscope}%
\pgfsys@transformshift{0.661284in}{2.093205in}%
\pgfsys@useobject{currentmarker}{}%
\end{pgfscope}%
\end{pgfscope}%
\begin{pgfscope}%
\definecolor{textcolor}{rgb}{0.000000,0.000000,0.000000}%
\pgfsetstrokecolor{textcolor}%
\pgfsetfillcolor{textcolor}%
\pgftext[x=0.256963in, y=2.054053in, left, base]{\color{textcolor}\rmfamily\fontsize{8.000000}{9.600000}\selectfont \(\displaystyle {10^{-11}}\)}%
\end{pgfscope}%
\begin{pgfscope}%
\pgfpathrectangle{\pgfqpoint{0.661284in}{0.417642in}}{\pgfqpoint{3.365288in}{2.055000in}}%
\pgfusepath{clip}%
\pgfsetrectcap%
\pgfsetroundjoin%
\pgfsetlinewidth{0.803000pt}%
\definecolor{currentstroke}{rgb}{0.850000,0.850000,0.850000}%
\pgfsetstrokecolor{currentstroke}%
\pgfsetdash{}{0pt}%
\pgfpathmoveto{\pgfqpoint{0.661284in}{0.560959in}}%
\pgfpathlineto{\pgfqpoint{4.026572in}{0.560959in}}%
\pgfusepath{stroke}%
\end{pgfscope}%
\begin{pgfscope}%
\pgfsetbuttcap%
\pgfsetroundjoin%
\definecolor{currentfill}{rgb}{0.000000,0.000000,0.000000}%
\pgfsetfillcolor{currentfill}%
\pgfsetlinewidth{0.602250pt}%
\definecolor{currentstroke}{rgb}{0.000000,0.000000,0.000000}%
\pgfsetstrokecolor{currentstroke}%
\pgfsetdash{}{0pt}%
\pgfsys@defobject{currentmarker}{\pgfqpoint{-0.027778in}{0.000000in}}{\pgfqpoint{-0.000000in}{0.000000in}}{%
\pgfpathmoveto{\pgfqpoint{-0.000000in}{0.000000in}}%
\pgfpathlineto{\pgfqpoint{-0.027778in}{0.000000in}}%
\pgfusepath{stroke,fill}%
}%
\begin{pgfscope}%
\pgfsys@transformshift{0.661284in}{0.560959in}%
\pgfsys@useobject{currentmarker}{}%
\end{pgfscope}%
\end{pgfscope}%
\begin{pgfscope}%
\pgfpathrectangle{\pgfqpoint{0.661284in}{0.417642in}}{\pgfqpoint{3.365288in}{2.055000in}}%
\pgfusepath{clip}%
\pgfsetrectcap%
\pgfsetroundjoin%
\pgfsetlinewidth{0.803000pt}%
\definecolor{currentstroke}{rgb}{0.850000,0.850000,0.850000}%
\pgfsetstrokecolor{currentstroke}%
\pgfsetdash{}{0pt}%
\pgfpathmoveto{\pgfqpoint{0.661284in}{0.660929in}}%
\pgfpathlineto{\pgfqpoint{4.026572in}{0.660929in}}%
\pgfusepath{stroke}%
\end{pgfscope}%
\begin{pgfscope}%
\pgfsetbuttcap%
\pgfsetroundjoin%
\definecolor{currentfill}{rgb}{0.000000,0.000000,0.000000}%
\pgfsetfillcolor{currentfill}%
\pgfsetlinewidth{0.602250pt}%
\definecolor{currentstroke}{rgb}{0.000000,0.000000,0.000000}%
\pgfsetstrokecolor{currentstroke}%
\pgfsetdash{}{0pt}%
\pgfsys@defobject{currentmarker}{\pgfqpoint{-0.027778in}{0.000000in}}{\pgfqpoint{-0.000000in}{0.000000in}}{%
\pgfpathmoveto{\pgfqpoint{-0.000000in}{0.000000in}}%
\pgfpathlineto{\pgfqpoint{-0.027778in}{0.000000in}}%
\pgfusepath{stroke,fill}%
}%
\begin{pgfscope}%
\pgfsys@transformshift{0.661284in}{0.660929in}%
\pgfsys@useobject{currentmarker}{}%
\end{pgfscope}%
\end{pgfscope}%
\begin{pgfscope}%
\pgfpathrectangle{\pgfqpoint{0.661284in}{0.417642in}}{\pgfqpoint{3.365288in}{2.055000in}}%
\pgfusepath{clip}%
\pgfsetrectcap%
\pgfsetroundjoin%
\pgfsetlinewidth{0.803000pt}%
\definecolor{currentstroke}{rgb}{0.850000,0.850000,0.850000}%
\pgfsetstrokecolor{currentstroke}%
\pgfsetdash{}{0pt}%
\pgfpathmoveto{\pgfqpoint{0.661284in}{0.731859in}}%
\pgfpathlineto{\pgfqpoint{4.026572in}{0.731859in}}%
\pgfusepath{stroke}%
\end{pgfscope}%
\begin{pgfscope}%
\pgfsetbuttcap%
\pgfsetroundjoin%
\definecolor{currentfill}{rgb}{0.000000,0.000000,0.000000}%
\pgfsetfillcolor{currentfill}%
\pgfsetlinewidth{0.602250pt}%
\definecolor{currentstroke}{rgb}{0.000000,0.000000,0.000000}%
\pgfsetstrokecolor{currentstroke}%
\pgfsetdash{}{0pt}%
\pgfsys@defobject{currentmarker}{\pgfqpoint{-0.027778in}{0.000000in}}{\pgfqpoint{-0.000000in}{0.000000in}}{%
\pgfpathmoveto{\pgfqpoint{-0.000000in}{0.000000in}}%
\pgfpathlineto{\pgfqpoint{-0.027778in}{0.000000in}}%
\pgfusepath{stroke,fill}%
}%
\begin{pgfscope}%
\pgfsys@transformshift{0.661284in}{0.731859in}%
\pgfsys@useobject{currentmarker}{}%
\end{pgfscope}%
\end{pgfscope}%
\begin{pgfscope}%
\pgfpathrectangle{\pgfqpoint{0.661284in}{0.417642in}}{\pgfqpoint{3.365288in}{2.055000in}}%
\pgfusepath{clip}%
\pgfsetrectcap%
\pgfsetroundjoin%
\pgfsetlinewidth{0.803000pt}%
\definecolor{currentstroke}{rgb}{0.850000,0.850000,0.850000}%
\pgfsetstrokecolor{currentstroke}%
\pgfsetdash{}{0pt}%
\pgfpathmoveto{\pgfqpoint{0.661284in}{0.786876in}}%
\pgfpathlineto{\pgfqpoint{4.026572in}{0.786876in}}%
\pgfusepath{stroke}%
\end{pgfscope}%
\begin{pgfscope}%
\pgfsetbuttcap%
\pgfsetroundjoin%
\definecolor{currentfill}{rgb}{0.000000,0.000000,0.000000}%
\pgfsetfillcolor{currentfill}%
\pgfsetlinewidth{0.602250pt}%
\definecolor{currentstroke}{rgb}{0.000000,0.000000,0.000000}%
\pgfsetstrokecolor{currentstroke}%
\pgfsetdash{}{0pt}%
\pgfsys@defobject{currentmarker}{\pgfqpoint{-0.027778in}{0.000000in}}{\pgfqpoint{-0.000000in}{0.000000in}}{%
\pgfpathmoveto{\pgfqpoint{-0.000000in}{0.000000in}}%
\pgfpathlineto{\pgfqpoint{-0.027778in}{0.000000in}}%
\pgfusepath{stroke,fill}%
}%
\begin{pgfscope}%
\pgfsys@transformshift{0.661284in}{0.786876in}%
\pgfsys@useobject{currentmarker}{}%
\end{pgfscope}%
\end{pgfscope}%
\begin{pgfscope}%
\pgfpathrectangle{\pgfqpoint{0.661284in}{0.417642in}}{\pgfqpoint{3.365288in}{2.055000in}}%
\pgfusepath{clip}%
\pgfsetrectcap%
\pgfsetroundjoin%
\pgfsetlinewidth{0.803000pt}%
\definecolor{currentstroke}{rgb}{0.850000,0.850000,0.850000}%
\pgfsetstrokecolor{currentstroke}%
\pgfsetdash{}{0pt}%
\pgfpathmoveto{\pgfqpoint{0.661284in}{0.831828in}}%
\pgfpathlineto{\pgfqpoint{4.026572in}{0.831828in}}%
\pgfusepath{stroke}%
\end{pgfscope}%
\begin{pgfscope}%
\pgfsetbuttcap%
\pgfsetroundjoin%
\definecolor{currentfill}{rgb}{0.000000,0.000000,0.000000}%
\pgfsetfillcolor{currentfill}%
\pgfsetlinewidth{0.602250pt}%
\definecolor{currentstroke}{rgb}{0.000000,0.000000,0.000000}%
\pgfsetstrokecolor{currentstroke}%
\pgfsetdash{}{0pt}%
\pgfsys@defobject{currentmarker}{\pgfqpoint{-0.027778in}{0.000000in}}{\pgfqpoint{-0.000000in}{0.000000in}}{%
\pgfpathmoveto{\pgfqpoint{-0.000000in}{0.000000in}}%
\pgfpathlineto{\pgfqpoint{-0.027778in}{0.000000in}}%
\pgfusepath{stroke,fill}%
}%
\begin{pgfscope}%
\pgfsys@transformshift{0.661284in}{0.831828in}%
\pgfsys@useobject{currentmarker}{}%
\end{pgfscope}%
\end{pgfscope}%
\begin{pgfscope}%
\pgfpathrectangle{\pgfqpoint{0.661284in}{0.417642in}}{\pgfqpoint{3.365288in}{2.055000in}}%
\pgfusepath{clip}%
\pgfsetrectcap%
\pgfsetroundjoin%
\pgfsetlinewidth{0.803000pt}%
\definecolor{currentstroke}{rgb}{0.850000,0.850000,0.850000}%
\pgfsetstrokecolor{currentstroke}%
\pgfsetdash{}{0pt}%
\pgfpathmoveto{\pgfqpoint{0.661284in}{0.869835in}}%
\pgfpathlineto{\pgfqpoint{4.026572in}{0.869835in}}%
\pgfusepath{stroke}%
\end{pgfscope}%
\begin{pgfscope}%
\pgfsetbuttcap%
\pgfsetroundjoin%
\definecolor{currentfill}{rgb}{0.000000,0.000000,0.000000}%
\pgfsetfillcolor{currentfill}%
\pgfsetlinewidth{0.602250pt}%
\definecolor{currentstroke}{rgb}{0.000000,0.000000,0.000000}%
\pgfsetstrokecolor{currentstroke}%
\pgfsetdash{}{0pt}%
\pgfsys@defobject{currentmarker}{\pgfqpoint{-0.027778in}{0.000000in}}{\pgfqpoint{-0.000000in}{0.000000in}}{%
\pgfpathmoveto{\pgfqpoint{-0.000000in}{0.000000in}}%
\pgfpathlineto{\pgfqpoint{-0.027778in}{0.000000in}}%
\pgfusepath{stroke,fill}%
}%
\begin{pgfscope}%
\pgfsys@transformshift{0.661284in}{0.869835in}%
\pgfsys@useobject{currentmarker}{}%
\end{pgfscope}%
\end{pgfscope}%
\begin{pgfscope}%
\pgfpathrectangle{\pgfqpoint{0.661284in}{0.417642in}}{\pgfqpoint{3.365288in}{2.055000in}}%
\pgfusepath{clip}%
\pgfsetrectcap%
\pgfsetroundjoin%
\pgfsetlinewidth{0.803000pt}%
\definecolor{currentstroke}{rgb}{0.850000,0.850000,0.850000}%
\pgfsetstrokecolor{currentstroke}%
\pgfsetdash{}{0pt}%
\pgfpathmoveto{\pgfqpoint{0.661284in}{0.902758in}}%
\pgfpathlineto{\pgfqpoint{4.026572in}{0.902758in}}%
\pgfusepath{stroke}%
\end{pgfscope}%
\begin{pgfscope}%
\pgfsetbuttcap%
\pgfsetroundjoin%
\definecolor{currentfill}{rgb}{0.000000,0.000000,0.000000}%
\pgfsetfillcolor{currentfill}%
\pgfsetlinewidth{0.602250pt}%
\definecolor{currentstroke}{rgb}{0.000000,0.000000,0.000000}%
\pgfsetstrokecolor{currentstroke}%
\pgfsetdash{}{0pt}%
\pgfsys@defobject{currentmarker}{\pgfqpoint{-0.027778in}{0.000000in}}{\pgfqpoint{-0.000000in}{0.000000in}}{%
\pgfpathmoveto{\pgfqpoint{-0.000000in}{0.000000in}}%
\pgfpathlineto{\pgfqpoint{-0.027778in}{0.000000in}}%
\pgfusepath{stroke,fill}%
}%
\begin{pgfscope}%
\pgfsys@transformshift{0.661284in}{0.902758in}%
\pgfsys@useobject{currentmarker}{}%
\end{pgfscope}%
\end{pgfscope}%
\begin{pgfscope}%
\pgfpathrectangle{\pgfqpoint{0.661284in}{0.417642in}}{\pgfqpoint{3.365288in}{2.055000in}}%
\pgfusepath{clip}%
\pgfsetrectcap%
\pgfsetroundjoin%
\pgfsetlinewidth{0.803000pt}%
\definecolor{currentstroke}{rgb}{0.850000,0.850000,0.850000}%
\pgfsetstrokecolor{currentstroke}%
\pgfsetdash{}{0pt}%
\pgfpathmoveto{\pgfqpoint{0.661284in}{0.931798in}}%
\pgfpathlineto{\pgfqpoint{4.026572in}{0.931798in}}%
\pgfusepath{stroke}%
\end{pgfscope}%
\begin{pgfscope}%
\pgfsetbuttcap%
\pgfsetroundjoin%
\definecolor{currentfill}{rgb}{0.000000,0.000000,0.000000}%
\pgfsetfillcolor{currentfill}%
\pgfsetlinewidth{0.602250pt}%
\definecolor{currentstroke}{rgb}{0.000000,0.000000,0.000000}%
\pgfsetstrokecolor{currentstroke}%
\pgfsetdash{}{0pt}%
\pgfsys@defobject{currentmarker}{\pgfqpoint{-0.027778in}{0.000000in}}{\pgfqpoint{-0.000000in}{0.000000in}}{%
\pgfpathmoveto{\pgfqpoint{-0.000000in}{0.000000in}}%
\pgfpathlineto{\pgfqpoint{-0.027778in}{0.000000in}}%
\pgfusepath{stroke,fill}%
}%
\begin{pgfscope}%
\pgfsys@transformshift{0.661284in}{0.931798in}%
\pgfsys@useobject{currentmarker}{}%
\end{pgfscope}%
\end{pgfscope}%
\begin{pgfscope}%
\pgfpathrectangle{\pgfqpoint{0.661284in}{0.417642in}}{\pgfqpoint{3.365288in}{2.055000in}}%
\pgfusepath{clip}%
\pgfsetrectcap%
\pgfsetroundjoin%
\pgfsetlinewidth{0.803000pt}%
\definecolor{currentstroke}{rgb}{0.850000,0.850000,0.850000}%
\pgfsetstrokecolor{currentstroke}%
\pgfsetdash{}{0pt}%
\pgfpathmoveto{\pgfqpoint{0.661284in}{1.128674in}}%
\pgfpathlineto{\pgfqpoint{4.026572in}{1.128674in}}%
\pgfusepath{stroke}%
\end{pgfscope}%
\begin{pgfscope}%
\pgfsetbuttcap%
\pgfsetroundjoin%
\definecolor{currentfill}{rgb}{0.000000,0.000000,0.000000}%
\pgfsetfillcolor{currentfill}%
\pgfsetlinewidth{0.602250pt}%
\definecolor{currentstroke}{rgb}{0.000000,0.000000,0.000000}%
\pgfsetstrokecolor{currentstroke}%
\pgfsetdash{}{0pt}%
\pgfsys@defobject{currentmarker}{\pgfqpoint{-0.027778in}{0.000000in}}{\pgfqpoint{-0.000000in}{0.000000in}}{%
\pgfpathmoveto{\pgfqpoint{-0.000000in}{0.000000in}}%
\pgfpathlineto{\pgfqpoint{-0.027778in}{0.000000in}}%
\pgfusepath{stroke,fill}%
}%
\begin{pgfscope}%
\pgfsys@transformshift{0.661284in}{1.128674in}%
\pgfsys@useobject{currentmarker}{}%
\end{pgfscope}%
\end{pgfscope}%
\begin{pgfscope}%
\pgfpathrectangle{\pgfqpoint{0.661284in}{0.417642in}}{\pgfqpoint{3.365288in}{2.055000in}}%
\pgfusepath{clip}%
\pgfsetrectcap%
\pgfsetroundjoin%
\pgfsetlinewidth{0.803000pt}%
\definecolor{currentstroke}{rgb}{0.850000,0.850000,0.850000}%
\pgfsetstrokecolor{currentstroke}%
\pgfsetdash{}{0pt}%
\pgfpathmoveto{\pgfqpoint{0.661284in}{1.228644in}}%
\pgfpathlineto{\pgfqpoint{4.026572in}{1.228644in}}%
\pgfusepath{stroke}%
\end{pgfscope}%
\begin{pgfscope}%
\pgfsetbuttcap%
\pgfsetroundjoin%
\definecolor{currentfill}{rgb}{0.000000,0.000000,0.000000}%
\pgfsetfillcolor{currentfill}%
\pgfsetlinewidth{0.602250pt}%
\definecolor{currentstroke}{rgb}{0.000000,0.000000,0.000000}%
\pgfsetstrokecolor{currentstroke}%
\pgfsetdash{}{0pt}%
\pgfsys@defobject{currentmarker}{\pgfqpoint{-0.027778in}{0.000000in}}{\pgfqpoint{-0.000000in}{0.000000in}}{%
\pgfpathmoveto{\pgfqpoint{-0.000000in}{0.000000in}}%
\pgfpathlineto{\pgfqpoint{-0.027778in}{0.000000in}}%
\pgfusepath{stroke,fill}%
}%
\begin{pgfscope}%
\pgfsys@transformshift{0.661284in}{1.228644in}%
\pgfsys@useobject{currentmarker}{}%
\end{pgfscope}%
\end{pgfscope}%
\begin{pgfscope}%
\pgfpathrectangle{\pgfqpoint{0.661284in}{0.417642in}}{\pgfqpoint{3.365288in}{2.055000in}}%
\pgfusepath{clip}%
\pgfsetrectcap%
\pgfsetroundjoin%
\pgfsetlinewidth{0.803000pt}%
\definecolor{currentstroke}{rgb}{0.850000,0.850000,0.850000}%
\pgfsetstrokecolor{currentstroke}%
\pgfsetdash{}{0pt}%
\pgfpathmoveto{\pgfqpoint{0.661284in}{1.299574in}}%
\pgfpathlineto{\pgfqpoint{4.026572in}{1.299574in}}%
\pgfusepath{stroke}%
\end{pgfscope}%
\begin{pgfscope}%
\pgfsetbuttcap%
\pgfsetroundjoin%
\definecolor{currentfill}{rgb}{0.000000,0.000000,0.000000}%
\pgfsetfillcolor{currentfill}%
\pgfsetlinewidth{0.602250pt}%
\definecolor{currentstroke}{rgb}{0.000000,0.000000,0.000000}%
\pgfsetstrokecolor{currentstroke}%
\pgfsetdash{}{0pt}%
\pgfsys@defobject{currentmarker}{\pgfqpoint{-0.027778in}{0.000000in}}{\pgfqpoint{-0.000000in}{0.000000in}}{%
\pgfpathmoveto{\pgfqpoint{-0.000000in}{0.000000in}}%
\pgfpathlineto{\pgfqpoint{-0.027778in}{0.000000in}}%
\pgfusepath{stroke,fill}%
}%
\begin{pgfscope}%
\pgfsys@transformshift{0.661284in}{1.299574in}%
\pgfsys@useobject{currentmarker}{}%
\end{pgfscope}%
\end{pgfscope}%
\begin{pgfscope}%
\pgfpathrectangle{\pgfqpoint{0.661284in}{0.417642in}}{\pgfqpoint{3.365288in}{2.055000in}}%
\pgfusepath{clip}%
\pgfsetrectcap%
\pgfsetroundjoin%
\pgfsetlinewidth{0.803000pt}%
\definecolor{currentstroke}{rgb}{0.850000,0.850000,0.850000}%
\pgfsetstrokecolor{currentstroke}%
\pgfsetdash{}{0pt}%
\pgfpathmoveto{\pgfqpoint{0.661284in}{1.354591in}}%
\pgfpathlineto{\pgfqpoint{4.026572in}{1.354591in}}%
\pgfusepath{stroke}%
\end{pgfscope}%
\begin{pgfscope}%
\pgfsetbuttcap%
\pgfsetroundjoin%
\definecolor{currentfill}{rgb}{0.000000,0.000000,0.000000}%
\pgfsetfillcolor{currentfill}%
\pgfsetlinewidth{0.602250pt}%
\definecolor{currentstroke}{rgb}{0.000000,0.000000,0.000000}%
\pgfsetstrokecolor{currentstroke}%
\pgfsetdash{}{0pt}%
\pgfsys@defobject{currentmarker}{\pgfqpoint{-0.027778in}{0.000000in}}{\pgfqpoint{-0.000000in}{0.000000in}}{%
\pgfpathmoveto{\pgfqpoint{-0.000000in}{0.000000in}}%
\pgfpathlineto{\pgfqpoint{-0.027778in}{0.000000in}}%
\pgfusepath{stroke,fill}%
}%
\begin{pgfscope}%
\pgfsys@transformshift{0.661284in}{1.354591in}%
\pgfsys@useobject{currentmarker}{}%
\end{pgfscope}%
\end{pgfscope}%
\begin{pgfscope}%
\pgfpathrectangle{\pgfqpoint{0.661284in}{0.417642in}}{\pgfqpoint{3.365288in}{2.055000in}}%
\pgfusepath{clip}%
\pgfsetrectcap%
\pgfsetroundjoin%
\pgfsetlinewidth{0.803000pt}%
\definecolor{currentstroke}{rgb}{0.850000,0.850000,0.850000}%
\pgfsetstrokecolor{currentstroke}%
\pgfsetdash{}{0pt}%
\pgfpathmoveto{\pgfqpoint{0.661284in}{1.399543in}}%
\pgfpathlineto{\pgfqpoint{4.026572in}{1.399543in}}%
\pgfusepath{stroke}%
\end{pgfscope}%
\begin{pgfscope}%
\pgfsetbuttcap%
\pgfsetroundjoin%
\definecolor{currentfill}{rgb}{0.000000,0.000000,0.000000}%
\pgfsetfillcolor{currentfill}%
\pgfsetlinewidth{0.602250pt}%
\definecolor{currentstroke}{rgb}{0.000000,0.000000,0.000000}%
\pgfsetstrokecolor{currentstroke}%
\pgfsetdash{}{0pt}%
\pgfsys@defobject{currentmarker}{\pgfqpoint{-0.027778in}{0.000000in}}{\pgfqpoint{-0.000000in}{0.000000in}}{%
\pgfpathmoveto{\pgfqpoint{-0.000000in}{0.000000in}}%
\pgfpathlineto{\pgfqpoint{-0.027778in}{0.000000in}}%
\pgfusepath{stroke,fill}%
}%
\begin{pgfscope}%
\pgfsys@transformshift{0.661284in}{1.399543in}%
\pgfsys@useobject{currentmarker}{}%
\end{pgfscope}%
\end{pgfscope}%
\begin{pgfscope}%
\pgfpathrectangle{\pgfqpoint{0.661284in}{0.417642in}}{\pgfqpoint{3.365288in}{2.055000in}}%
\pgfusepath{clip}%
\pgfsetrectcap%
\pgfsetroundjoin%
\pgfsetlinewidth{0.803000pt}%
\definecolor{currentstroke}{rgb}{0.850000,0.850000,0.850000}%
\pgfsetstrokecolor{currentstroke}%
\pgfsetdash{}{0pt}%
\pgfpathmoveto{\pgfqpoint{0.661284in}{1.437550in}}%
\pgfpathlineto{\pgfqpoint{4.026572in}{1.437550in}}%
\pgfusepath{stroke}%
\end{pgfscope}%
\begin{pgfscope}%
\pgfsetbuttcap%
\pgfsetroundjoin%
\definecolor{currentfill}{rgb}{0.000000,0.000000,0.000000}%
\pgfsetfillcolor{currentfill}%
\pgfsetlinewidth{0.602250pt}%
\definecolor{currentstroke}{rgb}{0.000000,0.000000,0.000000}%
\pgfsetstrokecolor{currentstroke}%
\pgfsetdash{}{0pt}%
\pgfsys@defobject{currentmarker}{\pgfqpoint{-0.027778in}{0.000000in}}{\pgfqpoint{-0.000000in}{0.000000in}}{%
\pgfpathmoveto{\pgfqpoint{-0.000000in}{0.000000in}}%
\pgfpathlineto{\pgfqpoint{-0.027778in}{0.000000in}}%
\pgfusepath{stroke,fill}%
}%
\begin{pgfscope}%
\pgfsys@transformshift{0.661284in}{1.437550in}%
\pgfsys@useobject{currentmarker}{}%
\end{pgfscope}%
\end{pgfscope}%
\begin{pgfscope}%
\pgfpathrectangle{\pgfqpoint{0.661284in}{0.417642in}}{\pgfqpoint{3.365288in}{2.055000in}}%
\pgfusepath{clip}%
\pgfsetrectcap%
\pgfsetroundjoin%
\pgfsetlinewidth{0.803000pt}%
\definecolor{currentstroke}{rgb}{0.850000,0.850000,0.850000}%
\pgfsetstrokecolor{currentstroke}%
\pgfsetdash{}{0pt}%
\pgfpathmoveto{\pgfqpoint{0.661284in}{1.470473in}}%
\pgfpathlineto{\pgfqpoint{4.026572in}{1.470473in}}%
\pgfusepath{stroke}%
\end{pgfscope}%
\begin{pgfscope}%
\pgfsetbuttcap%
\pgfsetroundjoin%
\definecolor{currentfill}{rgb}{0.000000,0.000000,0.000000}%
\pgfsetfillcolor{currentfill}%
\pgfsetlinewidth{0.602250pt}%
\definecolor{currentstroke}{rgb}{0.000000,0.000000,0.000000}%
\pgfsetstrokecolor{currentstroke}%
\pgfsetdash{}{0pt}%
\pgfsys@defobject{currentmarker}{\pgfqpoint{-0.027778in}{0.000000in}}{\pgfqpoint{-0.000000in}{0.000000in}}{%
\pgfpathmoveto{\pgfqpoint{-0.000000in}{0.000000in}}%
\pgfpathlineto{\pgfqpoint{-0.027778in}{0.000000in}}%
\pgfusepath{stroke,fill}%
}%
\begin{pgfscope}%
\pgfsys@transformshift{0.661284in}{1.470473in}%
\pgfsys@useobject{currentmarker}{}%
\end{pgfscope}%
\end{pgfscope}%
\begin{pgfscope}%
\pgfpathrectangle{\pgfqpoint{0.661284in}{0.417642in}}{\pgfqpoint{3.365288in}{2.055000in}}%
\pgfusepath{clip}%
\pgfsetrectcap%
\pgfsetroundjoin%
\pgfsetlinewidth{0.803000pt}%
\definecolor{currentstroke}{rgb}{0.850000,0.850000,0.850000}%
\pgfsetstrokecolor{currentstroke}%
\pgfsetdash{}{0pt}%
\pgfpathmoveto{\pgfqpoint{0.661284in}{1.499513in}}%
\pgfpathlineto{\pgfqpoint{4.026572in}{1.499513in}}%
\pgfusepath{stroke}%
\end{pgfscope}%
\begin{pgfscope}%
\pgfsetbuttcap%
\pgfsetroundjoin%
\definecolor{currentfill}{rgb}{0.000000,0.000000,0.000000}%
\pgfsetfillcolor{currentfill}%
\pgfsetlinewidth{0.602250pt}%
\definecolor{currentstroke}{rgb}{0.000000,0.000000,0.000000}%
\pgfsetstrokecolor{currentstroke}%
\pgfsetdash{}{0pt}%
\pgfsys@defobject{currentmarker}{\pgfqpoint{-0.027778in}{0.000000in}}{\pgfqpoint{-0.000000in}{0.000000in}}{%
\pgfpathmoveto{\pgfqpoint{-0.000000in}{0.000000in}}%
\pgfpathlineto{\pgfqpoint{-0.027778in}{0.000000in}}%
\pgfusepath{stroke,fill}%
}%
\begin{pgfscope}%
\pgfsys@transformshift{0.661284in}{1.499513in}%
\pgfsys@useobject{currentmarker}{}%
\end{pgfscope}%
\end{pgfscope}%
\begin{pgfscope}%
\pgfpathrectangle{\pgfqpoint{0.661284in}{0.417642in}}{\pgfqpoint{3.365288in}{2.055000in}}%
\pgfusepath{clip}%
\pgfsetrectcap%
\pgfsetroundjoin%
\pgfsetlinewidth{0.803000pt}%
\definecolor{currentstroke}{rgb}{0.850000,0.850000,0.850000}%
\pgfsetstrokecolor{currentstroke}%
\pgfsetdash{}{0pt}%
\pgfpathmoveto{\pgfqpoint{0.661284in}{1.696390in}}%
\pgfpathlineto{\pgfqpoint{4.026572in}{1.696390in}}%
\pgfusepath{stroke}%
\end{pgfscope}%
\begin{pgfscope}%
\pgfsetbuttcap%
\pgfsetroundjoin%
\definecolor{currentfill}{rgb}{0.000000,0.000000,0.000000}%
\pgfsetfillcolor{currentfill}%
\pgfsetlinewidth{0.602250pt}%
\definecolor{currentstroke}{rgb}{0.000000,0.000000,0.000000}%
\pgfsetstrokecolor{currentstroke}%
\pgfsetdash{}{0pt}%
\pgfsys@defobject{currentmarker}{\pgfqpoint{-0.027778in}{0.000000in}}{\pgfqpoint{-0.000000in}{0.000000in}}{%
\pgfpathmoveto{\pgfqpoint{-0.000000in}{0.000000in}}%
\pgfpathlineto{\pgfqpoint{-0.027778in}{0.000000in}}%
\pgfusepath{stroke,fill}%
}%
\begin{pgfscope}%
\pgfsys@transformshift{0.661284in}{1.696390in}%
\pgfsys@useobject{currentmarker}{}%
\end{pgfscope}%
\end{pgfscope}%
\begin{pgfscope}%
\pgfpathrectangle{\pgfqpoint{0.661284in}{0.417642in}}{\pgfqpoint{3.365288in}{2.055000in}}%
\pgfusepath{clip}%
\pgfsetrectcap%
\pgfsetroundjoin%
\pgfsetlinewidth{0.803000pt}%
\definecolor{currentstroke}{rgb}{0.850000,0.850000,0.850000}%
\pgfsetstrokecolor{currentstroke}%
\pgfsetdash{}{0pt}%
\pgfpathmoveto{\pgfqpoint{0.661284in}{1.796359in}}%
\pgfpathlineto{\pgfqpoint{4.026572in}{1.796359in}}%
\pgfusepath{stroke}%
\end{pgfscope}%
\begin{pgfscope}%
\pgfsetbuttcap%
\pgfsetroundjoin%
\definecolor{currentfill}{rgb}{0.000000,0.000000,0.000000}%
\pgfsetfillcolor{currentfill}%
\pgfsetlinewidth{0.602250pt}%
\definecolor{currentstroke}{rgb}{0.000000,0.000000,0.000000}%
\pgfsetstrokecolor{currentstroke}%
\pgfsetdash{}{0pt}%
\pgfsys@defobject{currentmarker}{\pgfqpoint{-0.027778in}{0.000000in}}{\pgfqpoint{-0.000000in}{0.000000in}}{%
\pgfpathmoveto{\pgfqpoint{-0.000000in}{0.000000in}}%
\pgfpathlineto{\pgfqpoint{-0.027778in}{0.000000in}}%
\pgfusepath{stroke,fill}%
}%
\begin{pgfscope}%
\pgfsys@transformshift{0.661284in}{1.796359in}%
\pgfsys@useobject{currentmarker}{}%
\end{pgfscope}%
\end{pgfscope}%
\begin{pgfscope}%
\pgfpathrectangle{\pgfqpoint{0.661284in}{0.417642in}}{\pgfqpoint{3.365288in}{2.055000in}}%
\pgfusepath{clip}%
\pgfsetrectcap%
\pgfsetroundjoin%
\pgfsetlinewidth{0.803000pt}%
\definecolor{currentstroke}{rgb}{0.850000,0.850000,0.850000}%
\pgfsetstrokecolor{currentstroke}%
\pgfsetdash{}{0pt}%
\pgfpathmoveto{\pgfqpoint{0.661284in}{1.867289in}}%
\pgfpathlineto{\pgfqpoint{4.026572in}{1.867289in}}%
\pgfusepath{stroke}%
\end{pgfscope}%
\begin{pgfscope}%
\pgfsetbuttcap%
\pgfsetroundjoin%
\definecolor{currentfill}{rgb}{0.000000,0.000000,0.000000}%
\pgfsetfillcolor{currentfill}%
\pgfsetlinewidth{0.602250pt}%
\definecolor{currentstroke}{rgb}{0.000000,0.000000,0.000000}%
\pgfsetstrokecolor{currentstroke}%
\pgfsetdash{}{0pt}%
\pgfsys@defobject{currentmarker}{\pgfqpoint{-0.027778in}{0.000000in}}{\pgfqpoint{-0.000000in}{0.000000in}}{%
\pgfpathmoveto{\pgfqpoint{-0.000000in}{0.000000in}}%
\pgfpathlineto{\pgfqpoint{-0.027778in}{0.000000in}}%
\pgfusepath{stroke,fill}%
}%
\begin{pgfscope}%
\pgfsys@transformshift{0.661284in}{1.867289in}%
\pgfsys@useobject{currentmarker}{}%
\end{pgfscope}%
\end{pgfscope}%
\begin{pgfscope}%
\pgfpathrectangle{\pgfqpoint{0.661284in}{0.417642in}}{\pgfqpoint{3.365288in}{2.055000in}}%
\pgfusepath{clip}%
\pgfsetrectcap%
\pgfsetroundjoin%
\pgfsetlinewidth{0.803000pt}%
\definecolor{currentstroke}{rgb}{0.850000,0.850000,0.850000}%
\pgfsetstrokecolor{currentstroke}%
\pgfsetdash{}{0pt}%
\pgfpathmoveto{\pgfqpoint{0.661284in}{1.922306in}}%
\pgfpathlineto{\pgfqpoint{4.026572in}{1.922306in}}%
\pgfusepath{stroke}%
\end{pgfscope}%
\begin{pgfscope}%
\pgfsetbuttcap%
\pgfsetroundjoin%
\definecolor{currentfill}{rgb}{0.000000,0.000000,0.000000}%
\pgfsetfillcolor{currentfill}%
\pgfsetlinewidth{0.602250pt}%
\definecolor{currentstroke}{rgb}{0.000000,0.000000,0.000000}%
\pgfsetstrokecolor{currentstroke}%
\pgfsetdash{}{0pt}%
\pgfsys@defobject{currentmarker}{\pgfqpoint{-0.027778in}{0.000000in}}{\pgfqpoint{-0.000000in}{0.000000in}}{%
\pgfpathmoveto{\pgfqpoint{-0.000000in}{0.000000in}}%
\pgfpathlineto{\pgfqpoint{-0.027778in}{0.000000in}}%
\pgfusepath{stroke,fill}%
}%
\begin{pgfscope}%
\pgfsys@transformshift{0.661284in}{1.922306in}%
\pgfsys@useobject{currentmarker}{}%
\end{pgfscope}%
\end{pgfscope}%
\begin{pgfscope}%
\pgfpathrectangle{\pgfqpoint{0.661284in}{0.417642in}}{\pgfqpoint{3.365288in}{2.055000in}}%
\pgfusepath{clip}%
\pgfsetrectcap%
\pgfsetroundjoin%
\pgfsetlinewidth{0.803000pt}%
\definecolor{currentstroke}{rgb}{0.850000,0.850000,0.850000}%
\pgfsetstrokecolor{currentstroke}%
\pgfsetdash{}{0pt}%
\pgfpathmoveto{\pgfqpoint{0.661284in}{1.967259in}}%
\pgfpathlineto{\pgfqpoint{4.026572in}{1.967259in}}%
\pgfusepath{stroke}%
\end{pgfscope}%
\begin{pgfscope}%
\pgfsetbuttcap%
\pgfsetroundjoin%
\definecolor{currentfill}{rgb}{0.000000,0.000000,0.000000}%
\pgfsetfillcolor{currentfill}%
\pgfsetlinewidth{0.602250pt}%
\definecolor{currentstroke}{rgb}{0.000000,0.000000,0.000000}%
\pgfsetstrokecolor{currentstroke}%
\pgfsetdash{}{0pt}%
\pgfsys@defobject{currentmarker}{\pgfqpoint{-0.027778in}{0.000000in}}{\pgfqpoint{-0.000000in}{0.000000in}}{%
\pgfpathmoveto{\pgfqpoint{-0.000000in}{0.000000in}}%
\pgfpathlineto{\pgfqpoint{-0.027778in}{0.000000in}}%
\pgfusepath{stroke,fill}%
}%
\begin{pgfscope}%
\pgfsys@transformshift{0.661284in}{1.967259in}%
\pgfsys@useobject{currentmarker}{}%
\end{pgfscope}%
\end{pgfscope}%
\begin{pgfscope}%
\pgfpathrectangle{\pgfqpoint{0.661284in}{0.417642in}}{\pgfqpoint{3.365288in}{2.055000in}}%
\pgfusepath{clip}%
\pgfsetrectcap%
\pgfsetroundjoin%
\pgfsetlinewidth{0.803000pt}%
\definecolor{currentstroke}{rgb}{0.850000,0.850000,0.850000}%
\pgfsetstrokecolor{currentstroke}%
\pgfsetdash{}{0pt}%
\pgfpathmoveto{\pgfqpoint{0.661284in}{2.005265in}}%
\pgfpathlineto{\pgfqpoint{4.026572in}{2.005265in}}%
\pgfusepath{stroke}%
\end{pgfscope}%
\begin{pgfscope}%
\pgfsetbuttcap%
\pgfsetroundjoin%
\definecolor{currentfill}{rgb}{0.000000,0.000000,0.000000}%
\pgfsetfillcolor{currentfill}%
\pgfsetlinewidth{0.602250pt}%
\definecolor{currentstroke}{rgb}{0.000000,0.000000,0.000000}%
\pgfsetstrokecolor{currentstroke}%
\pgfsetdash{}{0pt}%
\pgfsys@defobject{currentmarker}{\pgfqpoint{-0.027778in}{0.000000in}}{\pgfqpoint{-0.000000in}{0.000000in}}{%
\pgfpathmoveto{\pgfqpoint{-0.000000in}{0.000000in}}%
\pgfpathlineto{\pgfqpoint{-0.027778in}{0.000000in}}%
\pgfusepath{stroke,fill}%
}%
\begin{pgfscope}%
\pgfsys@transformshift{0.661284in}{2.005265in}%
\pgfsys@useobject{currentmarker}{}%
\end{pgfscope}%
\end{pgfscope}%
\begin{pgfscope}%
\pgfpathrectangle{\pgfqpoint{0.661284in}{0.417642in}}{\pgfqpoint{3.365288in}{2.055000in}}%
\pgfusepath{clip}%
\pgfsetrectcap%
\pgfsetroundjoin%
\pgfsetlinewidth{0.803000pt}%
\definecolor{currentstroke}{rgb}{0.850000,0.850000,0.850000}%
\pgfsetstrokecolor{currentstroke}%
\pgfsetdash{}{0pt}%
\pgfpathmoveto{\pgfqpoint{0.661284in}{2.038188in}}%
\pgfpathlineto{\pgfqpoint{4.026572in}{2.038188in}}%
\pgfusepath{stroke}%
\end{pgfscope}%
\begin{pgfscope}%
\pgfsetbuttcap%
\pgfsetroundjoin%
\definecolor{currentfill}{rgb}{0.000000,0.000000,0.000000}%
\pgfsetfillcolor{currentfill}%
\pgfsetlinewidth{0.602250pt}%
\definecolor{currentstroke}{rgb}{0.000000,0.000000,0.000000}%
\pgfsetstrokecolor{currentstroke}%
\pgfsetdash{}{0pt}%
\pgfsys@defobject{currentmarker}{\pgfqpoint{-0.027778in}{0.000000in}}{\pgfqpoint{-0.000000in}{0.000000in}}{%
\pgfpathmoveto{\pgfqpoint{-0.000000in}{0.000000in}}%
\pgfpathlineto{\pgfqpoint{-0.027778in}{0.000000in}}%
\pgfusepath{stroke,fill}%
}%
\begin{pgfscope}%
\pgfsys@transformshift{0.661284in}{2.038188in}%
\pgfsys@useobject{currentmarker}{}%
\end{pgfscope}%
\end{pgfscope}%
\begin{pgfscope}%
\pgfpathrectangle{\pgfqpoint{0.661284in}{0.417642in}}{\pgfqpoint{3.365288in}{2.055000in}}%
\pgfusepath{clip}%
\pgfsetrectcap%
\pgfsetroundjoin%
\pgfsetlinewidth{0.803000pt}%
\definecolor{currentstroke}{rgb}{0.850000,0.850000,0.850000}%
\pgfsetstrokecolor{currentstroke}%
\pgfsetdash{}{0pt}%
\pgfpathmoveto{\pgfqpoint{0.661284in}{2.067228in}}%
\pgfpathlineto{\pgfqpoint{4.026572in}{2.067228in}}%
\pgfusepath{stroke}%
\end{pgfscope}%
\begin{pgfscope}%
\pgfsetbuttcap%
\pgfsetroundjoin%
\definecolor{currentfill}{rgb}{0.000000,0.000000,0.000000}%
\pgfsetfillcolor{currentfill}%
\pgfsetlinewidth{0.602250pt}%
\definecolor{currentstroke}{rgb}{0.000000,0.000000,0.000000}%
\pgfsetstrokecolor{currentstroke}%
\pgfsetdash{}{0pt}%
\pgfsys@defobject{currentmarker}{\pgfqpoint{-0.027778in}{0.000000in}}{\pgfqpoint{-0.000000in}{0.000000in}}{%
\pgfpathmoveto{\pgfqpoint{-0.000000in}{0.000000in}}%
\pgfpathlineto{\pgfqpoint{-0.027778in}{0.000000in}}%
\pgfusepath{stroke,fill}%
}%
\begin{pgfscope}%
\pgfsys@transformshift{0.661284in}{2.067228in}%
\pgfsys@useobject{currentmarker}{}%
\end{pgfscope}%
\end{pgfscope}%
\begin{pgfscope}%
\pgfpathrectangle{\pgfqpoint{0.661284in}{0.417642in}}{\pgfqpoint{3.365288in}{2.055000in}}%
\pgfusepath{clip}%
\pgfsetrectcap%
\pgfsetroundjoin%
\pgfsetlinewidth{0.803000pt}%
\definecolor{currentstroke}{rgb}{0.850000,0.850000,0.850000}%
\pgfsetstrokecolor{currentstroke}%
\pgfsetdash{}{0pt}%
\pgfpathmoveto{\pgfqpoint{0.661284in}{2.264105in}}%
\pgfpathlineto{\pgfqpoint{4.026572in}{2.264105in}}%
\pgfusepath{stroke}%
\end{pgfscope}%
\begin{pgfscope}%
\pgfsetbuttcap%
\pgfsetroundjoin%
\definecolor{currentfill}{rgb}{0.000000,0.000000,0.000000}%
\pgfsetfillcolor{currentfill}%
\pgfsetlinewidth{0.602250pt}%
\definecolor{currentstroke}{rgb}{0.000000,0.000000,0.000000}%
\pgfsetstrokecolor{currentstroke}%
\pgfsetdash{}{0pt}%
\pgfsys@defobject{currentmarker}{\pgfqpoint{-0.027778in}{0.000000in}}{\pgfqpoint{-0.000000in}{0.000000in}}{%
\pgfpathmoveto{\pgfqpoint{-0.000000in}{0.000000in}}%
\pgfpathlineto{\pgfqpoint{-0.027778in}{0.000000in}}%
\pgfusepath{stroke,fill}%
}%
\begin{pgfscope}%
\pgfsys@transformshift{0.661284in}{2.264105in}%
\pgfsys@useobject{currentmarker}{}%
\end{pgfscope}%
\end{pgfscope}%
\begin{pgfscope}%
\pgfpathrectangle{\pgfqpoint{0.661284in}{0.417642in}}{\pgfqpoint{3.365288in}{2.055000in}}%
\pgfusepath{clip}%
\pgfsetrectcap%
\pgfsetroundjoin%
\pgfsetlinewidth{0.803000pt}%
\definecolor{currentstroke}{rgb}{0.850000,0.850000,0.850000}%
\pgfsetstrokecolor{currentstroke}%
\pgfsetdash{}{0pt}%
\pgfpathmoveto{\pgfqpoint{0.661284in}{2.364074in}}%
\pgfpathlineto{\pgfqpoint{4.026572in}{2.364074in}}%
\pgfusepath{stroke}%
\end{pgfscope}%
\begin{pgfscope}%
\pgfsetbuttcap%
\pgfsetroundjoin%
\definecolor{currentfill}{rgb}{0.000000,0.000000,0.000000}%
\pgfsetfillcolor{currentfill}%
\pgfsetlinewidth{0.602250pt}%
\definecolor{currentstroke}{rgb}{0.000000,0.000000,0.000000}%
\pgfsetstrokecolor{currentstroke}%
\pgfsetdash{}{0pt}%
\pgfsys@defobject{currentmarker}{\pgfqpoint{-0.027778in}{0.000000in}}{\pgfqpoint{-0.000000in}{0.000000in}}{%
\pgfpathmoveto{\pgfqpoint{-0.000000in}{0.000000in}}%
\pgfpathlineto{\pgfqpoint{-0.027778in}{0.000000in}}%
\pgfusepath{stroke,fill}%
}%
\begin{pgfscope}%
\pgfsys@transformshift{0.661284in}{2.364074in}%
\pgfsys@useobject{currentmarker}{}%
\end{pgfscope}%
\end{pgfscope}%
\begin{pgfscope}%
\pgfpathrectangle{\pgfqpoint{0.661284in}{0.417642in}}{\pgfqpoint{3.365288in}{2.055000in}}%
\pgfusepath{clip}%
\pgfsetrectcap%
\pgfsetroundjoin%
\pgfsetlinewidth{0.803000pt}%
\definecolor{currentstroke}{rgb}{0.850000,0.850000,0.850000}%
\pgfsetstrokecolor{currentstroke}%
\pgfsetdash{}{0pt}%
\pgfpathmoveto{\pgfqpoint{0.661284in}{2.435004in}}%
\pgfpathlineto{\pgfqpoint{4.026572in}{2.435004in}}%
\pgfusepath{stroke}%
\end{pgfscope}%
\begin{pgfscope}%
\pgfsetbuttcap%
\pgfsetroundjoin%
\definecolor{currentfill}{rgb}{0.000000,0.000000,0.000000}%
\pgfsetfillcolor{currentfill}%
\pgfsetlinewidth{0.602250pt}%
\definecolor{currentstroke}{rgb}{0.000000,0.000000,0.000000}%
\pgfsetstrokecolor{currentstroke}%
\pgfsetdash{}{0pt}%
\pgfsys@defobject{currentmarker}{\pgfqpoint{-0.027778in}{0.000000in}}{\pgfqpoint{-0.000000in}{0.000000in}}{%
\pgfpathmoveto{\pgfqpoint{-0.000000in}{0.000000in}}%
\pgfpathlineto{\pgfqpoint{-0.027778in}{0.000000in}}%
\pgfusepath{stroke,fill}%
}%
\begin{pgfscope}%
\pgfsys@transformshift{0.661284in}{2.435004in}%
\pgfsys@useobject{currentmarker}{}%
\end{pgfscope}%
\end{pgfscope}%
\begin{pgfscope}%
\definecolor{textcolor}{rgb}{0.000000,0.000000,0.000000}%
\pgfsetstrokecolor{textcolor}%
\pgfsetfillcolor{textcolor}%
\pgftext[x=0.201408in,y=1.445142in,,bottom,rotate=90.000000]{\color{textcolor}\rmfamily\fontsize{10.000000}{12.000000}\selectfont  \(\displaystyle S_y(f)\) in \(\displaystyle \unit{\V^2 \per \Hz}\)}%
\end{pgfscope}%
\begin{pgfscope}%
\pgfpathrectangle{\pgfqpoint{0.661284in}{0.417642in}}{\pgfqpoint{3.365288in}{2.055000in}}%
\pgfusepath{clip}%
\pgfsetbuttcap%
\pgfsetroundjoin%
\definecolor{currentfill}{rgb}{0.835294,0.368627,0.000000}%
\pgfsetfillcolor{currentfill}%
\pgfsetlinewidth{1.003750pt}%
\definecolor{currentstroke}{rgb}{0.835294,0.368627,0.000000}%
\pgfsetstrokecolor{currentstroke}%
\pgfsetdash{}{0pt}%
\pgfsys@defobject{currentmarker}{\pgfqpoint{-0.013889in}{-0.013889in}}{\pgfqpoint{0.013889in}{0.013889in}}{%
\pgfpathmoveto{\pgfqpoint{0.000000in}{-0.013889in}}%
\pgfpathcurveto{\pgfqpoint{0.003683in}{-0.013889in}}{\pgfqpoint{0.007216in}{-0.012425in}}{\pgfqpoint{0.009821in}{-0.009821in}}%
\pgfpathcurveto{\pgfqpoint{0.012425in}{-0.007216in}}{\pgfqpoint{0.013889in}{-0.003683in}}{\pgfqpoint{0.013889in}{0.000000in}}%
\pgfpathcurveto{\pgfqpoint{0.013889in}{0.003683in}}{\pgfqpoint{0.012425in}{0.007216in}}{\pgfqpoint{0.009821in}{0.009821in}}%
\pgfpathcurveto{\pgfqpoint{0.007216in}{0.012425in}}{\pgfqpoint{0.003683in}{0.013889in}}{\pgfqpoint{0.000000in}{0.013889in}}%
\pgfpathcurveto{\pgfqpoint{-0.003683in}{0.013889in}}{\pgfqpoint{-0.007216in}{0.012425in}}{\pgfqpoint{-0.009821in}{0.009821in}}%
\pgfpathcurveto{\pgfqpoint{-0.012425in}{0.007216in}}{\pgfqpoint{-0.013889in}{0.003683in}}{\pgfqpoint{-0.013889in}{0.000000in}}%
\pgfpathcurveto{\pgfqpoint{-0.013889in}{-0.003683in}}{\pgfqpoint{-0.012425in}{-0.007216in}}{\pgfqpoint{-0.009821in}{-0.009821in}}%
\pgfpathcurveto{\pgfqpoint{-0.007216in}{-0.012425in}}{\pgfqpoint{-0.003683in}{-0.013889in}}{\pgfqpoint{0.000000in}{-0.013889in}}%
\pgfpathlineto{\pgfqpoint{0.000000in}{-0.013889in}}%
\pgfpathclose%
\pgfusepath{stroke,fill}%
}%
\begin{pgfscope}%
\pgfsys@transformshift{-226.701573in}{2.212133in}%
\pgfsys@useobject{currentmarker}{}%
\end{pgfscope}%
\begin{pgfscope}%
\pgfsys@transformshift{0.814251in}{2.310535in}%
\pgfsys@useobject{currentmarker}{}%
\end{pgfscope}%
\begin{pgfscope}%
\pgfsys@transformshift{1.094118in}{2.228192in}%
\pgfsys@useobject{currentmarker}{}%
\end{pgfscope}%
\begin{pgfscope}%
\pgfsys@transformshift{1.257829in}{2.108021in}%
\pgfsys@useobject{currentmarker}{}%
\end{pgfscope}%
\begin{pgfscope}%
\pgfsys@transformshift{1.373984in}{2.030681in}%
\pgfsys@useobject{currentmarker}{}%
\end{pgfscope}%
\begin{pgfscope}%
\pgfsys@transformshift{1.464081in}{1.954583in}%
\pgfsys@useobject{currentmarker}{}%
\end{pgfscope}%
\begin{pgfscope}%
\pgfsys@transformshift{1.537695in}{1.913672in}%
\pgfsys@useobject{currentmarker}{}%
\end{pgfscope}%
\begin{pgfscope}%
\pgfsys@transformshift{1.599936in}{1.892939in}%
\pgfsys@useobject{currentmarker}{}%
\end{pgfscope}%
\begin{pgfscope}%
\pgfsys@transformshift{1.653850in}{1.853194in}%
\pgfsys@useobject{currentmarker}{}%
\end{pgfscope}%
\begin{pgfscope}%
\pgfsys@transformshift{1.701407in}{1.827688in}%
\pgfsys@useobject{currentmarker}{}%
\end{pgfscope}%
\begin{pgfscope}%
\pgfsys@transformshift{1.743947in}{1.823095in}%
\pgfsys@useobject{currentmarker}{}%
\end{pgfscope}%
\begin{pgfscope}%
\pgfsys@transformshift{1.782430in}{1.808177in}%
\pgfsys@useobject{currentmarker}{}%
\end{pgfscope}%
\begin{pgfscope}%
\pgfsys@transformshift{1.817562in}{1.791153in}%
\pgfsys@useobject{currentmarker}{}%
\end{pgfscope}%
\begin{pgfscope}%
\pgfsys@transformshift{1.849880in}{1.770742in}%
\pgfsys@useobject{currentmarker}{}%
\end{pgfscope}%
\begin{pgfscope}%
\pgfsys@transformshift{1.879802in}{1.736884in}%
\pgfsys@useobject{currentmarker}{}%
\end{pgfscope}%
\begin{pgfscope}%
\pgfsys@transformshift{1.907659in}{1.722085in}%
\pgfsys@useobject{currentmarker}{}%
\end{pgfscope}%
\begin{pgfscope}%
\pgfsys@transformshift{1.933717in}{1.693381in}%
\pgfsys@useobject{currentmarker}{}%
\end{pgfscope}%
\begin{pgfscope}%
\pgfsys@transformshift{1.958195in}{1.677993in}%
\pgfsys@useobject{currentmarker}{}%
\end{pgfscope}%
\begin{pgfscope}%
\pgfsys@transformshift{1.981273in}{1.654238in}%
\pgfsys@useobject{currentmarker}{}%
\end{pgfscope}%
\begin{pgfscope}%
\pgfsys@transformshift{2.003103in}{1.655341in}%
\pgfsys@useobject{currentmarker}{}%
\end{pgfscope}%
\begin{pgfscope}%
\pgfsys@transformshift{2.023814in}{1.661693in}%
\pgfsys@useobject{currentmarker}{}%
\end{pgfscope}%
\begin{pgfscope}%
\pgfsys@transformshift{2.043513in}{1.656601in}%
\pgfsys@useobject{currentmarker}{}%
\end{pgfscope}%
\begin{pgfscope}%
\pgfsys@transformshift{2.062296in}{1.641389in}%
\pgfsys@useobject{currentmarker}{}%
\end{pgfscope}%
\begin{pgfscope}%
\pgfsys@transformshift{2.080244in}{1.616631in}%
\pgfsys@useobject{currentmarker}{}%
\end{pgfscope}%
\begin{pgfscope}%
\pgfsys@transformshift{2.097428in}{1.610597in}%
\pgfsys@useobject{currentmarker}{}%
\end{pgfscope}%
\begin{pgfscope}%
\pgfsys@transformshift{2.113910in}{1.596785in}%
\pgfsys@useobject{currentmarker}{}%
\end{pgfscope}%
\begin{pgfscope}%
\pgfsys@transformshift{2.129746in}{1.571246in}%
\pgfsys@useobject{currentmarker}{}%
\end{pgfscope}%
\begin{pgfscope}%
\pgfsys@transformshift{2.144984in}{1.564287in}%
\pgfsys@useobject{currentmarker}{}%
\end{pgfscope}%
\begin{pgfscope}%
\pgfsys@transformshift{2.159668in}{1.562790in}%
\pgfsys@useobject{currentmarker}{}%
\end{pgfscope}%
\begin{pgfscope}%
\pgfsys@transformshift{2.173837in}{1.552798in}%
\pgfsys@useobject{currentmarker}{}%
\end{pgfscope}%
\begin{pgfscope}%
\pgfsys@transformshift{2.187525in}{1.550814in}%
\pgfsys@useobject{currentmarker}{}%
\end{pgfscope}%
\begin{pgfscope}%
\pgfsys@transformshift{2.200764in}{1.549961in}%
\pgfsys@useobject{currentmarker}{}%
\end{pgfscope}%
\begin{pgfscope}%
\pgfsys@transformshift{2.213583in}{1.527907in}%
\pgfsys@useobject{currentmarker}{}%
\end{pgfscope}%
\begin{pgfscope}%
\pgfsys@transformshift{2.226008in}{1.533921in}%
\pgfsys@useobject{currentmarker}{}%
\end{pgfscope}%
\begin{pgfscope}%
\pgfsys@transformshift{2.238061in}{1.510540in}%
\pgfsys@useobject{currentmarker}{}%
\end{pgfscope}%
\begin{pgfscope}%
\pgfsys@transformshift{2.249765in}{1.487995in}%
\pgfsys@useobject{currentmarker}{}%
\end{pgfscope}%
\begin{pgfscope}%
\pgfsys@transformshift{2.261139in}{1.485635in}%
\pgfsys@useobject{currentmarker}{}%
\end{pgfscope}%
\begin{pgfscope}%
\pgfsys@transformshift{2.272202in}{1.483556in}%
\pgfsys@useobject{currentmarker}{}%
\end{pgfscope}%
\begin{pgfscope}%
\pgfsys@transformshift{2.282970in}{1.465340in}%
\pgfsys@useobject{currentmarker}{}%
\end{pgfscope}%
\begin{pgfscope}%
\pgfsys@transformshift{2.293458in}{1.490102in}%
\pgfsys@useobject{currentmarker}{}%
\end{pgfscope}%
\begin{pgfscope}%
\pgfsys@transformshift{2.303680in}{1.481690in}%
\pgfsys@useobject{currentmarker}{}%
\end{pgfscope}%
\begin{pgfscope}%
\pgfsys@transformshift{2.313650in}{1.467273in}%
\pgfsys@useobject{currentmarker}{}%
\end{pgfscope}%
\begin{pgfscope}%
\pgfsys@transformshift{2.323380in}{1.469961in}%
\pgfsys@useobject{currentmarker}{}%
\end{pgfscope}%
\begin{pgfscope}%
\pgfsys@transformshift{2.332880in}{1.467646in}%
\pgfsys@useobject{currentmarker}{}%
\end{pgfscope}%
\begin{pgfscope}%
\pgfsys@transformshift{2.342163in}{1.452976in}%
\pgfsys@useobject{currentmarker}{}%
\end{pgfscope}%
\begin{pgfscope}%
\pgfsys@transformshift{2.351236in}{1.430003in}%
\pgfsys@useobject{currentmarker}{}%
\end{pgfscope}%
\begin{pgfscope}%
\pgfsys@transformshift{2.360111in}{1.444233in}%
\pgfsys@useobject{currentmarker}{}%
\end{pgfscope}%
\begin{pgfscope}%
\pgfsys@transformshift{2.368794in}{1.447669in}%
\pgfsys@useobject{currentmarker}{}%
\end{pgfscope}%
\begin{pgfscope}%
\pgfsys@transformshift{2.377294in}{1.411239in}%
\pgfsys@useobject{currentmarker}{}%
\end{pgfscope}%
\begin{pgfscope}%
\pgfsys@transformshift{2.385620in}{1.401743in}%
\pgfsys@useobject{currentmarker}{}%
\end{pgfscope}%
\begin{pgfscope}%
\pgfsys@transformshift{2.393777in}{1.420748in}%
\pgfsys@useobject{currentmarker}{}%
\end{pgfscope}%
\begin{pgfscope}%
\pgfsys@transformshift{2.401772in}{1.427543in}%
\pgfsys@useobject{currentmarker}{}%
\end{pgfscope}%
\begin{pgfscope}%
\pgfsys@transformshift{2.409613in}{1.430428in}%
\pgfsys@useobject{currentmarker}{}%
\end{pgfscope}%
\begin{pgfscope}%
\pgfsys@transformshift{2.417304in}{1.420557in}%
\pgfsys@useobject{currentmarker}{}%
\end{pgfscope}%
\begin{pgfscope}%
\pgfsys@transformshift{2.424851in}{1.404630in}%
\pgfsys@useobject{currentmarker}{}%
\end{pgfscope}%
\begin{pgfscope}%
\pgfsys@transformshift{2.432259in}{1.405701in}%
\pgfsys@useobject{currentmarker}{}%
\end{pgfscope}%
\begin{pgfscope}%
\pgfsys@transformshift{2.439535in}{1.416700in}%
\pgfsys@useobject{currentmarker}{}%
\end{pgfscope}%
\begin{pgfscope}%
\pgfsys@transformshift{2.446681in}{1.409368in}%
\pgfsys@useobject{currentmarker}{}%
\end{pgfscope}%
\begin{pgfscope}%
\pgfsys@transformshift{2.453703in}{1.393458in}%
\pgfsys@useobject{currentmarker}{}%
\end{pgfscope}%
\begin{pgfscope}%
\pgfsys@transformshift{2.460605in}{1.377241in}%
\pgfsys@useobject{currentmarker}{}%
\end{pgfscope}%
\begin{pgfscope}%
\pgfsys@transformshift{2.467391in}{1.370469in}%
\pgfsys@useobject{currentmarker}{}%
\end{pgfscope}%
\begin{pgfscope}%
\pgfsys@transformshift{2.474065in}{1.370036in}%
\pgfsys@useobject{currentmarker}{}%
\end{pgfscope}%
\begin{pgfscope}%
\pgfsys@transformshift{2.480631in}{1.348619in}%
\pgfsys@useobject{currentmarker}{}%
\end{pgfscope}%
\begin{pgfscope}%
\pgfsys@transformshift{2.487091in}{1.360328in}%
\pgfsys@useobject{currentmarker}{}%
\end{pgfscope}%
\begin{pgfscope}%
\pgfsys@transformshift{2.493450in}{1.368351in}%
\pgfsys@useobject{currentmarker}{}%
\end{pgfscope}%
\begin{pgfscope}%
\pgfsys@transformshift{2.499710in}{1.356290in}%
\pgfsys@useobject{currentmarker}{}%
\end{pgfscope}%
\begin{pgfscope}%
\pgfsys@transformshift{2.505874in}{1.338061in}%
\pgfsys@useobject{currentmarker}{}%
\end{pgfscope}%
\begin{pgfscope}%
\pgfsys@transformshift{2.511946in}{1.346403in}%
\pgfsys@useobject{currentmarker}{}%
\end{pgfscope}%
\begin{pgfscope}%
\pgfsys@transformshift{2.517927in}{1.363707in}%
\pgfsys@useobject{currentmarker}{}%
\end{pgfscope}%
\begin{pgfscope}%
\pgfsys@transformshift{2.523822in}{1.357537in}%
\pgfsys@useobject{currentmarker}{}%
\end{pgfscope}%
\begin{pgfscope}%
\pgfsys@transformshift{2.529632in}{1.346321in}%
\pgfsys@useobject{currentmarker}{}%
\end{pgfscope}%
\begin{pgfscope}%
\pgfsys@transformshift{2.535359in}{1.342653in}%
\pgfsys@useobject{currentmarker}{}%
\end{pgfscope}%
\begin{pgfscope}%
\pgfsys@transformshift{2.541006in}{1.338271in}%
\pgfsys@useobject{currentmarker}{}%
\end{pgfscope}%
\begin{pgfscope}%
\pgfsys@transformshift{2.546575in}{1.332067in}%
\pgfsys@useobject{currentmarker}{}%
\end{pgfscope}%
\begin{pgfscope}%
\pgfsys@transformshift{2.552068in}{1.322525in}%
\pgfsys@useobject{currentmarker}{}%
\end{pgfscope}%
\begin{pgfscope}%
\pgfsys@transformshift{2.557488in}{1.313946in}%
\pgfsys@useobject{currentmarker}{}%
\end{pgfscope}%
\begin{pgfscope}%
\pgfsys@transformshift{2.562836in}{1.309383in}%
\pgfsys@useobject{currentmarker}{}%
\end{pgfscope}%
\begin{pgfscope}%
\pgfsys@transformshift{2.568114in}{1.326445in}%
\pgfsys@useobject{currentmarker}{}%
\end{pgfscope}%
\begin{pgfscope}%
\pgfsys@transformshift{2.573324in}{1.326702in}%
\pgfsys@useobject{currentmarker}{}%
\end{pgfscope}%
\begin{pgfscope}%
\pgfsys@transformshift{2.578468in}{1.315245in}%
\pgfsys@useobject{currentmarker}{}%
\end{pgfscope}%
\begin{pgfscope}%
\pgfsys@transformshift{2.583546in}{1.310478in}%
\pgfsys@useobject{currentmarker}{}%
\end{pgfscope}%
\begin{pgfscope}%
\pgfsys@transformshift{2.588562in}{1.305512in}%
\pgfsys@useobject{currentmarker}{}%
\end{pgfscope}%
\begin{pgfscope}%
\pgfsys@transformshift{2.593516in}{1.303746in}%
\pgfsys@useobject{currentmarker}{}%
\end{pgfscope}%
\begin{pgfscope}%
\pgfsys@transformshift{2.598410in}{1.338708in}%
\pgfsys@useobject{currentmarker}{}%
\end{pgfscope}%
\begin{pgfscope}%
\pgfsys@transformshift{2.603246in}{1.326855in}%
\pgfsys@useobject{currentmarker}{}%
\end{pgfscope}%
\begin{pgfscope}%
\pgfsys@transformshift{2.608024in}{1.305610in}%
\pgfsys@useobject{currentmarker}{}%
\end{pgfscope}%
\begin{pgfscope}%
\pgfsys@transformshift{2.612747in}{1.290819in}%
\pgfsys@useobject{currentmarker}{}%
\end{pgfscope}%
\begin{pgfscope}%
\pgfsys@transformshift{2.617415in}{1.286288in}%
\pgfsys@useobject{currentmarker}{}%
\end{pgfscope}%
\begin{pgfscope}%
\pgfsys@transformshift{2.622029in}{1.296906in}%
\pgfsys@useobject{currentmarker}{}%
\end{pgfscope}%
\begin{pgfscope}%
\pgfsys@transformshift{2.626591in}{1.310062in}%
\pgfsys@useobject{currentmarker}{}%
\end{pgfscope}%
\begin{pgfscope}%
\pgfsys@transformshift{2.631103in}{1.291148in}%
\pgfsys@useobject{currentmarker}{}%
\end{pgfscope}%
\begin{pgfscope}%
\pgfsys@transformshift{2.635564in}{1.280497in}%
\pgfsys@useobject{currentmarker}{}%
\end{pgfscope}%
\begin{pgfscope}%
\pgfsys@transformshift{2.639977in}{1.306717in}%
\pgfsys@useobject{currentmarker}{}%
\end{pgfscope}%
\begin{pgfscope}%
\pgfsys@transformshift{2.644342in}{1.308180in}%
\pgfsys@useobject{currentmarker}{}%
\end{pgfscope}%
\begin{pgfscope}%
\pgfsys@transformshift{2.648660in}{1.283747in}%
\pgfsys@useobject{currentmarker}{}%
\end{pgfscope}%
\begin{pgfscope}%
\pgfsys@transformshift{2.652933in}{1.278693in}%
\pgfsys@useobject{currentmarker}{}%
\end{pgfscope}%
\begin{pgfscope}%
\pgfsys@transformshift{2.657161in}{1.287623in}%
\pgfsys@useobject{currentmarker}{}%
\end{pgfscope}%
\begin{pgfscope}%
\pgfsys@transformshift{2.661345in}{1.285816in}%
\pgfsys@useobject{currentmarker}{}%
\end{pgfscope}%
\begin{pgfscope}%
\pgfsys@transformshift{2.665486in}{1.283317in}%
\pgfsys@useobject{currentmarker}{}%
\end{pgfscope}%
\begin{pgfscope}%
\pgfsys@transformshift{2.669585in}{1.287230in}%
\pgfsys@useobject{currentmarker}{}%
\end{pgfscope}%
\begin{pgfscope}%
\pgfsys@transformshift{2.673643in}{1.262308in}%
\pgfsys@useobject{currentmarker}{}%
\end{pgfscope}%
\begin{pgfscope}%
\pgfsys@transformshift{2.677661in}{1.268521in}%
\pgfsys@useobject{currentmarker}{}%
\end{pgfscope}%
\begin{pgfscope}%
\pgfsys@transformshift{2.681639in}{1.272161in}%
\pgfsys@useobject{currentmarker}{}%
\end{pgfscope}%
\begin{pgfscope}%
\pgfsys@transformshift{2.685578in}{1.268702in}%
\pgfsys@useobject{currentmarker}{}%
\end{pgfscope}%
\begin{pgfscope}%
\pgfsys@transformshift{2.689479in}{1.249919in}%
\pgfsys@useobject{currentmarker}{}%
\end{pgfscope}%
\begin{pgfscope}%
\pgfsys@transformshift{2.693343in}{1.249705in}%
\pgfsys@useobject{currentmarker}{}%
\end{pgfscope}%
\begin{pgfscope}%
\pgfsys@transformshift{2.697170in}{1.244683in}%
\pgfsys@useobject{currentmarker}{}%
\end{pgfscope}%
\begin{pgfscope}%
\pgfsys@transformshift{2.700961in}{1.250174in}%
\pgfsys@useobject{currentmarker}{}%
\end{pgfscope}%
\begin{pgfscope}%
\pgfsys@transformshift{2.704717in}{1.248718in}%
\pgfsys@useobject{currentmarker}{}%
\end{pgfscope}%
\begin{pgfscope}%
\pgfsys@transformshift{2.708439in}{1.245844in}%
\pgfsys@useobject{currentmarker}{}%
\end{pgfscope}%
\begin{pgfscope}%
\pgfsys@transformshift{2.712126in}{1.239595in}%
\pgfsys@useobject{currentmarker}{}%
\end{pgfscope}%
\begin{pgfscope}%
\pgfsys@transformshift{2.715780in}{1.241355in}%
\pgfsys@useobject{currentmarker}{}%
\end{pgfscope}%
\begin{pgfscope}%
\pgfsys@transformshift{2.719401in}{1.233286in}%
\pgfsys@useobject{currentmarker}{}%
\end{pgfscope}%
\begin{pgfscope}%
\pgfsys@transformshift{2.722990in}{1.224701in}%
\pgfsys@useobject{currentmarker}{}%
\end{pgfscope}%
\begin{pgfscope}%
\pgfsys@transformshift{2.726547in}{1.220178in}%
\pgfsys@useobject{currentmarker}{}%
\end{pgfscope}%
\begin{pgfscope}%
\pgfsys@transformshift{2.730074in}{1.224061in}%
\pgfsys@useobject{currentmarker}{}%
\end{pgfscope}%
\begin{pgfscope}%
\pgfsys@transformshift{2.733570in}{1.236452in}%
\pgfsys@useobject{currentmarker}{}%
\end{pgfscope}%
\begin{pgfscope}%
\pgfsys@transformshift{2.737035in}{1.241673in}%
\pgfsys@useobject{currentmarker}{}%
\end{pgfscope}%
\begin{pgfscope}%
\pgfsys@transformshift{2.740472in}{1.233830in}%
\pgfsys@useobject{currentmarker}{}%
\end{pgfscope}%
\begin{pgfscope}%
\pgfsys@transformshift{2.743879in}{1.243708in}%
\pgfsys@useobject{currentmarker}{}%
\end{pgfscope}%
\begin{pgfscope}%
\pgfsys@transformshift{2.747258in}{1.243531in}%
\pgfsys@useobject{currentmarker}{}%
\end{pgfscope}%
\begin{pgfscope}%
\pgfsys@transformshift{2.750608in}{1.237728in}%
\pgfsys@useobject{currentmarker}{}%
\end{pgfscope}%
\begin{pgfscope}%
\pgfsys@transformshift{2.753932in}{1.217227in}%
\pgfsys@useobject{currentmarker}{}%
\end{pgfscope}%
\begin{pgfscope}%
\pgfsys@transformshift{2.757228in}{1.228406in}%
\pgfsys@useobject{currentmarker}{}%
\end{pgfscope}%
\begin{pgfscope}%
\pgfsys@transformshift{2.760497in}{1.223112in}%
\pgfsys@useobject{currentmarker}{}%
\end{pgfscope}%
\begin{pgfscope}%
\pgfsys@transformshift{2.763740in}{1.206827in}%
\pgfsys@useobject{currentmarker}{}%
\end{pgfscope}%
\begin{pgfscope}%
\pgfsys@transformshift{2.766957in}{1.193276in}%
\pgfsys@useobject{currentmarker}{}%
\end{pgfscope}%
\begin{pgfscope}%
\pgfsys@transformshift{2.770149in}{1.179966in}%
\pgfsys@useobject{currentmarker}{}%
\end{pgfscope}%
\begin{pgfscope}%
\pgfsys@transformshift{2.773316in}{1.192953in}%
\pgfsys@useobject{currentmarker}{}%
\end{pgfscope}%
\begin{pgfscope}%
\pgfsys@transformshift{2.776458in}{1.217930in}%
\pgfsys@useobject{currentmarker}{}%
\end{pgfscope}%
\begin{pgfscope}%
\pgfsys@transformshift{2.779576in}{1.223985in}%
\pgfsys@useobject{currentmarker}{}%
\end{pgfscope}%
\begin{pgfscope}%
\pgfsys@transformshift{2.782670in}{1.204218in}%
\pgfsys@useobject{currentmarker}{}%
\end{pgfscope}%
\begin{pgfscope}%
\pgfsys@transformshift{2.785740in}{1.194619in}%
\pgfsys@useobject{currentmarker}{}%
\end{pgfscope}%
\begin{pgfscope}%
\pgfsys@transformshift{2.788788in}{1.207988in}%
\pgfsys@useobject{currentmarker}{}%
\end{pgfscope}%
\begin{pgfscope}%
\pgfsys@transformshift{2.791812in}{1.195505in}%
\pgfsys@useobject{currentmarker}{}%
\end{pgfscope}%
\begin{pgfscope}%
\pgfsys@transformshift{2.794814in}{1.193870in}%
\pgfsys@useobject{currentmarker}{}%
\end{pgfscope}%
\begin{pgfscope}%
\pgfsys@transformshift{2.797794in}{1.182221in}%
\pgfsys@useobject{currentmarker}{}%
\end{pgfscope}%
\begin{pgfscope}%
\pgfsys@transformshift{2.800752in}{1.191484in}%
\pgfsys@useobject{currentmarker}{}%
\end{pgfscope}%
\begin{pgfscope}%
\pgfsys@transformshift{2.803688in}{1.189181in}%
\pgfsys@useobject{currentmarker}{}%
\end{pgfscope}%
\begin{pgfscope}%
\pgfsys@transformshift{2.806604in}{1.179064in}%
\pgfsys@useobject{currentmarker}{}%
\end{pgfscope}%
\begin{pgfscope}%
\pgfsys@transformshift{2.809498in}{1.170423in}%
\pgfsys@useobject{currentmarker}{}%
\end{pgfscope}%
\begin{pgfscope}%
\pgfsys@transformshift{2.812372in}{1.184721in}%
\pgfsys@useobject{currentmarker}{}%
\end{pgfscope}%
\begin{pgfscope}%
\pgfsys@transformshift{2.815225in}{1.199526in}%
\pgfsys@useobject{currentmarker}{}%
\end{pgfscope}%
\begin{pgfscope}%
\pgfsys@transformshift{2.818059in}{1.195215in}%
\pgfsys@useobject{currentmarker}{}%
\end{pgfscope}%
\begin{pgfscope}%
\pgfsys@transformshift{2.820872in}{1.185316in}%
\pgfsys@useobject{currentmarker}{}%
\end{pgfscope}%
\begin{pgfscope}%
\pgfsys@transformshift{2.823666in}{1.173783in}%
\pgfsys@useobject{currentmarker}{}%
\end{pgfscope}%
\begin{pgfscope}%
\pgfsys@transformshift{2.826441in}{1.167251in}%
\pgfsys@useobject{currentmarker}{}%
\end{pgfscope}%
\begin{pgfscope}%
\pgfsys@transformshift{2.829198in}{1.187221in}%
\pgfsys@useobject{currentmarker}{}%
\end{pgfscope}%
\begin{pgfscope}%
\pgfsys@transformshift{2.831935in}{1.203288in}%
\pgfsys@useobject{currentmarker}{}%
\end{pgfscope}%
\begin{pgfscope}%
\pgfsys@transformshift{2.834654in}{1.192929in}%
\pgfsys@useobject{currentmarker}{}%
\end{pgfscope}%
\begin{pgfscope}%
\pgfsys@transformshift{2.837355in}{1.164143in}%
\pgfsys@useobject{currentmarker}{}%
\end{pgfscope}%
\begin{pgfscope}%
\pgfsys@transformshift{2.840037in}{1.170534in}%
\pgfsys@useobject{currentmarker}{}%
\end{pgfscope}%
\begin{pgfscope}%
\pgfsys@transformshift{2.842703in}{1.143372in}%
\pgfsys@useobject{currentmarker}{}%
\end{pgfscope}%
\begin{pgfscope}%
\pgfsys@transformshift{2.845350in}{1.155456in}%
\pgfsys@useobject{currentmarker}{}%
\end{pgfscope}%
\begin{pgfscope}%
\pgfsys@transformshift{2.847981in}{1.151803in}%
\pgfsys@useobject{currentmarker}{}%
\end{pgfscope}%
\begin{pgfscope}%
\pgfsys@transformshift{2.850594in}{1.142216in}%
\pgfsys@useobject{currentmarker}{}%
\end{pgfscope}%
\begin{pgfscope}%
\pgfsys@transformshift{2.853190in}{1.156606in}%
\pgfsys@useobject{currentmarker}{}%
\end{pgfscope}%
\begin{pgfscope}%
\pgfsys@transformshift{2.855770in}{1.169372in}%
\pgfsys@useobject{currentmarker}{}%
\end{pgfscope}%
\begin{pgfscope}%
\pgfsys@transformshift{2.858334in}{1.173658in}%
\pgfsys@useobject{currentmarker}{}%
\end{pgfscope}%
\begin{pgfscope}%
\pgfsys@transformshift{2.860881in}{1.153869in}%
\pgfsys@useobject{currentmarker}{}%
\end{pgfscope}%
\begin{pgfscope}%
\pgfsys@transformshift{2.863413in}{1.148932in}%
\pgfsys@useobject{currentmarker}{}%
\end{pgfscope}%
\begin{pgfscope}%
\pgfsys@transformshift{2.865928in}{1.156379in}%
\pgfsys@useobject{currentmarker}{}%
\end{pgfscope}%
\begin{pgfscope}%
\pgfsys@transformshift{2.868429in}{1.156702in}%
\pgfsys@useobject{currentmarker}{}%
\end{pgfscope}%
\begin{pgfscope}%
\pgfsys@transformshift{2.870913in}{1.158872in}%
\pgfsys@useobject{currentmarker}{}%
\end{pgfscope}%
\begin{pgfscope}%
\pgfsys@transformshift{2.873383in}{1.136655in}%
\pgfsys@useobject{currentmarker}{}%
\end{pgfscope}%
\begin{pgfscope}%
\pgfsys@transformshift{2.875837in}{1.132707in}%
\pgfsys@useobject{currentmarker}{}%
\end{pgfscope}%
\begin{pgfscope}%
\pgfsys@transformshift{2.878277in}{1.145455in}%
\pgfsys@useobject{currentmarker}{}%
\end{pgfscope}%
\begin{pgfscope}%
\pgfsys@transformshift{2.880702in}{1.154187in}%
\pgfsys@useobject{currentmarker}{}%
\end{pgfscope}%
\begin{pgfscope}%
\pgfsys@transformshift{2.883112in}{1.149469in}%
\pgfsys@useobject{currentmarker}{}%
\end{pgfscope}%
\begin{pgfscope}%
\pgfsys@transformshift{2.885509in}{1.157849in}%
\pgfsys@useobject{currentmarker}{}%
\end{pgfscope}%
\begin{pgfscope}%
\pgfsys@transformshift{2.887891in}{1.147665in}%
\pgfsys@useobject{currentmarker}{}%
\end{pgfscope}%
\begin{pgfscope}%
\pgfsys@transformshift{2.890259in}{1.148214in}%
\pgfsys@useobject{currentmarker}{}%
\end{pgfscope}%
\begin{pgfscope}%
\pgfsys@transformshift{2.892613in}{1.142881in}%
\pgfsys@useobject{currentmarker}{}%
\end{pgfscope}%
\begin{pgfscope}%
\pgfsys@transformshift{2.894954in}{1.138376in}%
\pgfsys@useobject{currentmarker}{}%
\end{pgfscope}%
\begin{pgfscope}%
\pgfsys@transformshift{2.897281in}{1.139838in}%
\pgfsys@useobject{currentmarker}{}%
\end{pgfscope}%
\begin{pgfscope}%
\pgfsys@transformshift{2.899595in}{1.144699in}%
\pgfsys@useobject{currentmarker}{}%
\end{pgfscope}%
\begin{pgfscope}%
\pgfsys@transformshift{2.901895in}{1.132750in}%
\pgfsys@useobject{currentmarker}{}%
\end{pgfscope}%
\begin{pgfscope}%
\pgfsys@transformshift{2.904183in}{1.146038in}%
\pgfsys@useobject{currentmarker}{}%
\end{pgfscope}%
\begin{pgfscope}%
\pgfsys@transformshift{2.906458in}{1.160696in}%
\pgfsys@useobject{currentmarker}{}%
\end{pgfscope}%
\begin{pgfscope}%
\pgfsys@transformshift{2.908720in}{1.145672in}%
\pgfsys@useobject{currentmarker}{}%
\end{pgfscope}%
\begin{pgfscope}%
\pgfsys@transformshift{2.910969in}{1.155933in}%
\pgfsys@useobject{currentmarker}{}%
\end{pgfscope}%
\begin{pgfscope}%
\pgfsys@transformshift{2.913206in}{1.138720in}%
\pgfsys@useobject{currentmarker}{}%
\end{pgfscope}%
\begin{pgfscope}%
\pgfsys@transformshift{2.915431in}{1.118896in}%
\pgfsys@useobject{currentmarker}{}%
\end{pgfscope}%
\begin{pgfscope}%
\pgfsys@transformshift{2.917643in}{1.150976in}%
\pgfsys@useobject{currentmarker}{}%
\end{pgfscope}%
\begin{pgfscope}%
\pgfsys@transformshift{2.919843in}{1.161797in}%
\pgfsys@useobject{currentmarker}{}%
\end{pgfscope}%
\begin{pgfscope}%
\pgfsys@transformshift{2.922032in}{1.143359in}%
\pgfsys@useobject{currentmarker}{}%
\end{pgfscope}%
\begin{pgfscope}%
\pgfsys@transformshift{2.924208in}{1.126553in}%
\pgfsys@useobject{currentmarker}{}%
\end{pgfscope}%
\begin{pgfscope}%
\pgfsys@transformshift{2.926373in}{1.131387in}%
\pgfsys@useobject{currentmarker}{}%
\end{pgfscope}%
\begin{pgfscope}%
\pgfsys@transformshift{2.928527in}{1.144327in}%
\pgfsys@useobject{currentmarker}{}%
\end{pgfscope}%
\begin{pgfscope}%
\pgfsys@transformshift{2.930669in}{1.141488in}%
\pgfsys@useobject{currentmarker}{}%
\end{pgfscope}%
\begin{pgfscope}%
\pgfsys@transformshift{2.932799in}{1.143285in}%
\pgfsys@useobject{currentmarker}{}%
\end{pgfscope}%
\begin{pgfscope}%
\pgfsys@transformshift{2.934919in}{1.138486in}%
\pgfsys@useobject{currentmarker}{}%
\end{pgfscope}%
\begin{pgfscope}%
\pgfsys@transformshift{2.937027in}{1.118408in}%
\pgfsys@useobject{currentmarker}{}%
\end{pgfscope}%
\begin{pgfscope}%
\pgfsys@transformshift{2.939125in}{1.115551in}%
\pgfsys@useobject{currentmarker}{}%
\end{pgfscope}%
\begin{pgfscope}%
\pgfsys@transformshift{2.941211in}{1.120905in}%
\pgfsys@useobject{currentmarker}{}%
\end{pgfscope}%
\begin{pgfscope}%
\pgfsys@transformshift{2.943287in}{1.114205in}%
\pgfsys@useobject{currentmarker}{}%
\end{pgfscope}%
\begin{pgfscope}%
\pgfsys@transformshift{2.945353in}{1.109324in}%
\pgfsys@useobject{currentmarker}{}%
\end{pgfscope}%
\begin{pgfscope}%
\pgfsys@transformshift{2.947407in}{1.114602in}%
\pgfsys@useobject{currentmarker}{}%
\end{pgfscope}%
\begin{pgfscope}%
\pgfsys@transformshift{2.949452in}{1.123768in}%
\pgfsys@useobject{currentmarker}{}%
\end{pgfscope}%
\begin{pgfscope}%
\pgfsys@transformshift{2.951486in}{1.105784in}%
\pgfsys@useobject{currentmarker}{}%
\end{pgfscope}%
\begin{pgfscope}%
\pgfsys@transformshift{2.953510in}{1.093592in}%
\pgfsys@useobject{currentmarker}{}%
\end{pgfscope}%
\begin{pgfscope}%
\pgfsys@transformshift{2.955523in}{1.095857in}%
\pgfsys@useobject{currentmarker}{}%
\end{pgfscope}%
\begin{pgfscope}%
\pgfsys@transformshift{2.957527in}{1.106835in}%
\pgfsys@useobject{currentmarker}{}%
\end{pgfscope}%
\begin{pgfscope}%
\pgfsys@transformshift{2.959521in}{1.119947in}%
\pgfsys@useobject{currentmarker}{}%
\end{pgfscope}%
\begin{pgfscope}%
\pgfsys@transformshift{2.961505in}{1.105835in}%
\pgfsys@useobject{currentmarker}{}%
\end{pgfscope}%
\begin{pgfscope}%
\pgfsys@transformshift{2.963480in}{1.108336in}%
\pgfsys@useobject{currentmarker}{}%
\end{pgfscope}%
\begin{pgfscope}%
\pgfsys@transformshift{2.965444in}{1.107923in}%
\pgfsys@useobject{currentmarker}{}%
\end{pgfscope}%
\begin{pgfscope}%
\pgfsys@transformshift{2.967400in}{1.106871in}%
\pgfsys@useobject{currentmarker}{}%
\end{pgfscope}%
\begin{pgfscope}%
\pgfsys@transformshift{2.969345in}{1.100570in}%
\pgfsys@useobject{currentmarker}{}%
\end{pgfscope}%
\begin{pgfscope}%
\pgfsys@transformshift{2.971282in}{1.107368in}%
\pgfsys@useobject{currentmarker}{}%
\end{pgfscope}%
\begin{pgfscope}%
\pgfsys@transformshift{2.973209in}{1.117526in}%
\pgfsys@useobject{currentmarker}{}%
\end{pgfscope}%
\begin{pgfscope}%
\pgfsys@transformshift{2.975127in}{1.102677in}%
\pgfsys@useobject{currentmarker}{}%
\end{pgfscope}%
\begin{pgfscope}%
\pgfsys@transformshift{2.977036in}{1.105110in}%
\pgfsys@useobject{currentmarker}{}%
\end{pgfscope}%
\begin{pgfscope}%
\pgfsys@transformshift{2.978936in}{1.103224in}%
\pgfsys@useobject{currentmarker}{}%
\end{pgfscope}%
\begin{pgfscope}%
\pgfsys@transformshift{2.980828in}{1.092914in}%
\pgfsys@useobject{currentmarker}{}%
\end{pgfscope}%
\begin{pgfscope}%
\pgfsys@transformshift{2.982710in}{1.095705in}%
\pgfsys@useobject{currentmarker}{}%
\end{pgfscope}%
\begin{pgfscope}%
\pgfsys@transformshift{2.984584in}{1.100950in}%
\pgfsys@useobject{currentmarker}{}%
\end{pgfscope}%
\begin{pgfscope}%
\pgfsys@transformshift{2.986449in}{1.079925in}%
\pgfsys@useobject{currentmarker}{}%
\end{pgfscope}%
\begin{pgfscope}%
\pgfsys@transformshift{2.988305in}{1.078452in}%
\pgfsys@useobject{currentmarker}{}%
\end{pgfscope}%
\begin{pgfscope}%
\pgfsys@transformshift{2.990153in}{1.095143in}%
\pgfsys@useobject{currentmarker}{}%
\end{pgfscope}%
\begin{pgfscope}%
\pgfsys@transformshift{2.991992in}{1.084333in}%
\pgfsys@useobject{currentmarker}{}%
\end{pgfscope}%
\begin{pgfscope}%
\pgfsys@transformshift{2.993823in}{1.068674in}%
\pgfsys@useobject{currentmarker}{}%
\end{pgfscope}%
\begin{pgfscope}%
\pgfsys@transformshift{2.995646in}{1.071554in}%
\pgfsys@useobject{currentmarker}{}%
\end{pgfscope}%
\begin{pgfscope}%
\pgfsys@transformshift{2.997461in}{1.090487in}%
\pgfsys@useobject{currentmarker}{}%
\end{pgfscope}%
\begin{pgfscope}%
\pgfsys@transformshift{2.999267in}{1.089960in}%
\pgfsys@useobject{currentmarker}{}%
\end{pgfscope}%
\begin{pgfscope}%
\pgfsys@transformshift{3.001066in}{1.086554in}%
\pgfsys@useobject{currentmarker}{}%
\end{pgfscope}%
\begin{pgfscope}%
\pgfsys@transformshift{3.002856in}{1.084370in}%
\pgfsys@useobject{currentmarker}{}%
\end{pgfscope}%
\begin{pgfscope}%
\pgfsys@transformshift{3.004639in}{1.090671in}%
\pgfsys@useobject{currentmarker}{}%
\end{pgfscope}%
\begin{pgfscope}%
\pgfsys@transformshift{3.006414in}{1.081091in}%
\pgfsys@useobject{currentmarker}{}%
\end{pgfscope}%
\begin{pgfscope}%
\pgfsys@transformshift{3.008181in}{1.055300in}%
\pgfsys@useobject{currentmarker}{}%
\end{pgfscope}%
\begin{pgfscope}%
\pgfsys@transformshift{3.009940in}{1.070096in}%
\pgfsys@useobject{currentmarker}{}%
\end{pgfscope}%
\begin{pgfscope}%
\pgfsys@transformshift{3.011692in}{1.074515in}%
\pgfsys@useobject{currentmarker}{}%
\end{pgfscope}%
\begin{pgfscope}%
\pgfsys@transformshift{3.013436in}{1.091526in}%
\pgfsys@useobject{currentmarker}{}%
\end{pgfscope}%
\begin{pgfscope}%
\pgfsys@transformshift{3.015173in}{1.089347in}%
\pgfsys@useobject{currentmarker}{}%
\end{pgfscope}%
\begin{pgfscope}%
\pgfsys@transformshift{3.016902in}{1.088476in}%
\pgfsys@useobject{currentmarker}{}%
\end{pgfscope}%
\begin{pgfscope}%
\pgfsys@transformshift{3.018624in}{1.088989in}%
\pgfsys@useobject{currentmarker}{}%
\end{pgfscope}%
\begin{pgfscope}%
\pgfsys@transformshift{3.020338in}{1.081801in}%
\pgfsys@useobject{currentmarker}{}%
\end{pgfscope}%
\begin{pgfscope}%
\pgfsys@transformshift{3.022045in}{1.080246in}%
\pgfsys@useobject{currentmarker}{}%
\end{pgfscope}%
\begin{pgfscope}%
\pgfsys@transformshift{3.023745in}{1.062474in}%
\pgfsys@useobject{currentmarker}{}%
\end{pgfscope}%
\begin{pgfscope}%
\pgfsys@transformshift{3.025438in}{1.075953in}%
\pgfsys@useobject{currentmarker}{}%
\end{pgfscope}%
\begin{pgfscope}%
\pgfsys@transformshift{3.027124in}{1.088047in}%
\pgfsys@useobject{currentmarker}{}%
\end{pgfscope}%
\begin{pgfscope}%
\pgfsys@transformshift{3.028803in}{1.087707in}%
\pgfsys@useobject{currentmarker}{}%
\end{pgfscope}%
\begin{pgfscope}%
\pgfsys@transformshift{3.030475in}{1.080055in}%
\pgfsys@useobject{currentmarker}{}%
\end{pgfscope}%
\begin{pgfscope}%
\pgfsys@transformshift{3.032140in}{1.077134in}%
\pgfsys@useobject{currentmarker}{}%
\end{pgfscope}%
\begin{pgfscope}%
\pgfsys@transformshift{3.033798in}{1.077173in}%
\pgfsys@useobject{currentmarker}{}%
\end{pgfscope}%
\begin{pgfscope}%
\pgfsys@transformshift{3.035449in}{1.071720in}%
\pgfsys@useobject{currentmarker}{}%
\end{pgfscope}%
\begin{pgfscope}%
\pgfsys@transformshift{3.037094in}{1.057909in}%
\pgfsys@useobject{currentmarker}{}%
\end{pgfscope}%
\begin{pgfscope}%
\pgfsys@transformshift{3.038732in}{1.060890in}%
\pgfsys@useobject{currentmarker}{}%
\end{pgfscope}%
\begin{pgfscope}%
\pgfsys@transformshift{3.040363in}{1.074297in}%
\pgfsys@useobject{currentmarker}{}%
\end{pgfscope}%
\begin{pgfscope}%
\pgfsys@transformshift{3.041988in}{1.076855in}%
\pgfsys@useobject{currentmarker}{}%
\end{pgfscope}%
\begin{pgfscope}%
\pgfsys@transformshift{3.043606in}{1.071203in}%
\pgfsys@useobject{currentmarker}{}%
\end{pgfscope}%
\begin{pgfscope}%
\pgfsys@transformshift{3.045218in}{1.065219in}%
\pgfsys@useobject{currentmarker}{}%
\end{pgfscope}%
\begin{pgfscope}%
\pgfsys@transformshift{3.046824in}{1.054747in}%
\pgfsys@useobject{currentmarker}{}%
\end{pgfscope}%
\begin{pgfscope}%
\pgfsys@transformshift{3.048423in}{1.045368in}%
\pgfsys@useobject{currentmarker}{}%
\end{pgfscope}%
\begin{pgfscope}%
\pgfsys@transformshift{3.050016in}{1.036155in}%
\pgfsys@useobject{currentmarker}{}%
\end{pgfscope}%
\begin{pgfscope}%
\pgfsys@transformshift{3.051602in}{1.054057in}%
\pgfsys@useobject{currentmarker}{}%
\end{pgfscope}%
\begin{pgfscope}%
\pgfsys@transformshift{3.053182in}{1.071953in}%
\pgfsys@useobject{currentmarker}{}%
\end{pgfscope}%
\begin{pgfscope}%
\pgfsys@transformshift{3.054756in}{1.072210in}%
\pgfsys@useobject{currentmarker}{}%
\end{pgfscope}%
\begin{pgfscope}%
\pgfsys@transformshift{3.056324in}{1.061972in}%
\pgfsys@useobject{currentmarker}{}%
\end{pgfscope}%
\begin{pgfscope}%
\pgfsys@transformshift{3.057886in}{1.054887in}%
\pgfsys@useobject{currentmarker}{}%
\end{pgfscope}%
\begin{pgfscope}%
\pgfsys@transformshift{3.059442in}{1.059268in}%
\pgfsys@useobject{currentmarker}{}%
\end{pgfscope}%
\begin{pgfscope}%
\pgfsys@transformshift{3.060992in}{1.059757in}%
\pgfsys@useobject{currentmarker}{}%
\end{pgfscope}%
\begin{pgfscope}%
\pgfsys@transformshift{3.062536in}{1.054253in}%
\pgfsys@useobject{currentmarker}{}%
\end{pgfscope}%
\begin{pgfscope}%
\pgfsys@transformshift{3.064074in}{1.039848in}%
\pgfsys@useobject{currentmarker}{}%
\end{pgfscope}%
\begin{pgfscope}%
\pgfsys@transformshift{3.065607in}{1.056833in}%
\pgfsys@useobject{currentmarker}{}%
\end{pgfscope}%
\begin{pgfscope}%
\pgfsys@transformshift{3.067133in}{1.060574in}%
\pgfsys@useobject{currentmarker}{}%
\end{pgfscope}%
\begin{pgfscope}%
\pgfsys@transformshift{3.068654in}{1.057851in}%
\pgfsys@useobject{currentmarker}{}%
\end{pgfscope}%
\begin{pgfscope}%
\pgfsys@transformshift{3.070169in}{1.048560in}%
\pgfsys@useobject{currentmarker}{}%
\end{pgfscope}%
\begin{pgfscope}%
\pgfsys@transformshift{3.071678in}{1.043593in}%
\pgfsys@useobject{currentmarker}{}%
\end{pgfscope}%
\begin{pgfscope}%
\pgfsys@transformshift{3.073182in}{1.054540in}%
\pgfsys@useobject{currentmarker}{}%
\end{pgfscope}%
\begin{pgfscope}%
\pgfsys@transformshift{3.074680in}{1.043034in}%
\pgfsys@useobject{currentmarker}{}%
\end{pgfscope}%
\begin{pgfscope}%
\pgfsys@transformshift{3.076173in}{1.037924in}%
\pgfsys@useobject{currentmarker}{}%
\end{pgfscope}%
\begin{pgfscope}%
\pgfsys@transformshift{3.077660in}{1.044020in}%
\pgfsys@useobject{currentmarker}{}%
\end{pgfscope}%
\begin{pgfscope}%
\pgfsys@transformshift{3.079142in}{1.048210in}%
\pgfsys@useobject{currentmarker}{}%
\end{pgfscope}%
\begin{pgfscope}%
\pgfsys@transformshift{3.080618in}{1.072624in}%
\pgfsys@useobject{currentmarker}{}%
\end{pgfscope}%
\begin{pgfscope}%
\pgfsys@transformshift{3.082089in}{1.069239in}%
\pgfsys@useobject{currentmarker}{}%
\end{pgfscope}%
\begin{pgfscope}%
\pgfsys@transformshift{3.083555in}{1.070952in}%
\pgfsys@useobject{currentmarker}{}%
\end{pgfscope}%
\begin{pgfscope}%
\pgfsys@transformshift{3.085015in}{1.065764in}%
\pgfsys@useobject{currentmarker}{}%
\end{pgfscope}%
\begin{pgfscope}%
\pgfsys@transformshift{3.086470in}{1.061532in}%
\pgfsys@useobject{currentmarker}{}%
\end{pgfscope}%
\begin{pgfscope}%
\pgfsys@transformshift{3.087920in}{1.066140in}%
\pgfsys@useobject{currentmarker}{}%
\end{pgfscope}%
\begin{pgfscope}%
\pgfsys@transformshift{3.089364in}{1.063830in}%
\pgfsys@useobject{currentmarker}{}%
\end{pgfscope}%
\begin{pgfscope}%
\pgfsys@transformshift{3.090804in}{1.042154in}%
\pgfsys@useobject{currentmarker}{}%
\end{pgfscope}%
\begin{pgfscope}%
\pgfsys@transformshift{3.092238in}{1.028293in}%
\pgfsys@useobject{currentmarker}{}%
\end{pgfscope}%
\begin{pgfscope}%
\pgfsys@transformshift{3.093667in}{1.045212in}%
\pgfsys@useobject{currentmarker}{}%
\end{pgfscope}%
\begin{pgfscope}%
\pgfsys@transformshift{3.095091in}{1.018413in}%
\pgfsys@useobject{currentmarker}{}%
\end{pgfscope}%
\begin{pgfscope}%
\pgfsys@transformshift{3.096511in}{1.028347in}%
\pgfsys@useobject{currentmarker}{}%
\end{pgfscope}%
\begin{pgfscope}%
\pgfsys@transformshift{3.097925in}{1.017925in}%
\pgfsys@useobject{currentmarker}{}%
\end{pgfscope}%
\begin{pgfscope}%
\pgfsys@transformshift{3.099334in}{1.028672in}%
\pgfsys@useobject{currentmarker}{}%
\end{pgfscope}%
\begin{pgfscope}%
\pgfsys@transformshift{3.100739in}{1.058570in}%
\pgfsys@useobject{currentmarker}{}%
\end{pgfscope}%
\begin{pgfscope}%
\pgfsys@transformshift{3.102138in}{1.067796in}%
\pgfsys@useobject{currentmarker}{}%
\end{pgfscope}%
\begin{pgfscope}%
\pgfsys@transformshift{3.103533in}{1.033996in}%
\pgfsys@useobject{currentmarker}{}%
\end{pgfscope}%
\begin{pgfscope}%
\pgfsys@transformshift{3.104923in}{1.016518in}%
\pgfsys@useobject{currentmarker}{}%
\end{pgfscope}%
\begin{pgfscope}%
\pgfsys@transformshift{3.106308in}{1.022863in}%
\pgfsys@useobject{currentmarker}{}%
\end{pgfscope}%
\begin{pgfscope}%
\pgfsys@transformshift{3.107688in}{1.038041in}%
\pgfsys@useobject{currentmarker}{}%
\end{pgfscope}%
\begin{pgfscope}%
\pgfsys@transformshift{3.109064in}{1.045398in}%
\pgfsys@useobject{currentmarker}{}%
\end{pgfscope}%
\begin{pgfscope}%
\pgfsys@transformshift{3.110435in}{1.046932in}%
\pgfsys@useobject{currentmarker}{}%
\end{pgfscope}%
\begin{pgfscope}%
\pgfsys@transformshift{3.111801in}{1.043391in}%
\pgfsys@useobject{currentmarker}{}%
\end{pgfscope}%
\begin{pgfscope}%
\pgfsys@transformshift{3.113163in}{1.029832in}%
\pgfsys@useobject{currentmarker}{}%
\end{pgfscope}%
\begin{pgfscope}%
\pgfsys@transformshift{3.114520in}{1.028673in}%
\pgfsys@useobject{currentmarker}{}%
\end{pgfscope}%
\begin{pgfscope}%
\pgfsys@transformshift{3.115873in}{1.021616in}%
\pgfsys@useobject{currentmarker}{}%
\end{pgfscope}%
\begin{pgfscope}%
\pgfsys@transformshift{3.117221in}{1.029978in}%
\pgfsys@useobject{currentmarker}{}%
\end{pgfscope}%
\begin{pgfscope}%
\pgfsys@transformshift{3.118565in}{1.025541in}%
\pgfsys@useobject{currentmarker}{}%
\end{pgfscope}%
\begin{pgfscope}%
\pgfsys@transformshift{3.119904in}{1.041347in}%
\pgfsys@useobject{currentmarker}{}%
\end{pgfscope}%
\begin{pgfscope}%
\pgfsys@transformshift{3.121239in}{1.054193in}%
\pgfsys@useobject{currentmarker}{}%
\end{pgfscope}%
\begin{pgfscope}%
\pgfsys@transformshift{3.122569in}{1.038651in}%
\pgfsys@useobject{currentmarker}{}%
\end{pgfscope}%
\begin{pgfscope}%
\pgfsys@transformshift{3.123895in}{1.024660in}%
\pgfsys@useobject{currentmarker}{}%
\end{pgfscope}%
\begin{pgfscope}%
\pgfsys@transformshift{3.125217in}{1.007696in}%
\pgfsys@useobject{currentmarker}{}%
\end{pgfscope}%
\begin{pgfscope}%
\pgfsys@transformshift{3.126534in}{1.025055in}%
\pgfsys@useobject{currentmarker}{}%
\end{pgfscope}%
\begin{pgfscope}%
\pgfsys@transformshift{3.127847in}{1.029219in}%
\pgfsys@useobject{currentmarker}{}%
\end{pgfscope}%
\begin{pgfscope}%
\pgfsys@transformshift{3.129156in}{1.019081in}%
\pgfsys@useobject{currentmarker}{}%
\end{pgfscope}%
\begin{pgfscope}%
\pgfsys@transformshift{3.130460in}{1.012615in}%
\pgfsys@useobject{currentmarker}{}%
\end{pgfscope}%
\begin{pgfscope}%
\pgfsys@transformshift{3.131761in}{1.007861in}%
\pgfsys@useobject{currentmarker}{}%
\end{pgfscope}%
\begin{pgfscope}%
\pgfsys@transformshift{3.133057in}{1.012915in}%
\pgfsys@useobject{currentmarker}{}%
\end{pgfscope}%
\begin{pgfscope}%
\pgfsys@transformshift{3.134349in}{1.023084in}%
\pgfsys@useobject{currentmarker}{}%
\end{pgfscope}%
\begin{pgfscope}%
\pgfsys@transformshift{3.135637in}{1.026055in}%
\pgfsys@useobject{currentmarker}{}%
\end{pgfscope}%
\begin{pgfscope}%
\pgfsys@transformshift{3.136921in}{1.034861in}%
\pgfsys@useobject{currentmarker}{}%
\end{pgfscope}%
\begin{pgfscope}%
\pgfsys@transformshift{3.138200in}{1.031074in}%
\pgfsys@useobject{currentmarker}{}%
\end{pgfscope}%
\begin{pgfscope}%
\pgfsys@transformshift{3.139476in}{1.010926in}%
\pgfsys@useobject{currentmarker}{}%
\end{pgfscope}%
\begin{pgfscope}%
\pgfsys@transformshift{3.140748in}{0.990013in}%
\pgfsys@useobject{currentmarker}{}%
\end{pgfscope}%
\begin{pgfscope}%
\pgfsys@transformshift{3.142015in}{1.014794in}%
\pgfsys@useobject{currentmarker}{}%
\end{pgfscope}%
\begin{pgfscope}%
\pgfsys@transformshift{3.143279in}{1.017271in}%
\pgfsys@useobject{currentmarker}{}%
\end{pgfscope}%
\begin{pgfscope}%
\pgfsys@transformshift{3.144539in}{1.018787in}%
\pgfsys@useobject{currentmarker}{}%
\end{pgfscope}%
\begin{pgfscope}%
\pgfsys@transformshift{3.145795in}{1.005274in}%
\pgfsys@useobject{currentmarker}{}%
\end{pgfscope}%
\begin{pgfscope}%
\pgfsys@transformshift{3.147047in}{0.991690in}%
\pgfsys@useobject{currentmarker}{}%
\end{pgfscope}%
\begin{pgfscope}%
\pgfsys@transformshift{3.148295in}{1.015140in}%
\pgfsys@useobject{currentmarker}{}%
\end{pgfscope}%
\begin{pgfscope}%
\pgfsys@transformshift{3.149539in}{1.013215in}%
\pgfsys@useobject{currentmarker}{}%
\end{pgfscope}%
\begin{pgfscope}%
\pgfsys@transformshift{3.150780in}{1.025192in}%
\pgfsys@useobject{currentmarker}{}%
\end{pgfscope}%
\begin{pgfscope}%
\pgfsys@transformshift{3.152016in}{1.027299in}%
\pgfsys@useobject{currentmarker}{}%
\end{pgfscope}%
\begin{pgfscope}%
\pgfsys@transformshift{3.153249in}{1.021580in}%
\pgfsys@useobject{currentmarker}{}%
\end{pgfscope}%
\begin{pgfscope}%
\pgfsys@transformshift{3.154478in}{1.035342in}%
\pgfsys@useobject{currentmarker}{}%
\end{pgfscope}%
\begin{pgfscope}%
\pgfsys@transformshift{3.155704in}{1.018624in}%
\pgfsys@useobject{currentmarker}{}%
\end{pgfscope}%
\begin{pgfscope}%
\pgfsys@transformshift{3.156925in}{1.015797in}%
\pgfsys@useobject{currentmarker}{}%
\end{pgfscope}%
\begin{pgfscope}%
\pgfsys@transformshift{3.158143in}{1.012799in}%
\pgfsys@useobject{currentmarker}{}%
\end{pgfscope}%
\begin{pgfscope}%
\pgfsys@transformshift{3.159358in}{0.996381in}%
\pgfsys@useobject{currentmarker}{}%
\end{pgfscope}%
\begin{pgfscope}%
\pgfsys@transformshift{3.160568in}{1.023418in}%
\pgfsys@useobject{currentmarker}{}%
\end{pgfscope}%
\begin{pgfscope}%
\pgfsys@transformshift{3.161775in}{1.006915in}%
\pgfsys@useobject{currentmarker}{}%
\end{pgfscope}%
\begin{pgfscope}%
\pgfsys@transformshift{3.162979in}{0.995727in}%
\pgfsys@useobject{currentmarker}{}%
\end{pgfscope}%
\begin{pgfscope}%
\pgfsys@transformshift{3.164179in}{0.994052in}%
\pgfsys@useobject{currentmarker}{}%
\end{pgfscope}%
\begin{pgfscope}%
\pgfsys@transformshift{3.165375in}{1.001241in}%
\pgfsys@useobject{currentmarker}{}%
\end{pgfscope}%
\begin{pgfscope}%
\pgfsys@transformshift{3.166568in}{0.983763in}%
\pgfsys@useobject{currentmarker}{}%
\end{pgfscope}%
\begin{pgfscope}%
\pgfsys@transformshift{3.167757in}{0.985721in}%
\pgfsys@useobject{currentmarker}{}%
\end{pgfscope}%
\begin{pgfscope}%
\pgfsys@transformshift{3.168943in}{0.990351in}%
\pgfsys@useobject{currentmarker}{}%
\end{pgfscope}%
\begin{pgfscope}%
\pgfsys@transformshift{3.170125in}{0.999062in}%
\pgfsys@useobject{currentmarker}{}%
\end{pgfscope}%
\begin{pgfscope}%
\pgfsys@transformshift{3.171304in}{0.998911in}%
\pgfsys@useobject{currentmarker}{}%
\end{pgfscope}%
\begin{pgfscope}%
\pgfsys@transformshift{3.172479in}{1.004788in}%
\pgfsys@useobject{currentmarker}{}%
\end{pgfscope}%
\begin{pgfscope}%
\pgfsys@transformshift{3.173652in}{1.004097in}%
\pgfsys@useobject{currentmarker}{}%
\end{pgfscope}%
\begin{pgfscope}%
\pgfsys@transformshift{3.174820in}{0.998365in}%
\pgfsys@useobject{currentmarker}{}%
\end{pgfscope}%
\begin{pgfscope}%
\pgfsys@transformshift{3.175985in}{1.001533in}%
\pgfsys@useobject{currentmarker}{}%
\end{pgfscope}%
\begin{pgfscope}%
\pgfsys@transformshift{3.177147in}{1.004447in}%
\pgfsys@useobject{currentmarker}{}%
\end{pgfscope}%
\begin{pgfscope}%
\pgfsys@transformshift{3.178306in}{1.005587in}%
\pgfsys@useobject{currentmarker}{}%
\end{pgfscope}%
\begin{pgfscope}%
\pgfsys@transformshift{3.179461in}{0.983412in}%
\pgfsys@useobject{currentmarker}{}%
\end{pgfscope}%
\begin{pgfscope}%
\pgfsys@transformshift{3.180613in}{0.995140in}%
\pgfsys@useobject{currentmarker}{}%
\end{pgfscope}%
\begin{pgfscope}%
\pgfsys@transformshift{3.181762in}{0.998774in}%
\pgfsys@useobject{currentmarker}{}%
\end{pgfscope}%
\begin{pgfscope}%
\pgfsys@transformshift{3.182907in}{0.996884in}%
\pgfsys@useobject{currentmarker}{}%
\end{pgfscope}%
\begin{pgfscope}%
\pgfsys@transformshift{3.184049in}{0.990120in}%
\pgfsys@useobject{currentmarker}{}%
\end{pgfscope}%
\begin{pgfscope}%
\pgfsys@transformshift{3.185188in}{0.990175in}%
\pgfsys@useobject{currentmarker}{}%
\end{pgfscope}%
\begin{pgfscope}%
\pgfsys@transformshift{3.186324in}{0.991306in}%
\pgfsys@useobject{currentmarker}{}%
\end{pgfscope}%
\begin{pgfscope}%
\pgfsys@transformshift{3.187457in}{0.999029in}%
\pgfsys@useobject{currentmarker}{}%
\end{pgfscope}%
\begin{pgfscope}%
\pgfsys@transformshift{3.188586in}{0.986505in}%
\pgfsys@useobject{currentmarker}{}%
\end{pgfscope}%
\begin{pgfscope}%
\pgfsys@transformshift{3.189712in}{0.989067in}%
\pgfsys@useobject{currentmarker}{}%
\end{pgfscope}%
\begin{pgfscope}%
\pgfsys@transformshift{3.190835in}{0.978899in}%
\pgfsys@useobject{currentmarker}{}%
\end{pgfscope}%
\begin{pgfscope}%
\pgfsys@transformshift{3.191955in}{0.988474in}%
\pgfsys@useobject{currentmarker}{}%
\end{pgfscope}%
\begin{pgfscope}%
\pgfsys@transformshift{3.193072in}{1.010753in}%
\pgfsys@useobject{currentmarker}{}%
\end{pgfscope}%
\begin{pgfscope}%
\pgfsys@transformshift{3.194186in}{1.008064in}%
\pgfsys@useobject{currentmarker}{}%
\end{pgfscope}%
\begin{pgfscope}%
\pgfsys@transformshift{3.195297in}{1.000424in}%
\pgfsys@useobject{currentmarker}{}%
\end{pgfscope}%
\begin{pgfscope}%
\pgfsys@transformshift{3.196405in}{0.990929in}%
\pgfsys@useobject{currentmarker}{}%
\end{pgfscope}%
\begin{pgfscope}%
\pgfsys@transformshift{3.197509in}{1.007623in}%
\pgfsys@useobject{currentmarker}{}%
\end{pgfscope}%
\begin{pgfscope}%
\pgfsys@transformshift{3.198611in}{0.994371in}%
\pgfsys@useobject{currentmarker}{}%
\end{pgfscope}%
\begin{pgfscope}%
\pgfsys@transformshift{3.199710in}{0.974799in}%
\pgfsys@useobject{currentmarker}{}%
\end{pgfscope}%
\begin{pgfscope}%
\pgfsys@transformshift{3.200805in}{0.987268in}%
\pgfsys@useobject{currentmarker}{}%
\end{pgfscope}%
\begin{pgfscope}%
\pgfsys@transformshift{3.201898in}{0.973956in}%
\pgfsys@useobject{currentmarker}{}%
\end{pgfscope}%
\begin{pgfscope}%
\pgfsys@transformshift{3.202988in}{0.975147in}%
\pgfsys@useobject{currentmarker}{}%
\end{pgfscope}%
\begin{pgfscope}%
\pgfsys@transformshift{3.204075in}{1.004160in}%
\pgfsys@useobject{currentmarker}{}%
\end{pgfscope}%
\begin{pgfscope}%
\pgfsys@transformshift{3.205159in}{1.001146in}%
\pgfsys@useobject{currentmarker}{}%
\end{pgfscope}%
\begin{pgfscope}%
\pgfsys@transformshift{3.206240in}{0.982319in}%
\pgfsys@useobject{currentmarker}{}%
\end{pgfscope}%
\begin{pgfscope}%
\pgfsys@transformshift{3.207318in}{1.005391in}%
\pgfsys@useobject{currentmarker}{}%
\end{pgfscope}%
\begin{pgfscope}%
\pgfsys@transformshift{3.208393in}{1.020078in}%
\pgfsys@useobject{currentmarker}{}%
\end{pgfscope}%
\begin{pgfscope}%
\pgfsys@transformshift{3.209465in}{0.992390in}%
\pgfsys@useobject{currentmarker}{}%
\end{pgfscope}%
\begin{pgfscope}%
\pgfsys@transformshift{3.210535in}{0.978949in}%
\pgfsys@useobject{currentmarker}{}%
\end{pgfscope}%
\begin{pgfscope}%
\pgfsys@transformshift{3.211602in}{0.979093in}%
\pgfsys@useobject{currentmarker}{}%
\end{pgfscope}%
\begin{pgfscope}%
\pgfsys@transformshift{3.212666in}{0.986082in}%
\pgfsys@useobject{currentmarker}{}%
\end{pgfscope}%
\begin{pgfscope}%
\pgfsys@transformshift{3.213727in}{0.980121in}%
\pgfsys@useobject{currentmarker}{}%
\end{pgfscope}%
\begin{pgfscope}%
\pgfsys@transformshift{3.214785in}{0.977210in}%
\pgfsys@useobject{currentmarker}{}%
\end{pgfscope}%
\begin{pgfscope}%
\pgfsys@transformshift{3.215841in}{0.978557in}%
\pgfsys@useobject{currentmarker}{}%
\end{pgfscope}%
\begin{pgfscope}%
\pgfsys@transformshift{3.216894in}{0.979680in}%
\pgfsys@useobject{currentmarker}{}%
\end{pgfscope}%
\begin{pgfscope}%
\pgfsys@transformshift{3.217944in}{0.982758in}%
\pgfsys@useobject{currentmarker}{}%
\end{pgfscope}%
\begin{pgfscope}%
\pgfsys@transformshift{3.218991in}{0.991929in}%
\pgfsys@useobject{currentmarker}{}%
\end{pgfscope}%
\begin{pgfscope}%
\pgfsys@transformshift{3.220036in}{0.978910in}%
\pgfsys@useobject{currentmarker}{}%
\end{pgfscope}%
\begin{pgfscope}%
\pgfsys@transformshift{3.221078in}{0.985131in}%
\pgfsys@useobject{currentmarker}{}%
\end{pgfscope}%
\begin{pgfscope}%
\pgfsys@transformshift{3.222117in}{0.985535in}%
\pgfsys@useobject{currentmarker}{}%
\end{pgfscope}%
\begin{pgfscope}%
\pgfsys@transformshift{3.223154in}{0.980578in}%
\pgfsys@useobject{currentmarker}{}%
\end{pgfscope}%
\begin{pgfscope}%
\pgfsys@transformshift{3.224188in}{0.977162in}%
\pgfsys@useobject{currentmarker}{}%
\end{pgfscope}%
\begin{pgfscope}%
\pgfsys@transformshift{3.225219in}{0.970864in}%
\pgfsys@useobject{currentmarker}{}%
\end{pgfscope}%
\begin{pgfscope}%
\pgfsys@transformshift{3.226248in}{0.974486in}%
\pgfsys@useobject{currentmarker}{}%
\end{pgfscope}%
\begin{pgfscope}%
\pgfsys@transformshift{3.227274in}{0.966950in}%
\pgfsys@useobject{currentmarker}{}%
\end{pgfscope}%
\begin{pgfscope}%
\pgfsys@transformshift{3.228297in}{0.968223in}%
\pgfsys@useobject{currentmarker}{}%
\end{pgfscope}%
\begin{pgfscope}%
\pgfsys@transformshift{3.229318in}{0.978587in}%
\pgfsys@useobject{currentmarker}{}%
\end{pgfscope}%
\begin{pgfscope}%
\pgfsys@transformshift{3.230336in}{0.975673in}%
\pgfsys@useobject{currentmarker}{}%
\end{pgfscope}%
\begin{pgfscope}%
\pgfsys@transformshift{3.231352in}{0.986317in}%
\pgfsys@useobject{currentmarker}{}%
\end{pgfscope}%
\begin{pgfscope}%
\pgfsys@transformshift{3.232365in}{0.982326in}%
\pgfsys@useobject{currentmarker}{}%
\end{pgfscope}%
\begin{pgfscope}%
\pgfsys@transformshift{3.233376in}{0.985089in}%
\pgfsys@useobject{currentmarker}{}%
\end{pgfscope}%
\begin{pgfscope}%
\pgfsys@transformshift{3.234384in}{0.994577in}%
\pgfsys@useobject{currentmarker}{}%
\end{pgfscope}%
\begin{pgfscope}%
\pgfsys@transformshift{3.235390in}{0.987051in}%
\pgfsys@useobject{currentmarker}{}%
\end{pgfscope}%
\begin{pgfscope}%
\pgfsys@transformshift{3.236393in}{0.975234in}%
\pgfsys@useobject{currentmarker}{}%
\end{pgfscope}%
\begin{pgfscope}%
\pgfsys@transformshift{3.237394in}{0.979640in}%
\pgfsys@useobject{currentmarker}{}%
\end{pgfscope}%
\begin{pgfscope}%
\pgfsys@transformshift{3.238392in}{0.986156in}%
\pgfsys@useobject{currentmarker}{}%
\end{pgfscope}%
\begin{pgfscope}%
\pgfsys@transformshift{3.239387in}{0.988260in}%
\pgfsys@useobject{currentmarker}{}%
\end{pgfscope}%
\begin{pgfscope}%
\pgfsys@transformshift{3.240381in}{0.975073in}%
\pgfsys@useobject{currentmarker}{}%
\end{pgfscope}%
\begin{pgfscope}%
\pgfsys@transformshift{3.241372in}{0.964085in}%
\pgfsys@useobject{currentmarker}{}%
\end{pgfscope}%
\begin{pgfscope}%
\pgfsys@transformshift{3.242360in}{0.966061in}%
\pgfsys@useobject{currentmarker}{}%
\end{pgfscope}%
\begin{pgfscope}%
\pgfsys@transformshift{3.243346in}{0.962312in}%
\pgfsys@useobject{currentmarker}{}%
\end{pgfscope}%
\begin{pgfscope}%
\pgfsys@transformshift{3.244330in}{0.973201in}%
\pgfsys@useobject{currentmarker}{}%
\end{pgfscope}%
\begin{pgfscope}%
\pgfsys@transformshift{3.245311in}{0.973405in}%
\pgfsys@useobject{currentmarker}{}%
\end{pgfscope}%
\begin{pgfscope}%
\pgfsys@transformshift{3.246290in}{0.975052in}%
\pgfsys@useobject{currentmarker}{}%
\end{pgfscope}%
\begin{pgfscope}%
\pgfsys@transformshift{3.247266in}{0.976873in}%
\pgfsys@useobject{currentmarker}{}%
\end{pgfscope}%
\begin{pgfscope}%
\pgfsys@transformshift{3.248240in}{0.964186in}%
\pgfsys@useobject{currentmarker}{}%
\end{pgfscope}%
\begin{pgfscope}%
\pgfsys@transformshift{3.249212in}{0.960574in}%
\pgfsys@useobject{currentmarker}{}%
\end{pgfscope}%
\begin{pgfscope}%
\pgfsys@transformshift{3.250181in}{0.960859in}%
\pgfsys@useobject{currentmarker}{}%
\end{pgfscope}%
\begin{pgfscope}%
\pgfsys@transformshift{3.251148in}{0.967236in}%
\pgfsys@useobject{currentmarker}{}%
\end{pgfscope}%
\begin{pgfscope}%
\pgfsys@transformshift{3.252113in}{0.982371in}%
\pgfsys@useobject{currentmarker}{}%
\end{pgfscope}%
\begin{pgfscope}%
\pgfsys@transformshift{3.253076in}{0.979899in}%
\pgfsys@useobject{currentmarker}{}%
\end{pgfscope}%
\begin{pgfscope}%
\pgfsys@transformshift{3.254036in}{0.979619in}%
\pgfsys@useobject{currentmarker}{}%
\end{pgfscope}%
\begin{pgfscope}%
\pgfsys@transformshift{3.254994in}{0.971352in}%
\pgfsys@useobject{currentmarker}{}%
\end{pgfscope}%
\begin{pgfscope}%
\pgfsys@transformshift{3.255949in}{0.948243in}%
\pgfsys@useobject{currentmarker}{}%
\end{pgfscope}%
\begin{pgfscope}%
\pgfsys@transformshift{3.256903in}{0.962994in}%
\pgfsys@useobject{currentmarker}{}%
\end{pgfscope}%
\begin{pgfscope}%
\pgfsys@transformshift{3.257854in}{0.973464in}%
\pgfsys@useobject{currentmarker}{}%
\end{pgfscope}%
\begin{pgfscope}%
\pgfsys@transformshift{3.258803in}{0.965509in}%
\pgfsys@useobject{currentmarker}{}%
\end{pgfscope}%
\begin{pgfscope}%
\pgfsys@transformshift{3.259750in}{0.969650in}%
\pgfsys@useobject{currentmarker}{}%
\end{pgfscope}%
\begin{pgfscope}%
\pgfsys@transformshift{3.260694in}{0.956074in}%
\pgfsys@useobject{currentmarker}{}%
\end{pgfscope}%
\begin{pgfscope}%
\pgfsys@transformshift{3.261636in}{0.973691in}%
\pgfsys@useobject{currentmarker}{}%
\end{pgfscope}%
\begin{pgfscope}%
\pgfsys@transformshift{3.262576in}{0.975861in}%
\pgfsys@useobject{currentmarker}{}%
\end{pgfscope}%
\begin{pgfscope}%
\pgfsys@transformshift{3.263514in}{0.970735in}%
\pgfsys@useobject{currentmarker}{}%
\end{pgfscope}%
\begin{pgfscope}%
\pgfsys@transformshift{3.264450in}{0.960622in}%
\pgfsys@useobject{currentmarker}{}%
\end{pgfscope}%
\begin{pgfscope}%
\pgfsys@transformshift{3.265383in}{0.956437in}%
\pgfsys@useobject{currentmarker}{}%
\end{pgfscope}%
\begin{pgfscope}%
\pgfsys@transformshift{3.266315in}{0.983390in}%
\pgfsys@useobject{currentmarker}{}%
\end{pgfscope}%
\begin{pgfscope}%
\pgfsys@transformshift{3.267244in}{0.975429in}%
\pgfsys@useobject{currentmarker}{}%
\end{pgfscope}%
\begin{pgfscope}%
\pgfsys@transformshift{3.268171in}{0.965553in}%
\pgfsys@useobject{currentmarker}{}%
\end{pgfscope}%
\begin{pgfscope}%
\pgfsys@transformshift{3.269096in}{0.949367in}%
\pgfsys@useobject{currentmarker}{}%
\end{pgfscope}%
\begin{pgfscope}%
\pgfsys@transformshift{3.270019in}{0.960578in}%
\pgfsys@useobject{currentmarker}{}%
\end{pgfscope}%
\begin{pgfscope}%
\pgfsys@transformshift{3.270940in}{0.971392in}%
\pgfsys@useobject{currentmarker}{}%
\end{pgfscope}%
\begin{pgfscope}%
\pgfsys@transformshift{3.271859in}{0.965564in}%
\pgfsys@useobject{currentmarker}{}%
\end{pgfscope}%
\begin{pgfscope}%
\pgfsys@transformshift{3.272775in}{0.971464in}%
\pgfsys@useobject{currentmarker}{}%
\end{pgfscope}%
\begin{pgfscope}%
\pgfsys@transformshift{3.273690in}{0.982255in}%
\pgfsys@useobject{currentmarker}{}%
\end{pgfscope}%
\begin{pgfscope}%
\pgfsys@transformshift{3.274602in}{0.968804in}%
\pgfsys@useobject{currentmarker}{}%
\end{pgfscope}%
\begin{pgfscope}%
\pgfsys@transformshift{3.275513in}{0.963787in}%
\pgfsys@useobject{currentmarker}{}%
\end{pgfscope}%
\begin{pgfscope}%
\pgfsys@transformshift{3.276421in}{0.959708in}%
\pgfsys@useobject{currentmarker}{}%
\end{pgfscope}%
\begin{pgfscope}%
\pgfsys@transformshift{3.277327in}{0.940365in}%
\pgfsys@useobject{currentmarker}{}%
\end{pgfscope}%
\begin{pgfscope}%
\pgfsys@transformshift{3.278232in}{0.955078in}%
\pgfsys@useobject{currentmarker}{}%
\end{pgfscope}%
\begin{pgfscope}%
\pgfsys@transformshift{3.279134in}{0.960077in}%
\pgfsys@useobject{currentmarker}{}%
\end{pgfscope}%
\begin{pgfscope}%
\pgfsys@transformshift{3.280034in}{0.966244in}%
\pgfsys@useobject{currentmarker}{}%
\end{pgfscope}%
\begin{pgfscope}%
\pgfsys@transformshift{3.280932in}{0.960303in}%
\pgfsys@useobject{currentmarker}{}%
\end{pgfscope}%
\begin{pgfscope}%
\pgfsys@transformshift{3.281829in}{0.960809in}%
\pgfsys@useobject{currentmarker}{}%
\end{pgfscope}%
\begin{pgfscope}%
\pgfsys@transformshift{3.282723in}{0.965125in}%
\pgfsys@useobject{currentmarker}{}%
\end{pgfscope}%
\begin{pgfscope}%
\pgfsys@transformshift{3.283615in}{0.947064in}%
\pgfsys@useobject{currentmarker}{}%
\end{pgfscope}%
\begin{pgfscope}%
\pgfsys@transformshift{3.284505in}{0.949727in}%
\pgfsys@useobject{currentmarker}{}%
\end{pgfscope}%
\begin{pgfscope}%
\pgfsys@transformshift{3.285394in}{0.960823in}%
\pgfsys@useobject{currentmarker}{}%
\end{pgfscope}%
\begin{pgfscope}%
\pgfsys@transformshift{3.286280in}{0.969553in}%
\pgfsys@useobject{currentmarker}{}%
\end{pgfscope}%
\begin{pgfscope}%
\pgfsys@transformshift{3.287165in}{0.972565in}%
\pgfsys@useobject{currentmarker}{}%
\end{pgfscope}%
\begin{pgfscope}%
\pgfsys@transformshift{3.288047in}{0.958161in}%
\pgfsys@useobject{currentmarker}{}%
\end{pgfscope}%
\begin{pgfscope}%
\pgfsys@transformshift{3.288928in}{0.961548in}%
\pgfsys@useobject{currentmarker}{}%
\end{pgfscope}%
\begin{pgfscope}%
\pgfsys@transformshift{3.289807in}{0.943847in}%
\pgfsys@useobject{currentmarker}{}%
\end{pgfscope}%
\begin{pgfscope}%
\pgfsys@transformshift{3.290683in}{0.930144in}%
\pgfsys@useobject{currentmarker}{}%
\end{pgfscope}%
\begin{pgfscope}%
\pgfsys@transformshift{3.291558in}{0.913250in}%
\pgfsys@useobject{currentmarker}{}%
\end{pgfscope}%
\begin{pgfscope}%
\pgfsys@transformshift{3.292431in}{0.937747in}%
\pgfsys@useobject{currentmarker}{}%
\end{pgfscope}%
\begin{pgfscope}%
\pgfsys@transformshift{3.293302in}{0.959867in}%
\pgfsys@useobject{currentmarker}{}%
\end{pgfscope}%
\begin{pgfscope}%
\pgfsys@transformshift{3.294172in}{0.956046in}%
\pgfsys@useobject{currentmarker}{}%
\end{pgfscope}%
\begin{pgfscope}%
\pgfsys@transformshift{3.295039in}{0.976226in}%
\pgfsys@useobject{currentmarker}{}%
\end{pgfscope}%
\begin{pgfscope}%
\pgfsys@transformshift{3.295904in}{0.973309in}%
\pgfsys@useobject{currentmarker}{}%
\end{pgfscope}%
\begin{pgfscope}%
\pgfsys@transformshift{3.296768in}{0.951419in}%
\pgfsys@useobject{currentmarker}{}%
\end{pgfscope}%
\begin{pgfscope}%
\pgfsys@transformshift{3.297630in}{0.957470in}%
\pgfsys@useobject{currentmarker}{}%
\end{pgfscope}%
\begin{pgfscope}%
\pgfsys@transformshift{3.298490in}{0.967636in}%
\pgfsys@useobject{currentmarker}{}%
\end{pgfscope}%
\begin{pgfscope}%
\pgfsys@transformshift{3.299348in}{0.962160in}%
\pgfsys@useobject{currentmarker}{}%
\end{pgfscope}%
\begin{pgfscope}%
\pgfsys@transformshift{3.300204in}{0.957097in}%
\pgfsys@useobject{currentmarker}{}%
\end{pgfscope}%
\begin{pgfscope}%
\pgfsys@transformshift{3.301059in}{0.926436in}%
\pgfsys@useobject{currentmarker}{}%
\end{pgfscope}%
\begin{pgfscope}%
\pgfsys@transformshift{3.301912in}{0.927218in}%
\pgfsys@useobject{currentmarker}{}%
\end{pgfscope}%
\begin{pgfscope}%
\pgfsys@transformshift{3.302763in}{0.943130in}%
\pgfsys@useobject{currentmarker}{}%
\end{pgfscope}%
\begin{pgfscope}%
\pgfsys@transformshift{3.303612in}{0.953943in}%
\pgfsys@useobject{currentmarker}{}%
\end{pgfscope}%
\begin{pgfscope}%
\pgfsys@transformshift{3.304459in}{0.948315in}%
\pgfsys@useobject{currentmarker}{}%
\end{pgfscope}%
\begin{pgfscope}%
\pgfsys@transformshift{3.305305in}{0.951773in}%
\pgfsys@useobject{currentmarker}{}%
\end{pgfscope}%
\begin{pgfscope}%
\pgfsys@transformshift{3.306148in}{0.938070in}%
\pgfsys@useobject{currentmarker}{}%
\end{pgfscope}%
\begin{pgfscope}%
\pgfsys@transformshift{3.306990in}{0.930736in}%
\pgfsys@useobject{currentmarker}{}%
\end{pgfscope}%
\begin{pgfscope}%
\pgfsys@transformshift{3.307831in}{0.947911in}%
\pgfsys@useobject{currentmarker}{}%
\end{pgfscope}%
\begin{pgfscope}%
\pgfsys@transformshift{3.308669in}{0.941842in}%
\pgfsys@useobject{currentmarker}{}%
\end{pgfscope}%
\begin{pgfscope}%
\pgfsys@transformshift{3.309506in}{0.922205in}%
\pgfsys@useobject{currentmarker}{}%
\end{pgfscope}%
\begin{pgfscope}%
\pgfsys@transformshift{3.310341in}{0.917787in}%
\pgfsys@useobject{currentmarker}{}%
\end{pgfscope}%
\begin{pgfscope}%
\pgfsys@transformshift{3.311175in}{0.940747in}%
\pgfsys@useobject{currentmarker}{}%
\end{pgfscope}%
\begin{pgfscope}%
\pgfsys@transformshift{3.312006in}{0.938402in}%
\pgfsys@useobject{currentmarker}{}%
\end{pgfscope}%
\begin{pgfscope}%
\pgfsys@transformshift{3.312836in}{0.943605in}%
\pgfsys@useobject{currentmarker}{}%
\end{pgfscope}%
\begin{pgfscope}%
\pgfsys@transformshift{3.313664in}{0.945520in}%
\pgfsys@useobject{currentmarker}{}%
\end{pgfscope}%
\begin{pgfscope}%
\pgfsys@transformshift{3.314491in}{0.944715in}%
\pgfsys@useobject{currentmarker}{}%
\end{pgfscope}%
\begin{pgfscope}%
\pgfsys@transformshift{3.315316in}{0.947152in}%
\pgfsys@useobject{currentmarker}{}%
\end{pgfscope}%
\begin{pgfscope}%
\pgfsys@transformshift{3.316139in}{0.957920in}%
\pgfsys@useobject{currentmarker}{}%
\end{pgfscope}%
\begin{pgfscope}%
\pgfsys@transformshift{3.316960in}{0.957523in}%
\pgfsys@useobject{currentmarker}{}%
\end{pgfscope}%
\begin{pgfscope}%
\pgfsys@transformshift{3.317780in}{0.948501in}%
\pgfsys@useobject{currentmarker}{}%
\end{pgfscope}%
\begin{pgfscope}%
\pgfsys@transformshift{3.318598in}{0.917371in}%
\pgfsys@useobject{currentmarker}{}%
\end{pgfscope}%
\begin{pgfscope}%
\pgfsys@transformshift{3.319415in}{0.926623in}%
\pgfsys@useobject{currentmarker}{}%
\end{pgfscope}%
\begin{pgfscope}%
\pgfsys@transformshift{3.320230in}{0.937338in}%
\pgfsys@useobject{currentmarker}{}%
\end{pgfscope}%
\begin{pgfscope}%
\pgfsys@transformshift{3.321043in}{0.957542in}%
\pgfsys@useobject{currentmarker}{}%
\end{pgfscope}%
\begin{pgfscope}%
\pgfsys@transformshift{3.321855in}{0.929234in}%
\pgfsys@useobject{currentmarker}{}%
\end{pgfscope}%
\begin{pgfscope}%
\pgfsys@transformshift{3.322665in}{0.926647in}%
\pgfsys@useobject{currentmarker}{}%
\end{pgfscope}%
\begin{pgfscope}%
\pgfsys@transformshift{3.323473in}{0.937056in}%
\pgfsys@useobject{currentmarker}{}%
\end{pgfscope}%
\begin{pgfscope}%
\pgfsys@transformshift{3.324280in}{0.937869in}%
\pgfsys@useobject{currentmarker}{}%
\end{pgfscope}%
\begin{pgfscope}%
\pgfsys@transformshift{3.325085in}{0.939651in}%
\pgfsys@useobject{currentmarker}{}%
\end{pgfscope}%
\begin{pgfscope}%
\pgfsys@transformshift{3.325888in}{0.926794in}%
\pgfsys@useobject{currentmarker}{}%
\end{pgfscope}%
\begin{pgfscope}%
\pgfsys@transformshift{3.326690in}{0.935702in}%
\pgfsys@useobject{currentmarker}{}%
\end{pgfscope}%
\begin{pgfscope}%
\pgfsys@transformshift{3.327490in}{0.931606in}%
\pgfsys@useobject{currentmarker}{}%
\end{pgfscope}%
\begin{pgfscope}%
\pgfsys@transformshift{3.328289in}{0.942998in}%
\pgfsys@useobject{currentmarker}{}%
\end{pgfscope}%
\begin{pgfscope}%
\pgfsys@transformshift{3.329086in}{0.954852in}%
\pgfsys@useobject{currentmarker}{}%
\end{pgfscope}%
\begin{pgfscope}%
\pgfsys@transformshift{3.329882in}{0.938040in}%
\pgfsys@useobject{currentmarker}{}%
\end{pgfscope}%
\begin{pgfscope}%
\pgfsys@transformshift{3.330676in}{0.939529in}%
\pgfsys@useobject{currentmarker}{}%
\end{pgfscope}%
\begin{pgfscope}%
\pgfsys@transformshift{3.331468in}{0.934982in}%
\pgfsys@useobject{currentmarker}{}%
\end{pgfscope}%
\begin{pgfscope}%
\pgfsys@transformshift{3.332259in}{0.940322in}%
\pgfsys@useobject{currentmarker}{}%
\end{pgfscope}%
\begin{pgfscope}%
\pgfsys@transformshift{3.333049in}{0.953281in}%
\pgfsys@useobject{currentmarker}{}%
\end{pgfscope}%
\begin{pgfscope}%
\pgfsys@transformshift{3.333837in}{0.940721in}%
\pgfsys@useobject{currentmarker}{}%
\end{pgfscope}%
\begin{pgfscope}%
\pgfsys@transformshift{3.334623in}{0.934445in}%
\pgfsys@useobject{currentmarker}{}%
\end{pgfscope}%
\begin{pgfscope}%
\pgfsys@transformshift{3.335408in}{0.944134in}%
\pgfsys@useobject{currentmarker}{}%
\end{pgfscope}%
\begin{pgfscope}%
\pgfsys@transformshift{3.336191in}{0.946047in}%
\pgfsys@useobject{currentmarker}{}%
\end{pgfscope}%
\begin{pgfscope}%
\pgfsys@transformshift{3.336973in}{0.938175in}%
\pgfsys@useobject{currentmarker}{}%
\end{pgfscope}%
\begin{pgfscope}%
\pgfsys@transformshift{3.337753in}{0.926802in}%
\pgfsys@useobject{currentmarker}{}%
\end{pgfscope}%
\begin{pgfscope}%
\pgfsys@transformshift{3.338531in}{0.932429in}%
\pgfsys@useobject{currentmarker}{}%
\end{pgfscope}%
\begin{pgfscope}%
\pgfsys@transformshift{3.339309in}{0.945194in}%
\pgfsys@useobject{currentmarker}{}%
\end{pgfscope}%
\begin{pgfscope}%
\pgfsys@transformshift{3.340084in}{0.940250in}%
\pgfsys@useobject{currentmarker}{}%
\end{pgfscope}%
\begin{pgfscope}%
\pgfsys@transformshift{3.340859in}{0.956677in}%
\pgfsys@useobject{currentmarker}{}%
\end{pgfscope}%
\begin{pgfscope}%
\pgfsys@transformshift{3.341631in}{0.945970in}%
\pgfsys@useobject{currentmarker}{}%
\end{pgfscope}%
\begin{pgfscope}%
\pgfsys@transformshift{3.342403in}{0.946485in}%
\pgfsys@useobject{currentmarker}{}%
\end{pgfscope}%
\begin{pgfscope}%
\pgfsys@transformshift{3.343172in}{0.953667in}%
\pgfsys@useobject{currentmarker}{}%
\end{pgfscope}%
\begin{pgfscope}%
\pgfsys@transformshift{3.343941in}{0.954256in}%
\pgfsys@useobject{currentmarker}{}%
\end{pgfscope}%
\begin{pgfscope}%
\pgfsys@transformshift{3.344708in}{0.941144in}%
\pgfsys@useobject{currentmarker}{}%
\end{pgfscope}%
\begin{pgfscope}%
\pgfsys@transformshift{3.345473in}{0.928028in}%
\pgfsys@useobject{currentmarker}{}%
\end{pgfscope}%
\begin{pgfscope}%
\pgfsys@transformshift{3.346237in}{0.940092in}%
\pgfsys@useobject{currentmarker}{}%
\end{pgfscope}%
\begin{pgfscope}%
\pgfsys@transformshift{3.347000in}{0.943738in}%
\pgfsys@useobject{currentmarker}{}%
\end{pgfscope}%
\begin{pgfscope}%
\pgfsys@transformshift{3.347761in}{0.933839in}%
\pgfsys@useobject{currentmarker}{}%
\end{pgfscope}%
\begin{pgfscope}%
\pgfsys@transformshift{3.348520in}{0.931894in}%
\pgfsys@useobject{currentmarker}{}%
\end{pgfscope}%
\begin{pgfscope}%
\pgfsys@transformshift{3.349279in}{0.936246in}%
\pgfsys@useobject{currentmarker}{}%
\end{pgfscope}%
\begin{pgfscope}%
\pgfsys@transformshift{3.350035in}{0.936803in}%
\pgfsys@useobject{currentmarker}{}%
\end{pgfscope}%
\begin{pgfscope}%
\pgfsys@transformshift{3.350791in}{0.933558in}%
\pgfsys@useobject{currentmarker}{}%
\end{pgfscope}%
\begin{pgfscope}%
\pgfsys@transformshift{3.351545in}{0.931110in}%
\pgfsys@useobject{currentmarker}{}%
\end{pgfscope}%
\begin{pgfscope}%
\pgfsys@transformshift{3.352297in}{0.917416in}%
\pgfsys@useobject{currentmarker}{}%
\end{pgfscope}%
\begin{pgfscope}%
\pgfsys@transformshift{3.353049in}{0.905653in}%
\pgfsys@useobject{currentmarker}{}%
\end{pgfscope}%
\begin{pgfscope}%
\pgfsys@transformshift{3.353798in}{0.919656in}%
\pgfsys@useobject{currentmarker}{}%
\end{pgfscope}%
\begin{pgfscope}%
\pgfsys@transformshift{3.354547in}{0.931310in}%
\pgfsys@useobject{currentmarker}{}%
\end{pgfscope}%
\begin{pgfscope}%
\pgfsys@transformshift{3.355294in}{0.927533in}%
\pgfsys@useobject{currentmarker}{}%
\end{pgfscope}%
\begin{pgfscope}%
\pgfsys@transformshift{3.356039in}{0.909448in}%
\pgfsys@useobject{currentmarker}{}%
\end{pgfscope}%
\begin{pgfscope}%
\pgfsys@transformshift{3.356784in}{0.917889in}%
\pgfsys@useobject{currentmarker}{}%
\end{pgfscope}%
\begin{pgfscope}%
\pgfsys@transformshift{3.357527in}{0.922945in}%
\pgfsys@useobject{currentmarker}{}%
\end{pgfscope}%
\begin{pgfscope}%
\pgfsys@transformshift{3.358268in}{0.947336in}%
\pgfsys@useobject{currentmarker}{}%
\end{pgfscope}%
\begin{pgfscope}%
\pgfsys@transformshift{3.359008in}{0.947114in}%
\pgfsys@useobject{currentmarker}{}%
\end{pgfscope}%
\begin{pgfscope}%
\pgfsys@transformshift{3.359747in}{0.931497in}%
\pgfsys@useobject{currentmarker}{}%
\end{pgfscope}%
\begin{pgfscope}%
\pgfsys@transformshift{3.360485in}{0.940234in}%
\pgfsys@useobject{currentmarker}{}%
\end{pgfscope}%
\begin{pgfscope}%
\pgfsys@transformshift{3.361221in}{0.931788in}%
\pgfsys@useobject{currentmarker}{}%
\end{pgfscope}%
\begin{pgfscope}%
\pgfsys@transformshift{3.361955in}{0.935558in}%
\pgfsys@useobject{currentmarker}{}%
\end{pgfscope}%
\begin{pgfscope}%
\pgfsys@transformshift{3.362689in}{0.941072in}%
\pgfsys@useobject{currentmarker}{}%
\end{pgfscope}%
\begin{pgfscope}%
\pgfsys@transformshift{3.363421in}{0.949094in}%
\pgfsys@useobject{currentmarker}{}%
\end{pgfscope}%
\begin{pgfscope}%
\pgfsys@transformshift{3.364152in}{0.944139in}%
\pgfsys@useobject{currentmarker}{}%
\end{pgfscope}%
\begin{pgfscope}%
\pgfsys@transformshift{3.364881in}{0.927956in}%
\pgfsys@useobject{currentmarker}{}%
\end{pgfscope}%
\begin{pgfscope}%
\pgfsys@transformshift{3.365609in}{0.931912in}%
\pgfsys@useobject{currentmarker}{}%
\end{pgfscope}%
\begin{pgfscope}%
\pgfsys@transformshift{3.366336in}{0.931765in}%
\pgfsys@useobject{currentmarker}{}%
\end{pgfscope}%
\begin{pgfscope}%
\pgfsys@transformshift{3.367062in}{0.915523in}%
\pgfsys@useobject{currentmarker}{}%
\end{pgfscope}%
\begin{pgfscope}%
\pgfsys@transformshift{3.367786in}{0.919920in}%
\pgfsys@useobject{currentmarker}{}%
\end{pgfscope}%
\begin{pgfscope}%
\pgfsys@transformshift{3.368509in}{0.920189in}%
\pgfsys@useobject{currentmarker}{}%
\end{pgfscope}%
\begin{pgfscope}%
\pgfsys@transformshift{3.369231in}{0.917958in}%
\pgfsys@useobject{currentmarker}{}%
\end{pgfscope}%
\begin{pgfscope}%
\pgfsys@transformshift{3.369951in}{0.917563in}%
\pgfsys@useobject{currentmarker}{}%
\end{pgfscope}%
\begin{pgfscope}%
\pgfsys@transformshift{3.370670in}{0.918738in}%
\pgfsys@useobject{currentmarker}{}%
\end{pgfscope}%
\begin{pgfscope}%
\pgfsys@transformshift{3.371388in}{0.912577in}%
\pgfsys@useobject{currentmarker}{}%
\end{pgfscope}%
\begin{pgfscope}%
\pgfsys@transformshift{3.372104in}{0.915778in}%
\pgfsys@useobject{currentmarker}{}%
\end{pgfscope}%
\begin{pgfscope}%
\pgfsys@transformshift{3.372820in}{0.919455in}%
\pgfsys@useobject{currentmarker}{}%
\end{pgfscope}%
\begin{pgfscope}%
\pgfsys@transformshift{3.373534in}{0.912543in}%
\pgfsys@useobject{currentmarker}{}%
\end{pgfscope}%
\begin{pgfscope}%
\pgfsys@transformshift{3.374246in}{0.907907in}%
\pgfsys@useobject{currentmarker}{}%
\end{pgfscope}%
\begin{pgfscope}%
\pgfsys@transformshift{3.374958in}{0.910508in}%
\pgfsys@useobject{currentmarker}{}%
\end{pgfscope}%
\begin{pgfscope}%
\pgfsys@transformshift{3.375668in}{0.907977in}%
\pgfsys@useobject{currentmarker}{}%
\end{pgfscope}%
\begin{pgfscope}%
\pgfsys@transformshift{3.376377in}{0.905663in}%
\pgfsys@useobject{currentmarker}{}%
\end{pgfscope}%
\begin{pgfscope}%
\pgfsys@transformshift{3.377085in}{0.912452in}%
\pgfsys@useobject{currentmarker}{}%
\end{pgfscope}%
\begin{pgfscope}%
\pgfsys@transformshift{3.377791in}{0.919120in}%
\pgfsys@useobject{currentmarker}{}%
\end{pgfscope}%
\begin{pgfscope}%
\pgfsys@transformshift{3.378497in}{0.906635in}%
\pgfsys@useobject{currentmarker}{}%
\end{pgfscope}%
\begin{pgfscope}%
\pgfsys@transformshift{3.379201in}{0.924144in}%
\pgfsys@useobject{currentmarker}{}%
\end{pgfscope}%
\begin{pgfscope}%
\pgfsys@transformshift{3.379903in}{0.928586in}%
\pgfsys@useobject{currentmarker}{}%
\end{pgfscope}%
\begin{pgfscope}%
\pgfsys@transformshift{3.380605in}{0.930965in}%
\pgfsys@useobject{currentmarker}{}%
\end{pgfscope}%
\begin{pgfscope}%
\pgfsys@transformshift{3.381305in}{0.920272in}%
\pgfsys@useobject{currentmarker}{}%
\end{pgfscope}%
\begin{pgfscope}%
\pgfsys@transformshift{3.382005in}{0.910036in}%
\pgfsys@useobject{currentmarker}{}%
\end{pgfscope}%
\begin{pgfscope}%
\pgfsys@transformshift{3.382702in}{0.913628in}%
\pgfsys@useobject{currentmarker}{}%
\end{pgfscope}%
\begin{pgfscope}%
\pgfsys@transformshift{3.383399in}{0.906021in}%
\pgfsys@useobject{currentmarker}{}%
\end{pgfscope}%
\begin{pgfscope}%
\pgfsys@transformshift{3.384095in}{0.904670in}%
\pgfsys@useobject{currentmarker}{}%
\end{pgfscope}%
\begin{pgfscope}%
\pgfsys@transformshift{3.384789in}{0.902574in}%
\pgfsys@useobject{currentmarker}{}%
\end{pgfscope}%
\begin{pgfscope}%
\pgfsys@transformshift{3.385482in}{0.912397in}%
\pgfsys@useobject{currentmarker}{}%
\end{pgfscope}%
\begin{pgfscope}%
\pgfsys@transformshift{3.386174in}{0.910923in}%
\pgfsys@useobject{currentmarker}{}%
\end{pgfscope}%
\begin{pgfscope}%
\pgfsys@transformshift{3.386865in}{0.917275in}%
\pgfsys@useobject{currentmarker}{}%
\end{pgfscope}%
\begin{pgfscope}%
\pgfsys@transformshift{3.387555in}{0.923310in}%
\pgfsys@useobject{currentmarker}{}%
\end{pgfscope}%
\begin{pgfscope}%
\pgfsys@transformshift{3.388243in}{0.916566in}%
\pgfsys@useobject{currentmarker}{}%
\end{pgfscope}%
\begin{pgfscope}%
\pgfsys@transformshift{3.388930in}{0.908611in}%
\pgfsys@useobject{currentmarker}{}%
\end{pgfscope}%
\begin{pgfscope}%
\pgfsys@transformshift{3.389616in}{0.920628in}%
\pgfsys@useobject{currentmarker}{}%
\end{pgfscope}%
\begin{pgfscope}%
\pgfsys@transformshift{3.390301in}{0.919725in}%
\pgfsys@useobject{currentmarker}{}%
\end{pgfscope}%
\begin{pgfscope}%
\pgfsys@transformshift{3.390985in}{0.921437in}%
\pgfsys@useobject{currentmarker}{}%
\end{pgfscope}%
\begin{pgfscope}%
\pgfsys@transformshift{3.391668in}{0.915358in}%
\pgfsys@useobject{currentmarker}{}%
\end{pgfscope}%
\begin{pgfscope}%
\pgfsys@transformshift{3.392349in}{0.902086in}%
\pgfsys@useobject{currentmarker}{}%
\end{pgfscope}%
\begin{pgfscope}%
\pgfsys@transformshift{3.393029in}{0.911765in}%
\pgfsys@useobject{currentmarker}{}%
\end{pgfscope}%
\begin{pgfscope}%
\pgfsys@transformshift{3.393709in}{0.905078in}%
\pgfsys@useobject{currentmarker}{}%
\end{pgfscope}%
\begin{pgfscope}%
\pgfsys@transformshift{3.394387in}{0.904860in}%
\pgfsys@useobject{currentmarker}{}%
\end{pgfscope}%
\begin{pgfscope}%
\pgfsys@transformshift{3.395063in}{0.904438in}%
\pgfsys@useobject{currentmarker}{}%
\end{pgfscope}%
\begin{pgfscope}%
\pgfsys@transformshift{3.395739in}{0.909675in}%
\pgfsys@useobject{currentmarker}{}%
\end{pgfscope}%
\begin{pgfscope}%
\pgfsys@transformshift{3.396414in}{0.911392in}%
\pgfsys@useobject{currentmarker}{}%
\end{pgfscope}%
\begin{pgfscope}%
\pgfsys@transformshift{3.397087in}{0.915491in}%
\pgfsys@useobject{currentmarker}{}%
\end{pgfscope}%
\begin{pgfscope}%
\pgfsys@transformshift{3.397760in}{0.915845in}%
\pgfsys@useobject{currentmarker}{}%
\end{pgfscope}%
\begin{pgfscope}%
\pgfsys@transformshift{3.398431in}{0.928433in}%
\pgfsys@useobject{currentmarker}{}%
\end{pgfscope}%
\begin{pgfscope}%
\pgfsys@transformshift{3.399101in}{0.912115in}%
\pgfsys@useobject{currentmarker}{}%
\end{pgfscope}%
\begin{pgfscope}%
\pgfsys@transformshift{3.399770in}{0.892133in}%
\pgfsys@useobject{currentmarker}{}%
\end{pgfscope}%
\begin{pgfscope}%
\pgfsys@transformshift{3.400438in}{0.901891in}%
\pgfsys@useobject{currentmarker}{}%
\end{pgfscope}%
\begin{pgfscope}%
\pgfsys@transformshift{3.401105in}{0.906473in}%
\pgfsys@useobject{currentmarker}{}%
\end{pgfscope}%
\begin{pgfscope}%
\pgfsys@transformshift{3.401771in}{0.904452in}%
\pgfsys@useobject{currentmarker}{}%
\end{pgfscope}%
\begin{pgfscope}%
\pgfsys@transformshift{3.402435in}{0.881612in}%
\pgfsys@useobject{currentmarker}{}%
\end{pgfscope}%
\begin{pgfscope}%
\pgfsys@transformshift{3.403099in}{0.889285in}%
\pgfsys@useobject{currentmarker}{}%
\end{pgfscope}%
\begin{pgfscope}%
\pgfsys@transformshift{3.403761in}{0.905337in}%
\pgfsys@useobject{currentmarker}{}%
\end{pgfscope}%
\begin{pgfscope}%
\pgfsys@transformshift{3.404423in}{0.911861in}%
\pgfsys@useobject{currentmarker}{}%
\end{pgfscope}%
\begin{pgfscope}%
\pgfsys@transformshift{3.405083in}{0.910459in}%
\pgfsys@useobject{currentmarker}{}%
\end{pgfscope}%
\begin{pgfscope}%
\pgfsys@transformshift{3.405742in}{0.922249in}%
\pgfsys@useobject{currentmarker}{}%
\end{pgfscope}%
\begin{pgfscope}%
\pgfsys@transformshift{3.406400in}{0.917043in}%
\pgfsys@useobject{currentmarker}{}%
\end{pgfscope}%
\begin{pgfscope}%
\pgfsys@transformshift{3.407057in}{0.904986in}%
\pgfsys@useobject{currentmarker}{}%
\end{pgfscope}%
\begin{pgfscope}%
\pgfsys@transformshift{3.407713in}{0.912852in}%
\pgfsys@useobject{currentmarker}{}%
\end{pgfscope}%
\begin{pgfscope}%
\pgfsys@transformshift{3.408368in}{0.917129in}%
\pgfsys@useobject{currentmarker}{}%
\end{pgfscope}%
\begin{pgfscope}%
\pgfsys@transformshift{3.409022in}{0.911519in}%
\pgfsys@useobject{currentmarker}{}%
\end{pgfscope}%
\begin{pgfscope}%
\pgfsys@transformshift{3.409675in}{0.885751in}%
\pgfsys@useobject{currentmarker}{}%
\end{pgfscope}%
\begin{pgfscope}%
\pgfsys@transformshift{3.410327in}{0.912754in}%
\pgfsys@useobject{currentmarker}{}%
\end{pgfscope}%
\begin{pgfscope}%
\pgfsys@transformshift{3.410977in}{0.917936in}%
\pgfsys@useobject{currentmarker}{}%
\end{pgfscope}%
\begin{pgfscope}%
\pgfsys@transformshift{3.411627in}{0.909613in}%
\pgfsys@useobject{currentmarker}{}%
\end{pgfscope}%
\begin{pgfscope}%
\pgfsys@transformshift{3.412276in}{0.897568in}%
\pgfsys@useobject{currentmarker}{}%
\end{pgfscope}%
\begin{pgfscope}%
\pgfsys@transformshift{3.412923in}{0.903373in}%
\pgfsys@useobject{currentmarker}{}%
\end{pgfscope}%
\begin{pgfscope}%
\pgfsys@transformshift{3.413570in}{0.894126in}%
\pgfsys@useobject{currentmarker}{}%
\end{pgfscope}%
\begin{pgfscope}%
\pgfsys@transformshift{3.414215in}{0.908774in}%
\pgfsys@useobject{currentmarker}{}%
\end{pgfscope}%
\begin{pgfscope}%
\pgfsys@transformshift{3.414860in}{0.910331in}%
\pgfsys@useobject{currentmarker}{}%
\end{pgfscope}%
\begin{pgfscope}%
\pgfsys@transformshift{3.415503in}{0.916731in}%
\pgfsys@useobject{currentmarker}{}%
\end{pgfscope}%
\begin{pgfscope}%
\pgfsys@transformshift{3.416146in}{0.904320in}%
\pgfsys@useobject{currentmarker}{}%
\end{pgfscope}%
\begin{pgfscope}%
\pgfsys@transformshift{3.416787in}{0.901012in}%
\pgfsys@useobject{currentmarker}{}%
\end{pgfscope}%
\begin{pgfscope}%
\pgfsys@transformshift{3.417427in}{0.905031in}%
\pgfsys@useobject{currentmarker}{}%
\end{pgfscope}%
\begin{pgfscope}%
\pgfsys@transformshift{3.418067in}{0.923346in}%
\pgfsys@useobject{currentmarker}{}%
\end{pgfscope}%
\begin{pgfscope}%
\pgfsys@transformshift{3.418705in}{0.906556in}%
\pgfsys@useobject{currentmarker}{}%
\end{pgfscope}%
\begin{pgfscope}%
\pgfsys@transformshift{3.419342in}{0.899916in}%
\pgfsys@useobject{currentmarker}{}%
\end{pgfscope}%
\begin{pgfscope}%
\pgfsys@transformshift{3.419979in}{0.889678in}%
\pgfsys@useobject{currentmarker}{}%
\end{pgfscope}%
\begin{pgfscope}%
\pgfsys@transformshift{3.420614in}{0.886600in}%
\pgfsys@useobject{currentmarker}{}%
\end{pgfscope}%
\begin{pgfscope}%
\pgfsys@transformshift{3.421248in}{0.896644in}%
\pgfsys@useobject{currentmarker}{}%
\end{pgfscope}%
\begin{pgfscope}%
\pgfsys@transformshift{3.421882in}{0.898372in}%
\pgfsys@useobject{currentmarker}{}%
\end{pgfscope}%
\begin{pgfscope}%
\pgfsys@transformshift{3.422514in}{0.911271in}%
\pgfsys@useobject{currentmarker}{}%
\end{pgfscope}%
\begin{pgfscope}%
\pgfsys@transformshift{3.423146in}{0.895379in}%
\pgfsys@useobject{currentmarker}{}%
\end{pgfscope}%
\begin{pgfscope}%
\pgfsys@transformshift{3.423776in}{0.905894in}%
\pgfsys@useobject{currentmarker}{}%
\end{pgfscope}%
\begin{pgfscope}%
\pgfsys@transformshift{3.424405in}{0.911603in}%
\pgfsys@useobject{currentmarker}{}%
\end{pgfscope}%
\begin{pgfscope}%
\pgfsys@transformshift{3.425034in}{0.890495in}%
\pgfsys@useobject{currentmarker}{}%
\end{pgfscope}%
\begin{pgfscope}%
\pgfsys@transformshift{3.425661in}{0.900495in}%
\pgfsys@useobject{currentmarker}{}%
\end{pgfscope}%
\begin{pgfscope}%
\pgfsys@transformshift{3.426288in}{0.893926in}%
\pgfsys@useobject{currentmarker}{}%
\end{pgfscope}%
\begin{pgfscope}%
\pgfsys@transformshift{3.426913in}{0.888979in}%
\pgfsys@useobject{currentmarker}{}%
\end{pgfscope}%
\begin{pgfscope}%
\pgfsys@transformshift{3.427538in}{0.895430in}%
\pgfsys@useobject{currentmarker}{}%
\end{pgfscope}%
\begin{pgfscope}%
\pgfsys@transformshift{3.428161in}{0.896434in}%
\pgfsys@useobject{currentmarker}{}%
\end{pgfscope}%
\begin{pgfscope}%
\pgfsys@transformshift{3.428784in}{0.913221in}%
\pgfsys@useobject{currentmarker}{}%
\end{pgfscope}%
\begin{pgfscope}%
\pgfsys@transformshift{3.429406in}{0.903918in}%
\pgfsys@useobject{currentmarker}{}%
\end{pgfscope}%
\begin{pgfscope}%
\pgfsys@transformshift{3.430026in}{0.870362in}%
\pgfsys@useobject{currentmarker}{}%
\end{pgfscope}%
\begin{pgfscope}%
\pgfsys@transformshift{3.430646in}{0.889648in}%
\pgfsys@useobject{currentmarker}{}%
\end{pgfscope}%
\begin{pgfscope}%
\pgfsys@transformshift{3.431265in}{0.898366in}%
\pgfsys@useobject{currentmarker}{}%
\end{pgfscope}%
\begin{pgfscope}%
\pgfsys@transformshift{3.431883in}{0.915553in}%
\pgfsys@useobject{currentmarker}{}%
\end{pgfscope}%
\begin{pgfscope}%
\pgfsys@transformshift{3.432500in}{0.925471in}%
\pgfsys@useobject{currentmarker}{}%
\end{pgfscope}%
\begin{pgfscope}%
\pgfsys@transformshift{3.433115in}{0.927180in}%
\pgfsys@useobject{currentmarker}{}%
\end{pgfscope}%
\begin{pgfscope}%
\pgfsys@transformshift{3.433730in}{0.909628in}%
\pgfsys@useobject{currentmarker}{}%
\end{pgfscope}%
\begin{pgfscope}%
\pgfsys@transformshift{3.434345in}{0.897044in}%
\pgfsys@useobject{currentmarker}{}%
\end{pgfscope}%
\begin{pgfscope}%
\pgfsys@transformshift{3.434958in}{0.906200in}%
\pgfsys@useobject{currentmarker}{}%
\end{pgfscope}%
\begin{pgfscope}%
\pgfsys@transformshift{3.435570in}{0.900374in}%
\pgfsys@useobject{currentmarker}{}%
\end{pgfscope}%
\begin{pgfscope}%
\pgfsys@transformshift{3.436181in}{0.901127in}%
\pgfsys@useobject{currentmarker}{}%
\end{pgfscope}%
\begin{pgfscope}%
\pgfsys@transformshift{3.436792in}{0.896267in}%
\pgfsys@useobject{currentmarker}{}%
\end{pgfscope}%
\begin{pgfscope}%
\pgfsys@transformshift{3.437401in}{0.896956in}%
\pgfsys@useobject{currentmarker}{}%
\end{pgfscope}%
\begin{pgfscope}%
\pgfsys@transformshift{3.438010in}{0.887006in}%
\pgfsys@useobject{currentmarker}{}%
\end{pgfscope}%
\begin{pgfscope}%
\pgfsys@transformshift{3.438617in}{0.892842in}%
\pgfsys@useobject{currentmarker}{}%
\end{pgfscope}%
\begin{pgfscope}%
\pgfsys@transformshift{3.439224in}{0.892847in}%
\pgfsys@useobject{currentmarker}{}%
\end{pgfscope}%
\begin{pgfscope}%
\pgfsys@transformshift{3.439830in}{0.887776in}%
\pgfsys@useobject{currentmarker}{}%
\end{pgfscope}%
\begin{pgfscope}%
\pgfsys@transformshift{3.440435in}{0.889963in}%
\pgfsys@useobject{currentmarker}{}%
\end{pgfscope}%
\begin{pgfscope}%
\pgfsys@transformshift{3.441039in}{0.905006in}%
\pgfsys@useobject{currentmarker}{}%
\end{pgfscope}%
\begin{pgfscope}%
\pgfsys@transformshift{3.441642in}{0.892147in}%
\pgfsys@useobject{currentmarker}{}%
\end{pgfscope}%
\begin{pgfscope}%
\pgfsys@transformshift{3.442244in}{0.902496in}%
\pgfsys@useobject{currentmarker}{}%
\end{pgfscope}%
\begin{pgfscope}%
\pgfsys@transformshift{3.442845in}{0.908285in}%
\pgfsys@useobject{currentmarker}{}%
\end{pgfscope}%
\begin{pgfscope}%
\pgfsys@transformshift{3.443446in}{0.883795in}%
\pgfsys@useobject{currentmarker}{}%
\end{pgfscope}%
\begin{pgfscope}%
\pgfsys@transformshift{3.444045in}{0.904352in}%
\pgfsys@useobject{currentmarker}{}%
\end{pgfscope}%
\begin{pgfscope}%
\pgfsys@transformshift{3.444644in}{0.896557in}%
\pgfsys@useobject{currentmarker}{}%
\end{pgfscope}%
\begin{pgfscope}%
\pgfsys@transformshift{3.445241in}{0.894709in}%
\pgfsys@useobject{currentmarker}{}%
\end{pgfscope}%
\begin{pgfscope}%
\pgfsys@transformshift{3.445838in}{0.900394in}%
\pgfsys@useobject{currentmarker}{}%
\end{pgfscope}%
\begin{pgfscope}%
\pgfsys@transformshift{3.446434in}{0.899304in}%
\pgfsys@useobject{currentmarker}{}%
\end{pgfscope}%
\begin{pgfscope}%
\pgfsys@transformshift{3.447029in}{0.893973in}%
\pgfsys@useobject{currentmarker}{}%
\end{pgfscope}%
\begin{pgfscope}%
\pgfsys@transformshift{3.447623in}{0.897870in}%
\pgfsys@useobject{currentmarker}{}%
\end{pgfscope}%
\begin{pgfscope}%
\pgfsys@transformshift{3.448217in}{0.897919in}%
\pgfsys@useobject{currentmarker}{}%
\end{pgfscope}%
\begin{pgfscope}%
\pgfsys@transformshift{3.448809in}{0.892287in}%
\pgfsys@useobject{currentmarker}{}%
\end{pgfscope}%
\begin{pgfscope}%
\pgfsys@transformshift{3.449401in}{0.870414in}%
\pgfsys@useobject{currentmarker}{}%
\end{pgfscope}%
\begin{pgfscope}%
\pgfsys@transformshift{3.449992in}{0.873629in}%
\pgfsys@useobject{currentmarker}{}%
\end{pgfscope}%
\begin{pgfscope}%
\pgfsys@transformshift{3.450581in}{0.884442in}%
\pgfsys@useobject{currentmarker}{}%
\end{pgfscope}%
\begin{pgfscope}%
\pgfsys@transformshift{3.451170in}{0.876883in}%
\pgfsys@useobject{currentmarker}{}%
\end{pgfscope}%
\begin{pgfscope}%
\pgfsys@transformshift{3.451759in}{0.883546in}%
\pgfsys@useobject{currentmarker}{}%
\end{pgfscope}%
\begin{pgfscope}%
\pgfsys@transformshift{3.452346in}{0.895956in}%
\pgfsys@useobject{currentmarker}{}%
\end{pgfscope}%
\begin{pgfscope}%
\pgfsys@transformshift{3.452932in}{0.902661in}%
\pgfsys@useobject{currentmarker}{}%
\end{pgfscope}%
\begin{pgfscope}%
\pgfsys@transformshift{3.453518in}{0.901945in}%
\pgfsys@useobject{currentmarker}{}%
\end{pgfscope}%
\begin{pgfscope}%
\pgfsys@transformshift{3.454103in}{0.890613in}%
\pgfsys@useobject{currentmarker}{}%
\end{pgfscope}%
\begin{pgfscope}%
\pgfsys@transformshift{3.454687in}{0.874191in}%
\pgfsys@useobject{currentmarker}{}%
\end{pgfscope}%
\begin{pgfscope}%
\pgfsys@transformshift{3.455270in}{0.897142in}%
\pgfsys@useobject{currentmarker}{}%
\end{pgfscope}%
\begin{pgfscope}%
\pgfsys@transformshift{3.455852in}{0.888793in}%
\pgfsys@useobject{currentmarker}{}%
\end{pgfscope}%
\begin{pgfscope}%
\pgfsys@transformshift{3.456433in}{0.899691in}%
\pgfsys@useobject{currentmarker}{}%
\end{pgfscope}%
\begin{pgfscope}%
\pgfsys@transformshift{3.457014in}{0.914959in}%
\pgfsys@useobject{currentmarker}{}%
\end{pgfscope}%
\begin{pgfscope}%
\pgfsys@transformshift{3.457593in}{0.900395in}%
\pgfsys@useobject{currentmarker}{}%
\end{pgfscope}%
\begin{pgfscope}%
\pgfsys@transformshift{3.458172in}{0.888125in}%
\pgfsys@useobject{currentmarker}{}%
\end{pgfscope}%
\begin{pgfscope}%
\pgfsys@transformshift{3.458750in}{0.893705in}%
\pgfsys@useobject{currentmarker}{}%
\end{pgfscope}%
\begin{pgfscope}%
\pgfsys@transformshift{3.459328in}{0.895870in}%
\pgfsys@useobject{currentmarker}{}%
\end{pgfscope}%
\begin{pgfscope}%
\pgfsys@transformshift{3.459904in}{0.889337in}%
\pgfsys@useobject{currentmarker}{}%
\end{pgfscope}%
\begin{pgfscope}%
\pgfsys@transformshift{3.460479in}{0.902617in}%
\pgfsys@useobject{currentmarker}{}%
\end{pgfscope}%
\begin{pgfscope}%
\pgfsys@transformshift{3.461054in}{0.878704in}%
\pgfsys@useobject{currentmarker}{}%
\end{pgfscope}%
\begin{pgfscope}%
\pgfsys@transformshift{3.461628in}{0.893376in}%
\pgfsys@useobject{currentmarker}{}%
\end{pgfscope}%
\begin{pgfscope}%
\pgfsys@transformshift{3.462201in}{0.892168in}%
\pgfsys@useobject{currentmarker}{}%
\end{pgfscope}%
\begin{pgfscope}%
\pgfsys@transformshift{3.462774in}{0.888092in}%
\pgfsys@useobject{currentmarker}{}%
\end{pgfscope}%
\begin{pgfscope}%
\pgfsys@transformshift{3.463345in}{0.875326in}%
\pgfsys@useobject{currentmarker}{}%
\end{pgfscope}%
\begin{pgfscope}%
\pgfsys@transformshift{3.463916in}{0.877383in}%
\pgfsys@useobject{currentmarker}{}%
\end{pgfscope}%
\begin{pgfscope}%
\pgfsys@transformshift{3.464486in}{0.880163in}%
\pgfsys@useobject{currentmarker}{}%
\end{pgfscope}%
\begin{pgfscope}%
\pgfsys@transformshift{3.465055in}{0.884712in}%
\pgfsys@useobject{currentmarker}{}%
\end{pgfscope}%
\begin{pgfscope}%
\pgfsys@transformshift{3.465623in}{0.877915in}%
\pgfsys@useobject{currentmarker}{}%
\end{pgfscope}%
\begin{pgfscope}%
\pgfsys@transformshift{3.466190in}{0.877257in}%
\pgfsys@useobject{currentmarker}{}%
\end{pgfscope}%
\begin{pgfscope}%
\pgfsys@transformshift{3.466757in}{0.887183in}%
\pgfsys@useobject{currentmarker}{}%
\end{pgfscope}%
\begin{pgfscope}%
\pgfsys@transformshift{3.467323in}{0.883721in}%
\pgfsys@useobject{currentmarker}{}%
\end{pgfscope}%
\begin{pgfscope}%
\pgfsys@transformshift{3.467888in}{0.906094in}%
\pgfsys@useobject{currentmarker}{}%
\end{pgfscope}%
\begin{pgfscope}%
\pgfsys@transformshift{3.468452in}{0.900352in}%
\pgfsys@useobject{currentmarker}{}%
\end{pgfscope}%
\begin{pgfscope}%
\pgfsys@transformshift{3.469016in}{0.895041in}%
\pgfsys@useobject{currentmarker}{}%
\end{pgfscope}%
\begin{pgfscope}%
\pgfsys@transformshift{3.469579in}{0.906399in}%
\pgfsys@useobject{currentmarker}{}%
\end{pgfscope}%
\begin{pgfscope}%
\pgfsys@transformshift{3.470141in}{0.902587in}%
\pgfsys@useobject{currentmarker}{}%
\end{pgfscope}%
\begin{pgfscope}%
\pgfsys@transformshift{3.470702in}{0.922963in}%
\pgfsys@useobject{currentmarker}{}%
\end{pgfscope}%
\begin{pgfscope}%
\pgfsys@transformshift{3.471262in}{0.911693in}%
\pgfsys@useobject{currentmarker}{}%
\end{pgfscope}%
\begin{pgfscope}%
\pgfsys@transformshift{3.471822in}{0.888990in}%
\pgfsys@useobject{currentmarker}{}%
\end{pgfscope}%
\begin{pgfscope}%
\pgfsys@transformshift{3.472381in}{0.895693in}%
\pgfsys@useobject{currentmarker}{}%
\end{pgfscope}%
\begin{pgfscope}%
\pgfsys@transformshift{3.472939in}{0.887134in}%
\pgfsys@useobject{currentmarker}{}%
\end{pgfscope}%
\begin{pgfscope}%
\pgfsys@transformshift{3.473496in}{0.890131in}%
\pgfsys@useobject{currentmarker}{}%
\end{pgfscope}%
\begin{pgfscope}%
\pgfsys@transformshift{3.474053in}{0.883592in}%
\pgfsys@useobject{currentmarker}{}%
\end{pgfscope}%
\begin{pgfscope}%
\pgfsys@transformshift{3.474608in}{0.870188in}%
\pgfsys@useobject{currentmarker}{}%
\end{pgfscope}%
\begin{pgfscope}%
\pgfsys@transformshift{3.475163in}{0.875995in}%
\pgfsys@useobject{currentmarker}{}%
\end{pgfscope}%
\begin{pgfscope}%
\pgfsys@transformshift{3.475718in}{0.885120in}%
\pgfsys@useobject{currentmarker}{}%
\end{pgfscope}%
\begin{pgfscope}%
\pgfsys@transformshift{3.476271in}{0.892516in}%
\pgfsys@useobject{currentmarker}{}%
\end{pgfscope}%
\begin{pgfscope}%
\pgfsys@transformshift{3.476824in}{0.866394in}%
\pgfsys@useobject{currentmarker}{}%
\end{pgfscope}%
\begin{pgfscope}%
\pgfsys@transformshift{3.477376in}{0.836281in}%
\pgfsys@useobject{currentmarker}{}%
\end{pgfscope}%
\begin{pgfscope}%
\pgfsys@transformshift{3.477927in}{0.849968in}%
\pgfsys@useobject{currentmarker}{}%
\end{pgfscope}%
\begin{pgfscope}%
\pgfsys@transformshift{3.478477in}{0.880678in}%
\pgfsys@useobject{currentmarker}{}%
\end{pgfscope}%
\begin{pgfscope}%
\pgfsys@transformshift{3.479027in}{0.895457in}%
\pgfsys@useobject{currentmarker}{}%
\end{pgfscope}%
\begin{pgfscope}%
\pgfsys@transformshift{3.479576in}{0.901887in}%
\pgfsys@useobject{currentmarker}{}%
\end{pgfscope}%
\begin{pgfscope}%
\pgfsys@transformshift{3.480124in}{0.889432in}%
\pgfsys@useobject{currentmarker}{}%
\end{pgfscope}%
\begin{pgfscope}%
\pgfsys@transformshift{3.480672in}{0.880197in}%
\pgfsys@useobject{currentmarker}{}%
\end{pgfscope}%
\begin{pgfscope}%
\pgfsys@transformshift{3.481219in}{0.883781in}%
\pgfsys@useobject{currentmarker}{}%
\end{pgfscope}%
\begin{pgfscope}%
\pgfsys@transformshift{3.481764in}{0.890749in}%
\pgfsys@useobject{currentmarker}{}%
\end{pgfscope}%
\begin{pgfscope}%
\pgfsys@transformshift{3.482310in}{0.894817in}%
\pgfsys@useobject{currentmarker}{}%
\end{pgfscope}%
\begin{pgfscope}%
\pgfsys@transformshift{3.482854in}{0.871507in}%
\pgfsys@useobject{currentmarker}{}%
\end{pgfscope}%
\begin{pgfscope}%
\pgfsys@transformshift{3.483398in}{0.863437in}%
\pgfsys@useobject{currentmarker}{}%
\end{pgfscope}%
\begin{pgfscope}%
\pgfsys@transformshift{3.483941in}{0.876965in}%
\pgfsys@useobject{currentmarker}{}%
\end{pgfscope}%
\begin{pgfscope}%
\pgfsys@transformshift{3.484483in}{0.871045in}%
\pgfsys@useobject{currentmarker}{}%
\end{pgfscope}%
\begin{pgfscope}%
\pgfsys@transformshift{3.485025in}{0.873522in}%
\pgfsys@useobject{currentmarker}{}%
\end{pgfscope}%
\begin{pgfscope}%
\pgfsys@transformshift{3.485566in}{0.869696in}%
\pgfsys@useobject{currentmarker}{}%
\end{pgfscope}%
\begin{pgfscope}%
\pgfsys@transformshift{3.486106in}{0.873186in}%
\pgfsys@useobject{currentmarker}{}%
\end{pgfscope}%
\begin{pgfscope}%
\pgfsys@transformshift{3.486645in}{0.881885in}%
\pgfsys@useobject{currentmarker}{}%
\end{pgfscope}%
\begin{pgfscope}%
\pgfsys@transformshift{3.487184in}{0.874888in}%
\pgfsys@useobject{currentmarker}{}%
\end{pgfscope}%
\begin{pgfscope}%
\pgfsys@transformshift{3.487722in}{0.873700in}%
\pgfsys@useobject{currentmarker}{}%
\end{pgfscope}%
\begin{pgfscope}%
\pgfsys@transformshift{3.488259in}{0.879057in}%
\pgfsys@useobject{currentmarker}{}%
\end{pgfscope}%
\begin{pgfscope}%
\pgfsys@transformshift{3.488796in}{0.872951in}%
\pgfsys@useobject{currentmarker}{}%
\end{pgfscope}%
\begin{pgfscope}%
\pgfsys@transformshift{3.489332in}{0.867195in}%
\pgfsys@useobject{currentmarker}{}%
\end{pgfscope}%
\begin{pgfscope}%
\pgfsys@transformshift{3.489867in}{0.881322in}%
\pgfsys@useobject{currentmarker}{}%
\end{pgfscope}%
\begin{pgfscope}%
\pgfsys@transformshift{3.490401in}{0.881490in}%
\pgfsys@useobject{currentmarker}{}%
\end{pgfscope}%
\begin{pgfscope}%
\pgfsys@transformshift{3.490935in}{0.887068in}%
\pgfsys@useobject{currentmarker}{}%
\end{pgfscope}%
\begin{pgfscope}%
\pgfsys@transformshift{3.491468in}{0.890958in}%
\pgfsys@useobject{currentmarker}{}%
\end{pgfscope}%
\begin{pgfscope}%
\pgfsys@transformshift{3.492001in}{0.877578in}%
\pgfsys@useobject{currentmarker}{}%
\end{pgfscope}%
\begin{pgfscope}%
\pgfsys@transformshift{3.492532in}{0.886162in}%
\pgfsys@useobject{currentmarker}{}%
\end{pgfscope}%
\begin{pgfscope}%
\pgfsys@transformshift{3.493063in}{0.878437in}%
\pgfsys@useobject{currentmarker}{}%
\end{pgfscope}%
\begin{pgfscope}%
\pgfsys@transformshift{3.493593in}{0.864213in}%
\pgfsys@useobject{currentmarker}{}%
\end{pgfscope}%
\begin{pgfscope}%
\pgfsys@transformshift{3.494123in}{0.864108in}%
\pgfsys@useobject{currentmarker}{}%
\end{pgfscope}%
\begin{pgfscope}%
\pgfsys@transformshift{3.494652in}{0.868528in}%
\pgfsys@useobject{currentmarker}{}%
\end{pgfscope}%
\begin{pgfscope}%
\pgfsys@transformshift{3.495180in}{0.871055in}%
\pgfsys@useobject{currentmarker}{}%
\end{pgfscope}%
\begin{pgfscope}%
\pgfsys@transformshift{3.495707in}{0.879910in}%
\pgfsys@useobject{currentmarker}{}%
\end{pgfscope}%
\begin{pgfscope}%
\pgfsys@transformshift{3.496234in}{0.885867in}%
\pgfsys@useobject{currentmarker}{}%
\end{pgfscope}%
\begin{pgfscope}%
\pgfsys@transformshift{3.496760in}{0.891321in}%
\pgfsys@useobject{currentmarker}{}%
\end{pgfscope}%
\begin{pgfscope}%
\pgfsys@transformshift{3.497285in}{0.878203in}%
\pgfsys@useobject{currentmarker}{}%
\end{pgfscope}%
\begin{pgfscope}%
\pgfsys@transformshift{3.497810in}{0.854166in}%
\pgfsys@useobject{currentmarker}{}%
\end{pgfscope}%
\begin{pgfscope}%
\pgfsys@transformshift{3.498334in}{0.864715in}%
\pgfsys@useobject{currentmarker}{}%
\end{pgfscope}%
\begin{pgfscope}%
\pgfsys@transformshift{3.498857in}{0.880155in}%
\pgfsys@useobject{currentmarker}{}%
\end{pgfscope}%
\begin{pgfscope}%
\pgfsys@transformshift{3.499380in}{0.897646in}%
\pgfsys@useobject{currentmarker}{}%
\end{pgfscope}%
\begin{pgfscope}%
\pgfsys@transformshift{3.499902in}{0.883638in}%
\pgfsys@useobject{currentmarker}{}%
\end{pgfscope}%
\begin{pgfscope}%
\pgfsys@transformshift{3.500423in}{0.863463in}%
\pgfsys@useobject{currentmarker}{}%
\end{pgfscope}%
\begin{pgfscope}%
\pgfsys@transformshift{3.500944in}{0.872634in}%
\pgfsys@useobject{currentmarker}{}%
\end{pgfscope}%
\begin{pgfscope}%
\pgfsys@transformshift{3.501464in}{0.879173in}%
\pgfsys@useobject{currentmarker}{}%
\end{pgfscope}%
\begin{pgfscope}%
\pgfsys@transformshift{3.501983in}{0.864726in}%
\pgfsys@useobject{currentmarker}{}%
\end{pgfscope}%
\begin{pgfscope}%
\pgfsys@transformshift{3.502502in}{0.873718in}%
\pgfsys@useobject{currentmarker}{}%
\end{pgfscope}%
\begin{pgfscope}%
\pgfsys@transformshift{3.503020in}{0.876193in}%
\pgfsys@useobject{currentmarker}{}%
\end{pgfscope}%
\begin{pgfscope}%
\pgfsys@transformshift{3.503537in}{0.872241in}%
\pgfsys@useobject{currentmarker}{}%
\end{pgfscope}%
\begin{pgfscope}%
\pgfsys@transformshift{3.504054in}{0.860690in}%
\pgfsys@useobject{currentmarker}{}%
\end{pgfscope}%
\begin{pgfscope}%
\pgfsys@transformshift{3.504570in}{0.878778in}%
\pgfsys@useobject{currentmarker}{}%
\end{pgfscope}%
\begin{pgfscope}%
\pgfsys@transformshift{3.505085in}{0.869931in}%
\pgfsys@useobject{currentmarker}{}%
\end{pgfscope}%
\begin{pgfscope}%
\pgfsys@transformshift{3.505600in}{0.881290in}%
\pgfsys@useobject{currentmarker}{}%
\end{pgfscope}%
\begin{pgfscope}%
\pgfsys@transformshift{3.506114in}{0.891183in}%
\pgfsys@useobject{currentmarker}{}%
\end{pgfscope}%
\begin{pgfscope}%
\pgfsys@transformshift{3.506627in}{0.874393in}%
\pgfsys@useobject{currentmarker}{}%
\end{pgfscope}%
\begin{pgfscope}%
\pgfsys@transformshift{3.507140in}{0.884931in}%
\pgfsys@useobject{currentmarker}{}%
\end{pgfscope}%
\begin{pgfscope}%
\pgfsys@transformshift{3.507652in}{0.900213in}%
\pgfsys@useobject{currentmarker}{}%
\end{pgfscope}%
\begin{pgfscope}%
\pgfsys@transformshift{3.508164in}{0.876280in}%
\pgfsys@useobject{currentmarker}{}%
\end{pgfscope}%
\begin{pgfscope}%
\pgfsys@transformshift{3.508674in}{0.875103in}%
\pgfsys@useobject{currentmarker}{}%
\end{pgfscope}%
\begin{pgfscope}%
\pgfsys@transformshift{3.509184in}{0.857123in}%
\pgfsys@useobject{currentmarker}{}%
\end{pgfscope}%
\begin{pgfscope}%
\pgfsys@transformshift{3.509694in}{0.859982in}%
\pgfsys@useobject{currentmarker}{}%
\end{pgfscope}%
\begin{pgfscope}%
\pgfsys@transformshift{3.510203in}{0.863935in}%
\pgfsys@useobject{currentmarker}{}%
\end{pgfscope}%
\begin{pgfscope}%
\pgfsys@transformshift{3.510711in}{0.857639in}%
\pgfsys@useobject{currentmarker}{}%
\end{pgfscope}%
\begin{pgfscope}%
\pgfsys@transformshift{3.511219in}{0.857242in}%
\pgfsys@useobject{currentmarker}{}%
\end{pgfscope}%
\begin{pgfscope}%
\pgfsys@transformshift{3.511725in}{0.863391in}%
\pgfsys@useobject{currentmarker}{}%
\end{pgfscope}%
\begin{pgfscope}%
\pgfsys@transformshift{3.512232in}{0.877114in}%
\pgfsys@useobject{currentmarker}{}%
\end{pgfscope}%
\begin{pgfscope}%
\pgfsys@transformshift{3.512737in}{0.885391in}%
\pgfsys@useobject{currentmarker}{}%
\end{pgfscope}%
\begin{pgfscope}%
\pgfsys@transformshift{3.513242in}{0.865801in}%
\pgfsys@useobject{currentmarker}{}%
\end{pgfscope}%
\begin{pgfscope}%
\pgfsys@transformshift{3.513747in}{0.857944in}%
\pgfsys@useobject{currentmarker}{}%
\end{pgfscope}%
\begin{pgfscope}%
\pgfsys@transformshift{3.514251in}{0.883873in}%
\pgfsys@useobject{currentmarker}{}%
\end{pgfscope}%
\begin{pgfscope}%
\pgfsys@transformshift{3.514754in}{0.879341in}%
\pgfsys@useobject{currentmarker}{}%
\end{pgfscope}%
\begin{pgfscope}%
\pgfsys@transformshift{3.515256in}{0.870136in}%
\pgfsys@useobject{currentmarker}{}%
\end{pgfscope}%
\begin{pgfscope}%
\pgfsys@transformshift{3.515758in}{0.870421in}%
\pgfsys@useobject{currentmarker}{}%
\end{pgfscope}%
\begin{pgfscope}%
\pgfsys@transformshift{3.516259in}{0.864293in}%
\pgfsys@useobject{currentmarker}{}%
\end{pgfscope}%
\begin{pgfscope}%
\pgfsys@transformshift{3.516760in}{0.858843in}%
\pgfsys@useobject{currentmarker}{}%
\end{pgfscope}%
\begin{pgfscope}%
\pgfsys@transformshift{3.517260in}{0.868290in}%
\pgfsys@useobject{currentmarker}{}%
\end{pgfscope}%
\begin{pgfscope}%
\pgfsys@transformshift{3.517759in}{0.839322in}%
\pgfsys@useobject{currentmarker}{}%
\end{pgfscope}%
\begin{pgfscope}%
\pgfsys@transformshift{3.518258in}{0.850414in}%
\pgfsys@useobject{currentmarker}{}%
\end{pgfscope}%
\begin{pgfscope}%
\pgfsys@transformshift{3.518756in}{0.864965in}%
\pgfsys@useobject{currentmarker}{}%
\end{pgfscope}%
\begin{pgfscope}%
\pgfsys@transformshift{3.519254in}{0.873508in}%
\pgfsys@useobject{currentmarker}{}%
\end{pgfscope}%
\begin{pgfscope}%
\pgfsys@transformshift{3.519751in}{0.867029in}%
\pgfsys@useobject{currentmarker}{}%
\end{pgfscope}%
\begin{pgfscope}%
\pgfsys@transformshift{3.520247in}{0.852703in}%
\pgfsys@useobject{currentmarker}{}%
\end{pgfscope}%
\begin{pgfscope}%
\pgfsys@transformshift{3.520743in}{0.853820in}%
\pgfsys@useobject{currentmarker}{}%
\end{pgfscope}%
\begin{pgfscope}%
\pgfsys@transformshift{3.521238in}{0.872017in}%
\pgfsys@useobject{currentmarker}{}%
\end{pgfscope}%
\begin{pgfscope}%
\pgfsys@transformshift{3.521732in}{0.859599in}%
\pgfsys@useobject{currentmarker}{}%
\end{pgfscope}%
\begin{pgfscope}%
\pgfsys@transformshift{3.522226in}{0.865176in}%
\pgfsys@useobject{currentmarker}{}%
\end{pgfscope}%
\begin{pgfscope}%
\pgfsys@transformshift{3.522720in}{0.883648in}%
\pgfsys@useobject{currentmarker}{}%
\end{pgfscope}%
\begin{pgfscope}%
\pgfsys@transformshift{3.523212in}{0.885032in}%
\pgfsys@useobject{currentmarker}{}%
\end{pgfscope}%
\begin{pgfscope}%
\pgfsys@transformshift{3.523704in}{0.870219in}%
\pgfsys@useobject{currentmarker}{}%
\end{pgfscope}%
\begin{pgfscope}%
\pgfsys@transformshift{3.524196in}{0.856843in}%
\pgfsys@useobject{currentmarker}{}%
\end{pgfscope}%
\begin{pgfscope}%
\pgfsys@transformshift{3.524687in}{0.849812in}%
\pgfsys@useobject{currentmarker}{}%
\end{pgfscope}%
\begin{pgfscope}%
\pgfsys@transformshift{3.525177in}{0.848811in}%
\pgfsys@useobject{currentmarker}{}%
\end{pgfscope}%
\begin{pgfscope}%
\pgfsys@transformshift{3.525667in}{0.868059in}%
\pgfsys@useobject{currentmarker}{}%
\end{pgfscope}%
\begin{pgfscope}%
\pgfsys@transformshift{3.526156in}{0.881874in}%
\pgfsys@useobject{currentmarker}{}%
\end{pgfscope}%
\begin{pgfscope}%
\pgfsys@transformshift{3.526644in}{0.883111in}%
\pgfsys@useobject{currentmarker}{}%
\end{pgfscope}%
\begin{pgfscope}%
\pgfsys@transformshift{3.527132in}{0.875607in}%
\pgfsys@useobject{currentmarker}{}%
\end{pgfscope}%
\begin{pgfscope}%
\pgfsys@transformshift{3.527620in}{0.871852in}%
\pgfsys@useobject{currentmarker}{}%
\end{pgfscope}%
\begin{pgfscope}%
\pgfsys@transformshift{3.528106in}{0.857073in}%
\pgfsys@useobject{currentmarker}{}%
\end{pgfscope}%
\begin{pgfscope}%
\pgfsys@transformshift{3.528593in}{0.855423in}%
\pgfsys@useobject{currentmarker}{}%
\end{pgfscope}%
\begin{pgfscope}%
\pgfsys@transformshift{3.529078in}{0.857679in}%
\pgfsys@useobject{currentmarker}{}%
\end{pgfscope}%
\begin{pgfscope}%
\pgfsys@transformshift{3.529563in}{0.863009in}%
\pgfsys@useobject{currentmarker}{}%
\end{pgfscope}%
\begin{pgfscope}%
\pgfsys@transformshift{3.530048in}{0.874218in}%
\pgfsys@useobject{currentmarker}{}%
\end{pgfscope}%
\begin{pgfscope}%
\pgfsys@transformshift{3.530531in}{0.857468in}%
\pgfsys@useobject{currentmarker}{}%
\end{pgfscope}%
\begin{pgfscope}%
\pgfsys@transformshift{3.531015in}{0.878147in}%
\pgfsys@useobject{currentmarker}{}%
\end{pgfscope}%
\begin{pgfscope}%
\pgfsys@transformshift{3.531497in}{0.898823in}%
\pgfsys@useobject{currentmarker}{}%
\end{pgfscope}%
\begin{pgfscope}%
\pgfsys@transformshift{3.531980in}{0.863757in}%
\pgfsys@useobject{currentmarker}{}%
\end{pgfscope}%
\begin{pgfscope}%
\pgfsys@transformshift{3.532461in}{0.849858in}%
\pgfsys@useobject{currentmarker}{}%
\end{pgfscope}%
\begin{pgfscope}%
\pgfsys@transformshift{3.532942in}{0.869869in}%
\pgfsys@useobject{currentmarker}{}%
\end{pgfscope}%
\begin{pgfscope}%
\pgfsys@transformshift{3.533422in}{0.867348in}%
\pgfsys@useobject{currentmarker}{}%
\end{pgfscope}%
\begin{pgfscope}%
\pgfsys@transformshift{3.533902in}{0.867279in}%
\pgfsys@useobject{currentmarker}{}%
\end{pgfscope}%
\begin{pgfscope}%
\pgfsys@transformshift{3.534381in}{0.856707in}%
\pgfsys@useobject{currentmarker}{}%
\end{pgfscope}%
\begin{pgfscope}%
\pgfsys@transformshift{3.534860in}{0.846133in}%
\pgfsys@useobject{currentmarker}{}%
\end{pgfscope}%
\begin{pgfscope}%
\pgfsys@transformshift{3.535338in}{0.863737in}%
\pgfsys@useobject{currentmarker}{}%
\end{pgfscope}%
\begin{pgfscope}%
\pgfsys@transformshift{3.535816in}{0.866528in}%
\pgfsys@useobject{currentmarker}{}%
\end{pgfscope}%
\begin{pgfscope}%
\pgfsys@transformshift{3.536293in}{0.852152in}%
\pgfsys@useobject{currentmarker}{}%
\end{pgfscope}%
\begin{pgfscope}%
\pgfsys@transformshift{3.536769in}{0.861618in}%
\pgfsys@useobject{currentmarker}{}%
\end{pgfscope}%
\begin{pgfscope}%
\pgfsys@transformshift{3.537245in}{0.857049in}%
\pgfsys@useobject{currentmarker}{}%
\end{pgfscope}%
\begin{pgfscope}%
\pgfsys@transformshift{3.537720in}{0.872613in}%
\pgfsys@useobject{currentmarker}{}%
\end{pgfscope}%
\begin{pgfscope}%
\pgfsys@transformshift{3.538195in}{0.866712in}%
\pgfsys@useobject{currentmarker}{}%
\end{pgfscope}%
\begin{pgfscope}%
\pgfsys@transformshift{3.538669in}{0.852309in}%
\pgfsys@useobject{currentmarker}{}%
\end{pgfscope}%
\begin{pgfscope}%
\pgfsys@transformshift{3.539143in}{0.867626in}%
\pgfsys@useobject{currentmarker}{}%
\end{pgfscope}%
\begin{pgfscope}%
\pgfsys@transformshift{3.539616in}{0.856321in}%
\pgfsys@useobject{currentmarker}{}%
\end{pgfscope}%
\begin{pgfscope}%
\pgfsys@transformshift{3.540088in}{0.861750in}%
\pgfsys@useobject{currentmarker}{}%
\end{pgfscope}%
\begin{pgfscope}%
\pgfsys@transformshift{3.540560in}{0.875257in}%
\pgfsys@useobject{currentmarker}{}%
\end{pgfscope}%
\begin{pgfscope}%
\pgfsys@transformshift{3.541032in}{0.873839in}%
\pgfsys@useobject{currentmarker}{}%
\end{pgfscope}%
\begin{pgfscope}%
\pgfsys@transformshift{3.541503in}{0.858332in}%
\pgfsys@useobject{currentmarker}{}%
\end{pgfscope}%
\begin{pgfscope}%
\pgfsys@transformshift{3.541973in}{0.851344in}%
\pgfsys@useobject{currentmarker}{}%
\end{pgfscope}%
\begin{pgfscope}%
\pgfsys@transformshift{3.542443in}{0.864528in}%
\pgfsys@useobject{currentmarker}{}%
\end{pgfscope}%
\begin{pgfscope}%
\pgfsys@transformshift{3.542912in}{0.881282in}%
\pgfsys@useobject{currentmarker}{}%
\end{pgfscope}%
\begin{pgfscope}%
\pgfsys@transformshift{3.543381in}{0.864678in}%
\pgfsys@useobject{currentmarker}{}%
\end{pgfscope}%
\begin{pgfscope}%
\pgfsys@transformshift{3.543849in}{0.855804in}%
\pgfsys@useobject{currentmarker}{}%
\end{pgfscope}%
\begin{pgfscope}%
\pgfsys@transformshift{3.544316in}{0.857317in}%
\pgfsys@useobject{currentmarker}{}%
\end{pgfscope}%
\begin{pgfscope}%
\pgfsys@transformshift{3.544783in}{0.850025in}%
\pgfsys@useobject{currentmarker}{}%
\end{pgfscope}%
\begin{pgfscope}%
\pgfsys@transformshift{3.545250in}{0.853576in}%
\pgfsys@useobject{currentmarker}{}%
\end{pgfscope}%
\begin{pgfscope}%
\pgfsys@transformshift{3.545716in}{0.858821in}%
\pgfsys@useobject{currentmarker}{}%
\end{pgfscope}%
\begin{pgfscope}%
\pgfsys@transformshift{3.546181in}{0.874850in}%
\pgfsys@useobject{currentmarker}{}%
\end{pgfscope}%
\begin{pgfscope}%
\pgfsys@transformshift{3.546646in}{0.845335in}%
\pgfsys@useobject{currentmarker}{}%
\end{pgfscope}%
\begin{pgfscope}%
\pgfsys@transformshift{3.547111in}{0.853940in}%
\pgfsys@useobject{currentmarker}{}%
\end{pgfscope}%
\begin{pgfscope}%
\pgfsys@transformshift{3.547574in}{0.861170in}%
\pgfsys@useobject{currentmarker}{}%
\end{pgfscope}%
\begin{pgfscope}%
\pgfsys@transformshift{3.548038in}{0.852901in}%
\pgfsys@useobject{currentmarker}{}%
\end{pgfscope}%
\begin{pgfscope}%
\pgfsys@transformshift{3.548500in}{0.846718in}%
\pgfsys@useobject{currentmarker}{}%
\end{pgfscope}%
\begin{pgfscope}%
\pgfsys@transformshift{3.548963in}{0.854948in}%
\pgfsys@useobject{currentmarker}{}%
\end{pgfscope}%
\begin{pgfscope}%
\pgfsys@transformshift{3.549424in}{0.852590in}%
\pgfsys@useobject{currentmarker}{}%
\end{pgfscope}%
\begin{pgfscope}%
\pgfsys@transformshift{3.549886in}{0.852773in}%
\pgfsys@useobject{currentmarker}{}%
\end{pgfscope}%
\begin{pgfscope}%
\pgfsys@transformshift{3.550346in}{0.852799in}%
\pgfsys@useobject{currentmarker}{}%
\end{pgfscope}%
\begin{pgfscope}%
\pgfsys@transformshift{3.550806in}{0.846516in}%
\pgfsys@useobject{currentmarker}{}%
\end{pgfscope}%
\begin{pgfscope}%
\pgfsys@transformshift{3.551266in}{0.841264in}%
\pgfsys@useobject{currentmarker}{}%
\end{pgfscope}%
\begin{pgfscope}%
\pgfsys@transformshift{3.551725in}{0.866942in}%
\pgfsys@useobject{currentmarker}{}%
\end{pgfscope}%
\begin{pgfscope}%
\pgfsys@transformshift{3.552184in}{0.857605in}%
\pgfsys@useobject{currentmarker}{}%
\end{pgfscope}%
\begin{pgfscope}%
\pgfsys@transformshift{3.552642in}{0.852462in}%
\pgfsys@useobject{currentmarker}{}%
\end{pgfscope}%
\begin{pgfscope}%
\pgfsys@transformshift{3.553099in}{0.864022in}%
\pgfsys@useobject{currentmarker}{}%
\end{pgfscope}%
\begin{pgfscope}%
\pgfsys@transformshift{3.553556in}{0.873462in}%
\pgfsys@useobject{currentmarker}{}%
\end{pgfscope}%
\begin{pgfscope}%
\pgfsys@transformshift{3.554013in}{0.869767in}%
\pgfsys@useobject{currentmarker}{}%
\end{pgfscope}%
\begin{pgfscope}%
\pgfsys@transformshift{3.554469in}{0.849206in}%
\pgfsys@useobject{currentmarker}{}%
\end{pgfscope}%
\begin{pgfscope}%
\pgfsys@transformshift{3.554924in}{0.857992in}%
\pgfsys@useobject{currentmarker}{}%
\end{pgfscope}%
\begin{pgfscope}%
\pgfsys@transformshift{3.555379in}{0.873621in}%
\pgfsys@useobject{currentmarker}{}%
\end{pgfscope}%
\begin{pgfscope}%
\pgfsys@transformshift{3.555833in}{0.874390in}%
\pgfsys@useobject{currentmarker}{}%
\end{pgfscope}%
\begin{pgfscope}%
\pgfsys@transformshift{3.556287in}{0.866230in}%
\pgfsys@useobject{currentmarker}{}%
\end{pgfscope}%
\begin{pgfscope}%
\pgfsys@transformshift{3.556741in}{0.852690in}%
\pgfsys@useobject{currentmarker}{}%
\end{pgfscope}%
\begin{pgfscope}%
\pgfsys@transformshift{3.557194in}{0.847866in}%
\pgfsys@useobject{currentmarker}{}%
\end{pgfscope}%
\begin{pgfscope}%
\pgfsys@transformshift{3.557646in}{0.842038in}%
\pgfsys@useobject{currentmarker}{}%
\end{pgfscope}%
\begin{pgfscope}%
\pgfsys@transformshift{3.558098in}{0.857686in}%
\pgfsys@useobject{currentmarker}{}%
\end{pgfscope}%
\begin{pgfscope}%
\pgfsys@transformshift{3.558549in}{0.854792in}%
\pgfsys@useobject{currentmarker}{}%
\end{pgfscope}%
\begin{pgfscope}%
\pgfsys@transformshift{3.559000in}{0.855058in}%
\pgfsys@useobject{currentmarker}{}%
\end{pgfscope}%
\begin{pgfscope}%
\pgfsys@transformshift{3.559451in}{0.860997in}%
\pgfsys@useobject{currentmarker}{}%
\end{pgfscope}%
\begin{pgfscope}%
\pgfsys@transformshift{3.559900in}{0.841106in}%
\pgfsys@useobject{currentmarker}{}%
\end{pgfscope}%
\begin{pgfscope}%
\pgfsys@transformshift{3.560350in}{0.852173in}%
\pgfsys@useobject{currentmarker}{}%
\end{pgfscope}%
\begin{pgfscope}%
\pgfsys@transformshift{3.560799in}{0.868965in}%
\pgfsys@useobject{currentmarker}{}%
\end{pgfscope}%
\begin{pgfscope}%
\pgfsys@transformshift{3.561247in}{0.873678in}%
\pgfsys@useobject{currentmarker}{}%
\end{pgfscope}%
\begin{pgfscope}%
\pgfsys@transformshift{3.561695in}{0.862637in}%
\pgfsys@useobject{currentmarker}{}%
\end{pgfscope}%
\begin{pgfscope}%
\pgfsys@transformshift{3.562142in}{0.837815in}%
\pgfsys@useobject{currentmarker}{}%
\end{pgfscope}%
\begin{pgfscope}%
\pgfsys@transformshift{3.562589in}{0.833283in}%
\pgfsys@useobject{currentmarker}{}%
\end{pgfscope}%
\begin{pgfscope}%
\pgfsys@transformshift{3.563036in}{0.853876in}%
\pgfsys@useobject{currentmarker}{}%
\end{pgfscope}%
\begin{pgfscope}%
\pgfsys@transformshift{3.563482in}{0.855566in}%
\pgfsys@useobject{currentmarker}{}%
\end{pgfscope}%
\begin{pgfscope}%
\pgfsys@transformshift{3.563927in}{0.853458in}%
\pgfsys@useobject{currentmarker}{}%
\end{pgfscope}%
\begin{pgfscope}%
\pgfsys@transformshift{3.564372in}{0.848449in}%
\pgfsys@useobject{currentmarker}{}%
\end{pgfscope}%
\begin{pgfscope}%
\pgfsys@transformshift{3.564816in}{0.851141in}%
\pgfsys@useobject{currentmarker}{}%
\end{pgfscope}%
\begin{pgfscope}%
\pgfsys@transformshift{3.565260in}{0.862836in}%
\pgfsys@useobject{currentmarker}{}%
\end{pgfscope}%
\begin{pgfscope}%
\pgfsys@transformshift{3.565704in}{0.842810in}%
\pgfsys@useobject{currentmarker}{}%
\end{pgfscope}%
\begin{pgfscope}%
\pgfsys@transformshift{3.566147in}{0.844824in}%
\pgfsys@useobject{currentmarker}{}%
\end{pgfscope}%
\begin{pgfscope}%
\pgfsys@transformshift{3.566589in}{0.846826in}%
\pgfsys@useobject{currentmarker}{}%
\end{pgfscope}%
\begin{pgfscope}%
\pgfsys@transformshift{3.567031in}{0.839024in}%
\pgfsys@useobject{currentmarker}{}%
\end{pgfscope}%
\begin{pgfscope}%
\pgfsys@transformshift{3.567473in}{0.833998in}%
\pgfsys@useobject{currentmarker}{}%
\end{pgfscope}%
\begin{pgfscope}%
\pgfsys@transformshift{3.567914in}{0.842560in}%
\pgfsys@useobject{currentmarker}{}%
\end{pgfscope}%
\begin{pgfscope}%
\pgfsys@transformshift{3.568354in}{0.854805in}%
\pgfsys@useobject{currentmarker}{}%
\end{pgfscope}%
\begin{pgfscope}%
\pgfsys@transformshift{3.568794in}{0.845421in}%
\pgfsys@useobject{currentmarker}{}%
\end{pgfscope}%
\begin{pgfscope}%
\pgfsys@transformshift{3.569234in}{0.829016in}%
\pgfsys@useobject{currentmarker}{}%
\end{pgfscope}%
\begin{pgfscope}%
\pgfsys@transformshift{3.569673in}{0.834529in}%
\pgfsys@useobject{currentmarker}{}%
\end{pgfscope}%
\begin{pgfscope}%
\pgfsys@transformshift{3.570112in}{0.849444in}%
\pgfsys@useobject{currentmarker}{}%
\end{pgfscope}%
\begin{pgfscope}%
\pgfsys@transformshift{3.570550in}{0.847467in}%
\pgfsys@useobject{currentmarker}{}%
\end{pgfscope}%
\begin{pgfscope}%
\pgfsys@transformshift{3.570987in}{0.858113in}%
\pgfsys@useobject{currentmarker}{}%
\end{pgfscope}%
\begin{pgfscope}%
\pgfsys@transformshift{3.571425in}{0.861436in}%
\pgfsys@useobject{currentmarker}{}%
\end{pgfscope}%
\begin{pgfscope}%
\pgfsys@transformshift{3.571861in}{0.832020in}%
\pgfsys@useobject{currentmarker}{}%
\end{pgfscope}%
\begin{pgfscope}%
\pgfsys@transformshift{3.572298in}{0.830633in}%
\pgfsys@useobject{currentmarker}{}%
\end{pgfscope}%
\begin{pgfscope}%
\pgfsys@transformshift{3.572733in}{0.848792in}%
\pgfsys@useobject{currentmarker}{}%
\end{pgfscope}%
\begin{pgfscope}%
\pgfsys@transformshift{3.573169in}{0.850064in}%
\pgfsys@useobject{currentmarker}{}%
\end{pgfscope}%
\begin{pgfscope}%
\pgfsys@transformshift{3.573604in}{0.869253in}%
\pgfsys@useobject{currentmarker}{}%
\end{pgfscope}%
\begin{pgfscope}%
\pgfsys@transformshift{3.574038in}{0.859868in}%
\pgfsys@useobject{currentmarker}{}%
\end{pgfscope}%
\begin{pgfscope}%
\pgfsys@transformshift{3.574472in}{0.850794in}%
\pgfsys@useobject{currentmarker}{}%
\end{pgfscope}%
\begin{pgfscope}%
\pgfsys@transformshift{3.574905in}{0.834407in}%
\pgfsys@useobject{currentmarker}{}%
\end{pgfscope}%
\begin{pgfscope}%
\pgfsys@transformshift{3.575338in}{0.846671in}%
\pgfsys@useobject{currentmarker}{}%
\end{pgfscope}%
\begin{pgfscope}%
\pgfsys@transformshift{3.575771in}{0.853079in}%
\pgfsys@useobject{currentmarker}{}%
\end{pgfscope}%
\begin{pgfscope}%
\pgfsys@transformshift{3.576203in}{0.857694in}%
\pgfsys@useobject{currentmarker}{}%
\end{pgfscope}%
\begin{pgfscope}%
\pgfsys@transformshift{3.576635in}{0.861052in}%
\pgfsys@useobject{currentmarker}{}%
\end{pgfscope}%
\begin{pgfscope}%
\pgfsys@transformshift{3.577066in}{0.841474in}%
\pgfsys@useobject{currentmarker}{}%
\end{pgfscope}%
\begin{pgfscope}%
\pgfsys@transformshift{3.577496in}{0.851052in}%
\pgfsys@useobject{currentmarker}{}%
\end{pgfscope}%
\begin{pgfscope}%
\pgfsys@transformshift{3.577927in}{0.861388in}%
\pgfsys@useobject{currentmarker}{}%
\end{pgfscope}%
\begin{pgfscope}%
\pgfsys@transformshift{3.578356in}{0.864325in}%
\pgfsys@useobject{currentmarker}{}%
\end{pgfscope}%
\begin{pgfscope}%
\pgfsys@transformshift{3.578786in}{0.856402in}%
\pgfsys@useobject{currentmarker}{}%
\end{pgfscope}%
\begin{pgfscope}%
\pgfsys@transformshift{3.579214in}{0.839810in}%
\pgfsys@useobject{currentmarker}{}%
\end{pgfscope}%
\begin{pgfscope}%
\pgfsys@transformshift{3.579643in}{0.839020in}%
\pgfsys@useobject{currentmarker}{}%
\end{pgfscope}%
\begin{pgfscope}%
\pgfsys@transformshift{3.580071in}{0.852093in}%
\pgfsys@useobject{currentmarker}{}%
\end{pgfscope}%
\begin{pgfscope}%
\pgfsys@transformshift{3.580498in}{0.851879in}%
\pgfsys@useobject{currentmarker}{}%
\end{pgfscope}%
\begin{pgfscope}%
\pgfsys@transformshift{3.580925in}{0.867430in}%
\pgfsys@useobject{currentmarker}{}%
\end{pgfscope}%
\begin{pgfscope}%
\pgfsys@transformshift{3.581352in}{0.853405in}%
\pgfsys@useobject{currentmarker}{}%
\end{pgfscope}%
\begin{pgfscope}%
\pgfsys@transformshift{3.581778in}{0.842686in}%
\pgfsys@useobject{currentmarker}{}%
\end{pgfscope}%
\begin{pgfscope}%
\pgfsys@transformshift{3.582204in}{0.847070in}%
\pgfsys@useobject{currentmarker}{}%
\end{pgfscope}%
\begin{pgfscope}%
\pgfsys@transformshift{3.582629in}{0.836967in}%
\pgfsys@useobject{currentmarker}{}%
\end{pgfscope}%
\begin{pgfscope}%
\pgfsys@transformshift{3.583054in}{0.856661in}%
\pgfsys@useobject{currentmarker}{}%
\end{pgfscope}%
\begin{pgfscope}%
\pgfsys@transformshift{3.583478in}{0.863631in}%
\pgfsys@useobject{currentmarker}{}%
\end{pgfscope}%
\begin{pgfscope}%
\pgfsys@transformshift{3.583902in}{0.859059in}%
\pgfsys@useobject{currentmarker}{}%
\end{pgfscope}%
\begin{pgfscope}%
\pgfsys@transformshift{3.584325in}{0.857893in}%
\pgfsys@useobject{currentmarker}{}%
\end{pgfscope}%
\begin{pgfscope}%
\pgfsys@transformshift{3.584748in}{0.850568in}%
\pgfsys@useobject{currentmarker}{}%
\end{pgfscope}%
\begin{pgfscope}%
\pgfsys@transformshift{3.585171in}{0.849895in}%
\pgfsys@useobject{currentmarker}{}%
\end{pgfscope}%
\begin{pgfscope}%
\pgfsys@transformshift{3.585593in}{0.858869in}%
\pgfsys@useobject{currentmarker}{}%
\end{pgfscope}%
\begin{pgfscope}%
\pgfsys@transformshift{3.586015in}{0.852584in}%
\pgfsys@useobject{currentmarker}{}%
\end{pgfscope}%
\begin{pgfscope}%
\pgfsys@transformshift{3.586436in}{0.842100in}%
\pgfsys@useobject{currentmarker}{}%
\end{pgfscope}%
\begin{pgfscope}%
\pgfsys@transformshift{3.586857in}{0.845406in}%
\pgfsys@useobject{currentmarker}{}%
\end{pgfscope}%
\begin{pgfscope}%
\pgfsys@transformshift{3.587277in}{0.839119in}%
\pgfsys@useobject{currentmarker}{}%
\end{pgfscope}%
\begin{pgfscope}%
\pgfsys@transformshift{3.587697in}{0.834929in}%
\pgfsys@useobject{currentmarker}{}%
\end{pgfscope}%
\begin{pgfscope}%
\pgfsys@transformshift{3.588117in}{0.843013in}%
\pgfsys@useobject{currentmarker}{}%
\end{pgfscope}%
\begin{pgfscope}%
\pgfsys@transformshift{3.588536in}{0.843542in}%
\pgfsys@useobject{currentmarker}{}%
\end{pgfscope}%
\begin{pgfscope}%
\pgfsys@transformshift{3.588954in}{0.836663in}%
\pgfsys@useobject{currentmarker}{}%
\end{pgfscope}%
\begin{pgfscope}%
\pgfsys@transformshift{3.589373in}{0.831628in}%
\pgfsys@useobject{currentmarker}{}%
\end{pgfscope}%
\begin{pgfscope}%
\pgfsys@transformshift{3.589790in}{0.846656in}%
\pgfsys@useobject{currentmarker}{}%
\end{pgfscope}%
\begin{pgfscope}%
\pgfsys@transformshift{3.590208in}{0.836345in}%
\pgfsys@useobject{currentmarker}{}%
\end{pgfscope}%
\begin{pgfscope}%
\pgfsys@transformshift{3.590625in}{0.844131in}%
\pgfsys@useobject{currentmarker}{}%
\end{pgfscope}%
\begin{pgfscope}%
\pgfsys@transformshift{3.591041in}{0.858758in}%
\pgfsys@useobject{currentmarker}{}%
\end{pgfscope}%
\begin{pgfscope}%
\pgfsys@transformshift{3.591457in}{0.847983in}%
\pgfsys@useobject{currentmarker}{}%
\end{pgfscope}%
\begin{pgfscope}%
\pgfsys@transformshift{3.591873in}{0.826872in}%
\pgfsys@useobject{currentmarker}{}%
\end{pgfscope}%
\begin{pgfscope}%
\pgfsys@transformshift{3.592288in}{0.821743in}%
\pgfsys@useobject{currentmarker}{}%
\end{pgfscope}%
\begin{pgfscope}%
\pgfsys@transformshift{3.592703in}{0.845322in}%
\pgfsys@useobject{currentmarker}{}%
\end{pgfscope}%
\begin{pgfscope}%
\pgfsys@transformshift{3.593117in}{0.845614in}%
\pgfsys@useobject{currentmarker}{}%
\end{pgfscope}%
\begin{pgfscope}%
\pgfsys@transformshift{3.593531in}{0.841977in}%
\pgfsys@useobject{currentmarker}{}%
\end{pgfscope}%
\begin{pgfscope}%
\pgfsys@transformshift{3.593944in}{0.836274in}%
\pgfsys@useobject{currentmarker}{}%
\end{pgfscope}%
\begin{pgfscope}%
\pgfsys@transformshift{3.594357in}{0.846549in}%
\pgfsys@useobject{currentmarker}{}%
\end{pgfscope}%
\begin{pgfscope}%
\pgfsys@transformshift{3.594770in}{0.857440in}%
\pgfsys@useobject{currentmarker}{}%
\end{pgfscope}%
\begin{pgfscope}%
\pgfsys@transformshift{3.595182in}{0.856495in}%
\pgfsys@useobject{currentmarker}{}%
\end{pgfscope}%
\begin{pgfscope}%
\pgfsys@transformshift{3.595594in}{0.842270in}%
\pgfsys@useobject{currentmarker}{}%
\end{pgfscope}%
\begin{pgfscope}%
\pgfsys@transformshift{3.596005in}{0.846440in}%
\pgfsys@useobject{currentmarker}{}%
\end{pgfscope}%
\begin{pgfscope}%
\pgfsys@transformshift{3.596416in}{0.851595in}%
\pgfsys@useobject{currentmarker}{}%
\end{pgfscope}%
\begin{pgfscope}%
\pgfsys@transformshift{3.596827in}{0.860285in}%
\pgfsys@useobject{currentmarker}{}%
\end{pgfscope}%
\begin{pgfscope}%
\pgfsys@transformshift{3.597237in}{0.856508in}%
\pgfsys@useobject{currentmarker}{}%
\end{pgfscope}%
\begin{pgfscope}%
\pgfsys@transformshift{3.597647in}{0.847163in}%
\pgfsys@useobject{currentmarker}{}%
\end{pgfscope}%
\begin{pgfscope}%
\pgfsys@transformshift{3.598056in}{0.854040in}%
\pgfsys@useobject{currentmarker}{}%
\end{pgfscope}%
\begin{pgfscope}%
\pgfsys@transformshift{3.598465in}{0.858263in}%
\pgfsys@useobject{currentmarker}{}%
\end{pgfscope}%
\begin{pgfscope}%
\pgfsys@transformshift{3.598873in}{0.847109in}%
\pgfsys@useobject{currentmarker}{}%
\end{pgfscope}%
\begin{pgfscope}%
\pgfsys@transformshift{3.599281in}{0.847183in}%
\pgfsys@useobject{currentmarker}{}%
\end{pgfscope}%
\begin{pgfscope}%
\pgfsys@transformshift{3.599689in}{0.830086in}%
\pgfsys@useobject{currentmarker}{}%
\end{pgfscope}%
\begin{pgfscope}%
\pgfsys@transformshift{3.600096in}{0.833908in}%
\pgfsys@useobject{currentmarker}{}%
\end{pgfscope}%
\begin{pgfscope}%
\pgfsys@transformshift{3.600503in}{0.826436in}%
\pgfsys@useobject{currentmarker}{}%
\end{pgfscope}%
\begin{pgfscope}%
\pgfsys@transformshift{3.600909in}{0.824161in}%
\pgfsys@useobject{currentmarker}{}%
\end{pgfscope}%
\begin{pgfscope}%
\pgfsys@transformshift{3.601315in}{0.846690in}%
\pgfsys@useobject{currentmarker}{}%
\end{pgfscope}%
\begin{pgfscope}%
\pgfsys@transformshift{3.601721in}{0.849595in}%
\pgfsys@useobject{currentmarker}{}%
\end{pgfscope}%
\begin{pgfscope}%
\pgfsys@transformshift{3.602126in}{0.856939in}%
\pgfsys@useobject{currentmarker}{}%
\end{pgfscope}%
\begin{pgfscope}%
\pgfsys@transformshift{3.602531in}{0.855626in}%
\pgfsys@useobject{currentmarker}{}%
\end{pgfscope}%
\begin{pgfscope}%
\pgfsys@transformshift{3.602935in}{0.840881in}%
\pgfsys@useobject{currentmarker}{}%
\end{pgfscope}%
\begin{pgfscope}%
\pgfsys@transformshift{3.603339in}{0.850118in}%
\pgfsys@useobject{currentmarker}{}%
\end{pgfscope}%
\begin{pgfscope}%
\pgfsys@transformshift{3.603743in}{0.851251in}%
\pgfsys@useobject{currentmarker}{}%
\end{pgfscope}%
\begin{pgfscope}%
\pgfsys@transformshift{3.604146in}{0.835185in}%
\pgfsys@useobject{currentmarker}{}%
\end{pgfscope}%
\begin{pgfscope}%
\pgfsys@transformshift{3.604549in}{0.838117in}%
\pgfsys@useobject{currentmarker}{}%
\end{pgfscope}%
\begin{pgfscope}%
\pgfsys@transformshift{3.604951in}{0.857015in}%
\pgfsys@useobject{currentmarker}{}%
\end{pgfscope}%
\begin{pgfscope}%
\pgfsys@transformshift{3.605353in}{0.848586in}%
\pgfsys@useobject{currentmarker}{}%
\end{pgfscope}%
\begin{pgfscope}%
\pgfsys@transformshift{3.605755in}{0.839265in}%
\pgfsys@useobject{currentmarker}{}%
\end{pgfscope}%
\begin{pgfscope}%
\pgfsys@transformshift{3.606156in}{0.823943in}%
\pgfsys@useobject{currentmarker}{}%
\end{pgfscope}%
\begin{pgfscope}%
\pgfsys@transformshift{3.606556in}{0.836307in}%
\pgfsys@useobject{currentmarker}{}%
\end{pgfscope}%
\begin{pgfscope}%
\pgfsys@transformshift{3.606957in}{0.838047in}%
\pgfsys@useobject{currentmarker}{}%
\end{pgfscope}%
\begin{pgfscope}%
\pgfsys@transformshift{3.607357in}{0.833277in}%
\pgfsys@useobject{currentmarker}{}%
\end{pgfscope}%
\begin{pgfscope}%
\pgfsys@transformshift{3.607756in}{0.853135in}%
\pgfsys@useobject{currentmarker}{}%
\end{pgfscope}%
\begin{pgfscope}%
\pgfsys@transformshift{3.608156in}{0.844660in}%
\pgfsys@useobject{currentmarker}{}%
\end{pgfscope}%
\begin{pgfscope}%
\pgfsys@transformshift{3.608554in}{0.827289in}%
\pgfsys@useobject{currentmarker}{}%
\end{pgfscope}%
\begin{pgfscope}%
\pgfsys@transformshift{3.608953in}{0.838193in}%
\pgfsys@useobject{currentmarker}{}%
\end{pgfscope}%
\begin{pgfscope}%
\pgfsys@transformshift{3.609351in}{0.846710in}%
\pgfsys@useobject{currentmarker}{}%
\end{pgfscope}%
\begin{pgfscope}%
\pgfsys@transformshift{3.609748in}{0.835429in}%
\pgfsys@useobject{currentmarker}{}%
\end{pgfscope}%
\begin{pgfscope}%
\pgfsys@transformshift{3.610146in}{0.843235in}%
\pgfsys@useobject{currentmarker}{}%
\end{pgfscope}%
\begin{pgfscope}%
\pgfsys@transformshift{3.610542in}{0.836301in}%
\pgfsys@useobject{currentmarker}{}%
\end{pgfscope}%
\begin{pgfscope}%
\pgfsys@transformshift{3.610939in}{0.812805in}%
\pgfsys@useobject{currentmarker}{}%
\end{pgfscope}%
\begin{pgfscope}%
\pgfsys@transformshift{3.611335in}{0.835449in}%
\pgfsys@useobject{currentmarker}{}%
\end{pgfscope}%
\begin{pgfscope}%
\pgfsys@transformshift{3.611730in}{0.846209in}%
\pgfsys@useobject{currentmarker}{}%
\end{pgfscope}%
\begin{pgfscope}%
\pgfsys@transformshift{3.612126in}{0.834223in}%
\pgfsys@useobject{currentmarker}{}%
\end{pgfscope}%
\begin{pgfscope}%
\pgfsys@transformshift{3.612521in}{0.838330in}%
\pgfsys@useobject{currentmarker}{}%
\end{pgfscope}%
\begin{pgfscope}%
\pgfsys@transformshift{3.612915in}{0.853515in}%
\pgfsys@useobject{currentmarker}{}%
\end{pgfscope}%
\begin{pgfscope}%
\pgfsys@transformshift{3.613309in}{0.859247in}%
\pgfsys@useobject{currentmarker}{}%
\end{pgfscope}%
\begin{pgfscope}%
\pgfsys@transformshift{3.613703in}{0.854530in}%
\pgfsys@useobject{currentmarker}{}%
\end{pgfscope}%
\begin{pgfscope}%
\pgfsys@transformshift{3.614096in}{0.837253in}%
\pgfsys@useobject{currentmarker}{}%
\end{pgfscope}%
\begin{pgfscope}%
\pgfsys@transformshift{3.614489in}{0.831486in}%
\pgfsys@useobject{currentmarker}{}%
\end{pgfscope}%
\begin{pgfscope}%
\pgfsys@transformshift{3.614882in}{0.829175in}%
\pgfsys@useobject{currentmarker}{}%
\end{pgfscope}%
\begin{pgfscope}%
\pgfsys@transformshift{3.615274in}{0.834755in}%
\pgfsys@useobject{currentmarker}{}%
\end{pgfscope}%
\begin{pgfscope}%
\pgfsys@transformshift{3.615666in}{0.850496in}%
\pgfsys@useobject{currentmarker}{}%
\end{pgfscope}%
\begin{pgfscope}%
\pgfsys@transformshift{3.616057in}{0.850410in}%
\pgfsys@useobject{currentmarker}{}%
\end{pgfscope}%
\begin{pgfscope}%
\pgfsys@transformshift{3.616448in}{0.832319in}%
\pgfsys@useobject{currentmarker}{}%
\end{pgfscope}%
\begin{pgfscope}%
\pgfsys@transformshift{3.616839in}{0.832389in}%
\pgfsys@useobject{currentmarker}{}%
\end{pgfscope}%
\begin{pgfscope}%
\pgfsys@transformshift{3.617229in}{0.830181in}%
\pgfsys@useobject{currentmarker}{}%
\end{pgfscope}%
\begin{pgfscope}%
\pgfsys@transformshift{3.617619in}{0.838000in}%
\pgfsys@useobject{currentmarker}{}%
\end{pgfscope}%
\begin{pgfscope}%
\pgfsys@transformshift{3.618009in}{0.833799in}%
\pgfsys@useobject{currentmarker}{}%
\end{pgfscope}%
\begin{pgfscope}%
\pgfsys@transformshift{3.618398in}{0.841412in}%
\pgfsys@useobject{currentmarker}{}%
\end{pgfscope}%
\begin{pgfscope}%
\pgfsys@transformshift{3.618787in}{0.840289in}%
\pgfsys@useobject{currentmarker}{}%
\end{pgfscope}%
\begin{pgfscope}%
\pgfsys@transformshift{3.619175in}{0.836197in}%
\pgfsys@useobject{currentmarker}{}%
\end{pgfscope}%
\begin{pgfscope}%
\pgfsys@transformshift{3.619563in}{0.844364in}%
\pgfsys@useobject{currentmarker}{}%
\end{pgfscope}%
\begin{pgfscope}%
\pgfsys@transformshift{3.619951in}{0.851543in}%
\pgfsys@useobject{currentmarker}{}%
\end{pgfscope}%
\begin{pgfscope}%
\pgfsys@transformshift{3.620338in}{0.849784in}%
\pgfsys@useobject{currentmarker}{}%
\end{pgfscope}%
\begin{pgfscope}%
\pgfsys@transformshift{3.620725in}{0.836746in}%
\pgfsys@useobject{currentmarker}{}%
\end{pgfscope}%
\begin{pgfscope}%
\pgfsys@transformshift{3.621112in}{0.834583in}%
\pgfsys@useobject{currentmarker}{}%
\end{pgfscope}%
\begin{pgfscope}%
\pgfsys@transformshift{3.621498in}{0.823714in}%
\pgfsys@useobject{currentmarker}{}%
\end{pgfscope}%
\begin{pgfscope}%
\pgfsys@transformshift{3.621884in}{0.833020in}%
\pgfsys@useobject{currentmarker}{}%
\end{pgfscope}%
\begin{pgfscope}%
\pgfsys@transformshift{3.622269in}{0.824570in}%
\pgfsys@useobject{currentmarker}{}%
\end{pgfscope}%
\begin{pgfscope}%
\pgfsys@transformshift{3.622654in}{0.825688in}%
\pgfsys@useobject{currentmarker}{}%
\end{pgfscope}%
\begin{pgfscope}%
\pgfsys@transformshift{3.623039in}{0.838073in}%
\pgfsys@useobject{currentmarker}{}%
\end{pgfscope}%
\begin{pgfscope}%
\pgfsys@transformshift{3.623423in}{0.832710in}%
\pgfsys@useobject{currentmarker}{}%
\end{pgfscope}%
\begin{pgfscope}%
\pgfsys@transformshift{3.623807in}{0.843180in}%
\pgfsys@useobject{currentmarker}{}%
\end{pgfscope}%
\begin{pgfscope}%
\pgfsys@transformshift{3.624191in}{0.843659in}%
\pgfsys@useobject{currentmarker}{}%
\end{pgfscope}%
\begin{pgfscope}%
\pgfsys@transformshift{3.624574in}{0.824738in}%
\pgfsys@useobject{currentmarker}{}%
\end{pgfscope}%
\begin{pgfscope}%
\pgfsys@transformshift{3.624957in}{0.828035in}%
\pgfsys@useobject{currentmarker}{}%
\end{pgfscope}%
\begin{pgfscope}%
\pgfsys@transformshift{3.625339in}{0.844715in}%
\pgfsys@useobject{currentmarker}{}%
\end{pgfscope}%
\begin{pgfscope}%
\pgfsys@transformshift{3.625722in}{0.847946in}%
\pgfsys@useobject{currentmarker}{}%
\end{pgfscope}%
\begin{pgfscope}%
\pgfsys@transformshift{3.626103in}{0.852440in}%
\pgfsys@useobject{currentmarker}{}%
\end{pgfscope}%
\begin{pgfscope}%
\pgfsys@transformshift{3.626485in}{0.832655in}%
\pgfsys@useobject{currentmarker}{}%
\end{pgfscope}%
\begin{pgfscope}%
\pgfsys@transformshift{3.626866in}{0.812222in}%
\pgfsys@useobject{currentmarker}{}%
\end{pgfscope}%
\begin{pgfscope}%
\pgfsys@transformshift{3.627247in}{0.827140in}%
\pgfsys@useobject{currentmarker}{}%
\end{pgfscope}%
\begin{pgfscope}%
\pgfsys@transformshift{3.627627in}{0.856873in}%
\pgfsys@useobject{currentmarker}{}%
\end{pgfscope}%
\begin{pgfscope}%
\pgfsys@transformshift{3.628007in}{0.845705in}%
\pgfsys@useobject{currentmarker}{}%
\end{pgfscope}%
\begin{pgfscope}%
\pgfsys@transformshift{3.628387in}{0.830068in}%
\pgfsys@useobject{currentmarker}{}%
\end{pgfscope}%
\begin{pgfscope}%
\pgfsys@transformshift{3.628766in}{0.820769in}%
\pgfsys@useobject{currentmarker}{}%
\end{pgfscope}%
\begin{pgfscope}%
\pgfsys@transformshift{3.629145in}{0.834057in}%
\pgfsys@useobject{currentmarker}{}%
\end{pgfscope}%
\begin{pgfscope}%
\pgfsys@transformshift{3.629524in}{0.824859in}%
\pgfsys@useobject{currentmarker}{}%
\end{pgfscope}%
\begin{pgfscope}%
\pgfsys@transformshift{3.629902in}{0.822533in}%
\pgfsys@useobject{currentmarker}{}%
\end{pgfscope}%
\begin{pgfscope}%
\pgfsys@transformshift{3.630280in}{0.825211in}%
\pgfsys@useobject{currentmarker}{}%
\end{pgfscope}%
\begin{pgfscope}%
\pgfsys@transformshift{3.630657in}{0.836133in}%
\pgfsys@useobject{currentmarker}{}%
\end{pgfscope}%
\begin{pgfscope}%
\pgfsys@transformshift{3.631034in}{0.829992in}%
\pgfsys@useobject{currentmarker}{}%
\end{pgfscope}%
\begin{pgfscope}%
\pgfsys@transformshift{3.631411in}{0.823339in}%
\pgfsys@useobject{currentmarker}{}%
\end{pgfscope}%
\begin{pgfscope}%
\pgfsys@transformshift{3.631788in}{0.822866in}%
\pgfsys@useobject{currentmarker}{}%
\end{pgfscope}%
\begin{pgfscope}%
\pgfsys@transformshift{3.632164in}{0.834014in}%
\pgfsys@useobject{currentmarker}{}%
\end{pgfscope}%
\begin{pgfscope}%
\pgfsys@transformshift{3.632540in}{0.826121in}%
\pgfsys@useobject{currentmarker}{}%
\end{pgfscope}%
\begin{pgfscope}%
\pgfsys@transformshift{3.632915in}{0.801001in}%
\pgfsys@useobject{currentmarker}{}%
\end{pgfscope}%
\begin{pgfscope}%
\pgfsys@transformshift{3.633290in}{0.826752in}%
\pgfsys@useobject{currentmarker}{}%
\end{pgfscope}%
\begin{pgfscope}%
\pgfsys@transformshift{3.633665in}{0.831981in}%
\pgfsys@useobject{currentmarker}{}%
\end{pgfscope}%
\begin{pgfscope}%
\pgfsys@transformshift{3.634039in}{0.833283in}%
\pgfsys@useobject{currentmarker}{}%
\end{pgfscope}%
\begin{pgfscope}%
\pgfsys@transformshift{3.634413in}{0.832154in}%
\pgfsys@useobject{currentmarker}{}%
\end{pgfscope}%
\begin{pgfscope}%
\pgfsys@transformshift{3.634787in}{0.849131in}%
\pgfsys@useobject{currentmarker}{}%
\end{pgfscope}%
\begin{pgfscope}%
\pgfsys@transformshift{3.635160in}{0.836600in}%
\pgfsys@useobject{currentmarker}{}%
\end{pgfscope}%
\begin{pgfscope}%
\pgfsys@transformshift{3.635533in}{0.832067in}%
\pgfsys@useobject{currentmarker}{}%
\end{pgfscope}%
\begin{pgfscope}%
\pgfsys@transformshift{3.635906in}{0.837310in}%
\pgfsys@useobject{currentmarker}{}%
\end{pgfscope}%
\begin{pgfscope}%
\pgfsys@transformshift{3.636278in}{0.841158in}%
\pgfsys@useobject{currentmarker}{}%
\end{pgfscope}%
\begin{pgfscope}%
\pgfsys@transformshift{3.636650in}{0.844696in}%
\pgfsys@useobject{currentmarker}{}%
\end{pgfscope}%
\begin{pgfscope}%
\pgfsys@transformshift{3.637022in}{0.854934in}%
\pgfsys@useobject{currentmarker}{}%
\end{pgfscope}%
\begin{pgfscope}%
\pgfsys@transformshift{3.637393in}{0.847876in}%
\pgfsys@useobject{currentmarker}{}%
\end{pgfscope}%
\begin{pgfscope}%
\pgfsys@transformshift{3.637764in}{0.829449in}%
\pgfsys@useobject{currentmarker}{}%
\end{pgfscope}%
\begin{pgfscope}%
\pgfsys@transformshift{3.638135in}{0.848548in}%
\pgfsys@useobject{currentmarker}{}%
\end{pgfscope}%
\begin{pgfscope}%
\pgfsys@transformshift{3.638505in}{0.862039in}%
\pgfsys@useobject{currentmarker}{}%
\end{pgfscope}%
\begin{pgfscope}%
\pgfsys@transformshift{3.638875in}{0.854357in}%
\pgfsys@useobject{currentmarker}{}%
\end{pgfscope}%
\begin{pgfscope}%
\pgfsys@transformshift{3.639244in}{0.835035in}%
\pgfsys@useobject{currentmarker}{}%
\end{pgfscope}%
\begin{pgfscope}%
\pgfsys@transformshift{3.639613in}{0.853052in}%
\pgfsys@useobject{currentmarker}{}%
\end{pgfscope}%
\begin{pgfscope}%
\pgfsys@transformshift{3.639982in}{0.854413in}%
\pgfsys@useobject{currentmarker}{}%
\end{pgfscope}%
\begin{pgfscope}%
\pgfsys@transformshift{3.640351in}{0.843855in}%
\pgfsys@useobject{currentmarker}{}%
\end{pgfscope}%
\begin{pgfscope}%
\pgfsys@transformshift{3.640719in}{0.833935in}%
\pgfsys@useobject{currentmarker}{}%
\end{pgfscope}%
\begin{pgfscope}%
\pgfsys@transformshift{3.641087in}{0.830237in}%
\pgfsys@useobject{currentmarker}{}%
\end{pgfscope}%
\begin{pgfscope}%
\pgfsys@transformshift{3.641455in}{0.851573in}%
\pgfsys@useobject{currentmarker}{}%
\end{pgfscope}%
\begin{pgfscope}%
\pgfsys@transformshift{3.641822in}{0.862668in}%
\pgfsys@useobject{currentmarker}{}%
\end{pgfscope}%
\begin{pgfscope}%
\pgfsys@transformshift{3.642189in}{0.859618in}%
\pgfsys@useobject{currentmarker}{}%
\end{pgfscope}%
\begin{pgfscope}%
\pgfsys@transformshift{3.642555in}{0.836326in}%
\pgfsys@useobject{currentmarker}{}%
\end{pgfscope}%
\begin{pgfscope}%
\pgfsys@transformshift{3.642922in}{0.799992in}%
\pgfsys@useobject{currentmarker}{}%
\end{pgfscope}%
\begin{pgfscope}%
\pgfsys@transformshift{3.643287in}{0.826218in}%
\pgfsys@useobject{currentmarker}{}%
\end{pgfscope}%
\begin{pgfscope}%
\pgfsys@transformshift{3.643653in}{0.834111in}%
\pgfsys@useobject{currentmarker}{}%
\end{pgfscope}%
\begin{pgfscope}%
\pgfsys@transformshift{3.644018in}{0.835982in}%
\pgfsys@useobject{currentmarker}{}%
\end{pgfscope}%
\begin{pgfscope}%
\pgfsys@transformshift{3.644383in}{0.824879in}%
\pgfsys@useobject{currentmarker}{}%
\end{pgfscope}%
\begin{pgfscope}%
\pgfsys@transformshift{3.644748in}{0.828053in}%
\pgfsys@useobject{currentmarker}{}%
\end{pgfscope}%
\begin{pgfscope}%
\pgfsys@transformshift{3.645112in}{0.837456in}%
\pgfsys@useobject{currentmarker}{}%
\end{pgfscope}%
\begin{pgfscope}%
\pgfsys@transformshift{3.645476in}{0.825607in}%
\pgfsys@useobject{currentmarker}{}%
\end{pgfscope}%
\begin{pgfscope}%
\pgfsys@transformshift{3.645839in}{0.826556in}%
\pgfsys@useobject{currentmarker}{}%
\end{pgfscope}%
\begin{pgfscope}%
\pgfsys@transformshift{3.646203in}{0.820695in}%
\pgfsys@useobject{currentmarker}{}%
\end{pgfscope}%
\begin{pgfscope}%
\pgfsys@transformshift{3.646566in}{0.826816in}%
\pgfsys@useobject{currentmarker}{}%
\end{pgfscope}%
\begin{pgfscope}%
\pgfsys@transformshift{3.646928in}{0.851853in}%
\pgfsys@useobject{currentmarker}{}%
\end{pgfscope}%
\begin{pgfscope}%
\pgfsys@transformshift{3.647291in}{0.855526in}%
\pgfsys@useobject{currentmarker}{}%
\end{pgfscope}%
\begin{pgfscope}%
\pgfsys@transformshift{3.647652in}{0.843815in}%
\pgfsys@useobject{currentmarker}{}%
\end{pgfscope}%
\begin{pgfscope}%
\pgfsys@transformshift{3.648014in}{0.837363in}%
\pgfsys@useobject{currentmarker}{}%
\end{pgfscope}%
\begin{pgfscope}%
\pgfsys@transformshift{3.648375in}{0.836275in}%
\pgfsys@useobject{currentmarker}{}%
\end{pgfscope}%
\begin{pgfscope}%
\pgfsys@transformshift{3.648736in}{0.836581in}%
\pgfsys@useobject{currentmarker}{}%
\end{pgfscope}%
\begin{pgfscope}%
\pgfsys@transformshift{3.649097in}{0.836790in}%
\pgfsys@useobject{currentmarker}{}%
\end{pgfscope}%
\begin{pgfscope}%
\pgfsys@transformshift{3.649457in}{0.838743in}%
\pgfsys@useobject{currentmarker}{}%
\end{pgfscope}%
\begin{pgfscope}%
\pgfsys@transformshift{3.649817in}{0.836362in}%
\pgfsys@useobject{currentmarker}{}%
\end{pgfscope}%
\begin{pgfscope}%
\pgfsys@transformshift{3.650177in}{0.821544in}%
\pgfsys@useobject{currentmarker}{}%
\end{pgfscope}%
\begin{pgfscope}%
\pgfsys@transformshift{3.650536in}{0.822561in}%
\pgfsys@useobject{currentmarker}{}%
\end{pgfscope}%
\begin{pgfscope}%
\pgfsys@transformshift{3.650896in}{0.827141in}%
\pgfsys@useobject{currentmarker}{}%
\end{pgfscope}%
\begin{pgfscope}%
\pgfsys@transformshift{3.651254in}{0.808379in}%
\pgfsys@useobject{currentmarker}{}%
\end{pgfscope}%
\begin{pgfscope}%
\pgfsys@transformshift{3.651613in}{0.804527in}%
\pgfsys@useobject{currentmarker}{}%
\end{pgfscope}%
\begin{pgfscope}%
\pgfsys@transformshift{3.651971in}{0.830569in}%
\pgfsys@useobject{currentmarker}{}%
\end{pgfscope}%
\begin{pgfscope}%
\pgfsys@transformshift{3.652329in}{0.829108in}%
\pgfsys@useobject{currentmarker}{}%
\end{pgfscope}%
\begin{pgfscope}%
\pgfsys@transformshift{3.652686in}{0.833766in}%
\pgfsys@useobject{currentmarker}{}%
\end{pgfscope}%
\begin{pgfscope}%
\pgfsys@transformshift{3.653043in}{0.846256in}%
\pgfsys@useobject{currentmarker}{}%
\end{pgfscope}%
\begin{pgfscope}%
\pgfsys@transformshift{3.653400in}{0.851109in}%
\pgfsys@useobject{currentmarker}{}%
\end{pgfscope}%
\begin{pgfscope}%
\pgfsys@transformshift{3.653757in}{0.821715in}%
\pgfsys@useobject{currentmarker}{}%
\end{pgfscope}%
\begin{pgfscope}%
\pgfsys@transformshift{3.654113in}{0.813680in}%
\pgfsys@useobject{currentmarker}{}%
\end{pgfscope}%
\begin{pgfscope}%
\pgfsys@transformshift{3.654469in}{0.850251in}%
\pgfsys@useobject{currentmarker}{}%
\end{pgfscope}%
\begin{pgfscope}%
\pgfsys@transformshift{3.654824in}{0.833650in}%
\pgfsys@useobject{currentmarker}{}%
\end{pgfscope}%
\begin{pgfscope}%
\pgfsys@transformshift{3.655180in}{0.850918in}%
\pgfsys@useobject{currentmarker}{}%
\end{pgfscope}%
\begin{pgfscope}%
\pgfsys@transformshift{3.655534in}{0.839944in}%
\pgfsys@useobject{currentmarker}{}%
\end{pgfscope}%
\begin{pgfscope}%
\pgfsys@transformshift{3.655889in}{0.860423in}%
\pgfsys@useobject{currentmarker}{}%
\end{pgfscope}%
\begin{pgfscope}%
\pgfsys@transformshift{3.656243in}{0.845578in}%
\pgfsys@useobject{currentmarker}{}%
\end{pgfscope}%
\begin{pgfscope}%
\pgfsys@transformshift{3.656597in}{0.826260in}%
\pgfsys@useobject{currentmarker}{}%
\end{pgfscope}%
\begin{pgfscope}%
\pgfsys@transformshift{3.656951in}{0.833034in}%
\pgfsys@useobject{currentmarker}{}%
\end{pgfscope}%
\begin{pgfscope}%
\pgfsys@transformshift{3.657305in}{0.837756in}%
\pgfsys@useobject{currentmarker}{}%
\end{pgfscope}%
\begin{pgfscope}%
\pgfsys@transformshift{3.657658in}{0.830539in}%
\pgfsys@useobject{currentmarker}{}%
\end{pgfscope}%
\begin{pgfscope}%
\pgfsys@transformshift{3.658010in}{0.836556in}%
\pgfsys@useobject{currentmarker}{}%
\end{pgfscope}%
\begin{pgfscope}%
\pgfsys@transformshift{3.658363in}{0.846706in}%
\pgfsys@useobject{currentmarker}{}%
\end{pgfscope}%
\begin{pgfscope}%
\pgfsys@transformshift{3.658715in}{0.850013in}%
\pgfsys@useobject{currentmarker}{}%
\end{pgfscope}%
\begin{pgfscope}%
\pgfsys@transformshift{3.659067in}{0.820055in}%
\pgfsys@useobject{currentmarker}{}%
\end{pgfscope}%
\begin{pgfscope}%
\pgfsys@transformshift{3.659419in}{0.827730in}%
\pgfsys@useobject{currentmarker}{}%
\end{pgfscope}%
\begin{pgfscope}%
\pgfsys@transformshift{3.659770in}{0.835298in}%
\pgfsys@useobject{currentmarker}{}%
\end{pgfscope}%
\begin{pgfscope}%
\pgfsys@transformshift{3.660121in}{0.827793in}%
\pgfsys@useobject{currentmarker}{}%
\end{pgfscope}%
\begin{pgfscope}%
\pgfsys@transformshift{3.660471in}{0.803880in}%
\pgfsys@useobject{currentmarker}{}%
\end{pgfscope}%
\begin{pgfscope}%
\pgfsys@transformshift{3.660822in}{0.820865in}%
\pgfsys@useobject{currentmarker}{}%
\end{pgfscope}%
\begin{pgfscope}%
\pgfsys@transformshift{3.661172in}{0.833852in}%
\pgfsys@useobject{currentmarker}{}%
\end{pgfscope}%
\begin{pgfscope}%
\pgfsys@transformshift{3.661521in}{0.824786in}%
\pgfsys@useobject{currentmarker}{}%
\end{pgfscope}%
\begin{pgfscope}%
\pgfsys@transformshift{3.661871in}{0.831180in}%
\pgfsys@useobject{currentmarker}{}%
\end{pgfscope}%
\begin{pgfscope}%
\pgfsys@transformshift{3.662220in}{0.833227in}%
\pgfsys@useobject{currentmarker}{}%
\end{pgfscope}%
\begin{pgfscope}%
\pgfsys@transformshift{3.662569in}{0.825685in}%
\pgfsys@useobject{currentmarker}{}%
\end{pgfscope}%
\begin{pgfscope}%
\pgfsys@transformshift{3.662917in}{0.810126in}%
\pgfsys@useobject{currentmarker}{}%
\end{pgfscope}%
\begin{pgfscope}%
\pgfsys@transformshift{3.663266in}{0.834873in}%
\pgfsys@useobject{currentmarker}{}%
\end{pgfscope}%
\begin{pgfscope}%
\pgfsys@transformshift{3.663614in}{0.842123in}%
\pgfsys@useobject{currentmarker}{}%
\end{pgfscope}%
\begin{pgfscope}%
\pgfsys@transformshift{3.663961in}{0.814064in}%
\pgfsys@useobject{currentmarker}{}%
\end{pgfscope}%
\begin{pgfscope}%
\pgfsys@transformshift{3.664308in}{0.813957in}%
\pgfsys@useobject{currentmarker}{}%
\end{pgfscope}%
\begin{pgfscope}%
\pgfsys@transformshift{3.664655in}{0.837976in}%
\pgfsys@useobject{currentmarker}{}%
\end{pgfscope}%
\begin{pgfscope}%
\pgfsys@transformshift{3.665002in}{0.840907in}%
\pgfsys@useobject{currentmarker}{}%
\end{pgfscope}%
\begin{pgfscope}%
\pgfsys@transformshift{3.665349in}{0.847293in}%
\pgfsys@useobject{currentmarker}{}%
\end{pgfscope}%
\begin{pgfscope}%
\pgfsys@transformshift{3.665695in}{0.831118in}%
\pgfsys@useobject{currentmarker}{}%
\end{pgfscope}%
\begin{pgfscope}%
\pgfsys@transformshift{3.666041in}{0.828439in}%
\pgfsys@useobject{currentmarker}{}%
\end{pgfscope}%
\begin{pgfscope}%
\pgfsys@transformshift{3.666386in}{0.817233in}%
\pgfsys@useobject{currentmarker}{}%
\end{pgfscope}%
\begin{pgfscope}%
\pgfsys@transformshift{3.666731in}{0.838174in}%
\pgfsys@useobject{currentmarker}{}%
\end{pgfscope}%
\begin{pgfscope}%
\pgfsys@transformshift{3.667076in}{0.833546in}%
\pgfsys@useobject{currentmarker}{}%
\end{pgfscope}%
\begin{pgfscope}%
\pgfsys@transformshift{3.667421in}{0.805304in}%
\pgfsys@useobject{currentmarker}{}%
\end{pgfscope}%
\begin{pgfscope}%
\pgfsys@transformshift{3.667765in}{0.799002in}%
\pgfsys@useobject{currentmarker}{}%
\end{pgfscope}%
\begin{pgfscope}%
\pgfsys@transformshift{3.668109in}{0.807280in}%
\pgfsys@useobject{currentmarker}{}%
\end{pgfscope}%
\begin{pgfscope}%
\pgfsys@transformshift{3.668453in}{0.829045in}%
\pgfsys@useobject{currentmarker}{}%
\end{pgfscope}%
\begin{pgfscope}%
\pgfsys@transformshift{3.668797in}{0.835915in}%
\pgfsys@useobject{currentmarker}{}%
\end{pgfscope}%
\begin{pgfscope}%
\pgfsys@transformshift{3.669140in}{0.813211in}%
\pgfsys@useobject{currentmarker}{}%
\end{pgfscope}%
\begin{pgfscope}%
\pgfsys@transformshift{3.669483in}{0.811690in}%
\pgfsys@useobject{currentmarker}{}%
\end{pgfscope}%
\begin{pgfscope}%
\pgfsys@transformshift{3.669825in}{0.815709in}%
\pgfsys@useobject{currentmarker}{}%
\end{pgfscope}%
\begin{pgfscope}%
\pgfsys@transformshift{3.670168in}{0.809718in}%
\pgfsys@useobject{currentmarker}{}%
\end{pgfscope}%
\begin{pgfscope}%
\pgfsys@transformshift{3.670510in}{0.804010in}%
\pgfsys@useobject{currentmarker}{}%
\end{pgfscope}%
\begin{pgfscope}%
\pgfsys@transformshift{3.670851in}{0.814994in}%
\pgfsys@useobject{currentmarker}{}%
\end{pgfscope}%
\begin{pgfscope}%
\pgfsys@transformshift{3.671193in}{0.836094in}%
\pgfsys@useobject{currentmarker}{}%
\end{pgfscope}%
\begin{pgfscope}%
\pgfsys@transformshift{3.671534in}{0.827188in}%
\pgfsys@useobject{currentmarker}{}%
\end{pgfscope}%
\begin{pgfscope}%
\pgfsys@transformshift{3.671875in}{0.810679in}%
\pgfsys@useobject{currentmarker}{}%
\end{pgfscope}%
\begin{pgfscope}%
\pgfsys@transformshift{3.672215in}{0.802611in}%
\pgfsys@useobject{currentmarker}{}%
\end{pgfscope}%
\begin{pgfscope}%
\pgfsys@transformshift{3.672556in}{0.816780in}%
\pgfsys@useobject{currentmarker}{}%
\end{pgfscope}%
\begin{pgfscope}%
\pgfsys@transformshift{3.672896in}{0.839474in}%
\pgfsys@useobject{currentmarker}{}%
\end{pgfscope}%
\begin{pgfscope}%
\pgfsys@transformshift{3.673236in}{0.826777in}%
\pgfsys@useobject{currentmarker}{}%
\end{pgfscope}%
\begin{pgfscope}%
\pgfsys@transformshift{3.673575in}{0.836591in}%
\pgfsys@useobject{currentmarker}{}%
\end{pgfscope}%
\begin{pgfscope}%
\pgfsys@transformshift{3.673914in}{0.835653in}%
\pgfsys@useobject{currentmarker}{}%
\end{pgfscope}%
\begin{pgfscope}%
\pgfsys@transformshift{3.674253in}{0.840129in}%
\pgfsys@useobject{currentmarker}{}%
\end{pgfscope}%
\begin{pgfscope}%
\pgfsys@transformshift{3.674592in}{0.828303in}%
\pgfsys@useobject{currentmarker}{}%
\end{pgfscope}%
\begin{pgfscope}%
\pgfsys@transformshift{3.674930in}{0.835321in}%
\pgfsys@useobject{currentmarker}{}%
\end{pgfscope}%
\begin{pgfscope}%
\pgfsys@transformshift{3.675268in}{0.830285in}%
\pgfsys@useobject{currentmarker}{}%
\end{pgfscope}%
\begin{pgfscope}%
\pgfsys@transformshift{3.675606in}{0.834308in}%
\pgfsys@useobject{currentmarker}{}%
\end{pgfscope}%
\begin{pgfscope}%
\pgfsys@transformshift{3.675943in}{0.827509in}%
\pgfsys@useobject{currentmarker}{}%
\end{pgfscope}%
\begin{pgfscope}%
\pgfsys@transformshift{3.676280in}{0.828998in}%
\pgfsys@useobject{currentmarker}{}%
\end{pgfscope}%
\begin{pgfscope}%
\pgfsys@transformshift{3.676617in}{0.831600in}%
\pgfsys@useobject{currentmarker}{}%
\end{pgfscope}%
\begin{pgfscope}%
\pgfsys@transformshift{3.676954in}{0.817318in}%
\pgfsys@useobject{currentmarker}{}%
\end{pgfscope}%
\begin{pgfscope}%
\pgfsys@transformshift{3.677290in}{0.818946in}%
\pgfsys@useobject{currentmarker}{}%
\end{pgfscope}%
\begin{pgfscope}%
\pgfsys@transformshift{3.677626in}{0.824389in}%
\pgfsys@useobject{currentmarker}{}%
\end{pgfscope}%
\begin{pgfscope}%
\pgfsys@transformshift{3.677962in}{0.829939in}%
\pgfsys@useobject{currentmarker}{}%
\end{pgfscope}%
\begin{pgfscope}%
\pgfsys@transformshift{3.678297in}{0.836408in}%
\pgfsys@useobject{currentmarker}{}%
\end{pgfscope}%
\begin{pgfscope}%
\pgfsys@transformshift{3.678633in}{0.838284in}%
\pgfsys@useobject{currentmarker}{}%
\end{pgfscope}%
\begin{pgfscope}%
\pgfsys@transformshift{3.678968in}{0.820477in}%
\pgfsys@useobject{currentmarker}{}%
\end{pgfscope}%
\begin{pgfscope}%
\pgfsys@transformshift{3.679302in}{0.806286in}%
\pgfsys@useobject{currentmarker}{}%
\end{pgfscope}%
\begin{pgfscope}%
\pgfsys@transformshift{3.679637in}{0.818336in}%
\pgfsys@useobject{currentmarker}{}%
\end{pgfscope}%
\begin{pgfscope}%
\pgfsys@transformshift{3.679971in}{0.836650in}%
\pgfsys@useobject{currentmarker}{}%
\end{pgfscope}%
\begin{pgfscope}%
\pgfsys@transformshift{3.680304in}{0.820709in}%
\pgfsys@useobject{currentmarker}{}%
\end{pgfscope}%
\begin{pgfscope}%
\pgfsys@transformshift{3.680638in}{0.834548in}%
\pgfsys@useobject{currentmarker}{}%
\end{pgfscope}%
\begin{pgfscope}%
\pgfsys@transformshift{3.680971in}{0.841698in}%
\pgfsys@useobject{currentmarker}{}%
\end{pgfscope}%
\begin{pgfscope}%
\pgfsys@transformshift{3.681304in}{0.835731in}%
\pgfsys@useobject{currentmarker}{}%
\end{pgfscope}%
\begin{pgfscope}%
\pgfsys@transformshift{3.681637in}{0.809583in}%
\pgfsys@useobject{currentmarker}{}%
\end{pgfscope}%
\begin{pgfscope}%
\pgfsys@transformshift{3.681969in}{0.826824in}%
\pgfsys@useobject{currentmarker}{}%
\end{pgfscope}%
\begin{pgfscope}%
\pgfsys@transformshift{3.682302in}{0.824846in}%
\pgfsys@useobject{currentmarker}{}%
\end{pgfscope}%
\begin{pgfscope}%
\pgfsys@transformshift{3.682634in}{0.816308in}%
\pgfsys@useobject{currentmarker}{}%
\end{pgfscope}%
\begin{pgfscope}%
\pgfsys@transformshift{3.682965in}{0.817773in}%
\pgfsys@useobject{currentmarker}{}%
\end{pgfscope}%
\begin{pgfscope}%
\pgfsys@transformshift{3.683297in}{0.816362in}%
\pgfsys@useobject{currentmarker}{}%
\end{pgfscope}%
\begin{pgfscope}%
\pgfsys@transformshift{3.683628in}{0.825344in}%
\pgfsys@useobject{currentmarker}{}%
\end{pgfscope}%
\begin{pgfscope}%
\pgfsys@transformshift{3.683958in}{0.821476in}%
\pgfsys@useobject{currentmarker}{}%
\end{pgfscope}%
\begin{pgfscope}%
\pgfsys@transformshift{3.684289in}{0.826162in}%
\pgfsys@useobject{currentmarker}{}%
\end{pgfscope}%
\begin{pgfscope}%
\pgfsys@transformshift{3.684619in}{0.836846in}%
\pgfsys@useobject{currentmarker}{}%
\end{pgfscope}%
\begin{pgfscope}%
\pgfsys@transformshift{3.684949in}{0.826701in}%
\pgfsys@useobject{currentmarker}{}%
\end{pgfscope}%
\begin{pgfscope}%
\pgfsys@transformshift{3.685279in}{0.830940in}%
\pgfsys@useobject{currentmarker}{}%
\end{pgfscope}%
\begin{pgfscope}%
\pgfsys@transformshift{3.685608in}{0.830162in}%
\pgfsys@useobject{currentmarker}{}%
\end{pgfscope}%
\begin{pgfscope}%
\pgfsys@transformshift{3.685938in}{0.823417in}%
\pgfsys@useobject{currentmarker}{}%
\end{pgfscope}%
\begin{pgfscope}%
\pgfsys@transformshift{3.686267in}{0.830289in}%
\pgfsys@useobject{currentmarker}{}%
\end{pgfscope}%
\begin{pgfscope}%
\pgfsys@transformshift{3.686595in}{0.832288in}%
\pgfsys@useobject{currentmarker}{}%
\end{pgfscope}%
\begin{pgfscope}%
\pgfsys@transformshift{3.686924in}{0.837167in}%
\pgfsys@useobject{currentmarker}{}%
\end{pgfscope}%
\begin{pgfscope}%
\pgfsys@transformshift{3.687252in}{0.817960in}%
\pgfsys@useobject{currentmarker}{}%
\end{pgfscope}%
\begin{pgfscope}%
\pgfsys@transformshift{3.687580in}{0.819610in}%
\pgfsys@useobject{currentmarker}{}%
\end{pgfscope}%
\begin{pgfscope}%
\pgfsys@transformshift{3.687907in}{0.824189in}%
\pgfsys@useobject{currentmarker}{}%
\end{pgfscope}%
\begin{pgfscope}%
\pgfsys@transformshift{3.688235in}{0.808631in}%
\pgfsys@useobject{currentmarker}{}%
\end{pgfscope}%
\begin{pgfscope}%
\pgfsys@transformshift{3.688562in}{0.805152in}%
\pgfsys@useobject{currentmarker}{}%
\end{pgfscope}%
\begin{pgfscope}%
\pgfsys@transformshift{3.688888in}{0.800653in}%
\pgfsys@useobject{currentmarker}{}%
\end{pgfscope}%
\begin{pgfscope}%
\pgfsys@transformshift{3.689215in}{0.826882in}%
\pgfsys@useobject{currentmarker}{}%
\end{pgfscope}%
\begin{pgfscope}%
\pgfsys@transformshift{3.689541in}{0.838342in}%
\pgfsys@useobject{currentmarker}{}%
\end{pgfscope}%
\begin{pgfscope}%
\pgfsys@transformshift{3.689867in}{0.821203in}%
\pgfsys@useobject{currentmarker}{}%
\end{pgfscope}%
\begin{pgfscope}%
\pgfsys@transformshift{3.690193in}{0.826582in}%
\pgfsys@useobject{currentmarker}{}%
\end{pgfscope}%
\begin{pgfscope}%
\pgfsys@transformshift{3.690518in}{0.812730in}%
\pgfsys@useobject{currentmarker}{}%
\end{pgfscope}%
\begin{pgfscope}%
\pgfsys@transformshift{3.690844in}{0.806961in}%
\pgfsys@useobject{currentmarker}{}%
\end{pgfscope}%
\begin{pgfscope}%
\pgfsys@transformshift{3.691169in}{0.816942in}%
\pgfsys@useobject{currentmarker}{}%
\end{pgfscope}%
\begin{pgfscope}%
\pgfsys@transformshift{3.691493in}{0.793421in}%
\pgfsys@useobject{currentmarker}{}%
\end{pgfscope}%
\begin{pgfscope}%
\pgfsys@transformshift{3.691818in}{0.791417in}%
\pgfsys@useobject{currentmarker}{}%
\end{pgfscope}%
\begin{pgfscope}%
\pgfsys@transformshift{3.692142in}{0.833235in}%
\pgfsys@useobject{currentmarker}{}%
\end{pgfscope}%
\begin{pgfscope}%
\pgfsys@transformshift{3.692466in}{0.824130in}%
\pgfsys@useobject{currentmarker}{}%
\end{pgfscope}%
\begin{pgfscope}%
\pgfsys@transformshift{3.692790in}{0.806499in}%
\pgfsys@useobject{currentmarker}{}%
\end{pgfscope}%
\begin{pgfscope}%
\pgfsys@transformshift{3.693113in}{0.812252in}%
\pgfsys@useobject{currentmarker}{}%
\end{pgfscope}%
\begin{pgfscope}%
\pgfsys@transformshift{3.693436in}{0.822375in}%
\pgfsys@useobject{currentmarker}{}%
\end{pgfscope}%
\begin{pgfscope}%
\pgfsys@transformshift{3.693759in}{0.823872in}%
\pgfsys@useobject{currentmarker}{}%
\end{pgfscope}%
\begin{pgfscope}%
\pgfsys@transformshift{3.694082in}{0.813313in}%
\pgfsys@useobject{currentmarker}{}%
\end{pgfscope}%
\begin{pgfscope}%
\pgfsys@transformshift{3.694404in}{0.823998in}%
\pgfsys@useobject{currentmarker}{}%
\end{pgfscope}%
\begin{pgfscope}%
\pgfsys@transformshift{3.694726in}{0.818718in}%
\pgfsys@useobject{currentmarker}{}%
\end{pgfscope}%
\begin{pgfscope}%
\pgfsys@transformshift{3.695048in}{0.828134in}%
\pgfsys@useobject{currentmarker}{}%
\end{pgfscope}%
\begin{pgfscope}%
\pgfsys@transformshift{3.695370in}{0.821059in}%
\pgfsys@useobject{currentmarker}{}%
\end{pgfscope}%
\begin{pgfscope}%
\pgfsys@transformshift{3.695691in}{0.814699in}%
\pgfsys@useobject{currentmarker}{}%
\end{pgfscope}%
\begin{pgfscope}%
\pgfsys@transformshift{3.696012in}{0.820100in}%
\pgfsys@useobject{currentmarker}{}%
\end{pgfscope}%
\begin{pgfscope}%
\pgfsys@transformshift{3.696333in}{0.821284in}%
\pgfsys@useobject{currentmarker}{}%
\end{pgfscope}%
\begin{pgfscope}%
\pgfsys@transformshift{3.696653in}{0.827324in}%
\pgfsys@useobject{currentmarker}{}%
\end{pgfscope}%
\begin{pgfscope}%
\pgfsys@transformshift{3.696974in}{0.817973in}%
\pgfsys@useobject{currentmarker}{}%
\end{pgfscope}%
\begin{pgfscope}%
\pgfsys@transformshift{3.697294in}{0.800557in}%
\pgfsys@useobject{currentmarker}{}%
\end{pgfscope}%
\begin{pgfscope}%
\pgfsys@transformshift{3.697614in}{0.800353in}%
\pgfsys@useobject{currentmarker}{}%
\end{pgfscope}%
\begin{pgfscope}%
\pgfsys@transformshift{3.697933in}{0.808698in}%
\pgfsys@useobject{currentmarker}{}%
\end{pgfscope}%
\begin{pgfscope}%
\pgfsys@transformshift{3.698252in}{0.815146in}%
\pgfsys@useobject{currentmarker}{}%
\end{pgfscope}%
\begin{pgfscope}%
\pgfsys@transformshift{3.698571in}{0.819207in}%
\pgfsys@useobject{currentmarker}{}%
\end{pgfscope}%
\begin{pgfscope}%
\pgfsys@transformshift{3.698890in}{0.808524in}%
\pgfsys@useobject{currentmarker}{}%
\end{pgfscope}%
\begin{pgfscope}%
\pgfsys@transformshift{3.699209in}{0.794353in}%
\pgfsys@useobject{currentmarker}{}%
\end{pgfscope}%
\begin{pgfscope}%
\pgfsys@transformshift{3.699527in}{0.802125in}%
\pgfsys@useobject{currentmarker}{}%
\end{pgfscope}%
\begin{pgfscope}%
\pgfsys@transformshift{3.699845in}{0.808596in}%
\pgfsys@useobject{currentmarker}{}%
\end{pgfscope}%
\begin{pgfscope}%
\pgfsys@transformshift{3.700163in}{0.803803in}%
\pgfsys@useobject{currentmarker}{}%
\end{pgfscope}%
\begin{pgfscope}%
\pgfsys@transformshift{3.700480in}{0.819769in}%
\pgfsys@useobject{currentmarker}{}%
\end{pgfscope}%
\begin{pgfscope}%
\pgfsys@transformshift{3.700798in}{0.813199in}%
\pgfsys@useobject{currentmarker}{}%
\end{pgfscope}%
\begin{pgfscope}%
\pgfsys@transformshift{3.701115in}{0.819263in}%
\pgfsys@useobject{currentmarker}{}%
\end{pgfscope}%
\begin{pgfscope}%
\pgfsys@transformshift{3.701432in}{0.826394in}%
\pgfsys@useobject{currentmarker}{}%
\end{pgfscope}%
\begin{pgfscope}%
\pgfsys@transformshift{3.701748in}{0.817100in}%
\pgfsys@useobject{currentmarker}{}%
\end{pgfscope}%
\begin{pgfscope}%
\pgfsys@transformshift{3.702064in}{0.820690in}%
\pgfsys@useobject{currentmarker}{}%
\end{pgfscope}%
\begin{pgfscope}%
\pgfsys@transformshift{3.702381in}{0.810682in}%
\pgfsys@useobject{currentmarker}{}%
\end{pgfscope}%
\begin{pgfscope}%
\pgfsys@transformshift{3.702696in}{0.811865in}%
\pgfsys@useobject{currentmarker}{}%
\end{pgfscope}%
\begin{pgfscope}%
\pgfsys@transformshift{3.703012in}{0.820083in}%
\pgfsys@useobject{currentmarker}{}%
\end{pgfscope}%
\begin{pgfscope}%
\pgfsys@transformshift{3.703327in}{0.830076in}%
\pgfsys@useobject{currentmarker}{}%
\end{pgfscope}%
\begin{pgfscope}%
\pgfsys@transformshift{3.703642in}{0.836906in}%
\pgfsys@useobject{currentmarker}{}%
\end{pgfscope}%
\begin{pgfscope}%
\pgfsys@transformshift{3.703957in}{0.831119in}%
\pgfsys@useobject{currentmarker}{}%
\end{pgfscope}%
\begin{pgfscope}%
\pgfsys@transformshift{3.704272in}{0.816762in}%
\pgfsys@useobject{currentmarker}{}%
\end{pgfscope}%
\begin{pgfscope}%
\pgfsys@transformshift{3.704586in}{0.817923in}%
\pgfsys@useobject{currentmarker}{}%
\end{pgfscope}%
\begin{pgfscope}%
\pgfsys@transformshift{3.704900in}{0.818901in}%
\pgfsys@useobject{currentmarker}{}%
\end{pgfscope}%
\begin{pgfscope}%
\pgfsys@transformshift{3.705214in}{0.833851in}%
\pgfsys@useobject{currentmarker}{}%
\end{pgfscope}%
\begin{pgfscope}%
\pgfsys@transformshift{3.705528in}{0.823118in}%
\pgfsys@useobject{currentmarker}{}%
\end{pgfscope}%
\begin{pgfscope}%
\pgfsys@transformshift{3.705841in}{0.818455in}%
\pgfsys@useobject{currentmarker}{}%
\end{pgfscope}%
\begin{pgfscope}%
\pgfsys@transformshift{3.706154in}{0.828207in}%
\pgfsys@useobject{currentmarker}{}%
\end{pgfscope}%
\begin{pgfscope}%
\pgfsys@transformshift{3.706467in}{0.812300in}%
\pgfsys@useobject{currentmarker}{}%
\end{pgfscope}%
\begin{pgfscope}%
\pgfsys@transformshift{3.706780in}{0.810021in}%
\pgfsys@useobject{currentmarker}{}%
\end{pgfscope}%
\begin{pgfscope}%
\pgfsys@transformshift{3.707092in}{0.823444in}%
\pgfsys@useobject{currentmarker}{}%
\end{pgfscope}%
\begin{pgfscope}%
\pgfsys@transformshift{3.707404in}{0.825083in}%
\pgfsys@useobject{currentmarker}{}%
\end{pgfscope}%
\begin{pgfscope}%
\pgfsys@transformshift{3.707716in}{0.831111in}%
\pgfsys@useobject{currentmarker}{}%
\end{pgfscope}%
\begin{pgfscope}%
\pgfsys@transformshift{3.708028in}{0.814586in}%
\pgfsys@useobject{currentmarker}{}%
\end{pgfscope}%
\begin{pgfscope}%
\pgfsys@transformshift{3.708339in}{0.808384in}%
\pgfsys@useobject{currentmarker}{}%
\end{pgfscope}%
\begin{pgfscope}%
\pgfsys@transformshift{3.708650in}{0.812511in}%
\pgfsys@useobject{currentmarker}{}%
\end{pgfscope}%
\begin{pgfscope}%
\pgfsys@transformshift{3.708961in}{0.824690in}%
\pgfsys@useobject{currentmarker}{}%
\end{pgfscope}%
\begin{pgfscope}%
\pgfsys@transformshift{3.709272in}{0.833726in}%
\pgfsys@useobject{currentmarker}{}%
\end{pgfscope}%
\begin{pgfscope}%
\pgfsys@transformshift{3.709582in}{0.822154in}%
\pgfsys@useobject{currentmarker}{}%
\end{pgfscope}%
\begin{pgfscope}%
\pgfsys@transformshift{3.709893in}{0.815893in}%
\pgfsys@useobject{currentmarker}{}%
\end{pgfscope}%
\begin{pgfscope}%
\pgfsys@transformshift{3.710203in}{0.821860in}%
\pgfsys@useobject{currentmarker}{}%
\end{pgfscope}%
\begin{pgfscope}%
\pgfsys@transformshift{3.710512in}{0.811232in}%
\pgfsys@useobject{currentmarker}{}%
\end{pgfscope}%
\begin{pgfscope}%
\pgfsys@transformshift{3.710822in}{0.819721in}%
\pgfsys@useobject{currentmarker}{}%
\end{pgfscope}%
\begin{pgfscope}%
\pgfsys@transformshift{3.711131in}{0.838113in}%
\pgfsys@useobject{currentmarker}{}%
\end{pgfscope}%
\begin{pgfscope}%
\pgfsys@transformshift{3.711440in}{0.810799in}%
\pgfsys@useobject{currentmarker}{}%
\end{pgfscope}%
\begin{pgfscope}%
\pgfsys@transformshift{3.711749in}{0.811940in}%
\pgfsys@useobject{currentmarker}{}%
\end{pgfscope}%
\begin{pgfscope}%
\pgfsys@transformshift{3.712058in}{0.839419in}%
\pgfsys@useobject{currentmarker}{}%
\end{pgfscope}%
\begin{pgfscope}%
\pgfsys@transformshift{3.712366in}{0.831527in}%
\pgfsys@useobject{currentmarker}{}%
\end{pgfscope}%
\begin{pgfscope}%
\pgfsys@transformshift{3.712674in}{0.820987in}%
\pgfsys@useobject{currentmarker}{}%
\end{pgfscope}%
\begin{pgfscope}%
\pgfsys@transformshift{3.712982in}{0.811694in}%
\pgfsys@useobject{currentmarker}{}%
\end{pgfscope}%
\begin{pgfscope}%
\pgfsys@transformshift{3.713289in}{0.811127in}%
\pgfsys@useobject{currentmarker}{}%
\end{pgfscope}%
\begin{pgfscope}%
\pgfsys@transformshift{3.713597in}{0.810861in}%
\pgfsys@useobject{currentmarker}{}%
\end{pgfscope}%
\begin{pgfscope}%
\pgfsys@transformshift{3.713904in}{0.796312in}%
\pgfsys@useobject{currentmarker}{}%
\end{pgfscope}%
\begin{pgfscope}%
\pgfsys@transformshift{3.714211in}{0.795103in}%
\pgfsys@useobject{currentmarker}{}%
\end{pgfscope}%
\begin{pgfscope}%
\pgfsys@transformshift{3.714518in}{0.817841in}%
\pgfsys@useobject{currentmarker}{}%
\end{pgfscope}%
\begin{pgfscope}%
\pgfsys@transformshift{3.714824in}{0.828403in}%
\pgfsys@useobject{currentmarker}{}%
\end{pgfscope}%
\begin{pgfscope}%
\pgfsys@transformshift{3.715130in}{0.818993in}%
\pgfsys@useobject{currentmarker}{}%
\end{pgfscope}%
\begin{pgfscope}%
\pgfsys@transformshift{3.715436in}{0.822652in}%
\pgfsys@useobject{currentmarker}{}%
\end{pgfscope}%
\begin{pgfscope}%
\pgfsys@transformshift{3.715742in}{0.827435in}%
\pgfsys@useobject{currentmarker}{}%
\end{pgfscope}%
\begin{pgfscope}%
\pgfsys@transformshift{3.716048in}{0.819294in}%
\pgfsys@useobject{currentmarker}{}%
\end{pgfscope}%
\begin{pgfscope}%
\pgfsys@transformshift{3.716353in}{0.823993in}%
\pgfsys@useobject{currentmarker}{}%
\end{pgfscope}%
\begin{pgfscope}%
\pgfsys@transformshift{3.716658in}{0.835019in}%
\pgfsys@useobject{currentmarker}{}%
\end{pgfscope}%
\begin{pgfscope}%
\pgfsys@transformshift{3.716963in}{0.830031in}%
\pgfsys@useobject{currentmarker}{}%
\end{pgfscope}%
\begin{pgfscope}%
\pgfsys@transformshift{3.717267in}{0.817720in}%
\pgfsys@useobject{currentmarker}{}%
\end{pgfscope}%
\begin{pgfscope}%
\pgfsys@transformshift{3.717572in}{0.845534in}%
\pgfsys@useobject{currentmarker}{}%
\end{pgfscope}%
\begin{pgfscope}%
\pgfsys@transformshift{3.717876in}{0.831234in}%
\pgfsys@useobject{currentmarker}{}%
\end{pgfscope}%
\begin{pgfscope}%
\pgfsys@transformshift{3.718180in}{0.816804in}%
\pgfsys@useobject{currentmarker}{}%
\end{pgfscope}%
\begin{pgfscope}%
\pgfsys@transformshift{3.718484in}{0.807887in}%
\pgfsys@useobject{currentmarker}{}%
\end{pgfscope}%
\begin{pgfscope}%
\pgfsys@transformshift{3.718787in}{0.803672in}%
\pgfsys@useobject{currentmarker}{}%
\end{pgfscope}%
\begin{pgfscope}%
\pgfsys@transformshift{3.719090in}{0.801153in}%
\pgfsys@useobject{currentmarker}{}%
\end{pgfscope}%
\begin{pgfscope}%
\pgfsys@transformshift{3.719393in}{0.805038in}%
\pgfsys@useobject{currentmarker}{}%
\end{pgfscope}%
\begin{pgfscope}%
\pgfsys@transformshift{3.719696in}{0.810333in}%
\pgfsys@useobject{currentmarker}{}%
\end{pgfscope}%
\begin{pgfscope}%
\pgfsys@transformshift{3.719999in}{0.829061in}%
\pgfsys@useobject{currentmarker}{}%
\end{pgfscope}%
\begin{pgfscope}%
\pgfsys@transformshift{3.720301in}{0.821315in}%
\pgfsys@useobject{currentmarker}{}%
\end{pgfscope}%
\begin{pgfscope}%
\pgfsys@transformshift{3.720603in}{0.812723in}%
\pgfsys@useobject{currentmarker}{}%
\end{pgfscope}%
\begin{pgfscope}%
\pgfsys@transformshift{3.720905in}{0.816556in}%
\pgfsys@useobject{currentmarker}{}%
\end{pgfscope}%
\begin{pgfscope}%
\pgfsys@transformshift{3.721207in}{0.825754in}%
\pgfsys@useobject{currentmarker}{}%
\end{pgfscope}%
\begin{pgfscope}%
\pgfsys@transformshift{3.721508in}{0.829908in}%
\pgfsys@useobject{currentmarker}{}%
\end{pgfscope}%
\begin{pgfscope}%
\pgfsys@transformshift{3.721809in}{0.826288in}%
\pgfsys@useobject{currentmarker}{}%
\end{pgfscope}%
\begin{pgfscope}%
\pgfsys@transformshift{3.722110in}{0.837210in}%
\pgfsys@useobject{currentmarker}{}%
\end{pgfscope}%
\begin{pgfscope}%
\pgfsys@transformshift{3.722411in}{0.825947in}%
\pgfsys@useobject{currentmarker}{}%
\end{pgfscope}%
\begin{pgfscope}%
\pgfsys@transformshift{3.722712in}{0.815276in}%
\pgfsys@useobject{currentmarker}{}%
\end{pgfscope}%
\begin{pgfscope}%
\pgfsys@transformshift{3.723012in}{0.810722in}%
\pgfsys@useobject{currentmarker}{}%
\end{pgfscope}%
\begin{pgfscope}%
\pgfsys@transformshift{3.723312in}{0.813749in}%
\pgfsys@useobject{currentmarker}{}%
\end{pgfscope}%
\begin{pgfscope}%
\pgfsys@transformshift{3.723612in}{0.827566in}%
\pgfsys@useobject{currentmarker}{}%
\end{pgfscope}%
\begin{pgfscope}%
\pgfsys@transformshift{3.723911in}{0.830759in}%
\pgfsys@useobject{currentmarker}{}%
\end{pgfscope}%
\begin{pgfscope}%
\pgfsys@transformshift{3.724211in}{0.830634in}%
\pgfsys@useobject{currentmarker}{}%
\end{pgfscope}%
\begin{pgfscope}%
\pgfsys@transformshift{3.724510in}{0.820566in}%
\pgfsys@useobject{currentmarker}{}%
\end{pgfscope}%
\begin{pgfscope}%
\pgfsys@transformshift{3.724809in}{0.814869in}%
\pgfsys@useobject{currentmarker}{}%
\end{pgfscope}%
\begin{pgfscope}%
\pgfsys@transformshift{3.725108in}{0.830579in}%
\pgfsys@useobject{currentmarker}{}%
\end{pgfscope}%
\begin{pgfscope}%
\pgfsys@transformshift{3.725406in}{0.825473in}%
\pgfsys@useobject{currentmarker}{}%
\end{pgfscope}%
\begin{pgfscope}%
\pgfsys@transformshift{3.725705in}{0.807146in}%
\pgfsys@useobject{currentmarker}{}%
\end{pgfscope}%
\begin{pgfscope}%
\pgfsys@transformshift{3.726003in}{0.809788in}%
\pgfsys@useobject{currentmarker}{}%
\end{pgfscope}%
\begin{pgfscope}%
\pgfsys@transformshift{3.726301in}{0.806449in}%
\pgfsys@useobject{currentmarker}{}%
\end{pgfscope}%
\begin{pgfscope}%
\pgfsys@transformshift{3.726598in}{0.834823in}%
\pgfsys@useobject{currentmarker}{}%
\end{pgfscope}%
\begin{pgfscope}%
\pgfsys@transformshift{3.726896in}{0.838718in}%
\pgfsys@useobject{currentmarker}{}%
\end{pgfscope}%
\begin{pgfscope}%
\pgfsys@transformshift{3.727193in}{0.830426in}%
\pgfsys@useobject{currentmarker}{}%
\end{pgfscope}%
\begin{pgfscope}%
\pgfsys@transformshift{3.727490in}{0.837008in}%
\pgfsys@useobject{currentmarker}{}%
\end{pgfscope}%
\begin{pgfscope}%
\pgfsys@transformshift{3.727787in}{0.837297in}%
\pgfsys@useobject{currentmarker}{}%
\end{pgfscope}%
\begin{pgfscope}%
\pgfsys@transformshift{3.728083in}{0.840153in}%
\pgfsys@useobject{currentmarker}{}%
\end{pgfscope}%
\begin{pgfscope}%
\pgfsys@transformshift{3.728380in}{0.820689in}%
\pgfsys@useobject{currentmarker}{}%
\end{pgfscope}%
\begin{pgfscope}%
\pgfsys@transformshift{3.728676in}{0.821605in}%
\pgfsys@useobject{currentmarker}{}%
\end{pgfscope}%
\begin{pgfscope}%
\pgfsys@transformshift{3.728972in}{0.791983in}%
\pgfsys@useobject{currentmarker}{}%
\end{pgfscope}%
\begin{pgfscope}%
\pgfsys@transformshift{3.729267in}{0.788564in}%
\pgfsys@useobject{currentmarker}{}%
\end{pgfscope}%
\begin{pgfscope}%
\pgfsys@transformshift{3.729563in}{0.797049in}%
\pgfsys@useobject{currentmarker}{}%
\end{pgfscope}%
\begin{pgfscope}%
\pgfsys@transformshift{3.729858in}{0.811886in}%
\pgfsys@useobject{currentmarker}{}%
\end{pgfscope}%
\begin{pgfscope}%
\pgfsys@transformshift{3.730153in}{0.831684in}%
\pgfsys@useobject{currentmarker}{}%
\end{pgfscope}%
\begin{pgfscope}%
\pgfsys@transformshift{3.730448in}{0.803477in}%
\pgfsys@useobject{currentmarker}{}%
\end{pgfscope}%
\begin{pgfscope}%
\pgfsys@transformshift{3.730742in}{0.798778in}%
\pgfsys@useobject{currentmarker}{}%
\end{pgfscope}%
\begin{pgfscope}%
\pgfsys@transformshift{3.731037in}{0.810740in}%
\pgfsys@useobject{currentmarker}{}%
\end{pgfscope}%
\begin{pgfscope}%
\pgfsys@transformshift{3.731331in}{0.817053in}%
\pgfsys@useobject{currentmarker}{}%
\end{pgfscope}%
\begin{pgfscope}%
\pgfsys@transformshift{3.731625in}{0.827246in}%
\pgfsys@useobject{currentmarker}{}%
\end{pgfscope}%
\begin{pgfscope}%
\pgfsys@transformshift{3.731919in}{0.814134in}%
\pgfsys@useobject{currentmarker}{}%
\end{pgfscope}%
\begin{pgfscope}%
\pgfsys@transformshift{3.732212in}{0.812856in}%
\pgfsys@useobject{currentmarker}{}%
\end{pgfscope}%
\begin{pgfscope}%
\pgfsys@transformshift{3.732506in}{0.811657in}%
\pgfsys@useobject{currentmarker}{}%
\end{pgfscope}%
\begin{pgfscope}%
\pgfsys@transformshift{3.732799in}{0.804329in}%
\pgfsys@useobject{currentmarker}{}%
\end{pgfscope}%
\begin{pgfscope}%
\pgfsys@transformshift{3.733092in}{0.802041in}%
\pgfsys@useobject{currentmarker}{}%
\end{pgfscope}%
\begin{pgfscope}%
\pgfsys@transformshift{3.733384in}{0.807754in}%
\pgfsys@useobject{currentmarker}{}%
\end{pgfscope}%
\begin{pgfscope}%
\pgfsys@transformshift{3.733677in}{0.817732in}%
\pgfsys@useobject{currentmarker}{}%
\end{pgfscope}%
\begin{pgfscope}%
\pgfsys@transformshift{3.733969in}{0.822571in}%
\pgfsys@useobject{currentmarker}{}%
\end{pgfscope}%
\begin{pgfscope}%
\pgfsys@transformshift{3.734261in}{0.821179in}%
\pgfsys@useobject{currentmarker}{}%
\end{pgfscope}%
\begin{pgfscope}%
\pgfsys@transformshift{3.734553in}{0.819388in}%
\pgfsys@useobject{currentmarker}{}%
\end{pgfscope}%
\begin{pgfscope}%
\pgfsys@transformshift{3.734845in}{0.818226in}%
\pgfsys@useobject{currentmarker}{}%
\end{pgfscope}%
\begin{pgfscope}%
\pgfsys@transformshift{3.735136in}{0.810932in}%
\pgfsys@useobject{currentmarker}{}%
\end{pgfscope}%
\begin{pgfscope}%
\pgfsys@transformshift{3.735427in}{0.800405in}%
\pgfsys@useobject{currentmarker}{}%
\end{pgfscope}%
\begin{pgfscope}%
\pgfsys@transformshift{3.735718in}{0.809507in}%
\pgfsys@useobject{currentmarker}{}%
\end{pgfscope}%
\begin{pgfscope}%
\pgfsys@transformshift{3.736009in}{0.822384in}%
\pgfsys@useobject{currentmarker}{}%
\end{pgfscope}%
\begin{pgfscope}%
\pgfsys@transformshift{3.736300in}{0.822098in}%
\pgfsys@useobject{currentmarker}{}%
\end{pgfscope}%
\begin{pgfscope}%
\pgfsys@transformshift{3.736590in}{0.822785in}%
\pgfsys@useobject{currentmarker}{}%
\end{pgfscope}%
\begin{pgfscope}%
\pgfsys@transformshift{3.736880in}{0.818126in}%
\pgfsys@useobject{currentmarker}{}%
\end{pgfscope}%
\begin{pgfscope}%
\pgfsys@transformshift{3.737170in}{0.807657in}%
\pgfsys@useobject{currentmarker}{}%
\end{pgfscope}%
\begin{pgfscope}%
\pgfsys@transformshift{3.737460in}{0.813938in}%
\pgfsys@useobject{currentmarker}{}%
\end{pgfscope}%
\begin{pgfscope}%
\pgfsys@transformshift{3.737749in}{0.820327in}%
\pgfsys@useobject{currentmarker}{}%
\end{pgfscope}%
\begin{pgfscope}%
\pgfsys@transformshift{3.738039in}{0.833113in}%
\pgfsys@useobject{currentmarker}{}%
\end{pgfscope}%
\begin{pgfscope}%
\pgfsys@transformshift{3.738328in}{0.834381in}%
\pgfsys@useobject{currentmarker}{}%
\end{pgfscope}%
\begin{pgfscope}%
\pgfsys@transformshift{3.738617in}{0.819149in}%
\pgfsys@useobject{currentmarker}{}%
\end{pgfscope}%
\begin{pgfscope}%
\pgfsys@transformshift{3.738905in}{0.814758in}%
\pgfsys@useobject{currentmarker}{}%
\end{pgfscope}%
\begin{pgfscope}%
\pgfsys@transformshift{3.739194in}{0.816544in}%
\pgfsys@useobject{currentmarker}{}%
\end{pgfscope}%
\begin{pgfscope}%
\pgfsys@transformshift{3.739482in}{0.826955in}%
\pgfsys@useobject{currentmarker}{}%
\end{pgfscope}%
\begin{pgfscope}%
\pgfsys@transformshift{3.739770in}{0.833878in}%
\pgfsys@useobject{currentmarker}{}%
\end{pgfscope}%
\begin{pgfscope}%
\pgfsys@transformshift{3.740058in}{0.818663in}%
\pgfsys@useobject{currentmarker}{}%
\end{pgfscope}%
\begin{pgfscope}%
\pgfsys@transformshift{3.740346in}{0.815197in}%
\pgfsys@useobject{currentmarker}{}%
\end{pgfscope}%
\begin{pgfscope}%
\pgfsys@transformshift{3.740633in}{0.812408in}%
\pgfsys@useobject{currentmarker}{}%
\end{pgfscope}%
\begin{pgfscope}%
\pgfsys@transformshift{3.740921in}{0.800494in}%
\pgfsys@useobject{currentmarker}{}%
\end{pgfscope}%
\begin{pgfscope}%
\pgfsys@transformshift{3.741208in}{0.810399in}%
\pgfsys@useobject{currentmarker}{}%
\end{pgfscope}%
\begin{pgfscope}%
\pgfsys@transformshift{3.741495in}{0.812686in}%
\pgfsys@useobject{currentmarker}{}%
\end{pgfscope}%
\begin{pgfscope}%
\pgfsys@transformshift{3.741781in}{0.804620in}%
\pgfsys@useobject{currentmarker}{}%
\end{pgfscope}%
\begin{pgfscope}%
\pgfsys@transformshift{3.742068in}{0.828358in}%
\pgfsys@useobject{currentmarker}{}%
\end{pgfscope}%
\begin{pgfscope}%
\pgfsys@transformshift{3.742354in}{0.837558in}%
\pgfsys@useobject{currentmarker}{}%
\end{pgfscope}%
\begin{pgfscope}%
\pgfsys@transformshift{3.742640in}{0.799267in}%
\pgfsys@useobject{currentmarker}{}%
\end{pgfscope}%
\begin{pgfscope}%
\pgfsys@transformshift{3.742926in}{0.786159in}%
\pgfsys@useobject{currentmarker}{}%
\end{pgfscope}%
\begin{pgfscope}%
\pgfsys@transformshift{3.743211in}{0.786595in}%
\pgfsys@useobject{currentmarker}{}%
\end{pgfscope}%
\begin{pgfscope}%
\pgfsys@transformshift{3.743497in}{0.805131in}%
\pgfsys@useobject{currentmarker}{}%
\end{pgfscope}%
\begin{pgfscope}%
\pgfsys@transformshift{3.743782in}{0.780781in}%
\pgfsys@useobject{currentmarker}{}%
\end{pgfscope}%
\begin{pgfscope}%
\pgfsys@transformshift{3.744067in}{0.790452in}%
\pgfsys@useobject{currentmarker}{}%
\end{pgfscope}%
\begin{pgfscope}%
\pgfsys@transformshift{3.744352in}{0.787673in}%
\pgfsys@useobject{currentmarker}{}%
\end{pgfscope}%
\begin{pgfscope}%
\pgfsys@transformshift{3.744637in}{0.811635in}%
\pgfsys@useobject{currentmarker}{}%
\end{pgfscope}%
\begin{pgfscope}%
\pgfsys@transformshift{3.744921in}{0.800361in}%
\pgfsys@useobject{currentmarker}{}%
\end{pgfscope}%
\begin{pgfscope}%
\pgfsys@transformshift{3.745205in}{0.815793in}%
\pgfsys@useobject{currentmarker}{}%
\end{pgfscope}%
\begin{pgfscope}%
\pgfsys@transformshift{3.745489in}{0.812092in}%
\pgfsys@useobject{currentmarker}{}%
\end{pgfscope}%
\begin{pgfscope}%
\pgfsys@transformshift{3.745773in}{0.799063in}%
\pgfsys@useobject{currentmarker}{}%
\end{pgfscope}%
\begin{pgfscope}%
\pgfsys@transformshift{3.746057in}{0.803049in}%
\pgfsys@useobject{currentmarker}{}%
\end{pgfscope}%
\begin{pgfscope}%
\pgfsys@transformshift{3.746340in}{0.813250in}%
\pgfsys@useobject{currentmarker}{}%
\end{pgfscope}%
\begin{pgfscope}%
\pgfsys@transformshift{3.746624in}{0.792668in}%
\pgfsys@useobject{currentmarker}{}%
\end{pgfscope}%
\begin{pgfscope}%
\pgfsys@transformshift{3.746907in}{0.804321in}%
\pgfsys@useobject{currentmarker}{}%
\end{pgfscope}%
\begin{pgfscope}%
\pgfsys@transformshift{3.747189in}{0.817816in}%
\pgfsys@useobject{currentmarker}{}%
\end{pgfscope}%
\begin{pgfscope}%
\pgfsys@transformshift{3.747472in}{0.827703in}%
\pgfsys@useobject{currentmarker}{}%
\end{pgfscope}%
\begin{pgfscope}%
\pgfsys@transformshift{3.747755in}{0.825042in}%
\pgfsys@useobject{currentmarker}{}%
\end{pgfscope}%
\begin{pgfscope}%
\pgfsys@transformshift{3.748037in}{0.815959in}%
\pgfsys@useobject{currentmarker}{}%
\end{pgfscope}%
\begin{pgfscope}%
\pgfsys@transformshift{3.748319in}{0.801944in}%
\pgfsys@useobject{currentmarker}{}%
\end{pgfscope}%
\begin{pgfscope}%
\pgfsys@transformshift{3.748601in}{0.798569in}%
\pgfsys@useobject{currentmarker}{}%
\end{pgfscope}%
\begin{pgfscope}%
\pgfsys@transformshift{3.748882in}{0.795404in}%
\pgfsys@useobject{currentmarker}{}%
\end{pgfscope}%
\begin{pgfscope}%
\pgfsys@transformshift{3.749164in}{0.801428in}%
\pgfsys@useobject{currentmarker}{}%
\end{pgfscope}%
\begin{pgfscope}%
\pgfsys@transformshift{3.749445in}{0.812985in}%
\pgfsys@useobject{currentmarker}{}%
\end{pgfscope}%
\begin{pgfscope}%
\pgfsys@transformshift{3.749726in}{0.815925in}%
\pgfsys@useobject{currentmarker}{}%
\end{pgfscope}%
\begin{pgfscope}%
\pgfsys@transformshift{3.750007in}{0.796098in}%
\pgfsys@useobject{currentmarker}{}%
\end{pgfscope}%
\begin{pgfscope}%
\pgfsys@transformshift{3.750288in}{0.803030in}%
\pgfsys@useobject{currentmarker}{}%
\end{pgfscope}%
\begin{pgfscope}%
\pgfsys@transformshift{3.750568in}{0.813611in}%
\pgfsys@useobject{currentmarker}{}%
\end{pgfscope}%
\begin{pgfscope}%
\pgfsys@transformshift{3.750849in}{0.818515in}%
\pgfsys@useobject{currentmarker}{}%
\end{pgfscope}%
\begin{pgfscope}%
\pgfsys@transformshift{3.751129in}{0.814439in}%
\pgfsys@useobject{currentmarker}{}%
\end{pgfscope}%
\begin{pgfscope}%
\pgfsys@transformshift{3.751409in}{0.802646in}%
\pgfsys@useobject{currentmarker}{}%
\end{pgfscope}%
\begin{pgfscope}%
\pgfsys@transformshift{3.751688in}{0.801422in}%
\pgfsys@useobject{currentmarker}{}%
\end{pgfscope}%
\begin{pgfscope}%
\pgfsys@transformshift{3.751968in}{0.800769in}%
\pgfsys@useobject{currentmarker}{}%
\end{pgfscope}%
\begin{pgfscope}%
\pgfsys@transformshift{3.752247in}{0.806380in}%
\pgfsys@useobject{currentmarker}{}%
\end{pgfscope}%
\begin{pgfscope}%
\pgfsys@transformshift{3.752526in}{0.813689in}%
\pgfsys@useobject{currentmarker}{}%
\end{pgfscope}%
\begin{pgfscope}%
\pgfsys@transformshift{3.752805in}{0.794957in}%
\pgfsys@useobject{currentmarker}{}%
\end{pgfscope}%
\begin{pgfscope}%
\pgfsys@transformshift{3.753084in}{0.810583in}%
\pgfsys@useobject{currentmarker}{}%
\end{pgfscope}%
\begin{pgfscope}%
\pgfsys@transformshift{3.753362in}{0.828248in}%
\pgfsys@useobject{currentmarker}{}%
\end{pgfscope}%
\begin{pgfscope}%
\pgfsys@transformshift{3.753641in}{0.821867in}%
\pgfsys@useobject{currentmarker}{}%
\end{pgfscope}%
\begin{pgfscope}%
\pgfsys@transformshift{3.753919in}{0.810648in}%
\pgfsys@useobject{currentmarker}{}%
\end{pgfscope}%
\begin{pgfscope}%
\pgfsys@transformshift{3.754197in}{0.814510in}%
\pgfsys@useobject{currentmarker}{}%
\end{pgfscope}%
\begin{pgfscope}%
\pgfsys@transformshift{3.754475in}{0.793202in}%
\pgfsys@useobject{currentmarker}{}%
\end{pgfscope}%
\begin{pgfscope}%
\pgfsys@transformshift{3.754752in}{0.803884in}%
\pgfsys@useobject{currentmarker}{}%
\end{pgfscope}%
\begin{pgfscope}%
\pgfsys@transformshift{3.755030in}{0.800476in}%
\pgfsys@useobject{currentmarker}{}%
\end{pgfscope}%
\begin{pgfscope}%
\pgfsys@transformshift{3.755307in}{0.793983in}%
\pgfsys@useobject{currentmarker}{}%
\end{pgfscope}%
\begin{pgfscope}%
\pgfsys@transformshift{3.755584in}{0.796831in}%
\pgfsys@useobject{currentmarker}{}%
\end{pgfscope}%
\begin{pgfscope}%
\pgfsys@transformshift{3.755861in}{0.793071in}%
\pgfsys@useobject{currentmarker}{}%
\end{pgfscope}%
\begin{pgfscope}%
\pgfsys@transformshift{3.756137in}{0.795599in}%
\pgfsys@useobject{currentmarker}{}%
\end{pgfscope}%
\begin{pgfscope}%
\pgfsys@transformshift{3.756414in}{0.809042in}%
\pgfsys@useobject{currentmarker}{}%
\end{pgfscope}%
\begin{pgfscope}%
\pgfsys@transformshift{3.756690in}{0.815526in}%
\pgfsys@useobject{currentmarker}{}%
\end{pgfscope}%
\begin{pgfscope}%
\pgfsys@transformshift{3.756966in}{0.810063in}%
\pgfsys@useobject{currentmarker}{}%
\end{pgfscope}%
\begin{pgfscope}%
\pgfsys@transformshift{3.757242in}{0.818934in}%
\pgfsys@useobject{currentmarker}{}%
\end{pgfscope}%
\begin{pgfscope}%
\pgfsys@transformshift{3.757518in}{0.822216in}%
\pgfsys@useobject{currentmarker}{}%
\end{pgfscope}%
\begin{pgfscope}%
\pgfsys@transformshift{3.757793in}{0.817891in}%
\pgfsys@useobject{currentmarker}{}%
\end{pgfscope}%
\begin{pgfscope}%
\pgfsys@transformshift{3.758069in}{0.794665in}%
\pgfsys@useobject{currentmarker}{}%
\end{pgfscope}%
\begin{pgfscope}%
\pgfsys@transformshift{3.758344in}{0.797194in}%
\pgfsys@useobject{currentmarker}{}%
\end{pgfscope}%
\begin{pgfscope}%
\pgfsys@transformshift{3.758619in}{0.788786in}%
\pgfsys@useobject{currentmarker}{}%
\end{pgfscope}%
\begin{pgfscope}%
\pgfsys@transformshift{3.758894in}{0.792768in}%
\pgfsys@useobject{currentmarker}{}%
\end{pgfscope}%
\begin{pgfscope}%
\pgfsys@transformshift{3.759168in}{0.815814in}%
\pgfsys@useobject{currentmarker}{}%
\end{pgfscope}%
\begin{pgfscope}%
\pgfsys@transformshift{3.759442in}{0.826355in}%
\pgfsys@useobject{currentmarker}{}%
\end{pgfscope}%
\begin{pgfscope}%
\pgfsys@transformshift{3.759717in}{0.823250in}%
\pgfsys@useobject{currentmarker}{}%
\end{pgfscope}%
\begin{pgfscope}%
\pgfsys@transformshift{3.759991in}{0.815622in}%
\pgfsys@useobject{currentmarker}{}%
\end{pgfscope}%
\begin{pgfscope}%
\pgfsys@transformshift{3.760265in}{0.812109in}%
\pgfsys@useobject{currentmarker}{}%
\end{pgfscope}%
\begin{pgfscope}%
\pgfsys@transformshift{3.760538in}{0.810145in}%
\pgfsys@useobject{currentmarker}{}%
\end{pgfscope}%
\begin{pgfscope}%
\pgfsys@transformshift{3.760812in}{0.809753in}%
\pgfsys@useobject{currentmarker}{}%
\end{pgfscope}%
\begin{pgfscope}%
\pgfsys@transformshift{3.761085in}{0.823436in}%
\pgfsys@useobject{currentmarker}{}%
\end{pgfscope}%
\begin{pgfscope}%
\pgfsys@transformshift{3.761358in}{0.816730in}%
\pgfsys@useobject{currentmarker}{}%
\end{pgfscope}%
\begin{pgfscope}%
\pgfsys@transformshift{3.761631in}{0.818228in}%
\pgfsys@useobject{currentmarker}{}%
\end{pgfscope}%
\begin{pgfscope}%
\pgfsys@transformshift{3.761904in}{0.807779in}%
\pgfsys@useobject{currentmarker}{}%
\end{pgfscope}%
\begin{pgfscope}%
\pgfsys@transformshift{3.762176in}{0.822747in}%
\pgfsys@useobject{currentmarker}{}%
\end{pgfscope}%
\begin{pgfscope}%
\pgfsys@transformshift{3.762448in}{0.812532in}%
\pgfsys@useobject{currentmarker}{}%
\end{pgfscope}%
\begin{pgfscope}%
\pgfsys@transformshift{3.762721in}{0.806827in}%
\pgfsys@useobject{currentmarker}{}%
\end{pgfscope}%
\begin{pgfscope}%
\pgfsys@transformshift{3.762993in}{0.801553in}%
\pgfsys@useobject{currentmarker}{}%
\end{pgfscope}%
\begin{pgfscope}%
\pgfsys@transformshift{3.763264in}{0.814365in}%
\pgfsys@useobject{currentmarker}{}%
\end{pgfscope}%
\begin{pgfscope}%
\pgfsys@transformshift{3.763536in}{0.809098in}%
\pgfsys@useobject{currentmarker}{}%
\end{pgfscope}%
\begin{pgfscope}%
\pgfsys@transformshift{3.763808in}{0.806754in}%
\pgfsys@useobject{currentmarker}{}%
\end{pgfscope}%
\begin{pgfscope}%
\pgfsys@transformshift{3.764079in}{0.795202in}%
\pgfsys@useobject{currentmarker}{}%
\end{pgfscope}%
\begin{pgfscope}%
\pgfsys@transformshift{3.764350in}{0.806563in}%
\pgfsys@useobject{currentmarker}{}%
\end{pgfscope}%
\begin{pgfscope}%
\pgfsys@transformshift{3.764621in}{0.822408in}%
\pgfsys@useobject{currentmarker}{}%
\end{pgfscope}%
\begin{pgfscope}%
\pgfsys@transformshift{3.764891in}{0.816335in}%
\pgfsys@useobject{currentmarker}{}%
\end{pgfscope}%
\begin{pgfscope}%
\pgfsys@transformshift{3.765162in}{0.819516in}%
\pgfsys@useobject{currentmarker}{}%
\end{pgfscope}%
\begin{pgfscope}%
\pgfsys@transformshift{3.765432in}{0.814750in}%
\pgfsys@useobject{currentmarker}{}%
\end{pgfscope}%
\begin{pgfscope}%
\pgfsys@transformshift{3.765702in}{0.825225in}%
\pgfsys@useobject{currentmarker}{}%
\end{pgfscope}%
\begin{pgfscope}%
\pgfsys@transformshift{3.765972in}{0.830701in}%
\pgfsys@useobject{currentmarker}{}%
\end{pgfscope}%
\begin{pgfscope}%
\pgfsys@transformshift{3.766242in}{0.815175in}%
\pgfsys@useobject{currentmarker}{}%
\end{pgfscope}%
\begin{pgfscope}%
\pgfsys@transformshift{3.766512in}{0.795577in}%
\pgfsys@useobject{currentmarker}{}%
\end{pgfscope}%
\begin{pgfscope}%
\pgfsys@transformshift{3.766781in}{0.793516in}%
\pgfsys@useobject{currentmarker}{}%
\end{pgfscope}%
\begin{pgfscope}%
\pgfsys@transformshift{3.767051in}{0.807191in}%
\pgfsys@useobject{currentmarker}{}%
\end{pgfscope}%
\begin{pgfscope}%
\pgfsys@transformshift{3.767320in}{0.823167in}%
\pgfsys@useobject{currentmarker}{}%
\end{pgfscope}%
\begin{pgfscope}%
\pgfsys@transformshift{3.767589in}{0.843044in}%
\pgfsys@useobject{currentmarker}{}%
\end{pgfscope}%
\begin{pgfscope}%
\pgfsys@transformshift{3.767857in}{0.816877in}%
\pgfsys@useobject{currentmarker}{}%
\end{pgfscope}%
\begin{pgfscope}%
\pgfsys@transformshift{3.768126in}{0.794462in}%
\pgfsys@useobject{currentmarker}{}%
\end{pgfscope}%
\begin{pgfscope}%
\pgfsys@transformshift{3.768394in}{0.810152in}%
\pgfsys@useobject{currentmarker}{}%
\end{pgfscope}%
\begin{pgfscope}%
\pgfsys@transformshift{3.768662in}{0.800077in}%
\pgfsys@useobject{currentmarker}{}%
\end{pgfscope}%
\begin{pgfscope}%
\pgfsys@transformshift{3.768930in}{0.791989in}%
\pgfsys@useobject{currentmarker}{}%
\end{pgfscope}%
\begin{pgfscope}%
\pgfsys@transformshift{3.769198in}{0.829541in}%
\pgfsys@useobject{currentmarker}{}%
\end{pgfscope}%
\begin{pgfscope}%
\pgfsys@transformshift{3.769466in}{0.819834in}%
\pgfsys@useobject{currentmarker}{}%
\end{pgfscope}%
\begin{pgfscope}%
\pgfsys@transformshift{3.769733in}{0.790169in}%
\pgfsys@useobject{currentmarker}{}%
\end{pgfscope}%
\begin{pgfscope}%
\pgfsys@transformshift{3.770001in}{0.798896in}%
\pgfsys@useobject{currentmarker}{}%
\end{pgfscope}%
\begin{pgfscope}%
\pgfsys@transformshift{3.770268in}{0.802056in}%
\pgfsys@useobject{currentmarker}{}%
\end{pgfscope}%
\begin{pgfscope}%
\pgfsys@transformshift{3.770535in}{0.796120in}%
\pgfsys@useobject{currentmarker}{}%
\end{pgfscope}%
\begin{pgfscope}%
\pgfsys@transformshift{3.770802in}{0.800195in}%
\pgfsys@useobject{currentmarker}{}%
\end{pgfscope}%
\begin{pgfscope}%
\pgfsys@transformshift{3.771068in}{0.812072in}%
\pgfsys@useobject{currentmarker}{}%
\end{pgfscope}%
\begin{pgfscope}%
\pgfsys@transformshift{3.771335in}{0.822587in}%
\pgfsys@useobject{currentmarker}{}%
\end{pgfscope}%
\begin{pgfscope}%
\pgfsys@transformshift{3.771601in}{0.825084in}%
\pgfsys@useobject{currentmarker}{}%
\end{pgfscope}%
\begin{pgfscope}%
\pgfsys@transformshift{3.771867in}{0.832475in}%
\pgfsys@useobject{currentmarker}{}%
\end{pgfscope}%
\begin{pgfscope}%
\pgfsys@transformshift{3.772133in}{0.808414in}%
\pgfsys@useobject{currentmarker}{}%
\end{pgfscope}%
\begin{pgfscope}%
\pgfsys@transformshift{3.772399in}{0.800475in}%
\pgfsys@useobject{currentmarker}{}%
\end{pgfscope}%
\begin{pgfscope}%
\pgfsys@transformshift{3.772664in}{0.814076in}%
\pgfsys@useobject{currentmarker}{}%
\end{pgfscope}%
\begin{pgfscope}%
\pgfsys@transformshift{3.772929in}{0.803814in}%
\pgfsys@useobject{currentmarker}{}%
\end{pgfscope}%
\begin{pgfscope}%
\pgfsys@transformshift{3.773195in}{0.811697in}%
\pgfsys@useobject{currentmarker}{}%
\end{pgfscope}%
\begin{pgfscope}%
\pgfsys@transformshift{3.773460in}{0.816412in}%
\pgfsys@useobject{currentmarker}{}%
\end{pgfscope}%
\begin{pgfscope}%
\pgfsys@transformshift{3.773724in}{0.818500in}%
\pgfsys@useobject{currentmarker}{}%
\end{pgfscope}%
\begin{pgfscope}%
\pgfsys@transformshift{3.773989in}{0.800176in}%
\pgfsys@useobject{currentmarker}{}%
\end{pgfscope}%
\begin{pgfscope}%
\pgfsys@transformshift{3.774254in}{0.805661in}%
\pgfsys@useobject{currentmarker}{}%
\end{pgfscope}%
\begin{pgfscope}%
\pgfsys@transformshift{3.774518in}{0.829421in}%
\pgfsys@useobject{currentmarker}{}%
\end{pgfscope}%
\begin{pgfscope}%
\pgfsys@transformshift{3.774782in}{0.802095in}%
\pgfsys@useobject{currentmarker}{}%
\end{pgfscope}%
\begin{pgfscope}%
\pgfsys@transformshift{3.775046in}{0.796293in}%
\pgfsys@useobject{currentmarker}{}%
\end{pgfscope}%
\begin{pgfscope}%
\pgfsys@transformshift{3.775310in}{0.801596in}%
\pgfsys@useobject{currentmarker}{}%
\end{pgfscope}%
\begin{pgfscope}%
\pgfsys@transformshift{3.775574in}{0.806512in}%
\pgfsys@useobject{currentmarker}{}%
\end{pgfscope}%
\begin{pgfscope}%
\pgfsys@transformshift{3.775837in}{0.821926in}%
\pgfsys@useobject{currentmarker}{}%
\end{pgfscope}%
\begin{pgfscope}%
\pgfsys@transformshift{3.776100in}{0.840062in}%
\pgfsys@useobject{currentmarker}{}%
\end{pgfscope}%
\begin{pgfscope}%
\pgfsys@transformshift{3.776363in}{0.838099in}%
\pgfsys@useobject{currentmarker}{}%
\end{pgfscope}%
\begin{pgfscope}%
\pgfsys@transformshift{3.776626in}{0.824297in}%
\pgfsys@useobject{currentmarker}{}%
\end{pgfscope}%
\begin{pgfscope}%
\pgfsys@transformshift{3.776889in}{0.802431in}%
\pgfsys@useobject{currentmarker}{}%
\end{pgfscope}%
\begin{pgfscope}%
\pgfsys@transformshift{3.777152in}{0.806074in}%
\pgfsys@useobject{currentmarker}{}%
\end{pgfscope}%
\begin{pgfscope}%
\pgfsys@transformshift{3.777414in}{0.813010in}%
\pgfsys@useobject{currentmarker}{}%
\end{pgfscope}%
\begin{pgfscope}%
\pgfsys@transformshift{3.777677in}{0.814864in}%
\pgfsys@useobject{currentmarker}{}%
\end{pgfscope}%
\begin{pgfscope}%
\pgfsys@transformshift{3.777939in}{0.823830in}%
\pgfsys@useobject{currentmarker}{}%
\end{pgfscope}%
\begin{pgfscope}%
\pgfsys@transformshift{3.778201in}{0.816929in}%
\pgfsys@useobject{currentmarker}{}%
\end{pgfscope}%
\begin{pgfscope}%
\pgfsys@transformshift{3.778462in}{0.817474in}%
\pgfsys@useobject{currentmarker}{}%
\end{pgfscope}%
\begin{pgfscope}%
\pgfsys@transformshift{3.778724in}{0.810567in}%
\pgfsys@useobject{currentmarker}{}%
\end{pgfscope}%
\begin{pgfscope}%
\pgfsys@transformshift{3.778985in}{0.804068in}%
\pgfsys@useobject{currentmarker}{}%
\end{pgfscope}%
\begin{pgfscope}%
\pgfsys@transformshift{3.779247in}{0.808334in}%
\pgfsys@useobject{currentmarker}{}%
\end{pgfscope}%
\begin{pgfscope}%
\pgfsys@transformshift{3.779508in}{0.813864in}%
\pgfsys@useobject{currentmarker}{}%
\end{pgfscope}%
\begin{pgfscope}%
\pgfsys@transformshift{3.779769in}{0.809772in}%
\pgfsys@useobject{currentmarker}{}%
\end{pgfscope}%
\begin{pgfscope}%
\pgfsys@transformshift{3.780029in}{0.804997in}%
\pgfsys@useobject{currentmarker}{}%
\end{pgfscope}%
\begin{pgfscope}%
\pgfsys@transformshift{3.780290in}{0.819499in}%
\pgfsys@useobject{currentmarker}{}%
\end{pgfscope}%
\begin{pgfscope}%
\pgfsys@transformshift{3.780550in}{0.814277in}%
\pgfsys@useobject{currentmarker}{}%
\end{pgfscope}%
\begin{pgfscope}%
\pgfsys@transformshift{3.780811in}{0.802736in}%
\pgfsys@useobject{currentmarker}{}%
\end{pgfscope}%
\begin{pgfscope}%
\pgfsys@transformshift{3.781071in}{0.799485in}%
\pgfsys@useobject{currentmarker}{}%
\end{pgfscope}%
\begin{pgfscope}%
\pgfsys@transformshift{3.781330in}{0.808766in}%
\pgfsys@useobject{currentmarker}{}%
\end{pgfscope}%
\begin{pgfscope}%
\pgfsys@transformshift{3.781590in}{0.800079in}%
\pgfsys@useobject{currentmarker}{}%
\end{pgfscope}%
\begin{pgfscope}%
\pgfsys@transformshift{3.781850in}{0.804253in}%
\pgfsys@useobject{currentmarker}{}%
\end{pgfscope}%
\begin{pgfscope}%
\pgfsys@transformshift{3.782109in}{0.807086in}%
\pgfsys@useobject{currentmarker}{}%
\end{pgfscope}%
\begin{pgfscope}%
\pgfsys@transformshift{3.782368in}{0.805232in}%
\pgfsys@useobject{currentmarker}{}%
\end{pgfscope}%
\begin{pgfscope}%
\pgfsys@transformshift{3.782628in}{0.789128in}%
\pgfsys@useobject{currentmarker}{}%
\end{pgfscope}%
\begin{pgfscope}%
\pgfsys@transformshift{3.782886in}{0.814384in}%
\pgfsys@useobject{currentmarker}{}%
\end{pgfscope}%
\begin{pgfscope}%
\pgfsys@transformshift{3.783145in}{0.808884in}%
\pgfsys@useobject{currentmarker}{}%
\end{pgfscope}%
\begin{pgfscope}%
\pgfsys@transformshift{3.783404in}{0.791242in}%
\pgfsys@useobject{currentmarker}{}%
\end{pgfscope}%
\begin{pgfscope}%
\pgfsys@transformshift{3.783662in}{0.805871in}%
\pgfsys@useobject{currentmarker}{}%
\end{pgfscope}%
\begin{pgfscope}%
\pgfsys@transformshift{3.783920in}{0.793128in}%
\pgfsys@useobject{currentmarker}{}%
\end{pgfscope}%
\begin{pgfscope}%
\pgfsys@transformshift{3.784178in}{0.795409in}%
\pgfsys@useobject{currentmarker}{}%
\end{pgfscope}%
\begin{pgfscope}%
\pgfsys@transformshift{3.784436in}{0.804249in}%
\pgfsys@useobject{currentmarker}{}%
\end{pgfscope}%
\begin{pgfscope}%
\pgfsys@transformshift{3.784694in}{0.792557in}%
\pgfsys@useobject{currentmarker}{}%
\end{pgfscope}%
\begin{pgfscope}%
\pgfsys@transformshift{3.784952in}{0.800658in}%
\pgfsys@useobject{currentmarker}{}%
\end{pgfscope}%
\begin{pgfscope}%
\pgfsys@transformshift{3.785209in}{0.806729in}%
\pgfsys@useobject{currentmarker}{}%
\end{pgfscope}%
\begin{pgfscope}%
\pgfsys@transformshift{3.785466in}{0.815419in}%
\pgfsys@useobject{currentmarker}{}%
\end{pgfscope}%
\begin{pgfscope}%
\pgfsys@transformshift{3.785723in}{0.813168in}%
\pgfsys@useobject{currentmarker}{}%
\end{pgfscope}%
\begin{pgfscope}%
\pgfsys@transformshift{3.785980in}{0.811012in}%
\pgfsys@useobject{currentmarker}{}%
\end{pgfscope}%
\begin{pgfscope}%
\pgfsys@transformshift{3.786237in}{0.807131in}%
\pgfsys@useobject{currentmarker}{}%
\end{pgfscope}%
\begin{pgfscope}%
\pgfsys@transformshift{3.786494in}{0.810813in}%
\pgfsys@useobject{currentmarker}{}%
\end{pgfscope}%
\begin{pgfscope}%
\pgfsys@transformshift{3.786750in}{0.816786in}%
\pgfsys@useobject{currentmarker}{}%
\end{pgfscope}%
\begin{pgfscope}%
\pgfsys@transformshift{3.787006in}{0.816043in}%
\pgfsys@useobject{currentmarker}{}%
\end{pgfscope}%
\begin{pgfscope}%
\pgfsys@transformshift{3.787263in}{0.816465in}%
\pgfsys@useobject{currentmarker}{}%
\end{pgfscope}%
\begin{pgfscope}%
\pgfsys@transformshift{3.787519in}{0.826320in}%
\pgfsys@useobject{currentmarker}{}%
\end{pgfscope}%
\begin{pgfscope}%
\pgfsys@transformshift{3.787774in}{0.812825in}%
\pgfsys@useobject{currentmarker}{}%
\end{pgfscope}%
\begin{pgfscope}%
\pgfsys@transformshift{3.788030in}{0.805313in}%
\pgfsys@useobject{currentmarker}{}%
\end{pgfscope}%
\begin{pgfscope}%
\pgfsys@transformshift{3.788285in}{0.804929in}%
\pgfsys@useobject{currentmarker}{}%
\end{pgfscope}%
\begin{pgfscope}%
\pgfsys@transformshift{3.788541in}{0.801138in}%
\pgfsys@useobject{currentmarker}{}%
\end{pgfscope}%
\begin{pgfscope}%
\pgfsys@transformshift{3.788796in}{0.816587in}%
\pgfsys@useobject{currentmarker}{}%
\end{pgfscope}%
\begin{pgfscope}%
\pgfsys@transformshift{3.789051in}{0.817397in}%
\pgfsys@useobject{currentmarker}{}%
\end{pgfscope}%
\begin{pgfscope}%
\pgfsys@transformshift{3.789306in}{0.808312in}%
\pgfsys@useobject{currentmarker}{}%
\end{pgfscope}%
\begin{pgfscope}%
\pgfsys@transformshift{3.789560in}{0.813938in}%
\pgfsys@useobject{currentmarker}{}%
\end{pgfscope}%
\begin{pgfscope}%
\pgfsys@transformshift{3.789815in}{0.813736in}%
\pgfsys@useobject{currentmarker}{}%
\end{pgfscope}%
\begin{pgfscope}%
\pgfsys@transformshift{3.790069in}{0.792978in}%
\pgfsys@useobject{currentmarker}{}%
\end{pgfscope}%
\begin{pgfscope}%
\pgfsys@transformshift{3.790323in}{0.776298in}%
\pgfsys@useobject{currentmarker}{}%
\end{pgfscope}%
\begin{pgfscope}%
\pgfsys@transformshift{3.790577in}{0.782614in}%
\pgfsys@useobject{currentmarker}{}%
\end{pgfscope}%
\begin{pgfscope}%
\pgfsys@transformshift{3.790831in}{0.794069in}%
\pgfsys@useobject{currentmarker}{}%
\end{pgfscope}%
\begin{pgfscope}%
\pgfsys@transformshift{3.791085in}{0.819234in}%
\pgfsys@useobject{currentmarker}{}%
\end{pgfscope}%
\begin{pgfscope}%
\pgfsys@transformshift{3.791338in}{0.816001in}%
\pgfsys@useobject{currentmarker}{}%
\end{pgfscope}%
\begin{pgfscope}%
\pgfsys@transformshift{3.791592in}{0.812334in}%
\pgfsys@useobject{currentmarker}{}%
\end{pgfscope}%
\begin{pgfscope}%
\pgfsys@transformshift{3.791845in}{0.811122in}%
\pgfsys@useobject{currentmarker}{}%
\end{pgfscope}%
\begin{pgfscope}%
\pgfsys@transformshift{3.792098in}{0.802984in}%
\pgfsys@useobject{currentmarker}{}%
\end{pgfscope}%
\begin{pgfscope}%
\pgfsys@transformshift{3.792351in}{0.812001in}%
\pgfsys@useobject{currentmarker}{}%
\end{pgfscope}%
\begin{pgfscope}%
\pgfsys@transformshift{3.792604in}{0.806849in}%
\pgfsys@useobject{currentmarker}{}%
\end{pgfscope}%
\begin{pgfscope}%
\pgfsys@transformshift{3.792856in}{0.802143in}%
\pgfsys@useobject{currentmarker}{}%
\end{pgfscope}%
\begin{pgfscope}%
\pgfsys@transformshift{3.793109in}{0.811189in}%
\pgfsys@useobject{currentmarker}{}%
\end{pgfscope}%
\begin{pgfscope}%
\pgfsys@transformshift{3.793361in}{0.799982in}%
\pgfsys@useobject{currentmarker}{}%
\end{pgfscope}%
\begin{pgfscope}%
\pgfsys@transformshift{3.793613in}{0.804537in}%
\pgfsys@useobject{currentmarker}{}%
\end{pgfscope}%
\begin{pgfscope}%
\pgfsys@transformshift{3.793865in}{0.815635in}%
\pgfsys@useobject{currentmarker}{}%
\end{pgfscope}%
\begin{pgfscope}%
\pgfsys@transformshift{3.794117in}{0.818944in}%
\pgfsys@useobject{currentmarker}{}%
\end{pgfscope}%
\begin{pgfscope}%
\pgfsys@transformshift{3.794369in}{0.818203in}%
\pgfsys@useobject{currentmarker}{}%
\end{pgfscope}%
\begin{pgfscope}%
\pgfsys@transformshift{3.794620in}{0.819833in}%
\pgfsys@useobject{currentmarker}{}%
\end{pgfscope}%
\begin{pgfscope}%
\pgfsys@transformshift{3.794871in}{0.821651in}%
\pgfsys@useobject{currentmarker}{}%
\end{pgfscope}%
\begin{pgfscope}%
\pgfsys@transformshift{3.795123in}{0.801913in}%
\pgfsys@useobject{currentmarker}{}%
\end{pgfscope}%
\begin{pgfscope}%
\pgfsys@transformshift{3.795374in}{0.807585in}%
\pgfsys@useobject{currentmarker}{}%
\end{pgfscope}%
\begin{pgfscope}%
\pgfsys@transformshift{3.795624in}{0.809632in}%
\pgfsys@useobject{currentmarker}{}%
\end{pgfscope}%
\begin{pgfscope}%
\pgfsys@transformshift{3.795875in}{0.829172in}%
\pgfsys@useobject{currentmarker}{}%
\end{pgfscope}%
\begin{pgfscope}%
\pgfsys@transformshift{3.796126in}{0.821244in}%
\pgfsys@useobject{currentmarker}{}%
\end{pgfscope}%
\begin{pgfscope}%
\pgfsys@transformshift{3.796376in}{0.796412in}%
\pgfsys@useobject{currentmarker}{}%
\end{pgfscope}%
\begin{pgfscope}%
\pgfsys@transformshift{3.796626in}{0.822805in}%
\pgfsys@useobject{currentmarker}{}%
\end{pgfscope}%
\begin{pgfscope}%
\pgfsys@transformshift{3.796876in}{0.822178in}%
\pgfsys@useobject{currentmarker}{}%
\end{pgfscope}%
\begin{pgfscope}%
\pgfsys@transformshift{3.797126in}{0.820190in}%
\pgfsys@useobject{currentmarker}{}%
\end{pgfscope}%
\begin{pgfscope}%
\pgfsys@transformshift{3.797376in}{0.816816in}%
\pgfsys@useobject{currentmarker}{}%
\end{pgfscope}%
\begin{pgfscope}%
\pgfsys@transformshift{3.797626in}{0.810313in}%
\pgfsys@useobject{currentmarker}{}%
\end{pgfscope}%
\begin{pgfscope}%
\pgfsys@transformshift{3.797875in}{0.804484in}%
\pgfsys@useobject{currentmarker}{}%
\end{pgfscope}%
\begin{pgfscope}%
\pgfsys@transformshift{3.798125in}{0.812012in}%
\pgfsys@useobject{currentmarker}{}%
\end{pgfscope}%
\begin{pgfscope}%
\pgfsys@transformshift{3.798374in}{0.831037in}%
\pgfsys@useobject{currentmarker}{}%
\end{pgfscope}%
\begin{pgfscope}%
\pgfsys@transformshift{3.798623in}{0.822908in}%
\pgfsys@useobject{currentmarker}{}%
\end{pgfscope}%
\begin{pgfscope}%
\pgfsys@transformshift{3.798872in}{0.790520in}%
\pgfsys@useobject{currentmarker}{}%
\end{pgfscope}%
\begin{pgfscope}%
\pgfsys@transformshift{3.799120in}{0.786868in}%
\pgfsys@useobject{currentmarker}{}%
\end{pgfscope}%
\begin{pgfscope}%
\pgfsys@transformshift{3.799369in}{0.804295in}%
\pgfsys@useobject{currentmarker}{}%
\end{pgfscope}%
\begin{pgfscope}%
\pgfsys@transformshift{3.799617in}{0.811186in}%
\pgfsys@useobject{currentmarker}{}%
\end{pgfscope}%
\begin{pgfscope}%
\pgfsys@transformshift{3.799865in}{0.808544in}%
\pgfsys@useobject{currentmarker}{}%
\end{pgfscope}%
\begin{pgfscope}%
\pgfsys@transformshift{3.800113in}{0.808968in}%
\pgfsys@useobject{currentmarker}{}%
\end{pgfscope}%
\begin{pgfscope}%
\pgfsys@transformshift{3.800361in}{0.809900in}%
\pgfsys@useobject{currentmarker}{}%
\end{pgfscope}%
\begin{pgfscope}%
\pgfsys@transformshift{3.800609in}{0.795093in}%
\pgfsys@useobject{currentmarker}{}%
\end{pgfscope}%
\begin{pgfscope}%
\pgfsys@transformshift{3.800857in}{0.797241in}%
\pgfsys@useobject{currentmarker}{}%
\end{pgfscope}%
\begin{pgfscope}%
\pgfsys@transformshift{3.801104in}{0.811258in}%
\pgfsys@useobject{currentmarker}{}%
\end{pgfscope}%
\begin{pgfscope}%
\pgfsys@transformshift{3.801352in}{0.804513in}%
\pgfsys@useobject{currentmarker}{}%
\end{pgfscope}%
\begin{pgfscope}%
\pgfsys@transformshift{3.801599in}{0.807452in}%
\pgfsys@useobject{currentmarker}{}%
\end{pgfscope}%
\begin{pgfscope}%
\pgfsys@transformshift{3.801846in}{0.800425in}%
\pgfsys@useobject{currentmarker}{}%
\end{pgfscope}%
\begin{pgfscope}%
\pgfsys@transformshift{3.802093in}{0.799495in}%
\pgfsys@useobject{currentmarker}{}%
\end{pgfscope}%
\begin{pgfscope}%
\pgfsys@transformshift{3.802339in}{0.801246in}%
\pgfsys@useobject{currentmarker}{}%
\end{pgfscope}%
\begin{pgfscope}%
\pgfsys@transformshift{3.802586in}{0.800050in}%
\pgfsys@useobject{currentmarker}{}%
\end{pgfscope}%
\begin{pgfscope}%
\pgfsys@transformshift{3.802832in}{0.807252in}%
\pgfsys@useobject{currentmarker}{}%
\end{pgfscope}%
\begin{pgfscope}%
\pgfsys@transformshift{3.803079in}{0.802497in}%
\pgfsys@useobject{currentmarker}{}%
\end{pgfscope}%
\begin{pgfscope}%
\pgfsys@transformshift{3.803325in}{0.805585in}%
\pgfsys@useobject{currentmarker}{}%
\end{pgfscope}%
\begin{pgfscope}%
\pgfsys@transformshift{3.803571in}{0.793270in}%
\pgfsys@useobject{currentmarker}{}%
\end{pgfscope}%
\begin{pgfscope}%
\pgfsys@transformshift{3.803817in}{0.792221in}%
\pgfsys@useobject{currentmarker}{}%
\end{pgfscope}%
\begin{pgfscope}%
\pgfsys@transformshift{3.804062in}{0.804934in}%
\pgfsys@useobject{currentmarker}{}%
\end{pgfscope}%
\begin{pgfscope}%
\pgfsys@transformshift{3.804308in}{0.806764in}%
\pgfsys@useobject{currentmarker}{}%
\end{pgfscope}%
\begin{pgfscope}%
\pgfsys@transformshift{3.804553in}{0.789354in}%
\pgfsys@useobject{currentmarker}{}%
\end{pgfscope}%
\begin{pgfscope}%
\pgfsys@transformshift{3.804798in}{0.801327in}%
\pgfsys@useobject{currentmarker}{}%
\end{pgfscope}%
\begin{pgfscope}%
\pgfsys@transformshift{3.805043in}{0.795934in}%
\pgfsys@useobject{currentmarker}{}%
\end{pgfscope}%
\begin{pgfscope}%
\pgfsys@transformshift{3.805288in}{0.798724in}%
\pgfsys@useobject{currentmarker}{}%
\end{pgfscope}%
\begin{pgfscope}%
\pgfsys@transformshift{3.805533in}{0.795837in}%
\pgfsys@useobject{currentmarker}{}%
\end{pgfscope}%
\begin{pgfscope}%
\pgfsys@transformshift{3.805778in}{0.800970in}%
\pgfsys@useobject{currentmarker}{}%
\end{pgfscope}%
\begin{pgfscope}%
\pgfsys@transformshift{3.806022in}{0.803735in}%
\pgfsys@useobject{currentmarker}{}%
\end{pgfscope}%
\begin{pgfscope}%
\pgfsys@transformshift{3.806267in}{0.812411in}%
\pgfsys@useobject{currentmarker}{}%
\end{pgfscope}%
\begin{pgfscope}%
\pgfsys@transformshift{3.806511in}{0.811009in}%
\pgfsys@useobject{currentmarker}{}%
\end{pgfscope}%
\begin{pgfscope}%
\pgfsys@transformshift{3.806755in}{0.798421in}%
\pgfsys@useobject{currentmarker}{}%
\end{pgfscope}%
\begin{pgfscope}%
\pgfsys@transformshift{3.806999in}{0.798449in}%
\pgfsys@useobject{currentmarker}{}%
\end{pgfscope}%
\begin{pgfscope}%
\pgfsys@transformshift{3.807243in}{0.826638in}%
\pgfsys@useobject{currentmarker}{}%
\end{pgfscope}%
\begin{pgfscope}%
\pgfsys@transformshift{3.807486in}{0.810272in}%
\pgfsys@useobject{currentmarker}{}%
\end{pgfscope}%
\begin{pgfscope}%
\pgfsys@transformshift{3.807730in}{0.784938in}%
\pgfsys@useobject{currentmarker}{}%
\end{pgfscope}%
\begin{pgfscope}%
\pgfsys@transformshift{3.807973in}{0.798354in}%
\pgfsys@useobject{currentmarker}{}%
\end{pgfscope}%
\begin{pgfscope}%
\pgfsys@transformshift{3.808216in}{0.796859in}%
\pgfsys@useobject{currentmarker}{}%
\end{pgfscope}%
\begin{pgfscope}%
\pgfsys@transformshift{3.808459in}{0.778313in}%
\pgfsys@useobject{currentmarker}{}%
\end{pgfscope}%
\begin{pgfscope}%
\pgfsys@transformshift{3.808702in}{0.775062in}%
\pgfsys@useobject{currentmarker}{}%
\end{pgfscope}%
\begin{pgfscope}%
\pgfsys@transformshift{3.808945in}{0.786001in}%
\pgfsys@useobject{currentmarker}{}%
\end{pgfscope}%
\begin{pgfscope}%
\pgfsys@transformshift{3.809187in}{0.817119in}%
\pgfsys@useobject{currentmarker}{}%
\end{pgfscope}%
\begin{pgfscope}%
\pgfsys@transformshift{3.809430in}{0.821163in}%
\pgfsys@useobject{currentmarker}{}%
\end{pgfscope}%
\begin{pgfscope}%
\pgfsys@transformshift{3.809672in}{0.807139in}%
\pgfsys@useobject{currentmarker}{}%
\end{pgfscope}%
\begin{pgfscope}%
\pgfsys@transformshift{3.809914in}{0.797018in}%
\pgfsys@useobject{currentmarker}{}%
\end{pgfscope}%
\begin{pgfscope}%
\pgfsys@transformshift{3.810156in}{0.788179in}%
\pgfsys@useobject{currentmarker}{}%
\end{pgfscope}%
\begin{pgfscope}%
\pgfsys@transformshift{3.810398in}{0.804423in}%
\pgfsys@useobject{currentmarker}{}%
\end{pgfscope}%
\begin{pgfscope}%
\pgfsys@transformshift{3.810640in}{0.808418in}%
\pgfsys@useobject{currentmarker}{}%
\end{pgfscope}%
\begin{pgfscope}%
\pgfsys@transformshift{3.810881in}{0.797702in}%
\pgfsys@useobject{currentmarker}{}%
\end{pgfscope}%
\begin{pgfscope}%
\pgfsys@transformshift{3.811123in}{0.795773in}%
\pgfsys@useobject{currentmarker}{}%
\end{pgfscope}%
\begin{pgfscope}%
\pgfsys@transformshift{3.811364in}{0.793633in}%
\pgfsys@useobject{currentmarker}{}%
\end{pgfscope}%
\begin{pgfscope}%
\pgfsys@transformshift{3.811605in}{0.790999in}%
\pgfsys@useobject{currentmarker}{}%
\end{pgfscope}%
\begin{pgfscope}%
\pgfsys@transformshift{3.811846in}{0.801404in}%
\pgfsys@useobject{currentmarker}{}%
\end{pgfscope}%
\begin{pgfscope}%
\pgfsys@transformshift{3.812087in}{0.804216in}%
\pgfsys@useobject{currentmarker}{}%
\end{pgfscope}%
\begin{pgfscope}%
\pgfsys@transformshift{3.812327in}{0.799862in}%
\pgfsys@useobject{currentmarker}{}%
\end{pgfscope}%
\begin{pgfscope}%
\pgfsys@transformshift{3.812568in}{0.812665in}%
\pgfsys@useobject{currentmarker}{}%
\end{pgfscope}%
\begin{pgfscope}%
\pgfsys@transformshift{3.812808in}{0.787259in}%
\pgfsys@useobject{currentmarker}{}%
\end{pgfscope}%
\begin{pgfscope}%
\pgfsys@transformshift{3.813049in}{0.781526in}%
\pgfsys@useobject{currentmarker}{}%
\end{pgfscope}%
\begin{pgfscope}%
\pgfsys@transformshift{3.813289in}{0.805554in}%
\pgfsys@useobject{currentmarker}{}%
\end{pgfscope}%
\begin{pgfscope}%
\pgfsys@transformshift{3.813529in}{0.801028in}%
\pgfsys@useobject{currentmarker}{}%
\end{pgfscope}%
\begin{pgfscope}%
\pgfsys@transformshift{3.813769in}{0.793691in}%
\pgfsys@useobject{currentmarker}{}%
\end{pgfscope}%
\begin{pgfscope}%
\pgfsys@transformshift{3.814008in}{0.804914in}%
\pgfsys@useobject{currentmarker}{}%
\end{pgfscope}%
\begin{pgfscope}%
\pgfsys@transformshift{3.814248in}{0.822681in}%
\pgfsys@useobject{currentmarker}{}%
\end{pgfscope}%
\begin{pgfscope}%
\pgfsys@transformshift{3.814487in}{0.816275in}%
\pgfsys@useobject{currentmarker}{}%
\end{pgfscope}%
\begin{pgfscope}%
\pgfsys@transformshift{3.814726in}{0.800951in}%
\pgfsys@useobject{currentmarker}{}%
\end{pgfscope}%
\begin{pgfscope}%
\pgfsys@transformshift{3.814966in}{0.799784in}%
\pgfsys@useobject{currentmarker}{}%
\end{pgfscope}%
\begin{pgfscope}%
\pgfsys@transformshift{3.815205in}{0.807254in}%
\pgfsys@useobject{currentmarker}{}%
\end{pgfscope}%
\begin{pgfscope}%
\pgfsys@transformshift{3.815443in}{0.807271in}%
\pgfsys@useobject{currentmarker}{}%
\end{pgfscope}%
\begin{pgfscope}%
\pgfsys@transformshift{3.815682in}{0.810115in}%
\pgfsys@useobject{currentmarker}{}%
\end{pgfscope}%
\begin{pgfscope}%
\pgfsys@transformshift{3.815921in}{0.784948in}%
\pgfsys@useobject{currentmarker}{}%
\end{pgfscope}%
\begin{pgfscope}%
\pgfsys@transformshift{3.816159in}{0.799391in}%
\pgfsys@useobject{currentmarker}{}%
\end{pgfscope}%
\begin{pgfscope}%
\pgfsys@transformshift{3.816397in}{0.812365in}%
\pgfsys@useobject{currentmarker}{}%
\end{pgfscope}%
\begin{pgfscope}%
\pgfsys@transformshift{3.816636in}{0.813853in}%
\pgfsys@useobject{currentmarker}{}%
\end{pgfscope}%
\begin{pgfscope}%
\pgfsys@transformshift{3.816874in}{0.804799in}%
\pgfsys@useobject{currentmarker}{}%
\end{pgfscope}%
\begin{pgfscope}%
\pgfsys@transformshift{3.817111in}{0.813860in}%
\pgfsys@useobject{currentmarker}{}%
\end{pgfscope}%
\begin{pgfscope}%
\pgfsys@transformshift{3.817349in}{0.801215in}%
\pgfsys@useobject{currentmarker}{}%
\end{pgfscope}%
\begin{pgfscope}%
\pgfsys@transformshift{3.817587in}{0.811523in}%
\pgfsys@useobject{currentmarker}{}%
\end{pgfscope}%
\begin{pgfscope}%
\pgfsys@transformshift{3.817824in}{0.824746in}%
\pgfsys@useobject{currentmarker}{}%
\end{pgfscope}%
\begin{pgfscope}%
\pgfsys@transformshift{3.818061in}{0.805996in}%
\pgfsys@useobject{currentmarker}{}%
\end{pgfscope}%
\begin{pgfscope}%
\pgfsys@transformshift{3.818299in}{0.791977in}%
\pgfsys@useobject{currentmarker}{}%
\end{pgfscope}%
\begin{pgfscope}%
\pgfsys@transformshift{3.818536in}{0.795567in}%
\pgfsys@useobject{currentmarker}{}%
\end{pgfscope}%
\begin{pgfscope}%
\pgfsys@transformshift{3.818772in}{0.806491in}%
\pgfsys@useobject{currentmarker}{}%
\end{pgfscope}%
\begin{pgfscope}%
\pgfsys@transformshift{3.819009in}{0.797446in}%
\pgfsys@useobject{currentmarker}{}%
\end{pgfscope}%
\begin{pgfscope}%
\pgfsys@transformshift{3.819246in}{0.803651in}%
\pgfsys@useobject{currentmarker}{}%
\end{pgfscope}%
\begin{pgfscope}%
\pgfsys@transformshift{3.819482in}{0.797521in}%
\pgfsys@useobject{currentmarker}{}%
\end{pgfscope}%
\begin{pgfscope}%
\pgfsys@transformshift{3.819719in}{0.796584in}%
\pgfsys@useobject{currentmarker}{}%
\end{pgfscope}%
\begin{pgfscope}%
\pgfsys@transformshift{3.819955in}{0.799160in}%
\pgfsys@useobject{currentmarker}{}%
\end{pgfscope}%
\begin{pgfscope}%
\pgfsys@transformshift{3.820191in}{0.798893in}%
\pgfsys@useobject{currentmarker}{}%
\end{pgfscope}%
\begin{pgfscope}%
\pgfsys@transformshift{3.820427in}{0.782848in}%
\pgfsys@useobject{currentmarker}{}%
\end{pgfscope}%
\begin{pgfscope}%
\pgfsys@transformshift{3.820663in}{0.786754in}%
\pgfsys@useobject{currentmarker}{}%
\end{pgfscope}%
\begin{pgfscope}%
\pgfsys@transformshift{3.820898in}{0.792492in}%
\pgfsys@useobject{currentmarker}{}%
\end{pgfscope}%
\begin{pgfscope}%
\pgfsys@transformshift{3.821134in}{0.800865in}%
\pgfsys@useobject{currentmarker}{}%
\end{pgfscope}%
\begin{pgfscope}%
\pgfsys@transformshift{3.821369in}{0.801090in}%
\pgfsys@useobject{currentmarker}{}%
\end{pgfscope}%
\begin{pgfscope}%
\pgfsys@transformshift{3.821604in}{0.810812in}%
\pgfsys@useobject{currentmarker}{}%
\end{pgfscope}%
\begin{pgfscope}%
\pgfsys@transformshift{3.821839in}{0.808678in}%
\pgfsys@useobject{currentmarker}{}%
\end{pgfscope}%
\begin{pgfscope}%
\pgfsys@transformshift{3.822074in}{0.802800in}%
\pgfsys@useobject{currentmarker}{}%
\end{pgfscope}%
\begin{pgfscope}%
\pgfsys@transformshift{3.822309in}{0.815511in}%
\pgfsys@useobject{currentmarker}{}%
\end{pgfscope}%
\begin{pgfscope}%
\pgfsys@transformshift{3.822544in}{0.808168in}%
\pgfsys@useobject{currentmarker}{}%
\end{pgfscope}%
\begin{pgfscope}%
\pgfsys@transformshift{3.822778in}{0.799920in}%
\pgfsys@useobject{currentmarker}{}%
\end{pgfscope}%
\begin{pgfscope}%
\pgfsys@transformshift{3.823013in}{0.790015in}%
\pgfsys@useobject{currentmarker}{}%
\end{pgfscope}%
\begin{pgfscope}%
\pgfsys@transformshift{3.823247in}{0.787910in}%
\pgfsys@useobject{currentmarker}{}%
\end{pgfscope}%
\begin{pgfscope}%
\pgfsys@transformshift{3.823481in}{0.799648in}%
\pgfsys@useobject{currentmarker}{}%
\end{pgfscope}%
\begin{pgfscope}%
\pgfsys@transformshift{3.823715in}{0.815845in}%
\pgfsys@useobject{currentmarker}{}%
\end{pgfscope}%
\begin{pgfscope}%
\pgfsys@transformshift{3.823949in}{0.809499in}%
\pgfsys@useobject{currentmarker}{}%
\end{pgfscope}%
\begin{pgfscope}%
\pgfsys@transformshift{3.824183in}{0.791249in}%
\pgfsys@useobject{currentmarker}{}%
\end{pgfscope}%
\begin{pgfscope}%
\pgfsys@transformshift{3.824416in}{0.781158in}%
\pgfsys@useobject{currentmarker}{}%
\end{pgfscope}%
\begin{pgfscope}%
\pgfsys@transformshift{3.824650in}{0.798862in}%
\pgfsys@useobject{currentmarker}{}%
\end{pgfscope}%
\begin{pgfscope}%
\pgfsys@transformshift{3.824883in}{0.798166in}%
\pgfsys@useobject{currentmarker}{}%
\end{pgfscope}%
\begin{pgfscope}%
\pgfsys@transformshift{3.825116in}{0.804684in}%
\pgfsys@useobject{currentmarker}{}%
\end{pgfscope}%
\begin{pgfscope}%
\pgfsys@transformshift{3.825349in}{0.810297in}%
\pgfsys@useobject{currentmarker}{}%
\end{pgfscope}%
\begin{pgfscope}%
\pgfsys@transformshift{3.825582in}{0.811863in}%
\pgfsys@useobject{currentmarker}{}%
\end{pgfscope}%
\begin{pgfscope}%
\pgfsys@transformshift{3.825815in}{0.793787in}%
\pgfsys@useobject{currentmarker}{}%
\end{pgfscope}%
\begin{pgfscope}%
\pgfsys@transformshift{3.826048in}{0.797248in}%
\pgfsys@useobject{currentmarker}{}%
\end{pgfscope}%
\begin{pgfscope}%
\pgfsys@transformshift{3.826280in}{0.809517in}%
\pgfsys@useobject{currentmarker}{}%
\end{pgfscope}%
\begin{pgfscope}%
\pgfsys@transformshift{3.826513in}{0.793431in}%
\pgfsys@useobject{currentmarker}{}%
\end{pgfscope}%
\begin{pgfscope}%
\pgfsys@transformshift{3.826745in}{0.774780in}%
\pgfsys@useobject{currentmarker}{}%
\end{pgfscope}%
\begin{pgfscope}%
\pgfsys@transformshift{3.826977in}{0.797470in}%
\pgfsys@useobject{currentmarker}{}%
\end{pgfscope}%
\begin{pgfscope}%
\pgfsys@transformshift{3.827209in}{0.817348in}%
\pgfsys@useobject{currentmarker}{}%
\end{pgfscope}%
\begin{pgfscope}%
\pgfsys@transformshift{3.827441in}{0.806414in}%
\pgfsys@useobject{currentmarker}{}%
\end{pgfscope}%
\begin{pgfscope}%
\pgfsys@transformshift{3.827672in}{0.809707in}%
\pgfsys@useobject{currentmarker}{}%
\end{pgfscope}%
\begin{pgfscope}%
\pgfsys@transformshift{3.827904in}{0.796295in}%
\pgfsys@useobject{currentmarker}{}%
\end{pgfscope}%
\begin{pgfscope}%
\pgfsys@transformshift{3.828135in}{0.802652in}%
\pgfsys@useobject{currentmarker}{}%
\end{pgfscope}%
\begin{pgfscope}%
\pgfsys@transformshift{3.828367in}{0.809247in}%
\pgfsys@useobject{currentmarker}{}%
\end{pgfscope}%
\begin{pgfscope}%
\pgfsys@transformshift{3.828598in}{0.802941in}%
\pgfsys@useobject{currentmarker}{}%
\end{pgfscope}%
\begin{pgfscope}%
\pgfsys@transformshift{3.828829in}{0.790749in}%
\pgfsys@useobject{currentmarker}{}%
\end{pgfscope}%
\begin{pgfscope}%
\pgfsys@transformshift{3.829060in}{0.799998in}%
\pgfsys@useobject{currentmarker}{}%
\end{pgfscope}%
\begin{pgfscope}%
\pgfsys@transformshift{3.829291in}{0.787130in}%
\pgfsys@useobject{currentmarker}{}%
\end{pgfscope}%
\begin{pgfscope}%
\pgfsys@transformshift{3.829521in}{0.793036in}%
\pgfsys@useobject{currentmarker}{}%
\end{pgfscope}%
\begin{pgfscope}%
\pgfsys@transformshift{3.829752in}{0.807912in}%
\pgfsys@useobject{currentmarker}{}%
\end{pgfscope}%
\begin{pgfscope}%
\pgfsys@transformshift{3.829982in}{0.811240in}%
\pgfsys@useobject{currentmarker}{}%
\end{pgfscope}%
\begin{pgfscope}%
\pgfsys@transformshift{3.830213in}{0.813198in}%
\pgfsys@useobject{currentmarker}{}%
\end{pgfscope}%
\begin{pgfscope}%
\pgfsys@transformshift{3.830443in}{0.809282in}%
\pgfsys@useobject{currentmarker}{}%
\end{pgfscope}%
\begin{pgfscope}%
\pgfsys@transformshift{3.830673in}{0.802715in}%
\pgfsys@useobject{currentmarker}{}%
\end{pgfscope}%
\begin{pgfscope}%
\pgfsys@transformshift{3.830903in}{0.794041in}%
\pgfsys@useobject{currentmarker}{}%
\end{pgfscope}%
\begin{pgfscope}%
\pgfsys@transformshift{3.831132in}{0.806590in}%
\pgfsys@useobject{currentmarker}{}%
\end{pgfscope}%
\begin{pgfscope}%
\pgfsys@transformshift{3.831362in}{0.801119in}%
\pgfsys@useobject{currentmarker}{}%
\end{pgfscope}%
\begin{pgfscope}%
\pgfsys@transformshift{3.831591in}{0.792146in}%
\pgfsys@useobject{currentmarker}{}%
\end{pgfscope}%
\begin{pgfscope}%
\pgfsys@transformshift{3.831821in}{0.777559in}%
\pgfsys@useobject{currentmarker}{}%
\end{pgfscope}%
\begin{pgfscope}%
\pgfsys@transformshift{3.832050in}{0.791479in}%
\pgfsys@useobject{currentmarker}{}%
\end{pgfscope}%
\begin{pgfscope}%
\pgfsys@transformshift{3.832279in}{0.800456in}%
\pgfsys@useobject{currentmarker}{}%
\end{pgfscope}%
\begin{pgfscope}%
\pgfsys@transformshift{3.832508in}{0.802549in}%
\pgfsys@useobject{currentmarker}{}%
\end{pgfscope}%
\begin{pgfscope}%
\pgfsys@transformshift{3.832737in}{0.795241in}%
\pgfsys@useobject{currentmarker}{}%
\end{pgfscope}%
\begin{pgfscope}%
\pgfsys@transformshift{3.832966in}{0.788425in}%
\pgfsys@useobject{currentmarker}{}%
\end{pgfscope}%
\begin{pgfscope}%
\pgfsys@transformshift{3.833194in}{0.784597in}%
\pgfsys@useobject{currentmarker}{}%
\end{pgfscope}%
\begin{pgfscope}%
\pgfsys@transformshift{3.833423in}{0.806988in}%
\pgfsys@useobject{currentmarker}{}%
\end{pgfscope}%
\begin{pgfscope}%
\pgfsys@transformshift{3.833651in}{0.814759in}%
\pgfsys@useobject{currentmarker}{}%
\end{pgfscope}%
\begin{pgfscope}%
\pgfsys@transformshift{3.833879in}{0.797064in}%
\pgfsys@useobject{currentmarker}{}%
\end{pgfscope}%
\begin{pgfscope}%
\pgfsys@transformshift{3.834107in}{0.799591in}%
\pgfsys@useobject{currentmarker}{}%
\end{pgfscope}%
\begin{pgfscope}%
\pgfsys@transformshift{3.834335in}{0.814219in}%
\pgfsys@useobject{currentmarker}{}%
\end{pgfscope}%
\begin{pgfscope}%
\pgfsys@transformshift{3.834563in}{0.806355in}%
\pgfsys@useobject{currentmarker}{}%
\end{pgfscope}%
\begin{pgfscope}%
\pgfsys@transformshift{3.834790in}{0.796296in}%
\pgfsys@useobject{currentmarker}{}%
\end{pgfscope}%
\begin{pgfscope}%
\pgfsys@transformshift{3.835018in}{0.797068in}%
\pgfsys@useobject{currentmarker}{}%
\end{pgfscope}%
\begin{pgfscope}%
\pgfsys@transformshift{3.835245in}{0.801361in}%
\pgfsys@useobject{currentmarker}{}%
\end{pgfscope}%
\begin{pgfscope}%
\pgfsys@transformshift{3.835473in}{0.798993in}%
\pgfsys@useobject{currentmarker}{}%
\end{pgfscope}%
\begin{pgfscope}%
\pgfsys@transformshift{3.835700in}{0.799062in}%
\pgfsys@useobject{currentmarker}{}%
\end{pgfscope}%
\begin{pgfscope}%
\pgfsys@transformshift{3.835927in}{0.799717in}%
\pgfsys@useobject{currentmarker}{}%
\end{pgfscope}%
\begin{pgfscope}%
\pgfsys@transformshift{3.836154in}{0.798716in}%
\pgfsys@useobject{currentmarker}{}%
\end{pgfscope}%
\begin{pgfscope}%
\pgfsys@transformshift{3.836380in}{0.791820in}%
\pgfsys@useobject{currentmarker}{}%
\end{pgfscope}%
\begin{pgfscope}%
\pgfsys@transformshift{3.836607in}{0.799594in}%
\pgfsys@useobject{currentmarker}{}%
\end{pgfscope}%
\begin{pgfscope}%
\pgfsys@transformshift{3.836834in}{0.791946in}%
\pgfsys@useobject{currentmarker}{}%
\end{pgfscope}%
\begin{pgfscope}%
\pgfsys@transformshift{3.837060in}{0.819825in}%
\pgfsys@useobject{currentmarker}{}%
\end{pgfscope}%
\begin{pgfscope}%
\pgfsys@transformshift{3.837286in}{0.819903in}%
\pgfsys@useobject{currentmarker}{}%
\end{pgfscope}%
\begin{pgfscope}%
\pgfsys@transformshift{3.837512in}{0.801129in}%
\pgfsys@useobject{currentmarker}{}%
\end{pgfscope}%
\begin{pgfscope}%
\pgfsys@transformshift{3.837738in}{0.808219in}%
\pgfsys@useobject{currentmarker}{}%
\end{pgfscope}%
\begin{pgfscope}%
\pgfsys@transformshift{3.837964in}{0.797036in}%
\pgfsys@useobject{currentmarker}{}%
\end{pgfscope}%
\begin{pgfscope}%
\pgfsys@transformshift{3.838190in}{0.787107in}%
\pgfsys@useobject{currentmarker}{}%
\end{pgfscope}%
\begin{pgfscope}%
\pgfsys@transformshift{3.838416in}{0.777128in}%
\pgfsys@useobject{currentmarker}{}%
\end{pgfscope}%
\begin{pgfscope}%
\pgfsys@transformshift{3.838641in}{0.785859in}%
\pgfsys@useobject{currentmarker}{}%
\end{pgfscope}%
\begin{pgfscope}%
\pgfsys@transformshift{3.838867in}{0.812727in}%
\pgfsys@useobject{currentmarker}{}%
\end{pgfscope}%
\begin{pgfscope}%
\pgfsys@transformshift{3.839092in}{0.815736in}%
\pgfsys@useobject{currentmarker}{}%
\end{pgfscope}%
\begin{pgfscope}%
\pgfsys@transformshift{3.839317in}{0.801745in}%
\pgfsys@useobject{currentmarker}{}%
\end{pgfscope}%
\begin{pgfscope}%
\pgfsys@transformshift{3.839542in}{0.780815in}%
\pgfsys@useobject{currentmarker}{}%
\end{pgfscope}%
\begin{pgfscope}%
\pgfsys@transformshift{3.839767in}{0.786248in}%
\pgfsys@useobject{currentmarker}{}%
\end{pgfscope}%
\begin{pgfscope}%
\pgfsys@transformshift{3.839992in}{0.800855in}%
\pgfsys@useobject{currentmarker}{}%
\end{pgfscope}%
\begin{pgfscope}%
\pgfsys@transformshift{3.840216in}{0.791983in}%
\pgfsys@useobject{currentmarker}{}%
\end{pgfscope}%
\begin{pgfscope}%
\pgfsys@transformshift{3.840441in}{0.800071in}%
\pgfsys@useobject{currentmarker}{}%
\end{pgfscope}%
\begin{pgfscope}%
\pgfsys@transformshift{3.840665in}{0.802120in}%
\pgfsys@useobject{currentmarker}{}%
\end{pgfscope}%
\begin{pgfscope}%
\pgfsys@transformshift{3.840889in}{0.799886in}%
\pgfsys@useobject{currentmarker}{}%
\end{pgfscope}%
\begin{pgfscope}%
\pgfsys@transformshift{3.841113in}{0.814045in}%
\pgfsys@useobject{currentmarker}{}%
\end{pgfscope}%
\begin{pgfscope}%
\pgfsys@transformshift{3.841337in}{0.802534in}%
\pgfsys@useobject{currentmarker}{}%
\end{pgfscope}%
\begin{pgfscope}%
\pgfsys@transformshift{3.841561in}{0.792941in}%
\pgfsys@useobject{currentmarker}{}%
\end{pgfscope}%
\begin{pgfscope}%
\pgfsys@transformshift{3.841785in}{0.798449in}%
\pgfsys@useobject{currentmarker}{}%
\end{pgfscope}%
\begin{pgfscope}%
\pgfsys@transformshift{3.842009in}{0.776270in}%
\pgfsys@useobject{currentmarker}{}%
\end{pgfscope}%
\begin{pgfscope}%
\pgfsys@transformshift{3.842232in}{0.780129in}%
\pgfsys@useobject{currentmarker}{}%
\end{pgfscope}%
\begin{pgfscope}%
\pgfsys@transformshift{3.842456in}{0.799161in}%
\pgfsys@useobject{currentmarker}{}%
\end{pgfscope}%
\begin{pgfscope}%
\pgfsys@transformshift{3.842679in}{0.785970in}%
\pgfsys@useobject{currentmarker}{}%
\end{pgfscope}%
\begin{pgfscope}%
\pgfsys@transformshift{3.842902in}{0.794224in}%
\pgfsys@useobject{currentmarker}{}%
\end{pgfscope}%
\begin{pgfscope}%
\pgfsys@transformshift{3.843125in}{0.789204in}%
\pgfsys@useobject{currentmarker}{}%
\end{pgfscope}%
\begin{pgfscope}%
\pgfsys@transformshift{3.843348in}{0.789287in}%
\pgfsys@useobject{currentmarker}{}%
\end{pgfscope}%
\begin{pgfscope}%
\pgfsys@transformshift{3.843571in}{0.807426in}%
\pgfsys@useobject{currentmarker}{}%
\end{pgfscope}%
\begin{pgfscope}%
\pgfsys@transformshift{3.843793in}{0.802200in}%
\pgfsys@useobject{currentmarker}{}%
\end{pgfscope}%
\begin{pgfscope}%
\pgfsys@transformshift{3.844016in}{0.799880in}%
\pgfsys@useobject{currentmarker}{}%
\end{pgfscope}%
\begin{pgfscope}%
\pgfsys@transformshift{3.844238in}{0.806170in}%
\pgfsys@useobject{currentmarker}{}%
\end{pgfscope}%
\begin{pgfscope}%
\pgfsys@transformshift{3.844460in}{0.780180in}%
\pgfsys@useobject{currentmarker}{}%
\end{pgfscope}%
\begin{pgfscope}%
\pgfsys@transformshift{3.844683in}{0.808505in}%
\pgfsys@useobject{currentmarker}{}%
\end{pgfscope}%
\begin{pgfscope}%
\pgfsys@transformshift{3.844905in}{0.823083in}%
\pgfsys@useobject{currentmarker}{}%
\end{pgfscope}%
\begin{pgfscope}%
\pgfsys@transformshift{3.845127in}{0.812846in}%
\pgfsys@useobject{currentmarker}{}%
\end{pgfscope}%
\begin{pgfscope}%
\pgfsys@transformshift{3.845348in}{0.800623in}%
\pgfsys@useobject{currentmarker}{}%
\end{pgfscope}%
\begin{pgfscope}%
\pgfsys@transformshift{3.845570in}{0.788764in}%
\pgfsys@useobject{currentmarker}{}%
\end{pgfscope}%
\begin{pgfscope}%
\pgfsys@transformshift{3.845792in}{0.780013in}%
\pgfsys@useobject{currentmarker}{}%
\end{pgfscope}%
\begin{pgfscope}%
\pgfsys@transformshift{3.846013in}{0.796324in}%
\pgfsys@useobject{currentmarker}{}%
\end{pgfscope}%
\begin{pgfscope}%
\pgfsys@transformshift{3.846234in}{0.788679in}%
\pgfsys@useobject{currentmarker}{}%
\end{pgfscope}%
\begin{pgfscope}%
\pgfsys@transformshift{3.846455in}{0.820847in}%
\pgfsys@useobject{currentmarker}{}%
\end{pgfscope}%
\begin{pgfscope}%
\pgfsys@transformshift{3.846677in}{0.800181in}%
\pgfsys@useobject{currentmarker}{}%
\end{pgfscope}%
\begin{pgfscope}%
\pgfsys@transformshift{3.846897in}{0.790251in}%
\pgfsys@useobject{currentmarker}{}%
\end{pgfscope}%
\begin{pgfscope}%
\pgfsys@transformshift{3.847118in}{0.804872in}%
\pgfsys@useobject{currentmarker}{}%
\end{pgfscope}%
\begin{pgfscope}%
\pgfsys@transformshift{3.847339in}{0.811440in}%
\pgfsys@useobject{currentmarker}{}%
\end{pgfscope}%
\begin{pgfscope}%
\pgfsys@transformshift{3.847560in}{0.824380in}%
\pgfsys@useobject{currentmarker}{}%
\end{pgfscope}%
\begin{pgfscope}%
\pgfsys@transformshift{3.847780in}{0.821546in}%
\pgfsys@useobject{currentmarker}{}%
\end{pgfscope}%
\begin{pgfscope}%
\pgfsys@transformshift{3.848000in}{0.812029in}%
\pgfsys@useobject{currentmarker}{}%
\end{pgfscope}%
\begin{pgfscope}%
\pgfsys@transformshift{3.848221in}{0.816587in}%
\pgfsys@useobject{currentmarker}{}%
\end{pgfscope}%
\begin{pgfscope}%
\pgfsys@transformshift{3.848441in}{0.788760in}%
\pgfsys@useobject{currentmarker}{}%
\end{pgfscope}%
\begin{pgfscope}%
\pgfsys@transformshift{3.848661in}{0.788109in}%
\pgfsys@useobject{currentmarker}{}%
\end{pgfscope}%
\begin{pgfscope}%
\pgfsys@transformshift{3.848880in}{0.793053in}%
\pgfsys@useobject{currentmarker}{}%
\end{pgfscope}%
\begin{pgfscope}%
\pgfsys@transformshift{3.849100in}{0.803942in}%
\pgfsys@useobject{currentmarker}{}%
\end{pgfscope}%
\begin{pgfscope}%
\pgfsys@transformshift{3.849320in}{0.801347in}%
\pgfsys@useobject{currentmarker}{}%
\end{pgfscope}%
\begin{pgfscope}%
\pgfsys@transformshift{3.849539in}{0.813440in}%
\pgfsys@useobject{currentmarker}{}%
\end{pgfscope}%
\begin{pgfscope}%
\pgfsys@transformshift{3.849759in}{0.810601in}%
\pgfsys@useobject{currentmarker}{}%
\end{pgfscope}%
\begin{pgfscope}%
\pgfsys@transformshift{3.849978in}{0.815097in}%
\pgfsys@useobject{currentmarker}{}%
\end{pgfscope}%
\begin{pgfscope}%
\pgfsys@transformshift{3.850197in}{0.812278in}%
\pgfsys@useobject{currentmarker}{}%
\end{pgfscope}%
\begin{pgfscope}%
\pgfsys@transformshift{3.850416in}{0.795407in}%
\pgfsys@useobject{currentmarker}{}%
\end{pgfscope}%
\begin{pgfscope}%
\pgfsys@transformshift{3.850635in}{0.792361in}%
\pgfsys@useobject{currentmarker}{}%
\end{pgfscope}%
\begin{pgfscope}%
\pgfsys@transformshift{3.850854in}{0.800203in}%
\pgfsys@useobject{currentmarker}{}%
\end{pgfscope}%
\begin{pgfscope}%
\pgfsys@transformshift{3.851072in}{0.800724in}%
\pgfsys@useobject{currentmarker}{}%
\end{pgfscope}%
\begin{pgfscope}%
\pgfsys@transformshift{3.851291in}{0.798915in}%
\pgfsys@useobject{currentmarker}{}%
\end{pgfscope}%
\begin{pgfscope}%
\pgfsys@transformshift{3.851509in}{0.810571in}%
\pgfsys@useobject{currentmarker}{}%
\end{pgfscope}%
\begin{pgfscope}%
\pgfsys@transformshift{3.851728in}{0.815336in}%
\pgfsys@useobject{currentmarker}{}%
\end{pgfscope}%
\begin{pgfscope}%
\pgfsys@transformshift{3.851946in}{0.812839in}%
\pgfsys@useobject{currentmarker}{}%
\end{pgfscope}%
\begin{pgfscope}%
\pgfsys@transformshift{3.852164in}{0.808037in}%
\pgfsys@useobject{currentmarker}{}%
\end{pgfscope}%
\begin{pgfscope}%
\pgfsys@transformshift{3.852382in}{0.803892in}%
\pgfsys@useobject{currentmarker}{}%
\end{pgfscope}%
\begin{pgfscope}%
\pgfsys@transformshift{3.852600in}{0.793736in}%
\pgfsys@useobject{currentmarker}{}%
\end{pgfscope}%
\begin{pgfscope}%
\pgfsys@transformshift{3.852818in}{0.802907in}%
\pgfsys@useobject{currentmarker}{}%
\end{pgfscope}%
\begin{pgfscope}%
\pgfsys@transformshift{3.853035in}{0.811448in}%
\pgfsys@useobject{currentmarker}{}%
\end{pgfscope}%
\begin{pgfscope}%
\pgfsys@transformshift{3.853253in}{0.810464in}%
\pgfsys@useobject{currentmarker}{}%
\end{pgfscope}%
\begin{pgfscope}%
\pgfsys@transformshift{3.853470in}{0.809360in}%
\pgfsys@useobject{currentmarker}{}%
\end{pgfscope}%
\begin{pgfscope}%
\pgfsys@transformshift{3.853687in}{0.811264in}%
\pgfsys@useobject{currentmarker}{}%
\end{pgfscope}%
\begin{pgfscope}%
\pgfsys@transformshift{3.853904in}{0.804757in}%
\pgfsys@useobject{currentmarker}{}%
\end{pgfscope}%
\begin{pgfscope}%
\pgfsys@transformshift{3.854121in}{0.797561in}%
\pgfsys@useobject{currentmarker}{}%
\end{pgfscope}%
\begin{pgfscope}%
\pgfsys@transformshift{3.854338in}{0.794671in}%
\pgfsys@useobject{currentmarker}{}%
\end{pgfscope}%
\begin{pgfscope}%
\pgfsys@transformshift{3.854555in}{0.810904in}%
\pgfsys@useobject{currentmarker}{}%
\end{pgfscope}%
\begin{pgfscope}%
\pgfsys@transformshift{3.854772in}{0.803882in}%
\pgfsys@useobject{currentmarker}{}%
\end{pgfscope}%
\begin{pgfscope}%
\pgfsys@transformshift{3.854988in}{0.809895in}%
\pgfsys@useobject{currentmarker}{}%
\end{pgfscope}%
\begin{pgfscope}%
\pgfsys@transformshift{3.855205in}{0.820015in}%
\pgfsys@useobject{currentmarker}{}%
\end{pgfscope}%
\begin{pgfscope}%
\pgfsys@transformshift{3.855421in}{0.815026in}%
\pgfsys@useobject{currentmarker}{}%
\end{pgfscope}%
\begin{pgfscope}%
\pgfsys@transformshift{3.855637in}{0.808677in}%
\pgfsys@useobject{currentmarker}{}%
\end{pgfscope}%
\begin{pgfscope}%
\pgfsys@transformshift{3.855853in}{0.787729in}%
\pgfsys@useobject{currentmarker}{}%
\end{pgfscope}%
\begin{pgfscope}%
\pgfsys@transformshift{3.856069in}{0.790857in}%
\pgfsys@useobject{currentmarker}{}%
\end{pgfscope}%
\begin{pgfscope}%
\pgfsys@transformshift{3.856285in}{0.812845in}%
\pgfsys@useobject{currentmarker}{}%
\end{pgfscope}%
\begin{pgfscope}%
\pgfsys@transformshift{3.856501in}{0.804601in}%
\pgfsys@useobject{currentmarker}{}%
\end{pgfscope}%
\begin{pgfscope}%
\pgfsys@transformshift{3.856717in}{0.798043in}%
\pgfsys@useobject{currentmarker}{}%
\end{pgfscope}%
\begin{pgfscope}%
\pgfsys@transformshift{3.856932in}{0.782130in}%
\pgfsys@useobject{currentmarker}{}%
\end{pgfscope}%
\begin{pgfscope}%
\pgfsys@transformshift{3.857147in}{0.796883in}%
\pgfsys@useobject{currentmarker}{}%
\end{pgfscope}%
\begin{pgfscope}%
\pgfsys@transformshift{3.857363in}{0.811564in}%
\pgfsys@useobject{currentmarker}{}%
\end{pgfscope}%
\begin{pgfscope}%
\pgfsys@transformshift{3.857578in}{0.790270in}%
\pgfsys@useobject{currentmarker}{}%
\end{pgfscope}%
\begin{pgfscope}%
\pgfsys@transformshift{3.857793in}{0.801810in}%
\pgfsys@useobject{currentmarker}{}%
\end{pgfscope}%
\begin{pgfscope}%
\pgfsys@transformshift{3.858008in}{0.793095in}%
\pgfsys@useobject{currentmarker}{}%
\end{pgfscope}%
\begin{pgfscope}%
\pgfsys@transformshift{3.858223in}{0.798169in}%
\pgfsys@useobject{currentmarker}{}%
\end{pgfscope}%
\begin{pgfscope}%
\pgfsys@transformshift{3.858437in}{0.803402in}%
\pgfsys@useobject{currentmarker}{}%
\end{pgfscope}%
\begin{pgfscope}%
\pgfsys@transformshift{3.858652in}{0.792927in}%
\pgfsys@useobject{currentmarker}{}%
\end{pgfscope}%
\begin{pgfscope}%
\pgfsys@transformshift{3.858866in}{0.802566in}%
\pgfsys@useobject{currentmarker}{}%
\end{pgfscope}%
\begin{pgfscope}%
\pgfsys@transformshift{3.859081in}{0.799753in}%
\pgfsys@useobject{currentmarker}{}%
\end{pgfscope}%
\begin{pgfscope}%
\pgfsys@transformshift{3.859295in}{0.792222in}%
\pgfsys@useobject{currentmarker}{}%
\end{pgfscope}%
\begin{pgfscope}%
\pgfsys@transformshift{3.859509in}{0.808053in}%
\pgfsys@useobject{currentmarker}{}%
\end{pgfscope}%
\begin{pgfscope}%
\pgfsys@transformshift{3.859723in}{0.803664in}%
\pgfsys@useobject{currentmarker}{}%
\end{pgfscope}%
\begin{pgfscope}%
\pgfsys@transformshift{3.859937in}{0.797329in}%
\pgfsys@useobject{currentmarker}{}%
\end{pgfscope}%
\begin{pgfscope}%
\pgfsys@transformshift{3.860151in}{0.791336in}%
\pgfsys@useobject{currentmarker}{}%
\end{pgfscope}%
\begin{pgfscope}%
\pgfsys@transformshift{3.860365in}{0.807466in}%
\pgfsys@useobject{currentmarker}{}%
\end{pgfscope}%
\begin{pgfscope}%
\pgfsys@transformshift{3.860578in}{0.785413in}%
\pgfsys@useobject{currentmarker}{}%
\end{pgfscope}%
\begin{pgfscope}%
\pgfsys@transformshift{3.860792in}{0.796614in}%
\pgfsys@useobject{currentmarker}{}%
\end{pgfscope}%
\begin{pgfscope}%
\pgfsys@transformshift{3.861005in}{0.802057in}%
\pgfsys@useobject{currentmarker}{}%
\end{pgfscope}%
\begin{pgfscope}%
\pgfsys@transformshift{3.861218in}{0.796883in}%
\pgfsys@useobject{currentmarker}{}%
\end{pgfscope}%
\begin{pgfscope}%
\pgfsys@transformshift{3.861431in}{0.791457in}%
\pgfsys@useobject{currentmarker}{}%
\end{pgfscope}%
\begin{pgfscope}%
\pgfsys@transformshift{3.861644in}{0.803192in}%
\pgfsys@useobject{currentmarker}{}%
\end{pgfscope}%
\begin{pgfscope}%
\pgfsys@transformshift{3.861857in}{0.810039in}%
\pgfsys@useobject{currentmarker}{}%
\end{pgfscope}%
\begin{pgfscope}%
\pgfsys@transformshift{3.862070in}{0.820681in}%
\pgfsys@useobject{currentmarker}{}%
\end{pgfscope}%
\begin{pgfscope}%
\pgfsys@transformshift{3.862283in}{0.818743in}%
\pgfsys@useobject{currentmarker}{}%
\end{pgfscope}%
\begin{pgfscope}%
\pgfsys@transformshift{3.862495in}{0.794060in}%
\pgfsys@useobject{currentmarker}{}%
\end{pgfscope}%
\begin{pgfscope}%
\pgfsys@transformshift{3.862708in}{0.792699in}%
\pgfsys@useobject{currentmarker}{}%
\end{pgfscope}%
\begin{pgfscope}%
\pgfsys@transformshift{3.862920in}{0.796360in}%
\pgfsys@useobject{currentmarker}{}%
\end{pgfscope}%
\begin{pgfscope}%
\pgfsys@transformshift{3.863132in}{0.791251in}%
\pgfsys@useobject{currentmarker}{}%
\end{pgfscope}%
\begin{pgfscope}%
\pgfsys@transformshift{3.863344in}{0.788867in}%
\pgfsys@useobject{currentmarker}{}%
\end{pgfscope}%
\begin{pgfscope}%
\pgfsys@transformshift{3.863556in}{0.792161in}%
\pgfsys@useobject{currentmarker}{}%
\end{pgfscope}%
\begin{pgfscope}%
\pgfsys@transformshift{3.863768in}{0.801977in}%
\pgfsys@useobject{currentmarker}{}%
\end{pgfscope}%
\begin{pgfscope}%
\pgfsys@transformshift{3.863980in}{0.814640in}%
\pgfsys@useobject{currentmarker}{}%
\end{pgfscope}%
\begin{pgfscope}%
\pgfsys@transformshift{3.864192in}{0.807683in}%
\pgfsys@useobject{currentmarker}{}%
\end{pgfscope}%
\begin{pgfscope}%
\pgfsys@transformshift{3.864403in}{0.806381in}%
\pgfsys@useobject{currentmarker}{}%
\end{pgfscope}%
\begin{pgfscope}%
\pgfsys@transformshift{3.864615in}{0.814336in}%
\pgfsys@useobject{currentmarker}{}%
\end{pgfscope}%
\begin{pgfscope}%
\pgfsys@transformshift{3.864826in}{0.801431in}%
\pgfsys@useobject{currentmarker}{}%
\end{pgfscope}%
\begin{pgfscope}%
\pgfsys@transformshift{3.865037in}{0.787096in}%
\pgfsys@useobject{currentmarker}{}%
\end{pgfscope}%
\begin{pgfscope}%
\pgfsys@transformshift{3.865249in}{0.797677in}%
\pgfsys@useobject{currentmarker}{}%
\end{pgfscope}%
\begin{pgfscope}%
\pgfsys@transformshift{3.865460in}{0.804488in}%
\pgfsys@useobject{currentmarker}{}%
\end{pgfscope}%
\begin{pgfscope}%
\pgfsys@transformshift{3.865670in}{0.788676in}%
\pgfsys@useobject{currentmarker}{}%
\end{pgfscope}%
\begin{pgfscope}%
\pgfsys@transformshift{3.865881in}{0.785780in}%
\pgfsys@useobject{currentmarker}{}%
\end{pgfscope}%
\begin{pgfscope}%
\pgfsys@transformshift{3.866092in}{0.794863in}%
\pgfsys@useobject{currentmarker}{}%
\end{pgfscope}%
\begin{pgfscope}%
\pgfsys@transformshift{3.866302in}{0.814048in}%
\pgfsys@useobject{currentmarker}{}%
\end{pgfscope}%
\begin{pgfscope}%
\pgfsys@transformshift{3.866513in}{0.823098in}%
\pgfsys@useobject{currentmarker}{}%
\end{pgfscope}%
\begin{pgfscope}%
\pgfsys@transformshift{3.866723in}{0.823633in}%
\pgfsys@useobject{currentmarker}{}%
\end{pgfscope}%
\begin{pgfscope}%
\pgfsys@transformshift{3.866934in}{0.810006in}%
\pgfsys@useobject{currentmarker}{}%
\end{pgfscope}%
\begin{pgfscope}%
\pgfsys@transformshift{3.867144in}{0.787804in}%
\pgfsys@useobject{currentmarker}{}%
\end{pgfscope}%
\begin{pgfscope}%
\pgfsys@transformshift{3.867354in}{0.802091in}%
\pgfsys@useobject{currentmarker}{}%
\end{pgfscope}%
\begin{pgfscope}%
\pgfsys@transformshift{3.867564in}{0.819367in}%
\pgfsys@useobject{currentmarker}{}%
\end{pgfscope}%
\begin{pgfscope}%
\pgfsys@transformshift{3.867773in}{0.809021in}%
\pgfsys@useobject{currentmarker}{}%
\end{pgfscope}%
\begin{pgfscope}%
\pgfsys@transformshift{3.867983in}{0.800626in}%
\pgfsys@useobject{currentmarker}{}%
\end{pgfscope}%
\begin{pgfscope}%
\pgfsys@transformshift{3.868193in}{0.784866in}%
\pgfsys@useobject{currentmarker}{}%
\end{pgfscope}%
\begin{pgfscope}%
\pgfsys@transformshift{3.868402in}{0.791327in}%
\pgfsys@useobject{currentmarker}{}%
\end{pgfscope}%
\begin{pgfscope}%
\pgfsys@transformshift{3.868611in}{0.798977in}%
\pgfsys@useobject{currentmarker}{}%
\end{pgfscope}%
\begin{pgfscope}%
\pgfsys@transformshift{3.868821in}{0.806325in}%
\pgfsys@useobject{currentmarker}{}%
\end{pgfscope}%
\begin{pgfscope}%
\pgfsys@transformshift{3.869030in}{0.802456in}%
\pgfsys@useobject{currentmarker}{}%
\end{pgfscope}%
\begin{pgfscope}%
\pgfsys@transformshift{3.869239in}{0.798465in}%
\pgfsys@useobject{currentmarker}{}%
\end{pgfscope}%
\begin{pgfscope}%
\pgfsys@transformshift{3.869448in}{0.807238in}%
\pgfsys@useobject{currentmarker}{}%
\end{pgfscope}%
\begin{pgfscope}%
\pgfsys@transformshift{3.869657in}{0.801270in}%
\pgfsys@useobject{currentmarker}{}%
\end{pgfscope}%
\begin{pgfscope}%
\pgfsys@transformshift{3.869865in}{0.800794in}%
\pgfsys@useobject{currentmarker}{}%
\end{pgfscope}%
\begin{pgfscope}%
\pgfsys@transformshift{3.870074in}{0.796425in}%
\pgfsys@useobject{currentmarker}{}%
\end{pgfscope}%
\begin{pgfscope}%
\pgfsys@transformshift{3.870283in}{0.794943in}%
\pgfsys@useobject{currentmarker}{}%
\end{pgfscope}%
\begin{pgfscope}%
\pgfsys@transformshift{3.870491in}{0.780537in}%
\pgfsys@useobject{currentmarker}{}%
\end{pgfscope}%
\begin{pgfscope}%
\pgfsys@transformshift{3.870699in}{0.786664in}%
\pgfsys@useobject{currentmarker}{}%
\end{pgfscope}%
\begin{pgfscope}%
\pgfsys@transformshift{3.870907in}{0.787318in}%
\pgfsys@useobject{currentmarker}{}%
\end{pgfscope}%
\begin{pgfscope}%
\pgfsys@transformshift{3.871115in}{0.778799in}%
\pgfsys@useobject{currentmarker}{}%
\end{pgfscope}%
\begin{pgfscope}%
\pgfsys@transformshift{3.871323in}{0.788189in}%
\pgfsys@useobject{currentmarker}{}%
\end{pgfscope}%
\begin{pgfscope}%
\pgfsys@transformshift{3.871531in}{0.818758in}%
\pgfsys@useobject{currentmarker}{}%
\end{pgfscope}%
\begin{pgfscope}%
\pgfsys@transformshift{3.871739in}{0.814345in}%
\pgfsys@useobject{currentmarker}{}%
\end{pgfscope}%
\begin{pgfscope}%
\pgfsys@transformshift{3.871947in}{0.801397in}%
\pgfsys@useobject{currentmarker}{}%
\end{pgfscope}%
\begin{pgfscope}%
\pgfsys@transformshift{3.872154in}{0.808031in}%
\pgfsys@useobject{currentmarker}{}%
\end{pgfscope}%
\begin{pgfscope}%
\pgfsys@transformshift{3.872362in}{0.811153in}%
\pgfsys@useobject{currentmarker}{}%
\end{pgfscope}%
\begin{pgfscope}%
\pgfsys@transformshift{3.872569in}{0.799483in}%
\pgfsys@useobject{currentmarker}{}%
\end{pgfscope}%
\begin{pgfscope}%
\pgfsys@transformshift{3.872776in}{0.798053in}%
\pgfsys@useobject{currentmarker}{}%
\end{pgfscope}%
\begin{pgfscope}%
\pgfsys@transformshift{3.872983in}{0.815372in}%
\pgfsys@useobject{currentmarker}{}%
\end{pgfscope}%
\begin{pgfscope}%
\pgfsys@transformshift{3.873190in}{0.816723in}%
\pgfsys@useobject{currentmarker}{}%
\end{pgfscope}%
\begin{pgfscope}%
\pgfsys@transformshift{3.873397in}{0.809791in}%
\pgfsys@useobject{currentmarker}{}%
\end{pgfscope}%
\begin{pgfscope}%
\pgfsys@transformshift{3.873604in}{0.638789in}%
\pgfsys@useobject{currentmarker}{}%
\end{pgfscope}%
\end{pgfscope}%
\begin{pgfscope}%
\pgfpathrectangle{\pgfqpoint{0.661284in}{0.417642in}}{\pgfqpoint{3.365288in}{2.055000in}}%
\pgfusepath{clip}%
\pgfsetbuttcap%
\pgfsetroundjoin%
\pgfsetlinewidth{1.505625pt}%
\definecolor{currentstroke}{rgb}{0.003922,0.450980,0.698039}%
\pgfsetstrokecolor{currentstroke}%
\pgfsetdash{{5.550000pt}{2.400000pt}}{0.000000pt}%
\pgfpathmoveto{\pgfqpoint{0.814251in}{0.636998in}}%
\pgfpathlineto{\pgfqpoint{3.873604in}{0.636998in}}%
\pgfpathlineto{\pgfqpoint{3.873604in}{0.636998in}}%
\pgfusepath{stroke}%
\end{pgfscope}%
\begin{pgfscope}%
\pgfpathrectangle{\pgfqpoint{0.661284in}{0.417642in}}{\pgfqpoint{3.365288in}{2.055000in}}%
\pgfusepath{clip}%
\pgfsetbuttcap%
\pgfsetroundjoin%
\pgfsetlinewidth{1.505625pt}%
\definecolor{currentstroke}{rgb}{0.007843,0.619608,0.450980}%
\pgfsetstrokecolor{currentstroke}%
\pgfsetdash{{5.550000pt}{2.400000pt}}{0.000000pt}%
\pgfpathmoveto{\pgfqpoint{0.814251in}{2.379233in}}%
\pgfpathlineto{\pgfqpoint{3.873604in}{0.511051in}}%
\pgfpathlineto{\pgfqpoint{3.873604in}{0.511051in}}%
\pgfusepath{stroke}%
\end{pgfscope}%
\begin{pgfscope}%
\pgfsetrectcap%
\pgfsetmiterjoin%
\pgfsetlinewidth{0.803000pt}%
\definecolor{currentstroke}{rgb}{0.000000,0.000000,0.000000}%
\pgfsetstrokecolor{currentstroke}%
\pgfsetdash{}{0pt}%
\pgfpathmoveto{\pgfqpoint{0.661284in}{0.417642in}}%
\pgfpathlineto{\pgfqpoint{0.661284in}{2.472642in}}%
\pgfusepath{stroke}%
\end{pgfscope}%
\begin{pgfscope}%
\pgfsetrectcap%
\pgfsetmiterjoin%
\pgfsetlinewidth{0.803000pt}%
\definecolor{currentstroke}{rgb}{0.000000,0.000000,0.000000}%
\pgfsetstrokecolor{currentstroke}%
\pgfsetdash{}{0pt}%
\pgfpathmoveto{\pgfqpoint{4.026572in}{0.417642in}}%
\pgfpathlineto{\pgfqpoint{4.026572in}{2.472642in}}%
\pgfusepath{stroke}%
\end{pgfscope}%
\begin{pgfscope}%
\pgfsetrectcap%
\pgfsetmiterjoin%
\pgfsetlinewidth{0.803000pt}%
\definecolor{currentstroke}{rgb}{0.000000,0.000000,0.000000}%
\pgfsetstrokecolor{currentstroke}%
\pgfsetdash{}{0pt}%
\pgfpathmoveto{\pgfqpoint{0.661284in}{0.417642in}}%
\pgfpathlineto{\pgfqpoint{4.026572in}{0.417642in}}%
\pgfusepath{stroke}%
\end{pgfscope}%
\begin{pgfscope}%
\pgfsetrectcap%
\pgfsetmiterjoin%
\pgfsetlinewidth{0.803000pt}%
\definecolor{currentstroke}{rgb}{0.000000,0.000000,0.000000}%
\pgfsetstrokecolor{currentstroke}%
\pgfsetdash{}{0pt}%
\pgfpathmoveto{\pgfqpoint{0.661284in}{2.472642in}}%
\pgfpathlineto{\pgfqpoint{4.026572in}{2.472642in}}%
\pgfusepath{stroke}%
\end{pgfscope}%
\begin{pgfscope}%
\pgfsetbuttcap%
\pgfsetmiterjoin%
\definecolor{currentfill}{rgb}{1.000000,1.000000,1.000000}%
\pgfsetfillcolor{currentfill}%
\pgfsetfillopacity{0.800000}%
\pgfsetlinewidth{1.003750pt}%
\definecolor{currentstroke}{rgb}{0.800000,0.800000,0.800000}%
\pgfsetstrokecolor{currentstroke}%
\pgfsetstrokeopacity{0.800000}%
\pgfsetdash{}{0pt}%
\pgfpathmoveto{\pgfqpoint{2.948460in}{2.073975in}}%
\pgfpathlineto{\pgfqpoint{3.948794in}{2.073975in}}%
\pgfpathquadraticcurveto{\pgfqpoint{3.971016in}{2.073975in}}{\pgfqpoint{3.971016in}{2.096197in}}%
\pgfpathlineto{\pgfqpoint{3.971016in}{2.394864in}}%
\pgfpathquadraticcurveto{\pgfqpoint{3.971016in}{2.417086in}}{\pgfqpoint{3.948794in}{2.417086in}}%
\pgfpathlineto{\pgfqpoint{2.948460in}{2.417086in}}%
\pgfpathquadraticcurveto{\pgfqpoint{2.926238in}{2.417086in}}{\pgfqpoint{2.926238in}{2.394864in}}%
\pgfpathlineto{\pgfqpoint{2.926238in}{2.096197in}}%
\pgfpathquadraticcurveto{\pgfqpoint{2.926238in}{2.073975in}}{\pgfqpoint{2.948460in}{2.073975in}}%
\pgfpathlineto{\pgfqpoint{2.948460in}{2.073975in}}%
\pgfpathclose%
\pgfusepath{stroke,fill}%
\end{pgfscope}%
\begin{pgfscope}%
\pgfsetbuttcap%
\pgfsetroundjoin%
\pgfsetlinewidth{1.505625pt}%
\definecolor{currentstroke}{rgb}{0.003922,0.450980,0.698039}%
\pgfsetstrokecolor{currentstroke}%
\pgfsetdash{{5.550000pt}{2.400000pt}}{0.000000pt}%
\pgfpathmoveto{\pgfqpoint{2.970683in}{2.333753in}}%
\pgfpathlineto{\pgfqpoint{3.081794in}{2.333753in}}%
\pgfpathlineto{\pgfqpoint{3.192905in}{2.333753in}}%
\pgfusepath{stroke}%
\end{pgfscope}%
\begin{pgfscope}%
\definecolor{textcolor}{rgb}{0.000000,0.000000,0.000000}%
\pgfsetstrokecolor{textcolor}%
\pgfsetfillcolor{textcolor}%
\pgftext[x=3.281794in,y=2.294864in,left,base]{\color{textcolor}\rmfamily\fontsize{8.000000}{9.600000}\selectfont White noise}%
\end{pgfscope}%
\begin{pgfscope}%
\pgfsetbuttcap%
\pgfsetroundjoin%
\pgfsetlinewidth{1.505625pt}%
\definecolor{currentstroke}{rgb}{0.007843,0.619608,0.450980}%
\pgfsetstrokecolor{currentstroke}%
\pgfsetdash{{5.550000pt}{2.400000pt}}{0.000000pt}%
\pgfpathmoveto{\pgfqpoint{2.970683in}{2.178864in}}%
\pgfpathlineto{\pgfqpoint{3.081794in}{2.178864in}}%
\pgfpathlineto{\pgfqpoint{3.192905in}{2.178864in}}%
\pgfusepath{stroke}%
\end{pgfscope}%
\begin{pgfscope}%
\definecolor{textcolor}{rgb}{0.000000,0.000000,0.000000}%
\pgfsetstrokecolor{textcolor}%
\pgfsetfillcolor{textcolor}%
\pgftext[x=3.281794in,y=2.139975in,left,base]{\color{textcolor}\rmfamily\fontsize{8.000000}{9.600000}\selectfont Flicker noise}%
\end{pgfscope}%
\end{pgfpicture}%
\makeatother%
\endgroup%
% data/simulations/sim_autozero.py
    \caption{Simulated power spectrum of a Keysight \device{3458A} containing white noise and flicker noise. The line frequency is \qty{50}{\Hz}.}
    \label{fig:autozero_raw_psd}
\end{figure}

The noise power spectral density shown in figure \ref{fig:autozero_raw_psd} is calculated from the time series given in figure \ref{fig:autozero_raw_time} and confirms the flicker and white noise content. The theoretical white noise floor is shown as a horizontal dashed blue line and the flicker noise as a dashed green line. The \qty{1.5}{\Hz} corner frequency, which is defined as the intersection between the $f^{-1}$ noise and the white noise floor can be easily identified using those lines. It is evident that the \qty{5}{\Hz} sampling frequency with a \qty{2.5}{\Hz} bandwidth does not allow the spectral density to fully settle to the white noise floor.

From the power spectral density is can be seen that higher frequencies have a significantly lower noise spectral density than low frequencies. It is therefore most beneficial to do measurements at higher frequencies. To discuss the optimal measurement interval, the Allan deviation is an excellent tool.
\begin{figure}[ht]
    \centering
    %% Creator: Matplotlib, PGF backend
%%
%% To include the figure in your LaTeX document, write
%%   \input{<filename>.pgf}
%%
%% Make sure the required packages are loaded in your preamble
%%   \usepackage{pgf}
%%
%% Also ensure that all the required font packages are loaded; for instance,
%% the lmodern package is sometimes necessary when using math font.
%%   \usepackage{lmodern}
%%
%% Figures using additional raster images can only be included by \input if
%% they are in the same directory as the main LaTeX file. For loading figures
%% from other directories you can use the `import` package
%%   \usepackage{import}
%%
%% and then include the figures with
%%   \import{<path to file>}{<filename>.pgf}
%%
%% Matplotlib used the following preamble
%%   \usepackage{siunitx}
%%   \usepackage{fontspec}
%%   \makeatletter\@ifpackageloaded{underscore}{}{\usepackage[strings]{underscore}}\makeatother
%%
\begingroup%
\makeatletter%
\begin{pgfpicture}%
\pgfpathrectangle{\pgfpointorigin}{\pgfqpoint{4.068242in}{2.514312in}}%
\pgfusepath{use as bounding box, clip}%
\begin{pgfscope}%
\pgfsetbuttcap%
\pgfsetmiterjoin%
\definecolor{currentfill}{rgb}{1.000000,1.000000,1.000000}%
\pgfsetfillcolor{currentfill}%
\pgfsetlinewidth{0.000000pt}%
\definecolor{currentstroke}{rgb}{1.000000,1.000000,1.000000}%
\pgfsetstrokecolor{currentstroke}%
\pgfsetdash{}{0pt}%
\pgfpathmoveto{\pgfqpoint{0.000000in}{0.000000in}}%
\pgfpathlineto{\pgfqpoint{4.068242in}{0.000000in}}%
\pgfpathlineto{\pgfqpoint{4.068242in}{2.514312in}}%
\pgfpathlineto{\pgfqpoint{0.000000in}{2.514312in}}%
\pgfpathlineto{\pgfqpoint{0.000000in}{0.000000in}}%
\pgfpathclose%
\pgfusepath{fill}%
\end{pgfscope}%
\begin{pgfscope}%
\pgfsetbuttcap%
\pgfsetmiterjoin%
\definecolor{currentfill}{rgb}{1.000000,1.000000,1.000000}%
\pgfsetfillcolor{currentfill}%
\pgfsetlinewidth{0.000000pt}%
\definecolor{currentstroke}{rgb}{0.000000,0.000000,0.000000}%
\pgfsetstrokecolor{currentstroke}%
\pgfsetstrokeopacity{0.000000}%
\pgfsetdash{}{0pt}%
\pgfpathmoveto{\pgfqpoint{0.770608in}{0.417642in}}%
\pgfpathlineto{\pgfqpoint{3.941404in}{0.417642in}}%
\pgfpathlineto{\pgfqpoint{3.941404in}{2.438739in}}%
\pgfpathlineto{\pgfqpoint{0.770608in}{2.438739in}}%
\pgfpathlineto{\pgfqpoint{0.770608in}{0.417642in}}%
\pgfpathclose%
\pgfusepath{fill}%
\end{pgfscope}%
\begin{pgfscope}%
\pgfpathrectangle{\pgfqpoint{0.770608in}{0.417642in}}{\pgfqpoint{3.170796in}{2.021097in}}%
\pgfusepath{clip}%
\pgfsetrectcap%
\pgfsetroundjoin%
\pgfsetlinewidth{0.803000pt}%
\definecolor{currentstroke}{rgb}{0.450000,0.450000,0.450000}%
\pgfsetstrokecolor{currentstroke}%
\pgfsetdash{}{0pt}%
\pgfpathmoveto{\pgfqpoint{0.914735in}{0.417642in}}%
\pgfpathlineto{\pgfqpoint{0.914735in}{2.438739in}}%
\pgfusepath{stroke}%
\end{pgfscope}%
\begin{pgfscope}%
\pgfsetbuttcap%
\pgfsetroundjoin%
\definecolor{currentfill}{rgb}{0.000000,0.000000,0.000000}%
\pgfsetfillcolor{currentfill}%
\pgfsetlinewidth{0.803000pt}%
\definecolor{currentstroke}{rgb}{0.000000,0.000000,0.000000}%
\pgfsetstrokecolor{currentstroke}%
\pgfsetdash{}{0pt}%
\pgfsys@defobject{currentmarker}{\pgfqpoint{0.000000in}{-0.048611in}}{\pgfqpoint{0.000000in}{0.000000in}}{%
\pgfpathmoveto{\pgfqpoint{0.000000in}{0.000000in}}%
\pgfpathlineto{\pgfqpoint{0.000000in}{-0.048611in}}%
\pgfusepath{stroke,fill}%
}%
\begin{pgfscope}%
\pgfsys@transformshift{0.914735in}{0.417642in}%
\pgfsys@useobject{currentmarker}{}%
\end{pgfscope}%
\end{pgfscope}%
\begin{pgfscope}%
\definecolor{textcolor}{rgb}{0.000000,0.000000,0.000000}%
\pgfsetstrokecolor{textcolor}%
\pgfsetfillcolor{textcolor}%
\pgftext[x=0.914735in,y=0.320420in,,top]{\color{textcolor}\rmfamily\fontsize{8.000000}{9.600000}\selectfont \(\displaystyle {10^{-1}}\)}%
\end{pgfscope}%
\begin{pgfscope}%
\pgfpathrectangle{\pgfqpoint{0.770608in}{0.417642in}}{\pgfqpoint{3.170796in}{2.021097in}}%
\pgfusepath{clip}%
\pgfsetrectcap%
\pgfsetroundjoin%
\pgfsetlinewidth{0.803000pt}%
\definecolor{currentstroke}{rgb}{0.450000,0.450000,0.450000}%
\pgfsetstrokecolor{currentstroke}%
\pgfsetdash{}{0pt}%
\pgfpathmoveto{\pgfqpoint{1.418714in}{0.417642in}}%
\pgfpathlineto{\pgfqpoint{1.418714in}{2.438739in}}%
\pgfusepath{stroke}%
\end{pgfscope}%
\begin{pgfscope}%
\pgfsetbuttcap%
\pgfsetroundjoin%
\definecolor{currentfill}{rgb}{0.000000,0.000000,0.000000}%
\pgfsetfillcolor{currentfill}%
\pgfsetlinewidth{0.803000pt}%
\definecolor{currentstroke}{rgb}{0.000000,0.000000,0.000000}%
\pgfsetstrokecolor{currentstroke}%
\pgfsetdash{}{0pt}%
\pgfsys@defobject{currentmarker}{\pgfqpoint{0.000000in}{-0.048611in}}{\pgfqpoint{0.000000in}{0.000000in}}{%
\pgfpathmoveto{\pgfqpoint{0.000000in}{0.000000in}}%
\pgfpathlineto{\pgfqpoint{0.000000in}{-0.048611in}}%
\pgfusepath{stroke,fill}%
}%
\begin{pgfscope}%
\pgfsys@transformshift{1.418714in}{0.417642in}%
\pgfsys@useobject{currentmarker}{}%
\end{pgfscope}%
\end{pgfscope}%
\begin{pgfscope}%
\definecolor{textcolor}{rgb}{0.000000,0.000000,0.000000}%
\pgfsetstrokecolor{textcolor}%
\pgfsetfillcolor{textcolor}%
\pgftext[x=1.418714in,y=0.320420in,,top]{\color{textcolor}\rmfamily\fontsize{8.000000}{9.600000}\selectfont \(\displaystyle {10^{0}}\)}%
\end{pgfscope}%
\begin{pgfscope}%
\pgfpathrectangle{\pgfqpoint{0.770608in}{0.417642in}}{\pgfqpoint{3.170796in}{2.021097in}}%
\pgfusepath{clip}%
\pgfsetrectcap%
\pgfsetroundjoin%
\pgfsetlinewidth{0.803000pt}%
\definecolor{currentstroke}{rgb}{0.450000,0.450000,0.450000}%
\pgfsetstrokecolor{currentstroke}%
\pgfsetdash{}{0pt}%
\pgfpathmoveto{\pgfqpoint{1.922693in}{0.417642in}}%
\pgfpathlineto{\pgfqpoint{1.922693in}{2.438739in}}%
\pgfusepath{stroke}%
\end{pgfscope}%
\begin{pgfscope}%
\pgfsetbuttcap%
\pgfsetroundjoin%
\definecolor{currentfill}{rgb}{0.000000,0.000000,0.000000}%
\pgfsetfillcolor{currentfill}%
\pgfsetlinewidth{0.803000pt}%
\definecolor{currentstroke}{rgb}{0.000000,0.000000,0.000000}%
\pgfsetstrokecolor{currentstroke}%
\pgfsetdash{}{0pt}%
\pgfsys@defobject{currentmarker}{\pgfqpoint{0.000000in}{-0.048611in}}{\pgfqpoint{0.000000in}{0.000000in}}{%
\pgfpathmoveto{\pgfqpoint{0.000000in}{0.000000in}}%
\pgfpathlineto{\pgfqpoint{0.000000in}{-0.048611in}}%
\pgfusepath{stroke,fill}%
}%
\begin{pgfscope}%
\pgfsys@transformshift{1.922693in}{0.417642in}%
\pgfsys@useobject{currentmarker}{}%
\end{pgfscope}%
\end{pgfscope}%
\begin{pgfscope}%
\definecolor{textcolor}{rgb}{0.000000,0.000000,0.000000}%
\pgfsetstrokecolor{textcolor}%
\pgfsetfillcolor{textcolor}%
\pgftext[x=1.922693in,y=0.320420in,,top]{\color{textcolor}\rmfamily\fontsize{8.000000}{9.600000}\selectfont \(\displaystyle {10^{1}}\)}%
\end{pgfscope}%
\begin{pgfscope}%
\pgfpathrectangle{\pgfqpoint{0.770608in}{0.417642in}}{\pgfqpoint{3.170796in}{2.021097in}}%
\pgfusepath{clip}%
\pgfsetrectcap%
\pgfsetroundjoin%
\pgfsetlinewidth{0.803000pt}%
\definecolor{currentstroke}{rgb}{0.450000,0.450000,0.450000}%
\pgfsetstrokecolor{currentstroke}%
\pgfsetdash{}{0pt}%
\pgfpathmoveto{\pgfqpoint{2.426672in}{0.417642in}}%
\pgfpathlineto{\pgfqpoint{2.426672in}{2.438739in}}%
\pgfusepath{stroke}%
\end{pgfscope}%
\begin{pgfscope}%
\pgfsetbuttcap%
\pgfsetroundjoin%
\definecolor{currentfill}{rgb}{0.000000,0.000000,0.000000}%
\pgfsetfillcolor{currentfill}%
\pgfsetlinewidth{0.803000pt}%
\definecolor{currentstroke}{rgb}{0.000000,0.000000,0.000000}%
\pgfsetstrokecolor{currentstroke}%
\pgfsetdash{}{0pt}%
\pgfsys@defobject{currentmarker}{\pgfqpoint{0.000000in}{-0.048611in}}{\pgfqpoint{0.000000in}{0.000000in}}{%
\pgfpathmoveto{\pgfqpoint{0.000000in}{0.000000in}}%
\pgfpathlineto{\pgfqpoint{0.000000in}{-0.048611in}}%
\pgfusepath{stroke,fill}%
}%
\begin{pgfscope}%
\pgfsys@transformshift{2.426672in}{0.417642in}%
\pgfsys@useobject{currentmarker}{}%
\end{pgfscope}%
\end{pgfscope}%
\begin{pgfscope}%
\definecolor{textcolor}{rgb}{0.000000,0.000000,0.000000}%
\pgfsetstrokecolor{textcolor}%
\pgfsetfillcolor{textcolor}%
\pgftext[x=2.426672in,y=0.320420in,,top]{\color{textcolor}\rmfamily\fontsize{8.000000}{9.600000}\selectfont \(\displaystyle {10^{2}}\)}%
\end{pgfscope}%
\begin{pgfscope}%
\pgfpathrectangle{\pgfqpoint{0.770608in}{0.417642in}}{\pgfqpoint{3.170796in}{2.021097in}}%
\pgfusepath{clip}%
\pgfsetrectcap%
\pgfsetroundjoin%
\pgfsetlinewidth{0.803000pt}%
\definecolor{currentstroke}{rgb}{0.450000,0.450000,0.450000}%
\pgfsetstrokecolor{currentstroke}%
\pgfsetdash{}{0pt}%
\pgfpathmoveto{\pgfqpoint{2.930651in}{0.417642in}}%
\pgfpathlineto{\pgfqpoint{2.930651in}{2.438739in}}%
\pgfusepath{stroke}%
\end{pgfscope}%
\begin{pgfscope}%
\pgfsetbuttcap%
\pgfsetroundjoin%
\definecolor{currentfill}{rgb}{0.000000,0.000000,0.000000}%
\pgfsetfillcolor{currentfill}%
\pgfsetlinewidth{0.803000pt}%
\definecolor{currentstroke}{rgb}{0.000000,0.000000,0.000000}%
\pgfsetstrokecolor{currentstroke}%
\pgfsetdash{}{0pt}%
\pgfsys@defobject{currentmarker}{\pgfqpoint{0.000000in}{-0.048611in}}{\pgfqpoint{0.000000in}{0.000000in}}{%
\pgfpathmoveto{\pgfqpoint{0.000000in}{0.000000in}}%
\pgfpathlineto{\pgfqpoint{0.000000in}{-0.048611in}}%
\pgfusepath{stroke,fill}%
}%
\begin{pgfscope}%
\pgfsys@transformshift{2.930651in}{0.417642in}%
\pgfsys@useobject{currentmarker}{}%
\end{pgfscope}%
\end{pgfscope}%
\begin{pgfscope}%
\definecolor{textcolor}{rgb}{0.000000,0.000000,0.000000}%
\pgfsetstrokecolor{textcolor}%
\pgfsetfillcolor{textcolor}%
\pgftext[x=2.930651in,y=0.320420in,,top]{\color{textcolor}\rmfamily\fontsize{8.000000}{9.600000}\selectfont \(\displaystyle {10^{3}}\)}%
\end{pgfscope}%
\begin{pgfscope}%
\pgfpathrectangle{\pgfqpoint{0.770608in}{0.417642in}}{\pgfqpoint{3.170796in}{2.021097in}}%
\pgfusepath{clip}%
\pgfsetrectcap%
\pgfsetroundjoin%
\pgfsetlinewidth{0.803000pt}%
\definecolor{currentstroke}{rgb}{0.450000,0.450000,0.450000}%
\pgfsetstrokecolor{currentstroke}%
\pgfsetdash{}{0pt}%
\pgfpathmoveto{\pgfqpoint{3.434629in}{0.417642in}}%
\pgfpathlineto{\pgfqpoint{3.434629in}{2.438739in}}%
\pgfusepath{stroke}%
\end{pgfscope}%
\begin{pgfscope}%
\pgfsetbuttcap%
\pgfsetroundjoin%
\definecolor{currentfill}{rgb}{0.000000,0.000000,0.000000}%
\pgfsetfillcolor{currentfill}%
\pgfsetlinewidth{0.803000pt}%
\definecolor{currentstroke}{rgb}{0.000000,0.000000,0.000000}%
\pgfsetstrokecolor{currentstroke}%
\pgfsetdash{}{0pt}%
\pgfsys@defobject{currentmarker}{\pgfqpoint{0.000000in}{-0.048611in}}{\pgfqpoint{0.000000in}{0.000000in}}{%
\pgfpathmoveto{\pgfqpoint{0.000000in}{0.000000in}}%
\pgfpathlineto{\pgfqpoint{0.000000in}{-0.048611in}}%
\pgfusepath{stroke,fill}%
}%
\begin{pgfscope}%
\pgfsys@transformshift{3.434629in}{0.417642in}%
\pgfsys@useobject{currentmarker}{}%
\end{pgfscope}%
\end{pgfscope}%
\begin{pgfscope}%
\definecolor{textcolor}{rgb}{0.000000,0.000000,0.000000}%
\pgfsetstrokecolor{textcolor}%
\pgfsetfillcolor{textcolor}%
\pgftext[x=3.434629in,y=0.320420in,,top]{\color{textcolor}\rmfamily\fontsize{8.000000}{9.600000}\selectfont \(\displaystyle {10^{4}}\)}%
\end{pgfscope}%
\begin{pgfscope}%
\pgfpathrectangle{\pgfqpoint{0.770608in}{0.417642in}}{\pgfqpoint{3.170796in}{2.021097in}}%
\pgfusepath{clip}%
\pgfsetrectcap%
\pgfsetroundjoin%
\pgfsetlinewidth{0.803000pt}%
\definecolor{currentstroke}{rgb}{0.450000,0.450000,0.450000}%
\pgfsetstrokecolor{currentstroke}%
\pgfsetdash{}{0pt}%
\pgfpathmoveto{\pgfqpoint{3.938608in}{0.417642in}}%
\pgfpathlineto{\pgfqpoint{3.938608in}{2.438739in}}%
\pgfusepath{stroke}%
\end{pgfscope}%
\begin{pgfscope}%
\pgfsetbuttcap%
\pgfsetroundjoin%
\definecolor{currentfill}{rgb}{0.000000,0.000000,0.000000}%
\pgfsetfillcolor{currentfill}%
\pgfsetlinewidth{0.803000pt}%
\definecolor{currentstroke}{rgb}{0.000000,0.000000,0.000000}%
\pgfsetstrokecolor{currentstroke}%
\pgfsetdash{}{0pt}%
\pgfsys@defobject{currentmarker}{\pgfqpoint{0.000000in}{-0.048611in}}{\pgfqpoint{0.000000in}{0.000000in}}{%
\pgfpathmoveto{\pgfqpoint{0.000000in}{0.000000in}}%
\pgfpathlineto{\pgfqpoint{0.000000in}{-0.048611in}}%
\pgfusepath{stroke,fill}%
}%
\begin{pgfscope}%
\pgfsys@transformshift{3.938608in}{0.417642in}%
\pgfsys@useobject{currentmarker}{}%
\end{pgfscope}%
\end{pgfscope}%
\begin{pgfscope}%
\definecolor{textcolor}{rgb}{0.000000,0.000000,0.000000}%
\pgfsetstrokecolor{textcolor}%
\pgfsetfillcolor{textcolor}%
\pgftext[x=3.938608in,y=0.320420in,,top]{\color{textcolor}\rmfamily\fontsize{8.000000}{9.600000}\selectfont \(\displaystyle {10^{5}}\)}%
\end{pgfscope}%
\begin{pgfscope}%
\pgfpathrectangle{\pgfqpoint{0.770608in}{0.417642in}}{\pgfqpoint{3.170796in}{2.021097in}}%
\pgfusepath{clip}%
\pgfsetrectcap%
\pgfsetroundjoin%
\pgfsetlinewidth{0.803000pt}%
\definecolor{currentstroke}{rgb}{0.850000,0.850000,0.850000}%
\pgfsetstrokecolor{currentstroke}%
\pgfsetdash{}{0pt}%
\pgfpathmoveto{\pgfqpoint{0.802928in}{0.417642in}}%
\pgfpathlineto{\pgfqpoint{0.802928in}{2.438739in}}%
\pgfusepath{stroke}%
\end{pgfscope}%
\begin{pgfscope}%
\pgfsetbuttcap%
\pgfsetroundjoin%
\definecolor{currentfill}{rgb}{0.000000,0.000000,0.000000}%
\pgfsetfillcolor{currentfill}%
\pgfsetlinewidth{0.602250pt}%
\definecolor{currentstroke}{rgb}{0.000000,0.000000,0.000000}%
\pgfsetstrokecolor{currentstroke}%
\pgfsetdash{}{0pt}%
\pgfsys@defobject{currentmarker}{\pgfqpoint{0.000000in}{-0.027778in}}{\pgfqpoint{0.000000in}{0.000000in}}{%
\pgfpathmoveto{\pgfqpoint{0.000000in}{0.000000in}}%
\pgfpathlineto{\pgfqpoint{0.000000in}{-0.027778in}}%
\pgfusepath{stroke,fill}%
}%
\begin{pgfscope}%
\pgfsys@transformshift{0.802928in}{0.417642in}%
\pgfsys@useobject{currentmarker}{}%
\end{pgfscope}%
\end{pgfscope}%
\begin{pgfscope}%
\pgfpathrectangle{\pgfqpoint{0.770608in}{0.417642in}}{\pgfqpoint{3.170796in}{2.021097in}}%
\pgfusepath{clip}%
\pgfsetrectcap%
\pgfsetroundjoin%
\pgfsetlinewidth{0.803000pt}%
\definecolor{currentstroke}{rgb}{0.850000,0.850000,0.850000}%
\pgfsetstrokecolor{currentstroke}%
\pgfsetdash{}{0pt}%
\pgfpathmoveto{\pgfqpoint{0.836668in}{0.417642in}}%
\pgfpathlineto{\pgfqpoint{0.836668in}{2.438739in}}%
\pgfusepath{stroke}%
\end{pgfscope}%
\begin{pgfscope}%
\pgfsetbuttcap%
\pgfsetroundjoin%
\definecolor{currentfill}{rgb}{0.000000,0.000000,0.000000}%
\pgfsetfillcolor{currentfill}%
\pgfsetlinewidth{0.602250pt}%
\definecolor{currentstroke}{rgb}{0.000000,0.000000,0.000000}%
\pgfsetstrokecolor{currentstroke}%
\pgfsetdash{}{0pt}%
\pgfsys@defobject{currentmarker}{\pgfqpoint{0.000000in}{-0.027778in}}{\pgfqpoint{0.000000in}{0.000000in}}{%
\pgfpathmoveto{\pgfqpoint{0.000000in}{0.000000in}}%
\pgfpathlineto{\pgfqpoint{0.000000in}{-0.027778in}}%
\pgfusepath{stroke,fill}%
}%
\begin{pgfscope}%
\pgfsys@transformshift{0.836668in}{0.417642in}%
\pgfsys@useobject{currentmarker}{}%
\end{pgfscope}%
\end{pgfscope}%
\begin{pgfscope}%
\pgfpathrectangle{\pgfqpoint{0.770608in}{0.417642in}}{\pgfqpoint{3.170796in}{2.021097in}}%
\pgfusepath{clip}%
\pgfsetrectcap%
\pgfsetroundjoin%
\pgfsetlinewidth{0.803000pt}%
\definecolor{currentstroke}{rgb}{0.850000,0.850000,0.850000}%
\pgfsetstrokecolor{currentstroke}%
\pgfsetdash{}{0pt}%
\pgfpathmoveto{\pgfqpoint{0.865895in}{0.417642in}}%
\pgfpathlineto{\pgfqpoint{0.865895in}{2.438739in}}%
\pgfusepath{stroke}%
\end{pgfscope}%
\begin{pgfscope}%
\pgfsetbuttcap%
\pgfsetroundjoin%
\definecolor{currentfill}{rgb}{0.000000,0.000000,0.000000}%
\pgfsetfillcolor{currentfill}%
\pgfsetlinewidth{0.602250pt}%
\definecolor{currentstroke}{rgb}{0.000000,0.000000,0.000000}%
\pgfsetstrokecolor{currentstroke}%
\pgfsetdash{}{0pt}%
\pgfsys@defobject{currentmarker}{\pgfqpoint{0.000000in}{-0.027778in}}{\pgfqpoint{0.000000in}{0.000000in}}{%
\pgfpathmoveto{\pgfqpoint{0.000000in}{0.000000in}}%
\pgfpathlineto{\pgfqpoint{0.000000in}{-0.027778in}}%
\pgfusepath{stroke,fill}%
}%
\begin{pgfscope}%
\pgfsys@transformshift{0.865895in}{0.417642in}%
\pgfsys@useobject{currentmarker}{}%
\end{pgfscope}%
\end{pgfscope}%
\begin{pgfscope}%
\pgfpathrectangle{\pgfqpoint{0.770608in}{0.417642in}}{\pgfqpoint{3.170796in}{2.021097in}}%
\pgfusepath{clip}%
\pgfsetrectcap%
\pgfsetroundjoin%
\pgfsetlinewidth{0.803000pt}%
\definecolor{currentstroke}{rgb}{0.850000,0.850000,0.850000}%
\pgfsetstrokecolor{currentstroke}%
\pgfsetdash{}{0pt}%
\pgfpathmoveto{\pgfqpoint{0.891675in}{0.417642in}}%
\pgfpathlineto{\pgfqpoint{0.891675in}{2.438739in}}%
\pgfusepath{stroke}%
\end{pgfscope}%
\begin{pgfscope}%
\pgfsetbuttcap%
\pgfsetroundjoin%
\definecolor{currentfill}{rgb}{0.000000,0.000000,0.000000}%
\pgfsetfillcolor{currentfill}%
\pgfsetlinewidth{0.602250pt}%
\definecolor{currentstroke}{rgb}{0.000000,0.000000,0.000000}%
\pgfsetstrokecolor{currentstroke}%
\pgfsetdash{}{0pt}%
\pgfsys@defobject{currentmarker}{\pgfqpoint{0.000000in}{-0.027778in}}{\pgfqpoint{0.000000in}{0.000000in}}{%
\pgfpathmoveto{\pgfqpoint{0.000000in}{0.000000in}}%
\pgfpathlineto{\pgfqpoint{0.000000in}{-0.027778in}}%
\pgfusepath{stroke,fill}%
}%
\begin{pgfscope}%
\pgfsys@transformshift{0.891675in}{0.417642in}%
\pgfsys@useobject{currentmarker}{}%
\end{pgfscope}%
\end{pgfscope}%
\begin{pgfscope}%
\pgfpathrectangle{\pgfqpoint{0.770608in}{0.417642in}}{\pgfqpoint{3.170796in}{2.021097in}}%
\pgfusepath{clip}%
\pgfsetrectcap%
\pgfsetroundjoin%
\pgfsetlinewidth{0.803000pt}%
\definecolor{currentstroke}{rgb}{0.850000,0.850000,0.850000}%
\pgfsetstrokecolor{currentstroke}%
\pgfsetdash{}{0pt}%
\pgfpathmoveto{\pgfqpoint{1.066448in}{0.417642in}}%
\pgfpathlineto{\pgfqpoint{1.066448in}{2.438739in}}%
\pgfusepath{stroke}%
\end{pgfscope}%
\begin{pgfscope}%
\pgfsetbuttcap%
\pgfsetroundjoin%
\definecolor{currentfill}{rgb}{0.000000,0.000000,0.000000}%
\pgfsetfillcolor{currentfill}%
\pgfsetlinewidth{0.602250pt}%
\definecolor{currentstroke}{rgb}{0.000000,0.000000,0.000000}%
\pgfsetstrokecolor{currentstroke}%
\pgfsetdash{}{0pt}%
\pgfsys@defobject{currentmarker}{\pgfqpoint{0.000000in}{-0.027778in}}{\pgfqpoint{0.000000in}{0.000000in}}{%
\pgfpathmoveto{\pgfqpoint{0.000000in}{0.000000in}}%
\pgfpathlineto{\pgfqpoint{0.000000in}{-0.027778in}}%
\pgfusepath{stroke,fill}%
}%
\begin{pgfscope}%
\pgfsys@transformshift{1.066448in}{0.417642in}%
\pgfsys@useobject{currentmarker}{}%
\end{pgfscope}%
\end{pgfscope}%
\begin{pgfscope}%
\pgfpathrectangle{\pgfqpoint{0.770608in}{0.417642in}}{\pgfqpoint{3.170796in}{2.021097in}}%
\pgfusepath{clip}%
\pgfsetrectcap%
\pgfsetroundjoin%
\pgfsetlinewidth{0.803000pt}%
\definecolor{currentstroke}{rgb}{0.850000,0.850000,0.850000}%
\pgfsetstrokecolor{currentstroke}%
\pgfsetdash{}{0pt}%
\pgfpathmoveto{\pgfqpoint{1.155194in}{0.417642in}}%
\pgfpathlineto{\pgfqpoint{1.155194in}{2.438739in}}%
\pgfusepath{stroke}%
\end{pgfscope}%
\begin{pgfscope}%
\pgfsetbuttcap%
\pgfsetroundjoin%
\definecolor{currentfill}{rgb}{0.000000,0.000000,0.000000}%
\pgfsetfillcolor{currentfill}%
\pgfsetlinewidth{0.602250pt}%
\definecolor{currentstroke}{rgb}{0.000000,0.000000,0.000000}%
\pgfsetstrokecolor{currentstroke}%
\pgfsetdash{}{0pt}%
\pgfsys@defobject{currentmarker}{\pgfqpoint{0.000000in}{-0.027778in}}{\pgfqpoint{0.000000in}{0.000000in}}{%
\pgfpathmoveto{\pgfqpoint{0.000000in}{0.000000in}}%
\pgfpathlineto{\pgfqpoint{0.000000in}{-0.027778in}}%
\pgfusepath{stroke,fill}%
}%
\begin{pgfscope}%
\pgfsys@transformshift{1.155194in}{0.417642in}%
\pgfsys@useobject{currentmarker}{}%
\end{pgfscope}%
\end{pgfscope}%
\begin{pgfscope}%
\pgfpathrectangle{\pgfqpoint{0.770608in}{0.417642in}}{\pgfqpoint{3.170796in}{2.021097in}}%
\pgfusepath{clip}%
\pgfsetrectcap%
\pgfsetroundjoin%
\pgfsetlinewidth{0.803000pt}%
\definecolor{currentstroke}{rgb}{0.850000,0.850000,0.850000}%
\pgfsetstrokecolor{currentstroke}%
\pgfsetdash{}{0pt}%
\pgfpathmoveto{\pgfqpoint{1.218161in}{0.417642in}}%
\pgfpathlineto{\pgfqpoint{1.218161in}{2.438739in}}%
\pgfusepath{stroke}%
\end{pgfscope}%
\begin{pgfscope}%
\pgfsetbuttcap%
\pgfsetroundjoin%
\definecolor{currentfill}{rgb}{0.000000,0.000000,0.000000}%
\pgfsetfillcolor{currentfill}%
\pgfsetlinewidth{0.602250pt}%
\definecolor{currentstroke}{rgb}{0.000000,0.000000,0.000000}%
\pgfsetstrokecolor{currentstroke}%
\pgfsetdash{}{0pt}%
\pgfsys@defobject{currentmarker}{\pgfqpoint{0.000000in}{-0.027778in}}{\pgfqpoint{0.000000in}{0.000000in}}{%
\pgfpathmoveto{\pgfqpoint{0.000000in}{0.000000in}}%
\pgfpathlineto{\pgfqpoint{0.000000in}{-0.027778in}}%
\pgfusepath{stroke,fill}%
}%
\begin{pgfscope}%
\pgfsys@transformshift{1.218161in}{0.417642in}%
\pgfsys@useobject{currentmarker}{}%
\end{pgfscope}%
\end{pgfscope}%
\begin{pgfscope}%
\pgfpathrectangle{\pgfqpoint{0.770608in}{0.417642in}}{\pgfqpoint{3.170796in}{2.021097in}}%
\pgfusepath{clip}%
\pgfsetrectcap%
\pgfsetroundjoin%
\pgfsetlinewidth{0.803000pt}%
\definecolor{currentstroke}{rgb}{0.850000,0.850000,0.850000}%
\pgfsetstrokecolor{currentstroke}%
\pgfsetdash{}{0pt}%
\pgfpathmoveto{\pgfqpoint{1.267001in}{0.417642in}}%
\pgfpathlineto{\pgfqpoint{1.267001in}{2.438739in}}%
\pgfusepath{stroke}%
\end{pgfscope}%
\begin{pgfscope}%
\pgfsetbuttcap%
\pgfsetroundjoin%
\definecolor{currentfill}{rgb}{0.000000,0.000000,0.000000}%
\pgfsetfillcolor{currentfill}%
\pgfsetlinewidth{0.602250pt}%
\definecolor{currentstroke}{rgb}{0.000000,0.000000,0.000000}%
\pgfsetstrokecolor{currentstroke}%
\pgfsetdash{}{0pt}%
\pgfsys@defobject{currentmarker}{\pgfqpoint{0.000000in}{-0.027778in}}{\pgfqpoint{0.000000in}{0.000000in}}{%
\pgfpathmoveto{\pgfqpoint{0.000000in}{0.000000in}}%
\pgfpathlineto{\pgfqpoint{0.000000in}{-0.027778in}}%
\pgfusepath{stroke,fill}%
}%
\begin{pgfscope}%
\pgfsys@transformshift{1.267001in}{0.417642in}%
\pgfsys@useobject{currentmarker}{}%
\end{pgfscope}%
\end{pgfscope}%
\begin{pgfscope}%
\pgfpathrectangle{\pgfqpoint{0.770608in}{0.417642in}}{\pgfqpoint{3.170796in}{2.021097in}}%
\pgfusepath{clip}%
\pgfsetrectcap%
\pgfsetroundjoin%
\pgfsetlinewidth{0.803000pt}%
\definecolor{currentstroke}{rgb}{0.850000,0.850000,0.850000}%
\pgfsetstrokecolor{currentstroke}%
\pgfsetdash{}{0pt}%
\pgfpathmoveto{\pgfqpoint{1.306907in}{0.417642in}}%
\pgfpathlineto{\pgfqpoint{1.306907in}{2.438739in}}%
\pgfusepath{stroke}%
\end{pgfscope}%
\begin{pgfscope}%
\pgfsetbuttcap%
\pgfsetroundjoin%
\definecolor{currentfill}{rgb}{0.000000,0.000000,0.000000}%
\pgfsetfillcolor{currentfill}%
\pgfsetlinewidth{0.602250pt}%
\definecolor{currentstroke}{rgb}{0.000000,0.000000,0.000000}%
\pgfsetstrokecolor{currentstroke}%
\pgfsetdash{}{0pt}%
\pgfsys@defobject{currentmarker}{\pgfqpoint{0.000000in}{-0.027778in}}{\pgfqpoint{0.000000in}{0.000000in}}{%
\pgfpathmoveto{\pgfqpoint{0.000000in}{0.000000in}}%
\pgfpathlineto{\pgfqpoint{0.000000in}{-0.027778in}}%
\pgfusepath{stroke,fill}%
}%
\begin{pgfscope}%
\pgfsys@transformshift{1.306907in}{0.417642in}%
\pgfsys@useobject{currentmarker}{}%
\end{pgfscope}%
\end{pgfscope}%
\begin{pgfscope}%
\pgfpathrectangle{\pgfqpoint{0.770608in}{0.417642in}}{\pgfqpoint{3.170796in}{2.021097in}}%
\pgfusepath{clip}%
\pgfsetrectcap%
\pgfsetroundjoin%
\pgfsetlinewidth{0.803000pt}%
\definecolor{currentstroke}{rgb}{0.850000,0.850000,0.850000}%
\pgfsetstrokecolor{currentstroke}%
\pgfsetdash{}{0pt}%
\pgfpathmoveto{\pgfqpoint{1.340647in}{0.417642in}}%
\pgfpathlineto{\pgfqpoint{1.340647in}{2.438739in}}%
\pgfusepath{stroke}%
\end{pgfscope}%
\begin{pgfscope}%
\pgfsetbuttcap%
\pgfsetroundjoin%
\definecolor{currentfill}{rgb}{0.000000,0.000000,0.000000}%
\pgfsetfillcolor{currentfill}%
\pgfsetlinewidth{0.602250pt}%
\definecolor{currentstroke}{rgb}{0.000000,0.000000,0.000000}%
\pgfsetstrokecolor{currentstroke}%
\pgfsetdash{}{0pt}%
\pgfsys@defobject{currentmarker}{\pgfqpoint{0.000000in}{-0.027778in}}{\pgfqpoint{0.000000in}{0.000000in}}{%
\pgfpathmoveto{\pgfqpoint{0.000000in}{0.000000in}}%
\pgfpathlineto{\pgfqpoint{0.000000in}{-0.027778in}}%
\pgfusepath{stroke,fill}%
}%
\begin{pgfscope}%
\pgfsys@transformshift{1.340647in}{0.417642in}%
\pgfsys@useobject{currentmarker}{}%
\end{pgfscope}%
\end{pgfscope}%
\begin{pgfscope}%
\pgfpathrectangle{\pgfqpoint{0.770608in}{0.417642in}}{\pgfqpoint{3.170796in}{2.021097in}}%
\pgfusepath{clip}%
\pgfsetrectcap%
\pgfsetroundjoin%
\pgfsetlinewidth{0.803000pt}%
\definecolor{currentstroke}{rgb}{0.850000,0.850000,0.850000}%
\pgfsetstrokecolor{currentstroke}%
\pgfsetdash{}{0pt}%
\pgfpathmoveto{\pgfqpoint{1.369874in}{0.417642in}}%
\pgfpathlineto{\pgfqpoint{1.369874in}{2.438739in}}%
\pgfusepath{stroke}%
\end{pgfscope}%
\begin{pgfscope}%
\pgfsetbuttcap%
\pgfsetroundjoin%
\definecolor{currentfill}{rgb}{0.000000,0.000000,0.000000}%
\pgfsetfillcolor{currentfill}%
\pgfsetlinewidth{0.602250pt}%
\definecolor{currentstroke}{rgb}{0.000000,0.000000,0.000000}%
\pgfsetstrokecolor{currentstroke}%
\pgfsetdash{}{0pt}%
\pgfsys@defobject{currentmarker}{\pgfqpoint{0.000000in}{-0.027778in}}{\pgfqpoint{0.000000in}{0.000000in}}{%
\pgfpathmoveto{\pgfqpoint{0.000000in}{0.000000in}}%
\pgfpathlineto{\pgfqpoint{0.000000in}{-0.027778in}}%
\pgfusepath{stroke,fill}%
}%
\begin{pgfscope}%
\pgfsys@transformshift{1.369874in}{0.417642in}%
\pgfsys@useobject{currentmarker}{}%
\end{pgfscope}%
\end{pgfscope}%
\begin{pgfscope}%
\pgfpathrectangle{\pgfqpoint{0.770608in}{0.417642in}}{\pgfqpoint{3.170796in}{2.021097in}}%
\pgfusepath{clip}%
\pgfsetrectcap%
\pgfsetroundjoin%
\pgfsetlinewidth{0.803000pt}%
\definecolor{currentstroke}{rgb}{0.850000,0.850000,0.850000}%
\pgfsetstrokecolor{currentstroke}%
\pgfsetdash{}{0pt}%
\pgfpathmoveto{\pgfqpoint{1.395653in}{0.417642in}}%
\pgfpathlineto{\pgfqpoint{1.395653in}{2.438739in}}%
\pgfusepath{stroke}%
\end{pgfscope}%
\begin{pgfscope}%
\pgfsetbuttcap%
\pgfsetroundjoin%
\definecolor{currentfill}{rgb}{0.000000,0.000000,0.000000}%
\pgfsetfillcolor{currentfill}%
\pgfsetlinewidth{0.602250pt}%
\definecolor{currentstroke}{rgb}{0.000000,0.000000,0.000000}%
\pgfsetstrokecolor{currentstroke}%
\pgfsetdash{}{0pt}%
\pgfsys@defobject{currentmarker}{\pgfqpoint{0.000000in}{-0.027778in}}{\pgfqpoint{0.000000in}{0.000000in}}{%
\pgfpathmoveto{\pgfqpoint{0.000000in}{0.000000in}}%
\pgfpathlineto{\pgfqpoint{0.000000in}{-0.027778in}}%
\pgfusepath{stroke,fill}%
}%
\begin{pgfscope}%
\pgfsys@transformshift{1.395653in}{0.417642in}%
\pgfsys@useobject{currentmarker}{}%
\end{pgfscope}%
\end{pgfscope}%
\begin{pgfscope}%
\pgfpathrectangle{\pgfqpoint{0.770608in}{0.417642in}}{\pgfqpoint{3.170796in}{2.021097in}}%
\pgfusepath{clip}%
\pgfsetrectcap%
\pgfsetroundjoin%
\pgfsetlinewidth{0.803000pt}%
\definecolor{currentstroke}{rgb}{0.850000,0.850000,0.850000}%
\pgfsetstrokecolor{currentstroke}%
\pgfsetdash{}{0pt}%
\pgfpathmoveto{\pgfqpoint{1.570427in}{0.417642in}}%
\pgfpathlineto{\pgfqpoint{1.570427in}{2.438739in}}%
\pgfusepath{stroke}%
\end{pgfscope}%
\begin{pgfscope}%
\pgfsetbuttcap%
\pgfsetroundjoin%
\definecolor{currentfill}{rgb}{0.000000,0.000000,0.000000}%
\pgfsetfillcolor{currentfill}%
\pgfsetlinewidth{0.602250pt}%
\definecolor{currentstroke}{rgb}{0.000000,0.000000,0.000000}%
\pgfsetstrokecolor{currentstroke}%
\pgfsetdash{}{0pt}%
\pgfsys@defobject{currentmarker}{\pgfqpoint{0.000000in}{-0.027778in}}{\pgfqpoint{0.000000in}{0.000000in}}{%
\pgfpathmoveto{\pgfqpoint{0.000000in}{0.000000in}}%
\pgfpathlineto{\pgfqpoint{0.000000in}{-0.027778in}}%
\pgfusepath{stroke,fill}%
}%
\begin{pgfscope}%
\pgfsys@transformshift{1.570427in}{0.417642in}%
\pgfsys@useobject{currentmarker}{}%
\end{pgfscope}%
\end{pgfscope}%
\begin{pgfscope}%
\pgfpathrectangle{\pgfqpoint{0.770608in}{0.417642in}}{\pgfqpoint{3.170796in}{2.021097in}}%
\pgfusepath{clip}%
\pgfsetrectcap%
\pgfsetroundjoin%
\pgfsetlinewidth{0.803000pt}%
\definecolor{currentstroke}{rgb}{0.850000,0.850000,0.850000}%
\pgfsetstrokecolor{currentstroke}%
\pgfsetdash{}{0pt}%
\pgfpathmoveto{\pgfqpoint{1.659173in}{0.417642in}}%
\pgfpathlineto{\pgfqpoint{1.659173in}{2.438739in}}%
\pgfusepath{stroke}%
\end{pgfscope}%
\begin{pgfscope}%
\pgfsetbuttcap%
\pgfsetroundjoin%
\definecolor{currentfill}{rgb}{0.000000,0.000000,0.000000}%
\pgfsetfillcolor{currentfill}%
\pgfsetlinewidth{0.602250pt}%
\definecolor{currentstroke}{rgb}{0.000000,0.000000,0.000000}%
\pgfsetstrokecolor{currentstroke}%
\pgfsetdash{}{0pt}%
\pgfsys@defobject{currentmarker}{\pgfqpoint{0.000000in}{-0.027778in}}{\pgfqpoint{0.000000in}{0.000000in}}{%
\pgfpathmoveto{\pgfqpoint{0.000000in}{0.000000in}}%
\pgfpathlineto{\pgfqpoint{0.000000in}{-0.027778in}}%
\pgfusepath{stroke,fill}%
}%
\begin{pgfscope}%
\pgfsys@transformshift{1.659173in}{0.417642in}%
\pgfsys@useobject{currentmarker}{}%
\end{pgfscope}%
\end{pgfscope}%
\begin{pgfscope}%
\pgfpathrectangle{\pgfqpoint{0.770608in}{0.417642in}}{\pgfqpoint{3.170796in}{2.021097in}}%
\pgfusepath{clip}%
\pgfsetrectcap%
\pgfsetroundjoin%
\pgfsetlinewidth{0.803000pt}%
\definecolor{currentstroke}{rgb}{0.850000,0.850000,0.850000}%
\pgfsetstrokecolor{currentstroke}%
\pgfsetdash{}{0pt}%
\pgfpathmoveto{\pgfqpoint{1.722140in}{0.417642in}}%
\pgfpathlineto{\pgfqpoint{1.722140in}{2.438739in}}%
\pgfusepath{stroke}%
\end{pgfscope}%
\begin{pgfscope}%
\pgfsetbuttcap%
\pgfsetroundjoin%
\definecolor{currentfill}{rgb}{0.000000,0.000000,0.000000}%
\pgfsetfillcolor{currentfill}%
\pgfsetlinewidth{0.602250pt}%
\definecolor{currentstroke}{rgb}{0.000000,0.000000,0.000000}%
\pgfsetstrokecolor{currentstroke}%
\pgfsetdash{}{0pt}%
\pgfsys@defobject{currentmarker}{\pgfqpoint{0.000000in}{-0.027778in}}{\pgfqpoint{0.000000in}{0.000000in}}{%
\pgfpathmoveto{\pgfqpoint{0.000000in}{0.000000in}}%
\pgfpathlineto{\pgfqpoint{0.000000in}{-0.027778in}}%
\pgfusepath{stroke,fill}%
}%
\begin{pgfscope}%
\pgfsys@transformshift{1.722140in}{0.417642in}%
\pgfsys@useobject{currentmarker}{}%
\end{pgfscope}%
\end{pgfscope}%
\begin{pgfscope}%
\pgfpathrectangle{\pgfqpoint{0.770608in}{0.417642in}}{\pgfqpoint{3.170796in}{2.021097in}}%
\pgfusepath{clip}%
\pgfsetrectcap%
\pgfsetroundjoin%
\pgfsetlinewidth{0.803000pt}%
\definecolor{currentstroke}{rgb}{0.850000,0.850000,0.850000}%
\pgfsetstrokecolor{currentstroke}%
\pgfsetdash{}{0pt}%
\pgfpathmoveto{\pgfqpoint{1.770980in}{0.417642in}}%
\pgfpathlineto{\pgfqpoint{1.770980in}{2.438739in}}%
\pgfusepath{stroke}%
\end{pgfscope}%
\begin{pgfscope}%
\pgfsetbuttcap%
\pgfsetroundjoin%
\definecolor{currentfill}{rgb}{0.000000,0.000000,0.000000}%
\pgfsetfillcolor{currentfill}%
\pgfsetlinewidth{0.602250pt}%
\definecolor{currentstroke}{rgb}{0.000000,0.000000,0.000000}%
\pgfsetstrokecolor{currentstroke}%
\pgfsetdash{}{0pt}%
\pgfsys@defobject{currentmarker}{\pgfqpoint{0.000000in}{-0.027778in}}{\pgfqpoint{0.000000in}{0.000000in}}{%
\pgfpathmoveto{\pgfqpoint{0.000000in}{0.000000in}}%
\pgfpathlineto{\pgfqpoint{0.000000in}{-0.027778in}}%
\pgfusepath{stroke,fill}%
}%
\begin{pgfscope}%
\pgfsys@transformshift{1.770980in}{0.417642in}%
\pgfsys@useobject{currentmarker}{}%
\end{pgfscope}%
\end{pgfscope}%
\begin{pgfscope}%
\pgfpathrectangle{\pgfqpoint{0.770608in}{0.417642in}}{\pgfqpoint{3.170796in}{2.021097in}}%
\pgfusepath{clip}%
\pgfsetrectcap%
\pgfsetroundjoin%
\pgfsetlinewidth{0.803000pt}%
\definecolor{currentstroke}{rgb}{0.850000,0.850000,0.850000}%
\pgfsetstrokecolor{currentstroke}%
\pgfsetdash{}{0pt}%
\pgfpathmoveto{\pgfqpoint{1.810886in}{0.417642in}}%
\pgfpathlineto{\pgfqpoint{1.810886in}{2.438739in}}%
\pgfusepath{stroke}%
\end{pgfscope}%
\begin{pgfscope}%
\pgfsetbuttcap%
\pgfsetroundjoin%
\definecolor{currentfill}{rgb}{0.000000,0.000000,0.000000}%
\pgfsetfillcolor{currentfill}%
\pgfsetlinewidth{0.602250pt}%
\definecolor{currentstroke}{rgb}{0.000000,0.000000,0.000000}%
\pgfsetstrokecolor{currentstroke}%
\pgfsetdash{}{0pt}%
\pgfsys@defobject{currentmarker}{\pgfqpoint{0.000000in}{-0.027778in}}{\pgfqpoint{0.000000in}{0.000000in}}{%
\pgfpathmoveto{\pgfqpoint{0.000000in}{0.000000in}}%
\pgfpathlineto{\pgfqpoint{0.000000in}{-0.027778in}}%
\pgfusepath{stroke,fill}%
}%
\begin{pgfscope}%
\pgfsys@transformshift{1.810886in}{0.417642in}%
\pgfsys@useobject{currentmarker}{}%
\end{pgfscope}%
\end{pgfscope}%
\begin{pgfscope}%
\pgfpathrectangle{\pgfqpoint{0.770608in}{0.417642in}}{\pgfqpoint{3.170796in}{2.021097in}}%
\pgfusepath{clip}%
\pgfsetrectcap%
\pgfsetroundjoin%
\pgfsetlinewidth{0.803000pt}%
\definecolor{currentstroke}{rgb}{0.850000,0.850000,0.850000}%
\pgfsetstrokecolor{currentstroke}%
\pgfsetdash{}{0pt}%
\pgfpathmoveto{\pgfqpoint{1.844626in}{0.417642in}}%
\pgfpathlineto{\pgfqpoint{1.844626in}{2.438739in}}%
\pgfusepath{stroke}%
\end{pgfscope}%
\begin{pgfscope}%
\pgfsetbuttcap%
\pgfsetroundjoin%
\definecolor{currentfill}{rgb}{0.000000,0.000000,0.000000}%
\pgfsetfillcolor{currentfill}%
\pgfsetlinewidth{0.602250pt}%
\definecolor{currentstroke}{rgb}{0.000000,0.000000,0.000000}%
\pgfsetstrokecolor{currentstroke}%
\pgfsetdash{}{0pt}%
\pgfsys@defobject{currentmarker}{\pgfqpoint{0.000000in}{-0.027778in}}{\pgfqpoint{0.000000in}{0.000000in}}{%
\pgfpathmoveto{\pgfqpoint{0.000000in}{0.000000in}}%
\pgfpathlineto{\pgfqpoint{0.000000in}{-0.027778in}}%
\pgfusepath{stroke,fill}%
}%
\begin{pgfscope}%
\pgfsys@transformshift{1.844626in}{0.417642in}%
\pgfsys@useobject{currentmarker}{}%
\end{pgfscope}%
\end{pgfscope}%
\begin{pgfscope}%
\pgfpathrectangle{\pgfqpoint{0.770608in}{0.417642in}}{\pgfqpoint{3.170796in}{2.021097in}}%
\pgfusepath{clip}%
\pgfsetrectcap%
\pgfsetroundjoin%
\pgfsetlinewidth{0.803000pt}%
\definecolor{currentstroke}{rgb}{0.850000,0.850000,0.850000}%
\pgfsetstrokecolor{currentstroke}%
\pgfsetdash{}{0pt}%
\pgfpathmoveto{\pgfqpoint{1.873852in}{0.417642in}}%
\pgfpathlineto{\pgfqpoint{1.873852in}{2.438739in}}%
\pgfusepath{stroke}%
\end{pgfscope}%
\begin{pgfscope}%
\pgfsetbuttcap%
\pgfsetroundjoin%
\definecolor{currentfill}{rgb}{0.000000,0.000000,0.000000}%
\pgfsetfillcolor{currentfill}%
\pgfsetlinewidth{0.602250pt}%
\definecolor{currentstroke}{rgb}{0.000000,0.000000,0.000000}%
\pgfsetstrokecolor{currentstroke}%
\pgfsetdash{}{0pt}%
\pgfsys@defobject{currentmarker}{\pgfqpoint{0.000000in}{-0.027778in}}{\pgfqpoint{0.000000in}{0.000000in}}{%
\pgfpathmoveto{\pgfqpoint{0.000000in}{0.000000in}}%
\pgfpathlineto{\pgfqpoint{0.000000in}{-0.027778in}}%
\pgfusepath{stroke,fill}%
}%
\begin{pgfscope}%
\pgfsys@transformshift{1.873852in}{0.417642in}%
\pgfsys@useobject{currentmarker}{}%
\end{pgfscope}%
\end{pgfscope}%
\begin{pgfscope}%
\pgfpathrectangle{\pgfqpoint{0.770608in}{0.417642in}}{\pgfqpoint{3.170796in}{2.021097in}}%
\pgfusepath{clip}%
\pgfsetrectcap%
\pgfsetroundjoin%
\pgfsetlinewidth{0.803000pt}%
\definecolor{currentstroke}{rgb}{0.850000,0.850000,0.850000}%
\pgfsetstrokecolor{currentstroke}%
\pgfsetdash{}{0pt}%
\pgfpathmoveto{\pgfqpoint{1.899632in}{0.417642in}}%
\pgfpathlineto{\pgfqpoint{1.899632in}{2.438739in}}%
\pgfusepath{stroke}%
\end{pgfscope}%
\begin{pgfscope}%
\pgfsetbuttcap%
\pgfsetroundjoin%
\definecolor{currentfill}{rgb}{0.000000,0.000000,0.000000}%
\pgfsetfillcolor{currentfill}%
\pgfsetlinewidth{0.602250pt}%
\definecolor{currentstroke}{rgb}{0.000000,0.000000,0.000000}%
\pgfsetstrokecolor{currentstroke}%
\pgfsetdash{}{0pt}%
\pgfsys@defobject{currentmarker}{\pgfqpoint{0.000000in}{-0.027778in}}{\pgfqpoint{0.000000in}{0.000000in}}{%
\pgfpathmoveto{\pgfqpoint{0.000000in}{0.000000in}}%
\pgfpathlineto{\pgfqpoint{0.000000in}{-0.027778in}}%
\pgfusepath{stroke,fill}%
}%
\begin{pgfscope}%
\pgfsys@transformshift{1.899632in}{0.417642in}%
\pgfsys@useobject{currentmarker}{}%
\end{pgfscope}%
\end{pgfscope}%
\begin{pgfscope}%
\pgfpathrectangle{\pgfqpoint{0.770608in}{0.417642in}}{\pgfqpoint{3.170796in}{2.021097in}}%
\pgfusepath{clip}%
\pgfsetrectcap%
\pgfsetroundjoin%
\pgfsetlinewidth{0.803000pt}%
\definecolor{currentstroke}{rgb}{0.850000,0.850000,0.850000}%
\pgfsetstrokecolor{currentstroke}%
\pgfsetdash{}{0pt}%
\pgfpathmoveto{\pgfqpoint{2.074406in}{0.417642in}}%
\pgfpathlineto{\pgfqpoint{2.074406in}{2.438739in}}%
\pgfusepath{stroke}%
\end{pgfscope}%
\begin{pgfscope}%
\pgfsetbuttcap%
\pgfsetroundjoin%
\definecolor{currentfill}{rgb}{0.000000,0.000000,0.000000}%
\pgfsetfillcolor{currentfill}%
\pgfsetlinewidth{0.602250pt}%
\definecolor{currentstroke}{rgb}{0.000000,0.000000,0.000000}%
\pgfsetstrokecolor{currentstroke}%
\pgfsetdash{}{0pt}%
\pgfsys@defobject{currentmarker}{\pgfqpoint{0.000000in}{-0.027778in}}{\pgfqpoint{0.000000in}{0.000000in}}{%
\pgfpathmoveto{\pgfqpoint{0.000000in}{0.000000in}}%
\pgfpathlineto{\pgfqpoint{0.000000in}{-0.027778in}}%
\pgfusepath{stroke,fill}%
}%
\begin{pgfscope}%
\pgfsys@transformshift{2.074406in}{0.417642in}%
\pgfsys@useobject{currentmarker}{}%
\end{pgfscope}%
\end{pgfscope}%
\begin{pgfscope}%
\pgfpathrectangle{\pgfqpoint{0.770608in}{0.417642in}}{\pgfqpoint{3.170796in}{2.021097in}}%
\pgfusepath{clip}%
\pgfsetrectcap%
\pgfsetroundjoin%
\pgfsetlinewidth{0.803000pt}%
\definecolor{currentstroke}{rgb}{0.850000,0.850000,0.850000}%
\pgfsetstrokecolor{currentstroke}%
\pgfsetdash{}{0pt}%
\pgfpathmoveto{\pgfqpoint{2.163152in}{0.417642in}}%
\pgfpathlineto{\pgfqpoint{2.163152in}{2.438739in}}%
\pgfusepath{stroke}%
\end{pgfscope}%
\begin{pgfscope}%
\pgfsetbuttcap%
\pgfsetroundjoin%
\definecolor{currentfill}{rgb}{0.000000,0.000000,0.000000}%
\pgfsetfillcolor{currentfill}%
\pgfsetlinewidth{0.602250pt}%
\definecolor{currentstroke}{rgb}{0.000000,0.000000,0.000000}%
\pgfsetstrokecolor{currentstroke}%
\pgfsetdash{}{0pt}%
\pgfsys@defobject{currentmarker}{\pgfqpoint{0.000000in}{-0.027778in}}{\pgfqpoint{0.000000in}{0.000000in}}{%
\pgfpathmoveto{\pgfqpoint{0.000000in}{0.000000in}}%
\pgfpathlineto{\pgfqpoint{0.000000in}{-0.027778in}}%
\pgfusepath{stroke,fill}%
}%
\begin{pgfscope}%
\pgfsys@transformshift{2.163152in}{0.417642in}%
\pgfsys@useobject{currentmarker}{}%
\end{pgfscope}%
\end{pgfscope}%
\begin{pgfscope}%
\pgfpathrectangle{\pgfqpoint{0.770608in}{0.417642in}}{\pgfqpoint{3.170796in}{2.021097in}}%
\pgfusepath{clip}%
\pgfsetrectcap%
\pgfsetroundjoin%
\pgfsetlinewidth{0.803000pt}%
\definecolor{currentstroke}{rgb}{0.850000,0.850000,0.850000}%
\pgfsetstrokecolor{currentstroke}%
\pgfsetdash{}{0pt}%
\pgfpathmoveto{\pgfqpoint{2.226118in}{0.417642in}}%
\pgfpathlineto{\pgfqpoint{2.226118in}{2.438739in}}%
\pgfusepath{stroke}%
\end{pgfscope}%
\begin{pgfscope}%
\pgfsetbuttcap%
\pgfsetroundjoin%
\definecolor{currentfill}{rgb}{0.000000,0.000000,0.000000}%
\pgfsetfillcolor{currentfill}%
\pgfsetlinewidth{0.602250pt}%
\definecolor{currentstroke}{rgb}{0.000000,0.000000,0.000000}%
\pgfsetstrokecolor{currentstroke}%
\pgfsetdash{}{0pt}%
\pgfsys@defobject{currentmarker}{\pgfqpoint{0.000000in}{-0.027778in}}{\pgfqpoint{0.000000in}{0.000000in}}{%
\pgfpathmoveto{\pgfqpoint{0.000000in}{0.000000in}}%
\pgfpathlineto{\pgfqpoint{0.000000in}{-0.027778in}}%
\pgfusepath{stroke,fill}%
}%
\begin{pgfscope}%
\pgfsys@transformshift{2.226118in}{0.417642in}%
\pgfsys@useobject{currentmarker}{}%
\end{pgfscope}%
\end{pgfscope}%
\begin{pgfscope}%
\pgfpathrectangle{\pgfqpoint{0.770608in}{0.417642in}}{\pgfqpoint{3.170796in}{2.021097in}}%
\pgfusepath{clip}%
\pgfsetrectcap%
\pgfsetroundjoin%
\pgfsetlinewidth{0.803000pt}%
\definecolor{currentstroke}{rgb}{0.850000,0.850000,0.850000}%
\pgfsetstrokecolor{currentstroke}%
\pgfsetdash{}{0pt}%
\pgfpathmoveto{\pgfqpoint{2.274959in}{0.417642in}}%
\pgfpathlineto{\pgfqpoint{2.274959in}{2.438739in}}%
\pgfusepath{stroke}%
\end{pgfscope}%
\begin{pgfscope}%
\pgfsetbuttcap%
\pgfsetroundjoin%
\definecolor{currentfill}{rgb}{0.000000,0.000000,0.000000}%
\pgfsetfillcolor{currentfill}%
\pgfsetlinewidth{0.602250pt}%
\definecolor{currentstroke}{rgb}{0.000000,0.000000,0.000000}%
\pgfsetstrokecolor{currentstroke}%
\pgfsetdash{}{0pt}%
\pgfsys@defobject{currentmarker}{\pgfqpoint{0.000000in}{-0.027778in}}{\pgfqpoint{0.000000in}{0.000000in}}{%
\pgfpathmoveto{\pgfqpoint{0.000000in}{0.000000in}}%
\pgfpathlineto{\pgfqpoint{0.000000in}{-0.027778in}}%
\pgfusepath{stroke,fill}%
}%
\begin{pgfscope}%
\pgfsys@transformshift{2.274959in}{0.417642in}%
\pgfsys@useobject{currentmarker}{}%
\end{pgfscope}%
\end{pgfscope}%
\begin{pgfscope}%
\pgfpathrectangle{\pgfqpoint{0.770608in}{0.417642in}}{\pgfqpoint{3.170796in}{2.021097in}}%
\pgfusepath{clip}%
\pgfsetrectcap%
\pgfsetroundjoin%
\pgfsetlinewidth{0.803000pt}%
\definecolor{currentstroke}{rgb}{0.850000,0.850000,0.850000}%
\pgfsetstrokecolor{currentstroke}%
\pgfsetdash{}{0pt}%
\pgfpathmoveto{\pgfqpoint{2.314865in}{0.417642in}}%
\pgfpathlineto{\pgfqpoint{2.314865in}{2.438739in}}%
\pgfusepath{stroke}%
\end{pgfscope}%
\begin{pgfscope}%
\pgfsetbuttcap%
\pgfsetroundjoin%
\definecolor{currentfill}{rgb}{0.000000,0.000000,0.000000}%
\pgfsetfillcolor{currentfill}%
\pgfsetlinewidth{0.602250pt}%
\definecolor{currentstroke}{rgb}{0.000000,0.000000,0.000000}%
\pgfsetstrokecolor{currentstroke}%
\pgfsetdash{}{0pt}%
\pgfsys@defobject{currentmarker}{\pgfqpoint{0.000000in}{-0.027778in}}{\pgfqpoint{0.000000in}{0.000000in}}{%
\pgfpathmoveto{\pgfqpoint{0.000000in}{0.000000in}}%
\pgfpathlineto{\pgfqpoint{0.000000in}{-0.027778in}}%
\pgfusepath{stroke,fill}%
}%
\begin{pgfscope}%
\pgfsys@transformshift{2.314865in}{0.417642in}%
\pgfsys@useobject{currentmarker}{}%
\end{pgfscope}%
\end{pgfscope}%
\begin{pgfscope}%
\pgfpathrectangle{\pgfqpoint{0.770608in}{0.417642in}}{\pgfqpoint{3.170796in}{2.021097in}}%
\pgfusepath{clip}%
\pgfsetrectcap%
\pgfsetroundjoin%
\pgfsetlinewidth{0.803000pt}%
\definecolor{currentstroke}{rgb}{0.850000,0.850000,0.850000}%
\pgfsetstrokecolor{currentstroke}%
\pgfsetdash{}{0pt}%
\pgfpathmoveto{\pgfqpoint{2.348605in}{0.417642in}}%
\pgfpathlineto{\pgfqpoint{2.348605in}{2.438739in}}%
\pgfusepath{stroke}%
\end{pgfscope}%
\begin{pgfscope}%
\pgfsetbuttcap%
\pgfsetroundjoin%
\definecolor{currentfill}{rgb}{0.000000,0.000000,0.000000}%
\pgfsetfillcolor{currentfill}%
\pgfsetlinewidth{0.602250pt}%
\definecolor{currentstroke}{rgb}{0.000000,0.000000,0.000000}%
\pgfsetstrokecolor{currentstroke}%
\pgfsetdash{}{0pt}%
\pgfsys@defobject{currentmarker}{\pgfqpoint{0.000000in}{-0.027778in}}{\pgfqpoint{0.000000in}{0.000000in}}{%
\pgfpathmoveto{\pgfqpoint{0.000000in}{0.000000in}}%
\pgfpathlineto{\pgfqpoint{0.000000in}{-0.027778in}}%
\pgfusepath{stroke,fill}%
}%
\begin{pgfscope}%
\pgfsys@transformshift{2.348605in}{0.417642in}%
\pgfsys@useobject{currentmarker}{}%
\end{pgfscope}%
\end{pgfscope}%
\begin{pgfscope}%
\pgfpathrectangle{\pgfqpoint{0.770608in}{0.417642in}}{\pgfqpoint{3.170796in}{2.021097in}}%
\pgfusepath{clip}%
\pgfsetrectcap%
\pgfsetroundjoin%
\pgfsetlinewidth{0.803000pt}%
\definecolor{currentstroke}{rgb}{0.850000,0.850000,0.850000}%
\pgfsetstrokecolor{currentstroke}%
\pgfsetdash{}{0pt}%
\pgfpathmoveto{\pgfqpoint{2.377831in}{0.417642in}}%
\pgfpathlineto{\pgfqpoint{2.377831in}{2.438739in}}%
\pgfusepath{stroke}%
\end{pgfscope}%
\begin{pgfscope}%
\pgfsetbuttcap%
\pgfsetroundjoin%
\definecolor{currentfill}{rgb}{0.000000,0.000000,0.000000}%
\pgfsetfillcolor{currentfill}%
\pgfsetlinewidth{0.602250pt}%
\definecolor{currentstroke}{rgb}{0.000000,0.000000,0.000000}%
\pgfsetstrokecolor{currentstroke}%
\pgfsetdash{}{0pt}%
\pgfsys@defobject{currentmarker}{\pgfqpoint{0.000000in}{-0.027778in}}{\pgfqpoint{0.000000in}{0.000000in}}{%
\pgfpathmoveto{\pgfqpoint{0.000000in}{0.000000in}}%
\pgfpathlineto{\pgfqpoint{0.000000in}{-0.027778in}}%
\pgfusepath{stroke,fill}%
}%
\begin{pgfscope}%
\pgfsys@transformshift{2.377831in}{0.417642in}%
\pgfsys@useobject{currentmarker}{}%
\end{pgfscope}%
\end{pgfscope}%
\begin{pgfscope}%
\pgfpathrectangle{\pgfqpoint{0.770608in}{0.417642in}}{\pgfqpoint{3.170796in}{2.021097in}}%
\pgfusepath{clip}%
\pgfsetrectcap%
\pgfsetroundjoin%
\pgfsetlinewidth{0.803000pt}%
\definecolor{currentstroke}{rgb}{0.850000,0.850000,0.850000}%
\pgfsetstrokecolor{currentstroke}%
\pgfsetdash{}{0pt}%
\pgfpathmoveto{\pgfqpoint{2.403611in}{0.417642in}}%
\pgfpathlineto{\pgfqpoint{2.403611in}{2.438739in}}%
\pgfusepath{stroke}%
\end{pgfscope}%
\begin{pgfscope}%
\pgfsetbuttcap%
\pgfsetroundjoin%
\definecolor{currentfill}{rgb}{0.000000,0.000000,0.000000}%
\pgfsetfillcolor{currentfill}%
\pgfsetlinewidth{0.602250pt}%
\definecolor{currentstroke}{rgb}{0.000000,0.000000,0.000000}%
\pgfsetstrokecolor{currentstroke}%
\pgfsetdash{}{0pt}%
\pgfsys@defobject{currentmarker}{\pgfqpoint{0.000000in}{-0.027778in}}{\pgfqpoint{0.000000in}{0.000000in}}{%
\pgfpathmoveto{\pgfqpoint{0.000000in}{0.000000in}}%
\pgfpathlineto{\pgfqpoint{0.000000in}{-0.027778in}}%
\pgfusepath{stroke,fill}%
}%
\begin{pgfscope}%
\pgfsys@transformshift{2.403611in}{0.417642in}%
\pgfsys@useobject{currentmarker}{}%
\end{pgfscope}%
\end{pgfscope}%
\begin{pgfscope}%
\pgfpathrectangle{\pgfqpoint{0.770608in}{0.417642in}}{\pgfqpoint{3.170796in}{2.021097in}}%
\pgfusepath{clip}%
\pgfsetrectcap%
\pgfsetroundjoin%
\pgfsetlinewidth{0.803000pt}%
\definecolor{currentstroke}{rgb}{0.850000,0.850000,0.850000}%
\pgfsetstrokecolor{currentstroke}%
\pgfsetdash{}{0pt}%
\pgfpathmoveto{\pgfqpoint{2.578385in}{0.417642in}}%
\pgfpathlineto{\pgfqpoint{2.578385in}{2.438739in}}%
\pgfusepath{stroke}%
\end{pgfscope}%
\begin{pgfscope}%
\pgfsetbuttcap%
\pgfsetroundjoin%
\definecolor{currentfill}{rgb}{0.000000,0.000000,0.000000}%
\pgfsetfillcolor{currentfill}%
\pgfsetlinewidth{0.602250pt}%
\definecolor{currentstroke}{rgb}{0.000000,0.000000,0.000000}%
\pgfsetstrokecolor{currentstroke}%
\pgfsetdash{}{0pt}%
\pgfsys@defobject{currentmarker}{\pgfqpoint{0.000000in}{-0.027778in}}{\pgfqpoint{0.000000in}{0.000000in}}{%
\pgfpathmoveto{\pgfqpoint{0.000000in}{0.000000in}}%
\pgfpathlineto{\pgfqpoint{0.000000in}{-0.027778in}}%
\pgfusepath{stroke,fill}%
}%
\begin{pgfscope}%
\pgfsys@transformshift{2.578385in}{0.417642in}%
\pgfsys@useobject{currentmarker}{}%
\end{pgfscope}%
\end{pgfscope}%
\begin{pgfscope}%
\pgfpathrectangle{\pgfqpoint{0.770608in}{0.417642in}}{\pgfqpoint{3.170796in}{2.021097in}}%
\pgfusepath{clip}%
\pgfsetrectcap%
\pgfsetroundjoin%
\pgfsetlinewidth{0.803000pt}%
\definecolor{currentstroke}{rgb}{0.850000,0.850000,0.850000}%
\pgfsetstrokecolor{currentstroke}%
\pgfsetdash{}{0pt}%
\pgfpathmoveto{\pgfqpoint{2.667131in}{0.417642in}}%
\pgfpathlineto{\pgfqpoint{2.667131in}{2.438739in}}%
\pgfusepath{stroke}%
\end{pgfscope}%
\begin{pgfscope}%
\pgfsetbuttcap%
\pgfsetroundjoin%
\definecolor{currentfill}{rgb}{0.000000,0.000000,0.000000}%
\pgfsetfillcolor{currentfill}%
\pgfsetlinewidth{0.602250pt}%
\definecolor{currentstroke}{rgb}{0.000000,0.000000,0.000000}%
\pgfsetstrokecolor{currentstroke}%
\pgfsetdash{}{0pt}%
\pgfsys@defobject{currentmarker}{\pgfqpoint{0.000000in}{-0.027778in}}{\pgfqpoint{0.000000in}{0.000000in}}{%
\pgfpathmoveto{\pgfqpoint{0.000000in}{0.000000in}}%
\pgfpathlineto{\pgfqpoint{0.000000in}{-0.027778in}}%
\pgfusepath{stroke,fill}%
}%
\begin{pgfscope}%
\pgfsys@transformshift{2.667131in}{0.417642in}%
\pgfsys@useobject{currentmarker}{}%
\end{pgfscope}%
\end{pgfscope}%
\begin{pgfscope}%
\pgfpathrectangle{\pgfqpoint{0.770608in}{0.417642in}}{\pgfqpoint{3.170796in}{2.021097in}}%
\pgfusepath{clip}%
\pgfsetrectcap%
\pgfsetroundjoin%
\pgfsetlinewidth{0.803000pt}%
\definecolor{currentstroke}{rgb}{0.850000,0.850000,0.850000}%
\pgfsetstrokecolor{currentstroke}%
\pgfsetdash{}{0pt}%
\pgfpathmoveto{\pgfqpoint{2.730097in}{0.417642in}}%
\pgfpathlineto{\pgfqpoint{2.730097in}{2.438739in}}%
\pgfusepath{stroke}%
\end{pgfscope}%
\begin{pgfscope}%
\pgfsetbuttcap%
\pgfsetroundjoin%
\definecolor{currentfill}{rgb}{0.000000,0.000000,0.000000}%
\pgfsetfillcolor{currentfill}%
\pgfsetlinewidth{0.602250pt}%
\definecolor{currentstroke}{rgb}{0.000000,0.000000,0.000000}%
\pgfsetstrokecolor{currentstroke}%
\pgfsetdash{}{0pt}%
\pgfsys@defobject{currentmarker}{\pgfqpoint{0.000000in}{-0.027778in}}{\pgfqpoint{0.000000in}{0.000000in}}{%
\pgfpathmoveto{\pgfqpoint{0.000000in}{0.000000in}}%
\pgfpathlineto{\pgfqpoint{0.000000in}{-0.027778in}}%
\pgfusepath{stroke,fill}%
}%
\begin{pgfscope}%
\pgfsys@transformshift{2.730097in}{0.417642in}%
\pgfsys@useobject{currentmarker}{}%
\end{pgfscope}%
\end{pgfscope}%
\begin{pgfscope}%
\pgfpathrectangle{\pgfqpoint{0.770608in}{0.417642in}}{\pgfqpoint{3.170796in}{2.021097in}}%
\pgfusepath{clip}%
\pgfsetrectcap%
\pgfsetroundjoin%
\pgfsetlinewidth{0.803000pt}%
\definecolor{currentstroke}{rgb}{0.850000,0.850000,0.850000}%
\pgfsetstrokecolor{currentstroke}%
\pgfsetdash{}{0pt}%
\pgfpathmoveto{\pgfqpoint{2.778938in}{0.417642in}}%
\pgfpathlineto{\pgfqpoint{2.778938in}{2.438739in}}%
\pgfusepath{stroke}%
\end{pgfscope}%
\begin{pgfscope}%
\pgfsetbuttcap%
\pgfsetroundjoin%
\definecolor{currentfill}{rgb}{0.000000,0.000000,0.000000}%
\pgfsetfillcolor{currentfill}%
\pgfsetlinewidth{0.602250pt}%
\definecolor{currentstroke}{rgb}{0.000000,0.000000,0.000000}%
\pgfsetstrokecolor{currentstroke}%
\pgfsetdash{}{0pt}%
\pgfsys@defobject{currentmarker}{\pgfqpoint{0.000000in}{-0.027778in}}{\pgfqpoint{0.000000in}{0.000000in}}{%
\pgfpathmoveto{\pgfqpoint{0.000000in}{0.000000in}}%
\pgfpathlineto{\pgfqpoint{0.000000in}{-0.027778in}}%
\pgfusepath{stroke,fill}%
}%
\begin{pgfscope}%
\pgfsys@transformshift{2.778938in}{0.417642in}%
\pgfsys@useobject{currentmarker}{}%
\end{pgfscope}%
\end{pgfscope}%
\begin{pgfscope}%
\pgfpathrectangle{\pgfqpoint{0.770608in}{0.417642in}}{\pgfqpoint{3.170796in}{2.021097in}}%
\pgfusepath{clip}%
\pgfsetrectcap%
\pgfsetroundjoin%
\pgfsetlinewidth{0.803000pt}%
\definecolor{currentstroke}{rgb}{0.850000,0.850000,0.850000}%
\pgfsetstrokecolor{currentstroke}%
\pgfsetdash{}{0pt}%
\pgfpathmoveto{\pgfqpoint{2.818844in}{0.417642in}}%
\pgfpathlineto{\pgfqpoint{2.818844in}{2.438739in}}%
\pgfusepath{stroke}%
\end{pgfscope}%
\begin{pgfscope}%
\pgfsetbuttcap%
\pgfsetroundjoin%
\definecolor{currentfill}{rgb}{0.000000,0.000000,0.000000}%
\pgfsetfillcolor{currentfill}%
\pgfsetlinewidth{0.602250pt}%
\definecolor{currentstroke}{rgb}{0.000000,0.000000,0.000000}%
\pgfsetstrokecolor{currentstroke}%
\pgfsetdash{}{0pt}%
\pgfsys@defobject{currentmarker}{\pgfqpoint{0.000000in}{-0.027778in}}{\pgfqpoint{0.000000in}{0.000000in}}{%
\pgfpathmoveto{\pgfqpoint{0.000000in}{0.000000in}}%
\pgfpathlineto{\pgfqpoint{0.000000in}{-0.027778in}}%
\pgfusepath{stroke,fill}%
}%
\begin{pgfscope}%
\pgfsys@transformshift{2.818844in}{0.417642in}%
\pgfsys@useobject{currentmarker}{}%
\end{pgfscope}%
\end{pgfscope}%
\begin{pgfscope}%
\pgfpathrectangle{\pgfqpoint{0.770608in}{0.417642in}}{\pgfqpoint{3.170796in}{2.021097in}}%
\pgfusepath{clip}%
\pgfsetrectcap%
\pgfsetroundjoin%
\pgfsetlinewidth{0.803000pt}%
\definecolor{currentstroke}{rgb}{0.850000,0.850000,0.850000}%
\pgfsetstrokecolor{currentstroke}%
\pgfsetdash{}{0pt}%
\pgfpathmoveto{\pgfqpoint{2.852583in}{0.417642in}}%
\pgfpathlineto{\pgfqpoint{2.852583in}{2.438739in}}%
\pgfusepath{stroke}%
\end{pgfscope}%
\begin{pgfscope}%
\pgfsetbuttcap%
\pgfsetroundjoin%
\definecolor{currentfill}{rgb}{0.000000,0.000000,0.000000}%
\pgfsetfillcolor{currentfill}%
\pgfsetlinewidth{0.602250pt}%
\definecolor{currentstroke}{rgb}{0.000000,0.000000,0.000000}%
\pgfsetstrokecolor{currentstroke}%
\pgfsetdash{}{0pt}%
\pgfsys@defobject{currentmarker}{\pgfqpoint{0.000000in}{-0.027778in}}{\pgfqpoint{0.000000in}{0.000000in}}{%
\pgfpathmoveto{\pgfqpoint{0.000000in}{0.000000in}}%
\pgfpathlineto{\pgfqpoint{0.000000in}{-0.027778in}}%
\pgfusepath{stroke,fill}%
}%
\begin{pgfscope}%
\pgfsys@transformshift{2.852583in}{0.417642in}%
\pgfsys@useobject{currentmarker}{}%
\end{pgfscope}%
\end{pgfscope}%
\begin{pgfscope}%
\pgfpathrectangle{\pgfqpoint{0.770608in}{0.417642in}}{\pgfqpoint{3.170796in}{2.021097in}}%
\pgfusepath{clip}%
\pgfsetrectcap%
\pgfsetroundjoin%
\pgfsetlinewidth{0.803000pt}%
\definecolor{currentstroke}{rgb}{0.850000,0.850000,0.850000}%
\pgfsetstrokecolor{currentstroke}%
\pgfsetdash{}{0pt}%
\pgfpathmoveto{\pgfqpoint{2.881810in}{0.417642in}}%
\pgfpathlineto{\pgfqpoint{2.881810in}{2.438739in}}%
\pgfusepath{stroke}%
\end{pgfscope}%
\begin{pgfscope}%
\pgfsetbuttcap%
\pgfsetroundjoin%
\definecolor{currentfill}{rgb}{0.000000,0.000000,0.000000}%
\pgfsetfillcolor{currentfill}%
\pgfsetlinewidth{0.602250pt}%
\definecolor{currentstroke}{rgb}{0.000000,0.000000,0.000000}%
\pgfsetstrokecolor{currentstroke}%
\pgfsetdash{}{0pt}%
\pgfsys@defobject{currentmarker}{\pgfqpoint{0.000000in}{-0.027778in}}{\pgfqpoint{0.000000in}{0.000000in}}{%
\pgfpathmoveto{\pgfqpoint{0.000000in}{0.000000in}}%
\pgfpathlineto{\pgfqpoint{0.000000in}{-0.027778in}}%
\pgfusepath{stroke,fill}%
}%
\begin{pgfscope}%
\pgfsys@transformshift{2.881810in}{0.417642in}%
\pgfsys@useobject{currentmarker}{}%
\end{pgfscope}%
\end{pgfscope}%
\begin{pgfscope}%
\pgfpathrectangle{\pgfqpoint{0.770608in}{0.417642in}}{\pgfqpoint{3.170796in}{2.021097in}}%
\pgfusepath{clip}%
\pgfsetrectcap%
\pgfsetroundjoin%
\pgfsetlinewidth{0.803000pt}%
\definecolor{currentstroke}{rgb}{0.850000,0.850000,0.850000}%
\pgfsetstrokecolor{currentstroke}%
\pgfsetdash{}{0pt}%
\pgfpathmoveto{\pgfqpoint{2.907590in}{0.417642in}}%
\pgfpathlineto{\pgfqpoint{2.907590in}{2.438739in}}%
\pgfusepath{stroke}%
\end{pgfscope}%
\begin{pgfscope}%
\pgfsetbuttcap%
\pgfsetroundjoin%
\definecolor{currentfill}{rgb}{0.000000,0.000000,0.000000}%
\pgfsetfillcolor{currentfill}%
\pgfsetlinewidth{0.602250pt}%
\definecolor{currentstroke}{rgb}{0.000000,0.000000,0.000000}%
\pgfsetstrokecolor{currentstroke}%
\pgfsetdash{}{0pt}%
\pgfsys@defobject{currentmarker}{\pgfqpoint{0.000000in}{-0.027778in}}{\pgfqpoint{0.000000in}{0.000000in}}{%
\pgfpathmoveto{\pgfqpoint{0.000000in}{0.000000in}}%
\pgfpathlineto{\pgfqpoint{0.000000in}{-0.027778in}}%
\pgfusepath{stroke,fill}%
}%
\begin{pgfscope}%
\pgfsys@transformshift{2.907590in}{0.417642in}%
\pgfsys@useobject{currentmarker}{}%
\end{pgfscope}%
\end{pgfscope}%
\begin{pgfscope}%
\pgfpathrectangle{\pgfqpoint{0.770608in}{0.417642in}}{\pgfqpoint{3.170796in}{2.021097in}}%
\pgfusepath{clip}%
\pgfsetrectcap%
\pgfsetroundjoin%
\pgfsetlinewidth{0.803000pt}%
\definecolor{currentstroke}{rgb}{0.850000,0.850000,0.850000}%
\pgfsetstrokecolor{currentstroke}%
\pgfsetdash{}{0pt}%
\pgfpathmoveto{\pgfqpoint{3.082363in}{0.417642in}}%
\pgfpathlineto{\pgfqpoint{3.082363in}{2.438739in}}%
\pgfusepath{stroke}%
\end{pgfscope}%
\begin{pgfscope}%
\pgfsetbuttcap%
\pgfsetroundjoin%
\definecolor{currentfill}{rgb}{0.000000,0.000000,0.000000}%
\pgfsetfillcolor{currentfill}%
\pgfsetlinewidth{0.602250pt}%
\definecolor{currentstroke}{rgb}{0.000000,0.000000,0.000000}%
\pgfsetstrokecolor{currentstroke}%
\pgfsetdash{}{0pt}%
\pgfsys@defobject{currentmarker}{\pgfqpoint{0.000000in}{-0.027778in}}{\pgfqpoint{0.000000in}{0.000000in}}{%
\pgfpathmoveto{\pgfqpoint{0.000000in}{0.000000in}}%
\pgfpathlineto{\pgfqpoint{0.000000in}{-0.027778in}}%
\pgfusepath{stroke,fill}%
}%
\begin{pgfscope}%
\pgfsys@transformshift{3.082363in}{0.417642in}%
\pgfsys@useobject{currentmarker}{}%
\end{pgfscope}%
\end{pgfscope}%
\begin{pgfscope}%
\pgfpathrectangle{\pgfqpoint{0.770608in}{0.417642in}}{\pgfqpoint{3.170796in}{2.021097in}}%
\pgfusepath{clip}%
\pgfsetrectcap%
\pgfsetroundjoin%
\pgfsetlinewidth{0.803000pt}%
\definecolor{currentstroke}{rgb}{0.850000,0.850000,0.850000}%
\pgfsetstrokecolor{currentstroke}%
\pgfsetdash{}{0pt}%
\pgfpathmoveto{\pgfqpoint{3.171110in}{0.417642in}}%
\pgfpathlineto{\pgfqpoint{3.171110in}{2.438739in}}%
\pgfusepath{stroke}%
\end{pgfscope}%
\begin{pgfscope}%
\pgfsetbuttcap%
\pgfsetroundjoin%
\definecolor{currentfill}{rgb}{0.000000,0.000000,0.000000}%
\pgfsetfillcolor{currentfill}%
\pgfsetlinewidth{0.602250pt}%
\definecolor{currentstroke}{rgb}{0.000000,0.000000,0.000000}%
\pgfsetstrokecolor{currentstroke}%
\pgfsetdash{}{0pt}%
\pgfsys@defobject{currentmarker}{\pgfqpoint{0.000000in}{-0.027778in}}{\pgfqpoint{0.000000in}{0.000000in}}{%
\pgfpathmoveto{\pgfqpoint{0.000000in}{0.000000in}}%
\pgfpathlineto{\pgfqpoint{0.000000in}{-0.027778in}}%
\pgfusepath{stroke,fill}%
}%
\begin{pgfscope}%
\pgfsys@transformshift{3.171110in}{0.417642in}%
\pgfsys@useobject{currentmarker}{}%
\end{pgfscope}%
\end{pgfscope}%
\begin{pgfscope}%
\pgfpathrectangle{\pgfqpoint{0.770608in}{0.417642in}}{\pgfqpoint{3.170796in}{2.021097in}}%
\pgfusepath{clip}%
\pgfsetrectcap%
\pgfsetroundjoin%
\pgfsetlinewidth{0.803000pt}%
\definecolor{currentstroke}{rgb}{0.850000,0.850000,0.850000}%
\pgfsetstrokecolor{currentstroke}%
\pgfsetdash{}{0pt}%
\pgfpathmoveto{\pgfqpoint{3.234076in}{0.417642in}}%
\pgfpathlineto{\pgfqpoint{3.234076in}{2.438739in}}%
\pgfusepath{stroke}%
\end{pgfscope}%
\begin{pgfscope}%
\pgfsetbuttcap%
\pgfsetroundjoin%
\definecolor{currentfill}{rgb}{0.000000,0.000000,0.000000}%
\pgfsetfillcolor{currentfill}%
\pgfsetlinewidth{0.602250pt}%
\definecolor{currentstroke}{rgb}{0.000000,0.000000,0.000000}%
\pgfsetstrokecolor{currentstroke}%
\pgfsetdash{}{0pt}%
\pgfsys@defobject{currentmarker}{\pgfqpoint{0.000000in}{-0.027778in}}{\pgfqpoint{0.000000in}{0.000000in}}{%
\pgfpathmoveto{\pgfqpoint{0.000000in}{0.000000in}}%
\pgfpathlineto{\pgfqpoint{0.000000in}{-0.027778in}}%
\pgfusepath{stroke,fill}%
}%
\begin{pgfscope}%
\pgfsys@transformshift{3.234076in}{0.417642in}%
\pgfsys@useobject{currentmarker}{}%
\end{pgfscope}%
\end{pgfscope}%
\begin{pgfscope}%
\pgfpathrectangle{\pgfqpoint{0.770608in}{0.417642in}}{\pgfqpoint{3.170796in}{2.021097in}}%
\pgfusepath{clip}%
\pgfsetrectcap%
\pgfsetroundjoin%
\pgfsetlinewidth{0.803000pt}%
\definecolor{currentstroke}{rgb}{0.850000,0.850000,0.850000}%
\pgfsetstrokecolor{currentstroke}%
\pgfsetdash{}{0pt}%
\pgfpathmoveto{\pgfqpoint{3.282917in}{0.417642in}}%
\pgfpathlineto{\pgfqpoint{3.282917in}{2.438739in}}%
\pgfusepath{stroke}%
\end{pgfscope}%
\begin{pgfscope}%
\pgfsetbuttcap%
\pgfsetroundjoin%
\definecolor{currentfill}{rgb}{0.000000,0.000000,0.000000}%
\pgfsetfillcolor{currentfill}%
\pgfsetlinewidth{0.602250pt}%
\definecolor{currentstroke}{rgb}{0.000000,0.000000,0.000000}%
\pgfsetstrokecolor{currentstroke}%
\pgfsetdash{}{0pt}%
\pgfsys@defobject{currentmarker}{\pgfqpoint{0.000000in}{-0.027778in}}{\pgfqpoint{0.000000in}{0.000000in}}{%
\pgfpathmoveto{\pgfqpoint{0.000000in}{0.000000in}}%
\pgfpathlineto{\pgfqpoint{0.000000in}{-0.027778in}}%
\pgfusepath{stroke,fill}%
}%
\begin{pgfscope}%
\pgfsys@transformshift{3.282917in}{0.417642in}%
\pgfsys@useobject{currentmarker}{}%
\end{pgfscope}%
\end{pgfscope}%
\begin{pgfscope}%
\pgfpathrectangle{\pgfqpoint{0.770608in}{0.417642in}}{\pgfqpoint{3.170796in}{2.021097in}}%
\pgfusepath{clip}%
\pgfsetrectcap%
\pgfsetroundjoin%
\pgfsetlinewidth{0.803000pt}%
\definecolor{currentstroke}{rgb}{0.850000,0.850000,0.850000}%
\pgfsetstrokecolor{currentstroke}%
\pgfsetdash{}{0pt}%
\pgfpathmoveto{\pgfqpoint{3.322822in}{0.417642in}}%
\pgfpathlineto{\pgfqpoint{3.322822in}{2.438739in}}%
\pgfusepath{stroke}%
\end{pgfscope}%
\begin{pgfscope}%
\pgfsetbuttcap%
\pgfsetroundjoin%
\definecolor{currentfill}{rgb}{0.000000,0.000000,0.000000}%
\pgfsetfillcolor{currentfill}%
\pgfsetlinewidth{0.602250pt}%
\definecolor{currentstroke}{rgb}{0.000000,0.000000,0.000000}%
\pgfsetstrokecolor{currentstroke}%
\pgfsetdash{}{0pt}%
\pgfsys@defobject{currentmarker}{\pgfqpoint{0.000000in}{-0.027778in}}{\pgfqpoint{0.000000in}{0.000000in}}{%
\pgfpathmoveto{\pgfqpoint{0.000000in}{0.000000in}}%
\pgfpathlineto{\pgfqpoint{0.000000in}{-0.027778in}}%
\pgfusepath{stroke,fill}%
}%
\begin{pgfscope}%
\pgfsys@transformshift{3.322822in}{0.417642in}%
\pgfsys@useobject{currentmarker}{}%
\end{pgfscope}%
\end{pgfscope}%
\begin{pgfscope}%
\pgfpathrectangle{\pgfqpoint{0.770608in}{0.417642in}}{\pgfqpoint{3.170796in}{2.021097in}}%
\pgfusepath{clip}%
\pgfsetrectcap%
\pgfsetroundjoin%
\pgfsetlinewidth{0.803000pt}%
\definecolor{currentstroke}{rgb}{0.850000,0.850000,0.850000}%
\pgfsetstrokecolor{currentstroke}%
\pgfsetdash{}{0pt}%
\pgfpathmoveto{\pgfqpoint{3.356562in}{0.417642in}}%
\pgfpathlineto{\pgfqpoint{3.356562in}{2.438739in}}%
\pgfusepath{stroke}%
\end{pgfscope}%
\begin{pgfscope}%
\pgfsetbuttcap%
\pgfsetroundjoin%
\definecolor{currentfill}{rgb}{0.000000,0.000000,0.000000}%
\pgfsetfillcolor{currentfill}%
\pgfsetlinewidth{0.602250pt}%
\definecolor{currentstroke}{rgb}{0.000000,0.000000,0.000000}%
\pgfsetstrokecolor{currentstroke}%
\pgfsetdash{}{0pt}%
\pgfsys@defobject{currentmarker}{\pgfqpoint{0.000000in}{-0.027778in}}{\pgfqpoint{0.000000in}{0.000000in}}{%
\pgfpathmoveto{\pgfqpoint{0.000000in}{0.000000in}}%
\pgfpathlineto{\pgfqpoint{0.000000in}{-0.027778in}}%
\pgfusepath{stroke,fill}%
}%
\begin{pgfscope}%
\pgfsys@transformshift{3.356562in}{0.417642in}%
\pgfsys@useobject{currentmarker}{}%
\end{pgfscope}%
\end{pgfscope}%
\begin{pgfscope}%
\pgfpathrectangle{\pgfqpoint{0.770608in}{0.417642in}}{\pgfqpoint{3.170796in}{2.021097in}}%
\pgfusepath{clip}%
\pgfsetrectcap%
\pgfsetroundjoin%
\pgfsetlinewidth{0.803000pt}%
\definecolor{currentstroke}{rgb}{0.850000,0.850000,0.850000}%
\pgfsetstrokecolor{currentstroke}%
\pgfsetdash{}{0pt}%
\pgfpathmoveto{\pgfqpoint{3.385789in}{0.417642in}}%
\pgfpathlineto{\pgfqpoint{3.385789in}{2.438739in}}%
\pgfusepath{stroke}%
\end{pgfscope}%
\begin{pgfscope}%
\pgfsetbuttcap%
\pgfsetroundjoin%
\definecolor{currentfill}{rgb}{0.000000,0.000000,0.000000}%
\pgfsetfillcolor{currentfill}%
\pgfsetlinewidth{0.602250pt}%
\definecolor{currentstroke}{rgb}{0.000000,0.000000,0.000000}%
\pgfsetstrokecolor{currentstroke}%
\pgfsetdash{}{0pt}%
\pgfsys@defobject{currentmarker}{\pgfqpoint{0.000000in}{-0.027778in}}{\pgfqpoint{0.000000in}{0.000000in}}{%
\pgfpathmoveto{\pgfqpoint{0.000000in}{0.000000in}}%
\pgfpathlineto{\pgfqpoint{0.000000in}{-0.027778in}}%
\pgfusepath{stroke,fill}%
}%
\begin{pgfscope}%
\pgfsys@transformshift{3.385789in}{0.417642in}%
\pgfsys@useobject{currentmarker}{}%
\end{pgfscope}%
\end{pgfscope}%
\begin{pgfscope}%
\pgfpathrectangle{\pgfqpoint{0.770608in}{0.417642in}}{\pgfqpoint{3.170796in}{2.021097in}}%
\pgfusepath{clip}%
\pgfsetrectcap%
\pgfsetroundjoin%
\pgfsetlinewidth{0.803000pt}%
\definecolor{currentstroke}{rgb}{0.850000,0.850000,0.850000}%
\pgfsetstrokecolor{currentstroke}%
\pgfsetdash{}{0pt}%
\pgfpathmoveto{\pgfqpoint{3.411569in}{0.417642in}}%
\pgfpathlineto{\pgfqpoint{3.411569in}{2.438739in}}%
\pgfusepath{stroke}%
\end{pgfscope}%
\begin{pgfscope}%
\pgfsetbuttcap%
\pgfsetroundjoin%
\definecolor{currentfill}{rgb}{0.000000,0.000000,0.000000}%
\pgfsetfillcolor{currentfill}%
\pgfsetlinewidth{0.602250pt}%
\definecolor{currentstroke}{rgb}{0.000000,0.000000,0.000000}%
\pgfsetstrokecolor{currentstroke}%
\pgfsetdash{}{0pt}%
\pgfsys@defobject{currentmarker}{\pgfqpoint{0.000000in}{-0.027778in}}{\pgfqpoint{0.000000in}{0.000000in}}{%
\pgfpathmoveto{\pgfqpoint{0.000000in}{0.000000in}}%
\pgfpathlineto{\pgfqpoint{0.000000in}{-0.027778in}}%
\pgfusepath{stroke,fill}%
}%
\begin{pgfscope}%
\pgfsys@transformshift{3.411569in}{0.417642in}%
\pgfsys@useobject{currentmarker}{}%
\end{pgfscope}%
\end{pgfscope}%
\begin{pgfscope}%
\pgfpathrectangle{\pgfqpoint{0.770608in}{0.417642in}}{\pgfqpoint{3.170796in}{2.021097in}}%
\pgfusepath{clip}%
\pgfsetrectcap%
\pgfsetroundjoin%
\pgfsetlinewidth{0.803000pt}%
\definecolor{currentstroke}{rgb}{0.850000,0.850000,0.850000}%
\pgfsetstrokecolor{currentstroke}%
\pgfsetdash{}{0pt}%
\pgfpathmoveto{\pgfqpoint{3.586342in}{0.417642in}}%
\pgfpathlineto{\pgfqpoint{3.586342in}{2.438739in}}%
\pgfusepath{stroke}%
\end{pgfscope}%
\begin{pgfscope}%
\pgfsetbuttcap%
\pgfsetroundjoin%
\definecolor{currentfill}{rgb}{0.000000,0.000000,0.000000}%
\pgfsetfillcolor{currentfill}%
\pgfsetlinewidth{0.602250pt}%
\definecolor{currentstroke}{rgb}{0.000000,0.000000,0.000000}%
\pgfsetstrokecolor{currentstroke}%
\pgfsetdash{}{0pt}%
\pgfsys@defobject{currentmarker}{\pgfqpoint{0.000000in}{-0.027778in}}{\pgfqpoint{0.000000in}{0.000000in}}{%
\pgfpathmoveto{\pgfqpoint{0.000000in}{0.000000in}}%
\pgfpathlineto{\pgfqpoint{0.000000in}{-0.027778in}}%
\pgfusepath{stroke,fill}%
}%
\begin{pgfscope}%
\pgfsys@transformshift{3.586342in}{0.417642in}%
\pgfsys@useobject{currentmarker}{}%
\end{pgfscope}%
\end{pgfscope}%
\begin{pgfscope}%
\pgfpathrectangle{\pgfqpoint{0.770608in}{0.417642in}}{\pgfqpoint{3.170796in}{2.021097in}}%
\pgfusepath{clip}%
\pgfsetrectcap%
\pgfsetroundjoin%
\pgfsetlinewidth{0.803000pt}%
\definecolor{currentstroke}{rgb}{0.850000,0.850000,0.850000}%
\pgfsetstrokecolor{currentstroke}%
\pgfsetdash{}{0pt}%
\pgfpathmoveto{\pgfqpoint{3.675088in}{0.417642in}}%
\pgfpathlineto{\pgfqpoint{3.675088in}{2.438739in}}%
\pgfusepath{stroke}%
\end{pgfscope}%
\begin{pgfscope}%
\pgfsetbuttcap%
\pgfsetroundjoin%
\definecolor{currentfill}{rgb}{0.000000,0.000000,0.000000}%
\pgfsetfillcolor{currentfill}%
\pgfsetlinewidth{0.602250pt}%
\definecolor{currentstroke}{rgb}{0.000000,0.000000,0.000000}%
\pgfsetstrokecolor{currentstroke}%
\pgfsetdash{}{0pt}%
\pgfsys@defobject{currentmarker}{\pgfqpoint{0.000000in}{-0.027778in}}{\pgfqpoint{0.000000in}{0.000000in}}{%
\pgfpathmoveto{\pgfqpoint{0.000000in}{0.000000in}}%
\pgfpathlineto{\pgfqpoint{0.000000in}{-0.027778in}}%
\pgfusepath{stroke,fill}%
}%
\begin{pgfscope}%
\pgfsys@transformshift{3.675088in}{0.417642in}%
\pgfsys@useobject{currentmarker}{}%
\end{pgfscope}%
\end{pgfscope}%
\begin{pgfscope}%
\pgfpathrectangle{\pgfqpoint{0.770608in}{0.417642in}}{\pgfqpoint{3.170796in}{2.021097in}}%
\pgfusepath{clip}%
\pgfsetrectcap%
\pgfsetroundjoin%
\pgfsetlinewidth{0.803000pt}%
\definecolor{currentstroke}{rgb}{0.850000,0.850000,0.850000}%
\pgfsetstrokecolor{currentstroke}%
\pgfsetdash{}{0pt}%
\pgfpathmoveto{\pgfqpoint{3.738055in}{0.417642in}}%
\pgfpathlineto{\pgfqpoint{3.738055in}{2.438739in}}%
\pgfusepath{stroke}%
\end{pgfscope}%
\begin{pgfscope}%
\pgfsetbuttcap%
\pgfsetroundjoin%
\definecolor{currentfill}{rgb}{0.000000,0.000000,0.000000}%
\pgfsetfillcolor{currentfill}%
\pgfsetlinewidth{0.602250pt}%
\definecolor{currentstroke}{rgb}{0.000000,0.000000,0.000000}%
\pgfsetstrokecolor{currentstroke}%
\pgfsetdash{}{0pt}%
\pgfsys@defobject{currentmarker}{\pgfqpoint{0.000000in}{-0.027778in}}{\pgfqpoint{0.000000in}{0.000000in}}{%
\pgfpathmoveto{\pgfqpoint{0.000000in}{0.000000in}}%
\pgfpathlineto{\pgfqpoint{0.000000in}{-0.027778in}}%
\pgfusepath{stroke,fill}%
}%
\begin{pgfscope}%
\pgfsys@transformshift{3.738055in}{0.417642in}%
\pgfsys@useobject{currentmarker}{}%
\end{pgfscope}%
\end{pgfscope}%
\begin{pgfscope}%
\pgfpathrectangle{\pgfqpoint{0.770608in}{0.417642in}}{\pgfqpoint{3.170796in}{2.021097in}}%
\pgfusepath{clip}%
\pgfsetrectcap%
\pgfsetroundjoin%
\pgfsetlinewidth{0.803000pt}%
\definecolor{currentstroke}{rgb}{0.850000,0.850000,0.850000}%
\pgfsetstrokecolor{currentstroke}%
\pgfsetdash{}{0pt}%
\pgfpathmoveto{\pgfqpoint{3.786895in}{0.417642in}}%
\pgfpathlineto{\pgfqpoint{3.786895in}{2.438739in}}%
\pgfusepath{stroke}%
\end{pgfscope}%
\begin{pgfscope}%
\pgfsetbuttcap%
\pgfsetroundjoin%
\definecolor{currentfill}{rgb}{0.000000,0.000000,0.000000}%
\pgfsetfillcolor{currentfill}%
\pgfsetlinewidth{0.602250pt}%
\definecolor{currentstroke}{rgb}{0.000000,0.000000,0.000000}%
\pgfsetstrokecolor{currentstroke}%
\pgfsetdash{}{0pt}%
\pgfsys@defobject{currentmarker}{\pgfqpoint{0.000000in}{-0.027778in}}{\pgfqpoint{0.000000in}{0.000000in}}{%
\pgfpathmoveto{\pgfqpoint{0.000000in}{0.000000in}}%
\pgfpathlineto{\pgfqpoint{0.000000in}{-0.027778in}}%
\pgfusepath{stroke,fill}%
}%
\begin{pgfscope}%
\pgfsys@transformshift{3.786895in}{0.417642in}%
\pgfsys@useobject{currentmarker}{}%
\end{pgfscope}%
\end{pgfscope}%
\begin{pgfscope}%
\pgfpathrectangle{\pgfqpoint{0.770608in}{0.417642in}}{\pgfqpoint{3.170796in}{2.021097in}}%
\pgfusepath{clip}%
\pgfsetrectcap%
\pgfsetroundjoin%
\pgfsetlinewidth{0.803000pt}%
\definecolor{currentstroke}{rgb}{0.850000,0.850000,0.850000}%
\pgfsetstrokecolor{currentstroke}%
\pgfsetdash{}{0pt}%
\pgfpathmoveto{\pgfqpoint{3.826801in}{0.417642in}}%
\pgfpathlineto{\pgfqpoint{3.826801in}{2.438739in}}%
\pgfusepath{stroke}%
\end{pgfscope}%
\begin{pgfscope}%
\pgfsetbuttcap%
\pgfsetroundjoin%
\definecolor{currentfill}{rgb}{0.000000,0.000000,0.000000}%
\pgfsetfillcolor{currentfill}%
\pgfsetlinewidth{0.602250pt}%
\definecolor{currentstroke}{rgb}{0.000000,0.000000,0.000000}%
\pgfsetstrokecolor{currentstroke}%
\pgfsetdash{}{0pt}%
\pgfsys@defobject{currentmarker}{\pgfqpoint{0.000000in}{-0.027778in}}{\pgfqpoint{0.000000in}{0.000000in}}{%
\pgfpathmoveto{\pgfqpoint{0.000000in}{0.000000in}}%
\pgfpathlineto{\pgfqpoint{0.000000in}{-0.027778in}}%
\pgfusepath{stroke,fill}%
}%
\begin{pgfscope}%
\pgfsys@transformshift{3.826801in}{0.417642in}%
\pgfsys@useobject{currentmarker}{}%
\end{pgfscope}%
\end{pgfscope}%
\begin{pgfscope}%
\pgfpathrectangle{\pgfqpoint{0.770608in}{0.417642in}}{\pgfqpoint{3.170796in}{2.021097in}}%
\pgfusepath{clip}%
\pgfsetrectcap%
\pgfsetroundjoin%
\pgfsetlinewidth{0.803000pt}%
\definecolor{currentstroke}{rgb}{0.850000,0.850000,0.850000}%
\pgfsetstrokecolor{currentstroke}%
\pgfsetdash{}{0pt}%
\pgfpathmoveto{\pgfqpoint{3.860541in}{0.417642in}}%
\pgfpathlineto{\pgfqpoint{3.860541in}{2.438739in}}%
\pgfusepath{stroke}%
\end{pgfscope}%
\begin{pgfscope}%
\pgfsetbuttcap%
\pgfsetroundjoin%
\definecolor{currentfill}{rgb}{0.000000,0.000000,0.000000}%
\pgfsetfillcolor{currentfill}%
\pgfsetlinewidth{0.602250pt}%
\definecolor{currentstroke}{rgb}{0.000000,0.000000,0.000000}%
\pgfsetstrokecolor{currentstroke}%
\pgfsetdash{}{0pt}%
\pgfsys@defobject{currentmarker}{\pgfqpoint{0.000000in}{-0.027778in}}{\pgfqpoint{0.000000in}{0.000000in}}{%
\pgfpathmoveto{\pgfqpoint{0.000000in}{0.000000in}}%
\pgfpathlineto{\pgfqpoint{0.000000in}{-0.027778in}}%
\pgfusepath{stroke,fill}%
}%
\begin{pgfscope}%
\pgfsys@transformshift{3.860541in}{0.417642in}%
\pgfsys@useobject{currentmarker}{}%
\end{pgfscope}%
\end{pgfscope}%
\begin{pgfscope}%
\pgfpathrectangle{\pgfqpoint{0.770608in}{0.417642in}}{\pgfqpoint{3.170796in}{2.021097in}}%
\pgfusepath{clip}%
\pgfsetrectcap%
\pgfsetroundjoin%
\pgfsetlinewidth{0.803000pt}%
\definecolor{currentstroke}{rgb}{0.850000,0.850000,0.850000}%
\pgfsetstrokecolor{currentstroke}%
\pgfsetdash{}{0pt}%
\pgfpathmoveto{\pgfqpoint{3.889768in}{0.417642in}}%
\pgfpathlineto{\pgfqpoint{3.889768in}{2.438739in}}%
\pgfusepath{stroke}%
\end{pgfscope}%
\begin{pgfscope}%
\pgfsetbuttcap%
\pgfsetroundjoin%
\definecolor{currentfill}{rgb}{0.000000,0.000000,0.000000}%
\pgfsetfillcolor{currentfill}%
\pgfsetlinewidth{0.602250pt}%
\definecolor{currentstroke}{rgb}{0.000000,0.000000,0.000000}%
\pgfsetstrokecolor{currentstroke}%
\pgfsetdash{}{0pt}%
\pgfsys@defobject{currentmarker}{\pgfqpoint{0.000000in}{-0.027778in}}{\pgfqpoint{0.000000in}{0.000000in}}{%
\pgfpathmoveto{\pgfqpoint{0.000000in}{0.000000in}}%
\pgfpathlineto{\pgfqpoint{0.000000in}{-0.027778in}}%
\pgfusepath{stroke,fill}%
}%
\begin{pgfscope}%
\pgfsys@transformshift{3.889768in}{0.417642in}%
\pgfsys@useobject{currentmarker}{}%
\end{pgfscope}%
\end{pgfscope}%
\begin{pgfscope}%
\pgfpathrectangle{\pgfqpoint{0.770608in}{0.417642in}}{\pgfqpoint{3.170796in}{2.021097in}}%
\pgfusepath{clip}%
\pgfsetrectcap%
\pgfsetroundjoin%
\pgfsetlinewidth{0.803000pt}%
\definecolor{currentstroke}{rgb}{0.850000,0.850000,0.850000}%
\pgfsetstrokecolor{currentstroke}%
\pgfsetdash{}{0pt}%
\pgfpathmoveto{\pgfqpoint{3.915547in}{0.417642in}}%
\pgfpathlineto{\pgfqpoint{3.915547in}{2.438739in}}%
\pgfusepath{stroke}%
\end{pgfscope}%
\begin{pgfscope}%
\pgfsetbuttcap%
\pgfsetroundjoin%
\definecolor{currentfill}{rgb}{0.000000,0.000000,0.000000}%
\pgfsetfillcolor{currentfill}%
\pgfsetlinewidth{0.602250pt}%
\definecolor{currentstroke}{rgb}{0.000000,0.000000,0.000000}%
\pgfsetstrokecolor{currentstroke}%
\pgfsetdash{}{0pt}%
\pgfsys@defobject{currentmarker}{\pgfqpoint{0.000000in}{-0.027778in}}{\pgfqpoint{0.000000in}{0.000000in}}{%
\pgfpathmoveto{\pgfqpoint{0.000000in}{0.000000in}}%
\pgfpathlineto{\pgfqpoint{0.000000in}{-0.027778in}}%
\pgfusepath{stroke,fill}%
}%
\begin{pgfscope}%
\pgfsys@transformshift{3.915547in}{0.417642in}%
\pgfsys@useobject{currentmarker}{}%
\end{pgfscope}%
\end{pgfscope}%
\begin{pgfscope}%
\definecolor{textcolor}{rgb}{0.000000,0.000000,0.000000}%
\pgfsetstrokecolor{textcolor}%
\pgfsetfillcolor{textcolor}%
\pgftext[x=2.356006in,y=0.165003in,,top]{\color{textcolor}\rmfamily\fontsize{10.000000}{12.000000}\selectfont \(\displaystyle \tau\) in \unit{\second}}%
\end{pgfscope}%
\begin{pgfscope}%
\pgfpathrectangle{\pgfqpoint{0.770608in}{0.417642in}}{\pgfqpoint{3.170796in}{2.021097in}}%
\pgfusepath{clip}%
\pgfsetrectcap%
\pgfsetroundjoin%
\pgfsetlinewidth{0.803000pt}%
\definecolor{currentstroke}{rgb}{0.850000,0.850000,0.850000}%
\pgfsetstrokecolor{currentstroke}%
\pgfsetdash{}{0pt}%
\pgfpathmoveto{\pgfqpoint{0.770608in}{0.709769in}}%
\pgfpathlineto{\pgfqpoint{3.941404in}{0.709769in}}%
\pgfusepath{stroke}%
\end{pgfscope}%
\begin{pgfscope}%
\pgfsetbuttcap%
\pgfsetroundjoin%
\definecolor{currentfill}{rgb}{0.000000,0.000000,0.000000}%
\pgfsetfillcolor{currentfill}%
\pgfsetlinewidth{0.602250pt}%
\definecolor{currentstroke}{rgb}{0.000000,0.000000,0.000000}%
\pgfsetstrokecolor{currentstroke}%
\pgfsetdash{}{0pt}%
\pgfsys@defobject{currentmarker}{\pgfqpoint{-0.027778in}{0.000000in}}{\pgfqpoint{-0.000000in}{0.000000in}}{%
\pgfpathmoveto{\pgfqpoint{-0.000000in}{0.000000in}}%
\pgfpathlineto{\pgfqpoint{-0.027778in}{0.000000in}}%
\pgfusepath{stroke,fill}%
}%
\begin{pgfscope}%
\pgfsys@transformshift{0.770608in}{0.709769in}%
\pgfsys@useobject{currentmarker}{}%
\end{pgfscope}%
\end{pgfscope}%
\begin{pgfscope}%
\definecolor{textcolor}{rgb}{0.000000,0.000000,0.000000}%
\pgfsetstrokecolor{textcolor}%
\pgfsetfillcolor{textcolor}%
\pgftext[x=0.236114in, y=0.664829in, left, base]{\color{textcolor}\rmfamily\fontsize{8.000000}{9.600000}\selectfont \(\displaystyle {2\times10^{-7}}\)}%
\end{pgfscope}%
\begin{pgfscope}%
\pgfpathrectangle{\pgfqpoint{0.770608in}{0.417642in}}{\pgfqpoint{3.170796in}{2.021097in}}%
\pgfusepath{clip}%
\pgfsetrectcap%
\pgfsetroundjoin%
\pgfsetlinewidth{0.803000pt}%
\definecolor{currentstroke}{rgb}{0.850000,0.850000,0.850000}%
\pgfsetstrokecolor{currentstroke}%
\pgfsetdash{}{0pt}%
\pgfpathmoveto{\pgfqpoint{0.770608in}{1.714694in}}%
\pgfpathlineto{\pgfqpoint{3.941404in}{1.714694in}}%
\pgfusepath{stroke}%
\end{pgfscope}%
\begin{pgfscope}%
\pgfsetbuttcap%
\pgfsetroundjoin%
\definecolor{currentfill}{rgb}{0.000000,0.000000,0.000000}%
\pgfsetfillcolor{currentfill}%
\pgfsetlinewidth{0.602250pt}%
\definecolor{currentstroke}{rgb}{0.000000,0.000000,0.000000}%
\pgfsetstrokecolor{currentstroke}%
\pgfsetdash{}{0pt}%
\pgfsys@defobject{currentmarker}{\pgfqpoint{-0.027778in}{0.000000in}}{\pgfqpoint{-0.000000in}{0.000000in}}{%
\pgfpathmoveto{\pgfqpoint{-0.000000in}{0.000000in}}%
\pgfpathlineto{\pgfqpoint{-0.027778in}{0.000000in}}%
\pgfusepath{stroke,fill}%
}%
\begin{pgfscope}%
\pgfsys@transformshift{0.770608in}{1.714694in}%
\pgfsys@useobject{currentmarker}{}%
\end{pgfscope}%
\end{pgfscope}%
\begin{pgfscope}%
\definecolor{textcolor}{rgb}{0.000000,0.000000,0.000000}%
\pgfsetstrokecolor{textcolor}%
\pgfsetfillcolor{textcolor}%
\pgftext[x=0.236114in, y=1.669754in, left, base]{\color{textcolor}\rmfamily\fontsize{8.000000}{9.600000}\selectfont \(\displaystyle {3\times10^{-7}}\)}%
\end{pgfscope}%
\begin{pgfscope}%
\pgfpathrectangle{\pgfqpoint{0.770608in}{0.417642in}}{\pgfqpoint{3.170796in}{2.021097in}}%
\pgfusepath{clip}%
\pgfsetrectcap%
\pgfsetroundjoin%
\pgfsetlinewidth{0.803000pt}%
\definecolor{currentstroke}{rgb}{0.850000,0.850000,0.850000}%
\pgfsetstrokecolor{currentstroke}%
\pgfsetdash{}{0pt}%
\pgfpathmoveto{\pgfqpoint{0.770608in}{2.427699in}}%
\pgfpathlineto{\pgfqpoint{3.941404in}{2.427699in}}%
\pgfusepath{stroke}%
\end{pgfscope}%
\begin{pgfscope}%
\pgfsetbuttcap%
\pgfsetroundjoin%
\definecolor{currentfill}{rgb}{0.000000,0.000000,0.000000}%
\pgfsetfillcolor{currentfill}%
\pgfsetlinewidth{0.602250pt}%
\definecolor{currentstroke}{rgb}{0.000000,0.000000,0.000000}%
\pgfsetstrokecolor{currentstroke}%
\pgfsetdash{}{0pt}%
\pgfsys@defobject{currentmarker}{\pgfqpoint{-0.027778in}{0.000000in}}{\pgfqpoint{-0.000000in}{0.000000in}}{%
\pgfpathmoveto{\pgfqpoint{-0.000000in}{0.000000in}}%
\pgfpathlineto{\pgfqpoint{-0.027778in}{0.000000in}}%
\pgfusepath{stroke,fill}%
}%
\begin{pgfscope}%
\pgfsys@transformshift{0.770608in}{2.427699in}%
\pgfsys@useobject{currentmarker}{}%
\end{pgfscope}%
\end{pgfscope}%
\begin{pgfscope}%
\definecolor{textcolor}{rgb}{0.000000,0.000000,0.000000}%
\pgfsetstrokecolor{textcolor}%
\pgfsetfillcolor{textcolor}%
\pgftext[x=0.236114in, y=2.382759in, left, base]{\color{textcolor}\rmfamily\fontsize{8.000000}{9.600000}\selectfont \(\displaystyle {4\times10^{-7}}\)}%
\end{pgfscope}%
\begin{pgfscope}%
\definecolor{textcolor}{rgb}{0.000000,0.000000,0.000000}%
\pgfsetstrokecolor{textcolor}%
\pgfsetfillcolor{textcolor}%
\pgftext[x=0.180559in,y=1.428190in,,bottom,rotate=90.000000]{\color{textcolor}\rmfamily\fontsize{10.000000}{12.000000}\selectfont ADEV \(\displaystyle \sigma_A(\tau)\) in \unit{\V}}%
\end{pgfscope}%
\begin{pgfscope}%
\pgfpathrectangle{\pgfqpoint{0.770608in}{0.417642in}}{\pgfqpoint{3.170796in}{2.021097in}}%
\pgfusepath{clip}%
\pgfsetbuttcap%
\pgfsetroundjoin%
\definecolor{currentfill}{rgb}{0.835294,0.368627,0.000000}%
\pgfsetfillcolor{currentfill}%
\pgfsetlinewidth{1.003750pt}%
\definecolor{currentstroke}{rgb}{0.835294,0.368627,0.000000}%
\pgfsetstrokecolor{currentstroke}%
\pgfsetdash{}{0pt}%
\pgfsys@defobject{currentmarker}{\pgfqpoint{-0.020833in}{-0.020833in}}{\pgfqpoint{0.020833in}{0.020833in}}{%
\pgfpathmoveto{\pgfqpoint{0.000000in}{-0.020833in}}%
\pgfpathcurveto{\pgfqpoint{0.005525in}{-0.020833in}}{\pgfqpoint{0.010825in}{-0.018638in}}{\pgfqpoint{0.014731in}{-0.014731in}}%
\pgfpathcurveto{\pgfqpoint{0.018638in}{-0.010825in}}{\pgfqpoint{0.020833in}{-0.005525in}}{\pgfqpoint{0.020833in}{0.000000in}}%
\pgfpathcurveto{\pgfqpoint{0.020833in}{0.005525in}}{\pgfqpoint{0.018638in}{0.010825in}}{\pgfqpoint{0.014731in}{0.014731in}}%
\pgfpathcurveto{\pgfqpoint{0.010825in}{0.018638in}}{\pgfqpoint{0.005525in}{0.020833in}}{\pgfqpoint{0.000000in}{0.020833in}}%
\pgfpathcurveto{\pgfqpoint{-0.005525in}{0.020833in}}{\pgfqpoint{-0.010825in}{0.018638in}}{\pgfqpoint{-0.014731in}{0.014731in}}%
\pgfpathcurveto{\pgfqpoint{-0.018638in}{0.010825in}}{\pgfqpoint{-0.020833in}{0.005525in}}{\pgfqpoint{-0.020833in}{0.000000in}}%
\pgfpathcurveto{\pgfqpoint{-0.020833in}{-0.005525in}}{\pgfqpoint{-0.018638in}{-0.010825in}}{\pgfqpoint{-0.014731in}{-0.014731in}}%
\pgfpathcurveto{\pgfqpoint{-0.010825in}{-0.018638in}}{\pgfqpoint{-0.005525in}{-0.020833in}}{\pgfqpoint{0.000000in}{-0.020833in}}%
\pgfpathlineto{\pgfqpoint{0.000000in}{-0.020833in}}%
\pgfpathclose%
\pgfusepath{stroke,fill}%
}%
\begin{pgfscope}%
\pgfsys@transformshift{1.066448in}{2.346871in}%
\pgfsys@useobject{currentmarker}{}%
\end{pgfscope}%
\begin{pgfscope}%
\pgfsys@transformshift{1.218161in}{1.838602in}%
\pgfsys@useobject{currentmarker}{}%
\end{pgfscope}%
\begin{pgfscope}%
\pgfsys@transformshift{1.369874in}{1.512680in}%
\pgfsys@useobject{currentmarker}{}%
\end{pgfscope}%
\begin{pgfscope}%
\pgfsys@transformshift{1.521586in}{1.332431in}%
\pgfsys@useobject{currentmarker}{}%
\end{pgfscope}%
\begin{pgfscope}%
\pgfsys@transformshift{1.673299in}{1.238028in}%
\pgfsys@useobject{currentmarker}{}%
\end{pgfscope}%
\begin{pgfscope}%
\pgfsys@transformshift{1.825012in}{1.183337in}%
\pgfsys@useobject{currentmarker}{}%
\end{pgfscope}%
\begin{pgfscope}%
\pgfsys@transformshift{1.976725in}{1.169752in}%
\pgfsys@useobject{currentmarker}{}%
\end{pgfscope}%
\begin{pgfscope}%
\pgfsys@transformshift{2.128437in}{1.137687in}%
\pgfsys@useobject{currentmarker}{}%
\end{pgfscope}%
\begin{pgfscope}%
\pgfsys@transformshift{2.280150in}{1.076251in}%
\pgfsys@useobject{currentmarker}{}%
\end{pgfscope}%
\begin{pgfscope}%
\pgfsys@transformshift{2.431863in}{1.092924in}%
\pgfsys@useobject{currentmarker}{}%
\end{pgfscope}%
\begin{pgfscope}%
\pgfsys@transformshift{2.583575in}{1.080349in}%
\pgfsys@useobject{currentmarker}{}%
\end{pgfscope}%
\begin{pgfscope}%
\pgfsys@transformshift{2.735288in}{1.080493in}%
\pgfsys@useobject{currentmarker}{}%
\end{pgfscope}%
\begin{pgfscope}%
\pgfsys@transformshift{2.887001in}{1.080643in}%
\pgfsys@useobject{currentmarker}{}%
\end{pgfscope}%
\begin{pgfscope}%
\pgfsys@transformshift{3.038714in}{1.029718in}%
\pgfsys@useobject{currentmarker}{}%
\end{pgfscope}%
\begin{pgfscope}%
\pgfsys@transformshift{3.190426in}{1.138016in}%
\pgfsys@useobject{currentmarker}{}%
\end{pgfscope}%
\begin{pgfscope}%
\pgfsys@transformshift{3.342139in}{1.275055in}%
\pgfsys@useobject{currentmarker}{}%
\end{pgfscope}%
\begin{pgfscope}%
\pgfsys@transformshift{3.493852in}{1.341175in}%
\pgfsys@useobject{currentmarker}{}%
\end{pgfscope}%
\begin{pgfscope}%
\pgfsys@transformshift{3.645565in}{1.621756in}%
\pgfsys@useobject{currentmarker}{}%
\end{pgfscope}%
\begin{pgfscope}%
\pgfsys@transformshift{3.797277in}{0.824775in}%
\pgfsys@useobject{currentmarker}{}%
\end{pgfscope}%
\end{pgfscope}%
\begin{pgfscope}%
\pgfpathrectangle{\pgfqpoint{0.770608in}{0.417642in}}{\pgfqpoint{3.170796in}{2.021097in}}%
\pgfusepath{clip}%
\pgfsetbuttcap%
\pgfsetroundjoin%
\pgfsetlinewidth{1.505625pt}%
\definecolor{currentstroke}{rgb}{0.003922,0.450980,0.698039}%
\pgfsetstrokecolor{currentstroke}%
\pgfsetdash{{5.550000pt}{2.400000pt}}{0.000000pt}%
\pgfpathmoveto{\pgfqpoint{0.914735in}{2.227440in}}%
\pgfpathlineto{\pgfqpoint{1.066448in}{1.368475in}}%
\pgfpathlineto{\pgfqpoint{1.218161in}{0.509510in}}%
\pgfusepath{stroke}%
\end{pgfscope}%
\begin{pgfscope}%
\pgfpathrectangle{\pgfqpoint{0.770608in}{0.417642in}}{\pgfqpoint{3.170796in}{2.021097in}}%
\pgfusepath{clip}%
\pgfsetbuttcap%
\pgfsetroundjoin%
\pgfsetlinewidth{1.505625pt}%
\definecolor{currentstroke}{rgb}{0.007843,0.619608,0.450980}%
\pgfsetstrokecolor{currentstroke}%
\pgfsetdash{{5.550000pt}{2.400000pt}}{0.000000pt}%
\pgfpathmoveto{\pgfqpoint{1.066448in}{1.140221in}}%
\pgfpathlineto{\pgfqpoint{1.218161in}{1.140221in}}%
\pgfpathlineto{\pgfqpoint{1.369874in}{1.140221in}}%
\pgfpathlineto{\pgfqpoint{1.521586in}{1.140221in}}%
\pgfpathlineto{\pgfqpoint{1.673299in}{1.140221in}}%
\pgfpathlineto{\pgfqpoint{1.825012in}{1.140221in}}%
\pgfpathlineto{\pgfqpoint{1.976725in}{1.140221in}}%
\pgfpathlineto{\pgfqpoint{2.128437in}{1.140221in}}%
\pgfpathlineto{\pgfqpoint{2.280150in}{1.140221in}}%
\pgfpathlineto{\pgfqpoint{2.431863in}{1.140221in}}%
\pgfpathlineto{\pgfqpoint{2.583575in}{1.140221in}}%
\pgfpathlineto{\pgfqpoint{2.735288in}{1.140221in}}%
\pgfpathlineto{\pgfqpoint{2.887001in}{1.140221in}}%
\pgfpathlineto{\pgfqpoint{3.038714in}{1.140221in}}%
\pgfpathlineto{\pgfqpoint{3.190426in}{1.140221in}}%
\pgfpathlineto{\pgfqpoint{3.342139in}{1.140221in}}%
\pgfpathlineto{\pgfqpoint{3.493852in}{1.140221in}}%
\pgfpathlineto{\pgfqpoint{3.645565in}{1.140221in}}%
\pgfpathlineto{\pgfqpoint{3.797277in}{1.140221in}}%
\pgfusepath{stroke}%
\end{pgfscope}%
\begin{pgfscope}%
\pgfsetrectcap%
\pgfsetmiterjoin%
\pgfsetlinewidth{0.803000pt}%
\definecolor{currentstroke}{rgb}{0.000000,0.000000,0.000000}%
\pgfsetstrokecolor{currentstroke}%
\pgfsetdash{}{0pt}%
\pgfpathmoveto{\pgfqpoint{0.770608in}{0.417642in}}%
\pgfpathlineto{\pgfqpoint{0.770608in}{2.438739in}}%
\pgfusepath{stroke}%
\end{pgfscope}%
\begin{pgfscope}%
\pgfsetrectcap%
\pgfsetmiterjoin%
\pgfsetlinewidth{0.803000pt}%
\definecolor{currentstroke}{rgb}{0.000000,0.000000,0.000000}%
\pgfsetstrokecolor{currentstroke}%
\pgfsetdash{}{0pt}%
\pgfpathmoveto{\pgfqpoint{3.941404in}{0.417642in}}%
\pgfpathlineto{\pgfqpoint{3.941404in}{2.438739in}}%
\pgfusepath{stroke}%
\end{pgfscope}%
\begin{pgfscope}%
\pgfsetrectcap%
\pgfsetmiterjoin%
\pgfsetlinewidth{0.803000pt}%
\definecolor{currentstroke}{rgb}{0.000000,0.000000,0.000000}%
\pgfsetstrokecolor{currentstroke}%
\pgfsetdash{}{0pt}%
\pgfpathmoveto{\pgfqpoint{0.770608in}{0.417642in}}%
\pgfpathlineto{\pgfqpoint{3.941404in}{0.417642in}}%
\pgfusepath{stroke}%
\end{pgfscope}%
\begin{pgfscope}%
\pgfsetrectcap%
\pgfsetmiterjoin%
\pgfsetlinewidth{0.803000pt}%
\definecolor{currentstroke}{rgb}{0.000000,0.000000,0.000000}%
\pgfsetstrokecolor{currentstroke}%
\pgfsetdash{}{0pt}%
\pgfpathmoveto{\pgfqpoint{0.770608in}{2.438739in}}%
\pgfpathlineto{\pgfqpoint{3.941404in}{2.438739in}}%
\pgfusepath{stroke}%
\end{pgfscope}%
\begin{pgfscope}%
\pgfsetbuttcap%
\pgfsetmiterjoin%
\definecolor{currentfill}{rgb}{1.000000,1.000000,1.000000}%
\pgfsetfillcolor{currentfill}%
\pgfsetfillopacity{0.800000}%
\pgfsetlinewidth{1.003750pt}%
\definecolor{currentstroke}{rgb}{0.800000,0.800000,0.800000}%
\pgfsetstrokecolor{currentstroke}%
\pgfsetstrokeopacity{0.800000}%
\pgfsetdash{}{0pt}%
\pgfpathmoveto{\pgfqpoint{2.863293in}{2.040073in}}%
\pgfpathlineto{\pgfqpoint{3.863627in}{2.040073in}}%
\pgfpathquadraticcurveto{\pgfqpoint{3.885849in}{2.040073in}}{\pgfqpoint{3.885849in}{2.062295in}}%
\pgfpathlineto{\pgfqpoint{3.885849in}{2.360961in}}%
\pgfpathquadraticcurveto{\pgfqpoint{3.885849in}{2.383183in}}{\pgfqpoint{3.863627in}{2.383183in}}%
\pgfpathlineto{\pgfqpoint{2.863293in}{2.383183in}}%
\pgfpathquadraticcurveto{\pgfqpoint{2.841071in}{2.383183in}}{\pgfqpoint{2.841071in}{2.360961in}}%
\pgfpathlineto{\pgfqpoint{2.841071in}{2.062295in}}%
\pgfpathquadraticcurveto{\pgfqpoint{2.841071in}{2.040073in}}{\pgfqpoint{2.863293in}{2.040073in}}%
\pgfpathlineto{\pgfqpoint{2.863293in}{2.040073in}}%
\pgfpathclose%
\pgfusepath{stroke,fill}%
\end{pgfscope}%
\begin{pgfscope}%
\pgfsetbuttcap%
\pgfsetroundjoin%
\pgfsetlinewidth{1.505625pt}%
\definecolor{currentstroke}{rgb}{0.003922,0.450980,0.698039}%
\pgfsetstrokecolor{currentstroke}%
\pgfsetdash{{5.550000pt}{2.400000pt}}{0.000000pt}%
\pgfpathmoveto{\pgfqpoint{2.885516in}{2.299850in}}%
\pgfpathlineto{\pgfqpoint{2.996627in}{2.299850in}}%
\pgfpathlineto{\pgfqpoint{3.107738in}{2.299850in}}%
\pgfusepath{stroke}%
\end{pgfscope}%
\begin{pgfscope}%
\definecolor{textcolor}{rgb}{0.000000,0.000000,0.000000}%
\pgfsetstrokecolor{textcolor}%
\pgfsetfillcolor{textcolor}%
\pgftext[x=3.196627in,y=2.260961in,left,base]{\color{textcolor}\rmfamily\fontsize{8.000000}{9.600000}\selectfont White noise}%
\end{pgfscope}%
\begin{pgfscope}%
\pgfsetbuttcap%
\pgfsetroundjoin%
\pgfsetlinewidth{1.505625pt}%
\definecolor{currentstroke}{rgb}{0.007843,0.619608,0.450980}%
\pgfsetstrokecolor{currentstroke}%
\pgfsetdash{{5.550000pt}{2.400000pt}}{0.000000pt}%
\pgfpathmoveto{\pgfqpoint{2.885516in}{2.144961in}}%
\pgfpathlineto{\pgfqpoint{2.996627in}{2.144961in}}%
\pgfpathlineto{\pgfqpoint{3.107738in}{2.144961in}}%
\pgfusepath{stroke}%
\end{pgfscope}%
\begin{pgfscope}%
\definecolor{textcolor}{rgb}{0.000000,0.000000,0.000000}%
\pgfsetstrokecolor{textcolor}%
\pgfsetfillcolor{textcolor}%
\pgftext[x=3.196627in,y=2.106072in,left,base]{\color{textcolor}\rmfamily\fontsize{8.000000}{9.600000}\selectfont Flicker noise}%
\end{pgfscope}%
\end{pgfpicture}%
\makeatother%
\endgroup%
% data/simulations/sim_autozero.py
    \caption{Simulated Allan deviation of the input amplifier of a Keysight \device{3458A} containing white noise and flicker noise. The line frequency is \qty{50}{\Hz}.}
    \label{fig:autozero_raw_adev}
\end{figure}

The Allan deviation it plotted in figure \ref{fig:autozero_raw_adev} and shows two distinct regions. Short $\tau$ display an asymptotic behaviour towards white noise with a $\tau^{−0.5}$ dependence and at longer $\tau$ the constant flicker noise region can be identified. At very long $\tau$ typical end-of-data oscillations can be seen, which are the result of the limited confidence of the Allan deviation estimator as previously discussed and can therefore be safely ignored. The Allan deviation clearly demonstrates the performance of the device at longer integration times and it is obvious that beyond an integration time of about \qty{1}{\second} or \qty{50}{\plc} no additional information can be extracted from the measurement and the variance is constant. This leads to the need for autozeroing to remove the flicker noise. It can be shown \cite{autozero_with_dead_time} that subtracting a reference measurement from the actual measurement data removes all correlated effects. Since flicker noise is autocorrelated, it can be removed by subtracting a zero measurement.

To demonstrate autozeroing, two cases will be discussed. Going back to figure \ref{fig:dmm_autozer_offset_nulling} it can be seen that between switching inputs, a dead time $\theta$ is added. For a first discussion, this dead time is neglected and then the effect of adding a dead time is discussed in a next step.

Using figure \ref{fig:autozero_raw_adev} it was shown that integrating over flicker noise, does not reduce the variance. In order to have as little flicker noise content in the final measurement values as possible, it is clear that the autozeroing should be done as fast as feasible to keep the flicker noise content out. This allows to calculate the expected variance of the autozeroed measurement. The noise of the input measurement $x$ and the reference measurement $y$ are the same, because in this model the only noise source comes from the input amplifier, as the input signal is assumed to be noise-free. The zero level is, by definition, noise-free. As discussed above, the autozero interval is chosen, in such a way that its variance is dominated by white noise. The variance $\sigma^2$ of the combined measurement of $x-y$ can then be calculated using equation \ref{eqn:adding_white_noise}:
\begin{equation}
    \sigma_{x-y}^2 = \sigma_x^2 + \sigma_y^2 \label{eqn:autozeroing}
\end{equation}

By subtracting the zero reading the amplifier noise is effectively added twice to the final result, once for the input measurement and once for the zero measurement. Additional noise from the input signal noise would simply be added to this as it is uncorrelated as well.

Do note, that the number of samples is now half the number before applying autozeroing. This leads to an interesting effect. Taking for example a data set containing only white noise with a variance $\sigma^2$ and removing half the samples obviously does not change the variance as white noise is not correlated, but subtracting the samples is effectively decimating the data set and since the sampling rate is halved, the Nyquist band is halved as well. Unfortunately the input noise bandwidth stays the same. The second Nyquist band is then folded back into the first, thus doubling the noise power density.

To conclude, it is expected that the variance doubles and the power spectral density quadruples. These considerations can be compared to the simulated data. In order to apply the autozeroing algorithm to the simulated data set, the constant noise-free part, namely the \qty{10}{\V} signal, of the noisy input signal was nulled for every odd value and the residual noise was subtracted from the signal value. The result in the time domain is shown in figure \ref{fig:autozero_time}.
\begin{figure}[hb]
    \centering
    %% Creator: Matplotlib, PGF backend
%%
%% To include the figure in your LaTeX document, write
%%   \input{<filename>.pgf}
%%
%% Make sure the required packages are loaded in your preamble
%%   \usepackage{pgf}
%%
%% Also ensure that all the required font packages are loaded; for instance,
%% the lmodern package is sometimes necessary when using math font.
%%   \usepackage{lmodern}
%%
%% Figures using additional raster images can only be included by \input if
%% they are in the same directory as the main LaTeX file. For loading figures
%% from other directories you can use the `import` package
%%   \usepackage{import}
%%
%% and then include the figures with
%%   \import{<path to file>}{<filename>.pgf}
%%
%% Matplotlib used the following preamble
%%   \usepackage{siunitx}
%%   \sisetup{per-mode = symbol}%
%%   \usepackage{fontspec}
%%   \makeatletter\@ifpackageloaded{underscore}{}{\usepackage[strings]{underscore}}\makeatother
%%
\begingroup%
\makeatletter%
\begin{pgfpicture}%
\pgfpathrectangle{\pgfpointorigin}{\pgfqpoint{4.068242in}{2.514312in}}%
\pgfusepath{use as bounding box, clip}%
\begin{pgfscope}%
\pgfsetbuttcap%
\pgfsetmiterjoin%
\definecolor{currentfill}{rgb}{1.000000,1.000000,1.000000}%
\pgfsetfillcolor{currentfill}%
\pgfsetlinewidth{0.000000pt}%
\definecolor{currentstroke}{rgb}{1.000000,1.000000,1.000000}%
\pgfsetstrokecolor{currentstroke}%
\pgfsetdash{}{0pt}%
\pgfpathmoveto{\pgfqpoint{0.000000in}{0.000000in}}%
\pgfpathlineto{\pgfqpoint{4.068242in}{0.000000in}}%
\pgfpathlineto{\pgfqpoint{4.068242in}{2.514312in}}%
\pgfpathlineto{\pgfqpoint{0.000000in}{2.514312in}}%
\pgfpathlineto{\pgfqpoint{0.000000in}{0.000000in}}%
\pgfpathclose%
\pgfusepath{fill}%
\end{pgfscope}%
\begin{pgfscope}%
\pgfsetbuttcap%
\pgfsetmiterjoin%
\definecolor{currentfill}{rgb}{1.000000,1.000000,1.000000}%
\pgfsetfillcolor{currentfill}%
\pgfsetlinewidth{0.000000pt}%
\definecolor{currentstroke}{rgb}{0.000000,0.000000,0.000000}%
\pgfsetstrokecolor{currentstroke}%
\pgfsetstrokeopacity{0.000000}%
\pgfsetdash{}{0pt}%
\pgfpathmoveto{\pgfqpoint{0.471687in}{0.416447in}}%
\pgfpathlineto{\pgfqpoint{4.009530in}{0.416447in}}%
\pgfpathlineto{\pgfqpoint{4.009530in}{2.341095in}}%
\pgfpathlineto{\pgfqpoint{0.471687in}{2.341095in}}%
\pgfpathlineto{\pgfqpoint{0.471687in}{0.416447in}}%
\pgfpathclose%
\pgfusepath{fill}%
\end{pgfscope}%
\begin{pgfscope}%
\pgfpathrectangle{\pgfqpoint{0.471687in}{0.416447in}}{\pgfqpoint{3.537842in}{1.924647in}}%
\pgfusepath{clip}%
\pgfsetrectcap%
\pgfsetroundjoin%
\pgfsetlinewidth{0.803000pt}%
\definecolor{currentstroke}{rgb}{0.450000,0.450000,0.450000}%
\pgfsetstrokecolor{currentstroke}%
\pgfsetdash{}{0pt}%
\pgfpathmoveto{\pgfqpoint{0.632499in}{0.416447in}}%
\pgfpathlineto{\pgfqpoint{0.632499in}{2.341095in}}%
\pgfusepath{stroke}%
\end{pgfscope}%
\begin{pgfscope}%
\pgfsetbuttcap%
\pgfsetroundjoin%
\definecolor{currentfill}{rgb}{0.000000,0.000000,0.000000}%
\pgfsetfillcolor{currentfill}%
\pgfsetlinewidth{0.803000pt}%
\definecolor{currentstroke}{rgb}{0.000000,0.000000,0.000000}%
\pgfsetstrokecolor{currentstroke}%
\pgfsetdash{}{0pt}%
\pgfsys@defobject{currentmarker}{\pgfqpoint{0.000000in}{-0.048611in}}{\pgfqpoint{0.000000in}{0.000000in}}{%
\pgfpathmoveto{\pgfqpoint{0.000000in}{0.000000in}}%
\pgfpathlineto{\pgfqpoint{0.000000in}{-0.048611in}}%
\pgfusepath{stroke,fill}%
}%
\begin{pgfscope}%
\pgfsys@transformshift{0.632499in}{0.416447in}%
\pgfsys@useobject{currentmarker}{}%
\end{pgfscope}%
\end{pgfscope}%
\begin{pgfscope}%
\definecolor{textcolor}{rgb}{0.000000,0.000000,0.000000}%
\pgfsetstrokecolor{textcolor}%
\pgfsetfillcolor{textcolor}%
\pgftext[x=0.632499in,y=0.319225in,,top]{\color{textcolor}\rmfamily\fontsize{8.000000}{9.600000}\selectfont \(\displaystyle {0}\)}%
\end{pgfscope}%
\begin{pgfscope}%
\pgfpathrectangle{\pgfqpoint{0.471687in}{0.416447in}}{\pgfqpoint{3.537842in}{1.924647in}}%
\pgfusepath{clip}%
\pgfsetrectcap%
\pgfsetroundjoin%
\pgfsetlinewidth{0.803000pt}%
\definecolor{currentstroke}{rgb}{0.450000,0.450000,0.450000}%
\pgfsetstrokecolor{currentstroke}%
\pgfsetdash{}{0pt}%
\pgfpathmoveto{\pgfqpoint{1.034527in}{0.416447in}}%
\pgfpathlineto{\pgfqpoint{1.034527in}{2.341095in}}%
\pgfusepath{stroke}%
\end{pgfscope}%
\begin{pgfscope}%
\pgfsetbuttcap%
\pgfsetroundjoin%
\definecolor{currentfill}{rgb}{0.000000,0.000000,0.000000}%
\pgfsetfillcolor{currentfill}%
\pgfsetlinewidth{0.803000pt}%
\definecolor{currentstroke}{rgb}{0.000000,0.000000,0.000000}%
\pgfsetstrokecolor{currentstroke}%
\pgfsetdash{}{0pt}%
\pgfsys@defobject{currentmarker}{\pgfqpoint{0.000000in}{-0.048611in}}{\pgfqpoint{0.000000in}{0.000000in}}{%
\pgfpathmoveto{\pgfqpoint{0.000000in}{0.000000in}}%
\pgfpathlineto{\pgfqpoint{0.000000in}{-0.048611in}}%
\pgfusepath{stroke,fill}%
}%
\begin{pgfscope}%
\pgfsys@transformshift{1.034527in}{0.416447in}%
\pgfsys@useobject{currentmarker}{}%
\end{pgfscope}%
\end{pgfscope}%
\begin{pgfscope}%
\definecolor{textcolor}{rgb}{0.000000,0.000000,0.000000}%
\pgfsetstrokecolor{textcolor}%
\pgfsetfillcolor{textcolor}%
\pgftext[x=1.034527in,y=0.319225in,,top]{\color{textcolor}\rmfamily\fontsize{8.000000}{9.600000}\selectfont \(\displaystyle {25000}\)}%
\end{pgfscope}%
\begin{pgfscope}%
\pgfpathrectangle{\pgfqpoint{0.471687in}{0.416447in}}{\pgfqpoint{3.537842in}{1.924647in}}%
\pgfusepath{clip}%
\pgfsetrectcap%
\pgfsetroundjoin%
\pgfsetlinewidth{0.803000pt}%
\definecolor{currentstroke}{rgb}{0.450000,0.450000,0.450000}%
\pgfsetstrokecolor{currentstroke}%
\pgfsetdash{}{0pt}%
\pgfpathmoveto{\pgfqpoint{1.436555in}{0.416447in}}%
\pgfpathlineto{\pgfqpoint{1.436555in}{2.341095in}}%
\pgfusepath{stroke}%
\end{pgfscope}%
\begin{pgfscope}%
\pgfsetbuttcap%
\pgfsetroundjoin%
\definecolor{currentfill}{rgb}{0.000000,0.000000,0.000000}%
\pgfsetfillcolor{currentfill}%
\pgfsetlinewidth{0.803000pt}%
\definecolor{currentstroke}{rgb}{0.000000,0.000000,0.000000}%
\pgfsetstrokecolor{currentstroke}%
\pgfsetdash{}{0pt}%
\pgfsys@defobject{currentmarker}{\pgfqpoint{0.000000in}{-0.048611in}}{\pgfqpoint{0.000000in}{0.000000in}}{%
\pgfpathmoveto{\pgfqpoint{0.000000in}{0.000000in}}%
\pgfpathlineto{\pgfqpoint{0.000000in}{-0.048611in}}%
\pgfusepath{stroke,fill}%
}%
\begin{pgfscope}%
\pgfsys@transformshift{1.436555in}{0.416447in}%
\pgfsys@useobject{currentmarker}{}%
\end{pgfscope}%
\end{pgfscope}%
\begin{pgfscope}%
\definecolor{textcolor}{rgb}{0.000000,0.000000,0.000000}%
\pgfsetstrokecolor{textcolor}%
\pgfsetfillcolor{textcolor}%
\pgftext[x=1.436555in,y=0.319225in,,top]{\color{textcolor}\rmfamily\fontsize{8.000000}{9.600000}\selectfont \(\displaystyle {50000}\)}%
\end{pgfscope}%
\begin{pgfscope}%
\pgfpathrectangle{\pgfqpoint{0.471687in}{0.416447in}}{\pgfqpoint{3.537842in}{1.924647in}}%
\pgfusepath{clip}%
\pgfsetrectcap%
\pgfsetroundjoin%
\pgfsetlinewidth{0.803000pt}%
\definecolor{currentstroke}{rgb}{0.450000,0.450000,0.450000}%
\pgfsetstrokecolor{currentstroke}%
\pgfsetdash{}{0pt}%
\pgfpathmoveto{\pgfqpoint{1.838583in}{0.416447in}}%
\pgfpathlineto{\pgfqpoint{1.838583in}{2.341095in}}%
\pgfusepath{stroke}%
\end{pgfscope}%
\begin{pgfscope}%
\pgfsetbuttcap%
\pgfsetroundjoin%
\definecolor{currentfill}{rgb}{0.000000,0.000000,0.000000}%
\pgfsetfillcolor{currentfill}%
\pgfsetlinewidth{0.803000pt}%
\definecolor{currentstroke}{rgb}{0.000000,0.000000,0.000000}%
\pgfsetstrokecolor{currentstroke}%
\pgfsetdash{}{0pt}%
\pgfsys@defobject{currentmarker}{\pgfqpoint{0.000000in}{-0.048611in}}{\pgfqpoint{0.000000in}{0.000000in}}{%
\pgfpathmoveto{\pgfqpoint{0.000000in}{0.000000in}}%
\pgfpathlineto{\pgfqpoint{0.000000in}{-0.048611in}}%
\pgfusepath{stroke,fill}%
}%
\begin{pgfscope}%
\pgfsys@transformshift{1.838583in}{0.416447in}%
\pgfsys@useobject{currentmarker}{}%
\end{pgfscope}%
\end{pgfscope}%
\begin{pgfscope}%
\definecolor{textcolor}{rgb}{0.000000,0.000000,0.000000}%
\pgfsetstrokecolor{textcolor}%
\pgfsetfillcolor{textcolor}%
\pgftext[x=1.838583in,y=0.319225in,,top]{\color{textcolor}\rmfamily\fontsize{8.000000}{9.600000}\selectfont \(\displaystyle {75000}\)}%
\end{pgfscope}%
\begin{pgfscope}%
\pgfpathrectangle{\pgfqpoint{0.471687in}{0.416447in}}{\pgfqpoint{3.537842in}{1.924647in}}%
\pgfusepath{clip}%
\pgfsetrectcap%
\pgfsetroundjoin%
\pgfsetlinewidth{0.803000pt}%
\definecolor{currentstroke}{rgb}{0.450000,0.450000,0.450000}%
\pgfsetstrokecolor{currentstroke}%
\pgfsetdash{}{0pt}%
\pgfpathmoveto{\pgfqpoint{2.240612in}{0.416447in}}%
\pgfpathlineto{\pgfqpoint{2.240612in}{2.341095in}}%
\pgfusepath{stroke}%
\end{pgfscope}%
\begin{pgfscope}%
\pgfsetbuttcap%
\pgfsetroundjoin%
\definecolor{currentfill}{rgb}{0.000000,0.000000,0.000000}%
\pgfsetfillcolor{currentfill}%
\pgfsetlinewidth{0.803000pt}%
\definecolor{currentstroke}{rgb}{0.000000,0.000000,0.000000}%
\pgfsetstrokecolor{currentstroke}%
\pgfsetdash{}{0pt}%
\pgfsys@defobject{currentmarker}{\pgfqpoint{0.000000in}{-0.048611in}}{\pgfqpoint{0.000000in}{0.000000in}}{%
\pgfpathmoveto{\pgfqpoint{0.000000in}{0.000000in}}%
\pgfpathlineto{\pgfqpoint{0.000000in}{-0.048611in}}%
\pgfusepath{stroke,fill}%
}%
\begin{pgfscope}%
\pgfsys@transformshift{2.240612in}{0.416447in}%
\pgfsys@useobject{currentmarker}{}%
\end{pgfscope}%
\end{pgfscope}%
\begin{pgfscope}%
\definecolor{textcolor}{rgb}{0.000000,0.000000,0.000000}%
\pgfsetstrokecolor{textcolor}%
\pgfsetfillcolor{textcolor}%
\pgftext[x=2.240612in,y=0.319225in,,top]{\color{textcolor}\rmfamily\fontsize{8.000000}{9.600000}\selectfont \(\displaystyle {100000}\)}%
\end{pgfscope}%
\begin{pgfscope}%
\pgfpathrectangle{\pgfqpoint{0.471687in}{0.416447in}}{\pgfqpoint{3.537842in}{1.924647in}}%
\pgfusepath{clip}%
\pgfsetrectcap%
\pgfsetroundjoin%
\pgfsetlinewidth{0.803000pt}%
\definecolor{currentstroke}{rgb}{0.450000,0.450000,0.450000}%
\pgfsetstrokecolor{currentstroke}%
\pgfsetdash{}{0pt}%
\pgfpathmoveto{\pgfqpoint{2.642640in}{0.416447in}}%
\pgfpathlineto{\pgfqpoint{2.642640in}{2.341095in}}%
\pgfusepath{stroke}%
\end{pgfscope}%
\begin{pgfscope}%
\pgfsetbuttcap%
\pgfsetroundjoin%
\definecolor{currentfill}{rgb}{0.000000,0.000000,0.000000}%
\pgfsetfillcolor{currentfill}%
\pgfsetlinewidth{0.803000pt}%
\definecolor{currentstroke}{rgb}{0.000000,0.000000,0.000000}%
\pgfsetstrokecolor{currentstroke}%
\pgfsetdash{}{0pt}%
\pgfsys@defobject{currentmarker}{\pgfqpoint{0.000000in}{-0.048611in}}{\pgfqpoint{0.000000in}{0.000000in}}{%
\pgfpathmoveto{\pgfqpoint{0.000000in}{0.000000in}}%
\pgfpathlineto{\pgfqpoint{0.000000in}{-0.048611in}}%
\pgfusepath{stroke,fill}%
}%
\begin{pgfscope}%
\pgfsys@transformshift{2.642640in}{0.416447in}%
\pgfsys@useobject{currentmarker}{}%
\end{pgfscope}%
\end{pgfscope}%
\begin{pgfscope}%
\definecolor{textcolor}{rgb}{0.000000,0.000000,0.000000}%
\pgfsetstrokecolor{textcolor}%
\pgfsetfillcolor{textcolor}%
\pgftext[x=2.642640in,y=0.319225in,,top]{\color{textcolor}\rmfamily\fontsize{8.000000}{9.600000}\selectfont \(\displaystyle {125000}\)}%
\end{pgfscope}%
\begin{pgfscope}%
\pgfpathrectangle{\pgfqpoint{0.471687in}{0.416447in}}{\pgfqpoint{3.537842in}{1.924647in}}%
\pgfusepath{clip}%
\pgfsetrectcap%
\pgfsetroundjoin%
\pgfsetlinewidth{0.803000pt}%
\definecolor{currentstroke}{rgb}{0.450000,0.450000,0.450000}%
\pgfsetstrokecolor{currentstroke}%
\pgfsetdash{}{0pt}%
\pgfpathmoveto{\pgfqpoint{3.044668in}{0.416447in}}%
\pgfpathlineto{\pgfqpoint{3.044668in}{2.341095in}}%
\pgfusepath{stroke}%
\end{pgfscope}%
\begin{pgfscope}%
\pgfsetbuttcap%
\pgfsetroundjoin%
\definecolor{currentfill}{rgb}{0.000000,0.000000,0.000000}%
\pgfsetfillcolor{currentfill}%
\pgfsetlinewidth{0.803000pt}%
\definecolor{currentstroke}{rgb}{0.000000,0.000000,0.000000}%
\pgfsetstrokecolor{currentstroke}%
\pgfsetdash{}{0pt}%
\pgfsys@defobject{currentmarker}{\pgfqpoint{0.000000in}{-0.048611in}}{\pgfqpoint{0.000000in}{0.000000in}}{%
\pgfpathmoveto{\pgfqpoint{0.000000in}{0.000000in}}%
\pgfpathlineto{\pgfqpoint{0.000000in}{-0.048611in}}%
\pgfusepath{stroke,fill}%
}%
\begin{pgfscope}%
\pgfsys@transformshift{3.044668in}{0.416447in}%
\pgfsys@useobject{currentmarker}{}%
\end{pgfscope}%
\end{pgfscope}%
\begin{pgfscope}%
\definecolor{textcolor}{rgb}{0.000000,0.000000,0.000000}%
\pgfsetstrokecolor{textcolor}%
\pgfsetfillcolor{textcolor}%
\pgftext[x=3.044668in,y=0.319225in,,top]{\color{textcolor}\rmfamily\fontsize{8.000000}{9.600000}\selectfont \(\displaystyle {150000}\)}%
\end{pgfscope}%
\begin{pgfscope}%
\pgfpathrectangle{\pgfqpoint{0.471687in}{0.416447in}}{\pgfqpoint{3.537842in}{1.924647in}}%
\pgfusepath{clip}%
\pgfsetrectcap%
\pgfsetroundjoin%
\pgfsetlinewidth{0.803000pt}%
\definecolor{currentstroke}{rgb}{0.450000,0.450000,0.450000}%
\pgfsetstrokecolor{currentstroke}%
\pgfsetdash{}{0pt}%
\pgfpathmoveto{\pgfqpoint{3.446697in}{0.416447in}}%
\pgfpathlineto{\pgfqpoint{3.446697in}{2.341095in}}%
\pgfusepath{stroke}%
\end{pgfscope}%
\begin{pgfscope}%
\pgfsetbuttcap%
\pgfsetroundjoin%
\definecolor{currentfill}{rgb}{0.000000,0.000000,0.000000}%
\pgfsetfillcolor{currentfill}%
\pgfsetlinewidth{0.803000pt}%
\definecolor{currentstroke}{rgb}{0.000000,0.000000,0.000000}%
\pgfsetstrokecolor{currentstroke}%
\pgfsetdash{}{0pt}%
\pgfsys@defobject{currentmarker}{\pgfqpoint{0.000000in}{-0.048611in}}{\pgfqpoint{0.000000in}{0.000000in}}{%
\pgfpathmoveto{\pgfqpoint{0.000000in}{0.000000in}}%
\pgfpathlineto{\pgfqpoint{0.000000in}{-0.048611in}}%
\pgfusepath{stroke,fill}%
}%
\begin{pgfscope}%
\pgfsys@transformshift{3.446697in}{0.416447in}%
\pgfsys@useobject{currentmarker}{}%
\end{pgfscope}%
\end{pgfscope}%
\begin{pgfscope}%
\definecolor{textcolor}{rgb}{0.000000,0.000000,0.000000}%
\pgfsetstrokecolor{textcolor}%
\pgfsetfillcolor{textcolor}%
\pgftext[x=3.446697in,y=0.319225in,,top]{\color{textcolor}\rmfamily\fontsize{8.000000}{9.600000}\selectfont \(\displaystyle {175000}\)}%
\end{pgfscope}%
\begin{pgfscope}%
\pgfpathrectangle{\pgfqpoint{0.471687in}{0.416447in}}{\pgfqpoint{3.537842in}{1.924647in}}%
\pgfusepath{clip}%
\pgfsetrectcap%
\pgfsetroundjoin%
\pgfsetlinewidth{0.803000pt}%
\definecolor{currentstroke}{rgb}{0.450000,0.450000,0.450000}%
\pgfsetstrokecolor{currentstroke}%
\pgfsetdash{}{0pt}%
\pgfpathmoveto{\pgfqpoint{3.848725in}{0.416447in}}%
\pgfpathlineto{\pgfqpoint{3.848725in}{2.341095in}}%
\pgfusepath{stroke}%
\end{pgfscope}%
\begin{pgfscope}%
\pgfsetbuttcap%
\pgfsetroundjoin%
\definecolor{currentfill}{rgb}{0.000000,0.000000,0.000000}%
\pgfsetfillcolor{currentfill}%
\pgfsetlinewidth{0.803000pt}%
\definecolor{currentstroke}{rgb}{0.000000,0.000000,0.000000}%
\pgfsetstrokecolor{currentstroke}%
\pgfsetdash{}{0pt}%
\pgfsys@defobject{currentmarker}{\pgfqpoint{0.000000in}{-0.048611in}}{\pgfqpoint{0.000000in}{0.000000in}}{%
\pgfpathmoveto{\pgfqpoint{0.000000in}{0.000000in}}%
\pgfpathlineto{\pgfqpoint{0.000000in}{-0.048611in}}%
\pgfusepath{stroke,fill}%
}%
\begin{pgfscope}%
\pgfsys@transformshift{3.848725in}{0.416447in}%
\pgfsys@useobject{currentmarker}{}%
\end{pgfscope}%
\end{pgfscope}%
\begin{pgfscope}%
\definecolor{textcolor}{rgb}{0.000000,0.000000,0.000000}%
\pgfsetstrokecolor{textcolor}%
\pgfsetfillcolor{textcolor}%
\pgftext[x=3.848725in,y=0.319225in,,top]{\color{textcolor}\rmfamily\fontsize{8.000000}{9.600000}\selectfont \(\displaystyle {200000}\)}%
\end{pgfscope}%
\begin{pgfscope}%
\definecolor{textcolor}{rgb}{0.000000,0.000000,0.000000}%
\pgfsetstrokecolor{textcolor}%
\pgfsetfillcolor{textcolor}%
\pgftext[x=2.240609in,y=0.165003in,,top]{\color{textcolor}\rmfamily\fontsize{10.000000}{12.000000}\selectfont Time in \(\displaystyle \unit{\second}\)}%
\end{pgfscope}%
\begin{pgfscope}%
\pgfpathrectangle{\pgfqpoint{0.471687in}{0.416447in}}{\pgfqpoint{3.537842in}{1.924647in}}%
\pgfusepath{clip}%
\pgfsetrectcap%
\pgfsetroundjoin%
\pgfsetlinewidth{0.803000pt}%
\definecolor{currentstroke}{rgb}{0.450000,0.450000,0.450000}%
\pgfsetstrokecolor{currentstroke}%
\pgfsetdash{}{0pt}%
\pgfpathmoveto{\pgfqpoint{0.471687in}{0.523372in}}%
\pgfpathlineto{\pgfqpoint{4.009530in}{0.523372in}}%
\pgfusepath{stroke}%
\end{pgfscope}%
\begin{pgfscope}%
\pgfsetbuttcap%
\pgfsetroundjoin%
\definecolor{currentfill}{rgb}{0.000000,0.000000,0.000000}%
\pgfsetfillcolor{currentfill}%
\pgfsetlinewidth{0.803000pt}%
\definecolor{currentstroke}{rgb}{0.000000,0.000000,0.000000}%
\pgfsetstrokecolor{currentstroke}%
\pgfsetdash{}{0pt}%
\pgfsys@defobject{currentmarker}{\pgfqpoint{-0.048611in}{0.000000in}}{\pgfqpoint{-0.000000in}{0.000000in}}{%
\pgfpathmoveto{\pgfqpoint{-0.000000in}{0.000000in}}%
\pgfpathlineto{\pgfqpoint{-0.048611in}{0.000000in}}%
\pgfusepath{stroke,fill}%
}%
\begin{pgfscope}%
\pgfsys@transformshift{0.471687in}{0.523372in}%
\pgfsys@useobject{currentmarker}{}%
\end{pgfscope}%
\end{pgfscope}%
\begin{pgfscope}%
\definecolor{textcolor}{rgb}{0.000000,0.000000,0.000000}%
\pgfsetstrokecolor{textcolor}%
\pgfsetfillcolor{textcolor}%
\pgftext[x=0.223614in, y=0.484817in, left, base]{\color{textcolor}\rmfamily\fontsize{8.000000}{9.600000}\selectfont \(\displaystyle {\ensuremath{-}4}\)}%
\end{pgfscope}%
\begin{pgfscope}%
\pgfpathrectangle{\pgfqpoint{0.471687in}{0.416447in}}{\pgfqpoint{3.537842in}{1.924647in}}%
\pgfusepath{clip}%
\pgfsetrectcap%
\pgfsetroundjoin%
\pgfsetlinewidth{0.803000pt}%
\definecolor{currentstroke}{rgb}{0.450000,0.450000,0.450000}%
\pgfsetstrokecolor{currentstroke}%
\pgfsetdash{}{0pt}%
\pgfpathmoveto{\pgfqpoint{0.471687in}{0.951072in}}%
\pgfpathlineto{\pgfqpoint{4.009530in}{0.951072in}}%
\pgfusepath{stroke}%
\end{pgfscope}%
\begin{pgfscope}%
\pgfsetbuttcap%
\pgfsetroundjoin%
\definecolor{currentfill}{rgb}{0.000000,0.000000,0.000000}%
\pgfsetfillcolor{currentfill}%
\pgfsetlinewidth{0.803000pt}%
\definecolor{currentstroke}{rgb}{0.000000,0.000000,0.000000}%
\pgfsetstrokecolor{currentstroke}%
\pgfsetdash{}{0pt}%
\pgfsys@defobject{currentmarker}{\pgfqpoint{-0.048611in}{0.000000in}}{\pgfqpoint{-0.000000in}{0.000000in}}{%
\pgfpathmoveto{\pgfqpoint{-0.000000in}{0.000000in}}%
\pgfpathlineto{\pgfqpoint{-0.048611in}{0.000000in}}%
\pgfusepath{stroke,fill}%
}%
\begin{pgfscope}%
\pgfsys@transformshift{0.471687in}{0.951072in}%
\pgfsys@useobject{currentmarker}{}%
\end{pgfscope}%
\end{pgfscope}%
\begin{pgfscope}%
\definecolor{textcolor}{rgb}{0.000000,0.000000,0.000000}%
\pgfsetstrokecolor{textcolor}%
\pgfsetfillcolor{textcolor}%
\pgftext[x=0.223614in, y=0.912516in, left, base]{\color{textcolor}\rmfamily\fontsize{8.000000}{9.600000}\selectfont \(\displaystyle {\ensuremath{-}2}\)}%
\end{pgfscope}%
\begin{pgfscope}%
\pgfpathrectangle{\pgfqpoint{0.471687in}{0.416447in}}{\pgfqpoint{3.537842in}{1.924647in}}%
\pgfusepath{clip}%
\pgfsetrectcap%
\pgfsetroundjoin%
\pgfsetlinewidth{0.803000pt}%
\definecolor{currentstroke}{rgb}{0.450000,0.450000,0.450000}%
\pgfsetstrokecolor{currentstroke}%
\pgfsetdash{}{0pt}%
\pgfpathmoveto{\pgfqpoint{0.471687in}{1.378771in}}%
\pgfpathlineto{\pgfqpoint{4.009530in}{1.378771in}}%
\pgfusepath{stroke}%
\end{pgfscope}%
\begin{pgfscope}%
\pgfsetbuttcap%
\pgfsetroundjoin%
\definecolor{currentfill}{rgb}{0.000000,0.000000,0.000000}%
\pgfsetfillcolor{currentfill}%
\pgfsetlinewidth{0.803000pt}%
\definecolor{currentstroke}{rgb}{0.000000,0.000000,0.000000}%
\pgfsetstrokecolor{currentstroke}%
\pgfsetdash{}{0pt}%
\pgfsys@defobject{currentmarker}{\pgfqpoint{-0.048611in}{0.000000in}}{\pgfqpoint{-0.000000in}{0.000000in}}{%
\pgfpathmoveto{\pgfqpoint{-0.000000in}{0.000000in}}%
\pgfpathlineto{\pgfqpoint{-0.048611in}{0.000000in}}%
\pgfusepath{stroke,fill}%
}%
\begin{pgfscope}%
\pgfsys@transformshift{0.471687in}{1.378771in}%
\pgfsys@useobject{currentmarker}{}%
\end{pgfscope}%
\end{pgfscope}%
\begin{pgfscope}%
\definecolor{textcolor}{rgb}{0.000000,0.000000,0.000000}%
\pgfsetstrokecolor{textcolor}%
\pgfsetfillcolor{textcolor}%
\pgftext[x=0.315437in, y=1.340216in, left, base]{\color{textcolor}\rmfamily\fontsize{8.000000}{9.600000}\selectfont \(\displaystyle {0}\)}%
\end{pgfscope}%
\begin{pgfscope}%
\pgfpathrectangle{\pgfqpoint{0.471687in}{0.416447in}}{\pgfqpoint{3.537842in}{1.924647in}}%
\pgfusepath{clip}%
\pgfsetrectcap%
\pgfsetroundjoin%
\pgfsetlinewidth{0.803000pt}%
\definecolor{currentstroke}{rgb}{0.450000,0.450000,0.450000}%
\pgfsetstrokecolor{currentstroke}%
\pgfsetdash{}{0pt}%
\pgfpathmoveto{\pgfqpoint{0.471687in}{1.806471in}}%
\pgfpathlineto{\pgfqpoint{4.009530in}{1.806471in}}%
\pgfusepath{stroke}%
\end{pgfscope}%
\begin{pgfscope}%
\pgfsetbuttcap%
\pgfsetroundjoin%
\definecolor{currentfill}{rgb}{0.000000,0.000000,0.000000}%
\pgfsetfillcolor{currentfill}%
\pgfsetlinewidth{0.803000pt}%
\definecolor{currentstroke}{rgb}{0.000000,0.000000,0.000000}%
\pgfsetstrokecolor{currentstroke}%
\pgfsetdash{}{0pt}%
\pgfsys@defobject{currentmarker}{\pgfqpoint{-0.048611in}{0.000000in}}{\pgfqpoint{-0.000000in}{0.000000in}}{%
\pgfpathmoveto{\pgfqpoint{-0.000000in}{0.000000in}}%
\pgfpathlineto{\pgfqpoint{-0.048611in}{0.000000in}}%
\pgfusepath{stroke,fill}%
}%
\begin{pgfscope}%
\pgfsys@transformshift{0.471687in}{1.806471in}%
\pgfsys@useobject{currentmarker}{}%
\end{pgfscope}%
\end{pgfscope}%
\begin{pgfscope}%
\definecolor{textcolor}{rgb}{0.000000,0.000000,0.000000}%
\pgfsetstrokecolor{textcolor}%
\pgfsetfillcolor{textcolor}%
\pgftext[x=0.315437in, y=1.767915in, left, base]{\color{textcolor}\rmfamily\fontsize{8.000000}{9.600000}\selectfont \(\displaystyle {2}\)}%
\end{pgfscope}%
\begin{pgfscope}%
\pgfpathrectangle{\pgfqpoint{0.471687in}{0.416447in}}{\pgfqpoint{3.537842in}{1.924647in}}%
\pgfusepath{clip}%
\pgfsetrectcap%
\pgfsetroundjoin%
\pgfsetlinewidth{0.803000pt}%
\definecolor{currentstroke}{rgb}{0.450000,0.450000,0.450000}%
\pgfsetstrokecolor{currentstroke}%
\pgfsetdash{}{0pt}%
\pgfpathmoveto{\pgfqpoint{0.471687in}{2.234170in}}%
\pgfpathlineto{\pgfqpoint{4.009530in}{2.234170in}}%
\pgfusepath{stroke}%
\end{pgfscope}%
\begin{pgfscope}%
\pgfsetbuttcap%
\pgfsetroundjoin%
\definecolor{currentfill}{rgb}{0.000000,0.000000,0.000000}%
\pgfsetfillcolor{currentfill}%
\pgfsetlinewidth{0.803000pt}%
\definecolor{currentstroke}{rgb}{0.000000,0.000000,0.000000}%
\pgfsetstrokecolor{currentstroke}%
\pgfsetdash{}{0pt}%
\pgfsys@defobject{currentmarker}{\pgfqpoint{-0.048611in}{0.000000in}}{\pgfqpoint{-0.000000in}{0.000000in}}{%
\pgfpathmoveto{\pgfqpoint{-0.000000in}{0.000000in}}%
\pgfpathlineto{\pgfqpoint{-0.048611in}{0.000000in}}%
\pgfusepath{stroke,fill}%
}%
\begin{pgfscope}%
\pgfsys@transformshift{0.471687in}{2.234170in}%
\pgfsys@useobject{currentmarker}{}%
\end{pgfscope}%
\end{pgfscope}%
\begin{pgfscope}%
\definecolor{textcolor}{rgb}{0.000000,0.000000,0.000000}%
\pgfsetstrokecolor{textcolor}%
\pgfsetfillcolor{textcolor}%
\pgftext[x=0.315437in, y=2.195614in, left, base]{\color{textcolor}\rmfamily\fontsize{8.000000}{9.600000}\selectfont \(\displaystyle {4}\)}%
\end{pgfscope}%
\begin{pgfscope}%
\definecolor{textcolor}{rgb}{0.000000,0.000000,0.000000}%
\pgfsetstrokecolor{textcolor}%
\pgfsetfillcolor{textcolor}%
\pgftext[x=0.168059in,y=1.378771in,,bottom,rotate=90.000000]{\color{textcolor}\rmfamily\fontsize{10.000000}{12.000000}\selectfont Amplitude in \(\displaystyle \unit{\V}\)}%
\end{pgfscope}%
\begin{pgfscope}%
\definecolor{textcolor}{rgb}{0.000000,0.000000,0.000000}%
\pgfsetstrokecolor{textcolor}%
\pgfsetfillcolor{textcolor}%
\pgftext[x=0.471687in,y=2.382761in,left,base]{\color{textcolor}\rmfamily\fontsize{8.000000}{9.600000}\selectfont \(\displaystyle \times{10^{\ensuremath{-}6}}{+10^{1}}\)}%
\end{pgfscope}%
\begin{pgfscope}%
\pgfpathrectangle{\pgfqpoint{0.471687in}{0.416447in}}{\pgfqpoint{3.537842in}{1.924647in}}%
\pgfusepath{clip}%
\pgfsetrectcap%
\pgfsetroundjoin%
\pgfsetlinewidth{1.505625pt}%
\definecolor{currentstroke}{rgb}{0.925490,0.882353,0.200000}%
\pgfsetstrokecolor{currentstroke}%
\pgfsetdash{}{0pt}%
\pgfpathmoveto{\pgfqpoint{0.632499in}{1.287433in}}%
\pgfpathlineto{\pgfqpoint{0.633296in}{1.710792in}}%
\pgfpathlineto{\pgfqpoint{0.635908in}{1.021405in}}%
\pgfpathlineto{\pgfqpoint{0.641806in}{1.842120in}}%
\pgfpathlineto{\pgfqpoint{0.642314in}{1.004790in}}%
\pgfpathlineto{\pgfqpoint{0.646534in}{1.726777in}}%
\pgfpathlineto{\pgfqpoint{0.648650in}{1.115606in}}%
\pgfpathlineto{\pgfqpoint{0.654658in}{1.806626in}}%
\pgfpathlineto{\pgfqpoint{0.655713in}{0.975518in}}%
\pgfpathlineto{\pgfqpoint{0.658338in}{1.656328in}}%
\pgfpathlineto{\pgfqpoint{0.662416in}{1.015333in}}%
\pgfpathlineto{\pgfqpoint{0.664982in}{1.686220in}}%
\pgfpathlineto{\pgfqpoint{0.668597in}{1.079152in}}%
\pgfpathlineto{\pgfqpoint{0.671904in}{1.712408in}}%
\pgfpathlineto{\pgfqpoint{0.674933in}{1.096655in}}%
\pgfpathlineto{\pgfqpoint{0.677783in}{1.677530in}}%
\pgfpathlineto{\pgfqpoint{0.681752in}{1.040439in}}%
\pgfpathlineto{\pgfqpoint{0.685367in}{1.773041in}}%
\pgfpathlineto{\pgfqpoint{0.687335in}{0.932432in}}%
\pgfpathlineto{\pgfqpoint{0.691259in}{1.790099in}}%
\pgfpathlineto{\pgfqpoint{0.693986in}{1.092348in}}%
\pgfpathlineto{\pgfqpoint{0.697080in}{1.657695in}}%
\pgfpathlineto{\pgfqpoint{0.702027in}{0.919470in}}%
\pgfpathlineto{\pgfqpoint{0.704967in}{1.802841in}}%
\pgfpathlineto{\pgfqpoint{0.706864in}{1.101068in}}%
\pgfpathlineto{\pgfqpoint{0.710254in}{1.723504in}}%
\pgfpathlineto{\pgfqpoint{0.715812in}{0.978217in}}%
\pgfpathlineto{\pgfqpoint{0.716912in}{1.646696in}}%
\pgfpathlineto{\pgfqpoint{0.720417in}{0.969045in}}%
\pgfpathlineto{\pgfqpoint{0.722778in}{1.684075in}}%
\pgfpathlineto{\pgfqpoint{0.726432in}{1.039160in}}%
\pgfpathlineto{\pgfqpoint{0.730966in}{1.812285in}}%
\pgfpathlineto{\pgfqpoint{0.732440in}{1.002605in}}%
\pgfpathlineto{\pgfqpoint{0.735759in}{1.659799in}}%
\pgfpathlineto{\pgfqpoint{0.742069in}{0.891826in}}%
\pgfpathlineto{\pgfqpoint{0.742230in}{1.712848in}}%
\pgfpathlineto{\pgfqpoint{0.745414in}{1.087560in}}%
\pgfpathlineto{\pgfqpoint{0.750052in}{1.725644in}}%
\pgfpathlineto{\pgfqpoint{0.752020in}{1.087099in}}%
\pgfpathlineto{\pgfqpoint{0.755069in}{1.677292in}}%
\pgfpathlineto{\pgfqpoint{0.758755in}{1.013170in}}%
\pgfpathlineto{\pgfqpoint{0.761636in}{1.701235in}}%
\pgfpathlineto{\pgfqpoint{0.764975in}{1.026044in}}%
\pgfpathlineto{\pgfqpoint{0.769079in}{1.824066in}}%
\pgfpathlineto{\pgfqpoint{0.771234in}{1.091928in}}%
\pgfpathlineto{\pgfqpoint{0.776058in}{1.709336in}}%
\pgfpathlineto{\pgfqpoint{0.777531in}{1.049007in}}%
\pgfpathlineto{\pgfqpoint{0.781307in}{1.838824in}}%
\pgfpathlineto{\pgfqpoint{0.784568in}{1.037171in}}%
\pgfpathlineto{\pgfqpoint{0.787810in}{1.729629in}}%
\pgfpathlineto{\pgfqpoint{0.790486in}{1.028312in}}%
\pgfpathlineto{\pgfqpoint{0.794661in}{1.759188in}}%
\pgfpathlineto{\pgfqpoint{0.797465in}{1.124859in}}%
\pgfpathlineto{\pgfqpoint{0.800669in}{1.707749in}}%
\pgfpathlineto{\pgfqpoint{0.806020in}{0.951056in}}%
\pgfpathlineto{\pgfqpoint{0.806683in}{1.660959in}}%
\pgfpathlineto{\pgfqpoint{0.810780in}{1.029262in}}%
\pgfpathlineto{\pgfqpoint{0.813077in}{1.835055in}}%
\pgfpathlineto{\pgfqpoint{0.818133in}{1.009550in}}%
\pgfpathlineto{\pgfqpoint{0.819947in}{1.698485in}}%
\pgfpathlineto{\pgfqpoint{0.824237in}{1.090256in}}%
\pgfpathlineto{\pgfqpoint{0.825974in}{1.743554in}}%
\pgfpathlineto{\pgfqpoint{0.829267in}{1.007824in}}%
\pgfpathlineto{\pgfqpoint{0.832998in}{1.823798in}}%
\pgfpathlineto{\pgfqpoint{0.836620in}{1.002207in}}%
\pgfpathlineto{\pgfqpoint{0.839572in}{1.759965in}}%
\pgfpathlineto{\pgfqpoint{0.843612in}{1.026064in}}%
\pgfpathlineto{\pgfqpoint{0.845721in}{1.677591in}}%
\pgfpathlineto{\pgfqpoint{0.848751in}{1.036036in}}%
\pgfpathlineto{\pgfqpoint{0.851813in}{1.752759in}}%
\pgfpathlineto{\pgfqpoint{0.857486in}{1.007899in}}%
\pgfpathlineto{\pgfqpoint{0.859590in}{1.758885in}}%
\pgfpathlineto{\pgfqpoint{0.863288in}{1.018189in}}%
\pgfpathlineto{\pgfqpoint{0.864678in}{1.861767in}}%
\pgfpathlineto{\pgfqpoint{0.867868in}{1.004540in}}%
\pgfpathlineto{\pgfqpoint{0.872172in}{1.727079in}}%
\pgfpathlineto{\pgfqpoint{0.877253in}{0.905479in}}%
\pgfpathlineto{\pgfqpoint{0.877704in}{1.682164in}}%
\pgfpathlineto{\pgfqpoint{0.880997in}{1.083885in}}%
\pgfpathlineto{\pgfqpoint{0.885345in}{1.725641in}}%
\pgfpathlineto{\pgfqpoint{0.887372in}{1.121101in}}%
\pgfpathlineto{\pgfqpoint{0.890324in}{1.698718in}}%
\pgfpathlineto{\pgfqpoint{0.893708in}{1.105986in}}%
\pgfpathlineto{\pgfqpoint{0.897599in}{1.692155in}}%
\pgfpathlineto{\pgfqpoint{0.900185in}{1.120828in}}%
\pgfpathlineto{\pgfqpoint{0.904109in}{1.698380in}}%
\pgfpathlineto{\pgfqpoint{0.907910in}{0.987772in}}%
\pgfpathlineto{\pgfqpoint{0.909737in}{1.698417in}}%
\pgfpathlineto{\pgfqpoint{0.913976in}{1.037763in}}%
\pgfpathlineto{\pgfqpoint{0.916099in}{1.632923in}}%
\pgfpathlineto{\pgfqpoint{0.920081in}{1.057637in}}%
\pgfpathlineto{\pgfqpoint{0.923799in}{1.683956in}}%
\pgfpathlineto{\pgfqpoint{0.925896in}{1.003338in}}%
\pgfpathlineto{\pgfqpoint{0.930334in}{1.709928in}}%
\pgfpathlineto{\pgfqpoint{0.932373in}{1.067450in}}%
\pgfpathlineto{\pgfqpoint{0.936413in}{1.721890in}}%
\pgfpathlineto{\pgfqpoint{0.939860in}{1.040447in}}%
\pgfpathlineto{\pgfqpoint{0.943424in}{1.720759in}}%
\pgfpathlineto{\pgfqpoint{0.945116in}{1.059008in}}%
\pgfpathlineto{\pgfqpoint{0.949194in}{1.728121in}}%
\pgfpathlineto{\pgfqpoint{0.952539in}{1.028415in}}%
\pgfpathlineto{\pgfqpoint{0.955035in}{1.693352in}}%
\pgfpathlineto{\pgfqpoint{0.958836in}{1.094524in}}%
\pgfpathlineto{\pgfqpoint{0.961602in}{1.689254in}}%
\pgfpathlineto{\pgfqpoint{0.965455in}{1.128690in}}%
\pgfpathlineto{\pgfqpoint{0.968485in}{1.694511in}}%
\pgfpathlineto{\pgfqpoint{0.971161in}{0.972445in}}%
\pgfpathlineto{\pgfqpoint{0.975567in}{1.694261in}}%
\pgfpathlineto{\pgfqpoint{0.977625in}{0.980983in}}%
\pgfpathlineto{\pgfqpoint{0.980893in}{1.713750in}}%
\pgfpathlineto{\pgfqpoint{0.984347in}{1.026048in}}%
\pgfpathlineto{\pgfqpoint{0.987731in}{1.756009in}}%
\pgfpathlineto{\pgfqpoint{0.990394in}{1.032513in}}%
\pgfpathlineto{\pgfqpoint{0.993957in}{1.747289in}}%
\pgfpathlineto{\pgfqpoint{0.997431in}{0.906413in}}%
\pgfpathlineto{\pgfqpoint{1.001438in}{1.766895in}}%
\pgfpathlineto{\pgfqpoint{1.003169in}{1.125614in}}%
\pgfpathlineto{\pgfqpoint{1.008488in}{1.709350in}}%
\pgfpathlineto{\pgfqpoint{1.009607in}{1.081095in}}%
\pgfpathlineto{\pgfqpoint{1.013017in}{1.730557in}}%
\pgfpathlineto{\pgfqpoint{1.016477in}{1.012690in}}%
\pgfpathlineto{\pgfqpoint{1.019353in}{1.632734in}}%
\pgfpathlineto{\pgfqpoint{1.023971in}{0.895441in}}%
\pgfpathlineto{\pgfqpoint{1.025740in}{1.582666in}}%
\pgfpathlineto{\pgfqpoint{1.030088in}{0.979639in}}%
\pgfpathlineto{\pgfqpoint{1.033523in}{1.737372in}}%
\pgfpathlineto{\pgfqpoint{1.036090in}{1.087960in}}%
\pgfpathlineto{\pgfqpoint{1.039866in}{1.739275in}}%
\pgfpathlineto{\pgfqpoint{1.041828in}{1.079902in}}%
\pgfpathlineto{\pgfqpoint{1.046163in}{1.697898in}}%
\pgfpathlineto{\pgfqpoint{1.048826in}{1.079810in}}%
\pgfpathlineto{\pgfqpoint{1.052428in}{1.750116in}}%
\pgfpathlineto{\pgfqpoint{1.054873in}{1.083934in}}%
\pgfpathlineto{\pgfqpoint{1.059022in}{1.751654in}}%
\pgfpathlineto{\pgfqpoint{1.061151in}{1.098829in}}%
\pgfpathlineto{\pgfqpoint{1.064631in}{1.694007in}}%
\pgfpathlineto{\pgfqpoint{1.068992in}{0.921340in}}%
\pgfpathlineto{\pgfqpoint{1.072594in}{1.764964in}}%
\pgfpathlineto{\pgfqpoint{1.074029in}{0.971943in}}%
\pgfpathlineto{\pgfqpoint{1.077238in}{1.865404in}}%
\pgfpathlineto{\pgfqpoint{1.080937in}{0.947445in}}%
\pgfpathlineto{\pgfqpoint{1.083774in}{1.698554in}}%
\pgfpathlineto{\pgfqpoint{1.088064in}{0.965725in}}%
\pgfpathlineto{\pgfqpoint{1.091293in}{1.714397in}}%
\pgfpathlineto{\pgfqpoint{1.095268in}{0.941174in}}%
\pgfpathlineto{\pgfqpoint{1.097353in}{1.689092in}}%
\pgfpathlineto{\pgfqpoint{1.100209in}{1.076193in}}%
\pgfpathlineto{\pgfqpoint{1.105226in}{1.873081in}}%
\pgfpathlineto{\pgfqpoint{1.106281in}{1.104453in}}%
\pgfpathlineto{\pgfqpoint{1.109555in}{1.704374in}}%
\pgfpathlineto{\pgfqpoint{1.114630in}{1.037519in}}%
\pgfpathlineto{\pgfqpoint{1.117113in}{1.687282in}}%
\pgfpathlineto{\pgfqpoint{1.119474in}{1.044910in}}%
\pgfpathlineto{\pgfqpoint{1.124067in}{1.796341in}}%
\pgfpathlineto{\pgfqpoint{1.127746in}{0.979387in}}%
\pgfpathlineto{\pgfqpoint{1.129836in}{1.684247in}}%
\pgfpathlineto{\pgfqpoint{1.132924in}{1.086616in}}%
\pgfpathlineto{\pgfqpoint{1.137787in}{1.728482in}}%
\pgfpathlineto{\pgfqpoint{1.139369in}{1.069420in}}%
\pgfpathlineto{\pgfqpoint{1.141788in}{1.691017in}}%
\pgfpathlineto{\pgfqpoint{1.145133in}{1.070916in}}%
\pgfpathlineto{\pgfqpoint{1.149404in}{1.694255in}}%
\pgfpathlineto{\pgfqpoint{1.152537in}{1.019894in}}%
\pgfpathlineto{\pgfqpoint{1.154576in}{1.601672in}}%
\pgfpathlineto{\pgfqpoint{1.157940in}{1.056826in}}%
\pgfpathlineto{\pgfqpoint{1.162076in}{1.782408in}}%
\pgfpathlineto{\pgfqpoint{1.164314in}{1.057121in}}%
\pgfpathlineto{\pgfqpoint{1.167589in}{1.697024in}}%
\pgfpathlineto{\pgfqpoint{1.171345in}{1.099707in}}%
\pgfpathlineto{\pgfqpoint{1.174066in}{1.748616in}}%
\pgfpathlineto{\pgfqpoint{1.178504in}{1.016102in}}%
\pgfpathlineto{\pgfqpoint{1.180543in}{1.719853in}}%
\pgfpathlineto{\pgfqpoint{1.184178in}{1.053330in}}%
\pgfpathlineto{\pgfqpoint{1.186834in}{1.777559in}}%
\pgfpathlineto{\pgfqpoint{1.190752in}{1.048605in}}%
\pgfpathlineto{\pgfqpoint{1.194348in}{1.643124in}}%
\pgfpathlineto{\pgfqpoint{1.197345in}{1.081377in}}%
\pgfpathlineto{\pgfqpoint{1.200298in}{1.679423in}}%
\pgfpathlineto{\pgfqpoint{1.203012in}{1.147414in}}%
\pgfpathlineto{\pgfqpoint{1.206357in}{1.631527in}}%
\pgfpathlineto{\pgfqpoint{1.212397in}{0.986601in}}%
\pgfpathlineto{\pgfqpoint{1.212609in}{1.776572in}}%
\pgfpathlineto{\pgfqpoint{1.216063in}{1.082487in}}%
\pgfpathlineto{\pgfqpoint{1.220380in}{1.698489in}}%
\pgfpathlineto{\pgfqpoint{1.224149in}{0.986721in}}%
\pgfpathlineto{\pgfqpoint{1.225886in}{1.698095in}}%
\pgfpathlineto{\pgfqpoint{1.228871in}{1.052301in}}%
\pgfpathlineto{\pgfqpoint{1.232164in}{1.744315in}}%
\pgfpathlineto{\pgfqpoint{1.236596in}{1.007865in}}%
\pgfpathlineto{\pgfqpoint{1.238423in}{1.719021in}}%
\pgfpathlineto{\pgfqpoint{1.242668in}{0.968124in}}%
\pgfpathlineto{\pgfqpoint{1.244984in}{1.613648in}}%
\pgfpathlineto{\pgfqpoint{1.249557in}{0.957862in}}%
\pgfpathlineto{\pgfqpoint{1.251468in}{1.688941in}}%
\pgfpathlineto{\pgfqpoint{1.254652in}{0.987658in}}%
\pgfpathlineto{\pgfqpoint{1.257759in}{1.637700in}}%
\pgfpathlineto{\pgfqpoint{1.262911in}{1.029704in}}%
\pgfpathlineto{\pgfqpoint{1.264185in}{1.612969in}}%
\pgfpathlineto{\pgfqpoint{1.268758in}{1.051192in}}%
\pgfpathlineto{\pgfqpoint{1.270746in}{1.651709in}}%
\pgfpathlineto{\pgfqpoint{1.275023in}{1.019818in}}%
\pgfpathlineto{\pgfqpoint{1.278497in}{1.730637in}}%
\pgfpathlineto{\pgfqpoint{1.280510in}{1.074182in}}%
\pgfpathlineto{\pgfqpoint{1.283842in}{1.726622in}}%
\pgfpathlineto{\pgfqpoint{1.287644in}{1.080721in}}%
\pgfpathlineto{\pgfqpoint{1.291600in}{1.734733in}}%
\pgfpathlineto{\pgfqpoint{1.293394in}{1.073014in}}%
\pgfpathlineto{\pgfqpoint{1.298386in}{1.772280in}}%
\pgfpathlineto{\pgfqpoint{1.299949in}{1.100471in}}%
\pgfpathlineto{\pgfqpoint{1.304497in}{1.717193in}}%
\pgfpathlineto{\pgfqpoint{1.306967in}{0.917906in}}%
\pgfpathlineto{\pgfqpoint{1.310042in}{1.722505in}}%
\pgfpathlineto{\pgfqpoint{1.313547in}{0.934635in}}%
\pgfpathlineto{\pgfqpoint{1.316384in}{1.714465in}}%
\pgfpathlineto{\pgfqpoint{1.320765in}{1.036808in}}%
\pgfpathlineto{\pgfqpoint{1.322418in}{1.693686in}}%
\pgfpathlineto{\pgfqpoint{1.327499in}{0.996929in}}%
\pgfpathlineto{\pgfqpoint{1.329712in}{1.680006in}}%
\pgfpathlineto{\pgfqpoint{1.332324in}{1.092049in}}%
\pgfpathlineto{\pgfqpoint{1.337476in}{1.811466in}}%
\pgfpathlineto{\pgfqpoint{1.338332in}{1.029015in}}%
\pgfpathlineto{\pgfqpoint{1.342050in}{1.615097in}}%
\pgfpathlineto{\pgfqpoint{1.345324in}{1.073357in}}%
\pgfpathlineto{\pgfqpoint{1.348945in}{1.750824in}}%
\pgfpathlineto{\pgfqpoint{1.351608in}{1.150853in}}%
\pgfpathlineto{\pgfqpoint{1.354914in}{1.723853in}}%
\pgfpathlineto{\pgfqpoint{1.357764in}{1.037251in}}%
\pgfpathlineto{\pgfqpoint{1.360935in}{1.676405in}}%
\pgfpathlineto{\pgfqpoint{1.365046in}{1.089913in}}%
\pgfpathlineto{\pgfqpoint{1.367535in}{1.738079in}}%
\pgfpathlineto{\pgfqpoint{1.370764in}{1.068512in}}%
\pgfpathlineto{\pgfqpoint{1.374360in}{1.650361in}}%
\pgfpathlineto{\pgfqpoint{1.377344in}{1.000850in}}%
\pgfpathlineto{\pgfqpoint{1.381030in}{1.820204in}}%
\pgfpathlineto{\pgfqpoint{1.383590in}{1.046812in}}%
\pgfpathlineto{\pgfqpoint{1.386864in}{1.737731in}}%
\pgfpathlineto{\pgfqpoint{1.391760in}{1.018526in}}%
\pgfpathlineto{\pgfqpoint{1.394583in}{1.758213in}}%
\pgfpathlineto{\pgfqpoint{1.397304in}{1.012696in}}%
\pgfpathlineto{\pgfqpoint{1.399575in}{1.651315in}}%
\pgfpathlineto{\pgfqpoint{1.404136in}{0.985194in}}%
\pgfpathlineto{\pgfqpoint{1.406085in}{1.736685in}}%
\pgfpathlineto{\pgfqpoint{1.410549in}{1.044281in}}%
\pgfpathlineto{\pgfqpoint{1.413366in}{1.719072in}}%
\pgfpathlineto{\pgfqpoint{1.415817in}{1.051409in}}%
\pgfpathlineto{\pgfqpoint{1.420622in}{1.817249in}}%
\pgfpathlineto{\pgfqpoint{1.422860in}{1.060688in}}%
\pgfpathlineto{\pgfqpoint{1.425543in}{1.658240in}}%
\pgfpathlineto{\pgfqpoint{1.430232in}{1.043025in}}%
\pgfpathlineto{\pgfqpoint{1.431757in}{1.666235in}}%
\pgfpathlineto{\pgfqpoint{1.434986in}{1.074570in}}%
\pgfpathlineto{\pgfqpoint{1.438626in}{1.694186in}}%
\pgfpathlineto{\pgfqpoint{1.442627in}{1.029084in}}%
\pgfpathlineto{\pgfqpoint{1.445123in}{1.771999in}}%
\pgfpathlineto{\pgfqpoint{1.448101in}{1.012768in}}%
\pgfpathlineto{\pgfqpoint{1.451620in}{1.681373in}}%
\pgfpathlineto{\pgfqpoint{1.455499in}{1.086662in}}%
\pgfpathlineto{\pgfqpoint{1.457782in}{1.666179in}}%
\pgfpathlineto{\pgfqpoint{1.460973in}{0.984866in}}%
\pgfpathlineto{\pgfqpoint{1.466453in}{1.755599in}}%
\pgfpathlineto{\pgfqpoint{1.467688in}{1.043588in}}%
\pgfpathlineto{\pgfqpoint{1.470518in}{1.700088in}}%
\pgfpathlineto{\pgfqpoint{1.473992in}{1.057940in}}%
\pgfpathlineto{\pgfqpoint{1.477002in}{1.680953in}}%
\pgfpathlineto{\pgfqpoint{1.480952in}{0.990789in}}%
\pgfpathlineto{\pgfqpoint{1.483859in}{1.723901in}}%
\pgfpathlineto{\pgfqpoint{1.487056in}{1.079111in}}%
\pgfpathlineto{\pgfqpoint{1.489809in}{1.699458in}}%
\pgfpathlineto{\pgfqpoint{1.494196in}{1.026919in}}%
\pgfpathlineto{\pgfqpoint{1.497142in}{1.750842in}}%
\pgfpathlineto{\pgfqpoint{1.499677in}{1.054516in}}%
\pgfpathlineto{\pgfqpoint{1.502803in}{1.731890in}}%
\pgfpathlineto{\pgfqpoint{1.505871in}{0.997256in}}%
\pgfpathlineto{\pgfqpoint{1.509621in}{1.732776in}}%
\pgfpathlineto{\pgfqpoint{1.513738in}{1.013596in}}%
\pgfpathlineto{\pgfqpoint{1.515784in}{1.687130in}}%
\pgfpathlineto{\pgfqpoint{1.519006in}{1.030255in}}%
\pgfpathlineto{\pgfqpoint{1.522190in}{1.862694in}}%
\pgfpathlineto{\pgfqpoint{1.525355in}{0.991856in}}%
\pgfpathlineto{\pgfqpoint{1.528977in}{1.695608in}}%
\pgfpathlineto{\pgfqpoint{1.533807in}{1.024631in}}%
\pgfpathlineto{\pgfqpoint{1.535911in}{1.658778in}}%
\pgfpathlineto{\pgfqpoint{1.538838in}{1.028650in}}%
\pgfpathlineto{\pgfqpoint{1.541584in}{1.718313in}}%
\pgfpathlineto{\pgfqpoint{1.544646in}{1.056032in}}%
\pgfpathlineto{\pgfqpoint{1.548949in}{1.776379in}}%
\pgfpathlineto{\pgfqpoint{1.551561in}{1.050844in}}%
\pgfpathlineto{\pgfqpoint{1.554423in}{1.702553in}}%
\pgfpathlineto{\pgfqpoint{1.557530in}{1.127453in}}%
\pgfpathlineto{\pgfqpoint{1.560894in}{1.668934in}}%
\pgfpathlineto{\pgfqpoint{1.564625in}{1.103523in}}%
\pgfpathlineto{\pgfqpoint{1.567179in}{1.654155in}}%
\pgfpathlineto{\pgfqpoint{1.570344in}{1.047267in}}%
\pgfpathlineto{\pgfqpoint{1.573746in}{1.704602in}}%
\pgfpathlineto{\pgfqpoint{1.577291in}{1.013436in}}%
\pgfpathlineto{\pgfqpoint{1.580320in}{1.760947in}}%
\pgfpathlineto{\pgfqpoint{1.583344in}{1.144265in}}%
\pgfpathlineto{\pgfqpoint{1.586457in}{1.666995in}}%
\pgfpathlineto{\pgfqpoint{1.590066in}{1.080091in}}%
\pgfpathlineto{\pgfqpoint{1.594420in}{1.755200in}}%
\pgfpathlineto{\pgfqpoint{1.596890in}{1.060880in}}%
\pgfpathlineto{\pgfqpoint{1.600859in}{1.758513in}}%
\pgfpathlineto{\pgfqpoint{1.602680in}{1.078936in}}%
\pgfpathlineto{\pgfqpoint{1.607395in}{1.754808in}}%
\pgfpathlineto{\pgfqpoint{1.609003in}{1.044972in}}%
\pgfpathlineto{\pgfqpoint{1.612952in}{1.699934in}}%
\pgfpathlineto{\pgfqpoint{1.616445in}{1.060455in}}%
\pgfpathlineto{\pgfqpoint{1.619198in}{1.671192in}}%
\pgfpathlineto{\pgfqpoint{1.621906in}{1.067587in}}%
\pgfpathlineto{\pgfqpoint{1.625303in}{1.716161in}}%
\pgfpathlineto{\pgfqpoint{1.630873in}{0.989747in}}%
\pgfpathlineto{\pgfqpoint{1.632179in}{1.728743in}}%
\pgfpathlineto{\pgfqpoint{1.635369in}{0.976480in}}%
\pgfpathlineto{\pgfqpoint{1.638290in}{1.646435in}}%
\pgfpathlineto{\pgfqpoint{1.641197in}{1.005907in}}%
\pgfpathlineto{\pgfqpoint{1.645410in}{1.744916in}}%
\pgfpathlineto{\pgfqpoint{1.648723in}{1.068852in}}%
\pgfpathlineto{\pgfqpoint{1.651264in}{1.757923in}}%
\pgfpathlineto{\pgfqpoint{1.655085in}{1.037903in}}%
\pgfpathlineto{\pgfqpoint{1.657439in}{1.774435in}}%
\pgfpathlineto{\pgfqpoint{1.660900in}{1.066340in}}%
\pgfpathlineto{\pgfqpoint{1.664077in}{1.694483in}}%
\pgfpathlineto{\pgfqpoint{1.667261in}{1.112385in}}%
\pgfpathlineto{\pgfqpoint{1.670240in}{1.667197in}}%
\pgfpathlineto{\pgfqpoint{1.674582in}{0.978437in}}%
\pgfpathlineto{\pgfqpoint{1.678711in}{1.779933in}}%
\pgfpathlineto{\pgfqpoint{1.680313in}{1.079889in}}%
\pgfpathlineto{\pgfqpoint{1.683581in}{1.754525in}}%
\pgfpathlineto{\pgfqpoint{1.686649in}{1.100288in}}%
\pgfpathlineto{\pgfqpoint{1.689724in}{1.645567in}}%
\pgfpathlineto{\pgfqpoint{1.694741in}{0.904573in}}%
\pgfpathlineto{\pgfqpoint{1.695989in}{1.672558in}}%
\pgfpathlineto{\pgfqpoint{1.700125in}{1.055348in}}%
\pgfpathlineto{\pgfqpoint{1.702904in}{1.731354in}}%
\pgfpathlineto{\pgfqpoint{1.707837in}{0.954341in}}%
\pgfpathlineto{\pgfqpoint{1.709388in}{1.708244in}}%
\pgfpathlineto{\pgfqpoint{1.713131in}{1.072819in}}%
\pgfpathlineto{\pgfqpoint{1.715634in}{1.706967in}}%
\pgfpathlineto{\pgfqpoint{1.719146in}{1.016052in}}%
\pgfpathlineto{\pgfqpoint{1.722021in}{1.647776in}}%
\pgfpathlineto{\pgfqpoint{1.727842in}{0.964247in}}%
\pgfpathlineto{\pgfqpoint{1.729148in}{1.814330in}}%
\pgfpathlineto{\pgfqpoint{1.732094in}{1.098019in}}%
\pgfpathlineto{\pgfqpoint{1.735394in}{1.770304in}}%
\pgfpathlineto{\pgfqpoint{1.738205in}{0.982454in}}%
\pgfpathlineto{\pgfqpoint{1.741994in}{1.717171in}}%
\pgfpathlineto{\pgfqpoint{1.745634in}{1.038337in}}%
\pgfpathlineto{\pgfqpoint{1.748060in}{1.694708in}}%
\pgfpathlineto{\pgfqpoint{1.751089in}{1.008748in}}%
\pgfpathlineto{\pgfqpoint{1.754235in}{1.688121in}}%
\pgfpathlineto{\pgfqpoint{1.758705in}{1.010516in}}%
\pgfpathlineto{\pgfqpoint{1.760564in}{1.707939in}}%
\pgfpathlineto{\pgfqpoint{1.766386in}{0.982634in}}%
\pgfpathlineto{\pgfqpoint{1.767293in}{1.643282in}}%
\pgfpathlineto{\pgfqpoint{1.771326in}{1.001775in}}%
\pgfpathlineto{\pgfqpoint{1.774420in}{1.691661in}}%
\pgfpathlineto{\pgfqpoint{1.777681in}{0.993386in}}%
\pgfpathlineto{\pgfqpoint{1.780035in}{1.694648in}}%
\pgfpathlineto{\pgfqpoint{1.784506in}{0.980769in}}%
\pgfpathlineto{\pgfqpoint{1.786551in}{1.636501in}}%
\pgfpathlineto{\pgfqpoint{1.790694in}{1.023071in}}%
\pgfpathlineto{\pgfqpoint{1.793022in}{1.684629in}}%
\pgfpathlineto{\pgfqpoint{1.796509in}{1.040947in}}%
\pgfpathlineto{\pgfqpoint{1.799397in}{1.746873in}}%
\pgfpathlineto{\pgfqpoint{1.802491in}{1.050896in}}%
\pgfpathlineto{\pgfqpoint{1.805823in}{1.875203in}}%
\pgfpathlineto{\pgfqpoint{1.809534in}{1.020329in}}%
\pgfpathlineto{\pgfqpoint{1.813420in}{1.727086in}}%
\pgfpathlineto{\pgfqpoint{1.815221in}{1.134247in}}%
\pgfpathlineto{\pgfqpoint{1.818823in}{1.783109in}}%
\pgfpathlineto{\pgfqpoint{1.822702in}{1.025516in}}%
\pgfpathlineto{\pgfqpoint{1.825847in}{1.868272in}}%
\pgfpathlineto{\pgfqpoint{1.828890in}{1.027466in}}%
\pgfpathlineto{\pgfqpoint{1.831373in}{1.761167in}}%
\pgfpathlineto{\pgfqpoint{1.835329in}{1.083258in}}%
\pgfpathlineto{\pgfqpoint{1.837863in}{1.706374in}}%
\pgfpathlineto{\pgfqpoint{1.841414in}{0.965669in}}%
\pgfpathlineto{\pgfqpoint{1.844990in}{1.707238in}}%
\pgfpathlineto{\pgfqpoint{1.848959in}{1.051305in}}%
\pgfpathlineto{\pgfqpoint{1.851474in}{1.706287in}}%
\pgfpathlineto{\pgfqpoint{1.854099in}{1.094402in}}%
\pgfpathlineto{\pgfqpoint{1.857649in}{1.680098in}}%
\pgfpathlineto{\pgfqpoint{1.861148in}{0.997127in}}%
\pgfpathlineto{\pgfqpoint{1.863567in}{1.680462in}}%
\pgfpathlineto{\pgfqpoint{1.867163in}{1.055860in}}%
\pgfpathlineto{\pgfqpoint{1.870077in}{1.741682in}}%
\pgfpathlineto{\pgfqpoint{1.873434in}{0.964419in}}%
\pgfpathlineto{\pgfqpoint{1.877493in}{1.717346in}}%
\pgfpathlineto{\pgfqpoint{1.880298in}{0.900514in}}%
\pgfpathlineto{\pgfqpoint{1.883122in}{1.661398in}}%
\pgfpathlineto{\pgfqpoint{1.886788in}{0.994794in}}%
\pgfpathlineto{\pgfqpoint{1.889702in}{1.626603in}}%
\pgfpathlineto{\pgfqpoint{1.892815in}{1.125985in}}%
\pgfpathlineto{\pgfqpoint{1.896598in}{1.707556in}}%
\pgfpathlineto{\pgfqpoint{1.899113in}{1.055031in}}%
\pgfpathlineto{\pgfqpoint{1.903448in}{1.715055in}}%
\pgfpathlineto{\pgfqpoint{1.905944in}{1.081278in}}%
\pgfpathlineto{\pgfqpoint{1.908909in}{1.628979in}}%
\pgfpathlineto{\pgfqpoint{1.912499in}{1.077481in}}%
\pgfpathlineto{\pgfqpoint{1.915284in}{1.671403in}}%
\pgfpathlineto{\pgfqpoint{1.919761in}{1.068185in}}%
\pgfpathlineto{\pgfqpoint{1.922077in}{1.670612in}}%
\pgfpathlineto{\pgfqpoint{1.926927in}{0.913578in}}%
\pgfpathlineto{\pgfqpoint{1.928618in}{1.793778in}}%
\pgfpathlineto{\pgfqpoint{1.932819in}{1.022210in}}%
\pgfpathlineto{\pgfqpoint{1.934948in}{1.633031in}}%
\pgfpathlineto{\pgfqpoint{1.937759in}{1.049061in}}%
\pgfpathlineto{\pgfqpoint{1.943246in}{1.820656in}}%
\pgfpathlineto{\pgfqpoint{1.945857in}{0.959979in}}%
\pgfpathlineto{\pgfqpoint{1.947858in}{1.682445in}}%
\pgfpathlineto{\pgfqpoint{1.950849in}{1.100444in}}%
\pgfpathlineto{\pgfqpoint{1.953943in}{1.638548in}}%
\pgfpathlineto{\pgfqpoint{1.957719in}{0.960335in}}%
\pgfpathlineto{\pgfqpoint{1.960511in}{1.947774in}}%
\pgfpathlineto{\pgfqpoint{1.964647in}{1.039808in}}%
\pgfpathlineto{\pgfqpoint{1.966686in}{1.678991in}}%
\pgfpathlineto{\pgfqpoint{1.970011in}{1.016091in}}%
\pgfpathlineto{\pgfqpoint{1.973530in}{1.811810in}}%
\pgfpathlineto{\pgfqpoint{1.977203in}{1.028244in}}%
\pgfpathlineto{\pgfqpoint{1.979789in}{1.677622in}}%
\pgfpathlineto{\pgfqpoint{1.983732in}{1.085845in}}%
\pgfpathlineto{\pgfqpoint{1.986601in}{1.730023in}}%
\pgfpathlineto{\pgfqpoint{1.991065in}{0.973951in}}%
\pgfpathlineto{\pgfqpoint{1.992827in}{1.685812in}}%
\pgfpathlineto{\pgfqpoint{1.996545in}{1.102913in}}%
\pgfpathlineto{\pgfqpoint{1.999768in}{1.732266in}}%
\pgfpathlineto{\pgfqpoint{2.002791in}{1.056217in}}%
\pgfpathlineto{\pgfqpoint{2.006290in}{1.723855in}}%
\pgfpathlineto{\pgfqpoint{2.008651in}{1.015265in}}%
\pgfpathlineto{\pgfqpoint{2.012639in}{1.774132in}}%
\pgfpathlineto{\pgfqpoint{2.015122in}{1.022672in}}%
\pgfpathlineto{\pgfqpoint{2.018428in}{1.670106in}}%
\pgfpathlineto{\pgfqpoint{2.022236in}{0.997439in}}%
\pgfpathlineto{\pgfqpoint{2.026823in}{1.754252in}}%
\pgfpathlineto{\pgfqpoint{2.028000in}{1.126671in}}%
\pgfpathlineto{\pgfqpoint{2.032155in}{1.673254in}}%
\pgfpathlineto{\pgfqpoint{2.035500in}{1.062531in}}%
\pgfpathlineto{\pgfqpoint{2.037751in}{1.678140in}}%
\pgfpathlineto{\pgfqpoint{2.040839in}{1.081894in}}%
\pgfpathlineto{\pgfqpoint{2.044396in}{1.788377in}}%
\pgfpathlineto{\pgfqpoint{2.047638in}{1.080240in}}%
\pgfpathlineto{\pgfqpoint{2.050591in}{1.646668in}}%
\pgfpathlineto{\pgfqpoint{2.054309in}{0.988560in}}%
\pgfpathlineto{\pgfqpoint{2.057473in}{1.771910in}}%
\pgfpathlineto{\pgfqpoint{2.061281in}{1.041154in}}%
\pgfpathlineto{\pgfqpoint{2.063861in}{1.666433in}}%
\pgfpathlineto{\pgfqpoint{2.069348in}{0.980817in}}%
\pgfpathlineto{\pgfqpoint{2.069811in}{1.623705in}}%
\pgfpathlineto{\pgfqpoint{2.073464in}{1.026053in}}%
\pgfpathlineto{\pgfqpoint{2.076301in}{1.616043in}}%
\pgfpathlineto{\pgfqpoint{2.079685in}{1.085432in}}%
\pgfpathlineto{\pgfqpoint{2.083235in}{1.730784in}}%
\pgfpathlineto{\pgfqpoint{2.086156in}{1.110115in}}%
\pgfpathlineto{\pgfqpoint{2.090009in}{1.739015in}}%
\pgfpathlineto{\pgfqpoint{2.092427in}{1.115791in}}%
\pgfpathlineto{\pgfqpoint{2.096750in}{1.767449in}}%
\pgfpathlineto{\pgfqpoint{2.098892in}{1.063212in}}%
\pgfpathlineto{\pgfqpoint{2.102874in}{1.742336in}}%
\pgfpathlineto{\pgfqpoint{2.105672in}{1.098602in}}%
\pgfpathlineto{\pgfqpoint{2.108528in}{1.653116in}}%
\pgfpathlineto{\pgfqpoint{2.112696in}{1.032823in}}%
\pgfpathlineto{\pgfqpoint{2.115095in}{1.626489in}}%
\pgfpathlineto{\pgfqpoint{2.118582in}{1.090799in}}%
\pgfpathlineto{\pgfqpoint{2.122267in}{1.837597in}}%
\pgfpathlineto{\pgfqpoint{2.124802in}{1.076921in}}%
\pgfpathlineto{\pgfqpoint{2.128147in}{1.731559in}}%
\pgfpathlineto{\pgfqpoint{2.131588in}{1.060040in}}%
\pgfpathlineto{\pgfqpoint{2.134406in}{1.658820in}}%
\pgfpathlineto{\pgfqpoint{2.138690in}{1.019800in}}%
\pgfpathlineto{\pgfqpoint{2.142420in}{1.732543in}}%
\pgfpathlineto{\pgfqpoint{2.145032in}{1.065351in}}%
\pgfpathlineto{\pgfqpoint{2.147322in}{1.684383in}}%
\pgfpathlineto{\pgfqpoint{2.152481in}{0.872943in}}%
\pgfpathlineto{\pgfqpoint{2.153896in}{1.680875in}}%
\pgfpathlineto{\pgfqpoint{2.158714in}{0.994883in}}%
\pgfpathlineto{\pgfqpoint{2.160586in}{1.728221in}}%
\pgfpathlineto{\pgfqpoint{2.163918in}{0.972175in}}%
\pgfpathlineto{\pgfqpoint{2.166510in}{1.660251in}}%
\pgfpathlineto{\pgfqpoint{2.170228in}{1.100548in}}%
\pgfpathlineto{\pgfqpoint{2.173399in}{1.685156in}}%
\pgfpathlineto{\pgfqpoint{2.177394in}{0.955533in}}%
\pgfpathlineto{\pgfqpoint{2.180732in}{1.733711in}}%
\pgfpathlineto{\pgfqpoint{2.182823in}{1.060375in}}%
\pgfpathlineto{\pgfqpoint{2.186097in}{1.682178in}}%
\pgfpathlineto{\pgfqpoint{2.189107in}{1.076165in}}%
\pgfpathlineto{\pgfqpoint{2.192799in}{1.668635in}}%
\pgfpathlineto{\pgfqpoint{2.195713in}{1.129483in}}%
\pgfpathlineto{\pgfqpoint{2.201618in}{1.741373in}}%
\pgfpathlineto{\pgfqpoint{2.202146in}{1.045370in}}%
\pgfpathlineto{\pgfqpoint{2.205259in}{1.650958in}}%
\pgfpathlineto{\pgfqpoint{2.210932in}{0.916212in}}%
\pgfpathlineto{\pgfqpoint{2.212502in}{1.703987in}}%
\pgfpathlineto{\pgfqpoint{2.215068in}{1.036087in}}%
\pgfpathlineto{\pgfqpoint{2.219989in}{1.802478in}}%
\pgfpathlineto{\pgfqpoint{2.221308in}{1.104401in}}%
\pgfpathlineto{\pgfqpoint{2.226100in}{1.767899in}}%
\pgfpathlineto{\pgfqpoint{2.227991in}{1.053653in}}%
\pgfpathlineto{\pgfqpoint{2.231548in}{1.707349in}}%
\pgfpathlineto{\pgfqpoint{2.234848in}{1.064228in}}%
\pgfpathlineto{\pgfqpoint{2.237473in}{1.717969in}}%
\pgfpathlineto{\pgfqpoint{2.242683in}{1.019270in}}%
\pgfpathlineto{\pgfqpoint{2.244265in}{1.727336in}}%
\pgfpathlineto{\pgfqpoint{2.247289in}{1.056474in}}%
\pgfpathlineto{\pgfqpoint{2.250724in}{1.760882in}}%
\pgfpathlineto{\pgfqpoint{2.253528in}{1.047621in}}%
\pgfpathlineto{\pgfqpoint{2.256725in}{1.630592in}}%
\pgfpathlineto{\pgfqpoint{2.260424in}{1.014447in}}%
\pgfpathlineto{\pgfqpoint{2.263884in}{1.761710in}}%
\pgfpathlineto{\pgfqpoint{2.266965in}{1.113270in}}%
\pgfpathlineto{\pgfqpoint{2.271661in}{1.734161in}}%
\pgfpathlineto{\pgfqpoint{2.273662in}{1.032666in}}%
\pgfpathlineto{\pgfqpoint{2.276563in}{1.737323in}}%
\pgfpathlineto{\pgfqpoint{2.279580in}{1.103879in}}%
\pgfpathlineto{\pgfqpoint{2.283574in}{1.736643in}}%
\pgfpathlineto{\pgfqpoint{2.285780in}{1.013346in}}%
\pgfpathlineto{\pgfqpoint{2.289460in}{1.697539in}}%
\pgfpathlineto{\pgfqpoint{2.294181in}{0.973262in}}%
\pgfpathlineto{\pgfqpoint{2.296085in}{1.685866in}}%
\pgfpathlineto{\pgfqpoint{2.299771in}{0.970853in}}%
\pgfpathlineto{\pgfqpoint{2.301990in}{1.754538in}}%
\pgfpathlineto{\pgfqpoint{2.305187in}{0.997491in}}%
\pgfpathlineto{\pgfqpoint{2.310211in}{1.803744in}}%
\pgfpathlineto{\pgfqpoint{2.312462in}{1.049681in}}%
\pgfpathlineto{\pgfqpoint{2.314971in}{1.637390in}}%
\pgfpathlineto{\pgfqpoint{2.318888in}{1.092223in}}%
\pgfpathlineto{\pgfqpoint{2.322793in}{1.765593in}}%
\pgfpathlineto{\pgfqpoint{2.325070in}{1.081224in}}%
\pgfpathlineto{\pgfqpoint{2.327675in}{1.685935in}}%
\pgfpathlineto{\pgfqpoint{2.331296in}{1.104301in}}%
\pgfpathlineto{\pgfqpoint{2.334879in}{1.680239in}}%
\pgfpathlineto{\pgfqpoint{2.337587in}{1.005403in}}%
\pgfpathlineto{\pgfqpoint{2.340598in}{1.657421in}}%
\pgfpathlineto{\pgfqpoint{2.344071in}{1.051064in}}%
\pgfpathlineto{\pgfqpoint{2.347551in}{1.756920in}}%
\pgfpathlineto{\pgfqpoint{2.350491in}{1.027486in}}%
\pgfpathlineto{\pgfqpoint{2.354164in}{1.720583in}}%
\pgfpathlineto{\pgfqpoint{2.356775in}{1.051428in}}%
\pgfpathlineto{\pgfqpoint{2.360088in}{1.849528in}}%
\pgfpathlineto{\pgfqpoint{2.365453in}{0.953773in}}%
\pgfpathlineto{\pgfqpoint{2.366617in}{1.726026in}}%
\pgfpathlineto{\pgfqpoint{2.369640in}{1.092284in}}%
\pgfpathlineto{\pgfqpoint{2.372947in}{1.666959in}}%
\pgfpathlineto{\pgfqpoint{2.376465in}{0.976550in}}%
\pgfpathlineto{\pgfqpoint{2.380099in}{1.704649in}}%
\pgfpathlineto{\pgfqpoint{2.382647in}{1.052461in}}%
\pgfpathlineto{\pgfqpoint{2.386667in}{1.663344in}}%
\pgfpathlineto{\pgfqpoint{2.389446in}{1.105309in}}%
\pgfpathlineto{\pgfqpoint{2.392379in}{1.650447in}}%
\pgfpathlineto{\pgfqpoint{2.396026in}{0.979072in}}%
\pgfpathlineto{\pgfqpoint{2.398747in}{1.686516in}}%
\pgfpathlineto{\pgfqpoint{2.402478in}{1.064779in}}%
\pgfpathlineto{\pgfqpoint{2.405604in}{1.757573in}}%
\pgfpathlineto{\pgfqpoint{2.408402in}{1.066478in}}%
\pgfpathlineto{\pgfqpoint{2.411709in}{1.696068in}}%
\pgfpathlineto{\pgfqpoint{2.415201in}{0.998299in}}%
\pgfpathlineto{\pgfqpoint{2.418302in}{1.674986in}}%
\pgfpathlineto{\pgfqpoint{2.421653in}{1.115589in}}%
\pgfpathlineto{\pgfqpoint{2.425152in}{1.690816in}}%
\pgfpathlineto{\pgfqpoint{2.427616in}{1.014172in}}%
\pgfpathlineto{\pgfqpoint{2.432781in}{1.830374in}}%
\pgfpathlineto{\pgfqpoint{2.434222in}{1.112868in}}%
\pgfpathlineto{\pgfqpoint{2.438371in}{1.748726in}}%
\pgfpathlineto{\pgfqpoint{2.440790in}{1.056741in}}%
\pgfpathlineto{\pgfqpoint{2.445222in}{1.825243in}}%
\pgfpathlineto{\pgfqpoint{2.447003in}{1.118029in}}%
\pgfpathlineto{\pgfqpoint{2.452233in}{1.676063in}}%
\pgfpathlineto{\pgfqpoint{2.454047in}{0.963376in}}%
\pgfpathlineto{\pgfqpoint{2.456678in}{1.647299in}}%
\pgfpathlineto{\pgfqpoint{2.460383in}{1.062019in}}%
\pgfpathlineto{\pgfqpoint{2.464236in}{1.751301in}}%
\pgfpathlineto{\pgfqpoint{2.466307in}{1.073578in}}%
\pgfpathlineto{\pgfqpoint{2.471903in}{1.727528in}}%
\pgfpathlineto{\pgfqpoint{2.473203in}{1.077234in}}%
\pgfpathlineto{\pgfqpoint{2.476040in}{1.669492in}}%
\pgfpathlineto{\pgfqpoint{2.481571in}{0.941586in}}%
\pgfpathlineto{\pgfqpoint{2.482716in}{1.687168in}}%
\pgfpathlineto{\pgfqpoint{2.486100in}{1.038278in}}%
\pgfpathlineto{\pgfqpoint{2.489149in}{1.716565in}}%
\pgfpathlineto{\pgfqpoint{2.492166in}{1.057482in}}%
\pgfpathlineto{\pgfqpoint{2.495253in}{1.738459in}}%
\pgfpathlineto{\pgfqpoint{2.498830in}{1.075619in}}%
\pgfpathlineto{\pgfqpoint{2.503159in}{1.707566in}}%
\pgfpathlineto{\pgfqpoint{2.507932in}{0.932652in}}%
\pgfpathlineto{\pgfqpoint{2.508176in}{1.702261in}}%
\pgfpathlineto{\pgfqpoint{2.513856in}{1.026501in}}%
\pgfpathlineto{\pgfqpoint{2.516339in}{1.756979in}}%
\pgfpathlineto{\pgfqpoint{2.518114in}{1.093921in}}%
\pgfpathlineto{\pgfqpoint{2.521761in}{1.762727in}}%
\pgfpathlineto{\pgfqpoint{2.524373in}{1.064609in}}%
\pgfpathlineto{\pgfqpoint{2.528065in}{1.734702in}}%
\pgfpathlineto{\pgfqpoint{2.532999in}{0.995076in}}%
\pgfpathlineto{\pgfqpoint{2.534002in}{1.665870in}}%
\pgfpathlineto{\pgfqpoint{2.537791in}{1.051706in}}%
\pgfpathlineto{\pgfqpoint{2.541548in}{1.664855in}}%
\pgfpathlineto{\pgfqpoint{2.544236in}{1.090172in}}%
\pgfpathlineto{\pgfqpoint{2.547054in}{1.709043in}}%
\pgfpathlineto{\pgfqpoint{2.550289in}{1.022711in}}%
\pgfpathlineto{\pgfqpoint{2.553403in}{1.675269in}}%
\pgfpathlineto{\pgfqpoint{2.556452in}{1.025900in}}%
\pgfpathlineto{\pgfqpoint{2.560420in}{1.698929in}}%
\pgfpathlineto{\pgfqpoint{2.563797in}{1.077088in}}%
\pgfpathlineto{\pgfqpoint{2.566924in}{1.695170in}}%
\pgfpathlineto{\pgfqpoint{2.569426in}{1.029593in}}%
\pgfpathlineto{\pgfqpoint{2.573105in}{1.746534in}}%
\pgfpathlineto{\pgfqpoint{2.576321in}{1.078539in}}%
\pgfpathlineto{\pgfqpoint{2.580734in}{1.732503in}}%
\pgfpathlineto{\pgfqpoint{2.583442in}{0.933559in}}%
\pgfpathlineto{\pgfqpoint{2.585449in}{1.595901in}}%
\pgfpathlineto{\pgfqpoint{2.589933in}{1.021707in}}%
\pgfpathlineto{\pgfqpoint{2.593033in}{1.722210in}}%
\pgfpathlineto{\pgfqpoint{2.597092in}{1.024379in}}%
\pgfpathlineto{\pgfqpoint{2.598423in}{1.639352in}}%
\pgfpathlineto{\pgfqpoint{2.601575in}{1.076573in}}%
\pgfpathlineto{\pgfqpoint{2.605229in}{1.790035in}}%
\pgfpathlineto{\pgfqpoint{2.609030in}{1.026846in}}%
\pgfpathlineto{\pgfqpoint{2.611346in}{1.656296in}}%
\pgfpathlineto{\pgfqpoint{2.615694in}{1.006017in}}%
\pgfpathlineto{\pgfqpoint{2.617669in}{1.662642in}}%
\pgfpathlineto{\pgfqpoint{2.620963in}{1.008933in}}%
\pgfpathlineto{\pgfqpoint{2.624932in}{1.697785in}}%
\pgfpathlineto{\pgfqpoint{2.627550in}{1.001115in}}%
\pgfpathlineto{\pgfqpoint{2.632843in}{1.824192in}}%
\pgfpathlineto{\pgfqpoint{2.634117in}{1.023567in}}%
\pgfpathlineto{\pgfqpoint{2.637636in}{1.671156in}}%
\pgfpathlineto{\pgfqpoint{2.641225in}{1.041392in}}%
\pgfpathlineto{\pgfqpoint{2.643592in}{1.699752in}}%
\pgfpathlineto{\pgfqpoint{2.647690in}{0.959744in}}%
\pgfpathlineto{\pgfqpoint{2.649947in}{1.697753in}}%
\pgfpathlineto{\pgfqpoint{2.653742in}{1.066607in}}%
\pgfpathlineto{\pgfqpoint{2.657538in}{1.774060in}}%
\pgfpathlineto{\pgfqpoint{2.659570in}{1.095146in}}%
\pgfpathlineto{\pgfqpoint{2.663372in}{1.836417in}}%
\pgfpathlineto{\pgfqpoint{2.667540in}{1.021507in}}%
\pgfpathlineto{\pgfqpoint{2.669322in}{1.638758in}}%
\pgfpathlineto{\pgfqpoint{2.674609in}{0.984548in}}%
\pgfpathlineto{\pgfqpoint{2.675838in}{1.653392in}}%
\pgfpathlineto{\pgfqpoint{2.680064in}{1.036074in}}%
\pgfpathlineto{\pgfqpoint{2.682418in}{1.667701in}}%
\pgfpathlineto{\pgfqpoint{2.685699in}{1.001607in}}%
\pgfpathlineto{\pgfqpoint{2.689031in}{1.751981in}}%
\pgfpathlineto{\pgfqpoint{2.693977in}{0.863521in}}%
\pgfpathlineto{\pgfqpoint{2.695232in}{1.726652in}}%
\pgfpathlineto{\pgfqpoint{2.698866in}{1.002623in}}%
\pgfpathlineto{\pgfqpoint{2.701838in}{1.735450in}}%
\pgfpathlineto{\pgfqpoint{2.706894in}{0.975634in}}%
\pgfpathlineto{\pgfqpoint{2.708663in}{1.701857in}}%
\pgfpathlineto{\pgfqpoint{2.712419in}{1.032729in}}%
\pgfpathlineto{\pgfqpoint{2.716935in}{1.790945in}}%
\pgfpathlineto{\pgfqpoint{2.718318in}{1.021533in}}%
\pgfpathlineto{\pgfqpoint{2.721405in}{1.689192in}}%
\pgfpathlineto{\pgfqpoint{2.726461in}{1.049866in}}%
\pgfpathlineto{\pgfqpoint{2.727343in}{1.756016in}}%
\pgfpathlineto{\pgfqpoint{2.731845in}{1.041449in}}%
\pgfpathlineto{\pgfqpoint{2.733679in}{1.800535in}}%
\pgfpathlineto{\pgfqpoint{2.737853in}{0.960786in}}%
\pgfpathlineto{\pgfqpoint{2.740439in}{1.741966in}}%
\pgfpathlineto{\pgfqpoint{2.743527in}{1.108610in}}%
\pgfpathlineto{\pgfqpoint{2.747984in}{1.716518in}}%
\pgfpathlineto{\pgfqpoint{2.749824in}{1.105360in}}%
\pgfpathlineto{\pgfqpoint{2.754076in}{1.669388in}}%
\pgfpathlineto{\pgfqpoint{2.757125in}{1.030016in}}%
\pgfpathlineto{\pgfqpoint{2.759672in}{1.687574in}}%
\pgfpathlineto{\pgfqpoint{2.762946in}{1.013724in}}%
\pgfpathlineto{\pgfqpoint{2.766375in}{1.823340in}}%
\pgfpathlineto{\pgfqpoint{2.769411in}{1.093776in}}%
\pgfpathlineto{\pgfqpoint{2.772801in}{1.809590in}}%
\pgfpathlineto{\pgfqpoint{2.776782in}{1.034428in}}%
\pgfpathlineto{\pgfqpoint{2.779979in}{1.712715in}}%
\pgfpathlineto{\pgfqpoint{2.782366in}{1.093019in}}%
\pgfpathlineto{\pgfqpoint{2.786142in}{1.667318in}}%
\pgfpathlineto{\pgfqpoint{2.788978in}{1.071478in}}%
\pgfpathlineto{\pgfqpoint{2.791847in}{1.765720in}}%
\pgfpathlineto{\pgfqpoint{2.796266in}{1.014833in}}%
\pgfpathlineto{\pgfqpoint{2.798730in}{1.753926in}}%
\pgfpathlineto{\pgfqpoint{2.801670in}{1.034629in}}%
\pgfpathlineto{\pgfqpoint{2.805111in}{1.684145in}}%
\pgfpathlineto{\pgfqpoint{2.807980in}{0.968022in}}%
\pgfpathlineto{\pgfqpoint{2.811601in}{1.889899in}}%
\pgfpathlineto{\pgfqpoint{2.815815in}{1.037236in}}%
\pgfpathlineto{\pgfqpoint{2.817963in}{1.724750in}}%
\pgfpathlineto{\pgfqpoint{2.820780in}{1.058328in}}%
\pgfpathlineto{\pgfqpoint{2.824177in}{1.676183in}}%
\pgfpathlineto{\pgfqpoint{2.827650in}{1.035453in}}%
\pgfpathlineto{\pgfqpoint{2.831342in}{1.664177in}}%
\pgfpathlineto{\pgfqpoint{2.833613in}{1.099268in}}%
\pgfpathlineto{\pgfqpoint{2.837022in}{1.744259in}}%
\pgfpathlineto{\pgfqpoint{2.841171in}{0.855418in}}%
\pgfpathlineto{\pgfqpoint{2.843294in}{1.611215in}}%
\pgfpathlineto{\pgfqpoint{2.847462in}{0.956037in}}%
\pgfpathlineto{\pgfqpoint{2.850704in}{1.714463in}}%
\pgfpathlineto{\pgfqpoint{2.853657in}{1.061238in}}%
\pgfpathlineto{\pgfqpoint{2.856345in}{1.747986in}}%
\pgfpathlineto{\pgfqpoint{2.859581in}{1.090105in}}%
\pgfpathlineto{\pgfqpoint{2.863846in}{1.784938in}}%
\pgfpathlineto{\pgfqpoint{2.866354in}{1.003894in}}%
\pgfpathlineto{\pgfqpoint{2.870986in}{1.770678in}}%
\pgfpathlineto{\pgfqpoint{2.873070in}{1.054058in}}%
\pgfpathlineto{\pgfqpoint{2.876350in}{1.718228in}}%
\pgfpathlineto{\pgfqpoint{2.878904in}{1.044586in}}%
\pgfpathlineto{\pgfqpoint{2.883149in}{1.772013in}}%
\pgfpathlineto{\pgfqpoint{2.886121in}{1.050942in}}%
\pgfpathlineto{\pgfqpoint{2.889511in}{1.782941in}}%
\pgfpathlineto{\pgfqpoint{2.894496in}{0.924324in}}%
\pgfpathlineto{\pgfqpoint{2.895307in}{1.582453in}}%
\pgfpathlineto{\pgfqpoint{2.899115in}{1.009359in}}%
\pgfpathlineto{\pgfqpoint{2.901276in}{1.655236in}}%
\pgfpathlineto{\pgfqpoint{2.905599in}{0.989713in}}%
\pgfpathlineto{\pgfqpoint{2.907889in}{1.707504in}}%
\pgfpathlineto{\pgfqpoint{2.911517in}{1.059486in}}%
\pgfpathlineto{\pgfqpoint{2.914668in}{1.673703in}}%
\pgfpathlineto{\pgfqpoint{2.917544in}{1.071388in}}%
\pgfpathlineto{\pgfqpoint{2.920998in}{1.781848in}}%
\pgfpathlineto{\pgfqpoint{2.925475in}{1.006602in}}%
\pgfpathlineto{\pgfqpoint{2.927906in}{1.711819in}}%
\pgfpathlineto{\pgfqpoint{2.930415in}{1.128479in}}%
\pgfpathlineto{\pgfqpoint{2.934577in}{1.722809in}}%
\pgfpathlineto{\pgfqpoint{2.937459in}{1.007735in}}%
\pgfpathlineto{\pgfqpoint{2.940411in}{1.730940in}}%
\pgfpathlineto{\pgfqpoint{2.943434in}{1.121114in}}%
\pgfpathlineto{\pgfqpoint{2.946471in}{1.633855in}}%
\pgfpathlineto{\pgfqpoint{2.950568in}{1.062151in}}%
\pgfpathlineto{\pgfqpoint{2.953025in}{1.784029in}}%
\pgfpathlineto{\pgfqpoint{2.957335in}{1.070535in}}%
\pgfpathlineto{\pgfqpoint{2.959432in}{1.655773in}}%
\pgfpathlineto{\pgfqpoint{2.963291in}{1.053001in}}%
\pgfpathlineto{\pgfqpoint{2.968367in}{1.783011in}}%
\pgfpathlineto{\pgfqpoint{2.969602in}{1.049625in}}%
\pgfpathlineto{\pgfqpoint{2.972593in}{1.757808in}}%
\pgfpathlineto{\pgfqpoint{2.975507in}{1.072149in}}%
\pgfpathlineto{\pgfqpoint{2.979675in}{1.743777in}}%
\pgfpathlineto{\pgfqpoint{2.982029in}{1.085018in}}%
\pgfpathlineto{\pgfqpoint{2.985348in}{1.656212in}}%
\pgfpathlineto{\pgfqpoint{2.989626in}{1.067700in}}%
\pgfpathlineto{\pgfqpoint{2.991671in}{1.802546in}}%
\pgfpathlineto{\pgfqpoint{2.995183in}{1.039862in}}%
\pgfpathlineto{\pgfqpoint{2.997975in}{1.682744in}}%
\pgfpathlineto{\pgfqpoint{3.002857in}{1.082314in}}%
\pgfpathlineto{\pgfqpoint{3.005302in}{1.700011in}}%
\pgfpathlineto{\pgfqpoint{3.007907in}{1.076142in}}%
\pgfpathlineto{\pgfqpoint{3.011355in}{1.711187in}}%
\pgfpathlineto{\pgfqpoint{3.014359in}{1.043097in}}%
\pgfpathlineto{\pgfqpoint{3.018514in}{1.760539in}}%
\pgfpathlineto{\pgfqpoint{3.020836in}{1.011372in}}%
\pgfpathlineto{\pgfqpoint{3.023962in}{1.810936in}}%
\pgfpathlineto{\pgfqpoint{3.027770in}{1.044660in}}%
\pgfpathlineto{\pgfqpoint{3.030440in}{1.657536in}}%
\pgfpathlineto{\pgfqpoint{3.034100in}{1.021869in}}%
\pgfpathlineto{\pgfqpoint{3.037342in}{1.762652in}}%
\pgfpathlineto{\pgfqpoint{3.039928in}{1.125788in}}%
\pgfpathlineto{\pgfqpoint{3.043684in}{1.690410in}}%
\pgfpathlineto{\pgfqpoint{3.047119in}{1.083345in}}%
\pgfpathlineto{\pgfqpoint{3.049499in}{1.655792in}}%
\pgfpathlineto{\pgfqpoint{3.054156in}{0.988006in}}%
\pgfpathlineto{\pgfqpoint{3.056800in}{1.709857in}}%
\pgfpathlineto{\pgfqpoint{3.059489in}{1.037103in}}%
\pgfpathlineto{\pgfqpoint{3.063149in}{1.699223in}}%
\pgfpathlineto{\pgfqpoint{3.066088in}{1.044810in}}%
\pgfpathlineto{\pgfqpoint{3.069755in}{1.722534in}}%
\pgfpathlineto{\pgfqpoint{3.073749in}{1.033597in}}%
\pgfpathlineto{\pgfqpoint{3.075358in}{1.628300in}}%
\pgfpathlineto{\pgfqpoint{3.079288in}{1.075838in}}%
\pgfpathlineto{\pgfqpoint{3.083784in}{1.723418in}}%
\pgfpathlineto{\pgfqpoint{3.085328in}{1.099324in}}%
\pgfpathlineto{\pgfqpoint{3.089490in}{1.672158in}}%
\pgfpathlineto{\pgfqpoint{3.092654in}{1.005450in}}%
\pgfpathlineto{\pgfqpoint{3.095028in}{1.777837in}}%
\pgfpathlineto{\pgfqpoint{3.098502in}{1.061272in}}%
\pgfpathlineto{\pgfqpoint{3.101924in}{1.803839in}}%
\pgfpathlineto{\pgfqpoint{3.106806in}{0.925364in}}%
\pgfpathlineto{\pgfqpoint{3.107552in}{1.644708in}}%
\pgfpathlineto{\pgfqpoint{3.110929in}{1.019849in}}%
\pgfpathlineto{\pgfqpoint{3.115310in}{1.689756in}}%
\pgfpathlineto{\pgfqpoint{3.117259in}{0.949696in}}%
\pgfpathlineto{\pgfqpoint{3.120713in}{1.715638in}}%
\pgfpathlineto{\pgfqpoint{3.123839in}{1.052217in}}%
\pgfpathlineto{\pgfqpoint{3.128901in}{1.779101in}}%
\pgfpathlineto{\pgfqpoint{3.131095in}{0.996417in}}%
\pgfpathlineto{\pgfqpoint{3.133854in}{1.712008in}}%
\pgfpathlineto{\pgfqpoint{3.136807in}{1.066784in}}%
\pgfpathlineto{\pgfqpoint{3.139785in}{1.765972in}}%
\pgfpathlineto{\pgfqpoint{3.143863in}{1.077777in}}%
\pgfpathlineto{\pgfqpoint{3.146308in}{1.672419in}}%
\pgfpathlineto{\pgfqpoint{3.150463in}{1.099078in}}%
\pgfpathlineto{\pgfqpoint{3.152663in}{1.681492in}}%
\pgfpathlineto{\pgfqpoint{3.156664in}{0.989599in}}%
\pgfpathlineto{\pgfqpoint{3.159610in}{1.759912in}}%
\pgfpathlineto{\pgfqpoint{3.163669in}{1.055651in}}%
\pgfpathlineto{\pgfqpoint{3.166274in}{1.642031in}}%
\pgfpathlineto{\pgfqpoint{3.170371in}{0.991995in}}%
\pgfpathlineto{\pgfqpoint{3.172057in}{1.677317in}}%
\pgfpathlineto{\pgfqpoint{3.175775in}{1.110950in}}%
\pgfpathlineto{\pgfqpoint{3.178727in}{1.737489in}}%
\pgfpathlineto{\pgfqpoint{3.181821in}{1.067468in}}%
\pgfpathlineto{\pgfqpoint{3.185243in}{1.702592in}}%
\pgfpathlineto{\pgfqpoint{3.188356in}{1.101777in}}%
\pgfpathlineto{\pgfqpoint{3.192010in}{1.709892in}}%
\pgfpathlineto{\pgfqpoint{3.194873in}{1.086568in}}%
\pgfpathlineto{\pgfqpoint{3.199362in}{1.764492in}}%
\pgfpathlineto{\pgfqpoint{3.201980in}{1.069325in}}%
\pgfpathlineto{\pgfqpoint{3.205962in}{1.758812in}}%
\pgfpathlineto{\pgfqpoint{3.207635in}{1.104761in}}%
\pgfpathlineto{\pgfqpoint{3.211108in}{1.690293in}}%
\pgfpathlineto{\pgfqpoint{3.217052in}{0.976898in}}%
\pgfpathlineto{\pgfqpoint{3.217914in}{1.737093in}}%
\pgfpathlineto{\pgfqpoint{3.220397in}{1.097878in}}%
\pgfpathlineto{\pgfqpoint{3.223703in}{1.759405in}}%
\pgfpathlineto{\pgfqpoint{3.227627in}{1.019766in}}%
\pgfpathlineto{\pgfqpoint{3.231673in}{1.722190in}}%
\pgfpathlineto{\pgfqpoint{3.233982in}{1.011480in}}%
\pgfpathlineto{\pgfqpoint{3.237693in}{1.792475in}}%
\pgfpathlineto{\pgfqpoint{3.239655in}{1.154855in}}%
\pgfpathlineto{\pgfqpoint{3.243032in}{1.617066in}}%
\pgfpathlineto{\pgfqpoint{3.246873in}{1.069162in}}%
\pgfpathlineto{\pgfqpoint{3.250256in}{1.666556in}}%
\pgfpathlineto{\pgfqpoint{3.253228in}{1.105010in}}%
\pgfpathlineto{\pgfqpoint{3.258174in}{1.734685in}}%
\pgfpathlineto{\pgfqpoint{3.259435in}{1.083085in}}%
\pgfpathlineto{\pgfqpoint{3.264735in}{1.775132in}}%
\pgfpathlineto{\pgfqpoint{3.265855in}{1.084633in}}%
\pgfpathlineto{\pgfqpoint{3.268756in}{1.691167in}}%
\pgfpathlineto{\pgfqpoint{3.272017in}{0.991167in}}%
\pgfpathlineto{\pgfqpoint{3.275400in}{1.697113in}}%
\pgfpathlineto{\pgfqpoint{3.279530in}{0.989636in}}%
\pgfpathlineto{\pgfqpoint{3.282978in}{1.792538in}}%
\pgfpathlineto{\pgfqpoint{3.284985in}{1.072114in}}%
\pgfpathlineto{\pgfqpoint{3.288870in}{1.707650in}}%
\pgfpathlineto{\pgfqpoint{3.293199in}{0.995041in}}%
\pgfpathlineto{\pgfqpoint{3.294569in}{1.721006in}}%
\pgfpathlineto{\pgfqpoint{3.298133in}{1.100231in}}%
\pgfpathlineto{\pgfqpoint{3.301002in}{1.722447in}}%
\pgfpathlineto{\pgfqpoint{3.304868in}{1.040918in}}%
\pgfpathlineto{\pgfqpoint{3.310142in}{1.816055in}}%
\pgfpathlineto{\pgfqpoint{3.311010in}{1.148409in}}%
\pgfpathlineto{\pgfqpoint{3.314799in}{1.738707in}}%
\pgfpathlineto{\pgfqpoint{3.317263in}{0.991245in}}%
\pgfpathlineto{\pgfqpoint{3.321412in}{1.752259in}}%
\pgfpathlineto{\pgfqpoint{3.324403in}{0.929654in}}%
\pgfpathlineto{\pgfqpoint{3.326989in}{1.705928in}}%
\pgfpathlineto{\pgfqpoint{3.330546in}{1.043806in}}%
\pgfpathlineto{\pgfqpoint{3.333305in}{1.746657in}}%
\pgfpathlineto{\pgfqpoint{3.337043in}{1.069229in}}%
\pgfpathlineto{\pgfqpoint{3.339661in}{1.690732in}}%
\pgfpathlineto{\pgfqpoint{3.342883in}{1.087626in}}%
\pgfpathlineto{\pgfqpoint{3.346556in}{1.785941in}}%
\pgfpathlineto{\pgfqpoint{3.349341in}{1.091107in}}%
\pgfpathlineto{\pgfqpoint{3.353909in}{1.719077in}}%
\pgfpathlineto{\pgfqpoint{3.356018in}{1.081927in}}%
\pgfpathlineto{\pgfqpoint{3.360347in}{1.734646in}}%
\pgfpathlineto{\pgfqpoint{3.363197in}{1.023517in}}%
\pgfpathlineto{\pgfqpoint{3.365622in}{1.710975in}}%
\pgfpathlineto{\pgfqpoint{3.368735in}{1.098905in}}%
\pgfpathlineto{\pgfqpoint{3.372408in}{1.707342in}}%
\pgfpathlineto{\pgfqpoint{3.375097in}{1.042990in}}%
\pgfpathlineto{\pgfqpoint{3.379098in}{1.739864in}}%
\pgfpathlineto{\pgfqpoint{3.381439in}{1.088801in}}%
\pgfpathlineto{\pgfqpoint{3.385157in}{1.780475in}}%
\pgfpathlineto{\pgfqpoint{3.388239in}{1.074176in}}%
\pgfpathlineto{\pgfqpoint{3.392748in}{1.722384in}}%
\pgfpathlineto{\pgfqpoint{3.394967in}{1.015812in}}%
\pgfpathlineto{\pgfqpoint{3.398164in}{1.731401in}}%
\pgfpathlineto{\pgfqpoint{3.400853in}{1.038693in}}%
\pgfpathlineto{\pgfqpoint{3.404230in}{1.642334in}}%
\pgfpathlineto{\pgfqpoint{3.409324in}{1.034631in}}%
\pgfpathlineto{\pgfqpoint{3.410566in}{1.751663in}}%
\pgfpathlineto{\pgfqpoint{3.415744in}{1.048743in}}%
\pgfpathlineto{\pgfqpoint{3.417242in}{1.680309in}}%
\pgfpathlineto{\pgfqpoint{3.421095in}{1.085528in}}%
\pgfpathlineto{\pgfqpoint{3.424087in}{1.813433in}}%
\pgfpathlineto{\pgfqpoint{3.426859in}{0.999460in}}%
\pgfpathlineto{\pgfqpoint{3.430802in}{1.786779in}}%
\pgfpathlineto{\pgfqpoint{3.433388in}{1.118567in}}%
\pgfpathlineto{\pgfqpoint{3.437415in}{1.727936in}}%
\pgfpathlineto{\pgfqpoint{3.441705in}{0.944675in}}%
\pgfpathlineto{\pgfqpoint{3.443030in}{1.801888in}}%
\pgfpathlineto{\pgfqpoint{3.446066in}{1.003633in}}%
\pgfpathlineto{\pgfqpoint{3.451875in}{1.841777in}}%
\pgfpathlineto{\pgfqpoint{3.452415in}{1.144302in}}%
\pgfpathlineto{\pgfqpoint{3.455715in}{1.618312in}}%
\pgfpathlineto{\pgfqpoint{3.459124in}{1.054276in}}%
\pgfpathlineto{\pgfqpoint{3.462469in}{1.702744in}}%
\pgfpathlineto{\pgfqpoint{3.468014in}{0.967476in}}%
\pgfpathlineto{\pgfqpoint{3.469249in}{1.707144in}}%
\pgfpathlineto{\pgfqpoint{3.472343in}{1.034022in}}%
\pgfpathlineto{\pgfqpoint{3.476389in}{1.702924in}}%
\pgfpathlineto{\pgfqpoint{3.478891in}{1.012826in}}%
\pgfpathlineto{\pgfqpoint{3.481638in}{1.716113in}}%
\pgfpathlineto{\pgfqpoint{3.485812in}{1.062577in}}%
\pgfpathlineto{\pgfqpoint{3.487884in}{1.630581in}}%
\pgfpathlineto{\pgfqpoint{3.492399in}{1.128687in}}%
\pgfpathlineto{\pgfqpoint{3.494548in}{1.715954in}}%
\pgfpathlineto{\pgfqpoint{3.497577in}{1.119990in}}%
\pgfpathlineto{\pgfqpoint{3.501630in}{1.698806in}}%
\pgfpathlineto{\pgfqpoint{3.504364in}{1.047757in}}%
\pgfpathlineto{\pgfqpoint{3.507361in}{1.705934in}}%
\pgfpathlineto{\pgfqpoint{3.511362in}{0.996455in}}%
\pgfpathlineto{\pgfqpoint{3.515672in}{1.756648in}}%
\pgfpathlineto{\pgfqpoint{3.517216in}{0.977767in}}%
\pgfpathlineto{\pgfqpoint{3.520419in}{1.748743in}}%
\pgfpathlineto{\pgfqpoint{3.523397in}{0.973939in}}%
\pgfpathlineto{\pgfqpoint{3.526742in}{1.632029in}}%
\pgfpathlineto{\pgfqpoint{3.529913in}{1.024750in}}%
\pgfpathlineto{\pgfqpoint{3.533554in}{1.722254in}}%
\pgfpathlineto{\pgfqpoint{3.536359in}{0.909104in}}%
\pgfpathlineto{\pgfqpoint{3.539652in}{1.717383in}}%
\pgfpathlineto{\pgfqpoint{3.544599in}{0.992108in}}%
\pgfpathlineto{\pgfqpoint{3.546593in}{1.737043in}}%
\pgfpathlineto{\pgfqpoint{3.549635in}{1.044420in}}%
\pgfpathlineto{\pgfqpoint{3.552453in}{1.677772in}}%
\pgfpathlineto{\pgfqpoint{3.556261in}{0.998467in}}%
\pgfpathlineto{\pgfqpoint{3.558801in}{1.685640in}}%
\pgfpathlineto{\pgfqpoint{3.562609in}{1.055108in}}%
\pgfpathlineto{\pgfqpoint{3.565626in}{1.673295in}}%
\pgfpathlineto{\pgfqpoint{3.568759in}{1.127016in}}%
\pgfpathlineto{\pgfqpoint{3.571583in}{1.686821in}}%
\pgfpathlineto{\pgfqpoint{3.575204in}{1.073902in}}%
\pgfpathlineto{\pgfqpoint{3.578890in}{1.702938in}}%
\pgfpathlineto{\pgfqpoint{3.581251in}{0.994069in}}%
\pgfpathlineto{\pgfqpoint{3.585657in}{1.842380in}}%
\pgfpathlineto{\pgfqpoint{3.588153in}{1.104301in}}%
\pgfpathlineto{\pgfqpoint{3.591491in}{1.750942in}}%
\pgfpathlineto{\pgfqpoint{3.594418in}{1.112896in}}%
\pgfpathlineto{\pgfqpoint{3.597711in}{1.701869in}}%
\pgfpathlineto{\pgfqpoint{3.600760in}{1.003573in}}%
\pgfpathlineto{\pgfqpoint{3.604459in}{1.704761in}}%
\pgfpathlineto{\pgfqpoint{3.607939in}{1.059055in}}%
\pgfpathlineto{\pgfqpoint{3.612036in}{1.759855in}}%
\pgfpathlineto{\pgfqpoint{3.613715in}{1.124068in}}%
\pgfpathlineto{\pgfqpoint{3.618179in}{1.699424in}}%
\pgfpathlineto{\pgfqpoint{3.620495in}{1.115241in}}%
\pgfpathlineto{\pgfqpoint{3.623435in}{1.727609in}}%
\pgfpathlineto{\pgfqpoint{3.626529in}{1.166179in}}%
\pgfpathlineto{\pgfqpoint{3.630388in}{1.692283in}}%
\pgfpathlineto{\pgfqpoint{3.633173in}{1.065661in}}%
\pgfpathlineto{\pgfqpoint{3.636641in}{1.757566in}}%
\pgfpathlineto{\pgfqpoint{3.639259in}{1.075380in}}%
\pgfpathlineto{\pgfqpoint{3.642713in}{1.699389in}}%
\pgfpathlineto{\pgfqpoint{3.646334in}{1.071149in}}%
\pgfpathlineto{\pgfqpoint{3.648959in}{1.680073in}}%
\pgfpathlineto{\pgfqpoint{3.654182in}{1.022953in}}%
\pgfpathlineto{\pgfqpoint{3.656131in}{1.734149in}}%
\pgfpathlineto{\pgfqpoint{3.659862in}{0.999664in}}%
\pgfpathlineto{\pgfqpoint{3.662261in}{1.766728in}}%
\pgfpathlineto{\pgfqpoint{3.665690in}{1.041139in}}%
\pgfpathlineto{\pgfqpoint{3.668513in}{1.640082in}}%
\pgfpathlineto{\pgfqpoint{3.672315in}{1.035534in}}%
\pgfpathlineto{\pgfqpoint{3.674817in}{1.705390in}}%
\pgfpathlineto{\pgfqpoint{3.679153in}{0.995280in}}%
\pgfpathlineto{\pgfqpoint{3.681230in}{1.665942in}}%
\pgfpathlineto{\pgfqpoint{3.684504in}{1.067573in}}%
\pgfpathlineto{\pgfqpoint{3.688525in}{1.713221in}}%
\pgfpathlineto{\pgfqpoint{3.691239in}{0.980591in}}%
\pgfpathlineto{\pgfqpoint{3.694070in}{1.684161in}}%
\pgfpathlineto{\pgfqpoint{3.698186in}{1.063184in}}%
\pgfpathlineto{\pgfqpoint{3.700869in}{1.773994in}}%
\pgfpathlineto{\pgfqpoint{3.704162in}{1.073184in}}%
\pgfpathlineto{\pgfqpoint{3.708716in}{1.721017in}}%
\pgfpathlineto{\pgfqpoint{3.710672in}{1.010154in}}%
\pgfpathlineto{\pgfqpoint{3.715046in}{1.727325in}}%
\pgfpathlineto{\pgfqpoint{3.716622in}{1.122434in}}%
\pgfpathlineto{\pgfqpoint{3.720307in}{1.708249in}}%
\pgfpathlineto{\pgfqpoint{3.724598in}{1.008650in}}%
\pgfpathlineto{\pgfqpoint{3.726734in}{1.734248in}}%
\pgfpathlineto{\pgfqpoint{3.729776in}{1.019961in}}%
\pgfpathlineto{\pgfqpoint{3.733314in}{1.771625in}}%
\pgfpathlineto{\pgfqpoint{3.736601in}{1.031083in}}%
\pgfpathlineto{\pgfqpoint{3.739354in}{1.680053in}}%
\pgfpathlineto{\pgfqpoint{3.744075in}{1.006185in}}%
\pgfpathlineto{\pgfqpoint{3.745922in}{1.681025in}}%
\pgfpathlineto{\pgfqpoint{3.749099in}{1.072105in}}%
\pgfpathlineto{\pgfqpoint{3.754219in}{1.765227in}}%
\pgfpathlineto{\pgfqpoint{3.757140in}{0.980769in}}%
\pgfpathlineto{\pgfqpoint{3.759037in}{1.659036in}}%
\pgfpathlineto{\pgfqpoint{3.762273in}{1.101355in}}%
\pgfpathlineto{\pgfqpoint{3.765991in}{1.758950in}}%
\pgfpathlineto{\pgfqpoint{3.770474in}{0.974925in}}%
\pgfpathlineto{\pgfqpoint{3.771703in}{1.691693in}}%
\pgfpathlineto{\pgfqpoint{3.774983in}{1.060253in}}%
\pgfpathlineto{\pgfqpoint{3.778187in}{1.703440in}}%
\pgfpathlineto{\pgfqpoint{3.781827in}{1.093088in}}%
\pgfpathlineto{\pgfqpoint{3.784381in}{1.673610in}}%
\pgfpathlineto{\pgfqpoint{3.787777in}{1.058421in}}%
\pgfpathlineto{\pgfqpoint{3.790743in}{1.801085in}}%
\pgfpathlineto{\pgfqpoint{3.794956in}{1.038766in}}%
\pgfpathlineto{\pgfqpoint{3.798031in}{1.771641in}}%
\pgfpathlineto{\pgfqpoint{3.801749in}{1.012056in}}%
\pgfpathlineto{\pgfqpoint{3.804592in}{1.664721in}}%
\pgfpathlineto{\pgfqpoint{3.808329in}{0.954320in}}%
\pgfpathlineto{\pgfqpoint{3.810883in}{1.866207in}}%
\pgfpathlineto{\pgfqpoint{3.813880in}{1.052700in}}%
\pgfpathlineto{\pgfqpoint{3.816717in}{1.712802in}}%
\pgfpathlineto{\pgfqpoint{3.819978in}{1.075895in}}%
\pgfpathlineto{\pgfqpoint{3.824777in}{1.813769in}}%
\pgfpathlineto{\pgfqpoint{3.826231in}{0.961081in}}%
\pgfpathlineto{\pgfqpoint{3.829415in}{1.608505in}}%
\pgfpathlineto{\pgfqpoint{3.833892in}{0.952199in}}%
\pgfpathlineto{\pgfqpoint{3.835937in}{1.602145in}}%
\pgfpathlineto{\pgfqpoint{3.840581in}{0.996266in}}%
\pgfpathlineto{\pgfqpoint{3.842447in}{1.678111in}}%
\pgfpathlineto{\pgfqpoint{3.845901in}{0.987191in}}%
\pgfpathlineto{\pgfqpoint{3.848719in}{1.384537in}}%
\pgfpathlineto{\pgfqpoint{3.848719in}{1.384537in}}%
\pgfusepath{stroke}%
\end{pgfscope}%
\begin{pgfscope}%
\pgfsetrectcap%
\pgfsetmiterjoin%
\pgfsetlinewidth{0.803000pt}%
\definecolor{currentstroke}{rgb}{0.000000,0.000000,0.000000}%
\pgfsetstrokecolor{currentstroke}%
\pgfsetdash{}{0pt}%
\pgfpathmoveto{\pgfqpoint{0.471687in}{0.416447in}}%
\pgfpathlineto{\pgfqpoint{0.471687in}{2.341095in}}%
\pgfusepath{stroke}%
\end{pgfscope}%
\begin{pgfscope}%
\pgfsetrectcap%
\pgfsetmiterjoin%
\pgfsetlinewidth{0.803000pt}%
\definecolor{currentstroke}{rgb}{0.000000,0.000000,0.000000}%
\pgfsetstrokecolor{currentstroke}%
\pgfsetdash{}{0pt}%
\pgfpathmoveto{\pgfqpoint{4.009530in}{0.416447in}}%
\pgfpathlineto{\pgfqpoint{4.009530in}{2.341095in}}%
\pgfusepath{stroke}%
\end{pgfscope}%
\begin{pgfscope}%
\pgfsetrectcap%
\pgfsetmiterjoin%
\pgfsetlinewidth{0.803000pt}%
\definecolor{currentstroke}{rgb}{0.000000,0.000000,0.000000}%
\pgfsetstrokecolor{currentstroke}%
\pgfsetdash{}{0pt}%
\pgfpathmoveto{\pgfqpoint{0.471687in}{0.416447in}}%
\pgfpathlineto{\pgfqpoint{4.009530in}{0.416447in}}%
\pgfusepath{stroke}%
\end{pgfscope}%
\begin{pgfscope}%
\pgfsetrectcap%
\pgfsetmiterjoin%
\pgfsetlinewidth{0.803000pt}%
\definecolor{currentstroke}{rgb}{0.000000,0.000000,0.000000}%
\pgfsetstrokecolor{currentstroke}%
\pgfsetdash{}{0pt}%
\pgfpathmoveto{\pgfqpoint{0.471687in}{2.341095in}}%
\pgfpathlineto{\pgfqpoint{4.009530in}{2.341095in}}%
\pgfusepath{stroke}%
\end{pgfscope}%
\end{pgfpicture}%
\makeatother%
\endgroup%
% data/simulations/sim_autozero.py
    \caption{Simulated measurement with autozeroing applied.}
    \label{fig:autozero_time}
\end{figure}

When comparing figure \ref{fig:autozero_time} to figure \ref{fig:autozero_raw_time} on page \pageref{fig:autozero_raw_time} it is immediately evident that the $f^{-1}$ flicker noise component is no longer present. The difference in white noise strength is difficult to compare and it must be turned to the power spectral density again. When calculating the spectral density it is important to remember that the sampling rate is now halved because the odd samples were subtracted. The result is shown in figure \ref{fig:autozero_psd} along with dashed lines showing the noise content prior to applying the autozero algorithm as was done in figure \ref{fig:autozero_raw_psd} on page \pageref{fig:autozero_raw_psd}.

The power spectral density in figure \ref{fig:autozero_psd} confirms an increase in the white noise power as discussed above and using the graph it can be worked out that the white noise power $\sqrt{h_{-1}}$ has increased from \qty[power-half-as-sqrt, per-mode=symbol]{165}{\nV \Hz\tothe{-0.5}} to \qty[power-half-as-sqrt, per-mode=symbol]{489}{\nV \Hz\tothe{-0.5}}, an increase by a factor of $\sqrt{8.8}$, which is more than estimated from equation \ref{eqn:autozeroing}. Including the factor of $2$ introduced by the decimation, the increase of $\sqrt{h_{-1}}$ was gauged to be by a factor of $\sqrt{4}$ . The cause of the additional noise was already mentioned above. There is still some substantial $f^{-1}$ noise present at the autozero frequency of \qty{5}{\Hz}. This type of noise is not uncorrelated and therefore the covariance is not zero, hence equation \ref{eqn:adding_white_noise} does not strictly hold and additional correlated noise is leaking into the result. This hypothesis can be confirmed by increasing the sampling frequency by an order magnitude. Doing this, the white noise floor of the autozero measurement now only increases by a factor of $\sqrt{4.5}$, which is close to the expected factor of $\sqrt{4}$. This means that the autozeroing frequency should be chosen to be at least a decade above the noise corner frequency to be most effective.
\begin{figure}[ht]
    \centering
    %% Creator: Matplotlib, PGF backend
%%
%% To include the figure in your LaTeX document, write
%%   \input{<filename>.pgf}
%%
%% Make sure the required packages are loaded in your preamble
%%   \usepackage{pgf}
%%
%% Also ensure that all the required font packages are loaded; for instance,
%% the lmodern package is sometimes necessary when using math font.
%%   \usepackage{lmodern}
%%
%% Figures using additional raster images can only be included by \input if
%% they are in the same directory as the main LaTeX file. For loading figures
%% from other directories you can use the `import` package
%%   \usepackage{import}
%%
%% and then include the figures with
%%   \import{<path to file>}{<filename>.pgf}
%%
%% Matplotlib used the following preamble
%%   \usepackage{siunitx}
%%   \usepackage{fontspec}
%%   \makeatletter\@ifpackageloaded{underscore}{}{\usepackage[strings]{underscore}}\makeatother
%%
\begingroup%
\makeatletter%
\begin{pgfpicture}%
\pgfpathrectangle{\pgfpointorigin}{\pgfqpoint{4.060000in}{2.510000in}}%
\pgfusepath{use as bounding box, clip}%
\begin{pgfscope}%
\pgfsetbuttcap%
\pgfsetmiterjoin%
\definecolor{currentfill}{rgb}{1.000000,1.000000,1.000000}%
\pgfsetfillcolor{currentfill}%
\pgfsetlinewidth{0.000000pt}%
\definecolor{currentstroke}{rgb}{1.000000,1.000000,1.000000}%
\pgfsetstrokecolor{currentstroke}%
\pgfsetdash{}{0pt}%
\pgfpathmoveto{\pgfqpoint{0.000000in}{0.000000in}}%
\pgfpathlineto{\pgfqpoint{4.060000in}{0.000000in}}%
\pgfpathlineto{\pgfqpoint{4.060000in}{2.510000in}}%
\pgfpathlineto{\pgfqpoint{0.000000in}{2.510000in}}%
\pgfpathlineto{\pgfqpoint{0.000000in}{0.000000in}}%
\pgfpathclose%
\pgfusepath{fill}%
\end{pgfscope}%
\begin{pgfscope}%
\pgfsetbuttcap%
\pgfsetmiterjoin%
\definecolor{currentfill}{rgb}{1.000000,1.000000,1.000000}%
\pgfsetfillcolor{currentfill}%
\pgfsetlinewidth{0.000000pt}%
\definecolor{currentstroke}{rgb}{0.000000,0.000000,0.000000}%
\pgfsetstrokecolor{currentstroke}%
\pgfsetstrokeopacity{0.000000}%
\pgfsetdash{}{0pt}%
\pgfpathmoveto{\pgfqpoint{0.664463in}{0.417642in}}%
\pgfpathlineto{\pgfqpoint{4.018330in}{0.417642in}}%
\pgfpathlineto{\pgfqpoint{4.018330in}{2.468330in}}%
\pgfpathlineto{\pgfqpoint{0.664463in}{2.468330in}}%
\pgfpathlineto{\pgfqpoint{0.664463in}{0.417642in}}%
\pgfpathclose%
\pgfusepath{fill}%
\end{pgfscope}%
\begin{pgfscope}%
\pgfpathrectangle{\pgfqpoint{0.664463in}{0.417642in}}{\pgfqpoint{3.353867in}{2.050688in}}%
\pgfusepath{clip}%
\pgfsetrectcap%
\pgfsetroundjoin%
\pgfsetlinewidth{0.803000pt}%
\definecolor{currentstroke}{rgb}{0.450000,0.450000,0.450000}%
\pgfsetstrokecolor{currentstroke}%
\pgfsetdash{}{0pt}%
\pgfpathmoveto{\pgfqpoint{0.808808in}{0.417642in}}%
\pgfpathlineto{\pgfqpoint{0.808808in}{2.468330in}}%
\pgfusepath{stroke}%
\end{pgfscope}%
\begin{pgfscope}%
\pgfsetbuttcap%
\pgfsetroundjoin%
\definecolor{currentfill}{rgb}{0.000000,0.000000,0.000000}%
\pgfsetfillcolor{currentfill}%
\pgfsetlinewidth{0.803000pt}%
\definecolor{currentstroke}{rgb}{0.000000,0.000000,0.000000}%
\pgfsetstrokecolor{currentstroke}%
\pgfsetdash{}{0pt}%
\pgfsys@defobject{currentmarker}{\pgfqpoint{0.000000in}{-0.048611in}}{\pgfqpoint{0.000000in}{0.000000in}}{%
\pgfpathmoveto{\pgfqpoint{0.000000in}{0.000000in}}%
\pgfpathlineto{\pgfqpoint{0.000000in}{-0.048611in}}%
\pgfusepath{stroke,fill}%
}%
\begin{pgfscope}%
\pgfsys@transformshift{0.808808in}{0.417642in}%
\pgfsys@useobject{currentmarker}{}%
\end{pgfscope}%
\end{pgfscope}%
\begin{pgfscope}%
\definecolor{textcolor}{rgb}{0.000000,0.000000,0.000000}%
\pgfsetstrokecolor{textcolor}%
\pgfsetfillcolor{textcolor}%
\pgftext[x=0.808808in,y=0.320420in,,top]{\color{textcolor}\rmfamily\fontsize{8.000000}{9.600000}\selectfont \(\displaystyle {10^{-2}}\)}%
\end{pgfscope}%
\begin{pgfscope}%
\pgfpathrectangle{\pgfqpoint{0.664463in}{0.417642in}}{\pgfqpoint{3.353867in}{2.050688in}}%
\pgfusepath{clip}%
\pgfsetrectcap%
\pgfsetroundjoin%
\pgfsetlinewidth{0.803000pt}%
\definecolor{currentstroke}{rgb}{0.450000,0.450000,0.450000}%
\pgfsetstrokecolor{currentstroke}%
\pgfsetdash{}{0pt}%
\pgfpathmoveto{\pgfqpoint{2.083937in}{0.417642in}}%
\pgfpathlineto{\pgfqpoint{2.083937in}{2.468330in}}%
\pgfusepath{stroke}%
\end{pgfscope}%
\begin{pgfscope}%
\pgfsetbuttcap%
\pgfsetroundjoin%
\definecolor{currentfill}{rgb}{0.000000,0.000000,0.000000}%
\pgfsetfillcolor{currentfill}%
\pgfsetlinewidth{0.803000pt}%
\definecolor{currentstroke}{rgb}{0.000000,0.000000,0.000000}%
\pgfsetstrokecolor{currentstroke}%
\pgfsetdash{}{0pt}%
\pgfsys@defobject{currentmarker}{\pgfqpoint{0.000000in}{-0.048611in}}{\pgfqpoint{0.000000in}{0.000000in}}{%
\pgfpathmoveto{\pgfqpoint{0.000000in}{0.000000in}}%
\pgfpathlineto{\pgfqpoint{0.000000in}{-0.048611in}}%
\pgfusepath{stroke,fill}%
}%
\begin{pgfscope}%
\pgfsys@transformshift{2.083937in}{0.417642in}%
\pgfsys@useobject{currentmarker}{}%
\end{pgfscope}%
\end{pgfscope}%
\begin{pgfscope}%
\definecolor{textcolor}{rgb}{0.000000,0.000000,0.000000}%
\pgfsetstrokecolor{textcolor}%
\pgfsetfillcolor{textcolor}%
\pgftext[x=2.083937in,y=0.320420in,,top]{\color{textcolor}\rmfamily\fontsize{8.000000}{9.600000}\selectfont \(\displaystyle {10^{-1}}\)}%
\end{pgfscope}%
\begin{pgfscope}%
\pgfpathrectangle{\pgfqpoint{0.664463in}{0.417642in}}{\pgfqpoint{3.353867in}{2.050688in}}%
\pgfusepath{clip}%
\pgfsetrectcap%
\pgfsetroundjoin%
\pgfsetlinewidth{0.803000pt}%
\definecolor{currentstroke}{rgb}{0.450000,0.450000,0.450000}%
\pgfsetstrokecolor{currentstroke}%
\pgfsetdash{}{0pt}%
\pgfpathmoveto{\pgfqpoint{3.359065in}{0.417642in}}%
\pgfpathlineto{\pgfqpoint{3.359065in}{2.468330in}}%
\pgfusepath{stroke}%
\end{pgfscope}%
\begin{pgfscope}%
\pgfsetbuttcap%
\pgfsetroundjoin%
\definecolor{currentfill}{rgb}{0.000000,0.000000,0.000000}%
\pgfsetfillcolor{currentfill}%
\pgfsetlinewidth{0.803000pt}%
\definecolor{currentstroke}{rgb}{0.000000,0.000000,0.000000}%
\pgfsetstrokecolor{currentstroke}%
\pgfsetdash{}{0pt}%
\pgfsys@defobject{currentmarker}{\pgfqpoint{0.000000in}{-0.048611in}}{\pgfqpoint{0.000000in}{0.000000in}}{%
\pgfpathmoveto{\pgfqpoint{0.000000in}{0.000000in}}%
\pgfpathlineto{\pgfqpoint{0.000000in}{-0.048611in}}%
\pgfusepath{stroke,fill}%
}%
\begin{pgfscope}%
\pgfsys@transformshift{3.359065in}{0.417642in}%
\pgfsys@useobject{currentmarker}{}%
\end{pgfscope}%
\end{pgfscope}%
\begin{pgfscope}%
\definecolor{textcolor}{rgb}{0.000000,0.000000,0.000000}%
\pgfsetstrokecolor{textcolor}%
\pgfsetfillcolor{textcolor}%
\pgftext[x=3.359065in,y=0.320420in,,top]{\color{textcolor}\rmfamily\fontsize{8.000000}{9.600000}\selectfont \(\displaystyle {10^{0}}\)}%
\end{pgfscope}%
\begin{pgfscope}%
\pgfpathrectangle{\pgfqpoint{0.664463in}{0.417642in}}{\pgfqpoint{3.353867in}{2.050688in}}%
\pgfusepath{clip}%
\pgfsetrectcap%
\pgfsetroundjoin%
\pgfsetlinewidth{0.803000pt}%
\definecolor{currentstroke}{rgb}{0.850000,0.850000,0.850000}%
\pgfsetstrokecolor{currentstroke}%
\pgfsetdash{}{0pt}%
\pgfpathmoveto{\pgfqpoint{0.685235in}{0.417642in}}%
\pgfpathlineto{\pgfqpoint{0.685235in}{2.468330in}}%
\pgfusepath{stroke}%
\end{pgfscope}%
\begin{pgfscope}%
\pgfsetbuttcap%
\pgfsetroundjoin%
\definecolor{currentfill}{rgb}{0.000000,0.000000,0.000000}%
\pgfsetfillcolor{currentfill}%
\pgfsetlinewidth{0.602250pt}%
\definecolor{currentstroke}{rgb}{0.000000,0.000000,0.000000}%
\pgfsetstrokecolor{currentstroke}%
\pgfsetdash{}{0pt}%
\pgfsys@defobject{currentmarker}{\pgfqpoint{0.000000in}{-0.027778in}}{\pgfqpoint{0.000000in}{0.000000in}}{%
\pgfpathmoveto{\pgfqpoint{0.000000in}{0.000000in}}%
\pgfpathlineto{\pgfqpoint{0.000000in}{-0.027778in}}%
\pgfusepath{stroke,fill}%
}%
\begin{pgfscope}%
\pgfsys@transformshift{0.685235in}{0.417642in}%
\pgfsys@useobject{currentmarker}{}%
\end{pgfscope}%
\end{pgfscope}%
\begin{pgfscope}%
\pgfpathrectangle{\pgfqpoint{0.664463in}{0.417642in}}{\pgfqpoint{3.353867in}{2.050688in}}%
\pgfusepath{clip}%
\pgfsetrectcap%
\pgfsetroundjoin%
\pgfsetlinewidth{0.803000pt}%
\definecolor{currentstroke}{rgb}{0.850000,0.850000,0.850000}%
\pgfsetstrokecolor{currentstroke}%
\pgfsetdash{}{0pt}%
\pgfpathmoveto{\pgfqpoint{0.750461in}{0.417642in}}%
\pgfpathlineto{\pgfqpoint{0.750461in}{2.468330in}}%
\pgfusepath{stroke}%
\end{pgfscope}%
\begin{pgfscope}%
\pgfsetbuttcap%
\pgfsetroundjoin%
\definecolor{currentfill}{rgb}{0.000000,0.000000,0.000000}%
\pgfsetfillcolor{currentfill}%
\pgfsetlinewidth{0.602250pt}%
\definecolor{currentstroke}{rgb}{0.000000,0.000000,0.000000}%
\pgfsetstrokecolor{currentstroke}%
\pgfsetdash{}{0pt}%
\pgfsys@defobject{currentmarker}{\pgfqpoint{0.000000in}{-0.027778in}}{\pgfqpoint{0.000000in}{0.000000in}}{%
\pgfpathmoveto{\pgfqpoint{0.000000in}{0.000000in}}%
\pgfpathlineto{\pgfqpoint{0.000000in}{-0.027778in}}%
\pgfusepath{stroke,fill}%
}%
\begin{pgfscope}%
\pgfsys@transformshift{0.750461in}{0.417642in}%
\pgfsys@useobject{currentmarker}{}%
\end{pgfscope}%
\end{pgfscope}%
\begin{pgfscope}%
\pgfpathrectangle{\pgfqpoint{0.664463in}{0.417642in}}{\pgfqpoint{3.353867in}{2.050688in}}%
\pgfusepath{clip}%
\pgfsetrectcap%
\pgfsetroundjoin%
\pgfsetlinewidth{0.803000pt}%
\definecolor{currentstroke}{rgb}{0.850000,0.850000,0.850000}%
\pgfsetstrokecolor{currentstroke}%
\pgfsetdash{}{0pt}%
\pgfpathmoveto{\pgfqpoint{1.192660in}{0.417642in}}%
\pgfpathlineto{\pgfqpoint{1.192660in}{2.468330in}}%
\pgfusepath{stroke}%
\end{pgfscope}%
\begin{pgfscope}%
\pgfsetbuttcap%
\pgfsetroundjoin%
\definecolor{currentfill}{rgb}{0.000000,0.000000,0.000000}%
\pgfsetfillcolor{currentfill}%
\pgfsetlinewidth{0.602250pt}%
\definecolor{currentstroke}{rgb}{0.000000,0.000000,0.000000}%
\pgfsetstrokecolor{currentstroke}%
\pgfsetdash{}{0pt}%
\pgfsys@defobject{currentmarker}{\pgfqpoint{0.000000in}{-0.027778in}}{\pgfqpoint{0.000000in}{0.000000in}}{%
\pgfpathmoveto{\pgfqpoint{0.000000in}{0.000000in}}%
\pgfpathlineto{\pgfqpoint{0.000000in}{-0.027778in}}%
\pgfusepath{stroke,fill}%
}%
\begin{pgfscope}%
\pgfsys@transformshift{1.192660in}{0.417642in}%
\pgfsys@useobject{currentmarker}{}%
\end{pgfscope}%
\end{pgfscope}%
\begin{pgfscope}%
\pgfpathrectangle{\pgfqpoint{0.664463in}{0.417642in}}{\pgfqpoint{3.353867in}{2.050688in}}%
\pgfusepath{clip}%
\pgfsetrectcap%
\pgfsetroundjoin%
\pgfsetlinewidth{0.803000pt}%
\definecolor{currentstroke}{rgb}{0.850000,0.850000,0.850000}%
\pgfsetstrokecolor{currentstroke}%
\pgfsetdash{}{0pt}%
\pgfpathmoveto{\pgfqpoint{1.417199in}{0.417642in}}%
\pgfpathlineto{\pgfqpoint{1.417199in}{2.468330in}}%
\pgfusepath{stroke}%
\end{pgfscope}%
\begin{pgfscope}%
\pgfsetbuttcap%
\pgfsetroundjoin%
\definecolor{currentfill}{rgb}{0.000000,0.000000,0.000000}%
\pgfsetfillcolor{currentfill}%
\pgfsetlinewidth{0.602250pt}%
\definecolor{currentstroke}{rgb}{0.000000,0.000000,0.000000}%
\pgfsetstrokecolor{currentstroke}%
\pgfsetdash{}{0pt}%
\pgfsys@defobject{currentmarker}{\pgfqpoint{0.000000in}{-0.027778in}}{\pgfqpoint{0.000000in}{0.000000in}}{%
\pgfpathmoveto{\pgfqpoint{0.000000in}{0.000000in}}%
\pgfpathlineto{\pgfqpoint{0.000000in}{-0.027778in}}%
\pgfusepath{stroke,fill}%
}%
\begin{pgfscope}%
\pgfsys@transformshift{1.417199in}{0.417642in}%
\pgfsys@useobject{currentmarker}{}%
\end{pgfscope}%
\end{pgfscope}%
\begin{pgfscope}%
\pgfpathrectangle{\pgfqpoint{0.664463in}{0.417642in}}{\pgfqpoint{3.353867in}{2.050688in}}%
\pgfusepath{clip}%
\pgfsetrectcap%
\pgfsetroundjoin%
\pgfsetlinewidth{0.803000pt}%
\definecolor{currentstroke}{rgb}{0.850000,0.850000,0.850000}%
\pgfsetstrokecolor{currentstroke}%
\pgfsetdash{}{0pt}%
\pgfpathmoveto{\pgfqpoint{1.576512in}{0.417642in}}%
\pgfpathlineto{\pgfqpoint{1.576512in}{2.468330in}}%
\pgfusepath{stroke}%
\end{pgfscope}%
\begin{pgfscope}%
\pgfsetbuttcap%
\pgfsetroundjoin%
\definecolor{currentfill}{rgb}{0.000000,0.000000,0.000000}%
\pgfsetfillcolor{currentfill}%
\pgfsetlinewidth{0.602250pt}%
\definecolor{currentstroke}{rgb}{0.000000,0.000000,0.000000}%
\pgfsetstrokecolor{currentstroke}%
\pgfsetdash{}{0pt}%
\pgfsys@defobject{currentmarker}{\pgfqpoint{0.000000in}{-0.027778in}}{\pgfqpoint{0.000000in}{0.000000in}}{%
\pgfpathmoveto{\pgfqpoint{0.000000in}{0.000000in}}%
\pgfpathlineto{\pgfqpoint{0.000000in}{-0.027778in}}%
\pgfusepath{stroke,fill}%
}%
\begin{pgfscope}%
\pgfsys@transformshift{1.576512in}{0.417642in}%
\pgfsys@useobject{currentmarker}{}%
\end{pgfscope}%
\end{pgfscope}%
\begin{pgfscope}%
\pgfpathrectangle{\pgfqpoint{0.664463in}{0.417642in}}{\pgfqpoint{3.353867in}{2.050688in}}%
\pgfusepath{clip}%
\pgfsetrectcap%
\pgfsetroundjoin%
\pgfsetlinewidth{0.803000pt}%
\definecolor{currentstroke}{rgb}{0.850000,0.850000,0.850000}%
\pgfsetstrokecolor{currentstroke}%
\pgfsetdash{}{0pt}%
\pgfpathmoveto{\pgfqpoint{1.700084in}{0.417642in}}%
\pgfpathlineto{\pgfqpoint{1.700084in}{2.468330in}}%
\pgfusepath{stroke}%
\end{pgfscope}%
\begin{pgfscope}%
\pgfsetbuttcap%
\pgfsetroundjoin%
\definecolor{currentfill}{rgb}{0.000000,0.000000,0.000000}%
\pgfsetfillcolor{currentfill}%
\pgfsetlinewidth{0.602250pt}%
\definecolor{currentstroke}{rgb}{0.000000,0.000000,0.000000}%
\pgfsetstrokecolor{currentstroke}%
\pgfsetdash{}{0pt}%
\pgfsys@defobject{currentmarker}{\pgfqpoint{0.000000in}{-0.027778in}}{\pgfqpoint{0.000000in}{0.000000in}}{%
\pgfpathmoveto{\pgfqpoint{0.000000in}{0.000000in}}%
\pgfpathlineto{\pgfqpoint{0.000000in}{-0.027778in}}%
\pgfusepath{stroke,fill}%
}%
\begin{pgfscope}%
\pgfsys@transformshift{1.700084in}{0.417642in}%
\pgfsys@useobject{currentmarker}{}%
\end{pgfscope}%
\end{pgfscope}%
\begin{pgfscope}%
\pgfpathrectangle{\pgfqpoint{0.664463in}{0.417642in}}{\pgfqpoint{3.353867in}{2.050688in}}%
\pgfusepath{clip}%
\pgfsetrectcap%
\pgfsetroundjoin%
\pgfsetlinewidth{0.803000pt}%
\definecolor{currentstroke}{rgb}{0.850000,0.850000,0.850000}%
\pgfsetstrokecolor{currentstroke}%
\pgfsetdash{}{0pt}%
\pgfpathmoveto{\pgfqpoint{1.801051in}{0.417642in}}%
\pgfpathlineto{\pgfqpoint{1.801051in}{2.468330in}}%
\pgfusepath{stroke}%
\end{pgfscope}%
\begin{pgfscope}%
\pgfsetbuttcap%
\pgfsetroundjoin%
\definecolor{currentfill}{rgb}{0.000000,0.000000,0.000000}%
\pgfsetfillcolor{currentfill}%
\pgfsetlinewidth{0.602250pt}%
\definecolor{currentstroke}{rgb}{0.000000,0.000000,0.000000}%
\pgfsetstrokecolor{currentstroke}%
\pgfsetdash{}{0pt}%
\pgfsys@defobject{currentmarker}{\pgfqpoint{0.000000in}{-0.027778in}}{\pgfqpoint{0.000000in}{0.000000in}}{%
\pgfpathmoveto{\pgfqpoint{0.000000in}{0.000000in}}%
\pgfpathlineto{\pgfqpoint{0.000000in}{-0.027778in}}%
\pgfusepath{stroke,fill}%
}%
\begin{pgfscope}%
\pgfsys@transformshift{1.801051in}{0.417642in}%
\pgfsys@useobject{currentmarker}{}%
\end{pgfscope}%
\end{pgfscope}%
\begin{pgfscope}%
\pgfpathrectangle{\pgfqpoint{0.664463in}{0.417642in}}{\pgfqpoint{3.353867in}{2.050688in}}%
\pgfusepath{clip}%
\pgfsetrectcap%
\pgfsetroundjoin%
\pgfsetlinewidth{0.803000pt}%
\definecolor{currentstroke}{rgb}{0.850000,0.850000,0.850000}%
\pgfsetstrokecolor{currentstroke}%
\pgfsetdash{}{0pt}%
\pgfpathmoveto{\pgfqpoint{1.886417in}{0.417642in}}%
\pgfpathlineto{\pgfqpoint{1.886417in}{2.468330in}}%
\pgfusepath{stroke}%
\end{pgfscope}%
\begin{pgfscope}%
\pgfsetbuttcap%
\pgfsetroundjoin%
\definecolor{currentfill}{rgb}{0.000000,0.000000,0.000000}%
\pgfsetfillcolor{currentfill}%
\pgfsetlinewidth{0.602250pt}%
\definecolor{currentstroke}{rgb}{0.000000,0.000000,0.000000}%
\pgfsetstrokecolor{currentstroke}%
\pgfsetdash{}{0pt}%
\pgfsys@defobject{currentmarker}{\pgfqpoint{0.000000in}{-0.027778in}}{\pgfqpoint{0.000000in}{0.000000in}}{%
\pgfpathmoveto{\pgfqpoint{0.000000in}{0.000000in}}%
\pgfpathlineto{\pgfqpoint{0.000000in}{-0.027778in}}%
\pgfusepath{stroke,fill}%
}%
\begin{pgfscope}%
\pgfsys@transformshift{1.886417in}{0.417642in}%
\pgfsys@useobject{currentmarker}{}%
\end{pgfscope}%
\end{pgfscope}%
\begin{pgfscope}%
\pgfpathrectangle{\pgfqpoint{0.664463in}{0.417642in}}{\pgfqpoint{3.353867in}{2.050688in}}%
\pgfusepath{clip}%
\pgfsetrectcap%
\pgfsetroundjoin%
\pgfsetlinewidth{0.803000pt}%
\definecolor{currentstroke}{rgb}{0.850000,0.850000,0.850000}%
\pgfsetstrokecolor{currentstroke}%
\pgfsetdash{}{0pt}%
\pgfpathmoveto{\pgfqpoint{1.960364in}{0.417642in}}%
\pgfpathlineto{\pgfqpoint{1.960364in}{2.468330in}}%
\pgfusepath{stroke}%
\end{pgfscope}%
\begin{pgfscope}%
\pgfsetbuttcap%
\pgfsetroundjoin%
\definecolor{currentfill}{rgb}{0.000000,0.000000,0.000000}%
\pgfsetfillcolor{currentfill}%
\pgfsetlinewidth{0.602250pt}%
\definecolor{currentstroke}{rgb}{0.000000,0.000000,0.000000}%
\pgfsetstrokecolor{currentstroke}%
\pgfsetdash{}{0pt}%
\pgfsys@defobject{currentmarker}{\pgfqpoint{0.000000in}{-0.027778in}}{\pgfqpoint{0.000000in}{0.000000in}}{%
\pgfpathmoveto{\pgfqpoint{0.000000in}{0.000000in}}%
\pgfpathlineto{\pgfqpoint{0.000000in}{-0.027778in}}%
\pgfusepath{stroke,fill}%
}%
\begin{pgfscope}%
\pgfsys@transformshift{1.960364in}{0.417642in}%
\pgfsys@useobject{currentmarker}{}%
\end{pgfscope}%
\end{pgfscope}%
\begin{pgfscope}%
\pgfpathrectangle{\pgfqpoint{0.664463in}{0.417642in}}{\pgfqpoint{3.353867in}{2.050688in}}%
\pgfusepath{clip}%
\pgfsetrectcap%
\pgfsetroundjoin%
\pgfsetlinewidth{0.803000pt}%
\definecolor{currentstroke}{rgb}{0.850000,0.850000,0.850000}%
\pgfsetstrokecolor{currentstroke}%
\pgfsetdash{}{0pt}%
\pgfpathmoveto{\pgfqpoint{2.025590in}{0.417642in}}%
\pgfpathlineto{\pgfqpoint{2.025590in}{2.468330in}}%
\pgfusepath{stroke}%
\end{pgfscope}%
\begin{pgfscope}%
\pgfsetbuttcap%
\pgfsetroundjoin%
\definecolor{currentfill}{rgb}{0.000000,0.000000,0.000000}%
\pgfsetfillcolor{currentfill}%
\pgfsetlinewidth{0.602250pt}%
\definecolor{currentstroke}{rgb}{0.000000,0.000000,0.000000}%
\pgfsetstrokecolor{currentstroke}%
\pgfsetdash{}{0pt}%
\pgfsys@defobject{currentmarker}{\pgfqpoint{0.000000in}{-0.027778in}}{\pgfqpoint{0.000000in}{0.000000in}}{%
\pgfpathmoveto{\pgfqpoint{0.000000in}{0.000000in}}%
\pgfpathlineto{\pgfqpoint{0.000000in}{-0.027778in}}%
\pgfusepath{stroke,fill}%
}%
\begin{pgfscope}%
\pgfsys@transformshift{2.025590in}{0.417642in}%
\pgfsys@useobject{currentmarker}{}%
\end{pgfscope}%
\end{pgfscope}%
\begin{pgfscope}%
\pgfpathrectangle{\pgfqpoint{0.664463in}{0.417642in}}{\pgfqpoint{3.353867in}{2.050688in}}%
\pgfusepath{clip}%
\pgfsetrectcap%
\pgfsetroundjoin%
\pgfsetlinewidth{0.803000pt}%
\definecolor{currentstroke}{rgb}{0.850000,0.850000,0.850000}%
\pgfsetstrokecolor{currentstroke}%
\pgfsetdash{}{0pt}%
\pgfpathmoveto{\pgfqpoint{2.467789in}{0.417642in}}%
\pgfpathlineto{\pgfqpoint{2.467789in}{2.468330in}}%
\pgfusepath{stroke}%
\end{pgfscope}%
\begin{pgfscope}%
\pgfsetbuttcap%
\pgfsetroundjoin%
\definecolor{currentfill}{rgb}{0.000000,0.000000,0.000000}%
\pgfsetfillcolor{currentfill}%
\pgfsetlinewidth{0.602250pt}%
\definecolor{currentstroke}{rgb}{0.000000,0.000000,0.000000}%
\pgfsetstrokecolor{currentstroke}%
\pgfsetdash{}{0pt}%
\pgfsys@defobject{currentmarker}{\pgfqpoint{0.000000in}{-0.027778in}}{\pgfqpoint{0.000000in}{0.000000in}}{%
\pgfpathmoveto{\pgfqpoint{0.000000in}{0.000000in}}%
\pgfpathlineto{\pgfqpoint{0.000000in}{-0.027778in}}%
\pgfusepath{stroke,fill}%
}%
\begin{pgfscope}%
\pgfsys@transformshift{2.467789in}{0.417642in}%
\pgfsys@useobject{currentmarker}{}%
\end{pgfscope}%
\end{pgfscope}%
\begin{pgfscope}%
\pgfpathrectangle{\pgfqpoint{0.664463in}{0.417642in}}{\pgfqpoint{3.353867in}{2.050688in}}%
\pgfusepath{clip}%
\pgfsetrectcap%
\pgfsetroundjoin%
\pgfsetlinewidth{0.803000pt}%
\definecolor{currentstroke}{rgb}{0.850000,0.850000,0.850000}%
\pgfsetstrokecolor{currentstroke}%
\pgfsetdash{}{0pt}%
\pgfpathmoveto{\pgfqpoint{2.692328in}{0.417642in}}%
\pgfpathlineto{\pgfqpoint{2.692328in}{2.468330in}}%
\pgfusepath{stroke}%
\end{pgfscope}%
\begin{pgfscope}%
\pgfsetbuttcap%
\pgfsetroundjoin%
\definecolor{currentfill}{rgb}{0.000000,0.000000,0.000000}%
\pgfsetfillcolor{currentfill}%
\pgfsetlinewidth{0.602250pt}%
\definecolor{currentstroke}{rgb}{0.000000,0.000000,0.000000}%
\pgfsetstrokecolor{currentstroke}%
\pgfsetdash{}{0pt}%
\pgfsys@defobject{currentmarker}{\pgfqpoint{0.000000in}{-0.027778in}}{\pgfqpoint{0.000000in}{0.000000in}}{%
\pgfpathmoveto{\pgfqpoint{0.000000in}{0.000000in}}%
\pgfpathlineto{\pgfqpoint{0.000000in}{-0.027778in}}%
\pgfusepath{stroke,fill}%
}%
\begin{pgfscope}%
\pgfsys@transformshift{2.692328in}{0.417642in}%
\pgfsys@useobject{currentmarker}{}%
\end{pgfscope}%
\end{pgfscope}%
\begin{pgfscope}%
\pgfpathrectangle{\pgfqpoint{0.664463in}{0.417642in}}{\pgfqpoint{3.353867in}{2.050688in}}%
\pgfusepath{clip}%
\pgfsetrectcap%
\pgfsetroundjoin%
\pgfsetlinewidth{0.803000pt}%
\definecolor{currentstroke}{rgb}{0.850000,0.850000,0.850000}%
\pgfsetstrokecolor{currentstroke}%
\pgfsetdash{}{0pt}%
\pgfpathmoveto{\pgfqpoint{2.851641in}{0.417642in}}%
\pgfpathlineto{\pgfqpoint{2.851641in}{2.468330in}}%
\pgfusepath{stroke}%
\end{pgfscope}%
\begin{pgfscope}%
\pgfsetbuttcap%
\pgfsetroundjoin%
\definecolor{currentfill}{rgb}{0.000000,0.000000,0.000000}%
\pgfsetfillcolor{currentfill}%
\pgfsetlinewidth{0.602250pt}%
\definecolor{currentstroke}{rgb}{0.000000,0.000000,0.000000}%
\pgfsetstrokecolor{currentstroke}%
\pgfsetdash{}{0pt}%
\pgfsys@defobject{currentmarker}{\pgfqpoint{0.000000in}{-0.027778in}}{\pgfqpoint{0.000000in}{0.000000in}}{%
\pgfpathmoveto{\pgfqpoint{0.000000in}{0.000000in}}%
\pgfpathlineto{\pgfqpoint{0.000000in}{-0.027778in}}%
\pgfusepath{stroke,fill}%
}%
\begin{pgfscope}%
\pgfsys@transformshift{2.851641in}{0.417642in}%
\pgfsys@useobject{currentmarker}{}%
\end{pgfscope}%
\end{pgfscope}%
\begin{pgfscope}%
\pgfpathrectangle{\pgfqpoint{0.664463in}{0.417642in}}{\pgfqpoint{3.353867in}{2.050688in}}%
\pgfusepath{clip}%
\pgfsetrectcap%
\pgfsetroundjoin%
\pgfsetlinewidth{0.803000pt}%
\definecolor{currentstroke}{rgb}{0.850000,0.850000,0.850000}%
\pgfsetstrokecolor{currentstroke}%
\pgfsetdash{}{0pt}%
\pgfpathmoveto{\pgfqpoint{2.975213in}{0.417642in}}%
\pgfpathlineto{\pgfqpoint{2.975213in}{2.468330in}}%
\pgfusepath{stroke}%
\end{pgfscope}%
\begin{pgfscope}%
\pgfsetbuttcap%
\pgfsetroundjoin%
\definecolor{currentfill}{rgb}{0.000000,0.000000,0.000000}%
\pgfsetfillcolor{currentfill}%
\pgfsetlinewidth{0.602250pt}%
\definecolor{currentstroke}{rgb}{0.000000,0.000000,0.000000}%
\pgfsetstrokecolor{currentstroke}%
\pgfsetdash{}{0pt}%
\pgfsys@defobject{currentmarker}{\pgfqpoint{0.000000in}{-0.027778in}}{\pgfqpoint{0.000000in}{0.000000in}}{%
\pgfpathmoveto{\pgfqpoint{0.000000in}{0.000000in}}%
\pgfpathlineto{\pgfqpoint{0.000000in}{-0.027778in}}%
\pgfusepath{stroke,fill}%
}%
\begin{pgfscope}%
\pgfsys@transformshift{2.975213in}{0.417642in}%
\pgfsys@useobject{currentmarker}{}%
\end{pgfscope}%
\end{pgfscope}%
\begin{pgfscope}%
\pgfpathrectangle{\pgfqpoint{0.664463in}{0.417642in}}{\pgfqpoint{3.353867in}{2.050688in}}%
\pgfusepath{clip}%
\pgfsetrectcap%
\pgfsetroundjoin%
\pgfsetlinewidth{0.803000pt}%
\definecolor{currentstroke}{rgb}{0.850000,0.850000,0.850000}%
\pgfsetstrokecolor{currentstroke}%
\pgfsetdash{}{0pt}%
\pgfpathmoveto{\pgfqpoint{3.076180in}{0.417642in}}%
\pgfpathlineto{\pgfqpoint{3.076180in}{2.468330in}}%
\pgfusepath{stroke}%
\end{pgfscope}%
\begin{pgfscope}%
\pgfsetbuttcap%
\pgfsetroundjoin%
\definecolor{currentfill}{rgb}{0.000000,0.000000,0.000000}%
\pgfsetfillcolor{currentfill}%
\pgfsetlinewidth{0.602250pt}%
\definecolor{currentstroke}{rgb}{0.000000,0.000000,0.000000}%
\pgfsetstrokecolor{currentstroke}%
\pgfsetdash{}{0pt}%
\pgfsys@defobject{currentmarker}{\pgfqpoint{0.000000in}{-0.027778in}}{\pgfqpoint{0.000000in}{0.000000in}}{%
\pgfpathmoveto{\pgfqpoint{0.000000in}{0.000000in}}%
\pgfpathlineto{\pgfqpoint{0.000000in}{-0.027778in}}%
\pgfusepath{stroke,fill}%
}%
\begin{pgfscope}%
\pgfsys@transformshift{3.076180in}{0.417642in}%
\pgfsys@useobject{currentmarker}{}%
\end{pgfscope}%
\end{pgfscope}%
\begin{pgfscope}%
\pgfpathrectangle{\pgfqpoint{0.664463in}{0.417642in}}{\pgfqpoint{3.353867in}{2.050688in}}%
\pgfusepath{clip}%
\pgfsetrectcap%
\pgfsetroundjoin%
\pgfsetlinewidth{0.803000pt}%
\definecolor{currentstroke}{rgb}{0.850000,0.850000,0.850000}%
\pgfsetstrokecolor{currentstroke}%
\pgfsetdash{}{0pt}%
\pgfpathmoveto{\pgfqpoint{3.161545in}{0.417642in}}%
\pgfpathlineto{\pgfqpoint{3.161545in}{2.468330in}}%
\pgfusepath{stroke}%
\end{pgfscope}%
\begin{pgfscope}%
\pgfsetbuttcap%
\pgfsetroundjoin%
\definecolor{currentfill}{rgb}{0.000000,0.000000,0.000000}%
\pgfsetfillcolor{currentfill}%
\pgfsetlinewidth{0.602250pt}%
\definecolor{currentstroke}{rgb}{0.000000,0.000000,0.000000}%
\pgfsetstrokecolor{currentstroke}%
\pgfsetdash{}{0pt}%
\pgfsys@defobject{currentmarker}{\pgfqpoint{0.000000in}{-0.027778in}}{\pgfqpoint{0.000000in}{0.000000in}}{%
\pgfpathmoveto{\pgfqpoint{0.000000in}{0.000000in}}%
\pgfpathlineto{\pgfqpoint{0.000000in}{-0.027778in}}%
\pgfusepath{stroke,fill}%
}%
\begin{pgfscope}%
\pgfsys@transformshift{3.161545in}{0.417642in}%
\pgfsys@useobject{currentmarker}{}%
\end{pgfscope}%
\end{pgfscope}%
\begin{pgfscope}%
\pgfpathrectangle{\pgfqpoint{0.664463in}{0.417642in}}{\pgfqpoint{3.353867in}{2.050688in}}%
\pgfusepath{clip}%
\pgfsetrectcap%
\pgfsetroundjoin%
\pgfsetlinewidth{0.803000pt}%
\definecolor{currentstroke}{rgb}{0.850000,0.850000,0.850000}%
\pgfsetstrokecolor{currentstroke}%
\pgfsetdash{}{0pt}%
\pgfpathmoveto{\pgfqpoint{3.235493in}{0.417642in}}%
\pgfpathlineto{\pgfqpoint{3.235493in}{2.468330in}}%
\pgfusepath{stroke}%
\end{pgfscope}%
\begin{pgfscope}%
\pgfsetbuttcap%
\pgfsetroundjoin%
\definecolor{currentfill}{rgb}{0.000000,0.000000,0.000000}%
\pgfsetfillcolor{currentfill}%
\pgfsetlinewidth{0.602250pt}%
\definecolor{currentstroke}{rgb}{0.000000,0.000000,0.000000}%
\pgfsetstrokecolor{currentstroke}%
\pgfsetdash{}{0pt}%
\pgfsys@defobject{currentmarker}{\pgfqpoint{0.000000in}{-0.027778in}}{\pgfqpoint{0.000000in}{0.000000in}}{%
\pgfpathmoveto{\pgfqpoint{0.000000in}{0.000000in}}%
\pgfpathlineto{\pgfqpoint{0.000000in}{-0.027778in}}%
\pgfusepath{stroke,fill}%
}%
\begin{pgfscope}%
\pgfsys@transformshift{3.235493in}{0.417642in}%
\pgfsys@useobject{currentmarker}{}%
\end{pgfscope}%
\end{pgfscope}%
\begin{pgfscope}%
\pgfpathrectangle{\pgfqpoint{0.664463in}{0.417642in}}{\pgfqpoint{3.353867in}{2.050688in}}%
\pgfusepath{clip}%
\pgfsetrectcap%
\pgfsetroundjoin%
\pgfsetlinewidth{0.803000pt}%
\definecolor{currentstroke}{rgb}{0.850000,0.850000,0.850000}%
\pgfsetstrokecolor{currentstroke}%
\pgfsetdash{}{0pt}%
\pgfpathmoveto{\pgfqpoint{3.300719in}{0.417642in}}%
\pgfpathlineto{\pgfqpoint{3.300719in}{2.468330in}}%
\pgfusepath{stroke}%
\end{pgfscope}%
\begin{pgfscope}%
\pgfsetbuttcap%
\pgfsetroundjoin%
\definecolor{currentfill}{rgb}{0.000000,0.000000,0.000000}%
\pgfsetfillcolor{currentfill}%
\pgfsetlinewidth{0.602250pt}%
\definecolor{currentstroke}{rgb}{0.000000,0.000000,0.000000}%
\pgfsetstrokecolor{currentstroke}%
\pgfsetdash{}{0pt}%
\pgfsys@defobject{currentmarker}{\pgfqpoint{0.000000in}{-0.027778in}}{\pgfqpoint{0.000000in}{0.000000in}}{%
\pgfpathmoveto{\pgfqpoint{0.000000in}{0.000000in}}%
\pgfpathlineto{\pgfqpoint{0.000000in}{-0.027778in}}%
\pgfusepath{stroke,fill}%
}%
\begin{pgfscope}%
\pgfsys@transformshift{3.300719in}{0.417642in}%
\pgfsys@useobject{currentmarker}{}%
\end{pgfscope}%
\end{pgfscope}%
\begin{pgfscope}%
\pgfpathrectangle{\pgfqpoint{0.664463in}{0.417642in}}{\pgfqpoint{3.353867in}{2.050688in}}%
\pgfusepath{clip}%
\pgfsetrectcap%
\pgfsetroundjoin%
\pgfsetlinewidth{0.803000pt}%
\definecolor{currentstroke}{rgb}{0.850000,0.850000,0.850000}%
\pgfsetstrokecolor{currentstroke}%
\pgfsetdash{}{0pt}%
\pgfpathmoveto{\pgfqpoint{3.742917in}{0.417642in}}%
\pgfpathlineto{\pgfqpoint{3.742917in}{2.468330in}}%
\pgfusepath{stroke}%
\end{pgfscope}%
\begin{pgfscope}%
\pgfsetbuttcap%
\pgfsetroundjoin%
\definecolor{currentfill}{rgb}{0.000000,0.000000,0.000000}%
\pgfsetfillcolor{currentfill}%
\pgfsetlinewidth{0.602250pt}%
\definecolor{currentstroke}{rgb}{0.000000,0.000000,0.000000}%
\pgfsetstrokecolor{currentstroke}%
\pgfsetdash{}{0pt}%
\pgfsys@defobject{currentmarker}{\pgfqpoint{0.000000in}{-0.027778in}}{\pgfqpoint{0.000000in}{0.000000in}}{%
\pgfpathmoveto{\pgfqpoint{0.000000in}{0.000000in}}%
\pgfpathlineto{\pgfqpoint{0.000000in}{-0.027778in}}%
\pgfusepath{stroke,fill}%
}%
\begin{pgfscope}%
\pgfsys@transformshift{3.742917in}{0.417642in}%
\pgfsys@useobject{currentmarker}{}%
\end{pgfscope}%
\end{pgfscope}%
\begin{pgfscope}%
\pgfpathrectangle{\pgfqpoint{0.664463in}{0.417642in}}{\pgfqpoint{3.353867in}{2.050688in}}%
\pgfusepath{clip}%
\pgfsetrectcap%
\pgfsetroundjoin%
\pgfsetlinewidth{0.803000pt}%
\definecolor{currentstroke}{rgb}{0.850000,0.850000,0.850000}%
\pgfsetstrokecolor{currentstroke}%
\pgfsetdash{}{0pt}%
\pgfpathmoveto{\pgfqpoint{3.967457in}{0.417642in}}%
\pgfpathlineto{\pgfqpoint{3.967457in}{2.468330in}}%
\pgfusepath{stroke}%
\end{pgfscope}%
\begin{pgfscope}%
\pgfsetbuttcap%
\pgfsetroundjoin%
\definecolor{currentfill}{rgb}{0.000000,0.000000,0.000000}%
\pgfsetfillcolor{currentfill}%
\pgfsetlinewidth{0.602250pt}%
\definecolor{currentstroke}{rgb}{0.000000,0.000000,0.000000}%
\pgfsetstrokecolor{currentstroke}%
\pgfsetdash{}{0pt}%
\pgfsys@defobject{currentmarker}{\pgfqpoint{0.000000in}{-0.027778in}}{\pgfqpoint{0.000000in}{0.000000in}}{%
\pgfpathmoveto{\pgfqpoint{0.000000in}{0.000000in}}%
\pgfpathlineto{\pgfqpoint{0.000000in}{-0.027778in}}%
\pgfusepath{stroke,fill}%
}%
\begin{pgfscope}%
\pgfsys@transformshift{3.967457in}{0.417642in}%
\pgfsys@useobject{currentmarker}{}%
\end{pgfscope}%
\end{pgfscope}%
\begin{pgfscope}%
\definecolor{textcolor}{rgb}{0.000000,0.000000,0.000000}%
\pgfsetstrokecolor{textcolor}%
\pgfsetfillcolor{textcolor}%
\pgftext[x=2.341396in,y=0.165003in,,top]{\color{textcolor}\rmfamily\fontsize{10.000000}{12.000000}\selectfont Frequency in \(\displaystyle \unit{\Hz}\)}%
\end{pgfscope}%
\begin{pgfscope}%
\pgfpathrectangle{\pgfqpoint{0.664463in}{0.417642in}}{\pgfqpoint{3.353867in}{2.050688in}}%
\pgfusepath{clip}%
\pgfsetrectcap%
\pgfsetroundjoin%
\pgfsetlinewidth{0.803000pt}%
\definecolor{currentstroke}{rgb}{0.450000,0.450000,0.450000}%
\pgfsetstrokecolor{currentstroke}%
\pgfsetdash{}{0pt}%
\pgfpathmoveto{\pgfqpoint{0.664463in}{1.340072in}}%
\pgfpathlineto{\pgfqpoint{4.018330in}{1.340072in}}%
\pgfusepath{stroke}%
\end{pgfscope}%
\begin{pgfscope}%
\pgfsetbuttcap%
\pgfsetroundjoin%
\definecolor{currentfill}{rgb}{0.000000,0.000000,0.000000}%
\pgfsetfillcolor{currentfill}%
\pgfsetlinewidth{0.803000pt}%
\definecolor{currentstroke}{rgb}{0.000000,0.000000,0.000000}%
\pgfsetstrokecolor{currentstroke}%
\pgfsetdash{}{0pt}%
\pgfsys@defobject{currentmarker}{\pgfqpoint{-0.048611in}{0.000000in}}{\pgfqpoint{-0.000000in}{0.000000in}}{%
\pgfpathmoveto{\pgfqpoint{-0.000000in}{0.000000in}}%
\pgfpathlineto{\pgfqpoint{-0.048611in}{0.000000in}}%
\pgfusepath{stroke,fill}%
}%
\begin{pgfscope}%
\pgfsys@transformshift{0.664463in}{1.340072in}%
\pgfsys@useobject{currentmarker}{}%
\end{pgfscope}%
\end{pgfscope}%
\begin{pgfscope}%
\definecolor{textcolor}{rgb}{0.000000,0.000000,0.000000}%
\pgfsetstrokecolor{textcolor}%
\pgfsetfillcolor{textcolor}%
\pgftext[x=0.260142in, y=1.300920in, left, base]{\color{textcolor}\rmfamily\fontsize{8.000000}{9.600000}\selectfont \(\displaystyle {10^{-13}}\)}%
\end{pgfscope}%
\begin{pgfscope}%
\pgfpathrectangle{\pgfqpoint{0.664463in}{0.417642in}}{\pgfqpoint{3.353867in}{2.050688in}}%
\pgfusepath{clip}%
\pgfsetrectcap%
\pgfsetroundjoin%
\pgfsetlinewidth{0.803000pt}%
\definecolor{currentstroke}{rgb}{0.450000,0.450000,0.450000}%
\pgfsetstrokecolor{currentstroke}%
\pgfsetdash{}{0pt}%
\pgfpathmoveto{\pgfqpoint{0.664463in}{2.276326in}}%
\pgfpathlineto{\pgfqpoint{4.018330in}{2.276326in}}%
\pgfusepath{stroke}%
\end{pgfscope}%
\begin{pgfscope}%
\pgfsetbuttcap%
\pgfsetroundjoin%
\definecolor{currentfill}{rgb}{0.000000,0.000000,0.000000}%
\pgfsetfillcolor{currentfill}%
\pgfsetlinewidth{0.803000pt}%
\definecolor{currentstroke}{rgb}{0.000000,0.000000,0.000000}%
\pgfsetstrokecolor{currentstroke}%
\pgfsetdash{}{0pt}%
\pgfsys@defobject{currentmarker}{\pgfqpoint{-0.048611in}{0.000000in}}{\pgfqpoint{-0.000000in}{0.000000in}}{%
\pgfpathmoveto{\pgfqpoint{-0.000000in}{0.000000in}}%
\pgfpathlineto{\pgfqpoint{-0.048611in}{0.000000in}}%
\pgfusepath{stroke,fill}%
}%
\begin{pgfscope}%
\pgfsys@transformshift{0.664463in}{2.276326in}%
\pgfsys@useobject{currentmarker}{}%
\end{pgfscope}%
\end{pgfscope}%
\begin{pgfscope}%
\definecolor{textcolor}{rgb}{0.000000,0.000000,0.000000}%
\pgfsetstrokecolor{textcolor}%
\pgfsetfillcolor{textcolor}%
\pgftext[x=0.260142in, y=2.237174in, left, base]{\color{textcolor}\rmfamily\fontsize{8.000000}{9.600000}\selectfont \(\displaystyle {10^{-12}}\)}%
\end{pgfscope}%
\begin{pgfscope}%
\pgfpathrectangle{\pgfqpoint{0.664463in}{0.417642in}}{\pgfqpoint{3.353867in}{2.050688in}}%
\pgfusepath{clip}%
\pgfsetrectcap%
\pgfsetroundjoin%
\pgfsetlinewidth{0.803000pt}%
\definecolor{currentstroke}{rgb}{0.850000,0.850000,0.850000}%
\pgfsetstrokecolor{currentstroke}%
\pgfsetdash{}{0pt}%
\pgfpathmoveto{\pgfqpoint{0.664463in}{0.685659in}}%
\pgfpathlineto{\pgfqpoint{4.018330in}{0.685659in}}%
\pgfusepath{stroke}%
\end{pgfscope}%
\begin{pgfscope}%
\pgfsetbuttcap%
\pgfsetroundjoin%
\definecolor{currentfill}{rgb}{0.000000,0.000000,0.000000}%
\pgfsetfillcolor{currentfill}%
\pgfsetlinewidth{0.602250pt}%
\definecolor{currentstroke}{rgb}{0.000000,0.000000,0.000000}%
\pgfsetstrokecolor{currentstroke}%
\pgfsetdash{}{0pt}%
\pgfsys@defobject{currentmarker}{\pgfqpoint{-0.027778in}{0.000000in}}{\pgfqpoint{-0.000000in}{0.000000in}}{%
\pgfpathmoveto{\pgfqpoint{-0.000000in}{0.000000in}}%
\pgfpathlineto{\pgfqpoint{-0.027778in}{0.000000in}}%
\pgfusepath{stroke,fill}%
}%
\begin{pgfscope}%
\pgfsys@transformshift{0.664463in}{0.685659in}%
\pgfsys@useobject{currentmarker}{}%
\end{pgfscope}%
\end{pgfscope}%
\begin{pgfscope}%
\pgfpathrectangle{\pgfqpoint{0.664463in}{0.417642in}}{\pgfqpoint{3.353867in}{2.050688in}}%
\pgfusepath{clip}%
\pgfsetrectcap%
\pgfsetroundjoin%
\pgfsetlinewidth{0.803000pt}%
\definecolor{currentstroke}{rgb}{0.850000,0.850000,0.850000}%
\pgfsetstrokecolor{currentstroke}%
\pgfsetdash{}{0pt}%
\pgfpathmoveto{\pgfqpoint{0.664463in}{0.850525in}}%
\pgfpathlineto{\pgfqpoint{4.018330in}{0.850525in}}%
\pgfusepath{stroke}%
\end{pgfscope}%
\begin{pgfscope}%
\pgfsetbuttcap%
\pgfsetroundjoin%
\definecolor{currentfill}{rgb}{0.000000,0.000000,0.000000}%
\pgfsetfillcolor{currentfill}%
\pgfsetlinewidth{0.602250pt}%
\definecolor{currentstroke}{rgb}{0.000000,0.000000,0.000000}%
\pgfsetstrokecolor{currentstroke}%
\pgfsetdash{}{0pt}%
\pgfsys@defobject{currentmarker}{\pgfqpoint{-0.027778in}{0.000000in}}{\pgfqpoint{-0.000000in}{0.000000in}}{%
\pgfpathmoveto{\pgfqpoint{-0.000000in}{0.000000in}}%
\pgfpathlineto{\pgfqpoint{-0.027778in}{0.000000in}}%
\pgfusepath{stroke,fill}%
}%
\begin{pgfscope}%
\pgfsys@transformshift{0.664463in}{0.850525in}%
\pgfsys@useobject{currentmarker}{}%
\end{pgfscope}%
\end{pgfscope}%
\begin{pgfscope}%
\pgfpathrectangle{\pgfqpoint{0.664463in}{0.417642in}}{\pgfqpoint{3.353867in}{2.050688in}}%
\pgfusepath{clip}%
\pgfsetrectcap%
\pgfsetroundjoin%
\pgfsetlinewidth{0.803000pt}%
\definecolor{currentstroke}{rgb}{0.850000,0.850000,0.850000}%
\pgfsetstrokecolor{currentstroke}%
\pgfsetdash{}{0pt}%
\pgfpathmoveto{\pgfqpoint{0.664463in}{0.967500in}}%
\pgfpathlineto{\pgfqpoint{4.018330in}{0.967500in}}%
\pgfusepath{stroke}%
\end{pgfscope}%
\begin{pgfscope}%
\pgfsetbuttcap%
\pgfsetroundjoin%
\definecolor{currentfill}{rgb}{0.000000,0.000000,0.000000}%
\pgfsetfillcolor{currentfill}%
\pgfsetlinewidth{0.602250pt}%
\definecolor{currentstroke}{rgb}{0.000000,0.000000,0.000000}%
\pgfsetstrokecolor{currentstroke}%
\pgfsetdash{}{0pt}%
\pgfsys@defobject{currentmarker}{\pgfqpoint{-0.027778in}{0.000000in}}{\pgfqpoint{-0.000000in}{0.000000in}}{%
\pgfpathmoveto{\pgfqpoint{-0.000000in}{0.000000in}}%
\pgfpathlineto{\pgfqpoint{-0.027778in}{0.000000in}}%
\pgfusepath{stroke,fill}%
}%
\begin{pgfscope}%
\pgfsys@transformshift{0.664463in}{0.967500in}%
\pgfsys@useobject{currentmarker}{}%
\end{pgfscope}%
\end{pgfscope}%
\begin{pgfscope}%
\pgfpathrectangle{\pgfqpoint{0.664463in}{0.417642in}}{\pgfqpoint{3.353867in}{2.050688in}}%
\pgfusepath{clip}%
\pgfsetrectcap%
\pgfsetroundjoin%
\pgfsetlinewidth{0.803000pt}%
\definecolor{currentstroke}{rgb}{0.850000,0.850000,0.850000}%
\pgfsetstrokecolor{currentstroke}%
\pgfsetdash{}{0pt}%
\pgfpathmoveto{\pgfqpoint{0.664463in}{1.058232in}}%
\pgfpathlineto{\pgfqpoint{4.018330in}{1.058232in}}%
\pgfusepath{stroke}%
\end{pgfscope}%
\begin{pgfscope}%
\pgfsetbuttcap%
\pgfsetroundjoin%
\definecolor{currentfill}{rgb}{0.000000,0.000000,0.000000}%
\pgfsetfillcolor{currentfill}%
\pgfsetlinewidth{0.602250pt}%
\definecolor{currentstroke}{rgb}{0.000000,0.000000,0.000000}%
\pgfsetstrokecolor{currentstroke}%
\pgfsetdash{}{0pt}%
\pgfsys@defobject{currentmarker}{\pgfqpoint{-0.027778in}{0.000000in}}{\pgfqpoint{-0.000000in}{0.000000in}}{%
\pgfpathmoveto{\pgfqpoint{-0.000000in}{0.000000in}}%
\pgfpathlineto{\pgfqpoint{-0.027778in}{0.000000in}}%
\pgfusepath{stroke,fill}%
}%
\begin{pgfscope}%
\pgfsys@transformshift{0.664463in}{1.058232in}%
\pgfsys@useobject{currentmarker}{}%
\end{pgfscope}%
\end{pgfscope}%
\begin{pgfscope}%
\pgfpathrectangle{\pgfqpoint{0.664463in}{0.417642in}}{\pgfqpoint{3.353867in}{2.050688in}}%
\pgfusepath{clip}%
\pgfsetrectcap%
\pgfsetroundjoin%
\pgfsetlinewidth{0.803000pt}%
\definecolor{currentstroke}{rgb}{0.850000,0.850000,0.850000}%
\pgfsetstrokecolor{currentstroke}%
\pgfsetdash{}{0pt}%
\pgfpathmoveto{\pgfqpoint{0.664463in}{1.132366in}}%
\pgfpathlineto{\pgfqpoint{4.018330in}{1.132366in}}%
\pgfusepath{stroke}%
\end{pgfscope}%
\begin{pgfscope}%
\pgfsetbuttcap%
\pgfsetroundjoin%
\definecolor{currentfill}{rgb}{0.000000,0.000000,0.000000}%
\pgfsetfillcolor{currentfill}%
\pgfsetlinewidth{0.602250pt}%
\definecolor{currentstroke}{rgb}{0.000000,0.000000,0.000000}%
\pgfsetstrokecolor{currentstroke}%
\pgfsetdash{}{0pt}%
\pgfsys@defobject{currentmarker}{\pgfqpoint{-0.027778in}{0.000000in}}{\pgfqpoint{-0.000000in}{0.000000in}}{%
\pgfpathmoveto{\pgfqpoint{-0.000000in}{0.000000in}}%
\pgfpathlineto{\pgfqpoint{-0.027778in}{0.000000in}}%
\pgfusepath{stroke,fill}%
}%
\begin{pgfscope}%
\pgfsys@transformshift{0.664463in}{1.132366in}%
\pgfsys@useobject{currentmarker}{}%
\end{pgfscope}%
\end{pgfscope}%
\begin{pgfscope}%
\pgfpathrectangle{\pgfqpoint{0.664463in}{0.417642in}}{\pgfqpoint{3.353867in}{2.050688in}}%
\pgfusepath{clip}%
\pgfsetrectcap%
\pgfsetroundjoin%
\pgfsetlinewidth{0.803000pt}%
\definecolor{currentstroke}{rgb}{0.850000,0.850000,0.850000}%
\pgfsetstrokecolor{currentstroke}%
\pgfsetdash{}{0pt}%
\pgfpathmoveto{\pgfqpoint{0.664463in}{1.195045in}}%
\pgfpathlineto{\pgfqpoint{4.018330in}{1.195045in}}%
\pgfusepath{stroke}%
\end{pgfscope}%
\begin{pgfscope}%
\pgfsetbuttcap%
\pgfsetroundjoin%
\definecolor{currentfill}{rgb}{0.000000,0.000000,0.000000}%
\pgfsetfillcolor{currentfill}%
\pgfsetlinewidth{0.602250pt}%
\definecolor{currentstroke}{rgb}{0.000000,0.000000,0.000000}%
\pgfsetstrokecolor{currentstroke}%
\pgfsetdash{}{0pt}%
\pgfsys@defobject{currentmarker}{\pgfqpoint{-0.027778in}{0.000000in}}{\pgfqpoint{-0.000000in}{0.000000in}}{%
\pgfpathmoveto{\pgfqpoint{-0.000000in}{0.000000in}}%
\pgfpathlineto{\pgfqpoint{-0.027778in}{0.000000in}}%
\pgfusepath{stroke,fill}%
}%
\begin{pgfscope}%
\pgfsys@transformshift{0.664463in}{1.195045in}%
\pgfsys@useobject{currentmarker}{}%
\end{pgfscope}%
\end{pgfscope}%
\begin{pgfscope}%
\pgfpathrectangle{\pgfqpoint{0.664463in}{0.417642in}}{\pgfqpoint{3.353867in}{2.050688in}}%
\pgfusepath{clip}%
\pgfsetrectcap%
\pgfsetroundjoin%
\pgfsetlinewidth{0.803000pt}%
\definecolor{currentstroke}{rgb}{0.850000,0.850000,0.850000}%
\pgfsetstrokecolor{currentstroke}%
\pgfsetdash{}{0pt}%
\pgfpathmoveto{\pgfqpoint{0.664463in}{1.249340in}}%
\pgfpathlineto{\pgfqpoint{4.018330in}{1.249340in}}%
\pgfusepath{stroke}%
\end{pgfscope}%
\begin{pgfscope}%
\pgfsetbuttcap%
\pgfsetroundjoin%
\definecolor{currentfill}{rgb}{0.000000,0.000000,0.000000}%
\pgfsetfillcolor{currentfill}%
\pgfsetlinewidth{0.602250pt}%
\definecolor{currentstroke}{rgb}{0.000000,0.000000,0.000000}%
\pgfsetstrokecolor{currentstroke}%
\pgfsetdash{}{0pt}%
\pgfsys@defobject{currentmarker}{\pgfqpoint{-0.027778in}{0.000000in}}{\pgfqpoint{-0.000000in}{0.000000in}}{%
\pgfpathmoveto{\pgfqpoint{-0.000000in}{0.000000in}}%
\pgfpathlineto{\pgfqpoint{-0.027778in}{0.000000in}}%
\pgfusepath{stroke,fill}%
}%
\begin{pgfscope}%
\pgfsys@transformshift{0.664463in}{1.249340in}%
\pgfsys@useobject{currentmarker}{}%
\end{pgfscope}%
\end{pgfscope}%
\begin{pgfscope}%
\pgfpathrectangle{\pgfqpoint{0.664463in}{0.417642in}}{\pgfqpoint{3.353867in}{2.050688in}}%
\pgfusepath{clip}%
\pgfsetrectcap%
\pgfsetroundjoin%
\pgfsetlinewidth{0.803000pt}%
\definecolor{currentstroke}{rgb}{0.850000,0.850000,0.850000}%
\pgfsetstrokecolor{currentstroke}%
\pgfsetdash{}{0pt}%
\pgfpathmoveto{\pgfqpoint{0.664463in}{1.297232in}}%
\pgfpathlineto{\pgfqpoint{4.018330in}{1.297232in}}%
\pgfusepath{stroke}%
\end{pgfscope}%
\begin{pgfscope}%
\pgfsetbuttcap%
\pgfsetroundjoin%
\definecolor{currentfill}{rgb}{0.000000,0.000000,0.000000}%
\pgfsetfillcolor{currentfill}%
\pgfsetlinewidth{0.602250pt}%
\definecolor{currentstroke}{rgb}{0.000000,0.000000,0.000000}%
\pgfsetstrokecolor{currentstroke}%
\pgfsetdash{}{0pt}%
\pgfsys@defobject{currentmarker}{\pgfqpoint{-0.027778in}{0.000000in}}{\pgfqpoint{-0.000000in}{0.000000in}}{%
\pgfpathmoveto{\pgfqpoint{-0.000000in}{0.000000in}}%
\pgfpathlineto{\pgfqpoint{-0.027778in}{0.000000in}}%
\pgfusepath{stroke,fill}%
}%
\begin{pgfscope}%
\pgfsys@transformshift{0.664463in}{1.297232in}%
\pgfsys@useobject{currentmarker}{}%
\end{pgfscope}%
\end{pgfscope}%
\begin{pgfscope}%
\pgfpathrectangle{\pgfqpoint{0.664463in}{0.417642in}}{\pgfqpoint{3.353867in}{2.050688in}}%
\pgfusepath{clip}%
\pgfsetrectcap%
\pgfsetroundjoin%
\pgfsetlinewidth{0.803000pt}%
\definecolor{currentstroke}{rgb}{0.850000,0.850000,0.850000}%
\pgfsetstrokecolor{currentstroke}%
\pgfsetdash{}{0pt}%
\pgfpathmoveto{\pgfqpoint{0.664463in}{1.621913in}}%
\pgfpathlineto{\pgfqpoint{4.018330in}{1.621913in}}%
\pgfusepath{stroke}%
\end{pgfscope}%
\begin{pgfscope}%
\pgfsetbuttcap%
\pgfsetroundjoin%
\definecolor{currentfill}{rgb}{0.000000,0.000000,0.000000}%
\pgfsetfillcolor{currentfill}%
\pgfsetlinewidth{0.602250pt}%
\definecolor{currentstroke}{rgb}{0.000000,0.000000,0.000000}%
\pgfsetstrokecolor{currentstroke}%
\pgfsetdash{}{0pt}%
\pgfsys@defobject{currentmarker}{\pgfqpoint{-0.027778in}{0.000000in}}{\pgfqpoint{-0.000000in}{0.000000in}}{%
\pgfpathmoveto{\pgfqpoint{-0.000000in}{0.000000in}}%
\pgfpathlineto{\pgfqpoint{-0.027778in}{0.000000in}}%
\pgfusepath{stroke,fill}%
}%
\begin{pgfscope}%
\pgfsys@transformshift{0.664463in}{1.621913in}%
\pgfsys@useobject{currentmarker}{}%
\end{pgfscope}%
\end{pgfscope}%
\begin{pgfscope}%
\pgfpathrectangle{\pgfqpoint{0.664463in}{0.417642in}}{\pgfqpoint{3.353867in}{2.050688in}}%
\pgfusepath{clip}%
\pgfsetrectcap%
\pgfsetroundjoin%
\pgfsetlinewidth{0.803000pt}%
\definecolor{currentstroke}{rgb}{0.850000,0.850000,0.850000}%
\pgfsetstrokecolor{currentstroke}%
\pgfsetdash{}{0pt}%
\pgfpathmoveto{\pgfqpoint{0.664463in}{1.786779in}}%
\pgfpathlineto{\pgfqpoint{4.018330in}{1.786779in}}%
\pgfusepath{stroke}%
\end{pgfscope}%
\begin{pgfscope}%
\pgfsetbuttcap%
\pgfsetroundjoin%
\definecolor{currentfill}{rgb}{0.000000,0.000000,0.000000}%
\pgfsetfillcolor{currentfill}%
\pgfsetlinewidth{0.602250pt}%
\definecolor{currentstroke}{rgb}{0.000000,0.000000,0.000000}%
\pgfsetstrokecolor{currentstroke}%
\pgfsetdash{}{0pt}%
\pgfsys@defobject{currentmarker}{\pgfqpoint{-0.027778in}{0.000000in}}{\pgfqpoint{-0.000000in}{0.000000in}}{%
\pgfpathmoveto{\pgfqpoint{-0.000000in}{0.000000in}}%
\pgfpathlineto{\pgfqpoint{-0.027778in}{0.000000in}}%
\pgfusepath{stroke,fill}%
}%
\begin{pgfscope}%
\pgfsys@transformshift{0.664463in}{1.786779in}%
\pgfsys@useobject{currentmarker}{}%
\end{pgfscope}%
\end{pgfscope}%
\begin{pgfscope}%
\pgfpathrectangle{\pgfqpoint{0.664463in}{0.417642in}}{\pgfqpoint{3.353867in}{2.050688in}}%
\pgfusepath{clip}%
\pgfsetrectcap%
\pgfsetroundjoin%
\pgfsetlinewidth{0.803000pt}%
\definecolor{currentstroke}{rgb}{0.850000,0.850000,0.850000}%
\pgfsetstrokecolor{currentstroke}%
\pgfsetdash{}{0pt}%
\pgfpathmoveto{\pgfqpoint{0.664463in}{1.903753in}}%
\pgfpathlineto{\pgfqpoint{4.018330in}{1.903753in}}%
\pgfusepath{stroke}%
\end{pgfscope}%
\begin{pgfscope}%
\pgfsetbuttcap%
\pgfsetroundjoin%
\definecolor{currentfill}{rgb}{0.000000,0.000000,0.000000}%
\pgfsetfillcolor{currentfill}%
\pgfsetlinewidth{0.602250pt}%
\definecolor{currentstroke}{rgb}{0.000000,0.000000,0.000000}%
\pgfsetstrokecolor{currentstroke}%
\pgfsetdash{}{0pt}%
\pgfsys@defobject{currentmarker}{\pgfqpoint{-0.027778in}{0.000000in}}{\pgfqpoint{-0.000000in}{0.000000in}}{%
\pgfpathmoveto{\pgfqpoint{-0.000000in}{0.000000in}}%
\pgfpathlineto{\pgfqpoint{-0.027778in}{0.000000in}}%
\pgfusepath{stroke,fill}%
}%
\begin{pgfscope}%
\pgfsys@transformshift{0.664463in}{1.903753in}%
\pgfsys@useobject{currentmarker}{}%
\end{pgfscope}%
\end{pgfscope}%
\begin{pgfscope}%
\pgfpathrectangle{\pgfqpoint{0.664463in}{0.417642in}}{\pgfqpoint{3.353867in}{2.050688in}}%
\pgfusepath{clip}%
\pgfsetrectcap%
\pgfsetroundjoin%
\pgfsetlinewidth{0.803000pt}%
\definecolor{currentstroke}{rgb}{0.850000,0.850000,0.850000}%
\pgfsetstrokecolor{currentstroke}%
\pgfsetdash{}{0pt}%
\pgfpathmoveto{\pgfqpoint{0.664463in}{1.994486in}}%
\pgfpathlineto{\pgfqpoint{4.018330in}{1.994486in}}%
\pgfusepath{stroke}%
\end{pgfscope}%
\begin{pgfscope}%
\pgfsetbuttcap%
\pgfsetroundjoin%
\definecolor{currentfill}{rgb}{0.000000,0.000000,0.000000}%
\pgfsetfillcolor{currentfill}%
\pgfsetlinewidth{0.602250pt}%
\definecolor{currentstroke}{rgb}{0.000000,0.000000,0.000000}%
\pgfsetstrokecolor{currentstroke}%
\pgfsetdash{}{0pt}%
\pgfsys@defobject{currentmarker}{\pgfqpoint{-0.027778in}{0.000000in}}{\pgfqpoint{-0.000000in}{0.000000in}}{%
\pgfpathmoveto{\pgfqpoint{-0.000000in}{0.000000in}}%
\pgfpathlineto{\pgfqpoint{-0.027778in}{0.000000in}}%
\pgfusepath{stroke,fill}%
}%
\begin{pgfscope}%
\pgfsys@transformshift{0.664463in}{1.994486in}%
\pgfsys@useobject{currentmarker}{}%
\end{pgfscope}%
\end{pgfscope}%
\begin{pgfscope}%
\pgfpathrectangle{\pgfqpoint{0.664463in}{0.417642in}}{\pgfqpoint{3.353867in}{2.050688in}}%
\pgfusepath{clip}%
\pgfsetrectcap%
\pgfsetroundjoin%
\pgfsetlinewidth{0.803000pt}%
\definecolor{currentstroke}{rgb}{0.850000,0.850000,0.850000}%
\pgfsetstrokecolor{currentstroke}%
\pgfsetdash{}{0pt}%
\pgfpathmoveto{\pgfqpoint{0.664463in}{2.068620in}}%
\pgfpathlineto{\pgfqpoint{4.018330in}{2.068620in}}%
\pgfusepath{stroke}%
\end{pgfscope}%
\begin{pgfscope}%
\pgfsetbuttcap%
\pgfsetroundjoin%
\definecolor{currentfill}{rgb}{0.000000,0.000000,0.000000}%
\pgfsetfillcolor{currentfill}%
\pgfsetlinewidth{0.602250pt}%
\definecolor{currentstroke}{rgb}{0.000000,0.000000,0.000000}%
\pgfsetstrokecolor{currentstroke}%
\pgfsetdash{}{0pt}%
\pgfsys@defobject{currentmarker}{\pgfqpoint{-0.027778in}{0.000000in}}{\pgfqpoint{-0.000000in}{0.000000in}}{%
\pgfpathmoveto{\pgfqpoint{-0.000000in}{0.000000in}}%
\pgfpathlineto{\pgfqpoint{-0.027778in}{0.000000in}}%
\pgfusepath{stroke,fill}%
}%
\begin{pgfscope}%
\pgfsys@transformshift{0.664463in}{2.068620in}%
\pgfsys@useobject{currentmarker}{}%
\end{pgfscope}%
\end{pgfscope}%
\begin{pgfscope}%
\pgfpathrectangle{\pgfqpoint{0.664463in}{0.417642in}}{\pgfqpoint{3.353867in}{2.050688in}}%
\pgfusepath{clip}%
\pgfsetrectcap%
\pgfsetroundjoin%
\pgfsetlinewidth{0.803000pt}%
\definecolor{currentstroke}{rgb}{0.850000,0.850000,0.850000}%
\pgfsetstrokecolor{currentstroke}%
\pgfsetdash{}{0pt}%
\pgfpathmoveto{\pgfqpoint{0.664463in}{2.131299in}}%
\pgfpathlineto{\pgfqpoint{4.018330in}{2.131299in}}%
\pgfusepath{stroke}%
\end{pgfscope}%
\begin{pgfscope}%
\pgfsetbuttcap%
\pgfsetroundjoin%
\definecolor{currentfill}{rgb}{0.000000,0.000000,0.000000}%
\pgfsetfillcolor{currentfill}%
\pgfsetlinewidth{0.602250pt}%
\definecolor{currentstroke}{rgb}{0.000000,0.000000,0.000000}%
\pgfsetstrokecolor{currentstroke}%
\pgfsetdash{}{0pt}%
\pgfsys@defobject{currentmarker}{\pgfqpoint{-0.027778in}{0.000000in}}{\pgfqpoint{-0.000000in}{0.000000in}}{%
\pgfpathmoveto{\pgfqpoint{-0.000000in}{0.000000in}}%
\pgfpathlineto{\pgfqpoint{-0.027778in}{0.000000in}}%
\pgfusepath{stroke,fill}%
}%
\begin{pgfscope}%
\pgfsys@transformshift{0.664463in}{2.131299in}%
\pgfsys@useobject{currentmarker}{}%
\end{pgfscope}%
\end{pgfscope}%
\begin{pgfscope}%
\pgfpathrectangle{\pgfqpoint{0.664463in}{0.417642in}}{\pgfqpoint{3.353867in}{2.050688in}}%
\pgfusepath{clip}%
\pgfsetrectcap%
\pgfsetroundjoin%
\pgfsetlinewidth{0.803000pt}%
\definecolor{currentstroke}{rgb}{0.850000,0.850000,0.850000}%
\pgfsetstrokecolor{currentstroke}%
\pgfsetdash{}{0pt}%
\pgfpathmoveto{\pgfqpoint{0.664463in}{2.185594in}}%
\pgfpathlineto{\pgfqpoint{4.018330in}{2.185594in}}%
\pgfusepath{stroke}%
\end{pgfscope}%
\begin{pgfscope}%
\pgfsetbuttcap%
\pgfsetroundjoin%
\definecolor{currentfill}{rgb}{0.000000,0.000000,0.000000}%
\pgfsetfillcolor{currentfill}%
\pgfsetlinewidth{0.602250pt}%
\definecolor{currentstroke}{rgb}{0.000000,0.000000,0.000000}%
\pgfsetstrokecolor{currentstroke}%
\pgfsetdash{}{0pt}%
\pgfsys@defobject{currentmarker}{\pgfqpoint{-0.027778in}{0.000000in}}{\pgfqpoint{-0.000000in}{0.000000in}}{%
\pgfpathmoveto{\pgfqpoint{-0.000000in}{0.000000in}}%
\pgfpathlineto{\pgfqpoint{-0.027778in}{0.000000in}}%
\pgfusepath{stroke,fill}%
}%
\begin{pgfscope}%
\pgfsys@transformshift{0.664463in}{2.185594in}%
\pgfsys@useobject{currentmarker}{}%
\end{pgfscope}%
\end{pgfscope}%
\begin{pgfscope}%
\pgfpathrectangle{\pgfqpoint{0.664463in}{0.417642in}}{\pgfqpoint{3.353867in}{2.050688in}}%
\pgfusepath{clip}%
\pgfsetrectcap%
\pgfsetroundjoin%
\pgfsetlinewidth{0.803000pt}%
\definecolor{currentstroke}{rgb}{0.850000,0.850000,0.850000}%
\pgfsetstrokecolor{currentstroke}%
\pgfsetdash{}{0pt}%
\pgfpathmoveto{\pgfqpoint{0.664463in}{2.233486in}}%
\pgfpathlineto{\pgfqpoint{4.018330in}{2.233486in}}%
\pgfusepath{stroke}%
\end{pgfscope}%
\begin{pgfscope}%
\pgfsetbuttcap%
\pgfsetroundjoin%
\definecolor{currentfill}{rgb}{0.000000,0.000000,0.000000}%
\pgfsetfillcolor{currentfill}%
\pgfsetlinewidth{0.602250pt}%
\definecolor{currentstroke}{rgb}{0.000000,0.000000,0.000000}%
\pgfsetstrokecolor{currentstroke}%
\pgfsetdash{}{0pt}%
\pgfsys@defobject{currentmarker}{\pgfqpoint{-0.027778in}{0.000000in}}{\pgfqpoint{-0.000000in}{0.000000in}}{%
\pgfpathmoveto{\pgfqpoint{-0.000000in}{0.000000in}}%
\pgfpathlineto{\pgfqpoint{-0.027778in}{0.000000in}}%
\pgfusepath{stroke,fill}%
}%
\begin{pgfscope}%
\pgfsys@transformshift{0.664463in}{2.233486in}%
\pgfsys@useobject{currentmarker}{}%
\end{pgfscope}%
\end{pgfscope}%
\begin{pgfscope}%
\definecolor{textcolor}{rgb}{0.000000,0.000000,0.000000}%
\pgfsetstrokecolor{textcolor}%
\pgfsetfillcolor{textcolor}%
\pgftext[x=0.204587in,y=1.442986in,,bottom,rotate=90.000000]{\color{textcolor}\rmfamily\fontsize{10.000000}{12.000000}\selectfont  \(\displaystyle S_y(f)\) in \(\displaystyle \unit{\V \per \Hz}\)}%
\end{pgfscope}%
\begin{pgfscope}%
\pgfpathrectangle{\pgfqpoint{0.664463in}{0.417642in}}{\pgfqpoint{3.353867in}{2.050688in}}%
\pgfusepath{clip}%
\pgfsetbuttcap%
\pgfsetroundjoin%
\definecolor{currentfill}{rgb}{0.337255,0.705882,0.913725}%
\pgfsetfillcolor{currentfill}%
\pgfsetlinewidth{1.003750pt}%
\definecolor{currentstroke}{rgb}{0.337255,0.705882,0.913725}%
\pgfsetstrokecolor{currentstroke}%
\pgfsetdash{}{0pt}%
\pgfsys@defobject{currentmarker}{\pgfqpoint{-0.013889in}{-0.013889in}}{\pgfqpoint{0.013889in}{0.013889in}}{%
\pgfpathmoveto{\pgfqpoint{0.000000in}{-0.013889in}}%
\pgfpathcurveto{\pgfqpoint{0.003683in}{-0.013889in}}{\pgfqpoint{0.007216in}{-0.012425in}}{\pgfqpoint{0.009821in}{-0.009821in}}%
\pgfpathcurveto{\pgfqpoint{0.012425in}{-0.007216in}}{\pgfqpoint{0.013889in}{-0.003683in}}{\pgfqpoint{0.013889in}{0.000000in}}%
\pgfpathcurveto{\pgfqpoint{0.013889in}{0.003683in}}{\pgfqpoint{0.012425in}{0.007216in}}{\pgfqpoint{0.009821in}{0.009821in}}%
\pgfpathcurveto{\pgfqpoint{0.007216in}{0.012425in}}{\pgfqpoint{0.003683in}{0.013889in}}{\pgfqpoint{0.000000in}{0.013889in}}%
\pgfpathcurveto{\pgfqpoint{-0.003683in}{0.013889in}}{\pgfqpoint{-0.007216in}{0.012425in}}{\pgfqpoint{-0.009821in}{0.009821in}}%
\pgfpathcurveto{\pgfqpoint{-0.012425in}{0.007216in}}{\pgfqpoint{-0.013889in}{0.003683in}}{\pgfqpoint{-0.013889in}{0.000000in}}%
\pgfpathcurveto{\pgfqpoint{-0.013889in}{-0.003683in}}{\pgfqpoint{-0.012425in}{-0.007216in}}{\pgfqpoint{-0.009821in}{-0.009821in}}%
\pgfpathcurveto{\pgfqpoint{-0.007216in}{-0.012425in}}{\pgfqpoint{-0.003683in}{-0.013889in}}{\pgfqpoint{0.000000in}{-0.013889in}}%
\pgfpathlineto{\pgfqpoint{0.000000in}{-0.013889in}}%
\pgfpathclose%
\pgfusepath{stroke,fill}%
}%
\begin{pgfscope}%
\pgfsys@transformshift{0.816911in}{1.400394in}%
\pgfsys@useobject{currentmarker}{}%
\end{pgfscope}%
\begin{pgfscope}%
\pgfsys@transformshift{0.837348in}{1.413397in}%
\pgfsys@useobject{currentmarker}{}%
\end{pgfscope}%
\begin{pgfscope}%
\pgfsys@transformshift{0.857058in}{1.387691in}%
\pgfsys@useobject{currentmarker}{}%
\end{pgfscope}%
\begin{pgfscope}%
\pgfsys@transformshift{0.876090in}{1.418925in}%
\pgfsys@useobject{currentmarker}{}%
\end{pgfscope}%
\begin{pgfscope}%
\pgfsys@transformshift{0.894490in}{1.409499in}%
\pgfsys@useobject{currentmarker}{}%
\end{pgfscope}%
\begin{pgfscope}%
\pgfsys@transformshift{0.912298in}{1.421163in}%
\pgfsys@useobject{currentmarker}{}%
\end{pgfscope}%
\begin{pgfscope}%
\pgfsys@transformshift{0.932938in}{1.406174in}%
\pgfsys@useobject{currentmarker}{}%
\end{pgfscope}%
\begin{pgfscope}%
\pgfsys@transformshift{0.952836in}{1.441144in}%
\pgfsys@useobject{currentmarker}{}%
\end{pgfscope}%
\begin{pgfscope}%
\pgfsys@transformshift{0.972045in}{1.402009in}%
\pgfsys@useobject{currentmarker}{}%
\end{pgfscope}%
\begin{pgfscope}%
\pgfsys@transformshift{0.990609in}{1.420279in}%
\pgfsys@useobject{currentmarker}{}%
\end{pgfscope}%
\begin{pgfscope}%
\pgfsys@transformshift{1.011509in}{1.427179in}%
\pgfsys@useobject{currentmarker}{}%
\end{pgfscope}%
\begin{pgfscope}%
\pgfsys@transformshift{1.031648in}{1.423918in}%
\pgfsys@useobject{currentmarker}{}%
\end{pgfscope}%
\begin{pgfscope}%
\pgfsys@transformshift{1.048347in}{1.433959in}%
\pgfsys@useobject{currentmarker}{}%
\end{pgfscope}%
\begin{pgfscope}%
\pgfsys@transformshift{1.067212in}{1.420108in}%
\pgfsys@useobject{currentmarker}{}%
\end{pgfscope}%
\begin{pgfscope}%
\pgfsys@transformshift{1.088014in}{1.427578in}%
\pgfsys@useobject{currentmarker}{}%
\end{pgfscope}%
\begin{pgfscope}%
\pgfsys@transformshift{1.108062in}{1.420305in}%
\pgfsys@useobject{currentmarker}{}%
\end{pgfscope}%
\begin{pgfscope}%
\pgfsys@transformshift{1.127410in}{1.418129in}%
\pgfsys@useobject{currentmarker}{}%
\end{pgfscope}%
\begin{pgfscope}%
\pgfsys@transformshift{1.146105in}{1.417874in}%
\pgfsys@useobject{currentmarker}{}%
\end{pgfscope}%
\begin{pgfscope}%
\pgfsys@transformshift{1.164189in}{1.391397in}%
\pgfsys@useobject{currentmarker}{}%
\end{pgfscope}%
\begin{pgfscope}%
\pgfsys@transformshift{1.183852in}{1.412448in}%
\pgfsys@useobject{currentmarker}{}%
\end{pgfscope}%
\begin{pgfscope}%
\pgfsys@transformshift{1.202841in}{1.420232in}%
\pgfsys@useobject{currentmarker}{}%
\end{pgfscope}%
\begin{pgfscope}%
\pgfsys@transformshift{1.221200in}{1.411879in}%
\pgfsys@useobject{currentmarker}{}%
\end{pgfscope}%
\begin{pgfscope}%
\pgfsys@transformshift{1.240910in}{1.401599in}%
\pgfsys@useobject{currentmarker}{}%
\end{pgfscope}%
\begin{pgfscope}%
\pgfsys@transformshift{1.259942in}{1.403436in}%
\pgfsys@useobject{currentmarker}{}%
\end{pgfscope}%
\begin{pgfscope}%
\pgfsys@transformshift{1.280148in}{1.409583in}%
\pgfsys@useobject{currentmarker}{}%
\end{pgfscope}%
\begin{pgfscope}%
\pgfsys@transformshift{1.299644in}{1.423714in}%
\pgfsys@useobject{currentmarker}{}%
\end{pgfscope}%
\begin{pgfscope}%
\pgfsys@transformshift{1.318476in}{1.406162in}%
\pgfsys@useobject{currentmarker}{}%
\end{pgfscope}%
\begin{pgfscope}%
\pgfsys@transformshift{1.338315in}{1.425424in}%
\pgfsys@useobject{currentmarker}{}%
\end{pgfscope}%
\begin{pgfscope}%
\pgfsys@transformshift{1.357468in}{1.413868in}%
\pgfsys@useobject{currentmarker}{}%
\end{pgfscope}%
\begin{pgfscope}%
\pgfsys@transformshift{1.375980in}{1.421888in}%
\pgfsys@useobject{currentmarker}{}%
\end{pgfscope}%
\begin{pgfscope}%
\pgfsys@transformshift{1.395361in}{1.409347in}%
\pgfsys@useobject{currentmarker}{}%
\end{pgfscope}%
\begin{pgfscope}%
\pgfsys@transformshift{1.415500in}{1.404068in}%
\pgfsys@useobject{currentmarker}{}%
\end{pgfscope}%
\begin{pgfscope}%
\pgfsys@transformshift{1.434933in}{1.407810in}%
\pgfsys@useobject{currentmarker}{}%
\end{pgfscope}%
\begin{pgfscope}%
\pgfsys@transformshift{1.453707in}{1.422070in}%
\pgfsys@useobject{currentmarker}{}%
\end{pgfscope}%
\begin{pgfscope}%
\pgfsys@transformshift{1.471866in}{1.400545in}%
\pgfsys@useobject{currentmarker}{}%
\end{pgfscope}%
\begin{pgfscope}%
\pgfsys@transformshift{1.490682in}{1.413517in}%
\pgfsys@useobject{currentmarker}{}%
\end{pgfscope}%
\begin{pgfscope}%
\pgfsys@transformshift{1.510073in}{1.413115in}%
\pgfsys@useobject{currentmarker}{}%
\end{pgfscope}%
\begin{pgfscope}%
\pgfsys@transformshift{1.529957in}{1.407780in}%
\pgfsys@useobject{currentmarker}{}%
\end{pgfscope}%
\begin{pgfscope}%
\pgfsys@transformshift{1.549152in}{1.411188in}%
\pgfsys@useobject{currentmarker}{}%
\end{pgfscope}%
\begin{pgfscope}%
\pgfsys@transformshift{1.567704in}{1.414339in}%
\pgfsys@useobject{currentmarker}{}%
\end{pgfscope}%
\begin{pgfscope}%
\pgfsys@transformshift{1.586693in}{1.410853in}%
\pgfsys@useobject{currentmarker}{}%
\end{pgfscope}%
\begin{pgfscope}%
\pgfsys@transformshift{1.606054in}{1.408494in}%
\pgfsys@useobject{currentmarker}{}%
\end{pgfscope}%
\begin{pgfscope}%
\pgfsys@transformshift{1.625729in}{1.397773in}%
\pgfsys@useobject{currentmarker}{}%
\end{pgfscope}%
\begin{pgfscope}%
\pgfsys@transformshift{1.644729in}{1.420874in}%
\pgfsys@useobject{currentmarker}{}%
\end{pgfscope}%
\begin{pgfscope}%
\pgfsys@transformshift{1.664000in}{1.417574in}%
\pgfsys@useobject{currentmarker}{}%
\end{pgfscope}%
\begin{pgfscope}%
\pgfsys@transformshift{1.683496in}{1.421409in}%
\pgfsys@useobject{currentmarker}{}%
\end{pgfscope}%
\begin{pgfscope}%
\pgfsys@transformshift{1.702328in}{1.409679in}%
\pgfsys@useobject{currentmarker}{}%
\end{pgfscope}%
\begin{pgfscope}%
\pgfsys@transformshift{1.721354in}{1.405811in}%
\pgfsys@useobject{currentmarker}{}%
\end{pgfscope}%
\begin{pgfscope}%
\pgfsys@transformshift{1.741320in}{1.420658in}%
\pgfsys@useobject{currentmarker}{}%
\end{pgfscope}%
\begin{pgfscope}%
\pgfsys@transformshift{1.760590in}{1.402368in}%
\pgfsys@useobject{currentmarker}{}%
\end{pgfscope}%
\begin{pgfscope}%
\pgfsys@transformshift{1.779213in}{1.417545in}%
\pgfsys@useobject{currentmarker}{}%
\end{pgfscope}%
\begin{pgfscope}%
\pgfsys@transformshift{1.798646in}{1.419730in}%
\pgfsys@useobject{currentmarker}{}%
\end{pgfscope}%
\begin{pgfscope}%
\pgfsys@transformshift{1.818103in}{1.402487in}%
\pgfsys@useobject{currentmarker}{}%
\end{pgfscope}%
\begin{pgfscope}%
\pgfsys@transformshift{1.837559in}{1.412616in}%
\pgfsys@useobject{currentmarker}{}%
\end{pgfscope}%
\begin{pgfscope}%
\pgfsys@transformshift{1.856992in}{1.412337in}%
\pgfsys@useobject{currentmarker}{}%
\end{pgfscope}%
\begin{pgfscope}%
\pgfsys@transformshift{1.875766in}{1.410558in}%
\pgfsys@useobject{currentmarker}{}%
\end{pgfscope}%
\begin{pgfscope}%
\pgfsys@transformshift{1.894520in}{1.408102in}%
\pgfsys@useobject{currentmarker}{}%
\end{pgfscope}%
\begin{pgfscope}%
\pgfsys@transformshift{1.913809in}{1.415716in}%
\pgfsys@useobject{currentmarker}{}%
\end{pgfscope}%
\begin{pgfscope}%
\pgfsys@transformshift{1.933559in}{1.419159in}%
\pgfsys@useobject{currentmarker}{}%
\end{pgfscope}%
\begin{pgfscope}%
\pgfsys@transformshift{1.953164in}{1.427989in}%
\pgfsys@useobject{currentmarker}{}%
\end{pgfscope}%
\begin{pgfscope}%
\pgfsys@transformshift{1.972099in}{1.404642in}%
\pgfsys@useobject{currentmarker}{}%
\end{pgfscope}%
\begin{pgfscope}%
\pgfsys@transformshift{1.990907in}{1.415353in}%
\pgfsys@useobject{currentmarker}{}%
\end{pgfscope}%
\begin{pgfscope}%
\pgfsys@transformshift{2.010064in}{1.420134in}%
\pgfsys@useobject{currentmarker}{}%
\end{pgfscope}%
\begin{pgfscope}%
\pgfsys@transformshift{2.029514in}{1.419914in}%
\pgfsys@useobject{currentmarker}{}%
\end{pgfscope}%
\begin{pgfscope}%
\pgfsys@transformshift{2.048754in}{1.410638in}%
\pgfsys@useobject{currentmarker}{}%
\end{pgfscope}%
\begin{pgfscope}%
\pgfsys@transformshift{2.067783in}{1.412715in}%
\pgfsys@useobject{currentmarker}{}%
\end{pgfscope}%
\begin{pgfscope}%
\pgfsys@transformshift{2.087021in}{1.425545in}%
\pgfsys@useobject{currentmarker}{}%
\end{pgfscope}%
\begin{pgfscope}%
\pgfsys@transformshift{2.106425in}{1.415783in}%
\pgfsys@useobject{currentmarker}{}%
\end{pgfscope}%
\begin{pgfscope}%
\pgfsys@transformshift{2.125564in}{1.414950in}%
\pgfsys@useobject{currentmarker}{}%
\end{pgfscope}%
\begin{pgfscope}%
\pgfsys@transformshift{2.144821in}{1.408455in}%
\pgfsys@useobject{currentmarker}{}%
\end{pgfscope}%
\begin{pgfscope}%
\pgfsys@transformshift{2.164162in}{1.419282in}%
\pgfsys@useobject{currentmarker}{}%
\end{pgfscope}%
\begin{pgfscope}%
\pgfsys@transformshift{2.183205in}{1.413404in}%
\pgfsys@useobject{currentmarker}{}%
\end{pgfscope}%
\begin{pgfscope}%
\pgfsys@transformshift{2.202296in}{1.405310in}%
\pgfsys@useobject{currentmarker}{}%
\end{pgfscope}%
\begin{pgfscope}%
\pgfsys@transformshift{2.221741in}{1.419861in}%
\pgfsys@useobject{currentmarker}{}%
\end{pgfscope}%
\begin{pgfscope}%
\pgfsys@transformshift{2.241163in}{1.400520in}%
\pgfsys@useobject{currentmarker}{}%
\end{pgfscope}%
\begin{pgfscope}%
\pgfsys@transformshift{2.260233in}{1.418319in}%
\pgfsys@useobject{currentmarker}{}%
\end{pgfscope}%
\begin{pgfscope}%
\pgfsys@transformshift{2.279560in}{1.422351in}%
\pgfsys@useobject{currentmarker}{}%
\end{pgfscope}%
\begin{pgfscope}%
\pgfsys@transformshift{2.298809in}{1.416865in}%
\pgfsys@useobject{currentmarker}{}%
\end{pgfscope}%
\begin{pgfscope}%
\pgfsys@transformshift{2.317965in}{1.419139in}%
\pgfsys@useobject{currentmarker}{}%
\end{pgfscope}%
\begin{pgfscope}%
\pgfsys@transformshift{2.337284in}{1.427418in}%
\pgfsys@useobject{currentmarker}{}%
\end{pgfscope}%
\begin{pgfscope}%
\pgfsys@transformshift{2.356467in}{1.422218in}%
\pgfsys@useobject{currentmarker}{}%
\end{pgfscope}%
\begin{pgfscope}%
\pgfsys@transformshift{2.375508in}{1.412739in}%
\pgfsys@useobject{currentmarker}{}%
\end{pgfscope}%
\begin{pgfscope}%
\pgfsys@transformshift{2.394640in}{1.415303in}%
\pgfsys@useobject{currentmarker}{}%
\end{pgfscope}%
\begin{pgfscope}%
\pgfsys@transformshift{2.414065in}{1.423615in}%
\pgfsys@useobject{currentmarker}{}%
\end{pgfscope}%
\begin{pgfscope}%
\pgfsys@transformshift{2.433281in}{1.411777in}%
\pgfsys@useobject{currentmarker}{}%
\end{pgfscope}%
\begin{pgfscope}%
\pgfsys@transformshift{2.452504in}{1.418147in}%
\pgfsys@useobject{currentmarker}{}%
\end{pgfscope}%
\begin{pgfscope}%
\pgfsys@transformshift{2.471712in}{1.417963in}%
\pgfsys@useobject{currentmarker}{}%
\end{pgfscope}%
\begin{pgfscope}%
\pgfsys@transformshift{2.490885in}{1.413537in}%
\pgfsys@useobject{currentmarker}{}%
\end{pgfscope}%
\begin{pgfscope}%
\pgfsys@transformshift{2.510199in}{1.410583in}%
\pgfsys@useobject{currentmarker}{}%
\end{pgfscope}%
\begin{pgfscope}%
\pgfsys@transformshift{2.529430in}{1.418610in}%
\pgfsys@useobject{currentmarker}{}%
\end{pgfscope}%
\begin{pgfscope}%
\pgfsys@transformshift{2.548563in}{1.417126in}%
\pgfsys@useobject{currentmarker}{}%
\end{pgfscope}%
\begin{pgfscope}%
\pgfsys@transformshift{2.567762in}{1.417378in}%
\pgfsys@useobject{currentmarker}{}%
\end{pgfscope}%
\begin{pgfscope}%
\pgfsys@transformshift{2.587001in}{1.421773in}%
\pgfsys@useobject{currentmarker}{}%
\end{pgfscope}%
\begin{pgfscope}%
\pgfsys@transformshift{2.606252in}{1.420006in}%
\pgfsys@useobject{currentmarker}{}%
\end{pgfscope}%
\begin{pgfscope}%
\pgfsys@transformshift{2.625492in}{1.417020in}%
\pgfsys@useobject{currentmarker}{}%
\end{pgfscope}%
\begin{pgfscope}%
\pgfsys@transformshift{2.644700in}{1.416669in}%
\pgfsys@useobject{currentmarker}{}%
\end{pgfscope}%
\begin{pgfscope}%
\pgfsys@transformshift{2.664006in}{1.415057in}%
\pgfsys@useobject{currentmarker}{}%
\end{pgfscope}%
\begin{pgfscope}%
\pgfsys@transformshift{2.683234in}{1.421225in}%
\pgfsys@useobject{currentmarker}{}%
\end{pgfscope}%
\begin{pgfscope}%
\pgfsys@transformshift{2.702371in}{1.419912in}%
\pgfsys@useobject{currentmarker}{}%
\end{pgfscope}%
\begin{pgfscope}%
\pgfsys@transformshift{2.721537in}{1.422262in}%
\pgfsys@useobject{currentmarker}{}%
\end{pgfscope}%
\begin{pgfscope}%
\pgfsys@transformshift{2.740836in}{1.421311in}%
\pgfsys@useobject{currentmarker}{}%
\end{pgfscope}%
\begin{pgfscope}%
\pgfsys@transformshift{2.760109in}{1.418036in}%
\pgfsys@useobject{currentmarker}{}%
\end{pgfscope}%
\begin{pgfscope}%
\pgfsys@transformshift{2.779335in}{1.421700in}%
\pgfsys@useobject{currentmarker}{}%
\end{pgfscope}%
\begin{pgfscope}%
\pgfsys@transformshift{2.798498in}{1.416193in}%
\pgfsys@useobject{currentmarker}{}%
\end{pgfscope}%
\begin{pgfscope}%
\pgfsys@transformshift{2.817695in}{1.420415in}%
\pgfsys@useobject{currentmarker}{}%
\end{pgfscope}%
\begin{pgfscope}%
\pgfsys@transformshift{2.836899in}{1.419289in}%
\pgfsys@useobject{currentmarker}{}%
\end{pgfscope}%
\begin{pgfscope}%
\pgfsys@transformshift{2.856089in}{1.419681in}%
\pgfsys@useobject{currentmarker}{}%
\end{pgfscope}%
\begin{pgfscope}%
\pgfsys@transformshift{2.875345in}{1.416717in}%
\pgfsys@useobject{currentmarker}{}%
\end{pgfscope}%
\begin{pgfscope}%
\pgfsys@transformshift{2.894540in}{1.421057in}%
\pgfsys@useobject{currentmarker}{}%
\end{pgfscope}%
\begin{pgfscope}%
\pgfsys@transformshift{2.913754in}{1.419591in}%
\pgfsys@useobject{currentmarker}{}%
\end{pgfscope}%
\begin{pgfscope}%
\pgfsys@transformshift{2.933053in}{1.420413in}%
\pgfsys@useobject{currentmarker}{}%
\end{pgfscope}%
\begin{pgfscope}%
\pgfsys@transformshift{2.952320in}{1.417396in}%
\pgfsys@useobject{currentmarker}{}%
\end{pgfscope}%
\begin{pgfscope}%
\pgfsys@transformshift{2.971534in}{1.424017in}%
\pgfsys@useobject{currentmarker}{}%
\end{pgfscope}%
\begin{pgfscope}%
\pgfsys@transformshift{2.990761in}{1.418178in}%
\pgfsys@useobject{currentmarker}{}%
\end{pgfscope}%
\begin{pgfscope}%
\pgfsys@transformshift{3.009979in}{1.421931in}%
\pgfsys@useobject{currentmarker}{}%
\end{pgfscope}%
\begin{pgfscope}%
\pgfsys@transformshift{3.029165in}{1.420671in}%
\pgfsys@useobject{currentmarker}{}%
\end{pgfscope}%
\begin{pgfscope}%
\pgfsys@transformshift{3.048376in}{1.425369in}%
\pgfsys@useobject{currentmarker}{}%
\end{pgfscope}%
\begin{pgfscope}%
\pgfsys@transformshift{3.067587in}{1.424161in}%
\pgfsys@useobject{currentmarker}{}%
\end{pgfscope}%
\begin{pgfscope}%
\pgfsys@transformshift{3.086776in}{1.423966in}%
\pgfsys@useobject{currentmarker}{}%
\end{pgfscope}%
\begin{pgfscope}%
\pgfsys@transformshift{3.105989in}{1.422719in}%
\pgfsys@useobject{currentmarker}{}%
\end{pgfscope}%
\begin{pgfscope}%
\pgfsys@transformshift{3.125268in}{1.425511in}%
\pgfsys@useobject{currentmarker}{}%
\end{pgfscope}%
\begin{pgfscope}%
\pgfsys@transformshift{3.144521in}{1.423479in}%
\pgfsys@useobject{currentmarker}{}%
\end{pgfscope}%
\begin{pgfscope}%
\pgfsys@transformshift{3.163729in}{1.426171in}%
\pgfsys@useobject{currentmarker}{}%
\end{pgfscope}%
\begin{pgfscope}%
\pgfsys@transformshift{3.182931in}{1.421723in}%
\pgfsys@useobject{currentmarker}{}%
\end{pgfscope}%
\begin{pgfscope}%
\pgfsys@transformshift{3.202164in}{1.425145in}%
\pgfsys@useobject{currentmarker}{}%
\end{pgfscope}%
\begin{pgfscope}%
\pgfsys@transformshift{3.221401in}{1.425017in}%
\pgfsys@useobject{currentmarker}{}%
\end{pgfscope}%
\begin{pgfscope}%
\pgfsys@transformshift{3.240621in}{1.425965in}%
\pgfsys@useobject{currentmarker}{}%
\end{pgfscope}%
\begin{pgfscope}%
\pgfsys@transformshift{3.259854in}{1.426594in}%
\pgfsys@useobject{currentmarker}{}%
\end{pgfscope}%
\begin{pgfscope}%
\pgfsys@transformshift{3.279076in}{1.426754in}%
\pgfsys@useobject{currentmarker}{}%
\end{pgfscope}%
\begin{pgfscope}%
\pgfsys@transformshift{3.298314in}{1.430473in}%
\pgfsys@useobject{currentmarker}{}%
\end{pgfscope}%
\begin{pgfscope}%
\pgfsys@transformshift{3.317543in}{1.428202in}%
\pgfsys@useobject{currentmarker}{}%
\end{pgfscope}%
\begin{pgfscope}%
\pgfsys@transformshift{3.336744in}{1.428849in}%
\pgfsys@useobject{currentmarker}{}%
\end{pgfscope}%
\begin{pgfscope}%
\pgfsys@transformshift{3.355938in}{1.427126in}%
\pgfsys@useobject{currentmarker}{}%
\end{pgfscope}%
\begin{pgfscope}%
\pgfsys@transformshift{3.375147in}{1.428169in}%
\pgfsys@useobject{currentmarker}{}%
\end{pgfscope}%
\begin{pgfscope}%
\pgfsys@transformshift{3.394386in}{1.429502in}%
\pgfsys@useobject{currentmarker}{}%
\end{pgfscope}%
\begin{pgfscope}%
\pgfsys@transformshift{3.413630in}{1.428474in}%
\pgfsys@useobject{currentmarker}{}%
\end{pgfscope}%
\begin{pgfscope}%
\pgfsys@transformshift{3.432857in}{1.431914in}%
\pgfsys@useobject{currentmarker}{}%
\end{pgfscope}%
\begin{pgfscope}%
\pgfsys@transformshift{3.452082in}{1.430597in}%
\pgfsys@useobject{currentmarker}{}%
\end{pgfscope}%
\begin{pgfscope}%
\pgfsys@transformshift{3.471284in}{1.430566in}%
\pgfsys@useobject{currentmarker}{}%
\end{pgfscope}%
\begin{pgfscope}%
\pgfsys@transformshift{3.490508in}{1.428380in}%
\pgfsys@useobject{currentmarker}{}%
\end{pgfscope}%
\begin{pgfscope}%
\pgfsys@transformshift{3.509732in}{1.427936in}%
\pgfsys@useobject{currentmarker}{}%
\end{pgfscope}%
\begin{pgfscope}%
\pgfsys@transformshift{3.528933in}{1.433471in}%
\pgfsys@useobject{currentmarker}{}%
\end{pgfscope}%
\begin{pgfscope}%
\pgfsys@transformshift{3.548151in}{1.431974in}%
\pgfsys@useobject{currentmarker}{}%
\end{pgfscope}%
\begin{pgfscope}%
\pgfsys@transformshift{3.567393in}{1.432985in}%
\pgfsys@useobject{currentmarker}{}%
\end{pgfscope}%
\begin{pgfscope}%
\pgfsys@transformshift{3.586633in}{1.433360in}%
\pgfsys@useobject{currentmarker}{}%
\end{pgfscope}%
\begin{pgfscope}%
\pgfsys@transformshift{3.605849in}{1.436407in}%
\pgfsys@useobject{currentmarker}{}%
\end{pgfscope}%
\begin{pgfscope}%
\pgfsys@transformshift{3.625075in}{1.435924in}%
\pgfsys@useobject{currentmarker}{}%
\end{pgfscope}%
\begin{pgfscope}%
\pgfsys@transformshift{3.644287in}{1.433105in}%
\pgfsys@useobject{currentmarker}{}%
\end{pgfscope}%
\begin{pgfscope}%
\pgfsys@transformshift{3.663514in}{1.434069in}%
\pgfsys@useobject{currentmarker}{}%
\end{pgfscope}%
\begin{pgfscope}%
\pgfsys@transformshift{3.682755in}{1.434497in}%
\pgfsys@useobject{currentmarker}{}%
\end{pgfscope}%
\begin{pgfscope}%
\pgfsys@transformshift{3.701964in}{1.435812in}%
\pgfsys@useobject{currentmarker}{}%
\end{pgfscope}%
\begin{pgfscope}%
\pgfsys@transformshift{3.721189in}{1.435402in}%
\pgfsys@useobject{currentmarker}{}%
\end{pgfscope}%
\begin{pgfscope}%
\pgfsys@transformshift{3.740406in}{1.434185in}%
\pgfsys@useobject{currentmarker}{}%
\end{pgfscope}%
\begin{pgfscope}%
\pgfsys@transformshift{3.759614in}{1.437118in}%
\pgfsys@useobject{currentmarker}{}%
\end{pgfscope}%
\begin{pgfscope}%
\pgfsys@transformshift{3.778852in}{1.436114in}%
\pgfsys@useobject{currentmarker}{}%
\end{pgfscope}%
\begin{pgfscope}%
\pgfsys@transformshift{3.798076in}{1.438262in}%
\pgfsys@useobject{currentmarker}{}%
\end{pgfscope}%
\begin{pgfscope}%
\pgfsys@transformshift{3.817301in}{1.435053in}%
\pgfsys@useobject{currentmarker}{}%
\end{pgfscope}%
\begin{pgfscope}%
\pgfsys@transformshift{3.836523in}{1.437356in}%
\pgfsys@useobject{currentmarker}{}%
\end{pgfscope}%
\begin{pgfscope}%
\pgfsys@transformshift{3.855739in}{1.438027in}%
\pgfsys@useobject{currentmarker}{}%
\end{pgfscope}%
\begin{pgfscope}%
\pgfsys@transformshift{3.865881in}{1.440052in}%
\pgfsys@useobject{currentmarker}{}%
\end{pgfscope}%
\end{pgfscope}%
\begin{pgfscope}%
\pgfpathrectangle{\pgfqpoint{0.664463in}{0.417642in}}{\pgfqpoint{3.353867in}{2.050688in}}%
\pgfusepath{clip}%
\pgfsetbuttcap%
\pgfsetroundjoin%
\pgfsetlinewidth{1.505625pt}%
\definecolor{currentstroke}{rgb}{0.000000,0.447059,0.698039}%
\pgfsetstrokecolor{currentstroke}%
\pgfsetdash{{5.550000pt}{2.400000pt}}{0.000000pt}%
\pgfpathmoveto{\pgfqpoint{0.816911in}{0.791086in}}%
\pgfpathlineto{\pgfqpoint{3.865881in}{0.791086in}}%
\pgfpathlineto{\pgfqpoint{3.865881in}{0.791086in}}%
\pgfusepath{stroke}%
\end{pgfscope}%
\begin{pgfscope}%
\pgfpathrectangle{\pgfqpoint{0.664463in}{0.417642in}}{\pgfqpoint{3.353867in}{2.050688in}}%
\pgfusepath{clip}%
\pgfsetbuttcap%
\pgfsetroundjoin%
\pgfsetlinewidth{1.505625pt}%
\definecolor{currentstroke}{rgb}{0.000000,0.619608,0.450980}%
\pgfsetstrokecolor{currentstroke}%
\pgfsetdash{{5.550000pt}{2.400000pt}}{0.000000pt}%
\pgfpathmoveto{\pgfqpoint{0.816911in}{2.375117in}}%
\pgfpathlineto{\pgfqpoint{3.355938in}{0.510855in}}%
\pgfpathlineto{\pgfqpoint{3.355938in}{0.510855in}}%
\pgfusepath{stroke}%
\end{pgfscope}%
\begin{pgfscope}%
\pgfsetrectcap%
\pgfsetmiterjoin%
\pgfsetlinewidth{0.803000pt}%
\definecolor{currentstroke}{rgb}{0.000000,0.000000,0.000000}%
\pgfsetstrokecolor{currentstroke}%
\pgfsetdash{}{0pt}%
\pgfpathmoveto{\pgfqpoint{0.664463in}{0.417642in}}%
\pgfpathlineto{\pgfqpoint{0.664463in}{2.468330in}}%
\pgfusepath{stroke}%
\end{pgfscope}%
\begin{pgfscope}%
\pgfsetrectcap%
\pgfsetmiterjoin%
\pgfsetlinewidth{0.803000pt}%
\definecolor{currentstroke}{rgb}{0.000000,0.000000,0.000000}%
\pgfsetstrokecolor{currentstroke}%
\pgfsetdash{}{0pt}%
\pgfpathmoveto{\pgfqpoint{4.018330in}{0.417642in}}%
\pgfpathlineto{\pgfqpoint{4.018330in}{2.468330in}}%
\pgfusepath{stroke}%
\end{pgfscope}%
\begin{pgfscope}%
\pgfsetrectcap%
\pgfsetmiterjoin%
\pgfsetlinewidth{0.803000pt}%
\definecolor{currentstroke}{rgb}{0.000000,0.000000,0.000000}%
\pgfsetstrokecolor{currentstroke}%
\pgfsetdash{}{0pt}%
\pgfpathmoveto{\pgfqpoint{0.664463in}{0.417642in}}%
\pgfpathlineto{\pgfqpoint{4.018330in}{0.417642in}}%
\pgfusepath{stroke}%
\end{pgfscope}%
\begin{pgfscope}%
\pgfsetrectcap%
\pgfsetmiterjoin%
\pgfsetlinewidth{0.803000pt}%
\definecolor{currentstroke}{rgb}{0.000000,0.000000,0.000000}%
\pgfsetstrokecolor{currentstroke}%
\pgfsetdash{}{0pt}%
\pgfpathmoveto{\pgfqpoint{0.664463in}{2.468330in}}%
\pgfpathlineto{\pgfqpoint{4.018330in}{2.468330in}}%
\pgfusepath{stroke}%
\end{pgfscope}%
\begin{pgfscope}%
\pgfsetbuttcap%
\pgfsetmiterjoin%
\definecolor{currentfill}{rgb}{1.000000,1.000000,1.000000}%
\pgfsetfillcolor{currentfill}%
\pgfsetfillopacity{0.800000}%
\pgfsetlinewidth{1.003750pt}%
\definecolor{currentstroke}{rgb}{0.800000,0.800000,0.800000}%
\pgfsetstrokecolor{currentstroke}%
\pgfsetstrokeopacity{0.800000}%
\pgfsetdash{}{0pt}%
\pgfpathmoveto{\pgfqpoint{2.940219in}{2.069664in}}%
\pgfpathlineto{\pgfqpoint{3.940552in}{2.069664in}}%
\pgfpathquadraticcurveto{\pgfqpoint{3.962774in}{2.069664in}}{\pgfqpoint{3.962774in}{2.091886in}}%
\pgfpathlineto{\pgfqpoint{3.962774in}{2.390552in}}%
\pgfpathquadraticcurveto{\pgfqpoint{3.962774in}{2.412774in}}{\pgfqpoint{3.940552in}{2.412774in}}%
\pgfpathlineto{\pgfqpoint{2.940219in}{2.412774in}}%
\pgfpathquadraticcurveto{\pgfqpoint{2.917997in}{2.412774in}}{\pgfqpoint{2.917997in}{2.390552in}}%
\pgfpathlineto{\pgfqpoint{2.917997in}{2.091886in}}%
\pgfpathquadraticcurveto{\pgfqpoint{2.917997in}{2.069664in}}{\pgfqpoint{2.940219in}{2.069664in}}%
\pgfpathlineto{\pgfqpoint{2.940219in}{2.069664in}}%
\pgfpathclose%
\pgfusepath{stroke,fill}%
\end{pgfscope}%
\begin{pgfscope}%
\pgfsetbuttcap%
\pgfsetroundjoin%
\pgfsetlinewidth{1.505625pt}%
\definecolor{currentstroke}{rgb}{0.000000,0.447059,0.698039}%
\pgfsetstrokecolor{currentstroke}%
\pgfsetdash{{5.550000pt}{2.400000pt}}{0.000000pt}%
\pgfpathmoveto{\pgfqpoint{2.962441in}{2.329441in}}%
\pgfpathlineto{\pgfqpoint{3.073552in}{2.329441in}}%
\pgfpathlineto{\pgfqpoint{3.184663in}{2.329441in}}%
\pgfusepath{stroke}%
\end{pgfscope}%
\begin{pgfscope}%
\definecolor{textcolor}{rgb}{0.000000,0.000000,0.000000}%
\pgfsetstrokecolor{textcolor}%
\pgfsetfillcolor{textcolor}%
\pgftext[x=3.273552in,y=2.290552in,left,base]{\color{textcolor}\rmfamily\fontsize{8.000000}{9.600000}\selectfont White noise}%
\end{pgfscope}%
\begin{pgfscope}%
\pgfsetbuttcap%
\pgfsetroundjoin%
\pgfsetlinewidth{1.505625pt}%
\definecolor{currentstroke}{rgb}{0.000000,0.619608,0.450980}%
\pgfsetstrokecolor{currentstroke}%
\pgfsetdash{{5.550000pt}{2.400000pt}}{0.000000pt}%
\pgfpathmoveto{\pgfqpoint{2.962441in}{2.174552in}}%
\pgfpathlineto{\pgfqpoint{3.073552in}{2.174552in}}%
\pgfpathlineto{\pgfqpoint{3.184663in}{2.174552in}}%
\pgfusepath{stroke}%
\end{pgfscope}%
\begin{pgfscope}%
\definecolor{textcolor}{rgb}{0.000000,0.000000,0.000000}%
\pgfsetstrokecolor{textcolor}%
\pgfsetfillcolor{textcolor}%
\pgftext[x=3.273552in,y=2.135663in,left,base]{\color{textcolor}\rmfamily\fontsize{8.000000}{9.600000}\selectfont Flicker noise}%
\end{pgfscope}%
\end{pgfpicture}%
\makeatother%
\endgroup%
% data/simulations/sim_autozero.py
    \caption{Simulated power spectrum of a Keysight \device{3458A} with autozeroing applied. The dashed lines denote the noise present prior to applying the autozero algorithm. The line frequency is \qty{50}{\Hz}.}
    \label{fig:autozero_psd}
\end{figure}

Nonetheless, down to very low frequencies, $f^{-1}$ noise is effectively suppressed and the spectral density is almost perfectly flat. For reference, the dashed lines show the noise content that was present in the dataset prior to autozeroing, which is less white noise, but far more flicker noise.
\begin{figure}[hb]
    \centering
    %% Creator: Matplotlib, PGF backend
%%
%% To include the figure in your LaTeX document, write
%%   \input{<filename>.pgf}
%%
%% Make sure the required packages are loaded in your preamble
%%   \usepackage{pgf}
%%
%% Also ensure that all the required font packages are loaded; for instance,
%% the lmodern package is sometimes necessary when using math font.
%%   \usepackage{lmodern}
%%
%% Figures using additional raster images can only be included by \input if
%% they are in the same directory as the main LaTeX file. For loading figures
%% from other directories you can use the `import` package
%%   \usepackage{import}
%%
%% and then include the figures with
%%   \import{<path to file>}{<filename>.pgf}
%%
%% Matplotlib used the following preamble
%%   \usepackage{siunitx}
%%   \sisetup{per-mode = symbol}%
%%   \usepackage{fontspec}
%%   \makeatletter\@ifpackageloaded{underscore}{}{\usepackage[strings]{underscore}}\makeatother
%%
\begingroup%
\makeatletter%
\begin{pgfpicture}%
\pgfpathrectangle{\pgfpointorigin}{\pgfqpoint{4.068242in}{2.514312in}}%
\pgfusepath{use as bounding box, clip}%
\begin{pgfscope}%
\pgfsetbuttcap%
\pgfsetmiterjoin%
\definecolor{currentfill}{rgb}{1.000000,1.000000,1.000000}%
\pgfsetfillcolor{currentfill}%
\pgfsetlinewidth{0.000000pt}%
\definecolor{currentstroke}{rgb}{1.000000,1.000000,1.000000}%
\pgfsetstrokecolor{currentstroke}%
\pgfsetdash{}{0pt}%
\pgfpathmoveto{\pgfqpoint{0.000000in}{0.000000in}}%
\pgfpathlineto{\pgfqpoint{4.068242in}{0.000000in}}%
\pgfpathlineto{\pgfqpoint{4.068242in}{2.514312in}}%
\pgfpathlineto{\pgfqpoint{0.000000in}{2.514312in}}%
\pgfpathlineto{\pgfqpoint{0.000000in}{0.000000in}}%
\pgfpathclose%
\pgfusepath{fill}%
\end{pgfscope}%
\begin{pgfscope}%
\pgfsetbuttcap%
\pgfsetmiterjoin%
\definecolor{currentfill}{rgb}{1.000000,1.000000,1.000000}%
\pgfsetfillcolor{currentfill}%
\pgfsetlinewidth{0.000000pt}%
\definecolor{currentstroke}{rgb}{0.000000,0.000000,0.000000}%
\pgfsetstrokecolor{currentstroke}%
\pgfsetstrokeopacity{0.000000}%
\pgfsetdash{}{0pt}%
\pgfpathmoveto{\pgfqpoint{0.589510in}{0.417642in}}%
\pgfpathlineto{\pgfqpoint{4.026572in}{0.417642in}}%
\pgfpathlineto{\pgfqpoint{4.026572in}{2.472642in}}%
\pgfpathlineto{\pgfqpoint{0.589510in}{2.472642in}}%
\pgfpathlineto{\pgfqpoint{0.589510in}{0.417642in}}%
\pgfpathclose%
\pgfusepath{fill}%
\end{pgfscope}%
\begin{pgfscope}%
\pgfpathrectangle{\pgfqpoint{0.589510in}{0.417642in}}{\pgfqpoint{3.437062in}{2.055000in}}%
\pgfusepath{clip}%
\pgfsetrectcap%
\pgfsetroundjoin%
\pgfsetlinewidth{0.803000pt}%
\definecolor{currentstroke}{rgb}{0.450000,0.450000,0.450000}%
\pgfsetstrokecolor{currentstroke}%
\pgfsetdash{}{0pt}%
\pgfpathmoveto{\pgfqpoint{0.988710in}{0.417642in}}%
\pgfpathlineto{\pgfqpoint{0.988710in}{2.472642in}}%
\pgfusepath{stroke}%
\end{pgfscope}%
\begin{pgfscope}%
\pgfsetbuttcap%
\pgfsetroundjoin%
\definecolor{currentfill}{rgb}{0.000000,0.000000,0.000000}%
\pgfsetfillcolor{currentfill}%
\pgfsetlinewidth{0.803000pt}%
\definecolor{currentstroke}{rgb}{0.000000,0.000000,0.000000}%
\pgfsetstrokecolor{currentstroke}%
\pgfsetdash{}{0pt}%
\pgfsys@defobject{currentmarker}{\pgfqpoint{0.000000in}{-0.048611in}}{\pgfqpoint{0.000000in}{0.000000in}}{%
\pgfpathmoveto{\pgfqpoint{0.000000in}{0.000000in}}%
\pgfpathlineto{\pgfqpoint{0.000000in}{-0.048611in}}%
\pgfusepath{stroke,fill}%
}%
\begin{pgfscope}%
\pgfsys@transformshift{0.988710in}{0.417642in}%
\pgfsys@useobject{currentmarker}{}%
\end{pgfscope}%
\end{pgfscope}%
\begin{pgfscope}%
\definecolor{textcolor}{rgb}{0.000000,0.000000,0.000000}%
\pgfsetstrokecolor{textcolor}%
\pgfsetfillcolor{textcolor}%
\pgftext[x=0.988710in,y=0.320420in,,top]{\color{textcolor}\rmfamily\fontsize{8.000000}{9.600000}\selectfont \(\displaystyle {10^{0}}\)}%
\end{pgfscope}%
\begin{pgfscope}%
\pgfpathrectangle{\pgfqpoint{0.589510in}{0.417642in}}{\pgfqpoint{3.437062in}{2.055000in}}%
\pgfusepath{clip}%
\pgfsetrectcap%
\pgfsetroundjoin%
\pgfsetlinewidth{0.803000pt}%
\definecolor{currentstroke}{rgb}{0.450000,0.450000,0.450000}%
\pgfsetstrokecolor{currentstroke}%
\pgfsetdash{}{0pt}%
\pgfpathmoveto{\pgfqpoint{1.599281in}{0.417642in}}%
\pgfpathlineto{\pgfqpoint{1.599281in}{2.472642in}}%
\pgfusepath{stroke}%
\end{pgfscope}%
\begin{pgfscope}%
\pgfsetbuttcap%
\pgfsetroundjoin%
\definecolor{currentfill}{rgb}{0.000000,0.000000,0.000000}%
\pgfsetfillcolor{currentfill}%
\pgfsetlinewidth{0.803000pt}%
\definecolor{currentstroke}{rgb}{0.000000,0.000000,0.000000}%
\pgfsetstrokecolor{currentstroke}%
\pgfsetdash{}{0pt}%
\pgfsys@defobject{currentmarker}{\pgfqpoint{0.000000in}{-0.048611in}}{\pgfqpoint{0.000000in}{0.000000in}}{%
\pgfpathmoveto{\pgfqpoint{0.000000in}{0.000000in}}%
\pgfpathlineto{\pgfqpoint{0.000000in}{-0.048611in}}%
\pgfusepath{stroke,fill}%
}%
\begin{pgfscope}%
\pgfsys@transformshift{1.599281in}{0.417642in}%
\pgfsys@useobject{currentmarker}{}%
\end{pgfscope}%
\end{pgfscope}%
\begin{pgfscope}%
\definecolor{textcolor}{rgb}{0.000000,0.000000,0.000000}%
\pgfsetstrokecolor{textcolor}%
\pgfsetfillcolor{textcolor}%
\pgftext[x=1.599281in,y=0.320420in,,top]{\color{textcolor}\rmfamily\fontsize{8.000000}{9.600000}\selectfont \(\displaystyle {10^{1}}\)}%
\end{pgfscope}%
\begin{pgfscope}%
\pgfpathrectangle{\pgfqpoint{0.589510in}{0.417642in}}{\pgfqpoint{3.437062in}{2.055000in}}%
\pgfusepath{clip}%
\pgfsetrectcap%
\pgfsetroundjoin%
\pgfsetlinewidth{0.803000pt}%
\definecolor{currentstroke}{rgb}{0.450000,0.450000,0.450000}%
\pgfsetstrokecolor{currentstroke}%
\pgfsetdash{}{0pt}%
\pgfpathmoveto{\pgfqpoint{2.209852in}{0.417642in}}%
\pgfpathlineto{\pgfqpoint{2.209852in}{2.472642in}}%
\pgfusepath{stroke}%
\end{pgfscope}%
\begin{pgfscope}%
\pgfsetbuttcap%
\pgfsetroundjoin%
\definecolor{currentfill}{rgb}{0.000000,0.000000,0.000000}%
\pgfsetfillcolor{currentfill}%
\pgfsetlinewidth{0.803000pt}%
\definecolor{currentstroke}{rgb}{0.000000,0.000000,0.000000}%
\pgfsetstrokecolor{currentstroke}%
\pgfsetdash{}{0pt}%
\pgfsys@defobject{currentmarker}{\pgfqpoint{0.000000in}{-0.048611in}}{\pgfqpoint{0.000000in}{0.000000in}}{%
\pgfpathmoveto{\pgfqpoint{0.000000in}{0.000000in}}%
\pgfpathlineto{\pgfqpoint{0.000000in}{-0.048611in}}%
\pgfusepath{stroke,fill}%
}%
\begin{pgfscope}%
\pgfsys@transformshift{2.209852in}{0.417642in}%
\pgfsys@useobject{currentmarker}{}%
\end{pgfscope}%
\end{pgfscope}%
\begin{pgfscope}%
\definecolor{textcolor}{rgb}{0.000000,0.000000,0.000000}%
\pgfsetstrokecolor{textcolor}%
\pgfsetfillcolor{textcolor}%
\pgftext[x=2.209852in,y=0.320420in,,top]{\color{textcolor}\rmfamily\fontsize{8.000000}{9.600000}\selectfont \(\displaystyle {10^{2}}\)}%
\end{pgfscope}%
\begin{pgfscope}%
\pgfpathrectangle{\pgfqpoint{0.589510in}{0.417642in}}{\pgfqpoint{3.437062in}{2.055000in}}%
\pgfusepath{clip}%
\pgfsetrectcap%
\pgfsetroundjoin%
\pgfsetlinewidth{0.803000pt}%
\definecolor{currentstroke}{rgb}{0.450000,0.450000,0.450000}%
\pgfsetstrokecolor{currentstroke}%
\pgfsetdash{}{0pt}%
\pgfpathmoveto{\pgfqpoint{2.820422in}{0.417642in}}%
\pgfpathlineto{\pgfqpoint{2.820422in}{2.472642in}}%
\pgfusepath{stroke}%
\end{pgfscope}%
\begin{pgfscope}%
\pgfsetbuttcap%
\pgfsetroundjoin%
\definecolor{currentfill}{rgb}{0.000000,0.000000,0.000000}%
\pgfsetfillcolor{currentfill}%
\pgfsetlinewidth{0.803000pt}%
\definecolor{currentstroke}{rgb}{0.000000,0.000000,0.000000}%
\pgfsetstrokecolor{currentstroke}%
\pgfsetdash{}{0pt}%
\pgfsys@defobject{currentmarker}{\pgfqpoint{0.000000in}{-0.048611in}}{\pgfqpoint{0.000000in}{0.000000in}}{%
\pgfpathmoveto{\pgfqpoint{0.000000in}{0.000000in}}%
\pgfpathlineto{\pgfqpoint{0.000000in}{-0.048611in}}%
\pgfusepath{stroke,fill}%
}%
\begin{pgfscope}%
\pgfsys@transformshift{2.820422in}{0.417642in}%
\pgfsys@useobject{currentmarker}{}%
\end{pgfscope}%
\end{pgfscope}%
\begin{pgfscope}%
\definecolor{textcolor}{rgb}{0.000000,0.000000,0.000000}%
\pgfsetstrokecolor{textcolor}%
\pgfsetfillcolor{textcolor}%
\pgftext[x=2.820422in,y=0.320420in,,top]{\color{textcolor}\rmfamily\fontsize{8.000000}{9.600000}\selectfont \(\displaystyle {10^{3}}\)}%
\end{pgfscope}%
\begin{pgfscope}%
\pgfpathrectangle{\pgfqpoint{0.589510in}{0.417642in}}{\pgfqpoint{3.437062in}{2.055000in}}%
\pgfusepath{clip}%
\pgfsetrectcap%
\pgfsetroundjoin%
\pgfsetlinewidth{0.803000pt}%
\definecolor{currentstroke}{rgb}{0.450000,0.450000,0.450000}%
\pgfsetstrokecolor{currentstroke}%
\pgfsetdash{}{0pt}%
\pgfpathmoveto{\pgfqpoint{3.430993in}{0.417642in}}%
\pgfpathlineto{\pgfqpoint{3.430993in}{2.472642in}}%
\pgfusepath{stroke}%
\end{pgfscope}%
\begin{pgfscope}%
\pgfsetbuttcap%
\pgfsetroundjoin%
\definecolor{currentfill}{rgb}{0.000000,0.000000,0.000000}%
\pgfsetfillcolor{currentfill}%
\pgfsetlinewidth{0.803000pt}%
\definecolor{currentstroke}{rgb}{0.000000,0.000000,0.000000}%
\pgfsetstrokecolor{currentstroke}%
\pgfsetdash{}{0pt}%
\pgfsys@defobject{currentmarker}{\pgfqpoint{0.000000in}{-0.048611in}}{\pgfqpoint{0.000000in}{0.000000in}}{%
\pgfpathmoveto{\pgfqpoint{0.000000in}{0.000000in}}%
\pgfpathlineto{\pgfqpoint{0.000000in}{-0.048611in}}%
\pgfusepath{stroke,fill}%
}%
\begin{pgfscope}%
\pgfsys@transformshift{3.430993in}{0.417642in}%
\pgfsys@useobject{currentmarker}{}%
\end{pgfscope}%
\end{pgfscope}%
\begin{pgfscope}%
\definecolor{textcolor}{rgb}{0.000000,0.000000,0.000000}%
\pgfsetstrokecolor{textcolor}%
\pgfsetfillcolor{textcolor}%
\pgftext[x=3.430993in,y=0.320420in,,top]{\color{textcolor}\rmfamily\fontsize{8.000000}{9.600000}\selectfont \(\displaystyle {10^{4}}\)}%
\end{pgfscope}%
\begin{pgfscope}%
\pgfpathrectangle{\pgfqpoint{0.589510in}{0.417642in}}{\pgfqpoint{3.437062in}{2.055000in}}%
\pgfusepath{clip}%
\pgfsetrectcap%
\pgfsetroundjoin%
\pgfsetlinewidth{0.803000pt}%
\definecolor{currentstroke}{rgb}{0.850000,0.850000,0.850000}%
\pgfsetstrokecolor{currentstroke}%
\pgfsetdash{}{0pt}%
\pgfpathmoveto{\pgfqpoint{0.669456in}{0.417642in}}%
\pgfpathlineto{\pgfqpoint{0.669456in}{2.472642in}}%
\pgfusepath{stroke}%
\end{pgfscope}%
\begin{pgfscope}%
\pgfsetbuttcap%
\pgfsetroundjoin%
\definecolor{currentfill}{rgb}{0.000000,0.000000,0.000000}%
\pgfsetfillcolor{currentfill}%
\pgfsetlinewidth{0.602250pt}%
\definecolor{currentstroke}{rgb}{0.000000,0.000000,0.000000}%
\pgfsetstrokecolor{currentstroke}%
\pgfsetdash{}{0pt}%
\pgfsys@defobject{currentmarker}{\pgfqpoint{0.000000in}{-0.027778in}}{\pgfqpoint{0.000000in}{0.000000in}}{%
\pgfpathmoveto{\pgfqpoint{0.000000in}{0.000000in}}%
\pgfpathlineto{\pgfqpoint{0.000000in}{-0.027778in}}%
\pgfusepath{stroke,fill}%
}%
\begin{pgfscope}%
\pgfsys@transformshift{0.669456in}{0.417642in}%
\pgfsys@useobject{currentmarker}{}%
\end{pgfscope}%
\end{pgfscope}%
\begin{pgfscope}%
\pgfpathrectangle{\pgfqpoint{0.589510in}{0.417642in}}{\pgfqpoint{3.437062in}{2.055000in}}%
\pgfusepath{clip}%
\pgfsetrectcap%
\pgfsetroundjoin%
\pgfsetlinewidth{0.803000pt}%
\definecolor{currentstroke}{rgb}{0.850000,0.850000,0.850000}%
\pgfsetstrokecolor{currentstroke}%
\pgfsetdash{}{0pt}%
\pgfpathmoveto{\pgfqpoint{0.745740in}{0.417642in}}%
\pgfpathlineto{\pgfqpoint{0.745740in}{2.472642in}}%
\pgfusepath{stroke}%
\end{pgfscope}%
\begin{pgfscope}%
\pgfsetbuttcap%
\pgfsetroundjoin%
\definecolor{currentfill}{rgb}{0.000000,0.000000,0.000000}%
\pgfsetfillcolor{currentfill}%
\pgfsetlinewidth{0.602250pt}%
\definecolor{currentstroke}{rgb}{0.000000,0.000000,0.000000}%
\pgfsetstrokecolor{currentstroke}%
\pgfsetdash{}{0pt}%
\pgfsys@defobject{currentmarker}{\pgfqpoint{0.000000in}{-0.027778in}}{\pgfqpoint{0.000000in}{0.000000in}}{%
\pgfpathmoveto{\pgfqpoint{0.000000in}{0.000000in}}%
\pgfpathlineto{\pgfqpoint{0.000000in}{-0.027778in}}%
\pgfusepath{stroke,fill}%
}%
\begin{pgfscope}%
\pgfsys@transformshift{0.745740in}{0.417642in}%
\pgfsys@useobject{currentmarker}{}%
\end{pgfscope}%
\end{pgfscope}%
\begin{pgfscope}%
\pgfpathrectangle{\pgfqpoint{0.589510in}{0.417642in}}{\pgfqpoint{3.437062in}{2.055000in}}%
\pgfusepath{clip}%
\pgfsetrectcap%
\pgfsetroundjoin%
\pgfsetlinewidth{0.803000pt}%
\definecolor{currentstroke}{rgb}{0.850000,0.850000,0.850000}%
\pgfsetstrokecolor{currentstroke}%
\pgfsetdash{}{0pt}%
\pgfpathmoveto{\pgfqpoint{0.804910in}{0.417642in}}%
\pgfpathlineto{\pgfqpoint{0.804910in}{2.472642in}}%
\pgfusepath{stroke}%
\end{pgfscope}%
\begin{pgfscope}%
\pgfsetbuttcap%
\pgfsetroundjoin%
\definecolor{currentfill}{rgb}{0.000000,0.000000,0.000000}%
\pgfsetfillcolor{currentfill}%
\pgfsetlinewidth{0.602250pt}%
\definecolor{currentstroke}{rgb}{0.000000,0.000000,0.000000}%
\pgfsetstrokecolor{currentstroke}%
\pgfsetdash{}{0pt}%
\pgfsys@defobject{currentmarker}{\pgfqpoint{0.000000in}{-0.027778in}}{\pgfqpoint{0.000000in}{0.000000in}}{%
\pgfpathmoveto{\pgfqpoint{0.000000in}{0.000000in}}%
\pgfpathlineto{\pgfqpoint{0.000000in}{-0.027778in}}%
\pgfusepath{stroke,fill}%
}%
\begin{pgfscope}%
\pgfsys@transformshift{0.804910in}{0.417642in}%
\pgfsys@useobject{currentmarker}{}%
\end{pgfscope}%
\end{pgfscope}%
\begin{pgfscope}%
\pgfpathrectangle{\pgfqpoint{0.589510in}{0.417642in}}{\pgfqpoint{3.437062in}{2.055000in}}%
\pgfusepath{clip}%
\pgfsetrectcap%
\pgfsetroundjoin%
\pgfsetlinewidth{0.803000pt}%
\definecolor{currentstroke}{rgb}{0.850000,0.850000,0.850000}%
\pgfsetstrokecolor{currentstroke}%
\pgfsetdash{}{0pt}%
\pgfpathmoveto{\pgfqpoint{0.853256in}{0.417642in}}%
\pgfpathlineto{\pgfqpoint{0.853256in}{2.472642in}}%
\pgfusepath{stroke}%
\end{pgfscope}%
\begin{pgfscope}%
\pgfsetbuttcap%
\pgfsetroundjoin%
\definecolor{currentfill}{rgb}{0.000000,0.000000,0.000000}%
\pgfsetfillcolor{currentfill}%
\pgfsetlinewidth{0.602250pt}%
\definecolor{currentstroke}{rgb}{0.000000,0.000000,0.000000}%
\pgfsetstrokecolor{currentstroke}%
\pgfsetdash{}{0pt}%
\pgfsys@defobject{currentmarker}{\pgfqpoint{0.000000in}{-0.027778in}}{\pgfqpoint{0.000000in}{0.000000in}}{%
\pgfpathmoveto{\pgfqpoint{0.000000in}{0.000000in}}%
\pgfpathlineto{\pgfqpoint{0.000000in}{-0.027778in}}%
\pgfusepath{stroke,fill}%
}%
\begin{pgfscope}%
\pgfsys@transformshift{0.853256in}{0.417642in}%
\pgfsys@useobject{currentmarker}{}%
\end{pgfscope}%
\end{pgfscope}%
\begin{pgfscope}%
\pgfpathrectangle{\pgfqpoint{0.589510in}{0.417642in}}{\pgfqpoint{3.437062in}{2.055000in}}%
\pgfusepath{clip}%
\pgfsetrectcap%
\pgfsetroundjoin%
\pgfsetlinewidth{0.803000pt}%
\definecolor{currentstroke}{rgb}{0.850000,0.850000,0.850000}%
\pgfsetstrokecolor{currentstroke}%
\pgfsetdash{}{0pt}%
\pgfpathmoveto{\pgfqpoint{0.894132in}{0.417642in}}%
\pgfpathlineto{\pgfqpoint{0.894132in}{2.472642in}}%
\pgfusepath{stroke}%
\end{pgfscope}%
\begin{pgfscope}%
\pgfsetbuttcap%
\pgfsetroundjoin%
\definecolor{currentfill}{rgb}{0.000000,0.000000,0.000000}%
\pgfsetfillcolor{currentfill}%
\pgfsetlinewidth{0.602250pt}%
\definecolor{currentstroke}{rgb}{0.000000,0.000000,0.000000}%
\pgfsetstrokecolor{currentstroke}%
\pgfsetdash{}{0pt}%
\pgfsys@defobject{currentmarker}{\pgfqpoint{0.000000in}{-0.027778in}}{\pgfqpoint{0.000000in}{0.000000in}}{%
\pgfpathmoveto{\pgfqpoint{0.000000in}{0.000000in}}%
\pgfpathlineto{\pgfqpoint{0.000000in}{-0.027778in}}%
\pgfusepath{stroke,fill}%
}%
\begin{pgfscope}%
\pgfsys@transformshift{0.894132in}{0.417642in}%
\pgfsys@useobject{currentmarker}{}%
\end{pgfscope}%
\end{pgfscope}%
\begin{pgfscope}%
\pgfpathrectangle{\pgfqpoint{0.589510in}{0.417642in}}{\pgfqpoint{3.437062in}{2.055000in}}%
\pgfusepath{clip}%
\pgfsetrectcap%
\pgfsetroundjoin%
\pgfsetlinewidth{0.803000pt}%
\definecolor{currentstroke}{rgb}{0.850000,0.850000,0.850000}%
\pgfsetstrokecolor{currentstroke}%
\pgfsetdash{}{0pt}%
\pgfpathmoveto{\pgfqpoint{0.929540in}{0.417642in}}%
\pgfpathlineto{\pgfqpoint{0.929540in}{2.472642in}}%
\pgfusepath{stroke}%
\end{pgfscope}%
\begin{pgfscope}%
\pgfsetbuttcap%
\pgfsetroundjoin%
\definecolor{currentfill}{rgb}{0.000000,0.000000,0.000000}%
\pgfsetfillcolor{currentfill}%
\pgfsetlinewidth{0.602250pt}%
\definecolor{currentstroke}{rgb}{0.000000,0.000000,0.000000}%
\pgfsetstrokecolor{currentstroke}%
\pgfsetdash{}{0pt}%
\pgfsys@defobject{currentmarker}{\pgfqpoint{0.000000in}{-0.027778in}}{\pgfqpoint{0.000000in}{0.000000in}}{%
\pgfpathmoveto{\pgfqpoint{0.000000in}{0.000000in}}%
\pgfpathlineto{\pgfqpoint{0.000000in}{-0.027778in}}%
\pgfusepath{stroke,fill}%
}%
\begin{pgfscope}%
\pgfsys@transformshift{0.929540in}{0.417642in}%
\pgfsys@useobject{currentmarker}{}%
\end{pgfscope}%
\end{pgfscope}%
\begin{pgfscope}%
\pgfpathrectangle{\pgfqpoint{0.589510in}{0.417642in}}{\pgfqpoint{3.437062in}{2.055000in}}%
\pgfusepath{clip}%
\pgfsetrectcap%
\pgfsetroundjoin%
\pgfsetlinewidth{0.803000pt}%
\definecolor{currentstroke}{rgb}{0.850000,0.850000,0.850000}%
\pgfsetstrokecolor{currentstroke}%
\pgfsetdash{}{0pt}%
\pgfpathmoveto{\pgfqpoint{0.960772in}{0.417642in}}%
\pgfpathlineto{\pgfqpoint{0.960772in}{2.472642in}}%
\pgfusepath{stroke}%
\end{pgfscope}%
\begin{pgfscope}%
\pgfsetbuttcap%
\pgfsetroundjoin%
\definecolor{currentfill}{rgb}{0.000000,0.000000,0.000000}%
\pgfsetfillcolor{currentfill}%
\pgfsetlinewidth{0.602250pt}%
\definecolor{currentstroke}{rgb}{0.000000,0.000000,0.000000}%
\pgfsetstrokecolor{currentstroke}%
\pgfsetdash{}{0pt}%
\pgfsys@defobject{currentmarker}{\pgfqpoint{0.000000in}{-0.027778in}}{\pgfqpoint{0.000000in}{0.000000in}}{%
\pgfpathmoveto{\pgfqpoint{0.000000in}{0.000000in}}%
\pgfpathlineto{\pgfqpoint{0.000000in}{-0.027778in}}%
\pgfusepath{stroke,fill}%
}%
\begin{pgfscope}%
\pgfsys@transformshift{0.960772in}{0.417642in}%
\pgfsys@useobject{currentmarker}{}%
\end{pgfscope}%
\end{pgfscope}%
\begin{pgfscope}%
\pgfpathrectangle{\pgfqpoint{0.589510in}{0.417642in}}{\pgfqpoint{3.437062in}{2.055000in}}%
\pgfusepath{clip}%
\pgfsetrectcap%
\pgfsetroundjoin%
\pgfsetlinewidth{0.803000pt}%
\definecolor{currentstroke}{rgb}{0.850000,0.850000,0.850000}%
\pgfsetstrokecolor{currentstroke}%
\pgfsetdash{}{0pt}%
\pgfpathmoveto{\pgfqpoint{1.172510in}{0.417642in}}%
\pgfpathlineto{\pgfqpoint{1.172510in}{2.472642in}}%
\pgfusepath{stroke}%
\end{pgfscope}%
\begin{pgfscope}%
\pgfsetbuttcap%
\pgfsetroundjoin%
\definecolor{currentfill}{rgb}{0.000000,0.000000,0.000000}%
\pgfsetfillcolor{currentfill}%
\pgfsetlinewidth{0.602250pt}%
\definecolor{currentstroke}{rgb}{0.000000,0.000000,0.000000}%
\pgfsetstrokecolor{currentstroke}%
\pgfsetdash{}{0pt}%
\pgfsys@defobject{currentmarker}{\pgfqpoint{0.000000in}{-0.027778in}}{\pgfqpoint{0.000000in}{0.000000in}}{%
\pgfpathmoveto{\pgfqpoint{0.000000in}{0.000000in}}%
\pgfpathlineto{\pgfqpoint{0.000000in}{-0.027778in}}%
\pgfusepath{stroke,fill}%
}%
\begin{pgfscope}%
\pgfsys@transformshift{1.172510in}{0.417642in}%
\pgfsys@useobject{currentmarker}{}%
\end{pgfscope}%
\end{pgfscope}%
\begin{pgfscope}%
\pgfpathrectangle{\pgfqpoint{0.589510in}{0.417642in}}{\pgfqpoint{3.437062in}{2.055000in}}%
\pgfusepath{clip}%
\pgfsetrectcap%
\pgfsetroundjoin%
\pgfsetlinewidth{0.803000pt}%
\definecolor{currentstroke}{rgb}{0.850000,0.850000,0.850000}%
\pgfsetstrokecolor{currentstroke}%
\pgfsetdash{}{0pt}%
\pgfpathmoveto{\pgfqpoint{1.280027in}{0.417642in}}%
\pgfpathlineto{\pgfqpoint{1.280027in}{2.472642in}}%
\pgfusepath{stroke}%
\end{pgfscope}%
\begin{pgfscope}%
\pgfsetbuttcap%
\pgfsetroundjoin%
\definecolor{currentfill}{rgb}{0.000000,0.000000,0.000000}%
\pgfsetfillcolor{currentfill}%
\pgfsetlinewidth{0.602250pt}%
\definecolor{currentstroke}{rgb}{0.000000,0.000000,0.000000}%
\pgfsetstrokecolor{currentstroke}%
\pgfsetdash{}{0pt}%
\pgfsys@defobject{currentmarker}{\pgfqpoint{0.000000in}{-0.027778in}}{\pgfqpoint{0.000000in}{0.000000in}}{%
\pgfpathmoveto{\pgfqpoint{0.000000in}{0.000000in}}%
\pgfpathlineto{\pgfqpoint{0.000000in}{-0.027778in}}%
\pgfusepath{stroke,fill}%
}%
\begin{pgfscope}%
\pgfsys@transformshift{1.280027in}{0.417642in}%
\pgfsys@useobject{currentmarker}{}%
\end{pgfscope}%
\end{pgfscope}%
\begin{pgfscope}%
\pgfpathrectangle{\pgfqpoint{0.589510in}{0.417642in}}{\pgfqpoint{3.437062in}{2.055000in}}%
\pgfusepath{clip}%
\pgfsetrectcap%
\pgfsetroundjoin%
\pgfsetlinewidth{0.803000pt}%
\definecolor{currentstroke}{rgb}{0.850000,0.850000,0.850000}%
\pgfsetstrokecolor{currentstroke}%
\pgfsetdash{}{0pt}%
\pgfpathmoveto{\pgfqpoint{1.356311in}{0.417642in}}%
\pgfpathlineto{\pgfqpoint{1.356311in}{2.472642in}}%
\pgfusepath{stroke}%
\end{pgfscope}%
\begin{pgfscope}%
\pgfsetbuttcap%
\pgfsetroundjoin%
\definecolor{currentfill}{rgb}{0.000000,0.000000,0.000000}%
\pgfsetfillcolor{currentfill}%
\pgfsetlinewidth{0.602250pt}%
\definecolor{currentstroke}{rgb}{0.000000,0.000000,0.000000}%
\pgfsetstrokecolor{currentstroke}%
\pgfsetdash{}{0pt}%
\pgfsys@defobject{currentmarker}{\pgfqpoint{0.000000in}{-0.027778in}}{\pgfqpoint{0.000000in}{0.000000in}}{%
\pgfpathmoveto{\pgfqpoint{0.000000in}{0.000000in}}%
\pgfpathlineto{\pgfqpoint{0.000000in}{-0.027778in}}%
\pgfusepath{stroke,fill}%
}%
\begin{pgfscope}%
\pgfsys@transformshift{1.356311in}{0.417642in}%
\pgfsys@useobject{currentmarker}{}%
\end{pgfscope}%
\end{pgfscope}%
\begin{pgfscope}%
\pgfpathrectangle{\pgfqpoint{0.589510in}{0.417642in}}{\pgfqpoint{3.437062in}{2.055000in}}%
\pgfusepath{clip}%
\pgfsetrectcap%
\pgfsetroundjoin%
\pgfsetlinewidth{0.803000pt}%
\definecolor{currentstroke}{rgb}{0.850000,0.850000,0.850000}%
\pgfsetstrokecolor{currentstroke}%
\pgfsetdash{}{0pt}%
\pgfpathmoveto{\pgfqpoint{1.415481in}{0.417642in}}%
\pgfpathlineto{\pgfqpoint{1.415481in}{2.472642in}}%
\pgfusepath{stroke}%
\end{pgfscope}%
\begin{pgfscope}%
\pgfsetbuttcap%
\pgfsetroundjoin%
\definecolor{currentfill}{rgb}{0.000000,0.000000,0.000000}%
\pgfsetfillcolor{currentfill}%
\pgfsetlinewidth{0.602250pt}%
\definecolor{currentstroke}{rgb}{0.000000,0.000000,0.000000}%
\pgfsetstrokecolor{currentstroke}%
\pgfsetdash{}{0pt}%
\pgfsys@defobject{currentmarker}{\pgfqpoint{0.000000in}{-0.027778in}}{\pgfqpoint{0.000000in}{0.000000in}}{%
\pgfpathmoveto{\pgfqpoint{0.000000in}{0.000000in}}%
\pgfpathlineto{\pgfqpoint{0.000000in}{-0.027778in}}%
\pgfusepath{stroke,fill}%
}%
\begin{pgfscope}%
\pgfsys@transformshift{1.415481in}{0.417642in}%
\pgfsys@useobject{currentmarker}{}%
\end{pgfscope}%
\end{pgfscope}%
\begin{pgfscope}%
\pgfpathrectangle{\pgfqpoint{0.589510in}{0.417642in}}{\pgfqpoint{3.437062in}{2.055000in}}%
\pgfusepath{clip}%
\pgfsetrectcap%
\pgfsetroundjoin%
\pgfsetlinewidth{0.803000pt}%
\definecolor{currentstroke}{rgb}{0.850000,0.850000,0.850000}%
\pgfsetstrokecolor{currentstroke}%
\pgfsetdash{}{0pt}%
\pgfpathmoveto{\pgfqpoint{1.463827in}{0.417642in}}%
\pgfpathlineto{\pgfqpoint{1.463827in}{2.472642in}}%
\pgfusepath{stroke}%
\end{pgfscope}%
\begin{pgfscope}%
\pgfsetbuttcap%
\pgfsetroundjoin%
\definecolor{currentfill}{rgb}{0.000000,0.000000,0.000000}%
\pgfsetfillcolor{currentfill}%
\pgfsetlinewidth{0.602250pt}%
\definecolor{currentstroke}{rgb}{0.000000,0.000000,0.000000}%
\pgfsetstrokecolor{currentstroke}%
\pgfsetdash{}{0pt}%
\pgfsys@defobject{currentmarker}{\pgfqpoint{0.000000in}{-0.027778in}}{\pgfqpoint{0.000000in}{0.000000in}}{%
\pgfpathmoveto{\pgfqpoint{0.000000in}{0.000000in}}%
\pgfpathlineto{\pgfqpoint{0.000000in}{-0.027778in}}%
\pgfusepath{stroke,fill}%
}%
\begin{pgfscope}%
\pgfsys@transformshift{1.463827in}{0.417642in}%
\pgfsys@useobject{currentmarker}{}%
\end{pgfscope}%
\end{pgfscope}%
\begin{pgfscope}%
\pgfpathrectangle{\pgfqpoint{0.589510in}{0.417642in}}{\pgfqpoint{3.437062in}{2.055000in}}%
\pgfusepath{clip}%
\pgfsetrectcap%
\pgfsetroundjoin%
\pgfsetlinewidth{0.803000pt}%
\definecolor{currentstroke}{rgb}{0.850000,0.850000,0.850000}%
\pgfsetstrokecolor{currentstroke}%
\pgfsetdash{}{0pt}%
\pgfpathmoveto{\pgfqpoint{1.504702in}{0.417642in}}%
\pgfpathlineto{\pgfqpoint{1.504702in}{2.472642in}}%
\pgfusepath{stroke}%
\end{pgfscope}%
\begin{pgfscope}%
\pgfsetbuttcap%
\pgfsetroundjoin%
\definecolor{currentfill}{rgb}{0.000000,0.000000,0.000000}%
\pgfsetfillcolor{currentfill}%
\pgfsetlinewidth{0.602250pt}%
\definecolor{currentstroke}{rgb}{0.000000,0.000000,0.000000}%
\pgfsetstrokecolor{currentstroke}%
\pgfsetdash{}{0pt}%
\pgfsys@defobject{currentmarker}{\pgfqpoint{0.000000in}{-0.027778in}}{\pgfqpoint{0.000000in}{0.000000in}}{%
\pgfpathmoveto{\pgfqpoint{0.000000in}{0.000000in}}%
\pgfpathlineto{\pgfqpoint{0.000000in}{-0.027778in}}%
\pgfusepath{stroke,fill}%
}%
\begin{pgfscope}%
\pgfsys@transformshift{1.504702in}{0.417642in}%
\pgfsys@useobject{currentmarker}{}%
\end{pgfscope}%
\end{pgfscope}%
\begin{pgfscope}%
\pgfpathrectangle{\pgfqpoint{0.589510in}{0.417642in}}{\pgfqpoint{3.437062in}{2.055000in}}%
\pgfusepath{clip}%
\pgfsetrectcap%
\pgfsetroundjoin%
\pgfsetlinewidth{0.803000pt}%
\definecolor{currentstroke}{rgb}{0.850000,0.850000,0.850000}%
\pgfsetstrokecolor{currentstroke}%
\pgfsetdash{}{0pt}%
\pgfpathmoveto{\pgfqpoint{1.540111in}{0.417642in}}%
\pgfpathlineto{\pgfqpoint{1.540111in}{2.472642in}}%
\pgfusepath{stroke}%
\end{pgfscope}%
\begin{pgfscope}%
\pgfsetbuttcap%
\pgfsetroundjoin%
\definecolor{currentfill}{rgb}{0.000000,0.000000,0.000000}%
\pgfsetfillcolor{currentfill}%
\pgfsetlinewidth{0.602250pt}%
\definecolor{currentstroke}{rgb}{0.000000,0.000000,0.000000}%
\pgfsetstrokecolor{currentstroke}%
\pgfsetdash{}{0pt}%
\pgfsys@defobject{currentmarker}{\pgfqpoint{0.000000in}{-0.027778in}}{\pgfqpoint{0.000000in}{0.000000in}}{%
\pgfpathmoveto{\pgfqpoint{0.000000in}{0.000000in}}%
\pgfpathlineto{\pgfqpoint{0.000000in}{-0.027778in}}%
\pgfusepath{stroke,fill}%
}%
\begin{pgfscope}%
\pgfsys@transformshift{1.540111in}{0.417642in}%
\pgfsys@useobject{currentmarker}{}%
\end{pgfscope}%
\end{pgfscope}%
\begin{pgfscope}%
\pgfpathrectangle{\pgfqpoint{0.589510in}{0.417642in}}{\pgfqpoint{3.437062in}{2.055000in}}%
\pgfusepath{clip}%
\pgfsetrectcap%
\pgfsetroundjoin%
\pgfsetlinewidth{0.803000pt}%
\definecolor{currentstroke}{rgb}{0.850000,0.850000,0.850000}%
\pgfsetstrokecolor{currentstroke}%
\pgfsetdash{}{0pt}%
\pgfpathmoveto{\pgfqpoint{1.571343in}{0.417642in}}%
\pgfpathlineto{\pgfqpoint{1.571343in}{2.472642in}}%
\pgfusepath{stroke}%
\end{pgfscope}%
\begin{pgfscope}%
\pgfsetbuttcap%
\pgfsetroundjoin%
\definecolor{currentfill}{rgb}{0.000000,0.000000,0.000000}%
\pgfsetfillcolor{currentfill}%
\pgfsetlinewidth{0.602250pt}%
\definecolor{currentstroke}{rgb}{0.000000,0.000000,0.000000}%
\pgfsetstrokecolor{currentstroke}%
\pgfsetdash{}{0pt}%
\pgfsys@defobject{currentmarker}{\pgfqpoint{0.000000in}{-0.027778in}}{\pgfqpoint{0.000000in}{0.000000in}}{%
\pgfpathmoveto{\pgfqpoint{0.000000in}{0.000000in}}%
\pgfpathlineto{\pgfqpoint{0.000000in}{-0.027778in}}%
\pgfusepath{stroke,fill}%
}%
\begin{pgfscope}%
\pgfsys@transformshift{1.571343in}{0.417642in}%
\pgfsys@useobject{currentmarker}{}%
\end{pgfscope}%
\end{pgfscope}%
\begin{pgfscope}%
\pgfpathrectangle{\pgfqpoint{0.589510in}{0.417642in}}{\pgfqpoint{3.437062in}{2.055000in}}%
\pgfusepath{clip}%
\pgfsetrectcap%
\pgfsetroundjoin%
\pgfsetlinewidth{0.803000pt}%
\definecolor{currentstroke}{rgb}{0.850000,0.850000,0.850000}%
\pgfsetstrokecolor{currentstroke}%
\pgfsetdash{}{0pt}%
\pgfpathmoveto{\pgfqpoint{1.783081in}{0.417642in}}%
\pgfpathlineto{\pgfqpoint{1.783081in}{2.472642in}}%
\pgfusepath{stroke}%
\end{pgfscope}%
\begin{pgfscope}%
\pgfsetbuttcap%
\pgfsetroundjoin%
\definecolor{currentfill}{rgb}{0.000000,0.000000,0.000000}%
\pgfsetfillcolor{currentfill}%
\pgfsetlinewidth{0.602250pt}%
\definecolor{currentstroke}{rgb}{0.000000,0.000000,0.000000}%
\pgfsetstrokecolor{currentstroke}%
\pgfsetdash{}{0pt}%
\pgfsys@defobject{currentmarker}{\pgfqpoint{0.000000in}{-0.027778in}}{\pgfqpoint{0.000000in}{0.000000in}}{%
\pgfpathmoveto{\pgfqpoint{0.000000in}{0.000000in}}%
\pgfpathlineto{\pgfqpoint{0.000000in}{-0.027778in}}%
\pgfusepath{stroke,fill}%
}%
\begin{pgfscope}%
\pgfsys@transformshift{1.783081in}{0.417642in}%
\pgfsys@useobject{currentmarker}{}%
\end{pgfscope}%
\end{pgfscope}%
\begin{pgfscope}%
\pgfpathrectangle{\pgfqpoint{0.589510in}{0.417642in}}{\pgfqpoint{3.437062in}{2.055000in}}%
\pgfusepath{clip}%
\pgfsetrectcap%
\pgfsetroundjoin%
\pgfsetlinewidth{0.803000pt}%
\definecolor{currentstroke}{rgb}{0.850000,0.850000,0.850000}%
\pgfsetstrokecolor{currentstroke}%
\pgfsetdash{}{0pt}%
\pgfpathmoveto{\pgfqpoint{1.890597in}{0.417642in}}%
\pgfpathlineto{\pgfqpoint{1.890597in}{2.472642in}}%
\pgfusepath{stroke}%
\end{pgfscope}%
\begin{pgfscope}%
\pgfsetbuttcap%
\pgfsetroundjoin%
\definecolor{currentfill}{rgb}{0.000000,0.000000,0.000000}%
\pgfsetfillcolor{currentfill}%
\pgfsetlinewidth{0.602250pt}%
\definecolor{currentstroke}{rgb}{0.000000,0.000000,0.000000}%
\pgfsetstrokecolor{currentstroke}%
\pgfsetdash{}{0pt}%
\pgfsys@defobject{currentmarker}{\pgfqpoint{0.000000in}{-0.027778in}}{\pgfqpoint{0.000000in}{0.000000in}}{%
\pgfpathmoveto{\pgfqpoint{0.000000in}{0.000000in}}%
\pgfpathlineto{\pgfqpoint{0.000000in}{-0.027778in}}%
\pgfusepath{stroke,fill}%
}%
\begin{pgfscope}%
\pgfsys@transformshift{1.890597in}{0.417642in}%
\pgfsys@useobject{currentmarker}{}%
\end{pgfscope}%
\end{pgfscope}%
\begin{pgfscope}%
\pgfpathrectangle{\pgfqpoint{0.589510in}{0.417642in}}{\pgfqpoint{3.437062in}{2.055000in}}%
\pgfusepath{clip}%
\pgfsetrectcap%
\pgfsetroundjoin%
\pgfsetlinewidth{0.803000pt}%
\definecolor{currentstroke}{rgb}{0.850000,0.850000,0.850000}%
\pgfsetstrokecolor{currentstroke}%
\pgfsetdash{}{0pt}%
\pgfpathmoveto{\pgfqpoint{1.966881in}{0.417642in}}%
\pgfpathlineto{\pgfqpoint{1.966881in}{2.472642in}}%
\pgfusepath{stroke}%
\end{pgfscope}%
\begin{pgfscope}%
\pgfsetbuttcap%
\pgfsetroundjoin%
\definecolor{currentfill}{rgb}{0.000000,0.000000,0.000000}%
\pgfsetfillcolor{currentfill}%
\pgfsetlinewidth{0.602250pt}%
\definecolor{currentstroke}{rgb}{0.000000,0.000000,0.000000}%
\pgfsetstrokecolor{currentstroke}%
\pgfsetdash{}{0pt}%
\pgfsys@defobject{currentmarker}{\pgfqpoint{0.000000in}{-0.027778in}}{\pgfqpoint{0.000000in}{0.000000in}}{%
\pgfpathmoveto{\pgfqpoint{0.000000in}{0.000000in}}%
\pgfpathlineto{\pgfqpoint{0.000000in}{-0.027778in}}%
\pgfusepath{stroke,fill}%
}%
\begin{pgfscope}%
\pgfsys@transformshift{1.966881in}{0.417642in}%
\pgfsys@useobject{currentmarker}{}%
\end{pgfscope}%
\end{pgfscope}%
\begin{pgfscope}%
\pgfpathrectangle{\pgfqpoint{0.589510in}{0.417642in}}{\pgfqpoint{3.437062in}{2.055000in}}%
\pgfusepath{clip}%
\pgfsetrectcap%
\pgfsetroundjoin%
\pgfsetlinewidth{0.803000pt}%
\definecolor{currentstroke}{rgb}{0.850000,0.850000,0.850000}%
\pgfsetstrokecolor{currentstroke}%
\pgfsetdash{}{0pt}%
\pgfpathmoveto{\pgfqpoint{2.026052in}{0.417642in}}%
\pgfpathlineto{\pgfqpoint{2.026052in}{2.472642in}}%
\pgfusepath{stroke}%
\end{pgfscope}%
\begin{pgfscope}%
\pgfsetbuttcap%
\pgfsetroundjoin%
\definecolor{currentfill}{rgb}{0.000000,0.000000,0.000000}%
\pgfsetfillcolor{currentfill}%
\pgfsetlinewidth{0.602250pt}%
\definecolor{currentstroke}{rgb}{0.000000,0.000000,0.000000}%
\pgfsetstrokecolor{currentstroke}%
\pgfsetdash{}{0pt}%
\pgfsys@defobject{currentmarker}{\pgfqpoint{0.000000in}{-0.027778in}}{\pgfqpoint{0.000000in}{0.000000in}}{%
\pgfpathmoveto{\pgfqpoint{0.000000in}{0.000000in}}%
\pgfpathlineto{\pgfqpoint{0.000000in}{-0.027778in}}%
\pgfusepath{stroke,fill}%
}%
\begin{pgfscope}%
\pgfsys@transformshift{2.026052in}{0.417642in}%
\pgfsys@useobject{currentmarker}{}%
\end{pgfscope}%
\end{pgfscope}%
\begin{pgfscope}%
\pgfpathrectangle{\pgfqpoint{0.589510in}{0.417642in}}{\pgfqpoint{3.437062in}{2.055000in}}%
\pgfusepath{clip}%
\pgfsetrectcap%
\pgfsetroundjoin%
\pgfsetlinewidth{0.803000pt}%
\definecolor{currentstroke}{rgb}{0.850000,0.850000,0.850000}%
\pgfsetstrokecolor{currentstroke}%
\pgfsetdash{}{0pt}%
\pgfpathmoveto{\pgfqpoint{2.074397in}{0.417642in}}%
\pgfpathlineto{\pgfqpoint{2.074397in}{2.472642in}}%
\pgfusepath{stroke}%
\end{pgfscope}%
\begin{pgfscope}%
\pgfsetbuttcap%
\pgfsetroundjoin%
\definecolor{currentfill}{rgb}{0.000000,0.000000,0.000000}%
\pgfsetfillcolor{currentfill}%
\pgfsetlinewidth{0.602250pt}%
\definecolor{currentstroke}{rgb}{0.000000,0.000000,0.000000}%
\pgfsetstrokecolor{currentstroke}%
\pgfsetdash{}{0pt}%
\pgfsys@defobject{currentmarker}{\pgfqpoint{0.000000in}{-0.027778in}}{\pgfqpoint{0.000000in}{0.000000in}}{%
\pgfpathmoveto{\pgfqpoint{0.000000in}{0.000000in}}%
\pgfpathlineto{\pgfqpoint{0.000000in}{-0.027778in}}%
\pgfusepath{stroke,fill}%
}%
\begin{pgfscope}%
\pgfsys@transformshift{2.074397in}{0.417642in}%
\pgfsys@useobject{currentmarker}{}%
\end{pgfscope}%
\end{pgfscope}%
\begin{pgfscope}%
\pgfpathrectangle{\pgfqpoint{0.589510in}{0.417642in}}{\pgfqpoint{3.437062in}{2.055000in}}%
\pgfusepath{clip}%
\pgfsetrectcap%
\pgfsetroundjoin%
\pgfsetlinewidth{0.803000pt}%
\definecolor{currentstroke}{rgb}{0.850000,0.850000,0.850000}%
\pgfsetstrokecolor{currentstroke}%
\pgfsetdash{}{0pt}%
\pgfpathmoveto{\pgfqpoint{2.115273in}{0.417642in}}%
\pgfpathlineto{\pgfqpoint{2.115273in}{2.472642in}}%
\pgfusepath{stroke}%
\end{pgfscope}%
\begin{pgfscope}%
\pgfsetbuttcap%
\pgfsetroundjoin%
\definecolor{currentfill}{rgb}{0.000000,0.000000,0.000000}%
\pgfsetfillcolor{currentfill}%
\pgfsetlinewidth{0.602250pt}%
\definecolor{currentstroke}{rgb}{0.000000,0.000000,0.000000}%
\pgfsetstrokecolor{currentstroke}%
\pgfsetdash{}{0pt}%
\pgfsys@defobject{currentmarker}{\pgfqpoint{0.000000in}{-0.027778in}}{\pgfqpoint{0.000000in}{0.000000in}}{%
\pgfpathmoveto{\pgfqpoint{0.000000in}{0.000000in}}%
\pgfpathlineto{\pgfqpoint{0.000000in}{-0.027778in}}%
\pgfusepath{stroke,fill}%
}%
\begin{pgfscope}%
\pgfsys@transformshift{2.115273in}{0.417642in}%
\pgfsys@useobject{currentmarker}{}%
\end{pgfscope}%
\end{pgfscope}%
\begin{pgfscope}%
\pgfpathrectangle{\pgfqpoint{0.589510in}{0.417642in}}{\pgfqpoint{3.437062in}{2.055000in}}%
\pgfusepath{clip}%
\pgfsetrectcap%
\pgfsetroundjoin%
\pgfsetlinewidth{0.803000pt}%
\definecolor{currentstroke}{rgb}{0.850000,0.850000,0.850000}%
\pgfsetstrokecolor{currentstroke}%
\pgfsetdash{}{0pt}%
\pgfpathmoveto{\pgfqpoint{2.150681in}{0.417642in}}%
\pgfpathlineto{\pgfqpoint{2.150681in}{2.472642in}}%
\pgfusepath{stroke}%
\end{pgfscope}%
\begin{pgfscope}%
\pgfsetbuttcap%
\pgfsetroundjoin%
\definecolor{currentfill}{rgb}{0.000000,0.000000,0.000000}%
\pgfsetfillcolor{currentfill}%
\pgfsetlinewidth{0.602250pt}%
\definecolor{currentstroke}{rgb}{0.000000,0.000000,0.000000}%
\pgfsetstrokecolor{currentstroke}%
\pgfsetdash{}{0pt}%
\pgfsys@defobject{currentmarker}{\pgfqpoint{0.000000in}{-0.027778in}}{\pgfqpoint{0.000000in}{0.000000in}}{%
\pgfpathmoveto{\pgfqpoint{0.000000in}{0.000000in}}%
\pgfpathlineto{\pgfqpoint{0.000000in}{-0.027778in}}%
\pgfusepath{stroke,fill}%
}%
\begin{pgfscope}%
\pgfsys@transformshift{2.150681in}{0.417642in}%
\pgfsys@useobject{currentmarker}{}%
\end{pgfscope}%
\end{pgfscope}%
\begin{pgfscope}%
\pgfpathrectangle{\pgfqpoint{0.589510in}{0.417642in}}{\pgfqpoint{3.437062in}{2.055000in}}%
\pgfusepath{clip}%
\pgfsetrectcap%
\pgfsetroundjoin%
\pgfsetlinewidth{0.803000pt}%
\definecolor{currentstroke}{rgb}{0.850000,0.850000,0.850000}%
\pgfsetstrokecolor{currentstroke}%
\pgfsetdash{}{0pt}%
\pgfpathmoveto{\pgfqpoint{2.181914in}{0.417642in}}%
\pgfpathlineto{\pgfqpoint{2.181914in}{2.472642in}}%
\pgfusepath{stroke}%
\end{pgfscope}%
\begin{pgfscope}%
\pgfsetbuttcap%
\pgfsetroundjoin%
\definecolor{currentfill}{rgb}{0.000000,0.000000,0.000000}%
\pgfsetfillcolor{currentfill}%
\pgfsetlinewidth{0.602250pt}%
\definecolor{currentstroke}{rgb}{0.000000,0.000000,0.000000}%
\pgfsetstrokecolor{currentstroke}%
\pgfsetdash{}{0pt}%
\pgfsys@defobject{currentmarker}{\pgfqpoint{0.000000in}{-0.027778in}}{\pgfqpoint{0.000000in}{0.000000in}}{%
\pgfpathmoveto{\pgfqpoint{0.000000in}{0.000000in}}%
\pgfpathlineto{\pgfqpoint{0.000000in}{-0.027778in}}%
\pgfusepath{stroke,fill}%
}%
\begin{pgfscope}%
\pgfsys@transformshift{2.181914in}{0.417642in}%
\pgfsys@useobject{currentmarker}{}%
\end{pgfscope}%
\end{pgfscope}%
\begin{pgfscope}%
\pgfpathrectangle{\pgfqpoint{0.589510in}{0.417642in}}{\pgfqpoint{3.437062in}{2.055000in}}%
\pgfusepath{clip}%
\pgfsetrectcap%
\pgfsetroundjoin%
\pgfsetlinewidth{0.803000pt}%
\definecolor{currentstroke}{rgb}{0.850000,0.850000,0.850000}%
\pgfsetstrokecolor{currentstroke}%
\pgfsetdash{}{0pt}%
\pgfpathmoveto{\pgfqpoint{2.393652in}{0.417642in}}%
\pgfpathlineto{\pgfqpoint{2.393652in}{2.472642in}}%
\pgfusepath{stroke}%
\end{pgfscope}%
\begin{pgfscope}%
\pgfsetbuttcap%
\pgfsetroundjoin%
\definecolor{currentfill}{rgb}{0.000000,0.000000,0.000000}%
\pgfsetfillcolor{currentfill}%
\pgfsetlinewidth{0.602250pt}%
\definecolor{currentstroke}{rgb}{0.000000,0.000000,0.000000}%
\pgfsetstrokecolor{currentstroke}%
\pgfsetdash{}{0pt}%
\pgfsys@defobject{currentmarker}{\pgfqpoint{0.000000in}{-0.027778in}}{\pgfqpoint{0.000000in}{0.000000in}}{%
\pgfpathmoveto{\pgfqpoint{0.000000in}{0.000000in}}%
\pgfpathlineto{\pgfqpoint{0.000000in}{-0.027778in}}%
\pgfusepath{stroke,fill}%
}%
\begin{pgfscope}%
\pgfsys@transformshift{2.393652in}{0.417642in}%
\pgfsys@useobject{currentmarker}{}%
\end{pgfscope}%
\end{pgfscope}%
\begin{pgfscope}%
\pgfpathrectangle{\pgfqpoint{0.589510in}{0.417642in}}{\pgfqpoint{3.437062in}{2.055000in}}%
\pgfusepath{clip}%
\pgfsetrectcap%
\pgfsetroundjoin%
\pgfsetlinewidth{0.803000pt}%
\definecolor{currentstroke}{rgb}{0.850000,0.850000,0.850000}%
\pgfsetstrokecolor{currentstroke}%
\pgfsetdash{}{0pt}%
\pgfpathmoveto{\pgfqpoint{2.501168in}{0.417642in}}%
\pgfpathlineto{\pgfqpoint{2.501168in}{2.472642in}}%
\pgfusepath{stroke}%
\end{pgfscope}%
\begin{pgfscope}%
\pgfsetbuttcap%
\pgfsetroundjoin%
\definecolor{currentfill}{rgb}{0.000000,0.000000,0.000000}%
\pgfsetfillcolor{currentfill}%
\pgfsetlinewidth{0.602250pt}%
\definecolor{currentstroke}{rgb}{0.000000,0.000000,0.000000}%
\pgfsetstrokecolor{currentstroke}%
\pgfsetdash{}{0pt}%
\pgfsys@defobject{currentmarker}{\pgfqpoint{0.000000in}{-0.027778in}}{\pgfqpoint{0.000000in}{0.000000in}}{%
\pgfpathmoveto{\pgfqpoint{0.000000in}{0.000000in}}%
\pgfpathlineto{\pgfqpoint{0.000000in}{-0.027778in}}%
\pgfusepath{stroke,fill}%
}%
\begin{pgfscope}%
\pgfsys@transformshift{2.501168in}{0.417642in}%
\pgfsys@useobject{currentmarker}{}%
\end{pgfscope}%
\end{pgfscope}%
\begin{pgfscope}%
\pgfpathrectangle{\pgfqpoint{0.589510in}{0.417642in}}{\pgfqpoint{3.437062in}{2.055000in}}%
\pgfusepath{clip}%
\pgfsetrectcap%
\pgfsetroundjoin%
\pgfsetlinewidth{0.803000pt}%
\definecolor{currentstroke}{rgb}{0.850000,0.850000,0.850000}%
\pgfsetstrokecolor{currentstroke}%
\pgfsetdash{}{0pt}%
\pgfpathmoveto{\pgfqpoint{2.577452in}{0.417642in}}%
\pgfpathlineto{\pgfqpoint{2.577452in}{2.472642in}}%
\pgfusepath{stroke}%
\end{pgfscope}%
\begin{pgfscope}%
\pgfsetbuttcap%
\pgfsetroundjoin%
\definecolor{currentfill}{rgb}{0.000000,0.000000,0.000000}%
\pgfsetfillcolor{currentfill}%
\pgfsetlinewidth{0.602250pt}%
\definecolor{currentstroke}{rgb}{0.000000,0.000000,0.000000}%
\pgfsetstrokecolor{currentstroke}%
\pgfsetdash{}{0pt}%
\pgfsys@defobject{currentmarker}{\pgfqpoint{0.000000in}{-0.027778in}}{\pgfqpoint{0.000000in}{0.000000in}}{%
\pgfpathmoveto{\pgfqpoint{0.000000in}{0.000000in}}%
\pgfpathlineto{\pgfqpoint{0.000000in}{-0.027778in}}%
\pgfusepath{stroke,fill}%
}%
\begin{pgfscope}%
\pgfsys@transformshift{2.577452in}{0.417642in}%
\pgfsys@useobject{currentmarker}{}%
\end{pgfscope}%
\end{pgfscope}%
\begin{pgfscope}%
\pgfpathrectangle{\pgfqpoint{0.589510in}{0.417642in}}{\pgfqpoint{3.437062in}{2.055000in}}%
\pgfusepath{clip}%
\pgfsetrectcap%
\pgfsetroundjoin%
\pgfsetlinewidth{0.803000pt}%
\definecolor{currentstroke}{rgb}{0.850000,0.850000,0.850000}%
\pgfsetstrokecolor{currentstroke}%
\pgfsetdash{}{0pt}%
\pgfpathmoveto{\pgfqpoint{2.636622in}{0.417642in}}%
\pgfpathlineto{\pgfqpoint{2.636622in}{2.472642in}}%
\pgfusepath{stroke}%
\end{pgfscope}%
\begin{pgfscope}%
\pgfsetbuttcap%
\pgfsetroundjoin%
\definecolor{currentfill}{rgb}{0.000000,0.000000,0.000000}%
\pgfsetfillcolor{currentfill}%
\pgfsetlinewidth{0.602250pt}%
\definecolor{currentstroke}{rgb}{0.000000,0.000000,0.000000}%
\pgfsetstrokecolor{currentstroke}%
\pgfsetdash{}{0pt}%
\pgfsys@defobject{currentmarker}{\pgfqpoint{0.000000in}{-0.027778in}}{\pgfqpoint{0.000000in}{0.000000in}}{%
\pgfpathmoveto{\pgfqpoint{0.000000in}{0.000000in}}%
\pgfpathlineto{\pgfqpoint{0.000000in}{-0.027778in}}%
\pgfusepath{stroke,fill}%
}%
\begin{pgfscope}%
\pgfsys@transformshift{2.636622in}{0.417642in}%
\pgfsys@useobject{currentmarker}{}%
\end{pgfscope}%
\end{pgfscope}%
\begin{pgfscope}%
\pgfpathrectangle{\pgfqpoint{0.589510in}{0.417642in}}{\pgfqpoint{3.437062in}{2.055000in}}%
\pgfusepath{clip}%
\pgfsetrectcap%
\pgfsetroundjoin%
\pgfsetlinewidth{0.803000pt}%
\definecolor{currentstroke}{rgb}{0.850000,0.850000,0.850000}%
\pgfsetstrokecolor{currentstroke}%
\pgfsetdash{}{0pt}%
\pgfpathmoveto{\pgfqpoint{2.684968in}{0.417642in}}%
\pgfpathlineto{\pgfqpoint{2.684968in}{2.472642in}}%
\pgfusepath{stroke}%
\end{pgfscope}%
\begin{pgfscope}%
\pgfsetbuttcap%
\pgfsetroundjoin%
\definecolor{currentfill}{rgb}{0.000000,0.000000,0.000000}%
\pgfsetfillcolor{currentfill}%
\pgfsetlinewidth{0.602250pt}%
\definecolor{currentstroke}{rgb}{0.000000,0.000000,0.000000}%
\pgfsetstrokecolor{currentstroke}%
\pgfsetdash{}{0pt}%
\pgfsys@defobject{currentmarker}{\pgfqpoint{0.000000in}{-0.027778in}}{\pgfqpoint{0.000000in}{0.000000in}}{%
\pgfpathmoveto{\pgfqpoint{0.000000in}{0.000000in}}%
\pgfpathlineto{\pgfqpoint{0.000000in}{-0.027778in}}%
\pgfusepath{stroke,fill}%
}%
\begin{pgfscope}%
\pgfsys@transformshift{2.684968in}{0.417642in}%
\pgfsys@useobject{currentmarker}{}%
\end{pgfscope}%
\end{pgfscope}%
\begin{pgfscope}%
\pgfpathrectangle{\pgfqpoint{0.589510in}{0.417642in}}{\pgfqpoint{3.437062in}{2.055000in}}%
\pgfusepath{clip}%
\pgfsetrectcap%
\pgfsetroundjoin%
\pgfsetlinewidth{0.803000pt}%
\definecolor{currentstroke}{rgb}{0.850000,0.850000,0.850000}%
\pgfsetstrokecolor{currentstroke}%
\pgfsetdash{}{0pt}%
\pgfpathmoveto{\pgfqpoint{2.725844in}{0.417642in}}%
\pgfpathlineto{\pgfqpoint{2.725844in}{2.472642in}}%
\pgfusepath{stroke}%
\end{pgfscope}%
\begin{pgfscope}%
\pgfsetbuttcap%
\pgfsetroundjoin%
\definecolor{currentfill}{rgb}{0.000000,0.000000,0.000000}%
\pgfsetfillcolor{currentfill}%
\pgfsetlinewidth{0.602250pt}%
\definecolor{currentstroke}{rgb}{0.000000,0.000000,0.000000}%
\pgfsetstrokecolor{currentstroke}%
\pgfsetdash{}{0pt}%
\pgfsys@defobject{currentmarker}{\pgfqpoint{0.000000in}{-0.027778in}}{\pgfqpoint{0.000000in}{0.000000in}}{%
\pgfpathmoveto{\pgfqpoint{0.000000in}{0.000000in}}%
\pgfpathlineto{\pgfqpoint{0.000000in}{-0.027778in}}%
\pgfusepath{stroke,fill}%
}%
\begin{pgfscope}%
\pgfsys@transformshift{2.725844in}{0.417642in}%
\pgfsys@useobject{currentmarker}{}%
\end{pgfscope}%
\end{pgfscope}%
\begin{pgfscope}%
\pgfpathrectangle{\pgfqpoint{0.589510in}{0.417642in}}{\pgfqpoint{3.437062in}{2.055000in}}%
\pgfusepath{clip}%
\pgfsetrectcap%
\pgfsetroundjoin%
\pgfsetlinewidth{0.803000pt}%
\definecolor{currentstroke}{rgb}{0.850000,0.850000,0.850000}%
\pgfsetstrokecolor{currentstroke}%
\pgfsetdash{}{0pt}%
\pgfpathmoveto{\pgfqpoint{2.761252in}{0.417642in}}%
\pgfpathlineto{\pgfqpoint{2.761252in}{2.472642in}}%
\pgfusepath{stroke}%
\end{pgfscope}%
\begin{pgfscope}%
\pgfsetbuttcap%
\pgfsetroundjoin%
\definecolor{currentfill}{rgb}{0.000000,0.000000,0.000000}%
\pgfsetfillcolor{currentfill}%
\pgfsetlinewidth{0.602250pt}%
\definecolor{currentstroke}{rgb}{0.000000,0.000000,0.000000}%
\pgfsetstrokecolor{currentstroke}%
\pgfsetdash{}{0pt}%
\pgfsys@defobject{currentmarker}{\pgfqpoint{0.000000in}{-0.027778in}}{\pgfqpoint{0.000000in}{0.000000in}}{%
\pgfpathmoveto{\pgfqpoint{0.000000in}{0.000000in}}%
\pgfpathlineto{\pgfqpoint{0.000000in}{-0.027778in}}%
\pgfusepath{stroke,fill}%
}%
\begin{pgfscope}%
\pgfsys@transformshift{2.761252in}{0.417642in}%
\pgfsys@useobject{currentmarker}{}%
\end{pgfscope}%
\end{pgfscope}%
\begin{pgfscope}%
\pgfpathrectangle{\pgfqpoint{0.589510in}{0.417642in}}{\pgfqpoint{3.437062in}{2.055000in}}%
\pgfusepath{clip}%
\pgfsetrectcap%
\pgfsetroundjoin%
\pgfsetlinewidth{0.803000pt}%
\definecolor{currentstroke}{rgb}{0.850000,0.850000,0.850000}%
\pgfsetstrokecolor{currentstroke}%
\pgfsetdash{}{0pt}%
\pgfpathmoveto{\pgfqpoint{2.792484in}{0.417642in}}%
\pgfpathlineto{\pgfqpoint{2.792484in}{2.472642in}}%
\pgfusepath{stroke}%
\end{pgfscope}%
\begin{pgfscope}%
\pgfsetbuttcap%
\pgfsetroundjoin%
\definecolor{currentfill}{rgb}{0.000000,0.000000,0.000000}%
\pgfsetfillcolor{currentfill}%
\pgfsetlinewidth{0.602250pt}%
\definecolor{currentstroke}{rgb}{0.000000,0.000000,0.000000}%
\pgfsetstrokecolor{currentstroke}%
\pgfsetdash{}{0pt}%
\pgfsys@defobject{currentmarker}{\pgfqpoint{0.000000in}{-0.027778in}}{\pgfqpoint{0.000000in}{0.000000in}}{%
\pgfpathmoveto{\pgfqpoint{0.000000in}{0.000000in}}%
\pgfpathlineto{\pgfqpoint{0.000000in}{-0.027778in}}%
\pgfusepath{stroke,fill}%
}%
\begin{pgfscope}%
\pgfsys@transformshift{2.792484in}{0.417642in}%
\pgfsys@useobject{currentmarker}{}%
\end{pgfscope}%
\end{pgfscope}%
\begin{pgfscope}%
\pgfpathrectangle{\pgfqpoint{0.589510in}{0.417642in}}{\pgfqpoint{3.437062in}{2.055000in}}%
\pgfusepath{clip}%
\pgfsetrectcap%
\pgfsetroundjoin%
\pgfsetlinewidth{0.803000pt}%
\definecolor{currentstroke}{rgb}{0.850000,0.850000,0.850000}%
\pgfsetstrokecolor{currentstroke}%
\pgfsetdash{}{0pt}%
\pgfpathmoveto{\pgfqpoint{3.004223in}{0.417642in}}%
\pgfpathlineto{\pgfqpoint{3.004223in}{2.472642in}}%
\pgfusepath{stroke}%
\end{pgfscope}%
\begin{pgfscope}%
\pgfsetbuttcap%
\pgfsetroundjoin%
\definecolor{currentfill}{rgb}{0.000000,0.000000,0.000000}%
\pgfsetfillcolor{currentfill}%
\pgfsetlinewidth{0.602250pt}%
\definecolor{currentstroke}{rgb}{0.000000,0.000000,0.000000}%
\pgfsetstrokecolor{currentstroke}%
\pgfsetdash{}{0pt}%
\pgfsys@defobject{currentmarker}{\pgfqpoint{0.000000in}{-0.027778in}}{\pgfqpoint{0.000000in}{0.000000in}}{%
\pgfpathmoveto{\pgfqpoint{0.000000in}{0.000000in}}%
\pgfpathlineto{\pgfqpoint{0.000000in}{-0.027778in}}%
\pgfusepath{stroke,fill}%
}%
\begin{pgfscope}%
\pgfsys@transformshift{3.004223in}{0.417642in}%
\pgfsys@useobject{currentmarker}{}%
\end{pgfscope}%
\end{pgfscope}%
\begin{pgfscope}%
\pgfpathrectangle{\pgfqpoint{0.589510in}{0.417642in}}{\pgfqpoint{3.437062in}{2.055000in}}%
\pgfusepath{clip}%
\pgfsetrectcap%
\pgfsetroundjoin%
\pgfsetlinewidth{0.803000pt}%
\definecolor{currentstroke}{rgb}{0.850000,0.850000,0.850000}%
\pgfsetstrokecolor{currentstroke}%
\pgfsetdash{}{0pt}%
\pgfpathmoveto{\pgfqpoint{3.111739in}{0.417642in}}%
\pgfpathlineto{\pgfqpoint{3.111739in}{2.472642in}}%
\pgfusepath{stroke}%
\end{pgfscope}%
\begin{pgfscope}%
\pgfsetbuttcap%
\pgfsetroundjoin%
\definecolor{currentfill}{rgb}{0.000000,0.000000,0.000000}%
\pgfsetfillcolor{currentfill}%
\pgfsetlinewidth{0.602250pt}%
\definecolor{currentstroke}{rgb}{0.000000,0.000000,0.000000}%
\pgfsetstrokecolor{currentstroke}%
\pgfsetdash{}{0pt}%
\pgfsys@defobject{currentmarker}{\pgfqpoint{0.000000in}{-0.027778in}}{\pgfqpoint{0.000000in}{0.000000in}}{%
\pgfpathmoveto{\pgfqpoint{0.000000in}{0.000000in}}%
\pgfpathlineto{\pgfqpoint{0.000000in}{-0.027778in}}%
\pgfusepath{stroke,fill}%
}%
\begin{pgfscope}%
\pgfsys@transformshift{3.111739in}{0.417642in}%
\pgfsys@useobject{currentmarker}{}%
\end{pgfscope}%
\end{pgfscope}%
\begin{pgfscope}%
\pgfpathrectangle{\pgfqpoint{0.589510in}{0.417642in}}{\pgfqpoint{3.437062in}{2.055000in}}%
\pgfusepath{clip}%
\pgfsetrectcap%
\pgfsetroundjoin%
\pgfsetlinewidth{0.803000pt}%
\definecolor{currentstroke}{rgb}{0.850000,0.850000,0.850000}%
\pgfsetstrokecolor{currentstroke}%
\pgfsetdash{}{0pt}%
\pgfpathmoveto{\pgfqpoint{3.188023in}{0.417642in}}%
\pgfpathlineto{\pgfqpoint{3.188023in}{2.472642in}}%
\pgfusepath{stroke}%
\end{pgfscope}%
\begin{pgfscope}%
\pgfsetbuttcap%
\pgfsetroundjoin%
\definecolor{currentfill}{rgb}{0.000000,0.000000,0.000000}%
\pgfsetfillcolor{currentfill}%
\pgfsetlinewidth{0.602250pt}%
\definecolor{currentstroke}{rgb}{0.000000,0.000000,0.000000}%
\pgfsetstrokecolor{currentstroke}%
\pgfsetdash{}{0pt}%
\pgfsys@defobject{currentmarker}{\pgfqpoint{0.000000in}{-0.027778in}}{\pgfqpoint{0.000000in}{0.000000in}}{%
\pgfpathmoveto{\pgfqpoint{0.000000in}{0.000000in}}%
\pgfpathlineto{\pgfqpoint{0.000000in}{-0.027778in}}%
\pgfusepath{stroke,fill}%
}%
\begin{pgfscope}%
\pgfsys@transformshift{3.188023in}{0.417642in}%
\pgfsys@useobject{currentmarker}{}%
\end{pgfscope}%
\end{pgfscope}%
\begin{pgfscope}%
\pgfpathrectangle{\pgfqpoint{0.589510in}{0.417642in}}{\pgfqpoint{3.437062in}{2.055000in}}%
\pgfusepath{clip}%
\pgfsetrectcap%
\pgfsetroundjoin%
\pgfsetlinewidth{0.803000pt}%
\definecolor{currentstroke}{rgb}{0.850000,0.850000,0.850000}%
\pgfsetstrokecolor{currentstroke}%
\pgfsetdash{}{0pt}%
\pgfpathmoveto{\pgfqpoint{3.247193in}{0.417642in}}%
\pgfpathlineto{\pgfqpoint{3.247193in}{2.472642in}}%
\pgfusepath{stroke}%
\end{pgfscope}%
\begin{pgfscope}%
\pgfsetbuttcap%
\pgfsetroundjoin%
\definecolor{currentfill}{rgb}{0.000000,0.000000,0.000000}%
\pgfsetfillcolor{currentfill}%
\pgfsetlinewidth{0.602250pt}%
\definecolor{currentstroke}{rgb}{0.000000,0.000000,0.000000}%
\pgfsetstrokecolor{currentstroke}%
\pgfsetdash{}{0pt}%
\pgfsys@defobject{currentmarker}{\pgfqpoint{0.000000in}{-0.027778in}}{\pgfqpoint{0.000000in}{0.000000in}}{%
\pgfpathmoveto{\pgfqpoint{0.000000in}{0.000000in}}%
\pgfpathlineto{\pgfqpoint{0.000000in}{-0.027778in}}%
\pgfusepath{stroke,fill}%
}%
\begin{pgfscope}%
\pgfsys@transformshift{3.247193in}{0.417642in}%
\pgfsys@useobject{currentmarker}{}%
\end{pgfscope}%
\end{pgfscope}%
\begin{pgfscope}%
\pgfpathrectangle{\pgfqpoint{0.589510in}{0.417642in}}{\pgfqpoint{3.437062in}{2.055000in}}%
\pgfusepath{clip}%
\pgfsetrectcap%
\pgfsetroundjoin%
\pgfsetlinewidth{0.803000pt}%
\definecolor{currentstroke}{rgb}{0.850000,0.850000,0.850000}%
\pgfsetstrokecolor{currentstroke}%
\pgfsetdash{}{0pt}%
\pgfpathmoveto{\pgfqpoint{3.295539in}{0.417642in}}%
\pgfpathlineto{\pgfqpoint{3.295539in}{2.472642in}}%
\pgfusepath{stroke}%
\end{pgfscope}%
\begin{pgfscope}%
\pgfsetbuttcap%
\pgfsetroundjoin%
\definecolor{currentfill}{rgb}{0.000000,0.000000,0.000000}%
\pgfsetfillcolor{currentfill}%
\pgfsetlinewidth{0.602250pt}%
\definecolor{currentstroke}{rgb}{0.000000,0.000000,0.000000}%
\pgfsetstrokecolor{currentstroke}%
\pgfsetdash{}{0pt}%
\pgfsys@defobject{currentmarker}{\pgfqpoint{0.000000in}{-0.027778in}}{\pgfqpoint{0.000000in}{0.000000in}}{%
\pgfpathmoveto{\pgfqpoint{0.000000in}{0.000000in}}%
\pgfpathlineto{\pgfqpoint{0.000000in}{-0.027778in}}%
\pgfusepath{stroke,fill}%
}%
\begin{pgfscope}%
\pgfsys@transformshift{3.295539in}{0.417642in}%
\pgfsys@useobject{currentmarker}{}%
\end{pgfscope}%
\end{pgfscope}%
\begin{pgfscope}%
\pgfpathrectangle{\pgfqpoint{0.589510in}{0.417642in}}{\pgfqpoint{3.437062in}{2.055000in}}%
\pgfusepath{clip}%
\pgfsetrectcap%
\pgfsetroundjoin%
\pgfsetlinewidth{0.803000pt}%
\definecolor{currentstroke}{rgb}{0.850000,0.850000,0.850000}%
\pgfsetstrokecolor{currentstroke}%
\pgfsetdash{}{0pt}%
\pgfpathmoveto{\pgfqpoint{3.336415in}{0.417642in}}%
\pgfpathlineto{\pgfqpoint{3.336415in}{2.472642in}}%
\pgfusepath{stroke}%
\end{pgfscope}%
\begin{pgfscope}%
\pgfsetbuttcap%
\pgfsetroundjoin%
\definecolor{currentfill}{rgb}{0.000000,0.000000,0.000000}%
\pgfsetfillcolor{currentfill}%
\pgfsetlinewidth{0.602250pt}%
\definecolor{currentstroke}{rgb}{0.000000,0.000000,0.000000}%
\pgfsetstrokecolor{currentstroke}%
\pgfsetdash{}{0pt}%
\pgfsys@defobject{currentmarker}{\pgfqpoint{0.000000in}{-0.027778in}}{\pgfqpoint{0.000000in}{0.000000in}}{%
\pgfpathmoveto{\pgfqpoint{0.000000in}{0.000000in}}%
\pgfpathlineto{\pgfqpoint{0.000000in}{-0.027778in}}%
\pgfusepath{stroke,fill}%
}%
\begin{pgfscope}%
\pgfsys@transformshift{3.336415in}{0.417642in}%
\pgfsys@useobject{currentmarker}{}%
\end{pgfscope}%
\end{pgfscope}%
\begin{pgfscope}%
\pgfpathrectangle{\pgfqpoint{0.589510in}{0.417642in}}{\pgfqpoint{3.437062in}{2.055000in}}%
\pgfusepath{clip}%
\pgfsetrectcap%
\pgfsetroundjoin%
\pgfsetlinewidth{0.803000pt}%
\definecolor{currentstroke}{rgb}{0.850000,0.850000,0.850000}%
\pgfsetstrokecolor{currentstroke}%
\pgfsetdash{}{0pt}%
\pgfpathmoveto{\pgfqpoint{3.371823in}{0.417642in}}%
\pgfpathlineto{\pgfqpoint{3.371823in}{2.472642in}}%
\pgfusepath{stroke}%
\end{pgfscope}%
\begin{pgfscope}%
\pgfsetbuttcap%
\pgfsetroundjoin%
\definecolor{currentfill}{rgb}{0.000000,0.000000,0.000000}%
\pgfsetfillcolor{currentfill}%
\pgfsetlinewidth{0.602250pt}%
\definecolor{currentstroke}{rgb}{0.000000,0.000000,0.000000}%
\pgfsetstrokecolor{currentstroke}%
\pgfsetdash{}{0pt}%
\pgfsys@defobject{currentmarker}{\pgfqpoint{0.000000in}{-0.027778in}}{\pgfqpoint{0.000000in}{0.000000in}}{%
\pgfpathmoveto{\pgfqpoint{0.000000in}{0.000000in}}%
\pgfpathlineto{\pgfqpoint{0.000000in}{-0.027778in}}%
\pgfusepath{stroke,fill}%
}%
\begin{pgfscope}%
\pgfsys@transformshift{3.371823in}{0.417642in}%
\pgfsys@useobject{currentmarker}{}%
\end{pgfscope}%
\end{pgfscope}%
\begin{pgfscope}%
\pgfpathrectangle{\pgfqpoint{0.589510in}{0.417642in}}{\pgfqpoint{3.437062in}{2.055000in}}%
\pgfusepath{clip}%
\pgfsetrectcap%
\pgfsetroundjoin%
\pgfsetlinewidth{0.803000pt}%
\definecolor{currentstroke}{rgb}{0.850000,0.850000,0.850000}%
\pgfsetstrokecolor{currentstroke}%
\pgfsetdash{}{0pt}%
\pgfpathmoveto{\pgfqpoint{3.403055in}{0.417642in}}%
\pgfpathlineto{\pgfqpoint{3.403055in}{2.472642in}}%
\pgfusepath{stroke}%
\end{pgfscope}%
\begin{pgfscope}%
\pgfsetbuttcap%
\pgfsetroundjoin%
\definecolor{currentfill}{rgb}{0.000000,0.000000,0.000000}%
\pgfsetfillcolor{currentfill}%
\pgfsetlinewidth{0.602250pt}%
\definecolor{currentstroke}{rgb}{0.000000,0.000000,0.000000}%
\pgfsetstrokecolor{currentstroke}%
\pgfsetdash{}{0pt}%
\pgfsys@defobject{currentmarker}{\pgfqpoint{0.000000in}{-0.027778in}}{\pgfqpoint{0.000000in}{0.000000in}}{%
\pgfpathmoveto{\pgfqpoint{0.000000in}{0.000000in}}%
\pgfpathlineto{\pgfqpoint{0.000000in}{-0.027778in}}%
\pgfusepath{stroke,fill}%
}%
\begin{pgfscope}%
\pgfsys@transformshift{3.403055in}{0.417642in}%
\pgfsys@useobject{currentmarker}{}%
\end{pgfscope}%
\end{pgfscope}%
\begin{pgfscope}%
\pgfpathrectangle{\pgfqpoint{0.589510in}{0.417642in}}{\pgfqpoint{3.437062in}{2.055000in}}%
\pgfusepath{clip}%
\pgfsetrectcap%
\pgfsetroundjoin%
\pgfsetlinewidth{0.803000pt}%
\definecolor{currentstroke}{rgb}{0.850000,0.850000,0.850000}%
\pgfsetstrokecolor{currentstroke}%
\pgfsetdash{}{0pt}%
\pgfpathmoveto{\pgfqpoint{3.614793in}{0.417642in}}%
\pgfpathlineto{\pgfqpoint{3.614793in}{2.472642in}}%
\pgfusepath{stroke}%
\end{pgfscope}%
\begin{pgfscope}%
\pgfsetbuttcap%
\pgfsetroundjoin%
\definecolor{currentfill}{rgb}{0.000000,0.000000,0.000000}%
\pgfsetfillcolor{currentfill}%
\pgfsetlinewidth{0.602250pt}%
\definecolor{currentstroke}{rgb}{0.000000,0.000000,0.000000}%
\pgfsetstrokecolor{currentstroke}%
\pgfsetdash{}{0pt}%
\pgfsys@defobject{currentmarker}{\pgfqpoint{0.000000in}{-0.027778in}}{\pgfqpoint{0.000000in}{0.000000in}}{%
\pgfpathmoveto{\pgfqpoint{0.000000in}{0.000000in}}%
\pgfpathlineto{\pgfqpoint{0.000000in}{-0.027778in}}%
\pgfusepath{stroke,fill}%
}%
\begin{pgfscope}%
\pgfsys@transformshift{3.614793in}{0.417642in}%
\pgfsys@useobject{currentmarker}{}%
\end{pgfscope}%
\end{pgfscope}%
\begin{pgfscope}%
\pgfpathrectangle{\pgfqpoint{0.589510in}{0.417642in}}{\pgfqpoint{3.437062in}{2.055000in}}%
\pgfusepath{clip}%
\pgfsetrectcap%
\pgfsetroundjoin%
\pgfsetlinewidth{0.803000pt}%
\definecolor{currentstroke}{rgb}{0.850000,0.850000,0.850000}%
\pgfsetstrokecolor{currentstroke}%
\pgfsetdash{}{0pt}%
\pgfpathmoveto{\pgfqpoint{3.722309in}{0.417642in}}%
\pgfpathlineto{\pgfqpoint{3.722309in}{2.472642in}}%
\pgfusepath{stroke}%
\end{pgfscope}%
\begin{pgfscope}%
\pgfsetbuttcap%
\pgfsetroundjoin%
\definecolor{currentfill}{rgb}{0.000000,0.000000,0.000000}%
\pgfsetfillcolor{currentfill}%
\pgfsetlinewidth{0.602250pt}%
\definecolor{currentstroke}{rgb}{0.000000,0.000000,0.000000}%
\pgfsetstrokecolor{currentstroke}%
\pgfsetdash{}{0pt}%
\pgfsys@defobject{currentmarker}{\pgfqpoint{0.000000in}{-0.027778in}}{\pgfqpoint{0.000000in}{0.000000in}}{%
\pgfpathmoveto{\pgfqpoint{0.000000in}{0.000000in}}%
\pgfpathlineto{\pgfqpoint{0.000000in}{-0.027778in}}%
\pgfusepath{stroke,fill}%
}%
\begin{pgfscope}%
\pgfsys@transformshift{3.722309in}{0.417642in}%
\pgfsys@useobject{currentmarker}{}%
\end{pgfscope}%
\end{pgfscope}%
\begin{pgfscope}%
\pgfpathrectangle{\pgfqpoint{0.589510in}{0.417642in}}{\pgfqpoint{3.437062in}{2.055000in}}%
\pgfusepath{clip}%
\pgfsetrectcap%
\pgfsetroundjoin%
\pgfsetlinewidth{0.803000pt}%
\definecolor{currentstroke}{rgb}{0.850000,0.850000,0.850000}%
\pgfsetstrokecolor{currentstroke}%
\pgfsetdash{}{0pt}%
\pgfpathmoveto{\pgfqpoint{3.798593in}{0.417642in}}%
\pgfpathlineto{\pgfqpoint{3.798593in}{2.472642in}}%
\pgfusepath{stroke}%
\end{pgfscope}%
\begin{pgfscope}%
\pgfsetbuttcap%
\pgfsetroundjoin%
\definecolor{currentfill}{rgb}{0.000000,0.000000,0.000000}%
\pgfsetfillcolor{currentfill}%
\pgfsetlinewidth{0.602250pt}%
\definecolor{currentstroke}{rgb}{0.000000,0.000000,0.000000}%
\pgfsetstrokecolor{currentstroke}%
\pgfsetdash{}{0pt}%
\pgfsys@defobject{currentmarker}{\pgfqpoint{0.000000in}{-0.027778in}}{\pgfqpoint{0.000000in}{0.000000in}}{%
\pgfpathmoveto{\pgfqpoint{0.000000in}{0.000000in}}%
\pgfpathlineto{\pgfqpoint{0.000000in}{-0.027778in}}%
\pgfusepath{stroke,fill}%
}%
\begin{pgfscope}%
\pgfsys@transformshift{3.798593in}{0.417642in}%
\pgfsys@useobject{currentmarker}{}%
\end{pgfscope}%
\end{pgfscope}%
\begin{pgfscope}%
\pgfpathrectangle{\pgfqpoint{0.589510in}{0.417642in}}{\pgfqpoint{3.437062in}{2.055000in}}%
\pgfusepath{clip}%
\pgfsetrectcap%
\pgfsetroundjoin%
\pgfsetlinewidth{0.803000pt}%
\definecolor{currentstroke}{rgb}{0.850000,0.850000,0.850000}%
\pgfsetstrokecolor{currentstroke}%
\pgfsetdash{}{0pt}%
\pgfpathmoveto{\pgfqpoint{3.857764in}{0.417642in}}%
\pgfpathlineto{\pgfqpoint{3.857764in}{2.472642in}}%
\pgfusepath{stroke}%
\end{pgfscope}%
\begin{pgfscope}%
\pgfsetbuttcap%
\pgfsetroundjoin%
\definecolor{currentfill}{rgb}{0.000000,0.000000,0.000000}%
\pgfsetfillcolor{currentfill}%
\pgfsetlinewidth{0.602250pt}%
\definecolor{currentstroke}{rgb}{0.000000,0.000000,0.000000}%
\pgfsetstrokecolor{currentstroke}%
\pgfsetdash{}{0pt}%
\pgfsys@defobject{currentmarker}{\pgfqpoint{0.000000in}{-0.027778in}}{\pgfqpoint{0.000000in}{0.000000in}}{%
\pgfpathmoveto{\pgfqpoint{0.000000in}{0.000000in}}%
\pgfpathlineto{\pgfqpoint{0.000000in}{-0.027778in}}%
\pgfusepath{stroke,fill}%
}%
\begin{pgfscope}%
\pgfsys@transformshift{3.857764in}{0.417642in}%
\pgfsys@useobject{currentmarker}{}%
\end{pgfscope}%
\end{pgfscope}%
\begin{pgfscope}%
\pgfpathrectangle{\pgfqpoint{0.589510in}{0.417642in}}{\pgfqpoint{3.437062in}{2.055000in}}%
\pgfusepath{clip}%
\pgfsetrectcap%
\pgfsetroundjoin%
\pgfsetlinewidth{0.803000pt}%
\definecolor{currentstroke}{rgb}{0.850000,0.850000,0.850000}%
\pgfsetstrokecolor{currentstroke}%
\pgfsetdash{}{0pt}%
\pgfpathmoveto{\pgfqpoint{3.906110in}{0.417642in}}%
\pgfpathlineto{\pgfqpoint{3.906110in}{2.472642in}}%
\pgfusepath{stroke}%
\end{pgfscope}%
\begin{pgfscope}%
\pgfsetbuttcap%
\pgfsetroundjoin%
\definecolor{currentfill}{rgb}{0.000000,0.000000,0.000000}%
\pgfsetfillcolor{currentfill}%
\pgfsetlinewidth{0.602250pt}%
\definecolor{currentstroke}{rgb}{0.000000,0.000000,0.000000}%
\pgfsetstrokecolor{currentstroke}%
\pgfsetdash{}{0pt}%
\pgfsys@defobject{currentmarker}{\pgfqpoint{0.000000in}{-0.027778in}}{\pgfqpoint{0.000000in}{0.000000in}}{%
\pgfpathmoveto{\pgfqpoint{0.000000in}{0.000000in}}%
\pgfpathlineto{\pgfqpoint{0.000000in}{-0.027778in}}%
\pgfusepath{stroke,fill}%
}%
\begin{pgfscope}%
\pgfsys@transformshift{3.906110in}{0.417642in}%
\pgfsys@useobject{currentmarker}{}%
\end{pgfscope}%
\end{pgfscope}%
\begin{pgfscope}%
\pgfpathrectangle{\pgfqpoint{0.589510in}{0.417642in}}{\pgfqpoint{3.437062in}{2.055000in}}%
\pgfusepath{clip}%
\pgfsetrectcap%
\pgfsetroundjoin%
\pgfsetlinewidth{0.803000pt}%
\definecolor{currentstroke}{rgb}{0.850000,0.850000,0.850000}%
\pgfsetstrokecolor{currentstroke}%
\pgfsetdash{}{0pt}%
\pgfpathmoveto{\pgfqpoint{3.946985in}{0.417642in}}%
\pgfpathlineto{\pgfqpoint{3.946985in}{2.472642in}}%
\pgfusepath{stroke}%
\end{pgfscope}%
\begin{pgfscope}%
\pgfsetbuttcap%
\pgfsetroundjoin%
\definecolor{currentfill}{rgb}{0.000000,0.000000,0.000000}%
\pgfsetfillcolor{currentfill}%
\pgfsetlinewidth{0.602250pt}%
\definecolor{currentstroke}{rgb}{0.000000,0.000000,0.000000}%
\pgfsetstrokecolor{currentstroke}%
\pgfsetdash{}{0pt}%
\pgfsys@defobject{currentmarker}{\pgfqpoint{0.000000in}{-0.027778in}}{\pgfqpoint{0.000000in}{0.000000in}}{%
\pgfpathmoveto{\pgfqpoint{0.000000in}{0.000000in}}%
\pgfpathlineto{\pgfqpoint{0.000000in}{-0.027778in}}%
\pgfusepath{stroke,fill}%
}%
\begin{pgfscope}%
\pgfsys@transformshift{3.946985in}{0.417642in}%
\pgfsys@useobject{currentmarker}{}%
\end{pgfscope}%
\end{pgfscope}%
\begin{pgfscope}%
\pgfpathrectangle{\pgfqpoint{0.589510in}{0.417642in}}{\pgfqpoint{3.437062in}{2.055000in}}%
\pgfusepath{clip}%
\pgfsetrectcap%
\pgfsetroundjoin%
\pgfsetlinewidth{0.803000pt}%
\definecolor{currentstroke}{rgb}{0.850000,0.850000,0.850000}%
\pgfsetstrokecolor{currentstroke}%
\pgfsetdash{}{0pt}%
\pgfpathmoveto{\pgfqpoint{3.982393in}{0.417642in}}%
\pgfpathlineto{\pgfqpoint{3.982393in}{2.472642in}}%
\pgfusepath{stroke}%
\end{pgfscope}%
\begin{pgfscope}%
\pgfsetbuttcap%
\pgfsetroundjoin%
\definecolor{currentfill}{rgb}{0.000000,0.000000,0.000000}%
\pgfsetfillcolor{currentfill}%
\pgfsetlinewidth{0.602250pt}%
\definecolor{currentstroke}{rgb}{0.000000,0.000000,0.000000}%
\pgfsetstrokecolor{currentstroke}%
\pgfsetdash{}{0pt}%
\pgfsys@defobject{currentmarker}{\pgfqpoint{0.000000in}{-0.027778in}}{\pgfqpoint{0.000000in}{0.000000in}}{%
\pgfpathmoveto{\pgfqpoint{0.000000in}{0.000000in}}%
\pgfpathlineto{\pgfqpoint{0.000000in}{-0.027778in}}%
\pgfusepath{stroke,fill}%
}%
\begin{pgfscope}%
\pgfsys@transformshift{3.982393in}{0.417642in}%
\pgfsys@useobject{currentmarker}{}%
\end{pgfscope}%
\end{pgfscope}%
\begin{pgfscope}%
\pgfpathrectangle{\pgfqpoint{0.589510in}{0.417642in}}{\pgfqpoint{3.437062in}{2.055000in}}%
\pgfusepath{clip}%
\pgfsetrectcap%
\pgfsetroundjoin%
\pgfsetlinewidth{0.803000pt}%
\definecolor{currentstroke}{rgb}{0.850000,0.850000,0.850000}%
\pgfsetstrokecolor{currentstroke}%
\pgfsetdash{}{0pt}%
\pgfpathmoveto{\pgfqpoint{4.013626in}{0.417642in}}%
\pgfpathlineto{\pgfqpoint{4.013626in}{2.472642in}}%
\pgfusepath{stroke}%
\end{pgfscope}%
\begin{pgfscope}%
\pgfsetbuttcap%
\pgfsetroundjoin%
\definecolor{currentfill}{rgb}{0.000000,0.000000,0.000000}%
\pgfsetfillcolor{currentfill}%
\pgfsetlinewidth{0.602250pt}%
\definecolor{currentstroke}{rgb}{0.000000,0.000000,0.000000}%
\pgfsetstrokecolor{currentstroke}%
\pgfsetdash{}{0pt}%
\pgfsys@defobject{currentmarker}{\pgfqpoint{0.000000in}{-0.027778in}}{\pgfqpoint{0.000000in}{0.000000in}}{%
\pgfpathmoveto{\pgfqpoint{0.000000in}{0.000000in}}%
\pgfpathlineto{\pgfqpoint{0.000000in}{-0.027778in}}%
\pgfusepath{stroke,fill}%
}%
\begin{pgfscope}%
\pgfsys@transformshift{4.013626in}{0.417642in}%
\pgfsys@useobject{currentmarker}{}%
\end{pgfscope}%
\end{pgfscope}%
\begin{pgfscope}%
\definecolor{textcolor}{rgb}{0.000000,0.000000,0.000000}%
\pgfsetstrokecolor{textcolor}%
\pgfsetfillcolor{textcolor}%
\pgftext[x=2.308041in,y=0.165003in,,top]{\color{textcolor}\rmfamily\fontsize{10.000000}{12.000000}\selectfont \(\displaystyle \tau\) in \unit{\second}}%
\end{pgfscope}%
\begin{pgfscope}%
\pgfpathrectangle{\pgfqpoint{0.589510in}{0.417642in}}{\pgfqpoint{3.437062in}{2.055000in}}%
\pgfusepath{clip}%
\pgfsetrectcap%
\pgfsetroundjoin%
\pgfsetlinewidth{0.803000pt}%
\definecolor{currentstroke}{rgb}{0.450000,0.450000,0.450000}%
\pgfsetstrokecolor{currentstroke}%
\pgfsetdash{}{0pt}%
\pgfpathmoveto{\pgfqpoint{0.589510in}{0.505447in}}%
\pgfpathlineto{\pgfqpoint{4.026572in}{0.505447in}}%
\pgfusepath{stroke}%
\end{pgfscope}%
\begin{pgfscope}%
\pgfsetbuttcap%
\pgfsetroundjoin%
\definecolor{currentfill}{rgb}{0.000000,0.000000,0.000000}%
\pgfsetfillcolor{currentfill}%
\pgfsetlinewidth{0.803000pt}%
\definecolor{currentstroke}{rgb}{0.000000,0.000000,0.000000}%
\pgfsetstrokecolor{currentstroke}%
\pgfsetdash{}{0pt}%
\pgfsys@defobject{currentmarker}{\pgfqpoint{-0.048611in}{0.000000in}}{\pgfqpoint{-0.000000in}{0.000000in}}{%
\pgfpathmoveto{\pgfqpoint{-0.000000in}{0.000000in}}%
\pgfpathlineto{\pgfqpoint{-0.048611in}{0.000000in}}%
\pgfusepath{stroke,fill}%
}%
\begin{pgfscope}%
\pgfsys@transformshift{0.589510in}{0.505447in}%
\pgfsys@useobject{currentmarker}{}%
\end{pgfscope}%
\end{pgfscope}%
\begin{pgfscope}%
\definecolor{textcolor}{rgb}{0.000000,0.000000,0.000000}%
\pgfsetstrokecolor{textcolor}%
\pgfsetfillcolor{textcolor}%
\pgftext[x=0.236114in, y=0.466294in, left, base]{\color{textcolor}\rmfamily\fontsize{8.000000}{9.600000}\selectfont \(\displaystyle {10^{-9}}\)}%
\end{pgfscope}%
\begin{pgfscope}%
\pgfpathrectangle{\pgfqpoint{0.589510in}{0.417642in}}{\pgfqpoint{3.437062in}{2.055000in}}%
\pgfusepath{clip}%
\pgfsetrectcap%
\pgfsetroundjoin%
\pgfsetlinewidth{0.803000pt}%
\definecolor{currentstroke}{rgb}{0.450000,0.450000,0.450000}%
\pgfsetstrokecolor{currentstroke}%
\pgfsetdash{}{0pt}%
\pgfpathmoveto{\pgfqpoint{0.589510in}{1.187752in}}%
\pgfpathlineto{\pgfqpoint{4.026572in}{1.187752in}}%
\pgfusepath{stroke}%
\end{pgfscope}%
\begin{pgfscope}%
\pgfsetbuttcap%
\pgfsetroundjoin%
\definecolor{currentfill}{rgb}{0.000000,0.000000,0.000000}%
\pgfsetfillcolor{currentfill}%
\pgfsetlinewidth{0.803000pt}%
\definecolor{currentstroke}{rgb}{0.000000,0.000000,0.000000}%
\pgfsetstrokecolor{currentstroke}%
\pgfsetdash{}{0pt}%
\pgfsys@defobject{currentmarker}{\pgfqpoint{-0.048611in}{0.000000in}}{\pgfqpoint{-0.000000in}{0.000000in}}{%
\pgfpathmoveto{\pgfqpoint{-0.000000in}{0.000000in}}%
\pgfpathlineto{\pgfqpoint{-0.048611in}{0.000000in}}%
\pgfusepath{stroke,fill}%
}%
\begin{pgfscope}%
\pgfsys@transformshift{0.589510in}{1.187752in}%
\pgfsys@useobject{currentmarker}{}%
\end{pgfscope}%
\end{pgfscope}%
\begin{pgfscope}%
\definecolor{textcolor}{rgb}{0.000000,0.000000,0.000000}%
\pgfsetstrokecolor{textcolor}%
\pgfsetfillcolor{textcolor}%
\pgftext[x=0.236114in, y=1.148599in, left, base]{\color{textcolor}\rmfamily\fontsize{8.000000}{9.600000}\selectfont \(\displaystyle {10^{-8}}\)}%
\end{pgfscope}%
\begin{pgfscope}%
\pgfpathrectangle{\pgfqpoint{0.589510in}{0.417642in}}{\pgfqpoint{3.437062in}{2.055000in}}%
\pgfusepath{clip}%
\pgfsetrectcap%
\pgfsetroundjoin%
\pgfsetlinewidth{0.803000pt}%
\definecolor{currentstroke}{rgb}{0.450000,0.450000,0.450000}%
\pgfsetstrokecolor{currentstroke}%
\pgfsetdash{}{0pt}%
\pgfpathmoveto{\pgfqpoint{0.589510in}{1.870057in}}%
\pgfpathlineto{\pgfqpoint{4.026572in}{1.870057in}}%
\pgfusepath{stroke}%
\end{pgfscope}%
\begin{pgfscope}%
\pgfsetbuttcap%
\pgfsetroundjoin%
\definecolor{currentfill}{rgb}{0.000000,0.000000,0.000000}%
\pgfsetfillcolor{currentfill}%
\pgfsetlinewidth{0.803000pt}%
\definecolor{currentstroke}{rgb}{0.000000,0.000000,0.000000}%
\pgfsetstrokecolor{currentstroke}%
\pgfsetdash{}{0pt}%
\pgfsys@defobject{currentmarker}{\pgfqpoint{-0.048611in}{0.000000in}}{\pgfqpoint{-0.000000in}{0.000000in}}{%
\pgfpathmoveto{\pgfqpoint{-0.000000in}{0.000000in}}%
\pgfpathlineto{\pgfqpoint{-0.048611in}{0.000000in}}%
\pgfusepath{stroke,fill}%
}%
\begin{pgfscope}%
\pgfsys@transformshift{0.589510in}{1.870057in}%
\pgfsys@useobject{currentmarker}{}%
\end{pgfscope}%
\end{pgfscope}%
\begin{pgfscope}%
\definecolor{textcolor}{rgb}{0.000000,0.000000,0.000000}%
\pgfsetstrokecolor{textcolor}%
\pgfsetfillcolor{textcolor}%
\pgftext[x=0.236114in, y=1.830904in, left, base]{\color{textcolor}\rmfamily\fontsize{8.000000}{9.600000}\selectfont \(\displaystyle {10^{-7}}\)}%
\end{pgfscope}%
\begin{pgfscope}%
\pgfpathrectangle{\pgfqpoint{0.589510in}{0.417642in}}{\pgfqpoint{3.437062in}{2.055000in}}%
\pgfusepath{clip}%
\pgfsetrectcap%
\pgfsetroundjoin%
\pgfsetlinewidth{0.803000pt}%
\definecolor{currentstroke}{rgb}{0.850000,0.850000,0.850000}%
\pgfsetstrokecolor{currentstroke}%
\pgfsetdash{}{0pt}%
\pgfpathmoveto{\pgfqpoint{0.589510in}{0.439324in}}%
\pgfpathlineto{\pgfqpoint{4.026572in}{0.439324in}}%
\pgfusepath{stroke}%
\end{pgfscope}%
\begin{pgfscope}%
\pgfsetbuttcap%
\pgfsetroundjoin%
\definecolor{currentfill}{rgb}{0.000000,0.000000,0.000000}%
\pgfsetfillcolor{currentfill}%
\pgfsetlinewidth{0.602250pt}%
\definecolor{currentstroke}{rgb}{0.000000,0.000000,0.000000}%
\pgfsetstrokecolor{currentstroke}%
\pgfsetdash{}{0pt}%
\pgfsys@defobject{currentmarker}{\pgfqpoint{-0.027778in}{0.000000in}}{\pgfqpoint{-0.000000in}{0.000000in}}{%
\pgfpathmoveto{\pgfqpoint{-0.000000in}{0.000000in}}%
\pgfpathlineto{\pgfqpoint{-0.027778in}{0.000000in}}%
\pgfusepath{stroke,fill}%
}%
\begin{pgfscope}%
\pgfsys@transformshift{0.589510in}{0.439324in}%
\pgfsys@useobject{currentmarker}{}%
\end{pgfscope}%
\end{pgfscope}%
\begin{pgfscope}%
\pgfpathrectangle{\pgfqpoint{0.589510in}{0.417642in}}{\pgfqpoint{3.437062in}{2.055000in}}%
\pgfusepath{clip}%
\pgfsetrectcap%
\pgfsetroundjoin%
\pgfsetlinewidth{0.803000pt}%
\definecolor{currentstroke}{rgb}{0.850000,0.850000,0.850000}%
\pgfsetstrokecolor{currentstroke}%
\pgfsetdash{}{0pt}%
\pgfpathmoveto{\pgfqpoint{0.589510in}{0.474226in}}%
\pgfpathlineto{\pgfqpoint{4.026572in}{0.474226in}}%
\pgfusepath{stroke}%
\end{pgfscope}%
\begin{pgfscope}%
\pgfsetbuttcap%
\pgfsetroundjoin%
\definecolor{currentfill}{rgb}{0.000000,0.000000,0.000000}%
\pgfsetfillcolor{currentfill}%
\pgfsetlinewidth{0.602250pt}%
\definecolor{currentstroke}{rgb}{0.000000,0.000000,0.000000}%
\pgfsetstrokecolor{currentstroke}%
\pgfsetdash{}{0pt}%
\pgfsys@defobject{currentmarker}{\pgfqpoint{-0.027778in}{0.000000in}}{\pgfqpoint{-0.000000in}{0.000000in}}{%
\pgfpathmoveto{\pgfqpoint{-0.000000in}{0.000000in}}%
\pgfpathlineto{\pgfqpoint{-0.027778in}{0.000000in}}%
\pgfusepath{stroke,fill}%
}%
\begin{pgfscope}%
\pgfsys@transformshift{0.589510in}{0.474226in}%
\pgfsys@useobject{currentmarker}{}%
\end{pgfscope}%
\end{pgfscope}%
\begin{pgfscope}%
\pgfpathrectangle{\pgfqpoint{0.589510in}{0.417642in}}{\pgfqpoint{3.437062in}{2.055000in}}%
\pgfusepath{clip}%
\pgfsetrectcap%
\pgfsetroundjoin%
\pgfsetlinewidth{0.803000pt}%
\definecolor{currentstroke}{rgb}{0.850000,0.850000,0.850000}%
\pgfsetstrokecolor{currentstroke}%
\pgfsetdash{}{0pt}%
\pgfpathmoveto{\pgfqpoint{0.589510in}{0.710841in}}%
\pgfpathlineto{\pgfqpoint{4.026572in}{0.710841in}}%
\pgfusepath{stroke}%
\end{pgfscope}%
\begin{pgfscope}%
\pgfsetbuttcap%
\pgfsetroundjoin%
\definecolor{currentfill}{rgb}{0.000000,0.000000,0.000000}%
\pgfsetfillcolor{currentfill}%
\pgfsetlinewidth{0.602250pt}%
\definecolor{currentstroke}{rgb}{0.000000,0.000000,0.000000}%
\pgfsetstrokecolor{currentstroke}%
\pgfsetdash{}{0pt}%
\pgfsys@defobject{currentmarker}{\pgfqpoint{-0.027778in}{0.000000in}}{\pgfqpoint{-0.000000in}{0.000000in}}{%
\pgfpathmoveto{\pgfqpoint{-0.000000in}{0.000000in}}%
\pgfpathlineto{\pgfqpoint{-0.027778in}{0.000000in}}%
\pgfusepath{stroke,fill}%
}%
\begin{pgfscope}%
\pgfsys@transformshift{0.589510in}{0.710841in}%
\pgfsys@useobject{currentmarker}{}%
\end{pgfscope}%
\end{pgfscope}%
\begin{pgfscope}%
\pgfpathrectangle{\pgfqpoint{0.589510in}{0.417642in}}{\pgfqpoint{3.437062in}{2.055000in}}%
\pgfusepath{clip}%
\pgfsetrectcap%
\pgfsetroundjoin%
\pgfsetlinewidth{0.803000pt}%
\definecolor{currentstroke}{rgb}{0.850000,0.850000,0.850000}%
\pgfsetstrokecolor{currentstroke}%
\pgfsetdash{}{0pt}%
\pgfpathmoveto{\pgfqpoint{0.589510in}{0.830989in}}%
\pgfpathlineto{\pgfqpoint{4.026572in}{0.830989in}}%
\pgfusepath{stroke}%
\end{pgfscope}%
\begin{pgfscope}%
\pgfsetbuttcap%
\pgfsetroundjoin%
\definecolor{currentfill}{rgb}{0.000000,0.000000,0.000000}%
\pgfsetfillcolor{currentfill}%
\pgfsetlinewidth{0.602250pt}%
\definecolor{currentstroke}{rgb}{0.000000,0.000000,0.000000}%
\pgfsetstrokecolor{currentstroke}%
\pgfsetdash{}{0pt}%
\pgfsys@defobject{currentmarker}{\pgfqpoint{-0.027778in}{0.000000in}}{\pgfqpoint{-0.000000in}{0.000000in}}{%
\pgfpathmoveto{\pgfqpoint{-0.000000in}{0.000000in}}%
\pgfpathlineto{\pgfqpoint{-0.027778in}{0.000000in}}%
\pgfusepath{stroke,fill}%
}%
\begin{pgfscope}%
\pgfsys@transformshift{0.589510in}{0.830989in}%
\pgfsys@useobject{currentmarker}{}%
\end{pgfscope}%
\end{pgfscope}%
\begin{pgfscope}%
\pgfpathrectangle{\pgfqpoint{0.589510in}{0.417642in}}{\pgfqpoint{3.437062in}{2.055000in}}%
\pgfusepath{clip}%
\pgfsetrectcap%
\pgfsetroundjoin%
\pgfsetlinewidth{0.803000pt}%
\definecolor{currentstroke}{rgb}{0.850000,0.850000,0.850000}%
\pgfsetstrokecolor{currentstroke}%
\pgfsetdash{}{0pt}%
\pgfpathmoveto{\pgfqpoint{0.589510in}{0.916235in}}%
\pgfpathlineto{\pgfqpoint{4.026572in}{0.916235in}}%
\pgfusepath{stroke}%
\end{pgfscope}%
\begin{pgfscope}%
\pgfsetbuttcap%
\pgfsetroundjoin%
\definecolor{currentfill}{rgb}{0.000000,0.000000,0.000000}%
\pgfsetfillcolor{currentfill}%
\pgfsetlinewidth{0.602250pt}%
\definecolor{currentstroke}{rgb}{0.000000,0.000000,0.000000}%
\pgfsetstrokecolor{currentstroke}%
\pgfsetdash{}{0pt}%
\pgfsys@defobject{currentmarker}{\pgfqpoint{-0.027778in}{0.000000in}}{\pgfqpoint{-0.000000in}{0.000000in}}{%
\pgfpathmoveto{\pgfqpoint{-0.000000in}{0.000000in}}%
\pgfpathlineto{\pgfqpoint{-0.027778in}{0.000000in}}%
\pgfusepath{stroke,fill}%
}%
\begin{pgfscope}%
\pgfsys@transformshift{0.589510in}{0.916235in}%
\pgfsys@useobject{currentmarker}{}%
\end{pgfscope}%
\end{pgfscope}%
\begin{pgfscope}%
\pgfpathrectangle{\pgfqpoint{0.589510in}{0.417642in}}{\pgfqpoint{3.437062in}{2.055000in}}%
\pgfusepath{clip}%
\pgfsetrectcap%
\pgfsetroundjoin%
\pgfsetlinewidth{0.803000pt}%
\definecolor{currentstroke}{rgb}{0.850000,0.850000,0.850000}%
\pgfsetstrokecolor{currentstroke}%
\pgfsetdash{}{0pt}%
\pgfpathmoveto{\pgfqpoint{0.589510in}{0.982357in}}%
\pgfpathlineto{\pgfqpoint{4.026572in}{0.982357in}}%
\pgfusepath{stroke}%
\end{pgfscope}%
\begin{pgfscope}%
\pgfsetbuttcap%
\pgfsetroundjoin%
\definecolor{currentfill}{rgb}{0.000000,0.000000,0.000000}%
\pgfsetfillcolor{currentfill}%
\pgfsetlinewidth{0.602250pt}%
\definecolor{currentstroke}{rgb}{0.000000,0.000000,0.000000}%
\pgfsetstrokecolor{currentstroke}%
\pgfsetdash{}{0pt}%
\pgfsys@defobject{currentmarker}{\pgfqpoint{-0.027778in}{0.000000in}}{\pgfqpoint{-0.000000in}{0.000000in}}{%
\pgfpathmoveto{\pgfqpoint{-0.000000in}{0.000000in}}%
\pgfpathlineto{\pgfqpoint{-0.027778in}{0.000000in}}%
\pgfusepath{stroke,fill}%
}%
\begin{pgfscope}%
\pgfsys@transformshift{0.589510in}{0.982357in}%
\pgfsys@useobject{currentmarker}{}%
\end{pgfscope}%
\end{pgfscope}%
\begin{pgfscope}%
\pgfpathrectangle{\pgfqpoint{0.589510in}{0.417642in}}{\pgfqpoint{3.437062in}{2.055000in}}%
\pgfusepath{clip}%
\pgfsetrectcap%
\pgfsetroundjoin%
\pgfsetlinewidth{0.803000pt}%
\definecolor{currentstroke}{rgb}{0.850000,0.850000,0.850000}%
\pgfsetstrokecolor{currentstroke}%
\pgfsetdash{}{0pt}%
\pgfpathmoveto{\pgfqpoint{0.589510in}{1.036383in}}%
\pgfpathlineto{\pgfqpoint{4.026572in}{1.036383in}}%
\pgfusepath{stroke}%
\end{pgfscope}%
\begin{pgfscope}%
\pgfsetbuttcap%
\pgfsetroundjoin%
\definecolor{currentfill}{rgb}{0.000000,0.000000,0.000000}%
\pgfsetfillcolor{currentfill}%
\pgfsetlinewidth{0.602250pt}%
\definecolor{currentstroke}{rgb}{0.000000,0.000000,0.000000}%
\pgfsetstrokecolor{currentstroke}%
\pgfsetdash{}{0pt}%
\pgfsys@defobject{currentmarker}{\pgfqpoint{-0.027778in}{0.000000in}}{\pgfqpoint{-0.000000in}{0.000000in}}{%
\pgfpathmoveto{\pgfqpoint{-0.000000in}{0.000000in}}%
\pgfpathlineto{\pgfqpoint{-0.027778in}{0.000000in}}%
\pgfusepath{stroke,fill}%
}%
\begin{pgfscope}%
\pgfsys@transformshift{0.589510in}{1.036383in}%
\pgfsys@useobject{currentmarker}{}%
\end{pgfscope}%
\end{pgfscope}%
\begin{pgfscope}%
\pgfpathrectangle{\pgfqpoint{0.589510in}{0.417642in}}{\pgfqpoint{3.437062in}{2.055000in}}%
\pgfusepath{clip}%
\pgfsetrectcap%
\pgfsetroundjoin%
\pgfsetlinewidth{0.803000pt}%
\definecolor{currentstroke}{rgb}{0.850000,0.850000,0.850000}%
\pgfsetstrokecolor{currentstroke}%
\pgfsetdash{}{0pt}%
\pgfpathmoveto{\pgfqpoint{0.589510in}{1.082061in}}%
\pgfpathlineto{\pgfqpoint{4.026572in}{1.082061in}}%
\pgfusepath{stroke}%
\end{pgfscope}%
\begin{pgfscope}%
\pgfsetbuttcap%
\pgfsetroundjoin%
\definecolor{currentfill}{rgb}{0.000000,0.000000,0.000000}%
\pgfsetfillcolor{currentfill}%
\pgfsetlinewidth{0.602250pt}%
\definecolor{currentstroke}{rgb}{0.000000,0.000000,0.000000}%
\pgfsetstrokecolor{currentstroke}%
\pgfsetdash{}{0pt}%
\pgfsys@defobject{currentmarker}{\pgfqpoint{-0.027778in}{0.000000in}}{\pgfqpoint{-0.000000in}{0.000000in}}{%
\pgfpathmoveto{\pgfqpoint{-0.000000in}{0.000000in}}%
\pgfpathlineto{\pgfqpoint{-0.027778in}{0.000000in}}%
\pgfusepath{stroke,fill}%
}%
\begin{pgfscope}%
\pgfsys@transformshift{0.589510in}{1.082061in}%
\pgfsys@useobject{currentmarker}{}%
\end{pgfscope}%
\end{pgfscope}%
\begin{pgfscope}%
\pgfpathrectangle{\pgfqpoint{0.589510in}{0.417642in}}{\pgfqpoint{3.437062in}{2.055000in}}%
\pgfusepath{clip}%
\pgfsetrectcap%
\pgfsetroundjoin%
\pgfsetlinewidth{0.803000pt}%
\definecolor{currentstroke}{rgb}{0.850000,0.850000,0.850000}%
\pgfsetstrokecolor{currentstroke}%
\pgfsetdash{}{0pt}%
\pgfpathmoveto{\pgfqpoint{0.589510in}{1.121630in}}%
\pgfpathlineto{\pgfqpoint{4.026572in}{1.121630in}}%
\pgfusepath{stroke}%
\end{pgfscope}%
\begin{pgfscope}%
\pgfsetbuttcap%
\pgfsetroundjoin%
\definecolor{currentfill}{rgb}{0.000000,0.000000,0.000000}%
\pgfsetfillcolor{currentfill}%
\pgfsetlinewidth{0.602250pt}%
\definecolor{currentstroke}{rgb}{0.000000,0.000000,0.000000}%
\pgfsetstrokecolor{currentstroke}%
\pgfsetdash{}{0pt}%
\pgfsys@defobject{currentmarker}{\pgfqpoint{-0.027778in}{0.000000in}}{\pgfqpoint{-0.000000in}{0.000000in}}{%
\pgfpathmoveto{\pgfqpoint{-0.000000in}{0.000000in}}%
\pgfpathlineto{\pgfqpoint{-0.027778in}{0.000000in}}%
\pgfusepath{stroke,fill}%
}%
\begin{pgfscope}%
\pgfsys@transformshift{0.589510in}{1.121630in}%
\pgfsys@useobject{currentmarker}{}%
\end{pgfscope}%
\end{pgfscope}%
\begin{pgfscope}%
\pgfpathrectangle{\pgfqpoint{0.589510in}{0.417642in}}{\pgfqpoint{3.437062in}{2.055000in}}%
\pgfusepath{clip}%
\pgfsetrectcap%
\pgfsetroundjoin%
\pgfsetlinewidth{0.803000pt}%
\definecolor{currentstroke}{rgb}{0.850000,0.850000,0.850000}%
\pgfsetstrokecolor{currentstroke}%
\pgfsetdash{}{0pt}%
\pgfpathmoveto{\pgfqpoint{0.589510in}{1.156531in}}%
\pgfpathlineto{\pgfqpoint{4.026572in}{1.156531in}}%
\pgfusepath{stroke}%
\end{pgfscope}%
\begin{pgfscope}%
\pgfsetbuttcap%
\pgfsetroundjoin%
\definecolor{currentfill}{rgb}{0.000000,0.000000,0.000000}%
\pgfsetfillcolor{currentfill}%
\pgfsetlinewidth{0.602250pt}%
\definecolor{currentstroke}{rgb}{0.000000,0.000000,0.000000}%
\pgfsetstrokecolor{currentstroke}%
\pgfsetdash{}{0pt}%
\pgfsys@defobject{currentmarker}{\pgfqpoint{-0.027778in}{0.000000in}}{\pgfqpoint{-0.000000in}{0.000000in}}{%
\pgfpathmoveto{\pgfqpoint{-0.000000in}{0.000000in}}%
\pgfpathlineto{\pgfqpoint{-0.027778in}{0.000000in}}%
\pgfusepath{stroke,fill}%
}%
\begin{pgfscope}%
\pgfsys@transformshift{0.589510in}{1.156531in}%
\pgfsys@useobject{currentmarker}{}%
\end{pgfscope}%
\end{pgfscope}%
\begin{pgfscope}%
\pgfpathrectangle{\pgfqpoint{0.589510in}{0.417642in}}{\pgfqpoint{3.437062in}{2.055000in}}%
\pgfusepath{clip}%
\pgfsetrectcap%
\pgfsetroundjoin%
\pgfsetlinewidth{0.803000pt}%
\definecolor{currentstroke}{rgb}{0.850000,0.850000,0.850000}%
\pgfsetstrokecolor{currentstroke}%
\pgfsetdash{}{0pt}%
\pgfpathmoveto{\pgfqpoint{0.589510in}{1.393146in}}%
\pgfpathlineto{\pgfqpoint{4.026572in}{1.393146in}}%
\pgfusepath{stroke}%
\end{pgfscope}%
\begin{pgfscope}%
\pgfsetbuttcap%
\pgfsetroundjoin%
\definecolor{currentfill}{rgb}{0.000000,0.000000,0.000000}%
\pgfsetfillcolor{currentfill}%
\pgfsetlinewidth{0.602250pt}%
\definecolor{currentstroke}{rgb}{0.000000,0.000000,0.000000}%
\pgfsetstrokecolor{currentstroke}%
\pgfsetdash{}{0pt}%
\pgfsys@defobject{currentmarker}{\pgfqpoint{-0.027778in}{0.000000in}}{\pgfqpoint{-0.000000in}{0.000000in}}{%
\pgfpathmoveto{\pgfqpoint{-0.000000in}{0.000000in}}%
\pgfpathlineto{\pgfqpoint{-0.027778in}{0.000000in}}%
\pgfusepath{stroke,fill}%
}%
\begin{pgfscope}%
\pgfsys@transformshift{0.589510in}{1.393146in}%
\pgfsys@useobject{currentmarker}{}%
\end{pgfscope}%
\end{pgfscope}%
\begin{pgfscope}%
\pgfpathrectangle{\pgfqpoint{0.589510in}{0.417642in}}{\pgfqpoint{3.437062in}{2.055000in}}%
\pgfusepath{clip}%
\pgfsetrectcap%
\pgfsetroundjoin%
\pgfsetlinewidth{0.803000pt}%
\definecolor{currentstroke}{rgb}{0.850000,0.850000,0.850000}%
\pgfsetstrokecolor{currentstroke}%
\pgfsetdash{}{0pt}%
\pgfpathmoveto{\pgfqpoint{0.589510in}{1.513294in}}%
\pgfpathlineto{\pgfqpoint{4.026572in}{1.513294in}}%
\pgfusepath{stroke}%
\end{pgfscope}%
\begin{pgfscope}%
\pgfsetbuttcap%
\pgfsetroundjoin%
\definecolor{currentfill}{rgb}{0.000000,0.000000,0.000000}%
\pgfsetfillcolor{currentfill}%
\pgfsetlinewidth{0.602250pt}%
\definecolor{currentstroke}{rgb}{0.000000,0.000000,0.000000}%
\pgfsetstrokecolor{currentstroke}%
\pgfsetdash{}{0pt}%
\pgfsys@defobject{currentmarker}{\pgfqpoint{-0.027778in}{0.000000in}}{\pgfqpoint{-0.000000in}{0.000000in}}{%
\pgfpathmoveto{\pgfqpoint{-0.000000in}{0.000000in}}%
\pgfpathlineto{\pgfqpoint{-0.027778in}{0.000000in}}%
\pgfusepath{stroke,fill}%
}%
\begin{pgfscope}%
\pgfsys@transformshift{0.589510in}{1.513294in}%
\pgfsys@useobject{currentmarker}{}%
\end{pgfscope}%
\end{pgfscope}%
\begin{pgfscope}%
\pgfpathrectangle{\pgfqpoint{0.589510in}{0.417642in}}{\pgfqpoint{3.437062in}{2.055000in}}%
\pgfusepath{clip}%
\pgfsetrectcap%
\pgfsetroundjoin%
\pgfsetlinewidth{0.803000pt}%
\definecolor{currentstroke}{rgb}{0.850000,0.850000,0.850000}%
\pgfsetstrokecolor{currentstroke}%
\pgfsetdash{}{0pt}%
\pgfpathmoveto{\pgfqpoint{0.589510in}{1.598541in}}%
\pgfpathlineto{\pgfqpoint{4.026572in}{1.598541in}}%
\pgfusepath{stroke}%
\end{pgfscope}%
\begin{pgfscope}%
\pgfsetbuttcap%
\pgfsetroundjoin%
\definecolor{currentfill}{rgb}{0.000000,0.000000,0.000000}%
\pgfsetfillcolor{currentfill}%
\pgfsetlinewidth{0.602250pt}%
\definecolor{currentstroke}{rgb}{0.000000,0.000000,0.000000}%
\pgfsetstrokecolor{currentstroke}%
\pgfsetdash{}{0pt}%
\pgfsys@defobject{currentmarker}{\pgfqpoint{-0.027778in}{0.000000in}}{\pgfqpoint{-0.000000in}{0.000000in}}{%
\pgfpathmoveto{\pgfqpoint{-0.000000in}{0.000000in}}%
\pgfpathlineto{\pgfqpoint{-0.027778in}{0.000000in}}%
\pgfusepath{stroke,fill}%
}%
\begin{pgfscope}%
\pgfsys@transformshift{0.589510in}{1.598541in}%
\pgfsys@useobject{currentmarker}{}%
\end{pgfscope}%
\end{pgfscope}%
\begin{pgfscope}%
\pgfpathrectangle{\pgfqpoint{0.589510in}{0.417642in}}{\pgfqpoint{3.437062in}{2.055000in}}%
\pgfusepath{clip}%
\pgfsetrectcap%
\pgfsetroundjoin%
\pgfsetlinewidth{0.803000pt}%
\definecolor{currentstroke}{rgb}{0.850000,0.850000,0.850000}%
\pgfsetstrokecolor{currentstroke}%
\pgfsetdash{}{0pt}%
\pgfpathmoveto{\pgfqpoint{0.589510in}{1.664663in}}%
\pgfpathlineto{\pgfqpoint{4.026572in}{1.664663in}}%
\pgfusepath{stroke}%
\end{pgfscope}%
\begin{pgfscope}%
\pgfsetbuttcap%
\pgfsetroundjoin%
\definecolor{currentfill}{rgb}{0.000000,0.000000,0.000000}%
\pgfsetfillcolor{currentfill}%
\pgfsetlinewidth{0.602250pt}%
\definecolor{currentstroke}{rgb}{0.000000,0.000000,0.000000}%
\pgfsetstrokecolor{currentstroke}%
\pgfsetdash{}{0pt}%
\pgfsys@defobject{currentmarker}{\pgfqpoint{-0.027778in}{0.000000in}}{\pgfqpoint{-0.000000in}{0.000000in}}{%
\pgfpathmoveto{\pgfqpoint{-0.000000in}{0.000000in}}%
\pgfpathlineto{\pgfqpoint{-0.027778in}{0.000000in}}%
\pgfusepath{stroke,fill}%
}%
\begin{pgfscope}%
\pgfsys@transformshift{0.589510in}{1.664663in}%
\pgfsys@useobject{currentmarker}{}%
\end{pgfscope}%
\end{pgfscope}%
\begin{pgfscope}%
\pgfpathrectangle{\pgfqpoint{0.589510in}{0.417642in}}{\pgfqpoint{3.437062in}{2.055000in}}%
\pgfusepath{clip}%
\pgfsetrectcap%
\pgfsetroundjoin%
\pgfsetlinewidth{0.803000pt}%
\definecolor{currentstroke}{rgb}{0.850000,0.850000,0.850000}%
\pgfsetstrokecolor{currentstroke}%
\pgfsetdash{}{0pt}%
\pgfpathmoveto{\pgfqpoint{0.589510in}{1.718689in}}%
\pgfpathlineto{\pgfqpoint{4.026572in}{1.718689in}}%
\pgfusepath{stroke}%
\end{pgfscope}%
\begin{pgfscope}%
\pgfsetbuttcap%
\pgfsetroundjoin%
\definecolor{currentfill}{rgb}{0.000000,0.000000,0.000000}%
\pgfsetfillcolor{currentfill}%
\pgfsetlinewidth{0.602250pt}%
\definecolor{currentstroke}{rgb}{0.000000,0.000000,0.000000}%
\pgfsetstrokecolor{currentstroke}%
\pgfsetdash{}{0pt}%
\pgfsys@defobject{currentmarker}{\pgfqpoint{-0.027778in}{0.000000in}}{\pgfqpoint{-0.000000in}{0.000000in}}{%
\pgfpathmoveto{\pgfqpoint{-0.000000in}{0.000000in}}%
\pgfpathlineto{\pgfqpoint{-0.027778in}{0.000000in}}%
\pgfusepath{stroke,fill}%
}%
\begin{pgfscope}%
\pgfsys@transformshift{0.589510in}{1.718689in}%
\pgfsys@useobject{currentmarker}{}%
\end{pgfscope}%
\end{pgfscope}%
\begin{pgfscope}%
\pgfpathrectangle{\pgfqpoint{0.589510in}{0.417642in}}{\pgfqpoint{3.437062in}{2.055000in}}%
\pgfusepath{clip}%
\pgfsetrectcap%
\pgfsetroundjoin%
\pgfsetlinewidth{0.803000pt}%
\definecolor{currentstroke}{rgb}{0.850000,0.850000,0.850000}%
\pgfsetstrokecolor{currentstroke}%
\pgfsetdash{}{0pt}%
\pgfpathmoveto{\pgfqpoint{0.589510in}{1.764367in}}%
\pgfpathlineto{\pgfqpoint{4.026572in}{1.764367in}}%
\pgfusepath{stroke}%
\end{pgfscope}%
\begin{pgfscope}%
\pgfsetbuttcap%
\pgfsetroundjoin%
\definecolor{currentfill}{rgb}{0.000000,0.000000,0.000000}%
\pgfsetfillcolor{currentfill}%
\pgfsetlinewidth{0.602250pt}%
\definecolor{currentstroke}{rgb}{0.000000,0.000000,0.000000}%
\pgfsetstrokecolor{currentstroke}%
\pgfsetdash{}{0pt}%
\pgfsys@defobject{currentmarker}{\pgfqpoint{-0.027778in}{0.000000in}}{\pgfqpoint{-0.000000in}{0.000000in}}{%
\pgfpathmoveto{\pgfqpoint{-0.000000in}{0.000000in}}%
\pgfpathlineto{\pgfqpoint{-0.027778in}{0.000000in}}%
\pgfusepath{stroke,fill}%
}%
\begin{pgfscope}%
\pgfsys@transformshift{0.589510in}{1.764367in}%
\pgfsys@useobject{currentmarker}{}%
\end{pgfscope}%
\end{pgfscope}%
\begin{pgfscope}%
\pgfpathrectangle{\pgfqpoint{0.589510in}{0.417642in}}{\pgfqpoint{3.437062in}{2.055000in}}%
\pgfusepath{clip}%
\pgfsetrectcap%
\pgfsetroundjoin%
\pgfsetlinewidth{0.803000pt}%
\definecolor{currentstroke}{rgb}{0.850000,0.850000,0.850000}%
\pgfsetstrokecolor{currentstroke}%
\pgfsetdash{}{0pt}%
\pgfpathmoveto{\pgfqpoint{0.589510in}{1.803935in}}%
\pgfpathlineto{\pgfqpoint{4.026572in}{1.803935in}}%
\pgfusepath{stroke}%
\end{pgfscope}%
\begin{pgfscope}%
\pgfsetbuttcap%
\pgfsetroundjoin%
\definecolor{currentfill}{rgb}{0.000000,0.000000,0.000000}%
\pgfsetfillcolor{currentfill}%
\pgfsetlinewidth{0.602250pt}%
\definecolor{currentstroke}{rgb}{0.000000,0.000000,0.000000}%
\pgfsetstrokecolor{currentstroke}%
\pgfsetdash{}{0pt}%
\pgfsys@defobject{currentmarker}{\pgfqpoint{-0.027778in}{0.000000in}}{\pgfqpoint{-0.000000in}{0.000000in}}{%
\pgfpathmoveto{\pgfqpoint{-0.000000in}{0.000000in}}%
\pgfpathlineto{\pgfqpoint{-0.027778in}{0.000000in}}%
\pgfusepath{stroke,fill}%
}%
\begin{pgfscope}%
\pgfsys@transformshift{0.589510in}{1.803935in}%
\pgfsys@useobject{currentmarker}{}%
\end{pgfscope}%
\end{pgfscope}%
\begin{pgfscope}%
\pgfpathrectangle{\pgfqpoint{0.589510in}{0.417642in}}{\pgfqpoint{3.437062in}{2.055000in}}%
\pgfusepath{clip}%
\pgfsetrectcap%
\pgfsetroundjoin%
\pgfsetlinewidth{0.803000pt}%
\definecolor{currentstroke}{rgb}{0.850000,0.850000,0.850000}%
\pgfsetstrokecolor{currentstroke}%
\pgfsetdash{}{0pt}%
\pgfpathmoveto{\pgfqpoint{0.589510in}{1.838837in}}%
\pgfpathlineto{\pgfqpoint{4.026572in}{1.838837in}}%
\pgfusepath{stroke}%
\end{pgfscope}%
\begin{pgfscope}%
\pgfsetbuttcap%
\pgfsetroundjoin%
\definecolor{currentfill}{rgb}{0.000000,0.000000,0.000000}%
\pgfsetfillcolor{currentfill}%
\pgfsetlinewidth{0.602250pt}%
\definecolor{currentstroke}{rgb}{0.000000,0.000000,0.000000}%
\pgfsetstrokecolor{currentstroke}%
\pgfsetdash{}{0pt}%
\pgfsys@defobject{currentmarker}{\pgfqpoint{-0.027778in}{0.000000in}}{\pgfqpoint{-0.000000in}{0.000000in}}{%
\pgfpathmoveto{\pgfqpoint{-0.000000in}{0.000000in}}%
\pgfpathlineto{\pgfqpoint{-0.027778in}{0.000000in}}%
\pgfusepath{stroke,fill}%
}%
\begin{pgfscope}%
\pgfsys@transformshift{0.589510in}{1.838837in}%
\pgfsys@useobject{currentmarker}{}%
\end{pgfscope}%
\end{pgfscope}%
\begin{pgfscope}%
\pgfpathrectangle{\pgfqpoint{0.589510in}{0.417642in}}{\pgfqpoint{3.437062in}{2.055000in}}%
\pgfusepath{clip}%
\pgfsetrectcap%
\pgfsetroundjoin%
\pgfsetlinewidth{0.803000pt}%
\definecolor{currentstroke}{rgb}{0.850000,0.850000,0.850000}%
\pgfsetstrokecolor{currentstroke}%
\pgfsetdash{}{0pt}%
\pgfpathmoveto{\pgfqpoint{0.589510in}{2.075451in}}%
\pgfpathlineto{\pgfqpoint{4.026572in}{2.075451in}}%
\pgfusepath{stroke}%
\end{pgfscope}%
\begin{pgfscope}%
\pgfsetbuttcap%
\pgfsetroundjoin%
\definecolor{currentfill}{rgb}{0.000000,0.000000,0.000000}%
\pgfsetfillcolor{currentfill}%
\pgfsetlinewidth{0.602250pt}%
\definecolor{currentstroke}{rgb}{0.000000,0.000000,0.000000}%
\pgfsetstrokecolor{currentstroke}%
\pgfsetdash{}{0pt}%
\pgfsys@defobject{currentmarker}{\pgfqpoint{-0.027778in}{0.000000in}}{\pgfqpoint{-0.000000in}{0.000000in}}{%
\pgfpathmoveto{\pgfqpoint{-0.000000in}{0.000000in}}%
\pgfpathlineto{\pgfqpoint{-0.027778in}{0.000000in}}%
\pgfusepath{stroke,fill}%
}%
\begin{pgfscope}%
\pgfsys@transformshift{0.589510in}{2.075451in}%
\pgfsys@useobject{currentmarker}{}%
\end{pgfscope}%
\end{pgfscope}%
\begin{pgfscope}%
\pgfpathrectangle{\pgfqpoint{0.589510in}{0.417642in}}{\pgfqpoint{3.437062in}{2.055000in}}%
\pgfusepath{clip}%
\pgfsetrectcap%
\pgfsetroundjoin%
\pgfsetlinewidth{0.803000pt}%
\definecolor{currentstroke}{rgb}{0.850000,0.850000,0.850000}%
\pgfsetstrokecolor{currentstroke}%
\pgfsetdash{}{0pt}%
\pgfpathmoveto{\pgfqpoint{0.589510in}{2.195599in}}%
\pgfpathlineto{\pgfqpoint{4.026572in}{2.195599in}}%
\pgfusepath{stroke}%
\end{pgfscope}%
\begin{pgfscope}%
\pgfsetbuttcap%
\pgfsetroundjoin%
\definecolor{currentfill}{rgb}{0.000000,0.000000,0.000000}%
\pgfsetfillcolor{currentfill}%
\pgfsetlinewidth{0.602250pt}%
\definecolor{currentstroke}{rgb}{0.000000,0.000000,0.000000}%
\pgfsetstrokecolor{currentstroke}%
\pgfsetdash{}{0pt}%
\pgfsys@defobject{currentmarker}{\pgfqpoint{-0.027778in}{0.000000in}}{\pgfqpoint{-0.000000in}{0.000000in}}{%
\pgfpathmoveto{\pgfqpoint{-0.000000in}{0.000000in}}%
\pgfpathlineto{\pgfqpoint{-0.027778in}{0.000000in}}%
\pgfusepath{stroke,fill}%
}%
\begin{pgfscope}%
\pgfsys@transformshift{0.589510in}{2.195599in}%
\pgfsys@useobject{currentmarker}{}%
\end{pgfscope}%
\end{pgfscope}%
\begin{pgfscope}%
\pgfpathrectangle{\pgfqpoint{0.589510in}{0.417642in}}{\pgfqpoint{3.437062in}{2.055000in}}%
\pgfusepath{clip}%
\pgfsetrectcap%
\pgfsetroundjoin%
\pgfsetlinewidth{0.803000pt}%
\definecolor{currentstroke}{rgb}{0.850000,0.850000,0.850000}%
\pgfsetstrokecolor{currentstroke}%
\pgfsetdash{}{0pt}%
\pgfpathmoveto{\pgfqpoint{0.589510in}{2.280846in}}%
\pgfpathlineto{\pgfqpoint{4.026572in}{2.280846in}}%
\pgfusepath{stroke}%
\end{pgfscope}%
\begin{pgfscope}%
\pgfsetbuttcap%
\pgfsetroundjoin%
\definecolor{currentfill}{rgb}{0.000000,0.000000,0.000000}%
\pgfsetfillcolor{currentfill}%
\pgfsetlinewidth{0.602250pt}%
\definecolor{currentstroke}{rgb}{0.000000,0.000000,0.000000}%
\pgfsetstrokecolor{currentstroke}%
\pgfsetdash{}{0pt}%
\pgfsys@defobject{currentmarker}{\pgfqpoint{-0.027778in}{0.000000in}}{\pgfqpoint{-0.000000in}{0.000000in}}{%
\pgfpathmoveto{\pgfqpoint{-0.000000in}{0.000000in}}%
\pgfpathlineto{\pgfqpoint{-0.027778in}{0.000000in}}%
\pgfusepath{stroke,fill}%
}%
\begin{pgfscope}%
\pgfsys@transformshift{0.589510in}{2.280846in}%
\pgfsys@useobject{currentmarker}{}%
\end{pgfscope}%
\end{pgfscope}%
\begin{pgfscope}%
\pgfpathrectangle{\pgfqpoint{0.589510in}{0.417642in}}{\pgfqpoint{3.437062in}{2.055000in}}%
\pgfusepath{clip}%
\pgfsetrectcap%
\pgfsetroundjoin%
\pgfsetlinewidth{0.803000pt}%
\definecolor{currentstroke}{rgb}{0.850000,0.850000,0.850000}%
\pgfsetstrokecolor{currentstroke}%
\pgfsetdash{}{0pt}%
\pgfpathmoveto{\pgfqpoint{0.589510in}{2.346968in}}%
\pgfpathlineto{\pgfqpoint{4.026572in}{2.346968in}}%
\pgfusepath{stroke}%
\end{pgfscope}%
\begin{pgfscope}%
\pgfsetbuttcap%
\pgfsetroundjoin%
\definecolor{currentfill}{rgb}{0.000000,0.000000,0.000000}%
\pgfsetfillcolor{currentfill}%
\pgfsetlinewidth{0.602250pt}%
\definecolor{currentstroke}{rgb}{0.000000,0.000000,0.000000}%
\pgfsetstrokecolor{currentstroke}%
\pgfsetdash{}{0pt}%
\pgfsys@defobject{currentmarker}{\pgfqpoint{-0.027778in}{0.000000in}}{\pgfqpoint{-0.000000in}{0.000000in}}{%
\pgfpathmoveto{\pgfqpoint{-0.000000in}{0.000000in}}%
\pgfpathlineto{\pgfqpoint{-0.027778in}{0.000000in}}%
\pgfusepath{stroke,fill}%
}%
\begin{pgfscope}%
\pgfsys@transformshift{0.589510in}{2.346968in}%
\pgfsys@useobject{currentmarker}{}%
\end{pgfscope}%
\end{pgfscope}%
\begin{pgfscope}%
\pgfpathrectangle{\pgfqpoint{0.589510in}{0.417642in}}{\pgfqpoint{3.437062in}{2.055000in}}%
\pgfusepath{clip}%
\pgfsetrectcap%
\pgfsetroundjoin%
\pgfsetlinewidth{0.803000pt}%
\definecolor{currentstroke}{rgb}{0.850000,0.850000,0.850000}%
\pgfsetstrokecolor{currentstroke}%
\pgfsetdash{}{0pt}%
\pgfpathmoveto{\pgfqpoint{0.589510in}{2.400994in}}%
\pgfpathlineto{\pgfqpoint{4.026572in}{2.400994in}}%
\pgfusepath{stroke}%
\end{pgfscope}%
\begin{pgfscope}%
\pgfsetbuttcap%
\pgfsetroundjoin%
\definecolor{currentfill}{rgb}{0.000000,0.000000,0.000000}%
\pgfsetfillcolor{currentfill}%
\pgfsetlinewidth{0.602250pt}%
\definecolor{currentstroke}{rgb}{0.000000,0.000000,0.000000}%
\pgfsetstrokecolor{currentstroke}%
\pgfsetdash{}{0pt}%
\pgfsys@defobject{currentmarker}{\pgfqpoint{-0.027778in}{0.000000in}}{\pgfqpoint{-0.000000in}{0.000000in}}{%
\pgfpathmoveto{\pgfqpoint{-0.000000in}{0.000000in}}%
\pgfpathlineto{\pgfqpoint{-0.027778in}{0.000000in}}%
\pgfusepath{stroke,fill}%
}%
\begin{pgfscope}%
\pgfsys@transformshift{0.589510in}{2.400994in}%
\pgfsys@useobject{currentmarker}{}%
\end{pgfscope}%
\end{pgfscope}%
\begin{pgfscope}%
\pgfpathrectangle{\pgfqpoint{0.589510in}{0.417642in}}{\pgfqpoint{3.437062in}{2.055000in}}%
\pgfusepath{clip}%
\pgfsetrectcap%
\pgfsetroundjoin%
\pgfsetlinewidth{0.803000pt}%
\definecolor{currentstroke}{rgb}{0.850000,0.850000,0.850000}%
\pgfsetstrokecolor{currentstroke}%
\pgfsetdash{}{0pt}%
\pgfpathmoveto{\pgfqpoint{0.589510in}{2.446672in}}%
\pgfpathlineto{\pgfqpoint{4.026572in}{2.446672in}}%
\pgfusepath{stroke}%
\end{pgfscope}%
\begin{pgfscope}%
\pgfsetbuttcap%
\pgfsetroundjoin%
\definecolor{currentfill}{rgb}{0.000000,0.000000,0.000000}%
\pgfsetfillcolor{currentfill}%
\pgfsetlinewidth{0.602250pt}%
\definecolor{currentstroke}{rgb}{0.000000,0.000000,0.000000}%
\pgfsetstrokecolor{currentstroke}%
\pgfsetdash{}{0pt}%
\pgfsys@defobject{currentmarker}{\pgfqpoint{-0.027778in}{0.000000in}}{\pgfqpoint{-0.000000in}{0.000000in}}{%
\pgfpathmoveto{\pgfqpoint{-0.000000in}{0.000000in}}%
\pgfpathlineto{\pgfqpoint{-0.027778in}{0.000000in}}%
\pgfusepath{stroke,fill}%
}%
\begin{pgfscope}%
\pgfsys@transformshift{0.589510in}{2.446672in}%
\pgfsys@useobject{currentmarker}{}%
\end{pgfscope}%
\end{pgfscope}%
\begin{pgfscope}%
\definecolor{textcolor}{rgb}{0.000000,0.000000,0.000000}%
\pgfsetstrokecolor{textcolor}%
\pgfsetfillcolor{textcolor}%
\pgftext[x=0.180559in,y=1.445142in,,bottom,rotate=90.000000]{\color{textcolor}\rmfamily\fontsize{10.000000}{12.000000}\selectfont ADEV \(\displaystyle \sigma_A(\tau)\) in \unit{\V}}%
\end{pgfscope}%
\begin{pgfscope}%
\pgfpathrectangle{\pgfqpoint{0.589510in}{0.417642in}}{\pgfqpoint{3.437062in}{2.055000in}}%
\pgfusepath{clip}%
\pgfsetbuttcap%
\pgfsetroundjoin%
\definecolor{currentfill}{rgb}{0.925490,0.882353,0.200000}%
\pgfsetfillcolor{currentfill}%
\pgfsetlinewidth{1.003750pt}%
\definecolor{currentstroke}{rgb}{0.925490,0.882353,0.200000}%
\pgfsetstrokecolor{currentstroke}%
\pgfsetdash{}{0pt}%
\pgfsys@defobject{currentmarker}{\pgfqpoint{-0.020833in}{-0.020833in}}{\pgfqpoint{0.020833in}{0.020833in}}{%
\pgfpathmoveto{\pgfqpoint{0.000000in}{-0.020833in}}%
\pgfpathcurveto{\pgfqpoint{0.005525in}{-0.020833in}}{\pgfqpoint{0.010825in}{-0.018638in}}{\pgfqpoint{0.014731in}{-0.014731in}}%
\pgfpathcurveto{\pgfqpoint{0.018638in}{-0.010825in}}{\pgfqpoint{0.020833in}{-0.005525in}}{\pgfqpoint{0.020833in}{0.000000in}}%
\pgfpathcurveto{\pgfqpoint{0.020833in}{0.005525in}}{\pgfqpoint{0.018638in}{0.010825in}}{\pgfqpoint{0.014731in}{0.014731in}}%
\pgfpathcurveto{\pgfqpoint{0.010825in}{0.018638in}}{\pgfqpoint{0.005525in}{0.020833in}}{\pgfqpoint{0.000000in}{0.020833in}}%
\pgfpathcurveto{\pgfqpoint{-0.005525in}{0.020833in}}{\pgfqpoint{-0.010825in}{0.018638in}}{\pgfqpoint{-0.014731in}{0.014731in}}%
\pgfpathcurveto{\pgfqpoint{-0.018638in}{0.010825in}}{\pgfqpoint{-0.020833in}{0.005525in}}{\pgfqpoint{-0.020833in}{0.000000in}}%
\pgfpathcurveto{\pgfqpoint{-0.020833in}{-0.005525in}}{\pgfqpoint{-0.018638in}{-0.010825in}}{\pgfqpoint{-0.014731in}{-0.014731in}}%
\pgfpathcurveto{\pgfqpoint{-0.010825in}{-0.018638in}}{\pgfqpoint{-0.005525in}{-0.020833in}}{\pgfqpoint{0.000000in}{-0.020833in}}%
\pgfpathlineto{\pgfqpoint{0.000000in}{-0.020833in}}%
\pgfpathclose%
\pgfusepath{stroke,fill}%
}%
\begin{pgfscope}%
\pgfsys@transformshift{0.745740in}{2.379233in}%
\pgfsys@useobject{currentmarker}{}%
\end{pgfscope}%
\begin{pgfscope}%
\pgfsys@transformshift{0.929540in}{2.269764in}%
\pgfsys@useobject{currentmarker}{}%
\end{pgfscope}%
\begin{pgfscope}%
\pgfsys@transformshift{1.113340in}{2.160671in}%
\pgfsys@useobject{currentmarker}{}%
\end{pgfscope}%
\begin{pgfscope}%
\pgfsys@transformshift{1.297140in}{2.054742in}%
\pgfsys@useobject{currentmarker}{}%
\end{pgfscope}%
\begin{pgfscope}%
\pgfsys@transformshift{1.480940in}{1.949170in}%
\pgfsys@useobject{currentmarker}{}%
\end{pgfscope}%
\begin{pgfscope}%
\pgfsys@transformshift{1.664740in}{1.844707in}%
\pgfsys@useobject{currentmarker}{}%
\end{pgfscope}%
\begin{pgfscope}%
\pgfsys@transformshift{1.848540in}{1.739518in}%
\pgfsys@useobject{currentmarker}{}%
\end{pgfscope}%
\begin{pgfscope}%
\pgfsys@transformshift{2.032341in}{1.640852in}%
\pgfsys@useobject{currentmarker}{}%
\end{pgfscope}%
\begin{pgfscope}%
\pgfsys@transformshift{2.216141in}{1.539100in}%
\pgfsys@useobject{currentmarker}{}%
\end{pgfscope}%
\begin{pgfscope}%
\pgfsys@transformshift{2.399941in}{1.440819in}%
\pgfsys@useobject{currentmarker}{}%
\end{pgfscope}%
\begin{pgfscope}%
\pgfsys@transformshift{2.583741in}{1.331430in}%
\pgfsys@useobject{currentmarker}{}%
\end{pgfscope}%
\begin{pgfscope}%
\pgfsys@transformshift{2.767541in}{1.237471in}%
\pgfsys@useobject{currentmarker}{}%
\end{pgfscope}%
\begin{pgfscope}%
\pgfsys@transformshift{2.951341in}{1.138768in}%
\pgfsys@useobject{currentmarker}{}%
\end{pgfscope}%
\begin{pgfscope}%
\pgfsys@transformshift{3.135141in}{1.015049in}%
\pgfsys@useobject{currentmarker}{}%
\end{pgfscope}%
\begin{pgfscope}%
\pgfsys@transformshift{3.318941in}{0.900614in}%
\pgfsys@useobject{currentmarker}{}%
\end{pgfscope}%
\begin{pgfscope}%
\pgfsys@transformshift{3.502741in}{0.763555in}%
\pgfsys@useobject{currentmarker}{}%
\end{pgfscope}%
\begin{pgfscope}%
\pgfsys@transformshift{3.686541in}{0.686051in}%
\pgfsys@useobject{currentmarker}{}%
\end{pgfscope}%
\begin{pgfscope}%
\pgfsys@transformshift{3.870342in}{0.571408in}%
\pgfsys@useobject{currentmarker}{}%
\end{pgfscope}%
\end{pgfscope}%
\begin{pgfscope}%
\pgfpathrectangle{\pgfqpoint{0.589510in}{0.417642in}}{\pgfqpoint{3.437062in}{2.055000in}}%
\pgfusepath{clip}%
\pgfsetbuttcap%
\pgfsetroundjoin%
\pgfsetlinewidth{1.505625pt}%
\definecolor{currentstroke}{rgb}{0.003922,0.450980,0.698039}%
\pgfsetstrokecolor{currentstroke}%
\pgfsetdash{{5.550000pt}{2.400000pt}}{0.000000pt}%
\pgfpathmoveto{\pgfqpoint{0.745740in}{2.051509in}}%
\pgfpathlineto{\pgfqpoint{0.929540in}{1.948811in}}%
\pgfpathlineto{\pgfqpoint{1.113340in}{1.846114in}}%
\pgfpathlineto{\pgfqpoint{1.297140in}{1.743417in}}%
\pgfpathlineto{\pgfqpoint{1.480940in}{1.640720in}}%
\pgfpathlineto{\pgfqpoint{1.664740in}{1.538023in}}%
\pgfpathlineto{\pgfqpoint{1.848540in}{1.435326in}}%
\pgfpathlineto{\pgfqpoint{2.032341in}{1.332628in}}%
\pgfpathlineto{\pgfqpoint{2.216141in}{1.229931in}}%
\pgfpathlineto{\pgfqpoint{2.399941in}{1.127234in}}%
\pgfpathlineto{\pgfqpoint{2.583741in}{1.024537in}}%
\pgfpathlineto{\pgfqpoint{2.767541in}{0.921840in}}%
\pgfpathlineto{\pgfqpoint{2.951341in}{0.819143in}}%
\pgfpathlineto{\pgfqpoint{3.135141in}{0.716445in}}%
\pgfpathlineto{\pgfqpoint{3.318941in}{0.613748in}}%
\pgfpathlineto{\pgfqpoint{3.502741in}{0.511051in}}%
\pgfusepath{stroke}%
\end{pgfscope}%
\begin{pgfscope}%
\pgfpathrectangle{\pgfqpoint{0.589510in}{0.417642in}}{\pgfqpoint{3.437062in}{2.055000in}}%
\pgfusepath{clip}%
\pgfsetbuttcap%
\pgfsetroundjoin%
\pgfsetlinewidth{1.505625pt}%
\definecolor{currentstroke}{rgb}{0.007843,0.619608,0.450980}%
\pgfsetstrokecolor{currentstroke}%
\pgfsetdash{{5.550000pt}{2.400000pt}}{0.000000pt}%
\pgfpathmoveto{\pgfqpoint{0.745740in}{2.126916in}}%
\pgfpathlineto{\pgfqpoint{0.929540in}{2.126916in}}%
\pgfpathlineto{\pgfqpoint{1.113340in}{2.126916in}}%
\pgfpathlineto{\pgfqpoint{1.297140in}{2.126916in}}%
\pgfpathlineto{\pgfqpoint{1.480940in}{2.126916in}}%
\pgfpathlineto{\pgfqpoint{1.664740in}{2.126916in}}%
\pgfpathlineto{\pgfqpoint{1.848540in}{2.126916in}}%
\pgfpathlineto{\pgfqpoint{2.032341in}{2.126916in}}%
\pgfpathlineto{\pgfqpoint{2.216141in}{2.126916in}}%
\pgfpathlineto{\pgfqpoint{2.399941in}{2.126916in}}%
\pgfpathlineto{\pgfqpoint{2.583741in}{2.126916in}}%
\pgfpathlineto{\pgfqpoint{2.767541in}{2.126916in}}%
\pgfpathlineto{\pgfqpoint{2.951341in}{2.126916in}}%
\pgfpathlineto{\pgfqpoint{3.135141in}{2.126916in}}%
\pgfpathlineto{\pgfqpoint{3.318941in}{2.126916in}}%
\pgfpathlineto{\pgfqpoint{3.502741in}{2.126916in}}%
\pgfpathlineto{\pgfqpoint{3.686541in}{2.126916in}}%
\pgfpathlineto{\pgfqpoint{3.870342in}{2.126916in}}%
\pgfusepath{stroke}%
\end{pgfscope}%
\begin{pgfscope}%
\pgfsetrectcap%
\pgfsetmiterjoin%
\pgfsetlinewidth{0.803000pt}%
\definecolor{currentstroke}{rgb}{0.000000,0.000000,0.000000}%
\pgfsetstrokecolor{currentstroke}%
\pgfsetdash{}{0pt}%
\pgfpathmoveto{\pgfqpoint{0.589510in}{0.417642in}}%
\pgfpathlineto{\pgfqpoint{0.589510in}{2.472642in}}%
\pgfusepath{stroke}%
\end{pgfscope}%
\begin{pgfscope}%
\pgfsetrectcap%
\pgfsetmiterjoin%
\pgfsetlinewidth{0.803000pt}%
\definecolor{currentstroke}{rgb}{0.000000,0.000000,0.000000}%
\pgfsetstrokecolor{currentstroke}%
\pgfsetdash{}{0pt}%
\pgfpathmoveto{\pgfqpoint{4.026572in}{0.417642in}}%
\pgfpathlineto{\pgfqpoint{4.026572in}{2.472642in}}%
\pgfusepath{stroke}%
\end{pgfscope}%
\begin{pgfscope}%
\pgfsetrectcap%
\pgfsetmiterjoin%
\pgfsetlinewidth{0.803000pt}%
\definecolor{currentstroke}{rgb}{0.000000,0.000000,0.000000}%
\pgfsetstrokecolor{currentstroke}%
\pgfsetdash{}{0pt}%
\pgfpathmoveto{\pgfqpoint{0.589510in}{0.417642in}}%
\pgfpathlineto{\pgfqpoint{4.026572in}{0.417642in}}%
\pgfusepath{stroke}%
\end{pgfscope}%
\begin{pgfscope}%
\pgfsetrectcap%
\pgfsetmiterjoin%
\pgfsetlinewidth{0.803000pt}%
\definecolor{currentstroke}{rgb}{0.000000,0.000000,0.000000}%
\pgfsetstrokecolor{currentstroke}%
\pgfsetdash{}{0pt}%
\pgfpathmoveto{\pgfqpoint{0.589510in}{2.472642in}}%
\pgfpathlineto{\pgfqpoint{4.026572in}{2.472642in}}%
\pgfusepath{stroke}%
\end{pgfscope}%
\begin{pgfscope}%
\pgfsetbuttcap%
\pgfsetmiterjoin%
\definecolor{currentfill}{rgb}{1.000000,1.000000,1.000000}%
\pgfsetfillcolor{currentfill}%
\pgfsetfillopacity{0.800000}%
\pgfsetlinewidth{1.003750pt}%
\definecolor{currentstroke}{rgb}{0.800000,0.800000,0.800000}%
\pgfsetstrokecolor{currentstroke}%
\pgfsetstrokeopacity{0.800000}%
\pgfsetdash{}{0pt}%
\pgfpathmoveto{\pgfqpoint{2.948460in}{2.073975in}}%
\pgfpathlineto{\pgfqpoint{3.948794in}{2.073975in}}%
\pgfpathquadraticcurveto{\pgfqpoint{3.971016in}{2.073975in}}{\pgfqpoint{3.971016in}{2.096197in}}%
\pgfpathlineto{\pgfqpoint{3.971016in}{2.394864in}}%
\pgfpathquadraticcurveto{\pgfqpoint{3.971016in}{2.417086in}}{\pgfqpoint{3.948794in}{2.417086in}}%
\pgfpathlineto{\pgfqpoint{2.948460in}{2.417086in}}%
\pgfpathquadraticcurveto{\pgfqpoint{2.926238in}{2.417086in}}{\pgfqpoint{2.926238in}{2.394864in}}%
\pgfpathlineto{\pgfqpoint{2.926238in}{2.096197in}}%
\pgfpathquadraticcurveto{\pgfqpoint{2.926238in}{2.073975in}}{\pgfqpoint{2.948460in}{2.073975in}}%
\pgfpathlineto{\pgfqpoint{2.948460in}{2.073975in}}%
\pgfpathclose%
\pgfusepath{stroke,fill}%
\end{pgfscope}%
\begin{pgfscope}%
\pgfsetbuttcap%
\pgfsetroundjoin%
\pgfsetlinewidth{1.505625pt}%
\definecolor{currentstroke}{rgb}{0.003922,0.450980,0.698039}%
\pgfsetstrokecolor{currentstroke}%
\pgfsetdash{{5.550000pt}{2.400000pt}}{0.000000pt}%
\pgfpathmoveto{\pgfqpoint{2.970683in}{2.333753in}}%
\pgfpathlineto{\pgfqpoint{3.081794in}{2.333753in}}%
\pgfpathlineto{\pgfqpoint{3.192905in}{2.333753in}}%
\pgfusepath{stroke}%
\end{pgfscope}%
\begin{pgfscope}%
\definecolor{textcolor}{rgb}{0.000000,0.000000,0.000000}%
\pgfsetstrokecolor{textcolor}%
\pgfsetfillcolor{textcolor}%
\pgftext[x=3.281794in,y=2.294864in,left,base]{\color{textcolor}\rmfamily\fontsize{8.000000}{9.600000}\selectfont White noise}%
\end{pgfscope}%
\begin{pgfscope}%
\pgfsetbuttcap%
\pgfsetroundjoin%
\pgfsetlinewidth{1.505625pt}%
\definecolor{currentstroke}{rgb}{0.007843,0.619608,0.450980}%
\pgfsetstrokecolor{currentstroke}%
\pgfsetdash{{5.550000pt}{2.400000pt}}{0.000000pt}%
\pgfpathmoveto{\pgfqpoint{2.970683in}{2.178864in}}%
\pgfpathlineto{\pgfqpoint{3.081794in}{2.178864in}}%
\pgfpathlineto{\pgfqpoint{3.192905in}{2.178864in}}%
\pgfusepath{stroke}%
\end{pgfscope}%
\begin{pgfscope}%
\definecolor{textcolor}{rgb}{0.000000,0.000000,0.000000}%
\pgfsetstrokecolor{textcolor}%
\pgfsetfillcolor{textcolor}%
\pgftext[x=3.281794in,y=2.139975in,left,base]{\color{textcolor}\rmfamily\fontsize{8.000000}{9.600000}\selectfont Flicker noise}%
\end{pgfscope}%
\end{pgfpicture}%
\makeatother%
\endgroup%
% data/simulations/sim_autozero.py
    \caption{Simulated Allan deviation of a Keysight \device{3458A} with autozeroing applied. The dashed lines denote the deviation prior to applying the autozero algorithm. The line frequency is \qty{50}{\Hz}.}
    \label{fig:autozero_adev}
\end{figure}

The Allan deviation plot in figure \ref{fig:autozero_adev} also confirms that white noise is the only component and shows a $\tau^{-\frac 1 2}$ dependence for the full range of integration times.

From this plot it can be seen that for measurement times longer than about \qty{2}{\s} or \qty{100}{\plc}, autozeroing has a clear benefit over a measurement without autozeroing. It must be noted though that, judging from this simulation, the device would reach a noise floor of \qty[per-mode = symbol]{0.01}{\uV \per \V} only at integration times of slightly more than \qty{10}{\s}, while the datasheet claims \qty{2}{\s}. Do note, that this simulation is for the \qty{10}{\V} range of the DMM and therefore \qty[per-mode = symbol]{0.01}{\uV \per \V} is \qty{0.1}{\uV_{rms}}. It is therefore likely that the noise parameters of a real device are better than the numbers used in the simulation. Additionally, the datasheet likely refers to an instrument that is synced to a \qty{60}{\Hz} power line frequency which shifts the sampling frequency up by \qty{20}{\percent} and, as discussed, reduces the noise floor because more noise content is white noise at the autozero interval. In this simulation the \qty{0.01}{\uV \per \V} (\qty{0.1}{\uV_{rms}}) noise level would be reached at exactly \qty{10}{\s} when using a line frequency of \qty{60}{\Hz}. For the purpose of demonstrating the autozeroing algorithms these subtleties are irrelevant.

For the comparison of different ADC integration intervals before applying autozeroing figure \ref{fig:autozero_nplcs_adev} can be consulted. Using the Allan deviation makes it very simple to compare noise figures for identical measurement times $\tau$, yet different integration times, before autozeroing is applied. The simulation source code can be found in \external{data/simulations/sim\_optimal\_autozero.py} as part of the online supplemental material \cite{supplemental_material}.
\begin{figure}[ht]
    \centering
    %% Creator: Matplotlib, PGF backend
%%
%% To include the figure in your LaTeX document, write
%%   \input{<filename>.pgf}
%%
%% Make sure the required packages are loaded in your preamble
%%   \usepackage{pgf}
%%
%% Also ensure that all the required font packages are loaded; for instance,
%% the lmodern package is sometimes necessary when using math font.
%%   \usepackage{lmodern}
%%
%% Figures using additional raster images can only be included by \input if
%% they are in the same directory as the main LaTeX file. For loading figures
%% from other directories you can use the `import` package
%%   \usepackage{import}
%%
%% and then include the figures with
%%   \import{<path to file>}{<filename>.pgf}
%%
%% Matplotlib used the following preamble
%%   \def\mathdefault#1{#1}
%%   \everymath=\expandafter{\the\everymath\displaystyle}
%%   \usepackage{siunitx}
%%   \sisetup{per-mode = symbol}%
%%   \ifdefined\pdftexversion\else  % non-pdftex case.
%%     \usepackage{fontspec}
%%   \fi
%%   \makeatletter\@ifpackageloaded{underscore}{}{\usepackage[strings]{underscore}}\makeatother
%%
\begingroup%
\makeatletter%
\begin{pgfpicture}%
\pgfpathrectangle{\pgfpointorigin}{\pgfqpoint{4.068242in}{2.514312in}}%
\pgfusepath{use as bounding box, clip}%
\begin{pgfscope}%
\pgfsetbuttcap%
\pgfsetmiterjoin%
\definecolor{currentfill}{rgb}{1.000000,1.000000,1.000000}%
\pgfsetfillcolor{currentfill}%
\pgfsetlinewidth{0.000000pt}%
\definecolor{currentstroke}{rgb}{1.000000,1.000000,1.000000}%
\pgfsetstrokecolor{currentstroke}%
\pgfsetdash{}{0pt}%
\pgfpathmoveto{\pgfqpoint{0.000000in}{0.000000in}}%
\pgfpathlineto{\pgfqpoint{4.068242in}{0.000000in}}%
\pgfpathlineto{\pgfqpoint{4.068242in}{2.514312in}}%
\pgfpathlineto{\pgfqpoint{0.000000in}{2.514312in}}%
\pgfpathlineto{\pgfqpoint{0.000000in}{0.000000in}}%
\pgfpathclose%
\pgfusepath{fill}%
\end{pgfscope}%
\begin{pgfscope}%
\pgfsetbuttcap%
\pgfsetmiterjoin%
\definecolor{currentfill}{rgb}{1.000000,1.000000,1.000000}%
\pgfsetfillcolor{currentfill}%
\pgfsetlinewidth{0.000000pt}%
\definecolor{currentstroke}{rgb}{0.000000,0.000000,0.000000}%
\pgfsetstrokecolor{currentstroke}%
\pgfsetstrokeopacity{0.000000}%
\pgfsetdash{}{0pt}%
\pgfpathmoveto{\pgfqpoint{0.589510in}{0.417642in}}%
\pgfpathlineto{\pgfqpoint{4.026572in}{0.417642in}}%
\pgfpathlineto{\pgfqpoint{4.026572in}{2.472642in}}%
\pgfpathlineto{\pgfqpoint{0.589510in}{2.472642in}}%
\pgfpathlineto{\pgfqpoint{0.589510in}{0.417642in}}%
\pgfpathclose%
\pgfusepath{fill}%
\end{pgfscope}%
\begin{pgfscope}%
\pgfpathrectangle{\pgfqpoint{0.589510in}{0.417642in}}{\pgfqpoint{3.437062in}{2.055000in}}%
\pgfusepath{clip}%
\pgfsetrectcap%
\pgfsetroundjoin%
\pgfsetlinewidth{0.803000pt}%
\definecolor{currentstroke}{rgb}{0.450000,0.450000,0.450000}%
\pgfsetstrokecolor{currentstroke}%
\pgfsetdash{}{0pt}%
\pgfpathmoveto{\pgfqpoint{1.336175in}{0.417642in}}%
\pgfpathlineto{\pgfqpoint{1.336175in}{2.472642in}}%
\pgfusepath{stroke}%
\end{pgfscope}%
\begin{pgfscope}%
\pgfsetbuttcap%
\pgfsetroundjoin%
\definecolor{currentfill}{rgb}{0.000000,0.000000,0.000000}%
\pgfsetfillcolor{currentfill}%
\pgfsetlinewidth{0.803000pt}%
\definecolor{currentstroke}{rgb}{0.000000,0.000000,0.000000}%
\pgfsetstrokecolor{currentstroke}%
\pgfsetdash{}{0pt}%
\pgfsys@defobject{currentmarker}{\pgfqpoint{0.000000in}{-0.048611in}}{\pgfqpoint{0.000000in}{0.000000in}}{%
\pgfpathmoveto{\pgfqpoint{0.000000in}{0.000000in}}%
\pgfpathlineto{\pgfqpoint{0.000000in}{-0.048611in}}%
\pgfusepath{stroke,fill}%
}%
\begin{pgfscope}%
\pgfsys@transformshift{1.336175in}{0.417642in}%
\pgfsys@useobject{currentmarker}{}%
\end{pgfscope}%
\end{pgfscope}%
\begin{pgfscope}%
\definecolor{textcolor}{rgb}{0.000000,0.000000,0.000000}%
\pgfsetstrokecolor{textcolor}%
\pgfsetfillcolor{textcolor}%
\pgftext[x=1.336175in,y=0.320420in,,top]{\color{textcolor}{\rmfamily\fontsize{8.000000}{9.600000}\selectfont\catcode`\^=\active\def^{\ifmmode\sp\else\^{}\fi}\catcode`\%=\active\def%{\%}$\mathdefault{10^{-1}}$}}%
\end{pgfscope}%
\begin{pgfscope}%
\pgfpathrectangle{\pgfqpoint{0.589510in}{0.417642in}}{\pgfqpoint{3.437062in}{2.055000in}}%
\pgfusepath{clip}%
\pgfsetrectcap%
\pgfsetroundjoin%
\pgfsetlinewidth{0.803000pt}%
\definecolor{currentstroke}{rgb}{0.450000,0.450000,0.450000}%
\pgfsetstrokecolor{currentstroke}%
\pgfsetdash{}{0pt}%
\pgfpathmoveto{\pgfqpoint{2.180897in}{0.417642in}}%
\pgfpathlineto{\pgfqpoint{2.180897in}{2.472642in}}%
\pgfusepath{stroke}%
\end{pgfscope}%
\begin{pgfscope}%
\pgfsetbuttcap%
\pgfsetroundjoin%
\definecolor{currentfill}{rgb}{0.000000,0.000000,0.000000}%
\pgfsetfillcolor{currentfill}%
\pgfsetlinewidth{0.803000pt}%
\definecolor{currentstroke}{rgb}{0.000000,0.000000,0.000000}%
\pgfsetstrokecolor{currentstroke}%
\pgfsetdash{}{0pt}%
\pgfsys@defobject{currentmarker}{\pgfqpoint{0.000000in}{-0.048611in}}{\pgfqpoint{0.000000in}{0.000000in}}{%
\pgfpathmoveto{\pgfqpoint{0.000000in}{0.000000in}}%
\pgfpathlineto{\pgfqpoint{0.000000in}{-0.048611in}}%
\pgfusepath{stroke,fill}%
}%
\begin{pgfscope}%
\pgfsys@transformshift{2.180897in}{0.417642in}%
\pgfsys@useobject{currentmarker}{}%
\end{pgfscope}%
\end{pgfscope}%
\begin{pgfscope}%
\definecolor{textcolor}{rgb}{0.000000,0.000000,0.000000}%
\pgfsetstrokecolor{textcolor}%
\pgfsetfillcolor{textcolor}%
\pgftext[x=2.180897in,y=0.320420in,,top]{\color{textcolor}{\rmfamily\fontsize{8.000000}{9.600000}\selectfont\catcode`\^=\active\def^{\ifmmode\sp\else\^{}\fi}\catcode`\%=\active\def%{\%}$\mathdefault{10^{0}}$}}%
\end{pgfscope}%
\begin{pgfscope}%
\pgfpathrectangle{\pgfqpoint{0.589510in}{0.417642in}}{\pgfqpoint{3.437062in}{2.055000in}}%
\pgfusepath{clip}%
\pgfsetrectcap%
\pgfsetroundjoin%
\pgfsetlinewidth{0.803000pt}%
\definecolor{currentstroke}{rgb}{0.450000,0.450000,0.450000}%
\pgfsetstrokecolor{currentstroke}%
\pgfsetdash{}{0pt}%
\pgfpathmoveto{\pgfqpoint{3.025619in}{0.417642in}}%
\pgfpathlineto{\pgfqpoint{3.025619in}{2.472642in}}%
\pgfusepath{stroke}%
\end{pgfscope}%
\begin{pgfscope}%
\pgfsetbuttcap%
\pgfsetroundjoin%
\definecolor{currentfill}{rgb}{0.000000,0.000000,0.000000}%
\pgfsetfillcolor{currentfill}%
\pgfsetlinewidth{0.803000pt}%
\definecolor{currentstroke}{rgb}{0.000000,0.000000,0.000000}%
\pgfsetstrokecolor{currentstroke}%
\pgfsetdash{}{0pt}%
\pgfsys@defobject{currentmarker}{\pgfqpoint{0.000000in}{-0.048611in}}{\pgfqpoint{0.000000in}{0.000000in}}{%
\pgfpathmoveto{\pgfqpoint{0.000000in}{0.000000in}}%
\pgfpathlineto{\pgfqpoint{0.000000in}{-0.048611in}}%
\pgfusepath{stroke,fill}%
}%
\begin{pgfscope}%
\pgfsys@transformshift{3.025619in}{0.417642in}%
\pgfsys@useobject{currentmarker}{}%
\end{pgfscope}%
\end{pgfscope}%
\begin{pgfscope}%
\definecolor{textcolor}{rgb}{0.000000,0.000000,0.000000}%
\pgfsetstrokecolor{textcolor}%
\pgfsetfillcolor{textcolor}%
\pgftext[x=3.025619in,y=0.320420in,,top]{\color{textcolor}{\rmfamily\fontsize{8.000000}{9.600000}\selectfont\catcode`\^=\active\def^{\ifmmode\sp\else\^{}\fi}\catcode`\%=\active\def%{\%}$\mathdefault{10^{1}}$}}%
\end{pgfscope}%
\begin{pgfscope}%
\pgfpathrectangle{\pgfqpoint{0.589510in}{0.417642in}}{\pgfqpoint{3.437062in}{2.055000in}}%
\pgfusepath{clip}%
\pgfsetrectcap%
\pgfsetroundjoin%
\pgfsetlinewidth{0.803000pt}%
\definecolor{currentstroke}{rgb}{0.450000,0.450000,0.450000}%
\pgfsetstrokecolor{currentstroke}%
\pgfsetdash{}{0pt}%
\pgfpathmoveto{\pgfqpoint{3.870342in}{0.417642in}}%
\pgfpathlineto{\pgfqpoint{3.870342in}{2.472642in}}%
\pgfusepath{stroke}%
\end{pgfscope}%
\begin{pgfscope}%
\pgfsetbuttcap%
\pgfsetroundjoin%
\definecolor{currentfill}{rgb}{0.000000,0.000000,0.000000}%
\pgfsetfillcolor{currentfill}%
\pgfsetlinewidth{0.803000pt}%
\definecolor{currentstroke}{rgb}{0.000000,0.000000,0.000000}%
\pgfsetstrokecolor{currentstroke}%
\pgfsetdash{}{0pt}%
\pgfsys@defobject{currentmarker}{\pgfqpoint{0.000000in}{-0.048611in}}{\pgfqpoint{0.000000in}{0.000000in}}{%
\pgfpathmoveto{\pgfqpoint{0.000000in}{0.000000in}}%
\pgfpathlineto{\pgfqpoint{0.000000in}{-0.048611in}}%
\pgfusepath{stroke,fill}%
}%
\begin{pgfscope}%
\pgfsys@transformshift{3.870342in}{0.417642in}%
\pgfsys@useobject{currentmarker}{}%
\end{pgfscope}%
\end{pgfscope}%
\begin{pgfscope}%
\definecolor{textcolor}{rgb}{0.000000,0.000000,0.000000}%
\pgfsetstrokecolor{textcolor}%
\pgfsetfillcolor{textcolor}%
\pgftext[x=3.870342in,y=0.320420in,,top]{\color{textcolor}{\rmfamily\fontsize{8.000000}{9.600000}\selectfont\catcode`\^=\active\def^{\ifmmode\sp\else\^{}\fi}\catcode`\%=\active\def%{\%}$\mathdefault{10^{2}}$}}%
\end{pgfscope}%
\begin{pgfscope}%
\pgfpathrectangle{\pgfqpoint{0.589510in}{0.417642in}}{\pgfqpoint{3.437062in}{2.055000in}}%
\pgfusepath{clip}%
\pgfsetrectcap%
\pgfsetroundjoin%
\pgfsetlinewidth{0.803000pt}%
\definecolor{currentstroke}{rgb}{0.850000,0.850000,0.850000}%
\pgfsetstrokecolor{currentstroke}%
\pgfsetdash{}{0pt}%
\pgfpathmoveto{\pgfqpoint{0.745740in}{0.417642in}}%
\pgfpathlineto{\pgfqpoint{0.745740in}{2.472642in}}%
\pgfusepath{stroke}%
\end{pgfscope}%
\begin{pgfscope}%
\pgfsetbuttcap%
\pgfsetroundjoin%
\definecolor{currentfill}{rgb}{0.000000,0.000000,0.000000}%
\pgfsetfillcolor{currentfill}%
\pgfsetlinewidth{0.602250pt}%
\definecolor{currentstroke}{rgb}{0.000000,0.000000,0.000000}%
\pgfsetstrokecolor{currentstroke}%
\pgfsetdash{}{0pt}%
\pgfsys@defobject{currentmarker}{\pgfqpoint{0.000000in}{-0.027778in}}{\pgfqpoint{0.000000in}{0.000000in}}{%
\pgfpathmoveto{\pgfqpoint{0.000000in}{0.000000in}}%
\pgfpathlineto{\pgfqpoint{0.000000in}{-0.027778in}}%
\pgfusepath{stroke,fill}%
}%
\begin{pgfscope}%
\pgfsys@transformshift{0.745740in}{0.417642in}%
\pgfsys@useobject{currentmarker}{}%
\end{pgfscope}%
\end{pgfscope}%
\begin{pgfscope}%
\pgfpathrectangle{\pgfqpoint{0.589510in}{0.417642in}}{\pgfqpoint{3.437062in}{2.055000in}}%
\pgfusepath{clip}%
\pgfsetrectcap%
\pgfsetroundjoin%
\pgfsetlinewidth{0.803000pt}%
\definecolor{currentstroke}{rgb}{0.850000,0.850000,0.850000}%
\pgfsetstrokecolor{currentstroke}%
\pgfsetdash{}{0pt}%
\pgfpathmoveto{\pgfqpoint{0.894488in}{0.417642in}}%
\pgfpathlineto{\pgfqpoint{0.894488in}{2.472642in}}%
\pgfusepath{stroke}%
\end{pgfscope}%
\begin{pgfscope}%
\pgfsetbuttcap%
\pgfsetroundjoin%
\definecolor{currentfill}{rgb}{0.000000,0.000000,0.000000}%
\pgfsetfillcolor{currentfill}%
\pgfsetlinewidth{0.602250pt}%
\definecolor{currentstroke}{rgb}{0.000000,0.000000,0.000000}%
\pgfsetstrokecolor{currentstroke}%
\pgfsetdash{}{0pt}%
\pgfsys@defobject{currentmarker}{\pgfqpoint{0.000000in}{-0.027778in}}{\pgfqpoint{0.000000in}{0.000000in}}{%
\pgfpathmoveto{\pgfqpoint{0.000000in}{0.000000in}}%
\pgfpathlineto{\pgfqpoint{0.000000in}{-0.027778in}}%
\pgfusepath{stroke,fill}%
}%
\begin{pgfscope}%
\pgfsys@transformshift{0.894488in}{0.417642in}%
\pgfsys@useobject{currentmarker}{}%
\end{pgfscope}%
\end{pgfscope}%
\begin{pgfscope}%
\pgfpathrectangle{\pgfqpoint{0.589510in}{0.417642in}}{\pgfqpoint{3.437062in}{2.055000in}}%
\pgfusepath{clip}%
\pgfsetrectcap%
\pgfsetroundjoin%
\pgfsetlinewidth{0.803000pt}%
\definecolor{currentstroke}{rgb}{0.850000,0.850000,0.850000}%
\pgfsetstrokecolor{currentstroke}%
\pgfsetdash{}{0pt}%
\pgfpathmoveto{\pgfqpoint{1.000026in}{0.417642in}}%
\pgfpathlineto{\pgfqpoint{1.000026in}{2.472642in}}%
\pgfusepath{stroke}%
\end{pgfscope}%
\begin{pgfscope}%
\pgfsetbuttcap%
\pgfsetroundjoin%
\definecolor{currentfill}{rgb}{0.000000,0.000000,0.000000}%
\pgfsetfillcolor{currentfill}%
\pgfsetlinewidth{0.602250pt}%
\definecolor{currentstroke}{rgb}{0.000000,0.000000,0.000000}%
\pgfsetstrokecolor{currentstroke}%
\pgfsetdash{}{0pt}%
\pgfsys@defobject{currentmarker}{\pgfqpoint{0.000000in}{-0.027778in}}{\pgfqpoint{0.000000in}{0.000000in}}{%
\pgfpathmoveto{\pgfqpoint{0.000000in}{0.000000in}}%
\pgfpathlineto{\pgfqpoint{0.000000in}{-0.027778in}}%
\pgfusepath{stroke,fill}%
}%
\begin{pgfscope}%
\pgfsys@transformshift{1.000026in}{0.417642in}%
\pgfsys@useobject{currentmarker}{}%
\end{pgfscope}%
\end{pgfscope}%
\begin{pgfscope}%
\pgfpathrectangle{\pgfqpoint{0.589510in}{0.417642in}}{\pgfqpoint{3.437062in}{2.055000in}}%
\pgfusepath{clip}%
\pgfsetrectcap%
\pgfsetroundjoin%
\pgfsetlinewidth{0.803000pt}%
\definecolor{currentstroke}{rgb}{0.850000,0.850000,0.850000}%
\pgfsetstrokecolor{currentstroke}%
\pgfsetdash{}{0pt}%
\pgfpathmoveto{\pgfqpoint{1.081889in}{0.417642in}}%
\pgfpathlineto{\pgfqpoint{1.081889in}{2.472642in}}%
\pgfusepath{stroke}%
\end{pgfscope}%
\begin{pgfscope}%
\pgfsetbuttcap%
\pgfsetroundjoin%
\definecolor{currentfill}{rgb}{0.000000,0.000000,0.000000}%
\pgfsetfillcolor{currentfill}%
\pgfsetlinewidth{0.602250pt}%
\definecolor{currentstroke}{rgb}{0.000000,0.000000,0.000000}%
\pgfsetstrokecolor{currentstroke}%
\pgfsetdash{}{0pt}%
\pgfsys@defobject{currentmarker}{\pgfqpoint{0.000000in}{-0.027778in}}{\pgfqpoint{0.000000in}{0.000000in}}{%
\pgfpathmoveto{\pgfqpoint{0.000000in}{0.000000in}}%
\pgfpathlineto{\pgfqpoint{0.000000in}{-0.027778in}}%
\pgfusepath{stroke,fill}%
}%
\begin{pgfscope}%
\pgfsys@transformshift{1.081889in}{0.417642in}%
\pgfsys@useobject{currentmarker}{}%
\end{pgfscope}%
\end{pgfscope}%
\begin{pgfscope}%
\pgfpathrectangle{\pgfqpoint{0.589510in}{0.417642in}}{\pgfqpoint{3.437062in}{2.055000in}}%
\pgfusepath{clip}%
\pgfsetrectcap%
\pgfsetroundjoin%
\pgfsetlinewidth{0.803000pt}%
\definecolor{currentstroke}{rgb}{0.850000,0.850000,0.850000}%
\pgfsetstrokecolor{currentstroke}%
\pgfsetdash{}{0pt}%
\pgfpathmoveto{\pgfqpoint{1.148775in}{0.417642in}}%
\pgfpathlineto{\pgfqpoint{1.148775in}{2.472642in}}%
\pgfusepath{stroke}%
\end{pgfscope}%
\begin{pgfscope}%
\pgfsetbuttcap%
\pgfsetroundjoin%
\definecolor{currentfill}{rgb}{0.000000,0.000000,0.000000}%
\pgfsetfillcolor{currentfill}%
\pgfsetlinewidth{0.602250pt}%
\definecolor{currentstroke}{rgb}{0.000000,0.000000,0.000000}%
\pgfsetstrokecolor{currentstroke}%
\pgfsetdash{}{0pt}%
\pgfsys@defobject{currentmarker}{\pgfqpoint{0.000000in}{-0.027778in}}{\pgfqpoint{0.000000in}{0.000000in}}{%
\pgfpathmoveto{\pgfqpoint{0.000000in}{0.000000in}}%
\pgfpathlineto{\pgfqpoint{0.000000in}{-0.027778in}}%
\pgfusepath{stroke,fill}%
}%
\begin{pgfscope}%
\pgfsys@transformshift{1.148775in}{0.417642in}%
\pgfsys@useobject{currentmarker}{}%
\end{pgfscope}%
\end{pgfscope}%
\begin{pgfscope}%
\pgfpathrectangle{\pgfqpoint{0.589510in}{0.417642in}}{\pgfqpoint{3.437062in}{2.055000in}}%
\pgfusepath{clip}%
\pgfsetrectcap%
\pgfsetroundjoin%
\pgfsetlinewidth{0.803000pt}%
\definecolor{currentstroke}{rgb}{0.850000,0.850000,0.850000}%
\pgfsetstrokecolor{currentstroke}%
\pgfsetdash{}{0pt}%
\pgfpathmoveto{\pgfqpoint{1.205326in}{0.417642in}}%
\pgfpathlineto{\pgfqpoint{1.205326in}{2.472642in}}%
\pgfusepath{stroke}%
\end{pgfscope}%
\begin{pgfscope}%
\pgfsetbuttcap%
\pgfsetroundjoin%
\definecolor{currentfill}{rgb}{0.000000,0.000000,0.000000}%
\pgfsetfillcolor{currentfill}%
\pgfsetlinewidth{0.602250pt}%
\definecolor{currentstroke}{rgb}{0.000000,0.000000,0.000000}%
\pgfsetstrokecolor{currentstroke}%
\pgfsetdash{}{0pt}%
\pgfsys@defobject{currentmarker}{\pgfqpoint{0.000000in}{-0.027778in}}{\pgfqpoint{0.000000in}{0.000000in}}{%
\pgfpathmoveto{\pgfqpoint{0.000000in}{0.000000in}}%
\pgfpathlineto{\pgfqpoint{0.000000in}{-0.027778in}}%
\pgfusepath{stroke,fill}%
}%
\begin{pgfscope}%
\pgfsys@transformshift{1.205326in}{0.417642in}%
\pgfsys@useobject{currentmarker}{}%
\end{pgfscope}%
\end{pgfscope}%
\begin{pgfscope}%
\pgfpathrectangle{\pgfqpoint{0.589510in}{0.417642in}}{\pgfqpoint{3.437062in}{2.055000in}}%
\pgfusepath{clip}%
\pgfsetrectcap%
\pgfsetroundjoin%
\pgfsetlinewidth{0.803000pt}%
\definecolor{currentstroke}{rgb}{0.850000,0.850000,0.850000}%
\pgfsetstrokecolor{currentstroke}%
\pgfsetdash{}{0pt}%
\pgfpathmoveto{\pgfqpoint{1.254313in}{0.417642in}}%
\pgfpathlineto{\pgfqpoint{1.254313in}{2.472642in}}%
\pgfusepath{stroke}%
\end{pgfscope}%
\begin{pgfscope}%
\pgfsetbuttcap%
\pgfsetroundjoin%
\definecolor{currentfill}{rgb}{0.000000,0.000000,0.000000}%
\pgfsetfillcolor{currentfill}%
\pgfsetlinewidth{0.602250pt}%
\definecolor{currentstroke}{rgb}{0.000000,0.000000,0.000000}%
\pgfsetstrokecolor{currentstroke}%
\pgfsetdash{}{0pt}%
\pgfsys@defobject{currentmarker}{\pgfqpoint{0.000000in}{-0.027778in}}{\pgfqpoint{0.000000in}{0.000000in}}{%
\pgfpathmoveto{\pgfqpoint{0.000000in}{0.000000in}}%
\pgfpathlineto{\pgfqpoint{0.000000in}{-0.027778in}}%
\pgfusepath{stroke,fill}%
}%
\begin{pgfscope}%
\pgfsys@transformshift{1.254313in}{0.417642in}%
\pgfsys@useobject{currentmarker}{}%
\end{pgfscope}%
\end{pgfscope}%
\begin{pgfscope}%
\pgfpathrectangle{\pgfqpoint{0.589510in}{0.417642in}}{\pgfqpoint{3.437062in}{2.055000in}}%
\pgfusepath{clip}%
\pgfsetrectcap%
\pgfsetroundjoin%
\pgfsetlinewidth{0.803000pt}%
\definecolor{currentstroke}{rgb}{0.850000,0.850000,0.850000}%
\pgfsetstrokecolor{currentstroke}%
\pgfsetdash{}{0pt}%
\pgfpathmoveto{\pgfqpoint{1.297523in}{0.417642in}}%
\pgfpathlineto{\pgfqpoint{1.297523in}{2.472642in}}%
\pgfusepath{stroke}%
\end{pgfscope}%
\begin{pgfscope}%
\pgfsetbuttcap%
\pgfsetroundjoin%
\definecolor{currentfill}{rgb}{0.000000,0.000000,0.000000}%
\pgfsetfillcolor{currentfill}%
\pgfsetlinewidth{0.602250pt}%
\definecolor{currentstroke}{rgb}{0.000000,0.000000,0.000000}%
\pgfsetstrokecolor{currentstroke}%
\pgfsetdash{}{0pt}%
\pgfsys@defobject{currentmarker}{\pgfqpoint{0.000000in}{-0.027778in}}{\pgfqpoint{0.000000in}{0.000000in}}{%
\pgfpathmoveto{\pgfqpoint{0.000000in}{0.000000in}}%
\pgfpathlineto{\pgfqpoint{0.000000in}{-0.027778in}}%
\pgfusepath{stroke,fill}%
}%
\begin{pgfscope}%
\pgfsys@transformshift{1.297523in}{0.417642in}%
\pgfsys@useobject{currentmarker}{}%
\end{pgfscope}%
\end{pgfscope}%
\begin{pgfscope}%
\pgfpathrectangle{\pgfqpoint{0.589510in}{0.417642in}}{\pgfqpoint{3.437062in}{2.055000in}}%
\pgfusepath{clip}%
\pgfsetrectcap%
\pgfsetroundjoin%
\pgfsetlinewidth{0.803000pt}%
\definecolor{currentstroke}{rgb}{0.850000,0.850000,0.850000}%
\pgfsetstrokecolor{currentstroke}%
\pgfsetdash{}{0pt}%
\pgfpathmoveto{\pgfqpoint{1.590462in}{0.417642in}}%
\pgfpathlineto{\pgfqpoint{1.590462in}{2.472642in}}%
\pgfusepath{stroke}%
\end{pgfscope}%
\begin{pgfscope}%
\pgfsetbuttcap%
\pgfsetroundjoin%
\definecolor{currentfill}{rgb}{0.000000,0.000000,0.000000}%
\pgfsetfillcolor{currentfill}%
\pgfsetlinewidth{0.602250pt}%
\definecolor{currentstroke}{rgb}{0.000000,0.000000,0.000000}%
\pgfsetstrokecolor{currentstroke}%
\pgfsetdash{}{0pt}%
\pgfsys@defobject{currentmarker}{\pgfqpoint{0.000000in}{-0.027778in}}{\pgfqpoint{0.000000in}{0.000000in}}{%
\pgfpathmoveto{\pgfqpoint{0.000000in}{0.000000in}}%
\pgfpathlineto{\pgfqpoint{0.000000in}{-0.027778in}}%
\pgfusepath{stroke,fill}%
}%
\begin{pgfscope}%
\pgfsys@transformshift{1.590462in}{0.417642in}%
\pgfsys@useobject{currentmarker}{}%
\end{pgfscope}%
\end{pgfscope}%
\begin{pgfscope}%
\pgfpathrectangle{\pgfqpoint{0.589510in}{0.417642in}}{\pgfqpoint{3.437062in}{2.055000in}}%
\pgfusepath{clip}%
\pgfsetrectcap%
\pgfsetroundjoin%
\pgfsetlinewidth{0.803000pt}%
\definecolor{currentstroke}{rgb}{0.850000,0.850000,0.850000}%
\pgfsetstrokecolor{currentstroke}%
\pgfsetdash{}{0pt}%
\pgfpathmoveto{\pgfqpoint{1.739210in}{0.417642in}}%
\pgfpathlineto{\pgfqpoint{1.739210in}{2.472642in}}%
\pgfusepath{stroke}%
\end{pgfscope}%
\begin{pgfscope}%
\pgfsetbuttcap%
\pgfsetroundjoin%
\definecolor{currentfill}{rgb}{0.000000,0.000000,0.000000}%
\pgfsetfillcolor{currentfill}%
\pgfsetlinewidth{0.602250pt}%
\definecolor{currentstroke}{rgb}{0.000000,0.000000,0.000000}%
\pgfsetstrokecolor{currentstroke}%
\pgfsetdash{}{0pt}%
\pgfsys@defobject{currentmarker}{\pgfqpoint{0.000000in}{-0.027778in}}{\pgfqpoint{0.000000in}{0.000000in}}{%
\pgfpathmoveto{\pgfqpoint{0.000000in}{0.000000in}}%
\pgfpathlineto{\pgfqpoint{0.000000in}{-0.027778in}}%
\pgfusepath{stroke,fill}%
}%
\begin{pgfscope}%
\pgfsys@transformshift{1.739210in}{0.417642in}%
\pgfsys@useobject{currentmarker}{}%
\end{pgfscope}%
\end{pgfscope}%
\begin{pgfscope}%
\pgfpathrectangle{\pgfqpoint{0.589510in}{0.417642in}}{\pgfqpoint{3.437062in}{2.055000in}}%
\pgfusepath{clip}%
\pgfsetrectcap%
\pgfsetroundjoin%
\pgfsetlinewidth{0.803000pt}%
\definecolor{currentstroke}{rgb}{0.850000,0.850000,0.850000}%
\pgfsetstrokecolor{currentstroke}%
\pgfsetdash{}{0pt}%
\pgfpathmoveto{\pgfqpoint{1.844749in}{0.417642in}}%
\pgfpathlineto{\pgfqpoint{1.844749in}{2.472642in}}%
\pgfusepath{stroke}%
\end{pgfscope}%
\begin{pgfscope}%
\pgfsetbuttcap%
\pgfsetroundjoin%
\definecolor{currentfill}{rgb}{0.000000,0.000000,0.000000}%
\pgfsetfillcolor{currentfill}%
\pgfsetlinewidth{0.602250pt}%
\definecolor{currentstroke}{rgb}{0.000000,0.000000,0.000000}%
\pgfsetstrokecolor{currentstroke}%
\pgfsetdash{}{0pt}%
\pgfsys@defobject{currentmarker}{\pgfqpoint{0.000000in}{-0.027778in}}{\pgfqpoint{0.000000in}{0.000000in}}{%
\pgfpathmoveto{\pgfqpoint{0.000000in}{0.000000in}}%
\pgfpathlineto{\pgfqpoint{0.000000in}{-0.027778in}}%
\pgfusepath{stroke,fill}%
}%
\begin{pgfscope}%
\pgfsys@transformshift{1.844749in}{0.417642in}%
\pgfsys@useobject{currentmarker}{}%
\end{pgfscope}%
\end{pgfscope}%
\begin{pgfscope}%
\pgfpathrectangle{\pgfqpoint{0.589510in}{0.417642in}}{\pgfqpoint{3.437062in}{2.055000in}}%
\pgfusepath{clip}%
\pgfsetrectcap%
\pgfsetroundjoin%
\pgfsetlinewidth{0.803000pt}%
\definecolor{currentstroke}{rgb}{0.850000,0.850000,0.850000}%
\pgfsetstrokecolor{currentstroke}%
\pgfsetdash{}{0pt}%
\pgfpathmoveto{\pgfqpoint{1.926611in}{0.417642in}}%
\pgfpathlineto{\pgfqpoint{1.926611in}{2.472642in}}%
\pgfusepath{stroke}%
\end{pgfscope}%
\begin{pgfscope}%
\pgfsetbuttcap%
\pgfsetroundjoin%
\definecolor{currentfill}{rgb}{0.000000,0.000000,0.000000}%
\pgfsetfillcolor{currentfill}%
\pgfsetlinewidth{0.602250pt}%
\definecolor{currentstroke}{rgb}{0.000000,0.000000,0.000000}%
\pgfsetstrokecolor{currentstroke}%
\pgfsetdash{}{0pt}%
\pgfsys@defobject{currentmarker}{\pgfqpoint{0.000000in}{-0.027778in}}{\pgfqpoint{0.000000in}{0.000000in}}{%
\pgfpathmoveto{\pgfqpoint{0.000000in}{0.000000in}}%
\pgfpathlineto{\pgfqpoint{0.000000in}{-0.027778in}}%
\pgfusepath{stroke,fill}%
}%
\begin{pgfscope}%
\pgfsys@transformshift{1.926611in}{0.417642in}%
\pgfsys@useobject{currentmarker}{}%
\end{pgfscope}%
\end{pgfscope}%
\begin{pgfscope}%
\pgfpathrectangle{\pgfqpoint{0.589510in}{0.417642in}}{\pgfqpoint{3.437062in}{2.055000in}}%
\pgfusepath{clip}%
\pgfsetrectcap%
\pgfsetroundjoin%
\pgfsetlinewidth{0.803000pt}%
\definecolor{currentstroke}{rgb}{0.850000,0.850000,0.850000}%
\pgfsetstrokecolor{currentstroke}%
\pgfsetdash{}{0pt}%
\pgfpathmoveto{\pgfqpoint{1.993497in}{0.417642in}}%
\pgfpathlineto{\pgfqpoint{1.993497in}{2.472642in}}%
\pgfusepath{stroke}%
\end{pgfscope}%
\begin{pgfscope}%
\pgfsetbuttcap%
\pgfsetroundjoin%
\definecolor{currentfill}{rgb}{0.000000,0.000000,0.000000}%
\pgfsetfillcolor{currentfill}%
\pgfsetlinewidth{0.602250pt}%
\definecolor{currentstroke}{rgb}{0.000000,0.000000,0.000000}%
\pgfsetstrokecolor{currentstroke}%
\pgfsetdash{}{0pt}%
\pgfsys@defobject{currentmarker}{\pgfqpoint{0.000000in}{-0.027778in}}{\pgfqpoint{0.000000in}{0.000000in}}{%
\pgfpathmoveto{\pgfqpoint{0.000000in}{0.000000in}}%
\pgfpathlineto{\pgfqpoint{0.000000in}{-0.027778in}}%
\pgfusepath{stroke,fill}%
}%
\begin{pgfscope}%
\pgfsys@transformshift{1.993497in}{0.417642in}%
\pgfsys@useobject{currentmarker}{}%
\end{pgfscope}%
\end{pgfscope}%
\begin{pgfscope}%
\pgfpathrectangle{\pgfqpoint{0.589510in}{0.417642in}}{\pgfqpoint{3.437062in}{2.055000in}}%
\pgfusepath{clip}%
\pgfsetrectcap%
\pgfsetroundjoin%
\pgfsetlinewidth{0.803000pt}%
\definecolor{currentstroke}{rgb}{0.850000,0.850000,0.850000}%
\pgfsetstrokecolor{currentstroke}%
\pgfsetdash{}{0pt}%
\pgfpathmoveto{\pgfqpoint{2.050048in}{0.417642in}}%
\pgfpathlineto{\pgfqpoint{2.050048in}{2.472642in}}%
\pgfusepath{stroke}%
\end{pgfscope}%
\begin{pgfscope}%
\pgfsetbuttcap%
\pgfsetroundjoin%
\definecolor{currentfill}{rgb}{0.000000,0.000000,0.000000}%
\pgfsetfillcolor{currentfill}%
\pgfsetlinewidth{0.602250pt}%
\definecolor{currentstroke}{rgb}{0.000000,0.000000,0.000000}%
\pgfsetstrokecolor{currentstroke}%
\pgfsetdash{}{0pt}%
\pgfsys@defobject{currentmarker}{\pgfqpoint{0.000000in}{-0.027778in}}{\pgfqpoint{0.000000in}{0.000000in}}{%
\pgfpathmoveto{\pgfqpoint{0.000000in}{0.000000in}}%
\pgfpathlineto{\pgfqpoint{0.000000in}{-0.027778in}}%
\pgfusepath{stroke,fill}%
}%
\begin{pgfscope}%
\pgfsys@transformshift{2.050048in}{0.417642in}%
\pgfsys@useobject{currentmarker}{}%
\end{pgfscope}%
\end{pgfscope}%
\begin{pgfscope}%
\pgfpathrectangle{\pgfqpoint{0.589510in}{0.417642in}}{\pgfqpoint{3.437062in}{2.055000in}}%
\pgfusepath{clip}%
\pgfsetrectcap%
\pgfsetroundjoin%
\pgfsetlinewidth{0.803000pt}%
\definecolor{currentstroke}{rgb}{0.850000,0.850000,0.850000}%
\pgfsetstrokecolor{currentstroke}%
\pgfsetdash{}{0pt}%
\pgfpathmoveto{\pgfqpoint{2.099035in}{0.417642in}}%
\pgfpathlineto{\pgfqpoint{2.099035in}{2.472642in}}%
\pgfusepath{stroke}%
\end{pgfscope}%
\begin{pgfscope}%
\pgfsetbuttcap%
\pgfsetroundjoin%
\definecolor{currentfill}{rgb}{0.000000,0.000000,0.000000}%
\pgfsetfillcolor{currentfill}%
\pgfsetlinewidth{0.602250pt}%
\definecolor{currentstroke}{rgb}{0.000000,0.000000,0.000000}%
\pgfsetstrokecolor{currentstroke}%
\pgfsetdash{}{0pt}%
\pgfsys@defobject{currentmarker}{\pgfqpoint{0.000000in}{-0.027778in}}{\pgfqpoint{0.000000in}{0.000000in}}{%
\pgfpathmoveto{\pgfqpoint{0.000000in}{0.000000in}}%
\pgfpathlineto{\pgfqpoint{0.000000in}{-0.027778in}}%
\pgfusepath{stroke,fill}%
}%
\begin{pgfscope}%
\pgfsys@transformshift{2.099035in}{0.417642in}%
\pgfsys@useobject{currentmarker}{}%
\end{pgfscope}%
\end{pgfscope}%
\begin{pgfscope}%
\pgfpathrectangle{\pgfqpoint{0.589510in}{0.417642in}}{\pgfqpoint{3.437062in}{2.055000in}}%
\pgfusepath{clip}%
\pgfsetrectcap%
\pgfsetroundjoin%
\pgfsetlinewidth{0.803000pt}%
\definecolor{currentstroke}{rgb}{0.850000,0.850000,0.850000}%
\pgfsetstrokecolor{currentstroke}%
\pgfsetdash{}{0pt}%
\pgfpathmoveto{\pgfqpoint{2.142245in}{0.417642in}}%
\pgfpathlineto{\pgfqpoint{2.142245in}{2.472642in}}%
\pgfusepath{stroke}%
\end{pgfscope}%
\begin{pgfscope}%
\pgfsetbuttcap%
\pgfsetroundjoin%
\definecolor{currentfill}{rgb}{0.000000,0.000000,0.000000}%
\pgfsetfillcolor{currentfill}%
\pgfsetlinewidth{0.602250pt}%
\definecolor{currentstroke}{rgb}{0.000000,0.000000,0.000000}%
\pgfsetstrokecolor{currentstroke}%
\pgfsetdash{}{0pt}%
\pgfsys@defobject{currentmarker}{\pgfqpoint{0.000000in}{-0.027778in}}{\pgfqpoint{0.000000in}{0.000000in}}{%
\pgfpathmoveto{\pgfqpoint{0.000000in}{0.000000in}}%
\pgfpathlineto{\pgfqpoint{0.000000in}{-0.027778in}}%
\pgfusepath{stroke,fill}%
}%
\begin{pgfscope}%
\pgfsys@transformshift{2.142245in}{0.417642in}%
\pgfsys@useobject{currentmarker}{}%
\end{pgfscope}%
\end{pgfscope}%
\begin{pgfscope}%
\pgfpathrectangle{\pgfqpoint{0.589510in}{0.417642in}}{\pgfqpoint{3.437062in}{2.055000in}}%
\pgfusepath{clip}%
\pgfsetrectcap%
\pgfsetroundjoin%
\pgfsetlinewidth{0.803000pt}%
\definecolor{currentstroke}{rgb}{0.850000,0.850000,0.850000}%
\pgfsetstrokecolor{currentstroke}%
\pgfsetdash{}{0pt}%
\pgfpathmoveto{\pgfqpoint{2.435184in}{0.417642in}}%
\pgfpathlineto{\pgfqpoint{2.435184in}{2.472642in}}%
\pgfusepath{stroke}%
\end{pgfscope}%
\begin{pgfscope}%
\pgfsetbuttcap%
\pgfsetroundjoin%
\definecolor{currentfill}{rgb}{0.000000,0.000000,0.000000}%
\pgfsetfillcolor{currentfill}%
\pgfsetlinewidth{0.602250pt}%
\definecolor{currentstroke}{rgb}{0.000000,0.000000,0.000000}%
\pgfsetstrokecolor{currentstroke}%
\pgfsetdash{}{0pt}%
\pgfsys@defobject{currentmarker}{\pgfqpoint{0.000000in}{-0.027778in}}{\pgfqpoint{0.000000in}{0.000000in}}{%
\pgfpathmoveto{\pgfqpoint{0.000000in}{0.000000in}}%
\pgfpathlineto{\pgfqpoint{0.000000in}{-0.027778in}}%
\pgfusepath{stroke,fill}%
}%
\begin{pgfscope}%
\pgfsys@transformshift{2.435184in}{0.417642in}%
\pgfsys@useobject{currentmarker}{}%
\end{pgfscope}%
\end{pgfscope}%
\begin{pgfscope}%
\pgfpathrectangle{\pgfqpoint{0.589510in}{0.417642in}}{\pgfqpoint{3.437062in}{2.055000in}}%
\pgfusepath{clip}%
\pgfsetrectcap%
\pgfsetroundjoin%
\pgfsetlinewidth{0.803000pt}%
\definecolor{currentstroke}{rgb}{0.850000,0.850000,0.850000}%
\pgfsetstrokecolor{currentstroke}%
\pgfsetdash{}{0pt}%
\pgfpathmoveto{\pgfqpoint{2.583932in}{0.417642in}}%
\pgfpathlineto{\pgfqpoint{2.583932in}{2.472642in}}%
\pgfusepath{stroke}%
\end{pgfscope}%
\begin{pgfscope}%
\pgfsetbuttcap%
\pgfsetroundjoin%
\definecolor{currentfill}{rgb}{0.000000,0.000000,0.000000}%
\pgfsetfillcolor{currentfill}%
\pgfsetlinewidth{0.602250pt}%
\definecolor{currentstroke}{rgb}{0.000000,0.000000,0.000000}%
\pgfsetstrokecolor{currentstroke}%
\pgfsetdash{}{0pt}%
\pgfsys@defobject{currentmarker}{\pgfqpoint{0.000000in}{-0.027778in}}{\pgfqpoint{0.000000in}{0.000000in}}{%
\pgfpathmoveto{\pgfqpoint{0.000000in}{0.000000in}}%
\pgfpathlineto{\pgfqpoint{0.000000in}{-0.027778in}}%
\pgfusepath{stroke,fill}%
}%
\begin{pgfscope}%
\pgfsys@transformshift{2.583932in}{0.417642in}%
\pgfsys@useobject{currentmarker}{}%
\end{pgfscope}%
\end{pgfscope}%
\begin{pgfscope}%
\pgfpathrectangle{\pgfqpoint{0.589510in}{0.417642in}}{\pgfqpoint{3.437062in}{2.055000in}}%
\pgfusepath{clip}%
\pgfsetrectcap%
\pgfsetroundjoin%
\pgfsetlinewidth{0.803000pt}%
\definecolor{currentstroke}{rgb}{0.850000,0.850000,0.850000}%
\pgfsetstrokecolor{currentstroke}%
\pgfsetdash{}{0pt}%
\pgfpathmoveto{\pgfqpoint{2.689471in}{0.417642in}}%
\pgfpathlineto{\pgfqpoint{2.689471in}{2.472642in}}%
\pgfusepath{stroke}%
\end{pgfscope}%
\begin{pgfscope}%
\pgfsetbuttcap%
\pgfsetroundjoin%
\definecolor{currentfill}{rgb}{0.000000,0.000000,0.000000}%
\pgfsetfillcolor{currentfill}%
\pgfsetlinewidth{0.602250pt}%
\definecolor{currentstroke}{rgb}{0.000000,0.000000,0.000000}%
\pgfsetstrokecolor{currentstroke}%
\pgfsetdash{}{0pt}%
\pgfsys@defobject{currentmarker}{\pgfqpoint{0.000000in}{-0.027778in}}{\pgfqpoint{0.000000in}{0.000000in}}{%
\pgfpathmoveto{\pgfqpoint{0.000000in}{0.000000in}}%
\pgfpathlineto{\pgfqpoint{0.000000in}{-0.027778in}}%
\pgfusepath{stroke,fill}%
}%
\begin{pgfscope}%
\pgfsys@transformshift{2.689471in}{0.417642in}%
\pgfsys@useobject{currentmarker}{}%
\end{pgfscope}%
\end{pgfscope}%
\begin{pgfscope}%
\pgfpathrectangle{\pgfqpoint{0.589510in}{0.417642in}}{\pgfqpoint{3.437062in}{2.055000in}}%
\pgfusepath{clip}%
\pgfsetrectcap%
\pgfsetroundjoin%
\pgfsetlinewidth{0.803000pt}%
\definecolor{currentstroke}{rgb}{0.850000,0.850000,0.850000}%
\pgfsetstrokecolor{currentstroke}%
\pgfsetdash{}{0pt}%
\pgfpathmoveto{\pgfqpoint{2.771333in}{0.417642in}}%
\pgfpathlineto{\pgfqpoint{2.771333in}{2.472642in}}%
\pgfusepath{stroke}%
\end{pgfscope}%
\begin{pgfscope}%
\pgfsetbuttcap%
\pgfsetroundjoin%
\definecolor{currentfill}{rgb}{0.000000,0.000000,0.000000}%
\pgfsetfillcolor{currentfill}%
\pgfsetlinewidth{0.602250pt}%
\definecolor{currentstroke}{rgb}{0.000000,0.000000,0.000000}%
\pgfsetstrokecolor{currentstroke}%
\pgfsetdash{}{0pt}%
\pgfsys@defobject{currentmarker}{\pgfqpoint{0.000000in}{-0.027778in}}{\pgfqpoint{0.000000in}{0.000000in}}{%
\pgfpathmoveto{\pgfqpoint{0.000000in}{0.000000in}}%
\pgfpathlineto{\pgfqpoint{0.000000in}{-0.027778in}}%
\pgfusepath{stroke,fill}%
}%
\begin{pgfscope}%
\pgfsys@transformshift{2.771333in}{0.417642in}%
\pgfsys@useobject{currentmarker}{}%
\end{pgfscope}%
\end{pgfscope}%
\begin{pgfscope}%
\pgfpathrectangle{\pgfqpoint{0.589510in}{0.417642in}}{\pgfqpoint{3.437062in}{2.055000in}}%
\pgfusepath{clip}%
\pgfsetrectcap%
\pgfsetroundjoin%
\pgfsetlinewidth{0.803000pt}%
\definecolor{currentstroke}{rgb}{0.850000,0.850000,0.850000}%
\pgfsetstrokecolor{currentstroke}%
\pgfsetdash{}{0pt}%
\pgfpathmoveto{\pgfqpoint{2.838219in}{0.417642in}}%
\pgfpathlineto{\pgfqpoint{2.838219in}{2.472642in}}%
\pgfusepath{stroke}%
\end{pgfscope}%
\begin{pgfscope}%
\pgfsetbuttcap%
\pgfsetroundjoin%
\definecolor{currentfill}{rgb}{0.000000,0.000000,0.000000}%
\pgfsetfillcolor{currentfill}%
\pgfsetlinewidth{0.602250pt}%
\definecolor{currentstroke}{rgb}{0.000000,0.000000,0.000000}%
\pgfsetstrokecolor{currentstroke}%
\pgfsetdash{}{0pt}%
\pgfsys@defobject{currentmarker}{\pgfqpoint{0.000000in}{-0.027778in}}{\pgfqpoint{0.000000in}{0.000000in}}{%
\pgfpathmoveto{\pgfqpoint{0.000000in}{0.000000in}}%
\pgfpathlineto{\pgfqpoint{0.000000in}{-0.027778in}}%
\pgfusepath{stroke,fill}%
}%
\begin{pgfscope}%
\pgfsys@transformshift{2.838219in}{0.417642in}%
\pgfsys@useobject{currentmarker}{}%
\end{pgfscope}%
\end{pgfscope}%
\begin{pgfscope}%
\pgfpathrectangle{\pgfqpoint{0.589510in}{0.417642in}}{\pgfqpoint{3.437062in}{2.055000in}}%
\pgfusepath{clip}%
\pgfsetrectcap%
\pgfsetroundjoin%
\pgfsetlinewidth{0.803000pt}%
\definecolor{currentstroke}{rgb}{0.850000,0.850000,0.850000}%
\pgfsetstrokecolor{currentstroke}%
\pgfsetdash{}{0pt}%
\pgfpathmoveto{\pgfqpoint{2.894770in}{0.417642in}}%
\pgfpathlineto{\pgfqpoint{2.894770in}{2.472642in}}%
\pgfusepath{stroke}%
\end{pgfscope}%
\begin{pgfscope}%
\pgfsetbuttcap%
\pgfsetroundjoin%
\definecolor{currentfill}{rgb}{0.000000,0.000000,0.000000}%
\pgfsetfillcolor{currentfill}%
\pgfsetlinewidth{0.602250pt}%
\definecolor{currentstroke}{rgb}{0.000000,0.000000,0.000000}%
\pgfsetstrokecolor{currentstroke}%
\pgfsetdash{}{0pt}%
\pgfsys@defobject{currentmarker}{\pgfqpoint{0.000000in}{-0.027778in}}{\pgfqpoint{0.000000in}{0.000000in}}{%
\pgfpathmoveto{\pgfqpoint{0.000000in}{0.000000in}}%
\pgfpathlineto{\pgfqpoint{0.000000in}{-0.027778in}}%
\pgfusepath{stroke,fill}%
}%
\begin{pgfscope}%
\pgfsys@transformshift{2.894770in}{0.417642in}%
\pgfsys@useobject{currentmarker}{}%
\end{pgfscope}%
\end{pgfscope}%
\begin{pgfscope}%
\pgfpathrectangle{\pgfqpoint{0.589510in}{0.417642in}}{\pgfqpoint{3.437062in}{2.055000in}}%
\pgfusepath{clip}%
\pgfsetrectcap%
\pgfsetroundjoin%
\pgfsetlinewidth{0.803000pt}%
\definecolor{currentstroke}{rgb}{0.850000,0.850000,0.850000}%
\pgfsetstrokecolor{currentstroke}%
\pgfsetdash{}{0pt}%
\pgfpathmoveto{\pgfqpoint{2.943757in}{0.417642in}}%
\pgfpathlineto{\pgfqpoint{2.943757in}{2.472642in}}%
\pgfusepath{stroke}%
\end{pgfscope}%
\begin{pgfscope}%
\pgfsetbuttcap%
\pgfsetroundjoin%
\definecolor{currentfill}{rgb}{0.000000,0.000000,0.000000}%
\pgfsetfillcolor{currentfill}%
\pgfsetlinewidth{0.602250pt}%
\definecolor{currentstroke}{rgb}{0.000000,0.000000,0.000000}%
\pgfsetstrokecolor{currentstroke}%
\pgfsetdash{}{0pt}%
\pgfsys@defobject{currentmarker}{\pgfqpoint{0.000000in}{-0.027778in}}{\pgfqpoint{0.000000in}{0.000000in}}{%
\pgfpathmoveto{\pgfqpoint{0.000000in}{0.000000in}}%
\pgfpathlineto{\pgfqpoint{0.000000in}{-0.027778in}}%
\pgfusepath{stroke,fill}%
}%
\begin{pgfscope}%
\pgfsys@transformshift{2.943757in}{0.417642in}%
\pgfsys@useobject{currentmarker}{}%
\end{pgfscope}%
\end{pgfscope}%
\begin{pgfscope}%
\pgfpathrectangle{\pgfqpoint{0.589510in}{0.417642in}}{\pgfqpoint{3.437062in}{2.055000in}}%
\pgfusepath{clip}%
\pgfsetrectcap%
\pgfsetroundjoin%
\pgfsetlinewidth{0.803000pt}%
\definecolor{currentstroke}{rgb}{0.850000,0.850000,0.850000}%
\pgfsetstrokecolor{currentstroke}%
\pgfsetdash{}{0pt}%
\pgfpathmoveto{\pgfqpoint{2.986967in}{0.417642in}}%
\pgfpathlineto{\pgfqpoint{2.986967in}{2.472642in}}%
\pgfusepath{stroke}%
\end{pgfscope}%
\begin{pgfscope}%
\pgfsetbuttcap%
\pgfsetroundjoin%
\definecolor{currentfill}{rgb}{0.000000,0.000000,0.000000}%
\pgfsetfillcolor{currentfill}%
\pgfsetlinewidth{0.602250pt}%
\definecolor{currentstroke}{rgb}{0.000000,0.000000,0.000000}%
\pgfsetstrokecolor{currentstroke}%
\pgfsetdash{}{0pt}%
\pgfsys@defobject{currentmarker}{\pgfqpoint{0.000000in}{-0.027778in}}{\pgfqpoint{0.000000in}{0.000000in}}{%
\pgfpathmoveto{\pgfqpoint{0.000000in}{0.000000in}}%
\pgfpathlineto{\pgfqpoint{0.000000in}{-0.027778in}}%
\pgfusepath{stroke,fill}%
}%
\begin{pgfscope}%
\pgfsys@transformshift{2.986967in}{0.417642in}%
\pgfsys@useobject{currentmarker}{}%
\end{pgfscope}%
\end{pgfscope}%
\begin{pgfscope}%
\pgfpathrectangle{\pgfqpoint{0.589510in}{0.417642in}}{\pgfqpoint{3.437062in}{2.055000in}}%
\pgfusepath{clip}%
\pgfsetrectcap%
\pgfsetroundjoin%
\pgfsetlinewidth{0.803000pt}%
\definecolor{currentstroke}{rgb}{0.850000,0.850000,0.850000}%
\pgfsetstrokecolor{currentstroke}%
\pgfsetdash{}{0pt}%
\pgfpathmoveto{\pgfqpoint{3.279906in}{0.417642in}}%
\pgfpathlineto{\pgfqpoint{3.279906in}{2.472642in}}%
\pgfusepath{stroke}%
\end{pgfscope}%
\begin{pgfscope}%
\pgfsetbuttcap%
\pgfsetroundjoin%
\definecolor{currentfill}{rgb}{0.000000,0.000000,0.000000}%
\pgfsetfillcolor{currentfill}%
\pgfsetlinewidth{0.602250pt}%
\definecolor{currentstroke}{rgb}{0.000000,0.000000,0.000000}%
\pgfsetstrokecolor{currentstroke}%
\pgfsetdash{}{0pt}%
\pgfsys@defobject{currentmarker}{\pgfqpoint{0.000000in}{-0.027778in}}{\pgfqpoint{0.000000in}{0.000000in}}{%
\pgfpathmoveto{\pgfqpoint{0.000000in}{0.000000in}}%
\pgfpathlineto{\pgfqpoint{0.000000in}{-0.027778in}}%
\pgfusepath{stroke,fill}%
}%
\begin{pgfscope}%
\pgfsys@transformshift{3.279906in}{0.417642in}%
\pgfsys@useobject{currentmarker}{}%
\end{pgfscope}%
\end{pgfscope}%
\begin{pgfscope}%
\pgfpathrectangle{\pgfqpoint{0.589510in}{0.417642in}}{\pgfqpoint{3.437062in}{2.055000in}}%
\pgfusepath{clip}%
\pgfsetrectcap%
\pgfsetroundjoin%
\pgfsetlinewidth{0.803000pt}%
\definecolor{currentstroke}{rgb}{0.850000,0.850000,0.850000}%
\pgfsetstrokecolor{currentstroke}%
\pgfsetdash{}{0pt}%
\pgfpathmoveto{\pgfqpoint{3.428654in}{0.417642in}}%
\pgfpathlineto{\pgfqpoint{3.428654in}{2.472642in}}%
\pgfusepath{stroke}%
\end{pgfscope}%
\begin{pgfscope}%
\pgfsetbuttcap%
\pgfsetroundjoin%
\definecolor{currentfill}{rgb}{0.000000,0.000000,0.000000}%
\pgfsetfillcolor{currentfill}%
\pgfsetlinewidth{0.602250pt}%
\definecolor{currentstroke}{rgb}{0.000000,0.000000,0.000000}%
\pgfsetstrokecolor{currentstroke}%
\pgfsetdash{}{0pt}%
\pgfsys@defobject{currentmarker}{\pgfqpoint{0.000000in}{-0.027778in}}{\pgfqpoint{0.000000in}{0.000000in}}{%
\pgfpathmoveto{\pgfqpoint{0.000000in}{0.000000in}}%
\pgfpathlineto{\pgfqpoint{0.000000in}{-0.027778in}}%
\pgfusepath{stroke,fill}%
}%
\begin{pgfscope}%
\pgfsys@transformshift{3.428654in}{0.417642in}%
\pgfsys@useobject{currentmarker}{}%
\end{pgfscope}%
\end{pgfscope}%
\begin{pgfscope}%
\pgfpathrectangle{\pgfqpoint{0.589510in}{0.417642in}}{\pgfqpoint{3.437062in}{2.055000in}}%
\pgfusepath{clip}%
\pgfsetrectcap%
\pgfsetroundjoin%
\pgfsetlinewidth{0.803000pt}%
\definecolor{currentstroke}{rgb}{0.850000,0.850000,0.850000}%
\pgfsetstrokecolor{currentstroke}%
\pgfsetdash{}{0pt}%
\pgfpathmoveto{\pgfqpoint{3.534193in}{0.417642in}}%
\pgfpathlineto{\pgfqpoint{3.534193in}{2.472642in}}%
\pgfusepath{stroke}%
\end{pgfscope}%
\begin{pgfscope}%
\pgfsetbuttcap%
\pgfsetroundjoin%
\definecolor{currentfill}{rgb}{0.000000,0.000000,0.000000}%
\pgfsetfillcolor{currentfill}%
\pgfsetlinewidth{0.602250pt}%
\definecolor{currentstroke}{rgb}{0.000000,0.000000,0.000000}%
\pgfsetstrokecolor{currentstroke}%
\pgfsetdash{}{0pt}%
\pgfsys@defobject{currentmarker}{\pgfqpoint{0.000000in}{-0.027778in}}{\pgfqpoint{0.000000in}{0.000000in}}{%
\pgfpathmoveto{\pgfqpoint{0.000000in}{0.000000in}}%
\pgfpathlineto{\pgfqpoint{0.000000in}{-0.027778in}}%
\pgfusepath{stroke,fill}%
}%
\begin{pgfscope}%
\pgfsys@transformshift{3.534193in}{0.417642in}%
\pgfsys@useobject{currentmarker}{}%
\end{pgfscope}%
\end{pgfscope}%
\begin{pgfscope}%
\pgfpathrectangle{\pgfqpoint{0.589510in}{0.417642in}}{\pgfqpoint{3.437062in}{2.055000in}}%
\pgfusepath{clip}%
\pgfsetrectcap%
\pgfsetroundjoin%
\pgfsetlinewidth{0.803000pt}%
\definecolor{currentstroke}{rgb}{0.850000,0.850000,0.850000}%
\pgfsetstrokecolor{currentstroke}%
\pgfsetdash{}{0pt}%
\pgfpathmoveto{\pgfqpoint{3.616055in}{0.417642in}}%
\pgfpathlineto{\pgfqpoint{3.616055in}{2.472642in}}%
\pgfusepath{stroke}%
\end{pgfscope}%
\begin{pgfscope}%
\pgfsetbuttcap%
\pgfsetroundjoin%
\definecolor{currentfill}{rgb}{0.000000,0.000000,0.000000}%
\pgfsetfillcolor{currentfill}%
\pgfsetlinewidth{0.602250pt}%
\definecolor{currentstroke}{rgb}{0.000000,0.000000,0.000000}%
\pgfsetstrokecolor{currentstroke}%
\pgfsetdash{}{0pt}%
\pgfsys@defobject{currentmarker}{\pgfqpoint{0.000000in}{-0.027778in}}{\pgfqpoint{0.000000in}{0.000000in}}{%
\pgfpathmoveto{\pgfqpoint{0.000000in}{0.000000in}}%
\pgfpathlineto{\pgfqpoint{0.000000in}{-0.027778in}}%
\pgfusepath{stroke,fill}%
}%
\begin{pgfscope}%
\pgfsys@transformshift{3.616055in}{0.417642in}%
\pgfsys@useobject{currentmarker}{}%
\end{pgfscope}%
\end{pgfscope}%
\begin{pgfscope}%
\pgfpathrectangle{\pgfqpoint{0.589510in}{0.417642in}}{\pgfqpoint{3.437062in}{2.055000in}}%
\pgfusepath{clip}%
\pgfsetrectcap%
\pgfsetroundjoin%
\pgfsetlinewidth{0.803000pt}%
\definecolor{currentstroke}{rgb}{0.850000,0.850000,0.850000}%
\pgfsetstrokecolor{currentstroke}%
\pgfsetdash{}{0pt}%
\pgfpathmoveto{\pgfqpoint{3.682941in}{0.417642in}}%
\pgfpathlineto{\pgfqpoint{3.682941in}{2.472642in}}%
\pgfusepath{stroke}%
\end{pgfscope}%
\begin{pgfscope}%
\pgfsetbuttcap%
\pgfsetroundjoin%
\definecolor{currentfill}{rgb}{0.000000,0.000000,0.000000}%
\pgfsetfillcolor{currentfill}%
\pgfsetlinewidth{0.602250pt}%
\definecolor{currentstroke}{rgb}{0.000000,0.000000,0.000000}%
\pgfsetstrokecolor{currentstroke}%
\pgfsetdash{}{0pt}%
\pgfsys@defobject{currentmarker}{\pgfqpoint{0.000000in}{-0.027778in}}{\pgfqpoint{0.000000in}{0.000000in}}{%
\pgfpathmoveto{\pgfqpoint{0.000000in}{0.000000in}}%
\pgfpathlineto{\pgfqpoint{0.000000in}{-0.027778in}}%
\pgfusepath{stroke,fill}%
}%
\begin{pgfscope}%
\pgfsys@transformshift{3.682941in}{0.417642in}%
\pgfsys@useobject{currentmarker}{}%
\end{pgfscope}%
\end{pgfscope}%
\begin{pgfscope}%
\pgfpathrectangle{\pgfqpoint{0.589510in}{0.417642in}}{\pgfqpoint{3.437062in}{2.055000in}}%
\pgfusepath{clip}%
\pgfsetrectcap%
\pgfsetroundjoin%
\pgfsetlinewidth{0.803000pt}%
\definecolor{currentstroke}{rgb}{0.850000,0.850000,0.850000}%
\pgfsetstrokecolor{currentstroke}%
\pgfsetdash{}{0pt}%
\pgfpathmoveto{\pgfqpoint{3.739492in}{0.417642in}}%
\pgfpathlineto{\pgfqpoint{3.739492in}{2.472642in}}%
\pgfusepath{stroke}%
\end{pgfscope}%
\begin{pgfscope}%
\pgfsetbuttcap%
\pgfsetroundjoin%
\definecolor{currentfill}{rgb}{0.000000,0.000000,0.000000}%
\pgfsetfillcolor{currentfill}%
\pgfsetlinewidth{0.602250pt}%
\definecolor{currentstroke}{rgb}{0.000000,0.000000,0.000000}%
\pgfsetstrokecolor{currentstroke}%
\pgfsetdash{}{0pt}%
\pgfsys@defobject{currentmarker}{\pgfqpoint{0.000000in}{-0.027778in}}{\pgfqpoint{0.000000in}{0.000000in}}{%
\pgfpathmoveto{\pgfqpoint{0.000000in}{0.000000in}}%
\pgfpathlineto{\pgfqpoint{0.000000in}{-0.027778in}}%
\pgfusepath{stroke,fill}%
}%
\begin{pgfscope}%
\pgfsys@transformshift{3.739492in}{0.417642in}%
\pgfsys@useobject{currentmarker}{}%
\end{pgfscope}%
\end{pgfscope}%
\begin{pgfscope}%
\pgfpathrectangle{\pgfqpoint{0.589510in}{0.417642in}}{\pgfqpoint{3.437062in}{2.055000in}}%
\pgfusepath{clip}%
\pgfsetrectcap%
\pgfsetroundjoin%
\pgfsetlinewidth{0.803000pt}%
\definecolor{currentstroke}{rgb}{0.850000,0.850000,0.850000}%
\pgfsetstrokecolor{currentstroke}%
\pgfsetdash{}{0pt}%
\pgfpathmoveto{\pgfqpoint{3.788479in}{0.417642in}}%
\pgfpathlineto{\pgfqpoint{3.788479in}{2.472642in}}%
\pgfusepath{stroke}%
\end{pgfscope}%
\begin{pgfscope}%
\pgfsetbuttcap%
\pgfsetroundjoin%
\definecolor{currentfill}{rgb}{0.000000,0.000000,0.000000}%
\pgfsetfillcolor{currentfill}%
\pgfsetlinewidth{0.602250pt}%
\definecolor{currentstroke}{rgb}{0.000000,0.000000,0.000000}%
\pgfsetstrokecolor{currentstroke}%
\pgfsetdash{}{0pt}%
\pgfsys@defobject{currentmarker}{\pgfqpoint{0.000000in}{-0.027778in}}{\pgfqpoint{0.000000in}{0.000000in}}{%
\pgfpathmoveto{\pgfqpoint{0.000000in}{0.000000in}}%
\pgfpathlineto{\pgfqpoint{0.000000in}{-0.027778in}}%
\pgfusepath{stroke,fill}%
}%
\begin{pgfscope}%
\pgfsys@transformshift{3.788479in}{0.417642in}%
\pgfsys@useobject{currentmarker}{}%
\end{pgfscope}%
\end{pgfscope}%
\begin{pgfscope}%
\pgfpathrectangle{\pgfqpoint{0.589510in}{0.417642in}}{\pgfqpoint{3.437062in}{2.055000in}}%
\pgfusepath{clip}%
\pgfsetrectcap%
\pgfsetroundjoin%
\pgfsetlinewidth{0.803000pt}%
\definecolor{currentstroke}{rgb}{0.850000,0.850000,0.850000}%
\pgfsetstrokecolor{currentstroke}%
\pgfsetdash{}{0pt}%
\pgfpathmoveto{\pgfqpoint{3.831689in}{0.417642in}}%
\pgfpathlineto{\pgfqpoint{3.831689in}{2.472642in}}%
\pgfusepath{stroke}%
\end{pgfscope}%
\begin{pgfscope}%
\pgfsetbuttcap%
\pgfsetroundjoin%
\definecolor{currentfill}{rgb}{0.000000,0.000000,0.000000}%
\pgfsetfillcolor{currentfill}%
\pgfsetlinewidth{0.602250pt}%
\definecolor{currentstroke}{rgb}{0.000000,0.000000,0.000000}%
\pgfsetstrokecolor{currentstroke}%
\pgfsetdash{}{0pt}%
\pgfsys@defobject{currentmarker}{\pgfqpoint{0.000000in}{-0.027778in}}{\pgfqpoint{0.000000in}{0.000000in}}{%
\pgfpathmoveto{\pgfqpoint{0.000000in}{0.000000in}}%
\pgfpathlineto{\pgfqpoint{0.000000in}{-0.027778in}}%
\pgfusepath{stroke,fill}%
}%
\begin{pgfscope}%
\pgfsys@transformshift{3.831689in}{0.417642in}%
\pgfsys@useobject{currentmarker}{}%
\end{pgfscope}%
\end{pgfscope}%
\begin{pgfscope}%
\definecolor{textcolor}{rgb}{0.000000,0.000000,0.000000}%
\pgfsetstrokecolor{textcolor}%
\pgfsetfillcolor{textcolor}%
\pgftext[x=2.308041in,y=0.165003in,,top]{\color{textcolor}{\rmfamily\fontsize{10.000000}{12.000000}\selectfont\catcode`\^=\active\def^{\ifmmode\sp\else\^{}\fi}\catcode`\%=\active\def%{\%}$\tau$ in \unit{\second}}}%
\end{pgfscope}%
\begin{pgfscope}%
\pgfpathrectangle{\pgfqpoint{0.589510in}{0.417642in}}{\pgfqpoint{3.437062in}{2.055000in}}%
\pgfusepath{clip}%
\pgfsetrectcap%
\pgfsetroundjoin%
\pgfsetlinewidth{0.803000pt}%
\definecolor{currentstroke}{rgb}{0.450000,0.450000,0.450000}%
\pgfsetstrokecolor{currentstroke}%
\pgfsetdash{}{0pt}%
\pgfpathmoveto{\pgfqpoint{0.589510in}{0.417642in}}%
\pgfpathlineto{\pgfqpoint{4.026572in}{0.417642in}}%
\pgfusepath{stroke}%
\end{pgfscope}%
\begin{pgfscope}%
\pgfsetbuttcap%
\pgfsetroundjoin%
\definecolor{currentfill}{rgb}{0.000000,0.000000,0.000000}%
\pgfsetfillcolor{currentfill}%
\pgfsetlinewidth{0.803000pt}%
\definecolor{currentstroke}{rgb}{0.000000,0.000000,0.000000}%
\pgfsetstrokecolor{currentstroke}%
\pgfsetdash{}{0pt}%
\pgfsys@defobject{currentmarker}{\pgfqpoint{-0.048611in}{0.000000in}}{\pgfqpoint{-0.000000in}{0.000000in}}{%
\pgfpathmoveto{\pgfqpoint{-0.000000in}{0.000000in}}%
\pgfpathlineto{\pgfqpoint{-0.048611in}{0.000000in}}%
\pgfusepath{stroke,fill}%
}%
\begin{pgfscope}%
\pgfsys@transformshift{0.589510in}{0.417642in}%
\pgfsys@useobject{currentmarker}{}%
\end{pgfscope}%
\end{pgfscope}%
\begin{pgfscope}%
\definecolor{textcolor}{rgb}{0.000000,0.000000,0.000000}%
\pgfsetstrokecolor{textcolor}%
\pgfsetfillcolor{textcolor}%
\pgftext[x=0.236114in, y=0.378489in, left, base]{\color{textcolor}{\rmfamily\fontsize{8.000000}{9.600000}\selectfont\catcode`\^=\active\def^{\ifmmode\sp\else\^{}\fi}\catcode`\%=\active\def%{\%}$\mathdefault{10^{-8}}$}}%
\end{pgfscope}%
\begin{pgfscope}%
\pgfpathrectangle{\pgfqpoint{0.589510in}{0.417642in}}{\pgfqpoint{3.437062in}{2.055000in}}%
\pgfusepath{clip}%
\pgfsetrectcap%
\pgfsetroundjoin%
\pgfsetlinewidth{0.803000pt}%
\definecolor{currentstroke}{rgb}{0.450000,0.450000,0.450000}%
\pgfsetstrokecolor{currentstroke}%
\pgfsetdash{}{0pt}%
\pgfpathmoveto{\pgfqpoint{0.589510in}{1.310720in}}%
\pgfpathlineto{\pgfqpoint{4.026572in}{1.310720in}}%
\pgfusepath{stroke}%
\end{pgfscope}%
\begin{pgfscope}%
\pgfsetbuttcap%
\pgfsetroundjoin%
\definecolor{currentfill}{rgb}{0.000000,0.000000,0.000000}%
\pgfsetfillcolor{currentfill}%
\pgfsetlinewidth{0.803000pt}%
\definecolor{currentstroke}{rgb}{0.000000,0.000000,0.000000}%
\pgfsetstrokecolor{currentstroke}%
\pgfsetdash{}{0pt}%
\pgfsys@defobject{currentmarker}{\pgfqpoint{-0.048611in}{0.000000in}}{\pgfqpoint{-0.000000in}{0.000000in}}{%
\pgfpathmoveto{\pgfqpoint{-0.000000in}{0.000000in}}%
\pgfpathlineto{\pgfqpoint{-0.048611in}{0.000000in}}%
\pgfusepath{stroke,fill}%
}%
\begin{pgfscope}%
\pgfsys@transformshift{0.589510in}{1.310720in}%
\pgfsys@useobject{currentmarker}{}%
\end{pgfscope}%
\end{pgfscope}%
\begin{pgfscope}%
\definecolor{textcolor}{rgb}{0.000000,0.000000,0.000000}%
\pgfsetstrokecolor{textcolor}%
\pgfsetfillcolor{textcolor}%
\pgftext[x=0.236114in, y=1.271567in, left, base]{\color{textcolor}{\rmfamily\fontsize{8.000000}{9.600000}\selectfont\catcode`\^=\active\def^{\ifmmode\sp\else\^{}\fi}\catcode`\%=\active\def%{\%}$\mathdefault{10^{-7}}$}}%
\end{pgfscope}%
\begin{pgfscope}%
\pgfpathrectangle{\pgfqpoint{0.589510in}{0.417642in}}{\pgfqpoint{3.437062in}{2.055000in}}%
\pgfusepath{clip}%
\pgfsetrectcap%
\pgfsetroundjoin%
\pgfsetlinewidth{0.803000pt}%
\definecolor{currentstroke}{rgb}{0.450000,0.450000,0.450000}%
\pgfsetstrokecolor{currentstroke}%
\pgfsetdash{}{0pt}%
\pgfpathmoveto{\pgfqpoint{0.589510in}{2.203798in}}%
\pgfpathlineto{\pgfqpoint{4.026572in}{2.203798in}}%
\pgfusepath{stroke}%
\end{pgfscope}%
\begin{pgfscope}%
\pgfsetbuttcap%
\pgfsetroundjoin%
\definecolor{currentfill}{rgb}{0.000000,0.000000,0.000000}%
\pgfsetfillcolor{currentfill}%
\pgfsetlinewidth{0.803000pt}%
\definecolor{currentstroke}{rgb}{0.000000,0.000000,0.000000}%
\pgfsetstrokecolor{currentstroke}%
\pgfsetdash{}{0pt}%
\pgfsys@defobject{currentmarker}{\pgfqpoint{-0.048611in}{0.000000in}}{\pgfqpoint{-0.000000in}{0.000000in}}{%
\pgfpathmoveto{\pgfqpoint{-0.000000in}{0.000000in}}%
\pgfpathlineto{\pgfqpoint{-0.048611in}{0.000000in}}%
\pgfusepath{stroke,fill}%
}%
\begin{pgfscope}%
\pgfsys@transformshift{0.589510in}{2.203798in}%
\pgfsys@useobject{currentmarker}{}%
\end{pgfscope}%
\end{pgfscope}%
\begin{pgfscope}%
\definecolor{textcolor}{rgb}{0.000000,0.000000,0.000000}%
\pgfsetstrokecolor{textcolor}%
\pgfsetfillcolor{textcolor}%
\pgftext[x=0.236114in, y=2.164645in, left, base]{\color{textcolor}{\rmfamily\fontsize{8.000000}{9.600000}\selectfont\catcode`\^=\active\def^{\ifmmode\sp\else\^{}\fi}\catcode`\%=\active\def%{\%}$\mathdefault{10^{-6}}$}}%
\end{pgfscope}%
\begin{pgfscope}%
\pgfpathrectangle{\pgfqpoint{0.589510in}{0.417642in}}{\pgfqpoint{3.437062in}{2.055000in}}%
\pgfusepath{clip}%
\pgfsetrectcap%
\pgfsetroundjoin%
\pgfsetlinewidth{0.803000pt}%
\definecolor{currentstroke}{rgb}{0.850000,0.850000,0.850000}%
\pgfsetstrokecolor{currentstroke}%
\pgfsetdash{}{0pt}%
\pgfpathmoveto{\pgfqpoint{0.589510in}{0.686485in}}%
\pgfpathlineto{\pgfqpoint{4.026572in}{0.686485in}}%
\pgfusepath{stroke}%
\end{pgfscope}%
\begin{pgfscope}%
\pgfsetbuttcap%
\pgfsetroundjoin%
\definecolor{currentfill}{rgb}{0.000000,0.000000,0.000000}%
\pgfsetfillcolor{currentfill}%
\pgfsetlinewidth{0.602250pt}%
\definecolor{currentstroke}{rgb}{0.000000,0.000000,0.000000}%
\pgfsetstrokecolor{currentstroke}%
\pgfsetdash{}{0pt}%
\pgfsys@defobject{currentmarker}{\pgfqpoint{-0.027778in}{0.000000in}}{\pgfqpoint{-0.000000in}{0.000000in}}{%
\pgfpathmoveto{\pgfqpoint{-0.000000in}{0.000000in}}%
\pgfpathlineto{\pgfqpoint{-0.027778in}{0.000000in}}%
\pgfusepath{stroke,fill}%
}%
\begin{pgfscope}%
\pgfsys@transformshift{0.589510in}{0.686485in}%
\pgfsys@useobject{currentmarker}{}%
\end{pgfscope}%
\end{pgfscope}%
\begin{pgfscope}%
\pgfpathrectangle{\pgfqpoint{0.589510in}{0.417642in}}{\pgfqpoint{3.437062in}{2.055000in}}%
\pgfusepath{clip}%
\pgfsetrectcap%
\pgfsetroundjoin%
\pgfsetlinewidth{0.803000pt}%
\definecolor{currentstroke}{rgb}{0.850000,0.850000,0.850000}%
\pgfsetstrokecolor{currentstroke}%
\pgfsetdash{}{0pt}%
\pgfpathmoveto{\pgfqpoint{0.589510in}{0.843749in}}%
\pgfpathlineto{\pgfqpoint{4.026572in}{0.843749in}}%
\pgfusepath{stroke}%
\end{pgfscope}%
\begin{pgfscope}%
\pgfsetbuttcap%
\pgfsetroundjoin%
\definecolor{currentfill}{rgb}{0.000000,0.000000,0.000000}%
\pgfsetfillcolor{currentfill}%
\pgfsetlinewidth{0.602250pt}%
\definecolor{currentstroke}{rgb}{0.000000,0.000000,0.000000}%
\pgfsetstrokecolor{currentstroke}%
\pgfsetdash{}{0pt}%
\pgfsys@defobject{currentmarker}{\pgfqpoint{-0.027778in}{0.000000in}}{\pgfqpoint{-0.000000in}{0.000000in}}{%
\pgfpathmoveto{\pgfqpoint{-0.000000in}{0.000000in}}%
\pgfpathlineto{\pgfqpoint{-0.027778in}{0.000000in}}%
\pgfusepath{stroke,fill}%
}%
\begin{pgfscope}%
\pgfsys@transformshift{0.589510in}{0.843749in}%
\pgfsys@useobject{currentmarker}{}%
\end{pgfscope}%
\end{pgfscope}%
\begin{pgfscope}%
\pgfpathrectangle{\pgfqpoint{0.589510in}{0.417642in}}{\pgfqpoint{3.437062in}{2.055000in}}%
\pgfusepath{clip}%
\pgfsetrectcap%
\pgfsetroundjoin%
\pgfsetlinewidth{0.803000pt}%
\definecolor{currentstroke}{rgb}{0.850000,0.850000,0.850000}%
\pgfsetstrokecolor{currentstroke}%
\pgfsetdash{}{0pt}%
\pgfpathmoveto{\pgfqpoint{0.589510in}{0.955329in}}%
\pgfpathlineto{\pgfqpoint{4.026572in}{0.955329in}}%
\pgfusepath{stroke}%
\end{pgfscope}%
\begin{pgfscope}%
\pgfsetbuttcap%
\pgfsetroundjoin%
\definecolor{currentfill}{rgb}{0.000000,0.000000,0.000000}%
\pgfsetfillcolor{currentfill}%
\pgfsetlinewidth{0.602250pt}%
\definecolor{currentstroke}{rgb}{0.000000,0.000000,0.000000}%
\pgfsetstrokecolor{currentstroke}%
\pgfsetdash{}{0pt}%
\pgfsys@defobject{currentmarker}{\pgfqpoint{-0.027778in}{0.000000in}}{\pgfqpoint{-0.000000in}{0.000000in}}{%
\pgfpathmoveto{\pgfqpoint{-0.000000in}{0.000000in}}%
\pgfpathlineto{\pgfqpoint{-0.027778in}{0.000000in}}%
\pgfusepath{stroke,fill}%
}%
\begin{pgfscope}%
\pgfsys@transformshift{0.589510in}{0.955329in}%
\pgfsys@useobject{currentmarker}{}%
\end{pgfscope}%
\end{pgfscope}%
\begin{pgfscope}%
\pgfpathrectangle{\pgfqpoint{0.589510in}{0.417642in}}{\pgfqpoint{3.437062in}{2.055000in}}%
\pgfusepath{clip}%
\pgfsetrectcap%
\pgfsetroundjoin%
\pgfsetlinewidth{0.803000pt}%
\definecolor{currentstroke}{rgb}{0.850000,0.850000,0.850000}%
\pgfsetstrokecolor{currentstroke}%
\pgfsetdash{}{0pt}%
\pgfpathmoveto{\pgfqpoint{0.589510in}{1.041877in}}%
\pgfpathlineto{\pgfqpoint{4.026572in}{1.041877in}}%
\pgfusepath{stroke}%
\end{pgfscope}%
\begin{pgfscope}%
\pgfsetbuttcap%
\pgfsetroundjoin%
\definecolor{currentfill}{rgb}{0.000000,0.000000,0.000000}%
\pgfsetfillcolor{currentfill}%
\pgfsetlinewidth{0.602250pt}%
\definecolor{currentstroke}{rgb}{0.000000,0.000000,0.000000}%
\pgfsetstrokecolor{currentstroke}%
\pgfsetdash{}{0pt}%
\pgfsys@defobject{currentmarker}{\pgfqpoint{-0.027778in}{0.000000in}}{\pgfqpoint{-0.000000in}{0.000000in}}{%
\pgfpathmoveto{\pgfqpoint{-0.000000in}{0.000000in}}%
\pgfpathlineto{\pgfqpoint{-0.027778in}{0.000000in}}%
\pgfusepath{stroke,fill}%
}%
\begin{pgfscope}%
\pgfsys@transformshift{0.589510in}{1.041877in}%
\pgfsys@useobject{currentmarker}{}%
\end{pgfscope}%
\end{pgfscope}%
\begin{pgfscope}%
\pgfpathrectangle{\pgfqpoint{0.589510in}{0.417642in}}{\pgfqpoint{3.437062in}{2.055000in}}%
\pgfusepath{clip}%
\pgfsetrectcap%
\pgfsetroundjoin%
\pgfsetlinewidth{0.803000pt}%
\definecolor{currentstroke}{rgb}{0.850000,0.850000,0.850000}%
\pgfsetstrokecolor{currentstroke}%
\pgfsetdash{}{0pt}%
\pgfpathmoveto{\pgfqpoint{0.589510in}{1.112592in}}%
\pgfpathlineto{\pgfqpoint{4.026572in}{1.112592in}}%
\pgfusepath{stroke}%
\end{pgfscope}%
\begin{pgfscope}%
\pgfsetbuttcap%
\pgfsetroundjoin%
\definecolor{currentfill}{rgb}{0.000000,0.000000,0.000000}%
\pgfsetfillcolor{currentfill}%
\pgfsetlinewidth{0.602250pt}%
\definecolor{currentstroke}{rgb}{0.000000,0.000000,0.000000}%
\pgfsetstrokecolor{currentstroke}%
\pgfsetdash{}{0pt}%
\pgfsys@defobject{currentmarker}{\pgfqpoint{-0.027778in}{0.000000in}}{\pgfqpoint{-0.000000in}{0.000000in}}{%
\pgfpathmoveto{\pgfqpoint{-0.000000in}{0.000000in}}%
\pgfpathlineto{\pgfqpoint{-0.027778in}{0.000000in}}%
\pgfusepath{stroke,fill}%
}%
\begin{pgfscope}%
\pgfsys@transformshift{0.589510in}{1.112592in}%
\pgfsys@useobject{currentmarker}{}%
\end{pgfscope}%
\end{pgfscope}%
\begin{pgfscope}%
\pgfpathrectangle{\pgfqpoint{0.589510in}{0.417642in}}{\pgfqpoint{3.437062in}{2.055000in}}%
\pgfusepath{clip}%
\pgfsetrectcap%
\pgfsetroundjoin%
\pgfsetlinewidth{0.803000pt}%
\definecolor{currentstroke}{rgb}{0.850000,0.850000,0.850000}%
\pgfsetstrokecolor{currentstroke}%
\pgfsetdash{}{0pt}%
\pgfpathmoveto{\pgfqpoint{0.589510in}{1.172381in}}%
\pgfpathlineto{\pgfqpoint{4.026572in}{1.172381in}}%
\pgfusepath{stroke}%
\end{pgfscope}%
\begin{pgfscope}%
\pgfsetbuttcap%
\pgfsetroundjoin%
\definecolor{currentfill}{rgb}{0.000000,0.000000,0.000000}%
\pgfsetfillcolor{currentfill}%
\pgfsetlinewidth{0.602250pt}%
\definecolor{currentstroke}{rgb}{0.000000,0.000000,0.000000}%
\pgfsetstrokecolor{currentstroke}%
\pgfsetdash{}{0pt}%
\pgfsys@defobject{currentmarker}{\pgfqpoint{-0.027778in}{0.000000in}}{\pgfqpoint{-0.000000in}{0.000000in}}{%
\pgfpathmoveto{\pgfqpoint{-0.000000in}{0.000000in}}%
\pgfpathlineto{\pgfqpoint{-0.027778in}{0.000000in}}%
\pgfusepath{stroke,fill}%
}%
\begin{pgfscope}%
\pgfsys@transformshift{0.589510in}{1.172381in}%
\pgfsys@useobject{currentmarker}{}%
\end{pgfscope}%
\end{pgfscope}%
\begin{pgfscope}%
\pgfpathrectangle{\pgfqpoint{0.589510in}{0.417642in}}{\pgfqpoint{3.437062in}{2.055000in}}%
\pgfusepath{clip}%
\pgfsetrectcap%
\pgfsetroundjoin%
\pgfsetlinewidth{0.803000pt}%
\definecolor{currentstroke}{rgb}{0.850000,0.850000,0.850000}%
\pgfsetstrokecolor{currentstroke}%
\pgfsetdash{}{0pt}%
\pgfpathmoveto{\pgfqpoint{0.589510in}{1.224172in}}%
\pgfpathlineto{\pgfqpoint{4.026572in}{1.224172in}}%
\pgfusepath{stroke}%
\end{pgfscope}%
\begin{pgfscope}%
\pgfsetbuttcap%
\pgfsetroundjoin%
\definecolor{currentfill}{rgb}{0.000000,0.000000,0.000000}%
\pgfsetfillcolor{currentfill}%
\pgfsetlinewidth{0.602250pt}%
\definecolor{currentstroke}{rgb}{0.000000,0.000000,0.000000}%
\pgfsetstrokecolor{currentstroke}%
\pgfsetdash{}{0pt}%
\pgfsys@defobject{currentmarker}{\pgfqpoint{-0.027778in}{0.000000in}}{\pgfqpoint{-0.000000in}{0.000000in}}{%
\pgfpathmoveto{\pgfqpoint{-0.000000in}{0.000000in}}%
\pgfpathlineto{\pgfqpoint{-0.027778in}{0.000000in}}%
\pgfusepath{stroke,fill}%
}%
\begin{pgfscope}%
\pgfsys@transformshift{0.589510in}{1.224172in}%
\pgfsys@useobject{currentmarker}{}%
\end{pgfscope}%
\end{pgfscope}%
\begin{pgfscope}%
\pgfpathrectangle{\pgfqpoint{0.589510in}{0.417642in}}{\pgfqpoint{3.437062in}{2.055000in}}%
\pgfusepath{clip}%
\pgfsetrectcap%
\pgfsetroundjoin%
\pgfsetlinewidth{0.803000pt}%
\definecolor{currentstroke}{rgb}{0.850000,0.850000,0.850000}%
\pgfsetstrokecolor{currentstroke}%
\pgfsetdash{}{0pt}%
\pgfpathmoveto{\pgfqpoint{0.589510in}{1.269855in}}%
\pgfpathlineto{\pgfqpoint{4.026572in}{1.269855in}}%
\pgfusepath{stroke}%
\end{pgfscope}%
\begin{pgfscope}%
\pgfsetbuttcap%
\pgfsetroundjoin%
\definecolor{currentfill}{rgb}{0.000000,0.000000,0.000000}%
\pgfsetfillcolor{currentfill}%
\pgfsetlinewidth{0.602250pt}%
\definecolor{currentstroke}{rgb}{0.000000,0.000000,0.000000}%
\pgfsetstrokecolor{currentstroke}%
\pgfsetdash{}{0pt}%
\pgfsys@defobject{currentmarker}{\pgfqpoint{-0.027778in}{0.000000in}}{\pgfqpoint{-0.000000in}{0.000000in}}{%
\pgfpathmoveto{\pgfqpoint{-0.000000in}{0.000000in}}%
\pgfpathlineto{\pgfqpoint{-0.027778in}{0.000000in}}%
\pgfusepath{stroke,fill}%
}%
\begin{pgfscope}%
\pgfsys@transformshift{0.589510in}{1.269855in}%
\pgfsys@useobject{currentmarker}{}%
\end{pgfscope}%
\end{pgfscope}%
\begin{pgfscope}%
\pgfpathrectangle{\pgfqpoint{0.589510in}{0.417642in}}{\pgfqpoint{3.437062in}{2.055000in}}%
\pgfusepath{clip}%
\pgfsetrectcap%
\pgfsetroundjoin%
\pgfsetlinewidth{0.803000pt}%
\definecolor{currentstroke}{rgb}{0.850000,0.850000,0.850000}%
\pgfsetstrokecolor{currentstroke}%
\pgfsetdash{}{0pt}%
\pgfpathmoveto{\pgfqpoint{0.589510in}{1.579563in}}%
\pgfpathlineto{\pgfqpoint{4.026572in}{1.579563in}}%
\pgfusepath{stroke}%
\end{pgfscope}%
\begin{pgfscope}%
\pgfsetbuttcap%
\pgfsetroundjoin%
\definecolor{currentfill}{rgb}{0.000000,0.000000,0.000000}%
\pgfsetfillcolor{currentfill}%
\pgfsetlinewidth{0.602250pt}%
\definecolor{currentstroke}{rgb}{0.000000,0.000000,0.000000}%
\pgfsetstrokecolor{currentstroke}%
\pgfsetdash{}{0pt}%
\pgfsys@defobject{currentmarker}{\pgfqpoint{-0.027778in}{0.000000in}}{\pgfqpoint{-0.000000in}{0.000000in}}{%
\pgfpathmoveto{\pgfqpoint{-0.000000in}{0.000000in}}%
\pgfpathlineto{\pgfqpoint{-0.027778in}{0.000000in}}%
\pgfusepath{stroke,fill}%
}%
\begin{pgfscope}%
\pgfsys@transformshift{0.589510in}{1.579563in}%
\pgfsys@useobject{currentmarker}{}%
\end{pgfscope}%
\end{pgfscope}%
\begin{pgfscope}%
\pgfpathrectangle{\pgfqpoint{0.589510in}{0.417642in}}{\pgfqpoint{3.437062in}{2.055000in}}%
\pgfusepath{clip}%
\pgfsetrectcap%
\pgfsetroundjoin%
\pgfsetlinewidth{0.803000pt}%
\definecolor{currentstroke}{rgb}{0.850000,0.850000,0.850000}%
\pgfsetstrokecolor{currentstroke}%
\pgfsetdash{}{0pt}%
\pgfpathmoveto{\pgfqpoint{0.589510in}{1.736827in}}%
\pgfpathlineto{\pgfqpoint{4.026572in}{1.736827in}}%
\pgfusepath{stroke}%
\end{pgfscope}%
\begin{pgfscope}%
\pgfsetbuttcap%
\pgfsetroundjoin%
\definecolor{currentfill}{rgb}{0.000000,0.000000,0.000000}%
\pgfsetfillcolor{currentfill}%
\pgfsetlinewidth{0.602250pt}%
\definecolor{currentstroke}{rgb}{0.000000,0.000000,0.000000}%
\pgfsetstrokecolor{currentstroke}%
\pgfsetdash{}{0pt}%
\pgfsys@defobject{currentmarker}{\pgfqpoint{-0.027778in}{0.000000in}}{\pgfqpoint{-0.000000in}{0.000000in}}{%
\pgfpathmoveto{\pgfqpoint{-0.000000in}{0.000000in}}%
\pgfpathlineto{\pgfqpoint{-0.027778in}{0.000000in}}%
\pgfusepath{stroke,fill}%
}%
\begin{pgfscope}%
\pgfsys@transformshift{0.589510in}{1.736827in}%
\pgfsys@useobject{currentmarker}{}%
\end{pgfscope}%
\end{pgfscope}%
\begin{pgfscope}%
\pgfpathrectangle{\pgfqpoint{0.589510in}{0.417642in}}{\pgfqpoint{3.437062in}{2.055000in}}%
\pgfusepath{clip}%
\pgfsetrectcap%
\pgfsetroundjoin%
\pgfsetlinewidth{0.803000pt}%
\definecolor{currentstroke}{rgb}{0.850000,0.850000,0.850000}%
\pgfsetstrokecolor{currentstroke}%
\pgfsetdash{}{0pt}%
\pgfpathmoveto{\pgfqpoint{0.589510in}{1.848407in}}%
\pgfpathlineto{\pgfqpoint{4.026572in}{1.848407in}}%
\pgfusepath{stroke}%
\end{pgfscope}%
\begin{pgfscope}%
\pgfsetbuttcap%
\pgfsetroundjoin%
\definecolor{currentfill}{rgb}{0.000000,0.000000,0.000000}%
\pgfsetfillcolor{currentfill}%
\pgfsetlinewidth{0.602250pt}%
\definecolor{currentstroke}{rgb}{0.000000,0.000000,0.000000}%
\pgfsetstrokecolor{currentstroke}%
\pgfsetdash{}{0pt}%
\pgfsys@defobject{currentmarker}{\pgfqpoint{-0.027778in}{0.000000in}}{\pgfqpoint{-0.000000in}{0.000000in}}{%
\pgfpathmoveto{\pgfqpoint{-0.000000in}{0.000000in}}%
\pgfpathlineto{\pgfqpoint{-0.027778in}{0.000000in}}%
\pgfusepath{stroke,fill}%
}%
\begin{pgfscope}%
\pgfsys@transformshift{0.589510in}{1.848407in}%
\pgfsys@useobject{currentmarker}{}%
\end{pgfscope}%
\end{pgfscope}%
\begin{pgfscope}%
\pgfpathrectangle{\pgfqpoint{0.589510in}{0.417642in}}{\pgfqpoint{3.437062in}{2.055000in}}%
\pgfusepath{clip}%
\pgfsetrectcap%
\pgfsetroundjoin%
\pgfsetlinewidth{0.803000pt}%
\definecolor{currentstroke}{rgb}{0.850000,0.850000,0.850000}%
\pgfsetstrokecolor{currentstroke}%
\pgfsetdash{}{0pt}%
\pgfpathmoveto{\pgfqpoint{0.589510in}{1.934955in}}%
\pgfpathlineto{\pgfqpoint{4.026572in}{1.934955in}}%
\pgfusepath{stroke}%
\end{pgfscope}%
\begin{pgfscope}%
\pgfsetbuttcap%
\pgfsetroundjoin%
\definecolor{currentfill}{rgb}{0.000000,0.000000,0.000000}%
\pgfsetfillcolor{currentfill}%
\pgfsetlinewidth{0.602250pt}%
\definecolor{currentstroke}{rgb}{0.000000,0.000000,0.000000}%
\pgfsetstrokecolor{currentstroke}%
\pgfsetdash{}{0pt}%
\pgfsys@defobject{currentmarker}{\pgfqpoint{-0.027778in}{0.000000in}}{\pgfqpoint{-0.000000in}{0.000000in}}{%
\pgfpathmoveto{\pgfqpoint{-0.000000in}{0.000000in}}%
\pgfpathlineto{\pgfqpoint{-0.027778in}{0.000000in}}%
\pgfusepath{stroke,fill}%
}%
\begin{pgfscope}%
\pgfsys@transformshift{0.589510in}{1.934955in}%
\pgfsys@useobject{currentmarker}{}%
\end{pgfscope}%
\end{pgfscope}%
\begin{pgfscope}%
\pgfpathrectangle{\pgfqpoint{0.589510in}{0.417642in}}{\pgfqpoint{3.437062in}{2.055000in}}%
\pgfusepath{clip}%
\pgfsetrectcap%
\pgfsetroundjoin%
\pgfsetlinewidth{0.803000pt}%
\definecolor{currentstroke}{rgb}{0.850000,0.850000,0.850000}%
\pgfsetstrokecolor{currentstroke}%
\pgfsetdash{}{0pt}%
\pgfpathmoveto{\pgfqpoint{0.589510in}{2.005670in}}%
\pgfpathlineto{\pgfqpoint{4.026572in}{2.005670in}}%
\pgfusepath{stroke}%
\end{pgfscope}%
\begin{pgfscope}%
\pgfsetbuttcap%
\pgfsetroundjoin%
\definecolor{currentfill}{rgb}{0.000000,0.000000,0.000000}%
\pgfsetfillcolor{currentfill}%
\pgfsetlinewidth{0.602250pt}%
\definecolor{currentstroke}{rgb}{0.000000,0.000000,0.000000}%
\pgfsetstrokecolor{currentstroke}%
\pgfsetdash{}{0pt}%
\pgfsys@defobject{currentmarker}{\pgfqpoint{-0.027778in}{0.000000in}}{\pgfqpoint{-0.000000in}{0.000000in}}{%
\pgfpathmoveto{\pgfqpoint{-0.000000in}{0.000000in}}%
\pgfpathlineto{\pgfqpoint{-0.027778in}{0.000000in}}%
\pgfusepath{stroke,fill}%
}%
\begin{pgfscope}%
\pgfsys@transformshift{0.589510in}{2.005670in}%
\pgfsys@useobject{currentmarker}{}%
\end{pgfscope}%
\end{pgfscope}%
\begin{pgfscope}%
\pgfpathrectangle{\pgfqpoint{0.589510in}{0.417642in}}{\pgfqpoint{3.437062in}{2.055000in}}%
\pgfusepath{clip}%
\pgfsetrectcap%
\pgfsetroundjoin%
\pgfsetlinewidth{0.803000pt}%
\definecolor{currentstroke}{rgb}{0.850000,0.850000,0.850000}%
\pgfsetstrokecolor{currentstroke}%
\pgfsetdash{}{0pt}%
\pgfpathmoveto{\pgfqpoint{0.589510in}{2.065459in}}%
\pgfpathlineto{\pgfqpoint{4.026572in}{2.065459in}}%
\pgfusepath{stroke}%
\end{pgfscope}%
\begin{pgfscope}%
\pgfsetbuttcap%
\pgfsetroundjoin%
\definecolor{currentfill}{rgb}{0.000000,0.000000,0.000000}%
\pgfsetfillcolor{currentfill}%
\pgfsetlinewidth{0.602250pt}%
\definecolor{currentstroke}{rgb}{0.000000,0.000000,0.000000}%
\pgfsetstrokecolor{currentstroke}%
\pgfsetdash{}{0pt}%
\pgfsys@defobject{currentmarker}{\pgfqpoint{-0.027778in}{0.000000in}}{\pgfqpoint{-0.000000in}{0.000000in}}{%
\pgfpathmoveto{\pgfqpoint{-0.000000in}{0.000000in}}%
\pgfpathlineto{\pgfqpoint{-0.027778in}{0.000000in}}%
\pgfusepath{stroke,fill}%
}%
\begin{pgfscope}%
\pgfsys@transformshift{0.589510in}{2.065459in}%
\pgfsys@useobject{currentmarker}{}%
\end{pgfscope}%
\end{pgfscope}%
\begin{pgfscope}%
\pgfpathrectangle{\pgfqpoint{0.589510in}{0.417642in}}{\pgfqpoint{3.437062in}{2.055000in}}%
\pgfusepath{clip}%
\pgfsetrectcap%
\pgfsetroundjoin%
\pgfsetlinewidth{0.803000pt}%
\definecolor{currentstroke}{rgb}{0.850000,0.850000,0.850000}%
\pgfsetstrokecolor{currentstroke}%
\pgfsetdash{}{0pt}%
\pgfpathmoveto{\pgfqpoint{0.589510in}{2.117250in}}%
\pgfpathlineto{\pgfqpoint{4.026572in}{2.117250in}}%
\pgfusepath{stroke}%
\end{pgfscope}%
\begin{pgfscope}%
\pgfsetbuttcap%
\pgfsetroundjoin%
\definecolor{currentfill}{rgb}{0.000000,0.000000,0.000000}%
\pgfsetfillcolor{currentfill}%
\pgfsetlinewidth{0.602250pt}%
\definecolor{currentstroke}{rgb}{0.000000,0.000000,0.000000}%
\pgfsetstrokecolor{currentstroke}%
\pgfsetdash{}{0pt}%
\pgfsys@defobject{currentmarker}{\pgfqpoint{-0.027778in}{0.000000in}}{\pgfqpoint{-0.000000in}{0.000000in}}{%
\pgfpathmoveto{\pgfqpoint{-0.000000in}{0.000000in}}%
\pgfpathlineto{\pgfqpoint{-0.027778in}{0.000000in}}%
\pgfusepath{stroke,fill}%
}%
\begin{pgfscope}%
\pgfsys@transformshift{0.589510in}{2.117250in}%
\pgfsys@useobject{currentmarker}{}%
\end{pgfscope}%
\end{pgfscope}%
\begin{pgfscope}%
\pgfpathrectangle{\pgfqpoint{0.589510in}{0.417642in}}{\pgfqpoint{3.437062in}{2.055000in}}%
\pgfusepath{clip}%
\pgfsetrectcap%
\pgfsetroundjoin%
\pgfsetlinewidth{0.803000pt}%
\definecolor{currentstroke}{rgb}{0.850000,0.850000,0.850000}%
\pgfsetstrokecolor{currentstroke}%
\pgfsetdash{}{0pt}%
\pgfpathmoveto{\pgfqpoint{0.589510in}{2.162933in}}%
\pgfpathlineto{\pgfqpoint{4.026572in}{2.162933in}}%
\pgfusepath{stroke}%
\end{pgfscope}%
\begin{pgfscope}%
\pgfsetbuttcap%
\pgfsetroundjoin%
\definecolor{currentfill}{rgb}{0.000000,0.000000,0.000000}%
\pgfsetfillcolor{currentfill}%
\pgfsetlinewidth{0.602250pt}%
\definecolor{currentstroke}{rgb}{0.000000,0.000000,0.000000}%
\pgfsetstrokecolor{currentstroke}%
\pgfsetdash{}{0pt}%
\pgfsys@defobject{currentmarker}{\pgfqpoint{-0.027778in}{0.000000in}}{\pgfqpoint{-0.000000in}{0.000000in}}{%
\pgfpathmoveto{\pgfqpoint{-0.000000in}{0.000000in}}%
\pgfpathlineto{\pgfqpoint{-0.027778in}{0.000000in}}%
\pgfusepath{stroke,fill}%
}%
\begin{pgfscope}%
\pgfsys@transformshift{0.589510in}{2.162933in}%
\pgfsys@useobject{currentmarker}{}%
\end{pgfscope}%
\end{pgfscope}%
\begin{pgfscope}%
\pgfpathrectangle{\pgfqpoint{0.589510in}{0.417642in}}{\pgfqpoint{3.437062in}{2.055000in}}%
\pgfusepath{clip}%
\pgfsetrectcap%
\pgfsetroundjoin%
\pgfsetlinewidth{0.803000pt}%
\definecolor{currentstroke}{rgb}{0.850000,0.850000,0.850000}%
\pgfsetstrokecolor{currentstroke}%
\pgfsetdash{}{0pt}%
\pgfpathmoveto{\pgfqpoint{0.589510in}{2.472642in}}%
\pgfpathlineto{\pgfqpoint{4.026572in}{2.472642in}}%
\pgfusepath{stroke}%
\end{pgfscope}%
\begin{pgfscope}%
\pgfsetbuttcap%
\pgfsetroundjoin%
\definecolor{currentfill}{rgb}{0.000000,0.000000,0.000000}%
\pgfsetfillcolor{currentfill}%
\pgfsetlinewidth{0.602250pt}%
\definecolor{currentstroke}{rgb}{0.000000,0.000000,0.000000}%
\pgfsetstrokecolor{currentstroke}%
\pgfsetdash{}{0pt}%
\pgfsys@defobject{currentmarker}{\pgfqpoint{-0.027778in}{0.000000in}}{\pgfqpoint{-0.000000in}{0.000000in}}{%
\pgfpathmoveto{\pgfqpoint{-0.000000in}{0.000000in}}%
\pgfpathlineto{\pgfqpoint{-0.027778in}{0.000000in}}%
\pgfusepath{stroke,fill}%
}%
\begin{pgfscope}%
\pgfsys@transformshift{0.589510in}{2.472642in}%
\pgfsys@useobject{currentmarker}{}%
\end{pgfscope}%
\end{pgfscope}%
\begin{pgfscope}%
\definecolor{textcolor}{rgb}{0.000000,0.000000,0.000000}%
\pgfsetstrokecolor{textcolor}%
\pgfsetfillcolor{textcolor}%
\pgftext[x=0.180559in,y=1.445142in,,bottom,rotate=90.000000]{\color{textcolor}{\rmfamily\fontsize{10.000000}{12.000000}\selectfont\catcode`\^=\active\def^{\ifmmode\sp\else\^{}\fi}\catcode`\%=\active\def%{\%}ADEV $\sigma_A(\tau)$ in \unit{\V}}}%
\end{pgfscope}%
\begin{pgfscope}%
\pgfpathrectangle{\pgfqpoint{0.589510in}{0.417642in}}{\pgfqpoint{3.437062in}{2.055000in}}%
\pgfusepath{clip}%
\pgfsetrectcap%
\pgfsetroundjoin%
\pgfsetlinewidth{1.505625pt}%
\definecolor{currentstroke}{rgb}{0.121569,0.466667,0.705882}%
\pgfsetstrokecolor{currentstroke}%
\pgfsetdash{}{0pt}%
\pgfpathmoveto{\pgfqpoint{1.000026in}{2.287223in}}%
\pgfpathlineto{\pgfqpoint{1.254313in}{2.151260in}}%
\pgfpathlineto{\pgfqpoint{1.403061in}{2.071499in}}%
\pgfpathlineto{\pgfqpoint{1.508600in}{2.014735in}}%
\pgfpathlineto{\pgfqpoint{1.590462in}{1.970918in}}%
\pgfpathlineto{\pgfqpoint{1.657348in}{1.935132in}}%
\pgfpathlineto{\pgfqpoint{1.713899in}{1.905115in}}%
\pgfpathlineto{\pgfqpoint{1.806096in}{1.856144in}}%
\pgfpathlineto{\pgfqpoint{1.879714in}{1.817053in}}%
\pgfpathlineto{\pgfqpoint{1.940999in}{1.784421in}}%
\pgfpathlineto{\pgfqpoint{2.017173in}{1.743765in}}%
\pgfpathlineto{\pgfqpoint{2.080218in}{1.710367in}}%
\pgfpathlineto{\pgfqpoint{2.150308in}{1.673143in}}%
\pgfpathlineto{\pgfqpoint{2.209131in}{1.641765in}}%
\pgfpathlineto{\pgfqpoint{2.282749in}{1.602351in}}%
\pgfpathlineto{\pgfqpoint{2.353322in}{1.564824in}}%
\pgfpathlineto{\pgfqpoint{2.420208in}{1.529336in}}%
\pgfpathlineto{\pgfqpoint{2.489633in}{1.492938in}}%
\pgfpathlineto{\pgfqpoint{2.558622in}{1.457048in}}%
\pgfpathlineto{\pgfqpoint{2.629849in}{1.419796in}}%
\pgfpathlineto{\pgfqpoint{2.696735in}{1.384982in}}%
\pgfpathlineto{\pgfqpoint{2.768386in}{1.347457in}}%
\pgfpathlineto{\pgfqpoint{2.835765in}{1.311623in}}%
\pgfpathlineto{\pgfqpoint{2.905105in}{1.274788in}}%
\pgfpathlineto{\pgfqpoint{2.973686in}{1.238044in}}%
\pgfpathlineto{\pgfqpoint{3.042819in}{1.202044in}}%
\pgfpathlineto{\pgfqpoint{3.111567in}{1.167286in}}%
\pgfpathlineto{\pgfqpoint{3.181152in}{1.132867in}}%
\pgfpathlineto{\pgfqpoint{3.250114in}{1.098202in}}%
\pgfpathlineto{\pgfqpoint{3.318852in}{1.061766in}}%
\pgfpathlineto{\pgfqpoint{3.387821in}{1.024502in}}%
\pgfpathlineto{\pgfqpoint{3.456435in}{0.987765in}}%
\pgfpathlineto{\pgfqpoint{3.525657in}{0.951955in}}%
\pgfpathlineto{\pgfqpoint{3.594602in}{0.916492in}}%
\pgfpathlineto{\pgfqpoint{3.663350in}{0.879091in}}%
\pgfpathlineto{\pgfqpoint{3.732508in}{0.842603in}}%
\pgfpathlineto{\pgfqpoint{3.801454in}{0.804600in}}%
\pgfpathlineto{\pgfqpoint{3.870342in}{0.764922in}}%
\pgfusepath{stroke}%
\end{pgfscope}%
\begin{pgfscope}%
\pgfpathrectangle{\pgfqpoint{0.589510in}{0.417642in}}{\pgfqpoint{3.437062in}{2.055000in}}%
\pgfusepath{clip}%
\pgfsetrectcap%
\pgfsetroundjoin%
\pgfsetlinewidth{1.505625pt}%
\definecolor{currentstroke}{rgb}{1.000000,0.498039,0.054902}%
\pgfsetstrokecolor{currentstroke}%
\pgfsetdash{}{0pt}%
\pgfpathmoveto{\pgfqpoint{1.254313in}{2.166320in}}%
\pgfpathlineto{\pgfqpoint{1.508600in}{2.028731in}}%
\pgfpathlineto{\pgfqpoint{1.657348in}{1.947806in}}%
\pgfpathlineto{\pgfqpoint{1.762887in}{1.890151in}}%
\pgfpathlineto{\pgfqpoint{1.844749in}{1.845829in}}%
\pgfpathlineto{\pgfqpoint{1.911635in}{1.809666in}}%
\pgfpathlineto{\pgfqpoint{2.017173in}{1.752616in}}%
\pgfpathlineto{\pgfqpoint{2.060383in}{1.729449in}}%
\pgfpathlineto{\pgfqpoint{2.134001in}{1.689906in}}%
\pgfpathlineto{\pgfqpoint{2.222473in}{1.642287in}}%
\pgfpathlineto{\pgfqpoint{2.293701in}{1.604225in}}%
\pgfpathlineto{\pgfqpoint{2.353322in}{1.572322in}}%
\pgfpathlineto{\pgfqpoint{2.420208in}{1.536581in}}%
\pgfpathlineto{\pgfqpoint{2.489633in}{1.499968in}}%
\pgfpathlineto{\pgfqpoint{2.558622in}{1.463215in}}%
\pgfpathlineto{\pgfqpoint{2.625508in}{1.427677in}}%
\pgfpathlineto{\pgfqpoint{2.696735in}{1.389963in}}%
\pgfpathlineto{\pgfqpoint{2.768386in}{1.351651in}}%
\pgfpathlineto{\pgfqpoint{2.838219in}{1.314149in}}%
\pgfpathlineto{\pgfqpoint{2.905105in}{1.277991in}}%
\pgfpathlineto{\pgfqpoint{2.975372in}{1.240463in}}%
\pgfpathlineto{\pgfqpoint{3.042819in}{1.204961in}}%
\pgfpathlineto{\pgfqpoint{3.111567in}{1.168924in}}%
\pgfpathlineto{\pgfqpoint{3.181152in}{1.133136in}}%
\pgfpathlineto{\pgfqpoint{3.249317in}{1.096839in}}%
\pgfpathlineto{\pgfqpoint{3.318852in}{1.060394in}}%
\pgfpathlineto{\pgfqpoint{3.387274in}{1.025125in}}%
\pgfpathlineto{\pgfqpoint{3.456888in}{0.988914in}}%
\pgfpathlineto{\pgfqpoint{3.525281in}{0.954482in}}%
\pgfpathlineto{\pgfqpoint{3.594291in}{0.919858in}}%
\pgfpathlineto{\pgfqpoint{3.663350in}{0.883333in}}%
\pgfpathlineto{\pgfqpoint{3.732295in}{0.845417in}}%
\pgfpathlineto{\pgfqpoint{3.801454in}{0.804891in}}%
\pgfpathlineto{\pgfqpoint{3.870342in}{0.762758in}}%
\pgfusepath{stroke}%
\end{pgfscope}%
\begin{pgfscope}%
\pgfpathrectangle{\pgfqpoint{0.589510in}{0.417642in}}{\pgfqpoint{3.437062in}{2.055000in}}%
\pgfusepath{clip}%
\pgfsetrectcap%
\pgfsetroundjoin%
\pgfsetlinewidth{1.505625pt}%
\definecolor{currentstroke}{rgb}{0.172549,0.627451,0.172549}%
\pgfsetstrokecolor{currentstroke}%
\pgfsetdash{}{0pt}%
\pgfpathmoveto{\pgfqpoint{1.590462in}{2.027555in}}%
\pgfpathlineto{\pgfqpoint{1.844749in}{1.883255in}}%
\pgfpathlineto{\pgfqpoint{1.993497in}{1.800014in}}%
\pgfpathlineto{\pgfqpoint{2.099035in}{1.741812in}}%
\pgfpathlineto{\pgfqpoint{2.180897in}{1.697257in}}%
\pgfpathlineto{\pgfqpoint{2.304335in}{1.630048in}}%
\pgfpathlineto{\pgfqpoint{2.353322in}{1.603302in}}%
\pgfpathlineto{\pgfqpoint{2.435184in}{1.558190in}}%
\pgfpathlineto{\pgfqpoint{2.502070in}{1.521661in}}%
\pgfpathlineto{\pgfqpoint{2.558622in}{1.490942in}}%
\pgfpathlineto{\pgfqpoint{2.629849in}{1.452740in}}%
\pgfpathlineto{\pgfqpoint{2.689471in}{1.420850in}}%
\pgfpathlineto{\pgfqpoint{2.771333in}{1.376801in}}%
\pgfpathlineto{\pgfqpoint{2.838219in}{1.340625in}}%
\pgfpathlineto{\pgfqpoint{2.905105in}{1.304438in}}%
\pgfpathlineto{\pgfqpoint{2.970289in}{1.268853in}}%
\pgfpathlineto{\pgfqpoint{3.040008in}{1.230952in}}%
\pgfpathlineto{\pgfqpoint{3.110405in}{1.192831in}}%
\pgfpathlineto{\pgfqpoint{3.179227in}{1.155192in}}%
\pgfpathlineto{\pgfqpoint{3.249317in}{1.117349in}}%
\pgfpathlineto{\pgfqpoint{3.318191in}{1.079624in}}%
\pgfpathlineto{\pgfqpoint{3.387274in}{1.042096in}}%
\pgfpathlineto{\pgfqpoint{3.456888in}{1.004852in}}%
\pgfpathlineto{\pgfqpoint{3.524905in}{0.969920in}}%
\pgfpathlineto{\pgfqpoint{3.594913in}{0.935122in}}%
\pgfpathlineto{\pgfqpoint{3.662834in}{0.900161in}}%
\pgfpathlineto{\pgfqpoint{3.732081in}{0.864035in}}%
\pgfpathlineto{\pgfqpoint{3.801100in}{0.826825in}}%
\pgfpathlineto{\pgfqpoint{3.870342in}{0.789189in}}%
\pgfusepath{stroke}%
\end{pgfscope}%
\begin{pgfscope}%
\pgfpathrectangle{\pgfqpoint{0.589510in}{0.417642in}}{\pgfqpoint{3.437062in}{2.055000in}}%
\pgfusepath{clip}%
\pgfsetrectcap%
\pgfsetroundjoin%
\pgfsetlinewidth{1.505625pt}%
\definecolor{currentstroke}{rgb}{0.839216,0.152941,0.156863}%
\pgfsetstrokecolor{currentstroke}%
\pgfsetdash{}{0pt}%
\pgfpathmoveto{\pgfqpoint{1.844749in}{1.946045in}}%
\pgfpathlineto{\pgfqpoint{2.099035in}{1.798286in}}%
\pgfpathlineto{\pgfqpoint{2.247783in}{1.713004in}}%
\pgfpathlineto{\pgfqpoint{2.353322in}{1.653387in}}%
\pgfpathlineto{\pgfqpoint{2.435184in}{1.607437in}}%
\pgfpathlineto{\pgfqpoint{2.502070in}{1.570177in}}%
\pgfpathlineto{\pgfqpoint{2.558622in}{1.539063in}}%
\pgfpathlineto{\pgfqpoint{2.607609in}{1.512153in}}%
\pgfpathlineto{\pgfqpoint{2.689471in}{1.467682in}}%
\pgfpathlineto{\pgfqpoint{2.756357in}{1.431644in}}%
\pgfpathlineto{\pgfqpoint{2.838219in}{1.387730in}}%
\pgfpathlineto{\pgfqpoint{2.905105in}{1.352106in}}%
\pgfpathlineto{\pgfqpoint{2.978723in}{1.313043in}}%
\pgfpathlineto{\pgfqpoint{3.040008in}{1.280244in}}%
\pgfpathlineto{\pgfqpoint{3.116182in}{1.239844in}}%
\pgfpathlineto{\pgfqpoint{3.179227in}{1.206562in}}%
\pgfpathlineto{\pgfqpoint{3.249317in}{1.168776in}}%
\pgfpathlineto{\pgfqpoint{3.321482in}{1.130251in}}%
\pgfpathlineto{\pgfqpoint{3.387274in}{1.094892in}}%
\pgfpathlineto{\pgfqpoint{3.456888in}{1.057422in}}%
\pgfpathlineto{\pgfqpoint{3.526781in}{1.020048in}}%
\pgfpathlineto{\pgfqpoint{3.594913in}{0.981407in}}%
\pgfpathlineto{\pgfqpoint{3.662834in}{0.943932in}}%
\pgfpathlineto{\pgfqpoint{3.733149in}{0.904970in}}%
\pgfpathlineto{\pgfqpoint{3.801100in}{0.867607in}}%
\pgfpathlineto{\pgfqpoint{3.870342in}{0.831431in}}%
\pgfusepath{stroke}%
\end{pgfscope}%
\begin{pgfscope}%
\pgfpathrectangle{\pgfqpoint{0.589510in}{0.417642in}}{\pgfqpoint{3.437062in}{2.055000in}}%
\pgfusepath{clip}%
\pgfsetrectcap%
\pgfsetroundjoin%
\pgfsetlinewidth{1.505625pt}%
\definecolor{currentstroke}{rgb}{0.580392,0.403922,0.741176}%
\pgfsetstrokecolor{currentstroke}%
\pgfsetdash{}{0pt}%
\pgfpathmoveto{\pgfqpoint{2.099035in}{1.888053in}}%
\pgfpathlineto{\pgfqpoint{2.353322in}{1.734015in}}%
\pgfpathlineto{\pgfqpoint{2.502070in}{1.645554in}}%
\pgfpathlineto{\pgfqpoint{2.607609in}{1.583532in}}%
\pgfpathlineto{\pgfqpoint{2.689471in}{1.536271in}}%
\pgfpathlineto{\pgfqpoint{2.756357in}{1.497753in}}%
\pgfpathlineto{\pgfqpoint{2.812908in}{1.465823in}}%
\pgfpathlineto{\pgfqpoint{2.905105in}{1.414174in}}%
\pgfpathlineto{\pgfqpoint{2.978723in}{1.373428in}}%
\pgfpathlineto{\pgfqpoint{3.040008in}{1.339300in}}%
\pgfpathlineto{\pgfqpoint{3.116182in}{1.298519in}}%
\pgfpathlineto{\pgfqpoint{3.179227in}{1.264709in}}%
\pgfpathlineto{\pgfqpoint{3.249317in}{1.226950in}}%
\pgfpathlineto{\pgfqpoint{3.321482in}{1.187623in}}%
\pgfpathlineto{\pgfqpoint{3.392709in}{1.146836in}}%
\pgfpathlineto{\pgfqpoint{3.452331in}{1.112944in}}%
\pgfpathlineto{\pgfqpoint{3.526781in}{1.070971in}}%
\pgfpathlineto{\pgfqpoint{3.594913in}{1.033709in}}%
\pgfpathlineto{\pgfqpoint{3.662834in}{0.996299in}}%
\pgfpathlineto{\pgfqpoint{3.733149in}{0.958149in}}%
\pgfpathlineto{\pgfqpoint{3.802868in}{0.920578in}}%
\pgfpathlineto{\pgfqpoint{3.870342in}{0.881196in}}%
\pgfusepath{stroke}%
\end{pgfscope}%
\begin{pgfscope}%
\pgfpathrectangle{\pgfqpoint{0.589510in}{0.417642in}}{\pgfqpoint{3.437062in}{2.055000in}}%
\pgfusepath{clip}%
\pgfsetrectcap%
\pgfsetroundjoin%
\pgfsetlinewidth{1.505625pt}%
\definecolor{currentstroke}{rgb}{0.549020,0.337255,0.294118}%
\pgfsetstrokecolor{currentstroke}%
\pgfsetdash{}{0pt}%
\pgfpathmoveto{\pgfqpoint{2.435184in}{1.843551in}}%
\pgfpathlineto{\pgfqpoint{2.689471in}{1.682945in}}%
\pgfpathlineto{\pgfqpoint{2.838219in}{1.590848in}}%
\pgfpathlineto{\pgfqpoint{2.943757in}{1.526093in}}%
\pgfpathlineto{\pgfqpoint{3.025619in}{1.476983in}}%
\pgfpathlineto{\pgfqpoint{3.092506in}{1.437704in}}%
\pgfpathlineto{\pgfqpoint{3.198044in}{1.373739in}}%
\pgfpathlineto{\pgfqpoint{3.241254in}{1.347424in}}%
\pgfpathlineto{\pgfqpoint{3.314871in}{1.303724in}}%
\pgfpathlineto{\pgfqpoint{3.376157in}{1.269625in}}%
\pgfpathlineto{\pgfqpoint{3.452331in}{1.230018in}}%
\pgfpathlineto{\pgfqpoint{3.534193in}{1.185663in}}%
\pgfpathlineto{\pgfqpoint{3.601079in}{1.148566in}}%
\pgfpathlineto{\pgfqpoint{3.657630in}{1.117411in}}%
\pgfpathlineto{\pgfqpoint{3.728858in}{1.081426in}}%
\pgfpathlineto{\pgfqpoint{3.797538in}{1.046508in}}%
\pgfpathlineto{\pgfqpoint{3.870342in}{1.007077in}}%
\pgfusepath{stroke}%
\end{pgfscope}%
\begin{pgfscope}%
\pgfpathrectangle{\pgfqpoint{0.589510in}{0.417642in}}{\pgfqpoint{3.437062in}{2.055000in}}%
\pgfusepath{clip}%
\pgfsetbuttcap%
\pgfsetroundjoin%
\pgfsetlinewidth{1.505625pt}%
\definecolor{currentstroke}{rgb}{0.890196,0.466667,0.760784}%
\pgfsetstrokecolor{currentstroke}%
\pgfsetdash{{5.550000pt}{2.400000pt}}{0.000000pt}%
\pgfpathmoveto{\pgfqpoint{0.745740in}{2.151266in}}%
\pgfpathlineto{\pgfqpoint{1.000026in}{2.028694in}}%
\pgfpathlineto{\pgfqpoint{1.148775in}{1.962075in}}%
\pgfpathlineto{\pgfqpoint{1.254313in}{1.918150in}}%
\pgfpathlineto{\pgfqpoint{1.336175in}{1.886258in}}%
\pgfpathlineto{\pgfqpoint{1.403061in}{1.861837in}}%
\pgfpathlineto{\pgfqpoint{1.459613in}{1.842368in}}%
\pgfpathlineto{\pgfqpoint{1.508600in}{1.826357in}}%
\pgfpathlineto{\pgfqpoint{1.590462in}{1.801290in}}%
\pgfpathlineto{\pgfqpoint{1.657348in}{1.782546in}}%
\pgfpathlineto{\pgfqpoint{1.739210in}{1.761756in}}%
\pgfpathlineto{\pgfqpoint{1.806096in}{1.746691in}}%
\pgfpathlineto{\pgfqpoint{1.862648in}{1.735286in}}%
\pgfpathlineto{\pgfqpoint{1.940999in}{1.721226in}}%
\pgfpathlineto{\pgfqpoint{2.005526in}{1.711101in}}%
\pgfpathlineto{\pgfqpoint{2.080218in}{1.700856in}}%
\pgfpathlineto{\pgfqpoint{2.150308in}{1.692664in}}%
\pgfpathlineto{\pgfqpoint{2.215863in}{1.686126in}}%
\pgfpathlineto{\pgfqpoint{2.282749in}{1.680302in}}%
\pgfpathlineto{\pgfqpoint{2.353322in}{1.674843in}}%
\pgfpathlineto{\pgfqpoint{2.424010in}{1.670015in}}%
\pgfpathlineto{\pgfqpoint{2.489633in}{1.666012in}}%
\pgfpathlineto{\pgfqpoint{2.561233in}{1.662069in}}%
\pgfpathlineto{\pgfqpoint{2.629849in}{1.658849in}}%
\pgfpathlineto{\pgfqpoint{2.698529in}{1.656105in}}%
\pgfpathlineto{\pgfqpoint{2.766904in}{1.653981in}}%
\pgfpathlineto{\pgfqpoint{2.835765in}{1.652324in}}%
\pgfpathlineto{\pgfqpoint{2.905105in}{1.651063in}}%
\pgfpathlineto{\pgfqpoint{2.973686in}{1.649745in}}%
\pgfpathlineto{\pgfqpoint{3.042819in}{1.648067in}}%
\pgfpathlineto{\pgfqpoint{3.111567in}{1.646412in}}%
\pgfpathlineto{\pgfqpoint{3.180672in}{1.644719in}}%
\pgfpathlineto{\pgfqpoint{3.249715in}{1.643777in}}%
\pgfpathlineto{\pgfqpoint{3.318522in}{1.643776in}}%
\pgfpathlineto{\pgfqpoint{3.387548in}{1.644019in}}%
\pgfpathlineto{\pgfqpoint{3.456662in}{1.644022in}}%
\pgfpathlineto{\pgfqpoint{3.525469in}{1.643887in}}%
\pgfpathlineto{\pgfqpoint{3.594446in}{1.643112in}}%
\pgfpathlineto{\pgfqpoint{3.663479in}{1.642023in}}%
\pgfpathlineto{\pgfqpoint{3.732402in}{1.641065in}}%
\pgfpathlineto{\pgfqpoint{3.801366in}{1.639195in}}%
\pgfpathlineto{\pgfqpoint{3.870342in}{1.636771in}}%
\pgfusepath{stroke}%
\end{pgfscope}%
\begin{pgfscope}%
\pgfsetrectcap%
\pgfsetmiterjoin%
\pgfsetlinewidth{0.803000pt}%
\definecolor{currentstroke}{rgb}{0.000000,0.000000,0.000000}%
\pgfsetstrokecolor{currentstroke}%
\pgfsetdash{}{0pt}%
\pgfpathmoveto{\pgfqpoint{0.589510in}{0.417642in}}%
\pgfpathlineto{\pgfqpoint{0.589510in}{2.472642in}}%
\pgfusepath{stroke}%
\end{pgfscope}%
\begin{pgfscope}%
\pgfsetrectcap%
\pgfsetmiterjoin%
\pgfsetlinewidth{0.803000pt}%
\definecolor{currentstroke}{rgb}{0.000000,0.000000,0.000000}%
\pgfsetstrokecolor{currentstroke}%
\pgfsetdash{}{0pt}%
\pgfpathmoveto{\pgfqpoint{4.026572in}{0.417642in}}%
\pgfpathlineto{\pgfqpoint{4.026572in}{2.472642in}}%
\pgfusepath{stroke}%
\end{pgfscope}%
\begin{pgfscope}%
\pgfsetrectcap%
\pgfsetmiterjoin%
\pgfsetlinewidth{0.803000pt}%
\definecolor{currentstroke}{rgb}{0.000000,0.000000,0.000000}%
\pgfsetstrokecolor{currentstroke}%
\pgfsetdash{}{0pt}%
\pgfpathmoveto{\pgfqpoint{0.589510in}{0.417642in}}%
\pgfpathlineto{\pgfqpoint{4.026572in}{0.417642in}}%
\pgfusepath{stroke}%
\end{pgfscope}%
\begin{pgfscope}%
\pgfsetrectcap%
\pgfsetmiterjoin%
\pgfsetlinewidth{0.803000pt}%
\definecolor{currentstroke}{rgb}{0.000000,0.000000,0.000000}%
\pgfsetstrokecolor{currentstroke}%
\pgfsetdash{}{0pt}%
\pgfpathmoveto{\pgfqpoint{0.589510in}{2.472642in}}%
\pgfpathlineto{\pgfqpoint{4.026572in}{2.472642in}}%
\pgfusepath{stroke}%
\end{pgfscope}%
\begin{pgfscope}%
\pgfsetbuttcap%
\pgfsetmiterjoin%
\definecolor{currentfill}{rgb}{1.000000,1.000000,1.000000}%
\pgfsetfillcolor{currentfill}%
\pgfsetfillopacity{0.800000}%
\pgfsetlinewidth{1.003750pt}%
\definecolor{currentstroke}{rgb}{0.800000,0.800000,0.800000}%
\pgfsetstrokecolor{currentstroke}%
\pgfsetstrokeopacity{0.800000}%
\pgfsetdash{}{0pt}%
\pgfpathmoveto{\pgfqpoint{3.108127in}{1.454420in}}%
\pgfpathlineto{\pgfqpoint{3.948794in}{1.454420in}}%
\pgfpathquadraticcurveto{\pgfqpoint{3.971016in}{1.454420in}}{\pgfqpoint{3.971016in}{1.476642in}}%
\pgfpathlineto{\pgfqpoint{3.971016in}{2.394864in}}%
\pgfpathquadraticcurveto{\pgfqpoint{3.971016in}{2.417086in}}{\pgfqpoint{3.948794in}{2.417086in}}%
\pgfpathlineto{\pgfqpoint{3.108127in}{2.417086in}}%
\pgfpathquadraticcurveto{\pgfqpoint{3.085905in}{2.417086in}}{\pgfqpoint{3.085905in}{2.394864in}}%
\pgfpathlineto{\pgfqpoint{3.085905in}{1.476642in}}%
\pgfpathquadraticcurveto{\pgfqpoint{3.085905in}{1.454420in}}{\pgfqpoint{3.108127in}{1.454420in}}%
\pgfpathlineto{\pgfqpoint{3.108127in}{1.454420in}}%
\pgfpathclose%
\pgfusepath{stroke,fill}%
\end{pgfscope}%
\begin{pgfscope}%
\pgfsetrectcap%
\pgfsetroundjoin%
\pgfsetlinewidth{1.505625pt}%
\definecolor{currentstroke}{rgb}{0.121569,0.466667,0.705882}%
\pgfsetstrokecolor{currentstroke}%
\pgfsetdash{}{0pt}%
\pgfpathmoveto{\pgfqpoint{3.130349in}{2.333753in}}%
\pgfpathlineto{\pgfqpoint{3.241460in}{2.333753in}}%
\pgfpathlineto{\pgfqpoint{3.352571in}{2.333753in}}%
\pgfusepath{stroke}%
\end{pgfscope}%
\begin{pgfscope}%
\definecolor{textcolor}{rgb}{0.000000,0.000000,0.000000}%
\pgfsetstrokecolor{textcolor}%
\pgfsetfillcolor{textcolor}%
\pgftext[x=3.441460in,y=2.294864in,left,base]{\color{textcolor}{\rmfamily\fontsize{8.000000}{9.600000}\selectfont\catcode`\^=\active\def^{\ifmmode\sp\else\^{}\fi}\catcode`\%=\active\def%{\%}NPLC 1}}%
\end{pgfscope}%
\begin{pgfscope}%
\pgfsetrectcap%
\pgfsetroundjoin%
\pgfsetlinewidth{1.505625pt}%
\definecolor{currentstroke}{rgb}{1.000000,0.498039,0.054902}%
\pgfsetstrokecolor{currentstroke}%
\pgfsetdash{}{0pt}%
\pgfpathmoveto{\pgfqpoint{3.130349in}{2.178864in}}%
\pgfpathlineto{\pgfqpoint{3.241460in}{2.178864in}}%
\pgfpathlineto{\pgfqpoint{3.352571in}{2.178864in}}%
\pgfusepath{stroke}%
\end{pgfscope}%
\begin{pgfscope}%
\definecolor{textcolor}{rgb}{0.000000,0.000000,0.000000}%
\pgfsetstrokecolor{textcolor}%
\pgfsetfillcolor{textcolor}%
\pgftext[x=3.441460in,y=2.139975in,left,base]{\color{textcolor}{\rmfamily\fontsize{8.000000}{9.600000}\selectfont\catcode`\^=\active\def^{\ifmmode\sp\else\^{}\fi}\catcode`\%=\active\def%{\%}NPLC 2}}%
\end{pgfscope}%
\begin{pgfscope}%
\pgfsetrectcap%
\pgfsetroundjoin%
\pgfsetlinewidth{1.505625pt}%
\definecolor{currentstroke}{rgb}{0.172549,0.627451,0.172549}%
\pgfsetstrokecolor{currentstroke}%
\pgfsetdash{}{0pt}%
\pgfpathmoveto{\pgfqpoint{3.130349in}{2.023975in}}%
\pgfpathlineto{\pgfqpoint{3.241460in}{2.023975in}}%
\pgfpathlineto{\pgfqpoint{3.352571in}{2.023975in}}%
\pgfusepath{stroke}%
\end{pgfscope}%
\begin{pgfscope}%
\definecolor{textcolor}{rgb}{0.000000,0.000000,0.000000}%
\pgfsetstrokecolor{textcolor}%
\pgfsetfillcolor{textcolor}%
\pgftext[x=3.441460in,y=1.985086in,left,base]{\color{textcolor}{\rmfamily\fontsize{8.000000}{9.600000}\selectfont\catcode`\^=\active\def^{\ifmmode\sp\else\^{}\fi}\catcode`\%=\active\def%{\%}NPLC 5}}%
\end{pgfscope}%
\begin{pgfscope}%
\pgfsetrectcap%
\pgfsetroundjoin%
\pgfsetlinewidth{1.505625pt}%
\definecolor{currentstroke}{rgb}{0.839216,0.152941,0.156863}%
\pgfsetstrokecolor{currentstroke}%
\pgfsetdash{}{0pt}%
\pgfpathmoveto{\pgfqpoint{3.130349in}{1.869086in}}%
\pgfpathlineto{\pgfqpoint{3.241460in}{1.869086in}}%
\pgfpathlineto{\pgfqpoint{3.352571in}{1.869086in}}%
\pgfusepath{stroke}%
\end{pgfscope}%
\begin{pgfscope}%
\definecolor{textcolor}{rgb}{0.000000,0.000000,0.000000}%
\pgfsetstrokecolor{textcolor}%
\pgfsetfillcolor{textcolor}%
\pgftext[x=3.441460in,y=1.830198in,left,base]{\color{textcolor}{\rmfamily\fontsize{8.000000}{9.600000}\selectfont\catcode`\^=\active\def^{\ifmmode\sp\else\^{}\fi}\catcode`\%=\active\def%{\%}NPLC 10}}%
\end{pgfscope}%
\begin{pgfscope}%
\pgfsetrectcap%
\pgfsetroundjoin%
\pgfsetlinewidth{1.505625pt}%
\definecolor{currentstroke}{rgb}{0.580392,0.403922,0.741176}%
\pgfsetstrokecolor{currentstroke}%
\pgfsetdash{}{0pt}%
\pgfpathmoveto{\pgfqpoint{3.130349in}{1.714198in}}%
\pgfpathlineto{\pgfqpoint{3.241460in}{1.714198in}}%
\pgfpathlineto{\pgfqpoint{3.352571in}{1.714198in}}%
\pgfusepath{stroke}%
\end{pgfscope}%
\begin{pgfscope}%
\definecolor{textcolor}{rgb}{0.000000,0.000000,0.000000}%
\pgfsetstrokecolor{textcolor}%
\pgfsetfillcolor{textcolor}%
\pgftext[x=3.441460in,y=1.675309in,left,base]{\color{textcolor}{\rmfamily\fontsize{8.000000}{9.600000}\selectfont\catcode`\^=\active\def^{\ifmmode\sp\else\^{}\fi}\catcode`\%=\active\def%{\%}NPLC 20}}%
\end{pgfscope}%
\begin{pgfscope}%
\pgfsetrectcap%
\pgfsetroundjoin%
\pgfsetlinewidth{1.505625pt}%
\definecolor{currentstroke}{rgb}{0.549020,0.337255,0.294118}%
\pgfsetstrokecolor{currentstroke}%
\pgfsetdash{}{0pt}%
\pgfpathmoveto{\pgfqpoint{3.130349in}{1.559309in}}%
\pgfpathlineto{\pgfqpoint{3.241460in}{1.559309in}}%
\pgfpathlineto{\pgfqpoint{3.352571in}{1.559309in}}%
\pgfusepath{stroke}%
\end{pgfscope}%
\begin{pgfscope}%
\definecolor{textcolor}{rgb}{0.000000,0.000000,0.000000}%
\pgfsetstrokecolor{textcolor}%
\pgfsetfillcolor{textcolor}%
\pgftext[x=3.441460in,y=1.520420in,left,base]{\color{textcolor}{\rmfamily\fontsize{8.000000}{9.600000}\selectfont\catcode`\^=\active\def^{\ifmmode\sp\else\^{}\fi}\catcode`\%=\active\def%{\%}NPLC 50}}%
\end{pgfscope}%
\end{pgfpicture}%
\makeatother%
\endgroup%
% data/simulations/sim_optimal_autozero_v2.py
    \caption{Allan deviation for different ADC integration intervals before applying the AZ algorithm. Dead time $\theta = \qty{0}{\s}$. The dashed line denotes the Allan variance without autozeroing. The line frequency is \qty{50}{\Hz}.}
    \label{fig:autozero_nplcs_adev}
\end{figure}

It can be seen that with an increasing integration time before applying the AZ algorithm more uncertainty is accumulated due to the $f^{-1}$ content which cannot be filtered. As a result, after removing the $f^{-1}$ content using autozeroing more time is required for filtering until the same Allan deviation can be reached. From these simulations it can be concluded that if there is only a negligible dead time $\theta$ involved when switching the inputs, it is advantageous to switch early, while white noise is still dominating the noise content.% The two lines for \qty{1}{\PLC} and \qty{2}{\PLC} are clearly the ones with the lowest uncertainty

Finally, the case of a non-negligible dead time shall be treated. When the dead time has to be considered, it is clear that the autozero frequency cannot be arbitrarily increased, because an increasing proportion of sampling time is lost to the dead time. This effective loss in sampling time then increases the noise spectral density due to aliasing as discussed above. To show this effect, the simulation above is modified to include a dead time of \qty{1}{\plc} as detailed in figure \ref{fig:dmm_autozer_offset_nulling}. The dead time is added once after each measurement because the input is switched after each measurement. There are also alternative switching patterns like the one proposed by \citeauthor{autozero_with_dead_time} \cite{autozero_with_dead_time} splitting the measurement interval in two and instead of measuring HI-LO-HI-LO-HI-LO, to measure HI-LO-LO-HI-HI-LO. This scheme has both advantages and disadvantages, because $f^{-1}$ flicker noise is correlated and its autocorrelation function decays with $\text{const.} - \ln(\tau)$ \cite{flicker_noise_autocorrelation,flicker_noise_autocorrelation2}. Therefore constantly changing the order of subtracted samples is not as efficient in removing the noise as the normal autozero procedure because neighbouring samples are highly correlated. Only when the dead time is large in comparison to the measurement time, this method yields an advantage. Some measurements also allow for another scheme. If the measurement is differential, the HI and LO input can be inverted without incurring the noise penalty of equation \ref{eqn:autozeroing} because both measurements taken contain the desired data. This puts the autozeroing closer to a synchronous detection scheme, but this is outside the scope of this discussion. For the sake of simplicity, only the case of a HI-LO-HI-LO measurement mentioned first is treated here. To compare the zero dead time case with the non-negligible dead time case, the Allan deviation for different integration times is again evaluated in the same way as it was in figure \ref{fig:autozero_nplcs_adev}. The results are shown in figure \ref{fig:autozero_deadtime_nplcs_adev}.
\begin{figure}[ht]
    \centering
    %% Creator: Matplotlib, PGF backend
%%
%% To include the figure in your LaTeX document, write
%%   \input{<filename>.pgf}
%%
%% Make sure the required packages are loaded in your preamble
%%   \usepackage{pgf}
%%
%% Also ensure that all the required font packages are loaded; for instance,
%% the lmodern package is sometimes necessary when using math font.
%%   \usepackage{lmodern}
%%
%% Figures using additional raster images can only be included by \input if
%% they are in the same directory as the main LaTeX file. For loading figures
%% from other directories you can use the `import` package
%%   \usepackage{import}
%%
%% and then include the figures with
%%   \import{<path to file>}{<filename>.pgf}
%%
%% Matplotlib used the following preamble
%%   \def\mathdefault#1{#1}
%%   \everymath=\expandafter{\the\everymath\displaystyle}
%%   \usepackage{siunitx}
%%   \sisetup{per-mode = symbol}%
%%   \ifdefined\pdftexversion\else  % non-pdftex case.
%%     \usepackage{fontspec}
%%   \fi
%%   \makeatletter\@ifpackageloaded{underscore}{}{\usepackage[strings]{underscore}}\makeatother
%%
\begingroup%
\makeatletter%
\begin{pgfpicture}%
\pgfpathrectangle{\pgfpointorigin}{\pgfqpoint{4.068242in}{2.514312in}}%
\pgfusepath{use as bounding box, clip}%
\begin{pgfscope}%
\pgfsetbuttcap%
\pgfsetmiterjoin%
\definecolor{currentfill}{rgb}{1.000000,1.000000,1.000000}%
\pgfsetfillcolor{currentfill}%
\pgfsetlinewidth{0.000000pt}%
\definecolor{currentstroke}{rgb}{1.000000,1.000000,1.000000}%
\pgfsetstrokecolor{currentstroke}%
\pgfsetdash{}{0pt}%
\pgfpathmoveto{\pgfqpoint{0.000000in}{0.000000in}}%
\pgfpathlineto{\pgfqpoint{4.068242in}{0.000000in}}%
\pgfpathlineto{\pgfqpoint{4.068242in}{2.514312in}}%
\pgfpathlineto{\pgfqpoint{0.000000in}{2.514312in}}%
\pgfpathlineto{\pgfqpoint{0.000000in}{0.000000in}}%
\pgfpathclose%
\pgfusepath{fill}%
\end{pgfscope}%
\begin{pgfscope}%
\pgfsetbuttcap%
\pgfsetmiterjoin%
\definecolor{currentfill}{rgb}{1.000000,1.000000,1.000000}%
\pgfsetfillcolor{currentfill}%
\pgfsetlinewidth{0.000000pt}%
\definecolor{currentstroke}{rgb}{0.000000,0.000000,0.000000}%
\pgfsetstrokecolor{currentstroke}%
\pgfsetstrokeopacity{0.000000}%
\pgfsetdash{}{0pt}%
\pgfpathmoveto{\pgfqpoint{0.589510in}{0.417642in}}%
\pgfpathlineto{\pgfqpoint{4.026572in}{0.417642in}}%
\pgfpathlineto{\pgfqpoint{4.026572in}{2.472642in}}%
\pgfpathlineto{\pgfqpoint{0.589510in}{2.472642in}}%
\pgfpathlineto{\pgfqpoint{0.589510in}{0.417642in}}%
\pgfpathclose%
\pgfusepath{fill}%
\end{pgfscope}%
\begin{pgfscope}%
\pgfpathrectangle{\pgfqpoint{0.589510in}{0.417642in}}{\pgfqpoint{3.437062in}{2.055000in}}%
\pgfusepath{clip}%
\pgfsetrectcap%
\pgfsetroundjoin%
\pgfsetlinewidth{0.803000pt}%
\definecolor{currentstroke}{rgb}{0.450000,0.450000,0.450000}%
\pgfsetstrokecolor{currentstroke}%
\pgfsetdash{}{0pt}%
\pgfpathmoveto{\pgfqpoint{1.336175in}{0.417642in}}%
\pgfpathlineto{\pgfqpoint{1.336175in}{2.472642in}}%
\pgfusepath{stroke}%
\end{pgfscope}%
\begin{pgfscope}%
\pgfsetbuttcap%
\pgfsetroundjoin%
\definecolor{currentfill}{rgb}{0.000000,0.000000,0.000000}%
\pgfsetfillcolor{currentfill}%
\pgfsetlinewidth{0.803000pt}%
\definecolor{currentstroke}{rgb}{0.000000,0.000000,0.000000}%
\pgfsetstrokecolor{currentstroke}%
\pgfsetdash{}{0pt}%
\pgfsys@defobject{currentmarker}{\pgfqpoint{0.000000in}{-0.048611in}}{\pgfqpoint{0.000000in}{0.000000in}}{%
\pgfpathmoveto{\pgfqpoint{0.000000in}{0.000000in}}%
\pgfpathlineto{\pgfqpoint{0.000000in}{-0.048611in}}%
\pgfusepath{stroke,fill}%
}%
\begin{pgfscope}%
\pgfsys@transformshift{1.336175in}{0.417642in}%
\pgfsys@useobject{currentmarker}{}%
\end{pgfscope}%
\end{pgfscope}%
\begin{pgfscope}%
\definecolor{textcolor}{rgb}{0.000000,0.000000,0.000000}%
\pgfsetstrokecolor{textcolor}%
\pgfsetfillcolor{textcolor}%
\pgftext[x=1.336175in,y=0.320420in,,top]{\color{textcolor}{\rmfamily\fontsize{8.000000}{9.600000}\selectfont\catcode`\^=\active\def^{\ifmmode\sp\else\^{}\fi}\catcode`\%=\active\def%{\%}$\mathdefault{10^{-1}}$}}%
\end{pgfscope}%
\begin{pgfscope}%
\pgfpathrectangle{\pgfqpoint{0.589510in}{0.417642in}}{\pgfqpoint{3.437062in}{2.055000in}}%
\pgfusepath{clip}%
\pgfsetrectcap%
\pgfsetroundjoin%
\pgfsetlinewidth{0.803000pt}%
\definecolor{currentstroke}{rgb}{0.450000,0.450000,0.450000}%
\pgfsetstrokecolor{currentstroke}%
\pgfsetdash{}{0pt}%
\pgfpathmoveto{\pgfqpoint{2.180897in}{0.417642in}}%
\pgfpathlineto{\pgfqpoint{2.180897in}{2.472642in}}%
\pgfusepath{stroke}%
\end{pgfscope}%
\begin{pgfscope}%
\pgfsetbuttcap%
\pgfsetroundjoin%
\definecolor{currentfill}{rgb}{0.000000,0.000000,0.000000}%
\pgfsetfillcolor{currentfill}%
\pgfsetlinewidth{0.803000pt}%
\definecolor{currentstroke}{rgb}{0.000000,0.000000,0.000000}%
\pgfsetstrokecolor{currentstroke}%
\pgfsetdash{}{0pt}%
\pgfsys@defobject{currentmarker}{\pgfqpoint{0.000000in}{-0.048611in}}{\pgfqpoint{0.000000in}{0.000000in}}{%
\pgfpathmoveto{\pgfqpoint{0.000000in}{0.000000in}}%
\pgfpathlineto{\pgfqpoint{0.000000in}{-0.048611in}}%
\pgfusepath{stroke,fill}%
}%
\begin{pgfscope}%
\pgfsys@transformshift{2.180897in}{0.417642in}%
\pgfsys@useobject{currentmarker}{}%
\end{pgfscope}%
\end{pgfscope}%
\begin{pgfscope}%
\definecolor{textcolor}{rgb}{0.000000,0.000000,0.000000}%
\pgfsetstrokecolor{textcolor}%
\pgfsetfillcolor{textcolor}%
\pgftext[x=2.180897in,y=0.320420in,,top]{\color{textcolor}{\rmfamily\fontsize{8.000000}{9.600000}\selectfont\catcode`\^=\active\def^{\ifmmode\sp\else\^{}\fi}\catcode`\%=\active\def%{\%}$\mathdefault{10^{0}}$}}%
\end{pgfscope}%
\begin{pgfscope}%
\pgfpathrectangle{\pgfqpoint{0.589510in}{0.417642in}}{\pgfqpoint{3.437062in}{2.055000in}}%
\pgfusepath{clip}%
\pgfsetrectcap%
\pgfsetroundjoin%
\pgfsetlinewidth{0.803000pt}%
\definecolor{currentstroke}{rgb}{0.450000,0.450000,0.450000}%
\pgfsetstrokecolor{currentstroke}%
\pgfsetdash{}{0pt}%
\pgfpathmoveto{\pgfqpoint{3.025619in}{0.417642in}}%
\pgfpathlineto{\pgfqpoint{3.025619in}{2.472642in}}%
\pgfusepath{stroke}%
\end{pgfscope}%
\begin{pgfscope}%
\pgfsetbuttcap%
\pgfsetroundjoin%
\definecolor{currentfill}{rgb}{0.000000,0.000000,0.000000}%
\pgfsetfillcolor{currentfill}%
\pgfsetlinewidth{0.803000pt}%
\definecolor{currentstroke}{rgb}{0.000000,0.000000,0.000000}%
\pgfsetstrokecolor{currentstroke}%
\pgfsetdash{}{0pt}%
\pgfsys@defobject{currentmarker}{\pgfqpoint{0.000000in}{-0.048611in}}{\pgfqpoint{0.000000in}{0.000000in}}{%
\pgfpathmoveto{\pgfqpoint{0.000000in}{0.000000in}}%
\pgfpathlineto{\pgfqpoint{0.000000in}{-0.048611in}}%
\pgfusepath{stroke,fill}%
}%
\begin{pgfscope}%
\pgfsys@transformshift{3.025619in}{0.417642in}%
\pgfsys@useobject{currentmarker}{}%
\end{pgfscope}%
\end{pgfscope}%
\begin{pgfscope}%
\definecolor{textcolor}{rgb}{0.000000,0.000000,0.000000}%
\pgfsetstrokecolor{textcolor}%
\pgfsetfillcolor{textcolor}%
\pgftext[x=3.025619in,y=0.320420in,,top]{\color{textcolor}{\rmfamily\fontsize{8.000000}{9.600000}\selectfont\catcode`\^=\active\def^{\ifmmode\sp\else\^{}\fi}\catcode`\%=\active\def%{\%}$\mathdefault{10^{1}}$}}%
\end{pgfscope}%
\begin{pgfscope}%
\pgfpathrectangle{\pgfqpoint{0.589510in}{0.417642in}}{\pgfqpoint{3.437062in}{2.055000in}}%
\pgfusepath{clip}%
\pgfsetrectcap%
\pgfsetroundjoin%
\pgfsetlinewidth{0.803000pt}%
\definecolor{currentstroke}{rgb}{0.450000,0.450000,0.450000}%
\pgfsetstrokecolor{currentstroke}%
\pgfsetdash{}{0pt}%
\pgfpathmoveto{\pgfqpoint{3.870342in}{0.417642in}}%
\pgfpathlineto{\pgfqpoint{3.870342in}{2.472642in}}%
\pgfusepath{stroke}%
\end{pgfscope}%
\begin{pgfscope}%
\pgfsetbuttcap%
\pgfsetroundjoin%
\definecolor{currentfill}{rgb}{0.000000,0.000000,0.000000}%
\pgfsetfillcolor{currentfill}%
\pgfsetlinewidth{0.803000pt}%
\definecolor{currentstroke}{rgb}{0.000000,0.000000,0.000000}%
\pgfsetstrokecolor{currentstroke}%
\pgfsetdash{}{0pt}%
\pgfsys@defobject{currentmarker}{\pgfqpoint{0.000000in}{-0.048611in}}{\pgfqpoint{0.000000in}{0.000000in}}{%
\pgfpathmoveto{\pgfqpoint{0.000000in}{0.000000in}}%
\pgfpathlineto{\pgfqpoint{0.000000in}{-0.048611in}}%
\pgfusepath{stroke,fill}%
}%
\begin{pgfscope}%
\pgfsys@transformshift{3.870342in}{0.417642in}%
\pgfsys@useobject{currentmarker}{}%
\end{pgfscope}%
\end{pgfscope}%
\begin{pgfscope}%
\definecolor{textcolor}{rgb}{0.000000,0.000000,0.000000}%
\pgfsetstrokecolor{textcolor}%
\pgfsetfillcolor{textcolor}%
\pgftext[x=3.870342in,y=0.320420in,,top]{\color{textcolor}{\rmfamily\fontsize{8.000000}{9.600000}\selectfont\catcode`\^=\active\def^{\ifmmode\sp\else\^{}\fi}\catcode`\%=\active\def%{\%}$\mathdefault{10^{2}}$}}%
\end{pgfscope}%
\begin{pgfscope}%
\pgfpathrectangle{\pgfqpoint{0.589510in}{0.417642in}}{\pgfqpoint{3.437062in}{2.055000in}}%
\pgfusepath{clip}%
\pgfsetrectcap%
\pgfsetroundjoin%
\pgfsetlinewidth{0.803000pt}%
\definecolor{currentstroke}{rgb}{0.850000,0.850000,0.850000}%
\pgfsetstrokecolor{currentstroke}%
\pgfsetdash{}{0pt}%
\pgfpathmoveto{\pgfqpoint{0.745740in}{0.417642in}}%
\pgfpathlineto{\pgfqpoint{0.745740in}{2.472642in}}%
\pgfusepath{stroke}%
\end{pgfscope}%
\begin{pgfscope}%
\pgfsetbuttcap%
\pgfsetroundjoin%
\definecolor{currentfill}{rgb}{0.000000,0.000000,0.000000}%
\pgfsetfillcolor{currentfill}%
\pgfsetlinewidth{0.602250pt}%
\definecolor{currentstroke}{rgb}{0.000000,0.000000,0.000000}%
\pgfsetstrokecolor{currentstroke}%
\pgfsetdash{}{0pt}%
\pgfsys@defobject{currentmarker}{\pgfqpoint{0.000000in}{-0.027778in}}{\pgfqpoint{0.000000in}{0.000000in}}{%
\pgfpathmoveto{\pgfqpoint{0.000000in}{0.000000in}}%
\pgfpathlineto{\pgfqpoint{0.000000in}{-0.027778in}}%
\pgfusepath{stroke,fill}%
}%
\begin{pgfscope}%
\pgfsys@transformshift{0.745740in}{0.417642in}%
\pgfsys@useobject{currentmarker}{}%
\end{pgfscope}%
\end{pgfscope}%
\begin{pgfscope}%
\pgfpathrectangle{\pgfqpoint{0.589510in}{0.417642in}}{\pgfqpoint{3.437062in}{2.055000in}}%
\pgfusepath{clip}%
\pgfsetrectcap%
\pgfsetroundjoin%
\pgfsetlinewidth{0.803000pt}%
\definecolor{currentstroke}{rgb}{0.850000,0.850000,0.850000}%
\pgfsetstrokecolor{currentstroke}%
\pgfsetdash{}{0pt}%
\pgfpathmoveto{\pgfqpoint{0.894488in}{0.417642in}}%
\pgfpathlineto{\pgfqpoint{0.894488in}{2.472642in}}%
\pgfusepath{stroke}%
\end{pgfscope}%
\begin{pgfscope}%
\pgfsetbuttcap%
\pgfsetroundjoin%
\definecolor{currentfill}{rgb}{0.000000,0.000000,0.000000}%
\pgfsetfillcolor{currentfill}%
\pgfsetlinewidth{0.602250pt}%
\definecolor{currentstroke}{rgb}{0.000000,0.000000,0.000000}%
\pgfsetstrokecolor{currentstroke}%
\pgfsetdash{}{0pt}%
\pgfsys@defobject{currentmarker}{\pgfqpoint{0.000000in}{-0.027778in}}{\pgfqpoint{0.000000in}{0.000000in}}{%
\pgfpathmoveto{\pgfqpoint{0.000000in}{0.000000in}}%
\pgfpathlineto{\pgfqpoint{0.000000in}{-0.027778in}}%
\pgfusepath{stroke,fill}%
}%
\begin{pgfscope}%
\pgfsys@transformshift{0.894488in}{0.417642in}%
\pgfsys@useobject{currentmarker}{}%
\end{pgfscope}%
\end{pgfscope}%
\begin{pgfscope}%
\pgfpathrectangle{\pgfqpoint{0.589510in}{0.417642in}}{\pgfqpoint{3.437062in}{2.055000in}}%
\pgfusepath{clip}%
\pgfsetrectcap%
\pgfsetroundjoin%
\pgfsetlinewidth{0.803000pt}%
\definecolor{currentstroke}{rgb}{0.850000,0.850000,0.850000}%
\pgfsetstrokecolor{currentstroke}%
\pgfsetdash{}{0pt}%
\pgfpathmoveto{\pgfqpoint{1.000026in}{0.417642in}}%
\pgfpathlineto{\pgfqpoint{1.000026in}{2.472642in}}%
\pgfusepath{stroke}%
\end{pgfscope}%
\begin{pgfscope}%
\pgfsetbuttcap%
\pgfsetroundjoin%
\definecolor{currentfill}{rgb}{0.000000,0.000000,0.000000}%
\pgfsetfillcolor{currentfill}%
\pgfsetlinewidth{0.602250pt}%
\definecolor{currentstroke}{rgb}{0.000000,0.000000,0.000000}%
\pgfsetstrokecolor{currentstroke}%
\pgfsetdash{}{0pt}%
\pgfsys@defobject{currentmarker}{\pgfqpoint{0.000000in}{-0.027778in}}{\pgfqpoint{0.000000in}{0.000000in}}{%
\pgfpathmoveto{\pgfqpoint{0.000000in}{0.000000in}}%
\pgfpathlineto{\pgfqpoint{0.000000in}{-0.027778in}}%
\pgfusepath{stroke,fill}%
}%
\begin{pgfscope}%
\pgfsys@transformshift{1.000026in}{0.417642in}%
\pgfsys@useobject{currentmarker}{}%
\end{pgfscope}%
\end{pgfscope}%
\begin{pgfscope}%
\pgfpathrectangle{\pgfqpoint{0.589510in}{0.417642in}}{\pgfqpoint{3.437062in}{2.055000in}}%
\pgfusepath{clip}%
\pgfsetrectcap%
\pgfsetroundjoin%
\pgfsetlinewidth{0.803000pt}%
\definecolor{currentstroke}{rgb}{0.850000,0.850000,0.850000}%
\pgfsetstrokecolor{currentstroke}%
\pgfsetdash{}{0pt}%
\pgfpathmoveto{\pgfqpoint{1.081889in}{0.417642in}}%
\pgfpathlineto{\pgfqpoint{1.081889in}{2.472642in}}%
\pgfusepath{stroke}%
\end{pgfscope}%
\begin{pgfscope}%
\pgfsetbuttcap%
\pgfsetroundjoin%
\definecolor{currentfill}{rgb}{0.000000,0.000000,0.000000}%
\pgfsetfillcolor{currentfill}%
\pgfsetlinewidth{0.602250pt}%
\definecolor{currentstroke}{rgb}{0.000000,0.000000,0.000000}%
\pgfsetstrokecolor{currentstroke}%
\pgfsetdash{}{0pt}%
\pgfsys@defobject{currentmarker}{\pgfqpoint{0.000000in}{-0.027778in}}{\pgfqpoint{0.000000in}{0.000000in}}{%
\pgfpathmoveto{\pgfqpoint{0.000000in}{0.000000in}}%
\pgfpathlineto{\pgfqpoint{0.000000in}{-0.027778in}}%
\pgfusepath{stroke,fill}%
}%
\begin{pgfscope}%
\pgfsys@transformshift{1.081889in}{0.417642in}%
\pgfsys@useobject{currentmarker}{}%
\end{pgfscope}%
\end{pgfscope}%
\begin{pgfscope}%
\pgfpathrectangle{\pgfqpoint{0.589510in}{0.417642in}}{\pgfqpoint{3.437062in}{2.055000in}}%
\pgfusepath{clip}%
\pgfsetrectcap%
\pgfsetroundjoin%
\pgfsetlinewidth{0.803000pt}%
\definecolor{currentstroke}{rgb}{0.850000,0.850000,0.850000}%
\pgfsetstrokecolor{currentstroke}%
\pgfsetdash{}{0pt}%
\pgfpathmoveto{\pgfqpoint{1.148775in}{0.417642in}}%
\pgfpathlineto{\pgfqpoint{1.148775in}{2.472642in}}%
\pgfusepath{stroke}%
\end{pgfscope}%
\begin{pgfscope}%
\pgfsetbuttcap%
\pgfsetroundjoin%
\definecolor{currentfill}{rgb}{0.000000,0.000000,0.000000}%
\pgfsetfillcolor{currentfill}%
\pgfsetlinewidth{0.602250pt}%
\definecolor{currentstroke}{rgb}{0.000000,0.000000,0.000000}%
\pgfsetstrokecolor{currentstroke}%
\pgfsetdash{}{0pt}%
\pgfsys@defobject{currentmarker}{\pgfqpoint{0.000000in}{-0.027778in}}{\pgfqpoint{0.000000in}{0.000000in}}{%
\pgfpathmoveto{\pgfqpoint{0.000000in}{0.000000in}}%
\pgfpathlineto{\pgfqpoint{0.000000in}{-0.027778in}}%
\pgfusepath{stroke,fill}%
}%
\begin{pgfscope}%
\pgfsys@transformshift{1.148775in}{0.417642in}%
\pgfsys@useobject{currentmarker}{}%
\end{pgfscope}%
\end{pgfscope}%
\begin{pgfscope}%
\pgfpathrectangle{\pgfqpoint{0.589510in}{0.417642in}}{\pgfqpoint{3.437062in}{2.055000in}}%
\pgfusepath{clip}%
\pgfsetrectcap%
\pgfsetroundjoin%
\pgfsetlinewidth{0.803000pt}%
\definecolor{currentstroke}{rgb}{0.850000,0.850000,0.850000}%
\pgfsetstrokecolor{currentstroke}%
\pgfsetdash{}{0pt}%
\pgfpathmoveto{\pgfqpoint{1.205326in}{0.417642in}}%
\pgfpathlineto{\pgfqpoint{1.205326in}{2.472642in}}%
\pgfusepath{stroke}%
\end{pgfscope}%
\begin{pgfscope}%
\pgfsetbuttcap%
\pgfsetroundjoin%
\definecolor{currentfill}{rgb}{0.000000,0.000000,0.000000}%
\pgfsetfillcolor{currentfill}%
\pgfsetlinewidth{0.602250pt}%
\definecolor{currentstroke}{rgb}{0.000000,0.000000,0.000000}%
\pgfsetstrokecolor{currentstroke}%
\pgfsetdash{}{0pt}%
\pgfsys@defobject{currentmarker}{\pgfqpoint{0.000000in}{-0.027778in}}{\pgfqpoint{0.000000in}{0.000000in}}{%
\pgfpathmoveto{\pgfqpoint{0.000000in}{0.000000in}}%
\pgfpathlineto{\pgfqpoint{0.000000in}{-0.027778in}}%
\pgfusepath{stroke,fill}%
}%
\begin{pgfscope}%
\pgfsys@transformshift{1.205326in}{0.417642in}%
\pgfsys@useobject{currentmarker}{}%
\end{pgfscope}%
\end{pgfscope}%
\begin{pgfscope}%
\pgfpathrectangle{\pgfqpoint{0.589510in}{0.417642in}}{\pgfqpoint{3.437062in}{2.055000in}}%
\pgfusepath{clip}%
\pgfsetrectcap%
\pgfsetroundjoin%
\pgfsetlinewidth{0.803000pt}%
\definecolor{currentstroke}{rgb}{0.850000,0.850000,0.850000}%
\pgfsetstrokecolor{currentstroke}%
\pgfsetdash{}{0pt}%
\pgfpathmoveto{\pgfqpoint{1.254313in}{0.417642in}}%
\pgfpathlineto{\pgfqpoint{1.254313in}{2.472642in}}%
\pgfusepath{stroke}%
\end{pgfscope}%
\begin{pgfscope}%
\pgfsetbuttcap%
\pgfsetroundjoin%
\definecolor{currentfill}{rgb}{0.000000,0.000000,0.000000}%
\pgfsetfillcolor{currentfill}%
\pgfsetlinewidth{0.602250pt}%
\definecolor{currentstroke}{rgb}{0.000000,0.000000,0.000000}%
\pgfsetstrokecolor{currentstroke}%
\pgfsetdash{}{0pt}%
\pgfsys@defobject{currentmarker}{\pgfqpoint{0.000000in}{-0.027778in}}{\pgfqpoint{0.000000in}{0.000000in}}{%
\pgfpathmoveto{\pgfqpoint{0.000000in}{0.000000in}}%
\pgfpathlineto{\pgfqpoint{0.000000in}{-0.027778in}}%
\pgfusepath{stroke,fill}%
}%
\begin{pgfscope}%
\pgfsys@transformshift{1.254313in}{0.417642in}%
\pgfsys@useobject{currentmarker}{}%
\end{pgfscope}%
\end{pgfscope}%
\begin{pgfscope}%
\pgfpathrectangle{\pgfqpoint{0.589510in}{0.417642in}}{\pgfqpoint{3.437062in}{2.055000in}}%
\pgfusepath{clip}%
\pgfsetrectcap%
\pgfsetroundjoin%
\pgfsetlinewidth{0.803000pt}%
\definecolor{currentstroke}{rgb}{0.850000,0.850000,0.850000}%
\pgfsetstrokecolor{currentstroke}%
\pgfsetdash{}{0pt}%
\pgfpathmoveto{\pgfqpoint{1.297523in}{0.417642in}}%
\pgfpathlineto{\pgfqpoint{1.297523in}{2.472642in}}%
\pgfusepath{stroke}%
\end{pgfscope}%
\begin{pgfscope}%
\pgfsetbuttcap%
\pgfsetroundjoin%
\definecolor{currentfill}{rgb}{0.000000,0.000000,0.000000}%
\pgfsetfillcolor{currentfill}%
\pgfsetlinewidth{0.602250pt}%
\definecolor{currentstroke}{rgb}{0.000000,0.000000,0.000000}%
\pgfsetstrokecolor{currentstroke}%
\pgfsetdash{}{0pt}%
\pgfsys@defobject{currentmarker}{\pgfqpoint{0.000000in}{-0.027778in}}{\pgfqpoint{0.000000in}{0.000000in}}{%
\pgfpathmoveto{\pgfqpoint{0.000000in}{0.000000in}}%
\pgfpathlineto{\pgfqpoint{0.000000in}{-0.027778in}}%
\pgfusepath{stroke,fill}%
}%
\begin{pgfscope}%
\pgfsys@transformshift{1.297523in}{0.417642in}%
\pgfsys@useobject{currentmarker}{}%
\end{pgfscope}%
\end{pgfscope}%
\begin{pgfscope}%
\pgfpathrectangle{\pgfqpoint{0.589510in}{0.417642in}}{\pgfqpoint{3.437062in}{2.055000in}}%
\pgfusepath{clip}%
\pgfsetrectcap%
\pgfsetroundjoin%
\pgfsetlinewidth{0.803000pt}%
\definecolor{currentstroke}{rgb}{0.850000,0.850000,0.850000}%
\pgfsetstrokecolor{currentstroke}%
\pgfsetdash{}{0pt}%
\pgfpathmoveto{\pgfqpoint{1.590462in}{0.417642in}}%
\pgfpathlineto{\pgfqpoint{1.590462in}{2.472642in}}%
\pgfusepath{stroke}%
\end{pgfscope}%
\begin{pgfscope}%
\pgfsetbuttcap%
\pgfsetroundjoin%
\definecolor{currentfill}{rgb}{0.000000,0.000000,0.000000}%
\pgfsetfillcolor{currentfill}%
\pgfsetlinewidth{0.602250pt}%
\definecolor{currentstroke}{rgb}{0.000000,0.000000,0.000000}%
\pgfsetstrokecolor{currentstroke}%
\pgfsetdash{}{0pt}%
\pgfsys@defobject{currentmarker}{\pgfqpoint{0.000000in}{-0.027778in}}{\pgfqpoint{0.000000in}{0.000000in}}{%
\pgfpathmoveto{\pgfqpoint{0.000000in}{0.000000in}}%
\pgfpathlineto{\pgfqpoint{0.000000in}{-0.027778in}}%
\pgfusepath{stroke,fill}%
}%
\begin{pgfscope}%
\pgfsys@transformshift{1.590462in}{0.417642in}%
\pgfsys@useobject{currentmarker}{}%
\end{pgfscope}%
\end{pgfscope}%
\begin{pgfscope}%
\pgfpathrectangle{\pgfqpoint{0.589510in}{0.417642in}}{\pgfqpoint{3.437062in}{2.055000in}}%
\pgfusepath{clip}%
\pgfsetrectcap%
\pgfsetroundjoin%
\pgfsetlinewidth{0.803000pt}%
\definecolor{currentstroke}{rgb}{0.850000,0.850000,0.850000}%
\pgfsetstrokecolor{currentstroke}%
\pgfsetdash{}{0pt}%
\pgfpathmoveto{\pgfqpoint{1.739210in}{0.417642in}}%
\pgfpathlineto{\pgfqpoint{1.739210in}{2.472642in}}%
\pgfusepath{stroke}%
\end{pgfscope}%
\begin{pgfscope}%
\pgfsetbuttcap%
\pgfsetroundjoin%
\definecolor{currentfill}{rgb}{0.000000,0.000000,0.000000}%
\pgfsetfillcolor{currentfill}%
\pgfsetlinewidth{0.602250pt}%
\definecolor{currentstroke}{rgb}{0.000000,0.000000,0.000000}%
\pgfsetstrokecolor{currentstroke}%
\pgfsetdash{}{0pt}%
\pgfsys@defobject{currentmarker}{\pgfqpoint{0.000000in}{-0.027778in}}{\pgfqpoint{0.000000in}{0.000000in}}{%
\pgfpathmoveto{\pgfqpoint{0.000000in}{0.000000in}}%
\pgfpathlineto{\pgfqpoint{0.000000in}{-0.027778in}}%
\pgfusepath{stroke,fill}%
}%
\begin{pgfscope}%
\pgfsys@transformshift{1.739210in}{0.417642in}%
\pgfsys@useobject{currentmarker}{}%
\end{pgfscope}%
\end{pgfscope}%
\begin{pgfscope}%
\pgfpathrectangle{\pgfqpoint{0.589510in}{0.417642in}}{\pgfqpoint{3.437062in}{2.055000in}}%
\pgfusepath{clip}%
\pgfsetrectcap%
\pgfsetroundjoin%
\pgfsetlinewidth{0.803000pt}%
\definecolor{currentstroke}{rgb}{0.850000,0.850000,0.850000}%
\pgfsetstrokecolor{currentstroke}%
\pgfsetdash{}{0pt}%
\pgfpathmoveto{\pgfqpoint{1.844749in}{0.417642in}}%
\pgfpathlineto{\pgfqpoint{1.844749in}{2.472642in}}%
\pgfusepath{stroke}%
\end{pgfscope}%
\begin{pgfscope}%
\pgfsetbuttcap%
\pgfsetroundjoin%
\definecolor{currentfill}{rgb}{0.000000,0.000000,0.000000}%
\pgfsetfillcolor{currentfill}%
\pgfsetlinewidth{0.602250pt}%
\definecolor{currentstroke}{rgb}{0.000000,0.000000,0.000000}%
\pgfsetstrokecolor{currentstroke}%
\pgfsetdash{}{0pt}%
\pgfsys@defobject{currentmarker}{\pgfqpoint{0.000000in}{-0.027778in}}{\pgfqpoint{0.000000in}{0.000000in}}{%
\pgfpathmoveto{\pgfqpoint{0.000000in}{0.000000in}}%
\pgfpathlineto{\pgfqpoint{0.000000in}{-0.027778in}}%
\pgfusepath{stroke,fill}%
}%
\begin{pgfscope}%
\pgfsys@transformshift{1.844749in}{0.417642in}%
\pgfsys@useobject{currentmarker}{}%
\end{pgfscope}%
\end{pgfscope}%
\begin{pgfscope}%
\pgfpathrectangle{\pgfqpoint{0.589510in}{0.417642in}}{\pgfqpoint{3.437062in}{2.055000in}}%
\pgfusepath{clip}%
\pgfsetrectcap%
\pgfsetroundjoin%
\pgfsetlinewidth{0.803000pt}%
\definecolor{currentstroke}{rgb}{0.850000,0.850000,0.850000}%
\pgfsetstrokecolor{currentstroke}%
\pgfsetdash{}{0pt}%
\pgfpathmoveto{\pgfqpoint{1.926611in}{0.417642in}}%
\pgfpathlineto{\pgfqpoint{1.926611in}{2.472642in}}%
\pgfusepath{stroke}%
\end{pgfscope}%
\begin{pgfscope}%
\pgfsetbuttcap%
\pgfsetroundjoin%
\definecolor{currentfill}{rgb}{0.000000,0.000000,0.000000}%
\pgfsetfillcolor{currentfill}%
\pgfsetlinewidth{0.602250pt}%
\definecolor{currentstroke}{rgb}{0.000000,0.000000,0.000000}%
\pgfsetstrokecolor{currentstroke}%
\pgfsetdash{}{0pt}%
\pgfsys@defobject{currentmarker}{\pgfqpoint{0.000000in}{-0.027778in}}{\pgfqpoint{0.000000in}{0.000000in}}{%
\pgfpathmoveto{\pgfqpoint{0.000000in}{0.000000in}}%
\pgfpathlineto{\pgfqpoint{0.000000in}{-0.027778in}}%
\pgfusepath{stroke,fill}%
}%
\begin{pgfscope}%
\pgfsys@transformshift{1.926611in}{0.417642in}%
\pgfsys@useobject{currentmarker}{}%
\end{pgfscope}%
\end{pgfscope}%
\begin{pgfscope}%
\pgfpathrectangle{\pgfqpoint{0.589510in}{0.417642in}}{\pgfqpoint{3.437062in}{2.055000in}}%
\pgfusepath{clip}%
\pgfsetrectcap%
\pgfsetroundjoin%
\pgfsetlinewidth{0.803000pt}%
\definecolor{currentstroke}{rgb}{0.850000,0.850000,0.850000}%
\pgfsetstrokecolor{currentstroke}%
\pgfsetdash{}{0pt}%
\pgfpathmoveto{\pgfqpoint{1.993497in}{0.417642in}}%
\pgfpathlineto{\pgfqpoint{1.993497in}{2.472642in}}%
\pgfusepath{stroke}%
\end{pgfscope}%
\begin{pgfscope}%
\pgfsetbuttcap%
\pgfsetroundjoin%
\definecolor{currentfill}{rgb}{0.000000,0.000000,0.000000}%
\pgfsetfillcolor{currentfill}%
\pgfsetlinewidth{0.602250pt}%
\definecolor{currentstroke}{rgb}{0.000000,0.000000,0.000000}%
\pgfsetstrokecolor{currentstroke}%
\pgfsetdash{}{0pt}%
\pgfsys@defobject{currentmarker}{\pgfqpoint{0.000000in}{-0.027778in}}{\pgfqpoint{0.000000in}{0.000000in}}{%
\pgfpathmoveto{\pgfqpoint{0.000000in}{0.000000in}}%
\pgfpathlineto{\pgfqpoint{0.000000in}{-0.027778in}}%
\pgfusepath{stroke,fill}%
}%
\begin{pgfscope}%
\pgfsys@transformshift{1.993497in}{0.417642in}%
\pgfsys@useobject{currentmarker}{}%
\end{pgfscope}%
\end{pgfscope}%
\begin{pgfscope}%
\pgfpathrectangle{\pgfqpoint{0.589510in}{0.417642in}}{\pgfqpoint{3.437062in}{2.055000in}}%
\pgfusepath{clip}%
\pgfsetrectcap%
\pgfsetroundjoin%
\pgfsetlinewidth{0.803000pt}%
\definecolor{currentstroke}{rgb}{0.850000,0.850000,0.850000}%
\pgfsetstrokecolor{currentstroke}%
\pgfsetdash{}{0pt}%
\pgfpathmoveto{\pgfqpoint{2.050048in}{0.417642in}}%
\pgfpathlineto{\pgfqpoint{2.050048in}{2.472642in}}%
\pgfusepath{stroke}%
\end{pgfscope}%
\begin{pgfscope}%
\pgfsetbuttcap%
\pgfsetroundjoin%
\definecolor{currentfill}{rgb}{0.000000,0.000000,0.000000}%
\pgfsetfillcolor{currentfill}%
\pgfsetlinewidth{0.602250pt}%
\definecolor{currentstroke}{rgb}{0.000000,0.000000,0.000000}%
\pgfsetstrokecolor{currentstroke}%
\pgfsetdash{}{0pt}%
\pgfsys@defobject{currentmarker}{\pgfqpoint{0.000000in}{-0.027778in}}{\pgfqpoint{0.000000in}{0.000000in}}{%
\pgfpathmoveto{\pgfqpoint{0.000000in}{0.000000in}}%
\pgfpathlineto{\pgfqpoint{0.000000in}{-0.027778in}}%
\pgfusepath{stroke,fill}%
}%
\begin{pgfscope}%
\pgfsys@transformshift{2.050048in}{0.417642in}%
\pgfsys@useobject{currentmarker}{}%
\end{pgfscope}%
\end{pgfscope}%
\begin{pgfscope}%
\pgfpathrectangle{\pgfqpoint{0.589510in}{0.417642in}}{\pgfqpoint{3.437062in}{2.055000in}}%
\pgfusepath{clip}%
\pgfsetrectcap%
\pgfsetroundjoin%
\pgfsetlinewidth{0.803000pt}%
\definecolor{currentstroke}{rgb}{0.850000,0.850000,0.850000}%
\pgfsetstrokecolor{currentstroke}%
\pgfsetdash{}{0pt}%
\pgfpathmoveto{\pgfqpoint{2.099035in}{0.417642in}}%
\pgfpathlineto{\pgfqpoint{2.099035in}{2.472642in}}%
\pgfusepath{stroke}%
\end{pgfscope}%
\begin{pgfscope}%
\pgfsetbuttcap%
\pgfsetroundjoin%
\definecolor{currentfill}{rgb}{0.000000,0.000000,0.000000}%
\pgfsetfillcolor{currentfill}%
\pgfsetlinewidth{0.602250pt}%
\definecolor{currentstroke}{rgb}{0.000000,0.000000,0.000000}%
\pgfsetstrokecolor{currentstroke}%
\pgfsetdash{}{0pt}%
\pgfsys@defobject{currentmarker}{\pgfqpoint{0.000000in}{-0.027778in}}{\pgfqpoint{0.000000in}{0.000000in}}{%
\pgfpathmoveto{\pgfqpoint{0.000000in}{0.000000in}}%
\pgfpathlineto{\pgfqpoint{0.000000in}{-0.027778in}}%
\pgfusepath{stroke,fill}%
}%
\begin{pgfscope}%
\pgfsys@transformshift{2.099035in}{0.417642in}%
\pgfsys@useobject{currentmarker}{}%
\end{pgfscope}%
\end{pgfscope}%
\begin{pgfscope}%
\pgfpathrectangle{\pgfqpoint{0.589510in}{0.417642in}}{\pgfqpoint{3.437062in}{2.055000in}}%
\pgfusepath{clip}%
\pgfsetrectcap%
\pgfsetroundjoin%
\pgfsetlinewidth{0.803000pt}%
\definecolor{currentstroke}{rgb}{0.850000,0.850000,0.850000}%
\pgfsetstrokecolor{currentstroke}%
\pgfsetdash{}{0pt}%
\pgfpathmoveto{\pgfqpoint{2.142245in}{0.417642in}}%
\pgfpathlineto{\pgfqpoint{2.142245in}{2.472642in}}%
\pgfusepath{stroke}%
\end{pgfscope}%
\begin{pgfscope}%
\pgfsetbuttcap%
\pgfsetroundjoin%
\definecolor{currentfill}{rgb}{0.000000,0.000000,0.000000}%
\pgfsetfillcolor{currentfill}%
\pgfsetlinewidth{0.602250pt}%
\definecolor{currentstroke}{rgb}{0.000000,0.000000,0.000000}%
\pgfsetstrokecolor{currentstroke}%
\pgfsetdash{}{0pt}%
\pgfsys@defobject{currentmarker}{\pgfqpoint{0.000000in}{-0.027778in}}{\pgfqpoint{0.000000in}{0.000000in}}{%
\pgfpathmoveto{\pgfqpoint{0.000000in}{0.000000in}}%
\pgfpathlineto{\pgfqpoint{0.000000in}{-0.027778in}}%
\pgfusepath{stroke,fill}%
}%
\begin{pgfscope}%
\pgfsys@transformshift{2.142245in}{0.417642in}%
\pgfsys@useobject{currentmarker}{}%
\end{pgfscope}%
\end{pgfscope}%
\begin{pgfscope}%
\pgfpathrectangle{\pgfqpoint{0.589510in}{0.417642in}}{\pgfqpoint{3.437062in}{2.055000in}}%
\pgfusepath{clip}%
\pgfsetrectcap%
\pgfsetroundjoin%
\pgfsetlinewidth{0.803000pt}%
\definecolor{currentstroke}{rgb}{0.850000,0.850000,0.850000}%
\pgfsetstrokecolor{currentstroke}%
\pgfsetdash{}{0pt}%
\pgfpathmoveto{\pgfqpoint{2.435184in}{0.417642in}}%
\pgfpathlineto{\pgfqpoint{2.435184in}{2.472642in}}%
\pgfusepath{stroke}%
\end{pgfscope}%
\begin{pgfscope}%
\pgfsetbuttcap%
\pgfsetroundjoin%
\definecolor{currentfill}{rgb}{0.000000,0.000000,0.000000}%
\pgfsetfillcolor{currentfill}%
\pgfsetlinewidth{0.602250pt}%
\definecolor{currentstroke}{rgb}{0.000000,0.000000,0.000000}%
\pgfsetstrokecolor{currentstroke}%
\pgfsetdash{}{0pt}%
\pgfsys@defobject{currentmarker}{\pgfqpoint{0.000000in}{-0.027778in}}{\pgfqpoint{0.000000in}{0.000000in}}{%
\pgfpathmoveto{\pgfqpoint{0.000000in}{0.000000in}}%
\pgfpathlineto{\pgfqpoint{0.000000in}{-0.027778in}}%
\pgfusepath{stroke,fill}%
}%
\begin{pgfscope}%
\pgfsys@transformshift{2.435184in}{0.417642in}%
\pgfsys@useobject{currentmarker}{}%
\end{pgfscope}%
\end{pgfscope}%
\begin{pgfscope}%
\pgfpathrectangle{\pgfqpoint{0.589510in}{0.417642in}}{\pgfqpoint{3.437062in}{2.055000in}}%
\pgfusepath{clip}%
\pgfsetrectcap%
\pgfsetroundjoin%
\pgfsetlinewidth{0.803000pt}%
\definecolor{currentstroke}{rgb}{0.850000,0.850000,0.850000}%
\pgfsetstrokecolor{currentstroke}%
\pgfsetdash{}{0pt}%
\pgfpathmoveto{\pgfqpoint{2.583932in}{0.417642in}}%
\pgfpathlineto{\pgfqpoint{2.583932in}{2.472642in}}%
\pgfusepath{stroke}%
\end{pgfscope}%
\begin{pgfscope}%
\pgfsetbuttcap%
\pgfsetroundjoin%
\definecolor{currentfill}{rgb}{0.000000,0.000000,0.000000}%
\pgfsetfillcolor{currentfill}%
\pgfsetlinewidth{0.602250pt}%
\definecolor{currentstroke}{rgb}{0.000000,0.000000,0.000000}%
\pgfsetstrokecolor{currentstroke}%
\pgfsetdash{}{0pt}%
\pgfsys@defobject{currentmarker}{\pgfqpoint{0.000000in}{-0.027778in}}{\pgfqpoint{0.000000in}{0.000000in}}{%
\pgfpathmoveto{\pgfqpoint{0.000000in}{0.000000in}}%
\pgfpathlineto{\pgfqpoint{0.000000in}{-0.027778in}}%
\pgfusepath{stroke,fill}%
}%
\begin{pgfscope}%
\pgfsys@transformshift{2.583932in}{0.417642in}%
\pgfsys@useobject{currentmarker}{}%
\end{pgfscope}%
\end{pgfscope}%
\begin{pgfscope}%
\pgfpathrectangle{\pgfqpoint{0.589510in}{0.417642in}}{\pgfqpoint{3.437062in}{2.055000in}}%
\pgfusepath{clip}%
\pgfsetrectcap%
\pgfsetroundjoin%
\pgfsetlinewidth{0.803000pt}%
\definecolor{currentstroke}{rgb}{0.850000,0.850000,0.850000}%
\pgfsetstrokecolor{currentstroke}%
\pgfsetdash{}{0pt}%
\pgfpathmoveto{\pgfqpoint{2.689471in}{0.417642in}}%
\pgfpathlineto{\pgfqpoint{2.689471in}{2.472642in}}%
\pgfusepath{stroke}%
\end{pgfscope}%
\begin{pgfscope}%
\pgfsetbuttcap%
\pgfsetroundjoin%
\definecolor{currentfill}{rgb}{0.000000,0.000000,0.000000}%
\pgfsetfillcolor{currentfill}%
\pgfsetlinewidth{0.602250pt}%
\definecolor{currentstroke}{rgb}{0.000000,0.000000,0.000000}%
\pgfsetstrokecolor{currentstroke}%
\pgfsetdash{}{0pt}%
\pgfsys@defobject{currentmarker}{\pgfqpoint{0.000000in}{-0.027778in}}{\pgfqpoint{0.000000in}{0.000000in}}{%
\pgfpathmoveto{\pgfqpoint{0.000000in}{0.000000in}}%
\pgfpathlineto{\pgfqpoint{0.000000in}{-0.027778in}}%
\pgfusepath{stroke,fill}%
}%
\begin{pgfscope}%
\pgfsys@transformshift{2.689471in}{0.417642in}%
\pgfsys@useobject{currentmarker}{}%
\end{pgfscope}%
\end{pgfscope}%
\begin{pgfscope}%
\pgfpathrectangle{\pgfqpoint{0.589510in}{0.417642in}}{\pgfqpoint{3.437062in}{2.055000in}}%
\pgfusepath{clip}%
\pgfsetrectcap%
\pgfsetroundjoin%
\pgfsetlinewidth{0.803000pt}%
\definecolor{currentstroke}{rgb}{0.850000,0.850000,0.850000}%
\pgfsetstrokecolor{currentstroke}%
\pgfsetdash{}{0pt}%
\pgfpathmoveto{\pgfqpoint{2.771333in}{0.417642in}}%
\pgfpathlineto{\pgfqpoint{2.771333in}{2.472642in}}%
\pgfusepath{stroke}%
\end{pgfscope}%
\begin{pgfscope}%
\pgfsetbuttcap%
\pgfsetroundjoin%
\definecolor{currentfill}{rgb}{0.000000,0.000000,0.000000}%
\pgfsetfillcolor{currentfill}%
\pgfsetlinewidth{0.602250pt}%
\definecolor{currentstroke}{rgb}{0.000000,0.000000,0.000000}%
\pgfsetstrokecolor{currentstroke}%
\pgfsetdash{}{0pt}%
\pgfsys@defobject{currentmarker}{\pgfqpoint{0.000000in}{-0.027778in}}{\pgfqpoint{0.000000in}{0.000000in}}{%
\pgfpathmoveto{\pgfqpoint{0.000000in}{0.000000in}}%
\pgfpathlineto{\pgfqpoint{0.000000in}{-0.027778in}}%
\pgfusepath{stroke,fill}%
}%
\begin{pgfscope}%
\pgfsys@transformshift{2.771333in}{0.417642in}%
\pgfsys@useobject{currentmarker}{}%
\end{pgfscope}%
\end{pgfscope}%
\begin{pgfscope}%
\pgfpathrectangle{\pgfqpoint{0.589510in}{0.417642in}}{\pgfqpoint{3.437062in}{2.055000in}}%
\pgfusepath{clip}%
\pgfsetrectcap%
\pgfsetroundjoin%
\pgfsetlinewidth{0.803000pt}%
\definecolor{currentstroke}{rgb}{0.850000,0.850000,0.850000}%
\pgfsetstrokecolor{currentstroke}%
\pgfsetdash{}{0pt}%
\pgfpathmoveto{\pgfqpoint{2.838219in}{0.417642in}}%
\pgfpathlineto{\pgfqpoint{2.838219in}{2.472642in}}%
\pgfusepath{stroke}%
\end{pgfscope}%
\begin{pgfscope}%
\pgfsetbuttcap%
\pgfsetroundjoin%
\definecolor{currentfill}{rgb}{0.000000,0.000000,0.000000}%
\pgfsetfillcolor{currentfill}%
\pgfsetlinewidth{0.602250pt}%
\definecolor{currentstroke}{rgb}{0.000000,0.000000,0.000000}%
\pgfsetstrokecolor{currentstroke}%
\pgfsetdash{}{0pt}%
\pgfsys@defobject{currentmarker}{\pgfqpoint{0.000000in}{-0.027778in}}{\pgfqpoint{0.000000in}{0.000000in}}{%
\pgfpathmoveto{\pgfqpoint{0.000000in}{0.000000in}}%
\pgfpathlineto{\pgfqpoint{0.000000in}{-0.027778in}}%
\pgfusepath{stroke,fill}%
}%
\begin{pgfscope}%
\pgfsys@transformshift{2.838219in}{0.417642in}%
\pgfsys@useobject{currentmarker}{}%
\end{pgfscope}%
\end{pgfscope}%
\begin{pgfscope}%
\pgfpathrectangle{\pgfqpoint{0.589510in}{0.417642in}}{\pgfqpoint{3.437062in}{2.055000in}}%
\pgfusepath{clip}%
\pgfsetrectcap%
\pgfsetroundjoin%
\pgfsetlinewidth{0.803000pt}%
\definecolor{currentstroke}{rgb}{0.850000,0.850000,0.850000}%
\pgfsetstrokecolor{currentstroke}%
\pgfsetdash{}{0pt}%
\pgfpathmoveto{\pgfqpoint{2.894770in}{0.417642in}}%
\pgfpathlineto{\pgfqpoint{2.894770in}{2.472642in}}%
\pgfusepath{stroke}%
\end{pgfscope}%
\begin{pgfscope}%
\pgfsetbuttcap%
\pgfsetroundjoin%
\definecolor{currentfill}{rgb}{0.000000,0.000000,0.000000}%
\pgfsetfillcolor{currentfill}%
\pgfsetlinewidth{0.602250pt}%
\definecolor{currentstroke}{rgb}{0.000000,0.000000,0.000000}%
\pgfsetstrokecolor{currentstroke}%
\pgfsetdash{}{0pt}%
\pgfsys@defobject{currentmarker}{\pgfqpoint{0.000000in}{-0.027778in}}{\pgfqpoint{0.000000in}{0.000000in}}{%
\pgfpathmoveto{\pgfqpoint{0.000000in}{0.000000in}}%
\pgfpathlineto{\pgfqpoint{0.000000in}{-0.027778in}}%
\pgfusepath{stroke,fill}%
}%
\begin{pgfscope}%
\pgfsys@transformshift{2.894770in}{0.417642in}%
\pgfsys@useobject{currentmarker}{}%
\end{pgfscope}%
\end{pgfscope}%
\begin{pgfscope}%
\pgfpathrectangle{\pgfqpoint{0.589510in}{0.417642in}}{\pgfqpoint{3.437062in}{2.055000in}}%
\pgfusepath{clip}%
\pgfsetrectcap%
\pgfsetroundjoin%
\pgfsetlinewidth{0.803000pt}%
\definecolor{currentstroke}{rgb}{0.850000,0.850000,0.850000}%
\pgfsetstrokecolor{currentstroke}%
\pgfsetdash{}{0pt}%
\pgfpathmoveto{\pgfqpoint{2.943757in}{0.417642in}}%
\pgfpathlineto{\pgfqpoint{2.943757in}{2.472642in}}%
\pgfusepath{stroke}%
\end{pgfscope}%
\begin{pgfscope}%
\pgfsetbuttcap%
\pgfsetroundjoin%
\definecolor{currentfill}{rgb}{0.000000,0.000000,0.000000}%
\pgfsetfillcolor{currentfill}%
\pgfsetlinewidth{0.602250pt}%
\definecolor{currentstroke}{rgb}{0.000000,0.000000,0.000000}%
\pgfsetstrokecolor{currentstroke}%
\pgfsetdash{}{0pt}%
\pgfsys@defobject{currentmarker}{\pgfqpoint{0.000000in}{-0.027778in}}{\pgfqpoint{0.000000in}{0.000000in}}{%
\pgfpathmoveto{\pgfqpoint{0.000000in}{0.000000in}}%
\pgfpathlineto{\pgfqpoint{0.000000in}{-0.027778in}}%
\pgfusepath{stroke,fill}%
}%
\begin{pgfscope}%
\pgfsys@transformshift{2.943757in}{0.417642in}%
\pgfsys@useobject{currentmarker}{}%
\end{pgfscope}%
\end{pgfscope}%
\begin{pgfscope}%
\pgfpathrectangle{\pgfqpoint{0.589510in}{0.417642in}}{\pgfqpoint{3.437062in}{2.055000in}}%
\pgfusepath{clip}%
\pgfsetrectcap%
\pgfsetroundjoin%
\pgfsetlinewidth{0.803000pt}%
\definecolor{currentstroke}{rgb}{0.850000,0.850000,0.850000}%
\pgfsetstrokecolor{currentstroke}%
\pgfsetdash{}{0pt}%
\pgfpathmoveto{\pgfqpoint{2.986967in}{0.417642in}}%
\pgfpathlineto{\pgfqpoint{2.986967in}{2.472642in}}%
\pgfusepath{stroke}%
\end{pgfscope}%
\begin{pgfscope}%
\pgfsetbuttcap%
\pgfsetroundjoin%
\definecolor{currentfill}{rgb}{0.000000,0.000000,0.000000}%
\pgfsetfillcolor{currentfill}%
\pgfsetlinewidth{0.602250pt}%
\definecolor{currentstroke}{rgb}{0.000000,0.000000,0.000000}%
\pgfsetstrokecolor{currentstroke}%
\pgfsetdash{}{0pt}%
\pgfsys@defobject{currentmarker}{\pgfqpoint{0.000000in}{-0.027778in}}{\pgfqpoint{0.000000in}{0.000000in}}{%
\pgfpathmoveto{\pgfqpoint{0.000000in}{0.000000in}}%
\pgfpathlineto{\pgfqpoint{0.000000in}{-0.027778in}}%
\pgfusepath{stroke,fill}%
}%
\begin{pgfscope}%
\pgfsys@transformshift{2.986967in}{0.417642in}%
\pgfsys@useobject{currentmarker}{}%
\end{pgfscope}%
\end{pgfscope}%
\begin{pgfscope}%
\pgfpathrectangle{\pgfqpoint{0.589510in}{0.417642in}}{\pgfqpoint{3.437062in}{2.055000in}}%
\pgfusepath{clip}%
\pgfsetrectcap%
\pgfsetroundjoin%
\pgfsetlinewidth{0.803000pt}%
\definecolor{currentstroke}{rgb}{0.850000,0.850000,0.850000}%
\pgfsetstrokecolor{currentstroke}%
\pgfsetdash{}{0pt}%
\pgfpathmoveto{\pgfqpoint{3.279906in}{0.417642in}}%
\pgfpathlineto{\pgfqpoint{3.279906in}{2.472642in}}%
\pgfusepath{stroke}%
\end{pgfscope}%
\begin{pgfscope}%
\pgfsetbuttcap%
\pgfsetroundjoin%
\definecolor{currentfill}{rgb}{0.000000,0.000000,0.000000}%
\pgfsetfillcolor{currentfill}%
\pgfsetlinewidth{0.602250pt}%
\definecolor{currentstroke}{rgb}{0.000000,0.000000,0.000000}%
\pgfsetstrokecolor{currentstroke}%
\pgfsetdash{}{0pt}%
\pgfsys@defobject{currentmarker}{\pgfqpoint{0.000000in}{-0.027778in}}{\pgfqpoint{0.000000in}{0.000000in}}{%
\pgfpathmoveto{\pgfqpoint{0.000000in}{0.000000in}}%
\pgfpathlineto{\pgfqpoint{0.000000in}{-0.027778in}}%
\pgfusepath{stroke,fill}%
}%
\begin{pgfscope}%
\pgfsys@transformshift{3.279906in}{0.417642in}%
\pgfsys@useobject{currentmarker}{}%
\end{pgfscope}%
\end{pgfscope}%
\begin{pgfscope}%
\pgfpathrectangle{\pgfqpoint{0.589510in}{0.417642in}}{\pgfqpoint{3.437062in}{2.055000in}}%
\pgfusepath{clip}%
\pgfsetrectcap%
\pgfsetroundjoin%
\pgfsetlinewidth{0.803000pt}%
\definecolor{currentstroke}{rgb}{0.850000,0.850000,0.850000}%
\pgfsetstrokecolor{currentstroke}%
\pgfsetdash{}{0pt}%
\pgfpathmoveto{\pgfqpoint{3.428654in}{0.417642in}}%
\pgfpathlineto{\pgfqpoint{3.428654in}{2.472642in}}%
\pgfusepath{stroke}%
\end{pgfscope}%
\begin{pgfscope}%
\pgfsetbuttcap%
\pgfsetroundjoin%
\definecolor{currentfill}{rgb}{0.000000,0.000000,0.000000}%
\pgfsetfillcolor{currentfill}%
\pgfsetlinewidth{0.602250pt}%
\definecolor{currentstroke}{rgb}{0.000000,0.000000,0.000000}%
\pgfsetstrokecolor{currentstroke}%
\pgfsetdash{}{0pt}%
\pgfsys@defobject{currentmarker}{\pgfqpoint{0.000000in}{-0.027778in}}{\pgfqpoint{0.000000in}{0.000000in}}{%
\pgfpathmoveto{\pgfqpoint{0.000000in}{0.000000in}}%
\pgfpathlineto{\pgfqpoint{0.000000in}{-0.027778in}}%
\pgfusepath{stroke,fill}%
}%
\begin{pgfscope}%
\pgfsys@transformshift{3.428654in}{0.417642in}%
\pgfsys@useobject{currentmarker}{}%
\end{pgfscope}%
\end{pgfscope}%
\begin{pgfscope}%
\pgfpathrectangle{\pgfqpoint{0.589510in}{0.417642in}}{\pgfqpoint{3.437062in}{2.055000in}}%
\pgfusepath{clip}%
\pgfsetrectcap%
\pgfsetroundjoin%
\pgfsetlinewidth{0.803000pt}%
\definecolor{currentstroke}{rgb}{0.850000,0.850000,0.850000}%
\pgfsetstrokecolor{currentstroke}%
\pgfsetdash{}{0pt}%
\pgfpathmoveto{\pgfqpoint{3.534193in}{0.417642in}}%
\pgfpathlineto{\pgfqpoint{3.534193in}{2.472642in}}%
\pgfusepath{stroke}%
\end{pgfscope}%
\begin{pgfscope}%
\pgfsetbuttcap%
\pgfsetroundjoin%
\definecolor{currentfill}{rgb}{0.000000,0.000000,0.000000}%
\pgfsetfillcolor{currentfill}%
\pgfsetlinewidth{0.602250pt}%
\definecolor{currentstroke}{rgb}{0.000000,0.000000,0.000000}%
\pgfsetstrokecolor{currentstroke}%
\pgfsetdash{}{0pt}%
\pgfsys@defobject{currentmarker}{\pgfqpoint{0.000000in}{-0.027778in}}{\pgfqpoint{0.000000in}{0.000000in}}{%
\pgfpathmoveto{\pgfqpoint{0.000000in}{0.000000in}}%
\pgfpathlineto{\pgfqpoint{0.000000in}{-0.027778in}}%
\pgfusepath{stroke,fill}%
}%
\begin{pgfscope}%
\pgfsys@transformshift{3.534193in}{0.417642in}%
\pgfsys@useobject{currentmarker}{}%
\end{pgfscope}%
\end{pgfscope}%
\begin{pgfscope}%
\pgfpathrectangle{\pgfqpoint{0.589510in}{0.417642in}}{\pgfqpoint{3.437062in}{2.055000in}}%
\pgfusepath{clip}%
\pgfsetrectcap%
\pgfsetroundjoin%
\pgfsetlinewidth{0.803000pt}%
\definecolor{currentstroke}{rgb}{0.850000,0.850000,0.850000}%
\pgfsetstrokecolor{currentstroke}%
\pgfsetdash{}{0pt}%
\pgfpathmoveto{\pgfqpoint{3.616055in}{0.417642in}}%
\pgfpathlineto{\pgfqpoint{3.616055in}{2.472642in}}%
\pgfusepath{stroke}%
\end{pgfscope}%
\begin{pgfscope}%
\pgfsetbuttcap%
\pgfsetroundjoin%
\definecolor{currentfill}{rgb}{0.000000,0.000000,0.000000}%
\pgfsetfillcolor{currentfill}%
\pgfsetlinewidth{0.602250pt}%
\definecolor{currentstroke}{rgb}{0.000000,0.000000,0.000000}%
\pgfsetstrokecolor{currentstroke}%
\pgfsetdash{}{0pt}%
\pgfsys@defobject{currentmarker}{\pgfqpoint{0.000000in}{-0.027778in}}{\pgfqpoint{0.000000in}{0.000000in}}{%
\pgfpathmoveto{\pgfqpoint{0.000000in}{0.000000in}}%
\pgfpathlineto{\pgfqpoint{0.000000in}{-0.027778in}}%
\pgfusepath{stroke,fill}%
}%
\begin{pgfscope}%
\pgfsys@transformshift{3.616055in}{0.417642in}%
\pgfsys@useobject{currentmarker}{}%
\end{pgfscope}%
\end{pgfscope}%
\begin{pgfscope}%
\pgfpathrectangle{\pgfqpoint{0.589510in}{0.417642in}}{\pgfqpoint{3.437062in}{2.055000in}}%
\pgfusepath{clip}%
\pgfsetrectcap%
\pgfsetroundjoin%
\pgfsetlinewidth{0.803000pt}%
\definecolor{currentstroke}{rgb}{0.850000,0.850000,0.850000}%
\pgfsetstrokecolor{currentstroke}%
\pgfsetdash{}{0pt}%
\pgfpathmoveto{\pgfqpoint{3.682941in}{0.417642in}}%
\pgfpathlineto{\pgfqpoint{3.682941in}{2.472642in}}%
\pgfusepath{stroke}%
\end{pgfscope}%
\begin{pgfscope}%
\pgfsetbuttcap%
\pgfsetroundjoin%
\definecolor{currentfill}{rgb}{0.000000,0.000000,0.000000}%
\pgfsetfillcolor{currentfill}%
\pgfsetlinewidth{0.602250pt}%
\definecolor{currentstroke}{rgb}{0.000000,0.000000,0.000000}%
\pgfsetstrokecolor{currentstroke}%
\pgfsetdash{}{0pt}%
\pgfsys@defobject{currentmarker}{\pgfqpoint{0.000000in}{-0.027778in}}{\pgfqpoint{0.000000in}{0.000000in}}{%
\pgfpathmoveto{\pgfqpoint{0.000000in}{0.000000in}}%
\pgfpathlineto{\pgfqpoint{0.000000in}{-0.027778in}}%
\pgfusepath{stroke,fill}%
}%
\begin{pgfscope}%
\pgfsys@transformshift{3.682941in}{0.417642in}%
\pgfsys@useobject{currentmarker}{}%
\end{pgfscope}%
\end{pgfscope}%
\begin{pgfscope}%
\pgfpathrectangle{\pgfqpoint{0.589510in}{0.417642in}}{\pgfqpoint{3.437062in}{2.055000in}}%
\pgfusepath{clip}%
\pgfsetrectcap%
\pgfsetroundjoin%
\pgfsetlinewidth{0.803000pt}%
\definecolor{currentstroke}{rgb}{0.850000,0.850000,0.850000}%
\pgfsetstrokecolor{currentstroke}%
\pgfsetdash{}{0pt}%
\pgfpathmoveto{\pgfqpoint{3.739492in}{0.417642in}}%
\pgfpathlineto{\pgfqpoint{3.739492in}{2.472642in}}%
\pgfusepath{stroke}%
\end{pgfscope}%
\begin{pgfscope}%
\pgfsetbuttcap%
\pgfsetroundjoin%
\definecolor{currentfill}{rgb}{0.000000,0.000000,0.000000}%
\pgfsetfillcolor{currentfill}%
\pgfsetlinewidth{0.602250pt}%
\definecolor{currentstroke}{rgb}{0.000000,0.000000,0.000000}%
\pgfsetstrokecolor{currentstroke}%
\pgfsetdash{}{0pt}%
\pgfsys@defobject{currentmarker}{\pgfqpoint{0.000000in}{-0.027778in}}{\pgfqpoint{0.000000in}{0.000000in}}{%
\pgfpathmoveto{\pgfqpoint{0.000000in}{0.000000in}}%
\pgfpathlineto{\pgfqpoint{0.000000in}{-0.027778in}}%
\pgfusepath{stroke,fill}%
}%
\begin{pgfscope}%
\pgfsys@transformshift{3.739492in}{0.417642in}%
\pgfsys@useobject{currentmarker}{}%
\end{pgfscope}%
\end{pgfscope}%
\begin{pgfscope}%
\pgfpathrectangle{\pgfqpoint{0.589510in}{0.417642in}}{\pgfqpoint{3.437062in}{2.055000in}}%
\pgfusepath{clip}%
\pgfsetrectcap%
\pgfsetroundjoin%
\pgfsetlinewidth{0.803000pt}%
\definecolor{currentstroke}{rgb}{0.850000,0.850000,0.850000}%
\pgfsetstrokecolor{currentstroke}%
\pgfsetdash{}{0pt}%
\pgfpathmoveto{\pgfqpoint{3.788479in}{0.417642in}}%
\pgfpathlineto{\pgfqpoint{3.788479in}{2.472642in}}%
\pgfusepath{stroke}%
\end{pgfscope}%
\begin{pgfscope}%
\pgfsetbuttcap%
\pgfsetroundjoin%
\definecolor{currentfill}{rgb}{0.000000,0.000000,0.000000}%
\pgfsetfillcolor{currentfill}%
\pgfsetlinewidth{0.602250pt}%
\definecolor{currentstroke}{rgb}{0.000000,0.000000,0.000000}%
\pgfsetstrokecolor{currentstroke}%
\pgfsetdash{}{0pt}%
\pgfsys@defobject{currentmarker}{\pgfqpoint{0.000000in}{-0.027778in}}{\pgfqpoint{0.000000in}{0.000000in}}{%
\pgfpathmoveto{\pgfqpoint{0.000000in}{0.000000in}}%
\pgfpathlineto{\pgfqpoint{0.000000in}{-0.027778in}}%
\pgfusepath{stroke,fill}%
}%
\begin{pgfscope}%
\pgfsys@transformshift{3.788479in}{0.417642in}%
\pgfsys@useobject{currentmarker}{}%
\end{pgfscope}%
\end{pgfscope}%
\begin{pgfscope}%
\pgfpathrectangle{\pgfqpoint{0.589510in}{0.417642in}}{\pgfqpoint{3.437062in}{2.055000in}}%
\pgfusepath{clip}%
\pgfsetrectcap%
\pgfsetroundjoin%
\pgfsetlinewidth{0.803000pt}%
\definecolor{currentstroke}{rgb}{0.850000,0.850000,0.850000}%
\pgfsetstrokecolor{currentstroke}%
\pgfsetdash{}{0pt}%
\pgfpathmoveto{\pgfqpoint{3.831689in}{0.417642in}}%
\pgfpathlineto{\pgfqpoint{3.831689in}{2.472642in}}%
\pgfusepath{stroke}%
\end{pgfscope}%
\begin{pgfscope}%
\pgfsetbuttcap%
\pgfsetroundjoin%
\definecolor{currentfill}{rgb}{0.000000,0.000000,0.000000}%
\pgfsetfillcolor{currentfill}%
\pgfsetlinewidth{0.602250pt}%
\definecolor{currentstroke}{rgb}{0.000000,0.000000,0.000000}%
\pgfsetstrokecolor{currentstroke}%
\pgfsetdash{}{0pt}%
\pgfsys@defobject{currentmarker}{\pgfqpoint{0.000000in}{-0.027778in}}{\pgfqpoint{0.000000in}{0.000000in}}{%
\pgfpathmoveto{\pgfqpoint{0.000000in}{0.000000in}}%
\pgfpathlineto{\pgfqpoint{0.000000in}{-0.027778in}}%
\pgfusepath{stroke,fill}%
}%
\begin{pgfscope}%
\pgfsys@transformshift{3.831689in}{0.417642in}%
\pgfsys@useobject{currentmarker}{}%
\end{pgfscope}%
\end{pgfscope}%
\begin{pgfscope}%
\definecolor{textcolor}{rgb}{0.000000,0.000000,0.000000}%
\pgfsetstrokecolor{textcolor}%
\pgfsetfillcolor{textcolor}%
\pgftext[x=2.308041in,y=0.165003in,,top]{\color{textcolor}{\rmfamily\fontsize{10.000000}{12.000000}\selectfont\catcode`\^=\active\def^{\ifmmode\sp\else\^{}\fi}\catcode`\%=\active\def%{\%}$\tau$ in \unit{\second}}}%
\end{pgfscope}%
\begin{pgfscope}%
\pgfpathrectangle{\pgfqpoint{0.589510in}{0.417642in}}{\pgfqpoint{3.437062in}{2.055000in}}%
\pgfusepath{clip}%
\pgfsetrectcap%
\pgfsetroundjoin%
\pgfsetlinewidth{0.803000pt}%
\definecolor{currentstroke}{rgb}{0.450000,0.450000,0.450000}%
\pgfsetstrokecolor{currentstroke}%
\pgfsetdash{}{0pt}%
\pgfpathmoveto{\pgfqpoint{0.589510in}{0.417642in}}%
\pgfpathlineto{\pgfqpoint{4.026572in}{0.417642in}}%
\pgfusepath{stroke}%
\end{pgfscope}%
\begin{pgfscope}%
\pgfsetbuttcap%
\pgfsetroundjoin%
\definecolor{currentfill}{rgb}{0.000000,0.000000,0.000000}%
\pgfsetfillcolor{currentfill}%
\pgfsetlinewidth{0.803000pt}%
\definecolor{currentstroke}{rgb}{0.000000,0.000000,0.000000}%
\pgfsetstrokecolor{currentstroke}%
\pgfsetdash{}{0pt}%
\pgfsys@defobject{currentmarker}{\pgfqpoint{-0.048611in}{0.000000in}}{\pgfqpoint{-0.000000in}{0.000000in}}{%
\pgfpathmoveto{\pgfqpoint{-0.000000in}{0.000000in}}%
\pgfpathlineto{\pgfqpoint{-0.048611in}{0.000000in}}%
\pgfusepath{stroke,fill}%
}%
\begin{pgfscope}%
\pgfsys@transformshift{0.589510in}{0.417642in}%
\pgfsys@useobject{currentmarker}{}%
\end{pgfscope}%
\end{pgfscope}%
\begin{pgfscope}%
\definecolor{textcolor}{rgb}{0.000000,0.000000,0.000000}%
\pgfsetstrokecolor{textcolor}%
\pgfsetfillcolor{textcolor}%
\pgftext[x=0.236114in, y=0.378489in, left, base]{\color{textcolor}{\rmfamily\fontsize{8.000000}{9.600000}\selectfont\catcode`\^=\active\def^{\ifmmode\sp\else\^{}\fi}\catcode`\%=\active\def%{\%}$\mathdefault{10^{-8}}$}}%
\end{pgfscope}%
\begin{pgfscope}%
\pgfpathrectangle{\pgfqpoint{0.589510in}{0.417642in}}{\pgfqpoint{3.437062in}{2.055000in}}%
\pgfusepath{clip}%
\pgfsetrectcap%
\pgfsetroundjoin%
\pgfsetlinewidth{0.803000pt}%
\definecolor{currentstroke}{rgb}{0.450000,0.450000,0.450000}%
\pgfsetstrokecolor{currentstroke}%
\pgfsetdash{}{0pt}%
\pgfpathmoveto{\pgfqpoint{0.589510in}{1.310720in}}%
\pgfpathlineto{\pgfqpoint{4.026572in}{1.310720in}}%
\pgfusepath{stroke}%
\end{pgfscope}%
\begin{pgfscope}%
\pgfsetbuttcap%
\pgfsetroundjoin%
\definecolor{currentfill}{rgb}{0.000000,0.000000,0.000000}%
\pgfsetfillcolor{currentfill}%
\pgfsetlinewidth{0.803000pt}%
\definecolor{currentstroke}{rgb}{0.000000,0.000000,0.000000}%
\pgfsetstrokecolor{currentstroke}%
\pgfsetdash{}{0pt}%
\pgfsys@defobject{currentmarker}{\pgfqpoint{-0.048611in}{0.000000in}}{\pgfqpoint{-0.000000in}{0.000000in}}{%
\pgfpathmoveto{\pgfqpoint{-0.000000in}{0.000000in}}%
\pgfpathlineto{\pgfqpoint{-0.048611in}{0.000000in}}%
\pgfusepath{stroke,fill}%
}%
\begin{pgfscope}%
\pgfsys@transformshift{0.589510in}{1.310720in}%
\pgfsys@useobject{currentmarker}{}%
\end{pgfscope}%
\end{pgfscope}%
\begin{pgfscope}%
\definecolor{textcolor}{rgb}{0.000000,0.000000,0.000000}%
\pgfsetstrokecolor{textcolor}%
\pgfsetfillcolor{textcolor}%
\pgftext[x=0.236114in, y=1.271567in, left, base]{\color{textcolor}{\rmfamily\fontsize{8.000000}{9.600000}\selectfont\catcode`\^=\active\def^{\ifmmode\sp\else\^{}\fi}\catcode`\%=\active\def%{\%}$\mathdefault{10^{-7}}$}}%
\end{pgfscope}%
\begin{pgfscope}%
\pgfpathrectangle{\pgfqpoint{0.589510in}{0.417642in}}{\pgfqpoint{3.437062in}{2.055000in}}%
\pgfusepath{clip}%
\pgfsetrectcap%
\pgfsetroundjoin%
\pgfsetlinewidth{0.803000pt}%
\definecolor{currentstroke}{rgb}{0.450000,0.450000,0.450000}%
\pgfsetstrokecolor{currentstroke}%
\pgfsetdash{}{0pt}%
\pgfpathmoveto{\pgfqpoint{0.589510in}{2.203798in}}%
\pgfpathlineto{\pgfqpoint{4.026572in}{2.203798in}}%
\pgfusepath{stroke}%
\end{pgfscope}%
\begin{pgfscope}%
\pgfsetbuttcap%
\pgfsetroundjoin%
\definecolor{currentfill}{rgb}{0.000000,0.000000,0.000000}%
\pgfsetfillcolor{currentfill}%
\pgfsetlinewidth{0.803000pt}%
\definecolor{currentstroke}{rgb}{0.000000,0.000000,0.000000}%
\pgfsetstrokecolor{currentstroke}%
\pgfsetdash{}{0pt}%
\pgfsys@defobject{currentmarker}{\pgfqpoint{-0.048611in}{0.000000in}}{\pgfqpoint{-0.000000in}{0.000000in}}{%
\pgfpathmoveto{\pgfqpoint{-0.000000in}{0.000000in}}%
\pgfpathlineto{\pgfqpoint{-0.048611in}{0.000000in}}%
\pgfusepath{stroke,fill}%
}%
\begin{pgfscope}%
\pgfsys@transformshift{0.589510in}{2.203798in}%
\pgfsys@useobject{currentmarker}{}%
\end{pgfscope}%
\end{pgfscope}%
\begin{pgfscope}%
\definecolor{textcolor}{rgb}{0.000000,0.000000,0.000000}%
\pgfsetstrokecolor{textcolor}%
\pgfsetfillcolor{textcolor}%
\pgftext[x=0.236114in, y=2.164645in, left, base]{\color{textcolor}{\rmfamily\fontsize{8.000000}{9.600000}\selectfont\catcode`\^=\active\def^{\ifmmode\sp\else\^{}\fi}\catcode`\%=\active\def%{\%}$\mathdefault{10^{-6}}$}}%
\end{pgfscope}%
\begin{pgfscope}%
\pgfpathrectangle{\pgfqpoint{0.589510in}{0.417642in}}{\pgfqpoint{3.437062in}{2.055000in}}%
\pgfusepath{clip}%
\pgfsetrectcap%
\pgfsetroundjoin%
\pgfsetlinewidth{0.803000pt}%
\definecolor{currentstroke}{rgb}{0.850000,0.850000,0.850000}%
\pgfsetstrokecolor{currentstroke}%
\pgfsetdash{}{0pt}%
\pgfpathmoveto{\pgfqpoint{0.589510in}{0.686485in}}%
\pgfpathlineto{\pgfqpoint{4.026572in}{0.686485in}}%
\pgfusepath{stroke}%
\end{pgfscope}%
\begin{pgfscope}%
\pgfsetbuttcap%
\pgfsetroundjoin%
\definecolor{currentfill}{rgb}{0.000000,0.000000,0.000000}%
\pgfsetfillcolor{currentfill}%
\pgfsetlinewidth{0.602250pt}%
\definecolor{currentstroke}{rgb}{0.000000,0.000000,0.000000}%
\pgfsetstrokecolor{currentstroke}%
\pgfsetdash{}{0pt}%
\pgfsys@defobject{currentmarker}{\pgfqpoint{-0.027778in}{0.000000in}}{\pgfqpoint{-0.000000in}{0.000000in}}{%
\pgfpathmoveto{\pgfqpoint{-0.000000in}{0.000000in}}%
\pgfpathlineto{\pgfqpoint{-0.027778in}{0.000000in}}%
\pgfusepath{stroke,fill}%
}%
\begin{pgfscope}%
\pgfsys@transformshift{0.589510in}{0.686485in}%
\pgfsys@useobject{currentmarker}{}%
\end{pgfscope}%
\end{pgfscope}%
\begin{pgfscope}%
\pgfpathrectangle{\pgfqpoint{0.589510in}{0.417642in}}{\pgfqpoint{3.437062in}{2.055000in}}%
\pgfusepath{clip}%
\pgfsetrectcap%
\pgfsetroundjoin%
\pgfsetlinewidth{0.803000pt}%
\definecolor{currentstroke}{rgb}{0.850000,0.850000,0.850000}%
\pgfsetstrokecolor{currentstroke}%
\pgfsetdash{}{0pt}%
\pgfpathmoveto{\pgfqpoint{0.589510in}{0.843749in}}%
\pgfpathlineto{\pgfqpoint{4.026572in}{0.843749in}}%
\pgfusepath{stroke}%
\end{pgfscope}%
\begin{pgfscope}%
\pgfsetbuttcap%
\pgfsetroundjoin%
\definecolor{currentfill}{rgb}{0.000000,0.000000,0.000000}%
\pgfsetfillcolor{currentfill}%
\pgfsetlinewidth{0.602250pt}%
\definecolor{currentstroke}{rgb}{0.000000,0.000000,0.000000}%
\pgfsetstrokecolor{currentstroke}%
\pgfsetdash{}{0pt}%
\pgfsys@defobject{currentmarker}{\pgfqpoint{-0.027778in}{0.000000in}}{\pgfqpoint{-0.000000in}{0.000000in}}{%
\pgfpathmoveto{\pgfqpoint{-0.000000in}{0.000000in}}%
\pgfpathlineto{\pgfqpoint{-0.027778in}{0.000000in}}%
\pgfusepath{stroke,fill}%
}%
\begin{pgfscope}%
\pgfsys@transformshift{0.589510in}{0.843749in}%
\pgfsys@useobject{currentmarker}{}%
\end{pgfscope}%
\end{pgfscope}%
\begin{pgfscope}%
\pgfpathrectangle{\pgfqpoint{0.589510in}{0.417642in}}{\pgfqpoint{3.437062in}{2.055000in}}%
\pgfusepath{clip}%
\pgfsetrectcap%
\pgfsetroundjoin%
\pgfsetlinewidth{0.803000pt}%
\definecolor{currentstroke}{rgb}{0.850000,0.850000,0.850000}%
\pgfsetstrokecolor{currentstroke}%
\pgfsetdash{}{0pt}%
\pgfpathmoveto{\pgfqpoint{0.589510in}{0.955329in}}%
\pgfpathlineto{\pgfqpoint{4.026572in}{0.955329in}}%
\pgfusepath{stroke}%
\end{pgfscope}%
\begin{pgfscope}%
\pgfsetbuttcap%
\pgfsetroundjoin%
\definecolor{currentfill}{rgb}{0.000000,0.000000,0.000000}%
\pgfsetfillcolor{currentfill}%
\pgfsetlinewidth{0.602250pt}%
\definecolor{currentstroke}{rgb}{0.000000,0.000000,0.000000}%
\pgfsetstrokecolor{currentstroke}%
\pgfsetdash{}{0pt}%
\pgfsys@defobject{currentmarker}{\pgfqpoint{-0.027778in}{0.000000in}}{\pgfqpoint{-0.000000in}{0.000000in}}{%
\pgfpathmoveto{\pgfqpoint{-0.000000in}{0.000000in}}%
\pgfpathlineto{\pgfqpoint{-0.027778in}{0.000000in}}%
\pgfusepath{stroke,fill}%
}%
\begin{pgfscope}%
\pgfsys@transformshift{0.589510in}{0.955329in}%
\pgfsys@useobject{currentmarker}{}%
\end{pgfscope}%
\end{pgfscope}%
\begin{pgfscope}%
\pgfpathrectangle{\pgfqpoint{0.589510in}{0.417642in}}{\pgfqpoint{3.437062in}{2.055000in}}%
\pgfusepath{clip}%
\pgfsetrectcap%
\pgfsetroundjoin%
\pgfsetlinewidth{0.803000pt}%
\definecolor{currentstroke}{rgb}{0.850000,0.850000,0.850000}%
\pgfsetstrokecolor{currentstroke}%
\pgfsetdash{}{0pt}%
\pgfpathmoveto{\pgfqpoint{0.589510in}{1.041877in}}%
\pgfpathlineto{\pgfqpoint{4.026572in}{1.041877in}}%
\pgfusepath{stroke}%
\end{pgfscope}%
\begin{pgfscope}%
\pgfsetbuttcap%
\pgfsetroundjoin%
\definecolor{currentfill}{rgb}{0.000000,0.000000,0.000000}%
\pgfsetfillcolor{currentfill}%
\pgfsetlinewidth{0.602250pt}%
\definecolor{currentstroke}{rgb}{0.000000,0.000000,0.000000}%
\pgfsetstrokecolor{currentstroke}%
\pgfsetdash{}{0pt}%
\pgfsys@defobject{currentmarker}{\pgfqpoint{-0.027778in}{0.000000in}}{\pgfqpoint{-0.000000in}{0.000000in}}{%
\pgfpathmoveto{\pgfqpoint{-0.000000in}{0.000000in}}%
\pgfpathlineto{\pgfqpoint{-0.027778in}{0.000000in}}%
\pgfusepath{stroke,fill}%
}%
\begin{pgfscope}%
\pgfsys@transformshift{0.589510in}{1.041877in}%
\pgfsys@useobject{currentmarker}{}%
\end{pgfscope}%
\end{pgfscope}%
\begin{pgfscope}%
\pgfpathrectangle{\pgfqpoint{0.589510in}{0.417642in}}{\pgfqpoint{3.437062in}{2.055000in}}%
\pgfusepath{clip}%
\pgfsetrectcap%
\pgfsetroundjoin%
\pgfsetlinewidth{0.803000pt}%
\definecolor{currentstroke}{rgb}{0.850000,0.850000,0.850000}%
\pgfsetstrokecolor{currentstroke}%
\pgfsetdash{}{0pt}%
\pgfpathmoveto{\pgfqpoint{0.589510in}{1.112592in}}%
\pgfpathlineto{\pgfqpoint{4.026572in}{1.112592in}}%
\pgfusepath{stroke}%
\end{pgfscope}%
\begin{pgfscope}%
\pgfsetbuttcap%
\pgfsetroundjoin%
\definecolor{currentfill}{rgb}{0.000000,0.000000,0.000000}%
\pgfsetfillcolor{currentfill}%
\pgfsetlinewidth{0.602250pt}%
\definecolor{currentstroke}{rgb}{0.000000,0.000000,0.000000}%
\pgfsetstrokecolor{currentstroke}%
\pgfsetdash{}{0pt}%
\pgfsys@defobject{currentmarker}{\pgfqpoint{-0.027778in}{0.000000in}}{\pgfqpoint{-0.000000in}{0.000000in}}{%
\pgfpathmoveto{\pgfqpoint{-0.000000in}{0.000000in}}%
\pgfpathlineto{\pgfqpoint{-0.027778in}{0.000000in}}%
\pgfusepath{stroke,fill}%
}%
\begin{pgfscope}%
\pgfsys@transformshift{0.589510in}{1.112592in}%
\pgfsys@useobject{currentmarker}{}%
\end{pgfscope}%
\end{pgfscope}%
\begin{pgfscope}%
\pgfpathrectangle{\pgfqpoint{0.589510in}{0.417642in}}{\pgfqpoint{3.437062in}{2.055000in}}%
\pgfusepath{clip}%
\pgfsetrectcap%
\pgfsetroundjoin%
\pgfsetlinewidth{0.803000pt}%
\definecolor{currentstroke}{rgb}{0.850000,0.850000,0.850000}%
\pgfsetstrokecolor{currentstroke}%
\pgfsetdash{}{0pt}%
\pgfpathmoveto{\pgfqpoint{0.589510in}{1.172381in}}%
\pgfpathlineto{\pgfqpoint{4.026572in}{1.172381in}}%
\pgfusepath{stroke}%
\end{pgfscope}%
\begin{pgfscope}%
\pgfsetbuttcap%
\pgfsetroundjoin%
\definecolor{currentfill}{rgb}{0.000000,0.000000,0.000000}%
\pgfsetfillcolor{currentfill}%
\pgfsetlinewidth{0.602250pt}%
\definecolor{currentstroke}{rgb}{0.000000,0.000000,0.000000}%
\pgfsetstrokecolor{currentstroke}%
\pgfsetdash{}{0pt}%
\pgfsys@defobject{currentmarker}{\pgfqpoint{-0.027778in}{0.000000in}}{\pgfqpoint{-0.000000in}{0.000000in}}{%
\pgfpathmoveto{\pgfqpoint{-0.000000in}{0.000000in}}%
\pgfpathlineto{\pgfqpoint{-0.027778in}{0.000000in}}%
\pgfusepath{stroke,fill}%
}%
\begin{pgfscope}%
\pgfsys@transformshift{0.589510in}{1.172381in}%
\pgfsys@useobject{currentmarker}{}%
\end{pgfscope}%
\end{pgfscope}%
\begin{pgfscope}%
\pgfpathrectangle{\pgfqpoint{0.589510in}{0.417642in}}{\pgfqpoint{3.437062in}{2.055000in}}%
\pgfusepath{clip}%
\pgfsetrectcap%
\pgfsetroundjoin%
\pgfsetlinewidth{0.803000pt}%
\definecolor{currentstroke}{rgb}{0.850000,0.850000,0.850000}%
\pgfsetstrokecolor{currentstroke}%
\pgfsetdash{}{0pt}%
\pgfpathmoveto{\pgfqpoint{0.589510in}{1.224172in}}%
\pgfpathlineto{\pgfqpoint{4.026572in}{1.224172in}}%
\pgfusepath{stroke}%
\end{pgfscope}%
\begin{pgfscope}%
\pgfsetbuttcap%
\pgfsetroundjoin%
\definecolor{currentfill}{rgb}{0.000000,0.000000,0.000000}%
\pgfsetfillcolor{currentfill}%
\pgfsetlinewidth{0.602250pt}%
\definecolor{currentstroke}{rgb}{0.000000,0.000000,0.000000}%
\pgfsetstrokecolor{currentstroke}%
\pgfsetdash{}{0pt}%
\pgfsys@defobject{currentmarker}{\pgfqpoint{-0.027778in}{0.000000in}}{\pgfqpoint{-0.000000in}{0.000000in}}{%
\pgfpathmoveto{\pgfqpoint{-0.000000in}{0.000000in}}%
\pgfpathlineto{\pgfqpoint{-0.027778in}{0.000000in}}%
\pgfusepath{stroke,fill}%
}%
\begin{pgfscope}%
\pgfsys@transformshift{0.589510in}{1.224172in}%
\pgfsys@useobject{currentmarker}{}%
\end{pgfscope}%
\end{pgfscope}%
\begin{pgfscope}%
\pgfpathrectangle{\pgfqpoint{0.589510in}{0.417642in}}{\pgfqpoint{3.437062in}{2.055000in}}%
\pgfusepath{clip}%
\pgfsetrectcap%
\pgfsetroundjoin%
\pgfsetlinewidth{0.803000pt}%
\definecolor{currentstroke}{rgb}{0.850000,0.850000,0.850000}%
\pgfsetstrokecolor{currentstroke}%
\pgfsetdash{}{0pt}%
\pgfpathmoveto{\pgfqpoint{0.589510in}{1.269855in}}%
\pgfpathlineto{\pgfqpoint{4.026572in}{1.269855in}}%
\pgfusepath{stroke}%
\end{pgfscope}%
\begin{pgfscope}%
\pgfsetbuttcap%
\pgfsetroundjoin%
\definecolor{currentfill}{rgb}{0.000000,0.000000,0.000000}%
\pgfsetfillcolor{currentfill}%
\pgfsetlinewidth{0.602250pt}%
\definecolor{currentstroke}{rgb}{0.000000,0.000000,0.000000}%
\pgfsetstrokecolor{currentstroke}%
\pgfsetdash{}{0pt}%
\pgfsys@defobject{currentmarker}{\pgfqpoint{-0.027778in}{0.000000in}}{\pgfqpoint{-0.000000in}{0.000000in}}{%
\pgfpathmoveto{\pgfqpoint{-0.000000in}{0.000000in}}%
\pgfpathlineto{\pgfqpoint{-0.027778in}{0.000000in}}%
\pgfusepath{stroke,fill}%
}%
\begin{pgfscope}%
\pgfsys@transformshift{0.589510in}{1.269855in}%
\pgfsys@useobject{currentmarker}{}%
\end{pgfscope}%
\end{pgfscope}%
\begin{pgfscope}%
\pgfpathrectangle{\pgfqpoint{0.589510in}{0.417642in}}{\pgfqpoint{3.437062in}{2.055000in}}%
\pgfusepath{clip}%
\pgfsetrectcap%
\pgfsetroundjoin%
\pgfsetlinewidth{0.803000pt}%
\definecolor{currentstroke}{rgb}{0.850000,0.850000,0.850000}%
\pgfsetstrokecolor{currentstroke}%
\pgfsetdash{}{0pt}%
\pgfpathmoveto{\pgfqpoint{0.589510in}{1.579563in}}%
\pgfpathlineto{\pgfqpoint{4.026572in}{1.579563in}}%
\pgfusepath{stroke}%
\end{pgfscope}%
\begin{pgfscope}%
\pgfsetbuttcap%
\pgfsetroundjoin%
\definecolor{currentfill}{rgb}{0.000000,0.000000,0.000000}%
\pgfsetfillcolor{currentfill}%
\pgfsetlinewidth{0.602250pt}%
\definecolor{currentstroke}{rgb}{0.000000,0.000000,0.000000}%
\pgfsetstrokecolor{currentstroke}%
\pgfsetdash{}{0pt}%
\pgfsys@defobject{currentmarker}{\pgfqpoint{-0.027778in}{0.000000in}}{\pgfqpoint{-0.000000in}{0.000000in}}{%
\pgfpathmoveto{\pgfqpoint{-0.000000in}{0.000000in}}%
\pgfpathlineto{\pgfqpoint{-0.027778in}{0.000000in}}%
\pgfusepath{stroke,fill}%
}%
\begin{pgfscope}%
\pgfsys@transformshift{0.589510in}{1.579563in}%
\pgfsys@useobject{currentmarker}{}%
\end{pgfscope}%
\end{pgfscope}%
\begin{pgfscope}%
\pgfpathrectangle{\pgfqpoint{0.589510in}{0.417642in}}{\pgfqpoint{3.437062in}{2.055000in}}%
\pgfusepath{clip}%
\pgfsetrectcap%
\pgfsetroundjoin%
\pgfsetlinewidth{0.803000pt}%
\definecolor{currentstroke}{rgb}{0.850000,0.850000,0.850000}%
\pgfsetstrokecolor{currentstroke}%
\pgfsetdash{}{0pt}%
\pgfpathmoveto{\pgfqpoint{0.589510in}{1.736827in}}%
\pgfpathlineto{\pgfqpoint{4.026572in}{1.736827in}}%
\pgfusepath{stroke}%
\end{pgfscope}%
\begin{pgfscope}%
\pgfsetbuttcap%
\pgfsetroundjoin%
\definecolor{currentfill}{rgb}{0.000000,0.000000,0.000000}%
\pgfsetfillcolor{currentfill}%
\pgfsetlinewidth{0.602250pt}%
\definecolor{currentstroke}{rgb}{0.000000,0.000000,0.000000}%
\pgfsetstrokecolor{currentstroke}%
\pgfsetdash{}{0pt}%
\pgfsys@defobject{currentmarker}{\pgfqpoint{-0.027778in}{0.000000in}}{\pgfqpoint{-0.000000in}{0.000000in}}{%
\pgfpathmoveto{\pgfqpoint{-0.000000in}{0.000000in}}%
\pgfpathlineto{\pgfqpoint{-0.027778in}{0.000000in}}%
\pgfusepath{stroke,fill}%
}%
\begin{pgfscope}%
\pgfsys@transformshift{0.589510in}{1.736827in}%
\pgfsys@useobject{currentmarker}{}%
\end{pgfscope}%
\end{pgfscope}%
\begin{pgfscope}%
\pgfpathrectangle{\pgfqpoint{0.589510in}{0.417642in}}{\pgfqpoint{3.437062in}{2.055000in}}%
\pgfusepath{clip}%
\pgfsetrectcap%
\pgfsetroundjoin%
\pgfsetlinewidth{0.803000pt}%
\definecolor{currentstroke}{rgb}{0.850000,0.850000,0.850000}%
\pgfsetstrokecolor{currentstroke}%
\pgfsetdash{}{0pt}%
\pgfpathmoveto{\pgfqpoint{0.589510in}{1.848407in}}%
\pgfpathlineto{\pgfqpoint{4.026572in}{1.848407in}}%
\pgfusepath{stroke}%
\end{pgfscope}%
\begin{pgfscope}%
\pgfsetbuttcap%
\pgfsetroundjoin%
\definecolor{currentfill}{rgb}{0.000000,0.000000,0.000000}%
\pgfsetfillcolor{currentfill}%
\pgfsetlinewidth{0.602250pt}%
\definecolor{currentstroke}{rgb}{0.000000,0.000000,0.000000}%
\pgfsetstrokecolor{currentstroke}%
\pgfsetdash{}{0pt}%
\pgfsys@defobject{currentmarker}{\pgfqpoint{-0.027778in}{0.000000in}}{\pgfqpoint{-0.000000in}{0.000000in}}{%
\pgfpathmoveto{\pgfqpoint{-0.000000in}{0.000000in}}%
\pgfpathlineto{\pgfqpoint{-0.027778in}{0.000000in}}%
\pgfusepath{stroke,fill}%
}%
\begin{pgfscope}%
\pgfsys@transformshift{0.589510in}{1.848407in}%
\pgfsys@useobject{currentmarker}{}%
\end{pgfscope}%
\end{pgfscope}%
\begin{pgfscope}%
\pgfpathrectangle{\pgfqpoint{0.589510in}{0.417642in}}{\pgfqpoint{3.437062in}{2.055000in}}%
\pgfusepath{clip}%
\pgfsetrectcap%
\pgfsetroundjoin%
\pgfsetlinewidth{0.803000pt}%
\definecolor{currentstroke}{rgb}{0.850000,0.850000,0.850000}%
\pgfsetstrokecolor{currentstroke}%
\pgfsetdash{}{0pt}%
\pgfpathmoveto{\pgfqpoint{0.589510in}{1.934955in}}%
\pgfpathlineto{\pgfqpoint{4.026572in}{1.934955in}}%
\pgfusepath{stroke}%
\end{pgfscope}%
\begin{pgfscope}%
\pgfsetbuttcap%
\pgfsetroundjoin%
\definecolor{currentfill}{rgb}{0.000000,0.000000,0.000000}%
\pgfsetfillcolor{currentfill}%
\pgfsetlinewidth{0.602250pt}%
\definecolor{currentstroke}{rgb}{0.000000,0.000000,0.000000}%
\pgfsetstrokecolor{currentstroke}%
\pgfsetdash{}{0pt}%
\pgfsys@defobject{currentmarker}{\pgfqpoint{-0.027778in}{0.000000in}}{\pgfqpoint{-0.000000in}{0.000000in}}{%
\pgfpathmoveto{\pgfqpoint{-0.000000in}{0.000000in}}%
\pgfpathlineto{\pgfqpoint{-0.027778in}{0.000000in}}%
\pgfusepath{stroke,fill}%
}%
\begin{pgfscope}%
\pgfsys@transformshift{0.589510in}{1.934955in}%
\pgfsys@useobject{currentmarker}{}%
\end{pgfscope}%
\end{pgfscope}%
\begin{pgfscope}%
\pgfpathrectangle{\pgfqpoint{0.589510in}{0.417642in}}{\pgfqpoint{3.437062in}{2.055000in}}%
\pgfusepath{clip}%
\pgfsetrectcap%
\pgfsetroundjoin%
\pgfsetlinewidth{0.803000pt}%
\definecolor{currentstroke}{rgb}{0.850000,0.850000,0.850000}%
\pgfsetstrokecolor{currentstroke}%
\pgfsetdash{}{0pt}%
\pgfpathmoveto{\pgfqpoint{0.589510in}{2.005670in}}%
\pgfpathlineto{\pgfqpoint{4.026572in}{2.005670in}}%
\pgfusepath{stroke}%
\end{pgfscope}%
\begin{pgfscope}%
\pgfsetbuttcap%
\pgfsetroundjoin%
\definecolor{currentfill}{rgb}{0.000000,0.000000,0.000000}%
\pgfsetfillcolor{currentfill}%
\pgfsetlinewidth{0.602250pt}%
\definecolor{currentstroke}{rgb}{0.000000,0.000000,0.000000}%
\pgfsetstrokecolor{currentstroke}%
\pgfsetdash{}{0pt}%
\pgfsys@defobject{currentmarker}{\pgfqpoint{-0.027778in}{0.000000in}}{\pgfqpoint{-0.000000in}{0.000000in}}{%
\pgfpathmoveto{\pgfqpoint{-0.000000in}{0.000000in}}%
\pgfpathlineto{\pgfqpoint{-0.027778in}{0.000000in}}%
\pgfusepath{stroke,fill}%
}%
\begin{pgfscope}%
\pgfsys@transformshift{0.589510in}{2.005670in}%
\pgfsys@useobject{currentmarker}{}%
\end{pgfscope}%
\end{pgfscope}%
\begin{pgfscope}%
\pgfpathrectangle{\pgfqpoint{0.589510in}{0.417642in}}{\pgfqpoint{3.437062in}{2.055000in}}%
\pgfusepath{clip}%
\pgfsetrectcap%
\pgfsetroundjoin%
\pgfsetlinewidth{0.803000pt}%
\definecolor{currentstroke}{rgb}{0.850000,0.850000,0.850000}%
\pgfsetstrokecolor{currentstroke}%
\pgfsetdash{}{0pt}%
\pgfpathmoveto{\pgfqpoint{0.589510in}{2.065459in}}%
\pgfpathlineto{\pgfqpoint{4.026572in}{2.065459in}}%
\pgfusepath{stroke}%
\end{pgfscope}%
\begin{pgfscope}%
\pgfsetbuttcap%
\pgfsetroundjoin%
\definecolor{currentfill}{rgb}{0.000000,0.000000,0.000000}%
\pgfsetfillcolor{currentfill}%
\pgfsetlinewidth{0.602250pt}%
\definecolor{currentstroke}{rgb}{0.000000,0.000000,0.000000}%
\pgfsetstrokecolor{currentstroke}%
\pgfsetdash{}{0pt}%
\pgfsys@defobject{currentmarker}{\pgfqpoint{-0.027778in}{0.000000in}}{\pgfqpoint{-0.000000in}{0.000000in}}{%
\pgfpathmoveto{\pgfqpoint{-0.000000in}{0.000000in}}%
\pgfpathlineto{\pgfqpoint{-0.027778in}{0.000000in}}%
\pgfusepath{stroke,fill}%
}%
\begin{pgfscope}%
\pgfsys@transformshift{0.589510in}{2.065459in}%
\pgfsys@useobject{currentmarker}{}%
\end{pgfscope}%
\end{pgfscope}%
\begin{pgfscope}%
\pgfpathrectangle{\pgfqpoint{0.589510in}{0.417642in}}{\pgfqpoint{3.437062in}{2.055000in}}%
\pgfusepath{clip}%
\pgfsetrectcap%
\pgfsetroundjoin%
\pgfsetlinewidth{0.803000pt}%
\definecolor{currentstroke}{rgb}{0.850000,0.850000,0.850000}%
\pgfsetstrokecolor{currentstroke}%
\pgfsetdash{}{0pt}%
\pgfpathmoveto{\pgfqpoint{0.589510in}{2.117250in}}%
\pgfpathlineto{\pgfqpoint{4.026572in}{2.117250in}}%
\pgfusepath{stroke}%
\end{pgfscope}%
\begin{pgfscope}%
\pgfsetbuttcap%
\pgfsetroundjoin%
\definecolor{currentfill}{rgb}{0.000000,0.000000,0.000000}%
\pgfsetfillcolor{currentfill}%
\pgfsetlinewidth{0.602250pt}%
\definecolor{currentstroke}{rgb}{0.000000,0.000000,0.000000}%
\pgfsetstrokecolor{currentstroke}%
\pgfsetdash{}{0pt}%
\pgfsys@defobject{currentmarker}{\pgfqpoint{-0.027778in}{0.000000in}}{\pgfqpoint{-0.000000in}{0.000000in}}{%
\pgfpathmoveto{\pgfqpoint{-0.000000in}{0.000000in}}%
\pgfpathlineto{\pgfqpoint{-0.027778in}{0.000000in}}%
\pgfusepath{stroke,fill}%
}%
\begin{pgfscope}%
\pgfsys@transformshift{0.589510in}{2.117250in}%
\pgfsys@useobject{currentmarker}{}%
\end{pgfscope}%
\end{pgfscope}%
\begin{pgfscope}%
\pgfpathrectangle{\pgfqpoint{0.589510in}{0.417642in}}{\pgfqpoint{3.437062in}{2.055000in}}%
\pgfusepath{clip}%
\pgfsetrectcap%
\pgfsetroundjoin%
\pgfsetlinewidth{0.803000pt}%
\definecolor{currentstroke}{rgb}{0.850000,0.850000,0.850000}%
\pgfsetstrokecolor{currentstroke}%
\pgfsetdash{}{0pt}%
\pgfpathmoveto{\pgfqpoint{0.589510in}{2.162933in}}%
\pgfpathlineto{\pgfqpoint{4.026572in}{2.162933in}}%
\pgfusepath{stroke}%
\end{pgfscope}%
\begin{pgfscope}%
\pgfsetbuttcap%
\pgfsetroundjoin%
\definecolor{currentfill}{rgb}{0.000000,0.000000,0.000000}%
\pgfsetfillcolor{currentfill}%
\pgfsetlinewidth{0.602250pt}%
\definecolor{currentstroke}{rgb}{0.000000,0.000000,0.000000}%
\pgfsetstrokecolor{currentstroke}%
\pgfsetdash{}{0pt}%
\pgfsys@defobject{currentmarker}{\pgfqpoint{-0.027778in}{0.000000in}}{\pgfqpoint{-0.000000in}{0.000000in}}{%
\pgfpathmoveto{\pgfqpoint{-0.000000in}{0.000000in}}%
\pgfpathlineto{\pgfqpoint{-0.027778in}{0.000000in}}%
\pgfusepath{stroke,fill}%
}%
\begin{pgfscope}%
\pgfsys@transformshift{0.589510in}{2.162933in}%
\pgfsys@useobject{currentmarker}{}%
\end{pgfscope}%
\end{pgfscope}%
\begin{pgfscope}%
\pgfpathrectangle{\pgfqpoint{0.589510in}{0.417642in}}{\pgfqpoint{3.437062in}{2.055000in}}%
\pgfusepath{clip}%
\pgfsetrectcap%
\pgfsetroundjoin%
\pgfsetlinewidth{0.803000pt}%
\definecolor{currentstroke}{rgb}{0.850000,0.850000,0.850000}%
\pgfsetstrokecolor{currentstroke}%
\pgfsetdash{}{0pt}%
\pgfpathmoveto{\pgfqpoint{0.589510in}{2.472642in}}%
\pgfpathlineto{\pgfqpoint{4.026572in}{2.472642in}}%
\pgfusepath{stroke}%
\end{pgfscope}%
\begin{pgfscope}%
\pgfsetbuttcap%
\pgfsetroundjoin%
\definecolor{currentfill}{rgb}{0.000000,0.000000,0.000000}%
\pgfsetfillcolor{currentfill}%
\pgfsetlinewidth{0.602250pt}%
\definecolor{currentstroke}{rgb}{0.000000,0.000000,0.000000}%
\pgfsetstrokecolor{currentstroke}%
\pgfsetdash{}{0pt}%
\pgfsys@defobject{currentmarker}{\pgfqpoint{-0.027778in}{0.000000in}}{\pgfqpoint{-0.000000in}{0.000000in}}{%
\pgfpathmoveto{\pgfqpoint{-0.000000in}{0.000000in}}%
\pgfpathlineto{\pgfqpoint{-0.027778in}{0.000000in}}%
\pgfusepath{stroke,fill}%
}%
\begin{pgfscope}%
\pgfsys@transformshift{0.589510in}{2.472642in}%
\pgfsys@useobject{currentmarker}{}%
\end{pgfscope}%
\end{pgfscope}%
\begin{pgfscope}%
\definecolor{textcolor}{rgb}{0.000000,0.000000,0.000000}%
\pgfsetstrokecolor{textcolor}%
\pgfsetfillcolor{textcolor}%
\pgftext[x=0.180559in,y=1.445142in,,bottom,rotate=90.000000]{\color{textcolor}{\rmfamily\fontsize{10.000000}{12.000000}\selectfont\catcode`\^=\active\def^{\ifmmode\sp\else\^{}\fi}\catcode`\%=\active\def%{\%}ADEV $\sigma_A(\tau)$ in \unit{\V}}}%
\end{pgfscope}%
\begin{pgfscope}%
\pgfpathrectangle{\pgfqpoint{0.589510in}{0.417642in}}{\pgfqpoint{3.437062in}{2.055000in}}%
\pgfusepath{clip}%
\pgfsetrectcap%
\pgfsetroundjoin%
\pgfsetlinewidth{1.505625pt}%
\definecolor{currentstroke}{rgb}{0.121569,0.466667,0.705882}%
\pgfsetstrokecolor{currentstroke}%
\pgfsetdash{}{0pt}%
\pgfpathmoveto{\pgfqpoint{1.254313in}{2.293932in}}%
\pgfpathlineto{\pgfqpoint{1.508600in}{2.157806in}}%
\pgfpathlineto{\pgfqpoint{1.657348in}{2.077885in}}%
\pgfpathlineto{\pgfqpoint{1.762887in}{2.021270in}}%
\pgfpathlineto{\pgfqpoint{1.844749in}{1.977506in}}%
\pgfpathlineto{\pgfqpoint{1.911635in}{1.941868in}}%
\pgfpathlineto{\pgfqpoint{1.968186in}{1.911872in}}%
\pgfpathlineto{\pgfqpoint{2.060383in}{1.863285in}}%
\pgfpathlineto{\pgfqpoint{2.134001in}{1.824417in}}%
\pgfpathlineto{\pgfqpoint{2.195286in}{1.792052in}}%
\pgfpathlineto{\pgfqpoint{2.271460in}{1.751428in}}%
\pgfpathlineto{\pgfqpoint{2.334505in}{1.717693in}}%
\pgfpathlineto{\pgfqpoint{2.420208in}{1.671839in}}%
\pgfpathlineto{\pgfqpoint{2.489633in}{1.635207in}}%
\pgfpathlineto{\pgfqpoint{2.558622in}{1.599126in}}%
\pgfpathlineto{\pgfqpoint{2.625508in}{1.564015in}}%
\pgfpathlineto{\pgfqpoint{2.696735in}{1.526738in}}%
\pgfpathlineto{\pgfqpoint{2.762421in}{1.491641in}}%
\pgfpathlineto{\pgfqpoint{2.833295in}{1.454269in}}%
\pgfpathlineto{\pgfqpoint{2.901006in}{1.419321in}}%
\pgfpathlineto{\pgfqpoint{2.971991in}{1.382332in}}%
\pgfpathlineto{\pgfqpoint{3.042819in}{1.345038in}}%
\pgfpathlineto{\pgfqpoint{3.111567in}{1.307763in}}%
\pgfpathlineto{\pgfqpoint{3.179227in}{1.271807in}}%
\pgfpathlineto{\pgfqpoint{3.249317in}{1.235595in}}%
\pgfpathlineto{\pgfqpoint{3.317530in}{1.200622in}}%
\pgfpathlineto{\pgfqpoint{3.387274in}{1.163963in}}%
\pgfpathlineto{\pgfqpoint{3.455981in}{1.126154in}}%
\pgfpathlineto{\pgfqpoint{3.525281in}{1.089118in}}%
\pgfpathlineto{\pgfqpoint{3.594291in}{1.053179in}}%
\pgfpathlineto{\pgfqpoint{3.663350in}{1.016958in}}%
\pgfpathlineto{\pgfqpoint{3.732295in}{0.978817in}}%
\pgfpathlineto{\pgfqpoint{3.801100in}{0.941523in}}%
\pgfpathlineto{\pgfqpoint{3.870342in}{0.904478in}}%
\pgfusepath{stroke}%
\end{pgfscope}%
\begin{pgfscope}%
\pgfpathrectangle{\pgfqpoint{0.589510in}{0.417642in}}{\pgfqpoint{3.437062in}{2.055000in}}%
\pgfusepath{clip}%
\pgfsetrectcap%
\pgfsetroundjoin%
\pgfsetlinewidth{1.505625pt}%
\definecolor{currentstroke}{rgb}{1.000000,0.498039,0.054902}%
\pgfsetstrokecolor{currentstroke}%
\pgfsetdash{}{0pt}%
\pgfpathmoveto{\pgfqpoint{1.403061in}{2.174186in}}%
\pgfpathlineto{\pgfqpoint{1.657348in}{2.036924in}}%
\pgfpathlineto{\pgfqpoint{1.806096in}{1.957081in}}%
\pgfpathlineto{\pgfqpoint{1.911635in}{1.900275in}}%
\pgfpathlineto{\pgfqpoint{1.993497in}{1.855942in}}%
\pgfpathlineto{\pgfqpoint{2.060383in}{1.819785in}}%
\pgfpathlineto{\pgfqpoint{2.116934in}{1.789443in}}%
\pgfpathlineto{\pgfqpoint{2.209131in}{1.740031in}}%
\pgfpathlineto{\pgfqpoint{2.282749in}{1.700346in}}%
\pgfpathlineto{\pgfqpoint{2.344034in}{1.667391in}}%
\pgfpathlineto{\pgfqpoint{2.420208in}{1.626507in}}%
\pgfpathlineto{\pgfqpoint{2.483253in}{1.592803in}}%
\pgfpathlineto{\pgfqpoint{2.553343in}{1.555716in}}%
\pgfpathlineto{\pgfqpoint{2.625508in}{1.517146in}}%
\pgfpathlineto{\pgfqpoint{2.696735in}{1.478605in}}%
\pgfpathlineto{\pgfqpoint{2.765416in}{1.441666in}}%
\pgfpathlineto{\pgfqpoint{2.830807in}{1.406918in}}%
\pgfpathlineto{\pgfqpoint{2.898939in}{1.371065in}}%
\pgfpathlineto{\pgfqpoint{2.971991in}{1.333631in}}%
\pgfpathlineto{\pgfqpoint{3.041416in}{1.297878in}}%
\pgfpathlineto{\pgfqpoint{3.110405in}{1.262285in}}%
\pgfpathlineto{\pgfqpoint{3.180191in}{1.226774in}}%
\pgfpathlineto{\pgfqpoint{3.248518in}{1.191990in}}%
\pgfpathlineto{\pgfqpoint{3.318191in}{1.155717in}}%
\pgfpathlineto{\pgfqpoint{3.386726in}{1.118962in}}%
\pgfpathlineto{\pgfqpoint{3.455527in}{1.080317in}}%
\pgfpathlineto{\pgfqpoint{3.524905in}{1.040580in}}%
\pgfpathlineto{\pgfqpoint{3.593667in}{1.003186in}}%
\pgfpathlineto{\pgfqpoint{3.663350in}{0.968221in}}%
\pgfpathlineto{\pgfqpoint{3.732295in}{0.935784in}}%
\pgfpathlineto{\pgfqpoint{3.801100in}{0.904347in}}%
\pgfpathlineto{\pgfqpoint{3.870195in}{0.870741in}}%
\pgfusepath{stroke}%
\end{pgfscope}%
\begin{pgfscope}%
\pgfpathrectangle{\pgfqpoint{0.589510in}{0.417642in}}{\pgfqpoint{3.437062in}{2.055000in}}%
\pgfusepath{clip}%
\pgfsetrectcap%
\pgfsetroundjoin%
\pgfsetlinewidth{1.505625pt}%
\definecolor{currentstroke}{rgb}{0.172549,0.627451,0.172549}%
\pgfsetstrokecolor{currentstroke}%
\pgfsetdash{}{0pt}%
\pgfpathmoveto{\pgfqpoint{1.657348in}{2.035709in}}%
\pgfpathlineto{\pgfqpoint{1.911635in}{1.893571in}}%
\pgfpathlineto{\pgfqpoint{2.060383in}{1.810949in}}%
\pgfpathlineto{\pgfqpoint{2.165921in}{1.752614in}}%
\pgfpathlineto{\pgfqpoint{2.247783in}{1.707360in}}%
\pgfpathlineto{\pgfqpoint{2.314670in}{1.670536in}}%
\pgfpathlineto{\pgfqpoint{2.420208in}{1.612553in}}%
\pgfpathlineto{\pgfqpoint{2.463418in}{1.589045in}}%
\pgfpathlineto{\pgfqpoint{2.537035in}{1.549028in}}%
\pgfpathlineto{\pgfqpoint{2.625508in}{1.501398in}}%
\pgfpathlineto{\pgfqpoint{2.696735in}{1.463544in}}%
\pgfpathlineto{\pgfqpoint{2.756357in}{1.432148in}}%
\pgfpathlineto{\pgfqpoint{2.823243in}{1.397196in}}%
\pgfpathlineto{\pgfqpoint{2.892668in}{1.360536in}}%
\pgfpathlineto{\pgfqpoint{2.971991in}{1.317711in}}%
\pgfpathlineto{\pgfqpoint{3.037175in}{1.281881in}}%
\pgfpathlineto{\pgfqpoint{3.106894in}{1.244447in}}%
\pgfpathlineto{\pgfqpoint{3.177291in}{1.207672in}}%
\pgfpathlineto{\pgfqpoint{3.246113in}{1.172559in}}%
\pgfpathlineto{\pgfqpoint{3.316203in}{1.135499in}}%
\pgfpathlineto{\pgfqpoint{3.385078in}{1.097201in}}%
\pgfpathlineto{\pgfqpoint{3.454160in}{1.057084in}}%
\pgfpathlineto{\pgfqpoint{3.523774in}{1.017894in}}%
\pgfpathlineto{\pgfqpoint{3.593667in}{0.982493in}}%
\pgfpathlineto{\pgfqpoint{3.663350in}{0.947499in}}%
\pgfpathlineto{\pgfqpoint{3.732295in}{0.914827in}}%
\pgfpathlineto{\pgfqpoint{3.801100in}{0.880356in}}%
\pgfpathlineto{\pgfqpoint{3.869754in}{0.843899in}}%
\pgfusepath{stroke}%
\end{pgfscope}%
\begin{pgfscope}%
\pgfpathrectangle{\pgfqpoint{0.589510in}{0.417642in}}{\pgfqpoint{3.437062in}{2.055000in}}%
\pgfusepath{clip}%
\pgfsetrectcap%
\pgfsetroundjoin%
\pgfsetlinewidth{1.505625pt}%
\definecolor{currentstroke}{rgb}{0.839216,0.152941,0.156863}%
\pgfsetstrokecolor{currentstroke}%
\pgfsetdash{}{0pt}%
\pgfpathmoveto{\pgfqpoint{1.879714in}{1.953267in}}%
\pgfpathlineto{\pgfqpoint{2.134001in}{1.805094in}}%
\pgfpathlineto{\pgfqpoint{2.282749in}{1.719998in}}%
\pgfpathlineto{\pgfqpoint{2.388287in}{1.660587in}}%
\pgfpathlineto{\pgfqpoint{2.470149in}{1.614897in}}%
\pgfpathlineto{\pgfqpoint{2.537035in}{1.577575in}}%
\pgfpathlineto{\pgfqpoint{2.593587in}{1.546047in}}%
\pgfpathlineto{\pgfqpoint{2.685784in}{1.494734in}}%
\pgfpathlineto{\pgfqpoint{2.759401in}{1.454164in}}%
\pgfpathlineto{\pgfqpoint{2.820686in}{1.420765in}}%
\pgfpathlineto{\pgfqpoint{2.896861in}{1.379611in}}%
\pgfpathlineto{\pgfqpoint{2.959905in}{1.346091in}}%
\pgfpathlineto{\pgfqpoint{3.029996in}{1.308669in}}%
\pgfpathlineto{\pgfqpoint{3.102160in}{1.271064in}}%
\pgfpathlineto{\pgfqpoint{3.173388in}{1.233264in}}%
\pgfpathlineto{\pgfqpoint{3.242068in}{1.196509in}}%
\pgfpathlineto{\pgfqpoint{3.314871in}{1.156595in}}%
\pgfpathlineto{\pgfqpoint{3.381758in}{1.117987in}}%
\pgfpathlineto{\pgfqpoint{3.453704in}{1.077483in}}%
\pgfpathlineto{\pgfqpoint{3.522261in}{1.039499in}}%
\pgfpathlineto{\pgfqpoint{3.593979in}{1.002657in}}%
\pgfpathlineto{\pgfqpoint{3.662576in}{0.968528in}}%
\pgfpathlineto{\pgfqpoint{3.732295in}{0.932719in}}%
\pgfpathlineto{\pgfqpoint{3.800745in}{0.896407in}}%
\pgfpathlineto{\pgfqpoint{3.869901in}{0.858869in}}%
\pgfusepath{stroke}%
\end{pgfscope}%
\begin{pgfscope}%
\pgfpathrectangle{\pgfqpoint{0.589510in}{0.417642in}}{\pgfqpoint{3.437062in}{2.055000in}}%
\pgfusepath{clip}%
\pgfsetrectcap%
\pgfsetroundjoin%
\pgfsetlinewidth{1.505625pt}%
\definecolor{currentstroke}{rgb}{0.580392,0.403922,0.741176}%
\pgfsetstrokecolor{currentstroke}%
\pgfsetdash{}{0pt}%
\pgfpathmoveto{\pgfqpoint{2.116934in}{1.891593in}}%
\pgfpathlineto{\pgfqpoint{2.371221in}{1.739536in}}%
\pgfpathlineto{\pgfqpoint{2.519969in}{1.652019in}}%
\pgfpathlineto{\pgfqpoint{2.625508in}{1.589857in}}%
\pgfpathlineto{\pgfqpoint{2.707370in}{1.541466in}}%
\pgfpathlineto{\pgfqpoint{2.830807in}{1.470029in}}%
\pgfpathlineto{\pgfqpoint{2.879794in}{1.442174in}}%
\pgfpathlineto{\pgfqpoint{2.961656in}{1.396434in}}%
\pgfpathlineto{\pgfqpoint{3.028543in}{1.359747in}}%
\pgfpathlineto{\pgfqpoint{3.110405in}{1.315532in}}%
\pgfpathlineto{\pgfqpoint{3.177291in}{1.278959in}}%
\pgfpathlineto{\pgfqpoint{3.233842in}{1.249007in}}%
\pgfpathlineto{\pgfqpoint{3.312194in}{1.206584in}}%
\pgfpathlineto{\pgfqpoint{3.376721in}{1.172061in}}%
\pgfpathlineto{\pgfqpoint{3.451412in}{1.132550in}}%
\pgfpathlineto{\pgfqpoint{3.521503in}{1.095760in}}%
\pgfpathlineto{\pgfqpoint{3.593667in}{1.060493in}}%
\pgfpathlineto{\pgfqpoint{3.659460in}{1.026024in}}%
\pgfpathlineto{\pgfqpoint{3.729074in}{0.986495in}}%
\pgfpathlineto{\pgfqpoint{3.798967in}{0.944661in}}%
\pgfpathlineto{\pgfqpoint{3.870195in}{0.900558in}}%
\pgfusepath{stroke}%
\end{pgfscope}%
\begin{pgfscope}%
\pgfpathrectangle{\pgfqpoint{0.589510in}{0.417642in}}{\pgfqpoint{3.437062in}{2.055000in}}%
\pgfusepath{clip}%
\pgfsetrectcap%
\pgfsetroundjoin%
\pgfsetlinewidth{1.505625pt}%
\definecolor{currentstroke}{rgb}{0.549020,0.337255,0.294118}%
\pgfsetstrokecolor{currentstroke}%
\pgfsetdash{}{0pt}%
\pgfpathmoveto{\pgfqpoint{2.442449in}{1.847468in}}%
\pgfpathlineto{\pgfqpoint{2.696735in}{1.687470in}}%
\pgfpathlineto{\pgfqpoint{2.845484in}{1.595826in}}%
\pgfpathlineto{\pgfqpoint{2.951022in}{1.531059in}}%
\pgfpathlineto{\pgfqpoint{3.032884in}{1.481062in}}%
\pgfpathlineto{\pgfqpoint{3.099770in}{1.441495in}}%
\pgfpathlineto{\pgfqpoint{3.156322in}{1.407825in}}%
\pgfpathlineto{\pgfqpoint{3.248518in}{1.352909in}}%
\pgfpathlineto{\pgfqpoint{3.287171in}{1.329471in}}%
\pgfpathlineto{\pgfqpoint{3.383421in}{1.270862in}}%
\pgfpathlineto{\pgfqpoint{3.435919in}{1.240667in}}%
\pgfpathlineto{\pgfqpoint{3.522640in}{1.194014in}}%
\pgfpathlineto{\pgfqpoint{3.592730in}{1.156373in}}%
\pgfpathlineto{\pgfqpoint{3.651553in}{1.126412in}}%
\pgfpathlineto{\pgfqpoint{3.725171in}{1.089535in}}%
\pgfpathlineto{\pgfqpoint{3.795744in}{1.054580in}}%
\pgfpathlineto{\pgfqpoint{3.870195in}{1.017519in}}%
\pgfusepath{stroke}%
\end{pgfscope}%
\begin{pgfscope}%
\pgfpathrectangle{\pgfqpoint{0.589510in}{0.417642in}}{\pgfqpoint{3.437062in}{2.055000in}}%
\pgfusepath{clip}%
\pgfsetbuttcap%
\pgfsetroundjoin%
\pgfsetlinewidth{1.505625pt}%
\definecolor{currentstroke}{rgb}{0.890196,0.466667,0.760784}%
\pgfsetstrokecolor{currentstroke}%
\pgfsetdash{{5.550000pt}{2.400000pt}}{0.000000pt}%
\pgfpathmoveto{\pgfqpoint{0.745740in}{2.151346in}}%
\pgfpathlineto{\pgfqpoint{1.000026in}{2.028823in}}%
\pgfpathlineto{\pgfqpoint{1.148775in}{1.962259in}}%
\pgfpathlineto{\pgfqpoint{1.254313in}{1.918227in}}%
\pgfpathlineto{\pgfqpoint{1.336175in}{1.886150in}}%
\pgfpathlineto{\pgfqpoint{1.403061in}{1.861524in}}%
\pgfpathlineto{\pgfqpoint{1.508600in}{1.825772in}}%
\pgfpathlineto{\pgfqpoint{1.590462in}{1.800651in}}%
\pgfpathlineto{\pgfqpoint{1.657348in}{1.781913in}}%
\pgfpathlineto{\pgfqpoint{1.713899in}{1.767339in}}%
\pgfpathlineto{\pgfqpoint{1.785127in}{1.750670in}}%
\pgfpathlineto{\pgfqpoint{1.862648in}{1.734673in}}%
\pgfpathlineto{\pgfqpoint{1.926611in}{1.722957in}}%
\pgfpathlineto{\pgfqpoint{2.005526in}{1.710315in}}%
\pgfpathlineto{\pgfqpoint{2.070434in}{1.701278in}}%
\pgfpathlineto{\pgfqpoint{2.142245in}{1.692649in}}%
\pgfpathlineto{\pgfqpoint{2.209131in}{1.685621in}}%
\pgfpathlineto{\pgfqpoint{2.282749in}{1.678762in}}%
\pgfpathlineto{\pgfqpoint{2.348707in}{1.673282in}}%
\pgfpathlineto{\pgfqpoint{2.420208in}{1.668169in}}%
\pgfpathlineto{\pgfqpoint{2.489633in}{1.664113in}}%
\pgfpathlineto{\pgfqpoint{2.558622in}{1.660832in}}%
\pgfpathlineto{\pgfqpoint{2.627685in}{1.658276in}}%
\pgfpathlineto{\pgfqpoint{2.696735in}{1.656052in}}%
\pgfpathlineto{\pgfqpoint{2.766904in}{1.654249in}}%
\pgfpathlineto{\pgfqpoint{2.835765in}{1.652686in}}%
\pgfpathlineto{\pgfqpoint{2.904085in}{1.651415in}}%
\pgfpathlineto{\pgfqpoint{2.973686in}{1.650228in}}%
\pgfpathlineto{\pgfqpoint{3.042819in}{1.648969in}}%
\pgfpathlineto{\pgfqpoint{3.111567in}{1.647631in}}%
\pgfpathlineto{\pgfqpoint{3.180672in}{1.646110in}}%
\pgfpathlineto{\pgfqpoint{3.249715in}{1.644785in}}%
\pgfpathlineto{\pgfqpoint{3.318522in}{1.644018in}}%
\pgfpathlineto{\pgfqpoint{3.387548in}{1.643637in}}%
\pgfpathlineto{\pgfqpoint{3.456435in}{1.643197in}}%
\pgfpathlineto{\pgfqpoint{3.525469in}{1.642904in}}%
\pgfpathlineto{\pgfqpoint{3.594446in}{1.642354in}}%
\pgfpathlineto{\pgfqpoint{3.663350in}{1.642184in}}%
\pgfpathlineto{\pgfqpoint{3.732402in}{1.642802in}}%
\pgfpathlineto{\pgfqpoint{3.801366in}{1.642854in}}%
\pgfpathlineto{\pgfqpoint{3.870342in}{1.641794in}}%
\pgfusepath{stroke}%
\end{pgfscope}%
\begin{pgfscope}%
\pgfsetrectcap%
\pgfsetmiterjoin%
\pgfsetlinewidth{0.803000pt}%
\definecolor{currentstroke}{rgb}{0.000000,0.000000,0.000000}%
\pgfsetstrokecolor{currentstroke}%
\pgfsetdash{}{0pt}%
\pgfpathmoveto{\pgfqpoint{0.589510in}{0.417642in}}%
\pgfpathlineto{\pgfqpoint{0.589510in}{2.472642in}}%
\pgfusepath{stroke}%
\end{pgfscope}%
\begin{pgfscope}%
\pgfsetrectcap%
\pgfsetmiterjoin%
\pgfsetlinewidth{0.803000pt}%
\definecolor{currentstroke}{rgb}{0.000000,0.000000,0.000000}%
\pgfsetstrokecolor{currentstroke}%
\pgfsetdash{}{0pt}%
\pgfpathmoveto{\pgfqpoint{4.026572in}{0.417642in}}%
\pgfpathlineto{\pgfqpoint{4.026572in}{2.472642in}}%
\pgfusepath{stroke}%
\end{pgfscope}%
\begin{pgfscope}%
\pgfsetrectcap%
\pgfsetmiterjoin%
\pgfsetlinewidth{0.803000pt}%
\definecolor{currentstroke}{rgb}{0.000000,0.000000,0.000000}%
\pgfsetstrokecolor{currentstroke}%
\pgfsetdash{}{0pt}%
\pgfpathmoveto{\pgfqpoint{0.589510in}{0.417642in}}%
\pgfpathlineto{\pgfqpoint{4.026572in}{0.417642in}}%
\pgfusepath{stroke}%
\end{pgfscope}%
\begin{pgfscope}%
\pgfsetrectcap%
\pgfsetmiterjoin%
\pgfsetlinewidth{0.803000pt}%
\definecolor{currentstroke}{rgb}{0.000000,0.000000,0.000000}%
\pgfsetstrokecolor{currentstroke}%
\pgfsetdash{}{0pt}%
\pgfpathmoveto{\pgfqpoint{0.589510in}{2.472642in}}%
\pgfpathlineto{\pgfqpoint{4.026572in}{2.472642in}}%
\pgfusepath{stroke}%
\end{pgfscope}%
\begin{pgfscope}%
\pgfsetbuttcap%
\pgfsetmiterjoin%
\definecolor{currentfill}{rgb}{1.000000,1.000000,1.000000}%
\pgfsetfillcolor{currentfill}%
\pgfsetfillopacity{0.800000}%
\pgfsetlinewidth{1.003750pt}%
\definecolor{currentstroke}{rgb}{0.800000,0.800000,0.800000}%
\pgfsetstrokecolor{currentstroke}%
\pgfsetstrokeopacity{0.800000}%
\pgfsetdash{}{0pt}%
\pgfpathmoveto{\pgfqpoint{3.108127in}{1.454420in}}%
\pgfpathlineto{\pgfqpoint{3.948794in}{1.454420in}}%
\pgfpathquadraticcurveto{\pgfqpoint{3.971016in}{1.454420in}}{\pgfqpoint{3.971016in}{1.476642in}}%
\pgfpathlineto{\pgfqpoint{3.971016in}{2.394864in}}%
\pgfpathquadraticcurveto{\pgfqpoint{3.971016in}{2.417086in}}{\pgfqpoint{3.948794in}{2.417086in}}%
\pgfpathlineto{\pgfqpoint{3.108127in}{2.417086in}}%
\pgfpathquadraticcurveto{\pgfqpoint{3.085905in}{2.417086in}}{\pgfqpoint{3.085905in}{2.394864in}}%
\pgfpathlineto{\pgfqpoint{3.085905in}{1.476642in}}%
\pgfpathquadraticcurveto{\pgfqpoint{3.085905in}{1.454420in}}{\pgfqpoint{3.108127in}{1.454420in}}%
\pgfpathlineto{\pgfqpoint{3.108127in}{1.454420in}}%
\pgfpathclose%
\pgfusepath{stroke,fill}%
\end{pgfscope}%
\begin{pgfscope}%
\pgfsetrectcap%
\pgfsetroundjoin%
\pgfsetlinewidth{1.505625pt}%
\definecolor{currentstroke}{rgb}{0.121569,0.466667,0.705882}%
\pgfsetstrokecolor{currentstroke}%
\pgfsetdash{}{0pt}%
\pgfpathmoveto{\pgfqpoint{3.130349in}{2.333753in}}%
\pgfpathlineto{\pgfqpoint{3.241460in}{2.333753in}}%
\pgfpathlineto{\pgfqpoint{3.352571in}{2.333753in}}%
\pgfusepath{stroke}%
\end{pgfscope}%
\begin{pgfscope}%
\definecolor{textcolor}{rgb}{0.000000,0.000000,0.000000}%
\pgfsetstrokecolor{textcolor}%
\pgfsetfillcolor{textcolor}%
\pgftext[x=3.441460in,y=2.294864in,left,base]{\color{textcolor}{\rmfamily\fontsize{8.000000}{9.600000}\selectfont\catcode`\^=\active\def^{\ifmmode\sp\else\^{}\fi}\catcode`\%=\active\def%{\%}NPLC 1}}%
\end{pgfscope}%
\begin{pgfscope}%
\pgfsetrectcap%
\pgfsetroundjoin%
\pgfsetlinewidth{1.505625pt}%
\definecolor{currentstroke}{rgb}{1.000000,0.498039,0.054902}%
\pgfsetstrokecolor{currentstroke}%
\pgfsetdash{}{0pt}%
\pgfpathmoveto{\pgfqpoint{3.130349in}{2.178864in}}%
\pgfpathlineto{\pgfqpoint{3.241460in}{2.178864in}}%
\pgfpathlineto{\pgfqpoint{3.352571in}{2.178864in}}%
\pgfusepath{stroke}%
\end{pgfscope}%
\begin{pgfscope}%
\definecolor{textcolor}{rgb}{0.000000,0.000000,0.000000}%
\pgfsetstrokecolor{textcolor}%
\pgfsetfillcolor{textcolor}%
\pgftext[x=3.441460in,y=2.139975in,left,base]{\color{textcolor}{\rmfamily\fontsize{8.000000}{9.600000}\selectfont\catcode`\^=\active\def^{\ifmmode\sp\else\^{}\fi}\catcode`\%=\active\def%{\%}NPLC 2}}%
\end{pgfscope}%
\begin{pgfscope}%
\pgfsetrectcap%
\pgfsetroundjoin%
\pgfsetlinewidth{1.505625pt}%
\definecolor{currentstroke}{rgb}{0.172549,0.627451,0.172549}%
\pgfsetstrokecolor{currentstroke}%
\pgfsetdash{}{0pt}%
\pgfpathmoveto{\pgfqpoint{3.130349in}{2.023975in}}%
\pgfpathlineto{\pgfqpoint{3.241460in}{2.023975in}}%
\pgfpathlineto{\pgfqpoint{3.352571in}{2.023975in}}%
\pgfusepath{stroke}%
\end{pgfscope}%
\begin{pgfscope}%
\definecolor{textcolor}{rgb}{0.000000,0.000000,0.000000}%
\pgfsetstrokecolor{textcolor}%
\pgfsetfillcolor{textcolor}%
\pgftext[x=3.441460in,y=1.985086in,left,base]{\color{textcolor}{\rmfamily\fontsize{8.000000}{9.600000}\selectfont\catcode`\^=\active\def^{\ifmmode\sp\else\^{}\fi}\catcode`\%=\active\def%{\%}NPLC 5}}%
\end{pgfscope}%
\begin{pgfscope}%
\pgfsetrectcap%
\pgfsetroundjoin%
\pgfsetlinewidth{1.505625pt}%
\definecolor{currentstroke}{rgb}{0.839216,0.152941,0.156863}%
\pgfsetstrokecolor{currentstroke}%
\pgfsetdash{}{0pt}%
\pgfpathmoveto{\pgfqpoint{3.130349in}{1.869086in}}%
\pgfpathlineto{\pgfqpoint{3.241460in}{1.869086in}}%
\pgfpathlineto{\pgfqpoint{3.352571in}{1.869086in}}%
\pgfusepath{stroke}%
\end{pgfscope}%
\begin{pgfscope}%
\definecolor{textcolor}{rgb}{0.000000,0.000000,0.000000}%
\pgfsetstrokecolor{textcolor}%
\pgfsetfillcolor{textcolor}%
\pgftext[x=3.441460in,y=1.830198in,left,base]{\color{textcolor}{\rmfamily\fontsize{8.000000}{9.600000}\selectfont\catcode`\^=\active\def^{\ifmmode\sp\else\^{}\fi}\catcode`\%=\active\def%{\%}NPLC 10}}%
\end{pgfscope}%
\begin{pgfscope}%
\pgfsetrectcap%
\pgfsetroundjoin%
\pgfsetlinewidth{1.505625pt}%
\definecolor{currentstroke}{rgb}{0.580392,0.403922,0.741176}%
\pgfsetstrokecolor{currentstroke}%
\pgfsetdash{}{0pt}%
\pgfpathmoveto{\pgfqpoint{3.130349in}{1.714198in}}%
\pgfpathlineto{\pgfqpoint{3.241460in}{1.714198in}}%
\pgfpathlineto{\pgfqpoint{3.352571in}{1.714198in}}%
\pgfusepath{stroke}%
\end{pgfscope}%
\begin{pgfscope}%
\definecolor{textcolor}{rgb}{0.000000,0.000000,0.000000}%
\pgfsetstrokecolor{textcolor}%
\pgfsetfillcolor{textcolor}%
\pgftext[x=3.441460in,y=1.675309in,left,base]{\color{textcolor}{\rmfamily\fontsize{8.000000}{9.600000}\selectfont\catcode`\^=\active\def^{\ifmmode\sp\else\^{}\fi}\catcode`\%=\active\def%{\%}NPLC 20}}%
\end{pgfscope}%
\begin{pgfscope}%
\pgfsetrectcap%
\pgfsetroundjoin%
\pgfsetlinewidth{1.505625pt}%
\definecolor{currentstroke}{rgb}{0.549020,0.337255,0.294118}%
\pgfsetstrokecolor{currentstroke}%
\pgfsetdash{}{0pt}%
\pgfpathmoveto{\pgfqpoint{3.130349in}{1.559309in}}%
\pgfpathlineto{\pgfqpoint{3.241460in}{1.559309in}}%
\pgfpathlineto{\pgfqpoint{3.352571in}{1.559309in}}%
\pgfusepath{stroke}%
\end{pgfscope}%
\begin{pgfscope}%
\definecolor{textcolor}{rgb}{0.000000,0.000000,0.000000}%
\pgfsetstrokecolor{textcolor}%
\pgfsetfillcolor{textcolor}%
\pgftext[x=3.441460in,y=1.520420in,left,base]{\color{textcolor}{\rmfamily\fontsize{8.000000}{9.600000}\selectfont\catcode`\^=\active\def^{\ifmmode\sp\else\^{}\fi}\catcode`\%=\active\def%{\%}NPLC 50}}%
\end{pgfscope}%
\end{pgfpicture}%
\makeatother%
\endgroup%
% data/simulations/sim_optimal_autozero_v2.py
    \caption{Allan deviation for different ADC integration intervals before applying the AZ algorithm. Dead time $\theta = \qty{1}{\plc}$. The dashed line denotes the Allan variance without AZ. The line frequency is \qty{50}{\Hz}.}
    \label{fig:autozero_deadtime_nplcs_adev}
\end{figure}

Figure \ref{fig:autozero_deadtime_nplcs_adev} demonstrates that the effectiveness of the AZ scheme no longer keeps increasing with an ever rising switching frequency. Instead, there is an optimal autozero interval. Above this optimal frequency, the portion of time spent with dead time is getting too large and too little information is collected. For the parameters chosen for this simulation ($f_c = \qty{1.5}{\Hz}$ and \qty[power-half-as-sqrt, per-mode=symbol]{165}{\nV \Hz\tothe{-0.5}}), \qty{5}{\plc} at \qty{50}{\Hz} is the optimal interval. If the corner frequency is shifted to a lower frequency, the optimum shifts more towards \qty{10}{\plc}. The same goes for a higher line frequency of \qty{60}{\Hz}. This explains, why HP chose \qty{10}{\plc} as the maximum integration time. For integration times higher than that, software averaging is used delivering the performance shown in figure \ref{fig:autozero_deadtime_nplcs_adev} along the \qty{10}{\plc} line.

It should be stressed here that the dead time is not the only factor to consider when choosing the autozero interval. For example, in case of an amplifier, switching the input also adds an error current due to the charge injection of the switching transistors. This may negatively impact the measurement of a high impedance source. These additional drawbacks are implementation specific and must already be considered during the design phase.

\subsection{Gain Correction}%
\label{sec:autozero_gain}
The effect of the gain correction, where the input value $x$ is scaled by a scaling factor $y$ to adjust the gain error, can be calculated, assuming white noise, as follows:
\begin{align}
    \sigma_{x \cdot y}^2 &= \langle x^2 y^2 \rangle - \langle x y \rangle^2 \nonumber\\
    &= \langle x^2 \rangle \langle y^2 \rangle + \underbrace{2\,\mathrm{Cov}\left(x^2,y^2\right)}_{\text{uncorrelated} \, = \, 0} - \left( \langle x \rangle \langle y \rangle + \underbrace{2\,\mathrm{Cov}\left(x,y\right)}_{=\, 0} \right)^2 \nonumber\\
    &= \left(\sigma_x^2 + \langle x \rangle^2\right) \cdot \left(\sigma_y^2 + \langle y \rangle^2\right) - \langle x \rangle \langle y \rangle \nonumber\\
    &= \sigma_x^2 \sigma_y^2 + \sigma_x^2 \langle y \rangle^2 + \sigma_y^2 \langle x \rangle^2 \label{eqn:variance_multiplied}
\end{align}

With respect to the gain correction, equation \ref{eqn:variance_multiplied} can be further simplified. The scaling factor is derived from the reference voltage $V_{ref}$ and normalised using $\frac{V_{ref, measured}}{V_{ref}}$. The expected value, therefore is $\langle y \rangle \approx 1$, as the ADC full scale gain should not drift much. Furthermore, $\sigma_y^2$ is scaled by the constant $1/V_{ref}$ and $\sigma_x^2 \sigma_y^2 \ll \sigma_x^2$. The latter should be true for any measurement of significance.
\begin{equation}
    \sigma_{x \cdot y}^2 \approx \sigma_x^2 + \sigma_y^2 \langle x \rangle^2
\end{equation}

The gain correction noise therefore behaves similar to the offset correction case, except that it scales with the input voltage $x$ and has no effect with a shorted input, while fully introducing its additional noise when a full scale input is applied.


% check \cite{psd_to_adev} Appendix II for details on dead time
% Compare PSD in Generation-Recombination Noise, Allan Variance, and Low-Frequency Gain Instabilities in Microwave Amplifiers to our controller. The hump look similar. Due to popcorn noise

\clearpage
\section{Current Sources}
% TODO: The FET Constant-Current Source/Limiter
Throughout this work the concept of current sources is widely used, for example section \ref{sec:laser_current_driver} discusses a current source to drive laser diodes and the temperature controller discussed in section \ref{sec:temperature_controller} uses a current source to measure the resistance of a temperature sensitive resistor. While there are many more use cases, this section will limit the discussion to a few examples used by the devices presented in this work. Namely, a unidirectional transconductance amplifier with an operational-amplifier in conjunction with a field-effect transistor and a bidirectional Howland current pump invented by Bradford Howland in 1962 and first published in 1964 by \citeauthor{howland_current_source} \cite{howland_current_source}. The discussion will start with the properties of the ideal current source and, based on this, develop a more accurate model. The models developed typically represent the static, time-independent case unless explicitly stated. First, the unidirectional current source is treated, then the bidirectional Howland current pump is discussed.

\subsection{Current Sink and Current Source}%
\label{sec:current_sink_current_source}
The question whether to use a current source or a current sink is elemental for the design of a laser driver. Figure \ref{fig:current_sink_source} shows different configurations of current sinks and sources with respect to the laser diode.
\begin{figure}[ht]
    \centering
    \begin{subfigure}{0.225\linewidth}
        \centering
        \import{figures/}{current_source_high.tex}
        \caption{Source with\protect\\grounded LD.}
        \label{fig:current_source_high}
    \end{subfigure}
    \begin{subfigure}{0.225\linewidth}
        \centering
        \import{figures/}{current_source_low.tex}
        \caption{Source with\protect\\floating LD.}
        \label{fig:current_source_low}
    \end{subfigure}
    \begin{subfigure}{0.225\linewidth}
        \centering
        \import{figures/}{current_sink_high.tex}
        \caption{Sink with\protect\\grounded LD.}
        \label{fig:current_sink_high}
    \end{subfigure}
    \begin{subfigure}{0.225\linewidth}
        \centering
        \import{figures/}{current_sink_low.tex}
        \caption{Sink with\protect\\floating LD.}
        \label{fig:current_sink_low}
    \end{subfigure}
    \caption{Different configurations of current sinks and sources with respect to the laser diode. A green check mark denotes a fail-safe configuration when accidentally shorting one or more pins of the diode to the laser chassis, illustrated by a dashed connection.}
    \label{fig:current_sink_source}
\end{figure}

The optimal configuration depends on the laser diode and safety aspects in terms of protecting the laser diode. The protection of the laser diode is discussed first. The laser resonator is assumed grounded in the setup. This is not the design case, but incorrect assembly can facilitate this condition. While not intended, there are numerous ways to also accidentally short-the diode to ground and since there are no immediate consequences arising from it, when the controller is disconnected, it might easily be overlooked. This blunder should not bear the risk of destroying an expensive laser diode. To ensure this, a configuration where the laser diode is shorted out, instead of the current source or sink, must be chosen. That way, the laser diode is automatically removed from the circuit in case of an error condition.
Choosing between a current sink and a current source is more subtle. If the other shell of the laser diode is connected to the anode, a current sink can be considered to keep the diode can at ground potential. This is not an issue with the laser design in this group though, because the laser diode mount is floating. Another aspect is the electronics. A current source is typically implemented using p-channel field-effect transistors, while current sinks are using n-channel transistors and additionally the input of a current source is referenced to the positive supply, while the sink is referenced to the negative supply. Using the negative supply as a reference for control signals brings more challenges than vice versa, because typically integrated components like digital-to-analog converters prefer working with positive voltages and would need additional support to be floated to a negative reference. This makes a current source simpler to implement in this scenario and this work focuses on the current source. In principle all methods that will be discussed can be applied to a current sink as well.

\subsection{Ideal Current Source}
\label{sec:ideal_current_source}
The ideal current source as shown in figure \ref{fig:ideal_current_source} has two major properties besides the output current $I_{out}$, the output impedance $R_{out}$ and the compliance voltage, which are best understood when looking at the two equivalent representations of a current source separately. On the left in figure \ref{fig:ideal_current_source_norton}, the Norton representation can be seen. Norton's theorem reduces any linear circuit to a current source, shown in green, with a parallel resistance $R_{out}$, usually called output resistance or impedance. On the right, the Thévenin representation can be seen, which simplifies a circuit as a voltage source, also shown in green, with a series resistance.
\begin{figure}[ht]
    \centering
    \begin{subfigure}{0.4\linewidth}
        \centering
        \import{figures/}{current_source_norton.tex}
        \caption{Norton representation.}
        \label{fig:ideal_current_source_norton}
    \end{subfigure}
    \begin{subfigure}{0.4\linewidth}
        \centering
        \import{figures/}{current_source_thevenin.tex}
        \caption{Thévenin representation.}
        \label{fig:ideal_current_source_thevenin}
    \end{subfigure}
    \caption{An ideal current source with output impedance $R_{out}$ and noise $e_n$.}
    \label{fig:ideal_current_source}
\end{figure}

First, the output impedance is discussed. Ideally, $R_{out}$ is infinite and all current is forced to flow through the load. Given a finite output impedance leads to a decreased accuracy of $I_{out}$, because it is influenced by the load impedance as
\begin{equation}
    I_{out} = I_{set} \cdot \frac{R_{out}}{R_{load} + R_{out}} \, .
\end{equation}

In addition to a decreased accuracy, inserting a noise voltage source between the current source and the load as shown in figure \ref{fig:ideal_current_source} in orange, has the same effect as a changing load resistance and due to the finite output impedance $R_{out}$, any voltage noise $e_n$ translates to current noise $i_n$ through the load as
\begin{equation}
    i_n = \frac{e_n}{R_{load} + R_{out}} \approx \frac{e_n}{R_{out}} \, ,
\end{equation}

again making a high output impedance desirable to suppress noise sources between the current source and the load.

Going to figure \ref{fig:ideal_current_source_thevenin} of a current source in the Thévenin representation allows discussing the compliance voltage property. As it was said above, the output impedance of an ideal current source is infinite and so is the maximum output voltage of said current source. A finite output impedance immediately implies a finite supply voltage to keep the current to a finite limit, which dictates a maximum output voltage. This is called the compliance voltage.

\subsection{The Field-Effect Transistor Current Source}%
\label{sec:mosfet_current_source}
% Good slides can be found here: https://www.ittc.ku.edu/~jstiles/312/handouts/
Given the limited supply voltage of a real current source drives the need for a resistive element that has a finite resistance and infinite, or very high, frequency dependent dynamic impedance to react to load changes. One such pass element, having these properties, is a field-effect-transistor (FET). A junction-gate field-effect transistor (JFET) or metal–oxide–semiconductor field-effect transistor (MOSFET) can be used either as a current source or sink, depending on its doping. A p-channel FET, which uses a positive doping of the channel, is a current source, while an n-channel FET works as a current sink. This discussion is focussing on the p-channel FET with MOSFETs at its centre, because it covers the bulk of the laser current driver design in section \ref{sec:laser_current_driver}.
\begin{figure}[hb]
    \centering
    \begin{subfigure}{0.4\linewidth}
        \centering
        \import{figures/}{p-channel_jfet.tex}
        \caption{P-Channel JFET.}
        \label{fig:pjfet}
    \end{subfigure}
    \begin{subfigure}{0.4\linewidth}
        \centering
        \import{figures/}{p-channel_mosfet.tex}
        \caption{P-Channel MOSFET.}
        \label{fig:pmos}
    \end{subfigure}
    \caption{The simplified semiconductor structure of a JFET and a MOSFET.}
    \label{fig:FETs}
\end{figure}

The difference between a JFET and a MOSFET is the gate structure as illustrated in figure \ref{fig:FETs}. While a MOSFET has an insulated gate, the JFET does not. This reduces the gate leakage current, typically by about three orders of magnitude and allows to forward bias the device since there is no diode, resulting in larger current handling capacity. So for low currents up to a few \unit{\mA} or low noise applications, JFETs are preferred, while MOSFETs can handle several hundred ampere. The same mathematical approach can be applied to both types of FETs though. The other difference between a JFET and a MOSFET is the fact that JFETs are only available as depletion-mode (normally-on) devices, while MOSFETs are available as both depletion and enhancement (normally-off) devices. The reason is the gate structure as mentioned above. An enhancement-mode device does not conduct when the gate-to-source voltage $V_{GS} = \qty{0}{\V}$, so $V_{GS}$ must be decreased or the junction enhanced for the device to allow conduction. This is not possible with an uninsulated gate like a simple n-p junction of a JFET, which would then start conducting. A p-channel depletion-mode device on the other hand conducts at $V_{GS} = \qty{0}{\V}$ and $V_{GS}$ must be increased and the junction depleted to reduce the current, which is possible with the uninsulated gate, because the n-p junction is reverse biased. The annotated circuit symbol and the quantities used to discuss the device properties are shown in figure \ref{fig:fet_symbols}.
\begin{figure}[ht]
    \centering
    \begin{subfigure}{0.4\linewidth}
        \centering
        \import{figures/}{jfet_pins.tex}
        \caption{P-channel JFET.}
        \label{fig:fet_symbols_jfet}
    \end{subfigure}
    \begin{subfigure}{0.4\linewidth}
        \centering
        \import{figures/}{pmos_pins.tex}
        \caption{P-channel MOSFET.}
        \label{fig:fet_symbols_mosfet}
    \end{subfigure}
    \caption{Basic p-channel FET circuit.}
    \label{fig:fet_symbols}
\end{figure}

A p-channel FET has its source (S) connected to the positive supply and the drain (D) is connected to a more negative voltage, typically the load. For the MOSFET the gate (G) is biased below the source to allow conduction. The source is usually connected to the substrate for solitary devices as shown in figure \ref{fig:pmos}. This will be assumed in all further discussions and the consequences of a substrate that is biased differently are omitted here. The interested reader may look up these details in \cite{mosfet_details}.

As it was hinted above, if appropriately biased, a FET can be considered a voltage controlled current source. This property can be seen in figure \ref{fig:fet_curret_gate_bias}.
\begin{figure}[hb]
    \centering
    %% Creator: Matplotlib, PGF backend
%%
%% To include the figure in your LaTeX document, write
%%   \input{<filename>.pgf}
%%
%% Make sure the required packages are loaded in your preamble
%%   \usepackage{pgf}
%%
%% Also ensure that all the required font packages are loaded; for instance,
%% the lmodern package is sometimes necessary when using math font.
%%   \usepackage{lmodern}
%%
%% Figures using additional raster images can only be included by \input if
%% they are in the same directory as the main LaTeX file. For loading figures
%% from other directories you can use the `import` package
%%   \usepackage{import}
%%
%% and then include the figures with
%%   \import{<path to file>}{<filename>.pgf}
%%
%% Matplotlib used the following preamble
%%   \usepackage{siunitx}
%%   \usepackage{fontspec}
%%
\begingroup%
\makeatletter%
\begin{pgfpicture}%
\pgfpathrectangle{\pgfpointorigin}{\pgfqpoint{5.431103in}{3.356606in}}%
\pgfusepath{use as bounding box, clip}%
\begin{pgfscope}%
\pgfsetbuttcap%
\pgfsetmiterjoin%
\definecolor{currentfill}{rgb}{1.000000,1.000000,1.000000}%
\pgfsetfillcolor{currentfill}%
\pgfsetlinewidth{0.000000pt}%
\definecolor{currentstroke}{rgb}{1.000000,1.000000,1.000000}%
\pgfsetstrokecolor{currentstroke}%
\pgfsetdash{}{0pt}%
\pgfpathmoveto{\pgfqpoint{0.000000in}{0.000000in}}%
\pgfpathlineto{\pgfqpoint{5.431103in}{0.000000in}}%
\pgfpathlineto{\pgfqpoint{5.431103in}{3.356606in}}%
\pgfpathlineto{\pgfqpoint{0.000000in}{3.356606in}}%
\pgfpathlineto{\pgfqpoint{0.000000in}{0.000000in}}%
\pgfpathclose%
\pgfusepath{fill}%
\end{pgfscope}%
\begin{pgfscope}%
\pgfsetbuttcap%
\pgfsetmiterjoin%
\definecolor{currentfill}{rgb}{1.000000,1.000000,1.000000}%
\pgfsetfillcolor{currentfill}%
\pgfsetlinewidth{0.000000pt}%
\definecolor{currentstroke}{rgb}{0.000000,0.000000,0.000000}%
\pgfsetstrokecolor{currentstroke}%
\pgfsetstrokeopacity{0.000000}%
\pgfsetdash{}{0pt}%
\pgfpathmoveto{\pgfqpoint{0.693677in}{0.524170in}}%
\pgfpathlineto{\pgfqpoint{5.281103in}{0.524170in}}%
\pgfpathlineto{\pgfqpoint{5.281103in}{3.082363in}}%
\pgfpathlineto{\pgfqpoint{0.693677in}{3.082363in}}%
\pgfpathlineto{\pgfqpoint{0.693677in}{0.524170in}}%
\pgfpathclose%
\pgfusepath{fill}%
\end{pgfscope}%
\begin{pgfscope}%
\pgfpathrectangle{\pgfqpoint{0.693677in}{0.524170in}}{\pgfqpoint{4.587426in}{2.558193in}}%
\pgfusepath{clip}%
\pgfsetrectcap%
\pgfsetroundjoin%
\pgfsetlinewidth{0.803000pt}%
\definecolor{currentstroke}{rgb}{0.450000,0.450000,0.450000}%
\pgfsetstrokecolor{currentstroke}%
\pgfsetdash{}{0pt}%
\pgfpathmoveto{\pgfqpoint{5.072583in}{0.524170in}}%
\pgfpathlineto{\pgfqpoint{5.072583in}{3.082363in}}%
\pgfusepath{stroke}%
\end{pgfscope}%
\begin{pgfscope}%
\pgfsetbuttcap%
\pgfsetroundjoin%
\definecolor{currentfill}{rgb}{0.000000,0.000000,0.000000}%
\pgfsetfillcolor{currentfill}%
\pgfsetlinewidth{0.803000pt}%
\definecolor{currentstroke}{rgb}{0.000000,0.000000,0.000000}%
\pgfsetstrokecolor{currentstroke}%
\pgfsetdash{}{0pt}%
\pgfsys@defobject{currentmarker}{\pgfqpoint{0.000000in}{-0.048611in}}{\pgfqpoint{0.000000in}{0.000000in}}{%
\pgfpathmoveto{\pgfqpoint{0.000000in}{0.000000in}}%
\pgfpathlineto{\pgfqpoint{0.000000in}{-0.048611in}}%
\pgfusepath{stroke,fill}%
}%
\begin{pgfscope}%
\pgfsys@transformshift{5.072583in}{0.524170in}%
\pgfsys@useobject{currentmarker}{}%
\end{pgfscope}%
\end{pgfscope}%
\begin{pgfscope}%
\definecolor{textcolor}{rgb}{0.000000,0.000000,0.000000}%
\pgfsetstrokecolor{textcolor}%
\pgfsetfillcolor{textcolor}%
\pgftext[x=5.072583in,y=0.426948in,,top]{\color{textcolor}\rmfamily\fontsize{8.000000}{9.600000}\selectfont \(\displaystyle {\ensuremath{-}10}\)}%
\end{pgfscope}%
\begin{pgfscope}%
\pgfpathrectangle{\pgfqpoint{0.693677in}{0.524170in}}{\pgfqpoint{4.587426in}{2.558193in}}%
\pgfusepath{clip}%
\pgfsetrectcap%
\pgfsetroundjoin%
\pgfsetlinewidth{0.803000pt}%
\definecolor{currentstroke}{rgb}{0.450000,0.450000,0.450000}%
\pgfsetstrokecolor{currentstroke}%
\pgfsetdash{}{0pt}%
\pgfpathmoveto{\pgfqpoint{4.238506in}{0.524170in}}%
\pgfpathlineto{\pgfqpoint{4.238506in}{3.082363in}}%
\pgfusepath{stroke}%
\end{pgfscope}%
\begin{pgfscope}%
\pgfsetbuttcap%
\pgfsetroundjoin%
\definecolor{currentfill}{rgb}{0.000000,0.000000,0.000000}%
\pgfsetfillcolor{currentfill}%
\pgfsetlinewidth{0.803000pt}%
\definecolor{currentstroke}{rgb}{0.000000,0.000000,0.000000}%
\pgfsetstrokecolor{currentstroke}%
\pgfsetdash{}{0pt}%
\pgfsys@defobject{currentmarker}{\pgfqpoint{0.000000in}{-0.048611in}}{\pgfqpoint{0.000000in}{0.000000in}}{%
\pgfpathmoveto{\pgfqpoint{0.000000in}{0.000000in}}%
\pgfpathlineto{\pgfqpoint{0.000000in}{-0.048611in}}%
\pgfusepath{stroke,fill}%
}%
\begin{pgfscope}%
\pgfsys@transformshift{4.238506in}{0.524170in}%
\pgfsys@useobject{currentmarker}{}%
\end{pgfscope}%
\end{pgfscope}%
\begin{pgfscope}%
\definecolor{textcolor}{rgb}{0.000000,0.000000,0.000000}%
\pgfsetstrokecolor{textcolor}%
\pgfsetfillcolor{textcolor}%
\pgftext[x=4.238506in,y=0.426948in,,top]{\color{textcolor}\rmfamily\fontsize{8.000000}{9.600000}\selectfont \(\displaystyle {\ensuremath{-}8}\)}%
\end{pgfscope}%
\begin{pgfscope}%
\pgfpathrectangle{\pgfqpoint{0.693677in}{0.524170in}}{\pgfqpoint{4.587426in}{2.558193in}}%
\pgfusepath{clip}%
\pgfsetrectcap%
\pgfsetroundjoin%
\pgfsetlinewidth{0.803000pt}%
\definecolor{currentstroke}{rgb}{0.450000,0.450000,0.450000}%
\pgfsetstrokecolor{currentstroke}%
\pgfsetdash{}{0pt}%
\pgfpathmoveto{\pgfqpoint{3.404428in}{0.524170in}}%
\pgfpathlineto{\pgfqpoint{3.404428in}{3.082363in}}%
\pgfusepath{stroke}%
\end{pgfscope}%
\begin{pgfscope}%
\pgfsetbuttcap%
\pgfsetroundjoin%
\definecolor{currentfill}{rgb}{0.000000,0.000000,0.000000}%
\pgfsetfillcolor{currentfill}%
\pgfsetlinewidth{0.803000pt}%
\definecolor{currentstroke}{rgb}{0.000000,0.000000,0.000000}%
\pgfsetstrokecolor{currentstroke}%
\pgfsetdash{}{0pt}%
\pgfsys@defobject{currentmarker}{\pgfqpoint{0.000000in}{-0.048611in}}{\pgfqpoint{0.000000in}{0.000000in}}{%
\pgfpathmoveto{\pgfqpoint{0.000000in}{0.000000in}}%
\pgfpathlineto{\pgfqpoint{0.000000in}{-0.048611in}}%
\pgfusepath{stroke,fill}%
}%
\begin{pgfscope}%
\pgfsys@transformshift{3.404428in}{0.524170in}%
\pgfsys@useobject{currentmarker}{}%
\end{pgfscope}%
\end{pgfscope}%
\begin{pgfscope}%
\definecolor{textcolor}{rgb}{0.000000,0.000000,0.000000}%
\pgfsetstrokecolor{textcolor}%
\pgfsetfillcolor{textcolor}%
\pgftext[x=3.404428in,y=0.426948in,,top]{\color{textcolor}\rmfamily\fontsize{8.000000}{9.600000}\selectfont \(\displaystyle {\ensuremath{-}6}\)}%
\end{pgfscope}%
\begin{pgfscope}%
\pgfpathrectangle{\pgfqpoint{0.693677in}{0.524170in}}{\pgfqpoint{4.587426in}{2.558193in}}%
\pgfusepath{clip}%
\pgfsetrectcap%
\pgfsetroundjoin%
\pgfsetlinewidth{0.803000pt}%
\definecolor{currentstroke}{rgb}{0.450000,0.450000,0.450000}%
\pgfsetstrokecolor{currentstroke}%
\pgfsetdash{}{0pt}%
\pgfpathmoveto{\pgfqpoint{2.570351in}{0.524170in}}%
\pgfpathlineto{\pgfqpoint{2.570351in}{3.082363in}}%
\pgfusepath{stroke}%
\end{pgfscope}%
\begin{pgfscope}%
\pgfsetbuttcap%
\pgfsetroundjoin%
\definecolor{currentfill}{rgb}{0.000000,0.000000,0.000000}%
\pgfsetfillcolor{currentfill}%
\pgfsetlinewidth{0.803000pt}%
\definecolor{currentstroke}{rgb}{0.000000,0.000000,0.000000}%
\pgfsetstrokecolor{currentstroke}%
\pgfsetdash{}{0pt}%
\pgfsys@defobject{currentmarker}{\pgfqpoint{0.000000in}{-0.048611in}}{\pgfqpoint{0.000000in}{0.000000in}}{%
\pgfpathmoveto{\pgfqpoint{0.000000in}{0.000000in}}%
\pgfpathlineto{\pgfqpoint{0.000000in}{-0.048611in}}%
\pgfusepath{stroke,fill}%
}%
\begin{pgfscope}%
\pgfsys@transformshift{2.570351in}{0.524170in}%
\pgfsys@useobject{currentmarker}{}%
\end{pgfscope}%
\end{pgfscope}%
\begin{pgfscope}%
\definecolor{textcolor}{rgb}{0.000000,0.000000,0.000000}%
\pgfsetstrokecolor{textcolor}%
\pgfsetfillcolor{textcolor}%
\pgftext[x=2.570351in,y=0.426948in,,top]{\color{textcolor}\rmfamily\fontsize{8.000000}{9.600000}\selectfont \(\displaystyle {\ensuremath{-}4}\)}%
\end{pgfscope}%
\begin{pgfscope}%
\pgfpathrectangle{\pgfqpoint{0.693677in}{0.524170in}}{\pgfqpoint{4.587426in}{2.558193in}}%
\pgfusepath{clip}%
\pgfsetrectcap%
\pgfsetroundjoin%
\pgfsetlinewidth{0.803000pt}%
\definecolor{currentstroke}{rgb}{0.450000,0.450000,0.450000}%
\pgfsetstrokecolor{currentstroke}%
\pgfsetdash{}{0pt}%
\pgfpathmoveto{\pgfqpoint{1.736274in}{0.524170in}}%
\pgfpathlineto{\pgfqpoint{1.736274in}{3.082363in}}%
\pgfusepath{stroke}%
\end{pgfscope}%
\begin{pgfscope}%
\pgfsetbuttcap%
\pgfsetroundjoin%
\definecolor{currentfill}{rgb}{0.000000,0.000000,0.000000}%
\pgfsetfillcolor{currentfill}%
\pgfsetlinewidth{0.803000pt}%
\definecolor{currentstroke}{rgb}{0.000000,0.000000,0.000000}%
\pgfsetstrokecolor{currentstroke}%
\pgfsetdash{}{0pt}%
\pgfsys@defobject{currentmarker}{\pgfqpoint{0.000000in}{-0.048611in}}{\pgfqpoint{0.000000in}{0.000000in}}{%
\pgfpathmoveto{\pgfqpoint{0.000000in}{0.000000in}}%
\pgfpathlineto{\pgfqpoint{0.000000in}{-0.048611in}}%
\pgfusepath{stroke,fill}%
}%
\begin{pgfscope}%
\pgfsys@transformshift{1.736274in}{0.524170in}%
\pgfsys@useobject{currentmarker}{}%
\end{pgfscope}%
\end{pgfscope}%
\begin{pgfscope}%
\definecolor{textcolor}{rgb}{0.000000,0.000000,0.000000}%
\pgfsetstrokecolor{textcolor}%
\pgfsetfillcolor{textcolor}%
\pgftext[x=1.736274in,y=0.426948in,,top]{\color{textcolor}\rmfamily\fontsize{8.000000}{9.600000}\selectfont \(\displaystyle {\ensuremath{-}2}\)}%
\end{pgfscope}%
\begin{pgfscope}%
\pgfpathrectangle{\pgfqpoint{0.693677in}{0.524170in}}{\pgfqpoint{4.587426in}{2.558193in}}%
\pgfusepath{clip}%
\pgfsetrectcap%
\pgfsetroundjoin%
\pgfsetlinewidth{0.803000pt}%
\definecolor{currentstroke}{rgb}{0.450000,0.450000,0.450000}%
\pgfsetstrokecolor{currentstroke}%
\pgfsetdash{}{0pt}%
\pgfpathmoveto{\pgfqpoint{0.902196in}{0.524170in}}%
\pgfpathlineto{\pgfqpoint{0.902196in}{3.082363in}}%
\pgfusepath{stroke}%
\end{pgfscope}%
\begin{pgfscope}%
\pgfsetbuttcap%
\pgfsetroundjoin%
\definecolor{currentfill}{rgb}{0.000000,0.000000,0.000000}%
\pgfsetfillcolor{currentfill}%
\pgfsetlinewidth{0.803000pt}%
\definecolor{currentstroke}{rgb}{0.000000,0.000000,0.000000}%
\pgfsetstrokecolor{currentstroke}%
\pgfsetdash{}{0pt}%
\pgfsys@defobject{currentmarker}{\pgfqpoint{0.000000in}{-0.048611in}}{\pgfqpoint{0.000000in}{0.000000in}}{%
\pgfpathmoveto{\pgfqpoint{0.000000in}{0.000000in}}%
\pgfpathlineto{\pgfqpoint{0.000000in}{-0.048611in}}%
\pgfusepath{stroke,fill}%
}%
\begin{pgfscope}%
\pgfsys@transformshift{0.902196in}{0.524170in}%
\pgfsys@useobject{currentmarker}{}%
\end{pgfscope}%
\end{pgfscope}%
\begin{pgfscope}%
\definecolor{textcolor}{rgb}{0.000000,0.000000,0.000000}%
\pgfsetstrokecolor{textcolor}%
\pgfsetfillcolor{textcolor}%
\pgftext[x=0.902196in,y=0.426948in,,top]{\color{textcolor}\rmfamily\fontsize{8.000000}{9.600000}\selectfont \(\displaystyle {0}\)}%
\end{pgfscope}%
\begin{pgfscope}%
\definecolor{textcolor}{rgb}{0.000000,0.000000,0.000000}%
\pgfsetstrokecolor{textcolor}%
\pgfsetfillcolor{textcolor}%
\pgftext[x=2.987390in,y=0.272725in,,top]{\color{textcolor}\rmfamily\fontsize{10.000000}{12.000000}\selectfont Drain-Source Voltage \(\displaystyle V_{DS}\) in \unit{\V}}%
\end{pgfscope}%
\begin{pgfscope}%
\pgfpathrectangle{\pgfqpoint{0.693677in}{0.524170in}}{\pgfqpoint{4.587426in}{2.558193in}}%
\pgfusepath{clip}%
\pgfsetrectcap%
\pgfsetroundjoin%
\pgfsetlinewidth{0.803000pt}%
\definecolor{currentstroke}{rgb}{0.450000,0.450000,0.450000}%
\pgfsetstrokecolor{currentstroke}%
\pgfsetdash{}{0pt}%
\pgfpathmoveto{\pgfqpoint{0.693677in}{2.943608in}}%
\pgfpathlineto{\pgfqpoint{5.281103in}{2.943608in}}%
\pgfusepath{stroke}%
\end{pgfscope}%
\begin{pgfscope}%
\pgfsetbuttcap%
\pgfsetroundjoin%
\definecolor{currentfill}{rgb}{0.000000,0.000000,0.000000}%
\pgfsetfillcolor{currentfill}%
\pgfsetlinewidth{0.803000pt}%
\definecolor{currentstroke}{rgb}{0.000000,0.000000,0.000000}%
\pgfsetstrokecolor{currentstroke}%
\pgfsetdash{}{0pt}%
\pgfsys@defobject{currentmarker}{\pgfqpoint{-0.048611in}{0.000000in}}{\pgfqpoint{-0.000000in}{0.000000in}}{%
\pgfpathmoveto{\pgfqpoint{-0.000000in}{0.000000in}}%
\pgfpathlineto{\pgfqpoint{-0.048611in}{0.000000in}}%
\pgfusepath{stroke,fill}%
}%
\begin{pgfscope}%
\pgfsys@transformshift{0.693677in}{2.943608in}%
\pgfsys@useobject{currentmarker}{}%
\end{pgfscope}%
\end{pgfscope}%
\begin{pgfscope}%
\definecolor{textcolor}{rgb}{0.000000,0.000000,0.000000}%
\pgfsetstrokecolor{textcolor}%
\pgfsetfillcolor{textcolor}%
\pgftext[x=0.327546in, y=2.905053in, left, base]{\color{textcolor}\rmfamily\fontsize{8.000000}{9.600000}\selectfont \(\displaystyle {\ensuremath{-}500}\)}%
\end{pgfscope}%
\begin{pgfscope}%
\pgfpathrectangle{\pgfqpoint{0.693677in}{0.524170in}}{\pgfqpoint{4.587426in}{2.558193in}}%
\pgfusepath{clip}%
\pgfsetrectcap%
\pgfsetroundjoin%
\pgfsetlinewidth{0.803000pt}%
\definecolor{currentstroke}{rgb}{0.450000,0.450000,0.450000}%
\pgfsetstrokecolor{currentstroke}%
\pgfsetdash{}{0pt}%
\pgfpathmoveto{\pgfqpoint{0.693677in}{2.482977in}}%
\pgfpathlineto{\pgfqpoint{5.281103in}{2.482977in}}%
\pgfusepath{stroke}%
\end{pgfscope}%
\begin{pgfscope}%
\pgfsetbuttcap%
\pgfsetroundjoin%
\definecolor{currentfill}{rgb}{0.000000,0.000000,0.000000}%
\pgfsetfillcolor{currentfill}%
\pgfsetlinewidth{0.803000pt}%
\definecolor{currentstroke}{rgb}{0.000000,0.000000,0.000000}%
\pgfsetstrokecolor{currentstroke}%
\pgfsetdash{}{0pt}%
\pgfsys@defobject{currentmarker}{\pgfqpoint{-0.048611in}{0.000000in}}{\pgfqpoint{-0.000000in}{0.000000in}}{%
\pgfpathmoveto{\pgfqpoint{-0.000000in}{0.000000in}}%
\pgfpathlineto{\pgfqpoint{-0.048611in}{0.000000in}}%
\pgfusepath{stroke,fill}%
}%
\begin{pgfscope}%
\pgfsys@transformshift{0.693677in}{2.482977in}%
\pgfsys@useobject{currentmarker}{}%
\end{pgfscope}%
\end{pgfscope}%
\begin{pgfscope}%
\definecolor{textcolor}{rgb}{0.000000,0.000000,0.000000}%
\pgfsetstrokecolor{textcolor}%
\pgfsetfillcolor{textcolor}%
\pgftext[x=0.327546in, y=2.444421in, left, base]{\color{textcolor}\rmfamily\fontsize{8.000000}{9.600000}\selectfont \(\displaystyle {\ensuremath{-}400}\)}%
\end{pgfscope}%
\begin{pgfscope}%
\pgfpathrectangle{\pgfqpoint{0.693677in}{0.524170in}}{\pgfqpoint{4.587426in}{2.558193in}}%
\pgfusepath{clip}%
\pgfsetrectcap%
\pgfsetroundjoin%
\pgfsetlinewidth{0.803000pt}%
\definecolor{currentstroke}{rgb}{0.450000,0.450000,0.450000}%
\pgfsetstrokecolor{currentstroke}%
\pgfsetdash{}{0pt}%
\pgfpathmoveto{\pgfqpoint{0.693677in}{2.022346in}}%
\pgfpathlineto{\pgfqpoint{5.281103in}{2.022346in}}%
\pgfusepath{stroke}%
\end{pgfscope}%
\begin{pgfscope}%
\pgfsetbuttcap%
\pgfsetroundjoin%
\definecolor{currentfill}{rgb}{0.000000,0.000000,0.000000}%
\pgfsetfillcolor{currentfill}%
\pgfsetlinewidth{0.803000pt}%
\definecolor{currentstroke}{rgb}{0.000000,0.000000,0.000000}%
\pgfsetstrokecolor{currentstroke}%
\pgfsetdash{}{0pt}%
\pgfsys@defobject{currentmarker}{\pgfqpoint{-0.048611in}{0.000000in}}{\pgfqpoint{-0.000000in}{0.000000in}}{%
\pgfpathmoveto{\pgfqpoint{-0.000000in}{0.000000in}}%
\pgfpathlineto{\pgfqpoint{-0.048611in}{0.000000in}}%
\pgfusepath{stroke,fill}%
}%
\begin{pgfscope}%
\pgfsys@transformshift{0.693677in}{2.022346in}%
\pgfsys@useobject{currentmarker}{}%
\end{pgfscope}%
\end{pgfscope}%
\begin{pgfscope}%
\definecolor{textcolor}{rgb}{0.000000,0.000000,0.000000}%
\pgfsetstrokecolor{textcolor}%
\pgfsetfillcolor{textcolor}%
\pgftext[x=0.327546in, y=1.983790in, left, base]{\color{textcolor}\rmfamily\fontsize{8.000000}{9.600000}\selectfont \(\displaystyle {\ensuremath{-}300}\)}%
\end{pgfscope}%
\begin{pgfscope}%
\pgfpathrectangle{\pgfqpoint{0.693677in}{0.524170in}}{\pgfqpoint{4.587426in}{2.558193in}}%
\pgfusepath{clip}%
\pgfsetrectcap%
\pgfsetroundjoin%
\pgfsetlinewidth{0.803000pt}%
\definecolor{currentstroke}{rgb}{0.450000,0.450000,0.450000}%
\pgfsetstrokecolor{currentstroke}%
\pgfsetdash{}{0pt}%
\pgfpathmoveto{\pgfqpoint{0.693677in}{1.561714in}}%
\pgfpathlineto{\pgfqpoint{5.281103in}{1.561714in}}%
\pgfusepath{stroke}%
\end{pgfscope}%
\begin{pgfscope}%
\pgfsetbuttcap%
\pgfsetroundjoin%
\definecolor{currentfill}{rgb}{0.000000,0.000000,0.000000}%
\pgfsetfillcolor{currentfill}%
\pgfsetlinewidth{0.803000pt}%
\definecolor{currentstroke}{rgb}{0.000000,0.000000,0.000000}%
\pgfsetstrokecolor{currentstroke}%
\pgfsetdash{}{0pt}%
\pgfsys@defobject{currentmarker}{\pgfqpoint{-0.048611in}{0.000000in}}{\pgfqpoint{-0.000000in}{0.000000in}}{%
\pgfpathmoveto{\pgfqpoint{-0.000000in}{0.000000in}}%
\pgfpathlineto{\pgfqpoint{-0.048611in}{0.000000in}}%
\pgfusepath{stroke,fill}%
}%
\begin{pgfscope}%
\pgfsys@transformshift{0.693677in}{1.561714in}%
\pgfsys@useobject{currentmarker}{}%
\end{pgfscope}%
\end{pgfscope}%
\begin{pgfscope}%
\definecolor{textcolor}{rgb}{0.000000,0.000000,0.000000}%
\pgfsetstrokecolor{textcolor}%
\pgfsetfillcolor{textcolor}%
\pgftext[x=0.327546in, y=1.523159in, left, base]{\color{textcolor}\rmfamily\fontsize{8.000000}{9.600000}\selectfont \(\displaystyle {\ensuremath{-}200}\)}%
\end{pgfscope}%
\begin{pgfscope}%
\pgfpathrectangle{\pgfqpoint{0.693677in}{0.524170in}}{\pgfqpoint{4.587426in}{2.558193in}}%
\pgfusepath{clip}%
\pgfsetrectcap%
\pgfsetroundjoin%
\pgfsetlinewidth{0.803000pt}%
\definecolor{currentstroke}{rgb}{0.450000,0.450000,0.450000}%
\pgfsetstrokecolor{currentstroke}%
\pgfsetdash{}{0pt}%
\pgfpathmoveto{\pgfqpoint{0.693677in}{1.101083in}}%
\pgfpathlineto{\pgfqpoint{5.281103in}{1.101083in}}%
\pgfusepath{stroke}%
\end{pgfscope}%
\begin{pgfscope}%
\pgfsetbuttcap%
\pgfsetroundjoin%
\definecolor{currentfill}{rgb}{0.000000,0.000000,0.000000}%
\pgfsetfillcolor{currentfill}%
\pgfsetlinewidth{0.803000pt}%
\definecolor{currentstroke}{rgb}{0.000000,0.000000,0.000000}%
\pgfsetstrokecolor{currentstroke}%
\pgfsetdash{}{0pt}%
\pgfsys@defobject{currentmarker}{\pgfqpoint{-0.048611in}{0.000000in}}{\pgfqpoint{-0.000000in}{0.000000in}}{%
\pgfpathmoveto{\pgfqpoint{-0.000000in}{0.000000in}}%
\pgfpathlineto{\pgfqpoint{-0.048611in}{0.000000in}}%
\pgfusepath{stroke,fill}%
}%
\begin{pgfscope}%
\pgfsys@transformshift{0.693677in}{1.101083in}%
\pgfsys@useobject{currentmarker}{}%
\end{pgfscope}%
\end{pgfscope}%
\begin{pgfscope}%
\definecolor{textcolor}{rgb}{0.000000,0.000000,0.000000}%
\pgfsetstrokecolor{textcolor}%
\pgfsetfillcolor{textcolor}%
\pgftext[x=0.327546in, y=1.062527in, left, base]{\color{textcolor}\rmfamily\fontsize{8.000000}{9.600000}\selectfont \(\displaystyle {\ensuremath{-}100}\)}%
\end{pgfscope}%
\begin{pgfscope}%
\pgfpathrectangle{\pgfqpoint{0.693677in}{0.524170in}}{\pgfqpoint{4.587426in}{2.558193in}}%
\pgfusepath{clip}%
\pgfsetrectcap%
\pgfsetroundjoin%
\pgfsetlinewidth{0.803000pt}%
\definecolor{currentstroke}{rgb}{0.450000,0.450000,0.450000}%
\pgfsetstrokecolor{currentstroke}%
\pgfsetdash{}{0pt}%
\pgfpathmoveto{\pgfqpoint{0.693677in}{0.640451in}}%
\pgfpathlineto{\pgfqpoint{5.281103in}{0.640451in}}%
\pgfusepath{stroke}%
\end{pgfscope}%
\begin{pgfscope}%
\pgfsetbuttcap%
\pgfsetroundjoin%
\definecolor{currentfill}{rgb}{0.000000,0.000000,0.000000}%
\pgfsetfillcolor{currentfill}%
\pgfsetlinewidth{0.803000pt}%
\definecolor{currentstroke}{rgb}{0.000000,0.000000,0.000000}%
\pgfsetstrokecolor{currentstroke}%
\pgfsetdash{}{0pt}%
\pgfsys@defobject{currentmarker}{\pgfqpoint{-0.048611in}{0.000000in}}{\pgfqpoint{-0.000000in}{0.000000in}}{%
\pgfpathmoveto{\pgfqpoint{-0.000000in}{0.000000in}}%
\pgfpathlineto{\pgfqpoint{-0.048611in}{0.000000in}}%
\pgfusepath{stroke,fill}%
}%
\begin{pgfscope}%
\pgfsys@transformshift{0.693677in}{0.640451in}%
\pgfsys@useobject{currentmarker}{}%
\end{pgfscope}%
\end{pgfscope}%
\begin{pgfscope}%
\definecolor{textcolor}{rgb}{0.000000,0.000000,0.000000}%
\pgfsetstrokecolor{textcolor}%
\pgfsetfillcolor{textcolor}%
\pgftext[x=0.537426in, y=0.601896in, left, base]{\color{textcolor}\rmfamily\fontsize{8.000000}{9.600000}\selectfont \(\displaystyle {0}\)}%
\end{pgfscope}%
\begin{pgfscope}%
\definecolor{textcolor}{rgb}{0.000000,0.000000,0.000000}%
\pgfsetstrokecolor{textcolor}%
\pgfsetfillcolor{textcolor}%
\pgftext[x=0.271991in,y=1.803266in,,bottom,rotate=90.000000]{\color{textcolor}\rmfamily\fontsize{10.000000}{12.000000}\selectfont Drain Current \(\displaystyle I_D\) in \unit{\A}}%
\end{pgfscope}%
\begin{pgfscope}%
\definecolor{textcolor}{rgb}{0.000000,0.000000,0.000000}%
\pgfsetstrokecolor{textcolor}%
\pgfsetfillcolor{textcolor}%
\pgftext[x=0.693677in,y=3.124029in,left,base]{\color{textcolor}\rmfamily\fontsize{8.000000}{9.600000}\selectfont \(\displaystyle \times{10^{\ensuremath{-}3}}{}\)}%
\end{pgfscope}%
\begin{pgfscope}%
\pgfpathrectangle{\pgfqpoint{0.693677in}{0.524170in}}{\pgfqpoint{4.587426in}{2.558193in}}%
\pgfusepath{clip}%
\pgfsetrectcap%
\pgfsetroundjoin%
\pgfsetlinewidth{1.505625pt}%
\definecolor{currentstroke}{rgb}{0.003922,0.450980,0.698039}%
\pgfsetstrokecolor{currentstroke}%
\pgfsetstrokeopacity{0.700000}%
\pgfsetdash{}{0pt}%
\pgfpathmoveto{\pgfqpoint{0.902196in}{0.640451in}}%
\pgfpathlineto{\pgfqpoint{0.943900in}{0.649044in}}%
\pgfpathlineto{\pgfqpoint{0.985604in}{0.650106in}}%
\pgfpathlineto{\pgfqpoint{1.027308in}{0.650109in}}%
\pgfpathlineto{\pgfqpoint{1.069012in}{0.650113in}}%
\pgfpathlineto{\pgfqpoint{1.110716in}{0.650117in}}%
\pgfpathlineto{\pgfqpoint{1.152419in}{0.650121in}}%
\pgfpathlineto{\pgfqpoint{1.194123in}{0.650125in}}%
\pgfpathlineto{\pgfqpoint{1.235827in}{0.650129in}}%
\pgfpathlineto{\pgfqpoint{1.277531in}{0.650132in}}%
\pgfpathlineto{\pgfqpoint{1.319235in}{0.650136in}}%
\pgfpathlineto{\pgfqpoint{1.360939in}{0.650140in}}%
\pgfpathlineto{\pgfqpoint{1.402643in}{0.650144in}}%
\pgfpathlineto{\pgfqpoint{1.444346in}{0.650148in}}%
\pgfpathlineto{\pgfqpoint{1.486050in}{0.650152in}}%
\pgfpathlineto{\pgfqpoint{1.527754in}{0.650155in}}%
\pgfpathlineto{\pgfqpoint{1.569458in}{0.650159in}}%
\pgfpathlineto{\pgfqpoint{1.611162in}{0.650163in}}%
\pgfpathlineto{\pgfqpoint{1.652866in}{0.650167in}}%
\pgfpathlineto{\pgfqpoint{1.694570in}{0.650171in}}%
\pgfpathlineto{\pgfqpoint{1.736274in}{0.650174in}}%
\pgfpathlineto{\pgfqpoint{1.777977in}{0.650178in}}%
\pgfpathlineto{\pgfqpoint{1.819681in}{0.650182in}}%
\pgfpathlineto{\pgfqpoint{1.861385in}{0.650186in}}%
\pgfpathlineto{\pgfqpoint{1.903089in}{0.650190in}}%
\pgfpathlineto{\pgfqpoint{1.944793in}{0.650194in}}%
\pgfpathlineto{\pgfqpoint{1.986497in}{0.650197in}}%
\pgfpathlineto{\pgfqpoint{2.028201in}{0.650201in}}%
\pgfpathlineto{\pgfqpoint{2.069905in}{0.650205in}}%
\pgfpathlineto{\pgfqpoint{2.111608in}{0.650209in}}%
\pgfpathlineto{\pgfqpoint{2.153312in}{0.650213in}}%
\pgfpathlineto{\pgfqpoint{2.195016in}{0.650217in}}%
\pgfpathlineto{\pgfqpoint{2.236720in}{0.650220in}}%
\pgfpathlineto{\pgfqpoint{2.278424in}{0.650224in}}%
\pgfpathlineto{\pgfqpoint{2.320128in}{0.650228in}}%
\pgfpathlineto{\pgfqpoint{2.361832in}{0.650232in}}%
\pgfpathlineto{\pgfqpoint{2.403536in}{0.650236in}}%
\pgfpathlineto{\pgfqpoint{2.445239in}{0.650240in}}%
\pgfpathlineto{\pgfqpoint{2.486943in}{0.650243in}}%
\pgfpathlineto{\pgfqpoint{2.528647in}{0.650247in}}%
\pgfpathlineto{\pgfqpoint{2.570351in}{0.650251in}}%
\pgfpathlineto{\pgfqpoint{2.612055in}{0.650255in}}%
\pgfpathlineto{\pgfqpoint{2.653759in}{0.650259in}}%
\pgfpathlineto{\pgfqpoint{2.695463in}{0.650262in}}%
\pgfpathlineto{\pgfqpoint{2.737166in}{0.650266in}}%
\pgfpathlineto{\pgfqpoint{2.778870in}{0.650270in}}%
\pgfpathlineto{\pgfqpoint{2.820574in}{0.650274in}}%
\pgfpathlineto{\pgfqpoint{2.862278in}{0.650278in}}%
\pgfpathlineto{\pgfqpoint{2.903982in}{0.650282in}}%
\pgfpathlineto{\pgfqpoint{2.945686in}{0.650285in}}%
\pgfpathlineto{\pgfqpoint{2.987390in}{0.650289in}}%
\pgfpathlineto{\pgfqpoint{3.029094in}{0.650293in}}%
\pgfpathlineto{\pgfqpoint{3.070797in}{0.650297in}}%
\pgfpathlineto{\pgfqpoint{3.112501in}{0.650301in}}%
\pgfpathlineto{\pgfqpoint{3.154205in}{0.650305in}}%
\pgfpathlineto{\pgfqpoint{3.195909in}{0.650308in}}%
\pgfpathlineto{\pgfqpoint{3.237613in}{0.650312in}}%
\pgfpathlineto{\pgfqpoint{3.279317in}{0.650316in}}%
\pgfpathlineto{\pgfqpoint{3.321021in}{0.650320in}}%
\pgfpathlineto{\pgfqpoint{3.362725in}{0.650324in}}%
\pgfpathlineto{\pgfqpoint{3.404428in}{0.650327in}}%
\pgfpathlineto{\pgfqpoint{3.446132in}{0.650331in}}%
\pgfpathlineto{\pgfqpoint{3.487836in}{0.650335in}}%
\pgfpathlineto{\pgfqpoint{3.529540in}{0.650339in}}%
\pgfpathlineto{\pgfqpoint{3.571244in}{0.650343in}}%
\pgfpathlineto{\pgfqpoint{3.612948in}{0.650347in}}%
\pgfpathlineto{\pgfqpoint{3.654652in}{0.650350in}}%
\pgfpathlineto{\pgfqpoint{3.696355in}{0.650354in}}%
\pgfpathlineto{\pgfqpoint{3.738059in}{0.650358in}}%
\pgfpathlineto{\pgfqpoint{3.779763in}{0.650362in}}%
\pgfpathlineto{\pgfqpoint{3.821467in}{0.650366in}}%
\pgfpathlineto{\pgfqpoint{3.863171in}{0.650370in}}%
\pgfpathlineto{\pgfqpoint{3.904875in}{0.650373in}}%
\pgfpathlineto{\pgfqpoint{3.946579in}{0.650377in}}%
\pgfpathlineto{\pgfqpoint{3.988283in}{0.650381in}}%
\pgfpathlineto{\pgfqpoint{4.029986in}{0.650385in}}%
\pgfpathlineto{\pgfqpoint{4.071690in}{0.650389in}}%
\pgfpathlineto{\pgfqpoint{4.113394in}{0.650393in}}%
\pgfpathlineto{\pgfqpoint{4.155098in}{0.650396in}}%
\pgfpathlineto{\pgfqpoint{4.196802in}{0.650400in}}%
\pgfpathlineto{\pgfqpoint{4.238506in}{0.650404in}}%
\pgfpathlineto{\pgfqpoint{4.280210in}{0.650408in}}%
\pgfpathlineto{\pgfqpoint{4.321914in}{0.650412in}}%
\pgfpathlineto{\pgfqpoint{4.363617in}{0.650415in}}%
\pgfpathlineto{\pgfqpoint{4.405321in}{0.650419in}}%
\pgfpathlineto{\pgfqpoint{4.447025in}{0.650423in}}%
\pgfpathlineto{\pgfqpoint{4.488729in}{0.650427in}}%
\pgfpathlineto{\pgfqpoint{4.530433in}{0.650431in}}%
\pgfpathlineto{\pgfqpoint{4.572137in}{0.650435in}}%
\pgfpathlineto{\pgfqpoint{4.613841in}{0.650438in}}%
\pgfpathlineto{\pgfqpoint{4.655544in}{0.650442in}}%
\pgfpathlineto{\pgfqpoint{4.697248in}{0.650446in}}%
\pgfpathlineto{\pgfqpoint{4.738952in}{0.650450in}}%
\pgfpathlineto{\pgfqpoint{4.780656in}{0.650454in}}%
\pgfpathlineto{\pgfqpoint{4.822360in}{0.650458in}}%
\pgfpathlineto{\pgfqpoint{4.864064in}{0.650461in}}%
\pgfpathlineto{\pgfqpoint{4.905768in}{0.650465in}}%
\pgfpathlineto{\pgfqpoint{4.947472in}{0.650469in}}%
\pgfpathlineto{\pgfqpoint{4.989175in}{0.650473in}}%
\pgfpathlineto{\pgfqpoint{5.030879in}{0.650477in}}%
\pgfpathlineto{\pgfqpoint{5.072583in}{0.650480in}}%
\pgfusepath{stroke}%
\end{pgfscope}%
\begin{pgfscope}%
\pgfpathrectangle{\pgfqpoint{0.693677in}{0.524170in}}{\pgfqpoint{4.587426in}{2.558193in}}%
\pgfusepath{clip}%
\pgfsetrectcap%
\pgfsetroundjoin%
\pgfsetlinewidth{1.505625pt}%
\definecolor{currentstroke}{rgb}{0.870588,0.560784,0.019608}%
\pgfsetstrokecolor{currentstroke}%
\pgfsetstrokeopacity{0.700000}%
\pgfsetdash{}{0pt}%
\pgfpathmoveto{\pgfqpoint{0.902196in}{0.640451in}}%
\pgfpathlineto{\pgfqpoint{0.943900in}{0.689725in}}%
\pgfpathlineto{\pgfqpoint{0.985604in}{0.733798in}}%
\pgfpathlineto{\pgfqpoint{1.027308in}{0.772344in}}%
\pgfpathlineto{\pgfqpoint{1.069012in}{0.805006in}}%
\pgfpathlineto{\pgfqpoint{1.110716in}{0.831383in}}%
\pgfpathlineto{\pgfqpoint{1.152419in}{0.851031in}}%
\pgfpathlineto{\pgfqpoint{1.194123in}{0.863451in}}%
\pgfpathlineto{\pgfqpoint{1.235827in}{0.868085in}}%
\pgfpathlineto{\pgfqpoint{1.277531in}{0.868179in}}%
\pgfpathlineto{\pgfqpoint{1.319235in}{0.868261in}}%
\pgfpathlineto{\pgfqpoint{1.360939in}{0.868342in}}%
\pgfpathlineto{\pgfqpoint{1.402643in}{0.868424in}}%
\pgfpathlineto{\pgfqpoint{1.444346in}{0.868506in}}%
\pgfpathlineto{\pgfqpoint{1.486050in}{0.868588in}}%
\pgfpathlineto{\pgfqpoint{1.527754in}{0.868669in}}%
\pgfpathlineto{\pgfqpoint{1.569458in}{0.868751in}}%
\pgfpathlineto{\pgfqpoint{1.611162in}{0.868833in}}%
\pgfpathlineto{\pgfqpoint{1.652866in}{0.868914in}}%
\pgfpathlineto{\pgfqpoint{1.694570in}{0.868996in}}%
\pgfpathlineto{\pgfqpoint{1.736274in}{0.869078in}}%
\pgfpathlineto{\pgfqpoint{1.777977in}{0.869159in}}%
\pgfpathlineto{\pgfqpoint{1.819681in}{0.869241in}}%
\pgfpathlineto{\pgfqpoint{1.861385in}{0.869323in}}%
\pgfpathlineto{\pgfqpoint{1.903089in}{0.869404in}}%
\pgfpathlineto{\pgfqpoint{1.944793in}{0.869486in}}%
\pgfpathlineto{\pgfqpoint{1.986497in}{0.869567in}}%
\pgfpathlineto{\pgfqpoint{2.028201in}{0.869649in}}%
\pgfpathlineto{\pgfqpoint{2.069905in}{0.869731in}}%
\pgfpathlineto{\pgfqpoint{2.111608in}{0.869812in}}%
\pgfpathlineto{\pgfqpoint{2.153312in}{0.869894in}}%
\pgfpathlineto{\pgfqpoint{2.195016in}{0.869975in}}%
\pgfpathlineto{\pgfqpoint{2.236720in}{0.870057in}}%
\pgfpathlineto{\pgfqpoint{2.278424in}{0.870139in}}%
\pgfpathlineto{\pgfqpoint{2.320128in}{0.870220in}}%
\pgfpathlineto{\pgfqpoint{2.361832in}{0.870302in}}%
\pgfpathlineto{\pgfqpoint{2.403536in}{0.870383in}}%
\pgfpathlineto{\pgfqpoint{2.445239in}{0.870465in}}%
\pgfpathlineto{\pgfqpoint{2.486943in}{0.870546in}}%
\pgfpathlineto{\pgfqpoint{2.528647in}{0.870628in}}%
\pgfpathlineto{\pgfqpoint{2.570351in}{0.870709in}}%
\pgfpathlineto{\pgfqpoint{2.612055in}{0.870791in}}%
\pgfpathlineto{\pgfqpoint{2.653759in}{0.870872in}}%
\pgfpathlineto{\pgfqpoint{2.695463in}{0.870954in}}%
\pgfpathlineto{\pgfqpoint{2.737166in}{0.871035in}}%
\pgfpathlineto{\pgfqpoint{2.778870in}{0.871117in}}%
\pgfpathlineto{\pgfqpoint{2.820574in}{0.871198in}}%
\pgfpathlineto{\pgfqpoint{2.862278in}{0.871280in}}%
\pgfpathlineto{\pgfqpoint{2.903982in}{0.871361in}}%
\pgfpathlineto{\pgfqpoint{2.945686in}{0.871443in}}%
\pgfpathlineto{\pgfqpoint{2.987390in}{0.871524in}}%
\pgfpathlineto{\pgfqpoint{3.029094in}{0.871606in}}%
\pgfpathlineto{\pgfqpoint{3.070797in}{0.871687in}}%
\pgfpathlineto{\pgfqpoint{3.112501in}{0.871769in}}%
\pgfpathlineto{\pgfqpoint{3.154205in}{0.871850in}}%
\pgfpathlineto{\pgfqpoint{3.195909in}{0.871932in}}%
\pgfpathlineto{\pgfqpoint{3.237613in}{0.872013in}}%
\pgfpathlineto{\pgfqpoint{3.279317in}{0.872094in}}%
\pgfpathlineto{\pgfqpoint{3.321021in}{0.872176in}}%
\pgfpathlineto{\pgfqpoint{3.362725in}{0.872257in}}%
\pgfpathlineto{\pgfqpoint{3.404428in}{0.872339in}}%
\pgfpathlineto{\pgfqpoint{3.446132in}{0.872420in}}%
\pgfpathlineto{\pgfqpoint{3.487836in}{0.872502in}}%
\pgfpathlineto{\pgfqpoint{3.529540in}{0.872583in}}%
\pgfpathlineto{\pgfqpoint{3.571244in}{0.872664in}}%
\pgfpathlineto{\pgfqpoint{3.612948in}{0.872746in}}%
\pgfpathlineto{\pgfqpoint{3.654652in}{0.872827in}}%
\pgfpathlineto{\pgfqpoint{3.696355in}{0.872908in}}%
\pgfpathlineto{\pgfqpoint{3.738059in}{0.872990in}}%
\pgfpathlineto{\pgfqpoint{3.779763in}{0.873071in}}%
\pgfpathlineto{\pgfqpoint{3.821467in}{0.873152in}}%
\pgfpathlineto{\pgfqpoint{3.863171in}{0.873234in}}%
\pgfpathlineto{\pgfqpoint{3.904875in}{0.873315in}}%
\pgfpathlineto{\pgfqpoint{3.946579in}{0.873396in}}%
\pgfpathlineto{\pgfqpoint{3.988283in}{0.873478in}}%
\pgfpathlineto{\pgfqpoint{4.029986in}{0.873559in}}%
\pgfpathlineto{\pgfqpoint{4.071690in}{0.873640in}}%
\pgfpathlineto{\pgfqpoint{4.113394in}{0.873722in}}%
\pgfpathlineto{\pgfqpoint{4.155098in}{0.873803in}}%
\pgfpathlineto{\pgfqpoint{4.196802in}{0.873884in}}%
\pgfpathlineto{\pgfqpoint{4.238506in}{0.873966in}}%
\pgfpathlineto{\pgfqpoint{4.280210in}{0.874047in}}%
\pgfpathlineto{\pgfqpoint{4.321914in}{0.874128in}}%
\pgfpathlineto{\pgfqpoint{4.363617in}{0.874209in}}%
\pgfpathlineto{\pgfqpoint{4.405321in}{0.874291in}}%
\pgfpathlineto{\pgfqpoint{4.447025in}{0.874372in}}%
\pgfpathlineto{\pgfqpoint{4.488729in}{0.874453in}}%
\pgfpathlineto{\pgfqpoint{4.530433in}{0.874534in}}%
\pgfpathlineto{\pgfqpoint{4.572137in}{0.874616in}}%
\pgfpathlineto{\pgfqpoint{4.613841in}{0.874697in}}%
\pgfpathlineto{\pgfqpoint{4.655544in}{0.874778in}}%
\pgfpathlineto{\pgfqpoint{4.697248in}{0.874859in}}%
\pgfpathlineto{\pgfqpoint{4.738952in}{0.874940in}}%
\pgfpathlineto{\pgfqpoint{4.780656in}{0.875022in}}%
\pgfpathlineto{\pgfqpoint{4.822360in}{0.875103in}}%
\pgfpathlineto{\pgfqpoint{4.864064in}{0.875184in}}%
\pgfpathlineto{\pgfqpoint{4.905768in}{0.875265in}}%
\pgfpathlineto{\pgfqpoint{4.947472in}{0.875346in}}%
\pgfpathlineto{\pgfqpoint{4.989175in}{0.875428in}}%
\pgfpathlineto{\pgfqpoint{5.030879in}{0.875509in}}%
\pgfpathlineto{\pgfqpoint{5.072583in}{0.875590in}}%
\pgfusepath{stroke}%
\end{pgfscope}%
\begin{pgfscope}%
\pgfpathrectangle{\pgfqpoint{0.693677in}{0.524170in}}{\pgfqpoint{4.587426in}{2.558193in}}%
\pgfusepath{clip}%
\pgfsetrectcap%
\pgfsetroundjoin%
\pgfsetlinewidth{1.505625pt}%
\definecolor{currentstroke}{rgb}{0.007843,0.619608,0.450980}%
\pgfsetstrokecolor{currentstroke}%
\pgfsetstrokeopacity{0.700000}%
\pgfsetdash{}{0pt}%
\pgfpathmoveto{\pgfqpoint{0.902196in}{0.640451in}}%
\pgfpathlineto{\pgfqpoint{0.943900in}{0.728494in}}%
\pgfpathlineto{\pgfqpoint{0.985604in}{0.813592in}}%
\pgfpathlineto{\pgfqpoint{1.027308in}{0.895624in}}%
\pgfpathlineto{\pgfqpoint{1.069012in}{0.974463in}}%
\pgfpathlineto{\pgfqpoint{1.110716in}{1.049972in}}%
\pgfpathlineto{\pgfqpoint{1.152419in}{1.122007in}}%
\pgfpathlineto{\pgfqpoint{1.194123in}{1.190414in}}%
\pgfpathlineto{\pgfqpoint{1.235827in}{1.255027in}}%
\pgfpathlineto{\pgfqpoint{1.277531in}{1.315669in}}%
\pgfpathlineto{\pgfqpoint{1.319235in}{1.372152in}}%
\pgfpathlineto{\pgfqpoint{1.360939in}{1.424273in}}%
\pgfpathlineto{\pgfqpoint{1.402643in}{1.471814in}}%
\pgfpathlineto{\pgfqpoint{1.444346in}{1.514540in}}%
\pgfpathlineto{\pgfqpoint{1.486050in}{1.552196in}}%
\pgfpathlineto{\pgfqpoint{1.527754in}{1.584509in}}%
\pgfpathlineto{\pgfqpoint{1.569458in}{1.611183in}}%
\pgfpathlineto{\pgfqpoint{1.611162in}{1.631894in}}%
\pgfpathlineto{\pgfqpoint{1.652866in}{1.646292in}}%
\pgfpathlineto{\pgfqpoint{1.694570in}{1.653994in}}%
\pgfpathlineto{\pgfqpoint{1.736274in}{1.655368in}}%
\pgfpathlineto{\pgfqpoint{1.777977in}{1.655683in}}%
\pgfpathlineto{\pgfqpoint{1.819681in}{1.655999in}}%
\pgfpathlineto{\pgfqpoint{1.861385in}{1.656314in}}%
\pgfpathlineto{\pgfqpoint{1.903089in}{1.656630in}}%
\pgfpathlineto{\pgfqpoint{1.944793in}{1.656946in}}%
\pgfpathlineto{\pgfqpoint{1.986497in}{1.657261in}}%
\pgfpathlineto{\pgfqpoint{2.028201in}{1.657576in}}%
\pgfpathlineto{\pgfqpoint{2.069905in}{1.657891in}}%
\pgfpathlineto{\pgfqpoint{2.111608in}{1.658206in}}%
\pgfpathlineto{\pgfqpoint{2.153312in}{1.658522in}}%
\pgfpathlineto{\pgfqpoint{2.195016in}{1.658837in}}%
\pgfpathlineto{\pgfqpoint{2.236720in}{1.659152in}}%
\pgfpathlineto{\pgfqpoint{2.278424in}{1.659467in}}%
\pgfpathlineto{\pgfqpoint{2.320128in}{1.659782in}}%
\pgfpathlineto{\pgfqpoint{2.361832in}{1.660097in}}%
\pgfpathlineto{\pgfqpoint{2.403536in}{1.660411in}}%
\pgfpathlineto{\pgfqpoint{2.445239in}{1.660726in}}%
\pgfpathlineto{\pgfqpoint{2.486943in}{1.661041in}}%
\pgfpathlineto{\pgfqpoint{2.528647in}{1.661356in}}%
\pgfpathlineto{\pgfqpoint{2.570351in}{1.661670in}}%
\pgfpathlineto{\pgfqpoint{2.612055in}{1.661985in}}%
\pgfpathlineto{\pgfqpoint{2.653759in}{1.662299in}}%
\pgfpathlineto{\pgfqpoint{2.695463in}{1.662614in}}%
\pgfpathlineto{\pgfqpoint{2.737166in}{1.662928in}}%
\pgfpathlineto{\pgfqpoint{2.778870in}{1.663242in}}%
\pgfpathlineto{\pgfqpoint{2.820574in}{1.663557in}}%
\pgfpathlineto{\pgfqpoint{2.862278in}{1.663871in}}%
\pgfpathlineto{\pgfqpoint{2.903982in}{1.664185in}}%
\pgfpathlineto{\pgfqpoint{2.945686in}{1.664499in}}%
\pgfpathlineto{\pgfqpoint{2.987390in}{1.664814in}}%
\pgfpathlineto{\pgfqpoint{3.029094in}{1.665128in}}%
\pgfpathlineto{\pgfqpoint{3.070797in}{1.665442in}}%
\pgfpathlineto{\pgfqpoint{3.112501in}{1.665756in}}%
\pgfpathlineto{\pgfqpoint{3.154205in}{1.666070in}}%
\pgfpathlineto{\pgfqpoint{3.195909in}{1.666383in}}%
\pgfpathlineto{\pgfqpoint{3.237613in}{1.666698in}}%
\pgfpathlineto{\pgfqpoint{3.279317in}{1.667011in}}%
\pgfpathlineto{\pgfqpoint{3.321021in}{1.667325in}}%
\pgfpathlineto{\pgfqpoint{3.362725in}{1.667639in}}%
\pgfpathlineto{\pgfqpoint{3.404428in}{1.667952in}}%
\pgfpathlineto{\pgfqpoint{3.446132in}{1.668266in}}%
\pgfpathlineto{\pgfqpoint{3.487836in}{1.668579in}}%
\pgfpathlineto{\pgfqpoint{3.529540in}{1.668892in}}%
\pgfpathlineto{\pgfqpoint{3.571244in}{1.669206in}}%
\pgfpathlineto{\pgfqpoint{3.612948in}{1.669519in}}%
\pgfpathlineto{\pgfqpoint{3.654652in}{1.669833in}}%
\pgfpathlineto{\pgfqpoint{3.696355in}{1.670146in}}%
\pgfpathlineto{\pgfqpoint{3.738059in}{1.670459in}}%
\pgfpathlineto{\pgfqpoint{3.779763in}{1.670772in}}%
\pgfpathlineto{\pgfqpoint{3.821467in}{1.671086in}}%
\pgfpathlineto{\pgfqpoint{3.863171in}{1.671399in}}%
\pgfpathlineto{\pgfqpoint{3.904875in}{1.671712in}}%
\pgfpathlineto{\pgfqpoint{3.946579in}{1.672025in}}%
\pgfpathlineto{\pgfqpoint{3.988283in}{1.672338in}}%
\pgfpathlineto{\pgfqpoint{4.029986in}{1.672650in}}%
\pgfpathlineto{\pgfqpoint{4.071690in}{1.672963in}}%
\pgfpathlineto{\pgfqpoint{4.113394in}{1.673276in}}%
\pgfpathlineto{\pgfqpoint{4.155098in}{1.673589in}}%
\pgfpathlineto{\pgfqpoint{4.196802in}{1.673901in}}%
\pgfpathlineto{\pgfqpoint{4.238506in}{1.674214in}}%
\pgfpathlineto{\pgfqpoint{4.280210in}{1.674526in}}%
\pgfpathlineto{\pgfqpoint{4.321914in}{1.674839in}}%
\pgfpathlineto{\pgfqpoint{4.363617in}{1.675152in}}%
\pgfpathlineto{\pgfqpoint{4.405321in}{1.675464in}}%
\pgfpathlineto{\pgfqpoint{4.447025in}{1.675776in}}%
\pgfpathlineto{\pgfqpoint{4.488729in}{1.676088in}}%
\pgfpathlineto{\pgfqpoint{4.530433in}{1.676401in}}%
\pgfpathlineto{\pgfqpoint{4.572137in}{1.676713in}}%
\pgfpathlineto{\pgfqpoint{4.613841in}{1.677025in}}%
\pgfpathlineto{\pgfqpoint{4.655544in}{1.677337in}}%
\pgfpathlineto{\pgfqpoint{4.697248in}{1.677649in}}%
\pgfpathlineto{\pgfqpoint{4.738952in}{1.677961in}}%
\pgfpathlineto{\pgfqpoint{4.780656in}{1.678273in}}%
\pgfpathlineto{\pgfqpoint{4.822360in}{1.678585in}}%
\pgfpathlineto{\pgfqpoint{4.864064in}{1.678897in}}%
\pgfpathlineto{\pgfqpoint{4.905768in}{1.679209in}}%
\pgfpathlineto{\pgfqpoint{4.947472in}{1.679521in}}%
\pgfpathlineto{\pgfqpoint{4.989175in}{1.679832in}}%
\pgfpathlineto{\pgfqpoint{5.030879in}{1.680144in}}%
\pgfpathlineto{\pgfqpoint{5.072583in}{1.680456in}}%
\pgfusepath{stroke}%
\end{pgfscope}%
\begin{pgfscope}%
\pgfpathrectangle{\pgfqpoint{0.693677in}{0.524170in}}{\pgfqpoint{4.587426in}{2.558193in}}%
\pgfusepath{clip}%
\pgfsetrectcap%
\pgfsetroundjoin%
\pgfsetlinewidth{1.505625pt}%
\definecolor{currentstroke}{rgb}{0.835294,0.368627,0.000000}%
\pgfsetstrokecolor{currentstroke}%
\pgfsetstrokeopacity{0.700000}%
\pgfsetdash{}{0pt}%
\pgfpathmoveto{\pgfqpoint{0.902196in}{0.640451in}}%
\pgfpathlineto{\pgfqpoint{0.943900in}{0.753268in}}%
\pgfpathlineto{\pgfqpoint{0.985604in}{0.864259in}}%
\pgfpathlineto{\pgfqpoint{1.027308in}{0.973367in}}%
\pgfpathlineto{\pgfqpoint{1.069012in}{1.080532in}}%
\pgfpathlineto{\pgfqpoint{1.110716in}{1.185691in}}%
\pgfpathlineto{\pgfqpoint{1.152419in}{1.288781in}}%
\pgfpathlineto{\pgfqpoint{1.194123in}{1.389732in}}%
\pgfpathlineto{\pgfqpoint{1.235827in}{1.488473in}}%
\pgfpathlineto{\pgfqpoint{1.277531in}{1.584929in}}%
\pgfpathlineto{\pgfqpoint{1.319235in}{1.679020in}}%
\pgfpathlineto{\pgfqpoint{1.360939in}{1.770664in}}%
\pgfpathlineto{\pgfqpoint{1.402643in}{1.859775in}}%
\pgfpathlineto{\pgfqpoint{1.444346in}{1.946260in}}%
\pgfpathlineto{\pgfqpoint{1.486050in}{2.030024in}}%
\pgfpathlineto{\pgfqpoint{1.527754in}{2.110964in}}%
\pgfpathlineto{\pgfqpoint{1.569458in}{2.188972in}}%
\pgfpathlineto{\pgfqpoint{1.611162in}{2.263937in}}%
\pgfpathlineto{\pgfqpoint{1.652866in}{2.335737in}}%
\pgfpathlineto{\pgfqpoint{1.694570in}{2.404246in}}%
\pgfpathlineto{\pgfqpoint{1.736274in}{2.469330in}}%
\pgfpathlineto{\pgfqpoint{1.777977in}{2.530846in}}%
\pgfpathlineto{\pgfqpoint{1.819681in}{2.588642in}}%
\pgfpathlineto{\pgfqpoint{1.861385in}{2.642557in}}%
\pgfpathlineto{\pgfqpoint{1.903089in}{2.692420in}}%
\pgfpathlineto{\pgfqpoint{1.944793in}{2.738048in}}%
\pgfpathlineto{\pgfqpoint{1.986497in}{2.779244in}}%
\pgfpathlineto{\pgfqpoint{2.028201in}{2.815801in}}%
\pgfpathlineto{\pgfqpoint{2.069905in}{2.847493in}}%
\pgfpathlineto{\pgfqpoint{2.111608in}{2.874082in}}%
\pgfpathlineto{\pgfqpoint{2.153312in}{2.895307in}}%
\pgfpathlineto{\pgfqpoint{2.195016in}{2.910892in}}%
\pgfpathlineto{\pgfqpoint{2.236720in}{2.920536in}}%
\pgfpathlineto{\pgfqpoint{2.278424in}{2.923934in}}%
\pgfpathlineto{\pgfqpoint{2.320128in}{2.924567in}}%
\pgfpathlineto{\pgfqpoint{2.361832in}{2.925200in}}%
\pgfpathlineto{\pgfqpoint{2.403536in}{2.925834in}}%
\pgfpathlineto{\pgfqpoint{2.445239in}{2.926467in}}%
\pgfpathlineto{\pgfqpoint{2.486943in}{2.927100in}}%
\pgfpathlineto{\pgfqpoint{2.528647in}{2.927733in}}%
\pgfpathlineto{\pgfqpoint{2.570351in}{2.928365in}}%
\pgfpathlineto{\pgfqpoint{2.612055in}{2.928998in}}%
\pgfpathlineto{\pgfqpoint{2.653759in}{2.929631in}}%
\pgfpathlineto{\pgfqpoint{2.695463in}{2.930263in}}%
\pgfpathlineto{\pgfqpoint{2.737166in}{2.930895in}}%
\pgfpathlineto{\pgfqpoint{2.778870in}{2.931527in}}%
\pgfpathlineto{\pgfqpoint{2.820574in}{2.932159in}}%
\pgfpathlineto{\pgfqpoint{2.862278in}{2.932791in}}%
\pgfpathlineto{\pgfqpoint{2.903982in}{2.933423in}}%
\pgfpathlineto{\pgfqpoint{2.945686in}{2.934054in}}%
\pgfpathlineto{\pgfqpoint{2.987390in}{2.934685in}}%
\pgfpathlineto{\pgfqpoint{3.029094in}{2.935317in}}%
\pgfpathlineto{\pgfqpoint{3.070797in}{2.935948in}}%
\pgfpathlineto{\pgfqpoint{3.112501in}{2.936579in}}%
\pgfpathlineto{\pgfqpoint{3.154205in}{2.937209in}}%
\pgfpathlineto{\pgfqpoint{3.195909in}{2.937840in}}%
\pgfpathlineto{\pgfqpoint{3.237613in}{2.938471in}}%
\pgfpathlineto{\pgfqpoint{3.279317in}{2.939101in}}%
\pgfpathlineto{\pgfqpoint{3.321021in}{2.939732in}}%
\pgfpathlineto{\pgfqpoint{3.362725in}{2.940361in}}%
\pgfpathlineto{\pgfqpoint{3.404428in}{2.940992in}}%
\pgfpathlineto{\pgfqpoint{3.446132in}{2.941621in}}%
\pgfpathlineto{\pgfqpoint{3.487836in}{2.942251in}}%
\pgfpathlineto{\pgfqpoint{3.529540in}{2.942881in}}%
\pgfpathlineto{\pgfqpoint{3.571244in}{2.943510in}}%
\pgfpathlineto{\pgfqpoint{3.612948in}{2.944140in}}%
\pgfpathlineto{\pgfqpoint{3.654652in}{2.944769in}}%
\pgfpathlineto{\pgfqpoint{3.696355in}{2.945398in}}%
\pgfpathlineto{\pgfqpoint{3.738059in}{2.946027in}}%
\pgfpathlineto{\pgfqpoint{3.779763in}{2.946656in}}%
\pgfpathlineto{\pgfqpoint{3.821467in}{2.947284in}}%
\pgfpathlineto{\pgfqpoint{3.863171in}{2.947913in}}%
\pgfpathlineto{\pgfqpoint{3.904875in}{2.948541in}}%
\pgfpathlineto{\pgfqpoint{3.946579in}{2.949170in}}%
\pgfpathlineto{\pgfqpoint{3.988283in}{2.949798in}}%
\pgfpathlineto{\pgfqpoint{4.029986in}{2.950426in}}%
\pgfpathlineto{\pgfqpoint{4.071690in}{2.951053in}}%
\pgfpathlineto{\pgfqpoint{4.113394in}{2.951681in}}%
\pgfpathlineto{\pgfqpoint{4.155098in}{2.952309in}}%
\pgfpathlineto{\pgfqpoint{4.196802in}{2.952936in}}%
\pgfpathlineto{\pgfqpoint{4.238506in}{2.953563in}}%
\pgfpathlineto{\pgfqpoint{4.280210in}{2.954191in}}%
\pgfpathlineto{\pgfqpoint{4.321914in}{2.954818in}}%
\pgfpathlineto{\pgfqpoint{4.363617in}{2.955445in}}%
\pgfpathlineto{\pgfqpoint{4.405321in}{2.956071in}}%
\pgfpathlineto{\pgfqpoint{4.447025in}{2.956698in}}%
\pgfpathlineto{\pgfqpoint{4.488729in}{2.957325in}}%
\pgfpathlineto{\pgfqpoint{4.530433in}{2.957951in}}%
\pgfpathlineto{\pgfqpoint{4.572137in}{2.958577in}}%
\pgfpathlineto{\pgfqpoint{4.613841in}{2.959203in}}%
\pgfpathlineto{\pgfqpoint{4.655544in}{2.959829in}}%
\pgfpathlineto{\pgfqpoint{4.697248in}{2.960455in}}%
\pgfpathlineto{\pgfqpoint{4.738952in}{2.961081in}}%
\pgfpathlineto{\pgfqpoint{4.780656in}{2.961706in}}%
\pgfpathlineto{\pgfqpoint{4.822360in}{2.962332in}}%
\pgfpathlineto{\pgfqpoint{4.864064in}{2.962957in}}%
\pgfpathlineto{\pgfqpoint{4.905768in}{2.963582in}}%
\pgfpathlineto{\pgfqpoint{4.947472in}{2.964207in}}%
\pgfpathlineto{\pgfqpoint{4.989175in}{2.964832in}}%
\pgfpathlineto{\pgfqpoint{5.030879in}{2.965457in}}%
\pgfpathlineto{\pgfqpoint{5.072583in}{2.966081in}}%
\pgfusepath{stroke}%
\end{pgfscope}%
\begin{pgfscope}%
\pgfsetrectcap%
\pgfsetmiterjoin%
\pgfsetlinewidth{0.803000pt}%
\definecolor{currentstroke}{rgb}{0.000000,0.000000,0.000000}%
\pgfsetstrokecolor{currentstroke}%
\pgfsetdash{}{0pt}%
\pgfpathmoveto{\pgfqpoint{0.693677in}{0.524170in}}%
\pgfpathlineto{\pgfqpoint{0.693677in}{3.082363in}}%
\pgfusepath{stroke}%
\end{pgfscope}%
\begin{pgfscope}%
\pgfsetrectcap%
\pgfsetmiterjoin%
\pgfsetlinewidth{0.803000pt}%
\definecolor{currentstroke}{rgb}{0.000000,0.000000,0.000000}%
\pgfsetstrokecolor{currentstroke}%
\pgfsetdash{}{0pt}%
\pgfpathmoveto{\pgfqpoint{5.281103in}{0.524170in}}%
\pgfpathlineto{\pgfqpoint{5.281103in}{3.082363in}}%
\pgfusepath{stroke}%
\end{pgfscope}%
\begin{pgfscope}%
\pgfsetrectcap%
\pgfsetmiterjoin%
\pgfsetlinewidth{0.803000pt}%
\definecolor{currentstroke}{rgb}{0.000000,0.000000,0.000000}%
\pgfsetstrokecolor{currentstroke}%
\pgfsetdash{}{0pt}%
\pgfpathmoveto{\pgfqpoint{0.693677in}{0.524170in}}%
\pgfpathlineto{\pgfqpoint{5.281103in}{0.524170in}}%
\pgfusepath{stroke}%
\end{pgfscope}%
\begin{pgfscope}%
\pgfsetrectcap%
\pgfsetmiterjoin%
\pgfsetlinewidth{0.803000pt}%
\definecolor{currentstroke}{rgb}{0.000000,0.000000,0.000000}%
\pgfsetstrokecolor{currentstroke}%
\pgfsetdash{}{0pt}%
\pgfpathmoveto{\pgfqpoint{0.693677in}{3.082363in}}%
\pgfpathlineto{\pgfqpoint{5.281103in}{3.082363in}}%
\pgfusepath{stroke}%
\end{pgfscope}%
\begin{pgfscope}%
\pgfsetbuttcap%
\pgfsetmiterjoin%
\definecolor{currentfill}{rgb}{1.000000,1.000000,1.000000}%
\pgfsetfillcolor{currentfill}%
\pgfsetfillopacity{0.800000}%
\pgfsetlinewidth{1.003750pt}%
\definecolor{currentstroke}{rgb}{0.800000,0.800000,0.800000}%
\pgfsetstrokecolor{currentstroke}%
\pgfsetstrokeopacity{0.800000}%
\pgfsetdash{}{0pt}%
\pgfpathmoveto{\pgfqpoint{0.771455in}{2.373919in}}%
\pgfpathlineto{\pgfqpoint{1.857795in}{2.373919in}}%
\pgfpathquadraticcurveto{\pgfqpoint{1.880017in}{2.373919in}}{\pgfqpoint{1.880017in}{2.396141in}}%
\pgfpathlineto{\pgfqpoint{1.880017in}{3.004585in}}%
\pgfpathquadraticcurveto{\pgfqpoint{1.880017in}{3.026807in}}{\pgfqpoint{1.857795in}{3.026807in}}%
\pgfpathlineto{\pgfqpoint{0.771455in}{3.026807in}}%
\pgfpathquadraticcurveto{\pgfqpoint{0.749232in}{3.026807in}}{\pgfqpoint{0.749232in}{3.004585in}}%
\pgfpathlineto{\pgfqpoint{0.749232in}{2.396141in}}%
\pgfpathquadraticcurveto{\pgfqpoint{0.749232in}{2.373919in}}{\pgfqpoint{0.771455in}{2.373919in}}%
\pgfpathlineto{\pgfqpoint{0.771455in}{2.373919in}}%
\pgfpathclose%
\pgfusepath{stroke,fill}%
\end{pgfscope}%
\begin{pgfscope}%
\pgfsetrectcap%
\pgfsetroundjoin%
\pgfsetlinewidth{1.505625pt}%
\definecolor{currentstroke}{rgb}{0.003922,0.450980,0.698039}%
\pgfsetstrokecolor{currentstroke}%
\pgfsetstrokeopacity{0.700000}%
\pgfsetdash{}{0pt}%
\pgfpathmoveto{\pgfqpoint{0.793677in}{2.943474in}}%
\pgfpathlineto{\pgfqpoint{0.904788in}{2.943474in}}%
\pgfpathlineto{\pgfqpoint{1.015899in}{2.943474in}}%
\pgfusepath{stroke}%
\end{pgfscope}%
\begin{pgfscope}%
\definecolor{textcolor}{rgb}{0.000000,0.000000,0.000000}%
\pgfsetstrokecolor{textcolor}%
\pgfsetfillcolor{textcolor}%
\pgftext[x=1.104788in,y=2.904585in,left,base]{\color{textcolor}\rmfamily\fontsize{8.000000}{9.600000}\selectfont \(\displaystyle V_{GS} = \qty{-3.5}{\V}\)}%
\end{pgfscope}%
\begin{pgfscope}%
\pgfsetrectcap%
\pgfsetroundjoin%
\pgfsetlinewidth{1.505625pt}%
\definecolor{currentstroke}{rgb}{0.870588,0.560784,0.019608}%
\pgfsetstrokecolor{currentstroke}%
\pgfsetstrokeopacity{0.700000}%
\pgfsetdash{}{0pt}%
\pgfpathmoveto{\pgfqpoint{0.793677in}{2.788585in}}%
\pgfpathlineto{\pgfqpoint{0.904788in}{2.788585in}}%
\pgfpathlineto{\pgfqpoint{1.015899in}{2.788585in}}%
\pgfusepath{stroke}%
\end{pgfscope}%
\begin{pgfscope}%
\definecolor{textcolor}{rgb}{0.000000,0.000000,0.000000}%
\pgfsetstrokecolor{textcolor}%
\pgfsetfillcolor{textcolor}%
\pgftext[x=1.104788in,y=2.749696in,left,base]{\color{textcolor}\rmfamily\fontsize{8.000000}{9.600000}\selectfont \(\displaystyle V_{GS} = \qty{-4}{\V}\)}%
\end{pgfscope}%
\begin{pgfscope}%
\pgfsetrectcap%
\pgfsetroundjoin%
\pgfsetlinewidth{1.505625pt}%
\definecolor{currentstroke}{rgb}{0.007843,0.619608,0.450980}%
\pgfsetstrokecolor{currentstroke}%
\pgfsetstrokeopacity{0.700000}%
\pgfsetdash{}{0pt}%
\pgfpathmoveto{\pgfqpoint{0.793677in}{2.633696in}}%
\pgfpathlineto{\pgfqpoint{0.904788in}{2.633696in}}%
\pgfpathlineto{\pgfqpoint{1.015899in}{2.633696in}}%
\pgfusepath{stroke}%
\end{pgfscope}%
\begin{pgfscope}%
\definecolor{textcolor}{rgb}{0.000000,0.000000,0.000000}%
\pgfsetstrokecolor{textcolor}%
\pgfsetfillcolor{textcolor}%
\pgftext[x=1.104788in,y=2.594808in,left,base]{\color{textcolor}\rmfamily\fontsize{8.000000}{9.600000}\selectfont \(\displaystyle V_{GS} = \qty{-4.5}{\V}\)}%
\end{pgfscope}%
\begin{pgfscope}%
\pgfsetrectcap%
\pgfsetroundjoin%
\pgfsetlinewidth{1.505625pt}%
\definecolor{currentstroke}{rgb}{0.835294,0.368627,0.000000}%
\pgfsetstrokecolor{currentstroke}%
\pgfsetstrokeopacity{0.700000}%
\pgfsetdash{}{0pt}%
\pgfpathmoveto{\pgfqpoint{0.793677in}{2.478808in}}%
\pgfpathlineto{\pgfqpoint{0.904788in}{2.478808in}}%
\pgfpathlineto{\pgfqpoint{1.015899in}{2.478808in}}%
\pgfusepath{stroke}%
\end{pgfscope}%
\begin{pgfscope}%
\definecolor{textcolor}{rgb}{0.000000,0.000000,0.000000}%
\pgfsetstrokecolor{textcolor}%
\pgfsetfillcolor{textcolor}%
\pgftext[x=1.104788in,y=2.439919in,left,base]{\color{textcolor}\rmfamily\fontsize{8.000000}{9.600000}\selectfont \(\displaystyle V_{GS} = \qty{-5}{\V}\)}%
\end{pgfscope}%
\end{pgfpicture}%
\makeatother%
\endgroup%
% data/plot_generic.py
    \caption{Simulated drain current for different gate bias voltages of an \device{IRF9610} p-channel MOSFET.}
    \label{fig:fet_curret_gate_bias}
\end{figure}

Figure \ref{fig:fet_curret_gate_bias} shows the current $I_D$ flowing out of the drain of a p-channel MOSFET over the drain-to-source voltage $V_{DS}$ which is applied across the FET. For illustrative purposes an example p-channel MOSFET was chosen and its \textit{Simulation Program with Integrated Circuit Emphasis} (SPICE) model \cite{irf9610_spice,irf9610_spice_better} was used to generate the data, yet the overall shape is the same for all FETs. For more information on modelling MOSFETs in SPICE, \citep[p. 442]{spice_mosfets} can be consulted. There are two regions, the first region, where $V_{DS} > V_{GS} - V_{th}$, demonstrates an almost linear correlation of the channel current and the voltage across the device. This is called the ohmic region, where the MOSFET behaves much like a (gate-) voltage controlled resistor and can be described \cite{shockley_fet_equations} as
\begin{equation}
    I_{D,ohmic} = \underbrace{\kappa (V_{GS} - V_{th}) V_{DS}}_{\text{ohmic}} - \underbrace{\frac 1 2 \kappa V_{DS}^2}_{\text{pinch off}} \, .
\end{equation}
For small voltages $V_{DS}$ the output current is proportional to the applied voltage $V_{DS}$ across the channel just like a normal resistor, giving rise to its name ohmic region. As the voltage increases further $I_D$ starts leveling off because $V_{DS}$ starts affecting the channel conductivity. The channel is slowly getting pinched off at one end and becomes tapered. The reason is that the voltage $V_{DS}$ is dropped across the length of the channel. This voltage drop is linear with $V_{DS}$, resulting in a $-V_{DS}^2$ dependency of the current, reducing the conductivity of the channel. $V_{th}$ is called the threshold voltage of a MOSFET or pinch-off voltage $V_p$ in case of a JFET and is the voltage at which a current starts flowing.

The parameter $\kappa$ is a device specific parameter and depends on process parameters and the geometry of the device.
\begin{equation}
    \kappa = \kappa' \frac W L = \mu C_{ox} \frac W L
\end{equation}
$\mu$ is the electron mobility, which is about \qty{1350}{\square \cm \per \V} for n-channel MOSFETs and about \qty{540}{\square \cm \per \V} for p-channel MOSFETs \cite{fet_equations}. $C_{ox}$ is the gate-oxide capacitance per unit area and determined by the thickness $t_{ox}$ of the silicon dioxide layer of the gate
\begin{equation}
    C_{ox} = \frac{\epsilon_{ox}}{t_{ox}} \approx \frac{3.9 \cdot \epsilon_{0}}{t_{ox}} \approx \frac{\qty{3.45e-11}{\F \per \m}}{t_{ox}}\,,
\end{equation}
$W$ is the width of the channel, and $L$ is the length of the channel.

The letter $\kappa$ is used here instead of the usual $k$ as it is used by \citeauthor{fet_equations} \cite{fet_equations} to avoid confusion with the Boltzmann constant $k_B$. Unfortunately, $\kappa$ is not well controlled \cite{horowitz1989}, because it is not just determined by the size, but also the doping of the material. While the size of the structure can be well controlled to within a few \unit{\nm} using lithography masks, the doping is a matter of temperature and time in a diffusion furnace. The ohmic mode of operation is, for example, used in switches or linear voltage regulators to control the output voltage of the regulator, forming a low impedance voltage source and not the desired current source. This brings up the next region to discuss.

Once the voltage $V_{DS}$ has reached $V_{GS} - V_{th}$, the channel is fully pinched off, any further increase in $V_{DS}$ will not lead to an increase in $I_D$, in other words the output resistance becomes infinite. The MOSFET is said to be pinched-off or in saturation. In practice there still is a small influence of $V_{DS}$ on the channel. While the depth can no longer decrease as its length is \num{0} at one end already, the channel will retract a small amount in length with increasing $V_{DS}$. This is taken into account by the factor $\lambda$, called channel-length modulation. The drain current in saturation can now be described \cite{shockley_fet_equations} as
\begin{equation}
    I_{D,sat} = \underbrace{\frac 1 2 \kappa \left(V_{GS} - V_{th} \right)^2}_{\text{ideal FET}} (1 + \lambda V_{DS}) \, . \label{eqn:mosfet_saturation}
\end{equation}

The parameter $\lambda$ is the first order Taylor expansion of the length dependence of $\kappa$ and typically is small and on the order of \qtyrange[range-units = single, per-mode=power]{0.01}{0.05}{\per \volt} for p-channel MOSFETs \citep[p. 23]{mosfet_flicker_noise}. It mainly depends on the length of the channel to which it is inversely proportional, since the channel length defines the slope of the tapered channel. Sometimes the value $\frac{1}{\lambda}$ is also referred to as the Early voltage $V_A$. It is noteworthy that more modern processes choose a smaller channel length to reduce the on-state resistance of the MOSFET because the main application of a MOSFET nowadays is as a switch. The reduced channel length makes the MOSFET more susceptible to the channel length modulation effect. This will be discussed in more detail in section \ref{sec:component_selection}, when choosing a suitable MOSFET.

Going back to figure \ref{fig:fet_curret_gate_bias} the effect of the channel-length modulation can be seen as a small slope of $I_D$ in the saturation region.

Combining the previous equations, the FET drain current behaviour can be summed up as
\begin{equation}
    I_D = \begin{cases}
        0 & \text{if } V_{GS} - V_{th} < 0\\
        \kappa (V_{GS} - V_{th}) V_{DS} - \frac 1 2 \kappa V_{DS}^2 & \text{if } V_{GS} - V_{th} >= 0 \text{ and } V_{DS} < V_{GS} - V_{th}\\
        \frac 1 2 \kappa \left(V_{GS} - V_{th} \right)^2 (1 + \lambda V_{DS}) & \text{if } V_{GS} - V_{th} >= 0 \text{ and } V_{DS} \geq V_{GS} - V_{th}
    \end{cases}
    \label{eqn:mosfet_id_large_signal}
\end{equation}

The saturation region is the region of interest for building a high output impedance current source, because for a wide range of $V_{DS}$ the current remains almost constant and can be adjusted using the gate voltage $V_{GS}$. As a reminder, for the p-channel MOSFET, all voltages are reversed. $V_{GS}$, $V_{th}$, $V_{DS}$, $\kappa$ and $I_D$ are negative. Some datasheets therefore only give the magnitude of those quantities. The important aspect to remember is that for the p-channel enhancement-mode MOSFET the gate must be biased negative with respect to the source pin by a least the threshold voltage ($V_{GS} < V_{th}$ or $|V_{GS}| > |V_{th}|$) to turn the transistor on and allow current to flow.

Before proceeding to the precision current source in section \ref{sec:precision_current_source}, the concept of conductance and transconductance must be explored. The transconductance describes the relationship of the input voltage with the output current. The conductance is a measure for how well current flows from input to output. The transconductance $g_m$ and the channel conductance $g_{DS}$ are defined as
\begin{align}
    g_{m, sat} &\coloneqq \left. \frac{\partial I_{D,sat}}{\partial V_{GS}} \right|_{V_{DS} = const} = \kappa \left(V_{GS} - V_{th} \right) (1 + \lambda V_{DS}) \, , \label{eqn:mosfet_gm}\\
    &= \sqrt{2 \kappa I_D \left(1+ \lambda V_{DS}\right)} \approx \sqrt{2 \kappa I_D} \label{eqn:mosfet_gm_approximation} \\
    g_{DS, sat} &\coloneqq \left. \frac{\partial I_{D,sat}}{\partial V_{DS}} \right|_{V_{GS} = const} = \frac{1}{2} \kappa \left(V_{GS} - V_{th} \right)^2 \lambda\\
    &= \frac{I_D}{\frac{1}{\lambda} + V_{DS}} = \frac{1}{R_o} \approx I_D \lambda \label{eqn:mosfet_gds}\,.
\end{align}
The transconductance $g_m$, as a measure of the current gain with respect to the gate-source voltage of the MOSFET, is proportional to the square root of the drain current $I_D$. The inverse of the channel conductance $g_{DS}$ is called output resistance $R_o$ and discussed below. Typically the $V_{DS}$ term in the denominator of the output resistance in equation \ref{eqn:mosfet_gds} can be neglected.

The meaning of $g_{m}$ and $g_{GS}$ can be best understood when looking at a mathematical model of the MOSFET. These models come in varying complexity and either as a large-signal or small-signal model. Only the latter is used here. The small-signal model, is a first order Taylor approximation around the working point, for a constant gate-source voltage $V_{GS}$ and constant drain-source $V_{DS}$, hence both $g_{m}$ and $g_{GS}$ are constants.
\begin{align}
    I_D &\approx \frac{\partial I_D}{\partial V_{GS}} \Delta V_{GS} + \frac{\partial I_D}{\partial V_{DS}} \Delta V_{DS}\\
    &= g_{m} \Delta V_{GS} + g_{DS} \Delta V_{DS}\\
    &= g_{m} v_{GS} + \frac{1}{R_o} v_{DS} = i_D \label{eqn:mosfet_id_small_signal}
\end{align}
The lower case letters denote the variables of the small-signal model as they only change very little compared to the working point parameters.
From \ref{eqn:mosfet_id_small_signal} it can be seen that the $g_{DS}$ term adds to the output current and is proportional to $v_{DS}$. Comparing this with figure \ref{fig:ideal_current_source_norton}, the proportionality constant can be identified as $\frac{1}{R_o}$ like proposed above. Just like the ideal current source in figure \ref{fig:ideal_current_source}, the model can be given in the Norton or Thévenin representation, both shown in figure \ref{fig:mostfet_small_signa_model}.
\begin{figure}[hb]
    \centering
    \begin{subfigure}{0.43\linewidth}
        \centering
        \import{figures/}{mosfet_small_signal.tex}
        \caption{Small-signal model of a saturated MOSFET including the output resistance. The output resistance models the channel-length modulation as given by equation \ref{eqn:mosfet_id_small_signal}.}
        \label{fig:mostfet_small_signa_model_model_norton}
    \end{subfigure}
    \begin{subfigure}{0.43\linewidth}
        \centering
        \import{figures/}{mosfet_small_signal_t-model.tex}
        \caption{MOSFET model in Thévenin representation.}
        \label{fig:mostfet_small_signa_model_thevenin}
    \end{subfigure}
    \caption{Equivalent MOSFET models in Norton and Thévenin representations.}
    \label{fig:mostfet_small_signa_model}
\end{figure}

A detailed graphic derivation of the Thévenin representation can be found in \cite{fet_equations}. The Thévenin representation will prove especially valuable when treating circuits with a resistance in the source leg.
The small-signal model now shows that the output impedance is dependent on the channel-length modulation $\lambda$ and $v_{DS}$. Typically, $\frac{1}{\lambda} \gg v_{DS}$, so $\lambda$ is the most important factor governing the output impedance of a MOSFET.

To give an example of the output impedance of a MOSFET, parameters were taken from the aforementioned SPICE model of the \device{IRF9610}. Do note that these parameters of the model are tuned to match certain operating conditions by their creators and only present an estimation of the real MOSFET. Using the example parameters from table \ref{tab:current_source_parameters}, $I_D=\qty{250}{\mA}$, $\lambda = \qty[per-mode=power]{4}{\per \milli \volt}$, $V_{DS}=\qty{3.5}{\V}$ equation \ref{eqn:mosfet_gds} yields
\begin{equation}
    R_{out} = R_{o}\left(I_D=\qty{250}{\mA}, \lambda = \qty[per-mode=power]{4}{\per \milli \volt}\right) = \qty{1014}{\ohm} \overset{V_{DS} = 0}{\approx} \qty{1}{\kilo \ohm} \, , \label{eqn:mosfet_rout_irf9610}
\end{equation}
which is not very convincing as a current source. The insignificant impact of $V_{DS}$ on the output impedance can be seen when dropping the $V_{DS}$ term, which leads to an output impedance of \qty{1}{\kilo \ohm}. In textbooks this dependence is therefore usually neglected. To improve $R_{out}$, the focus thus lies on the $\lambda$ dependence. The model derived from equation \ref{eqn:mosfet_id_small_signal} can be used to do so, leading to the precision current source presented next.

% This can be demonstrated building a simple current source and then cascoding it. A very simple current source can be built using a JFET. As mentioned above, a JFET is a depletion-mode device and is already turned on at $V_{GS} = \qty{0}{\V}$. To turn it off the gate voltage must be increased above the source leg. For an illustration, refer to figure \ref{fig:jfet_curret_gate_bias}, which is very similar to the MOSFET behaviour.
%
% \begin{figure}[ht]
%     \centering
%     \input{images/jfet_current_gate_bias.pgf}
%     \caption{Simulated drain current for different gate bias voltages of a \device{2N5460} p-channel JSFET.}
%     \label{fig:jfet_curret_gate_bias}
% \end{figure}
%
% The topmost curve of figure \ref{fig:jfet_curret_gate_bias} is the case with a direct connection of the gate to the source. Above about $V_{DS}=\qty{4}{\V}$, the JFET works as a current source, although the effect of the Early voltage given in equation \ref{eqn:mosfet_saturation} can be clearly seen with a slight dependence of the output current on $V_{DS}$. With increasing $V_{GS}$, it can be observed that the slope of $I_D$ flattens.
%
%
% An example for a cascode is shown in figure \ref{fig:current_source_jfet_cascode}.
%
% \begin{figure}[ht]
%     \centering
%     \begin{subfigure}[t]{0.3\linewidth}
%         \centering
%         \import{figures/}{current_source_fet_no_bias.tex}
%         \caption{JFET current source.}
%         \label{fig:current_source_jfet_no_bias}
%     \end{subfigure}%
%     %\hfill%
%     \begin{subfigure}[t]{0.3\linewidth}
%         \centering
%         \import{figures/}{current_source_fet_bias.tex}
%         \caption{Self-biased JFET current source.}
%         \label{fig:current_source_jfet_bias}
%     \end{subfigure}%
%     %\hfill%
%     \begin{subfigure}[t]{0.3\linewidth}
%         \centering
%         \import{figures/}{current_source_fet_cascode.tex}
%         \caption{Cascoded JFET current source.}
%         \label{fig:current_source_jfet_cascode}
%     \end{subfigure}
%     \caption{Different types of JFET current sources with increasing output impedance.}
%     \label{fig:current_source_jfet}
% \end{figure}

\subsection{Precision Current Source}%
\label{sec:precision_current_source}
In the previous section \ref{sec:mosfet_current_source} it was shown in equation \ref{eqn:mosfet_id_small_signal} that the output impedance of a MOSFET depends on the channel-length modulation $\lambda$ and is too low for practical purposes. On the quest to improve the output impedance of the MOSFET circuit in figure \ref{fig:mostfet_small_signa_model_model_norton}, the most obvious solution would be to simply add a source resistor $R_s$ into the circuit as shown in in figure \ref{fig:pmos_current_source_resistor}. At first glance this may seem to only add a series resistance to $R_o$, but the attempt is more intriguing and will lead to an even better solution.
\begin{figure}[ht]
    \centering
    \begin{subfigure}[t]{0.45\linewidth}
        \centering
        \import{figures/}{current_source_resistor.tex}
        \caption{MOSFET with source resistor $R_s$ to improve the output impedance $R_{out}$.}
        \label{fig:pmos_current_source_resistor}
    \end{subfigure}%
    \begin{subfigure}[t]{0.45\linewidth}
         \centering
         \import{figures/}{pmos_small_signal_resistor.tex}
         \caption{Small-signal Thévenin model.}
         \label{fig:pmos_current_source_resistor_small_signal}
     \end{subfigure}%
     \caption{Circuit of a MOSFET with source degeneration resistor and equivalent Thévenin model.}
\end{figure}

Before calculating the output impedance, we shall have a look at $v_{GS}$ and the input signal $v_i$ derived from it. With the introduction of the source resistor $R_s$, $v_i$ no longer equals $v_{GS}$, because $\frac{1}{g_m}$ now forms a voltage divider with $R_s$ and it follows
\begin{equation}
    v_{GS} = v_i \frac{\frac{1}{g_m}}{R_s + \frac{1}{g_m}} = v_i \frac{1}{1 + g_m R_s} \,.
\end{equation}
This implies a reduction in gain by the factor $\frac{1}{1 + R_s g_m}$ compared to the previously discussed approach. The cause of this reduction is negative feedback. To understand this, imagine that with a constant $v_i$ and hence a constant current $I_D$ flowing, a changing load resistance is trying to modulate $I_D$. Any increase in $I_D$ will cause the voltage across $R_s$ to rise, reducing $v_{GS}$, because $v_i$ is still constant. The decreasing $v_{GS}$ will then reduce $I_D$, thus introducing negative feedback. Having realized there is negative feedback present, it can be postulated that the reduction in input sensitivity, or effective transconductance, will be passed on to the output impedance. This very interesting relationship will now be derived.

To calculate the output impedance, figure \ref{fig:pmos_current_source_resistor_small_signal} can be simplified by grounding $v_i$, because there is no AC component as there is no current flowing through the insulated MOSFET gate and is not modulated. The load $R_{load}$ resistance must is replaced by an AC test voltage $v_{load}$ to modulate $I_D$. These changes result in the small-signal model shown in figure \ref{fig:pmos_common_gate_amplifier}. This configuration is also called a common-gate amplifier.
\begin{figure}[ht]
    \centering
    \import{figures/}{mosfet_small_signal_cg.tex}
    \caption{Small-signal model of the common-gate amplifier with source resistance $R_s$.}
    \label{fig:pmos_common_gate_amplifier}
\end{figure}

The (dynamic) output impedance is given by
\begin{equation}
    R_{out,cg} = \frac{v_{load}}{i_D}\,, \label{eqn:mosfet_rout}
\end{equation}
with $i_D = i_S$, since there is no gate current. $v_{load}$ can easily be calculated by looking at figure \ref{fig:pmos_common_gate_amplifier} and equals the total voltage across $R_o$ and $R_s$. $v_{GS}$ can also be found, because the gate is grounded. With the resistance $\frac{1}{g_m}$ at one end, the voltage at the source pin must be $-v_{GS}$.
\begin{align}
    v_{load} &= \left(i_D - i\right) R_o + i_S R_s \nonumber\\
    &= \left(i_D - g_m v_{gs}\right) R_o + i_D R_s \nonumber\\
    &= \left(i_D + g_m i_D R_s\right) R_o + i_D R_s \label{eqn:mosfet_cg_vout}
\end{align}
Using equations \ref{eqn:mosfet_rout} and \ref{eqn:mosfet_cg_vout} gives
\begin{equation}
    R_{out,cg} = \left(1 + g_m R_s\right) R_o + R_s \label{eqn:mosfet_cg_rout}
\end{equation}
for the output impedance.

This result is interesting, as it can be be immediately seen that the output impedance scales very quickly with the transconductance $g_m$ and $R_s$. As it was already speculated above, the reduction in the transconductance $\frac{1}{1 + g_m R_s}$ of the MOSFET is transferred to the output impedance, which is increasing by the inverse of the loss in transconductance.

Going back to the quest for an increased output impedance, it is apparent that increasing $R_s$ quickly raises the output impedance, as it scales with $gm_m R_o$, but it would come at the cost of a significantly reduced compliance voltage. Therefore, other means need to be explored. As we have seen, the scale factor $gm_m R_o$ is explained by feedback and this leads to another solution. The amount of feedback can be increased further using an operational amplifier (op-amp) as shown in figure \ref{fig:precision_current_source}.
\begin{figure}[ht]
    \centering
    \import{figures/}{precision_current_source.tex}
    \caption{Transconductance amplifier with a p-channel MOSFET.}
    \label{fig:precision_current_source}
\end{figure}

The output impedance of this transconductance amplifier is amplified by the open-loop gain of the op-amp as shown in appendix \ref{sec:transfer_function_transconductance}, while the transfer function greatly simplifies to
\begin{align}
    R_{out} &\approx A_{ol} \left(g_m R_o R_s + R_o + R_s \right) \nonumber\\
    I_{out} &\approx \frac{V_{ref}}{R_s} \label{eqn:current_source_transfer_function}
\end{align}

In addition to the increased output impedance, the current $I_D = I_{out}$ can now steered by adjusting $V_{ref}$ and is, given sufficient loop gain of the op-amp, no longer dependent on the MOSFET but rather only on the sense resistor $R_s$.

This has the added benefit that it is possible to leverage the tight accuracy and precision of a resistor over the poor specifications of a MOSFET. Resistors can be manufactured with tolerances of less than \qty{100}{\micro \ohm \per \ohm}, which is orders of magnitude better than FETs, which can be matched to low \unit{\percent} values with patience.

Using the example parameters from table \ref{tab:current_source_parameters}, the output impedance in saturation can now be calculated again for $I_{out}=\qty{250}{\mA}$ and the ideal \device{IRF9610} model with the addition of an idealized \device{AD797} op-amp using the worst-case specifications.
\begin{equation}
    R_{out} \approx \qty[per-mode=power]{2}{\volt \per \uV} \left(\qty{0.64}{\siemens}\cdot \qty{1014}{\ohm} \cdot \qty{30}{\ohm} + \qty{1014}{\ohm} + \qty{30}{\ohm} \right) \approx \qty{40}{\giga\ohm} \label{eqn:current_source_output_impedance}
\end{equation}

From these consideration, it can be seen that the open-loop gain and the unity-gain bandwidth of the op-amp essentially determine the properties of the current source, given that $R_{id} \gg R_s$ and $R_o \gg R_s$. This will be important for selecting an operational amplifier later.

The next section will focus on the MOSFET and discuss the compliance voltage of the current source, which was only briefly touched during the introduction. It will give rise to criteria for selecting a MOSFET for the precision current source.

\subsection{Compliance Voltage}%
\label{sec:compliance_voltage}
The compliance voltage of a current source is the maximum voltage it can output to maintain the requested output current. For an ideal current source, the compliance voltage is infinite, but it is obviously limited in the physical world.

The precision current source discussed in section \ref{sec:precision_current_source} has several limiting factors of the compliance voltage, which shall be discussed now. The compliance voltage is taxed most at the maximum output current $I_{out,max}$. Thus for the following discussion, the output is always treated as set to maximum.

Looking at figure \ref{fig:precision_current_source} of the precision current source it is immediately evident that the output voltage can be calculated by subtracting the voltage across the source resistor $V_{R_s}$ and the MOSFET $V_{DS}$ from the supply voltage $V_{sup}$
\begin{equation*}
    V_{out} = V_{sup} - V_{R_s} - V_{DS} = V_{sup} - V_{ref} - V_{DS}\,.
\end{equation*}

The voltage $V_{R_s}$ is given by equation \ref{eqn:current_source_transfer_function} and equal to the setpoint voltage and hence given by the system parameters. This leads to the question of the minimum working point voltage $V_{DS}$ at $I_{out,max}$. As a reminder, from equation \ref{eqn:mosfet_id_large_signal} and figure \ref{fig:fet_curret_gate_bias} one can see that the drain current is almost constant over $V_{DS}$ in the saturation region, and in the ohmic region is proportional to $V_{DS}$. The transition point from the ohmic region to the saturation region is at $V_{DS} = V_{GS} - V_{th}$ and putting this into equation \ref{eqn:mosfet_id_large_signal} yields for the drain current
\begin{align}
    I_D &= \frac{1}{2} \kappa V_{DS}^2 \left(1+ \lambda V_{DS}\right) \nonumber\\
    \Rightarrow V_{DS} &\approx \sqrt{\frac{2 I_D}{\kappa}}\\
    &\approx \qty{784}{\mV}
\end{align}

The latter result was calculated using the example parameters from table \ref{tab:current_source_parameters}. At this point it can already be postulated that the MOSFET will severely change in its function as a current source for $V_{DS} < \qty{0.78}{\V}$. To quantify this, one has to look at the output impedance of the transconductance amplifier once again. In the last section, the output impedance was only treated for the saturation region, but this time, $R_{out}$ must be considered over a wide range of $V_{DS}$, thus not only in the saturation region but also in the ohmic region. Instead of using the small-signal model as before, which assumed only small changes of $V_{DS}$, a large-signal model must be applied, which also includes the non-linear nature of the piece-wise defined equation \ref{eqn:mosfet_id_large_signal} of the drain current.

For the sake of simplicity, a SPICE simulation of figure \ref{fig:precision_current_source} was carried out in LTSpice \cite{ltspice}. Solving this analytically bears no educational value over the numerical solution shown below as will be seen. Additionally, the SPICE simulation also offers the opportunity to add additional, parasitic elements to the model to evaluate their effect, for example, the capacitive nature of the MOSFET gate.

The simulation itself is numerically challenging and the typical approaches will lead to the limits of the numerical precision. To make the simulation feasible, the large-signal model is broken down into several small segments. For each of these segments, the small-signal model at its respective working point is evaluated and then the result joined back together to reconstruct the large-signal model sought. How this is done in detail, is shown in appendix \ref{sec:ltspice_current_source} as it is beyond the scope of this section. The final result was calculated for two different frequencies, one frequency was deliberately chosen so low (\qty{1}{\micro\Hz}) that it is well below the dominant pole of the op-amp, meaning that the full open-loop gain applies and the other frequency chosen was \qty{1}{\MHz}, were the gain had dropped to \qty[per-mode=power]{10}{\V \per V}. This is shown in figure \ref{fig:ltspice_output_impedance_simulation}.
\begin{figure}[ht]
    \centering
    %% Creator: Matplotlib, PGF backend
%%
%% To include the figure in your LaTeX document, write
%%   \input{<filename>.pgf}
%%
%% Make sure the required packages are loaded in your preamble
%%   \usepackage{pgf}
%%
%% Also ensure that all the required font packages are loaded; for instance,
%% the lmodern package is sometimes necessary when using math font.
%%   \usepackage{lmodern}
%%
%% Figures using additional raster images can only be included by \input if
%% they are in the same directory as the main LaTeX file. For loading figures
%% from other directories you can use the `import` package
%%   \usepackage{import}
%%
%% and then include the figures with
%%   \import{<path to file>}{<filename>.pgf}
%%
%% Matplotlib used the following preamble
%%   \usepackage{siunitx}
%%   \usepackage{fontspec}
%%
\begingroup%
\makeatletter%
\begin{pgfpicture}%
\pgfpathrectangle{\pgfpointorigin}{\pgfqpoint{5.431103in}{3.356606in}}%
\pgfusepath{use as bounding box, clip}%
\begin{pgfscope}%
\pgfsetbuttcap%
\pgfsetmiterjoin%
\definecolor{currentfill}{rgb}{1.000000,1.000000,1.000000}%
\pgfsetfillcolor{currentfill}%
\pgfsetlinewidth{0.000000pt}%
\definecolor{currentstroke}{rgb}{1.000000,1.000000,1.000000}%
\pgfsetstrokecolor{currentstroke}%
\pgfsetdash{}{0pt}%
\pgfpathmoveto{\pgfqpoint{0.000000in}{0.000000in}}%
\pgfpathlineto{\pgfqpoint{5.431103in}{0.000000in}}%
\pgfpathlineto{\pgfqpoint{5.431103in}{3.356606in}}%
\pgfpathlineto{\pgfqpoint{0.000000in}{3.356606in}}%
\pgfpathlineto{\pgfqpoint{0.000000in}{0.000000in}}%
\pgfpathclose%
\pgfusepath{fill}%
\end{pgfscope}%
\begin{pgfscope}%
\pgfsetbuttcap%
\pgfsetmiterjoin%
\definecolor{currentfill}{rgb}{1.000000,1.000000,1.000000}%
\pgfsetfillcolor{currentfill}%
\pgfsetlinewidth{0.000000pt}%
\definecolor{currentstroke}{rgb}{0.000000,0.000000,0.000000}%
\pgfsetstrokecolor{currentstroke}%
\pgfsetstrokeopacity{0.000000}%
\pgfsetdash{}{0pt}%
\pgfpathmoveto{\pgfqpoint{0.644859in}{0.524170in}}%
\pgfpathlineto{\pgfqpoint{5.281103in}{0.524170in}}%
\pgfpathlineto{\pgfqpoint{5.281103in}{3.189255in}}%
\pgfpathlineto{\pgfqpoint{0.644859in}{3.189255in}}%
\pgfpathlineto{\pgfqpoint{0.644859in}{0.524170in}}%
\pgfpathclose%
\pgfusepath{fill}%
\end{pgfscope}%
\begin{pgfscope}%
\pgfpathrectangle{\pgfqpoint{0.644859in}{0.524170in}}{\pgfqpoint{4.636243in}{2.665085in}}%
\pgfusepath{clip}%
\pgfsetrectcap%
\pgfsetroundjoin%
\pgfsetlinewidth{0.803000pt}%
\definecolor{currentstroke}{rgb}{0.450000,0.450000,0.450000}%
\pgfsetstrokecolor{currentstroke}%
\pgfsetdash{}{0pt}%
\pgfpathmoveto{\pgfqpoint{0.855598in}{0.524170in}}%
\pgfpathlineto{\pgfqpoint{0.855598in}{3.189255in}}%
\pgfusepath{stroke}%
\end{pgfscope}%
\begin{pgfscope}%
\pgfsetbuttcap%
\pgfsetroundjoin%
\definecolor{currentfill}{rgb}{0.000000,0.000000,0.000000}%
\pgfsetfillcolor{currentfill}%
\pgfsetlinewidth{0.803000pt}%
\definecolor{currentstroke}{rgb}{0.000000,0.000000,0.000000}%
\pgfsetstrokecolor{currentstroke}%
\pgfsetdash{}{0pt}%
\pgfsys@defobject{currentmarker}{\pgfqpoint{0.000000in}{-0.048611in}}{\pgfqpoint{0.000000in}{0.000000in}}{%
\pgfpathmoveto{\pgfqpoint{0.000000in}{0.000000in}}%
\pgfpathlineto{\pgfqpoint{0.000000in}{-0.048611in}}%
\pgfusepath{stroke,fill}%
}%
\begin{pgfscope}%
\pgfsys@transformshift{0.855598in}{0.524170in}%
\pgfsys@useobject{currentmarker}{}%
\end{pgfscope}%
\end{pgfscope}%
\begin{pgfscope}%
\definecolor{textcolor}{rgb}{0.000000,0.000000,0.000000}%
\pgfsetstrokecolor{textcolor}%
\pgfsetfillcolor{textcolor}%
\pgftext[x=0.855598in,y=0.426948in,,top]{\color{textcolor}\rmfamily\fontsize{8.000000}{9.600000}\selectfont \(\displaystyle {0.0}\)}%
\end{pgfscope}%
\begin{pgfscope}%
\pgfpathrectangle{\pgfqpoint{0.644859in}{0.524170in}}{\pgfqpoint{4.636243in}{2.665085in}}%
\pgfusepath{clip}%
\pgfsetrectcap%
\pgfsetroundjoin%
\pgfsetlinewidth{0.803000pt}%
\definecolor{currentstroke}{rgb}{0.450000,0.450000,0.450000}%
\pgfsetstrokecolor{currentstroke}%
\pgfsetdash{}{0pt}%
\pgfpathmoveto{\pgfqpoint{1.698551in}{0.524170in}}%
\pgfpathlineto{\pgfqpoint{1.698551in}{3.189255in}}%
\pgfusepath{stroke}%
\end{pgfscope}%
\begin{pgfscope}%
\pgfsetbuttcap%
\pgfsetroundjoin%
\definecolor{currentfill}{rgb}{0.000000,0.000000,0.000000}%
\pgfsetfillcolor{currentfill}%
\pgfsetlinewidth{0.803000pt}%
\definecolor{currentstroke}{rgb}{0.000000,0.000000,0.000000}%
\pgfsetstrokecolor{currentstroke}%
\pgfsetdash{}{0pt}%
\pgfsys@defobject{currentmarker}{\pgfqpoint{0.000000in}{-0.048611in}}{\pgfqpoint{0.000000in}{0.000000in}}{%
\pgfpathmoveto{\pgfqpoint{0.000000in}{0.000000in}}%
\pgfpathlineto{\pgfqpoint{0.000000in}{-0.048611in}}%
\pgfusepath{stroke,fill}%
}%
\begin{pgfscope}%
\pgfsys@transformshift{1.698551in}{0.524170in}%
\pgfsys@useobject{currentmarker}{}%
\end{pgfscope}%
\end{pgfscope}%
\begin{pgfscope}%
\definecolor{textcolor}{rgb}{0.000000,0.000000,0.000000}%
\pgfsetstrokecolor{textcolor}%
\pgfsetfillcolor{textcolor}%
\pgftext[x=1.698551in,y=0.426948in,,top]{\color{textcolor}\rmfamily\fontsize{8.000000}{9.600000}\selectfont \(\displaystyle {0.2}\)}%
\end{pgfscope}%
\begin{pgfscope}%
\pgfpathrectangle{\pgfqpoint{0.644859in}{0.524170in}}{\pgfqpoint{4.636243in}{2.665085in}}%
\pgfusepath{clip}%
\pgfsetrectcap%
\pgfsetroundjoin%
\pgfsetlinewidth{0.803000pt}%
\definecolor{currentstroke}{rgb}{0.450000,0.450000,0.450000}%
\pgfsetstrokecolor{currentstroke}%
\pgfsetdash{}{0pt}%
\pgfpathmoveto{\pgfqpoint{2.541504in}{0.524170in}}%
\pgfpathlineto{\pgfqpoint{2.541504in}{3.189255in}}%
\pgfusepath{stroke}%
\end{pgfscope}%
\begin{pgfscope}%
\pgfsetbuttcap%
\pgfsetroundjoin%
\definecolor{currentfill}{rgb}{0.000000,0.000000,0.000000}%
\pgfsetfillcolor{currentfill}%
\pgfsetlinewidth{0.803000pt}%
\definecolor{currentstroke}{rgb}{0.000000,0.000000,0.000000}%
\pgfsetstrokecolor{currentstroke}%
\pgfsetdash{}{0pt}%
\pgfsys@defobject{currentmarker}{\pgfqpoint{0.000000in}{-0.048611in}}{\pgfqpoint{0.000000in}{0.000000in}}{%
\pgfpathmoveto{\pgfqpoint{0.000000in}{0.000000in}}%
\pgfpathlineto{\pgfqpoint{0.000000in}{-0.048611in}}%
\pgfusepath{stroke,fill}%
}%
\begin{pgfscope}%
\pgfsys@transformshift{2.541504in}{0.524170in}%
\pgfsys@useobject{currentmarker}{}%
\end{pgfscope}%
\end{pgfscope}%
\begin{pgfscope}%
\definecolor{textcolor}{rgb}{0.000000,0.000000,0.000000}%
\pgfsetstrokecolor{textcolor}%
\pgfsetfillcolor{textcolor}%
\pgftext[x=2.541504in,y=0.426948in,,top]{\color{textcolor}\rmfamily\fontsize{8.000000}{9.600000}\selectfont \(\displaystyle {0.4}\)}%
\end{pgfscope}%
\begin{pgfscope}%
\pgfpathrectangle{\pgfqpoint{0.644859in}{0.524170in}}{\pgfqpoint{4.636243in}{2.665085in}}%
\pgfusepath{clip}%
\pgfsetrectcap%
\pgfsetroundjoin%
\pgfsetlinewidth{0.803000pt}%
\definecolor{currentstroke}{rgb}{0.450000,0.450000,0.450000}%
\pgfsetstrokecolor{currentstroke}%
\pgfsetdash{}{0pt}%
\pgfpathmoveto{\pgfqpoint{3.384458in}{0.524170in}}%
\pgfpathlineto{\pgfqpoint{3.384458in}{3.189255in}}%
\pgfusepath{stroke}%
\end{pgfscope}%
\begin{pgfscope}%
\pgfsetbuttcap%
\pgfsetroundjoin%
\definecolor{currentfill}{rgb}{0.000000,0.000000,0.000000}%
\pgfsetfillcolor{currentfill}%
\pgfsetlinewidth{0.803000pt}%
\definecolor{currentstroke}{rgb}{0.000000,0.000000,0.000000}%
\pgfsetstrokecolor{currentstroke}%
\pgfsetdash{}{0pt}%
\pgfsys@defobject{currentmarker}{\pgfqpoint{0.000000in}{-0.048611in}}{\pgfqpoint{0.000000in}{0.000000in}}{%
\pgfpathmoveto{\pgfqpoint{0.000000in}{0.000000in}}%
\pgfpathlineto{\pgfqpoint{0.000000in}{-0.048611in}}%
\pgfusepath{stroke,fill}%
}%
\begin{pgfscope}%
\pgfsys@transformshift{3.384458in}{0.524170in}%
\pgfsys@useobject{currentmarker}{}%
\end{pgfscope}%
\end{pgfscope}%
\begin{pgfscope}%
\definecolor{textcolor}{rgb}{0.000000,0.000000,0.000000}%
\pgfsetstrokecolor{textcolor}%
\pgfsetfillcolor{textcolor}%
\pgftext[x=3.384458in,y=0.426948in,,top]{\color{textcolor}\rmfamily\fontsize{8.000000}{9.600000}\selectfont \(\displaystyle {0.6}\)}%
\end{pgfscope}%
\begin{pgfscope}%
\pgfpathrectangle{\pgfqpoint{0.644859in}{0.524170in}}{\pgfqpoint{4.636243in}{2.665085in}}%
\pgfusepath{clip}%
\pgfsetrectcap%
\pgfsetroundjoin%
\pgfsetlinewidth{0.803000pt}%
\definecolor{currentstroke}{rgb}{0.450000,0.450000,0.450000}%
\pgfsetstrokecolor{currentstroke}%
\pgfsetdash{}{0pt}%
\pgfpathmoveto{\pgfqpoint{4.227411in}{0.524170in}}%
\pgfpathlineto{\pgfqpoint{4.227411in}{3.189255in}}%
\pgfusepath{stroke}%
\end{pgfscope}%
\begin{pgfscope}%
\pgfsetbuttcap%
\pgfsetroundjoin%
\definecolor{currentfill}{rgb}{0.000000,0.000000,0.000000}%
\pgfsetfillcolor{currentfill}%
\pgfsetlinewidth{0.803000pt}%
\definecolor{currentstroke}{rgb}{0.000000,0.000000,0.000000}%
\pgfsetstrokecolor{currentstroke}%
\pgfsetdash{}{0pt}%
\pgfsys@defobject{currentmarker}{\pgfqpoint{0.000000in}{-0.048611in}}{\pgfqpoint{0.000000in}{0.000000in}}{%
\pgfpathmoveto{\pgfqpoint{0.000000in}{0.000000in}}%
\pgfpathlineto{\pgfqpoint{0.000000in}{-0.048611in}}%
\pgfusepath{stroke,fill}%
}%
\begin{pgfscope}%
\pgfsys@transformshift{4.227411in}{0.524170in}%
\pgfsys@useobject{currentmarker}{}%
\end{pgfscope}%
\end{pgfscope}%
\begin{pgfscope}%
\definecolor{textcolor}{rgb}{0.000000,0.000000,0.000000}%
\pgfsetstrokecolor{textcolor}%
\pgfsetfillcolor{textcolor}%
\pgftext[x=4.227411in,y=0.426948in,,top]{\color{textcolor}\rmfamily\fontsize{8.000000}{9.600000}\selectfont \(\displaystyle {0.8}\)}%
\end{pgfscope}%
\begin{pgfscope}%
\pgfpathrectangle{\pgfqpoint{0.644859in}{0.524170in}}{\pgfqpoint{4.636243in}{2.665085in}}%
\pgfusepath{clip}%
\pgfsetrectcap%
\pgfsetroundjoin%
\pgfsetlinewidth{0.803000pt}%
\definecolor{currentstroke}{rgb}{0.450000,0.450000,0.450000}%
\pgfsetstrokecolor{currentstroke}%
\pgfsetdash{}{0pt}%
\pgfpathmoveto{\pgfqpoint{5.070364in}{0.524170in}}%
\pgfpathlineto{\pgfqpoint{5.070364in}{3.189255in}}%
\pgfusepath{stroke}%
\end{pgfscope}%
\begin{pgfscope}%
\pgfsetbuttcap%
\pgfsetroundjoin%
\definecolor{currentfill}{rgb}{0.000000,0.000000,0.000000}%
\pgfsetfillcolor{currentfill}%
\pgfsetlinewidth{0.803000pt}%
\definecolor{currentstroke}{rgb}{0.000000,0.000000,0.000000}%
\pgfsetstrokecolor{currentstroke}%
\pgfsetdash{}{0pt}%
\pgfsys@defobject{currentmarker}{\pgfqpoint{0.000000in}{-0.048611in}}{\pgfqpoint{0.000000in}{0.000000in}}{%
\pgfpathmoveto{\pgfqpoint{0.000000in}{0.000000in}}%
\pgfpathlineto{\pgfqpoint{0.000000in}{-0.048611in}}%
\pgfusepath{stroke,fill}%
}%
\begin{pgfscope}%
\pgfsys@transformshift{5.070364in}{0.524170in}%
\pgfsys@useobject{currentmarker}{}%
\end{pgfscope}%
\end{pgfscope}%
\begin{pgfscope}%
\definecolor{textcolor}{rgb}{0.000000,0.000000,0.000000}%
\pgfsetstrokecolor{textcolor}%
\pgfsetfillcolor{textcolor}%
\pgftext[x=5.070364in,y=0.426948in,,top]{\color{textcolor}\rmfamily\fontsize{8.000000}{9.600000}\selectfont \(\displaystyle {1.0}\)}%
\end{pgfscope}%
\begin{pgfscope}%
\definecolor{textcolor}{rgb}{0.000000,0.000000,0.000000}%
\pgfsetstrokecolor{textcolor}%
\pgfsetfillcolor{textcolor}%
\pgftext[x=2.962981in,y=0.272725in,,top]{\color{textcolor}\rmfamily\fontsize{10.000000}{12.000000}\selectfont Drain-source voltage \(\displaystyle V_{DS}\) in \unit{\V}}%
\end{pgfscope}%
\begin{pgfscope}%
\pgfpathrectangle{\pgfqpoint{0.644859in}{0.524170in}}{\pgfqpoint{4.636243in}{2.665085in}}%
\pgfusepath{clip}%
\pgfsetrectcap%
\pgfsetroundjoin%
\pgfsetlinewidth{0.803000pt}%
\definecolor{currentstroke}{rgb}{0.450000,0.450000,0.450000}%
\pgfsetstrokecolor{currentstroke}%
\pgfsetdash{}{0pt}%
\pgfpathmoveto{\pgfqpoint{0.644859in}{0.770636in}}%
\pgfpathlineto{\pgfqpoint{5.281103in}{0.770636in}}%
\pgfusepath{stroke}%
\end{pgfscope}%
\begin{pgfscope}%
\pgfsetbuttcap%
\pgfsetroundjoin%
\definecolor{currentfill}{rgb}{0.000000,0.000000,0.000000}%
\pgfsetfillcolor{currentfill}%
\pgfsetlinewidth{0.803000pt}%
\definecolor{currentstroke}{rgb}{0.000000,0.000000,0.000000}%
\pgfsetstrokecolor{currentstroke}%
\pgfsetdash{}{0pt}%
\pgfsys@defobject{currentmarker}{\pgfqpoint{-0.048611in}{0.000000in}}{\pgfqpoint{-0.000000in}{0.000000in}}{%
\pgfpathmoveto{\pgfqpoint{-0.000000in}{0.000000in}}%
\pgfpathlineto{\pgfqpoint{-0.048611in}{0.000000in}}%
\pgfusepath{stroke,fill}%
}%
\begin{pgfscope}%
\pgfsys@transformshift{0.644859in}{0.770636in}%
\pgfsys@useobject{currentmarker}{}%
\end{pgfscope}%
\end{pgfscope}%
\begin{pgfscope}%
\definecolor{textcolor}{rgb}{0.000000,0.000000,0.000000}%
\pgfsetstrokecolor{textcolor}%
\pgfsetfillcolor{textcolor}%
\pgftext[x=0.371710in, y=0.731483in, left, base]{\color{textcolor}\rmfamily\fontsize{8.000000}{9.600000}\selectfont \(\displaystyle {10^{2}}\)}%
\end{pgfscope}%
\begin{pgfscope}%
\pgfpathrectangle{\pgfqpoint{0.644859in}{0.524170in}}{\pgfqpoint{4.636243in}{2.665085in}}%
\pgfusepath{clip}%
\pgfsetrectcap%
\pgfsetroundjoin%
\pgfsetlinewidth{0.803000pt}%
\definecolor{currentstroke}{rgb}{0.450000,0.450000,0.450000}%
\pgfsetstrokecolor{currentstroke}%
\pgfsetdash{}{0pt}%
\pgfpathmoveto{\pgfqpoint{0.644859in}{1.250011in}}%
\pgfpathlineto{\pgfqpoint{5.281103in}{1.250011in}}%
\pgfusepath{stroke}%
\end{pgfscope}%
\begin{pgfscope}%
\pgfsetbuttcap%
\pgfsetroundjoin%
\definecolor{currentfill}{rgb}{0.000000,0.000000,0.000000}%
\pgfsetfillcolor{currentfill}%
\pgfsetlinewidth{0.803000pt}%
\definecolor{currentstroke}{rgb}{0.000000,0.000000,0.000000}%
\pgfsetstrokecolor{currentstroke}%
\pgfsetdash{}{0pt}%
\pgfsys@defobject{currentmarker}{\pgfqpoint{-0.048611in}{0.000000in}}{\pgfqpoint{-0.000000in}{0.000000in}}{%
\pgfpathmoveto{\pgfqpoint{-0.000000in}{0.000000in}}%
\pgfpathlineto{\pgfqpoint{-0.048611in}{0.000000in}}%
\pgfusepath{stroke,fill}%
}%
\begin{pgfscope}%
\pgfsys@transformshift{0.644859in}{1.250011in}%
\pgfsys@useobject{currentmarker}{}%
\end{pgfscope}%
\end{pgfscope}%
\begin{pgfscope}%
\definecolor{textcolor}{rgb}{0.000000,0.000000,0.000000}%
\pgfsetstrokecolor{textcolor}%
\pgfsetfillcolor{textcolor}%
\pgftext[x=0.371710in, y=1.210858in, left, base]{\color{textcolor}\rmfamily\fontsize{8.000000}{9.600000}\selectfont \(\displaystyle {10^{4}}\)}%
\end{pgfscope}%
\begin{pgfscope}%
\pgfpathrectangle{\pgfqpoint{0.644859in}{0.524170in}}{\pgfqpoint{4.636243in}{2.665085in}}%
\pgfusepath{clip}%
\pgfsetrectcap%
\pgfsetroundjoin%
\pgfsetlinewidth{0.803000pt}%
\definecolor{currentstroke}{rgb}{0.450000,0.450000,0.450000}%
\pgfsetstrokecolor{currentstroke}%
\pgfsetdash{}{0pt}%
\pgfpathmoveto{\pgfqpoint{0.644859in}{1.729387in}}%
\pgfpathlineto{\pgfqpoint{5.281103in}{1.729387in}}%
\pgfusepath{stroke}%
\end{pgfscope}%
\begin{pgfscope}%
\pgfsetbuttcap%
\pgfsetroundjoin%
\definecolor{currentfill}{rgb}{0.000000,0.000000,0.000000}%
\pgfsetfillcolor{currentfill}%
\pgfsetlinewidth{0.803000pt}%
\definecolor{currentstroke}{rgb}{0.000000,0.000000,0.000000}%
\pgfsetstrokecolor{currentstroke}%
\pgfsetdash{}{0pt}%
\pgfsys@defobject{currentmarker}{\pgfqpoint{-0.048611in}{0.000000in}}{\pgfqpoint{-0.000000in}{0.000000in}}{%
\pgfpathmoveto{\pgfqpoint{-0.000000in}{0.000000in}}%
\pgfpathlineto{\pgfqpoint{-0.048611in}{0.000000in}}%
\pgfusepath{stroke,fill}%
}%
\begin{pgfscope}%
\pgfsys@transformshift{0.644859in}{1.729387in}%
\pgfsys@useobject{currentmarker}{}%
\end{pgfscope}%
\end{pgfscope}%
\begin{pgfscope}%
\definecolor{textcolor}{rgb}{0.000000,0.000000,0.000000}%
\pgfsetstrokecolor{textcolor}%
\pgfsetfillcolor{textcolor}%
\pgftext[x=0.371710in, y=1.690234in, left, base]{\color{textcolor}\rmfamily\fontsize{8.000000}{9.600000}\selectfont \(\displaystyle {10^{6}}\)}%
\end{pgfscope}%
\begin{pgfscope}%
\pgfpathrectangle{\pgfqpoint{0.644859in}{0.524170in}}{\pgfqpoint{4.636243in}{2.665085in}}%
\pgfusepath{clip}%
\pgfsetrectcap%
\pgfsetroundjoin%
\pgfsetlinewidth{0.803000pt}%
\definecolor{currentstroke}{rgb}{0.450000,0.450000,0.450000}%
\pgfsetstrokecolor{currentstroke}%
\pgfsetdash{}{0pt}%
\pgfpathmoveto{\pgfqpoint{0.644859in}{2.208762in}}%
\pgfpathlineto{\pgfqpoint{5.281103in}{2.208762in}}%
\pgfusepath{stroke}%
\end{pgfscope}%
\begin{pgfscope}%
\pgfsetbuttcap%
\pgfsetroundjoin%
\definecolor{currentfill}{rgb}{0.000000,0.000000,0.000000}%
\pgfsetfillcolor{currentfill}%
\pgfsetlinewidth{0.803000pt}%
\definecolor{currentstroke}{rgb}{0.000000,0.000000,0.000000}%
\pgfsetstrokecolor{currentstroke}%
\pgfsetdash{}{0pt}%
\pgfsys@defobject{currentmarker}{\pgfqpoint{-0.048611in}{0.000000in}}{\pgfqpoint{-0.000000in}{0.000000in}}{%
\pgfpathmoveto{\pgfqpoint{-0.000000in}{0.000000in}}%
\pgfpathlineto{\pgfqpoint{-0.048611in}{0.000000in}}%
\pgfusepath{stroke,fill}%
}%
\begin{pgfscope}%
\pgfsys@transformshift{0.644859in}{2.208762in}%
\pgfsys@useobject{currentmarker}{}%
\end{pgfscope}%
\end{pgfscope}%
\begin{pgfscope}%
\definecolor{textcolor}{rgb}{0.000000,0.000000,0.000000}%
\pgfsetstrokecolor{textcolor}%
\pgfsetfillcolor{textcolor}%
\pgftext[x=0.371710in, y=2.169609in, left, base]{\color{textcolor}\rmfamily\fontsize{8.000000}{9.600000}\selectfont \(\displaystyle {10^{8}}\)}%
\end{pgfscope}%
\begin{pgfscope}%
\pgfpathrectangle{\pgfqpoint{0.644859in}{0.524170in}}{\pgfqpoint{4.636243in}{2.665085in}}%
\pgfusepath{clip}%
\pgfsetrectcap%
\pgfsetroundjoin%
\pgfsetlinewidth{0.803000pt}%
\definecolor{currentstroke}{rgb}{0.450000,0.450000,0.450000}%
\pgfsetstrokecolor{currentstroke}%
\pgfsetdash{}{0pt}%
\pgfpathmoveto{\pgfqpoint{0.644859in}{2.688138in}}%
\pgfpathlineto{\pgfqpoint{5.281103in}{2.688138in}}%
\pgfusepath{stroke}%
\end{pgfscope}%
\begin{pgfscope}%
\pgfsetbuttcap%
\pgfsetroundjoin%
\definecolor{currentfill}{rgb}{0.000000,0.000000,0.000000}%
\pgfsetfillcolor{currentfill}%
\pgfsetlinewidth{0.803000pt}%
\definecolor{currentstroke}{rgb}{0.000000,0.000000,0.000000}%
\pgfsetstrokecolor{currentstroke}%
\pgfsetdash{}{0pt}%
\pgfsys@defobject{currentmarker}{\pgfqpoint{-0.048611in}{0.000000in}}{\pgfqpoint{-0.000000in}{0.000000in}}{%
\pgfpathmoveto{\pgfqpoint{-0.000000in}{0.000000in}}%
\pgfpathlineto{\pgfqpoint{-0.048611in}{0.000000in}}%
\pgfusepath{stroke,fill}%
}%
\begin{pgfscope}%
\pgfsys@transformshift{0.644859in}{2.688138in}%
\pgfsys@useobject{currentmarker}{}%
\end{pgfscope}%
\end{pgfscope}%
\begin{pgfscope}%
\definecolor{textcolor}{rgb}{0.000000,0.000000,0.000000}%
\pgfsetstrokecolor{textcolor}%
\pgfsetfillcolor{textcolor}%
\pgftext[x=0.320785in, y=2.648985in, left, base]{\color{textcolor}\rmfamily\fontsize{8.000000}{9.600000}\selectfont \(\displaystyle {10^{10}}\)}%
\end{pgfscope}%
\begin{pgfscope}%
\pgfpathrectangle{\pgfqpoint{0.644859in}{0.524170in}}{\pgfqpoint{4.636243in}{2.665085in}}%
\pgfusepath{clip}%
\pgfsetrectcap%
\pgfsetroundjoin%
\pgfsetlinewidth{0.803000pt}%
\definecolor{currentstroke}{rgb}{0.450000,0.450000,0.450000}%
\pgfsetstrokecolor{currentstroke}%
\pgfsetdash{}{0pt}%
\pgfpathmoveto{\pgfqpoint{0.644859in}{3.167513in}}%
\pgfpathlineto{\pgfqpoint{5.281103in}{3.167513in}}%
\pgfusepath{stroke}%
\end{pgfscope}%
\begin{pgfscope}%
\pgfsetbuttcap%
\pgfsetroundjoin%
\definecolor{currentfill}{rgb}{0.000000,0.000000,0.000000}%
\pgfsetfillcolor{currentfill}%
\pgfsetlinewidth{0.803000pt}%
\definecolor{currentstroke}{rgb}{0.000000,0.000000,0.000000}%
\pgfsetstrokecolor{currentstroke}%
\pgfsetdash{}{0pt}%
\pgfsys@defobject{currentmarker}{\pgfqpoint{-0.048611in}{0.000000in}}{\pgfqpoint{-0.000000in}{0.000000in}}{%
\pgfpathmoveto{\pgfqpoint{-0.000000in}{0.000000in}}%
\pgfpathlineto{\pgfqpoint{-0.048611in}{0.000000in}}%
\pgfusepath{stroke,fill}%
}%
\begin{pgfscope}%
\pgfsys@transformshift{0.644859in}{3.167513in}%
\pgfsys@useobject{currentmarker}{}%
\end{pgfscope}%
\end{pgfscope}%
\begin{pgfscope}%
\definecolor{textcolor}{rgb}{0.000000,0.000000,0.000000}%
\pgfsetstrokecolor{textcolor}%
\pgfsetfillcolor{textcolor}%
\pgftext[x=0.320785in, y=3.128360in, left, base]{\color{textcolor}\rmfamily\fontsize{8.000000}{9.600000}\selectfont \(\displaystyle {10^{12}}\)}%
\end{pgfscope}%
\begin{pgfscope}%
\definecolor{textcolor}{rgb}{0.000000,0.000000,0.000000}%
\pgfsetstrokecolor{textcolor}%
\pgfsetfillcolor{textcolor}%
\pgftext[x=0.265230in,y=1.856712in,,bottom,rotate=90.000000]{\color{textcolor}\rmfamily\fontsize{10.000000}{12.000000}\selectfont Ouput Impedance \(\displaystyle R_{out}\) in \unit{\ohm}}%
\end{pgfscope}%
\begin{pgfscope}%
\pgfpathrectangle{\pgfqpoint{0.644859in}{0.524170in}}{\pgfqpoint{4.636243in}{2.665085in}}%
\pgfusepath{clip}%
\pgfsetrectcap%
\pgfsetroundjoin%
\pgfsetlinewidth{1.505625pt}%
\definecolor{currentstroke}{rgb}{0.003922,0.450980,0.698039}%
\pgfsetstrokecolor{currentstroke}%
\pgfsetstrokeopacity{0.700000}%
\pgfsetdash{}{0pt}%
\pgfpathmoveto{\pgfqpoint{5.070364in}{3.068115in}}%
\pgfpathlineto{\pgfqpoint{4.155760in}{3.067971in}}%
\pgfpathlineto{\pgfqpoint{4.151545in}{3.005604in}}%
\pgfpathlineto{\pgfqpoint{4.147330in}{2.966691in}}%
\pgfpathlineto{\pgfqpoint{4.138901in}{2.915985in}}%
\pgfpathlineto{\pgfqpoint{4.130471in}{2.881817in}}%
\pgfpathlineto{\pgfqpoint{4.122042in}{2.855955in}}%
\pgfpathlineto{\pgfqpoint{4.109397in}{2.826030in}}%
\pgfpathlineto{\pgfqpoint{4.096753in}{2.802565in}}%
\pgfpathlineto{\pgfqpoint{4.084109in}{2.783222in}}%
\pgfpathlineto{\pgfqpoint{4.067250in}{2.761746in}}%
\pgfpathlineto{\pgfqpoint{4.050391in}{2.743685in}}%
\pgfpathlineto{\pgfqpoint{4.029317in}{2.724474in}}%
\pgfpathlineto{\pgfqpoint{4.004028in}{2.704916in}}%
\pgfpathlineto{\pgfqpoint{3.978740in}{2.688090in}}%
\pgfpathlineto{\pgfqpoint{3.949236in}{2.670964in}}%
\pgfpathlineto{\pgfqpoint{3.915518in}{2.653850in}}%
\pgfpathlineto{\pgfqpoint{3.873370in}{2.635155in}}%
\pgfpathlineto{\pgfqpoint{3.827008in}{2.617130in}}%
\pgfpathlineto{\pgfqpoint{3.768001in}{2.596996in}}%
\pgfpathlineto{\pgfqpoint{3.704780in}{2.577965in}}%
\pgfpathlineto{\pgfqpoint{3.633129in}{2.558694in}}%
\pgfpathlineto{\pgfqpoint{3.553048in}{2.539280in}}%
\pgfpathlineto{\pgfqpoint{3.451894in}{2.517060in}}%
\pgfpathlineto{\pgfqpoint{3.321236in}{2.491042in}}%
\pgfpathlineto{\pgfqpoint{3.173719in}{2.464042in}}%
\pgfpathlineto{\pgfqpoint{2.967196in}{2.428736in}}%
\pgfpathlineto{\pgfqpoint{2.634229in}{2.374003in}}%
\pgfpathlineto{\pgfqpoint{2.356055in}{2.327024in}}%
\pgfpathlineto{\pgfqpoint{2.191679in}{2.297423in}}%
\pgfpathlineto{\pgfqpoint{2.027303in}{2.265496in}}%
\pgfpathlineto{\pgfqpoint{1.909289in}{2.240581in}}%
\pgfpathlineto{\pgfqpoint{1.787061in}{2.212357in}}%
\pgfpathlineto{\pgfqpoint{1.694336in}{2.188844in}}%
\pgfpathlineto{\pgfqpoint{1.610041in}{2.165439in}}%
\pgfpathlineto{\pgfqpoint{1.529960in}{2.140930in}}%
\pgfpathlineto{\pgfqpoint{1.462524in}{2.118136in}}%
\pgfpathlineto{\pgfqpoint{1.399302in}{2.094527in}}%
\pgfpathlineto{\pgfqpoint{1.340296in}{2.070019in}}%
\pgfpathlineto{\pgfqpoint{1.289719in}{2.046625in}}%
\pgfpathlineto{\pgfqpoint{1.243356in}{2.022740in}}%
\pgfpathlineto{\pgfqpoint{1.201208in}{1.998484in}}%
\pgfpathlineto{\pgfqpoint{1.163276in}{1.974048in}}%
\pgfpathlineto{\pgfqpoint{1.129557in}{1.949696in}}%
\pgfpathlineto{\pgfqpoint{1.100054in}{1.925823in}}%
\pgfpathlineto{\pgfqpoint{1.070551in}{1.898918in}}%
\pgfpathlineto{\pgfqpoint{1.045262in}{1.872756in}}%
\pgfpathlineto{\pgfqpoint{1.024188in}{1.848164in}}%
\pgfpathlineto{\pgfqpoint{1.003114in}{1.820300in}}%
\pgfpathlineto{\pgfqpoint{0.982041in}{1.788158in}}%
\pgfpathlineto{\pgfqpoint{0.965182in}{1.758329in}}%
\pgfpathlineto{\pgfqpoint{0.948322in}{1.723526in}}%
\pgfpathlineto{\pgfqpoint{0.935678in}{1.692978in}}%
\pgfpathlineto{\pgfqpoint{0.923034in}{1.657193in}}%
\pgfpathlineto{\pgfqpoint{0.910390in}{1.613965in}}%
\pgfpathlineto{\pgfqpoint{0.901960in}{1.579186in}}%
\pgfpathlineto{\pgfqpoint{0.893530in}{1.537421in}}%
\pgfpathlineto{\pgfqpoint{0.885101in}{1.485130in}}%
\pgfpathlineto{\pgfqpoint{0.876671in}{1.415170in}}%
\pgfpathlineto{\pgfqpoint{0.872457in}{1.368821in}}%
\pgfpathlineto{\pgfqpoint{0.868242in}{1.309160in}}%
\pgfpathlineto{\pgfqpoint{0.864027in}{1.225405in}}%
\pgfpathlineto{\pgfqpoint{0.859812in}{1.084579in}}%
\pgfpathlineto{\pgfqpoint{0.855598in}{0.717466in}}%
\pgfpathlineto{\pgfqpoint{0.855598in}{0.717466in}}%
\pgfusepath{stroke}%
\end{pgfscope}%
\begin{pgfscope}%
\pgfpathrectangle{\pgfqpoint{0.644859in}{0.524170in}}{\pgfqpoint{4.636243in}{2.665085in}}%
\pgfusepath{clip}%
\pgfsetrectcap%
\pgfsetroundjoin%
\pgfsetlinewidth{1.505625pt}%
\definecolor{currentstroke}{rgb}{0.870588,0.560784,0.019608}%
\pgfsetstrokecolor{currentstroke}%
\pgfsetstrokeopacity{0.700000}%
\pgfsetdash{}{0pt}%
\pgfpathmoveto{\pgfqpoint{5.070364in}{1.558405in}}%
\pgfpathlineto{\pgfqpoint{4.155760in}{1.558273in}}%
\pgfpathlineto{\pgfqpoint{4.151545in}{1.495907in}}%
\pgfpathlineto{\pgfqpoint{4.147330in}{1.456984in}}%
\pgfpathlineto{\pgfqpoint{4.138901in}{1.406289in}}%
\pgfpathlineto{\pgfqpoint{4.130471in}{1.372122in}}%
\pgfpathlineto{\pgfqpoint{4.122042in}{1.346267in}}%
\pgfpathlineto{\pgfqpoint{4.109397in}{1.316340in}}%
\pgfpathlineto{\pgfqpoint{4.096753in}{1.292880in}}%
\pgfpathlineto{\pgfqpoint{4.084109in}{1.273545in}}%
\pgfpathlineto{\pgfqpoint{4.067250in}{1.252069in}}%
\pgfpathlineto{\pgfqpoint{4.050391in}{1.234019in}}%
\pgfpathlineto{\pgfqpoint{4.029317in}{1.214819in}}%
\pgfpathlineto{\pgfqpoint{4.004028in}{1.195271in}}%
\pgfpathlineto{\pgfqpoint{3.978740in}{1.178453in}}%
\pgfpathlineto{\pgfqpoint{3.949236in}{1.161343in}}%
\pgfpathlineto{\pgfqpoint{3.915518in}{1.144240in}}%
\pgfpathlineto{\pgfqpoint{3.873370in}{1.125565in}}%
\pgfpathlineto{\pgfqpoint{3.827008in}{1.107569in}}%
\pgfpathlineto{\pgfqpoint{3.772216in}{1.088820in}}%
\pgfpathlineto{\pgfqpoint{3.708995in}{1.069671in}}%
\pgfpathlineto{\pgfqpoint{3.637344in}{1.050321in}}%
\pgfpathlineto{\pgfqpoint{3.553048in}{1.029888in}}%
\pgfpathlineto{\pgfqpoint{3.451894in}{1.007762in}}%
\pgfpathlineto{\pgfqpoint{3.333880in}{0.984272in}}%
\pgfpathlineto{\pgfqpoint{3.186364in}{0.957285in}}%
\pgfpathlineto{\pgfqpoint{2.996699in}{0.924994in}}%
\pgfpathlineto{\pgfqpoint{2.697451in}{0.876580in}}%
\pgfpathlineto{\pgfqpoint{2.288618in}{0.810148in}}%
\pgfpathlineto{\pgfqpoint{2.031517in}{0.766054in}}%
\pgfpathlineto{\pgfqpoint{1.601611in}{0.691613in}}%
\pgfpathlineto{\pgfqpoint{1.500457in}{0.676961in}}%
\pgfpathlineto{\pgfqpoint{1.411947in}{0.666368in}}%
\pgfpathlineto{\pgfqpoint{1.327651in}{0.658578in}}%
\pgfpathlineto{\pgfqpoint{1.243356in}{0.653042in}}%
\pgfpathlineto{\pgfqpoint{1.150631in}{0.649187in}}%
\pgfpathlineto{\pgfqpoint{1.032618in}{0.646652in}}%
\pgfpathlineto{\pgfqpoint{0.855598in}{0.645310in}}%
\pgfpathlineto{\pgfqpoint{0.855598in}{0.645310in}}%
\pgfusepath{stroke}%
\end{pgfscope}%
\begin{pgfscope}%
\pgfsetrectcap%
\pgfsetmiterjoin%
\pgfsetlinewidth{0.803000pt}%
\definecolor{currentstroke}{rgb}{0.000000,0.000000,0.000000}%
\pgfsetstrokecolor{currentstroke}%
\pgfsetdash{}{0pt}%
\pgfpathmoveto{\pgfqpoint{0.644859in}{0.524170in}}%
\pgfpathlineto{\pgfqpoint{0.644859in}{3.189255in}}%
\pgfusepath{stroke}%
\end{pgfscope}%
\begin{pgfscope}%
\pgfsetrectcap%
\pgfsetmiterjoin%
\pgfsetlinewidth{0.803000pt}%
\definecolor{currentstroke}{rgb}{0.000000,0.000000,0.000000}%
\pgfsetstrokecolor{currentstroke}%
\pgfsetdash{}{0pt}%
\pgfpathmoveto{\pgfqpoint{5.281103in}{0.524170in}}%
\pgfpathlineto{\pgfqpoint{5.281103in}{3.189255in}}%
\pgfusepath{stroke}%
\end{pgfscope}%
\begin{pgfscope}%
\pgfsetrectcap%
\pgfsetmiterjoin%
\pgfsetlinewidth{0.803000pt}%
\definecolor{currentstroke}{rgb}{0.000000,0.000000,0.000000}%
\pgfsetstrokecolor{currentstroke}%
\pgfsetdash{}{0pt}%
\pgfpathmoveto{\pgfqpoint{0.644859in}{0.524170in}}%
\pgfpathlineto{\pgfqpoint{5.281103in}{0.524170in}}%
\pgfusepath{stroke}%
\end{pgfscope}%
\begin{pgfscope}%
\pgfsetrectcap%
\pgfsetmiterjoin%
\pgfsetlinewidth{0.803000pt}%
\definecolor{currentstroke}{rgb}{0.000000,0.000000,0.000000}%
\pgfsetstrokecolor{currentstroke}%
\pgfsetdash{}{0pt}%
\pgfpathmoveto{\pgfqpoint{0.644859in}{3.189255in}}%
\pgfpathlineto{\pgfqpoint{5.281103in}{3.189255in}}%
\pgfusepath{stroke}%
\end{pgfscope}%
\begin{pgfscope}%
\pgfsetbuttcap%
\pgfsetmiterjoin%
\definecolor{currentfill}{rgb}{1.000000,1.000000,1.000000}%
\pgfsetfillcolor{currentfill}%
\pgfsetfillopacity{0.800000}%
\pgfsetlinewidth{1.003750pt}%
\definecolor{currentstroke}{rgb}{0.800000,0.800000,0.800000}%
\pgfsetstrokecolor{currentstroke}%
\pgfsetstrokeopacity{0.800000}%
\pgfsetdash{}{0pt}%
\pgfpathmoveto{\pgfqpoint{0.722637in}{2.790588in}}%
\pgfpathlineto{\pgfqpoint{1.404740in}{2.790588in}}%
\pgfpathquadraticcurveto{\pgfqpoint{1.426962in}{2.790588in}}{\pgfqpoint{1.426962in}{2.812811in}}%
\pgfpathlineto{\pgfqpoint{1.426962in}{3.111477in}}%
\pgfpathquadraticcurveto{\pgfqpoint{1.426962in}{3.133699in}}{\pgfqpoint{1.404740in}{3.133699in}}%
\pgfpathlineto{\pgfqpoint{0.722637in}{3.133699in}}%
\pgfpathquadraticcurveto{\pgfqpoint{0.700415in}{3.133699in}}{\pgfqpoint{0.700415in}{3.111477in}}%
\pgfpathlineto{\pgfqpoint{0.700415in}{2.812811in}}%
\pgfpathquadraticcurveto{\pgfqpoint{0.700415in}{2.790588in}}{\pgfqpoint{0.722637in}{2.790588in}}%
\pgfpathlineto{\pgfqpoint{0.722637in}{2.790588in}}%
\pgfpathclose%
\pgfusepath{stroke,fill}%
\end{pgfscope}%
\begin{pgfscope}%
\pgfsetrectcap%
\pgfsetroundjoin%
\pgfsetlinewidth{1.505625pt}%
\definecolor{currentstroke}{rgb}{0.003922,0.450980,0.698039}%
\pgfsetstrokecolor{currentstroke}%
\pgfsetstrokeopacity{0.700000}%
\pgfsetdash{}{0pt}%
\pgfpathmoveto{\pgfqpoint{0.744859in}{3.050366in}}%
\pgfpathlineto{\pgfqpoint{0.855970in}{3.050366in}}%
\pgfpathlineto{\pgfqpoint{0.967081in}{3.050366in}}%
\pgfusepath{stroke}%
\end{pgfscope}%
\begin{pgfscope}%
\definecolor{textcolor}{rgb}{0.000000,0.000000,0.000000}%
\pgfsetstrokecolor{textcolor}%
\pgfsetfillcolor{textcolor}%
\pgftext[x=1.055970in,y=3.011477in,left,base]{\color{textcolor}\rmfamily\fontsize{8.000000}{9.600000}\selectfont DC}%
\end{pgfscope}%
\begin{pgfscope}%
\pgfsetrectcap%
\pgfsetroundjoin%
\pgfsetlinewidth{1.505625pt}%
\definecolor{currentstroke}{rgb}{0.870588,0.560784,0.019608}%
\pgfsetstrokecolor{currentstroke}%
\pgfsetstrokeopacity{0.700000}%
\pgfsetdash{}{0pt}%
\pgfpathmoveto{\pgfqpoint{0.744859in}{2.895477in}}%
\pgfpathlineto{\pgfqpoint{0.855970in}{2.895477in}}%
\pgfpathlineto{\pgfqpoint{0.967081in}{2.895477in}}%
\pgfusepath{stroke}%
\end{pgfscope}%
\begin{pgfscope}%
\definecolor{textcolor}{rgb}{0.000000,0.000000,0.000000}%
\pgfsetstrokecolor{textcolor}%
\pgfsetfillcolor{textcolor}%
\pgftext[x=1.055970in,y=2.856588in,left,base]{\color{textcolor}\rmfamily\fontsize{8.000000}{9.600000}\selectfont \qty{1}{\MHz}}%
\end{pgfscope}%
\end{pgfpicture}%
\makeatother%
\endgroup%
% data/plot_generic.py
    \caption{Simulated output impedance for the precision current source from figure \ref{fig:precision_current_source} at DC and \qty{1}{\MHz} over the drain-source voltage.}
    \label{fig:ltspice_output_impedance_simulation}
\end{figure}

Looking at figure \ref{fig:ltspice_output_impedance_simulation} clearly shows the effect of entering the ohmic region of the MOSFET. Over a range of \qty{100}{\mV} starting at the \qty{0.78}{\V} calculated above, the output impedance drops by two orders of magnitude and then keeps dropping at an exponential rate with decreasing $V_{DS}$. The same effect applies to the output impedance at \qty{1}{\MHz}, although the starting value of the output impedance is around \qty{200}{\kilo\ohm} due to the reduced gain from the op-amp at \qty{1}{\MHz}. It can also be seen that $R_{out}$ levels off at \qty{30}{\ohm}, the value of the sense resistor.

This overall effect of leaving the saturation region is so drastic that the compliance voltage must be defined in such a way that the MOSFET remains in saturation and this leads to
\begin{equation}
    V_{comp} = V_{sup} - V_{ref} - \sqrt{\frac{2 I_D}{\kappa}} \,.
\end{equation}

Now turning to the supply voltage, it is limited by the op-amp which must drive the gate of the MOSFET all the way up to the supply to turn off the current source. The reference voltage is, unless one divides it down dictated by the reference chosen. This, unfortunately, leaves only little room for the MOSFET and it must be carefully chosen not limit the compliance voltage too much.

At this point a fallacy the author has observed multiple times must be addressed. In order to address the limited compliance voltage, one may be tempted to use multiple MOSFETs in parallel to divide the current between the MOSFETs and thereby reduce the voltage that needs to be dropped across the FET proportional to $\frac{1}{\sqrt{N}}$, where $N$ is the number of MOSFETs paralleled.

Imagine the following modified circuit of the precision current source shown in figure \ref{fig:precision_current_source_two_mosfets} with two MOSFETs in parallel. For clarity the gate resistors required are not included.
\begin{figure}[ht]
    \centering
    \import{figures/}{precision_current_source_2fets.tex}
    \caption{Transconductance amplifier with two p-channel MOSFETs in parallel.}
    \label{fig:precision_current_source_two_mosfets}
\end{figure}

While at first this seems like a solution to the limited $V_{DS}$, it is not recommended for a number of reasons given here.

The first reason is, MOSFET specifications are very loose, notably the threshold voltage $V_{th}$, the transconductance $g_m$ and the capacitances, but the latter is of little concern here. These tolerances limit the usefulness of paralleling MOSFETs to certain conditions, for example, when using the MOSFETs as a switch ad not as a current source. The difference between its use as a switch and a current source is the thermal load, which is a lot higher when using the MOSFET as current source. In this respect it seems to be a common misunderstanding that MOSFETs are immune to thermal runaway. This is mostly true when using them as a switch, fully turned on and in the ohmic region. In this case, there are two effects occurring, the first is that the (absolute) value of $V_{th}$ decreases with temperature, thus increasing $I_D$ and the second effect is, $R_{DS,on}$ is rising with temperature \cite{mosfet_thermal_runaway}. The latter effect is, depending on the physical design, stronger, but it depends on the specific MOSFET. A detailed analysis of paralleling MOSFETs as switches can be found in \cite{paralleling_mosfets}. In case the MOSFET is operating in pinch-off and not the ohmic region, $R_{DS,on}$ has no influence on the current, therefore, the only effect at work is the decreasing $V_{th}$. Depending on the difference in $V_{th}$ between the paralleled MOSFETs, one MOSFET will take most of the current and power. Adding source resistors can compensate for this by pushing down the source voltage as the current goes up. This will then reduce $V_{GS}$. The size of the resistor depends on the transconductance $g_m$ and the temperature coefficient of $V_{GS}$, which is around \qtyrange[range-units = single]{1.5}{2}{\mV \per \K} \cite{mosfet_vgs_tempco}. Unfortunately, \qty{1}{\ohm} or \qty{2}{\ohm}, will already use up, most of the benefits gained in compliance voltage as will be shown below.

The second reason why paralleling MOSFETs is not desirable can be seen when
remembering equation \ref{eqn:mosfet_id_large_signal}. It is known that the transition from the unwanted ohmic region to the saturation region is
\begin{equation}
    V_{DS} \geq V_{GS} − V_{th}\,.
\end{equation}

Looking at \ref{fig:precision_current_source_two_mosfets}, it can be seen that $V_{GS}$ is set by the op-amp and is the same for both MOSFETs because their gates and sources are connected. However, $V_{th}$ is device specific and according to the datasheet of the example \device{IRF9610} \cite{datasheet_IRF9610} $V_{th}$ values can show a spread of as much as \qtyrange[range-units = single, range-phrase={~to~}]{-2}{-4}{\V}, although \cite{appnote_mosfet_parameter_spread} suggests that MOSFETs from the same reel show a spread of only \qty{\pm 125}{\mV} of $V_{th}$  within the same batch for consecutive devices. The \qty{125}{\mV} was found for the \device{BUK7S1R5-40H} \cite{datasheet_BUK7S1R5}, which was sampled in this report. The number given in the report is for $3\sigma$ and, assuming the datasheet values for the spread are also referring to $3\sigma$, the spread found in the report is about twice as high as the datasheet value of \qtyrange[range-units = single]{2.4}{3.6}{\V}. Assuming similar numbers for \device{IRF9610} MOSFET used in our examples, this leads to \qty{\pm 208}{\mV} for the \device{IRF9610}, again applying $3\sigma$. Using this number, a Monte Carlo simulation (not quite, because the dice were biased to yield a Gaussian distribution instead of equal probabilities) was run using LTSpice, simulating the circuit shown in figure \ref{fig:precision_current_source_two_mosfets} and also the original circuit using only one MOSFET. For this simulation the current source was set to \qty{250}{\mA} as per table \ref{tab:current_source_parameters}. The load voltage was set to
\begin{equation*}
    V_{DS, parallel} = \sqrt{\frac{2 \frac{I_d}{2}}{\kappa}} \approx \qty{555}{\mV}\,,
    V_{DS, single} = \sqrt{\frac{2 I_d}{\kappa}} \approx \qty{784}{\mV} \,.
\end{equation*}

$\frac{I_D}{2}$ was used to calculate $V_{DS, parallel}$ for the parallel configuration to show the effect assuming perfect current sharing between the MOSFETs. Additionally, a configuration with an increased safety margin of $1 \sigma = \qty{70}{\mV}$ added to $V_{DS, parallel}$ was investigated. \num{4000} samples were drawn and the spread of the output impedance was calculated for each circuit. The results are shown as a histogram in figure \ref{fig:ltpsice_mosfet_mc_output_impedance}. The counts give the number of cases for each bin of the output impedance.
\begin{figure}[ht]
    \centering
    %% Creator: Matplotlib, PGF backend
%%
%% To include the figure in your LaTeX document, write
%%   \input{<filename>.pgf}
%%
%% Make sure the required packages are loaded in your preamble
%%   \usepackage{pgf}
%%
%% Also ensure that all the required font packages are loaded; for instance,
%% the lmodern package is sometimes necessary when using math font.
%%   \usepackage{lmodern}
%%
%% Figures using additional raster images can only be included by \input if
%% they are in the same directory as the main LaTeX file. For loading figures
%% from other directories you can use the `import` package
%%   \usepackage{import}
%%
%% and then include the figures with
%%   \import{<path to file>}{<filename>.pgf}
%%
%% Matplotlib used the following preamble
%%   \usepackage{siunitx}
%%   \usepackage{fontspec}
%%   \makeatletter\@ifpackageloaded{underscore}{}{\usepackage[strings]{underscore}}\makeatother
%%
\begingroup%
\makeatletter%
\begin{pgfpicture}%
\pgfpathrectangle{\pgfpointorigin}{\pgfqpoint{5.431103in}{3.356606in}}%
\pgfusepath{use as bounding box, clip}%
\begin{pgfscope}%
\pgfsetbuttcap%
\pgfsetmiterjoin%
\definecolor{currentfill}{rgb}{1.000000,1.000000,1.000000}%
\pgfsetfillcolor{currentfill}%
\pgfsetlinewidth{0.000000pt}%
\definecolor{currentstroke}{rgb}{1.000000,1.000000,1.000000}%
\pgfsetstrokecolor{currentstroke}%
\pgfsetdash{}{0pt}%
\pgfpathmoveto{\pgfqpoint{0.000000in}{0.000000in}}%
\pgfpathlineto{\pgfqpoint{5.431103in}{0.000000in}}%
\pgfpathlineto{\pgfqpoint{5.431103in}{3.356606in}}%
\pgfpathlineto{\pgfqpoint{0.000000in}{3.356606in}}%
\pgfpathlineto{\pgfqpoint{0.000000in}{0.000000in}}%
\pgfpathclose%
\pgfusepath{fill}%
\end{pgfscope}%
\begin{pgfscope}%
\pgfsetbuttcap%
\pgfsetmiterjoin%
\definecolor{currentfill}{rgb}{1.000000,1.000000,1.000000}%
\pgfsetfillcolor{currentfill}%
\pgfsetlinewidth{0.000000pt}%
\definecolor{currentstroke}{rgb}{0.000000,0.000000,0.000000}%
\pgfsetstrokecolor{currentstroke}%
\pgfsetstrokeopacity{0.000000}%
\pgfsetdash{}{0pt}%
\pgfpathmoveto{\pgfqpoint{0.661006in}{0.524170in}}%
\pgfpathlineto{\pgfqpoint{5.281103in}{0.524170in}}%
\pgfpathlineto{\pgfqpoint{5.281103in}{3.206606in}}%
\pgfpathlineto{\pgfqpoint{0.661006in}{3.206606in}}%
\pgfpathlineto{\pgfqpoint{0.661006in}{0.524170in}}%
\pgfpathclose%
\pgfusepath{fill}%
\end{pgfscope}%
\begin{pgfscope}%
\pgfpathrectangle{\pgfqpoint{0.661006in}{0.524170in}}{\pgfqpoint{4.620097in}{2.682436in}}%
\pgfusepath{clip}%
\pgfsetbuttcap%
\pgfsetmiterjoin%
\definecolor{currentfill}{rgb}{0.870588,0.560784,0.019608}%
\pgfsetfillcolor{currentfill}%
\pgfsetfillopacity{0.700000}%
\pgfsetlinewidth{0.000000pt}%
\definecolor{currentstroke}{rgb}{0.000000,0.000000,0.000000}%
\pgfsetstrokecolor{currentstroke}%
\pgfsetstrokeopacity{0.700000}%
\pgfsetdash{}{0pt}%
\pgfpathmoveto{\pgfqpoint{0.871010in}{0.524170in}}%
\pgfpathlineto{\pgfqpoint{1.151016in}{0.524170in}}%
\pgfpathlineto{\pgfqpoint{1.151016in}{2.406985in}}%
\pgfpathlineto{\pgfqpoint{0.871010in}{2.406985in}}%
\pgfpathlineto{\pgfqpoint{0.871010in}{0.524170in}}%
\pgfpathclose%
\pgfusepath{fill}%
\end{pgfscope}%
\begin{pgfscope}%
\pgfpathrectangle{\pgfqpoint{0.661006in}{0.524170in}}{\pgfqpoint{4.620097in}{2.682436in}}%
\pgfusepath{clip}%
\pgfsetbuttcap%
\pgfsetmiterjoin%
\definecolor{currentfill}{rgb}{0.870588,0.560784,0.019608}%
\pgfsetfillcolor{currentfill}%
\pgfsetfillopacity{0.700000}%
\pgfsetlinewidth{0.000000pt}%
\definecolor{currentstroke}{rgb}{0.000000,0.000000,0.000000}%
\pgfsetstrokecolor{currentstroke}%
\pgfsetstrokeopacity{0.700000}%
\pgfsetdash{}{0pt}%
\pgfpathmoveto{\pgfqpoint{1.151016in}{0.524170in}}%
\pgfpathlineto{\pgfqpoint{1.431022in}{0.524170in}}%
\pgfpathlineto{\pgfqpoint{1.431022in}{0.818599in}}%
\pgfpathlineto{\pgfqpoint{1.151016in}{0.818599in}}%
\pgfpathlineto{\pgfqpoint{1.151016in}{0.524170in}}%
\pgfpathclose%
\pgfusepath{fill}%
\end{pgfscope}%
\begin{pgfscope}%
\pgfpathrectangle{\pgfqpoint{0.661006in}{0.524170in}}{\pgfqpoint{4.620097in}{2.682436in}}%
\pgfusepath{clip}%
\pgfsetbuttcap%
\pgfsetmiterjoin%
\definecolor{currentfill}{rgb}{0.870588,0.560784,0.019608}%
\pgfsetfillcolor{currentfill}%
\pgfsetfillopacity{0.700000}%
\pgfsetlinewidth{0.000000pt}%
\definecolor{currentstroke}{rgb}{0.000000,0.000000,0.000000}%
\pgfsetstrokecolor{currentstroke}%
\pgfsetstrokeopacity{0.700000}%
\pgfsetdash{}{0pt}%
\pgfpathmoveto{\pgfqpoint{1.431022in}{0.524170in}}%
\pgfpathlineto{\pgfqpoint{1.711028in}{0.524170in}}%
\pgfpathlineto{\pgfqpoint{1.711028in}{0.642325in}}%
\pgfpathlineto{\pgfqpoint{1.431022in}{0.642325in}}%
\pgfpathlineto{\pgfqpoint{1.431022in}{0.524170in}}%
\pgfpathclose%
\pgfusepath{fill}%
\end{pgfscope}%
\begin{pgfscope}%
\pgfpathrectangle{\pgfqpoint{0.661006in}{0.524170in}}{\pgfqpoint{4.620097in}{2.682436in}}%
\pgfusepath{clip}%
\pgfsetbuttcap%
\pgfsetmiterjoin%
\definecolor{currentfill}{rgb}{0.870588,0.560784,0.019608}%
\pgfsetfillcolor{currentfill}%
\pgfsetfillopacity{0.700000}%
\pgfsetlinewidth{0.000000pt}%
\definecolor{currentstroke}{rgb}{0.000000,0.000000,0.000000}%
\pgfsetstrokecolor{currentstroke}%
\pgfsetstrokeopacity{0.700000}%
\pgfsetdash{}{0pt}%
\pgfpathmoveto{\pgfqpoint{1.711028in}{0.524170in}}%
\pgfpathlineto{\pgfqpoint{1.991034in}{0.524170in}}%
\pgfpathlineto{\pgfqpoint{1.991034in}{0.594424in}}%
\pgfpathlineto{\pgfqpoint{1.711028in}{0.594424in}}%
\pgfpathlineto{\pgfqpoint{1.711028in}{0.524170in}}%
\pgfpathclose%
\pgfusepath{fill}%
\end{pgfscope}%
\begin{pgfscope}%
\pgfpathrectangle{\pgfqpoint{0.661006in}{0.524170in}}{\pgfqpoint{4.620097in}{2.682436in}}%
\pgfusepath{clip}%
\pgfsetbuttcap%
\pgfsetmiterjoin%
\definecolor{currentfill}{rgb}{0.870588,0.560784,0.019608}%
\pgfsetfillcolor{currentfill}%
\pgfsetfillopacity{0.700000}%
\pgfsetlinewidth{0.000000pt}%
\definecolor{currentstroke}{rgb}{0.000000,0.000000,0.000000}%
\pgfsetstrokecolor{currentstroke}%
\pgfsetstrokeopacity{0.700000}%
\pgfsetdash{}{0pt}%
\pgfpathmoveto{\pgfqpoint{1.991034in}{0.524170in}}%
\pgfpathlineto{\pgfqpoint{2.271039in}{0.524170in}}%
\pgfpathlineto{\pgfqpoint{2.271039in}{0.566961in}}%
\pgfpathlineto{\pgfqpoint{1.991034in}{0.566961in}}%
\pgfpathlineto{\pgfqpoint{1.991034in}{0.524170in}}%
\pgfpathclose%
\pgfusepath{fill}%
\end{pgfscope}%
\begin{pgfscope}%
\pgfpathrectangle{\pgfqpoint{0.661006in}{0.524170in}}{\pgfqpoint{4.620097in}{2.682436in}}%
\pgfusepath{clip}%
\pgfsetbuttcap%
\pgfsetmiterjoin%
\definecolor{currentfill}{rgb}{0.870588,0.560784,0.019608}%
\pgfsetfillcolor{currentfill}%
\pgfsetfillopacity{0.700000}%
\pgfsetlinewidth{0.000000pt}%
\definecolor{currentstroke}{rgb}{0.000000,0.000000,0.000000}%
\pgfsetstrokecolor{currentstroke}%
\pgfsetstrokeopacity{0.700000}%
\pgfsetdash{}{0pt}%
\pgfpathmoveto{\pgfqpoint{2.271039in}{0.524170in}}%
\pgfpathlineto{\pgfqpoint{2.551045in}{0.524170in}}%
\pgfpathlineto{\pgfqpoint{2.551045in}{0.548439in}}%
\pgfpathlineto{\pgfqpoint{2.271039in}{0.548439in}}%
\pgfpathlineto{\pgfqpoint{2.271039in}{0.524170in}}%
\pgfpathclose%
\pgfusepath{fill}%
\end{pgfscope}%
\begin{pgfscope}%
\pgfpathrectangle{\pgfqpoint{0.661006in}{0.524170in}}{\pgfqpoint{4.620097in}{2.682436in}}%
\pgfusepath{clip}%
\pgfsetbuttcap%
\pgfsetmiterjoin%
\definecolor{currentfill}{rgb}{0.870588,0.560784,0.019608}%
\pgfsetfillcolor{currentfill}%
\pgfsetfillopacity{0.700000}%
\pgfsetlinewidth{0.000000pt}%
\definecolor{currentstroke}{rgb}{0.000000,0.000000,0.000000}%
\pgfsetstrokecolor{currentstroke}%
\pgfsetstrokeopacity{0.700000}%
\pgfsetdash{}{0pt}%
\pgfpathmoveto{\pgfqpoint{2.551045in}{0.524170in}}%
\pgfpathlineto{\pgfqpoint{2.831051in}{0.524170in}}%
\pgfpathlineto{\pgfqpoint{2.831051in}{0.545885in}}%
\pgfpathlineto{\pgfqpoint{2.551045in}{0.545885in}}%
\pgfpathlineto{\pgfqpoint{2.551045in}{0.524170in}}%
\pgfpathclose%
\pgfusepath{fill}%
\end{pgfscope}%
\begin{pgfscope}%
\pgfpathrectangle{\pgfqpoint{0.661006in}{0.524170in}}{\pgfqpoint{4.620097in}{2.682436in}}%
\pgfusepath{clip}%
\pgfsetbuttcap%
\pgfsetmiterjoin%
\definecolor{currentfill}{rgb}{0.870588,0.560784,0.019608}%
\pgfsetfillcolor{currentfill}%
\pgfsetfillopacity{0.700000}%
\pgfsetlinewidth{0.000000pt}%
\definecolor{currentstroke}{rgb}{0.000000,0.000000,0.000000}%
\pgfsetstrokecolor{currentstroke}%
\pgfsetstrokeopacity{0.700000}%
\pgfsetdash{}{0pt}%
\pgfpathmoveto{\pgfqpoint{2.831051in}{0.524170in}}%
\pgfpathlineto{\pgfqpoint{3.111057in}{0.524170in}}%
\pgfpathlineto{\pgfqpoint{3.111057in}{0.536943in}}%
\pgfpathlineto{\pgfqpoint{2.831051in}{0.536943in}}%
\pgfpathlineto{\pgfqpoint{2.831051in}{0.524170in}}%
\pgfpathclose%
\pgfusepath{fill}%
\end{pgfscope}%
\begin{pgfscope}%
\pgfpathrectangle{\pgfqpoint{0.661006in}{0.524170in}}{\pgfqpoint{4.620097in}{2.682436in}}%
\pgfusepath{clip}%
\pgfsetbuttcap%
\pgfsetmiterjoin%
\definecolor{currentfill}{rgb}{0.870588,0.560784,0.019608}%
\pgfsetfillcolor{currentfill}%
\pgfsetfillopacity{0.700000}%
\pgfsetlinewidth{0.000000pt}%
\definecolor{currentstroke}{rgb}{0.000000,0.000000,0.000000}%
\pgfsetstrokecolor{currentstroke}%
\pgfsetstrokeopacity{0.700000}%
\pgfsetdash{}{0pt}%
\pgfpathmoveto{\pgfqpoint{3.111057in}{0.524170in}}%
\pgfpathlineto{\pgfqpoint{3.391063in}{0.524170in}}%
\pgfpathlineto{\pgfqpoint{3.391063in}{0.535666in}}%
\pgfpathlineto{\pgfqpoint{3.111057in}{0.535666in}}%
\pgfpathlineto{\pgfqpoint{3.111057in}{0.524170in}}%
\pgfpathclose%
\pgfusepath{fill}%
\end{pgfscope}%
\begin{pgfscope}%
\pgfpathrectangle{\pgfqpoint{0.661006in}{0.524170in}}{\pgfqpoint{4.620097in}{2.682436in}}%
\pgfusepath{clip}%
\pgfsetbuttcap%
\pgfsetmiterjoin%
\definecolor{currentfill}{rgb}{0.870588,0.560784,0.019608}%
\pgfsetfillcolor{currentfill}%
\pgfsetfillopacity{0.700000}%
\pgfsetlinewidth{0.000000pt}%
\definecolor{currentstroke}{rgb}{0.000000,0.000000,0.000000}%
\pgfsetstrokecolor{currentstroke}%
\pgfsetstrokeopacity{0.700000}%
\pgfsetdash{}{0pt}%
\pgfpathmoveto{\pgfqpoint{3.391063in}{0.524170in}}%
\pgfpathlineto{\pgfqpoint{3.671069in}{0.524170in}}%
\pgfpathlineto{\pgfqpoint{3.671069in}{0.534389in}}%
\pgfpathlineto{\pgfqpoint{3.391063in}{0.534389in}}%
\pgfpathlineto{\pgfqpoint{3.391063in}{0.524170in}}%
\pgfpathclose%
\pgfusepath{fill}%
\end{pgfscope}%
\begin{pgfscope}%
\pgfpathrectangle{\pgfqpoint{0.661006in}{0.524170in}}{\pgfqpoint{4.620097in}{2.682436in}}%
\pgfusepath{clip}%
\pgfsetbuttcap%
\pgfsetmiterjoin%
\definecolor{currentfill}{rgb}{0.870588,0.560784,0.019608}%
\pgfsetfillcolor{currentfill}%
\pgfsetfillopacity{0.700000}%
\pgfsetlinewidth{0.000000pt}%
\definecolor{currentstroke}{rgb}{0.000000,0.000000,0.000000}%
\pgfsetstrokecolor{currentstroke}%
\pgfsetstrokeopacity{0.700000}%
\pgfsetdash{}{0pt}%
\pgfpathmoveto{\pgfqpoint{3.671069in}{0.524170in}}%
\pgfpathlineto{\pgfqpoint{3.951075in}{0.524170in}}%
\pgfpathlineto{\pgfqpoint{3.951075in}{0.527363in}}%
\pgfpathlineto{\pgfqpoint{3.671069in}{0.527363in}}%
\pgfpathlineto{\pgfqpoint{3.671069in}{0.524170in}}%
\pgfpathclose%
\pgfusepath{fill}%
\end{pgfscope}%
\begin{pgfscope}%
\pgfpathrectangle{\pgfqpoint{0.661006in}{0.524170in}}{\pgfqpoint{4.620097in}{2.682436in}}%
\pgfusepath{clip}%
\pgfsetbuttcap%
\pgfsetmiterjoin%
\definecolor{currentfill}{rgb}{0.870588,0.560784,0.019608}%
\pgfsetfillcolor{currentfill}%
\pgfsetfillopacity{0.700000}%
\pgfsetlinewidth{0.000000pt}%
\definecolor{currentstroke}{rgb}{0.000000,0.000000,0.000000}%
\pgfsetstrokecolor{currentstroke}%
\pgfsetstrokeopacity{0.700000}%
\pgfsetdash{}{0pt}%
\pgfpathmoveto{\pgfqpoint{3.951075in}{0.524170in}}%
\pgfpathlineto{\pgfqpoint{4.231081in}{0.524170in}}%
\pgfpathlineto{\pgfqpoint{4.231081in}{0.529918in}}%
\pgfpathlineto{\pgfqpoint{3.951075in}{0.529918in}}%
\pgfpathlineto{\pgfqpoint{3.951075in}{0.524170in}}%
\pgfpathclose%
\pgfusepath{fill}%
\end{pgfscope}%
\begin{pgfscope}%
\pgfpathrectangle{\pgfqpoint{0.661006in}{0.524170in}}{\pgfqpoint{4.620097in}{2.682436in}}%
\pgfusepath{clip}%
\pgfsetbuttcap%
\pgfsetmiterjoin%
\definecolor{currentfill}{rgb}{0.870588,0.560784,0.019608}%
\pgfsetfillcolor{currentfill}%
\pgfsetfillopacity{0.700000}%
\pgfsetlinewidth{0.000000pt}%
\definecolor{currentstroke}{rgb}{0.000000,0.000000,0.000000}%
\pgfsetstrokecolor{currentstroke}%
\pgfsetstrokeopacity{0.700000}%
\pgfsetdash{}{0pt}%
\pgfpathmoveto{\pgfqpoint{4.231081in}{0.524170in}}%
\pgfpathlineto{\pgfqpoint{4.511086in}{0.524170in}}%
\pgfpathlineto{\pgfqpoint{4.511086in}{0.529279in}}%
\pgfpathlineto{\pgfqpoint{4.231081in}{0.529279in}}%
\pgfpathlineto{\pgfqpoint{4.231081in}{0.524170in}}%
\pgfpathclose%
\pgfusepath{fill}%
\end{pgfscope}%
\begin{pgfscope}%
\pgfpathrectangle{\pgfqpoint{0.661006in}{0.524170in}}{\pgfqpoint{4.620097in}{2.682436in}}%
\pgfusepath{clip}%
\pgfsetbuttcap%
\pgfsetmiterjoin%
\definecolor{currentfill}{rgb}{0.870588,0.560784,0.019608}%
\pgfsetfillcolor{currentfill}%
\pgfsetfillopacity{0.700000}%
\pgfsetlinewidth{0.000000pt}%
\definecolor{currentstroke}{rgb}{0.000000,0.000000,0.000000}%
\pgfsetstrokecolor{currentstroke}%
\pgfsetstrokeopacity{0.700000}%
\pgfsetdash{}{0pt}%
\pgfpathmoveto{\pgfqpoint{4.511086in}{0.524170in}}%
\pgfpathlineto{\pgfqpoint{4.791092in}{0.524170in}}%
\pgfpathlineto{\pgfqpoint{4.791092in}{0.526086in}}%
\pgfpathlineto{\pgfqpoint{4.511086in}{0.526086in}}%
\pgfpathlineto{\pgfqpoint{4.511086in}{0.524170in}}%
\pgfpathclose%
\pgfusepath{fill}%
\end{pgfscope}%
\begin{pgfscope}%
\pgfpathrectangle{\pgfqpoint{0.661006in}{0.524170in}}{\pgfqpoint{4.620097in}{2.682436in}}%
\pgfusepath{clip}%
\pgfsetbuttcap%
\pgfsetmiterjoin%
\definecolor{currentfill}{rgb}{0.870588,0.560784,0.019608}%
\pgfsetfillcolor{currentfill}%
\pgfsetfillopacity{0.700000}%
\pgfsetlinewidth{0.000000pt}%
\definecolor{currentstroke}{rgb}{0.000000,0.000000,0.000000}%
\pgfsetstrokecolor{currentstroke}%
\pgfsetstrokeopacity{0.700000}%
\pgfsetdash{}{0pt}%
\pgfpathmoveto{\pgfqpoint{4.791092in}{0.524170in}}%
\pgfpathlineto{\pgfqpoint{5.071098in}{0.524170in}}%
\pgfpathlineto{\pgfqpoint{5.071098in}{0.573986in}}%
\pgfpathlineto{\pgfqpoint{4.791092in}{0.573986in}}%
\pgfpathlineto{\pgfqpoint{4.791092in}{0.524170in}}%
\pgfpathclose%
\pgfusepath{fill}%
\end{pgfscope}%
\begin{pgfscope}%
\pgfpathrectangle{\pgfqpoint{0.661006in}{0.524170in}}{\pgfqpoint{4.620097in}{2.682436in}}%
\pgfusepath{clip}%
\pgfsetbuttcap%
\pgfsetmiterjoin%
\definecolor{currentfill}{rgb}{0.007843,0.619608,0.450980}%
\pgfsetfillcolor{currentfill}%
\pgfsetfillopacity{0.700000}%
\pgfsetlinewidth{0.000000pt}%
\definecolor{currentstroke}{rgb}{0.000000,0.000000,0.000000}%
\pgfsetstrokecolor{currentstroke}%
\pgfsetstrokeopacity{0.700000}%
\pgfsetdash{}{0pt}%
\pgfpathmoveto{\pgfqpoint{0.871010in}{0.524170in}}%
\pgfpathlineto{\pgfqpoint{1.151016in}{0.524170in}}%
\pgfpathlineto{\pgfqpoint{1.151016in}{0.524170in}}%
\pgfpathlineto{\pgfqpoint{0.871010in}{0.524170in}}%
\pgfpathlineto{\pgfqpoint{0.871010in}{0.524170in}}%
\pgfpathclose%
\pgfusepath{fill}%
\end{pgfscope}%
\begin{pgfscope}%
\pgfpathrectangle{\pgfqpoint{0.661006in}{0.524170in}}{\pgfqpoint{4.620097in}{2.682436in}}%
\pgfusepath{clip}%
\pgfsetbuttcap%
\pgfsetmiterjoin%
\definecolor{currentfill}{rgb}{0.007843,0.619608,0.450980}%
\pgfsetfillcolor{currentfill}%
\pgfsetfillopacity{0.700000}%
\pgfsetlinewidth{0.000000pt}%
\definecolor{currentstroke}{rgb}{0.000000,0.000000,0.000000}%
\pgfsetstrokecolor{currentstroke}%
\pgfsetstrokeopacity{0.700000}%
\pgfsetdash{}{0pt}%
\pgfpathmoveto{\pgfqpoint{1.151016in}{0.524170in}}%
\pgfpathlineto{\pgfqpoint{1.431022in}{0.524170in}}%
\pgfpathlineto{\pgfqpoint{1.431022in}{0.524170in}}%
\pgfpathlineto{\pgfqpoint{1.151016in}{0.524170in}}%
\pgfpathlineto{\pgfqpoint{1.151016in}{0.524170in}}%
\pgfpathclose%
\pgfusepath{fill}%
\end{pgfscope}%
\begin{pgfscope}%
\pgfpathrectangle{\pgfqpoint{0.661006in}{0.524170in}}{\pgfqpoint{4.620097in}{2.682436in}}%
\pgfusepath{clip}%
\pgfsetbuttcap%
\pgfsetmiterjoin%
\definecolor{currentfill}{rgb}{0.007843,0.619608,0.450980}%
\pgfsetfillcolor{currentfill}%
\pgfsetfillopacity{0.700000}%
\pgfsetlinewidth{0.000000pt}%
\definecolor{currentstroke}{rgb}{0.000000,0.000000,0.000000}%
\pgfsetstrokecolor{currentstroke}%
\pgfsetstrokeopacity{0.700000}%
\pgfsetdash{}{0pt}%
\pgfpathmoveto{\pgfqpoint{1.431022in}{0.524170in}}%
\pgfpathlineto{\pgfqpoint{1.711028in}{0.524170in}}%
\pgfpathlineto{\pgfqpoint{1.711028in}{0.524170in}}%
\pgfpathlineto{\pgfqpoint{1.431022in}{0.524170in}}%
\pgfpathlineto{\pgfqpoint{1.431022in}{0.524170in}}%
\pgfpathclose%
\pgfusepath{fill}%
\end{pgfscope}%
\begin{pgfscope}%
\pgfpathrectangle{\pgfqpoint{0.661006in}{0.524170in}}{\pgfqpoint{4.620097in}{2.682436in}}%
\pgfusepath{clip}%
\pgfsetbuttcap%
\pgfsetmiterjoin%
\definecolor{currentfill}{rgb}{0.007843,0.619608,0.450980}%
\pgfsetfillcolor{currentfill}%
\pgfsetfillopacity{0.700000}%
\pgfsetlinewidth{0.000000pt}%
\definecolor{currentstroke}{rgb}{0.000000,0.000000,0.000000}%
\pgfsetstrokecolor{currentstroke}%
\pgfsetstrokeopacity{0.700000}%
\pgfsetdash{}{0pt}%
\pgfpathmoveto{\pgfqpoint{1.711028in}{0.524170in}}%
\pgfpathlineto{\pgfqpoint{1.991034in}{0.524170in}}%
\pgfpathlineto{\pgfqpoint{1.991034in}{0.524170in}}%
\pgfpathlineto{\pgfqpoint{1.711028in}{0.524170in}}%
\pgfpathlineto{\pgfqpoint{1.711028in}{0.524170in}}%
\pgfpathclose%
\pgfusepath{fill}%
\end{pgfscope}%
\begin{pgfscope}%
\pgfpathrectangle{\pgfqpoint{0.661006in}{0.524170in}}{\pgfqpoint{4.620097in}{2.682436in}}%
\pgfusepath{clip}%
\pgfsetbuttcap%
\pgfsetmiterjoin%
\definecolor{currentfill}{rgb}{0.007843,0.619608,0.450980}%
\pgfsetfillcolor{currentfill}%
\pgfsetfillopacity{0.700000}%
\pgfsetlinewidth{0.000000pt}%
\definecolor{currentstroke}{rgb}{0.000000,0.000000,0.000000}%
\pgfsetstrokecolor{currentstroke}%
\pgfsetstrokeopacity{0.700000}%
\pgfsetdash{}{0pt}%
\pgfpathmoveto{\pgfqpoint{1.991034in}{0.524170in}}%
\pgfpathlineto{\pgfqpoint{2.271039in}{0.524170in}}%
\pgfpathlineto{\pgfqpoint{2.271039in}{0.524170in}}%
\pgfpathlineto{\pgfqpoint{1.991034in}{0.524170in}}%
\pgfpathlineto{\pgfqpoint{1.991034in}{0.524170in}}%
\pgfpathclose%
\pgfusepath{fill}%
\end{pgfscope}%
\begin{pgfscope}%
\pgfpathrectangle{\pgfqpoint{0.661006in}{0.524170in}}{\pgfqpoint{4.620097in}{2.682436in}}%
\pgfusepath{clip}%
\pgfsetbuttcap%
\pgfsetmiterjoin%
\definecolor{currentfill}{rgb}{0.007843,0.619608,0.450980}%
\pgfsetfillcolor{currentfill}%
\pgfsetfillopacity{0.700000}%
\pgfsetlinewidth{0.000000pt}%
\definecolor{currentstroke}{rgb}{0.000000,0.000000,0.000000}%
\pgfsetstrokecolor{currentstroke}%
\pgfsetstrokeopacity{0.700000}%
\pgfsetdash{}{0pt}%
\pgfpathmoveto{\pgfqpoint{2.271039in}{0.524170in}}%
\pgfpathlineto{\pgfqpoint{2.551045in}{0.524170in}}%
\pgfpathlineto{\pgfqpoint{2.551045in}{0.524170in}}%
\pgfpathlineto{\pgfqpoint{2.271039in}{0.524170in}}%
\pgfpathlineto{\pgfqpoint{2.271039in}{0.524170in}}%
\pgfpathclose%
\pgfusepath{fill}%
\end{pgfscope}%
\begin{pgfscope}%
\pgfpathrectangle{\pgfqpoint{0.661006in}{0.524170in}}{\pgfqpoint{4.620097in}{2.682436in}}%
\pgfusepath{clip}%
\pgfsetbuttcap%
\pgfsetmiterjoin%
\definecolor{currentfill}{rgb}{0.007843,0.619608,0.450980}%
\pgfsetfillcolor{currentfill}%
\pgfsetfillopacity{0.700000}%
\pgfsetlinewidth{0.000000pt}%
\definecolor{currentstroke}{rgb}{0.000000,0.000000,0.000000}%
\pgfsetstrokecolor{currentstroke}%
\pgfsetstrokeopacity{0.700000}%
\pgfsetdash{}{0pt}%
\pgfpathmoveto{\pgfqpoint{2.551045in}{0.524170in}}%
\pgfpathlineto{\pgfqpoint{2.831051in}{0.524170in}}%
\pgfpathlineto{\pgfqpoint{2.831051in}{0.524170in}}%
\pgfpathlineto{\pgfqpoint{2.551045in}{0.524170in}}%
\pgfpathlineto{\pgfqpoint{2.551045in}{0.524170in}}%
\pgfpathclose%
\pgfusepath{fill}%
\end{pgfscope}%
\begin{pgfscope}%
\pgfpathrectangle{\pgfqpoint{0.661006in}{0.524170in}}{\pgfqpoint{4.620097in}{2.682436in}}%
\pgfusepath{clip}%
\pgfsetbuttcap%
\pgfsetmiterjoin%
\definecolor{currentfill}{rgb}{0.007843,0.619608,0.450980}%
\pgfsetfillcolor{currentfill}%
\pgfsetfillopacity{0.700000}%
\pgfsetlinewidth{0.000000pt}%
\definecolor{currentstroke}{rgb}{0.000000,0.000000,0.000000}%
\pgfsetstrokecolor{currentstroke}%
\pgfsetstrokeopacity{0.700000}%
\pgfsetdash{}{0pt}%
\pgfpathmoveto{\pgfqpoint{2.831051in}{0.524170in}}%
\pgfpathlineto{\pgfqpoint{3.111057in}{0.524170in}}%
\pgfpathlineto{\pgfqpoint{3.111057in}{0.524170in}}%
\pgfpathlineto{\pgfqpoint{2.831051in}{0.524170in}}%
\pgfpathlineto{\pgfqpoint{2.831051in}{0.524170in}}%
\pgfpathclose%
\pgfusepath{fill}%
\end{pgfscope}%
\begin{pgfscope}%
\pgfpathrectangle{\pgfqpoint{0.661006in}{0.524170in}}{\pgfqpoint{4.620097in}{2.682436in}}%
\pgfusepath{clip}%
\pgfsetbuttcap%
\pgfsetmiterjoin%
\definecolor{currentfill}{rgb}{0.007843,0.619608,0.450980}%
\pgfsetfillcolor{currentfill}%
\pgfsetfillopacity{0.700000}%
\pgfsetlinewidth{0.000000pt}%
\definecolor{currentstroke}{rgb}{0.000000,0.000000,0.000000}%
\pgfsetstrokecolor{currentstroke}%
\pgfsetstrokeopacity{0.700000}%
\pgfsetdash{}{0pt}%
\pgfpathmoveto{\pgfqpoint{3.111057in}{0.524170in}}%
\pgfpathlineto{\pgfqpoint{3.391063in}{0.524170in}}%
\pgfpathlineto{\pgfqpoint{3.391063in}{0.524170in}}%
\pgfpathlineto{\pgfqpoint{3.111057in}{0.524170in}}%
\pgfpathlineto{\pgfqpoint{3.111057in}{0.524170in}}%
\pgfpathclose%
\pgfusepath{fill}%
\end{pgfscope}%
\begin{pgfscope}%
\pgfpathrectangle{\pgfqpoint{0.661006in}{0.524170in}}{\pgfqpoint{4.620097in}{2.682436in}}%
\pgfusepath{clip}%
\pgfsetbuttcap%
\pgfsetmiterjoin%
\definecolor{currentfill}{rgb}{0.007843,0.619608,0.450980}%
\pgfsetfillcolor{currentfill}%
\pgfsetfillopacity{0.700000}%
\pgfsetlinewidth{0.000000pt}%
\definecolor{currentstroke}{rgb}{0.000000,0.000000,0.000000}%
\pgfsetstrokecolor{currentstroke}%
\pgfsetstrokeopacity{0.700000}%
\pgfsetdash{}{0pt}%
\pgfpathmoveto{\pgfqpoint{3.391063in}{0.524170in}}%
\pgfpathlineto{\pgfqpoint{3.671069in}{0.524170in}}%
\pgfpathlineto{\pgfqpoint{3.671069in}{0.524170in}}%
\pgfpathlineto{\pgfqpoint{3.391063in}{0.524170in}}%
\pgfpathlineto{\pgfqpoint{3.391063in}{0.524170in}}%
\pgfpathclose%
\pgfusepath{fill}%
\end{pgfscope}%
\begin{pgfscope}%
\pgfpathrectangle{\pgfqpoint{0.661006in}{0.524170in}}{\pgfqpoint{4.620097in}{2.682436in}}%
\pgfusepath{clip}%
\pgfsetbuttcap%
\pgfsetmiterjoin%
\definecolor{currentfill}{rgb}{0.007843,0.619608,0.450980}%
\pgfsetfillcolor{currentfill}%
\pgfsetfillopacity{0.700000}%
\pgfsetlinewidth{0.000000pt}%
\definecolor{currentstroke}{rgb}{0.000000,0.000000,0.000000}%
\pgfsetstrokecolor{currentstroke}%
\pgfsetstrokeopacity{0.700000}%
\pgfsetdash{}{0pt}%
\pgfpathmoveto{\pgfqpoint{3.671069in}{0.524170in}}%
\pgfpathlineto{\pgfqpoint{3.951075in}{0.524170in}}%
\pgfpathlineto{\pgfqpoint{3.951075in}{3.078871in}}%
\pgfpathlineto{\pgfqpoint{3.671069in}{3.078871in}}%
\pgfpathlineto{\pgfqpoint{3.671069in}{0.524170in}}%
\pgfpathclose%
\pgfusepath{fill}%
\end{pgfscope}%
\begin{pgfscope}%
\pgfpathrectangle{\pgfqpoint{0.661006in}{0.524170in}}{\pgfqpoint{4.620097in}{2.682436in}}%
\pgfusepath{clip}%
\pgfsetbuttcap%
\pgfsetmiterjoin%
\definecolor{currentfill}{rgb}{0.007843,0.619608,0.450980}%
\pgfsetfillcolor{currentfill}%
\pgfsetfillopacity{0.700000}%
\pgfsetlinewidth{0.000000pt}%
\definecolor{currentstroke}{rgb}{0.000000,0.000000,0.000000}%
\pgfsetstrokecolor{currentstroke}%
\pgfsetstrokeopacity{0.700000}%
\pgfsetdash{}{0pt}%
\pgfpathmoveto{\pgfqpoint{3.951075in}{0.524170in}}%
\pgfpathlineto{\pgfqpoint{4.231081in}{0.524170in}}%
\pgfpathlineto{\pgfqpoint{4.231081in}{0.524170in}}%
\pgfpathlineto{\pgfqpoint{3.951075in}{0.524170in}}%
\pgfpathlineto{\pgfqpoint{3.951075in}{0.524170in}}%
\pgfpathclose%
\pgfusepath{fill}%
\end{pgfscope}%
\begin{pgfscope}%
\pgfpathrectangle{\pgfqpoint{0.661006in}{0.524170in}}{\pgfqpoint{4.620097in}{2.682436in}}%
\pgfusepath{clip}%
\pgfsetbuttcap%
\pgfsetmiterjoin%
\definecolor{currentfill}{rgb}{0.007843,0.619608,0.450980}%
\pgfsetfillcolor{currentfill}%
\pgfsetfillopacity{0.700000}%
\pgfsetlinewidth{0.000000pt}%
\definecolor{currentstroke}{rgb}{0.000000,0.000000,0.000000}%
\pgfsetstrokecolor{currentstroke}%
\pgfsetstrokeopacity{0.700000}%
\pgfsetdash{}{0pt}%
\pgfpathmoveto{\pgfqpoint{4.231081in}{0.524170in}}%
\pgfpathlineto{\pgfqpoint{4.511086in}{0.524170in}}%
\pgfpathlineto{\pgfqpoint{4.511086in}{0.524170in}}%
\pgfpathlineto{\pgfqpoint{4.231081in}{0.524170in}}%
\pgfpathlineto{\pgfqpoint{4.231081in}{0.524170in}}%
\pgfpathclose%
\pgfusepath{fill}%
\end{pgfscope}%
\begin{pgfscope}%
\pgfpathrectangle{\pgfqpoint{0.661006in}{0.524170in}}{\pgfqpoint{4.620097in}{2.682436in}}%
\pgfusepath{clip}%
\pgfsetbuttcap%
\pgfsetmiterjoin%
\definecolor{currentfill}{rgb}{0.007843,0.619608,0.450980}%
\pgfsetfillcolor{currentfill}%
\pgfsetfillopacity{0.700000}%
\pgfsetlinewidth{0.000000pt}%
\definecolor{currentstroke}{rgb}{0.000000,0.000000,0.000000}%
\pgfsetstrokecolor{currentstroke}%
\pgfsetstrokeopacity{0.700000}%
\pgfsetdash{}{0pt}%
\pgfpathmoveto{\pgfqpoint{4.511086in}{0.524170in}}%
\pgfpathlineto{\pgfqpoint{4.791092in}{0.524170in}}%
\pgfpathlineto{\pgfqpoint{4.791092in}{0.524170in}}%
\pgfpathlineto{\pgfqpoint{4.511086in}{0.524170in}}%
\pgfpathlineto{\pgfqpoint{4.511086in}{0.524170in}}%
\pgfpathclose%
\pgfusepath{fill}%
\end{pgfscope}%
\begin{pgfscope}%
\pgfpathrectangle{\pgfqpoint{0.661006in}{0.524170in}}{\pgfqpoint{4.620097in}{2.682436in}}%
\pgfusepath{clip}%
\pgfsetbuttcap%
\pgfsetmiterjoin%
\definecolor{currentfill}{rgb}{0.007843,0.619608,0.450980}%
\pgfsetfillcolor{currentfill}%
\pgfsetfillopacity{0.700000}%
\pgfsetlinewidth{0.000000pt}%
\definecolor{currentstroke}{rgb}{0.000000,0.000000,0.000000}%
\pgfsetstrokecolor{currentstroke}%
\pgfsetstrokeopacity{0.700000}%
\pgfsetdash{}{0pt}%
\pgfpathmoveto{\pgfqpoint{4.791092in}{0.524170in}}%
\pgfpathlineto{\pgfqpoint{5.071098in}{0.524170in}}%
\pgfpathlineto{\pgfqpoint{5.071098in}{0.524170in}}%
\pgfpathlineto{\pgfqpoint{4.791092in}{0.524170in}}%
\pgfpathlineto{\pgfqpoint{4.791092in}{0.524170in}}%
\pgfpathclose%
\pgfusepath{fill}%
\end{pgfscope}%
\begin{pgfscope}%
\pgfpathrectangle{\pgfqpoint{0.661006in}{0.524170in}}{\pgfqpoint{4.620097in}{2.682436in}}%
\pgfusepath{clip}%
\pgfsetbuttcap%
\pgfsetmiterjoin%
\definecolor{currentfill}{rgb}{0.003922,0.450980,0.698039}%
\pgfsetfillcolor{currentfill}%
\pgfsetfillopacity{0.700000}%
\pgfsetlinewidth{0.000000pt}%
\definecolor{currentstroke}{rgb}{0.000000,0.000000,0.000000}%
\pgfsetstrokecolor{currentstroke}%
\pgfsetstrokeopacity{0.700000}%
\pgfsetdash{}{0pt}%
\pgfpathmoveto{\pgfqpoint{0.871010in}{0.524170in}}%
\pgfpathlineto{\pgfqpoint{1.151016in}{0.524170in}}%
\pgfpathlineto{\pgfqpoint{1.151016in}{0.685116in}}%
\pgfpathlineto{\pgfqpoint{0.871010in}{0.685116in}}%
\pgfpathlineto{\pgfqpoint{0.871010in}{0.524170in}}%
\pgfpathclose%
\pgfusepath{fill}%
\end{pgfscope}%
\begin{pgfscope}%
\pgfpathrectangle{\pgfqpoint{0.661006in}{0.524170in}}{\pgfqpoint{4.620097in}{2.682436in}}%
\pgfusepath{clip}%
\pgfsetbuttcap%
\pgfsetmiterjoin%
\definecolor{currentfill}{rgb}{0.003922,0.450980,0.698039}%
\pgfsetfillcolor{currentfill}%
\pgfsetfillopacity{0.700000}%
\pgfsetlinewidth{0.000000pt}%
\definecolor{currentstroke}{rgb}{0.000000,0.000000,0.000000}%
\pgfsetstrokecolor{currentstroke}%
\pgfsetstrokeopacity{0.700000}%
\pgfsetdash{}{0pt}%
\pgfpathmoveto{\pgfqpoint{1.151016in}{0.524170in}}%
\pgfpathlineto{\pgfqpoint{1.431022in}{0.524170in}}%
\pgfpathlineto{\pgfqpoint{1.431022in}{0.591869in}}%
\pgfpathlineto{\pgfqpoint{1.151016in}{0.591869in}}%
\pgfpathlineto{\pgfqpoint{1.151016in}{0.524170in}}%
\pgfpathclose%
\pgfusepath{fill}%
\end{pgfscope}%
\begin{pgfscope}%
\pgfpathrectangle{\pgfqpoint{0.661006in}{0.524170in}}{\pgfqpoint{4.620097in}{2.682436in}}%
\pgfusepath{clip}%
\pgfsetbuttcap%
\pgfsetmiterjoin%
\definecolor{currentfill}{rgb}{0.003922,0.450980,0.698039}%
\pgfsetfillcolor{currentfill}%
\pgfsetfillopacity{0.700000}%
\pgfsetlinewidth{0.000000pt}%
\definecolor{currentstroke}{rgb}{0.000000,0.000000,0.000000}%
\pgfsetstrokecolor{currentstroke}%
\pgfsetstrokeopacity{0.700000}%
\pgfsetdash{}{0pt}%
\pgfpathmoveto{\pgfqpoint{1.431022in}{0.524170in}}%
\pgfpathlineto{\pgfqpoint{1.711028in}{0.524170in}}%
\pgfpathlineto{\pgfqpoint{1.711028in}{0.556742in}}%
\pgfpathlineto{\pgfqpoint{1.431022in}{0.556742in}}%
\pgfpathlineto{\pgfqpoint{1.431022in}{0.524170in}}%
\pgfpathclose%
\pgfusepath{fill}%
\end{pgfscope}%
\begin{pgfscope}%
\pgfpathrectangle{\pgfqpoint{0.661006in}{0.524170in}}{\pgfqpoint{4.620097in}{2.682436in}}%
\pgfusepath{clip}%
\pgfsetbuttcap%
\pgfsetmiterjoin%
\definecolor{currentfill}{rgb}{0.003922,0.450980,0.698039}%
\pgfsetfillcolor{currentfill}%
\pgfsetfillopacity{0.700000}%
\pgfsetlinewidth{0.000000pt}%
\definecolor{currentstroke}{rgb}{0.000000,0.000000,0.000000}%
\pgfsetstrokecolor{currentstroke}%
\pgfsetstrokeopacity{0.700000}%
\pgfsetdash{}{0pt}%
\pgfpathmoveto{\pgfqpoint{1.711028in}{0.524170in}}%
\pgfpathlineto{\pgfqpoint{1.991034in}{0.524170in}}%
\pgfpathlineto{\pgfqpoint{1.991034in}{0.542691in}}%
\pgfpathlineto{\pgfqpoint{1.711028in}{0.542691in}}%
\pgfpathlineto{\pgfqpoint{1.711028in}{0.524170in}}%
\pgfpathclose%
\pgfusepath{fill}%
\end{pgfscope}%
\begin{pgfscope}%
\pgfpathrectangle{\pgfqpoint{0.661006in}{0.524170in}}{\pgfqpoint{4.620097in}{2.682436in}}%
\pgfusepath{clip}%
\pgfsetbuttcap%
\pgfsetmiterjoin%
\definecolor{currentfill}{rgb}{0.003922,0.450980,0.698039}%
\pgfsetfillcolor{currentfill}%
\pgfsetfillopacity{0.700000}%
\pgfsetlinewidth{0.000000pt}%
\definecolor{currentstroke}{rgb}{0.000000,0.000000,0.000000}%
\pgfsetstrokecolor{currentstroke}%
\pgfsetstrokeopacity{0.700000}%
\pgfsetdash{}{0pt}%
\pgfpathmoveto{\pgfqpoint{1.991034in}{0.524170in}}%
\pgfpathlineto{\pgfqpoint{2.271039in}{0.524170in}}%
\pgfpathlineto{\pgfqpoint{2.271039in}{0.534389in}}%
\pgfpathlineto{\pgfqpoint{1.991034in}{0.534389in}}%
\pgfpathlineto{\pgfqpoint{1.991034in}{0.524170in}}%
\pgfpathclose%
\pgfusepath{fill}%
\end{pgfscope}%
\begin{pgfscope}%
\pgfpathrectangle{\pgfqpoint{0.661006in}{0.524170in}}{\pgfqpoint{4.620097in}{2.682436in}}%
\pgfusepath{clip}%
\pgfsetbuttcap%
\pgfsetmiterjoin%
\definecolor{currentfill}{rgb}{0.003922,0.450980,0.698039}%
\pgfsetfillcolor{currentfill}%
\pgfsetfillopacity{0.700000}%
\pgfsetlinewidth{0.000000pt}%
\definecolor{currentstroke}{rgb}{0.000000,0.000000,0.000000}%
\pgfsetstrokecolor{currentstroke}%
\pgfsetstrokeopacity{0.700000}%
\pgfsetdash{}{0pt}%
\pgfpathmoveto{\pgfqpoint{2.271039in}{0.524170in}}%
\pgfpathlineto{\pgfqpoint{2.551045in}{0.524170in}}%
\pgfpathlineto{\pgfqpoint{2.551045in}{0.533111in}}%
\pgfpathlineto{\pgfqpoint{2.271039in}{0.533111in}}%
\pgfpathlineto{\pgfqpoint{2.271039in}{0.524170in}}%
\pgfpathclose%
\pgfusepath{fill}%
\end{pgfscope}%
\begin{pgfscope}%
\pgfpathrectangle{\pgfqpoint{0.661006in}{0.524170in}}{\pgfqpoint{4.620097in}{2.682436in}}%
\pgfusepath{clip}%
\pgfsetbuttcap%
\pgfsetmiterjoin%
\definecolor{currentfill}{rgb}{0.003922,0.450980,0.698039}%
\pgfsetfillcolor{currentfill}%
\pgfsetfillopacity{0.700000}%
\pgfsetlinewidth{0.000000pt}%
\definecolor{currentstroke}{rgb}{0.000000,0.000000,0.000000}%
\pgfsetstrokecolor{currentstroke}%
\pgfsetstrokeopacity{0.700000}%
\pgfsetdash{}{0pt}%
\pgfpathmoveto{\pgfqpoint{2.551045in}{0.524170in}}%
\pgfpathlineto{\pgfqpoint{2.831051in}{0.524170in}}%
\pgfpathlineto{\pgfqpoint{2.831051in}{0.530557in}}%
\pgfpathlineto{\pgfqpoint{2.551045in}{0.530557in}}%
\pgfpathlineto{\pgfqpoint{2.551045in}{0.524170in}}%
\pgfpathclose%
\pgfusepath{fill}%
\end{pgfscope}%
\begin{pgfscope}%
\pgfpathrectangle{\pgfqpoint{0.661006in}{0.524170in}}{\pgfqpoint{4.620097in}{2.682436in}}%
\pgfusepath{clip}%
\pgfsetbuttcap%
\pgfsetmiterjoin%
\definecolor{currentfill}{rgb}{0.003922,0.450980,0.698039}%
\pgfsetfillcolor{currentfill}%
\pgfsetfillopacity{0.700000}%
\pgfsetlinewidth{0.000000pt}%
\definecolor{currentstroke}{rgb}{0.000000,0.000000,0.000000}%
\pgfsetstrokecolor{currentstroke}%
\pgfsetstrokeopacity{0.700000}%
\pgfsetdash{}{0pt}%
\pgfpathmoveto{\pgfqpoint{2.831051in}{0.524170in}}%
\pgfpathlineto{\pgfqpoint{3.111057in}{0.524170in}}%
\pgfpathlineto{\pgfqpoint{3.111057in}{0.530557in}}%
\pgfpathlineto{\pgfqpoint{2.831051in}{0.530557in}}%
\pgfpathlineto{\pgfqpoint{2.831051in}{0.524170in}}%
\pgfpathclose%
\pgfusepath{fill}%
\end{pgfscope}%
\begin{pgfscope}%
\pgfpathrectangle{\pgfqpoint{0.661006in}{0.524170in}}{\pgfqpoint{4.620097in}{2.682436in}}%
\pgfusepath{clip}%
\pgfsetbuttcap%
\pgfsetmiterjoin%
\definecolor{currentfill}{rgb}{0.003922,0.450980,0.698039}%
\pgfsetfillcolor{currentfill}%
\pgfsetfillopacity{0.700000}%
\pgfsetlinewidth{0.000000pt}%
\definecolor{currentstroke}{rgb}{0.000000,0.000000,0.000000}%
\pgfsetstrokecolor{currentstroke}%
\pgfsetstrokeopacity{0.700000}%
\pgfsetdash{}{0pt}%
\pgfpathmoveto{\pgfqpoint{3.111057in}{0.524170in}}%
\pgfpathlineto{\pgfqpoint{3.391063in}{0.524170in}}%
\pgfpathlineto{\pgfqpoint{3.391063in}{0.527363in}}%
\pgfpathlineto{\pgfqpoint{3.111057in}{0.527363in}}%
\pgfpathlineto{\pgfqpoint{3.111057in}{0.524170in}}%
\pgfpathclose%
\pgfusepath{fill}%
\end{pgfscope}%
\begin{pgfscope}%
\pgfpathrectangle{\pgfqpoint{0.661006in}{0.524170in}}{\pgfqpoint{4.620097in}{2.682436in}}%
\pgfusepath{clip}%
\pgfsetbuttcap%
\pgfsetmiterjoin%
\definecolor{currentfill}{rgb}{0.003922,0.450980,0.698039}%
\pgfsetfillcolor{currentfill}%
\pgfsetfillopacity{0.700000}%
\pgfsetlinewidth{0.000000pt}%
\definecolor{currentstroke}{rgb}{0.000000,0.000000,0.000000}%
\pgfsetstrokecolor{currentstroke}%
\pgfsetstrokeopacity{0.700000}%
\pgfsetdash{}{0pt}%
\pgfpathmoveto{\pgfqpoint{3.391063in}{0.524170in}}%
\pgfpathlineto{\pgfqpoint{3.671069in}{0.524170in}}%
\pgfpathlineto{\pgfqpoint{3.671069in}{0.526724in}}%
\pgfpathlineto{\pgfqpoint{3.391063in}{0.526724in}}%
\pgfpathlineto{\pgfqpoint{3.391063in}{0.524170in}}%
\pgfpathclose%
\pgfusepath{fill}%
\end{pgfscope}%
\begin{pgfscope}%
\pgfpathrectangle{\pgfqpoint{0.661006in}{0.524170in}}{\pgfqpoint{4.620097in}{2.682436in}}%
\pgfusepath{clip}%
\pgfsetbuttcap%
\pgfsetmiterjoin%
\definecolor{currentfill}{rgb}{0.003922,0.450980,0.698039}%
\pgfsetfillcolor{currentfill}%
\pgfsetfillopacity{0.700000}%
\pgfsetlinewidth{0.000000pt}%
\definecolor{currentstroke}{rgb}{0.000000,0.000000,0.000000}%
\pgfsetstrokecolor{currentstroke}%
\pgfsetstrokeopacity{0.700000}%
\pgfsetdash{}{0pt}%
\pgfpathmoveto{\pgfqpoint{3.671069in}{0.524170in}}%
\pgfpathlineto{\pgfqpoint{3.951075in}{0.524170in}}%
\pgfpathlineto{\pgfqpoint{3.951075in}{0.529279in}}%
\pgfpathlineto{\pgfqpoint{3.671069in}{0.529279in}}%
\pgfpathlineto{\pgfqpoint{3.671069in}{0.524170in}}%
\pgfpathclose%
\pgfusepath{fill}%
\end{pgfscope}%
\begin{pgfscope}%
\pgfpathrectangle{\pgfqpoint{0.661006in}{0.524170in}}{\pgfqpoint{4.620097in}{2.682436in}}%
\pgfusepath{clip}%
\pgfsetbuttcap%
\pgfsetmiterjoin%
\definecolor{currentfill}{rgb}{0.003922,0.450980,0.698039}%
\pgfsetfillcolor{currentfill}%
\pgfsetfillopacity{0.700000}%
\pgfsetlinewidth{0.000000pt}%
\definecolor{currentstroke}{rgb}{0.000000,0.000000,0.000000}%
\pgfsetstrokecolor{currentstroke}%
\pgfsetstrokeopacity{0.700000}%
\pgfsetdash{}{0pt}%
\pgfpathmoveto{\pgfqpoint{3.951075in}{0.524170in}}%
\pgfpathlineto{\pgfqpoint{4.231081in}{0.524170in}}%
\pgfpathlineto{\pgfqpoint{4.231081in}{0.525447in}}%
\pgfpathlineto{\pgfqpoint{3.951075in}{0.525447in}}%
\pgfpathlineto{\pgfqpoint{3.951075in}{0.524170in}}%
\pgfpathclose%
\pgfusepath{fill}%
\end{pgfscope}%
\begin{pgfscope}%
\pgfpathrectangle{\pgfqpoint{0.661006in}{0.524170in}}{\pgfqpoint{4.620097in}{2.682436in}}%
\pgfusepath{clip}%
\pgfsetbuttcap%
\pgfsetmiterjoin%
\definecolor{currentfill}{rgb}{0.003922,0.450980,0.698039}%
\pgfsetfillcolor{currentfill}%
\pgfsetfillopacity{0.700000}%
\pgfsetlinewidth{0.000000pt}%
\definecolor{currentstroke}{rgb}{0.000000,0.000000,0.000000}%
\pgfsetstrokecolor{currentstroke}%
\pgfsetstrokeopacity{0.700000}%
\pgfsetdash{}{0pt}%
\pgfpathmoveto{\pgfqpoint{4.231081in}{0.524170in}}%
\pgfpathlineto{\pgfqpoint{4.511086in}{0.524170in}}%
\pgfpathlineto{\pgfqpoint{4.511086in}{0.524808in}}%
\pgfpathlineto{\pgfqpoint{4.231081in}{0.524808in}}%
\pgfpathlineto{\pgfqpoint{4.231081in}{0.524170in}}%
\pgfpathclose%
\pgfusepath{fill}%
\end{pgfscope}%
\begin{pgfscope}%
\pgfpathrectangle{\pgfqpoint{0.661006in}{0.524170in}}{\pgfqpoint{4.620097in}{2.682436in}}%
\pgfusepath{clip}%
\pgfsetbuttcap%
\pgfsetmiterjoin%
\definecolor{currentfill}{rgb}{0.003922,0.450980,0.698039}%
\pgfsetfillcolor{currentfill}%
\pgfsetfillopacity{0.700000}%
\pgfsetlinewidth{0.000000pt}%
\definecolor{currentstroke}{rgb}{0.000000,0.000000,0.000000}%
\pgfsetstrokecolor{currentstroke}%
\pgfsetstrokeopacity{0.700000}%
\pgfsetdash{}{0pt}%
\pgfpathmoveto{\pgfqpoint{4.511086in}{0.524170in}}%
\pgfpathlineto{\pgfqpoint{4.791092in}{0.524170in}}%
\pgfpathlineto{\pgfqpoint{4.791092in}{0.524808in}}%
\pgfpathlineto{\pgfqpoint{4.511086in}{0.524808in}}%
\pgfpathlineto{\pgfqpoint{4.511086in}{0.524170in}}%
\pgfpathclose%
\pgfusepath{fill}%
\end{pgfscope}%
\begin{pgfscope}%
\pgfpathrectangle{\pgfqpoint{0.661006in}{0.524170in}}{\pgfqpoint{4.620097in}{2.682436in}}%
\pgfusepath{clip}%
\pgfsetbuttcap%
\pgfsetmiterjoin%
\definecolor{currentfill}{rgb}{0.003922,0.450980,0.698039}%
\pgfsetfillcolor{currentfill}%
\pgfsetfillopacity{0.700000}%
\pgfsetlinewidth{0.000000pt}%
\definecolor{currentstroke}{rgb}{0.000000,0.000000,0.000000}%
\pgfsetstrokecolor{currentstroke}%
\pgfsetstrokeopacity{0.700000}%
\pgfsetdash{}{0pt}%
\pgfpathmoveto{\pgfqpoint{4.791092in}{0.524170in}}%
\pgfpathlineto{\pgfqpoint{5.071098in}{0.524170in}}%
\pgfpathlineto{\pgfqpoint{5.071098in}{2.753785in}}%
\pgfpathlineto{\pgfqpoint{4.791092in}{2.753785in}}%
\pgfpathlineto{\pgfqpoint{4.791092in}{0.524170in}}%
\pgfpathclose%
\pgfusepath{fill}%
\end{pgfscope}%
\begin{pgfscope}%
\pgfpathrectangle{\pgfqpoint{0.661006in}{0.524170in}}{\pgfqpoint{4.620097in}{2.682436in}}%
\pgfusepath{clip}%
\pgfsetrectcap%
\pgfsetroundjoin%
\pgfsetlinewidth{0.803000pt}%
\definecolor{currentstroke}{rgb}{0.450000,0.450000,0.450000}%
\pgfsetstrokecolor{currentstroke}%
\pgfsetdash{}{0pt}%
\pgfpathmoveto{\pgfqpoint{0.845195in}{0.524170in}}%
\pgfpathlineto{\pgfqpoint{0.845195in}{3.206606in}}%
\pgfusepath{stroke}%
\end{pgfscope}%
\begin{pgfscope}%
\pgfsetbuttcap%
\pgfsetroundjoin%
\definecolor{currentfill}{rgb}{0.000000,0.000000,0.000000}%
\pgfsetfillcolor{currentfill}%
\pgfsetlinewidth{0.803000pt}%
\definecolor{currentstroke}{rgb}{0.000000,0.000000,0.000000}%
\pgfsetstrokecolor{currentstroke}%
\pgfsetdash{}{0pt}%
\pgfsys@defobject{currentmarker}{\pgfqpoint{0.000000in}{-0.048611in}}{\pgfqpoint{0.000000in}{0.000000in}}{%
\pgfpathmoveto{\pgfqpoint{0.000000in}{0.000000in}}%
\pgfpathlineto{\pgfqpoint{0.000000in}{-0.048611in}}%
\pgfusepath{stroke,fill}%
}%
\begin{pgfscope}%
\pgfsys@transformshift{0.845195in}{0.524170in}%
\pgfsys@useobject{currentmarker}{}%
\end{pgfscope}%
\end{pgfscope}%
\begin{pgfscope}%
\definecolor{textcolor}{rgb}{0.000000,0.000000,0.000000}%
\pgfsetstrokecolor{textcolor}%
\pgfsetfillcolor{textcolor}%
\pgftext[x=0.845195in,y=0.426948in,,top]{\color{textcolor}\rmfamily\fontsize{8.000000}{9.600000}\selectfont \(\displaystyle {0}\)}%
\end{pgfscope}%
\begin{pgfscope}%
\pgfpathrectangle{\pgfqpoint{0.661006in}{0.524170in}}{\pgfqpoint{4.620097in}{2.682436in}}%
\pgfusepath{clip}%
\pgfsetrectcap%
\pgfsetroundjoin%
\pgfsetlinewidth{0.803000pt}%
\definecolor{currentstroke}{rgb}{0.450000,0.450000,0.450000}%
\pgfsetstrokecolor{currentstroke}%
\pgfsetdash{}{0pt}%
\pgfpathmoveto{\pgfqpoint{1.623446in}{0.524170in}}%
\pgfpathlineto{\pgfqpoint{1.623446in}{3.206606in}}%
\pgfusepath{stroke}%
\end{pgfscope}%
\begin{pgfscope}%
\pgfsetbuttcap%
\pgfsetroundjoin%
\definecolor{currentfill}{rgb}{0.000000,0.000000,0.000000}%
\pgfsetfillcolor{currentfill}%
\pgfsetlinewidth{0.803000pt}%
\definecolor{currentstroke}{rgb}{0.000000,0.000000,0.000000}%
\pgfsetstrokecolor{currentstroke}%
\pgfsetdash{}{0pt}%
\pgfsys@defobject{currentmarker}{\pgfqpoint{0.000000in}{-0.048611in}}{\pgfqpoint{0.000000in}{0.000000in}}{%
\pgfpathmoveto{\pgfqpoint{0.000000in}{0.000000in}}%
\pgfpathlineto{\pgfqpoint{0.000000in}{-0.048611in}}%
\pgfusepath{stroke,fill}%
}%
\begin{pgfscope}%
\pgfsys@transformshift{1.623446in}{0.524170in}%
\pgfsys@useobject{currentmarker}{}%
\end{pgfscope}%
\end{pgfscope}%
\begin{pgfscope}%
\definecolor{textcolor}{rgb}{0.000000,0.000000,0.000000}%
\pgfsetstrokecolor{textcolor}%
\pgfsetfillcolor{textcolor}%
\pgftext[x=1.623446in,y=0.426948in,,top]{\color{textcolor}\rmfamily\fontsize{8.000000}{9.600000}\selectfont \(\displaystyle {1}\)}%
\end{pgfscope}%
\begin{pgfscope}%
\pgfpathrectangle{\pgfqpoint{0.661006in}{0.524170in}}{\pgfqpoint{4.620097in}{2.682436in}}%
\pgfusepath{clip}%
\pgfsetrectcap%
\pgfsetroundjoin%
\pgfsetlinewidth{0.803000pt}%
\definecolor{currentstroke}{rgb}{0.450000,0.450000,0.450000}%
\pgfsetstrokecolor{currentstroke}%
\pgfsetdash{}{0pt}%
\pgfpathmoveto{\pgfqpoint{2.401697in}{0.524170in}}%
\pgfpathlineto{\pgfqpoint{2.401697in}{3.206606in}}%
\pgfusepath{stroke}%
\end{pgfscope}%
\begin{pgfscope}%
\pgfsetbuttcap%
\pgfsetroundjoin%
\definecolor{currentfill}{rgb}{0.000000,0.000000,0.000000}%
\pgfsetfillcolor{currentfill}%
\pgfsetlinewidth{0.803000pt}%
\definecolor{currentstroke}{rgb}{0.000000,0.000000,0.000000}%
\pgfsetstrokecolor{currentstroke}%
\pgfsetdash{}{0pt}%
\pgfsys@defobject{currentmarker}{\pgfqpoint{0.000000in}{-0.048611in}}{\pgfqpoint{0.000000in}{0.000000in}}{%
\pgfpathmoveto{\pgfqpoint{0.000000in}{0.000000in}}%
\pgfpathlineto{\pgfqpoint{0.000000in}{-0.048611in}}%
\pgfusepath{stroke,fill}%
}%
\begin{pgfscope}%
\pgfsys@transformshift{2.401697in}{0.524170in}%
\pgfsys@useobject{currentmarker}{}%
\end{pgfscope}%
\end{pgfscope}%
\begin{pgfscope}%
\definecolor{textcolor}{rgb}{0.000000,0.000000,0.000000}%
\pgfsetstrokecolor{textcolor}%
\pgfsetfillcolor{textcolor}%
\pgftext[x=2.401697in,y=0.426948in,,top]{\color{textcolor}\rmfamily\fontsize{8.000000}{9.600000}\selectfont \(\displaystyle {2}\)}%
\end{pgfscope}%
\begin{pgfscope}%
\pgfpathrectangle{\pgfqpoint{0.661006in}{0.524170in}}{\pgfqpoint{4.620097in}{2.682436in}}%
\pgfusepath{clip}%
\pgfsetrectcap%
\pgfsetroundjoin%
\pgfsetlinewidth{0.803000pt}%
\definecolor{currentstroke}{rgb}{0.450000,0.450000,0.450000}%
\pgfsetstrokecolor{currentstroke}%
\pgfsetdash{}{0pt}%
\pgfpathmoveto{\pgfqpoint{3.179947in}{0.524170in}}%
\pgfpathlineto{\pgfqpoint{3.179947in}{3.206606in}}%
\pgfusepath{stroke}%
\end{pgfscope}%
\begin{pgfscope}%
\pgfsetbuttcap%
\pgfsetroundjoin%
\definecolor{currentfill}{rgb}{0.000000,0.000000,0.000000}%
\pgfsetfillcolor{currentfill}%
\pgfsetlinewidth{0.803000pt}%
\definecolor{currentstroke}{rgb}{0.000000,0.000000,0.000000}%
\pgfsetstrokecolor{currentstroke}%
\pgfsetdash{}{0pt}%
\pgfsys@defobject{currentmarker}{\pgfqpoint{0.000000in}{-0.048611in}}{\pgfqpoint{0.000000in}{0.000000in}}{%
\pgfpathmoveto{\pgfqpoint{0.000000in}{0.000000in}}%
\pgfpathlineto{\pgfqpoint{0.000000in}{-0.048611in}}%
\pgfusepath{stroke,fill}%
}%
\begin{pgfscope}%
\pgfsys@transformshift{3.179947in}{0.524170in}%
\pgfsys@useobject{currentmarker}{}%
\end{pgfscope}%
\end{pgfscope}%
\begin{pgfscope}%
\definecolor{textcolor}{rgb}{0.000000,0.000000,0.000000}%
\pgfsetstrokecolor{textcolor}%
\pgfsetfillcolor{textcolor}%
\pgftext[x=3.179947in,y=0.426948in,,top]{\color{textcolor}\rmfamily\fontsize{8.000000}{9.600000}\selectfont \(\displaystyle {3}\)}%
\end{pgfscope}%
\begin{pgfscope}%
\pgfpathrectangle{\pgfqpoint{0.661006in}{0.524170in}}{\pgfqpoint{4.620097in}{2.682436in}}%
\pgfusepath{clip}%
\pgfsetrectcap%
\pgfsetroundjoin%
\pgfsetlinewidth{0.803000pt}%
\definecolor{currentstroke}{rgb}{0.450000,0.450000,0.450000}%
\pgfsetstrokecolor{currentstroke}%
\pgfsetdash{}{0pt}%
\pgfpathmoveto{\pgfqpoint{3.958198in}{0.524170in}}%
\pgfpathlineto{\pgfqpoint{3.958198in}{3.206606in}}%
\pgfusepath{stroke}%
\end{pgfscope}%
\begin{pgfscope}%
\pgfsetbuttcap%
\pgfsetroundjoin%
\definecolor{currentfill}{rgb}{0.000000,0.000000,0.000000}%
\pgfsetfillcolor{currentfill}%
\pgfsetlinewidth{0.803000pt}%
\definecolor{currentstroke}{rgb}{0.000000,0.000000,0.000000}%
\pgfsetstrokecolor{currentstroke}%
\pgfsetdash{}{0pt}%
\pgfsys@defobject{currentmarker}{\pgfqpoint{0.000000in}{-0.048611in}}{\pgfqpoint{0.000000in}{0.000000in}}{%
\pgfpathmoveto{\pgfqpoint{0.000000in}{0.000000in}}%
\pgfpathlineto{\pgfqpoint{0.000000in}{-0.048611in}}%
\pgfusepath{stroke,fill}%
}%
\begin{pgfscope}%
\pgfsys@transformshift{3.958198in}{0.524170in}%
\pgfsys@useobject{currentmarker}{}%
\end{pgfscope}%
\end{pgfscope}%
\begin{pgfscope}%
\definecolor{textcolor}{rgb}{0.000000,0.000000,0.000000}%
\pgfsetstrokecolor{textcolor}%
\pgfsetfillcolor{textcolor}%
\pgftext[x=3.958198in,y=0.426948in,,top]{\color{textcolor}\rmfamily\fontsize{8.000000}{9.600000}\selectfont \(\displaystyle {4}\)}%
\end{pgfscope}%
\begin{pgfscope}%
\pgfpathrectangle{\pgfqpoint{0.661006in}{0.524170in}}{\pgfqpoint{4.620097in}{2.682436in}}%
\pgfusepath{clip}%
\pgfsetrectcap%
\pgfsetroundjoin%
\pgfsetlinewidth{0.803000pt}%
\definecolor{currentstroke}{rgb}{0.450000,0.450000,0.450000}%
\pgfsetstrokecolor{currentstroke}%
\pgfsetdash{}{0pt}%
\pgfpathmoveto{\pgfqpoint{4.736449in}{0.524170in}}%
\pgfpathlineto{\pgfqpoint{4.736449in}{3.206606in}}%
\pgfusepath{stroke}%
\end{pgfscope}%
\begin{pgfscope}%
\pgfsetbuttcap%
\pgfsetroundjoin%
\definecolor{currentfill}{rgb}{0.000000,0.000000,0.000000}%
\pgfsetfillcolor{currentfill}%
\pgfsetlinewidth{0.803000pt}%
\definecolor{currentstroke}{rgb}{0.000000,0.000000,0.000000}%
\pgfsetstrokecolor{currentstroke}%
\pgfsetdash{}{0pt}%
\pgfsys@defobject{currentmarker}{\pgfqpoint{0.000000in}{-0.048611in}}{\pgfqpoint{0.000000in}{0.000000in}}{%
\pgfpathmoveto{\pgfqpoint{0.000000in}{0.000000in}}%
\pgfpathlineto{\pgfqpoint{0.000000in}{-0.048611in}}%
\pgfusepath{stroke,fill}%
}%
\begin{pgfscope}%
\pgfsys@transformshift{4.736449in}{0.524170in}%
\pgfsys@useobject{currentmarker}{}%
\end{pgfscope}%
\end{pgfscope}%
\begin{pgfscope}%
\definecolor{textcolor}{rgb}{0.000000,0.000000,0.000000}%
\pgfsetstrokecolor{textcolor}%
\pgfsetfillcolor{textcolor}%
\pgftext[x=4.736449in,y=0.426948in,,top]{\color{textcolor}\rmfamily\fontsize{8.000000}{9.600000}\selectfont \(\displaystyle {5}\)}%
\end{pgfscope}%
\begin{pgfscope}%
\definecolor{textcolor}{rgb}{0.000000,0.000000,0.000000}%
\pgfsetstrokecolor{textcolor}%
\pgfsetfillcolor{textcolor}%
\pgftext[x=2.971054in,y=0.272725in,,top]{\color{textcolor}\rmfamily\fontsize{10.000000}{12.000000}\selectfont Output impedance in \unit{\ohm}}%
\end{pgfscope}%
\begin{pgfscope}%
\definecolor{textcolor}{rgb}{0.000000,0.000000,0.000000}%
\pgfsetstrokecolor{textcolor}%
\pgfsetfillcolor{textcolor}%
\pgftext[x=5.281103in,y=0.286614in,right,top]{\color{textcolor}\rmfamily\fontsize{8.000000}{9.600000}\selectfont \(\displaystyle \times{10^{10}}{}\)}%
\end{pgfscope}%
\begin{pgfscope}%
\pgfpathrectangle{\pgfqpoint{0.661006in}{0.524170in}}{\pgfqpoint{4.620097in}{2.682436in}}%
\pgfusepath{clip}%
\pgfsetrectcap%
\pgfsetroundjoin%
\pgfsetlinewidth{0.803000pt}%
\definecolor{currentstroke}{rgb}{0.450000,0.450000,0.450000}%
\pgfsetstrokecolor{currentstroke}%
\pgfsetdash{}{0pt}%
\pgfpathmoveto{\pgfqpoint{0.661006in}{0.524170in}}%
\pgfpathlineto{\pgfqpoint{5.281103in}{0.524170in}}%
\pgfusepath{stroke}%
\end{pgfscope}%
\begin{pgfscope}%
\pgfsetbuttcap%
\pgfsetroundjoin%
\definecolor{currentfill}{rgb}{0.000000,0.000000,0.000000}%
\pgfsetfillcolor{currentfill}%
\pgfsetlinewidth{0.803000pt}%
\definecolor{currentstroke}{rgb}{0.000000,0.000000,0.000000}%
\pgfsetstrokecolor{currentstroke}%
\pgfsetdash{}{0pt}%
\pgfsys@defobject{currentmarker}{\pgfqpoint{-0.048611in}{0.000000in}}{\pgfqpoint{-0.000000in}{0.000000in}}{%
\pgfpathmoveto{\pgfqpoint{-0.000000in}{0.000000in}}%
\pgfpathlineto{\pgfqpoint{-0.048611in}{0.000000in}}%
\pgfusepath{stroke,fill}%
}%
\begin{pgfscope}%
\pgfsys@transformshift{0.661006in}{0.524170in}%
\pgfsys@useobject{currentmarker}{}%
\end{pgfscope}%
\end{pgfscope}%
\begin{pgfscope}%
\definecolor{textcolor}{rgb}{0.000000,0.000000,0.000000}%
\pgfsetstrokecolor{textcolor}%
\pgfsetfillcolor{textcolor}%
\pgftext[x=0.504755in, y=0.485614in, left, base]{\color{textcolor}\rmfamily\fontsize{8.000000}{9.600000}\selectfont \(\displaystyle {0}\)}%
\end{pgfscope}%
\begin{pgfscope}%
\pgfpathrectangle{\pgfqpoint{0.661006in}{0.524170in}}{\pgfqpoint{4.620097in}{2.682436in}}%
\pgfusepath{clip}%
\pgfsetrectcap%
\pgfsetroundjoin%
\pgfsetlinewidth{0.803000pt}%
\definecolor{currentstroke}{rgb}{0.450000,0.450000,0.450000}%
\pgfsetstrokecolor{currentstroke}%
\pgfsetdash{}{0pt}%
\pgfpathmoveto{\pgfqpoint{0.661006in}{0.843507in}}%
\pgfpathlineto{\pgfqpoint{5.281103in}{0.843507in}}%
\pgfusepath{stroke}%
\end{pgfscope}%
\begin{pgfscope}%
\pgfsetbuttcap%
\pgfsetroundjoin%
\definecolor{currentfill}{rgb}{0.000000,0.000000,0.000000}%
\pgfsetfillcolor{currentfill}%
\pgfsetlinewidth{0.803000pt}%
\definecolor{currentstroke}{rgb}{0.000000,0.000000,0.000000}%
\pgfsetstrokecolor{currentstroke}%
\pgfsetdash{}{0pt}%
\pgfsys@defobject{currentmarker}{\pgfqpoint{-0.048611in}{0.000000in}}{\pgfqpoint{-0.000000in}{0.000000in}}{%
\pgfpathmoveto{\pgfqpoint{-0.000000in}{0.000000in}}%
\pgfpathlineto{\pgfqpoint{-0.048611in}{0.000000in}}%
\pgfusepath{stroke,fill}%
}%
\begin{pgfscope}%
\pgfsys@transformshift{0.661006in}{0.843507in}%
\pgfsys@useobject{currentmarker}{}%
\end{pgfscope}%
\end{pgfscope}%
\begin{pgfscope}%
\definecolor{textcolor}{rgb}{0.000000,0.000000,0.000000}%
\pgfsetstrokecolor{textcolor}%
\pgfsetfillcolor{textcolor}%
\pgftext[x=0.386698in, y=0.804952in, left, base]{\color{textcolor}\rmfamily\fontsize{8.000000}{9.600000}\selectfont \(\displaystyle {500}\)}%
\end{pgfscope}%
\begin{pgfscope}%
\pgfpathrectangle{\pgfqpoint{0.661006in}{0.524170in}}{\pgfqpoint{4.620097in}{2.682436in}}%
\pgfusepath{clip}%
\pgfsetrectcap%
\pgfsetroundjoin%
\pgfsetlinewidth{0.803000pt}%
\definecolor{currentstroke}{rgb}{0.450000,0.450000,0.450000}%
\pgfsetstrokecolor{currentstroke}%
\pgfsetdash{}{0pt}%
\pgfpathmoveto{\pgfqpoint{0.661006in}{1.162845in}}%
\pgfpathlineto{\pgfqpoint{5.281103in}{1.162845in}}%
\pgfusepath{stroke}%
\end{pgfscope}%
\begin{pgfscope}%
\pgfsetbuttcap%
\pgfsetroundjoin%
\definecolor{currentfill}{rgb}{0.000000,0.000000,0.000000}%
\pgfsetfillcolor{currentfill}%
\pgfsetlinewidth{0.803000pt}%
\definecolor{currentstroke}{rgb}{0.000000,0.000000,0.000000}%
\pgfsetstrokecolor{currentstroke}%
\pgfsetdash{}{0pt}%
\pgfsys@defobject{currentmarker}{\pgfqpoint{-0.048611in}{0.000000in}}{\pgfqpoint{-0.000000in}{0.000000in}}{%
\pgfpathmoveto{\pgfqpoint{-0.000000in}{0.000000in}}%
\pgfpathlineto{\pgfqpoint{-0.048611in}{0.000000in}}%
\pgfusepath{stroke,fill}%
}%
\begin{pgfscope}%
\pgfsys@transformshift{0.661006in}{1.162845in}%
\pgfsys@useobject{currentmarker}{}%
\end{pgfscope}%
\end{pgfscope}%
\begin{pgfscope}%
\definecolor{textcolor}{rgb}{0.000000,0.000000,0.000000}%
\pgfsetstrokecolor{textcolor}%
\pgfsetfillcolor{textcolor}%
\pgftext[x=0.327669in, y=1.124290in, left, base]{\color{textcolor}\rmfamily\fontsize{8.000000}{9.600000}\selectfont \(\displaystyle {1000}\)}%
\end{pgfscope}%
\begin{pgfscope}%
\pgfpathrectangle{\pgfqpoint{0.661006in}{0.524170in}}{\pgfqpoint{4.620097in}{2.682436in}}%
\pgfusepath{clip}%
\pgfsetrectcap%
\pgfsetroundjoin%
\pgfsetlinewidth{0.803000pt}%
\definecolor{currentstroke}{rgb}{0.450000,0.450000,0.450000}%
\pgfsetstrokecolor{currentstroke}%
\pgfsetdash{}{0pt}%
\pgfpathmoveto{\pgfqpoint{0.661006in}{1.482183in}}%
\pgfpathlineto{\pgfqpoint{5.281103in}{1.482183in}}%
\pgfusepath{stroke}%
\end{pgfscope}%
\begin{pgfscope}%
\pgfsetbuttcap%
\pgfsetroundjoin%
\definecolor{currentfill}{rgb}{0.000000,0.000000,0.000000}%
\pgfsetfillcolor{currentfill}%
\pgfsetlinewidth{0.803000pt}%
\definecolor{currentstroke}{rgb}{0.000000,0.000000,0.000000}%
\pgfsetstrokecolor{currentstroke}%
\pgfsetdash{}{0pt}%
\pgfsys@defobject{currentmarker}{\pgfqpoint{-0.048611in}{0.000000in}}{\pgfqpoint{-0.000000in}{0.000000in}}{%
\pgfpathmoveto{\pgfqpoint{-0.000000in}{0.000000in}}%
\pgfpathlineto{\pgfqpoint{-0.048611in}{0.000000in}}%
\pgfusepath{stroke,fill}%
}%
\begin{pgfscope}%
\pgfsys@transformshift{0.661006in}{1.482183in}%
\pgfsys@useobject{currentmarker}{}%
\end{pgfscope}%
\end{pgfscope}%
\begin{pgfscope}%
\definecolor{textcolor}{rgb}{0.000000,0.000000,0.000000}%
\pgfsetstrokecolor{textcolor}%
\pgfsetfillcolor{textcolor}%
\pgftext[x=0.327669in, y=1.443627in, left, base]{\color{textcolor}\rmfamily\fontsize{8.000000}{9.600000}\selectfont \(\displaystyle {1500}\)}%
\end{pgfscope}%
\begin{pgfscope}%
\pgfpathrectangle{\pgfqpoint{0.661006in}{0.524170in}}{\pgfqpoint{4.620097in}{2.682436in}}%
\pgfusepath{clip}%
\pgfsetrectcap%
\pgfsetroundjoin%
\pgfsetlinewidth{0.803000pt}%
\definecolor{currentstroke}{rgb}{0.450000,0.450000,0.450000}%
\pgfsetstrokecolor{currentstroke}%
\pgfsetdash{}{0pt}%
\pgfpathmoveto{\pgfqpoint{0.661006in}{1.801520in}}%
\pgfpathlineto{\pgfqpoint{5.281103in}{1.801520in}}%
\pgfusepath{stroke}%
\end{pgfscope}%
\begin{pgfscope}%
\pgfsetbuttcap%
\pgfsetroundjoin%
\definecolor{currentfill}{rgb}{0.000000,0.000000,0.000000}%
\pgfsetfillcolor{currentfill}%
\pgfsetlinewidth{0.803000pt}%
\definecolor{currentstroke}{rgb}{0.000000,0.000000,0.000000}%
\pgfsetstrokecolor{currentstroke}%
\pgfsetdash{}{0pt}%
\pgfsys@defobject{currentmarker}{\pgfqpoint{-0.048611in}{0.000000in}}{\pgfqpoint{-0.000000in}{0.000000in}}{%
\pgfpathmoveto{\pgfqpoint{-0.000000in}{0.000000in}}%
\pgfpathlineto{\pgfqpoint{-0.048611in}{0.000000in}}%
\pgfusepath{stroke,fill}%
}%
\begin{pgfscope}%
\pgfsys@transformshift{0.661006in}{1.801520in}%
\pgfsys@useobject{currentmarker}{}%
\end{pgfscope}%
\end{pgfscope}%
\begin{pgfscope}%
\definecolor{textcolor}{rgb}{0.000000,0.000000,0.000000}%
\pgfsetstrokecolor{textcolor}%
\pgfsetfillcolor{textcolor}%
\pgftext[x=0.327669in, y=1.762965in, left, base]{\color{textcolor}\rmfamily\fontsize{8.000000}{9.600000}\selectfont \(\displaystyle {2000}\)}%
\end{pgfscope}%
\begin{pgfscope}%
\pgfpathrectangle{\pgfqpoint{0.661006in}{0.524170in}}{\pgfqpoint{4.620097in}{2.682436in}}%
\pgfusepath{clip}%
\pgfsetrectcap%
\pgfsetroundjoin%
\pgfsetlinewidth{0.803000pt}%
\definecolor{currentstroke}{rgb}{0.450000,0.450000,0.450000}%
\pgfsetstrokecolor{currentstroke}%
\pgfsetdash{}{0pt}%
\pgfpathmoveto{\pgfqpoint{0.661006in}{2.120858in}}%
\pgfpathlineto{\pgfqpoint{5.281103in}{2.120858in}}%
\pgfusepath{stroke}%
\end{pgfscope}%
\begin{pgfscope}%
\pgfsetbuttcap%
\pgfsetroundjoin%
\definecolor{currentfill}{rgb}{0.000000,0.000000,0.000000}%
\pgfsetfillcolor{currentfill}%
\pgfsetlinewidth{0.803000pt}%
\definecolor{currentstroke}{rgb}{0.000000,0.000000,0.000000}%
\pgfsetstrokecolor{currentstroke}%
\pgfsetdash{}{0pt}%
\pgfsys@defobject{currentmarker}{\pgfqpoint{-0.048611in}{0.000000in}}{\pgfqpoint{-0.000000in}{0.000000in}}{%
\pgfpathmoveto{\pgfqpoint{-0.000000in}{0.000000in}}%
\pgfpathlineto{\pgfqpoint{-0.048611in}{0.000000in}}%
\pgfusepath{stroke,fill}%
}%
\begin{pgfscope}%
\pgfsys@transformshift{0.661006in}{2.120858in}%
\pgfsys@useobject{currentmarker}{}%
\end{pgfscope}%
\end{pgfscope}%
\begin{pgfscope}%
\definecolor{textcolor}{rgb}{0.000000,0.000000,0.000000}%
\pgfsetstrokecolor{textcolor}%
\pgfsetfillcolor{textcolor}%
\pgftext[x=0.327669in, y=2.082302in, left, base]{\color{textcolor}\rmfamily\fontsize{8.000000}{9.600000}\selectfont \(\displaystyle {2500}\)}%
\end{pgfscope}%
\begin{pgfscope}%
\pgfpathrectangle{\pgfqpoint{0.661006in}{0.524170in}}{\pgfqpoint{4.620097in}{2.682436in}}%
\pgfusepath{clip}%
\pgfsetrectcap%
\pgfsetroundjoin%
\pgfsetlinewidth{0.803000pt}%
\definecolor{currentstroke}{rgb}{0.450000,0.450000,0.450000}%
\pgfsetstrokecolor{currentstroke}%
\pgfsetdash{}{0pt}%
\pgfpathmoveto{\pgfqpoint{0.661006in}{2.440196in}}%
\pgfpathlineto{\pgfqpoint{5.281103in}{2.440196in}}%
\pgfusepath{stroke}%
\end{pgfscope}%
\begin{pgfscope}%
\pgfsetbuttcap%
\pgfsetroundjoin%
\definecolor{currentfill}{rgb}{0.000000,0.000000,0.000000}%
\pgfsetfillcolor{currentfill}%
\pgfsetlinewidth{0.803000pt}%
\definecolor{currentstroke}{rgb}{0.000000,0.000000,0.000000}%
\pgfsetstrokecolor{currentstroke}%
\pgfsetdash{}{0pt}%
\pgfsys@defobject{currentmarker}{\pgfqpoint{-0.048611in}{0.000000in}}{\pgfqpoint{-0.000000in}{0.000000in}}{%
\pgfpathmoveto{\pgfqpoint{-0.000000in}{0.000000in}}%
\pgfpathlineto{\pgfqpoint{-0.048611in}{0.000000in}}%
\pgfusepath{stroke,fill}%
}%
\begin{pgfscope}%
\pgfsys@transformshift{0.661006in}{2.440196in}%
\pgfsys@useobject{currentmarker}{}%
\end{pgfscope}%
\end{pgfscope}%
\begin{pgfscope}%
\definecolor{textcolor}{rgb}{0.000000,0.000000,0.000000}%
\pgfsetstrokecolor{textcolor}%
\pgfsetfillcolor{textcolor}%
\pgftext[x=0.327669in, y=2.401640in, left, base]{\color{textcolor}\rmfamily\fontsize{8.000000}{9.600000}\selectfont \(\displaystyle {3000}\)}%
\end{pgfscope}%
\begin{pgfscope}%
\pgfpathrectangle{\pgfqpoint{0.661006in}{0.524170in}}{\pgfqpoint{4.620097in}{2.682436in}}%
\pgfusepath{clip}%
\pgfsetrectcap%
\pgfsetroundjoin%
\pgfsetlinewidth{0.803000pt}%
\definecolor{currentstroke}{rgb}{0.450000,0.450000,0.450000}%
\pgfsetstrokecolor{currentstroke}%
\pgfsetdash{}{0pt}%
\pgfpathmoveto{\pgfqpoint{0.661006in}{2.759533in}}%
\pgfpathlineto{\pgfqpoint{5.281103in}{2.759533in}}%
\pgfusepath{stroke}%
\end{pgfscope}%
\begin{pgfscope}%
\pgfsetbuttcap%
\pgfsetroundjoin%
\definecolor{currentfill}{rgb}{0.000000,0.000000,0.000000}%
\pgfsetfillcolor{currentfill}%
\pgfsetlinewidth{0.803000pt}%
\definecolor{currentstroke}{rgb}{0.000000,0.000000,0.000000}%
\pgfsetstrokecolor{currentstroke}%
\pgfsetdash{}{0pt}%
\pgfsys@defobject{currentmarker}{\pgfqpoint{-0.048611in}{0.000000in}}{\pgfqpoint{-0.000000in}{0.000000in}}{%
\pgfpathmoveto{\pgfqpoint{-0.000000in}{0.000000in}}%
\pgfpathlineto{\pgfqpoint{-0.048611in}{0.000000in}}%
\pgfusepath{stroke,fill}%
}%
\begin{pgfscope}%
\pgfsys@transformshift{0.661006in}{2.759533in}%
\pgfsys@useobject{currentmarker}{}%
\end{pgfscope}%
\end{pgfscope}%
\begin{pgfscope}%
\definecolor{textcolor}{rgb}{0.000000,0.000000,0.000000}%
\pgfsetstrokecolor{textcolor}%
\pgfsetfillcolor{textcolor}%
\pgftext[x=0.327669in, y=2.720978in, left, base]{\color{textcolor}\rmfamily\fontsize{8.000000}{9.600000}\selectfont \(\displaystyle {3500}\)}%
\end{pgfscope}%
\begin{pgfscope}%
\pgfpathrectangle{\pgfqpoint{0.661006in}{0.524170in}}{\pgfqpoint{4.620097in}{2.682436in}}%
\pgfusepath{clip}%
\pgfsetrectcap%
\pgfsetroundjoin%
\pgfsetlinewidth{0.803000pt}%
\definecolor{currentstroke}{rgb}{0.450000,0.450000,0.450000}%
\pgfsetstrokecolor{currentstroke}%
\pgfsetdash{}{0pt}%
\pgfpathmoveto{\pgfqpoint{0.661006in}{3.078871in}}%
\pgfpathlineto{\pgfqpoint{5.281103in}{3.078871in}}%
\pgfusepath{stroke}%
\end{pgfscope}%
\begin{pgfscope}%
\pgfsetbuttcap%
\pgfsetroundjoin%
\definecolor{currentfill}{rgb}{0.000000,0.000000,0.000000}%
\pgfsetfillcolor{currentfill}%
\pgfsetlinewidth{0.803000pt}%
\definecolor{currentstroke}{rgb}{0.000000,0.000000,0.000000}%
\pgfsetstrokecolor{currentstroke}%
\pgfsetdash{}{0pt}%
\pgfsys@defobject{currentmarker}{\pgfqpoint{-0.048611in}{0.000000in}}{\pgfqpoint{-0.000000in}{0.000000in}}{%
\pgfpathmoveto{\pgfqpoint{-0.000000in}{0.000000in}}%
\pgfpathlineto{\pgfqpoint{-0.048611in}{0.000000in}}%
\pgfusepath{stroke,fill}%
}%
\begin{pgfscope}%
\pgfsys@transformshift{0.661006in}{3.078871in}%
\pgfsys@useobject{currentmarker}{}%
\end{pgfscope}%
\end{pgfscope}%
\begin{pgfscope}%
\definecolor{textcolor}{rgb}{0.000000,0.000000,0.000000}%
\pgfsetstrokecolor{textcolor}%
\pgfsetfillcolor{textcolor}%
\pgftext[x=0.327669in, y=3.040315in, left, base]{\color{textcolor}\rmfamily\fontsize{8.000000}{9.600000}\selectfont \(\displaystyle {4000}\)}%
\end{pgfscope}%
\begin{pgfscope}%
\definecolor{textcolor}{rgb}{0.000000,0.000000,0.000000}%
\pgfsetstrokecolor{textcolor}%
\pgfsetfillcolor{textcolor}%
\pgftext[x=0.272113in,y=1.865388in,,bottom,rotate=90.000000]{\color{textcolor}\rmfamily\fontsize{10.000000}{12.000000}\selectfont Counts}%
\end{pgfscope}%
\begin{pgfscope}%
\pgfsetrectcap%
\pgfsetmiterjoin%
\pgfsetlinewidth{0.803000pt}%
\definecolor{currentstroke}{rgb}{0.000000,0.000000,0.000000}%
\pgfsetstrokecolor{currentstroke}%
\pgfsetdash{}{0pt}%
\pgfpathmoveto{\pgfqpoint{0.661006in}{0.524170in}}%
\pgfpathlineto{\pgfqpoint{0.661006in}{3.206606in}}%
\pgfusepath{stroke}%
\end{pgfscope}%
\begin{pgfscope}%
\pgfsetrectcap%
\pgfsetmiterjoin%
\pgfsetlinewidth{0.803000pt}%
\definecolor{currentstroke}{rgb}{0.000000,0.000000,0.000000}%
\pgfsetstrokecolor{currentstroke}%
\pgfsetdash{}{0pt}%
\pgfpathmoveto{\pgfqpoint{5.281103in}{0.524170in}}%
\pgfpathlineto{\pgfqpoint{5.281103in}{3.206606in}}%
\pgfusepath{stroke}%
\end{pgfscope}%
\begin{pgfscope}%
\pgfsetrectcap%
\pgfsetmiterjoin%
\pgfsetlinewidth{0.803000pt}%
\definecolor{currentstroke}{rgb}{0.000000,0.000000,0.000000}%
\pgfsetstrokecolor{currentstroke}%
\pgfsetdash{}{0pt}%
\pgfpathmoveto{\pgfqpoint{0.661006in}{0.524170in}}%
\pgfpathlineto{\pgfqpoint{5.281103in}{0.524170in}}%
\pgfusepath{stroke}%
\end{pgfscope}%
\begin{pgfscope}%
\pgfsetrectcap%
\pgfsetmiterjoin%
\pgfsetlinewidth{0.803000pt}%
\definecolor{currentstroke}{rgb}{0.000000,0.000000,0.000000}%
\pgfsetstrokecolor{currentstroke}%
\pgfsetdash{}{0pt}%
\pgfpathmoveto{\pgfqpoint{0.661006in}{3.206606in}}%
\pgfpathlineto{\pgfqpoint{5.281103in}{3.206606in}}%
\pgfusepath{stroke}%
\end{pgfscope}%
\begin{pgfscope}%
\pgfsetbuttcap%
\pgfsetmiterjoin%
\definecolor{currentfill}{rgb}{1.000000,1.000000,1.000000}%
\pgfsetfillcolor{currentfill}%
\pgfsetfillopacity{0.800000}%
\pgfsetlinewidth{1.003750pt}%
\definecolor{currentstroke}{rgb}{0.800000,0.800000,0.800000}%
\pgfsetstrokecolor{currentstroke}%
\pgfsetstrokeopacity{0.800000}%
\pgfsetdash{}{0pt}%
\pgfpathmoveto{\pgfqpoint{0.738783in}{2.651384in}}%
\pgfpathlineto{\pgfqpoint{2.576618in}{2.651384in}}%
\pgfpathquadraticcurveto{\pgfqpoint{2.598840in}{2.651384in}}{\pgfqpoint{2.598840in}{2.673607in}}%
\pgfpathlineto{\pgfqpoint{2.598840in}{3.128828in}}%
\pgfpathquadraticcurveto{\pgfqpoint{2.598840in}{3.151050in}}{\pgfqpoint{2.576618in}{3.151050in}}%
\pgfpathlineto{\pgfqpoint{0.738783in}{3.151050in}}%
\pgfpathquadraticcurveto{\pgfqpoint{0.716561in}{3.151050in}}{\pgfqpoint{0.716561in}{3.128828in}}%
\pgfpathlineto{\pgfqpoint{0.716561in}{2.673607in}}%
\pgfpathquadraticcurveto{\pgfqpoint{0.716561in}{2.651384in}}{\pgfqpoint{0.738783in}{2.651384in}}%
\pgfpathlineto{\pgfqpoint{0.738783in}{2.651384in}}%
\pgfpathclose%
\pgfusepath{stroke,fill}%
\end{pgfscope}%
\begin{pgfscope}%
\pgfsetbuttcap%
\pgfsetmiterjoin%
\definecolor{currentfill}{rgb}{0.870588,0.560784,0.019608}%
\pgfsetfillcolor{currentfill}%
\pgfsetfillopacity{0.700000}%
\pgfsetlinewidth{0.000000pt}%
\definecolor{currentstroke}{rgb}{0.000000,0.000000,0.000000}%
\pgfsetstrokecolor{currentstroke}%
\pgfsetstrokeopacity{0.700000}%
\pgfsetdash{}{0pt}%
\pgfpathmoveto{\pgfqpoint{0.761006in}{3.028828in}}%
\pgfpathlineto{\pgfqpoint{0.983228in}{3.028828in}}%
\pgfpathlineto{\pgfqpoint{0.983228in}{3.106606in}}%
\pgfpathlineto{\pgfqpoint{0.761006in}{3.106606in}}%
\pgfpathlineto{\pgfqpoint{0.761006in}{3.028828in}}%
\pgfpathclose%
\pgfusepath{fill}%
\end{pgfscope}%
\begin{pgfscope}%
\definecolor{textcolor}{rgb}{0.000000,0.000000,0.000000}%
\pgfsetstrokecolor{textcolor}%
\pgfsetfillcolor{textcolor}%
\pgftext[x=1.072117in,y=3.028828in,left,base]{\color{textcolor}\rmfamily\fontsize{8.000000}{9.600000}\selectfont Parallel MOSFETs}%
\end{pgfscope}%
\begin{pgfscope}%
\pgfsetbuttcap%
\pgfsetmiterjoin%
\definecolor{currentfill}{rgb}{0.007843,0.619608,0.450980}%
\pgfsetfillcolor{currentfill}%
\pgfsetfillopacity{0.700000}%
\pgfsetlinewidth{0.000000pt}%
\definecolor{currentstroke}{rgb}{0.000000,0.000000,0.000000}%
\pgfsetstrokecolor{currentstroke}%
\pgfsetstrokeopacity{0.700000}%
\pgfsetdash{}{0pt}%
\pgfpathmoveto{\pgfqpoint{0.761006in}{2.873495in}}%
\pgfpathlineto{\pgfqpoint{0.983228in}{2.873495in}}%
\pgfpathlineto{\pgfqpoint{0.983228in}{2.951273in}}%
\pgfpathlineto{\pgfqpoint{0.761006in}{2.951273in}}%
\pgfpathlineto{\pgfqpoint{0.761006in}{2.873495in}}%
\pgfpathclose%
\pgfusepath{fill}%
\end{pgfscope}%
\begin{pgfscope}%
\definecolor{textcolor}{rgb}{0.000000,0.000000,0.000000}%
\pgfsetstrokecolor{textcolor}%
\pgfsetfillcolor{textcolor}%
\pgftext[x=1.072117in,y=2.873495in,left,base]{\color{textcolor}\rmfamily\fontsize{8.000000}{9.600000}\selectfont Single MOSFET}%
\end{pgfscope}%
\begin{pgfscope}%
\pgfsetbuttcap%
\pgfsetmiterjoin%
\definecolor{currentfill}{rgb}{0.003922,0.450980,0.698039}%
\pgfsetfillcolor{currentfill}%
\pgfsetfillopacity{0.700000}%
\pgfsetlinewidth{0.000000pt}%
\definecolor{currentstroke}{rgb}{0.000000,0.000000,0.000000}%
\pgfsetstrokecolor{currentstroke}%
\pgfsetstrokeopacity{0.700000}%
\pgfsetdash{}{0pt}%
\pgfpathmoveto{\pgfqpoint{0.761006in}{2.717384in}}%
\pgfpathlineto{\pgfqpoint{0.983228in}{2.717384in}}%
\pgfpathlineto{\pgfqpoint{0.983228in}{2.795162in}}%
\pgfpathlineto{\pgfqpoint{0.761006in}{2.795162in}}%
\pgfpathlineto{\pgfqpoint{0.761006in}{2.717384in}}%
\pgfpathclose%
\pgfusepath{fill}%
\end{pgfscope}%
\begin{pgfscope}%
\definecolor{textcolor}{rgb}{0.000000,0.000000,0.000000}%
\pgfsetstrokecolor{textcolor}%
\pgfsetfillcolor{textcolor}%
\pgftext[x=1.072117in,y=2.717384in,left,base]{\color{textcolor}\rmfamily\fontsize{8.000000}{9.600000}\selectfont Parallel MOSFET \(\displaystyle V_{DS}+1\sigma\)}%
\end{pgfscope}%
\end{pgfpicture}%
\makeatother%
\endgroup%
% data/plot_ltspice_monte-carlo.py
    \caption{Results of a Monte Carlo simulation of the output impedance for different configurations of MOSFETs.}
    \label{fig:ltpsice_mosfet_mc_output_impedance}
\end{figure}

Unsurprisingly, there is no variance of the output impedance in the single MOSFET case in accordance to what can be seen in appendix \ref{sec:transfer_function_transconductance}. The op-amp gain simply suppresses all device properties of the MOSFET. The slight variation of $g_m$ for different samples was not simulated, because this variation stems from the variation of $\kappa$ and goes as $\frac{1}{\sqrt{\kappa}}$, so its effect is not as pronounced as the threshold.

In case of two MOSFETs, the output impedance varies over an order of magnitude from about \qtyrange[range-units = single]{1.8}{52}{\giga \ohm}. Even when increasing the drain-source voltage by $1 \sigma = \qty{70}{\mV}$ to \qty{625}{\mV} on average, the spread is still an order of magnitude. Only when increasing $V_{DS}$ to around \qty{700}{\mV}, the situation stabilises, but then the net gain from this measure has shrunk to a meager \qty{84}{\mV}. It can be seen from this simulation that the system-to-system spread becomes very unstable in tough situations. Such instability can also be brought into the system by temperature effects as $V_{th}$ is temperature dependent as discussed above. Additionally, it may suffer from thermal runaway unless each individual MOSFET is laid out to carry the full current.

\subsection{Noise Sources}%
\label{sec:current_source_noise}
The fundamentals of different types of noise were already introduced in section \ref{sec:allan_deviation}. Here, a subset of those noise types is revisited. It is expected that the dominant noise observed in the current source circuit is $\frac{1}{f}$-noise at low frequencies and white noise towards higher frequencies. All noise components will be converted to the so-called input referred notation to make the noise sources comparable. This can be easily understood when looking at two amplifiers with different gain. If both of them add a fixed amount of noise to the output signal, the absolute amount of noise may be the same, but the signal to noise ratio shows a different picture. To compare these amplifiers it is useful to divide the noise by the transfer function (gain) of the amplifier. This is called the input-referred noise, since it treats the noise in relation to the input signal. Additionally, when calculating noise figures, the noise bandwidth is always considered to be \qty{1}{\Hz} unless specified otherwise.
\begin{figure}[ht]
    \centering
    \import{figures/}{precision_current_source.tex}
    \caption{Transconductance amplifier with a p-channel MOSFET. Copied from page \pageref{fig:precision_current_source}.}
    \label{fig:precision_current_source_noise}
\end{figure}

Noise sources are ubiquitous in the circuit shown in figure \ref{fig:precision_current_source_noise}, which is repeated here from figure \ref{fig:precision_current_source} on page \pageref{fig:precision_current_source} for clarity. The resistor $R_s$, the MOSFET, the op-amp, the setpoint voltage $V_{ref}$ and the supply voltage $V_{sup}$ can all contribute noise to the output current. Fortunately, some of those noise contributions are either very small or well suppressed in this design, so each component must be briefly discussed to see if they can indeed be safely neglected.

Starting with the supply voltage $V_{sup}$, it can be seen that any change of this voltage affects the string $R_s$-$Q$-$R_{load}$. From equation \ref{eqn:transconductance_amplifier_transfer_function}, it is known that if the op-amp gain is high (true within the bandwidth of the op-amp) all disturbances of the voltage across $R_s$ will be suppressed and the output current is only defined by the reference input and $R_s$. Looking closer, the supply noise is present at the inverting and non-inverting input of the op-amp with the same magnitude. If there is no current flowing into the op-amp pins, which is true for low frequencies, the noise is affecting both pins equally and it will be suppressed by the common-mode-rejection ratio (CMRR) of the device. Fortunately, this is a strong quality of precision op-amps and values of more than \qty[per-mode=power]{1}{\uV \per \volt} are not uncommon. The op-amp will therefore take care of the supply noise at low frequencies. At high frequencies the parasitic capacitance of the input pins and the reduced gain and CMRR come into play. To take care of this, it is therefore prudent to filter the supply voltage for high frequency noise.

The next noise source is the reference voltage. The reference is directly connected to the input and its noise dictates most of the circuit noise. While the high-frequency noise can again be filtered to some extend, the low frequency noise, which is mostly $\frac{1}{f}$-noise cannot be filtered as was shown in section \ref{sec:flicker_noise}, so it must be kept low from the start and the reference selected for low flicker noise.

The MOSFET as a noise source is considered in appendix \ref{sec:mosfet_noise} and the interested reader may find the derivation of the MOSFET noise component there. The two types of noise that need to be considered are the flicker noise of the MOSFET and its wideband thermal noise as calculated in equation \ref{eqn:current_noise_mosfet} on page \pageref{eqn:current_noise_mosfet}
\begin{equation*}
    i_{n} = \sqrt{\underbrace{4 k_B T \frac{2}{3} g_m}_{\text{thermal}} + \underbrace{\frac{K_f I_D}{C_{ox} L^2} \frac{1}{f}}_{\text{flicker}}} \,.
\end{equation*}

To calculate the input referred noise in order to show that the MOSFET noise will be suppressed by the op-amp, the current noise needs to be divided by the open-loop gain derived as equation \ref{eqn:transconductance_amplifier_open_loop_gain} on page \pageref{eqn:transconductance_amplifier_open_loop_gain}
\begin{equation}
    e_{n,FET} = \frac{i_n}{A_f} = \frac{\sqrt{4 k_B T \frac{2}{3} g_m + \frac{K_f I_D}{C_{ox} L^2} \frac{1}{f}}}{\frac{A_{op}}{R_s} \frac{g_m \left(R_o || R_s || R_{id}\right)}{g_m \left(R_o || R_s || R_{id}\right) + 1}} \,.
\end{equation}

Looking at the parameters from table \ref{tab:current_source_parameters}, it is found that $\left(R_o || R_s || R_{id}\right) \approx R_s$ and $e_n$ can be simplified to
\begin{align*}
    e_{n,FET} &\approx \frac{\sqrt{4 k_B T \frac{2}{3} g_m + \frac{K_f I_D}{C_{ox} L^2} \frac{1}{f}}}{A_1 \frac{1}{R_s + \frac{1}{g_m}}}\\
    &\approx \frac{R_s + \frac{1}{g_m}}{A_1} \sqrt{4 k_B T \frac{2}{3} g_m + \frac{K_f I_D}{C_{ox} L^2} \frac{1}{f}}\\
    \overset{A_1 \to \infty}&{=} 0
\end{align*}

Unless the MOSFET transconductance $g_m$ or the gain of the op-amp $A_1$ become very small, the noise of the MOSFET is very well suppressed. If the wideband thermal noise contribution is small (which it is, see \ref{sec:mosfet_noise}) and the flicker noise corner frequency is within the bandwidth of the op-amp, the noise contribution from the MOSFET can be neglected.

The noise contribution from the sense resistor $R_s$ is the Johnson–Nyquist noise, see section \ref{sec:white_noise}, which when transformed to its Norton representation can be written as a current noise
\begin{equation}
    i_{n,R} = \sqrt{\frac{4 k_B T}{R_s}} \label{eqn:current_noise_resistor}\,.
\end{equation}

Additionally, it was shown that depending on the material of the resistive element, a flicker noise component can also be present. This is especially prevalent in carbon and thick-film resistors \cite{flicker_noise_carbon_film,1_f_noise_thick_film}. While thin-film resistors are less noisy, their performance varies greatly between different models \cite{resistor_current_noise_ligo}, so their make and model must be carefully selected for the application. Metal foil and wirewound resistors were shown to perform best and have almost no flicker noise \cite{resistor_current_noise_ligo,flicker_noise_foil_resistor_beev}. Using a high quality resistor, the flicker noise can be neglected and only the thermal noise must be taken into account.

The sense resistor is part of the feedback network and therefore it contributes fully to the noise of the transconductance amplifier. Input referred, the current noise must be divided by the closed-loop gain $A_f$ given by \ref{eqn:transfer_function_closed_loop} on page \pageref{eqn:transfer_function_closed_loop}.
\begin{equation}
    e_{n,R} = i_{n,R} \cdot \beta \approx i_n \cdot R_s = \sqrt{4 k_B T R_s} \label{eqn:noise_sense_resistor}
\end{equation}

The final component to be discussed is the operational amplifier. Although the op-amp is a rather complex device, its noise can be modeled with sufficient accuracy by a small number of noise sources. The typical noise model of an op-amp is shown in figure \ref{fig:op-amp_noise_model}.
\begin{figure}[ht]
    \centering
    %\scalebox{1} % scalebox
    \caption{Noise model of the operational amplifier.}
    \label{fig:op-amp_noise_model}
\end{figure}

In figure \ref{fig:op-amp_noise_model} one can see that there are three noise sources required to treat the op-amp. The input voltage noise source $e_{n}$ and two input current noise sources $i_n$. The current noise noise sources $i_n$ are assumed to be mostly uncorrelated. This assumption leads to an upper bound as can be seen from figure \ref{fig:op-amp_input_stage}, which shows the input differential amplifier, which is the first stage of a typical bipolar op-amp.
\begin{figure}[hb]
    \centering
    %\scalebox{1} % scalebox
    \caption{Bipolar op-amp input stage with noise sources.}
    \label{fig:op-amp_input_stage}
\end{figure}

Of the three noise sources $i_{n,p}$, $i_{n,n}$ and $i_{n,EE}$ only $i_{n,p}$ and $i_{n,n}$ are uncorrelated, because it is the input bias current of the individual transistors, and only the effect of $i_{n,EE}$ is correlated, because the current of the emitter bias current source is equally distributed between the two input transistors. Since effects of equal magnitude and sign cancel out due to the differential nature of the input stage, correlated effects are suppressed. An equal magnitude can be assumed, because the gain of the two transistors is well matched, due to their close proximity on the semiconductor die. Therefore assuming all noise is uncorrelated presents an upper bound for the current noise $i_n$. A more detailed analysis can be found in \cite{op-amp_noise_correlation}. Due to the matching of the transistors, the magnitude $i_{n,p}$ and $i_{n,n}$ are also closely matched, hence, in the model used here, they are assumed equal and referred to as $i_n$ from now on. These two current noise sources cannot be combined with the voltage noise source, because they depend on the external impedance connected to the op-amp, so this final step must be done by the circuit designer as shown next.

Both the voltage noise $e_n$ and the current noise $i_n$ can be found in the datasheet of the op-amp. Typically, these values given are input-referred values. For the complete circuit as in figure \ref{fig:precision_current_source_noise} it is possible to calculate the full noise contribution of the op-amp as
\begin{equation}
    e_{n,op} = \sqrt{e_n^2 + e_{n,+}^2 + e_{n,-}^2}\,,
\end{equation}
assuming the noise sources are uncorrelated. The input referred voltage noise $e_{n,-}$ of the inverting input resulting from the current noise can be calculated in a similar fashion as $e_{n_R}$ in equation \ref{eqn:noise_sense_resistor}. It is likewise part of the feedback network and must therefore be divided by the closed-loop gain $A_f$ as before.
\begin{equation}
    e_{n,-} \approx i_n \cdot R_s
\end{equation}

The current noise of the non-inverting input can be translated by considering its input impedance. This is determined by the filter circuit of the reference voltage which is required to remove the high frequency noise as discussed above. Assuming an RC-filter of first order,
the output impedance can be calculated from the transfer function of the low-pass filter, derived in equation \ref{eqn:first-order_model}
\begin{align}
    R_{out,f} &= R_{f} \cdot A = \frac{R_{f}}{1+sR_{f}C}  \label{eqn:output_impedance_rc_filter}\\
    \lim_{s \to 0} R_{out,f} &= R_{f} \nonumber\\
    \lim_{s \to \infty} R_{out,f} &= 0 \,.\nonumber\\
     e_{n,+} &\approx \frac{i_n R_{f}}{1+sR_{f}C}
\end{align}

Looking at the output impedance of the filter, it can be seen that for high frequencies, the output impedance approaches \num{0}, while for low frequencies it is $R_{f}$. This has the effect that if the filter corner frequency $\omega_0 = \frac{1}{RC}$ is close to the flicker noise corner frequency of the reference voltage there is is almost no wideband current noise contribution from $e_{n,+}$ as well. Only the $\frac{1}{f}$ component of the op-amp current noise multiplied with $R_{f}$ is left. Ideally, this is lower than the reference noise to have negligible impact and must be kept in mind when selecting an op-amp.
This leads to the total noise of the op amp
\begin{equation}
    e_{n,op} = \sqrt{e_n^2 + (i_n R_s)^2 + \left|\frac{i_n R_{f}}{1+sRC}\right|^2} \,.
\end{equation}

To conclude, table \ref{tab:current_source_noise_contributers} is given as a reference for the noise contributions in the low-frequency and also the wideband domain. From this table, it can be seen that the only wideband-noise contributors are the reference resistor and the op-amp. The low-frequency contributors are the voltage reference and the op-amp, since they have a strong flicker noise component. A low-noise, precision op-amp typically has far less low frequency noise than a voltage reference and the dominant low frequency contributor remains the voltage reference.
\begin{table}[ht]
    \centering
    \begin{tabular}{lll}
        \toprule
        Noise component& Low frequency& Wideband \\
        \midrule
        $V_{sup}$ & $\approx 0$ & $\approx 0$\\
        MOSFET & $\approx 0$ & $\approx 0$\\
        $V_{ref}$ & $\sqrt{e_{n,ref}^2 + 4 k_B T R_{f}} $ & $\approx 0$\\
        $R_s$ & $\sqrt{4 k_B T R_s}$ & $\sqrt{4 k_B T R_s}$\\
        Op-amp & $\sqrt{e_n^2 + i_n^2 (R_s^2 + R_{f}^2)}$ & $\sqrt{e_n^2 + i_n^2 R_s^2}$\\
        \bottomrule
    \end{tabular}
    \caption{Input referred noise components of the transconductance amplifier. Multiply by $\frac{1}{R_s}$ to get the output referred current noise.}
    \label{tab:current_source_noise_contributers}
\end{table}

From those findings it is clear that the most important choices regarding the noise contributions are a good quality metal-foil or wirewound sense resistor $R_s$, a low noise voltage reference and a low noise op-amp. Regarding the low-noise op-amp it is critical to decide between low voltage noise or low current noise. This choice depends on the value of $R_s$. For typical values of $R_s$ below \qty{1}{\kilo\ohm}, voltage noise is the dominating effect. The reference should be chosen for low flicker noise.

\subsection{Component Selection}%
\label{sec:component_selection}
This section deals with selecting the right components for the precision current source presented in section \ref{sec:precision_current_source}. The focus lies on the requirements defined in section \ref{sec:laser_current_driver}, notably specifications \ref{lst:dgDrive_specs_environment} and \ref{lst:dgDrive_specs_electrical}. Most attention will be on the MOSFET, the operational amplifier, and the voltage reference. For these components examples from literature are given and are compared to the requirements. Discussed first is the voltage reference, because this will define several parameters down the road. Then the op-amp is considered, for which several examples from scientific publications and other alternatives are shown and the best solution is presented. Finally, the selection parameters for the MOSFET will be elaborated. The reader must be warned though that the lineup of p-channel MOSFETs in production is decreasing, with more and more products being discontinued in favour of n-channel MOSFETs. The component recommendations may therefore already be outdated.

Numerous laser driver designs for different applications and laser diodes can be found in literature \cite{libbrecht_hall, laser_driver_mosfet_noise, laser_driver_digital, laser_driver_digital_update, laser_driver_qcl_space, laser_driver_qcl_taubman, laser_driver_qcl_taubman_multiplexer}. Even though \citeauthor{libbrecht_hall} \cite{libbrecht_hall} were not the first to present their circuit, a similar solution can already be found in \cite{laser_driver_old}, their design stands out for its simplicity. The designs mentioned can be divided into two groups. High power drivers for quantum cascade lasers (QCL) typically featuring a compliance voltage $V_c$ of more than \qty{10}{\V} and output currents of up to several ampere based on the work of \citeauthor{laser_driver_qcl_taubman} \cite{laser_driver_qcl_taubman} and medium power devices for laser diodes having a lower compliance voltage of around \qty{2}{\V} and capable of driving a few hundred \unit{\mA} based on the work of \citeauthor{libbrecht_hall} \cite{libbrecht_hall}. The requirements of this work mostly fall into the latter category, except for the compliance voltage, which is targeted for blue laser diodes and $V_c \ge \qty{8}{\V}$. All these designs share one common aspect though, the type of voltage reference. Most laser drivers in literature and also commercial products are designed around low-noise, low-drift buried Zener diode voltage references, namely the Analog Devices (ADI) \device{LM399} \cite{datasheet_LM399} or ADI \device{LTZ1000} \cite{datasheet_LTZ1000}.

The buried types of voltage references are Zener diodes. that are created within the bulk silicon using ion implantation. This reduces noise caused by surface contamination \cite{zener_diode_stability}. These diodes are not true Zener diodes though, but called Zeners nonetheless, and use a mix of Zener and avalanche breakdown to compensate the temperature coefficient.

The Zener effect is the tunneling of electrons through the barrier formed between the valence band and conduction band. It has a negative temperature coefficient, because an increase in temperature reduces the size of the band gap. This effect dominates at low currents.

Avalanche breakdown, on the other, hand describes a mechanism in which free electrons (due to temperature) are accelerated to such energies that they knock out other electrons, causing an avalanche of electrons. This effect has a positive temperature coefficient \cite{tempco_avalanche_breackdown} and occurs at higher currents. While the zero temperature coefficient point is around \qty{5}{\V}, this operating point implies a high susceptibility to changes in the reverse current. Typically, the Zener voltage is shifted slightly upwards to result in a net positive coefficient, which is then compensated by the negative temperature coefficient of a forward biased diode \cite{zener_diode_stability}. This results in the common Zener diode voltage of around $\qty{6.2}{\V} + \qty{0.7}{\V} = \qty{6.9}{\V}$. In comparison to other types of diodes, buried Zeners have the best stability and lowest noise. $V_{ref} \approx \qty{7}{\V}$ is therefore set in stone for the reasons given above. Table \ref{tab:overview_buried_zener_diodes} lists some commercially available buried Zener diodes. All of these diodes are manufactured by ADI as they are the sole manufacturer left on the market to produce this kind of diodes.
\begin{table}[ht]
    \centering
    \begin{tabular}{lllll}
        \toprule
        Component& Voltage& Temperature coefficient & Stability& Package\\
        \midrule
        \device{LT1021} & \qty{7}{\V} & \qtyrange[range-units = single]{2}{5}{\uV \per \V \per \K} & \qty{15}{\uV \per \V \per \kilo\hour\tothe{0.5}} & SO-8\\
        \device{LT1027} & \qty{5}{\V} & \qtyrange[range-units = single]{1}{2}{\uV \per \V \per \K} & not specified & SO-8\\
        \device{LM399} & \qty{7}{\V} & \qtyrange[range-units = single]{0.3}{1}{\uV \per \V \per \K} & \qty{8}{\uV \per \V \per \kilo\hour\tothe{0.5}} & TO-46\\
        \device{ADR1399} & \qty{7}{\V} & \qtyrange[range-units = single]{0.2}{1}{\uV \per \V \per \K} & \qty{7}{\uV \per \V \per \kilo\hour\tothe{0.5}} & TO-46\\
        \device{LTZ1000} & \qty{7.2}{\V} & \qty{0.05}{\uV \per \V \per \K} & \qty{0.3}{\uV \per \V \per \kilo\hour\tothe{0.5}} & TO-99\\
        \device{ADR1000} & \qty{6.6}{\V} & \qty{<0.2}{\uV \per \V \per \K} & \qty{0.2}{\uV \per \V \per \kilo\hour\tothe{0.5}} & TO-99\\
        \bottomrule
    \end{tabular}
    \caption{List of commercially available buried Zener diodes and selected properties.}
    \label{tab:overview_buried_zener_diodes}
\end{table}

Choosing a voltage reference can be done according to specification \ref{lst:dgDrive_specs_environment}. A temperature coefficient of \qty{<= 1}{\uA \per \A \per \K} rules out any non-hermetic unheated voltage reference. Using a hermetic package also improves the stability against humidity as the epoxy used for an SO-8 package is hydrophilic and swells when exposed to water vapour causing pressure on the die, resulting in a change of the output voltage. The hermetic voltage references can be divided into two groups, the \device{LM399} and the newer \device{ADR1399} in one group and the \device{LTZ1000} and its newer counterpart \device{ADR1000} in another. While the \device{LM399} requires very few external components, the external circuit for the \device{LTZ1000} is far more elaborate requiring more parts and space. Additionally, the \device{LTZ1000} is more than four times the price of the \device{LM399} in quantities of \num{10} at the time of writing. Last but not least, the stability and temperature coefficient of the \device{LTZ1000} cannot be matched by the performance of the sense resistor in the precision current source, so the sense resistor gives a lower bound of about \qty{0.5}{\uA \per \A \per \K}. Unless the better low frequency noise performance is required, the \device{LM399} and \device{ADR1399} are the more economical parts. The performance of the latter two references will be discussed in section \ref{sec:zener_diode_selection} and the sense resistor is considered next.

With the maximum reference voltage of \qty{7}{\V} known, a sense resistor between \qty{14}{\ohm} (\qty{500}{\mA}) and \qty{28}{\ohm} (\qty{250}{\mA}) is required. The Vishay \device{VPR221Z} are high power low drift metal foil resistors in a TO-220 package and a solid choice here. The low value resistors, combined with the requirement for a low current noise output of the source, limits the choice of op-amps to bipolar low-noise devices or discrete implementations. Table \ref{tab:overview_bipolar_op-amps} lists some choices compiled from the literature sources, which will now be discussed.
\begin{table}[ht]
    \centering
    \begin{tabular}{llll}
        \toprule
        Component& Wideband-noise& Low frequency noise & Temperature coefficient \\
        \midrule
        \device{LT1028} & \qty[power-half-as-sqrt]{0.85}{\nV \per \Hz\tothe{0.5}} & \qty{35}{\nV_{p-p}} & \qty{0.2}{\uV \per \K}\\
        \device{AD797} & \qty[power-half-as-sqrt]{0.9}{\nV \per \Hz\tothe{0.5}} & \qty{50}{\nV_{p-p}} & \qty{0.2}{\uV \per \K}\\
        \device{ADA4898} & \qty[power-half-as-sqrt]{0.9}{\nV \per \Hz\tothe{0.5}} & not specified & \qty{1}{\uV \per \K}\\
        \device{ADA4004} & \qty[power-half-as-sqrt]{1.8}{\nV \per \Hz\tothe{0.5}} & \qty{150}{\nV_{p-p}} & \qty{0.7}{\uV \per \K}\\
        \device{AD8671} & \qty[power-half-as-sqrt]{2.8}{\nV \per \Hz\tothe{0.5}} & \qty{77}{\nV_{p-p}} & \qty{0.3}{\uV \per \K}\\
        \bottomrule
    \end{tabular}
    \caption{List of low-noise precision bipolar operational amplifiers with typical performance properties.}
    \label{tab:overview_bipolar_op-amps}
\end{table}

The low value of the sense resistor makes a bipolar op-amp the preferred choice, because they have a very low voltage noise and their current noise and input bias current do not interfere with such a low value resistor. While a discrete solution using matched JFETs or bipolar transistors may push the input noise even lower, the temperature stability, circuit complexity and again the size speaks against this option, so the discussion will be limited to integrated solutions only.

To find a reference point for the choice of op-amp, the thermal noise of the sense resistor must be looked at. A \qty{28}{\ohm} sense resistor has a thermal noise of
\begin{equation*}
    e_n\left(R = \qty{28}{\ohm} T = \qty{23}{\celsius}\right) = \qty[power-half-as-sqrt]{0.67}{\nV \per \Hz\tothe{0.5}} \,.
\end{equation*}

This means that even the lowest noise op-amp from table \ref{tab:overview_bipolar_op-amps} dominates the wideband-noise. With this said, the component choices made in literate can be discussed. The \device{AD8671} chosen by \citeauthor{laser_driver_digital} \cite{laser_driver_digital} only makes sense, because they have chosen a very large filter resistor $R_{f}$ of $2 \times \qty{3}{\kilo\ohm}$. The \device{ADA4004} was used by Moglabs in the \device{DLC-102}, again likely due to the high values of $R_{f}$ used. The \device{ADA4898} might seem like a good choice at first sight but the very limited (in terms of precision op-amps) open-loop gain of \qty{0.14}{\V \per \uV} makes this op-amp an inexpensive, but poor choice. The final choice is between the \device{AD797} and the \device{LT1028}, both op-amps have very similar specifications, but there is a peculiarity in the datasheet of the \device{LT1028} \cite{datasheet_LT1028}. The \textit{High Frequency Voltage Noise
vs Frequency} plot shows a noise bump at around \qty{400}{\kHz}. The original application of the \device{LT1x28} is for audio frequencies, which is well below that bump, but in this case poses a problem. The publication by \citeauthor{libbrecht_hall} \cite{libbrecht_hall} also blames the \device{LT1028} for the noise peak around \qty{400}{\kHz}. Moreover, this peak is the reason why \citeauthor{laser_driver_mosfet_noise} \cite{laser_driver_mosfet_noise} found the \device{LT1028} to have a higher integrated noise than the \device{AD797}. Additionally to the superior noise performance, the \device{AD797} (B-grade) has excellent specifications overall. The open-loop gain is between \qtyrange[range-units = single]{2}{20}{\V \per \uV}, the supply rejection is greater than \qty{1}{\uV \per \V}, the bias current is almost constant between \qtyrange[range-units = single]{20}{100}{\celsius} and the unity gain bandwidth is around \qty{10}{\MHz}. Finally it does have a high output drive capability of \qty{50}{\mA}, which allows to drive large MOSFETs. These features make the \device{AD797} the ideal op-amp for low-value sense resistors, although it puts limits on the maximum filter resistor to limit the low frequency current noise contribution.

Finally, the choice of MOSFETs can be discussed. As it was shown in section \ref{sec:mosfet_current_source} in equation \ref{eqn:mosfet_saturation}, the channel length modulation plays an important role in increasing the channel conductance $g_{DS}$ and limiting the output impedance. To reduce the channel length modulation a longer channel is preferred. Manufacturers do not give these numbers, nor information on the manufacturing process. Older technologies like the planar (lateral) FET are better suited for operating in the saturation region than the modern trench (vertical) FET. Trench MOSFETs are geared towards a low on-state resistance $R_{DS,on}$, which is important for switching applications, but their lower resistance comes from a shorter channel. One of the few planar MOSFETs still available on the market is the HEXFET, which was designed for switching applications, but proves useful nonetheless. High voltage MOSFETs also have longer channels than low voltage MOSFETs, so searching for MOSFETs that are rated for \qtyrange[range-units = single]{60}{100}{\V} or more can narrow down the candidates. While the output impedance is a factor worth keeping in mind, the most important aspect is whether the MOSFET can drive the load regarding the compliance voltage. To outline the problem, one can refer again to the example parameters from table \ref{tab:current_source_parameters}.

Assuming a supply voltage of \qty{15}{\V} and the \device{AD797} op-amp, the current source supply voltage $V_{sup}$ is limited to about \qtyrange[range-units = single]{11}{12}{\V}, because the \device{AD797} is no rail-to-rail op-amp and its output only swings within \qty{3}{\V} of the rail (minimum) and the input is limited to within \qty{2}{\V} of the rail (minimum). Considering the maximum $V_{ref}$ at full output of \qty{7}{\V} and a load voltage of \qty{3}{\V} in case of the \device{L785H1} \cite{datasheet_thorlabs_780nm} laser diode used as an example in this section leaves only
\begin{equation}
    V_{DS,min} = V_{sup} - V_{ref} - V_{load} = \qty{12}{\V} - \qty{7}{\V} - \qty{3}{\V} = \qty{2}{\V} \label{eqn:minimum_mosfet_vds}
\end{equation}
for the MOSFET -- a serious challenge.

To find a suitable MOSFET, one has to consult the \textit{Typical Output Characteristics} graph in the datasheet. By using the maximum output current specification it is possible to estimate the minimum drain-source voltage $V_{DS}$ to keep the MOSFET in saturation at the given maximum output current. This again narrows down the list of candidates.

The final aspect is the capacitive nature of the MOSFET gate. This property was briefly brushed in appendix \ref{sec:mosfet_noise} and the parasitic capacitances can be found in figure \ref{fig:mosfet_parasitic_capacitors}. The \device{AD797} can drive fairly large capacitive loads and several hundred \unit{\pF} are possible. It is best to keep the input capacitance $C_{iss}$ below \qty{500}{\pF}. The output impedance of the \device{AD797}, is about \qty{10}{\ohm} at \qty{1}{\MHz} and rising by an order of magnitude at \qty{10}{\MHz}. \qty{500}{\pF} results in an impedance of around \qty{300}{\ohm} dropping by an order of magnitude at \qty{10}{\MHz}, so keeping the capacitance low, allows for a higher bandwidth of the current source.

Using these guidelines, searching a MOSFET across a lot of manufactures can still be tedious, but the distributor Digikey, for example, allows filtering and sorting by voltage and input capacitance. The following MOSFETs in table are given as an example and can be chosen for their respective current ranges.
\begin{table}[ht]
    \centering
    \begin{tabular}{llll}
        \toprule
        MOSFET& Maximum $V_{DS}$& Input capacitance $C_{iss}$ & Current range \\
        \midrule
        \device{IRF9610} & \qty{200}{\V}& \qty{170}{\pF} & \qtyrange[range-units = single]{100}{250}{\mA}\\
        \device{IRF9Z10} & \qty{50}{\V}& \qty{270}{\pF} & \qtyrange[range-units = single]{250}{500}{\mA}\\
        \device{IRF9Z14} & \qty{60}{\V}& \qty{270}{\pF} & \qtyrange[range-units = single]{250}{500}{\mA}\\
        \bottomrule
    \end{tabular}
    \caption{Example MOSFETs for a current source and recommended current ranges.}
    \label{tab:example_mosfet_selection}
\end{table}

The current range of the MOSFETs in table \ref{tab:example_mosfet_selection} is given based on the datasheet, making sure that the MOSFET can be biased into saturation for the estimated minimum $V_{DS}$ according to \ref{eqn:minimum_mosfet_vds}. The \device{IRF9Z10} is a lower voltage version of the \device{IRF9Z14} and the \device{IRF9Z14} should be preferred if available. Those MOSFETs starting with \textit{IRF} are all HEXFETs formerly made by International Rectifier, whose MOSFET business was bought by Vishay in 2007.

To summarize the component selection. The ADI \device{AD797} op-amp is a superior choice among the op-amps compared and the recommended choice for being low noise with enormous gain, high bandwidth and a strong drive current. A MOSFET based on an older process, to ensure stable performance in saturation, like the Vishay \device{IRF9Z14} is a good choice for medium power applications. The reference of choice is a a buried Zener diode, because of their low noise and high stability. The ADI \device{LM399} or \device{ADR1399} is the most economical choice. Regarding the sense resistor, it must be able to dissipate up to $\qty{500}{\mA} \cdot \qty{7}{\V} = \qty{3.5}{\W}$ with minimal drift, making the Vishay \device{VPR221Z} a very good choice.

\subsection{Current Source Example Parameters}%
\label{sec:current_source_summary}
Throughout this section, example calculations are performed to give the reader an idea of real-life parameters that can be applied with the theoretical models. These parameters are summarised in table \ref{tab:current_source_parameters}, including their origin.
\begin{table}[ht]
    \centering
    \begin{tabular}{lll}
        \toprule
        Parameter& Value& Source \\
        \midrule
        MOSFET drain current $I_D$ & \qty{250}{\mA} & \device{L785H1} \cite{datasheet_thorlabs_780nm}\\
        MOSFET $\kappa$ & \qty[per-mode=power]{0.813}{\ampere \per \square\volt} & \device{IRF9610} SPICE model \cite{irf9610_spice}\\
        MOSFET channel length modulation $\lambda$ & \qty[per-mode=power]{4}{\per \milli \volt} & \device{IRF9610} SPICE model \cite{irf9610_spice}\\
        MOSFET source voltage & \qtyrange{3.5}{4}{\V} & section \ref{sec:component_selection}\\
        Source/Sense Resistor $R_s$ & \qty{14}{\ohm} or \qty{28}{\ohm} & section \ref{sec:component_selection}\\
        Op-amp differential input impedance $R_{id}$ & \qty{7.5}{\kilo\ohm} & \device{AD797} \cite{datasheet_AD797}\\
        Op-amp open-loop gain $A_{ol}$ & \qty[per-mode=power]{2}{\volt \per \uV} & \device{AD797} \cite{datasheet_AD797}\\
        Op-amp gain bandwidth product $GBP$ & \qty{10}{\MHz} & \device{AD797} \cite{datasheet_AD797}\\
        \bottomrule
    \end{tabular}
    \caption{Parameters used throughout this section and their sources.}
    \label{tab:current_source_parameters}
\end{table}

\subsection{Howland Current Pump}%
\label{sec:howland_current_source}
This section will discuss two versions of the popular Howland current source (HCS) \cite{howland_current_source}. Both the traditional Howland current pump and the so-called \textit{improved} Howland current pump which is similar, but changes some of its properties for good and bad. Both are bidirectional current sources and can be used for high frequency current modulation or the generation of a precision floating current source. The Howland current source is most useful for small currents of up to a few \unit{mA}, because its output impedance is limited and mostly depends on the resistors surrounding the op-amp as will be shown below. The design discussed here aims at an output current gain of \qty{1}{\mA \per \V}, which has proven useful for diode lasers over time. The full circuit is shown in figure \ref{fig:howland_current_source}. It is shown as the improved Howland current source but can be configured as the classic Howland current source by shorting out $R_{2a}$.
\begin{figure}[ht]
    \centering
    %\scalebox{0.5} % scalebox
    \caption{The Howland current source. Using $R_{2a} = \qty{0}{\ohm}$ is the classic version, while $R_{2a} \neq \qty{0}{\ohm}$ is the \textit{improved} version. $C_c$ is a compensation capacitor to enhance stability.}
    \label{fig:howland_current_source}
\end{figure}

As can be seen, these two current sources designs require a set of either \num{4} or \num{5} resistors depending on the desired configuration as the classic Howland current source does not need $R_{2a}$. The output impedance of the Howland current source is derived in appendix \ref{sec:appendix_howland_current_source} and the result is equation \ref{eqn:appendix_howland_output_impedance_resistors_gain} repeated here,
\begin{equation}
    R_{o,m,A} = \left(\frac{R^2 - R_{2a}^2}{R}\right) \frac{\left(A R + R - \epsilon + 1\right)}{A \left(R + \epsilon - 1\right) + 2 R - 2 \epsilon + 2}\,,\label{eqn:howland_output_impedance_resistors_gain}
\end{equation}
where $A$ is the frequency dependent gain of the op-amp as discussed in equation \ref{eqn:op-amp_gain}, $R = R_1 = R_2 = R_3 = R_4$ and $\epsilon$ is the resistor mismatch factor introduced in \ref{eqn:howland_mismatching_factor}. The mismatch factor can be applied for worst-case calculations for a given resistor tolerance $T$ using equation \ref{eqn:howland_current_source_worst_case_mismatch}.

By using the formula above, it is possible to calculate the output impedance for a modulation current source of a laser driver. Since the the output impedance of the \textit{improved} Howland current source is worse than that of the classic Howland current source according to equation \ref{eqn:howland_output_impedance_resistors_gain}, the discussion will begin here. Experience has shown that an input sensitivity of \qty{1}{\mA \per \V} is a reasonable choice, giving enough headroom to steer the laser and apply a modulation as mentioned above. The resistor value can be calculated from this input sensitivity to be \qty{1}{\kilo\ohm}. To reduce manual labor it is recommended to use an array of \num{4} matched resistors like the Vishay \device{MORN} \cite{datasheet_MORN} or the ADI \device{LT5400} \cite{datasheet_LT5400}, both quad resistor arrays offering a ratio matching between \qty{0.5}{\percent} and \qty{0.01}{\percent}. Commonly available is the \device{MORN} array with \qty{0.05}{\percent} tolerance and the \device{LT5400} with \qty{0.01}{\percent} tolerance and even \qty{0.005}{\percent} between neighbouring resistor pairs. For such arrays and an amplifier with $A=\num{e7}$, the worst case output impedance can be calculated as
\begin{align*}
    R_{out}(R=\qty{1}{\kilo\ohm}, T=\qty{0.05}{\percent}) &\approx \qty{499}{\kilo\ohm} & R_{out}(R=\qty{1}{\kilo\ohm}, T=\qty{0.01}{\percent}) &\approx \qty{2.5}{\mega\ohm} \,.
\end{align*}

$R_{out}$ is proportional to $\frac{1}{T}$, since the amplifier gain is very high and the tolerances are small. The upper limit imposed by the op-amp gain can be calculated according to equation \ref{eqn:howland_output_impedance_loop_gain} as
\begin{equation*}
    R_{out, max} (A=\qty{10}{\V \per \uV}) \approx \qty{2.5}{\giga\ohm}\,.
\end{equation*}

Comparing these values with the requirements from specification \ref{lst:dgDrive_specs_electrical}, $R_{out} \qty{\geq 7.5}{\mega \ohm}$, it is obvious that the Howland current source can meet them, but not without extra trimming or a tighter matched resistor array (T \qty{<= 0.01}{\percent}) which is hard to come by and expensive. In order to get a better understanding of the output impedance and to show the effect of different tolerances, a Monte Carlo simulation was conducted. The parameters assumed a Gaussian distribution of the resistor values with $3 \sigma = T$ due to the tolerance. The Monte Carlo simulation was then run \num{e5} times and the output impedance was extracted for each run. Finally, the absolute value (there are positive and negative values, see below) was binned into \num{100} bins of \qty{500}{\kilo\ohm} width and plotted as a histogram.

The LTSpice simulation file can be found at \external{source/spice/howland\_current\_source\_ideal.asc} and the result, limited to bin values $\qty{\leq 50}{\mega\ohm}$ for better readability, is shown in figure \ref{fig:ltpsice_howland_mc_output_impedance}.
\begin{figure}[ht]
    \centering
    %% Creator: Matplotlib, PGF backend
%%
%% To include the figure in your LaTeX document, write
%%   \input{<filename>.pgf}
%%
%% Make sure the required packages are loaded in your preamble
%%   \usepackage{pgf}
%%
%% Also ensure that all the required font packages are loaded; for instance,
%% the lmodern package is sometimes necessary when using math font.
%%   \usepackage{lmodern}
%%
%% Figures using additional raster images can only be included by \input if
%% they are in the same directory as the main LaTeX file. For loading figures
%% from other directories you can use the `import` package
%%   \usepackage{import}
%%
%% and then include the figures with
%%   \import{<path to file>}{<filename>.pgf}
%%
%% Matplotlib used the following preamble
%%   \usepackage{siunitx}
%%   \usepackage{fontspec}
%%
\begingroup%
\makeatletter%
\begin{pgfpicture}%
\pgfpathrectangle{\pgfpointorigin}{\pgfqpoint{5.431103in}{3.356606in}}%
\pgfusepath{use as bounding box, clip}%
\begin{pgfscope}%
\pgfsetbuttcap%
\pgfsetmiterjoin%
\definecolor{currentfill}{rgb}{1.000000,1.000000,1.000000}%
\pgfsetfillcolor{currentfill}%
\pgfsetlinewidth{0.000000pt}%
\definecolor{currentstroke}{rgb}{1.000000,1.000000,1.000000}%
\pgfsetstrokecolor{currentstroke}%
\pgfsetdash{}{0pt}%
\pgfpathmoveto{\pgfqpoint{0.000000in}{0.000000in}}%
\pgfpathlineto{\pgfqpoint{5.431103in}{0.000000in}}%
\pgfpathlineto{\pgfqpoint{5.431103in}{3.356606in}}%
\pgfpathlineto{\pgfqpoint{0.000000in}{3.356606in}}%
\pgfpathlineto{\pgfqpoint{0.000000in}{0.000000in}}%
\pgfpathclose%
\pgfusepath{fill}%
\end{pgfscope}%
\begin{pgfscope}%
\pgfsetbuttcap%
\pgfsetmiterjoin%
\definecolor{currentfill}{rgb}{1.000000,1.000000,1.000000}%
\pgfsetfillcolor{currentfill}%
\pgfsetlinewidth{0.000000pt}%
\definecolor{currentstroke}{rgb}{0.000000,0.000000,0.000000}%
\pgfsetstrokecolor{currentstroke}%
\pgfsetstrokeopacity{0.000000}%
\pgfsetdash{}{0pt}%
\pgfpathmoveto{\pgfqpoint{0.651412in}{0.524170in}}%
\pgfpathlineto{\pgfqpoint{5.281103in}{0.524170in}}%
\pgfpathlineto{\pgfqpoint{5.281103in}{3.082363in}}%
\pgfpathlineto{\pgfqpoint{0.651412in}{3.082363in}}%
\pgfpathlineto{\pgfqpoint{0.651412in}{0.524170in}}%
\pgfpathclose%
\pgfusepath{fill}%
\end{pgfscope}%
\begin{pgfscope}%
\pgfpathrectangle{\pgfqpoint{0.651412in}{0.524170in}}{\pgfqpoint{4.629690in}{2.558193in}}%
\pgfusepath{clip}%
\pgfsetbuttcap%
\pgfsetmiterjoin%
\definecolor{currentfill}{rgb}{0.003922,0.450980,0.698039}%
\pgfsetfillcolor{currentfill}%
\pgfsetfillopacity{0.700000}%
\pgfsetlinewidth{0.000000pt}%
\definecolor{currentstroke}{rgb}{0.000000,0.000000,0.000000}%
\pgfsetstrokecolor{currentstroke}%
\pgfsetstrokeopacity{0.700000}%
\pgfsetdash{}{0pt}%
\pgfpathmoveto{\pgfqpoint{0.861853in}{0.524170in}}%
\pgfpathlineto{\pgfqpoint{0.903941in}{0.524170in}}%
\pgfpathlineto{\pgfqpoint{0.903941in}{0.524170in}}%
\pgfpathlineto{\pgfqpoint{0.861853in}{0.524170in}}%
\pgfpathlineto{\pgfqpoint{0.861853in}{0.524170in}}%
\pgfpathclose%
\pgfusepath{fill}%
\end{pgfscope}%
\begin{pgfscope}%
\pgfpathrectangle{\pgfqpoint{0.651412in}{0.524170in}}{\pgfqpoint{4.629690in}{2.558193in}}%
\pgfusepath{clip}%
\pgfsetbuttcap%
\pgfsetmiterjoin%
\definecolor{currentfill}{rgb}{0.003922,0.450980,0.698039}%
\pgfsetfillcolor{currentfill}%
\pgfsetfillopacity{0.700000}%
\pgfsetlinewidth{0.000000pt}%
\definecolor{currentstroke}{rgb}{0.000000,0.000000,0.000000}%
\pgfsetstrokecolor{currentstroke}%
\pgfsetstrokeopacity{0.700000}%
\pgfsetdash{}{0pt}%
\pgfpathmoveto{\pgfqpoint{0.903941in}{0.524170in}}%
\pgfpathlineto{\pgfqpoint{0.946029in}{0.524170in}}%
\pgfpathlineto{\pgfqpoint{0.946029in}{0.584019in}}%
\pgfpathlineto{\pgfqpoint{0.903941in}{0.584019in}}%
\pgfpathlineto{\pgfqpoint{0.903941in}{0.524170in}}%
\pgfpathclose%
\pgfusepath{fill}%
\end{pgfscope}%
\begin{pgfscope}%
\pgfpathrectangle{\pgfqpoint{0.651412in}{0.524170in}}{\pgfqpoint{4.629690in}{2.558193in}}%
\pgfusepath{clip}%
\pgfsetbuttcap%
\pgfsetmiterjoin%
\definecolor{currentfill}{rgb}{0.003922,0.450980,0.698039}%
\pgfsetfillcolor{currentfill}%
\pgfsetfillopacity{0.700000}%
\pgfsetlinewidth{0.000000pt}%
\definecolor{currentstroke}{rgb}{0.000000,0.000000,0.000000}%
\pgfsetstrokecolor{currentstroke}%
\pgfsetstrokeopacity{0.700000}%
\pgfsetdash{}{0pt}%
\pgfpathmoveto{\pgfqpoint{0.946029in}{0.524170in}}%
\pgfpathlineto{\pgfqpoint{0.988117in}{0.524170in}}%
\pgfpathlineto{\pgfqpoint{0.988117in}{1.559068in}}%
\pgfpathlineto{\pgfqpoint{0.946029in}{1.559068in}}%
\pgfpathlineto{\pgfqpoint{0.946029in}{0.524170in}}%
\pgfpathclose%
\pgfusepath{fill}%
\end{pgfscope}%
\begin{pgfscope}%
\pgfpathrectangle{\pgfqpoint{0.651412in}{0.524170in}}{\pgfqpoint{4.629690in}{2.558193in}}%
\pgfusepath{clip}%
\pgfsetbuttcap%
\pgfsetmiterjoin%
\definecolor{currentfill}{rgb}{0.003922,0.450980,0.698039}%
\pgfsetfillcolor{currentfill}%
\pgfsetfillopacity{0.700000}%
\pgfsetlinewidth{0.000000pt}%
\definecolor{currentstroke}{rgb}{0.000000,0.000000,0.000000}%
\pgfsetstrokecolor{currentstroke}%
\pgfsetstrokeopacity{0.700000}%
\pgfsetdash{}{0pt}%
\pgfpathmoveto{\pgfqpoint{0.988117in}{0.524170in}}%
\pgfpathlineto{\pgfqpoint{1.030205in}{0.524170in}}%
\pgfpathlineto{\pgfqpoint{1.030205in}{2.681246in}}%
\pgfpathlineto{\pgfqpoint{0.988117in}{2.681246in}}%
\pgfpathlineto{\pgfqpoint{0.988117in}{0.524170in}}%
\pgfpathclose%
\pgfusepath{fill}%
\end{pgfscope}%
\begin{pgfscope}%
\pgfpathrectangle{\pgfqpoint{0.651412in}{0.524170in}}{\pgfqpoint{4.629690in}{2.558193in}}%
\pgfusepath{clip}%
\pgfsetbuttcap%
\pgfsetmiterjoin%
\definecolor{currentfill}{rgb}{0.003922,0.450980,0.698039}%
\pgfsetfillcolor{currentfill}%
\pgfsetfillopacity{0.700000}%
\pgfsetlinewidth{0.000000pt}%
\definecolor{currentstroke}{rgb}{0.000000,0.000000,0.000000}%
\pgfsetstrokecolor{currentstroke}%
\pgfsetstrokeopacity{0.700000}%
\pgfsetdash{}{0pt}%
\pgfpathmoveto{\pgfqpoint{1.030205in}{0.524170in}}%
\pgfpathlineto{\pgfqpoint{1.072293in}{0.524170in}}%
\pgfpathlineto{\pgfqpoint{1.072293in}{2.960544in}}%
\pgfpathlineto{\pgfqpoint{1.030205in}{2.960544in}}%
\pgfpathlineto{\pgfqpoint{1.030205in}{0.524170in}}%
\pgfpathclose%
\pgfusepath{fill}%
\end{pgfscope}%
\begin{pgfscope}%
\pgfpathrectangle{\pgfqpoint{0.651412in}{0.524170in}}{\pgfqpoint{4.629690in}{2.558193in}}%
\pgfusepath{clip}%
\pgfsetbuttcap%
\pgfsetmiterjoin%
\definecolor{currentfill}{rgb}{0.003922,0.450980,0.698039}%
\pgfsetfillcolor{currentfill}%
\pgfsetfillopacity{0.700000}%
\pgfsetlinewidth{0.000000pt}%
\definecolor{currentstroke}{rgb}{0.000000,0.000000,0.000000}%
\pgfsetstrokecolor{currentstroke}%
\pgfsetstrokeopacity{0.700000}%
\pgfsetdash{}{0pt}%
\pgfpathmoveto{\pgfqpoint{1.072293in}{0.524170in}}%
\pgfpathlineto{\pgfqpoint{1.114381in}{0.524170in}}%
\pgfpathlineto{\pgfqpoint{1.114381in}{2.733615in}}%
\pgfpathlineto{\pgfqpoint{1.072293in}{2.733615in}}%
\pgfpathlineto{\pgfqpoint{1.072293in}{0.524170in}}%
\pgfpathclose%
\pgfusepath{fill}%
\end{pgfscope}%
\begin{pgfscope}%
\pgfpathrectangle{\pgfqpoint{0.651412in}{0.524170in}}{\pgfqpoint{4.629690in}{2.558193in}}%
\pgfusepath{clip}%
\pgfsetbuttcap%
\pgfsetmiterjoin%
\definecolor{currentfill}{rgb}{0.003922,0.450980,0.698039}%
\pgfsetfillcolor{currentfill}%
\pgfsetfillopacity{0.700000}%
\pgfsetlinewidth{0.000000pt}%
\definecolor{currentstroke}{rgb}{0.000000,0.000000,0.000000}%
\pgfsetstrokecolor{currentstroke}%
\pgfsetstrokeopacity{0.700000}%
\pgfsetdash{}{0pt}%
\pgfpathmoveto{\pgfqpoint{1.114381in}{0.524170in}}%
\pgfpathlineto{\pgfqpoint{1.156469in}{0.524170in}}%
\pgfpathlineto{\pgfqpoint{1.156469in}{2.401949in}}%
\pgfpathlineto{\pgfqpoint{1.114381in}{2.401949in}}%
\pgfpathlineto{\pgfqpoint{1.114381in}{0.524170in}}%
\pgfpathclose%
\pgfusepath{fill}%
\end{pgfscope}%
\begin{pgfscope}%
\pgfpathrectangle{\pgfqpoint{0.651412in}{0.524170in}}{\pgfqpoint{4.629690in}{2.558193in}}%
\pgfusepath{clip}%
\pgfsetbuttcap%
\pgfsetmiterjoin%
\definecolor{currentfill}{rgb}{0.003922,0.450980,0.698039}%
\pgfsetfillcolor{currentfill}%
\pgfsetfillopacity{0.700000}%
\pgfsetlinewidth{0.000000pt}%
\definecolor{currentstroke}{rgb}{0.000000,0.000000,0.000000}%
\pgfsetstrokecolor{currentstroke}%
\pgfsetstrokeopacity{0.700000}%
\pgfsetdash{}{0pt}%
\pgfpathmoveto{\pgfqpoint{1.156469in}{0.524170in}}%
\pgfpathlineto{\pgfqpoint{1.198557in}{0.524170in}}%
\pgfpathlineto{\pgfqpoint{1.198557in}{2.010433in}}%
\pgfpathlineto{\pgfqpoint{1.156469in}{2.010433in}}%
\pgfpathlineto{\pgfqpoint{1.156469in}{0.524170in}}%
\pgfpathclose%
\pgfusepath{fill}%
\end{pgfscope}%
\begin{pgfscope}%
\pgfpathrectangle{\pgfqpoint{0.651412in}{0.524170in}}{\pgfqpoint{4.629690in}{2.558193in}}%
\pgfusepath{clip}%
\pgfsetbuttcap%
\pgfsetmiterjoin%
\definecolor{currentfill}{rgb}{0.003922,0.450980,0.698039}%
\pgfsetfillcolor{currentfill}%
\pgfsetfillopacity{0.700000}%
\pgfsetlinewidth{0.000000pt}%
\definecolor{currentstroke}{rgb}{0.000000,0.000000,0.000000}%
\pgfsetstrokecolor{currentstroke}%
\pgfsetstrokeopacity{0.700000}%
\pgfsetdash{}{0pt}%
\pgfpathmoveto{\pgfqpoint{1.198557in}{0.524170in}}%
\pgfpathlineto{\pgfqpoint{1.240645in}{0.524170in}}%
\pgfpathlineto{\pgfqpoint{1.240645in}{1.808441in}}%
\pgfpathlineto{\pgfqpoint{1.198557in}{1.808441in}}%
\pgfpathlineto{\pgfqpoint{1.198557in}{0.524170in}}%
\pgfpathclose%
\pgfusepath{fill}%
\end{pgfscope}%
\begin{pgfscope}%
\pgfpathrectangle{\pgfqpoint{0.651412in}{0.524170in}}{\pgfqpoint{4.629690in}{2.558193in}}%
\pgfusepath{clip}%
\pgfsetbuttcap%
\pgfsetmiterjoin%
\definecolor{currentfill}{rgb}{0.003922,0.450980,0.698039}%
\pgfsetfillcolor{currentfill}%
\pgfsetfillopacity{0.700000}%
\pgfsetlinewidth{0.000000pt}%
\definecolor{currentstroke}{rgb}{0.000000,0.000000,0.000000}%
\pgfsetstrokecolor{currentstroke}%
\pgfsetstrokeopacity{0.700000}%
\pgfsetdash{}{0pt}%
\pgfpathmoveto{\pgfqpoint{1.240645in}{0.524170in}}%
\pgfpathlineto{\pgfqpoint{1.282734in}{0.524170in}}%
\pgfpathlineto{\pgfqpoint{1.282734in}{1.593980in}}%
\pgfpathlineto{\pgfqpoint{1.240645in}{1.593980in}}%
\pgfpathlineto{\pgfqpoint{1.240645in}{0.524170in}}%
\pgfpathclose%
\pgfusepath{fill}%
\end{pgfscope}%
\begin{pgfscope}%
\pgfpathrectangle{\pgfqpoint{0.651412in}{0.524170in}}{\pgfqpoint{4.629690in}{2.558193in}}%
\pgfusepath{clip}%
\pgfsetbuttcap%
\pgfsetmiterjoin%
\definecolor{currentfill}{rgb}{0.003922,0.450980,0.698039}%
\pgfsetfillcolor{currentfill}%
\pgfsetfillopacity{0.700000}%
\pgfsetlinewidth{0.000000pt}%
\definecolor{currentstroke}{rgb}{0.000000,0.000000,0.000000}%
\pgfsetstrokecolor{currentstroke}%
\pgfsetstrokeopacity{0.700000}%
\pgfsetdash{}{0pt}%
\pgfpathmoveto{\pgfqpoint{1.282734in}{0.524170in}}%
\pgfpathlineto{\pgfqpoint{1.324822in}{0.524170in}}%
\pgfpathlineto{\pgfqpoint{1.324822in}{1.399469in}}%
\pgfpathlineto{\pgfqpoint{1.282734in}{1.399469in}}%
\pgfpathlineto{\pgfqpoint{1.282734in}{0.524170in}}%
\pgfpathclose%
\pgfusepath{fill}%
\end{pgfscope}%
\begin{pgfscope}%
\pgfpathrectangle{\pgfqpoint{0.651412in}{0.524170in}}{\pgfqpoint{4.629690in}{2.558193in}}%
\pgfusepath{clip}%
\pgfsetbuttcap%
\pgfsetmiterjoin%
\definecolor{currentfill}{rgb}{0.003922,0.450980,0.698039}%
\pgfsetfillcolor{currentfill}%
\pgfsetfillopacity{0.700000}%
\pgfsetlinewidth{0.000000pt}%
\definecolor{currentstroke}{rgb}{0.000000,0.000000,0.000000}%
\pgfsetstrokecolor{currentstroke}%
\pgfsetstrokeopacity{0.700000}%
\pgfsetdash{}{0pt}%
\pgfpathmoveto{\pgfqpoint{1.324822in}{0.524170in}}%
\pgfpathlineto{\pgfqpoint{1.366910in}{0.524170in}}%
\pgfpathlineto{\pgfqpoint{1.366910in}{1.307201in}}%
\pgfpathlineto{\pgfqpoint{1.324822in}{1.307201in}}%
\pgfpathlineto{\pgfqpoint{1.324822in}{0.524170in}}%
\pgfpathclose%
\pgfusepath{fill}%
\end{pgfscope}%
\begin{pgfscope}%
\pgfpathrectangle{\pgfqpoint{0.651412in}{0.524170in}}{\pgfqpoint{4.629690in}{2.558193in}}%
\pgfusepath{clip}%
\pgfsetbuttcap%
\pgfsetmiterjoin%
\definecolor{currentfill}{rgb}{0.003922,0.450980,0.698039}%
\pgfsetfillcolor{currentfill}%
\pgfsetfillopacity{0.700000}%
\pgfsetlinewidth{0.000000pt}%
\definecolor{currentstroke}{rgb}{0.000000,0.000000,0.000000}%
\pgfsetstrokecolor{currentstroke}%
\pgfsetstrokeopacity{0.700000}%
\pgfsetdash{}{0pt}%
\pgfpathmoveto{\pgfqpoint{1.366910in}{0.524170in}}%
\pgfpathlineto{\pgfqpoint{1.408998in}{0.524170in}}%
\pgfpathlineto{\pgfqpoint{1.408998in}{1.172540in}}%
\pgfpathlineto{\pgfqpoint{1.366910in}{1.172540in}}%
\pgfpathlineto{\pgfqpoint{1.366910in}{0.524170in}}%
\pgfpathclose%
\pgfusepath{fill}%
\end{pgfscope}%
\begin{pgfscope}%
\pgfpathrectangle{\pgfqpoint{0.651412in}{0.524170in}}{\pgfqpoint{4.629690in}{2.558193in}}%
\pgfusepath{clip}%
\pgfsetbuttcap%
\pgfsetmiterjoin%
\definecolor{currentfill}{rgb}{0.003922,0.450980,0.698039}%
\pgfsetfillcolor{currentfill}%
\pgfsetfillopacity{0.700000}%
\pgfsetlinewidth{0.000000pt}%
\definecolor{currentstroke}{rgb}{0.000000,0.000000,0.000000}%
\pgfsetstrokecolor{currentstroke}%
\pgfsetstrokeopacity{0.700000}%
\pgfsetdash{}{0pt}%
\pgfpathmoveto{\pgfqpoint{1.408998in}{0.524170in}}%
\pgfpathlineto{\pgfqpoint{1.451086in}{0.524170in}}%
\pgfpathlineto{\pgfqpoint{1.451086in}{1.160071in}}%
\pgfpathlineto{\pgfqpoint{1.408998in}{1.160071in}}%
\pgfpathlineto{\pgfqpoint{1.408998in}{0.524170in}}%
\pgfpathclose%
\pgfusepath{fill}%
\end{pgfscope}%
\begin{pgfscope}%
\pgfpathrectangle{\pgfqpoint{0.651412in}{0.524170in}}{\pgfqpoint{4.629690in}{2.558193in}}%
\pgfusepath{clip}%
\pgfsetbuttcap%
\pgfsetmiterjoin%
\definecolor{currentfill}{rgb}{0.003922,0.450980,0.698039}%
\pgfsetfillcolor{currentfill}%
\pgfsetfillopacity{0.700000}%
\pgfsetlinewidth{0.000000pt}%
\definecolor{currentstroke}{rgb}{0.000000,0.000000,0.000000}%
\pgfsetstrokecolor{currentstroke}%
\pgfsetstrokeopacity{0.700000}%
\pgfsetdash{}{0pt}%
\pgfpathmoveto{\pgfqpoint{1.451086in}{0.524170in}}%
\pgfpathlineto{\pgfqpoint{1.493174in}{0.524170in}}%
\pgfpathlineto{\pgfqpoint{1.493174in}{1.042866in}}%
\pgfpathlineto{\pgfqpoint{1.451086in}{1.042866in}}%
\pgfpathlineto{\pgfqpoint{1.451086in}{0.524170in}}%
\pgfpathclose%
\pgfusepath{fill}%
\end{pgfscope}%
\begin{pgfscope}%
\pgfpathrectangle{\pgfqpoint{0.651412in}{0.524170in}}{\pgfqpoint{4.629690in}{2.558193in}}%
\pgfusepath{clip}%
\pgfsetbuttcap%
\pgfsetmiterjoin%
\definecolor{currentfill}{rgb}{0.003922,0.450980,0.698039}%
\pgfsetfillcolor{currentfill}%
\pgfsetfillopacity{0.700000}%
\pgfsetlinewidth{0.000000pt}%
\definecolor{currentstroke}{rgb}{0.000000,0.000000,0.000000}%
\pgfsetstrokecolor{currentstroke}%
\pgfsetstrokeopacity{0.700000}%
\pgfsetdash{}{0pt}%
\pgfpathmoveto{\pgfqpoint{1.493174in}{0.524170in}}%
\pgfpathlineto{\pgfqpoint{1.535262in}{0.524170in}}%
\pgfpathlineto{\pgfqpoint{1.535262in}{1.025410in}}%
\pgfpathlineto{\pgfqpoint{1.493174in}{1.025410in}}%
\pgfpathlineto{\pgfqpoint{1.493174in}{0.524170in}}%
\pgfpathclose%
\pgfusepath{fill}%
\end{pgfscope}%
\begin{pgfscope}%
\pgfpathrectangle{\pgfqpoint{0.651412in}{0.524170in}}{\pgfqpoint{4.629690in}{2.558193in}}%
\pgfusepath{clip}%
\pgfsetbuttcap%
\pgfsetmiterjoin%
\definecolor{currentfill}{rgb}{0.003922,0.450980,0.698039}%
\pgfsetfillcolor{currentfill}%
\pgfsetfillopacity{0.700000}%
\pgfsetlinewidth{0.000000pt}%
\definecolor{currentstroke}{rgb}{0.000000,0.000000,0.000000}%
\pgfsetstrokecolor{currentstroke}%
\pgfsetstrokeopacity{0.700000}%
\pgfsetdash{}{0pt}%
\pgfpathmoveto{\pgfqpoint{1.535262in}{0.524170in}}%
\pgfpathlineto{\pgfqpoint{1.577350in}{0.524170in}}%
\pgfpathlineto{\pgfqpoint{1.577350in}{0.943116in}}%
\pgfpathlineto{\pgfqpoint{1.535262in}{0.943116in}}%
\pgfpathlineto{\pgfqpoint{1.535262in}{0.524170in}}%
\pgfpathclose%
\pgfusepath{fill}%
\end{pgfscope}%
\begin{pgfscope}%
\pgfpathrectangle{\pgfqpoint{0.651412in}{0.524170in}}{\pgfqpoint{4.629690in}{2.558193in}}%
\pgfusepath{clip}%
\pgfsetbuttcap%
\pgfsetmiterjoin%
\definecolor{currentfill}{rgb}{0.003922,0.450980,0.698039}%
\pgfsetfillcolor{currentfill}%
\pgfsetfillopacity{0.700000}%
\pgfsetlinewidth{0.000000pt}%
\definecolor{currentstroke}{rgb}{0.000000,0.000000,0.000000}%
\pgfsetstrokecolor{currentstroke}%
\pgfsetstrokeopacity{0.700000}%
\pgfsetdash{}{0pt}%
\pgfpathmoveto{\pgfqpoint{1.577350in}{0.524170in}}%
\pgfpathlineto{\pgfqpoint{1.619438in}{0.524170in}}%
\pgfpathlineto{\pgfqpoint{1.619438in}{0.903217in}}%
\pgfpathlineto{\pgfqpoint{1.577350in}{0.903217in}}%
\pgfpathlineto{\pgfqpoint{1.577350in}{0.524170in}}%
\pgfpathclose%
\pgfusepath{fill}%
\end{pgfscope}%
\begin{pgfscope}%
\pgfpathrectangle{\pgfqpoint{0.651412in}{0.524170in}}{\pgfqpoint{4.629690in}{2.558193in}}%
\pgfusepath{clip}%
\pgfsetbuttcap%
\pgfsetmiterjoin%
\definecolor{currentfill}{rgb}{0.003922,0.450980,0.698039}%
\pgfsetfillcolor{currentfill}%
\pgfsetfillopacity{0.700000}%
\pgfsetlinewidth{0.000000pt}%
\definecolor{currentstroke}{rgb}{0.000000,0.000000,0.000000}%
\pgfsetstrokecolor{currentstroke}%
\pgfsetstrokeopacity{0.700000}%
\pgfsetdash{}{0pt}%
\pgfpathmoveto{\pgfqpoint{1.619438in}{0.524170in}}%
\pgfpathlineto{\pgfqpoint{1.661526in}{0.524170in}}%
\pgfpathlineto{\pgfqpoint{1.661526in}{0.828405in}}%
\pgfpathlineto{\pgfqpoint{1.619438in}{0.828405in}}%
\pgfpathlineto{\pgfqpoint{1.619438in}{0.524170in}}%
\pgfpathclose%
\pgfusepath{fill}%
\end{pgfscope}%
\begin{pgfscope}%
\pgfpathrectangle{\pgfqpoint{0.651412in}{0.524170in}}{\pgfqpoint{4.629690in}{2.558193in}}%
\pgfusepath{clip}%
\pgfsetbuttcap%
\pgfsetmiterjoin%
\definecolor{currentfill}{rgb}{0.003922,0.450980,0.698039}%
\pgfsetfillcolor{currentfill}%
\pgfsetfillopacity{0.700000}%
\pgfsetlinewidth{0.000000pt}%
\definecolor{currentstroke}{rgb}{0.000000,0.000000,0.000000}%
\pgfsetstrokecolor{currentstroke}%
\pgfsetstrokeopacity{0.700000}%
\pgfsetdash{}{0pt}%
\pgfpathmoveto{\pgfqpoint{1.661526in}{0.524170in}}%
\pgfpathlineto{\pgfqpoint{1.703614in}{0.524170in}}%
\pgfpathlineto{\pgfqpoint{1.703614in}{0.805961in}}%
\pgfpathlineto{\pgfqpoint{1.661526in}{0.805961in}}%
\pgfpathlineto{\pgfqpoint{1.661526in}{0.524170in}}%
\pgfpathclose%
\pgfusepath{fill}%
\end{pgfscope}%
\begin{pgfscope}%
\pgfpathrectangle{\pgfqpoint{0.651412in}{0.524170in}}{\pgfqpoint{4.629690in}{2.558193in}}%
\pgfusepath{clip}%
\pgfsetbuttcap%
\pgfsetmiterjoin%
\definecolor{currentfill}{rgb}{0.003922,0.450980,0.698039}%
\pgfsetfillcolor{currentfill}%
\pgfsetfillopacity{0.700000}%
\pgfsetlinewidth{0.000000pt}%
\definecolor{currentstroke}{rgb}{0.000000,0.000000,0.000000}%
\pgfsetstrokecolor{currentstroke}%
\pgfsetstrokeopacity{0.700000}%
\pgfsetdash{}{0pt}%
\pgfpathmoveto{\pgfqpoint{1.703614in}{0.524170in}}%
\pgfpathlineto{\pgfqpoint{1.745703in}{0.524170in}}%
\pgfpathlineto{\pgfqpoint{1.745703in}{0.828405in}}%
\pgfpathlineto{\pgfqpoint{1.703614in}{0.828405in}}%
\pgfpathlineto{\pgfqpoint{1.703614in}{0.524170in}}%
\pgfpathclose%
\pgfusepath{fill}%
\end{pgfscope}%
\begin{pgfscope}%
\pgfpathrectangle{\pgfqpoint{0.651412in}{0.524170in}}{\pgfqpoint{4.629690in}{2.558193in}}%
\pgfusepath{clip}%
\pgfsetbuttcap%
\pgfsetmiterjoin%
\definecolor{currentfill}{rgb}{0.003922,0.450980,0.698039}%
\pgfsetfillcolor{currentfill}%
\pgfsetfillopacity{0.700000}%
\pgfsetlinewidth{0.000000pt}%
\definecolor{currentstroke}{rgb}{0.000000,0.000000,0.000000}%
\pgfsetstrokecolor{currentstroke}%
\pgfsetstrokeopacity{0.700000}%
\pgfsetdash{}{0pt}%
\pgfpathmoveto{\pgfqpoint{1.745703in}{0.524170in}}%
\pgfpathlineto{\pgfqpoint{1.787791in}{0.524170in}}%
\pgfpathlineto{\pgfqpoint{1.787791in}{0.800974in}}%
\pgfpathlineto{\pgfqpoint{1.745703in}{0.800974in}}%
\pgfpathlineto{\pgfqpoint{1.745703in}{0.524170in}}%
\pgfpathclose%
\pgfusepath{fill}%
\end{pgfscope}%
\begin{pgfscope}%
\pgfpathrectangle{\pgfqpoint{0.651412in}{0.524170in}}{\pgfqpoint{4.629690in}{2.558193in}}%
\pgfusepath{clip}%
\pgfsetbuttcap%
\pgfsetmiterjoin%
\definecolor{currentfill}{rgb}{0.003922,0.450980,0.698039}%
\pgfsetfillcolor{currentfill}%
\pgfsetfillopacity{0.700000}%
\pgfsetlinewidth{0.000000pt}%
\definecolor{currentstroke}{rgb}{0.000000,0.000000,0.000000}%
\pgfsetstrokecolor{currentstroke}%
\pgfsetstrokeopacity{0.700000}%
\pgfsetdash{}{0pt}%
\pgfpathmoveto{\pgfqpoint{1.787791in}{0.524170in}}%
\pgfpathlineto{\pgfqpoint{1.829879in}{0.524170in}}%
\pgfpathlineto{\pgfqpoint{1.829879in}{0.751099in}}%
\pgfpathlineto{\pgfqpoint{1.787791in}{0.751099in}}%
\pgfpathlineto{\pgfqpoint{1.787791in}{0.524170in}}%
\pgfpathclose%
\pgfusepath{fill}%
\end{pgfscope}%
\begin{pgfscope}%
\pgfpathrectangle{\pgfqpoint{0.651412in}{0.524170in}}{\pgfqpoint{4.629690in}{2.558193in}}%
\pgfusepath{clip}%
\pgfsetbuttcap%
\pgfsetmiterjoin%
\definecolor{currentfill}{rgb}{0.003922,0.450980,0.698039}%
\pgfsetfillcolor{currentfill}%
\pgfsetfillopacity{0.700000}%
\pgfsetlinewidth{0.000000pt}%
\definecolor{currentstroke}{rgb}{0.000000,0.000000,0.000000}%
\pgfsetstrokecolor{currentstroke}%
\pgfsetstrokeopacity{0.700000}%
\pgfsetdash{}{0pt}%
\pgfpathmoveto{\pgfqpoint{1.829879in}{0.524170in}}%
\pgfpathlineto{\pgfqpoint{1.871967in}{0.524170in}}%
\pgfpathlineto{\pgfqpoint{1.871967in}{0.741124in}}%
\pgfpathlineto{\pgfqpoint{1.829879in}{0.741124in}}%
\pgfpathlineto{\pgfqpoint{1.829879in}{0.524170in}}%
\pgfpathclose%
\pgfusepath{fill}%
\end{pgfscope}%
\begin{pgfscope}%
\pgfpathrectangle{\pgfqpoint{0.651412in}{0.524170in}}{\pgfqpoint{4.629690in}{2.558193in}}%
\pgfusepath{clip}%
\pgfsetbuttcap%
\pgfsetmiterjoin%
\definecolor{currentfill}{rgb}{0.003922,0.450980,0.698039}%
\pgfsetfillcolor{currentfill}%
\pgfsetfillopacity{0.700000}%
\pgfsetlinewidth{0.000000pt}%
\definecolor{currentstroke}{rgb}{0.000000,0.000000,0.000000}%
\pgfsetstrokecolor{currentstroke}%
\pgfsetstrokeopacity{0.700000}%
\pgfsetdash{}{0pt}%
\pgfpathmoveto{\pgfqpoint{1.871967in}{0.524170in}}%
\pgfpathlineto{\pgfqpoint{1.914055in}{0.524170in}}%
\pgfpathlineto{\pgfqpoint{1.914055in}{0.748605in}}%
\pgfpathlineto{\pgfqpoint{1.871967in}{0.748605in}}%
\pgfpathlineto{\pgfqpoint{1.871967in}{0.524170in}}%
\pgfpathclose%
\pgfusepath{fill}%
\end{pgfscope}%
\begin{pgfscope}%
\pgfpathrectangle{\pgfqpoint{0.651412in}{0.524170in}}{\pgfqpoint{4.629690in}{2.558193in}}%
\pgfusepath{clip}%
\pgfsetbuttcap%
\pgfsetmiterjoin%
\definecolor{currentfill}{rgb}{0.003922,0.450980,0.698039}%
\pgfsetfillcolor{currentfill}%
\pgfsetfillopacity{0.700000}%
\pgfsetlinewidth{0.000000pt}%
\definecolor{currentstroke}{rgb}{0.000000,0.000000,0.000000}%
\pgfsetstrokecolor{currentstroke}%
\pgfsetstrokeopacity{0.700000}%
\pgfsetdash{}{0pt}%
\pgfpathmoveto{\pgfqpoint{1.914055in}{0.524170in}}%
\pgfpathlineto{\pgfqpoint{1.956143in}{0.524170in}}%
\pgfpathlineto{\pgfqpoint{1.956143in}{0.691250in}}%
\pgfpathlineto{\pgfqpoint{1.914055in}{0.691250in}}%
\pgfpathlineto{\pgfqpoint{1.914055in}{0.524170in}}%
\pgfpathclose%
\pgfusepath{fill}%
\end{pgfscope}%
\begin{pgfscope}%
\pgfpathrectangle{\pgfqpoint{0.651412in}{0.524170in}}{\pgfqpoint{4.629690in}{2.558193in}}%
\pgfusepath{clip}%
\pgfsetbuttcap%
\pgfsetmiterjoin%
\definecolor{currentfill}{rgb}{0.003922,0.450980,0.698039}%
\pgfsetfillcolor{currentfill}%
\pgfsetfillopacity{0.700000}%
\pgfsetlinewidth{0.000000pt}%
\definecolor{currentstroke}{rgb}{0.000000,0.000000,0.000000}%
\pgfsetstrokecolor{currentstroke}%
\pgfsetstrokeopacity{0.700000}%
\pgfsetdash{}{0pt}%
\pgfpathmoveto{\pgfqpoint{1.956143in}{0.524170in}}%
\pgfpathlineto{\pgfqpoint{1.998231in}{0.524170in}}%
\pgfpathlineto{\pgfqpoint{1.998231in}{0.656337in}}%
\pgfpathlineto{\pgfqpoint{1.956143in}{0.656337in}}%
\pgfpathlineto{\pgfqpoint{1.956143in}{0.524170in}}%
\pgfpathclose%
\pgfusepath{fill}%
\end{pgfscope}%
\begin{pgfscope}%
\pgfpathrectangle{\pgfqpoint{0.651412in}{0.524170in}}{\pgfqpoint{4.629690in}{2.558193in}}%
\pgfusepath{clip}%
\pgfsetbuttcap%
\pgfsetmiterjoin%
\definecolor{currentfill}{rgb}{0.003922,0.450980,0.698039}%
\pgfsetfillcolor{currentfill}%
\pgfsetfillopacity{0.700000}%
\pgfsetlinewidth{0.000000pt}%
\definecolor{currentstroke}{rgb}{0.000000,0.000000,0.000000}%
\pgfsetstrokecolor{currentstroke}%
\pgfsetstrokeopacity{0.700000}%
\pgfsetdash{}{0pt}%
\pgfpathmoveto{\pgfqpoint{1.998231in}{0.524170in}}%
\pgfpathlineto{\pgfqpoint{2.040319in}{0.524170in}}%
\pgfpathlineto{\pgfqpoint{2.040319in}{0.688756in}}%
\pgfpathlineto{\pgfqpoint{1.998231in}{0.688756in}}%
\pgfpathlineto{\pgfqpoint{1.998231in}{0.524170in}}%
\pgfpathclose%
\pgfusepath{fill}%
\end{pgfscope}%
\begin{pgfscope}%
\pgfpathrectangle{\pgfqpoint{0.651412in}{0.524170in}}{\pgfqpoint{4.629690in}{2.558193in}}%
\pgfusepath{clip}%
\pgfsetbuttcap%
\pgfsetmiterjoin%
\definecolor{currentfill}{rgb}{0.003922,0.450980,0.698039}%
\pgfsetfillcolor{currentfill}%
\pgfsetfillopacity{0.700000}%
\pgfsetlinewidth{0.000000pt}%
\definecolor{currentstroke}{rgb}{0.000000,0.000000,0.000000}%
\pgfsetstrokecolor{currentstroke}%
\pgfsetstrokeopacity{0.700000}%
\pgfsetdash{}{0pt}%
\pgfpathmoveto{\pgfqpoint{2.040319in}{0.524170in}}%
\pgfpathlineto{\pgfqpoint{2.082407in}{0.524170in}}%
\pgfpathlineto{\pgfqpoint{2.082407in}{0.696237in}}%
\pgfpathlineto{\pgfqpoint{2.040319in}{0.696237in}}%
\pgfpathlineto{\pgfqpoint{2.040319in}{0.524170in}}%
\pgfpathclose%
\pgfusepath{fill}%
\end{pgfscope}%
\begin{pgfscope}%
\pgfpathrectangle{\pgfqpoint{0.651412in}{0.524170in}}{\pgfqpoint{4.629690in}{2.558193in}}%
\pgfusepath{clip}%
\pgfsetbuttcap%
\pgfsetmiterjoin%
\definecolor{currentfill}{rgb}{0.003922,0.450980,0.698039}%
\pgfsetfillcolor{currentfill}%
\pgfsetfillopacity{0.700000}%
\pgfsetlinewidth{0.000000pt}%
\definecolor{currentstroke}{rgb}{0.000000,0.000000,0.000000}%
\pgfsetstrokecolor{currentstroke}%
\pgfsetstrokeopacity{0.700000}%
\pgfsetdash{}{0pt}%
\pgfpathmoveto{\pgfqpoint{2.082407in}{0.524170in}}%
\pgfpathlineto{\pgfqpoint{2.124495in}{0.524170in}}%
\pgfpathlineto{\pgfqpoint{2.124495in}{0.633894in}}%
\pgfpathlineto{\pgfqpoint{2.082407in}{0.633894in}}%
\pgfpathlineto{\pgfqpoint{2.082407in}{0.524170in}}%
\pgfpathclose%
\pgfusepath{fill}%
\end{pgfscope}%
\begin{pgfscope}%
\pgfpathrectangle{\pgfqpoint{0.651412in}{0.524170in}}{\pgfqpoint{4.629690in}{2.558193in}}%
\pgfusepath{clip}%
\pgfsetbuttcap%
\pgfsetmiterjoin%
\definecolor{currentfill}{rgb}{0.003922,0.450980,0.698039}%
\pgfsetfillcolor{currentfill}%
\pgfsetfillopacity{0.700000}%
\pgfsetlinewidth{0.000000pt}%
\definecolor{currentstroke}{rgb}{0.000000,0.000000,0.000000}%
\pgfsetstrokecolor{currentstroke}%
\pgfsetstrokeopacity{0.700000}%
\pgfsetdash{}{0pt}%
\pgfpathmoveto{\pgfqpoint{2.124495in}{0.524170in}}%
\pgfpathlineto{\pgfqpoint{2.166583in}{0.524170in}}%
\pgfpathlineto{\pgfqpoint{2.166583in}{0.663819in}}%
\pgfpathlineto{\pgfqpoint{2.124495in}{0.663819in}}%
\pgfpathlineto{\pgfqpoint{2.124495in}{0.524170in}}%
\pgfpathclose%
\pgfusepath{fill}%
\end{pgfscope}%
\begin{pgfscope}%
\pgfpathrectangle{\pgfqpoint{0.651412in}{0.524170in}}{\pgfqpoint{4.629690in}{2.558193in}}%
\pgfusepath{clip}%
\pgfsetbuttcap%
\pgfsetmiterjoin%
\definecolor{currentfill}{rgb}{0.003922,0.450980,0.698039}%
\pgfsetfillcolor{currentfill}%
\pgfsetfillopacity{0.700000}%
\pgfsetlinewidth{0.000000pt}%
\definecolor{currentstroke}{rgb}{0.000000,0.000000,0.000000}%
\pgfsetstrokecolor{currentstroke}%
\pgfsetstrokeopacity{0.700000}%
\pgfsetdash{}{0pt}%
\pgfpathmoveto{\pgfqpoint{2.166583in}{0.524170in}}%
\pgfpathlineto{\pgfqpoint{2.208672in}{0.524170in}}%
\pgfpathlineto{\pgfqpoint{2.208672in}{0.633894in}}%
\pgfpathlineto{\pgfqpoint{2.166583in}{0.633894in}}%
\pgfpathlineto{\pgfqpoint{2.166583in}{0.524170in}}%
\pgfpathclose%
\pgfusepath{fill}%
\end{pgfscope}%
\begin{pgfscope}%
\pgfpathrectangle{\pgfqpoint{0.651412in}{0.524170in}}{\pgfqpoint{4.629690in}{2.558193in}}%
\pgfusepath{clip}%
\pgfsetbuttcap%
\pgfsetmiterjoin%
\definecolor{currentfill}{rgb}{0.003922,0.450980,0.698039}%
\pgfsetfillcolor{currentfill}%
\pgfsetfillopacity{0.700000}%
\pgfsetlinewidth{0.000000pt}%
\definecolor{currentstroke}{rgb}{0.000000,0.000000,0.000000}%
\pgfsetstrokecolor{currentstroke}%
\pgfsetstrokeopacity{0.700000}%
\pgfsetdash{}{0pt}%
\pgfpathmoveto{\pgfqpoint{2.208672in}{0.524170in}}%
\pgfpathlineto{\pgfqpoint{2.250760in}{0.524170in}}%
\pgfpathlineto{\pgfqpoint{2.250760in}{0.661325in}}%
\pgfpathlineto{\pgfqpoint{2.208672in}{0.661325in}}%
\pgfpathlineto{\pgfqpoint{2.208672in}{0.524170in}}%
\pgfpathclose%
\pgfusepath{fill}%
\end{pgfscope}%
\begin{pgfscope}%
\pgfpathrectangle{\pgfqpoint{0.651412in}{0.524170in}}{\pgfqpoint{4.629690in}{2.558193in}}%
\pgfusepath{clip}%
\pgfsetbuttcap%
\pgfsetmiterjoin%
\definecolor{currentfill}{rgb}{0.003922,0.450980,0.698039}%
\pgfsetfillcolor{currentfill}%
\pgfsetfillopacity{0.700000}%
\pgfsetlinewidth{0.000000pt}%
\definecolor{currentstroke}{rgb}{0.000000,0.000000,0.000000}%
\pgfsetstrokecolor{currentstroke}%
\pgfsetstrokeopacity{0.700000}%
\pgfsetdash{}{0pt}%
\pgfpathmoveto{\pgfqpoint{2.250760in}{0.524170in}}%
\pgfpathlineto{\pgfqpoint{2.292848in}{0.524170in}}%
\pgfpathlineto{\pgfqpoint{2.292848in}{0.618932in}}%
\pgfpathlineto{\pgfqpoint{2.250760in}{0.618932in}}%
\pgfpathlineto{\pgfqpoint{2.250760in}{0.524170in}}%
\pgfpathclose%
\pgfusepath{fill}%
\end{pgfscope}%
\begin{pgfscope}%
\pgfpathrectangle{\pgfqpoint{0.651412in}{0.524170in}}{\pgfqpoint{4.629690in}{2.558193in}}%
\pgfusepath{clip}%
\pgfsetbuttcap%
\pgfsetmiterjoin%
\definecolor{currentfill}{rgb}{0.003922,0.450980,0.698039}%
\pgfsetfillcolor{currentfill}%
\pgfsetfillopacity{0.700000}%
\pgfsetlinewidth{0.000000pt}%
\definecolor{currentstroke}{rgb}{0.000000,0.000000,0.000000}%
\pgfsetstrokecolor{currentstroke}%
\pgfsetstrokeopacity{0.700000}%
\pgfsetdash{}{0pt}%
\pgfpathmoveto{\pgfqpoint{2.292848in}{0.524170in}}%
\pgfpathlineto{\pgfqpoint{2.334936in}{0.524170in}}%
\pgfpathlineto{\pgfqpoint{2.334936in}{0.608957in}}%
\pgfpathlineto{\pgfqpoint{2.292848in}{0.608957in}}%
\pgfpathlineto{\pgfqpoint{2.292848in}{0.524170in}}%
\pgfpathclose%
\pgfusepath{fill}%
\end{pgfscope}%
\begin{pgfscope}%
\pgfpathrectangle{\pgfqpoint{0.651412in}{0.524170in}}{\pgfqpoint{4.629690in}{2.558193in}}%
\pgfusepath{clip}%
\pgfsetbuttcap%
\pgfsetmiterjoin%
\definecolor{currentfill}{rgb}{0.003922,0.450980,0.698039}%
\pgfsetfillcolor{currentfill}%
\pgfsetfillopacity{0.700000}%
\pgfsetlinewidth{0.000000pt}%
\definecolor{currentstroke}{rgb}{0.000000,0.000000,0.000000}%
\pgfsetstrokecolor{currentstroke}%
\pgfsetstrokeopacity{0.700000}%
\pgfsetdash{}{0pt}%
\pgfpathmoveto{\pgfqpoint{2.334936in}{0.524170in}}%
\pgfpathlineto{\pgfqpoint{2.377024in}{0.524170in}}%
\pgfpathlineto{\pgfqpoint{2.377024in}{0.623919in}}%
\pgfpathlineto{\pgfqpoint{2.334936in}{0.623919in}}%
\pgfpathlineto{\pgfqpoint{2.334936in}{0.524170in}}%
\pgfpathclose%
\pgfusepath{fill}%
\end{pgfscope}%
\begin{pgfscope}%
\pgfpathrectangle{\pgfqpoint{0.651412in}{0.524170in}}{\pgfqpoint{4.629690in}{2.558193in}}%
\pgfusepath{clip}%
\pgfsetbuttcap%
\pgfsetmiterjoin%
\definecolor{currentfill}{rgb}{0.003922,0.450980,0.698039}%
\pgfsetfillcolor{currentfill}%
\pgfsetfillopacity{0.700000}%
\pgfsetlinewidth{0.000000pt}%
\definecolor{currentstroke}{rgb}{0.000000,0.000000,0.000000}%
\pgfsetstrokecolor{currentstroke}%
\pgfsetstrokeopacity{0.700000}%
\pgfsetdash{}{0pt}%
\pgfpathmoveto{\pgfqpoint{2.377024in}{0.524170in}}%
\pgfpathlineto{\pgfqpoint{2.419112in}{0.524170in}}%
\pgfpathlineto{\pgfqpoint{2.419112in}{0.596488in}}%
\pgfpathlineto{\pgfqpoint{2.377024in}{0.596488in}}%
\pgfpathlineto{\pgfqpoint{2.377024in}{0.524170in}}%
\pgfpathclose%
\pgfusepath{fill}%
\end{pgfscope}%
\begin{pgfscope}%
\pgfpathrectangle{\pgfqpoint{0.651412in}{0.524170in}}{\pgfqpoint{4.629690in}{2.558193in}}%
\pgfusepath{clip}%
\pgfsetbuttcap%
\pgfsetmiterjoin%
\definecolor{currentfill}{rgb}{0.003922,0.450980,0.698039}%
\pgfsetfillcolor{currentfill}%
\pgfsetfillopacity{0.700000}%
\pgfsetlinewidth{0.000000pt}%
\definecolor{currentstroke}{rgb}{0.000000,0.000000,0.000000}%
\pgfsetstrokecolor{currentstroke}%
\pgfsetstrokeopacity{0.700000}%
\pgfsetdash{}{0pt}%
\pgfpathmoveto{\pgfqpoint{2.419112in}{0.524170in}}%
\pgfpathlineto{\pgfqpoint{2.461200in}{0.524170in}}%
\pgfpathlineto{\pgfqpoint{2.461200in}{0.596488in}}%
\pgfpathlineto{\pgfqpoint{2.419112in}{0.596488in}}%
\pgfpathlineto{\pgfqpoint{2.419112in}{0.524170in}}%
\pgfpathclose%
\pgfusepath{fill}%
\end{pgfscope}%
\begin{pgfscope}%
\pgfpathrectangle{\pgfqpoint{0.651412in}{0.524170in}}{\pgfqpoint{4.629690in}{2.558193in}}%
\pgfusepath{clip}%
\pgfsetbuttcap%
\pgfsetmiterjoin%
\definecolor{currentfill}{rgb}{0.003922,0.450980,0.698039}%
\pgfsetfillcolor{currentfill}%
\pgfsetfillopacity{0.700000}%
\pgfsetlinewidth{0.000000pt}%
\definecolor{currentstroke}{rgb}{0.000000,0.000000,0.000000}%
\pgfsetstrokecolor{currentstroke}%
\pgfsetstrokeopacity{0.700000}%
\pgfsetdash{}{0pt}%
\pgfpathmoveto{\pgfqpoint{2.461200in}{0.524170in}}%
\pgfpathlineto{\pgfqpoint{2.503288in}{0.524170in}}%
\pgfpathlineto{\pgfqpoint{2.503288in}{0.603969in}}%
\pgfpathlineto{\pgfqpoint{2.461200in}{0.603969in}}%
\pgfpathlineto{\pgfqpoint{2.461200in}{0.524170in}}%
\pgfpathclose%
\pgfusepath{fill}%
\end{pgfscope}%
\begin{pgfscope}%
\pgfpathrectangle{\pgfqpoint{0.651412in}{0.524170in}}{\pgfqpoint{4.629690in}{2.558193in}}%
\pgfusepath{clip}%
\pgfsetbuttcap%
\pgfsetmiterjoin%
\definecolor{currentfill}{rgb}{0.003922,0.450980,0.698039}%
\pgfsetfillcolor{currentfill}%
\pgfsetfillopacity{0.700000}%
\pgfsetlinewidth{0.000000pt}%
\definecolor{currentstroke}{rgb}{0.000000,0.000000,0.000000}%
\pgfsetstrokecolor{currentstroke}%
\pgfsetstrokeopacity{0.700000}%
\pgfsetdash{}{0pt}%
\pgfpathmoveto{\pgfqpoint{2.503288in}{0.524170in}}%
\pgfpathlineto{\pgfqpoint{2.545376in}{0.524170in}}%
\pgfpathlineto{\pgfqpoint{2.545376in}{0.616438in}}%
\pgfpathlineto{\pgfqpoint{2.503288in}{0.616438in}}%
\pgfpathlineto{\pgfqpoint{2.503288in}{0.524170in}}%
\pgfpathclose%
\pgfusepath{fill}%
\end{pgfscope}%
\begin{pgfscope}%
\pgfpathrectangle{\pgfqpoint{0.651412in}{0.524170in}}{\pgfqpoint{4.629690in}{2.558193in}}%
\pgfusepath{clip}%
\pgfsetbuttcap%
\pgfsetmiterjoin%
\definecolor{currentfill}{rgb}{0.003922,0.450980,0.698039}%
\pgfsetfillcolor{currentfill}%
\pgfsetfillopacity{0.700000}%
\pgfsetlinewidth{0.000000pt}%
\definecolor{currentstroke}{rgb}{0.000000,0.000000,0.000000}%
\pgfsetstrokecolor{currentstroke}%
\pgfsetstrokeopacity{0.700000}%
\pgfsetdash{}{0pt}%
\pgfpathmoveto{\pgfqpoint{2.545376in}{0.524170in}}%
\pgfpathlineto{\pgfqpoint{2.587464in}{0.524170in}}%
\pgfpathlineto{\pgfqpoint{2.587464in}{0.593994in}}%
\pgfpathlineto{\pgfqpoint{2.545376in}{0.593994in}}%
\pgfpathlineto{\pgfqpoint{2.545376in}{0.524170in}}%
\pgfpathclose%
\pgfusepath{fill}%
\end{pgfscope}%
\begin{pgfscope}%
\pgfpathrectangle{\pgfqpoint{0.651412in}{0.524170in}}{\pgfqpoint{4.629690in}{2.558193in}}%
\pgfusepath{clip}%
\pgfsetbuttcap%
\pgfsetmiterjoin%
\definecolor{currentfill}{rgb}{0.003922,0.450980,0.698039}%
\pgfsetfillcolor{currentfill}%
\pgfsetfillopacity{0.700000}%
\pgfsetlinewidth{0.000000pt}%
\definecolor{currentstroke}{rgb}{0.000000,0.000000,0.000000}%
\pgfsetstrokecolor{currentstroke}%
\pgfsetstrokeopacity{0.700000}%
\pgfsetdash{}{0pt}%
\pgfpathmoveto{\pgfqpoint{2.587464in}{0.524170in}}%
\pgfpathlineto{\pgfqpoint{2.629553in}{0.524170in}}%
\pgfpathlineto{\pgfqpoint{2.629553in}{0.608957in}}%
\pgfpathlineto{\pgfqpoint{2.587464in}{0.608957in}}%
\pgfpathlineto{\pgfqpoint{2.587464in}{0.524170in}}%
\pgfpathclose%
\pgfusepath{fill}%
\end{pgfscope}%
\begin{pgfscope}%
\pgfpathrectangle{\pgfqpoint{0.651412in}{0.524170in}}{\pgfqpoint{4.629690in}{2.558193in}}%
\pgfusepath{clip}%
\pgfsetbuttcap%
\pgfsetmiterjoin%
\definecolor{currentfill}{rgb}{0.003922,0.450980,0.698039}%
\pgfsetfillcolor{currentfill}%
\pgfsetfillopacity{0.700000}%
\pgfsetlinewidth{0.000000pt}%
\definecolor{currentstroke}{rgb}{0.000000,0.000000,0.000000}%
\pgfsetstrokecolor{currentstroke}%
\pgfsetstrokeopacity{0.700000}%
\pgfsetdash{}{0pt}%
\pgfpathmoveto{\pgfqpoint{2.629553in}{0.524170in}}%
\pgfpathlineto{\pgfqpoint{2.671641in}{0.524170in}}%
\pgfpathlineto{\pgfqpoint{2.671641in}{0.598982in}}%
\pgfpathlineto{\pgfqpoint{2.629553in}{0.598982in}}%
\pgfpathlineto{\pgfqpoint{2.629553in}{0.524170in}}%
\pgfpathclose%
\pgfusepath{fill}%
\end{pgfscope}%
\begin{pgfscope}%
\pgfpathrectangle{\pgfqpoint{0.651412in}{0.524170in}}{\pgfqpoint{4.629690in}{2.558193in}}%
\pgfusepath{clip}%
\pgfsetbuttcap%
\pgfsetmiterjoin%
\definecolor{currentfill}{rgb}{0.003922,0.450980,0.698039}%
\pgfsetfillcolor{currentfill}%
\pgfsetfillopacity{0.700000}%
\pgfsetlinewidth{0.000000pt}%
\definecolor{currentstroke}{rgb}{0.000000,0.000000,0.000000}%
\pgfsetstrokecolor{currentstroke}%
\pgfsetstrokeopacity{0.700000}%
\pgfsetdash{}{0pt}%
\pgfpathmoveto{\pgfqpoint{2.671641in}{0.524170in}}%
\pgfpathlineto{\pgfqpoint{2.713729in}{0.524170in}}%
\pgfpathlineto{\pgfqpoint{2.713729in}{0.579032in}}%
\pgfpathlineto{\pgfqpoint{2.671641in}{0.579032in}}%
\pgfpathlineto{\pgfqpoint{2.671641in}{0.524170in}}%
\pgfpathclose%
\pgfusepath{fill}%
\end{pgfscope}%
\begin{pgfscope}%
\pgfpathrectangle{\pgfqpoint{0.651412in}{0.524170in}}{\pgfqpoint{4.629690in}{2.558193in}}%
\pgfusepath{clip}%
\pgfsetbuttcap%
\pgfsetmiterjoin%
\definecolor{currentfill}{rgb}{0.003922,0.450980,0.698039}%
\pgfsetfillcolor{currentfill}%
\pgfsetfillopacity{0.700000}%
\pgfsetlinewidth{0.000000pt}%
\definecolor{currentstroke}{rgb}{0.000000,0.000000,0.000000}%
\pgfsetstrokecolor{currentstroke}%
\pgfsetstrokeopacity{0.700000}%
\pgfsetdash{}{0pt}%
\pgfpathmoveto{\pgfqpoint{2.713729in}{0.524170in}}%
\pgfpathlineto{\pgfqpoint{2.755817in}{0.524170in}}%
\pgfpathlineto{\pgfqpoint{2.755817in}{0.589007in}}%
\pgfpathlineto{\pgfqpoint{2.713729in}{0.589007in}}%
\pgfpathlineto{\pgfqpoint{2.713729in}{0.524170in}}%
\pgfpathclose%
\pgfusepath{fill}%
\end{pgfscope}%
\begin{pgfscope}%
\pgfpathrectangle{\pgfqpoint{0.651412in}{0.524170in}}{\pgfqpoint{4.629690in}{2.558193in}}%
\pgfusepath{clip}%
\pgfsetbuttcap%
\pgfsetmiterjoin%
\definecolor{currentfill}{rgb}{0.003922,0.450980,0.698039}%
\pgfsetfillcolor{currentfill}%
\pgfsetfillopacity{0.700000}%
\pgfsetlinewidth{0.000000pt}%
\definecolor{currentstroke}{rgb}{0.000000,0.000000,0.000000}%
\pgfsetstrokecolor{currentstroke}%
\pgfsetstrokeopacity{0.700000}%
\pgfsetdash{}{0pt}%
\pgfpathmoveto{\pgfqpoint{2.755817in}{0.524170in}}%
\pgfpathlineto{\pgfqpoint{2.797905in}{0.524170in}}%
\pgfpathlineto{\pgfqpoint{2.797905in}{0.584019in}}%
\pgfpathlineto{\pgfqpoint{2.755817in}{0.584019in}}%
\pgfpathlineto{\pgfqpoint{2.755817in}{0.524170in}}%
\pgfpathclose%
\pgfusepath{fill}%
\end{pgfscope}%
\begin{pgfscope}%
\pgfpathrectangle{\pgfqpoint{0.651412in}{0.524170in}}{\pgfqpoint{4.629690in}{2.558193in}}%
\pgfusepath{clip}%
\pgfsetbuttcap%
\pgfsetmiterjoin%
\definecolor{currentfill}{rgb}{0.003922,0.450980,0.698039}%
\pgfsetfillcolor{currentfill}%
\pgfsetfillopacity{0.700000}%
\pgfsetlinewidth{0.000000pt}%
\definecolor{currentstroke}{rgb}{0.000000,0.000000,0.000000}%
\pgfsetstrokecolor{currentstroke}%
\pgfsetstrokeopacity{0.700000}%
\pgfsetdash{}{0pt}%
\pgfpathmoveto{\pgfqpoint{2.797905in}{0.524170in}}%
\pgfpathlineto{\pgfqpoint{2.839993in}{0.524170in}}%
\pgfpathlineto{\pgfqpoint{2.839993in}{0.556588in}}%
\pgfpathlineto{\pgfqpoint{2.797905in}{0.556588in}}%
\pgfpathlineto{\pgfqpoint{2.797905in}{0.524170in}}%
\pgfpathclose%
\pgfusepath{fill}%
\end{pgfscope}%
\begin{pgfscope}%
\pgfpathrectangle{\pgfqpoint{0.651412in}{0.524170in}}{\pgfqpoint{4.629690in}{2.558193in}}%
\pgfusepath{clip}%
\pgfsetbuttcap%
\pgfsetmiterjoin%
\definecolor{currentfill}{rgb}{0.003922,0.450980,0.698039}%
\pgfsetfillcolor{currentfill}%
\pgfsetfillopacity{0.700000}%
\pgfsetlinewidth{0.000000pt}%
\definecolor{currentstroke}{rgb}{0.000000,0.000000,0.000000}%
\pgfsetstrokecolor{currentstroke}%
\pgfsetstrokeopacity{0.700000}%
\pgfsetdash{}{0pt}%
\pgfpathmoveto{\pgfqpoint{2.839993in}{0.524170in}}%
\pgfpathlineto{\pgfqpoint{2.882081in}{0.524170in}}%
\pgfpathlineto{\pgfqpoint{2.882081in}{0.586513in}}%
\pgfpathlineto{\pgfqpoint{2.839993in}{0.586513in}}%
\pgfpathlineto{\pgfqpoint{2.839993in}{0.524170in}}%
\pgfpathclose%
\pgfusepath{fill}%
\end{pgfscope}%
\begin{pgfscope}%
\pgfpathrectangle{\pgfqpoint{0.651412in}{0.524170in}}{\pgfqpoint{4.629690in}{2.558193in}}%
\pgfusepath{clip}%
\pgfsetbuttcap%
\pgfsetmiterjoin%
\definecolor{currentfill}{rgb}{0.003922,0.450980,0.698039}%
\pgfsetfillcolor{currentfill}%
\pgfsetfillopacity{0.700000}%
\pgfsetlinewidth{0.000000pt}%
\definecolor{currentstroke}{rgb}{0.000000,0.000000,0.000000}%
\pgfsetstrokecolor{currentstroke}%
\pgfsetstrokeopacity{0.700000}%
\pgfsetdash{}{0pt}%
\pgfpathmoveto{\pgfqpoint{2.882081in}{0.524170in}}%
\pgfpathlineto{\pgfqpoint{2.924169in}{0.524170in}}%
\pgfpathlineto{\pgfqpoint{2.924169in}{0.569057in}}%
\pgfpathlineto{\pgfqpoint{2.882081in}{0.569057in}}%
\pgfpathlineto{\pgfqpoint{2.882081in}{0.524170in}}%
\pgfpathclose%
\pgfusepath{fill}%
\end{pgfscope}%
\begin{pgfscope}%
\pgfpathrectangle{\pgfqpoint{0.651412in}{0.524170in}}{\pgfqpoint{4.629690in}{2.558193in}}%
\pgfusepath{clip}%
\pgfsetbuttcap%
\pgfsetmiterjoin%
\definecolor{currentfill}{rgb}{0.003922,0.450980,0.698039}%
\pgfsetfillcolor{currentfill}%
\pgfsetfillopacity{0.700000}%
\pgfsetlinewidth{0.000000pt}%
\definecolor{currentstroke}{rgb}{0.000000,0.000000,0.000000}%
\pgfsetstrokecolor{currentstroke}%
\pgfsetstrokeopacity{0.700000}%
\pgfsetdash{}{0pt}%
\pgfpathmoveto{\pgfqpoint{2.924169in}{0.524170in}}%
\pgfpathlineto{\pgfqpoint{2.966257in}{0.524170in}}%
\pgfpathlineto{\pgfqpoint{2.966257in}{0.579032in}}%
\pgfpathlineto{\pgfqpoint{2.924169in}{0.579032in}}%
\pgfpathlineto{\pgfqpoint{2.924169in}{0.524170in}}%
\pgfpathclose%
\pgfusepath{fill}%
\end{pgfscope}%
\begin{pgfscope}%
\pgfpathrectangle{\pgfqpoint{0.651412in}{0.524170in}}{\pgfqpoint{4.629690in}{2.558193in}}%
\pgfusepath{clip}%
\pgfsetbuttcap%
\pgfsetmiterjoin%
\definecolor{currentfill}{rgb}{0.003922,0.450980,0.698039}%
\pgfsetfillcolor{currentfill}%
\pgfsetfillopacity{0.700000}%
\pgfsetlinewidth{0.000000pt}%
\definecolor{currentstroke}{rgb}{0.000000,0.000000,0.000000}%
\pgfsetstrokecolor{currentstroke}%
\pgfsetstrokeopacity{0.700000}%
\pgfsetdash{}{0pt}%
\pgfpathmoveto{\pgfqpoint{2.966257in}{0.524170in}}%
\pgfpathlineto{\pgfqpoint{3.008345in}{0.524170in}}%
\pgfpathlineto{\pgfqpoint{3.008345in}{0.556588in}}%
\pgfpathlineto{\pgfqpoint{2.966257in}{0.556588in}}%
\pgfpathlineto{\pgfqpoint{2.966257in}{0.524170in}}%
\pgfpathclose%
\pgfusepath{fill}%
\end{pgfscope}%
\begin{pgfscope}%
\pgfpathrectangle{\pgfqpoint{0.651412in}{0.524170in}}{\pgfqpoint{4.629690in}{2.558193in}}%
\pgfusepath{clip}%
\pgfsetbuttcap%
\pgfsetmiterjoin%
\definecolor{currentfill}{rgb}{0.003922,0.450980,0.698039}%
\pgfsetfillcolor{currentfill}%
\pgfsetfillopacity{0.700000}%
\pgfsetlinewidth{0.000000pt}%
\definecolor{currentstroke}{rgb}{0.000000,0.000000,0.000000}%
\pgfsetstrokecolor{currentstroke}%
\pgfsetstrokeopacity{0.700000}%
\pgfsetdash{}{0pt}%
\pgfpathmoveto{\pgfqpoint{3.008345in}{0.524170in}}%
\pgfpathlineto{\pgfqpoint{3.050433in}{0.524170in}}%
\pgfpathlineto{\pgfqpoint{3.050433in}{0.554095in}}%
\pgfpathlineto{\pgfqpoint{3.008345in}{0.554095in}}%
\pgfpathlineto{\pgfqpoint{3.008345in}{0.524170in}}%
\pgfpathclose%
\pgfusepath{fill}%
\end{pgfscope}%
\begin{pgfscope}%
\pgfpathrectangle{\pgfqpoint{0.651412in}{0.524170in}}{\pgfqpoint{4.629690in}{2.558193in}}%
\pgfusepath{clip}%
\pgfsetbuttcap%
\pgfsetmiterjoin%
\definecolor{currentfill}{rgb}{0.003922,0.450980,0.698039}%
\pgfsetfillcolor{currentfill}%
\pgfsetfillopacity{0.700000}%
\pgfsetlinewidth{0.000000pt}%
\definecolor{currentstroke}{rgb}{0.000000,0.000000,0.000000}%
\pgfsetstrokecolor{currentstroke}%
\pgfsetstrokeopacity{0.700000}%
\pgfsetdash{}{0pt}%
\pgfpathmoveto{\pgfqpoint{3.050433in}{0.524170in}}%
\pgfpathlineto{\pgfqpoint{3.092522in}{0.524170in}}%
\pgfpathlineto{\pgfqpoint{3.092522in}{0.556588in}}%
\pgfpathlineto{\pgfqpoint{3.050433in}{0.556588in}}%
\pgfpathlineto{\pgfqpoint{3.050433in}{0.524170in}}%
\pgfpathclose%
\pgfusepath{fill}%
\end{pgfscope}%
\begin{pgfscope}%
\pgfpathrectangle{\pgfqpoint{0.651412in}{0.524170in}}{\pgfqpoint{4.629690in}{2.558193in}}%
\pgfusepath{clip}%
\pgfsetbuttcap%
\pgfsetmiterjoin%
\definecolor{currentfill}{rgb}{0.003922,0.450980,0.698039}%
\pgfsetfillcolor{currentfill}%
\pgfsetfillopacity{0.700000}%
\pgfsetlinewidth{0.000000pt}%
\definecolor{currentstroke}{rgb}{0.000000,0.000000,0.000000}%
\pgfsetstrokecolor{currentstroke}%
\pgfsetstrokeopacity{0.700000}%
\pgfsetdash{}{0pt}%
\pgfpathmoveto{\pgfqpoint{3.092522in}{0.524170in}}%
\pgfpathlineto{\pgfqpoint{3.134610in}{0.524170in}}%
\pgfpathlineto{\pgfqpoint{3.134610in}{0.566563in}}%
\pgfpathlineto{\pgfqpoint{3.092522in}{0.566563in}}%
\pgfpathlineto{\pgfqpoint{3.092522in}{0.524170in}}%
\pgfpathclose%
\pgfusepath{fill}%
\end{pgfscope}%
\begin{pgfscope}%
\pgfpathrectangle{\pgfqpoint{0.651412in}{0.524170in}}{\pgfqpoint{4.629690in}{2.558193in}}%
\pgfusepath{clip}%
\pgfsetbuttcap%
\pgfsetmiterjoin%
\definecolor{currentfill}{rgb}{0.003922,0.450980,0.698039}%
\pgfsetfillcolor{currentfill}%
\pgfsetfillopacity{0.700000}%
\pgfsetlinewidth{0.000000pt}%
\definecolor{currentstroke}{rgb}{0.000000,0.000000,0.000000}%
\pgfsetstrokecolor{currentstroke}%
\pgfsetstrokeopacity{0.700000}%
\pgfsetdash{}{0pt}%
\pgfpathmoveto{\pgfqpoint{3.134610in}{0.524170in}}%
\pgfpathlineto{\pgfqpoint{3.176698in}{0.524170in}}%
\pgfpathlineto{\pgfqpoint{3.176698in}{0.559082in}}%
\pgfpathlineto{\pgfqpoint{3.134610in}{0.559082in}}%
\pgfpathlineto{\pgfqpoint{3.134610in}{0.524170in}}%
\pgfpathclose%
\pgfusepath{fill}%
\end{pgfscope}%
\begin{pgfscope}%
\pgfpathrectangle{\pgfqpoint{0.651412in}{0.524170in}}{\pgfqpoint{4.629690in}{2.558193in}}%
\pgfusepath{clip}%
\pgfsetbuttcap%
\pgfsetmiterjoin%
\definecolor{currentfill}{rgb}{0.003922,0.450980,0.698039}%
\pgfsetfillcolor{currentfill}%
\pgfsetfillopacity{0.700000}%
\pgfsetlinewidth{0.000000pt}%
\definecolor{currentstroke}{rgb}{0.000000,0.000000,0.000000}%
\pgfsetstrokecolor{currentstroke}%
\pgfsetstrokeopacity{0.700000}%
\pgfsetdash{}{0pt}%
\pgfpathmoveto{\pgfqpoint{3.176698in}{0.524170in}}%
\pgfpathlineto{\pgfqpoint{3.218786in}{0.524170in}}%
\pgfpathlineto{\pgfqpoint{3.218786in}{0.554095in}}%
\pgfpathlineto{\pgfqpoint{3.176698in}{0.554095in}}%
\pgfpathlineto{\pgfqpoint{3.176698in}{0.524170in}}%
\pgfpathclose%
\pgfusepath{fill}%
\end{pgfscope}%
\begin{pgfscope}%
\pgfpathrectangle{\pgfqpoint{0.651412in}{0.524170in}}{\pgfqpoint{4.629690in}{2.558193in}}%
\pgfusepath{clip}%
\pgfsetbuttcap%
\pgfsetmiterjoin%
\definecolor{currentfill}{rgb}{0.003922,0.450980,0.698039}%
\pgfsetfillcolor{currentfill}%
\pgfsetfillopacity{0.700000}%
\pgfsetlinewidth{0.000000pt}%
\definecolor{currentstroke}{rgb}{0.000000,0.000000,0.000000}%
\pgfsetstrokecolor{currentstroke}%
\pgfsetstrokeopacity{0.700000}%
\pgfsetdash{}{0pt}%
\pgfpathmoveto{\pgfqpoint{3.218786in}{0.524170in}}%
\pgfpathlineto{\pgfqpoint{3.260874in}{0.524170in}}%
\pgfpathlineto{\pgfqpoint{3.260874in}{0.579032in}}%
\pgfpathlineto{\pgfqpoint{3.218786in}{0.579032in}}%
\pgfpathlineto{\pgfqpoint{3.218786in}{0.524170in}}%
\pgfpathclose%
\pgfusepath{fill}%
\end{pgfscope}%
\begin{pgfscope}%
\pgfpathrectangle{\pgfqpoint{0.651412in}{0.524170in}}{\pgfqpoint{4.629690in}{2.558193in}}%
\pgfusepath{clip}%
\pgfsetbuttcap%
\pgfsetmiterjoin%
\definecolor{currentfill}{rgb}{0.003922,0.450980,0.698039}%
\pgfsetfillcolor{currentfill}%
\pgfsetfillopacity{0.700000}%
\pgfsetlinewidth{0.000000pt}%
\definecolor{currentstroke}{rgb}{0.000000,0.000000,0.000000}%
\pgfsetstrokecolor{currentstroke}%
\pgfsetstrokeopacity{0.700000}%
\pgfsetdash{}{0pt}%
\pgfpathmoveto{\pgfqpoint{3.260874in}{0.524170in}}%
\pgfpathlineto{\pgfqpoint{3.302962in}{0.524170in}}%
\pgfpathlineto{\pgfqpoint{3.302962in}{0.556588in}}%
\pgfpathlineto{\pgfqpoint{3.260874in}{0.556588in}}%
\pgfpathlineto{\pgfqpoint{3.260874in}{0.524170in}}%
\pgfpathclose%
\pgfusepath{fill}%
\end{pgfscope}%
\begin{pgfscope}%
\pgfpathrectangle{\pgfqpoint{0.651412in}{0.524170in}}{\pgfqpoint{4.629690in}{2.558193in}}%
\pgfusepath{clip}%
\pgfsetbuttcap%
\pgfsetmiterjoin%
\definecolor{currentfill}{rgb}{0.003922,0.450980,0.698039}%
\pgfsetfillcolor{currentfill}%
\pgfsetfillopacity{0.700000}%
\pgfsetlinewidth{0.000000pt}%
\definecolor{currentstroke}{rgb}{0.000000,0.000000,0.000000}%
\pgfsetstrokecolor{currentstroke}%
\pgfsetstrokeopacity{0.700000}%
\pgfsetdash{}{0pt}%
\pgfpathmoveto{\pgfqpoint{3.302962in}{0.524170in}}%
\pgfpathlineto{\pgfqpoint{3.345050in}{0.524170in}}%
\pgfpathlineto{\pgfqpoint{3.345050in}{0.566563in}}%
\pgfpathlineto{\pgfqpoint{3.302962in}{0.566563in}}%
\pgfpathlineto{\pgfqpoint{3.302962in}{0.524170in}}%
\pgfpathclose%
\pgfusepath{fill}%
\end{pgfscope}%
\begin{pgfscope}%
\pgfpathrectangle{\pgfqpoint{0.651412in}{0.524170in}}{\pgfqpoint{4.629690in}{2.558193in}}%
\pgfusepath{clip}%
\pgfsetbuttcap%
\pgfsetmiterjoin%
\definecolor{currentfill}{rgb}{0.003922,0.450980,0.698039}%
\pgfsetfillcolor{currentfill}%
\pgfsetfillopacity{0.700000}%
\pgfsetlinewidth{0.000000pt}%
\definecolor{currentstroke}{rgb}{0.000000,0.000000,0.000000}%
\pgfsetstrokecolor{currentstroke}%
\pgfsetstrokeopacity{0.700000}%
\pgfsetdash{}{0pt}%
\pgfpathmoveto{\pgfqpoint{3.345050in}{0.524170in}}%
\pgfpathlineto{\pgfqpoint{3.387138in}{0.524170in}}%
\pgfpathlineto{\pgfqpoint{3.387138in}{0.554095in}}%
\pgfpathlineto{\pgfqpoint{3.345050in}{0.554095in}}%
\pgfpathlineto{\pgfqpoint{3.345050in}{0.524170in}}%
\pgfpathclose%
\pgfusepath{fill}%
\end{pgfscope}%
\begin{pgfscope}%
\pgfpathrectangle{\pgfqpoint{0.651412in}{0.524170in}}{\pgfqpoint{4.629690in}{2.558193in}}%
\pgfusepath{clip}%
\pgfsetbuttcap%
\pgfsetmiterjoin%
\definecolor{currentfill}{rgb}{0.003922,0.450980,0.698039}%
\pgfsetfillcolor{currentfill}%
\pgfsetfillopacity{0.700000}%
\pgfsetlinewidth{0.000000pt}%
\definecolor{currentstroke}{rgb}{0.000000,0.000000,0.000000}%
\pgfsetstrokecolor{currentstroke}%
\pgfsetstrokeopacity{0.700000}%
\pgfsetdash{}{0pt}%
\pgfpathmoveto{\pgfqpoint{3.387138in}{0.524170in}}%
\pgfpathlineto{\pgfqpoint{3.429226in}{0.524170in}}%
\pgfpathlineto{\pgfqpoint{3.429226in}{0.559082in}}%
\pgfpathlineto{\pgfqpoint{3.387138in}{0.559082in}}%
\pgfpathlineto{\pgfqpoint{3.387138in}{0.524170in}}%
\pgfpathclose%
\pgfusepath{fill}%
\end{pgfscope}%
\begin{pgfscope}%
\pgfpathrectangle{\pgfqpoint{0.651412in}{0.524170in}}{\pgfqpoint{4.629690in}{2.558193in}}%
\pgfusepath{clip}%
\pgfsetbuttcap%
\pgfsetmiterjoin%
\definecolor{currentfill}{rgb}{0.003922,0.450980,0.698039}%
\pgfsetfillcolor{currentfill}%
\pgfsetfillopacity{0.700000}%
\pgfsetlinewidth{0.000000pt}%
\definecolor{currentstroke}{rgb}{0.000000,0.000000,0.000000}%
\pgfsetstrokecolor{currentstroke}%
\pgfsetstrokeopacity{0.700000}%
\pgfsetdash{}{0pt}%
\pgfpathmoveto{\pgfqpoint{3.429226in}{0.524170in}}%
\pgfpathlineto{\pgfqpoint{3.471314in}{0.524170in}}%
\pgfpathlineto{\pgfqpoint{3.471314in}{0.559082in}}%
\pgfpathlineto{\pgfqpoint{3.429226in}{0.559082in}}%
\pgfpathlineto{\pgfqpoint{3.429226in}{0.524170in}}%
\pgfpathclose%
\pgfusepath{fill}%
\end{pgfscope}%
\begin{pgfscope}%
\pgfpathrectangle{\pgfqpoint{0.651412in}{0.524170in}}{\pgfqpoint{4.629690in}{2.558193in}}%
\pgfusepath{clip}%
\pgfsetbuttcap%
\pgfsetmiterjoin%
\definecolor{currentfill}{rgb}{0.003922,0.450980,0.698039}%
\pgfsetfillcolor{currentfill}%
\pgfsetfillopacity{0.700000}%
\pgfsetlinewidth{0.000000pt}%
\definecolor{currentstroke}{rgb}{0.000000,0.000000,0.000000}%
\pgfsetstrokecolor{currentstroke}%
\pgfsetstrokeopacity{0.700000}%
\pgfsetdash{}{0pt}%
\pgfpathmoveto{\pgfqpoint{3.471314in}{0.524170in}}%
\pgfpathlineto{\pgfqpoint{3.513403in}{0.524170in}}%
\pgfpathlineto{\pgfqpoint{3.513403in}{0.566563in}}%
\pgfpathlineto{\pgfqpoint{3.471314in}{0.566563in}}%
\pgfpathlineto{\pgfqpoint{3.471314in}{0.524170in}}%
\pgfpathclose%
\pgfusepath{fill}%
\end{pgfscope}%
\begin{pgfscope}%
\pgfpathrectangle{\pgfqpoint{0.651412in}{0.524170in}}{\pgfqpoint{4.629690in}{2.558193in}}%
\pgfusepath{clip}%
\pgfsetbuttcap%
\pgfsetmiterjoin%
\definecolor{currentfill}{rgb}{0.003922,0.450980,0.698039}%
\pgfsetfillcolor{currentfill}%
\pgfsetfillopacity{0.700000}%
\pgfsetlinewidth{0.000000pt}%
\definecolor{currentstroke}{rgb}{0.000000,0.000000,0.000000}%
\pgfsetstrokecolor{currentstroke}%
\pgfsetstrokeopacity{0.700000}%
\pgfsetdash{}{0pt}%
\pgfpathmoveto{\pgfqpoint{3.513403in}{0.524170in}}%
\pgfpathlineto{\pgfqpoint{3.555491in}{0.524170in}}%
\pgfpathlineto{\pgfqpoint{3.555491in}{0.554095in}}%
\pgfpathlineto{\pgfqpoint{3.513403in}{0.554095in}}%
\pgfpathlineto{\pgfqpoint{3.513403in}{0.524170in}}%
\pgfpathclose%
\pgfusepath{fill}%
\end{pgfscope}%
\begin{pgfscope}%
\pgfpathrectangle{\pgfqpoint{0.651412in}{0.524170in}}{\pgfqpoint{4.629690in}{2.558193in}}%
\pgfusepath{clip}%
\pgfsetbuttcap%
\pgfsetmiterjoin%
\definecolor{currentfill}{rgb}{0.003922,0.450980,0.698039}%
\pgfsetfillcolor{currentfill}%
\pgfsetfillopacity{0.700000}%
\pgfsetlinewidth{0.000000pt}%
\definecolor{currentstroke}{rgb}{0.000000,0.000000,0.000000}%
\pgfsetstrokecolor{currentstroke}%
\pgfsetstrokeopacity{0.700000}%
\pgfsetdash{}{0pt}%
\pgfpathmoveto{\pgfqpoint{3.555491in}{0.524170in}}%
\pgfpathlineto{\pgfqpoint{3.597579in}{0.524170in}}%
\pgfpathlineto{\pgfqpoint{3.597579in}{0.561576in}}%
\pgfpathlineto{\pgfqpoint{3.555491in}{0.561576in}}%
\pgfpathlineto{\pgfqpoint{3.555491in}{0.524170in}}%
\pgfpathclose%
\pgfusepath{fill}%
\end{pgfscope}%
\begin{pgfscope}%
\pgfpathrectangle{\pgfqpoint{0.651412in}{0.524170in}}{\pgfqpoint{4.629690in}{2.558193in}}%
\pgfusepath{clip}%
\pgfsetbuttcap%
\pgfsetmiterjoin%
\definecolor{currentfill}{rgb}{0.003922,0.450980,0.698039}%
\pgfsetfillcolor{currentfill}%
\pgfsetfillopacity{0.700000}%
\pgfsetlinewidth{0.000000pt}%
\definecolor{currentstroke}{rgb}{0.000000,0.000000,0.000000}%
\pgfsetstrokecolor{currentstroke}%
\pgfsetstrokeopacity{0.700000}%
\pgfsetdash{}{0pt}%
\pgfpathmoveto{\pgfqpoint{3.597579in}{0.524170in}}%
\pgfpathlineto{\pgfqpoint{3.639667in}{0.524170in}}%
\pgfpathlineto{\pgfqpoint{3.639667in}{0.549107in}}%
\pgfpathlineto{\pgfqpoint{3.597579in}{0.549107in}}%
\pgfpathlineto{\pgfqpoint{3.597579in}{0.524170in}}%
\pgfpathclose%
\pgfusepath{fill}%
\end{pgfscope}%
\begin{pgfscope}%
\pgfpathrectangle{\pgfqpoint{0.651412in}{0.524170in}}{\pgfqpoint{4.629690in}{2.558193in}}%
\pgfusepath{clip}%
\pgfsetbuttcap%
\pgfsetmiterjoin%
\definecolor{currentfill}{rgb}{0.003922,0.450980,0.698039}%
\pgfsetfillcolor{currentfill}%
\pgfsetfillopacity{0.700000}%
\pgfsetlinewidth{0.000000pt}%
\definecolor{currentstroke}{rgb}{0.000000,0.000000,0.000000}%
\pgfsetstrokecolor{currentstroke}%
\pgfsetstrokeopacity{0.700000}%
\pgfsetdash{}{0pt}%
\pgfpathmoveto{\pgfqpoint{3.639667in}{0.524170in}}%
\pgfpathlineto{\pgfqpoint{3.681755in}{0.524170in}}%
\pgfpathlineto{\pgfqpoint{3.681755in}{0.556588in}}%
\pgfpathlineto{\pgfqpoint{3.639667in}{0.556588in}}%
\pgfpathlineto{\pgfqpoint{3.639667in}{0.524170in}}%
\pgfpathclose%
\pgfusepath{fill}%
\end{pgfscope}%
\begin{pgfscope}%
\pgfpathrectangle{\pgfqpoint{0.651412in}{0.524170in}}{\pgfqpoint{4.629690in}{2.558193in}}%
\pgfusepath{clip}%
\pgfsetbuttcap%
\pgfsetmiterjoin%
\definecolor{currentfill}{rgb}{0.003922,0.450980,0.698039}%
\pgfsetfillcolor{currentfill}%
\pgfsetfillopacity{0.700000}%
\pgfsetlinewidth{0.000000pt}%
\definecolor{currentstroke}{rgb}{0.000000,0.000000,0.000000}%
\pgfsetstrokecolor{currentstroke}%
\pgfsetstrokeopacity{0.700000}%
\pgfsetdash{}{0pt}%
\pgfpathmoveto{\pgfqpoint{3.681755in}{0.524170in}}%
\pgfpathlineto{\pgfqpoint{3.723843in}{0.524170in}}%
\pgfpathlineto{\pgfqpoint{3.723843in}{0.544120in}}%
\pgfpathlineto{\pgfqpoint{3.681755in}{0.544120in}}%
\pgfpathlineto{\pgfqpoint{3.681755in}{0.524170in}}%
\pgfpathclose%
\pgfusepath{fill}%
\end{pgfscope}%
\begin{pgfscope}%
\pgfpathrectangle{\pgfqpoint{0.651412in}{0.524170in}}{\pgfqpoint{4.629690in}{2.558193in}}%
\pgfusepath{clip}%
\pgfsetbuttcap%
\pgfsetmiterjoin%
\definecolor{currentfill}{rgb}{0.003922,0.450980,0.698039}%
\pgfsetfillcolor{currentfill}%
\pgfsetfillopacity{0.700000}%
\pgfsetlinewidth{0.000000pt}%
\definecolor{currentstroke}{rgb}{0.000000,0.000000,0.000000}%
\pgfsetstrokecolor{currentstroke}%
\pgfsetstrokeopacity{0.700000}%
\pgfsetdash{}{0pt}%
\pgfpathmoveto{\pgfqpoint{3.723843in}{0.524170in}}%
\pgfpathlineto{\pgfqpoint{3.765931in}{0.524170in}}%
\pgfpathlineto{\pgfqpoint{3.765931in}{0.549107in}}%
\pgfpathlineto{\pgfqpoint{3.723843in}{0.549107in}}%
\pgfpathlineto{\pgfqpoint{3.723843in}{0.524170in}}%
\pgfpathclose%
\pgfusepath{fill}%
\end{pgfscope}%
\begin{pgfscope}%
\pgfpathrectangle{\pgfqpoint{0.651412in}{0.524170in}}{\pgfqpoint{4.629690in}{2.558193in}}%
\pgfusepath{clip}%
\pgfsetbuttcap%
\pgfsetmiterjoin%
\definecolor{currentfill}{rgb}{0.003922,0.450980,0.698039}%
\pgfsetfillcolor{currentfill}%
\pgfsetfillopacity{0.700000}%
\pgfsetlinewidth{0.000000pt}%
\definecolor{currentstroke}{rgb}{0.000000,0.000000,0.000000}%
\pgfsetstrokecolor{currentstroke}%
\pgfsetstrokeopacity{0.700000}%
\pgfsetdash{}{0pt}%
\pgfpathmoveto{\pgfqpoint{3.765931in}{0.524170in}}%
\pgfpathlineto{\pgfqpoint{3.808019in}{0.524170in}}%
\pgfpathlineto{\pgfqpoint{3.808019in}{0.554095in}}%
\pgfpathlineto{\pgfqpoint{3.765931in}{0.554095in}}%
\pgfpathlineto{\pgfqpoint{3.765931in}{0.524170in}}%
\pgfpathclose%
\pgfusepath{fill}%
\end{pgfscope}%
\begin{pgfscope}%
\pgfpathrectangle{\pgfqpoint{0.651412in}{0.524170in}}{\pgfqpoint{4.629690in}{2.558193in}}%
\pgfusepath{clip}%
\pgfsetbuttcap%
\pgfsetmiterjoin%
\definecolor{currentfill}{rgb}{0.003922,0.450980,0.698039}%
\pgfsetfillcolor{currentfill}%
\pgfsetfillopacity{0.700000}%
\pgfsetlinewidth{0.000000pt}%
\definecolor{currentstroke}{rgb}{0.000000,0.000000,0.000000}%
\pgfsetstrokecolor{currentstroke}%
\pgfsetstrokeopacity{0.700000}%
\pgfsetdash{}{0pt}%
\pgfpathmoveto{\pgfqpoint{3.808019in}{0.524170in}}%
\pgfpathlineto{\pgfqpoint{3.850107in}{0.524170in}}%
\pgfpathlineto{\pgfqpoint{3.850107in}{0.539132in}}%
\pgfpathlineto{\pgfqpoint{3.808019in}{0.539132in}}%
\pgfpathlineto{\pgfqpoint{3.808019in}{0.524170in}}%
\pgfpathclose%
\pgfusepath{fill}%
\end{pgfscope}%
\begin{pgfscope}%
\pgfpathrectangle{\pgfqpoint{0.651412in}{0.524170in}}{\pgfqpoint{4.629690in}{2.558193in}}%
\pgfusepath{clip}%
\pgfsetbuttcap%
\pgfsetmiterjoin%
\definecolor{currentfill}{rgb}{0.003922,0.450980,0.698039}%
\pgfsetfillcolor{currentfill}%
\pgfsetfillopacity{0.700000}%
\pgfsetlinewidth{0.000000pt}%
\definecolor{currentstroke}{rgb}{0.000000,0.000000,0.000000}%
\pgfsetstrokecolor{currentstroke}%
\pgfsetstrokeopacity{0.700000}%
\pgfsetdash{}{0pt}%
\pgfpathmoveto{\pgfqpoint{3.850107in}{0.524170in}}%
\pgfpathlineto{\pgfqpoint{3.892195in}{0.524170in}}%
\pgfpathlineto{\pgfqpoint{3.892195in}{0.549107in}}%
\pgfpathlineto{\pgfqpoint{3.850107in}{0.549107in}}%
\pgfpathlineto{\pgfqpoint{3.850107in}{0.524170in}}%
\pgfpathclose%
\pgfusepath{fill}%
\end{pgfscope}%
\begin{pgfscope}%
\pgfpathrectangle{\pgfqpoint{0.651412in}{0.524170in}}{\pgfqpoint{4.629690in}{2.558193in}}%
\pgfusepath{clip}%
\pgfsetbuttcap%
\pgfsetmiterjoin%
\definecolor{currentfill}{rgb}{0.003922,0.450980,0.698039}%
\pgfsetfillcolor{currentfill}%
\pgfsetfillopacity{0.700000}%
\pgfsetlinewidth{0.000000pt}%
\definecolor{currentstroke}{rgb}{0.000000,0.000000,0.000000}%
\pgfsetstrokecolor{currentstroke}%
\pgfsetstrokeopacity{0.700000}%
\pgfsetdash{}{0pt}%
\pgfpathmoveto{\pgfqpoint{3.892195in}{0.524170in}}%
\pgfpathlineto{\pgfqpoint{3.934283in}{0.524170in}}%
\pgfpathlineto{\pgfqpoint{3.934283in}{0.551601in}}%
\pgfpathlineto{\pgfqpoint{3.892195in}{0.551601in}}%
\pgfpathlineto{\pgfqpoint{3.892195in}{0.524170in}}%
\pgfpathclose%
\pgfusepath{fill}%
\end{pgfscope}%
\begin{pgfscope}%
\pgfpathrectangle{\pgfqpoint{0.651412in}{0.524170in}}{\pgfqpoint{4.629690in}{2.558193in}}%
\pgfusepath{clip}%
\pgfsetbuttcap%
\pgfsetmiterjoin%
\definecolor{currentfill}{rgb}{0.003922,0.450980,0.698039}%
\pgfsetfillcolor{currentfill}%
\pgfsetfillopacity{0.700000}%
\pgfsetlinewidth{0.000000pt}%
\definecolor{currentstroke}{rgb}{0.000000,0.000000,0.000000}%
\pgfsetstrokecolor{currentstroke}%
\pgfsetstrokeopacity{0.700000}%
\pgfsetdash{}{0pt}%
\pgfpathmoveto{\pgfqpoint{3.934283in}{0.524170in}}%
\pgfpathlineto{\pgfqpoint{3.976372in}{0.524170in}}%
\pgfpathlineto{\pgfqpoint{3.976372in}{0.541626in}}%
\pgfpathlineto{\pgfqpoint{3.934283in}{0.541626in}}%
\pgfpathlineto{\pgfqpoint{3.934283in}{0.524170in}}%
\pgfpathclose%
\pgfusepath{fill}%
\end{pgfscope}%
\begin{pgfscope}%
\pgfpathrectangle{\pgfqpoint{0.651412in}{0.524170in}}{\pgfqpoint{4.629690in}{2.558193in}}%
\pgfusepath{clip}%
\pgfsetbuttcap%
\pgfsetmiterjoin%
\definecolor{currentfill}{rgb}{0.003922,0.450980,0.698039}%
\pgfsetfillcolor{currentfill}%
\pgfsetfillopacity{0.700000}%
\pgfsetlinewidth{0.000000pt}%
\definecolor{currentstroke}{rgb}{0.000000,0.000000,0.000000}%
\pgfsetstrokecolor{currentstroke}%
\pgfsetstrokeopacity{0.700000}%
\pgfsetdash{}{0pt}%
\pgfpathmoveto{\pgfqpoint{3.976372in}{0.524170in}}%
\pgfpathlineto{\pgfqpoint{4.018460in}{0.524170in}}%
\pgfpathlineto{\pgfqpoint{4.018460in}{0.556588in}}%
\pgfpathlineto{\pgfqpoint{3.976372in}{0.556588in}}%
\pgfpathlineto{\pgfqpoint{3.976372in}{0.524170in}}%
\pgfpathclose%
\pgfusepath{fill}%
\end{pgfscope}%
\begin{pgfscope}%
\pgfpathrectangle{\pgfqpoint{0.651412in}{0.524170in}}{\pgfqpoint{4.629690in}{2.558193in}}%
\pgfusepath{clip}%
\pgfsetbuttcap%
\pgfsetmiterjoin%
\definecolor{currentfill}{rgb}{0.003922,0.450980,0.698039}%
\pgfsetfillcolor{currentfill}%
\pgfsetfillopacity{0.700000}%
\pgfsetlinewidth{0.000000pt}%
\definecolor{currentstroke}{rgb}{0.000000,0.000000,0.000000}%
\pgfsetstrokecolor{currentstroke}%
\pgfsetstrokeopacity{0.700000}%
\pgfsetdash{}{0pt}%
\pgfpathmoveto{\pgfqpoint{4.018460in}{0.524170in}}%
\pgfpathlineto{\pgfqpoint{4.060548in}{0.524170in}}%
\pgfpathlineto{\pgfqpoint{4.060548in}{0.536638in}}%
\pgfpathlineto{\pgfqpoint{4.018460in}{0.536638in}}%
\pgfpathlineto{\pgfqpoint{4.018460in}{0.524170in}}%
\pgfpathclose%
\pgfusepath{fill}%
\end{pgfscope}%
\begin{pgfscope}%
\pgfpathrectangle{\pgfqpoint{0.651412in}{0.524170in}}{\pgfqpoint{4.629690in}{2.558193in}}%
\pgfusepath{clip}%
\pgfsetbuttcap%
\pgfsetmiterjoin%
\definecolor{currentfill}{rgb}{0.003922,0.450980,0.698039}%
\pgfsetfillcolor{currentfill}%
\pgfsetfillopacity{0.700000}%
\pgfsetlinewidth{0.000000pt}%
\definecolor{currentstroke}{rgb}{0.000000,0.000000,0.000000}%
\pgfsetstrokecolor{currentstroke}%
\pgfsetstrokeopacity{0.700000}%
\pgfsetdash{}{0pt}%
\pgfpathmoveto{\pgfqpoint{4.060548in}{0.524170in}}%
\pgfpathlineto{\pgfqpoint{4.102636in}{0.524170in}}%
\pgfpathlineto{\pgfqpoint{4.102636in}{0.549107in}}%
\pgfpathlineto{\pgfqpoint{4.060548in}{0.549107in}}%
\pgfpathlineto{\pgfqpoint{4.060548in}{0.524170in}}%
\pgfpathclose%
\pgfusepath{fill}%
\end{pgfscope}%
\begin{pgfscope}%
\pgfpathrectangle{\pgfqpoint{0.651412in}{0.524170in}}{\pgfqpoint{4.629690in}{2.558193in}}%
\pgfusepath{clip}%
\pgfsetbuttcap%
\pgfsetmiterjoin%
\definecolor{currentfill}{rgb}{0.003922,0.450980,0.698039}%
\pgfsetfillcolor{currentfill}%
\pgfsetfillopacity{0.700000}%
\pgfsetlinewidth{0.000000pt}%
\definecolor{currentstroke}{rgb}{0.000000,0.000000,0.000000}%
\pgfsetstrokecolor{currentstroke}%
\pgfsetstrokeopacity{0.700000}%
\pgfsetdash{}{0pt}%
\pgfpathmoveto{\pgfqpoint{4.102636in}{0.524170in}}%
\pgfpathlineto{\pgfqpoint{4.144724in}{0.524170in}}%
\pgfpathlineto{\pgfqpoint{4.144724in}{0.529157in}}%
\pgfpathlineto{\pgfqpoint{4.102636in}{0.529157in}}%
\pgfpathlineto{\pgfqpoint{4.102636in}{0.524170in}}%
\pgfpathclose%
\pgfusepath{fill}%
\end{pgfscope}%
\begin{pgfscope}%
\pgfpathrectangle{\pgfqpoint{0.651412in}{0.524170in}}{\pgfqpoint{4.629690in}{2.558193in}}%
\pgfusepath{clip}%
\pgfsetbuttcap%
\pgfsetmiterjoin%
\definecolor{currentfill}{rgb}{0.003922,0.450980,0.698039}%
\pgfsetfillcolor{currentfill}%
\pgfsetfillopacity{0.700000}%
\pgfsetlinewidth{0.000000pt}%
\definecolor{currentstroke}{rgb}{0.000000,0.000000,0.000000}%
\pgfsetstrokecolor{currentstroke}%
\pgfsetstrokeopacity{0.700000}%
\pgfsetdash{}{0pt}%
\pgfpathmoveto{\pgfqpoint{4.144724in}{0.524170in}}%
\pgfpathlineto{\pgfqpoint{4.186812in}{0.524170in}}%
\pgfpathlineto{\pgfqpoint{4.186812in}{0.534145in}}%
\pgfpathlineto{\pgfqpoint{4.144724in}{0.534145in}}%
\pgfpathlineto{\pgfqpoint{4.144724in}{0.524170in}}%
\pgfpathclose%
\pgfusepath{fill}%
\end{pgfscope}%
\begin{pgfscope}%
\pgfpathrectangle{\pgfqpoint{0.651412in}{0.524170in}}{\pgfqpoint{4.629690in}{2.558193in}}%
\pgfusepath{clip}%
\pgfsetbuttcap%
\pgfsetmiterjoin%
\definecolor{currentfill}{rgb}{0.003922,0.450980,0.698039}%
\pgfsetfillcolor{currentfill}%
\pgfsetfillopacity{0.700000}%
\pgfsetlinewidth{0.000000pt}%
\definecolor{currentstroke}{rgb}{0.000000,0.000000,0.000000}%
\pgfsetstrokecolor{currentstroke}%
\pgfsetstrokeopacity{0.700000}%
\pgfsetdash{}{0pt}%
\pgfpathmoveto{\pgfqpoint{4.186812in}{0.524170in}}%
\pgfpathlineto{\pgfqpoint{4.228900in}{0.524170in}}%
\pgfpathlineto{\pgfqpoint{4.228900in}{0.539132in}}%
\pgfpathlineto{\pgfqpoint{4.186812in}{0.539132in}}%
\pgfpathlineto{\pgfqpoint{4.186812in}{0.524170in}}%
\pgfpathclose%
\pgfusepath{fill}%
\end{pgfscope}%
\begin{pgfscope}%
\pgfpathrectangle{\pgfqpoint{0.651412in}{0.524170in}}{\pgfqpoint{4.629690in}{2.558193in}}%
\pgfusepath{clip}%
\pgfsetbuttcap%
\pgfsetmiterjoin%
\definecolor{currentfill}{rgb}{0.003922,0.450980,0.698039}%
\pgfsetfillcolor{currentfill}%
\pgfsetfillopacity{0.700000}%
\pgfsetlinewidth{0.000000pt}%
\definecolor{currentstroke}{rgb}{0.000000,0.000000,0.000000}%
\pgfsetstrokecolor{currentstroke}%
\pgfsetstrokeopacity{0.700000}%
\pgfsetdash{}{0pt}%
\pgfpathmoveto{\pgfqpoint{4.228900in}{0.524170in}}%
\pgfpathlineto{\pgfqpoint{4.270988in}{0.524170in}}%
\pgfpathlineto{\pgfqpoint{4.270988in}{0.536638in}}%
\pgfpathlineto{\pgfqpoint{4.228900in}{0.536638in}}%
\pgfpathlineto{\pgfqpoint{4.228900in}{0.524170in}}%
\pgfpathclose%
\pgfusepath{fill}%
\end{pgfscope}%
\begin{pgfscope}%
\pgfpathrectangle{\pgfqpoint{0.651412in}{0.524170in}}{\pgfqpoint{4.629690in}{2.558193in}}%
\pgfusepath{clip}%
\pgfsetbuttcap%
\pgfsetmiterjoin%
\definecolor{currentfill}{rgb}{0.003922,0.450980,0.698039}%
\pgfsetfillcolor{currentfill}%
\pgfsetfillopacity{0.700000}%
\pgfsetlinewidth{0.000000pt}%
\definecolor{currentstroke}{rgb}{0.000000,0.000000,0.000000}%
\pgfsetstrokecolor{currentstroke}%
\pgfsetstrokeopacity{0.700000}%
\pgfsetdash{}{0pt}%
\pgfpathmoveto{\pgfqpoint{4.270988in}{0.524170in}}%
\pgfpathlineto{\pgfqpoint{4.313076in}{0.524170in}}%
\pgfpathlineto{\pgfqpoint{4.313076in}{0.539132in}}%
\pgfpathlineto{\pgfqpoint{4.270988in}{0.539132in}}%
\pgfpathlineto{\pgfqpoint{4.270988in}{0.524170in}}%
\pgfpathclose%
\pgfusepath{fill}%
\end{pgfscope}%
\begin{pgfscope}%
\pgfpathrectangle{\pgfqpoint{0.651412in}{0.524170in}}{\pgfqpoint{4.629690in}{2.558193in}}%
\pgfusepath{clip}%
\pgfsetbuttcap%
\pgfsetmiterjoin%
\definecolor{currentfill}{rgb}{0.003922,0.450980,0.698039}%
\pgfsetfillcolor{currentfill}%
\pgfsetfillopacity{0.700000}%
\pgfsetlinewidth{0.000000pt}%
\definecolor{currentstroke}{rgb}{0.000000,0.000000,0.000000}%
\pgfsetstrokecolor{currentstroke}%
\pgfsetstrokeopacity{0.700000}%
\pgfsetdash{}{0pt}%
\pgfpathmoveto{\pgfqpoint{4.313076in}{0.524170in}}%
\pgfpathlineto{\pgfqpoint{4.355164in}{0.524170in}}%
\pgfpathlineto{\pgfqpoint{4.355164in}{0.531651in}}%
\pgfpathlineto{\pgfqpoint{4.313076in}{0.531651in}}%
\pgfpathlineto{\pgfqpoint{4.313076in}{0.524170in}}%
\pgfpathclose%
\pgfusepath{fill}%
\end{pgfscope}%
\begin{pgfscope}%
\pgfpathrectangle{\pgfqpoint{0.651412in}{0.524170in}}{\pgfqpoint{4.629690in}{2.558193in}}%
\pgfusepath{clip}%
\pgfsetbuttcap%
\pgfsetmiterjoin%
\definecolor{currentfill}{rgb}{0.003922,0.450980,0.698039}%
\pgfsetfillcolor{currentfill}%
\pgfsetfillopacity{0.700000}%
\pgfsetlinewidth{0.000000pt}%
\definecolor{currentstroke}{rgb}{0.000000,0.000000,0.000000}%
\pgfsetstrokecolor{currentstroke}%
\pgfsetstrokeopacity{0.700000}%
\pgfsetdash{}{0pt}%
\pgfpathmoveto{\pgfqpoint{4.355164in}{0.524170in}}%
\pgfpathlineto{\pgfqpoint{4.397253in}{0.524170in}}%
\pgfpathlineto{\pgfqpoint{4.397253in}{0.546613in}}%
\pgfpathlineto{\pgfqpoint{4.355164in}{0.546613in}}%
\pgfpathlineto{\pgfqpoint{4.355164in}{0.524170in}}%
\pgfpathclose%
\pgfusepath{fill}%
\end{pgfscope}%
\begin{pgfscope}%
\pgfpathrectangle{\pgfqpoint{0.651412in}{0.524170in}}{\pgfqpoint{4.629690in}{2.558193in}}%
\pgfusepath{clip}%
\pgfsetbuttcap%
\pgfsetmiterjoin%
\definecolor{currentfill}{rgb}{0.003922,0.450980,0.698039}%
\pgfsetfillcolor{currentfill}%
\pgfsetfillopacity{0.700000}%
\pgfsetlinewidth{0.000000pt}%
\definecolor{currentstroke}{rgb}{0.000000,0.000000,0.000000}%
\pgfsetstrokecolor{currentstroke}%
\pgfsetstrokeopacity{0.700000}%
\pgfsetdash{}{0pt}%
\pgfpathmoveto{\pgfqpoint{4.397253in}{0.524170in}}%
\pgfpathlineto{\pgfqpoint{4.439341in}{0.524170in}}%
\pgfpathlineto{\pgfqpoint{4.439341in}{0.539132in}}%
\pgfpathlineto{\pgfqpoint{4.397253in}{0.539132in}}%
\pgfpathlineto{\pgfqpoint{4.397253in}{0.524170in}}%
\pgfpathclose%
\pgfusepath{fill}%
\end{pgfscope}%
\begin{pgfscope}%
\pgfpathrectangle{\pgfqpoint{0.651412in}{0.524170in}}{\pgfqpoint{4.629690in}{2.558193in}}%
\pgfusepath{clip}%
\pgfsetbuttcap%
\pgfsetmiterjoin%
\definecolor{currentfill}{rgb}{0.003922,0.450980,0.698039}%
\pgfsetfillcolor{currentfill}%
\pgfsetfillopacity{0.700000}%
\pgfsetlinewidth{0.000000pt}%
\definecolor{currentstroke}{rgb}{0.000000,0.000000,0.000000}%
\pgfsetstrokecolor{currentstroke}%
\pgfsetstrokeopacity{0.700000}%
\pgfsetdash{}{0pt}%
\pgfpathmoveto{\pgfqpoint{4.439341in}{0.524170in}}%
\pgfpathlineto{\pgfqpoint{4.481429in}{0.524170in}}%
\pgfpathlineto{\pgfqpoint{4.481429in}{0.549107in}}%
\pgfpathlineto{\pgfqpoint{4.439341in}{0.549107in}}%
\pgfpathlineto{\pgfqpoint{4.439341in}{0.524170in}}%
\pgfpathclose%
\pgfusepath{fill}%
\end{pgfscope}%
\begin{pgfscope}%
\pgfpathrectangle{\pgfqpoint{0.651412in}{0.524170in}}{\pgfqpoint{4.629690in}{2.558193in}}%
\pgfusepath{clip}%
\pgfsetbuttcap%
\pgfsetmiterjoin%
\definecolor{currentfill}{rgb}{0.003922,0.450980,0.698039}%
\pgfsetfillcolor{currentfill}%
\pgfsetfillopacity{0.700000}%
\pgfsetlinewidth{0.000000pt}%
\definecolor{currentstroke}{rgb}{0.000000,0.000000,0.000000}%
\pgfsetstrokecolor{currentstroke}%
\pgfsetstrokeopacity{0.700000}%
\pgfsetdash{}{0pt}%
\pgfpathmoveto{\pgfqpoint{4.481429in}{0.524170in}}%
\pgfpathlineto{\pgfqpoint{4.523517in}{0.524170in}}%
\pgfpathlineto{\pgfqpoint{4.523517in}{0.551601in}}%
\pgfpathlineto{\pgfqpoint{4.481429in}{0.551601in}}%
\pgfpathlineto{\pgfqpoint{4.481429in}{0.524170in}}%
\pgfpathclose%
\pgfusepath{fill}%
\end{pgfscope}%
\begin{pgfscope}%
\pgfpathrectangle{\pgfqpoint{0.651412in}{0.524170in}}{\pgfqpoint{4.629690in}{2.558193in}}%
\pgfusepath{clip}%
\pgfsetbuttcap%
\pgfsetmiterjoin%
\definecolor{currentfill}{rgb}{0.003922,0.450980,0.698039}%
\pgfsetfillcolor{currentfill}%
\pgfsetfillopacity{0.700000}%
\pgfsetlinewidth{0.000000pt}%
\definecolor{currentstroke}{rgb}{0.000000,0.000000,0.000000}%
\pgfsetstrokecolor{currentstroke}%
\pgfsetstrokeopacity{0.700000}%
\pgfsetdash{}{0pt}%
\pgfpathmoveto{\pgfqpoint{4.523517in}{0.524170in}}%
\pgfpathlineto{\pgfqpoint{4.565605in}{0.524170in}}%
\pgfpathlineto{\pgfqpoint{4.565605in}{0.539132in}}%
\pgfpathlineto{\pgfqpoint{4.523517in}{0.539132in}}%
\pgfpathlineto{\pgfqpoint{4.523517in}{0.524170in}}%
\pgfpathclose%
\pgfusepath{fill}%
\end{pgfscope}%
\begin{pgfscope}%
\pgfpathrectangle{\pgfqpoint{0.651412in}{0.524170in}}{\pgfqpoint{4.629690in}{2.558193in}}%
\pgfusepath{clip}%
\pgfsetbuttcap%
\pgfsetmiterjoin%
\definecolor{currentfill}{rgb}{0.003922,0.450980,0.698039}%
\pgfsetfillcolor{currentfill}%
\pgfsetfillopacity{0.700000}%
\pgfsetlinewidth{0.000000pt}%
\definecolor{currentstroke}{rgb}{0.000000,0.000000,0.000000}%
\pgfsetstrokecolor{currentstroke}%
\pgfsetstrokeopacity{0.700000}%
\pgfsetdash{}{0pt}%
\pgfpathmoveto{\pgfqpoint{4.565605in}{0.524170in}}%
\pgfpathlineto{\pgfqpoint{4.607693in}{0.524170in}}%
\pgfpathlineto{\pgfqpoint{4.607693in}{0.536638in}}%
\pgfpathlineto{\pgfqpoint{4.565605in}{0.536638in}}%
\pgfpathlineto{\pgfqpoint{4.565605in}{0.524170in}}%
\pgfpathclose%
\pgfusepath{fill}%
\end{pgfscope}%
\begin{pgfscope}%
\pgfpathrectangle{\pgfqpoint{0.651412in}{0.524170in}}{\pgfqpoint{4.629690in}{2.558193in}}%
\pgfusepath{clip}%
\pgfsetbuttcap%
\pgfsetmiterjoin%
\definecolor{currentfill}{rgb}{0.003922,0.450980,0.698039}%
\pgfsetfillcolor{currentfill}%
\pgfsetfillopacity{0.700000}%
\pgfsetlinewidth{0.000000pt}%
\definecolor{currentstroke}{rgb}{0.000000,0.000000,0.000000}%
\pgfsetstrokecolor{currentstroke}%
\pgfsetstrokeopacity{0.700000}%
\pgfsetdash{}{0pt}%
\pgfpathmoveto{\pgfqpoint{4.607693in}{0.524170in}}%
\pgfpathlineto{\pgfqpoint{4.649781in}{0.524170in}}%
\pgfpathlineto{\pgfqpoint{4.649781in}{0.549107in}}%
\pgfpathlineto{\pgfqpoint{4.607693in}{0.549107in}}%
\pgfpathlineto{\pgfqpoint{4.607693in}{0.524170in}}%
\pgfpathclose%
\pgfusepath{fill}%
\end{pgfscope}%
\begin{pgfscope}%
\pgfpathrectangle{\pgfqpoint{0.651412in}{0.524170in}}{\pgfqpoint{4.629690in}{2.558193in}}%
\pgfusepath{clip}%
\pgfsetbuttcap%
\pgfsetmiterjoin%
\definecolor{currentfill}{rgb}{0.003922,0.450980,0.698039}%
\pgfsetfillcolor{currentfill}%
\pgfsetfillopacity{0.700000}%
\pgfsetlinewidth{0.000000pt}%
\definecolor{currentstroke}{rgb}{0.000000,0.000000,0.000000}%
\pgfsetstrokecolor{currentstroke}%
\pgfsetstrokeopacity{0.700000}%
\pgfsetdash{}{0pt}%
\pgfpathmoveto{\pgfqpoint{4.649781in}{0.524170in}}%
\pgfpathlineto{\pgfqpoint{4.691869in}{0.524170in}}%
\pgfpathlineto{\pgfqpoint{4.691869in}{0.531651in}}%
\pgfpathlineto{\pgfqpoint{4.649781in}{0.531651in}}%
\pgfpathlineto{\pgfqpoint{4.649781in}{0.524170in}}%
\pgfpathclose%
\pgfusepath{fill}%
\end{pgfscope}%
\begin{pgfscope}%
\pgfpathrectangle{\pgfqpoint{0.651412in}{0.524170in}}{\pgfqpoint{4.629690in}{2.558193in}}%
\pgfusepath{clip}%
\pgfsetbuttcap%
\pgfsetmiterjoin%
\definecolor{currentfill}{rgb}{0.003922,0.450980,0.698039}%
\pgfsetfillcolor{currentfill}%
\pgfsetfillopacity{0.700000}%
\pgfsetlinewidth{0.000000pt}%
\definecolor{currentstroke}{rgb}{0.000000,0.000000,0.000000}%
\pgfsetstrokecolor{currentstroke}%
\pgfsetstrokeopacity{0.700000}%
\pgfsetdash{}{0pt}%
\pgfpathmoveto{\pgfqpoint{4.691869in}{0.524170in}}%
\pgfpathlineto{\pgfqpoint{4.733957in}{0.524170in}}%
\pgfpathlineto{\pgfqpoint{4.733957in}{0.526664in}}%
\pgfpathlineto{\pgfqpoint{4.691869in}{0.526664in}}%
\pgfpathlineto{\pgfqpoint{4.691869in}{0.524170in}}%
\pgfpathclose%
\pgfusepath{fill}%
\end{pgfscope}%
\begin{pgfscope}%
\pgfpathrectangle{\pgfqpoint{0.651412in}{0.524170in}}{\pgfqpoint{4.629690in}{2.558193in}}%
\pgfusepath{clip}%
\pgfsetbuttcap%
\pgfsetmiterjoin%
\definecolor{currentfill}{rgb}{0.003922,0.450980,0.698039}%
\pgfsetfillcolor{currentfill}%
\pgfsetfillopacity{0.700000}%
\pgfsetlinewidth{0.000000pt}%
\definecolor{currentstroke}{rgb}{0.000000,0.000000,0.000000}%
\pgfsetstrokecolor{currentstroke}%
\pgfsetstrokeopacity{0.700000}%
\pgfsetdash{}{0pt}%
\pgfpathmoveto{\pgfqpoint{4.733957in}{0.524170in}}%
\pgfpathlineto{\pgfqpoint{4.776045in}{0.524170in}}%
\pgfpathlineto{\pgfqpoint{4.776045in}{0.536638in}}%
\pgfpathlineto{\pgfqpoint{4.733957in}{0.536638in}}%
\pgfpathlineto{\pgfqpoint{4.733957in}{0.524170in}}%
\pgfpathclose%
\pgfusepath{fill}%
\end{pgfscope}%
\begin{pgfscope}%
\pgfpathrectangle{\pgfqpoint{0.651412in}{0.524170in}}{\pgfqpoint{4.629690in}{2.558193in}}%
\pgfusepath{clip}%
\pgfsetbuttcap%
\pgfsetmiterjoin%
\definecolor{currentfill}{rgb}{0.003922,0.450980,0.698039}%
\pgfsetfillcolor{currentfill}%
\pgfsetfillopacity{0.700000}%
\pgfsetlinewidth{0.000000pt}%
\definecolor{currentstroke}{rgb}{0.000000,0.000000,0.000000}%
\pgfsetstrokecolor{currentstroke}%
\pgfsetstrokeopacity{0.700000}%
\pgfsetdash{}{0pt}%
\pgfpathmoveto{\pgfqpoint{4.776045in}{0.524170in}}%
\pgfpathlineto{\pgfqpoint{4.818133in}{0.524170in}}%
\pgfpathlineto{\pgfqpoint{4.818133in}{0.549107in}}%
\pgfpathlineto{\pgfqpoint{4.776045in}{0.549107in}}%
\pgfpathlineto{\pgfqpoint{4.776045in}{0.524170in}}%
\pgfpathclose%
\pgfusepath{fill}%
\end{pgfscope}%
\begin{pgfscope}%
\pgfpathrectangle{\pgfqpoint{0.651412in}{0.524170in}}{\pgfqpoint{4.629690in}{2.558193in}}%
\pgfusepath{clip}%
\pgfsetbuttcap%
\pgfsetmiterjoin%
\definecolor{currentfill}{rgb}{0.003922,0.450980,0.698039}%
\pgfsetfillcolor{currentfill}%
\pgfsetfillopacity{0.700000}%
\pgfsetlinewidth{0.000000pt}%
\definecolor{currentstroke}{rgb}{0.000000,0.000000,0.000000}%
\pgfsetstrokecolor{currentstroke}%
\pgfsetstrokeopacity{0.700000}%
\pgfsetdash{}{0pt}%
\pgfpathmoveto{\pgfqpoint{4.818133in}{0.524170in}}%
\pgfpathlineto{\pgfqpoint{4.860222in}{0.524170in}}%
\pgfpathlineto{\pgfqpoint{4.860222in}{0.536638in}}%
\pgfpathlineto{\pgfqpoint{4.818133in}{0.536638in}}%
\pgfpathlineto{\pgfqpoint{4.818133in}{0.524170in}}%
\pgfpathclose%
\pgfusepath{fill}%
\end{pgfscope}%
\begin{pgfscope}%
\pgfpathrectangle{\pgfqpoint{0.651412in}{0.524170in}}{\pgfqpoint{4.629690in}{2.558193in}}%
\pgfusepath{clip}%
\pgfsetbuttcap%
\pgfsetmiterjoin%
\definecolor{currentfill}{rgb}{0.003922,0.450980,0.698039}%
\pgfsetfillcolor{currentfill}%
\pgfsetfillopacity{0.700000}%
\pgfsetlinewidth{0.000000pt}%
\definecolor{currentstroke}{rgb}{0.000000,0.000000,0.000000}%
\pgfsetstrokecolor{currentstroke}%
\pgfsetstrokeopacity{0.700000}%
\pgfsetdash{}{0pt}%
\pgfpathmoveto{\pgfqpoint{4.860222in}{0.524170in}}%
\pgfpathlineto{\pgfqpoint{4.902310in}{0.524170in}}%
\pgfpathlineto{\pgfqpoint{4.902310in}{0.529157in}}%
\pgfpathlineto{\pgfqpoint{4.860222in}{0.529157in}}%
\pgfpathlineto{\pgfqpoint{4.860222in}{0.524170in}}%
\pgfpathclose%
\pgfusepath{fill}%
\end{pgfscope}%
\begin{pgfscope}%
\pgfpathrectangle{\pgfqpoint{0.651412in}{0.524170in}}{\pgfqpoint{4.629690in}{2.558193in}}%
\pgfusepath{clip}%
\pgfsetbuttcap%
\pgfsetmiterjoin%
\definecolor{currentfill}{rgb}{0.003922,0.450980,0.698039}%
\pgfsetfillcolor{currentfill}%
\pgfsetfillopacity{0.700000}%
\pgfsetlinewidth{0.000000pt}%
\definecolor{currentstroke}{rgb}{0.000000,0.000000,0.000000}%
\pgfsetstrokecolor{currentstroke}%
\pgfsetstrokeopacity{0.700000}%
\pgfsetdash{}{0pt}%
\pgfpathmoveto{\pgfqpoint{4.902310in}{0.524170in}}%
\pgfpathlineto{\pgfqpoint{4.944398in}{0.524170in}}%
\pgfpathlineto{\pgfqpoint{4.944398in}{0.541626in}}%
\pgfpathlineto{\pgfqpoint{4.902310in}{0.541626in}}%
\pgfpathlineto{\pgfqpoint{4.902310in}{0.524170in}}%
\pgfpathclose%
\pgfusepath{fill}%
\end{pgfscope}%
\begin{pgfscope}%
\pgfpathrectangle{\pgfqpoint{0.651412in}{0.524170in}}{\pgfqpoint{4.629690in}{2.558193in}}%
\pgfusepath{clip}%
\pgfsetbuttcap%
\pgfsetmiterjoin%
\definecolor{currentfill}{rgb}{0.003922,0.450980,0.698039}%
\pgfsetfillcolor{currentfill}%
\pgfsetfillopacity{0.700000}%
\pgfsetlinewidth{0.000000pt}%
\definecolor{currentstroke}{rgb}{0.000000,0.000000,0.000000}%
\pgfsetstrokecolor{currentstroke}%
\pgfsetstrokeopacity{0.700000}%
\pgfsetdash{}{0pt}%
\pgfpathmoveto{\pgfqpoint{4.944398in}{0.524170in}}%
\pgfpathlineto{\pgfqpoint{4.986486in}{0.524170in}}%
\pgfpathlineto{\pgfqpoint{4.986486in}{0.529157in}}%
\pgfpathlineto{\pgfqpoint{4.944398in}{0.529157in}}%
\pgfpathlineto{\pgfqpoint{4.944398in}{0.524170in}}%
\pgfpathclose%
\pgfusepath{fill}%
\end{pgfscope}%
\begin{pgfscope}%
\pgfpathrectangle{\pgfqpoint{0.651412in}{0.524170in}}{\pgfqpoint{4.629690in}{2.558193in}}%
\pgfusepath{clip}%
\pgfsetbuttcap%
\pgfsetmiterjoin%
\definecolor{currentfill}{rgb}{0.003922,0.450980,0.698039}%
\pgfsetfillcolor{currentfill}%
\pgfsetfillopacity{0.700000}%
\pgfsetlinewidth{0.000000pt}%
\definecolor{currentstroke}{rgb}{0.000000,0.000000,0.000000}%
\pgfsetstrokecolor{currentstroke}%
\pgfsetstrokeopacity{0.700000}%
\pgfsetdash{}{0pt}%
\pgfpathmoveto{\pgfqpoint{4.986486in}{0.524170in}}%
\pgfpathlineto{\pgfqpoint{5.028574in}{0.524170in}}%
\pgfpathlineto{\pgfqpoint{5.028574in}{0.539132in}}%
\pgfpathlineto{\pgfqpoint{4.986486in}{0.539132in}}%
\pgfpathlineto{\pgfqpoint{4.986486in}{0.524170in}}%
\pgfpathclose%
\pgfusepath{fill}%
\end{pgfscope}%
\begin{pgfscope}%
\pgfpathrectangle{\pgfqpoint{0.651412in}{0.524170in}}{\pgfqpoint{4.629690in}{2.558193in}}%
\pgfusepath{clip}%
\pgfsetbuttcap%
\pgfsetmiterjoin%
\definecolor{currentfill}{rgb}{0.003922,0.450980,0.698039}%
\pgfsetfillcolor{currentfill}%
\pgfsetfillopacity{0.700000}%
\pgfsetlinewidth{0.000000pt}%
\definecolor{currentstroke}{rgb}{0.000000,0.000000,0.000000}%
\pgfsetstrokecolor{currentstroke}%
\pgfsetstrokeopacity{0.700000}%
\pgfsetdash{}{0pt}%
\pgfpathmoveto{\pgfqpoint{5.028574in}{0.524170in}}%
\pgfpathlineto{\pgfqpoint{5.070662in}{0.524170in}}%
\pgfpathlineto{\pgfqpoint{5.070662in}{0.534145in}}%
\pgfpathlineto{\pgfqpoint{5.028574in}{0.534145in}}%
\pgfpathlineto{\pgfqpoint{5.028574in}{0.524170in}}%
\pgfpathclose%
\pgfusepath{fill}%
\end{pgfscope}%
\begin{pgfscope}%
\pgfpathrectangle{\pgfqpoint{0.651412in}{0.524170in}}{\pgfqpoint{4.629690in}{2.558193in}}%
\pgfusepath{clip}%
\pgfsetbuttcap%
\pgfsetmiterjoin%
\definecolor{currentfill}{rgb}{0.870588,0.560784,0.019608}%
\pgfsetfillcolor{currentfill}%
\pgfsetfillopacity{0.700000}%
\pgfsetlinewidth{0.000000pt}%
\definecolor{currentstroke}{rgb}{0.000000,0.000000,0.000000}%
\pgfsetstrokecolor{currentstroke}%
\pgfsetstrokeopacity{0.700000}%
\pgfsetdash{}{0pt}%
\pgfpathmoveto{\pgfqpoint{0.861853in}{0.524170in}}%
\pgfpathlineto{\pgfqpoint{0.903941in}{0.524170in}}%
\pgfpathlineto{\pgfqpoint{0.903941in}{0.524170in}}%
\pgfpathlineto{\pgfqpoint{0.861853in}{0.524170in}}%
\pgfpathlineto{\pgfqpoint{0.861853in}{0.524170in}}%
\pgfpathclose%
\pgfusepath{fill}%
\end{pgfscope}%
\begin{pgfscope}%
\pgfpathrectangle{\pgfqpoint{0.651412in}{0.524170in}}{\pgfqpoint{4.629690in}{2.558193in}}%
\pgfusepath{clip}%
\pgfsetbuttcap%
\pgfsetmiterjoin%
\definecolor{currentfill}{rgb}{0.870588,0.560784,0.019608}%
\pgfsetfillcolor{currentfill}%
\pgfsetfillopacity{0.700000}%
\pgfsetlinewidth{0.000000pt}%
\definecolor{currentstroke}{rgb}{0.000000,0.000000,0.000000}%
\pgfsetstrokecolor{currentstroke}%
\pgfsetstrokeopacity{0.700000}%
\pgfsetdash{}{0pt}%
\pgfpathmoveto{\pgfqpoint{0.903941in}{0.524170in}}%
\pgfpathlineto{\pgfqpoint{0.946029in}{0.524170in}}%
\pgfpathlineto{\pgfqpoint{0.946029in}{0.524170in}}%
\pgfpathlineto{\pgfqpoint{0.903941in}{0.524170in}}%
\pgfpathlineto{\pgfqpoint{0.903941in}{0.524170in}}%
\pgfpathclose%
\pgfusepath{fill}%
\end{pgfscope}%
\begin{pgfscope}%
\pgfpathrectangle{\pgfqpoint{0.651412in}{0.524170in}}{\pgfqpoint{4.629690in}{2.558193in}}%
\pgfusepath{clip}%
\pgfsetbuttcap%
\pgfsetmiterjoin%
\definecolor{currentfill}{rgb}{0.870588,0.560784,0.019608}%
\pgfsetfillcolor{currentfill}%
\pgfsetfillopacity{0.700000}%
\pgfsetlinewidth{0.000000pt}%
\definecolor{currentstroke}{rgb}{0.000000,0.000000,0.000000}%
\pgfsetstrokecolor{currentstroke}%
\pgfsetstrokeopacity{0.700000}%
\pgfsetdash{}{0pt}%
\pgfpathmoveto{\pgfqpoint{0.946029in}{0.524170in}}%
\pgfpathlineto{\pgfqpoint{0.988117in}{0.524170in}}%
\pgfpathlineto{\pgfqpoint{0.988117in}{0.524170in}}%
\pgfpathlineto{\pgfqpoint{0.946029in}{0.524170in}}%
\pgfpathlineto{\pgfqpoint{0.946029in}{0.524170in}}%
\pgfpathclose%
\pgfusepath{fill}%
\end{pgfscope}%
\begin{pgfscope}%
\pgfpathrectangle{\pgfqpoint{0.651412in}{0.524170in}}{\pgfqpoint{4.629690in}{2.558193in}}%
\pgfusepath{clip}%
\pgfsetbuttcap%
\pgfsetmiterjoin%
\definecolor{currentfill}{rgb}{0.870588,0.560784,0.019608}%
\pgfsetfillcolor{currentfill}%
\pgfsetfillopacity{0.700000}%
\pgfsetlinewidth{0.000000pt}%
\definecolor{currentstroke}{rgb}{0.000000,0.000000,0.000000}%
\pgfsetstrokecolor{currentstroke}%
\pgfsetstrokeopacity{0.700000}%
\pgfsetdash{}{0pt}%
\pgfpathmoveto{\pgfqpoint{0.988117in}{0.524170in}}%
\pgfpathlineto{\pgfqpoint{1.030205in}{0.524170in}}%
\pgfpathlineto{\pgfqpoint{1.030205in}{0.524170in}}%
\pgfpathlineto{\pgfqpoint{0.988117in}{0.524170in}}%
\pgfpathlineto{\pgfqpoint{0.988117in}{0.524170in}}%
\pgfpathclose%
\pgfusepath{fill}%
\end{pgfscope}%
\begin{pgfscope}%
\pgfpathrectangle{\pgfqpoint{0.651412in}{0.524170in}}{\pgfqpoint{4.629690in}{2.558193in}}%
\pgfusepath{clip}%
\pgfsetbuttcap%
\pgfsetmiterjoin%
\definecolor{currentfill}{rgb}{0.870588,0.560784,0.019608}%
\pgfsetfillcolor{currentfill}%
\pgfsetfillopacity{0.700000}%
\pgfsetlinewidth{0.000000pt}%
\definecolor{currentstroke}{rgb}{0.000000,0.000000,0.000000}%
\pgfsetstrokecolor{currentstroke}%
\pgfsetstrokeopacity{0.700000}%
\pgfsetdash{}{0pt}%
\pgfpathmoveto{\pgfqpoint{1.030205in}{0.524170in}}%
\pgfpathlineto{\pgfqpoint{1.072293in}{0.524170in}}%
\pgfpathlineto{\pgfqpoint{1.072293in}{0.524170in}}%
\pgfpathlineto{\pgfqpoint{1.030205in}{0.524170in}}%
\pgfpathlineto{\pgfqpoint{1.030205in}{0.524170in}}%
\pgfpathclose%
\pgfusepath{fill}%
\end{pgfscope}%
\begin{pgfscope}%
\pgfpathrectangle{\pgfqpoint{0.651412in}{0.524170in}}{\pgfqpoint{4.629690in}{2.558193in}}%
\pgfusepath{clip}%
\pgfsetbuttcap%
\pgfsetmiterjoin%
\definecolor{currentfill}{rgb}{0.870588,0.560784,0.019608}%
\pgfsetfillcolor{currentfill}%
\pgfsetfillopacity{0.700000}%
\pgfsetlinewidth{0.000000pt}%
\definecolor{currentstroke}{rgb}{0.000000,0.000000,0.000000}%
\pgfsetstrokecolor{currentstroke}%
\pgfsetstrokeopacity{0.700000}%
\pgfsetdash{}{0pt}%
\pgfpathmoveto{\pgfqpoint{1.072293in}{0.524170in}}%
\pgfpathlineto{\pgfqpoint{1.114381in}{0.524170in}}%
\pgfpathlineto{\pgfqpoint{1.114381in}{0.524170in}}%
\pgfpathlineto{\pgfqpoint{1.072293in}{0.524170in}}%
\pgfpathlineto{\pgfqpoint{1.072293in}{0.524170in}}%
\pgfpathclose%
\pgfusepath{fill}%
\end{pgfscope}%
\begin{pgfscope}%
\pgfpathrectangle{\pgfqpoint{0.651412in}{0.524170in}}{\pgfqpoint{4.629690in}{2.558193in}}%
\pgfusepath{clip}%
\pgfsetbuttcap%
\pgfsetmiterjoin%
\definecolor{currentfill}{rgb}{0.870588,0.560784,0.019608}%
\pgfsetfillcolor{currentfill}%
\pgfsetfillopacity{0.700000}%
\pgfsetlinewidth{0.000000pt}%
\definecolor{currentstroke}{rgb}{0.000000,0.000000,0.000000}%
\pgfsetstrokecolor{currentstroke}%
\pgfsetstrokeopacity{0.700000}%
\pgfsetdash{}{0pt}%
\pgfpathmoveto{\pgfqpoint{1.114381in}{0.524170in}}%
\pgfpathlineto{\pgfqpoint{1.156469in}{0.524170in}}%
\pgfpathlineto{\pgfqpoint{1.156469in}{0.524170in}}%
\pgfpathlineto{\pgfqpoint{1.114381in}{0.524170in}}%
\pgfpathlineto{\pgfqpoint{1.114381in}{0.524170in}}%
\pgfpathclose%
\pgfusepath{fill}%
\end{pgfscope}%
\begin{pgfscope}%
\pgfpathrectangle{\pgfqpoint{0.651412in}{0.524170in}}{\pgfqpoint{4.629690in}{2.558193in}}%
\pgfusepath{clip}%
\pgfsetbuttcap%
\pgfsetmiterjoin%
\definecolor{currentfill}{rgb}{0.870588,0.560784,0.019608}%
\pgfsetfillcolor{currentfill}%
\pgfsetfillopacity{0.700000}%
\pgfsetlinewidth{0.000000pt}%
\definecolor{currentstroke}{rgb}{0.000000,0.000000,0.000000}%
\pgfsetstrokecolor{currentstroke}%
\pgfsetstrokeopacity{0.700000}%
\pgfsetdash{}{0pt}%
\pgfpathmoveto{\pgfqpoint{1.156469in}{0.524170in}}%
\pgfpathlineto{\pgfqpoint{1.198557in}{0.524170in}}%
\pgfpathlineto{\pgfqpoint{1.198557in}{0.524170in}}%
\pgfpathlineto{\pgfqpoint{1.156469in}{0.524170in}}%
\pgfpathlineto{\pgfqpoint{1.156469in}{0.524170in}}%
\pgfpathclose%
\pgfusepath{fill}%
\end{pgfscope}%
\begin{pgfscope}%
\pgfpathrectangle{\pgfqpoint{0.651412in}{0.524170in}}{\pgfqpoint{4.629690in}{2.558193in}}%
\pgfusepath{clip}%
\pgfsetbuttcap%
\pgfsetmiterjoin%
\definecolor{currentfill}{rgb}{0.870588,0.560784,0.019608}%
\pgfsetfillcolor{currentfill}%
\pgfsetfillopacity{0.700000}%
\pgfsetlinewidth{0.000000pt}%
\definecolor{currentstroke}{rgb}{0.000000,0.000000,0.000000}%
\pgfsetstrokecolor{currentstroke}%
\pgfsetstrokeopacity{0.700000}%
\pgfsetdash{}{0pt}%
\pgfpathmoveto{\pgfqpoint{1.198557in}{0.524170in}}%
\pgfpathlineto{\pgfqpoint{1.240645in}{0.524170in}}%
\pgfpathlineto{\pgfqpoint{1.240645in}{0.549048in}}%
\pgfpathlineto{\pgfqpoint{1.198557in}{0.549048in}}%
\pgfpathlineto{\pgfqpoint{1.198557in}{0.524170in}}%
\pgfpathclose%
\pgfusepath{fill}%
\end{pgfscope}%
\begin{pgfscope}%
\pgfpathrectangle{\pgfqpoint{0.651412in}{0.524170in}}{\pgfqpoint{4.629690in}{2.558193in}}%
\pgfusepath{clip}%
\pgfsetbuttcap%
\pgfsetmiterjoin%
\definecolor{currentfill}{rgb}{0.870588,0.560784,0.019608}%
\pgfsetfillcolor{currentfill}%
\pgfsetfillopacity{0.700000}%
\pgfsetlinewidth{0.000000pt}%
\definecolor{currentstroke}{rgb}{0.000000,0.000000,0.000000}%
\pgfsetstrokecolor{currentstroke}%
\pgfsetstrokeopacity{0.700000}%
\pgfsetdash{}{0pt}%
\pgfpathmoveto{\pgfqpoint{1.240645in}{0.524170in}}%
\pgfpathlineto{\pgfqpoint{1.282734in}{0.524170in}}%
\pgfpathlineto{\pgfqpoint{1.282734in}{0.586366in}}%
\pgfpathlineto{\pgfqpoint{1.240645in}{0.586366in}}%
\pgfpathlineto{\pgfqpoint{1.240645in}{0.524170in}}%
\pgfpathclose%
\pgfusepath{fill}%
\end{pgfscope}%
\begin{pgfscope}%
\pgfpathrectangle{\pgfqpoint{0.651412in}{0.524170in}}{\pgfqpoint{4.629690in}{2.558193in}}%
\pgfusepath{clip}%
\pgfsetbuttcap%
\pgfsetmiterjoin%
\definecolor{currentfill}{rgb}{0.870588,0.560784,0.019608}%
\pgfsetfillcolor{currentfill}%
\pgfsetfillopacity{0.700000}%
\pgfsetlinewidth{0.000000pt}%
\definecolor{currentstroke}{rgb}{0.000000,0.000000,0.000000}%
\pgfsetstrokecolor{currentstroke}%
\pgfsetstrokeopacity{0.700000}%
\pgfsetdash{}{0pt}%
\pgfpathmoveto{\pgfqpoint{1.282734in}{0.524170in}}%
\pgfpathlineto{\pgfqpoint{1.324822in}{0.524170in}}%
\pgfpathlineto{\pgfqpoint{1.324822in}{0.617464in}}%
\pgfpathlineto{\pgfqpoint{1.282734in}{0.617464in}}%
\pgfpathlineto{\pgfqpoint{1.282734in}{0.524170in}}%
\pgfpathclose%
\pgfusepath{fill}%
\end{pgfscope}%
\begin{pgfscope}%
\pgfpathrectangle{\pgfqpoint{0.651412in}{0.524170in}}{\pgfqpoint{4.629690in}{2.558193in}}%
\pgfusepath{clip}%
\pgfsetbuttcap%
\pgfsetmiterjoin%
\definecolor{currentfill}{rgb}{0.870588,0.560784,0.019608}%
\pgfsetfillcolor{currentfill}%
\pgfsetfillopacity{0.700000}%
\pgfsetlinewidth{0.000000pt}%
\definecolor{currentstroke}{rgb}{0.000000,0.000000,0.000000}%
\pgfsetstrokecolor{currentstroke}%
\pgfsetstrokeopacity{0.700000}%
\pgfsetdash{}{0pt}%
\pgfpathmoveto{\pgfqpoint{1.324822in}{0.524170in}}%
\pgfpathlineto{\pgfqpoint{1.366910in}{0.524170in}}%
\pgfpathlineto{\pgfqpoint{1.366910in}{0.716977in}}%
\pgfpathlineto{\pgfqpoint{1.324822in}{0.716977in}}%
\pgfpathlineto{\pgfqpoint{1.324822in}{0.524170in}}%
\pgfpathclose%
\pgfusepath{fill}%
\end{pgfscope}%
\begin{pgfscope}%
\pgfpathrectangle{\pgfqpoint{0.651412in}{0.524170in}}{\pgfqpoint{4.629690in}{2.558193in}}%
\pgfusepath{clip}%
\pgfsetbuttcap%
\pgfsetmiterjoin%
\definecolor{currentfill}{rgb}{0.870588,0.560784,0.019608}%
\pgfsetfillcolor{currentfill}%
\pgfsetfillopacity{0.700000}%
\pgfsetlinewidth{0.000000pt}%
\definecolor{currentstroke}{rgb}{0.000000,0.000000,0.000000}%
\pgfsetstrokecolor{currentstroke}%
\pgfsetstrokeopacity{0.700000}%
\pgfsetdash{}{0pt}%
\pgfpathmoveto{\pgfqpoint{1.366910in}{0.524170in}}%
\pgfpathlineto{\pgfqpoint{1.408998in}{0.524170in}}%
\pgfpathlineto{\pgfqpoint{1.408998in}{0.766734in}}%
\pgfpathlineto{\pgfqpoint{1.366910in}{0.766734in}}%
\pgfpathlineto{\pgfqpoint{1.366910in}{0.524170in}}%
\pgfpathclose%
\pgfusepath{fill}%
\end{pgfscope}%
\begin{pgfscope}%
\pgfpathrectangle{\pgfqpoint{0.651412in}{0.524170in}}{\pgfqpoint{4.629690in}{2.558193in}}%
\pgfusepath{clip}%
\pgfsetbuttcap%
\pgfsetmiterjoin%
\definecolor{currentfill}{rgb}{0.870588,0.560784,0.019608}%
\pgfsetfillcolor{currentfill}%
\pgfsetfillopacity{0.700000}%
\pgfsetlinewidth{0.000000pt}%
\definecolor{currentstroke}{rgb}{0.000000,0.000000,0.000000}%
\pgfsetstrokecolor{currentstroke}%
\pgfsetstrokeopacity{0.700000}%
\pgfsetdash{}{0pt}%
\pgfpathmoveto{\pgfqpoint{1.408998in}{0.524170in}}%
\pgfpathlineto{\pgfqpoint{1.451086in}{0.524170in}}%
\pgfpathlineto{\pgfqpoint{1.451086in}{0.881796in}}%
\pgfpathlineto{\pgfqpoint{1.408998in}{0.881796in}}%
\pgfpathlineto{\pgfqpoint{1.408998in}{0.524170in}}%
\pgfpathclose%
\pgfusepath{fill}%
\end{pgfscope}%
\begin{pgfscope}%
\pgfpathrectangle{\pgfqpoint{0.651412in}{0.524170in}}{\pgfqpoint{4.629690in}{2.558193in}}%
\pgfusepath{clip}%
\pgfsetbuttcap%
\pgfsetmiterjoin%
\definecolor{currentfill}{rgb}{0.870588,0.560784,0.019608}%
\pgfsetfillcolor{currentfill}%
\pgfsetfillopacity{0.700000}%
\pgfsetlinewidth{0.000000pt}%
\definecolor{currentstroke}{rgb}{0.000000,0.000000,0.000000}%
\pgfsetstrokecolor{currentstroke}%
\pgfsetstrokeopacity{0.700000}%
\pgfsetdash{}{0pt}%
\pgfpathmoveto{\pgfqpoint{1.451086in}{0.524170in}}%
\pgfpathlineto{\pgfqpoint{1.493174in}{0.524170in}}%
\pgfpathlineto{\pgfqpoint{1.493174in}{0.906675in}}%
\pgfpathlineto{\pgfqpoint{1.451086in}{0.906675in}}%
\pgfpathlineto{\pgfqpoint{1.451086in}{0.524170in}}%
\pgfpathclose%
\pgfusepath{fill}%
\end{pgfscope}%
\begin{pgfscope}%
\pgfpathrectangle{\pgfqpoint{0.651412in}{0.524170in}}{\pgfqpoint{4.629690in}{2.558193in}}%
\pgfusepath{clip}%
\pgfsetbuttcap%
\pgfsetmiterjoin%
\definecolor{currentfill}{rgb}{0.870588,0.560784,0.019608}%
\pgfsetfillcolor{currentfill}%
\pgfsetfillopacity{0.700000}%
\pgfsetlinewidth{0.000000pt}%
\definecolor{currentstroke}{rgb}{0.000000,0.000000,0.000000}%
\pgfsetstrokecolor{currentstroke}%
\pgfsetstrokeopacity{0.700000}%
\pgfsetdash{}{0pt}%
\pgfpathmoveto{\pgfqpoint{1.493174in}{0.524170in}}%
\pgfpathlineto{\pgfqpoint{1.535262in}{0.524170in}}%
\pgfpathlineto{\pgfqpoint{1.535262in}{1.015518in}}%
\pgfpathlineto{\pgfqpoint{1.493174in}{1.015518in}}%
\pgfpathlineto{\pgfqpoint{1.493174in}{0.524170in}}%
\pgfpathclose%
\pgfusepath{fill}%
\end{pgfscope}%
\begin{pgfscope}%
\pgfpathrectangle{\pgfqpoint{0.651412in}{0.524170in}}{\pgfqpoint{4.629690in}{2.558193in}}%
\pgfusepath{clip}%
\pgfsetbuttcap%
\pgfsetmiterjoin%
\definecolor{currentfill}{rgb}{0.870588,0.560784,0.019608}%
\pgfsetfillcolor{currentfill}%
\pgfsetfillopacity{0.700000}%
\pgfsetlinewidth{0.000000pt}%
\definecolor{currentstroke}{rgb}{0.000000,0.000000,0.000000}%
\pgfsetstrokecolor{currentstroke}%
\pgfsetstrokeopacity{0.700000}%
\pgfsetdash{}{0pt}%
\pgfpathmoveto{\pgfqpoint{1.535262in}{0.524170in}}%
\pgfpathlineto{\pgfqpoint{1.577350in}{0.524170in}}%
\pgfpathlineto{\pgfqpoint{1.577350in}{1.046616in}}%
\pgfpathlineto{\pgfqpoint{1.535262in}{1.046616in}}%
\pgfpathlineto{\pgfqpoint{1.535262in}{0.524170in}}%
\pgfpathclose%
\pgfusepath{fill}%
\end{pgfscope}%
\begin{pgfscope}%
\pgfpathrectangle{\pgfqpoint{0.651412in}{0.524170in}}{\pgfqpoint{4.629690in}{2.558193in}}%
\pgfusepath{clip}%
\pgfsetbuttcap%
\pgfsetmiterjoin%
\definecolor{currentfill}{rgb}{0.870588,0.560784,0.019608}%
\pgfsetfillcolor{currentfill}%
\pgfsetfillopacity{0.700000}%
\pgfsetlinewidth{0.000000pt}%
\definecolor{currentstroke}{rgb}{0.000000,0.000000,0.000000}%
\pgfsetstrokecolor{currentstroke}%
\pgfsetstrokeopacity{0.700000}%
\pgfsetdash{}{0pt}%
\pgfpathmoveto{\pgfqpoint{1.577350in}{0.524170in}}%
\pgfpathlineto{\pgfqpoint{1.619438in}{0.524170in}}%
\pgfpathlineto{\pgfqpoint{1.619438in}{1.118141in}}%
\pgfpathlineto{\pgfqpoint{1.577350in}{1.118141in}}%
\pgfpathlineto{\pgfqpoint{1.577350in}{0.524170in}}%
\pgfpathclose%
\pgfusepath{fill}%
\end{pgfscope}%
\begin{pgfscope}%
\pgfpathrectangle{\pgfqpoint{0.651412in}{0.524170in}}{\pgfqpoint{4.629690in}{2.558193in}}%
\pgfusepath{clip}%
\pgfsetbuttcap%
\pgfsetmiterjoin%
\definecolor{currentfill}{rgb}{0.870588,0.560784,0.019608}%
\pgfsetfillcolor{currentfill}%
\pgfsetfillopacity{0.700000}%
\pgfsetlinewidth{0.000000pt}%
\definecolor{currentstroke}{rgb}{0.000000,0.000000,0.000000}%
\pgfsetstrokecolor{currentstroke}%
\pgfsetstrokeopacity{0.700000}%
\pgfsetdash{}{0pt}%
\pgfpathmoveto{\pgfqpoint{1.619438in}{0.524170in}}%
\pgfpathlineto{\pgfqpoint{1.661526in}{0.524170in}}%
\pgfpathlineto{\pgfqpoint{1.661526in}{1.046616in}}%
\pgfpathlineto{\pgfqpoint{1.619438in}{1.046616in}}%
\pgfpathlineto{\pgfqpoint{1.619438in}{0.524170in}}%
\pgfpathclose%
\pgfusepath{fill}%
\end{pgfscope}%
\begin{pgfscope}%
\pgfpathrectangle{\pgfqpoint{0.651412in}{0.524170in}}{\pgfqpoint{4.629690in}{2.558193in}}%
\pgfusepath{clip}%
\pgfsetbuttcap%
\pgfsetmiterjoin%
\definecolor{currentfill}{rgb}{0.870588,0.560784,0.019608}%
\pgfsetfillcolor{currentfill}%
\pgfsetfillopacity{0.700000}%
\pgfsetlinewidth{0.000000pt}%
\definecolor{currentstroke}{rgb}{0.000000,0.000000,0.000000}%
\pgfsetstrokecolor{currentstroke}%
\pgfsetstrokeopacity{0.700000}%
\pgfsetdash{}{0pt}%
\pgfpathmoveto{\pgfqpoint{1.661526in}{0.524170in}}%
\pgfpathlineto{\pgfqpoint{1.703614in}{0.524170in}}%
\pgfpathlineto{\pgfqpoint{1.703614in}{1.108812in}}%
\pgfpathlineto{\pgfqpoint{1.661526in}{1.108812in}}%
\pgfpathlineto{\pgfqpoint{1.661526in}{0.524170in}}%
\pgfpathclose%
\pgfusepath{fill}%
\end{pgfscope}%
\begin{pgfscope}%
\pgfpathrectangle{\pgfqpoint{0.651412in}{0.524170in}}{\pgfqpoint{4.629690in}{2.558193in}}%
\pgfusepath{clip}%
\pgfsetbuttcap%
\pgfsetmiterjoin%
\definecolor{currentfill}{rgb}{0.870588,0.560784,0.019608}%
\pgfsetfillcolor{currentfill}%
\pgfsetfillopacity{0.700000}%
\pgfsetlinewidth{0.000000pt}%
\definecolor{currentstroke}{rgb}{0.000000,0.000000,0.000000}%
\pgfsetstrokecolor{currentstroke}%
\pgfsetstrokeopacity{0.700000}%
\pgfsetdash{}{0pt}%
\pgfpathmoveto{\pgfqpoint{1.703614in}{0.524170in}}%
\pgfpathlineto{\pgfqpoint{1.745703in}{0.524170in}}%
\pgfpathlineto{\pgfqpoint{1.745703in}{1.167898in}}%
\pgfpathlineto{\pgfqpoint{1.703614in}{1.167898in}}%
\pgfpathlineto{\pgfqpoint{1.703614in}{0.524170in}}%
\pgfpathclose%
\pgfusepath{fill}%
\end{pgfscope}%
\begin{pgfscope}%
\pgfpathrectangle{\pgfqpoint{0.651412in}{0.524170in}}{\pgfqpoint{4.629690in}{2.558193in}}%
\pgfusepath{clip}%
\pgfsetbuttcap%
\pgfsetmiterjoin%
\definecolor{currentfill}{rgb}{0.870588,0.560784,0.019608}%
\pgfsetfillcolor{currentfill}%
\pgfsetfillopacity{0.700000}%
\pgfsetlinewidth{0.000000pt}%
\definecolor{currentstroke}{rgb}{0.000000,0.000000,0.000000}%
\pgfsetstrokecolor{currentstroke}%
\pgfsetstrokeopacity{0.700000}%
\pgfsetdash{}{0pt}%
\pgfpathmoveto{\pgfqpoint{1.745703in}{0.524170in}}%
\pgfpathlineto{\pgfqpoint{1.787791in}{0.524170in}}%
\pgfpathlineto{\pgfqpoint{1.787791in}{1.208325in}}%
\pgfpathlineto{\pgfqpoint{1.745703in}{1.208325in}}%
\pgfpathlineto{\pgfqpoint{1.745703in}{0.524170in}}%
\pgfpathclose%
\pgfusepath{fill}%
\end{pgfscope}%
\begin{pgfscope}%
\pgfpathrectangle{\pgfqpoint{0.651412in}{0.524170in}}{\pgfqpoint{4.629690in}{2.558193in}}%
\pgfusepath{clip}%
\pgfsetbuttcap%
\pgfsetmiterjoin%
\definecolor{currentfill}{rgb}{0.870588,0.560784,0.019608}%
\pgfsetfillcolor{currentfill}%
\pgfsetfillopacity{0.700000}%
\pgfsetlinewidth{0.000000pt}%
\definecolor{currentstroke}{rgb}{0.000000,0.000000,0.000000}%
\pgfsetstrokecolor{currentstroke}%
\pgfsetstrokeopacity{0.700000}%
\pgfsetdash{}{0pt}%
\pgfpathmoveto{\pgfqpoint{1.787791in}{0.524170in}}%
\pgfpathlineto{\pgfqpoint{1.829879in}{0.524170in}}%
\pgfpathlineto{\pgfqpoint{1.829879in}{1.083933in}}%
\pgfpathlineto{\pgfqpoint{1.787791in}{1.083933in}}%
\pgfpathlineto{\pgfqpoint{1.787791in}{0.524170in}}%
\pgfpathclose%
\pgfusepath{fill}%
\end{pgfscope}%
\begin{pgfscope}%
\pgfpathrectangle{\pgfqpoint{0.651412in}{0.524170in}}{\pgfqpoint{4.629690in}{2.558193in}}%
\pgfusepath{clip}%
\pgfsetbuttcap%
\pgfsetmiterjoin%
\definecolor{currentfill}{rgb}{0.870588,0.560784,0.019608}%
\pgfsetfillcolor{currentfill}%
\pgfsetfillopacity{0.700000}%
\pgfsetlinewidth{0.000000pt}%
\definecolor{currentstroke}{rgb}{0.000000,0.000000,0.000000}%
\pgfsetstrokecolor{currentstroke}%
\pgfsetstrokeopacity{0.700000}%
\pgfsetdash{}{0pt}%
\pgfpathmoveto{\pgfqpoint{1.829879in}{0.524170in}}%
\pgfpathlineto{\pgfqpoint{1.871967in}{0.524170in}}%
\pgfpathlineto{\pgfqpoint{1.871967in}{1.090153in}}%
\pgfpathlineto{\pgfqpoint{1.829879in}{1.090153in}}%
\pgfpathlineto{\pgfqpoint{1.829879in}{0.524170in}}%
\pgfpathclose%
\pgfusepath{fill}%
\end{pgfscope}%
\begin{pgfscope}%
\pgfpathrectangle{\pgfqpoint{0.651412in}{0.524170in}}{\pgfqpoint{4.629690in}{2.558193in}}%
\pgfusepath{clip}%
\pgfsetbuttcap%
\pgfsetmiterjoin%
\definecolor{currentfill}{rgb}{0.870588,0.560784,0.019608}%
\pgfsetfillcolor{currentfill}%
\pgfsetfillopacity{0.700000}%
\pgfsetlinewidth{0.000000pt}%
\definecolor{currentstroke}{rgb}{0.000000,0.000000,0.000000}%
\pgfsetstrokecolor{currentstroke}%
\pgfsetstrokeopacity{0.700000}%
\pgfsetdash{}{0pt}%
\pgfpathmoveto{\pgfqpoint{1.871967in}{0.524170in}}%
\pgfpathlineto{\pgfqpoint{1.914055in}{0.524170in}}%
\pgfpathlineto{\pgfqpoint{1.914055in}{1.096372in}}%
\pgfpathlineto{\pgfqpoint{1.871967in}{1.096372in}}%
\pgfpathlineto{\pgfqpoint{1.871967in}{0.524170in}}%
\pgfpathclose%
\pgfusepath{fill}%
\end{pgfscope}%
\begin{pgfscope}%
\pgfpathrectangle{\pgfqpoint{0.651412in}{0.524170in}}{\pgfqpoint{4.629690in}{2.558193in}}%
\pgfusepath{clip}%
\pgfsetbuttcap%
\pgfsetmiterjoin%
\definecolor{currentfill}{rgb}{0.870588,0.560784,0.019608}%
\pgfsetfillcolor{currentfill}%
\pgfsetfillopacity{0.700000}%
\pgfsetlinewidth{0.000000pt}%
\definecolor{currentstroke}{rgb}{0.000000,0.000000,0.000000}%
\pgfsetstrokecolor{currentstroke}%
\pgfsetstrokeopacity{0.700000}%
\pgfsetdash{}{0pt}%
\pgfpathmoveto{\pgfqpoint{1.914055in}{0.524170in}}%
\pgfpathlineto{\pgfqpoint{1.956143in}{0.524170in}}%
\pgfpathlineto{\pgfqpoint{1.956143in}{1.115031in}}%
\pgfpathlineto{\pgfqpoint{1.914055in}{1.115031in}}%
\pgfpathlineto{\pgfqpoint{1.914055in}{0.524170in}}%
\pgfpathclose%
\pgfusepath{fill}%
\end{pgfscope}%
\begin{pgfscope}%
\pgfpathrectangle{\pgfqpoint{0.651412in}{0.524170in}}{\pgfqpoint{4.629690in}{2.558193in}}%
\pgfusepath{clip}%
\pgfsetbuttcap%
\pgfsetmiterjoin%
\definecolor{currentfill}{rgb}{0.870588,0.560784,0.019608}%
\pgfsetfillcolor{currentfill}%
\pgfsetfillopacity{0.700000}%
\pgfsetlinewidth{0.000000pt}%
\definecolor{currentstroke}{rgb}{0.000000,0.000000,0.000000}%
\pgfsetstrokecolor{currentstroke}%
\pgfsetstrokeopacity{0.700000}%
\pgfsetdash{}{0pt}%
\pgfpathmoveto{\pgfqpoint{1.956143in}{0.524170in}}%
\pgfpathlineto{\pgfqpoint{1.998231in}{0.524170in}}%
\pgfpathlineto{\pgfqpoint{1.998231in}{1.096372in}}%
\pgfpathlineto{\pgfqpoint{1.956143in}{1.096372in}}%
\pgfpathlineto{\pgfqpoint{1.956143in}{0.524170in}}%
\pgfpathclose%
\pgfusepath{fill}%
\end{pgfscope}%
\begin{pgfscope}%
\pgfpathrectangle{\pgfqpoint{0.651412in}{0.524170in}}{\pgfqpoint{4.629690in}{2.558193in}}%
\pgfusepath{clip}%
\pgfsetbuttcap%
\pgfsetmiterjoin%
\definecolor{currentfill}{rgb}{0.870588,0.560784,0.019608}%
\pgfsetfillcolor{currentfill}%
\pgfsetfillopacity{0.700000}%
\pgfsetlinewidth{0.000000pt}%
\definecolor{currentstroke}{rgb}{0.000000,0.000000,0.000000}%
\pgfsetstrokecolor{currentstroke}%
\pgfsetstrokeopacity{0.700000}%
\pgfsetdash{}{0pt}%
\pgfpathmoveto{\pgfqpoint{1.998231in}{0.524170in}}%
\pgfpathlineto{\pgfqpoint{2.040319in}{0.524170in}}%
\pgfpathlineto{\pgfqpoint{2.040319in}{1.111921in}}%
\pgfpathlineto{\pgfqpoint{1.998231in}{1.111921in}}%
\pgfpathlineto{\pgfqpoint{1.998231in}{0.524170in}}%
\pgfpathclose%
\pgfusepath{fill}%
\end{pgfscope}%
\begin{pgfscope}%
\pgfpathrectangle{\pgfqpoint{0.651412in}{0.524170in}}{\pgfqpoint{4.629690in}{2.558193in}}%
\pgfusepath{clip}%
\pgfsetbuttcap%
\pgfsetmiterjoin%
\definecolor{currentfill}{rgb}{0.870588,0.560784,0.019608}%
\pgfsetfillcolor{currentfill}%
\pgfsetfillopacity{0.700000}%
\pgfsetlinewidth{0.000000pt}%
\definecolor{currentstroke}{rgb}{0.000000,0.000000,0.000000}%
\pgfsetstrokecolor{currentstroke}%
\pgfsetstrokeopacity{0.700000}%
\pgfsetdash{}{0pt}%
\pgfpathmoveto{\pgfqpoint{2.040319in}{0.524170in}}%
\pgfpathlineto{\pgfqpoint{2.082407in}{0.524170in}}%
\pgfpathlineto{\pgfqpoint{2.082407in}{1.021737in}}%
\pgfpathlineto{\pgfqpoint{2.040319in}{1.021737in}}%
\pgfpathlineto{\pgfqpoint{2.040319in}{0.524170in}}%
\pgfpathclose%
\pgfusepath{fill}%
\end{pgfscope}%
\begin{pgfscope}%
\pgfpathrectangle{\pgfqpoint{0.651412in}{0.524170in}}{\pgfqpoint{4.629690in}{2.558193in}}%
\pgfusepath{clip}%
\pgfsetbuttcap%
\pgfsetmiterjoin%
\definecolor{currentfill}{rgb}{0.870588,0.560784,0.019608}%
\pgfsetfillcolor{currentfill}%
\pgfsetfillopacity{0.700000}%
\pgfsetlinewidth{0.000000pt}%
\definecolor{currentstroke}{rgb}{0.000000,0.000000,0.000000}%
\pgfsetstrokecolor{currentstroke}%
\pgfsetstrokeopacity{0.700000}%
\pgfsetdash{}{0pt}%
\pgfpathmoveto{\pgfqpoint{2.082407in}{0.524170in}}%
\pgfpathlineto{\pgfqpoint{2.124495in}{0.524170in}}%
\pgfpathlineto{\pgfqpoint{2.124495in}{1.093263in}}%
\pgfpathlineto{\pgfqpoint{2.082407in}{1.093263in}}%
\pgfpathlineto{\pgfqpoint{2.082407in}{0.524170in}}%
\pgfpathclose%
\pgfusepath{fill}%
\end{pgfscope}%
\begin{pgfscope}%
\pgfpathrectangle{\pgfqpoint{0.651412in}{0.524170in}}{\pgfqpoint{4.629690in}{2.558193in}}%
\pgfusepath{clip}%
\pgfsetbuttcap%
\pgfsetmiterjoin%
\definecolor{currentfill}{rgb}{0.870588,0.560784,0.019608}%
\pgfsetfillcolor{currentfill}%
\pgfsetfillopacity{0.700000}%
\pgfsetlinewidth{0.000000pt}%
\definecolor{currentstroke}{rgb}{0.000000,0.000000,0.000000}%
\pgfsetstrokecolor{currentstroke}%
\pgfsetstrokeopacity{0.700000}%
\pgfsetdash{}{0pt}%
\pgfpathmoveto{\pgfqpoint{2.124495in}{0.524170in}}%
\pgfpathlineto{\pgfqpoint{2.166583in}{0.524170in}}%
\pgfpathlineto{\pgfqpoint{2.166583in}{0.987530in}}%
\pgfpathlineto{\pgfqpoint{2.124495in}{0.987530in}}%
\pgfpathlineto{\pgfqpoint{2.124495in}{0.524170in}}%
\pgfpathclose%
\pgfusepath{fill}%
\end{pgfscope}%
\begin{pgfscope}%
\pgfpathrectangle{\pgfqpoint{0.651412in}{0.524170in}}{\pgfqpoint{4.629690in}{2.558193in}}%
\pgfusepath{clip}%
\pgfsetbuttcap%
\pgfsetmiterjoin%
\definecolor{currentfill}{rgb}{0.870588,0.560784,0.019608}%
\pgfsetfillcolor{currentfill}%
\pgfsetfillopacity{0.700000}%
\pgfsetlinewidth{0.000000pt}%
\definecolor{currentstroke}{rgb}{0.000000,0.000000,0.000000}%
\pgfsetstrokecolor{currentstroke}%
\pgfsetstrokeopacity{0.700000}%
\pgfsetdash{}{0pt}%
\pgfpathmoveto{\pgfqpoint{2.166583in}{0.524170in}}%
\pgfpathlineto{\pgfqpoint{2.208672in}{0.524170in}}%
\pgfpathlineto{\pgfqpoint{2.208672in}{0.990639in}}%
\pgfpathlineto{\pgfqpoint{2.166583in}{0.990639in}}%
\pgfpathlineto{\pgfqpoint{2.166583in}{0.524170in}}%
\pgfpathclose%
\pgfusepath{fill}%
\end{pgfscope}%
\begin{pgfscope}%
\pgfpathrectangle{\pgfqpoint{0.651412in}{0.524170in}}{\pgfqpoint{4.629690in}{2.558193in}}%
\pgfusepath{clip}%
\pgfsetbuttcap%
\pgfsetmiterjoin%
\definecolor{currentfill}{rgb}{0.870588,0.560784,0.019608}%
\pgfsetfillcolor{currentfill}%
\pgfsetfillopacity{0.700000}%
\pgfsetlinewidth{0.000000pt}%
\definecolor{currentstroke}{rgb}{0.000000,0.000000,0.000000}%
\pgfsetstrokecolor{currentstroke}%
\pgfsetstrokeopacity{0.700000}%
\pgfsetdash{}{0pt}%
\pgfpathmoveto{\pgfqpoint{2.208672in}{0.524170in}}%
\pgfpathlineto{\pgfqpoint{2.250760in}{0.524170in}}%
\pgfpathlineto{\pgfqpoint{2.250760in}{0.984420in}}%
\pgfpathlineto{\pgfqpoint{2.208672in}{0.984420in}}%
\pgfpathlineto{\pgfqpoint{2.208672in}{0.524170in}}%
\pgfpathclose%
\pgfusepath{fill}%
\end{pgfscope}%
\begin{pgfscope}%
\pgfpathrectangle{\pgfqpoint{0.651412in}{0.524170in}}{\pgfqpoint{4.629690in}{2.558193in}}%
\pgfusepath{clip}%
\pgfsetbuttcap%
\pgfsetmiterjoin%
\definecolor{currentfill}{rgb}{0.870588,0.560784,0.019608}%
\pgfsetfillcolor{currentfill}%
\pgfsetfillopacity{0.700000}%
\pgfsetlinewidth{0.000000pt}%
\definecolor{currentstroke}{rgb}{0.000000,0.000000,0.000000}%
\pgfsetstrokecolor{currentstroke}%
\pgfsetstrokeopacity{0.700000}%
\pgfsetdash{}{0pt}%
\pgfpathmoveto{\pgfqpoint{2.250760in}{0.524170in}}%
\pgfpathlineto{\pgfqpoint{2.292848in}{0.524170in}}%
\pgfpathlineto{\pgfqpoint{2.292848in}{0.950212in}}%
\pgfpathlineto{\pgfqpoint{2.250760in}{0.950212in}}%
\pgfpathlineto{\pgfqpoint{2.250760in}{0.524170in}}%
\pgfpathclose%
\pgfusepath{fill}%
\end{pgfscope}%
\begin{pgfscope}%
\pgfpathrectangle{\pgfqpoint{0.651412in}{0.524170in}}{\pgfqpoint{4.629690in}{2.558193in}}%
\pgfusepath{clip}%
\pgfsetbuttcap%
\pgfsetmiterjoin%
\definecolor{currentfill}{rgb}{0.870588,0.560784,0.019608}%
\pgfsetfillcolor{currentfill}%
\pgfsetfillopacity{0.700000}%
\pgfsetlinewidth{0.000000pt}%
\definecolor{currentstroke}{rgb}{0.000000,0.000000,0.000000}%
\pgfsetstrokecolor{currentstroke}%
\pgfsetstrokeopacity{0.700000}%
\pgfsetdash{}{0pt}%
\pgfpathmoveto{\pgfqpoint{2.292848in}{0.524170in}}%
\pgfpathlineto{\pgfqpoint{2.334936in}{0.524170in}}%
\pgfpathlineto{\pgfqpoint{2.334936in}{0.937773in}}%
\pgfpathlineto{\pgfqpoint{2.292848in}{0.937773in}}%
\pgfpathlineto{\pgfqpoint{2.292848in}{0.524170in}}%
\pgfpathclose%
\pgfusepath{fill}%
\end{pgfscope}%
\begin{pgfscope}%
\pgfpathrectangle{\pgfqpoint{0.651412in}{0.524170in}}{\pgfqpoint{4.629690in}{2.558193in}}%
\pgfusepath{clip}%
\pgfsetbuttcap%
\pgfsetmiterjoin%
\definecolor{currentfill}{rgb}{0.870588,0.560784,0.019608}%
\pgfsetfillcolor{currentfill}%
\pgfsetfillopacity{0.700000}%
\pgfsetlinewidth{0.000000pt}%
\definecolor{currentstroke}{rgb}{0.000000,0.000000,0.000000}%
\pgfsetstrokecolor{currentstroke}%
\pgfsetstrokeopacity{0.700000}%
\pgfsetdash{}{0pt}%
\pgfpathmoveto{\pgfqpoint{2.334936in}{0.524170in}}%
\pgfpathlineto{\pgfqpoint{2.377024in}{0.524170in}}%
\pgfpathlineto{\pgfqpoint{2.377024in}{0.959541in}}%
\pgfpathlineto{\pgfqpoint{2.334936in}{0.959541in}}%
\pgfpathlineto{\pgfqpoint{2.334936in}{0.524170in}}%
\pgfpathclose%
\pgfusepath{fill}%
\end{pgfscope}%
\begin{pgfscope}%
\pgfpathrectangle{\pgfqpoint{0.651412in}{0.524170in}}{\pgfqpoint{4.629690in}{2.558193in}}%
\pgfusepath{clip}%
\pgfsetbuttcap%
\pgfsetmiterjoin%
\definecolor{currentfill}{rgb}{0.870588,0.560784,0.019608}%
\pgfsetfillcolor{currentfill}%
\pgfsetfillopacity{0.700000}%
\pgfsetlinewidth{0.000000pt}%
\definecolor{currentstroke}{rgb}{0.000000,0.000000,0.000000}%
\pgfsetstrokecolor{currentstroke}%
\pgfsetstrokeopacity{0.700000}%
\pgfsetdash{}{0pt}%
\pgfpathmoveto{\pgfqpoint{2.377024in}{0.524170in}}%
\pgfpathlineto{\pgfqpoint{2.419112in}{0.524170in}}%
\pgfpathlineto{\pgfqpoint{2.419112in}{0.978200in}}%
\pgfpathlineto{\pgfqpoint{2.377024in}{0.978200in}}%
\pgfpathlineto{\pgfqpoint{2.377024in}{0.524170in}}%
\pgfpathclose%
\pgfusepath{fill}%
\end{pgfscope}%
\begin{pgfscope}%
\pgfpathrectangle{\pgfqpoint{0.651412in}{0.524170in}}{\pgfqpoint{4.629690in}{2.558193in}}%
\pgfusepath{clip}%
\pgfsetbuttcap%
\pgfsetmiterjoin%
\definecolor{currentfill}{rgb}{0.870588,0.560784,0.019608}%
\pgfsetfillcolor{currentfill}%
\pgfsetfillopacity{0.700000}%
\pgfsetlinewidth{0.000000pt}%
\definecolor{currentstroke}{rgb}{0.000000,0.000000,0.000000}%
\pgfsetstrokecolor{currentstroke}%
\pgfsetstrokeopacity{0.700000}%
\pgfsetdash{}{0pt}%
\pgfpathmoveto{\pgfqpoint{2.419112in}{0.524170in}}%
\pgfpathlineto{\pgfqpoint{2.461200in}{0.524170in}}%
\pgfpathlineto{\pgfqpoint{2.461200in}{0.928443in}}%
\pgfpathlineto{\pgfqpoint{2.419112in}{0.928443in}}%
\pgfpathlineto{\pgfqpoint{2.419112in}{0.524170in}}%
\pgfpathclose%
\pgfusepath{fill}%
\end{pgfscope}%
\begin{pgfscope}%
\pgfpathrectangle{\pgfqpoint{0.651412in}{0.524170in}}{\pgfqpoint{4.629690in}{2.558193in}}%
\pgfusepath{clip}%
\pgfsetbuttcap%
\pgfsetmiterjoin%
\definecolor{currentfill}{rgb}{0.870588,0.560784,0.019608}%
\pgfsetfillcolor{currentfill}%
\pgfsetfillopacity{0.700000}%
\pgfsetlinewidth{0.000000pt}%
\definecolor{currentstroke}{rgb}{0.000000,0.000000,0.000000}%
\pgfsetstrokecolor{currentstroke}%
\pgfsetstrokeopacity{0.700000}%
\pgfsetdash{}{0pt}%
\pgfpathmoveto{\pgfqpoint{2.461200in}{0.524170in}}%
\pgfpathlineto{\pgfqpoint{2.503288in}{0.524170in}}%
\pgfpathlineto{\pgfqpoint{2.503288in}{0.869357in}}%
\pgfpathlineto{\pgfqpoint{2.461200in}{0.869357in}}%
\pgfpathlineto{\pgfqpoint{2.461200in}{0.524170in}}%
\pgfpathclose%
\pgfusepath{fill}%
\end{pgfscope}%
\begin{pgfscope}%
\pgfpathrectangle{\pgfqpoint{0.651412in}{0.524170in}}{\pgfqpoint{4.629690in}{2.558193in}}%
\pgfusepath{clip}%
\pgfsetbuttcap%
\pgfsetmiterjoin%
\definecolor{currentfill}{rgb}{0.870588,0.560784,0.019608}%
\pgfsetfillcolor{currentfill}%
\pgfsetfillopacity{0.700000}%
\pgfsetlinewidth{0.000000pt}%
\definecolor{currentstroke}{rgb}{0.000000,0.000000,0.000000}%
\pgfsetstrokecolor{currentstroke}%
\pgfsetstrokeopacity{0.700000}%
\pgfsetdash{}{0pt}%
\pgfpathmoveto{\pgfqpoint{2.503288in}{0.524170in}}%
\pgfpathlineto{\pgfqpoint{2.545376in}{0.524170in}}%
\pgfpathlineto{\pgfqpoint{2.545376in}{0.884906in}}%
\pgfpathlineto{\pgfqpoint{2.503288in}{0.884906in}}%
\pgfpathlineto{\pgfqpoint{2.503288in}{0.524170in}}%
\pgfpathclose%
\pgfusepath{fill}%
\end{pgfscope}%
\begin{pgfscope}%
\pgfpathrectangle{\pgfqpoint{0.651412in}{0.524170in}}{\pgfqpoint{4.629690in}{2.558193in}}%
\pgfusepath{clip}%
\pgfsetbuttcap%
\pgfsetmiterjoin%
\definecolor{currentfill}{rgb}{0.870588,0.560784,0.019608}%
\pgfsetfillcolor{currentfill}%
\pgfsetfillopacity{0.700000}%
\pgfsetlinewidth{0.000000pt}%
\definecolor{currentstroke}{rgb}{0.000000,0.000000,0.000000}%
\pgfsetstrokecolor{currentstroke}%
\pgfsetstrokeopacity{0.700000}%
\pgfsetdash{}{0pt}%
\pgfpathmoveto{\pgfqpoint{2.545376in}{0.524170in}}%
\pgfpathlineto{\pgfqpoint{2.587464in}{0.524170in}}%
\pgfpathlineto{\pgfqpoint{2.587464in}{0.788503in}}%
\pgfpathlineto{\pgfqpoint{2.545376in}{0.788503in}}%
\pgfpathlineto{\pgfqpoint{2.545376in}{0.524170in}}%
\pgfpathclose%
\pgfusepath{fill}%
\end{pgfscope}%
\begin{pgfscope}%
\pgfpathrectangle{\pgfqpoint{0.651412in}{0.524170in}}{\pgfqpoint{4.629690in}{2.558193in}}%
\pgfusepath{clip}%
\pgfsetbuttcap%
\pgfsetmiterjoin%
\definecolor{currentfill}{rgb}{0.870588,0.560784,0.019608}%
\pgfsetfillcolor{currentfill}%
\pgfsetfillopacity{0.700000}%
\pgfsetlinewidth{0.000000pt}%
\definecolor{currentstroke}{rgb}{0.000000,0.000000,0.000000}%
\pgfsetstrokecolor{currentstroke}%
\pgfsetstrokeopacity{0.700000}%
\pgfsetdash{}{0pt}%
\pgfpathmoveto{\pgfqpoint{2.587464in}{0.524170in}}%
\pgfpathlineto{\pgfqpoint{2.629553in}{0.524170in}}%
\pgfpathlineto{\pgfqpoint{2.629553in}{0.850698in}}%
\pgfpathlineto{\pgfqpoint{2.587464in}{0.850698in}}%
\pgfpathlineto{\pgfqpoint{2.587464in}{0.524170in}}%
\pgfpathclose%
\pgfusepath{fill}%
\end{pgfscope}%
\begin{pgfscope}%
\pgfpathrectangle{\pgfqpoint{0.651412in}{0.524170in}}{\pgfqpoint{4.629690in}{2.558193in}}%
\pgfusepath{clip}%
\pgfsetbuttcap%
\pgfsetmiterjoin%
\definecolor{currentfill}{rgb}{0.870588,0.560784,0.019608}%
\pgfsetfillcolor{currentfill}%
\pgfsetfillopacity{0.700000}%
\pgfsetlinewidth{0.000000pt}%
\definecolor{currentstroke}{rgb}{0.000000,0.000000,0.000000}%
\pgfsetstrokecolor{currentstroke}%
\pgfsetstrokeopacity{0.700000}%
\pgfsetdash{}{0pt}%
\pgfpathmoveto{\pgfqpoint{2.629553in}{0.524170in}}%
\pgfpathlineto{\pgfqpoint{2.671641in}{0.524170in}}%
\pgfpathlineto{\pgfqpoint{2.671641in}{0.822710in}}%
\pgfpathlineto{\pgfqpoint{2.629553in}{0.822710in}}%
\pgfpathlineto{\pgfqpoint{2.629553in}{0.524170in}}%
\pgfpathclose%
\pgfusepath{fill}%
\end{pgfscope}%
\begin{pgfscope}%
\pgfpathrectangle{\pgfqpoint{0.651412in}{0.524170in}}{\pgfqpoint{4.629690in}{2.558193in}}%
\pgfusepath{clip}%
\pgfsetbuttcap%
\pgfsetmiterjoin%
\definecolor{currentfill}{rgb}{0.870588,0.560784,0.019608}%
\pgfsetfillcolor{currentfill}%
\pgfsetfillopacity{0.700000}%
\pgfsetlinewidth{0.000000pt}%
\definecolor{currentstroke}{rgb}{0.000000,0.000000,0.000000}%
\pgfsetstrokecolor{currentstroke}%
\pgfsetstrokeopacity{0.700000}%
\pgfsetdash{}{0pt}%
\pgfpathmoveto{\pgfqpoint{2.671641in}{0.524170in}}%
\pgfpathlineto{\pgfqpoint{2.713729in}{0.524170in}}%
\pgfpathlineto{\pgfqpoint{2.713729in}{0.825820in}}%
\pgfpathlineto{\pgfqpoint{2.671641in}{0.825820in}}%
\pgfpathlineto{\pgfqpoint{2.671641in}{0.524170in}}%
\pgfpathclose%
\pgfusepath{fill}%
\end{pgfscope}%
\begin{pgfscope}%
\pgfpathrectangle{\pgfqpoint{0.651412in}{0.524170in}}{\pgfqpoint{4.629690in}{2.558193in}}%
\pgfusepath{clip}%
\pgfsetbuttcap%
\pgfsetmiterjoin%
\definecolor{currentfill}{rgb}{0.870588,0.560784,0.019608}%
\pgfsetfillcolor{currentfill}%
\pgfsetfillopacity{0.700000}%
\pgfsetlinewidth{0.000000pt}%
\definecolor{currentstroke}{rgb}{0.000000,0.000000,0.000000}%
\pgfsetstrokecolor{currentstroke}%
\pgfsetstrokeopacity{0.700000}%
\pgfsetdash{}{0pt}%
\pgfpathmoveto{\pgfqpoint{2.713729in}{0.524170in}}%
\pgfpathlineto{\pgfqpoint{2.755817in}{0.524170in}}%
\pgfpathlineto{\pgfqpoint{2.755817in}{0.816491in}}%
\pgfpathlineto{\pgfqpoint{2.713729in}{0.816491in}}%
\pgfpathlineto{\pgfqpoint{2.713729in}{0.524170in}}%
\pgfpathclose%
\pgfusepath{fill}%
\end{pgfscope}%
\begin{pgfscope}%
\pgfpathrectangle{\pgfqpoint{0.651412in}{0.524170in}}{\pgfqpoint{4.629690in}{2.558193in}}%
\pgfusepath{clip}%
\pgfsetbuttcap%
\pgfsetmiterjoin%
\definecolor{currentfill}{rgb}{0.870588,0.560784,0.019608}%
\pgfsetfillcolor{currentfill}%
\pgfsetfillopacity{0.700000}%
\pgfsetlinewidth{0.000000pt}%
\definecolor{currentstroke}{rgb}{0.000000,0.000000,0.000000}%
\pgfsetstrokecolor{currentstroke}%
\pgfsetstrokeopacity{0.700000}%
\pgfsetdash{}{0pt}%
\pgfpathmoveto{\pgfqpoint{2.755817in}{0.524170in}}%
\pgfpathlineto{\pgfqpoint{2.797905in}{0.524170in}}%
\pgfpathlineto{\pgfqpoint{2.797905in}{0.819601in}}%
\pgfpathlineto{\pgfqpoint{2.755817in}{0.819601in}}%
\pgfpathlineto{\pgfqpoint{2.755817in}{0.524170in}}%
\pgfpathclose%
\pgfusepath{fill}%
\end{pgfscope}%
\begin{pgfscope}%
\pgfpathrectangle{\pgfqpoint{0.651412in}{0.524170in}}{\pgfqpoint{4.629690in}{2.558193in}}%
\pgfusepath{clip}%
\pgfsetbuttcap%
\pgfsetmiterjoin%
\definecolor{currentfill}{rgb}{0.870588,0.560784,0.019608}%
\pgfsetfillcolor{currentfill}%
\pgfsetfillopacity{0.700000}%
\pgfsetlinewidth{0.000000pt}%
\definecolor{currentstroke}{rgb}{0.000000,0.000000,0.000000}%
\pgfsetstrokecolor{currentstroke}%
\pgfsetstrokeopacity{0.700000}%
\pgfsetdash{}{0pt}%
\pgfpathmoveto{\pgfqpoint{2.797905in}{0.524170in}}%
\pgfpathlineto{\pgfqpoint{2.839993in}{0.524170in}}%
\pgfpathlineto{\pgfqpoint{2.839993in}{0.816491in}}%
\pgfpathlineto{\pgfqpoint{2.797905in}{0.816491in}}%
\pgfpathlineto{\pgfqpoint{2.797905in}{0.524170in}}%
\pgfpathclose%
\pgfusepath{fill}%
\end{pgfscope}%
\begin{pgfscope}%
\pgfpathrectangle{\pgfqpoint{0.651412in}{0.524170in}}{\pgfqpoint{4.629690in}{2.558193in}}%
\pgfusepath{clip}%
\pgfsetbuttcap%
\pgfsetmiterjoin%
\definecolor{currentfill}{rgb}{0.870588,0.560784,0.019608}%
\pgfsetfillcolor{currentfill}%
\pgfsetfillopacity{0.700000}%
\pgfsetlinewidth{0.000000pt}%
\definecolor{currentstroke}{rgb}{0.000000,0.000000,0.000000}%
\pgfsetstrokecolor{currentstroke}%
\pgfsetstrokeopacity{0.700000}%
\pgfsetdash{}{0pt}%
\pgfpathmoveto{\pgfqpoint{2.839993in}{0.524170in}}%
\pgfpathlineto{\pgfqpoint{2.882081in}{0.524170in}}%
\pgfpathlineto{\pgfqpoint{2.882081in}{0.788503in}}%
\pgfpathlineto{\pgfqpoint{2.839993in}{0.788503in}}%
\pgfpathlineto{\pgfqpoint{2.839993in}{0.524170in}}%
\pgfpathclose%
\pgfusepath{fill}%
\end{pgfscope}%
\begin{pgfscope}%
\pgfpathrectangle{\pgfqpoint{0.651412in}{0.524170in}}{\pgfqpoint{4.629690in}{2.558193in}}%
\pgfusepath{clip}%
\pgfsetbuttcap%
\pgfsetmiterjoin%
\definecolor{currentfill}{rgb}{0.870588,0.560784,0.019608}%
\pgfsetfillcolor{currentfill}%
\pgfsetfillopacity{0.700000}%
\pgfsetlinewidth{0.000000pt}%
\definecolor{currentstroke}{rgb}{0.000000,0.000000,0.000000}%
\pgfsetstrokecolor{currentstroke}%
\pgfsetstrokeopacity{0.700000}%
\pgfsetdash{}{0pt}%
\pgfpathmoveto{\pgfqpoint{2.882081in}{0.524170in}}%
\pgfpathlineto{\pgfqpoint{2.924169in}{0.524170in}}%
\pgfpathlineto{\pgfqpoint{2.924169in}{0.797832in}}%
\pgfpathlineto{\pgfqpoint{2.882081in}{0.797832in}}%
\pgfpathlineto{\pgfqpoint{2.882081in}{0.524170in}}%
\pgfpathclose%
\pgfusepath{fill}%
\end{pgfscope}%
\begin{pgfscope}%
\pgfpathrectangle{\pgfqpoint{0.651412in}{0.524170in}}{\pgfqpoint{4.629690in}{2.558193in}}%
\pgfusepath{clip}%
\pgfsetbuttcap%
\pgfsetmiterjoin%
\definecolor{currentfill}{rgb}{0.870588,0.560784,0.019608}%
\pgfsetfillcolor{currentfill}%
\pgfsetfillopacity{0.700000}%
\pgfsetlinewidth{0.000000pt}%
\definecolor{currentstroke}{rgb}{0.000000,0.000000,0.000000}%
\pgfsetstrokecolor{currentstroke}%
\pgfsetstrokeopacity{0.700000}%
\pgfsetdash{}{0pt}%
\pgfpathmoveto{\pgfqpoint{2.924169in}{0.524170in}}%
\pgfpathlineto{\pgfqpoint{2.966257in}{0.524170in}}%
\pgfpathlineto{\pgfqpoint{2.966257in}{0.751185in}}%
\pgfpathlineto{\pgfqpoint{2.924169in}{0.751185in}}%
\pgfpathlineto{\pgfqpoint{2.924169in}{0.524170in}}%
\pgfpathclose%
\pgfusepath{fill}%
\end{pgfscope}%
\begin{pgfscope}%
\pgfpathrectangle{\pgfqpoint{0.651412in}{0.524170in}}{\pgfqpoint{4.629690in}{2.558193in}}%
\pgfusepath{clip}%
\pgfsetbuttcap%
\pgfsetmiterjoin%
\definecolor{currentfill}{rgb}{0.870588,0.560784,0.019608}%
\pgfsetfillcolor{currentfill}%
\pgfsetfillopacity{0.700000}%
\pgfsetlinewidth{0.000000pt}%
\definecolor{currentstroke}{rgb}{0.000000,0.000000,0.000000}%
\pgfsetstrokecolor{currentstroke}%
\pgfsetstrokeopacity{0.700000}%
\pgfsetdash{}{0pt}%
\pgfpathmoveto{\pgfqpoint{2.966257in}{0.524170in}}%
\pgfpathlineto{\pgfqpoint{3.008345in}{0.524170in}}%
\pgfpathlineto{\pgfqpoint{3.008345in}{0.816491in}}%
\pgfpathlineto{\pgfqpoint{2.966257in}{0.816491in}}%
\pgfpathlineto{\pgfqpoint{2.966257in}{0.524170in}}%
\pgfpathclose%
\pgfusepath{fill}%
\end{pgfscope}%
\begin{pgfscope}%
\pgfpathrectangle{\pgfqpoint{0.651412in}{0.524170in}}{\pgfqpoint{4.629690in}{2.558193in}}%
\pgfusepath{clip}%
\pgfsetbuttcap%
\pgfsetmiterjoin%
\definecolor{currentfill}{rgb}{0.870588,0.560784,0.019608}%
\pgfsetfillcolor{currentfill}%
\pgfsetfillopacity{0.700000}%
\pgfsetlinewidth{0.000000pt}%
\definecolor{currentstroke}{rgb}{0.000000,0.000000,0.000000}%
\pgfsetstrokecolor{currentstroke}%
\pgfsetstrokeopacity{0.700000}%
\pgfsetdash{}{0pt}%
\pgfpathmoveto{\pgfqpoint{3.008345in}{0.524170in}}%
\pgfpathlineto{\pgfqpoint{3.050433in}{0.524170in}}%
\pgfpathlineto{\pgfqpoint{3.050433in}{0.776063in}}%
\pgfpathlineto{\pgfqpoint{3.008345in}{0.776063in}}%
\pgfpathlineto{\pgfqpoint{3.008345in}{0.524170in}}%
\pgfpathclose%
\pgfusepath{fill}%
\end{pgfscope}%
\begin{pgfscope}%
\pgfpathrectangle{\pgfqpoint{0.651412in}{0.524170in}}{\pgfqpoint{4.629690in}{2.558193in}}%
\pgfusepath{clip}%
\pgfsetbuttcap%
\pgfsetmiterjoin%
\definecolor{currentfill}{rgb}{0.870588,0.560784,0.019608}%
\pgfsetfillcolor{currentfill}%
\pgfsetfillopacity{0.700000}%
\pgfsetlinewidth{0.000000pt}%
\definecolor{currentstroke}{rgb}{0.000000,0.000000,0.000000}%
\pgfsetstrokecolor{currentstroke}%
\pgfsetstrokeopacity{0.700000}%
\pgfsetdash{}{0pt}%
\pgfpathmoveto{\pgfqpoint{3.050433in}{0.524170in}}%
\pgfpathlineto{\pgfqpoint{3.092522in}{0.524170in}}%
\pgfpathlineto{\pgfqpoint{3.092522in}{0.769844in}}%
\pgfpathlineto{\pgfqpoint{3.050433in}{0.769844in}}%
\pgfpathlineto{\pgfqpoint{3.050433in}{0.524170in}}%
\pgfpathclose%
\pgfusepath{fill}%
\end{pgfscope}%
\begin{pgfscope}%
\pgfpathrectangle{\pgfqpoint{0.651412in}{0.524170in}}{\pgfqpoint{4.629690in}{2.558193in}}%
\pgfusepath{clip}%
\pgfsetbuttcap%
\pgfsetmiterjoin%
\definecolor{currentfill}{rgb}{0.870588,0.560784,0.019608}%
\pgfsetfillcolor{currentfill}%
\pgfsetfillopacity{0.700000}%
\pgfsetlinewidth{0.000000pt}%
\definecolor{currentstroke}{rgb}{0.000000,0.000000,0.000000}%
\pgfsetstrokecolor{currentstroke}%
\pgfsetstrokeopacity{0.700000}%
\pgfsetdash{}{0pt}%
\pgfpathmoveto{\pgfqpoint{3.092522in}{0.524170in}}%
\pgfpathlineto{\pgfqpoint{3.134610in}{0.524170in}}%
\pgfpathlineto{\pgfqpoint{3.134610in}{0.723197in}}%
\pgfpathlineto{\pgfqpoint{3.092522in}{0.723197in}}%
\pgfpathlineto{\pgfqpoint{3.092522in}{0.524170in}}%
\pgfpathclose%
\pgfusepath{fill}%
\end{pgfscope}%
\begin{pgfscope}%
\pgfpathrectangle{\pgfqpoint{0.651412in}{0.524170in}}{\pgfqpoint{4.629690in}{2.558193in}}%
\pgfusepath{clip}%
\pgfsetbuttcap%
\pgfsetmiterjoin%
\definecolor{currentfill}{rgb}{0.870588,0.560784,0.019608}%
\pgfsetfillcolor{currentfill}%
\pgfsetfillopacity{0.700000}%
\pgfsetlinewidth{0.000000pt}%
\definecolor{currentstroke}{rgb}{0.000000,0.000000,0.000000}%
\pgfsetstrokecolor{currentstroke}%
\pgfsetstrokeopacity{0.700000}%
\pgfsetdash{}{0pt}%
\pgfpathmoveto{\pgfqpoint{3.134610in}{0.524170in}}%
\pgfpathlineto{\pgfqpoint{3.176698in}{0.524170in}}%
\pgfpathlineto{\pgfqpoint{3.176698in}{0.779173in}}%
\pgfpathlineto{\pgfqpoint{3.134610in}{0.779173in}}%
\pgfpathlineto{\pgfqpoint{3.134610in}{0.524170in}}%
\pgfpathclose%
\pgfusepath{fill}%
\end{pgfscope}%
\begin{pgfscope}%
\pgfpathrectangle{\pgfqpoint{0.651412in}{0.524170in}}{\pgfqpoint{4.629690in}{2.558193in}}%
\pgfusepath{clip}%
\pgfsetbuttcap%
\pgfsetmiterjoin%
\definecolor{currentfill}{rgb}{0.870588,0.560784,0.019608}%
\pgfsetfillcolor{currentfill}%
\pgfsetfillopacity{0.700000}%
\pgfsetlinewidth{0.000000pt}%
\definecolor{currentstroke}{rgb}{0.000000,0.000000,0.000000}%
\pgfsetstrokecolor{currentstroke}%
\pgfsetstrokeopacity{0.700000}%
\pgfsetdash{}{0pt}%
\pgfpathmoveto{\pgfqpoint{3.176698in}{0.524170in}}%
\pgfpathlineto{\pgfqpoint{3.218786in}{0.524170in}}%
\pgfpathlineto{\pgfqpoint{3.218786in}{0.726307in}}%
\pgfpathlineto{\pgfqpoint{3.176698in}{0.726307in}}%
\pgfpathlineto{\pgfqpoint{3.176698in}{0.524170in}}%
\pgfpathclose%
\pgfusepath{fill}%
\end{pgfscope}%
\begin{pgfscope}%
\pgfpathrectangle{\pgfqpoint{0.651412in}{0.524170in}}{\pgfqpoint{4.629690in}{2.558193in}}%
\pgfusepath{clip}%
\pgfsetbuttcap%
\pgfsetmiterjoin%
\definecolor{currentfill}{rgb}{0.870588,0.560784,0.019608}%
\pgfsetfillcolor{currentfill}%
\pgfsetfillopacity{0.700000}%
\pgfsetlinewidth{0.000000pt}%
\definecolor{currentstroke}{rgb}{0.000000,0.000000,0.000000}%
\pgfsetstrokecolor{currentstroke}%
\pgfsetstrokeopacity{0.700000}%
\pgfsetdash{}{0pt}%
\pgfpathmoveto{\pgfqpoint{3.218786in}{0.524170in}}%
\pgfpathlineto{\pgfqpoint{3.260874in}{0.524170in}}%
\pgfpathlineto{\pgfqpoint{3.260874in}{0.710758in}}%
\pgfpathlineto{\pgfqpoint{3.218786in}{0.710758in}}%
\pgfpathlineto{\pgfqpoint{3.218786in}{0.524170in}}%
\pgfpathclose%
\pgfusepath{fill}%
\end{pgfscope}%
\begin{pgfscope}%
\pgfpathrectangle{\pgfqpoint{0.651412in}{0.524170in}}{\pgfqpoint{4.629690in}{2.558193in}}%
\pgfusepath{clip}%
\pgfsetbuttcap%
\pgfsetmiterjoin%
\definecolor{currentfill}{rgb}{0.870588,0.560784,0.019608}%
\pgfsetfillcolor{currentfill}%
\pgfsetfillopacity{0.700000}%
\pgfsetlinewidth{0.000000pt}%
\definecolor{currentstroke}{rgb}{0.000000,0.000000,0.000000}%
\pgfsetstrokecolor{currentstroke}%
\pgfsetstrokeopacity{0.700000}%
\pgfsetdash{}{0pt}%
\pgfpathmoveto{\pgfqpoint{3.260874in}{0.524170in}}%
\pgfpathlineto{\pgfqpoint{3.302962in}{0.524170in}}%
\pgfpathlineto{\pgfqpoint{3.302962in}{0.673440in}}%
\pgfpathlineto{\pgfqpoint{3.260874in}{0.673440in}}%
\pgfpathlineto{\pgfqpoint{3.260874in}{0.524170in}}%
\pgfpathclose%
\pgfusepath{fill}%
\end{pgfscope}%
\begin{pgfscope}%
\pgfpathrectangle{\pgfqpoint{0.651412in}{0.524170in}}{\pgfqpoint{4.629690in}{2.558193in}}%
\pgfusepath{clip}%
\pgfsetbuttcap%
\pgfsetmiterjoin%
\definecolor{currentfill}{rgb}{0.870588,0.560784,0.019608}%
\pgfsetfillcolor{currentfill}%
\pgfsetfillopacity{0.700000}%
\pgfsetlinewidth{0.000000pt}%
\definecolor{currentstroke}{rgb}{0.000000,0.000000,0.000000}%
\pgfsetstrokecolor{currentstroke}%
\pgfsetstrokeopacity{0.700000}%
\pgfsetdash{}{0pt}%
\pgfpathmoveto{\pgfqpoint{3.302962in}{0.524170in}}%
\pgfpathlineto{\pgfqpoint{3.345050in}{0.524170in}}%
\pgfpathlineto{\pgfqpoint{3.345050in}{0.667220in}}%
\pgfpathlineto{\pgfqpoint{3.302962in}{0.667220in}}%
\pgfpathlineto{\pgfqpoint{3.302962in}{0.524170in}}%
\pgfpathclose%
\pgfusepath{fill}%
\end{pgfscope}%
\begin{pgfscope}%
\pgfpathrectangle{\pgfqpoint{0.651412in}{0.524170in}}{\pgfqpoint{4.629690in}{2.558193in}}%
\pgfusepath{clip}%
\pgfsetbuttcap%
\pgfsetmiterjoin%
\definecolor{currentfill}{rgb}{0.870588,0.560784,0.019608}%
\pgfsetfillcolor{currentfill}%
\pgfsetfillopacity{0.700000}%
\pgfsetlinewidth{0.000000pt}%
\definecolor{currentstroke}{rgb}{0.000000,0.000000,0.000000}%
\pgfsetstrokecolor{currentstroke}%
\pgfsetstrokeopacity{0.700000}%
\pgfsetdash{}{0pt}%
\pgfpathmoveto{\pgfqpoint{3.345050in}{0.524170in}}%
\pgfpathlineto{\pgfqpoint{3.387138in}{0.524170in}}%
\pgfpathlineto{\pgfqpoint{3.387138in}{0.682769in}}%
\pgfpathlineto{\pgfqpoint{3.345050in}{0.682769in}}%
\pgfpathlineto{\pgfqpoint{3.345050in}{0.524170in}}%
\pgfpathclose%
\pgfusepath{fill}%
\end{pgfscope}%
\begin{pgfscope}%
\pgfpathrectangle{\pgfqpoint{0.651412in}{0.524170in}}{\pgfqpoint{4.629690in}{2.558193in}}%
\pgfusepath{clip}%
\pgfsetbuttcap%
\pgfsetmiterjoin%
\definecolor{currentfill}{rgb}{0.870588,0.560784,0.019608}%
\pgfsetfillcolor{currentfill}%
\pgfsetfillopacity{0.700000}%
\pgfsetlinewidth{0.000000pt}%
\definecolor{currentstroke}{rgb}{0.000000,0.000000,0.000000}%
\pgfsetstrokecolor{currentstroke}%
\pgfsetstrokeopacity{0.700000}%
\pgfsetdash{}{0pt}%
\pgfpathmoveto{\pgfqpoint{3.387138in}{0.524170in}}%
\pgfpathlineto{\pgfqpoint{3.429226in}{0.524170in}}%
\pgfpathlineto{\pgfqpoint{3.429226in}{0.720087in}}%
\pgfpathlineto{\pgfqpoint{3.387138in}{0.720087in}}%
\pgfpathlineto{\pgfqpoint{3.387138in}{0.524170in}}%
\pgfpathclose%
\pgfusepath{fill}%
\end{pgfscope}%
\begin{pgfscope}%
\pgfpathrectangle{\pgfqpoint{0.651412in}{0.524170in}}{\pgfqpoint{4.629690in}{2.558193in}}%
\pgfusepath{clip}%
\pgfsetbuttcap%
\pgfsetmiterjoin%
\definecolor{currentfill}{rgb}{0.870588,0.560784,0.019608}%
\pgfsetfillcolor{currentfill}%
\pgfsetfillopacity{0.700000}%
\pgfsetlinewidth{0.000000pt}%
\definecolor{currentstroke}{rgb}{0.000000,0.000000,0.000000}%
\pgfsetstrokecolor{currentstroke}%
\pgfsetstrokeopacity{0.700000}%
\pgfsetdash{}{0pt}%
\pgfpathmoveto{\pgfqpoint{3.429226in}{0.524170in}}%
\pgfpathlineto{\pgfqpoint{3.471314in}{0.524170in}}%
\pgfpathlineto{\pgfqpoint{3.471314in}{0.723197in}}%
\pgfpathlineto{\pgfqpoint{3.429226in}{0.723197in}}%
\pgfpathlineto{\pgfqpoint{3.429226in}{0.524170in}}%
\pgfpathclose%
\pgfusepath{fill}%
\end{pgfscope}%
\begin{pgfscope}%
\pgfpathrectangle{\pgfqpoint{0.651412in}{0.524170in}}{\pgfqpoint{4.629690in}{2.558193in}}%
\pgfusepath{clip}%
\pgfsetbuttcap%
\pgfsetmiterjoin%
\definecolor{currentfill}{rgb}{0.870588,0.560784,0.019608}%
\pgfsetfillcolor{currentfill}%
\pgfsetfillopacity{0.700000}%
\pgfsetlinewidth{0.000000pt}%
\definecolor{currentstroke}{rgb}{0.000000,0.000000,0.000000}%
\pgfsetstrokecolor{currentstroke}%
\pgfsetstrokeopacity{0.700000}%
\pgfsetdash{}{0pt}%
\pgfpathmoveto{\pgfqpoint{3.471314in}{0.524170in}}%
\pgfpathlineto{\pgfqpoint{3.513403in}{0.524170in}}%
\pgfpathlineto{\pgfqpoint{3.513403in}{0.679660in}}%
\pgfpathlineto{\pgfqpoint{3.471314in}{0.679660in}}%
\pgfpathlineto{\pgfqpoint{3.471314in}{0.524170in}}%
\pgfpathclose%
\pgfusepath{fill}%
\end{pgfscope}%
\begin{pgfscope}%
\pgfpathrectangle{\pgfqpoint{0.651412in}{0.524170in}}{\pgfqpoint{4.629690in}{2.558193in}}%
\pgfusepath{clip}%
\pgfsetbuttcap%
\pgfsetmiterjoin%
\definecolor{currentfill}{rgb}{0.870588,0.560784,0.019608}%
\pgfsetfillcolor{currentfill}%
\pgfsetfillopacity{0.700000}%
\pgfsetlinewidth{0.000000pt}%
\definecolor{currentstroke}{rgb}{0.000000,0.000000,0.000000}%
\pgfsetstrokecolor{currentstroke}%
\pgfsetstrokeopacity{0.700000}%
\pgfsetdash{}{0pt}%
\pgfpathmoveto{\pgfqpoint{3.513403in}{0.524170in}}%
\pgfpathlineto{\pgfqpoint{3.555491in}{0.524170in}}%
\pgfpathlineto{\pgfqpoint{3.555491in}{0.701428in}}%
\pgfpathlineto{\pgfqpoint{3.513403in}{0.701428in}}%
\pgfpathlineto{\pgfqpoint{3.513403in}{0.524170in}}%
\pgfpathclose%
\pgfusepath{fill}%
\end{pgfscope}%
\begin{pgfscope}%
\pgfpathrectangle{\pgfqpoint{0.651412in}{0.524170in}}{\pgfqpoint{4.629690in}{2.558193in}}%
\pgfusepath{clip}%
\pgfsetbuttcap%
\pgfsetmiterjoin%
\definecolor{currentfill}{rgb}{0.870588,0.560784,0.019608}%
\pgfsetfillcolor{currentfill}%
\pgfsetfillopacity{0.700000}%
\pgfsetlinewidth{0.000000pt}%
\definecolor{currentstroke}{rgb}{0.000000,0.000000,0.000000}%
\pgfsetstrokecolor{currentstroke}%
\pgfsetstrokeopacity{0.700000}%
\pgfsetdash{}{0pt}%
\pgfpathmoveto{\pgfqpoint{3.555491in}{0.524170in}}%
\pgfpathlineto{\pgfqpoint{3.597579in}{0.524170in}}%
\pgfpathlineto{\pgfqpoint{3.597579in}{0.657891in}}%
\pgfpathlineto{\pgfqpoint{3.555491in}{0.657891in}}%
\pgfpathlineto{\pgfqpoint{3.555491in}{0.524170in}}%
\pgfpathclose%
\pgfusepath{fill}%
\end{pgfscope}%
\begin{pgfscope}%
\pgfpathrectangle{\pgfqpoint{0.651412in}{0.524170in}}{\pgfqpoint{4.629690in}{2.558193in}}%
\pgfusepath{clip}%
\pgfsetbuttcap%
\pgfsetmiterjoin%
\definecolor{currentfill}{rgb}{0.870588,0.560784,0.019608}%
\pgfsetfillcolor{currentfill}%
\pgfsetfillopacity{0.700000}%
\pgfsetlinewidth{0.000000pt}%
\definecolor{currentstroke}{rgb}{0.000000,0.000000,0.000000}%
\pgfsetstrokecolor{currentstroke}%
\pgfsetstrokeopacity{0.700000}%
\pgfsetdash{}{0pt}%
\pgfpathmoveto{\pgfqpoint{3.597579in}{0.524170in}}%
\pgfpathlineto{\pgfqpoint{3.639667in}{0.524170in}}%
\pgfpathlineto{\pgfqpoint{3.639667in}{0.682769in}}%
\pgfpathlineto{\pgfqpoint{3.597579in}{0.682769in}}%
\pgfpathlineto{\pgfqpoint{3.597579in}{0.524170in}}%
\pgfpathclose%
\pgfusepath{fill}%
\end{pgfscope}%
\begin{pgfscope}%
\pgfpathrectangle{\pgfqpoint{0.651412in}{0.524170in}}{\pgfqpoint{4.629690in}{2.558193in}}%
\pgfusepath{clip}%
\pgfsetbuttcap%
\pgfsetmiterjoin%
\definecolor{currentfill}{rgb}{0.870588,0.560784,0.019608}%
\pgfsetfillcolor{currentfill}%
\pgfsetfillopacity{0.700000}%
\pgfsetlinewidth{0.000000pt}%
\definecolor{currentstroke}{rgb}{0.000000,0.000000,0.000000}%
\pgfsetstrokecolor{currentstroke}%
\pgfsetstrokeopacity{0.700000}%
\pgfsetdash{}{0pt}%
\pgfpathmoveto{\pgfqpoint{3.639667in}{0.524170in}}%
\pgfpathlineto{\pgfqpoint{3.681755in}{0.524170in}}%
\pgfpathlineto{\pgfqpoint{3.681755in}{0.651671in}}%
\pgfpathlineto{\pgfqpoint{3.639667in}{0.651671in}}%
\pgfpathlineto{\pgfqpoint{3.639667in}{0.524170in}}%
\pgfpathclose%
\pgfusepath{fill}%
\end{pgfscope}%
\begin{pgfscope}%
\pgfpathrectangle{\pgfqpoint{0.651412in}{0.524170in}}{\pgfqpoint{4.629690in}{2.558193in}}%
\pgfusepath{clip}%
\pgfsetbuttcap%
\pgfsetmiterjoin%
\definecolor{currentfill}{rgb}{0.870588,0.560784,0.019608}%
\pgfsetfillcolor{currentfill}%
\pgfsetfillopacity{0.700000}%
\pgfsetlinewidth{0.000000pt}%
\definecolor{currentstroke}{rgb}{0.000000,0.000000,0.000000}%
\pgfsetstrokecolor{currentstroke}%
\pgfsetstrokeopacity{0.700000}%
\pgfsetdash{}{0pt}%
\pgfpathmoveto{\pgfqpoint{3.681755in}{0.524170in}}%
\pgfpathlineto{\pgfqpoint{3.723843in}{0.524170in}}%
\pgfpathlineto{\pgfqpoint{3.723843in}{0.670330in}}%
\pgfpathlineto{\pgfqpoint{3.681755in}{0.670330in}}%
\pgfpathlineto{\pgfqpoint{3.681755in}{0.524170in}}%
\pgfpathclose%
\pgfusepath{fill}%
\end{pgfscope}%
\begin{pgfscope}%
\pgfpathrectangle{\pgfqpoint{0.651412in}{0.524170in}}{\pgfqpoint{4.629690in}{2.558193in}}%
\pgfusepath{clip}%
\pgfsetbuttcap%
\pgfsetmiterjoin%
\definecolor{currentfill}{rgb}{0.870588,0.560784,0.019608}%
\pgfsetfillcolor{currentfill}%
\pgfsetfillopacity{0.700000}%
\pgfsetlinewidth{0.000000pt}%
\definecolor{currentstroke}{rgb}{0.000000,0.000000,0.000000}%
\pgfsetstrokecolor{currentstroke}%
\pgfsetstrokeopacity{0.700000}%
\pgfsetdash{}{0pt}%
\pgfpathmoveto{\pgfqpoint{3.723843in}{0.524170in}}%
\pgfpathlineto{\pgfqpoint{3.765931in}{0.524170in}}%
\pgfpathlineto{\pgfqpoint{3.765931in}{0.698318in}}%
\pgfpathlineto{\pgfqpoint{3.723843in}{0.698318in}}%
\pgfpathlineto{\pgfqpoint{3.723843in}{0.524170in}}%
\pgfpathclose%
\pgfusepath{fill}%
\end{pgfscope}%
\begin{pgfscope}%
\pgfpathrectangle{\pgfqpoint{0.651412in}{0.524170in}}{\pgfqpoint{4.629690in}{2.558193in}}%
\pgfusepath{clip}%
\pgfsetbuttcap%
\pgfsetmiterjoin%
\definecolor{currentfill}{rgb}{0.870588,0.560784,0.019608}%
\pgfsetfillcolor{currentfill}%
\pgfsetfillopacity{0.700000}%
\pgfsetlinewidth{0.000000pt}%
\definecolor{currentstroke}{rgb}{0.000000,0.000000,0.000000}%
\pgfsetstrokecolor{currentstroke}%
\pgfsetstrokeopacity{0.700000}%
\pgfsetdash{}{0pt}%
\pgfpathmoveto{\pgfqpoint{3.765931in}{0.524170in}}%
\pgfpathlineto{\pgfqpoint{3.808019in}{0.524170in}}%
\pgfpathlineto{\pgfqpoint{3.808019in}{0.654781in}}%
\pgfpathlineto{\pgfqpoint{3.765931in}{0.654781in}}%
\pgfpathlineto{\pgfqpoint{3.765931in}{0.524170in}}%
\pgfpathclose%
\pgfusepath{fill}%
\end{pgfscope}%
\begin{pgfscope}%
\pgfpathrectangle{\pgfqpoint{0.651412in}{0.524170in}}{\pgfqpoint{4.629690in}{2.558193in}}%
\pgfusepath{clip}%
\pgfsetbuttcap%
\pgfsetmiterjoin%
\definecolor{currentfill}{rgb}{0.870588,0.560784,0.019608}%
\pgfsetfillcolor{currentfill}%
\pgfsetfillopacity{0.700000}%
\pgfsetlinewidth{0.000000pt}%
\definecolor{currentstroke}{rgb}{0.000000,0.000000,0.000000}%
\pgfsetstrokecolor{currentstroke}%
\pgfsetstrokeopacity{0.700000}%
\pgfsetdash{}{0pt}%
\pgfpathmoveto{\pgfqpoint{3.808019in}{0.524170in}}%
\pgfpathlineto{\pgfqpoint{3.850107in}{0.524170in}}%
\pgfpathlineto{\pgfqpoint{3.850107in}{0.642342in}}%
\pgfpathlineto{\pgfqpoint{3.808019in}{0.642342in}}%
\pgfpathlineto{\pgfqpoint{3.808019in}{0.524170in}}%
\pgfpathclose%
\pgfusepath{fill}%
\end{pgfscope}%
\begin{pgfscope}%
\pgfpathrectangle{\pgfqpoint{0.651412in}{0.524170in}}{\pgfqpoint{4.629690in}{2.558193in}}%
\pgfusepath{clip}%
\pgfsetbuttcap%
\pgfsetmiterjoin%
\definecolor{currentfill}{rgb}{0.870588,0.560784,0.019608}%
\pgfsetfillcolor{currentfill}%
\pgfsetfillopacity{0.700000}%
\pgfsetlinewidth{0.000000pt}%
\definecolor{currentstroke}{rgb}{0.000000,0.000000,0.000000}%
\pgfsetstrokecolor{currentstroke}%
\pgfsetstrokeopacity{0.700000}%
\pgfsetdash{}{0pt}%
\pgfpathmoveto{\pgfqpoint{3.850107in}{0.524170in}}%
\pgfpathlineto{\pgfqpoint{3.892195in}{0.524170in}}%
\pgfpathlineto{\pgfqpoint{3.892195in}{0.651671in}}%
\pgfpathlineto{\pgfqpoint{3.850107in}{0.651671in}}%
\pgfpathlineto{\pgfqpoint{3.850107in}{0.524170in}}%
\pgfpathclose%
\pgfusepath{fill}%
\end{pgfscope}%
\begin{pgfscope}%
\pgfpathrectangle{\pgfqpoint{0.651412in}{0.524170in}}{\pgfqpoint{4.629690in}{2.558193in}}%
\pgfusepath{clip}%
\pgfsetbuttcap%
\pgfsetmiterjoin%
\definecolor{currentfill}{rgb}{0.870588,0.560784,0.019608}%
\pgfsetfillcolor{currentfill}%
\pgfsetfillopacity{0.700000}%
\pgfsetlinewidth{0.000000pt}%
\definecolor{currentstroke}{rgb}{0.000000,0.000000,0.000000}%
\pgfsetstrokecolor{currentstroke}%
\pgfsetstrokeopacity{0.700000}%
\pgfsetdash{}{0pt}%
\pgfpathmoveto{\pgfqpoint{3.892195in}{0.524170in}}%
\pgfpathlineto{\pgfqpoint{3.934283in}{0.524170in}}%
\pgfpathlineto{\pgfqpoint{3.934283in}{0.670330in}}%
\pgfpathlineto{\pgfqpoint{3.892195in}{0.670330in}}%
\pgfpathlineto{\pgfqpoint{3.892195in}{0.524170in}}%
\pgfpathclose%
\pgfusepath{fill}%
\end{pgfscope}%
\begin{pgfscope}%
\pgfpathrectangle{\pgfqpoint{0.651412in}{0.524170in}}{\pgfqpoint{4.629690in}{2.558193in}}%
\pgfusepath{clip}%
\pgfsetbuttcap%
\pgfsetmiterjoin%
\definecolor{currentfill}{rgb}{0.870588,0.560784,0.019608}%
\pgfsetfillcolor{currentfill}%
\pgfsetfillopacity{0.700000}%
\pgfsetlinewidth{0.000000pt}%
\definecolor{currentstroke}{rgb}{0.000000,0.000000,0.000000}%
\pgfsetstrokecolor{currentstroke}%
\pgfsetstrokeopacity{0.700000}%
\pgfsetdash{}{0pt}%
\pgfpathmoveto{\pgfqpoint{3.934283in}{0.524170in}}%
\pgfpathlineto{\pgfqpoint{3.976372in}{0.524170in}}%
\pgfpathlineto{\pgfqpoint{3.976372in}{0.636122in}}%
\pgfpathlineto{\pgfqpoint{3.934283in}{0.636122in}}%
\pgfpathlineto{\pgfqpoint{3.934283in}{0.524170in}}%
\pgfpathclose%
\pgfusepath{fill}%
\end{pgfscope}%
\begin{pgfscope}%
\pgfpathrectangle{\pgfqpoint{0.651412in}{0.524170in}}{\pgfqpoint{4.629690in}{2.558193in}}%
\pgfusepath{clip}%
\pgfsetbuttcap%
\pgfsetmiterjoin%
\definecolor{currentfill}{rgb}{0.870588,0.560784,0.019608}%
\pgfsetfillcolor{currentfill}%
\pgfsetfillopacity{0.700000}%
\pgfsetlinewidth{0.000000pt}%
\definecolor{currentstroke}{rgb}{0.000000,0.000000,0.000000}%
\pgfsetstrokecolor{currentstroke}%
\pgfsetstrokeopacity{0.700000}%
\pgfsetdash{}{0pt}%
\pgfpathmoveto{\pgfqpoint{3.976372in}{0.524170in}}%
\pgfpathlineto{\pgfqpoint{4.018460in}{0.524170in}}%
\pgfpathlineto{\pgfqpoint{4.018460in}{0.642342in}}%
\pgfpathlineto{\pgfqpoint{3.976372in}{0.642342in}}%
\pgfpathlineto{\pgfqpoint{3.976372in}{0.524170in}}%
\pgfpathclose%
\pgfusepath{fill}%
\end{pgfscope}%
\begin{pgfscope}%
\pgfpathrectangle{\pgfqpoint{0.651412in}{0.524170in}}{\pgfqpoint{4.629690in}{2.558193in}}%
\pgfusepath{clip}%
\pgfsetbuttcap%
\pgfsetmiterjoin%
\definecolor{currentfill}{rgb}{0.870588,0.560784,0.019608}%
\pgfsetfillcolor{currentfill}%
\pgfsetfillopacity{0.700000}%
\pgfsetlinewidth{0.000000pt}%
\definecolor{currentstroke}{rgb}{0.000000,0.000000,0.000000}%
\pgfsetstrokecolor{currentstroke}%
\pgfsetstrokeopacity{0.700000}%
\pgfsetdash{}{0pt}%
\pgfpathmoveto{\pgfqpoint{4.018460in}{0.524170in}}%
\pgfpathlineto{\pgfqpoint{4.060548in}{0.524170in}}%
\pgfpathlineto{\pgfqpoint{4.060548in}{0.648562in}}%
\pgfpathlineto{\pgfqpoint{4.018460in}{0.648562in}}%
\pgfpathlineto{\pgfqpoint{4.018460in}{0.524170in}}%
\pgfpathclose%
\pgfusepath{fill}%
\end{pgfscope}%
\begin{pgfscope}%
\pgfpathrectangle{\pgfqpoint{0.651412in}{0.524170in}}{\pgfqpoint{4.629690in}{2.558193in}}%
\pgfusepath{clip}%
\pgfsetbuttcap%
\pgfsetmiterjoin%
\definecolor{currentfill}{rgb}{0.870588,0.560784,0.019608}%
\pgfsetfillcolor{currentfill}%
\pgfsetfillopacity{0.700000}%
\pgfsetlinewidth{0.000000pt}%
\definecolor{currentstroke}{rgb}{0.000000,0.000000,0.000000}%
\pgfsetstrokecolor{currentstroke}%
\pgfsetstrokeopacity{0.700000}%
\pgfsetdash{}{0pt}%
\pgfpathmoveto{\pgfqpoint{4.060548in}{0.524170in}}%
\pgfpathlineto{\pgfqpoint{4.102636in}{0.524170in}}%
\pgfpathlineto{\pgfqpoint{4.102636in}{0.664111in}}%
\pgfpathlineto{\pgfqpoint{4.060548in}{0.664111in}}%
\pgfpathlineto{\pgfqpoint{4.060548in}{0.524170in}}%
\pgfpathclose%
\pgfusepath{fill}%
\end{pgfscope}%
\begin{pgfscope}%
\pgfpathrectangle{\pgfqpoint{0.651412in}{0.524170in}}{\pgfqpoint{4.629690in}{2.558193in}}%
\pgfusepath{clip}%
\pgfsetbuttcap%
\pgfsetmiterjoin%
\definecolor{currentfill}{rgb}{0.870588,0.560784,0.019608}%
\pgfsetfillcolor{currentfill}%
\pgfsetfillopacity{0.700000}%
\pgfsetlinewidth{0.000000pt}%
\definecolor{currentstroke}{rgb}{0.000000,0.000000,0.000000}%
\pgfsetstrokecolor{currentstroke}%
\pgfsetstrokeopacity{0.700000}%
\pgfsetdash{}{0pt}%
\pgfpathmoveto{\pgfqpoint{4.102636in}{0.524170in}}%
\pgfpathlineto{\pgfqpoint{4.144724in}{0.524170in}}%
\pgfpathlineto{\pgfqpoint{4.144724in}{0.611244in}}%
\pgfpathlineto{\pgfqpoint{4.102636in}{0.611244in}}%
\pgfpathlineto{\pgfqpoint{4.102636in}{0.524170in}}%
\pgfpathclose%
\pgfusepath{fill}%
\end{pgfscope}%
\begin{pgfscope}%
\pgfpathrectangle{\pgfqpoint{0.651412in}{0.524170in}}{\pgfqpoint{4.629690in}{2.558193in}}%
\pgfusepath{clip}%
\pgfsetbuttcap%
\pgfsetmiterjoin%
\definecolor{currentfill}{rgb}{0.870588,0.560784,0.019608}%
\pgfsetfillcolor{currentfill}%
\pgfsetfillopacity{0.700000}%
\pgfsetlinewidth{0.000000pt}%
\definecolor{currentstroke}{rgb}{0.000000,0.000000,0.000000}%
\pgfsetstrokecolor{currentstroke}%
\pgfsetstrokeopacity{0.700000}%
\pgfsetdash{}{0pt}%
\pgfpathmoveto{\pgfqpoint{4.144724in}{0.524170in}}%
\pgfpathlineto{\pgfqpoint{4.186812in}{0.524170in}}%
\pgfpathlineto{\pgfqpoint{4.186812in}{0.645452in}}%
\pgfpathlineto{\pgfqpoint{4.144724in}{0.645452in}}%
\pgfpathlineto{\pgfqpoint{4.144724in}{0.524170in}}%
\pgfpathclose%
\pgfusepath{fill}%
\end{pgfscope}%
\begin{pgfscope}%
\pgfpathrectangle{\pgfqpoint{0.651412in}{0.524170in}}{\pgfqpoint{4.629690in}{2.558193in}}%
\pgfusepath{clip}%
\pgfsetbuttcap%
\pgfsetmiterjoin%
\definecolor{currentfill}{rgb}{0.870588,0.560784,0.019608}%
\pgfsetfillcolor{currentfill}%
\pgfsetfillopacity{0.700000}%
\pgfsetlinewidth{0.000000pt}%
\definecolor{currentstroke}{rgb}{0.000000,0.000000,0.000000}%
\pgfsetstrokecolor{currentstroke}%
\pgfsetstrokeopacity{0.700000}%
\pgfsetdash{}{0pt}%
\pgfpathmoveto{\pgfqpoint{4.186812in}{0.524170in}}%
\pgfpathlineto{\pgfqpoint{4.228900in}{0.524170in}}%
\pgfpathlineto{\pgfqpoint{4.228900in}{0.598805in}}%
\pgfpathlineto{\pgfqpoint{4.186812in}{0.598805in}}%
\pgfpathlineto{\pgfqpoint{4.186812in}{0.524170in}}%
\pgfpathclose%
\pgfusepath{fill}%
\end{pgfscope}%
\begin{pgfscope}%
\pgfpathrectangle{\pgfqpoint{0.651412in}{0.524170in}}{\pgfqpoint{4.629690in}{2.558193in}}%
\pgfusepath{clip}%
\pgfsetbuttcap%
\pgfsetmiterjoin%
\definecolor{currentfill}{rgb}{0.870588,0.560784,0.019608}%
\pgfsetfillcolor{currentfill}%
\pgfsetfillopacity{0.700000}%
\pgfsetlinewidth{0.000000pt}%
\definecolor{currentstroke}{rgb}{0.000000,0.000000,0.000000}%
\pgfsetstrokecolor{currentstroke}%
\pgfsetstrokeopacity{0.700000}%
\pgfsetdash{}{0pt}%
\pgfpathmoveto{\pgfqpoint{4.228900in}{0.524170in}}%
\pgfpathlineto{\pgfqpoint{4.270988in}{0.524170in}}%
\pgfpathlineto{\pgfqpoint{4.270988in}{0.614354in}}%
\pgfpathlineto{\pgfqpoint{4.228900in}{0.614354in}}%
\pgfpathlineto{\pgfqpoint{4.228900in}{0.524170in}}%
\pgfpathclose%
\pgfusepath{fill}%
\end{pgfscope}%
\begin{pgfscope}%
\pgfpathrectangle{\pgfqpoint{0.651412in}{0.524170in}}{\pgfqpoint{4.629690in}{2.558193in}}%
\pgfusepath{clip}%
\pgfsetbuttcap%
\pgfsetmiterjoin%
\definecolor{currentfill}{rgb}{0.870588,0.560784,0.019608}%
\pgfsetfillcolor{currentfill}%
\pgfsetfillopacity{0.700000}%
\pgfsetlinewidth{0.000000pt}%
\definecolor{currentstroke}{rgb}{0.000000,0.000000,0.000000}%
\pgfsetstrokecolor{currentstroke}%
\pgfsetstrokeopacity{0.700000}%
\pgfsetdash{}{0pt}%
\pgfpathmoveto{\pgfqpoint{4.270988in}{0.524170in}}%
\pgfpathlineto{\pgfqpoint{4.313076in}{0.524170in}}%
\pgfpathlineto{\pgfqpoint{4.313076in}{0.636122in}}%
\pgfpathlineto{\pgfqpoint{4.270988in}{0.636122in}}%
\pgfpathlineto{\pgfqpoint{4.270988in}{0.524170in}}%
\pgfpathclose%
\pgfusepath{fill}%
\end{pgfscope}%
\begin{pgfscope}%
\pgfpathrectangle{\pgfqpoint{0.651412in}{0.524170in}}{\pgfqpoint{4.629690in}{2.558193in}}%
\pgfusepath{clip}%
\pgfsetbuttcap%
\pgfsetmiterjoin%
\definecolor{currentfill}{rgb}{0.870588,0.560784,0.019608}%
\pgfsetfillcolor{currentfill}%
\pgfsetfillopacity{0.700000}%
\pgfsetlinewidth{0.000000pt}%
\definecolor{currentstroke}{rgb}{0.000000,0.000000,0.000000}%
\pgfsetstrokecolor{currentstroke}%
\pgfsetstrokeopacity{0.700000}%
\pgfsetdash{}{0pt}%
\pgfpathmoveto{\pgfqpoint{4.313076in}{0.524170in}}%
\pgfpathlineto{\pgfqpoint{4.355164in}{0.524170in}}%
\pgfpathlineto{\pgfqpoint{4.355164in}{0.614354in}}%
\pgfpathlineto{\pgfqpoint{4.313076in}{0.614354in}}%
\pgfpathlineto{\pgfqpoint{4.313076in}{0.524170in}}%
\pgfpathclose%
\pgfusepath{fill}%
\end{pgfscope}%
\begin{pgfscope}%
\pgfpathrectangle{\pgfqpoint{0.651412in}{0.524170in}}{\pgfqpoint{4.629690in}{2.558193in}}%
\pgfusepath{clip}%
\pgfsetbuttcap%
\pgfsetmiterjoin%
\definecolor{currentfill}{rgb}{0.870588,0.560784,0.019608}%
\pgfsetfillcolor{currentfill}%
\pgfsetfillopacity{0.700000}%
\pgfsetlinewidth{0.000000pt}%
\definecolor{currentstroke}{rgb}{0.000000,0.000000,0.000000}%
\pgfsetstrokecolor{currentstroke}%
\pgfsetstrokeopacity{0.700000}%
\pgfsetdash{}{0pt}%
\pgfpathmoveto{\pgfqpoint{4.355164in}{0.524170in}}%
\pgfpathlineto{\pgfqpoint{4.397253in}{0.524170in}}%
\pgfpathlineto{\pgfqpoint{4.397253in}{0.639232in}}%
\pgfpathlineto{\pgfqpoint{4.355164in}{0.639232in}}%
\pgfpathlineto{\pgfqpoint{4.355164in}{0.524170in}}%
\pgfpathclose%
\pgfusepath{fill}%
\end{pgfscope}%
\begin{pgfscope}%
\pgfpathrectangle{\pgfqpoint{0.651412in}{0.524170in}}{\pgfqpoint{4.629690in}{2.558193in}}%
\pgfusepath{clip}%
\pgfsetbuttcap%
\pgfsetmiterjoin%
\definecolor{currentfill}{rgb}{0.870588,0.560784,0.019608}%
\pgfsetfillcolor{currentfill}%
\pgfsetfillopacity{0.700000}%
\pgfsetlinewidth{0.000000pt}%
\definecolor{currentstroke}{rgb}{0.000000,0.000000,0.000000}%
\pgfsetstrokecolor{currentstroke}%
\pgfsetstrokeopacity{0.700000}%
\pgfsetdash{}{0pt}%
\pgfpathmoveto{\pgfqpoint{4.397253in}{0.524170in}}%
\pgfpathlineto{\pgfqpoint{4.439341in}{0.524170in}}%
\pgfpathlineto{\pgfqpoint{4.439341in}{0.633013in}}%
\pgfpathlineto{\pgfqpoint{4.397253in}{0.633013in}}%
\pgfpathlineto{\pgfqpoint{4.397253in}{0.524170in}}%
\pgfpathclose%
\pgfusepath{fill}%
\end{pgfscope}%
\begin{pgfscope}%
\pgfpathrectangle{\pgfqpoint{0.651412in}{0.524170in}}{\pgfqpoint{4.629690in}{2.558193in}}%
\pgfusepath{clip}%
\pgfsetbuttcap%
\pgfsetmiterjoin%
\definecolor{currentfill}{rgb}{0.870588,0.560784,0.019608}%
\pgfsetfillcolor{currentfill}%
\pgfsetfillopacity{0.700000}%
\pgfsetlinewidth{0.000000pt}%
\definecolor{currentstroke}{rgb}{0.000000,0.000000,0.000000}%
\pgfsetstrokecolor{currentstroke}%
\pgfsetstrokeopacity{0.700000}%
\pgfsetdash{}{0pt}%
\pgfpathmoveto{\pgfqpoint{4.439341in}{0.524170in}}%
\pgfpathlineto{\pgfqpoint{4.481429in}{0.524170in}}%
\pgfpathlineto{\pgfqpoint{4.481429in}{0.601915in}}%
\pgfpathlineto{\pgfqpoint{4.439341in}{0.601915in}}%
\pgfpathlineto{\pgfqpoint{4.439341in}{0.524170in}}%
\pgfpathclose%
\pgfusepath{fill}%
\end{pgfscope}%
\begin{pgfscope}%
\pgfpathrectangle{\pgfqpoint{0.651412in}{0.524170in}}{\pgfqpoint{4.629690in}{2.558193in}}%
\pgfusepath{clip}%
\pgfsetbuttcap%
\pgfsetmiterjoin%
\definecolor{currentfill}{rgb}{0.870588,0.560784,0.019608}%
\pgfsetfillcolor{currentfill}%
\pgfsetfillopacity{0.700000}%
\pgfsetlinewidth{0.000000pt}%
\definecolor{currentstroke}{rgb}{0.000000,0.000000,0.000000}%
\pgfsetstrokecolor{currentstroke}%
\pgfsetstrokeopacity{0.700000}%
\pgfsetdash{}{0pt}%
\pgfpathmoveto{\pgfqpoint{4.481429in}{0.524170in}}%
\pgfpathlineto{\pgfqpoint{4.523517in}{0.524170in}}%
\pgfpathlineto{\pgfqpoint{4.523517in}{0.657891in}}%
\pgfpathlineto{\pgfqpoint{4.481429in}{0.657891in}}%
\pgfpathlineto{\pgfqpoint{4.481429in}{0.524170in}}%
\pgfpathclose%
\pgfusepath{fill}%
\end{pgfscope}%
\begin{pgfscope}%
\pgfpathrectangle{\pgfqpoint{0.651412in}{0.524170in}}{\pgfqpoint{4.629690in}{2.558193in}}%
\pgfusepath{clip}%
\pgfsetbuttcap%
\pgfsetmiterjoin%
\definecolor{currentfill}{rgb}{0.870588,0.560784,0.019608}%
\pgfsetfillcolor{currentfill}%
\pgfsetfillopacity{0.700000}%
\pgfsetlinewidth{0.000000pt}%
\definecolor{currentstroke}{rgb}{0.000000,0.000000,0.000000}%
\pgfsetstrokecolor{currentstroke}%
\pgfsetstrokeopacity{0.700000}%
\pgfsetdash{}{0pt}%
\pgfpathmoveto{\pgfqpoint{4.523517in}{0.524170in}}%
\pgfpathlineto{\pgfqpoint{4.565605in}{0.524170in}}%
\pgfpathlineto{\pgfqpoint{4.565605in}{0.623683in}}%
\pgfpathlineto{\pgfqpoint{4.523517in}{0.623683in}}%
\pgfpathlineto{\pgfqpoint{4.523517in}{0.524170in}}%
\pgfpathclose%
\pgfusepath{fill}%
\end{pgfscope}%
\begin{pgfscope}%
\pgfpathrectangle{\pgfqpoint{0.651412in}{0.524170in}}{\pgfqpoint{4.629690in}{2.558193in}}%
\pgfusepath{clip}%
\pgfsetbuttcap%
\pgfsetmiterjoin%
\definecolor{currentfill}{rgb}{0.870588,0.560784,0.019608}%
\pgfsetfillcolor{currentfill}%
\pgfsetfillopacity{0.700000}%
\pgfsetlinewidth{0.000000pt}%
\definecolor{currentstroke}{rgb}{0.000000,0.000000,0.000000}%
\pgfsetstrokecolor{currentstroke}%
\pgfsetstrokeopacity{0.700000}%
\pgfsetdash{}{0pt}%
\pgfpathmoveto{\pgfqpoint{4.565605in}{0.524170in}}%
\pgfpathlineto{\pgfqpoint{4.607693in}{0.524170in}}%
\pgfpathlineto{\pgfqpoint{4.607693in}{0.601915in}}%
\pgfpathlineto{\pgfqpoint{4.565605in}{0.601915in}}%
\pgfpathlineto{\pgfqpoint{4.565605in}{0.524170in}}%
\pgfpathclose%
\pgfusepath{fill}%
\end{pgfscope}%
\begin{pgfscope}%
\pgfpathrectangle{\pgfqpoint{0.651412in}{0.524170in}}{\pgfqpoint{4.629690in}{2.558193in}}%
\pgfusepath{clip}%
\pgfsetbuttcap%
\pgfsetmiterjoin%
\definecolor{currentfill}{rgb}{0.870588,0.560784,0.019608}%
\pgfsetfillcolor{currentfill}%
\pgfsetfillopacity{0.700000}%
\pgfsetlinewidth{0.000000pt}%
\definecolor{currentstroke}{rgb}{0.000000,0.000000,0.000000}%
\pgfsetstrokecolor{currentstroke}%
\pgfsetstrokeopacity{0.700000}%
\pgfsetdash{}{0pt}%
\pgfpathmoveto{\pgfqpoint{4.607693in}{0.524170in}}%
\pgfpathlineto{\pgfqpoint{4.649781in}{0.524170in}}%
\pgfpathlineto{\pgfqpoint{4.649781in}{0.605025in}}%
\pgfpathlineto{\pgfqpoint{4.607693in}{0.605025in}}%
\pgfpathlineto{\pgfqpoint{4.607693in}{0.524170in}}%
\pgfpathclose%
\pgfusepath{fill}%
\end{pgfscope}%
\begin{pgfscope}%
\pgfpathrectangle{\pgfqpoint{0.651412in}{0.524170in}}{\pgfqpoint{4.629690in}{2.558193in}}%
\pgfusepath{clip}%
\pgfsetbuttcap%
\pgfsetmiterjoin%
\definecolor{currentfill}{rgb}{0.870588,0.560784,0.019608}%
\pgfsetfillcolor{currentfill}%
\pgfsetfillopacity{0.700000}%
\pgfsetlinewidth{0.000000pt}%
\definecolor{currentstroke}{rgb}{0.000000,0.000000,0.000000}%
\pgfsetstrokecolor{currentstroke}%
\pgfsetstrokeopacity{0.700000}%
\pgfsetdash{}{0pt}%
\pgfpathmoveto{\pgfqpoint{4.649781in}{0.524170in}}%
\pgfpathlineto{\pgfqpoint{4.691869in}{0.524170in}}%
\pgfpathlineto{\pgfqpoint{4.691869in}{0.629903in}}%
\pgfpathlineto{\pgfqpoint{4.649781in}{0.629903in}}%
\pgfpathlineto{\pgfqpoint{4.649781in}{0.524170in}}%
\pgfpathclose%
\pgfusepath{fill}%
\end{pgfscope}%
\begin{pgfscope}%
\pgfpathrectangle{\pgfqpoint{0.651412in}{0.524170in}}{\pgfqpoint{4.629690in}{2.558193in}}%
\pgfusepath{clip}%
\pgfsetbuttcap%
\pgfsetmiterjoin%
\definecolor{currentfill}{rgb}{0.870588,0.560784,0.019608}%
\pgfsetfillcolor{currentfill}%
\pgfsetfillopacity{0.700000}%
\pgfsetlinewidth{0.000000pt}%
\definecolor{currentstroke}{rgb}{0.000000,0.000000,0.000000}%
\pgfsetstrokecolor{currentstroke}%
\pgfsetstrokeopacity{0.700000}%
\pgfsetdash{}{0pt}%
\pgfpathmoveto{\pgfqpoint{4.691869in}{0.524170in}}%
\pgfpathlineto{\pgfqpoint{4.733957in}{0.524170in}}%
\pgfpathlineto{\pgfqpoint{4.733957in}{0.614354in}}%
\pgfpathlineto{\pgfqpoint{4.691869in}{0.614354in}}%
\pgfpathlineto{\pgfqpoint{4.691869in}{0.524170in}}%
\pgfpathclose%
\pgfusepath{fill}%
\end{pgfscope}%
\begin{pgfscope}%
\pgfpathrectangle{\pgfqpoint{0.651412in}{0.524170in}}{\pgfqpoint{4.629690in}{2.558193in}}%
\pgfusepath{clip}%
\pgfsetbuttcap%
\pgfsetmiterjoin%
\definecolor{currentfill}{rgb}{0.870588,0.560784,0.019608}%
\pgfsetfillcolor{currentfill}%
\pgfsetfillopacity{0.700000}%
\pgfsetlinewidth{0.000000pt}%
\definecolor{currentstroke}{rgb}{0.000000,0.000000,0.000000}%
\pgfsetstrokecolor{currentstroke}%
\pgfsetstrokeopacity{0.700000}%
\pgfsetdash{}{0pt}%
\pgfpathmoveto{\pgfqpoint{4.733957in}{0.524170in}}%
\pgfpathlineto{\pgfqpoint{4.776045in}{0.524170in}}%
\pgfpathlineto{\pgfqpoint{4.776045in}{0.620573in}}%
\pgfpathlineto{\pgfqpoint{4.733957in}{0.620573in}}%
\pgfpathlineto{\pgfqpoint{4.733957in}{0.524170in}}%
\pgfpathclose%
\pgfusepath{fill}%
\end{pgfscope}%
\begin{pgfscope}%
\pgfpathrectangle{\pgfqpoint{0.651412in}{0.524170in}}{\pgfqpoint{4.629690in}{2.558193in}}%
\pgfusepath{clip}%
\pgfsetbuttcap%
\pgfsetmiterjoin%
\definecolor{currentfill}{rgb}{0.870588,0.560784,0.019608}%
\pgfsetfillcolor{currentfill}%
\pgfsetfillopacity{0.700000}%
\pgfsetlinewidth{0.000000pt}%
\definecolor{currentstroke}{rgb}{0.000000,0.000000,0.000000}%
\pgfsetstrokecolor{currentstroke}%
\pgfsetstrokeopacity{0.700000}%
\pgfsetdash{}{0pt}%
\pgfpathmoveto{\pgfqpoint{4.776045in}{0.524170in}}%
\pgfpathlineto{\pgfqpoint{4.818133in}{0.524170in}}%
\pgfpathlineto{\pgfqpoint{4.818133in}{0.639232in}}%
\pgfpathlineto{\pgfqpoint{4.776045in}{0.639232in}}%
\pgfpathlineto{\pgfqpoint{4.776045in}{0.524170in}}%
\pgfpathclose%
\pgfusepath{fill}%
\end{pgfscope}%
\begin{pgfscope}%
\pgfpathrectangle{\pgfqpoint{0.651412in}{0.524170in}}{\pgfqpoint{4.629690in}{2.558193in}}%
\pgfusepath{clip}%
\pgfsetbuttcap%
\pgfsetmiterjoin%
\definecolor{currentfill}{rgb}{0.870588,0.560784,0.019608}%
\pgfsetfillcolor{currentfill}%
\pgfsetfillopacity{0.700000}%
\pgfsetlinewidth{0.000000pt}%
\definecolor{currentstroke}{rgb}{0.000000,0.000000,0.000000}%
\pgfsetstrokecolor{currentstroke}%
\pgfsetstrokeopacity{0.700000}%
\pgfsetdash{}{0pt}%
\pgfpathmoveto{\pgfqpoint{4.818133in}{0.524170in}}%
\pgfpathlineto{\pgfqpoint{4.860222in}{0.524170in}}%
\pgfpathlineto{\pgfqpoint{4.860222in}{0.617464in}}%
\pgfpathlineto{\pgfqpoint{4.818133in}{0.617464in}}%
\pgfpathlineto{\pgfqpoint{4.818133in}{0.524170in}}%
\pgfpathclose%
\pgfusepath{fill}%
\end{pgfscope}%
\begin{pgfscope}%
\pgfpathrectangle{\pgfqpoint{0.651412in}{0.524170in}}{\pgfqpoint{4.629690in}{2.558193in}}%
\pgfusepath{clip}%
\pgfsetbuttcap%
\pgfsetmiterjoin%
\definecolor{currentfill}{rgb}{0.870588,0.560784,0.019608}%
\pgfsetfillcolor{currentfill}%
\pgfsetfillopacity{0.700000}%
\pgfsetlinewidth{0.000000pt}%
\definecolor{currentstroke}{rgb}{0.000000,0.000000,0.000000}%
\pgfsetstrokecolor{currentstroke}%
\pgfsetstrokeopacity{0.700000}%
\pgfsetdash{}{0pt}%
\pgfpathmoveto{\pgfqpoint{4.860222in}{0.524170in}}%
\pgfpathlineto{\pgfqpoint{4.902310in}{0.524170in}}%
\pgfpathlineto{\pgfqpoint{4.902310in}{0.586366in}}%
\pgfpathlineto{\pgfqpoint{4.860222in}{0.586366in}}%
\pgfpathlineto{\pgfqpoint{4.860222in}{0.524170in}}%
\pgfpathclose%
\pgfusepath{fill}%
\end{pgfscope}%
\begin{pgfscope}%
\pgfpathrectangle{\pgfqpoint{0.651412in}{0.524170in}}{\pgfqpoint{4.629690in}{2.558193in}}%
\pgfusepath{clip}%
\pgfsetbuttcap%
\pgfsetmiterjoin%
\definecolor{currentfill}{rgb}{0.870588,0.560784,0.019608}%
\pgfsetfillcolor{currentfill}%
\pgfsetfillopacity{0.700000}%
\pgfsetlinewidth{0.000000pt}%
\definecolor{currentstroke}{rgb}{0.000000,0.000000,0.000000}%
\pgfsetstrokecolor{currentstroke}%
\pgfsetstrokeopacity{0.700000}%
\pgfsetdash{}{0pt}%
\pgfpathmoveto{\pgfqpoint{4.902310in}{0.524170in}}%
\pgfpathlineto{\pgfqpoint{4.944398in}{0.524170in}}%
\pgfpathlineto{\pgfqpoint{4.944398in}{0.608134in}}%
\pgfpathlineto{\pgfqpoint{4.902310in}{0.608134in}}%
\pgfpathlineto{\pgfqpoint{4.902310in}{0.524170in}}%
\pgfpathclose%
\pgfusepath{fill}%
\end{pgfscope}%
\begin{pgfscope}%
\pgfpathrectangle{\pgfqpoint{0.651412in}{0.524170in}}{\pgfqpoint{4.629690in}{2.558193in}}%
\pgfusepath{clip}%
\pgfsetbuttcap%
\pgfsetmiterjoin%
\definecolor{currentfill}{rgb}{0.870588,0.560784,0.019608}%
\pgfsetfillcolor{currentfill}%
\pgfsetfillopacity{0.700000}%
\pgfsetlinewidth{0.000000pt}%
\definecolor{currentstroke}{rgb}{0.000000,0.000000,0.000000}%
\pgfsetstrokecolor{currentstroke}%
\pgfsetstrokeopacity{0.700000}%
\pgfsetdash{}{0pt}%
\pgfpathmoveto{\pgfqpoint{4.944398in}{0.524170in}}%
\pgfpathlineto{\pgfqpoint{4.986486in}{0.524170in}}%
\pgfpathlineto{\pgfqpoint{4.986486in}{0.605025in}}%
\pgfpathlineto{\pgfqpoint{4.944398in}{0.605025in}}%
\pgfpathlineto{\pgfqpoint{4.944398in}{0.524170in}}%
\pgfpathclose%
\pgfusepath{fill}%
\end{pgfscope}%
\begin{pgfscope}%
\pgfpathrectangle{\pgfqpoint{0.651412in}{0.524170in}}{\pgfqpoint{4.629690in}{2.558193in}}%
\pgfusepath{clip}%
\pgfsetbuttcap%
\pgfsetmiterjoin%
\definecolor{currentfill}{rgb}{0.870588,0.560784,0.019608}%
\pgfsetfillcolor{currentfill}%
\pgfsetfillopacity{0.700000}%
\pgfsetlinewidth{0.000000pt}%
\definecolor{currentstroke}{rgb}{0.000000,0.000000,0.000000}%
\pgfsetstrokecolor{currentstroke}%
\pgfsetstrokeopacity{0.700000}%
\pgfsetdash{}{0pt}%
\pgfpathmoveto{\pgfqpoint{4.986486in}{0.524170in}}%
\pgfpathlineto{\pgfqpoint{5.028574in}{0.524170in}}%
\pgfpathlineto{\pgfqpoint{5.028574in}{0.589476in}}%
\pgfpathlineto{\pgfqpoint{4.986486in}{0.589476in}}%
\pgfpathlineto{\pgfqpoint{4.986486in}{0.524170in}}%
\pgfpathclose%
\pgfusepath{fill}%
\end{pgfscope}%
\begin{pgfscope}%
\pgfpathrectangle{\pgfqpoint{0.651412in}{0.524170in}}{\pgfqpoint{4.629690in}{2.558193in}}%
\pgfusepath{clip}%
\pgfsetbuttcap%
\pgfsetmiterjoin%
\definecolor{currentfill}{rgb}{0.870588,0.560784,0.019608}%
\pgfsetfillcolor{currentfill}%
\pgfsetfillopacity{0.700000}%
\pgfsetlinewidth{0.000000pt}%
\definecolor{currentstroke}{rgb}{0.000000,0.000000,0.000000}%
\pgfsetstrokecolor{currentstroke}%
\pgfsetstrokeopacity{0.700000}%
\pgfsetdash{}{0pt}%
\pgfpathmoveto{\pgfqpoint{5.028574in}{0.524170in}}%
\pgfpathlineto{\pgfqpoint{5.070662in}{0.524170in}}%
\pgfpathlineto{\pgfqpoint{5.070662in}{0.605025in}}%
\pgfpathlineto{\pgfqpoint{5.028574in}{0.605025in}}%
\pgfpathlineto{\pgfqpoint{5.028574in}{0.524170in}}%
\pgfpathclose%
\pgfusepath{fill}%
\end{pgfscope}%
\begin{pgfscope}%
\pgfpathrectangle{\pgfqpoint{0.651412in}{0.524170in}}{\pgfqpoint{4.629690in}{2.558193in}}%
\pgfusepath{clip}%
\pgfsetbuttcap%
\pgfsetmiterjoin%
\definecolor{currentfill}{rgb}{0.835294,0.368627,0.000000}%
\pgfsetfillcolor{currentfill}%
\pgfsetfillopacity{0.700000}%
\pgfsetlinewidth{0.000000pt}%
\definecolor{currentstroke}{rgb}{0.000000,0.000000,0.000000}%
\pgfsetstrokecolor{currentstroke}%
\pgfsetstrokeopacity{0.700000}%
\pgfsetdash{}{0pt}%
\pgfpathmoveto{\pgfqpoint{0.861853in}{0.524170in}}%
\pgfpathlineto{\pgfqpoint{0.903941in}{0.524170in}}%
\pgfpathlineto{\pgfqpoint{0.903941in}{0.524170in}}%
\pgfpathlineto{\pgfqpoint{0.861853in}{0.524170in}}%
\pgfpathlineto{\pgfqpoint{0.861853in}{0.524170in}}%
\pgfpathclose%
\pgfusepath{fill}%
\end{pgfscope}%
\begin{pgfscope}%
\pgfpathrectangle{\pgfqpoint{0.651412in}{0.524170in}}{\pgfqpoint{4.629690in}{2.558193in}}%
\pgfusepath{clip}%
\pgfsetbuttcap%
\pgfsetmiterjoin%
\definecolor{currentfill}{rgb}{0.835294,0.368627,0.000000}%
\pgfsetfillcolor{currentfill}%
\pgfsetfillopacity{0.700000}%
\pgfsetlinewidth{0.000000pt}%
\definecolor{currentstroke}{rgb}{0.000000,0.000000,0.000000}%
\pgfsetstrokecolor{currentstroke}%
\pgfsetstrokeopacity{0.700000}%
\pgfsetdash{}{0pt}%
\pgfpathmoveto{\pgfqpoint{0.903941in}{0.524170in}}%
\pgfpathlineto{\pgfqpoint{0.946029in}{0.524170in}}%
\pgfpathlineto{\pgfqpoint{0.946029in}{0.524170in}}%
\pgfpathlineto{\pgfqpoint{0.903941in}{0.524170in}}%
\pgfpathlineto{\pgfqpoint{0.903941in}{0.524170in}}%
\pgfpathclose%
\pgfusepath{fill}%
\end{pgfscope}%
\begin{pgfscope}%
\pgfpathrectangle{\pgfqpoint{0.651412in}{0.524170in}}{\pgfqpoint{4.629690in}{2.558193in}}%
\pgfusepath{clip}%
\pgfsetbuttcap%
\pgfsetmiterjoin%
\definecolor{currentfill}{rgb}{0.835294,0.368627,0.000000}%
\pgfsetfillcolor{currentfill}%
\pgfsetfillopacity{0.700000}%
\pgfsetlinewidth{0.000000pt}%
\definecolor{currentstroke}{rgb}{0.000000,0.000000,0.000000}%
\pgfsetstrokecolor{currentstroke}%
\pgfsetstrokeopacity{0.700000}%
\pgfsetdash{}{0pt}%
\pgfpathmoveto{\pgfqpoint{0.946029in}{0.524170in}}%
\pgfpathlineto{\pgfqpoint{0.988117in}{0.524170in}}%
\pgfpathlineto{\pgfqpoint{0.988117in}{0.524170in}}%
\pgfpathlineto{\pgfqpoint{0.946029in}{0.524170in}}%
\pgfpathlineto{\pgfqpoint{0.946029in}{0.524170in}}%
\pgfpathclose%
\pgfusepath{fill}%
\end{pgfscope}%
\begin{pgfscope}%
\pgfpathrectangle{\pgfqpoint{0.651412in}{0.524170in}}{\pgfqpoint{4.629690in}{2.558193in}}%
\pgfusepath{clip}%
\pgfsetbuttcap%
\pgfsetmiterjoin%
\definecolor{currentfill}{rgb}{0.835294,0.368627,0.000000}%
\pgfsetfillcolor{currentfill}%
\pgfsetfillopacity{0.700000}%
\pgfsetlinewidth{0.000000pt}%
\definecolor{currentstroke}{rgb}{0.000000,0.000000,0.000000}%
\pgfsetstrokecolor{currentstroke}%
\pgfsetstrokeopacity{0.700000}%
\pgfsetdash{}{0pt}%
\pgfpathmoveto{\pgfqpoint{0.988117in}{0.524170in}}%
\pgfpathlineto{\pgfqpoint{1.030205in}{0.524170in}}%
\pgfpathlineto{\pgfqpoint{1.030205in}{0.593832in}}%
\pgfpathlineto{\pgfqpoint{0.988117in}{0.593832in}}%
\pgfpathlineto{\pgfqpoint{0.988117in}{0.524170in}}%
\pgfpathclose%
\pgfusepath{fill}%
\end{pgfscope}%
\begin{pgfscope}%
\pgfpathrectangle{\pgfqpoint{0.651412in}{0.524170in}}{\pgfqpoint{4.629690in}{2.558193in}}%
\pgfusepath{clip}%
\pgfsetbuttcap%
\pgfsetmiterjoin%
\definecolor{currentfill}{rgb}{0.835294,0.368627,0.000000}%
\pgfsetfillcolor{currentfill}%
\pgfsetfillopacity{0.700000}%
\pgfsetlinewidth{0.000000pt}%
\definecolor{currentstroke}{rgb}{0.000000,0.000000,0.000000}%
\pgfsetstrokecolor{currentstroke}%
\pgfsetstrokeopacity{0.700000}%
\pgfsetdash{}{0pt}%
\pgfpathmoveto{\pgfqpoint{1.030205in}{0.524170in}}%
\pgfpathlineto{\pgfqpoint{1.072293in}{0.524170in}}%
\pgfpathlineto{\pgfqpoint{1.072293in}{0.947500in}}%
\pgfpathlineto{\pgfqpoint{1.030205in}{0.947500in}}%
\pgfpathlineto{\pgfqpoint{1.030205in}{0.524170in}}%
\pgfpathclose%
\pgfusepath{fill}%
\end{pgfscope}%
\begin{pgfscope}%
\pgfpathrectangle{\pgfqpoint{0.651412in}{0.524170in}}{\pgfqpoint{4.629690in}{2.558193in}}%
\pgfusepath{clip}%
\pgfsetbuttcap%
\pgfsetmiterjoin%
\definecolor{currentfill}{rgb}{0.835294,0.368627,0.000000}%
\pgfsetfillcolor{currentfill}%
\pgfsetfillopacity{0.700000}%
\pgfsetlinewidth{0.000000pt}%
\definecolor{currentstroke}{rgb}{0.000000,0.000000,0.000000}%
\pgfsetstrokecolor{currentstroke}%
\pgfsetstrokeopacity{0.700000}%
\pgfsetdash{}{0pt}%
\pgfpathmoveto{\pgfqpoint{1.072293in}{0.524170in}}%
\pgfpathlineto{\pgfqpoint{1.114381in}{0.524170in}}%
\pgfpathlineto{\pgfqpoint{1.114381in}{1.327962in}}%
\pgfpathlineto{\pgfqpoint{1.072293in}{1.327962in}}%
\pgfpathlineto{\pgfqpoint{1.072293in}{0.524170in}}%
\pgfpathclose%
\pgfusepath{fill}%
\end{pgfscope}%
\begin{pgfscope}%
\pgfpathrectangle{\pgfqpoint{0.651412in}{0.524170in}}{\pgfqpoint{4.629690in}{2.558193in}}%
\pgfusepath{clip}%
\pgfsetbuttcap%
\pgfsetmiterjoin%
\definecolor{currentfill}{rgb}{0.835294,0.368627,0.000000}%
\pgfsetfillcolor{currentfill}%
\pgfsetfillopacity{0.700000}%
\pgfsetlinewidth{0.000000pt}%
\definecolor{currentstroke}{rgb}{0.000000,0.000000,0.000000}%
\pgfsetstrokecolor{currentstroke}%
\pgfsetstrokeopacity{0.700000}%
\pgfsetdash{}{0pt}%
\pgfpathmoveto{\pgfqpoint{1.114381in}{0.524170in}}%
\pgfpathlineto{\pgfqpoint{1.156469in}{0.524170in}}%
\pgfpathlineto{\pgfqpoint{1.156469in}{1.633403in}}%
\pgfpathlineto{\pgfqpoint{1.114381in}{1.633403in}}%
\pgfpathlineto{\pgfqpoint{1.114381in}{0.524170in}}%
\pgfpathclose%
\pgfusepath{fill}%
\end{pgfscope}%
\begin{pgfscope}%
\pgfpathrectangle{\pgfqpoint{0.651412in}{0.524170in}}{\pgfqpoint{4.629690in}{2.558193in}}%
\pgfusepath{clip}%
\pgfsetbuttcap%
\pgfsetmiterjoin%
\definecolor{currentfill}{rgb}{0.835294,0.368627,0.000000}%
\pgfsetfillcolor{currentfill}%
\pgfsetfillopacity{0.700000}%
\pgfsetlinewidth{0.000000pt}%
\definecolor{currentstroke}{rgb}{0.000000,0.000000,0.000000}%
\pgfsetstrokecolor{currentstroke}%
\pgfsetstrokeopacity{0.700000}%
\pgfsetdash{}{0pt}%
\pgfpathmoveto{\pgfqpoint{1.156469in}{0.524170in}}%
\pgfpathlineto{\pgfqpoint{1.198557in}{0.524170in}}%
\pgfpathlineto{\pgfqpoint{1.198557in}{1.783444in}}%
\pgfpathlineto{\pgfqpoint{1.156469in}{1.783444in}}%
\pgfpathlineto{\pgfqpoint{1.156469in}{0.524170in}}%
\pgfpathclose%
\pgfusepath{fill}%
\end{pgfscope}%
\begin{pgfscope}%
\pgfpathrectangle{\pgfqpoint{0.651412in}{0.524170in}}{\pgfqpoint{4.629690in}{2.558193in}}%
\pgfusepath{clip}%
\pgfsetbuttcap%
\pgfsetmiterjoin%
\definecolor{currentfill}{rgb}{0.835294,0.368627,0.000000}%
\pgfsetfillcolor{currentfill}%
\pgfsetfillopacity{0.700000}%
\pgfsetlinewidth{0.000000pt}%
\definecolor{currentstroke}{rgb}{0.000000,0.000000,0.000000}%
\pgfsetstrokecolor{currentstroke}%
\pgfsetstrokeopacity{0.700000}%
\pgfsetdash{}{0pt}%
\pgfpathmoveto{\pgfqpoint{1.198557in}{0.524170in}}%
\pgfpathlineto{\pgfqpoint{1.240645in}{0.524170in}}%
\pgfpathlineto{\pgfqpoint{1.240645in}{1.676272in}}%
\pgfpathlineto{\pgfqpoint{1.198557in}{1.676272in}}%
\pgfpathlineto{\pgfqpoint{1.198557in}{0.524170in}}%
\pgfpathclose%
\pgfusepath{fill}%
\end{pgfscope}%
\begin{pgfscope}%
\pgfpathrectangle{\pgfqpoint{0.651412in}{0.524170in}}{\pgfqpoint{4.629690in}{2.558193in}}%
\pgfusepath{clip}%
\pgfsetbuttcap%
\pgfsetmiterjoin%
\definecolor{currentfill}{rgb}{0.835294,0.368627,0.000000}%
\pgfsetfillcolor{currentfill}%
\pgfsetfillopacity{0.700000}%
\pgfsetlinewidth{0.000000pt}%
\definecolor{currentstroke}{rgb}{0.000000,0.000000,0.000000}%
\pgfsetstrokecolor{currentstroke}%
\pgfsetstrokeopacity{0.700000}%
\pgfsetdash{}{0pt}%
\pgfpathmoveto{\pgfqpoint{1.240645in}{0.524170in}}%
\pgfpathlineto{\pgfqpoint{1.282734in}{0.524170in}}%
\pgfpathlineto{\pgfqpoint{1.282734in}{1.810237in}}%
\pgfpathlineto{\pgfqpoint{1.240645in}{1.810237in}}%
\pgfpathlineto{\pgfqpoint{1.240645in}{0.524170in}}%
\pgfpathclose%
\pgfusepath{fill}%
\end{pgfscope}%
\begin{pgfscope}%
\pgfpathrectangle{\pgfqpoint{0.651412in}{0.524170in}}{\pgfqpoint{4.629690in}{2.558193in}}%
\pgfusepath{clip}%
\pgfsetbuttcap%
\pgfsetmiterjoin%
\definecolor{currentfill}{rgb}{0.835294,0.368627,0.000000}%
\pgfsetfillcolor{currentfill}%
\pgfsetfillopacity{0.700000}%
\pgfsetlinewidth{0.000000pt}%
\definecolor{currentstroke}{rgb}{0.000000,0.000000,0.000000}%
\pgfsetstrokecolor{currentstroke}%
\pgfsetstrokeopacity{0.700000}%
\pgfsetdash{}{0pt}%
\pgfpathmoveto{\pgfqpoint{1.282734in}{0.524170in}}%
\pgfpathlineto{\pgfqpoint{1.324822in}{0.524170in}}%
\pgfpathlineto{\pgfqpoint{1.324822in}{1.585176in}}%
\pgfpathlineto{\pgfqpoint{1.282734in}{1.585176in}}%
\pgfpathlineto{\pgfqpoint{1.282734in}{0.524170in}}%
\pgfpathclose%
\pgfusepath{fill}%
\end{pgfscope}%
\begin{pgfscope}%
\pgfpathrectangle{\pgfqpoint{0.651412in}{0.524170in}}{\pgfqpoint{4.629690in}{2.558193in}}%
\pgfusepath{clip}%
\pgfsetbuttcap%
\pgfsetmiterjoin%
\definecolor{currentfill}{rgb}{0.835294,0.368627,0.000000}%
\pgfsetfillcolor{currentfill}%
\pgfsetfillopacity{0.700000}%
\pgfsetlinewidth{0.000000pt}%
\definecolor{currentstroke}{rgb}{0.000000,0.000000,0.000000}%
\pgfsetstrokecolor{currentstroke}%
\pgfsetstrokeopacity{0.700000}%
\pgfsetdash{}{0pt}%
\pgfpathmoveto{\pgfqpoint{1.324822in}{0.524170in}}%
\pgfpathlineto{\pgfqpoint{1.366910in}{0.524170in}}%
\pgfpathlineto{\pgfqpoint{1.366910in}{1.660196in}}%
\pgfpathlineto{\pgfqpoint{1.324822in}{1.660196in}}%
\pgfpathlineto{\pgfqpoint{1.324822in}{0.524170in}}%
\pgfpathclose%
\pgfusepath{fill}%
\end{pgfscope}%
\begin{pgfscope}%
\pgfpathrectangle{\pgfqpoint{0.651412in}{0.524170in}}{\pgfqpoint{4.629690in}{2.558193in}}%
\pgfusepath{clip}%
\pgfsetbuttcap%
\pgfsetmiterjoin%
\definecolor{currentfill}{rgb}{0.835294,0.368627,0.000000}%
\pgfsetfillcolor{currentfill}%
\pgfsetfillopacity{0.700000}%
\pgfsetlinewidth{0.000000pt}%
\definecolor{currentstroke}{rgb}{0.000000,0.000000,0.000000}%
\pgfsetstrokecolor{currentstroke}%
\pgfsetstrokeopacity{0.700000}%
\pgfsetdash{}{0pt}%
\pgfpathmoveto{\pgfqpoint{1.366910in}{0.524170in}}%
\pgfpathlineto{\pgfqpoint{1.408998in}{0.524170in}}%
\pgfpathlineto{\pgfqpoint{1.408998in}{1.461927in}}%
\pgfpathlineto{\pgfqpoint{1.366910in}{1.461927in}}%
\pgfpathlineto{\pgfqpoint{1.366910in}{0.524170in}}%
\pgfpathclose%
\pgfusepath{fill}%
\end{pgfscope}%
\begin{pgfscope}%
\pgfpathrectangle{\pgfqpoint{0.651412in}{0.524170in}}{\pgfqpoint{4.629690in}{2.558193in}}%
\pgfusepath{clip}%
\pgfsetbuttcap%
\pgfsetmiterjoin%
\definecolor{currentfill}{rgb}{0.835294,0.368627,0.000000}%
\pgfsetfillcolor{currentfill}%
\pgfsetfillopacity{0.700000}%
\pgfsetlinewidth{0.000000pt}%
\definecolor{currentstroke}{rgb}{0.000000,0.000000,0.000000}%
\pgfsetstrokecolor{currentstroke}%
\pgfsetstrokeopacity{0.700000}%
\pgfsetdash{}{0pt}%
\pgfpathmoveto{\pgfqpoint{1.408998in}{0.524170in}}%
\pgfpathlineto{\pgfqpoint{1.451086in}{0.524170in}}%
\pgfpathlineto{\pgfqpoint{1.451086in}{1.472645in}}%
\pgfpathlineto{\pgfqpoint{1.408998in}{1.472645in}}%
\pgfpathlineto{\pgfqpoint{1.408998in}{0.524170in}}%
\pgfpathclose%
\pgfusepath{fill}%
\end{pgfscope}%
\begin{pgfscope}%
\pgfpathrectangle{\pgfqpoint{0.651412in}{0.524170in}}{\pgfqpoint{4.629690in}{2.558193in}}%
\pgfusepath{clip}%
\pgfsetbuttcap%
\pgfsetmiterjoin%
\definecolor{currentfill}{rgb}{0.835294,0.368627,0.000000}%
\pgfsetfillcolor{currentfill}%
\pgfsetfillopacity{0.700000}%
\pgfsetlinewidth{0.000000pt}%
\definecolor{currentstroke}{rgb}{0.000000,0.000000,0.000000}%
\pgfsetstrokecolor{currentstroke}%
\pgfsetstrokeopacity{0.700000}%
\pgfsetdash{}{0pt}%
\pgfpathmoveto{\pgfqpoint{1.451086in}{0.524170in}}%
\pgfpathlineto{\pgfqpoint{1.493174in}{0.524170in}}%
\pgfpathlineto{\pgfqpoint{1.493174in}{1.236866in}}%
\pgfpathlineto{\pgfqpoint{1.451086in}{1.236866in}}%
\pgfpathlineto{\pgfqpoint{1.451086in}{0.524170in}}%
\pgfpathclose%
\pgfusepath{fill}%
\end{pgfscope}%
\begin{pgfscope}%
\pgfpathrectangle{\pgfqpoint{0.651412in}{0.524170in}}{\pgfqpoint{4.629690in}{2.558193in}}%
\pgfusepath{clip}%
\pgfsetbuttcap%
\pgfsetmiterjoin%
\definecolor{currentfill}{rgb}{0.835294,0.368627,0.000000}%
\pgfsetfillcolor{currentfill}%
\pgfsetfillopacity{0.700000}%
\pgfsetlinewidth{0.000000pt}%
\definecolor{currentstroke}{rgb}{0.000000,0.000000,0.000000}%
\pgfsetstrokecolor{currentstroke}%
\pgfsetstrokeopacity{0.700000}%
\pgfsetdash{}{0pt}%
\pgfpathmoveto{\pgfqpoint{1.493174in}{0.524170in}}%
\pgfpathlineto{\pgfqpoint{1.535262in}{0.524170in}}%
\pgfpathlineto{\pgfqpoint{1.535262in}{1.413700in}}%
\pgfpathlineto{\pgfqpoint{1.493174in}{1.413700in}}%
\pgfpathlineto{\pgfqpoint{1.493174in}{0.524170in}}%
\pgfpathclose%
\pgfusepath{fill}%
\end{pgfscope}%
\begin{pgfscope}%
\pgfpathrectangle{\pgfqpoint{0.651412in}{0.524170in}}{\pgfqpoint{4.629690in}{2.558193in}}%
\pgfusepath{clip}%
\pgfsetbuttcap%
\pgfsetmiterjoin%
\definecolor{currentfill}{rgb}{0.835294,0.368627,0.000000}%
\pgfsetfillcolor{currentfill}%
\pgfsetfillopacity{0.700000}%
\pgfsetlinewidth{0.000000pt}%
\definecolor{currentstroke}{rgb}{0.000000,0.000000,0.000000}%
\pgfsetstrokecolor{currentstroke}%
\pgfsetstrokeopacity{0.700000}%
\pgfsetdash{}{0pt}%
\pgfpathmoveto{\pgfqpoint{1.535262in}{0.524170in}}%
\pgfpathlineto{\pgfqpoint{1.577350in}{0.524170in}}%
\pgfpathlineto{\pgfqpoint{1.577350in}{1.188638in}}%
\pgfpathlineto{\pgfqpoint{1.535262in}{1.188638in}}%
\pgfpathlineto{\pgfqpoint{1.535262in}{0.524170in}}%
\pgfpathclose%
\pgfusepath{fill}%
\end{pgfscope}%
\begin{pgfscope}%
\pgfpathrectangle{\pgfqpoint{0.651412in}{0.524170in}}{\pgfqpoint{4.629690in}{2.558193in}}%
\pgfusepath{clip}%
\pgfsetbuttcap%
\pgfsetmiterjoin%
\definecolor{currentfill}{rgb}{0.835294,0.368627,0.000000}%
\pgfsetfillcolor{currentfill}%
\pgfsetfillopacity{0.700000}%
\pgfsetlinewidth{0.000000pt}%
\definecolor{currentstroke}{rgb}{0.000000,0.000000,0.000000}%
\pgfsetstrokecolor{currentstroke}%
\pgfsetstrokeopacity{0.700000}%
\pgfsetdash{}{0pt}%
\pgfpathmoveto{\pgfqpoint{1.577350in}{0.524170in}}%
\pgfpathlineto{\pgfqpoint{1.619438in}{0.524170in}}%
\pgfpathlineto{\pgfqpoint{1.619438in}{1.231507in}}%
\pgfpathlineto{\pgfqpoint{1.577350in}{1.231507in}}%
\pgfpathlineto{\pgfqpoint{1.577350in}{0.524170in}}%
\pgfpathclose%
\pgfusepath{fill}%
\end{pgfscope}%
\begin{pgfscope}%
\pgfpathrectangle{\pgfqpoint{0.651412in}{0.524170in}}{\pgfqpoint{4.629690in}{2.558193in}}%
\pgfusepath{clip}%
\pgfsetbuttcap%
\pgfsetmiterjoin%
\definecolor{currentfill}{rgb}{0.835294,0.368627,0.000000}%
\pgfsetfillcolor{currentfill}%
\pgfsetfillopacity{0.700000}%
\pgfsetlinewidth{0.000000pt}%
\definecolor{currentstroke}{rgb}{0.000000,0.000000,0.000000}%
\pgfsetstrokecolor{currentstroke}%
\pgfsetstrokeopacity{0.700000}%
\pgfsetdash{}{0pt}%
\pgfpathmoveto{\pgfqpoint{1.619438in}{0.524170in}}%
\pgfpathlineto{\pgfqpoint{1.661526in}{0.524170in}}%
\pgfpathlineto{\pgfqpoint{1.661526in}{1.135052in}}%
\pgfpathlineto{\pgfqpoint{1.619438in}{1.135052in}}%
\pgfpathlineto{\pgfqpoint{1.619438in}{0.524170in}}%
\pgfpathclose%
\pgfusepath{fill}%
\end{pgfscope}%
\begin{pgfscope}%
\pgfpathrectangle{\pgfqpoint{0.651412in}{0.524170in}}{\pgfqpoint{4.629690in}{2.558193in}}%
\pgfusepath{clip}%
\pgfsetbuttcap%
\pgfsetmiterjoin%
\definecolor{currentfill}{rgb}{0.835294,0.368627,0.000000}%
\pgfsetfillcolor{currentfill}%
\pgfsetfillopacity{0.700000}%
\pgfsetlinewidth{0.000000pt}%
\definecolor{currentstroke}{rgb}{0.000000,0.000000,0.000000}%
\pgfsetstrokecolor{currentstroke}%
\pgfsetstrokeopacity{0.700000}%
\pgfsetdash{}{0pt}%
\pgfpathmoveto{\pgfqpoint{1.661526in}{0.524170in}}%
\pgfpathlineto{\pgfqpoint{1.703614in}{0.524170in}}%
\pgfpathlineto{\pgfqpoint{1.703614in}{1.118976in}}%
\pgfpathlineto{\pgfqpoint{1.661526in}{1.118976in}}%
\pgfpathlineto{\pgfqpoint{1.661526in}{0.524170in}}%
\pgfpathclose%
\pgfusepath{fill}%
\end{pgfscope}%
\begin{pgfscope}%
\pgfpathrectangle{\pgfqpoint{0.651412in}{0.524170in}}{\pgfqpoint{4.629690in}{2.558193in}}%
\pgfusepath{clip}%
\pgfsetbuttcap%
\pgfsetmiterjoin%
\definecolor{currentfill}{rgb}{0.835294,0.368627,0.000000}%
\pgfsetfillcolor{currentfill}%
\pgfsetfillopacity{0.700000}%
\pgfsetlinewidth{0.000000pt}%
\definecolor{currentstroke}{rgb}{0.000000,0.000000,0.000000}%
\pgfsetstrokecolor{currentstroke}%
\pgfsetstrokeopacity{0.700000}%
\pgfsetdash{}{0pt}%
\pgfpathmoveto{\pgfqpoint{1.703614in}{0.524170in}}%
\pgfpathlineto{\pgfqpoint{1.745703in}{0.524170in}}%
\pgfpathlineto{\pgfqpoint{1.745703in}{1.049314in}}%
\pgfpathlineto{\pgfqpoint{1.703614in}{1.049314in}}%
\pgfpathlineto{\pgfqpoint{1.703614in}{0.524170in}}%
\pgfpathclose%
\pgfusepath{fill}%
\end{pgfscope}%
\begin{pgfscope}%
\pgfpathrectangle{\pgfqpoint{0.651412in}{0.524170in}}{\pgfqpoint{4.629690in}{2.558193in}}%
\pgfusepath{clip}%
\pgfsetbuttcap%
\pgfsetmiterjoin%
\definecolor{currentfill}{rgb}{0.835294,0.368627,0.000000}%
\pgfsetfillcolor{currentfill}%
\pgfsetfillopacity{0.700000}%
\pgfsetlinewidth{0.000000pt}%
\definecolor{currentstroke}{rgb}{0.000000,0.000000,0.000000}%
\pgfsetstrokecolor{currentstroke}%
\pgfsetstrokeopacity{0.700000}%
\pgfsetdash{}{0pt}%
\pgfpathmoveto{\pgfqpoint{1.745703in}{0.524170in}}%
\pgfpathlineto{\pgfqpoint{1.787791in}{0.524170in}}%
\pgfpathlineto{\pgfqpoint{1.787791in}{0.995728in}}%
\pgfpathlineto{\pgfqpoint{1.745703in}{0.995728in}}%
\pgfpathlineto{\pgfqpoint{1.745703in}{0.524170in}}%
\pgfpathclose%
\pgfusepath{fill}%
\end{pgfscope}%
\begin{pgfscope}%
\pgfpathrectangle{\pgfqpoint{0.651412in}{0.524170in}}{\pgfqpoint{4.629690in}{2.558193in}}%
\pgfusepath{clip}%
\pgfsetbuttcap%
\pgfsetmiterjoin%
\definecolor{currentfill}{rgb}{0.835294,0.368627,0.000000}%
\pgfsetfillcolor{currentfill}%
\pgfsetfillopacity{0.700000}%
\pgfsetlinewidth{0.000000pt}%
\definecolor{currentstroke}{rgb}{0.000000,0.000000,0.000000}%
\pgfsetstrokecolor{currentstroke}%
\pgfsetstrokeopacity{0.700000}%
\pgfsetdash{}{0pt}%
\pgfpathmoveto{\pgfqpoint{1.787791in}{0.524170in}}%
\pgfpathlineto{\pgfqpoint{1.829879in}{0.524170in}}%
\pgfpathlineto{\pgfqpoint{1.829879in}{0.931425in}}%
\pgfpathlineto{\pgfqpoint{1.787791in}{0.931425in}}%
\pgfpathlineto{\pgfqpoint{1.787791in}{0.524170in}}%
\pgfpathclose%
\pgfusepath{fill}%
\end{pgfscope}%
\begin{pgfscope}%
\pgfpathrectangle{\pgfqpoint{0.651412in}{0.524170in}}{\pgfqpoint{4.629690in}{2.558193in}}%
\pgfusepath{clip}%
\pgfsetbuttcap%
\pgfsetmiterjoin%
\definecolor{currentfill}{rgb}{0.835294,0.368627,0.000000}%
\pgfsetfillcolor{currentfill}%
\pgfsetfillopacity{0.700000}%
\pgfsetlinewidth{0.000000pt}%
\definecolor{currentstroke}{rgb}{0.000000,0.000000,0.000000}%
\pgfsetstrokecolor{currentstroke}%
\pgfsetstrokeopacity{0.700000}%
\pgfsetdash{}{0pt}%
\pgfpathmoveto{\pgfqpoint{1.829879in}{0.524170in}}%
\pgfpathlineto{\pgfqpoint{1.871967in}{0.524170in}}%
\pgfpathlineto{\pgfqpoint{1.871967in}{0.915349in}}%
\pgfpathlineto{\pgfqpoint{1.829879in}{0.915349in}}%
\pgfpathlineto{\pgfqpoint{1.829879in}{0.524170in}}%
\pgfpathclose%
\pgfusepath{fill}%
\end{pgfscope}%
\begin{pgfscope}%
\pgfpathrectangle{\pgfqpoint{0.651412in}{0.524170in}}{\pgfqpoint{4.629690in}{2.558193in}}%
\pgfusepath{clip}%
\pgfsetbuttcap%
\pgfsetmiterjoin%
\definecolor{currentfill}{rgb}{0.835294,0.368627,0.000000}%
\pgfsetfillcolor{currentfill}%
\pgfsetfillopacity{0.700000}%
\pgfsetlinewidth{0.000000pt}%
\definecolor{currentstroke}{rgb}{0.000000,0.000000,0.000000}%
\pgfsetstrokecolor{currentstroke}%
\pgfsetstrokeopacity{0.700000}%
\pgfsetdash{}{0pt}%
\pgfpathmoveto{\pgfqpoint{1.871967in}{0.524170in}}%
\pgfpathlineto{\pgfqpoint{1.914055in}{0.524170in}}%
\pgfpathlineto{\pgfqpoint{1.914055in}{0.947500in}}%
\pgfpathlineto{\pgfqpoint{1.871967in}{0.947500in}}%
\pgfpathlineto{\pgfqpoint{1.871967in}{0.524170in}}%
\pgfpathclose%
\pgfusepath{fill}%
\end{pgfscope}%
\begin{pgfscope}%
\pgfpathrectangle{\pgfqpoint{0.651412in}{0.524170in}}{\pgfqpoint{4.629690in}{2.558193in}}%
\pgfusepath{clip}%
\pgfsetbuttcap%
\pgfsetmiterjoin%
\definecolor{currentfill}{rgb}{0.835294,0.368627,0.000000}%
\pgfsetfillcolor{currentfill}%
\pgfsetfillopacity{0.700000}%
\pgfsetlinewidth{0.000000pt}%
\definecolor{currentstroke}{rgb}{0.000000,0.000000,0.000000}%
\pgfsetstrokecolor{currentstroke}%
\pgfsetstrokeopacity{0.700000}%
\pgfsetdash{}{0pt}%
\pgfpathmoveto{\pgfqpoint{1.914055in}{0.524170in}}%
\pgfpathlineto{\pgfqpoint{1.956143in}{0.524170in}}%
\pgfpathlineto{\pgfqpoint{1.956143in}{0.834969in}}%
\pgfpathlineto{\pgfqpoint{1.914055in}{0.834969in}}%
\pgfpathlineto{\pgfqpoint{1.914055in}{0.524170in}}%
\pgfpathclose%
\pgfusepath{fill}%
\end{pgfscope}%
\begin{pgfscope}%
\pgfpathrectangle{\pgfqpoint{0.651412in}{0.524170in}}{\pgfqpoint{4.629690in}{2.558193in}}%
\pgfusepath{clip}%
\pgfsetbuttcap%
\pgfsetmiterjoin%
\definecolor{currentfill}{rgb}{0.835294,0.368627,0.000000}%
\pgfsetfillcolor{currentfill}%
\pgfsetfillopacity{0.700000}%
\pgfsetlinewidth{0.000000pt}%
\definecolor{currentstroke}{rgb}{0.000000,0.000000,0.000000}%
\pgfsetstrokecolor{currentstroke}%
\pgfsetstrokeopacity{0.700000}%
\pgfsetdash{}{0pt}%
\pgfpathmoveto{\pgfqpoint{1.956143in}{0.524170in}}%
\pgfpathlineto{\pgfqpoint{1.998231in}{0.524170in}}%
\pgfpathlineto{\pgfqpoint{1.998231in}{0.899273in}}%
\pgfpathlineto{\pgfqpoint{1.956143in}{0.899273in}}%
\pgfpathlineto{\pgfqpoint{1.956143in}{0.524170in}}%
\pgfpathclose%
\pgfusepath{fill}%
\end{pgfscope}%
\begin{pgfscope}%
\pgfpathrectangle{\pgfqpoint{0.651412in}{0.524170in}}{\pgfqpoint{4.629690in}{2.558193in}}%
\pgfusepath{clip}%
\pgfsetbuttcap%
\pgfsetmiterjoin%
\definecolor{currentfill}{rgb}{0.835294,0.368627,0.000000}%
\pgfsetfillcolor{currentfill}%
\pgfsetfillopacity{0.700000}%
\pgfsetlinewidth{0.000000pt}%
\definecolor{currentstroke}{rgb}{0.000000,0.000000,0.000000}%
\pgfsetstrokecolor{currentstroke}%
\pgfsetstrokeopacity{0.700000}%
\pgfsetdash{}{0pt}%
\pgfpathmoveto{\pgfqpoint{1.998231in}{0.524170in}}%
\pgfpathlineto{\pgfqpoint{2.040319in}{0.524170in}}%
\pgfpathlineto{\pgfqpoint{2.040319in}{0.834969in}}%
\pgfpathlineto{\pgfqpoint{1.998231in}{0.834969in}}%
\pgfpathlineto{\pgfqpoint{1.998231in}{0.524170in}}%
\pgfpathclose%
\pgfusepath{fill}%
\end{pgfscope}%
\begin{pgfscope}%
\pgfpathrectangle{\pgfqpoint{0.651412in}{0.524170in}}{\pgfqpoint{4.629690in}{2.558193in}}%
\pgfusepath{clip}%
\pgfsetbuttcap%
\pgfsetmiterjoin%
\definecolor{currentfill}{rgb}{0.835294,0.368627,0.000000}%
\pgfsetfillcolor{currentfill}%
\pgfsetfillopacity{0.700000}%
\pgfsetlinewidth{0.000000pt}%
\definecolor{currentstroke}{rgb}{0.000000,0.000000,0.000000}%
\pgfsetstrokecolor{currentstroke}%
\pgfsetstrokeopacity{0.700000}%
\pgfsetdash{}{0pt}%
\pgfpathmoveto{\pgfqpoint{2.040319in}{0.524170in}}%
\pgfpathlineto{\pgfqpoint{2.082407in}{0.524170in}}%
\pgfpathlineto{\pgfqpoint{2.082407in}{0.770666in}}%
\pgfpathlineto{\pgfqpoint{2.040319in}{0.770666in}}%
\pgfpathlineto{\pgfqpoint{2.040319in}{0.524170in}}%
\pgfpathclose%
\pgfusepath{fill}%
\end{pgfscope}%
\begin{pgfscope}%
\pgfpathrectangle{\pgfqpoint{0.651412in}{0.524170in}}{\pgfqpoint{4.629690in}{2.558193in}}%
\pgfusepath{clip}%
\pgfsetbuttcap%
\pgfsetmiterjoin%
\definecolor{currentfill}{rgb}{0.835294,0.368627,0.000000}%
\pgfsetfillcolor{currentfill}%
\pgfsetfillopacity{0.700000}%
\pgfsetlinewidth{0.000000pt}%
\definecolor{currentstroke}{rgb}{0.000000,0.000000,0.000000}%
\pgfsetstrokecolor{currentstroke}%
\pgfsetstrokeopacity{0.700000}%
\pgfsetdash{}{0pt}%
\pgfpathmoveto{\pgfqpoint{2.082407in}{0.524170in}}%
\pgfpathlineto{\pgfqpoint{2.124495in}{0.524170in}}%
\pgfpathlineto{\pgfqpoint{2.124495in}{0.813535in}}%
\pgfpathlineto{\pgfqpoint{2.082407in}{0.813535in}}%
\pgfpathlineto{\pgfqpoint{2.082407in}{0.524170in}}%
\pgfpathclose%
\pgfusepath{fill}%
\end{pgfscope}%
\begin{pgfscope}%
\pgfpathrectangle{\pgfqpoint{0.651412in}{0.524170in}}{\pgfqpoint{4.629690in}{2.558193in}}%
\pgfusepath{clip}%
\pgfsetbuttcap%
\pgfsetmiterjoin%
\definecolor{currentfill}{rgb}{0.835294,0.368627,0.000000}%
\pgfsetfillcolor{currentfill}%
\pgfsetfillopacity{0.700000}%
\pgfsetlinewidth{0.000000pt}%
\definecolor{currentstroke}{rgb}{0.000000,0.000000,0.000000}%
\pgfsetstrokecolor{currentstroke}%
\pgfsetstrokeopacity{0.700000}%
\pgfsetdash{}{0pt}%
\pgfpathmoveto{\pgfqpoint{2.124495in}{0.524170in}}%
\pgfpathlineto{\pgfqpoint{2.166583in}{0.524170in}}%
\pgfpathlineto{\pgfqpoint{2.166583in}{0.776025in}}%
\pgfpathlineto{\pgfqpoint{2.124495in}{0.776025in}}%
\pgfpathlineto{\pgfqpoint{2.124495in}{0.524170in}}%
\pgfpathclose%
\pgfusepath{fill}%
\end{pgfscope}%
\begin{pgfscope}%
\pgfpathrectangle{\pgfqpoint{0.651412in}{0.524170in}}{\pgfqpoint{4.629690in}{2.558193in}}%
\pgfusepath{clip}%
\pgfsetbuttcap%
\pgfsetmiterjoin%
\definecolor{currentfill}{rgb}{0.835294,0.368627,0.000000}%
\pgfsetfillcolor{currentfill}%
\pgfsetfillopacity{0.700000}%
\pgfsetlinewidth{0.000000pt}%
\definecolor{currentstroke}{rgb}{0.000000,0.000000,0.000000}%
\pgfsetstrokecolor{currentstroke}%
\pgfsetstrokeopacity{0.700000}%
\pgfsetdash{}{0pt}%
\pgfpathmoveto{\pgfqpoint{2.166583in}{0.524170in}}%
\pgfpathlineto{\pgfqpoint{2.208672in}{0.524170in}}%
\pgfpathlineto{\pgfqpoint{2.208672in}{0.754590in}}%
\pgfpathlineto{\pgfqpoint{2.166583in}{0.754590in}}%
\pgfpathlineto{\pgfqpoint{2.166583in}{0.524170in}}%
\pgfpathclose%
\pgfusepath{fill}%
\end{pgfscope}%
\begin{pgfscope}%
\pgfpathrectangle{\pgfqpoint{0.651412in}{0.524170in}}{\pgfqpoint{4.629690in}{2.558193in}}%
\pgfusepath{clip}%
\pgfsetbuttcap%
\pgfsetmiterjoin%
\definecolor{currentfill}{rgb}{0.835294,0.368627,0.000000}%
\pgfsetfillcolor{currentfill}%
\pgfsetfillopacity{0.700000}%
\pgfsetlinewidth{0.000000pt}%
\definecolor{currentstroke}{rgb}{0.000000,0.000000,0.000000}%
\pgfsetstrokecolor{currentstroke}%
\pgfsetstrokeopacity{0.700000}%
\pgfsetdash{}{0pt}%
\pgfpathmoveto{\pgfqpoint{2.208672in}{0.524170in}}%
\pgfpathlineto{\pgfqpoint{2.250760in}{0.524170in}}%
\pgfpathlineto{\pgfqpoint{2.250760in}{0.770666in}}%
\pgfpathlineto{\pgfqpoint{2.208672in}{0.770666in}}%
\pgfpathlineto{\pgfqpoint{2.208672in}{0.524170in}}%
\pgfpathclose%
\pgfusepath{fill}%
\end{pgfscope}%
\begin{pgfscope}%
\pgfpathrectangle{\pgfqpoint{0.651412in}{0.524170in}}{\pgfqpoint{4.629690in}{2.558193in}}%
\pgfusepath{clip}%
\pgfsetbuttcap%
\pgfsetmiterjoin%
\definecolor{currentfill}{rgb}{0.835294,0.368627,0.000000}%
\pgfsetfillcolor{currentfill}%
\pgfsetfillopacity{0.700000}%
\pgfsetlinewidth{0.000000pt}%
\definecolor{currentstroke}{rgb}{0.000000,0.000000,0.000000}%
\pgfsetstrokecolor{currentstroke}%
\pgfsetstrokeopacity{0.700000}%
\pgfsetdash{}{0pt}%
\pgfpathmoveto{\pgfqpoint{2.250760in}{0.524170in}}%
\pgfpathlineto{\pgfqpoint{2.292848in}{0.524170in}}%
\pgfpathlineto{\pgfqpoint{2.292848in}{0.754590in}}%
\pgfpathlineto{\pgfqpoint{2.250760in}{0.754590in}}%
\pgfpathlineto{\pgfqpoint{2.250760in}{0.524170in}}%
\pgfpathclose%
\pgfusepath{fill}%
\end{pgfscope}%
\begin{pgfscope}%
\pgfpathrectangle{\pgfqpoint{0.651412in}{0.524170in}}{\pgfqpoint{4.629690in}{2.558193in}}%
\pgfusepath{clip}%
\pgfsetbuttcap%
\pgfsetmiterjoin%
\definecolor{currentfill}{rgb}{0.835294,0.368627,0.000000}%
\pgfsetfillcolor{currentfill}%
\pgfsetfillopacity{0.700000}%
\pgfsetlinewidth{0.000000pt}%
\definecolor{currentstroke}{rgb}{0.000000,0.000000,0.000000}%
\pgfsetstrokecolor{currentstroke}%
\pgfsetstrokeopacity{0.700000}%
\pgfsetdash{}{0pt}%
\pgfpathmoveto{\pgfqpoint{2.292848in}{0.524170in}}%
\pgfpathlineto{\pgfqpoint{2.334936in}{0.524170in}}%
\pgfpathlineto{\pgfqpoint{2.334936in}{0.754590in}}%
\pgfpathlineto{\pgfqpoint{2.292848in}{0.754590in}}%
\pgfpathlineto{\pgfqpoint{2.292848in}{0.524170in}}%
\pgfpathclose%
\pgfusepath{fill}%
\end{pgfscope}%
\begin{pgfscope}%
\pgfpathrectangle{\pgfqpoint{0.651412in}{0.524170in}}{\pgfqpoint{4.629690in}{2.558193in}}%
\pgfusepath{clip}%
\pgfsetbuttcap%
\pgfsetmiterjoin%
\definecolor{currentfill}{rgb}{0.835294,0.368627,0.000000}%
\pgfsetfillcolor{currentfill}%
\pgfsetfillopacity{0.700000}%
\pgfsetlinewidth{0.000000pt}%
\definecolor{currentstroke}{rgb}{0.000000,0.000000,0.000000}%
\pgfsetstrokecolor{currentstroke}%
\pgfsetstrokeopacity{0.700000}%
\pgfsetdash{}{0pt}%
\pgfpathmoveto{\pgfqpoint{2.334936in}{0.524170in}}%
\pgfpathlineto{\pgfqpoint{2.377024in}{0.524170in}}%
\pgfpathlineto{\pgfqpoint{2.377024in}{0.695645in}}%
\pgfpathlineto{\pgfqpoint{2.334936in}{0.695645in}}%
\pgfpathlineto{\pgfqpoint{2.334936in}{0.524170in}}%
\pgfpathclose%
\pgfusepath{fill}%
\end{pgfscope}%
\begin{pgfscope}%
\pgfpathrectangle{\pgfqpoint{0.651412in}{0.524170in}}{\pgfqpoint{4.629690in}{2.558193in}}%
\pgfusepath{clip}%
\pgfsetbuttcap%
\pgfsetmiterjoin%
\definecolor{currentfill}{rgb}{0.835294,0.368627,0.000000}%
\pgfsetfillcolor{currentfill}%
\pgfsetfillopacity{0.700000}%
\pgfsetlinewidth{0.000000pt}%
\definecolor{currentstroke}{rgb}{0.000000,0.000000,0.000000}%
\pgfsetstrokecolor{currentstroke}%
\pgfsetstrokeopacity{0.700000}%
\pgfsetdash{}{0pt}%
\pgfpathmoveto{\pgfqpoint{2.377024in}{0.524170in}}%
\pgfpathlineto{\pgfqpoint{2.419112in}{0.524170in}}%
\pgfpathlineto{\pgfqpoint{2.419112in}{0.722439in}}%
\pgfpathlineto{\pgfqpoint{2.377024in}{0.722439in}}%
\pgfpathlineto{\pgfqpoint{2.377024in}{0.524170in}}%
\pgfpathclose%
\pgfusepath{fill}%
\end{pgfscope}%
\begin{pgfscope}%
\pgfpathrectangle{\pgfqpoint{0.651412in}{0.524170in}}{\pgfqpoint{4.629690in}{2.558193in}}%
\pgfusepath{clip}%
\pgfsetbuttcap%
\pgfsetmiterjoin%
\definecolor{currentfill}{rgb}{0.835294,0.368627,0.000000}%
\pgfsetfillcolor{currentfill}%
\pgfsetfillopacity{0.700000}%
\pgfsetlinewidth{0.000000pt}%
\definecolor{currentstroke}{rgb}{0.000000,0.000000,0.000000}%
\pgfsetstrokecolor{currentstroke}%
\pgfsetstrokeopacity{0.700000}%
\pgfsetdash{}{0pt}%
\pgfpathmoveto{\pgfqpoint{2.419112in}{0.524170in}}%
\pgfpathlineto{\pgfqpoint{2.461200in}{0.524170in}}%
\pgfpathlineto{\pgfqpoint{2.461200in}{0.701004in}}%
\pgfpathlineto{\pgfqpoint{2.419112in}{0.701004in}}%
\pgfpathlineto{\pgfqpoint{2.419112in}{0.524170in}}%
\pgfpathclose%
\pgfusepath{fill}%
\end{pgfscope}%
\begin{pgfscope}%
\pgfpathrectangle{\pgfqpoint{0.651412in}{0.524170in}}{\pgfqpoint{4.629690in}{2.558193in}}%
\pgfusepath{clip}%
\pgfsetbuttcap%
\pgfsetmiterjoin%
\definecolor{currentfill}{rgb}{0.835294,0.368627,0.000000}%
\pgfsetfillcolor{currentfill}%
\pgfsetfillopacity{0.700000}%
\pgfsetlinewidth{0.000000pt}%
\definecolor{currentstroke}{rgb}{0.000000,0.000000,0.000000}%
\pgfsetstrokecolor{currentstroke}%
\pgfsetstrokeopacity{0.700000}%
\pgfsetdash{}{0pt}%
\pgfpathmoveto{\pgfqpoint{2.461200in}{0.524170in}}%
\pgfpathlineto{\pgfqpoint{2.503288in}{0.524170in}}%
\pgfpathlineto{\pgfqpoint{2.503288in}{0.663494in}}%
\pgfpathlineto{\pgfqpoint{2.461200in}{0.663494in}}%
\pgfpathlineto{\pgfqpoint{2.461200in}{0.524170in}}%
\pgfpathclose%
\pgfusepath{fill}%
\end{pgfscope}%
\begin{pgfscope}%
\pgfpathrectangle{\pgfqpoint{0.651412in}{0.524170in}}{\pgfqpoint{4.629690in}{2.558193in}}%
\pgfusepath{clip}%
\pgfsetbuttcap%
\pgfsetmiterjoin%
\definecolor{currentfill}{rgb}{0.835294,0.368627,0.000000}%
\pgfsetfillcolor{currentfill}%
\pgfsetfillopacity{0.700000}%
\pgfsetlinewidth{0.000000pt}%
\definecolor{currentstroke}{rgb}{0.000000,0.000000,0.000000}%
\pgfsetstrokecolor{currentstroke}%
\pgfsetstrokeopacity{0.700000}%
\pgfsetdash{}{0pt}%
\pgfpathmoveto{\pgfqpoint{2.503288in}{0.524170in}}%
\pgfpathlineto{\pgfqpoint{2.545376in}{0.524170in}}%
\pgfpathlineto{\pgfqpoint{2.545376in}{0.647418in}}%
\pgfpathlineto{\pgfqpoint{2.503288in}{0.647418in}}%
\pgfpathlineto{\pgfqpoint{2.503288in}{0.524170in}}%
\pgfpathclose%
\pgfusepath{fill}%
\end{pgfscope}%
\begin{pgfscope}%
\pgfpathrectangle{\pgfqpoint{0.651412in}{0.524170in}}{\pgfqpoint{4.629690in}{2.558193in}}%
\pgfusepath{clip}%
\pgfsetbuttcap%
\pgfsetmiterjoin%
\definecolor{currentfill}{rgb}{0.835294,0.368627,0.000000}%
\pgfsetfillcolor{currentfill}%
\pgfsetfillopacity{0.700000}%
\pgfsetlinewidth{0.000000pt}%
\definecolor{currentstroke}{rgb}{0.000000,0.000000,0.000000}%
\pgfsetstrokecolor{currentstroke}%
\pgfsetstrokeopacity{0.700000}%
\pgfsetdash{}{0pt}%
\pgfpathmoveto{\pgfqpoint{2.545376in}{0.524170in}}%
\pgfpathlineto{\pgfqpoint{2.587464in}{0.524170in}}%
\pgfpathlineto{\pgfqpoint{2.587464in}{0.684928in}}%
\pgfpathlineto{\pgfqpoint{2.545376in}{0.684928in}}%
\pgfpathlineto{\pgfqpoint{2.545376in}{0.524170in}}%
\pgfpathclose%
\pgfusepath{fill}%
\end{pgfscope}%
\begin{pgfscope}%
\pgfpathrectangle{\pgfqpoint{0.651412in}{0.524170in}}{\pgfqpoint{4.629690in}{2.558193in}}%
\pgfusepath{clip}%
\pgfsetbuttcap%
\pgfsetmiterjoin%
\definecolor{currentfill}{rgb}{0.835294,0.368627,0.000000}%
\pgfsetfillcolor{currentfill}%
\pgfsetfillopacity{0.700000}%
\pgfsetlinewidth{0.000000pt}%
\definecolor{currentstroke}{rgb}{0.000000,0.000000,0.000000}%
\pgfsetstrokecolor{currentstroke}%
\pgfsetstrokeopacity{0.700000}%
\pgfsetdash{}{0pt}%
\pgfpathmoveto{\pgfqpoint{2.587464in}{0.524170in}}%
\pgfpathlineto{\pgfqpoint{2.629553in}{0.524170in}}%
\pgfpathlineto{\pgfqpoint{2.629553in}{0.625983in}}%
\pgfpathlineto{\pgfqpoint{2.587464in}{0.625983in}}%
\pgfpathlineto{\pgfqpoint{2.587464in}{0.524170in}}%
\pgfpathclose%
\pgfusepath{fill}%
\end{pgfscope}%
\begin{pgfscope}%
\pgfpathrectangle{\pgfqpoint{0.651412in}{0.524170in}}{\pgfqpoint{4.629690in}{2.558193in}}%
\pgfusepath{clip}%
\pgfsetbuttcap%
\pgfsetmiterjoin%
\definecolor{currentfill}{rgb}{0.835294,0.368627,0.000000}%
\pgfsetfillcolor{currentfill}%
\pgfsetfillopacity{0.700000}%
\pgfsetlinewidth{0.000000pt}%
\definecolor{currentstroke}{rgb}{0.000000,0.000000,0.000000}%
\pgfsetstrokecolor{currentstroke}%
\pgfsetstrokeopacity{0.700000}%
\pgfsetdash{}{0pt}%
\pgfpathmoveto{\pgfqpoint{2.629553in}{0.524170in}}%
\pgfpathlineto{\pgfqpoint{2.671641in}{0.524170in}}%
\pgfpathlineto{\pgfqpoint{2.671641in}{0.642059in}}%
\pgfpathlineto{\pgfqpoint{2.629553in}{0.642059in}}%
\pgfpathlineto{\pgfqpoint{2.629553in}{0.524170in}}%
\pgfpathclose%
\pgfusepath{fill}%
\end{pgfscope}%
\begin{pgfscope}%
\pgfpathrectangle{\pgfqpoint{0.651412in}{0.524170in}}{\pgfqpoint{4.629690in}{2.558193in}}%
\pgfusepath{clip}%
\pgfsetbuttcap%
\pgfsetmiterjoin%
\definecolor{currentfill}{rgb}{0.835294,0.368627,0.000000}%
\pgfsetfillcolor{currentfill}%
\pgfsetfillopacity{0.700000}%
\pgfsetlinewidth{0.000000pt}%
\definecolor{currentstroke}{rgb}{0.000000,0.000000,0.000000}%
\pgfsetstrokecolor{currentstroke}%
\pgfsetstrokeopacity{0.700000}%
\pgfsetdash{}{0pt}%
\pgfpathmoveto{\pgfqpoint{2.671641in}{0.524170in}}%
\pgfpathlineto{\pgfqpoint{2.713729in}{0.524170in}}%
\pgfpathlineto{\pgfqpoint{2.713729in}{0.684928in}}%
\pgfpathlineto{\pgfqpoint{2.671641in}{0.684928in}}%
\pgfpathlineto{\pgfqpoint{2.671641in}{0.524170in}}%
\pgfpathclose%
\pgfusepath{fill}%
\end{pgfscope}%
\begin{pgfscope}%
\pgfpathrectangle{\pgfqpoint{0.651412in}{0.524170in}}{\pgfqpoint{4.629690in}{2.558193in}}%
\pgfusepath{clip}%
\pgfsetbuttcap%
\pgfsetmiterjoin%
\definecolor{currentfill}{rgb}{0.835294,0.368627,0.000000}%
\pgfsetfillcolor{currentfill}%
\pgfsetfillopacity{0.700000}%
\pgfsetlinewidth{0.000000pt}%
\definecolor{currentstroke}{rgb}{0.000000,0.000000,0.000000}%
\pgfsetstrokecolor{currentstroke}%
\pgfsetstrokeopacity{0.700000}%
\pgfsetdash{}{0pt}%
\pgfpathmoveto{\pgfqpoint{2.713729in}{0.524170in}}%
\pgfpathlineto{\pgfqpoint{2.755817in}{0.524170in}}%
\pgfpathlineto{\pgfqpoint{2.755817in}{0.674211in}}%
\pgfpathlineto{\pgfqpoint{2.713729in}{0.674211in}}%
\pgfpathlineto{\pgfqpoint{2.713729in}{0.524170in}}%
\pgfpathclose%
\pgfusepath{fill}%
\end{pgfscope}%
\begin{pgfscope}%
\pgfpathrectangle{\pgfqpoint{0.651412in}{0.524170in}}{\pgfqpoint{4.629690in}{2.558193in}}%
\pgfusepath{clip}%
\pgfsetbuttcap%
\pgfsetmiterjoin%
\definecolor{currentfill}{rgb}{0.835294,0.368627,0.000000}%
\pgfsetfillcolor{currentfill}%
\pgfsetfillopacity{0.700000}%
\pgfsetlinewidth{0.000000pt}%
\definecolor{currentstroke}{rgb}{0.000000,0.000000,0.000000}%
\pgfsetstrokecolor{currentstroke}%
\pgfsetstrokeopacity{0.700000}%
\pgfsetdash{}{0pt}%
\pgfpathmoveto{\pgfqpoint{2.755817in}{0.524170in}}%
\pgfpathlineto{\pgfqpoint{2.797905in}{0.524170in}}%
\pgfpathlineto{\pgfqpoint{2.797905in}{0.615266in}}%
\pgfpathlineto{\pgfqpoint{2.755817in}{0.615266in}}%
\pgfpathlineto{\pgfqpoint{2.755817in}{0.524170in}}%
\pgfpathclose%
\pgfusepath{fill}%
\end{pgfscope}%
\begin{pgfscope}%
\pgfpathrectangle{\pgfqpoint{0.651412in}{0.524170in}}{\pgfqpoint{4.629690in}{2.558193in}}%
\pgfusepath{clip}%
\pgfsetbuttcap%
\pgfsetmiterjoin%
\definecolor{currentfill}{rgb}{0.835294,0.368627,0.000000}%
\pgfsetfillcolor{currentfill}%
\pgfsetfillopacity{0.700000}%
\pgfsetlinewidth{0.000000pt}%
\definecolor{currentstroke}{rgb}{0.000000,0.000000,0.000000}%
\pgfsetstrokecolor{currentstroke}%
\pgfsetstrokeopacity{0.700000}%
\pgfsetdash{}{0pt}%
\pgfpathmoveto{\pgfqpoint{2.797905in}{0.524170in}}%
\pgfpathlineto{\pgfqpoint{2.839993in}{0.524170in}}%
\pgfpathlineto{\pgfqpoint{2.839993in}{0.620625in}}%
\pgfpathlineto{\pgfqpoint{2.797905in}{0.620625in}}%
\pgfpathlineto{\pgfqpoint{2.797905in}{0.524170in}}%
\pgfpathclose%
\pgfusepath{fill}%
\end{pgfscope}%
\begin{pgfscope}%
\pgfpathrectangle{\pgfqpoint{0.651412in}{0.524170in}}{\pgfqpoint{4.629690in}{2.558193in}}%
\pgfusepath{clip}%
\pgfsetbuttcap%
\pgfsetmiterjoin%
\definecolor{currentfill}{rgb}{0.835294,0.368627,0.000000}%
\pgfsetfillcolor{currentfill}%
\pgfsetfillopacity{0.700000}%
\pgfsetlinewidth{0.000000pt}%
\definecolor{currentstroke}{rgb}{0.000000,0.000000,0.000000}%
\pgfsetstrokecolor{currentstroke}%
\pgfsetstrokeopacity{0.700000}%
\pgfsetdash{}{0pt}%
\pgfpathmoveto{\pgfqpoint{2.839993in}{0.524170in}}%
\pgfpathlineto{\pgfqpoint{2.882081in}{0.524170in}}%
\pgfpathlineto{\pgfqpoint{2.882081in}{0.663494in}}%
\pgfpathlineto{\pgfqpoint{2.839993in}{0.663494in}}%
\pgfpathlineto{\pgfqpoint{2.839993in}{0.524170in}}%
\pgfpathclose%
\pgfusepath{fill}%
\end{pgfscope}%
\begin{pgfscope}%
\pgfpathrectangle{\pgfqpoint{0.651412in}{0.524170in}}{\pgfqpoint{4.629690in}{2.558193in}}%
\pgfusepath{clip}%
\pgfsetbuttcap%
\pgfsetmiterjoin%
\definecolor{currentfill}{rgb}{0.835294,0.368627,0.000000}%
\pgfsetfillcolor{currentfill}%
\pgfsetfillopacity{0.700000}%
\pgfsetlinewidth{0.000000pt}%
\definecolor{currentstroke}{rgb}{0.000000,0.000000,0.000000}%
\pgfsetstrokecolor{currentstroke}%
\pgfsetstrokeopacity{0.700000}%
\pgfsetdash{}{0pt}%
\pgfpathmoveto{\pgfqpoint{2.882081in}{0.524170in}}%
\pgfpathlineto{\pgfqpoint{2.924169in}{0.524170in}}%
\pgfpathlineto{\pgfqpoint{2.924169in}{0.625983in}}%
\pgfpathlineto{\pgfqpoint{2.882081in}{0.625983in}}%
\pgfpathlineto{\pgfqpoint{2.882081in}{0.524170in}}%
\pgfpathclose%
\pgfusepath{fill}%
\end{pgfscope}%
\begin{pgfscope}%
\pgfpathrectangle{\pgfqpoint{0.651412in}{0.524170in}}{\pgfqpoint{4.629690in}{2.558193in}}%
\pgfusepath{clip}%
\pgfsetbuttcap%
\pgfsetmiterjoin%
\definecolor{currentfill}{rgb}{0.835294,0.368627,0.000000}%
\pgfsetfillcolor{currentfill}%
\pgfsetfillopacity{0.700000}%
\pgfsetlinewidth{0.000000pt}%
\definecolor{currentstroke}{rgb}{0.000000,0.000000,0.000000}%
\pgfsetstrokecolor{currentstroke}%
\pgfsetstrokeopacity{0.700000}%
\pgfsetdash{}{0pt}%
\pgfpathmoveto{\pgfqpoint{2.924169in}{0.524170in}}%
\pgfpathlineto{\pgfqpoint{2.966257in}{0.524170in}}%
\pgfpathlineto{\pgfqpoint{2.966257in}{0.625983in}}%
\pgfpathlineto{\pgfqpoint{2.924169in}{0.625983in}}%
\pgfpathlineto{\pgfqpoint{2.924169in}{0.524170in}}%
\pgfpathclose%
\pgfusepath{fill}%
\end{pgfscope}%
\begin{pgfscope}%
\pgfpathrectangle{\pgfqpoint{0.651412in}{0.524170in}}{\pgfqpoint{4.629690in}{2.558193in}}%
\pgfusepath{clip}%
\pgfsetbuttcap%
\pgfsetmiterjoin%
\definecolor{currentfill}{rgb}{0.835294,0.368627,0.000000}%
\pgfsetfillcolor{currentfill}%
\pgfsetfillopacity{0.700000}%
\pgfsetlinewidth{0.000000pt}%
\definecolor{currentstroke}{rgb}{0.000000,0.000000,0.000000}%
\pgfsetstrokecolor{currentstroke}%
\pgfsetstrokeopacity{0.700000}%
\pgfsetdash{}{0pt}%
\pgfpathmoveto{\pgfqpoint{2.966257in}{0.524170in}}%
\pgfpathlineto{\pgfqpoint{3.008345in}{0.524170in}}%
\pgfpathlineto{\pgfqpoint{3.008345in}{0.604549in}}%
\pgfpathlineto{\pgfqpoint{2.966257in}{0.604549in}}%
\pgfpathlineto{\pgfqpoint{2.966257in}{0.524170in}}%
\pgfpathclose%
\pgfusepath{fill}%
\end{pgfscope}%
\begin{pgfscope}%
\pgfpathrectangle{\pgfqpoint{0.651412in}{0.524170in}}{\pgfqpoint{4.629690in}{2.558193in}}%
\pgfusepath{clip}%
\pgfsetbuttcap%
\pgfsetmiterjoin%
\definecolor{currentfill}{rgb}{0.835294,0.368627,0.000000}%
\pgfsetfillcolor{currentfill}%
\pgfsetfillopacity{0.700000}%
\pgfsetlinewidth{0.000000pt}%
\definecolor{currentstroke}{rgb}{0.000000,0.000000,0.000000}%
\pgfsetstrokecolor{currentstroke}%
\pgfsetstrokeopacity{0.700000}%
\pgfsetdash{}{0pt}%
\pgfpathmoveto{\pgfqpoint{3.008345in}{0.524170in}}%
\pgfpathlineto{\pgfqpoint{3.050433in}{0.524170in}}%
\pgfpathlineto{\pgfqpoint{3.050433in}{0.609908in}}%
\pgfpathlineto{\pgfqpoint{3.008345in}{0.609908in}}%
\pgfpathlineto{\pgfqpoint{3.008345in}{0.524170in}}%
\pgfpathclose%
\pgfusepath{fill}%
\end{pgfscope}%
\begin{pgfscope}%
\pgfpathrectangle{\pgfqpoint{0.651412in}{0.524170in}}{\pgfqpoint{4.629690in}{2.558193in}}%
\pgfusepath{clip}%
\pgfsetbuttcap%
\pgfsetmiterjoin%
\definecolor{currentfill}{rgb}{0.835294,0.368627,0.000000}%
\pgfsetfillcolor{currentfill}%
\pgfsetfillopacity{0.700000}%
\pgfsetlinewidth{0.000000pt}%
\definecolor{currentstroke}{rgb}{0.000000,0.000000,0.000000}%
\pgfsetstrokecolor{currentstroke}%
\pgfsetstrokeopacity{0.700000}%
\pgfsetdash{}{0pt}%
\pgfpathmoveto{\pgfqpoint{3.050433in}{0.524170in}}%
\pgfpathlineto{\pgfqpoint{3.092522in}{0.524170in}}%
\pgfpathlineto{\pgfqpoint{3.092522in}{0.625983in}}%
\pgfpathlineto{\pgfqpoint{3.050433in}{0.625983in}}%
\pgfpathlineto{\pgfqpoint{3.050433in}{0.524170in}}%
\pgfpathclose%
\pgfusepath{fill}%
\end{pgfscope}%
\begin{pgfscope}%
\pgfpathrectangle{\pgfqpoint{0.651412in}{0.524170in}}{\pgfqpoint{4.629690in}{2.558193in}}%
\pgfusepath{clip}%
\pgfsetbuttcap%
\pgfsetmiterjoin%
\definecolor{currentfill}{rgb}{0.835294,0.368627,0.000000}%
\pgfsetfillcolor{currentfill}%
\pgfsetfillopacity{0.700000}%
\pgfsetlinewidth{0.000000pt}%
\definecolor{currentstroke}{rgb}{0.000000,0.000000,0.000000}%
\pgfsetstrokecolor{currentstroke}%
\pgfsetstrokeopacity{0.700000}%
\pgfsetdash{}{0pt}%
\pgfpathmoveto{\pgfqpoint{3.092522in}{0.524170in}}%
\pgfpathlineto{\pgfqpoint{3.134610in}{0.524170in}}%
\pgfpathlineto{\pgfqpoint{3.134610in}{0.604549in}}%
\pgfpathlineto{\pgfqpoint{3.092522in}{0.604549in}}%
\pgfpathlineto{\pgfqpoint{3.092522in}{0.524170in}}%
\pgfpathclose%
\pgfusepath{fill}%
\end{pgfscope}%
\begin{pgfscope}%
\pgfpathrectangle{\pgfqpoint{0.651412in}{0.524170in}}{\pgfqpoint{4.629690in}{2.558193in}}%
\pgfusepath{clip}%
\pgfsetbuttcap%
\pgfsetmiterjoin%
\definecolor{currentfill}{rgb}{0.835294,0.368627,0.000000}%
\pgfsetfillcolor{currentfill}%
\pgfsetfillopacity{0.700000}%
\pgfsetlinewidth{0.000000pt}%
\definecolor{currentstroke}{rgb}{0.000000,0.000000,0.000000}%
\pgfsetstrokecolor{currentstroke}%
\pgfsetstrokeopacity{0.700000}%
\pgfsetdash{}{0pt}%
\pgfpathmoveto{\pgfqpoint{3.134610in}{0.524170in}}%
\pgfpathlineto{\pgfqpoint{3.176698in}{0.524170in}}%
\pgfpathlineto{\pgfqpoint{3.176698in}{0.577756in}}%
\pgfpathlineto{\pgfqpoint{3.134610in}{0.577756in}}%
\pgfpathlineto{\pgfqpoint{3.134610in}{0.524170in}}%
\pgfpathclose%
\pgfusepath{fill}%
\end{pgfscope}%
\begin{pgfscope}%
\pgfpathrectangle{\pgfqpoint{0.651412in}{0.524170in}}{\pgfqpoint{4.629690in}{2.558193in}}%
\pgfusepath{clip}%
\pgfsetbuttcap%
\pgfsetmiterjoin%
\definecolor{currentfill}{rgb}{0.835294,0.368627,0.000000}%
\pgfsetfillcolor{currentfill}%
\pgfsetfillopacity{0.700000}%
\pgfsetlinewidth{0.000000pt}%
\definecolor{currentstroke}{rgb}{0.000000,0.000000,0.000000}%
\pgfsetstrokecolor{currentstroke}%
\pgfsetstrokeopacity{0.700000}%
\pgfsetdash{}{0pt}%
\pgfpathmoveto{\pgfqpoint{3.176698in}{0.524170in}}%
\pgfpathlineto{\pgfqpoint{3.218786in}{0.524170in}}%
\pgfpathlineto{\pgfqpoint{3.218786in}{0.636701in}}%
\pgfpathlineto{\pgfqpoint{3.176698in}{0.636701in}}%
\pgfpathlineto{\pgfqpoint{3.176698in}{0.524170in}}%
\pgfpathclose%
\pgfusepath{fill}%
\end{pgfscope}%
\begin{pgfscope}%
\pgfpathrectangle{\pgfqpoint{0.651412in}{0.524170in}}{\pgfqpoint{4.629690in}{2.558193in}}%
\pgfusepath{clip}%
\pgfsetbuttcap%
\pgfsetmiterjoin%
\definecolor{currentfill}{rgb}{0.835294,0.368627,0.000000}%
\pgfsetfillcolor{currentfill}%
\pgfsetfillopacity{0.700000}%
\pgfsetlinewidth{0.000000pt}%
\definecolor{currentstroke}{rgb}{0.000000,0.000000,0.000000}%
\pgfsetstrokecolor{currentstroke}%
\pgfsetstrokeopacity{0.700000}%
\pgfsetdash{}{0pt}%
\pgfpathmoveto{\pgfqpoint{3.218786in}{0.524170in}}%
\pgfpathlineto{\pgfqpoint{3.260874in}{0.524170in}}%
\pgfpathlineto{\pgfqpoint{3.260874in}{0.577756in}}%
\pgfpathlineto{\pgfqpoint{3.218786in}{0.577756in}}%
\pgfpathlineto{\pgfqpoint{3.218786in}{0.524170in}}%
\pgfpathclose%
\pgfusepath{fill}%
\end{pgfscope}%
\begin{pgfscope}%
\pgfpathrectangle{\pgfqpoint{0.651412in}{0.524170in}}{\pgfqpoint{4.629690in}{2.558193in}}%
\pgfusepath{clip}%
\pgfsetbuttcap%
\pgfsetmiterjoin%
\definecolor{currentfill}{rgb}{0.835294,0.368627,0.000000}%
\pgfsetfillcolor{currentfill}%
\pgfsetfillopacity{0.700000}%
\pgfsetlinewidth{0.000000pt}%
\definecolor{currentstroke}{rgb}{0.000000,0.000000,0.000000}%
\pgfsetstrokecolor{currentstroke}%
\pgfsetstrokeopacity{0.700000}%
\pgfsetdash{}{0pt}%
\pgfpathmoveto{\pgfqpoint{3.260874in}{0.524170in}}%
\pgfpathlineto{\pgfqpoint{3.302962in}{0.524170in}}%
\pgfpathlineto{\pgfqpoint{3.302962in}{0.577756in}}%
\pgfpathlineto{\pgfqpoint{3.260874in}{0.577756in}}%
\pgfpathlineto{\pgfqpoint{3.260874in}{0.524170in}}%
\pgfpathclose%
\pgfusepath{fill}%
\end{pgfscope}%
\begin{pgfscope}%
\pgfpathrectangle{\pgfqpoint{0.651412in}{0.524170in}}{\pgfqpoint{4.629690in}{2.558193in}}%
\pgfusepath{clip}%
\pgfsetbuttcap%
\pgfsetmiterjoin%
\definecolor{currentfill}{rgb}{0.835294,0.368627,0.000000}%
\pgfsetfillcolor{currentfill}%
\pgfsetfillopacity{0.700000}%
\pgfsetlinewidth{0.000000pt}%
\definecolor{currentstroke}{rgb}{0.000000,0.000000,0.000000}%
\pgfsetstrokecolor{currentstroke}%
\pgfsetstrokeopacity{0.700000}%
\pgfsetdash{}{0pt}%
\pgfpathmoveto{\pgfqpoint{3.302962in}{0.524170in}}%
\pgfpathlineto{\pgfqpoint{3.345050in}{0.524170in}}%
\pgfpathlineto{\pgfqpoint{3.345050in}{0.572397in}}%
\pgfpathlineto{\pgfqpoint{3.302962in}{0.572397in}}%
\pgfpathlineto{\pgfqpoint{3.302962in}{0.524170in}}%
\pgfpathclose%
\pgfusepath{fill}%
\end{pgfscope}%
\begin{pgfscope}%
\pgfpathrectangle{\pgfqpoint{0.651412in}{0.524170in}}{\pgfqpoint{4.629690in}{2.558193in}}%
\pgfusepath{clip}%
\pgfsetbuttcap%
\pgfsetmiterjoin%
\definecolor{currentfill}{rgb}{0.835294,0.368627,0.000000}%
\pgfsetfillcolor{currentfill}%
\pgfsetfillopacity{0.700000}%
\pgfsetlinewidth{0.000000pt}%
\definecolor{currentstroke}{rgb}{0.000000,0.000000,0.000000}%
\pgfsetstrokecolor{currentstroke}%
\pgfsetstrokeopacity{0.700000}%
\pgfsetdash{}{0pt}%
\pgfpathmoveto{\pgfqpoint{3.345050in}{0.524170in}}%
\pgfpathlineto{\pgfqpoint{3.387138in}{0.524170in}}%
\pgfpathlineto{\pgfqpoint{3.387138in}{0.599190in}}%
\pgfpathlineto{\pgfqpoint{3.345050in}{0.599190in}}%
\pgfpathlineto{\pgfqpoint{3.345050in}{0.524170in}}%
\pgfpathclose%
\pgfusepath{fill}%
\end{pgfscope}%
\begin{pgfscope}%
\pgfpathrectangle{\pgfqpoint{0.651412in}{0.524170in}}{\pgfqpoint{4.629690in}{2.558193in}}%
\pgfusepath{clip}%
\pgfsetbuttcap%
\pgfsetmiterjoin%
\definecolor{currentfill}{rgb}{0.835294,0.368627,0.000000}%
\pgfsetfillcolor{currentfill}%
\pgfsetfillopacity{0.700000}%
\pgfsetlinewidth{0.000000pt}%
\definecolor{currentstroke}{rgb}{0.000000,0.000000,0.000000}%
\pgfsetstrokecolor{currentstroke}%
\pgfsetstrokeopacity{0.700000}%
\pgfsetdash{}{0pt}%
\pgfpathmoveto{\pgfqpoint{3.387138in}{0.524170in}}%
\pgfpathlineto{\pgfqpoint{3.429226in}{0.524170in}}%
\pgfpathlineto{\pgfqpoint{3.429226in}{0.625983in}}%
\pgfpathlineto{\pgfqpoint{3.387138in}{0.625983in}}%
\pgfpathlineto{\pgfqpoint{3.387138in}{0.524170in}}%
\pgfpathclose%
\pgfusepath{fill}%
\end{pgfscope}%
\begin{pgfscope}%
\pgfpathrectangle{\pgfqpoint{0.651412in}{0.524170in}}{\pgfqpoint{4.629690in}{2.558193in}}%
\pgfusepath{clip}%
\pgfsetbuttcap%
\pgfsetmiterjoin%
\definecolor{currentfill}{rgb}{0.835294,0.368627,0.000000}%
\pgfsetfillcolor{currentfill}%
\pgfsetfillopacity{0.700000}%
\pgfsetlinewidth{0.000000pt}%
\definecolor{currentstroke}{rgb}{0.000000,0.000000,0.000000}%
\pgfsetstrokecolor{currentstroke}%
\pgfsetstrokeopacity{0.700000}%
\pgfsetdash{}{0pt}%
\pgfpathmoveto{\pgfqpoint{3.429226in}{0.524170in}}%
\pgfpathlineto{\pgfqpoint{3.471314in}{0.524170in}}%
\pgfpathlineto{\pgfqpoint{3.471314in}{0.550963in}}%
\pgfpathlineto{\pgfqpoint{3.429226in}{0.550963in}}%
\pgfpathlineto{\pgfqpoint{3.429226in}{0.524170in}}%
\pgfpathclose%
\pgfusepath{fill}%
\end{pgfscope}%
\begin{pgfscope}%
\pgfpathrectangle{\pgfqpoint{0.651412in}{0.524170in}}{\pgfqpoint{4.629690in}{2.558193in}}%
\pgfusepath{clip}%
\pgfsetbuttcap%
\pgfsetmiterjoin%
\definecolor{currentfill}{rgb}{0.835294,0.368627,0.000000}%
\pgfsetfillcolor{currentfill}%
\pgfsetfillopacity{0.700000}%
\pgfsetlinewidth{0.000000pt}%
\definecolor{currentstroke}{rgb}{0.000000,0.000000,0.000000}%
\pgfsetstrokecolor{currentstroke}%
\pgfsetstrokeopacity{0.700000}%
\pgfsetdash{}{0pt}%
\pgfpathmoveto{\pgfqpoint{3.471314in}{0.524170in}}%
\pgfpathlineto{\pgfqpoint{3.513403in}{0.524170in}}%
\pgfpathlineto{\pgfqpoint{3.513403in}{0.593832in}}%
\pgfpathlineto{\pgfqpoint{3.471314in}{0.593832in}}%
\pgfpathlineto{\pgfqpoint{3.471314in}{0.524170in}}%
\pgfpathclose%
\pgfusepath{fill}%
\end{pgfscope}%
\begin{pgfscope}%
\pgfpathrectangle{\pgfqpoint{0.651412in}{0.524170in}}{\pgfqpoint{4.629690in}{2.558193in}}%
\pgfusepath{clip}%
\pgfsetbuttcap%
\pgfsetmiterjoin%
\definecolor{currentfill}{rgb}{0.835294,0.368627,0.000000}%
\pgfsetfillcolor{currentfill}%
\pgfsetfillopacity{0.700000}%
\pgfsetlinewidth{0.000000pt}%
\definecolor{currentstroke}{rgb}{0.000000,0.000000,0.000000}%
\pgfsetstrokecolor{currentstroke}%
\pgfsetstrokeopacity{0.700000}%
\pgfsetdash{}{0pt}%
\pgfpathmoveto{\pgfqpoint{3.513403in}{0.524170in}}%
\pgfpathlineto{\pgfqpoint{3.555491in}{0.524170in}}%
\pgfpathlineto{\pgfqpoint{3.555491in}{0.588473in}}%
\pgfpathlineto{\pgfqpoint{3.513403in}{0.588473in}}%
\pgfpathlineto{\pgfqpoint{3.513403in}{0.524170in}}%
\pgfpathclose%
\pgfusepath{fill}%
\end{pgfscope}%
\begin{pgfscope}%
\pgfpathrectangle{\pgfqpoint{0.651412in}{0.524170in}}{\pgfqpoint{4.629690in}{2.558193in}}%
\pgfusepath{clip}%
\pgfsetbuttcap%
\pgfsetmiterjoin%
\definecolor{currentfill}{rgb}{0.835294,0.368627,0.000000}%
\pgfsetfillcolor{currentfill}%
\pgfsetfillopacity{0.700000}%
\pgfsetlinewidth{0.000000pt}%
\definecolor{currentstroke}{rgb}{0.000000,0.000000,0.000000}%
\pgfsetstrokecolor{currentstroke}%
\pgfsetstrokeopacity{0.700000}%
\pgfsetdash{}{0pt}%
\pgfpathmoveto{\pgfqpoint{3.555491in}{0.524170in}}%
\pgfpathlineto{\pgfqpoint{3.597579in}{0.524170in}}%
\pgfpathlineto{\pgfqpoint{3.597579in}{0.572397in}}%
\pgfpathlineto{\pgfqpoint{3.555491in}{0.572397in}}%
\pgfpathlineto{\pgfqpoint{3.555491in}{0.524170in}}%
\pgfpathclose%
\pgfusepath{fill}%
\end{pgfscope}%
\begin{pgfscope}%
\pgfpathrectangle{\pgfqpoint{0.651412in}{0.524170in}}{\pgfqpoint{4.629690in}{2.558193in}}%
\pgfusepath{clip}%
\pgfsetbuttcap%
\pgfsetmiterjoin%
\definecolor{currentfill}{rgb}{0.835294,0.368627,0.000000}%
\pgfsetfillcolor{currentfill}%
\pgfsetfillopacity{0.700000}%
\pgfsetlinewidth{0.000000pt}%
\definecolor{currentstroke}{rgb}{0.000000,0.000000,0.000000}%
\pgfsetstrokecolor{currentstroke}%
\pgfsetstrokeopacity{0.700000}%
\pgfsetdash{}{0pt}%
\pgfpathmoveto{\pgfqpoint{3.597579in}{0.524170in}}%
\pgfpathlineto{\pgfqpoint{3.639667in}{0.524170in}}%
\pgfpathlineto{\pgfqpoint{3.639667in}{0.604549in}}%
\pgfpathlineto{\pgfqpoint{3.597579in}{0.604549in}}%
\pgfpathlineto{\pgfqpoint{3.597579in}{0.524170in}}%
\pgfpathclose%
\pgfusepath{fill}%
\end{pgfscope}%
\begin{pgfscope}%
\pgfpathrectangle{\pgfqpoint{0.651412in}{0.524170in}}{\pgfqpoint{4.629690in}{2.558193in}}%
\pgfusepath{clip}%
\pgfsetbuttcap%
\pgfsetmiterjoin%
\definecolor{currentfill}{rgb}{0.835294,0.368627,0.000000}%
\pgfsetfillcolor{currentfill}%
\pgfsetfillopacity{0.700000}%
\pgfsetlinewidth{0.000000pt}%
\definecolor{currentstroke}{rgb}{0.000000,0.000000,0.000000}%
\pgfsetstrokecolor{currentstroke}%
\pgfsetstrokeopacity{0.700000}%
\pgfsetdash{}{0pt}%
\pgfpathmoveto{\pgfqpoint{3.639667in}{0.524170in}}%
\pgfpathlineto{\pgfqpoint{3.681755in}{0.524170in}}%
\pgfpathlineto{\pgfqpoint{3.681755in}{0.550963in}}%
\pgfpathlineto{\pgfqpoint{3.639667in}{0.550963in}}%
\pgfpathlineto{\pgfqpoint{3.639667in}{0.524170in}}%
\pgfpathclose%
\pgfusepath{fill}%
\end{pgfscope}%
\begin{pgfscope}%
\pgfpathrectangle{\pgfqpoint{0.651412in}{0.524170in}}{\pgfqpoint{4.629690in}{2.558193in}}%
\pgfusepath{clip}%
\pgfsetbuttcap%
\pgfsetmiterjoin%
\definecolor{currentfill}{rgb}{0.835294,0.368627,0.000000}%
\pgfsetfillcolor{currentfill}%
\pgfsetfillopacity{0.700000}%
\pgfsetlinewidth{0.000000pt}%
\definecolor{currentstroke}{rgb}{0.000000,0.000000,0.000000}%
\pgfsetstrokecolor{currentstroke}%
\pgfsetstrokeopacity{0.700000}%
\pgfsetdash{}{0pt}%
\pgfpathmoveto{\pgfqpoint{3.681755in}{0.524170in}}%
\pgfpathlineto{\pgfqpoint{3.723843in}{0.524170in}}%
\pgfpathlineto{\pgfqpoint{3.723843in}{0.609908in}}%
\pgfpathlineto{\pgfqpoint{3.681755in}{0.609908in}}%
\pgfpathlineto{\pgfqpoint{3.681755in}{0.524170in}}%
\pgfpathclose%
\pgfusepath{fill}%
\end{pgfscope}%
\begin{pgfscope}%
\pgfpathrectangle{\pgfqpoint{0.651412in}{0.524170in}}{\pgfqpoint{4.629690in}{2.558193in}}%
\pgfusepath{clip}%
\pgfsetbuttcap%
\pgfsetmiterjoin%
\definecolor{currentfill}{rgb}{0.835294,0.368627,0.000000}%
\pgfsetfillcolor{currentfill}%
\pgfsetfillopacity{0.700000}%
\pgfsetlinewidth{0.000000pt}%
\definecolor{currentstroke}{rgb}{0.000000,0.000000,0.000000}%
\pgfsetstrokecolor{currentstroke}%
\pgfsetstrokeopacity{0.700000}%
\pgfsetdash{}{0pt}%
\pgfpathmoveto{\pgfqpoint{3.723843in}{0.524170in}}%
\pgfpathlineto{\pgfqpoint{3.765931in}{0.524170in}}%
\pgfpathlineto{\pgfqpoint{3.765931in}{0.561680in}}%
\pgfpathlineto{\pgfqpoint{3.723843in}{0.561680in}}%
\pgfpathlineto{\pgfqpoint{3.723843in}{0.524170in}}%
\pgfpathclose%
\pgfusepath{fill}%
\end{pgfscope}%
\begin{pgfscope}%
\pgfpathrectangle{\pgfqpoint{0.651412in}{0.524170in}}{\pgfqpoint{4.629690in}{2.558193in}}%
\pgfusepath{clip}%
\pgfsetbuttcap%
\pgfsetmiterjoin%
\definecolor{currentfill}{rgb}{0.835294,0.368627,0.000000}%
\pgfsetfillcolor{currentfill}%
\pgfsetfillopacity{0.700000}%
\pgfsetlinewidth{0.000000pt}%
\definecolor{currentstroke}{rgb}{0.000000,0.000000,0.000000}%
\pgfsetstrokecolor{currentstroke}%
\pgfsetstrokeopacity{0.700000}%
\pgfsetdash{}{0pt}%
\pgfpathmoveto{\pgfqpoint{3.765931in}{0.524170in}}%
\pgfpathlineto{\pgfqpoint{3.808019in}{0.524170in}}%
\pgfpathlineto{\pgfqpoint{3.808019in}{0.561680in}}%
\pgfpathlineto{\pgfqpoint{3.765931in}{0.561680in}}%
\pgfpathlineto{\pgfqpoint{3.765931in}{0.524170in}}%
\pgfpathclose%
\pgfusepath{fill}%
\end{pgfscope}%
\begin{pgfscope}%
\pgfpathrectangle{\pgfqpoint{0.651412in}{0.524170in}}{\pgfqpoint{4.629690in}{2.558193in}}%
\pgfusepath{clip}%
\pgfsetbuttcap%
\pgfsetmiterjoin%
\definecolor{currentfill}{rgb}{0.835294,0.368627,0.000000}%
\pgfsetfillcolor{currentfill}%
\pgfsetfillopacity{0.700000}%
\pgfsetlinewidth{0.000000pt}%
\definecolor{currentstroke}{rgb}{0.000000,0.000000,0.000000}%
\pgfsetstrokecolor{currentstroke}%
\pgfsetstrokeopacity{0.700000}%
\pgfsetdash{}{0pt}%
\pgfpathmoveto{\pgfqpoint{3.808019in}{0.524170in}}%
\pgfpathlineto{\pgfqpoint{3.850107in}{0.524170in}}%
\pgfpathlineto{\pgfqpoint{3.850107in}{0.588473in}}%
\pgfpathlineto{\pgfqpoint{3.808019in}{0.588473in}}%
\pgfpathlineto{\pgfqpoint{3.808019in}{0.524170in}}%
\pgfpathclose%
\pgfusepath{fill}%
\end{pgfscope}%
\begin{pgfscope}%
\pgfpathrectangle{\pgfqpoint{0.651412in}{0.524170in}}{\pgfqpoint{4.629690in}{2.558193in}}%
\pgfusepath{clip}%
\pgfsetbuttcap%
\pgfsetmiterjoin%
\definecolor{currentfill}{rgb}{0.835294,0.368627,0.000000}%
\pgfsetfillcolor{currentfill}%
\pgfsetfillopacity{0.700000}%
\pgfsetlinewidth{0.000000pt}%
\definecolor{currentstroke}{rgb}{0.000000,0.000000,0.000000}%
\pgfsetstrokecolor{currentstroke}%
\pgfsetstrokeopacity{0.700000}%
\pgfsetdash{}{0pt}%
\pgfpathmoveto{\pgfqpoint{3.850107in}{0.524170in}}%
\pgfpathlineto{\pgfqpoint{3.892195in}{0.524170in}}%
\pgfpathlineto{\pgfqpoint{3.892195in}{0.583115in}}%
\pgfpathlineto{\pgfqpoint{3.850107in}{0.583115in}}%
\pgfpathlineto{\pgfqpoint{3.850107in}{0.524170in}}%
\pgfpathclose%
\pgfusepath{fill}%
\end{pgfscope}%
\begin{pgfscope}%
\pgfpathrectangle{\pgfqpoint{0.651412in}{0.524170in}}{\pgfqpoint{4.629690in}{2.558193in}}%
\pgfusepath{clip}%
\pgfsetbuttcap%
\pgfsetmiterjoin%
\definecolor{currentfill}{rgb}{0.835294,0.368627,0.000000}%
\pgfsetfillcolor{currentfill}%
\pgfsetfillopacity{0.700000}%
\pgfsetlinewidth{0.000000pt}%
\definecolor{currentstroke}{rgb}{0.000000,0.000000,0.000000}%
\pgfsetstrokecolor{currentstroke}%
\pgfsetstrokeopacity{0.700000}%
\pgfsetdash{}{0pt}%
\pgfpathmoveto{\pgfqpoint{3.892195in}{0.524170in}}%
\pgfpathlineto{\pgfqpoint{3.934283in}{0.524170in}}%
\pgfpathlineto{\pgfqpoint{3.934283in}{0.540246in}}%
\pgfpathlineto{\pgfqpoint{3.892195in}{0.540246in}}%
\pgfpathlineto{\pgfqpoint{3.892195in}{0.524170in}}%
\pgfpathclose%
\pgfusepath{fill}%
\end{pgfscope}%
\begin{pgfscope}%
\pgfpathrectangle{\pgfqpoint{0.651412in}{0.524170in}}{\pgfqpoint{4.629690in}{2.558193in}}%
\pgfusepath{clip}%
\pgfsetbuttcap%
\pgfsetmiterjoin%
\definecolor{currentfill}{rgb}{0.835294,0.368627,0.000000}%
\pgfsetfillcolor{currentfill}%
\pgfsetfillopacity{0.700000}%
\pgfsetlinewidth{0.000000pt}%
\definecolor{currentstroke}{rgb}{0.000000,0.000000,0.000000}%
\pgfsetstrokecolor{currentstroke}%
\pgfsetstrokeopacity{0.700000}%
\pgfsetdash{}{0pt}%
\pgfpathmoveto{\pgfqpoint{3.934283in}{0.524170in}}%
\pgfpathlineto{\pgfqpoint{3.976372in}{0.524170in}}%
\pgfpathlineto{\pgfqpoint{3.976372in}{0.593832in}}%
\pgfpathlineto{\pgfqpoint{3.934283in}{0.593832in}}%
\pgfpathlineto{\pgfqpoint{3.934283in}{0.524170in}}%
\pgfpathclose%
\pgfusepath{fill}%
\end{pgfscope}%
\begin{pgfscope}%
\pgfpathrectangle{\pgfqpoint{0.651412in}{0.524170in}}{\pgfqpoint{4.629690in}{2.558193in}}%
\pgfusepath{clip}%
\pgfsetbuttcap%
\pgfsetmiterjoin%
\definecolor{currentfill}{rgb}{0.835294,0.368627,0.000000}%
\pgfsetfillcolor{currentfill}%
\pgfsetfillopacity{0.700000}%
\pgfsetlinewidth{0.000000pt}%
\definecolor{currentstroke}{rgb}{0.000000,0.000000,0.000000}%
\pgfsetstrokecolor{currentstroke}%
\pgfsetstrokeopacity{0.700000}%
\pgfsetdash{}{0pt}%
\pgfpathmoveto{\pgfqpoint{3.976372in}{0.524170in}}%
\pgfpathlineto{\pgfqpoint{4.018460in}{0.524170in}}%
\pgfpathlineto{\pgfqpoint{4.018460in}{0.572397in}}%
\pgfpathlineto{\pgfqpoint{3.976372in}{0.572397in}}%
\pgfpathlineto{\pgfqpoint{3.976372in}{0.524170in}}%
\pgfpathclose%
\pgfusepath{fill}%
\end{pgfscope}%
\begin{pgfscope}%
\pgfpathrectangle{\pgfqpoint{0.651412in}{0.524170in}}{\pgfqpoint{4.629690in}{2.558193in}}%
\pgfusepath{clip}%
\pgfsetbuttcap%
\pgfsetmiterjoin%
\definecolor{currentfill}{rgb}{0.835294,0.368627,0.000000}%
\pgfsetfillcolor{currentfill}%
\pgfsetfillopacity{0.700000}%
\pgfsetlinewidth{0.000000pt}%
\definecolor{currentstroke}{rgb}{0.000000,0.000000,0.000000}%
\pgfsetstrokecolor{currentstroke}%
\pgfsetstrokeopacity{0.700000}%
\pgfsetdash{}{0pt}%
\pgfpathmoveto{\pgfqpoint{4.018460in}{0.524170in}}%
\pgfpathlineto{\pgfqpoint{4.060548in}{0.524170in}}%
\pgfpathlineto{\pgfqpoint{4.060548in}{0.540246in}}%
\pgfpathlineto{\pgfqpoint{4.018460in}{0.540246in}}%
\pgfpathlineto{\pgfqpoint{4.018460in}{0.524170in}}%
\pgfpathclose%
\pgfusepath{fill}%
\end{pgfscope}%
\begin{pgfscope}%
\pgfpathrectangle{\pgfqpoint{0.651412in}{0.524170in}}{\pgfqpoint{4.629690in}{2.558193in}}%
\pgfusepath{clip}%
\pgfsetbuttcap%
\pgfsetmiterjoin%
\definecolor{currentfill}{rgb}{0.835294,0.368627,0.000000}%
\pgfsetfillcolor{currentfill}%
\pgfsetfillopacity{0.700000}%
\pgfsetlinewidth{0.000000pt}%
\definecolor{currentstroke}{rgb}{0.000000,0.000000,0.000000}%
\pgfsetstrokecolor{currentstroke}%
\pgfsetstrokeopacity{0.700000}%
\pgfsetdash{}{0pt}%
\pgfpathmoveto{\pgfqpoint{4.060548in}{0.524170in}}%
\pgfpathlineto{\pgfqpoint{4.102636in}{0.524170in}}%
\pgfpathlineto{\pgfqpoint{4.102636in}{0.588473in}}%
\pgfpathlineto{\pgfqpoint{4.060548in}{0.588473in}}%
\pgfpathlineto{\pgfqpoint{4.060548in}{0.524170in}}%
\pgfpathclose%
\pgfusepath{fill}%
\end{pgfscope}%
\begin{pgfscope}%
\pgfpathrectangle{\pgfqpoint{0.651412in}{0.524170in}}{\pgfqpoint{4.629690in}{2.558193in}}%
\pgfusepath{clip}%
\pgfsetbuttcap%
\pgfsetmiterjoin%
\definecolor{currentfill}{rgb}{0.835294,0.368627,0.000000}%
\pgfsetfillcolor{currentfill}%
\pgfsetfillopacity{0.700000}%
\pgfsetlinewidth{0.000000pt}%
\definecolor{currentstroke}{rgb}{0.000000,0.000000,0.000000}%
\pgfsetstrokecolor{currentstroke}%
\pgfsetstrokeopacity{0.700000}%
\pgfsetdash{}{0pt}%
\pgfpathmoveto{\pgfqpoint{4.102636in}{0.524170in}}%
\pgfpathlineto{\pgfqpoint{4.144724in}{0.524170in}}%
\pgfpathlineto{\pgfqpoint{4.144724in}{0.529528in}}%
\pgfpathlineto{\pgfqpoint{4.102636in}{0.529528in}}%
\pgfpathlineto{\pgfqpoint{4.102636in}{0.524170in}}%
\pgfpathclose%
\pgfusepath{fill}%
\end{pgfscope}%
\begin{pgfscope}%
\pgfpathrectangle{\pgfqpoint{0.651412in}{0.524170in}}{\pgfqpoint{4.629690in}{2.558193in}}%
\pgfusepath{clip}%
\pgfsetbuttcap%
\pgfsetmiterjoin%
\definecolor{currentfill}{rgb}{0.835294,0.368627,0.000000}%
\pgfsetfillcolor{currentfill}%
\pgfsetfillopacity{0.700000}%
\pgfsetlinewidth{0.000000pt}%
\definecolor{currentstroke}{rgb}{0.000000,0.000000,0.000000}%
\pgfsetstrokecolor{currentstroke}%
\pgfsetstrokeopacity{0.700000}%
\pgfsetdash{}{0pt}%
\pgfpathmoveto{\pgfqpoint{4.144724in}{0.524170in}}%
\pgfpathlineto{\pgfqpoint{4.186812in}{0.524170in}}%
\pgfpathlineto{\pgfqpoint{4.186812in}{0.556321in}}%
\pgfpathlineto{\pgfqpoint{4.144724in}{0.556321in}}%
\pgfpathlineto{\pgfqpoint{4.144724in}{0.524170in}}%
\pgfpathclose%
\pgfusepath{fill}%
\end{pgfscope}%
\begin{pgfscope}%
\pgfpathrectangle{\pgfqpoint{0.651412in}{0.524170in}}{\pgfqpoint{4.629690in}{2.558193in}}%
\pgfusepath{clip}%
\pgfsetbuttcap%
\pgfsetmiterjoin%
\definecolor{currentfill}{rgb}{0.835294,0.368627,0.000000}%
\pgfsetfillcolor{currentfill}%
\pgfsetfillopacity{0.700000}%
\pgfsetlinewidth{0.000000pt}%
\definecolor{currentstroke}{rgb}{0.000000,0.000000,0.000000}%
\pgfsetstrokecolor{currentstroke}%
\pgfsetstrokeopacity{0.700000}%
\pgfsetdash{}{0pt}%
\pgfpathmoveto{\pgfqpoint{4.186812in}{0.524170in}}%
\pgfpathlineto{\pgfqpoint{4.228900in}{0.524170in}}%
\pgfpathlineto{\pgfqpoint{4.228900in}{0.577756in}}%
\pgfpathlineto{\pgfqpoint{4.186812in}{0.577756in}}%
\pgfpathlineto{\pgfqpoint{4.186812in}{0.524170in}}%
\pgfpathclose%
\pgfusepath{fill}%
\end{pgfscope}%
\begin{pgfscope}%
\pgfpathrectangle{\pgfqpoint{0.651412in}{0.524170in}}{\pgfqpoint{4.629690in}{2.558193in}}%
\pgfusepath{clip}%
\pgfsetbuttcap%
\pgfsetmiterjoin%
\definecolor{currentfill}{rgb}{0.835294,0.368627,0.000000}%
\pgfsetfillcolor{currentfill}%
\pgfsetfillopacity{0.700000}%
\pgfsetlinewidth{0.000000pt}%
\definecolor{currentstroke}{rgb}{0.000000,0.000000,0.000000}%
\pgfsetstrokecolor{currentstroke}%
\pgfsetstrokeopacity{0.700000}%
\pgfsetdash{}{0pt}%
\pgfpathmoveto{\pgfqpoint{4.228900in}{0.524170in}}%
\pgfpathlineto{\pgfqpoint{4.270988in}{0.524170in}}%
\pgfpathlineto{\pgfqpoint{4.270988in}{0.545604in}}%
\pgfpathlineto{\pgfqpoint{4.228900in}{0.545604in}}%
\pgfpathlineto{\pgfqpoint{4.228900in}{0.524170in}}%
\pgfpathclose%
\pgfusepath{fill}%
\end{pgfscope}%
\begin{pgfscope}%
\pgfpathrectangle{\pgfqpoint{0.651412in}{0.524170in}}{\pgfqpoint{4.629690in}{2.558193in}}%
\pgfusepath{clip}%
\pgfsetbuttcap%
\pgfsetmiterjoin%
\definecolor{currentfill}{rgb}{0.835294,0.368627,0.000000}%
\pgfsetfillcolor{currentfill}%
\pgfsetfillopacity{0.700000}%
\pgfsetlinewidth{0.000000pt}%
\definecolor{currentstroke}{rgb}{0.000000,0.000000,0.000000}%
\pgfsetstrokecolor{currentstroke}%
\pgfsetstrokeopacity{0.700000}%
\pgfsetdash{}{0pt}%
\pgfpathmoveto{\pgfqpoint{4.270988in}{0.524170in}}%
\pgfpathlineto{\pgfqpoint{4.313076in}{0.524170in}}%
\pgfpathlineto{\pgfqpoint{4.313076in}{0.567039in}}%
\pgfpathlineto{\pgfqpoint{4.270988in}{0.567039in}}%
\pgfpathlineto{\pgfqpoint{4.270988in}{0.524170in}}%
\pgfpathclose%
\pgfusepath{fill}%
\end{pgfscope}%
\begin{pgfscope}%
\pgfpathrectangle{\pgfqpoint{0.651412in}{0.524170in}}{\pgfqpoint{4.629690in}{2.558193in}}%
\pgfusepath{clip}%
\pgfsetbuttcap%
\pgfsetmiterjoin%
\definecolor{currentfill}{rgb}{0.835294,0.368627,0.000000}%
\pgfsetfillcolor{currentfill}%
\pgfsetfillopacity{0.700000}%
\pgfsetlinewidth{0.000000pt}%
\definecolor{currentstroke}{rgb}{0.000000,0.000000,0.000000}%
\pgfsetstrokecolor{currentstroke}%
\pgfsetstrokeopacity{0.700000}%
\pgfsetdash{}{0pt}%
\pgfpathmoveto{\pgfqpoint{4.313076in}{0.524170in}}%
\pgfpathlineto{\pgfqpoint{4.355164in}{0.524170in}}%
\pgfpathlineto{\pgfqpoint{4.355164in}{0.545604in}}%
\pgfpathlineto{\pgfqpoint{4.313076in}{0.545604in}}%
\pgfpathlineto{\pgfqpoint{4.313076in}{0.524170in}}%
\pgfpathclose%
\pgfusepath{fill}%
\end{pgfscope}%
\begin{pgfscope}%
\pgfpathrectangle{\pgfqpoint{0.651412in}{0.524170in}}{\pgfqpoint{4.629690in}{2.558193in}}%
\pgfusepath{clip}%
\pgfsetbuttcap%
\pgfsetmiterjoin%
\definecolor{currentfill}{rgb}{0.835294,0.368627,0.000000}%
\pgfsetfillcolor{currentfill}%
\pgfsetfillopacity{0.700000}%
\pgfsetlinewidth{0.000000pt}%
\definecolor{currentstroke}{rgb}{0.000000,0.000000,0.000000}%
\pgfsetstrokecolor{currentstroke}%
\pgfsetstrokeopacity{0.700000}%
\pgfsetdash{}{0pt}%
\pgfpathmoveto{\pgfqpoint{4.355164in}{0.524170in}}%
\pgfpathlineto{\pgfqpoint{4.397253in}{0.524170in}}%
\pgfpathlineto{\pgfqpoint{4.397253in}{0.556321in}}%
\pgfpathlineto{\pgfqpoint{4.355164in}{0.556321in}}%
\pgfpathlineto{\pgfqpoint{4.355164in}{0.524170in}}%
\pgfpathclose%
\pgfusepath{fill}%
\end{pgfscope}%
\begin{pgfscope}%
\pgfpathrectangle{\pgfqpoint{0.651412in}{0.524170in}}{\pgfqpoint{4.629690in}{2.558193in}}%
\pgfusepath{clip}%
\pgfsetbuttcap%
\pgfsetmiterjoin%
\definecolor{currentfill}{rgb}{0.835294,0.368627,0.000000}%
\pgfsetfillcolor{currentfill}%
\pgfsetfillopacity{0.700000}%
\pgfsetlinewidth{0.000000pt}%
\definecolor{currentstroke}{rgb}{0.000000,0.000000,0.000000}%
\pgfsetstrokecolor{currentstroke}%
\pgfsetstrokeopacity{0.700000}%
\pgfsetdash{}{0pt}%
\pgfpathmoveto{\pgfqpoint{4.397253in}{0.524170in}}%
\pgfpathlineto{\pgfqpoint{4.439341in}{0.524170in}}%
\pgfpathlineto{\pgfqpoint{4.439341in}{0.556321in}}%
\pgfpathlineto{\pgfqpoint{4.397253in}{0.556321in}}%
\pgfpathlineto{\pgfqpoint{4.397253in}{0.524170in}}%
\pgfpathclose%
\pgfusepath{fill}%
\end{pgfscope}%
\begin{pgfscope}%
\pgfpathrectangle{\pgfqpoint{0.651412in}{0.524170in}}{\pgfqpoint{4.629690in}{2.558193in}}%
\pgfusepath{clip}%
\pgfsetbuttcap%
\pgfsetmiterjoin%
\definecolor{currentfill}{rgb}{0.835294,0.368627,0.000000}%
\pgfsetfillcolor{currentfill}%
\pgfsetfillopacity{0.700000}%
\pgfsetlinewidth{0.000000pt}%
\definecolor{currentstroke}{rgb}{0.000000,0.000000,0.000000}%
\pgfsetstrokecolor{currentstroke}%
\pgfsetstrokeopacity{0.700000}%
\pgfsetdash{}{0pt}%
\pgfpathmoveto{\pgfqpoint{4.439341in}{0.524170in}}%
\pgfpathlineto{\pgfqpoint{4.481429in}{0.524170in}}%
\pgfpathlineto{\pgfqpoint{4.481429in}{0.556321in}}%
\pgfpathlineto{\pgfqpoint{4.439341in}{0.556321in}}%
\pgfpathlineto{\pgfqpoint{4.439341in}{0.524170in}}%
\pgfpathclose%
\pgfusepath{fill}%
\end{pgfscope}%
\begin{pgfscope}%
\pgfpathrectangle{\pgfqpoint{0.651412in}{0.524170in}}{\pgfqpoint{4.629690in}{2.558193in}}%
\pgfusepath{clip}%
\pgfsetbuttcap%
\pgfsetmiterjoin%
\definecolor{currentfill}{rgb}{0.835294,0.368627,0.000000}%
\pgfsetfillcolor{currentfill}%
\pgfsetfillopacity{0.700000}%
\pgfsetlinewidth{0.000000pt}%
\definecolor{currentstroke}{rgb}{0.000000,0.000000,0.000000}%
\pgfsetstrokecolor{currentstroke}%
\pgfsetstrokeopacity{0.700000}%
\pgfsetdash{}{0pt}%
\pgfpathmoveto{\pgfqpoint{4.481429in}{0.524170in}}%
\pgfpathlineto{\pgfqpoint{4.523517in}{0.524170in}}%
\pgfpathlineto{\pgfqpoint{4.523517in}{0.583115in}}%
\pgfpathlineto{\pgfqpoint{4.481429in}{0.583115in}}%
\pgfpathlineto{\pgfqpoint{4.481429in}{0.524170in}}%
\pgfpathclose%
\pgfusepath{fill}%
\end{pgfscope}%
\begin{pgfscope}%
\pgfpathrectangle{\pgfqpoint{0.651412in}{0.524170in}}{\pgfqpoint{4.629690in}{2.558193in}}%
\pgfusepath{clip}%
\pgfsetbuttcap%
\pgfsetmiterjoin%
\definecolor{currentfill}{rgb}{0.835294,0.368627,0.000000}%
\pgfsetfillcolor{currentfill}%
\pgfsetfillopacity{0.700000}%
\pgfsetlinewidth{0.000000pt}%
\definecolor{currentstroke}{rgb}{0.000000,0.000000,0.000000}%
\pgfsetstrokecolor{currentstroke}%
\pgfsetstrokeopacity{0.700000}%
\pgfsetdash{}{0pt}%
\pgfpathmoveto{\pgfqpoint{4.523517in}{0.524170in}}%
\pgfpathlineto{\pgfqpoint{4.565605in}{0.524170in}}%
\pgfpathlineto{\pgfqpoint{4.565605in}{0.556321in}}%
\pgfpathlineto{\pgfqpoint{4.523517in}{0.556321in}}%
\pgfpathlineto{\pgfqpoint{4.523517in}{0.524170in}}%
\pgfpathclose%
\pgfusepath{fill}%
\end{pgfscope}%
\begin{pgfscope}%
\pgfpathrectangle{\pgfqpoint{0.651412in}{0.524170in}}{\pgfqpoint{4.629690in}{2.558193in}}%
\pgfusepath{clip}%
\pgfsetbuttcap%
\pgfsetmiterjoin%
\definecolor{currentfill}{rgb}{0.835294,0.368627,0.000000}%
\pgfsetfillcolor{currentfill}%
\pgfsetfillopacity{0.700000}%
\pgfsetlinewidth{0.000000pt}%
\definecolor{currentstroke}{rgb}{0.000000,0.000000,0.000000}%
\pgfsetstrokecolor{currentstroke}%
\pgfsetstrokeopacity{0.700000}%
\pgfsetdash{}{0pt}%
\pgfpathmoveto{\pgfqpoint{4.565605in}{0.524170in}}%
\pgfpathlineto{\pgfqpoint{4.607693in}{0.524170in}}%
\pgfpathlineto{\pgfqpoint{4.607693in}{0.556321in}}%
\pgfpathlineto{\pgfqpoint{4.565605in}{0.556321in}}%
\pgfpathlineto{\pgfqpoint{4.565605in}{0.524170in}}%
\pgfpathclose%
\pgfusepath{fill}%
\end{pgfscope}%
\begin{pgfscope}%
\pgfpathrectangle{\pgfqpoint{0.651412in}{0.524170in}}{\pgfqpoint{4.629690in}{2.558193in}}%
\pgfusepath{clip}%
\pgfsetbuttcap%
\pgfsetmiterjoin%
\definecolor{currentfill}{rgb}{0.835294,0.368627,0.000000}%
\pgfsetfillcolor{currentfill}%
\pgfsetfillopacity{0.700000}%
\pgfsetlinewidth{0.000000pt}%
\definecolor{currentstroke}{rgb}{0.000000,0.000000,0.000000}%
\pgfsetstrokecolor{currentstroke}%
\pgfsetstrokeopacity{0.700000}%
\pgfsetdash{}{0pt}%
\pgfpathmoveto{\pgfqpoint{4.607693in}{0.524170in}}%
\pgfpathlineto{\pgfqpoint{4.649781in}{0.524170in}}%
\pgfpathlineto{\pgfqpoint{4.649781in}{0.540246in}}%
\pgfpathlineto{\pgfqpoint{4.607693in}{0.540246in}}%
\pgfpathlineto{\pgfqpoint{4.607693in}{0.524170in}}%
\pgfpathclose%
\pgfusepath{fill}%
\end{pgfscope}%
\begin{pgfscope}%
\pgfpathrectangle{\pgfqpoint{0.651412in}{0.524170in}}{\pgfqpoint{4.629690in}{2.558193in}}%
\pgfusepath{clip}%
\pgfsetbuttcap%
\pgfsetmiterjoin%
\definecolor{currentfill}{rgb}{0.835294,0.368627,0.000000}%
\pgfsetfillcolor{currentfill}%
\pgfsetfillopacity{0.700000}%
\pgfsetlinewidth{0.000000pt}%
\definecolor{currentstroke}{rgb}{0.000000,0.000000,0.000000}%
\pgfsetstrokecolor{currentstroke}%
\pgfsetstrokeopacity{0.700000}%
\pgfsetdash{}{0pt}%
\pgfpathmoveto{\pgfqpoint{4.649781in}{0.524170in}}%
\pgfpathlineto{\pgfqpoint{4.691869in}{0.524170in}}%
\pgfpathlineto{\pgfqpoint{4.691869in}{0.545604in}}%
\pgfpathlineto{\pgfqpoint{4.649781in}{0.545604in}}%
\pgfpathlineto{\pgfqpoint{4.649781in}{0.524170in}}%
\pgfpathclose%
\pgfusepath{fill}%
\end{pgfscope}%
\begin{pgfscope}%
\pgfpathrectangle{\pgfqpoint{0.651412in}{0.524170in}}{\pgfqpoint{4.629690in}{2.558193in}}%
\pgfusepath{clip}%
\pgfsetbuttcap%
\pgfsetmiterjoin%
\definecolor{currentfill}{rgb}{0.835294,0.368627,0.000000}%
\pgfsetfillcolor{currentfill}%
\pgfsetfillopacity{0.700000}%
\pgfsetlinewidth{0.000000pt}%
\definecolor{currentstroke}{rgb}{0.000000,0.000000,0.000000}%
\pgfsetstrokecolor{currentstroke}%
\pgfsetstrokeopacity{0.700000}%
\pgfsetdash{}{0pt}%
\pgfpathmoveto{\pgfqpoint{4.691869in}{0.524170in}}%
\pgfpathlineto{\pgfqpoint{4.733957in}{0.524170in}}%
\pgfpathlineto{\pgfqpoint{4.733957in}{0.556321in}}%
\pgfpathlineto{\pgfqpoint{4.691869in}{0.556321in}}%
\pgfpathlineto{\pgfqpoint{4.691869in}{0.524170in}}%
\pgfpathclose%
\pgfusepath{fill}%
\end{pgfscope}%
\begin{pgfscope}%
\pgfpathrectangle{\pgfqpoint{0.651412in}{0.524170in}}{\pgfqpoint{4.629690in}{2.558193in}}%
\pgfusepath{clip}%
\pgfsetbuttcap%
\pgfsetmiterjoin%
\definecolor{currentfill}{rgb}{0.835294,0.368627,0.000000}%
\pgfsetfillcolor{currentfill}%
\pgfsetfillopacity{0.700000}%
\pgfsetlinewidth{0.000000pt}%
\definecolor{currentstroke}{rgb}{0.000000,0.000000,0.000000}%
\pgfsetstrokecolor{currentstroke}%
\pgfsetstrokeopacity{0.700000}%
\pgfsetdash{}{0pt}%
\pgfpathmoveto{\pgfqpoint{4.733957in}{0.524170in}}%
\pgfpathlineto{\pgfqpoint{4.776045in}{0.524170in}}%
\pgfpathlineto{\pgfqpoint{4.776045in}{0.540246in}}%
\pgfpathlineto{\pgfqpoint{4.733957in}{0.540246in}}%
\pgfpathlineto{\pgfqpoint{4.733957in}{0.524170in}}%
\pgfpathclose%
\pgfusepath{fill}%
\end{pgfscope}%
\begin{pgfscope}%
\pgfpathrectangle{\pgfqpoint{0.651412in}{0.524170in}}{\pgfqpoint{4.629690in}{2.558193in}}%
\pgfusepath{clip}%
\pgfsetbuttcap%
\pgfsetmiterjoin%
\definecolor{currentfill}{rgb}{0.835294,0.368627,0.000000}%
\pgfsetfillcolor{currentfill}%
\pgfsetfillopacity{0.700000}%
\pgfsetlinewidth{0.000000pt}%
\definecolor{currentstroke}{rgb}{0.000000,0.000000,0.000000}%
\pgfsetstrokecolor{currentstroke}%
\pgfsetstrokeopacity{0.700000}%
\pgfsetdash{}{0pt}%
\pgfpathmoveto{\pgfqpoint{4.776045in}{0.524170in}}%
\pgfpathlineto{\pgfqpoint{4.818133in}{0.524170in}}%
\pgfpathlineto{\pgfqpoint{4.818133in}{0.545604in}}%
\pgfpathlineto{\pgfqpoint{4.776045in}{0.545604in}}%
\pgfpathlineto{\pgfqpoint{4.776045in}{0.524170in}}%
\pgfpathclose%
\pgfusepath{fill}%
\end{pgfscope}%
\begin{pgfscope}%
\pgfpathrectangle{\pgfqpoint{0.651412in}{0.524170in}}{\pgfqpoint{4.629690in}{2.558193in}}%
\pgfusepath{clip}%
\pgfsetbuttcap%
\pgfsetmiterjoin%
\definecolor{currentfill}{rgb}{0.835294,0.368627,0.000000}%
\pgfsetfillcolor{currentfill}%
\pgfsetfillopacity{0.700000}%
\pgfsetlinewidth{0.000000pt}%
\definecolor{currentstroke}{rgb}{0.000000,0.000000,0.000000}%
\pgfsetstrokecolor{currentstroke}%
\pgfsetstrokeopacity{0.700000}%
\pgfsetdash{}{0pt}%
\pgfpathmoveto{\pgfqpoint{4.818133in}{0.524170in}}%
\pgfpathlineto{\pgfqpoint{4.860222in}{0.524170in}}%
\pgfpathlineto{\pgfqpoint{4.860222in}{0.540246in}}%
\pgfpathlineto{\pgfqpoint{4.818133in}{0.540246in}}%
\pgfpathlineto{\pgfqpoint{4.818133in}{0.524170in}}%
\pgfpathclose%
\pgfusepath{fill}%
\end{pgfscope}%
\begin{pgfscope}%
\pgfpathrectangle{\pgfqpoint{0.651412in}{0.524170in}}{\pgfqpoint{4.629690in}{2.558193in}}%
\pgfusepath{clip}%
\pgfsetbuttcap%
\pgfsetmiterjoin%
\definecolor{currentfill}{rgb}{0.835294,0.368627,0.000000}%
\pgfsetfillcolor{currentfill}%
\pgfsetfillopacity{0.700000}%
\pgfsetlinewidth{0.000000pt}%
\definecolor{currentstroke}{rgb}{0.000000,0.000000,0.000000}%
\pgfsetstrokecolor{currentstroke}%
\pgfsetstrokeopacity{0.700000}%
\pgfsetdash{}{0pt}%
\pgfpathmoveto{\pgfqpoint{4.860222in}{0.524170in}}%
\pgfpathlineto{\pgfqpoint{4.902310in}{0.524170in}}%
\pgfpathlineto{\pgfqpoint{4.902310in}{0.550963in}}%
\pgfpathlineto{\pgfqpoint{4.860222in}{0.550963in}}%
\pgfpathlineto{\pgfqpoint{4.860222in}{0.524170in}}%
\pgfpathclose%
\pgfusepath{fill}%
\end{pgfscope}%
\begin{pgfscope}%
\pgfpathrectangle{\pgfqpoint{0.651412in}{0.524170in}}{\pgfqpoint{4.629690in}{2.558193in}}%
\pgfusepath{clip}%
\pgfsetbuttcap%
\pgfsetmiterjoin%
\definecolor{currentfill}{rgb}{0.835294,0.368627,0.000000}%
\pgfsetfillcolor{currentfill}%
\pgfsetfillopacity{0.700000}%
\pgfsetlinewidth{0.000000pt}%
\definecolor{currentstroke}{rgb}{0.000000,0.000000,0.000000}%
\pgfsetstrokecolor{currentstroke}%
\pgfsetstrokeopacity{0.700000}%
\pgfsetdash{}{0pt}%
\pgfpathmoveto{\pgfqpoint{4.902310in}{0.524170in}}%
\pgfpathlineto{\pgfqpoint{4.944398in}{0.524170in}}%
\pgfpathlineto{\pgfqpoint{4.944398in}{0.561680in}}%
\pgfpathlineto{\pgfqpoint{4.902310in}{0.561680in}}%
\pgfpathlineto{\pgfqpoint{4.902310in}{0.524170in}}%
\pgfpathclose%
\pgfusepath{fill}%
\end{pgfscope}%
\begin{pgfscope}%
\pgfpathrectangle{\pgfqpoint{0.651412in}{0.524170in}}{\pgfqpoint{4.629690in}{2.558193in}}%
\pgfusepath{clip}%
\pgfsetbuttcap%
\pgfsetmiterjoin%
\definecolor{currentfill}{rgb}{0.835294,0.368627,0.000000}%
\pgfsetfillcolor{currentfill}%
\pgfsetfillopacity{0.700000}%
\pgfsetlinewidth{0.000000pt}%
\definecolor{currentstroke}{rgb}{0.000000,0.000000,0.000000}%
\pgfsetstrokecolor{currentstroke}%
\pgfsetstrokeopacity{0.700000}%
\pgfsetdash{}{0pt}%
\pgfpathmoveto{\pgfqpoint{4.944398in}{0.524170in}}%
\pgfpathlineto{\pgfqpoint{4.986486in}{0.524170in}}%
\pgfpathlineto{\pgfqpoint{4.986486in}{0.561680in}}%
\pgfpathlineto{\pgfqpoint{4.944398in}{0.561680in}}%
\pgfpathlineto{\pgfqpoint{4.944398in}{0.524170in}}%
\pgfpathclose%
\pgfusepath{fill}%
\end{pgfscope}%
\begin{pgfscope}%
\pgfpathrectangle{\pgfqpoint{0.651412in}{0.524170in}}{\pgfqpoint{4.629690in}{2.558193in}}%
\pgfusepath{clip}%
\pgfsetbuttcap%
\pgfsetmiterjoin%
\definecolor{currentfill}{rgb}{0.835294,0.368627,0.000000}%
\pgfsetfillcolor{currentfill}%
\pgfsetfillopacity{0.700000}%
\pgfsetlinewidth{0.000000pt}%
\definecolor{currentstroke}{rgb}{0.000000,0.000000,0.000000}%
\pgfsetstrokecolor{currentstroke}%
\pgfsetstrokeopacity{0.700000}%
\pgfsetdash{}{0pt}%
\pgfpathmoveto{\pgfqpoint{4.986486in}{0.524170in}}%
\pgfpathlineto{\pgfqpoint{5.028574in}{0.524170in}}%
\pgfpathlineto{\pgfqpoint{5.028574in}{0.567039in}}%
\pgfpathlineto{\pgfqpoint{4.986486in}{0.567039in}}%
\pgfpathlineto{\pgfqpoint{4.986486in}{0.524170in}}%
\pgfpathclose%
\pgfusepath{fill}%
\end{pgfscope}%
\begin{pgfscope}%
\pgfpathrectangle{\pgfqpoint{0.651412in}{0.524170in}}{\pgfqpoint{4.629690in}{2.558193in}}%
\pgfusepath{clip}%
\pgfsetbuttcap%
\pgfsetmiterjoin%
\definecolor{currentfill}{rgb}{0.835294,0.368627,0.000000}%
\pgfsetfillcolor{currentfill}%
\pgfsetfillopacity{0.700000}%
\pgfsetlinewidth{0.000000pt}%
\definecolor{currentstroke}{rgb}{0.000000,0.000000,0.000000}%
\pgfsetstrokecolor{currentstroke}%
\pgfsetstrokeopacity{0.700000}%
\pgfsetdash{}{0pt}%
\pgfpathmoveto{\pgfqpoint{5.028574in}{0.524170in}}%
\pgfpathlineto{\pgfqpoint{5.070662in}{0.524170in}}%
\pgfpathlineto{\pgfqpoint{5.070662in}{0.550963in}}%
\pgfpathlineto{\pgfqpoint{5.028574in}{0.550963in}}%
\pgfpathlineto{\pgfqpoint{5.028574in}{0.524170in}}%
\pgfpathclose%
\pgfusepath{fill}%
\end{pgfscope}%
\begin{pgfscope}%
\pgfpathrectangle{\pgfqpoint{0.651412in}{0.524170in}}{\pgfqpoint{4.629690in}{2.558193in}}%
\pgfusepath{clip}%
\pgfsetbuttcap%
\pgfsetmiterjoin%
\definecolor{currentfill}{rgb}{0.007843,0.619608,0.450980}%
\pgfsetfillcolor{currentfill}%
\pgfsetfillopacity{0.700000}%
\pgfsetlinewidth{0.000000pt}%
\definecolor{currentstroke}{rgb}{0.000000,0.000000,0.000000}%
\pgfsetstrokecolor{currentstroke}%
\pgfsetstrokeopacity{0.700000}%
\pgfsetdash{}{0pt}%
\pgfpathmoveto{\pgfqpoint{0.861853in}{0.524170in}}%
\pgfpathlineto{\pgfqpoint{0.903941in}{0.524170in}}%
\pgfpathlineto{\pgfqpoint{0.903941in}{0.524170in}}%
\pgfpathlineto{\pgfqpoint{0.861853in}{0.524170in}}%
\pgfpathlineto{\pgfqpoint{0.861853in}{0.524170in}}%
\pgfpathclose%
\pgfusepath{fill}%
\end{pgfscope}%
\begin{pgfscope}%
\pgfpathrectangle{\pgfqpoint{0.651412in}{0.524170in}}{\pgfqpoint{4.629690in}{2.558193in}}%
\pgfusepath{clip}%
\pgfsetbuttcap%
\pgfsetmiterjoin%
\definecolor{currentfill}{rgb}{0.007843,0.619608,0.450980}%
\pgfsetfillcolor{currentfill}%
\pgfsetfillopacity{0.700000}%
\pgfsetlinewidth{0.000000pt}%
\definecolor{currentstroke}{rgb}{0.000000,0.000000,0.000000}%
\pgfsetstrokecolor{currentstroke}%
\pgfsetstrokeopacity{0.700000}%
\pgfsetdash{}{0pt}%
\pgfpathmoveto{\pgfqpoint{0.903941in}{0.524170in}}%
\pgfpathlineto{\pgfqpoint{0.946029in}{0.524170in}}%
\pgfpathlineto{\pgfqpoint{0.946029in}{0.524170in}}%
\pgfpathlineto{\pgfqpoint{0.903941in}{0.524170in}}%
\pgfpathlineto{\pgfqpoint{0.903941in}{0.524170in}}%
\pgfpathclose%
\pgfusepath{fill}%
\end{pgfscope}%
\begin{pgfscope}%
\pgfpathrectangle{\pgfqpoint{0.651412in}{0.524170in}}{\pgfqpoint{4.629690in}{2.558193in}}%
\pgfusepath{clip}%
\pgfsetbuttcap%
\pgfsetmiterjoin%
\definecolor{currentfill}{rgb}{0.007843,0.619608,0.450980}%
\pgfsetfillcolor{currentfill}%
\pgfsetfillopacity{0.700000}%
\pgfsetlinewidth{0.000000pt}%
\definecolor{currentstroke}{rgb}{0.000000,0.000000,0.000000}%
\pgfsetstrokecolor{currentstroke}%
\pgfsetstrokeopacity{0.700000}%
\pgfsetdash{}{0pt}%
\pgfpathmoveto{\pgfqpoint{0.946029in}{0.524170in}}%
\pgfpathlineto{\pgfqpoint{0.988117in}{0.524170in}}%
\pgfpathlineto{\pgfqpoint{0.988117in}{0.524170in}}%
\pgfpathlineto{\pgfqpoint{0.946029in}{0.524170in}}%
\pgfpathlineto{\pgfqpoint{0.946029in}{0.524170in}}%
\pgfpathclose%
\pgfusepath{fill}%
\end{pgfscope}%
\begin{pgfscope}%
\pgfpathrectangle{\pgfqpoint{0.651412in}{0.524170in}}{\pgfqpoint{4.629690in}{2.558193in}}%
\pgfusepath{clip}%
\pgfsetbuttcap%
\pgfsetmiterjoin%
\definecolor{currentfill}{rgb}{0.007843,0.619608,0.450980}%
\pgfsetfillcolor{currentfill}%
\pgfsetfillopacity{0.700000}%
\pgfsetlinewidth{0.000000pt}%
\definecolor{currentstroke}{rgb}{0.000000,0.000000,0.000000}%
\pgfsetstrokecolor{currentstroke}%
\pgfsetstrokeopacity{0.700000}%
\pgfsetdash{}{0pt}%
\pgfpathmoveto{\pgfqpoint{0.988117in}{0.524170in}}%
\pgfpathlineto{\pgfqpoint{1.030205in}{0.524170in}}%
\pgfpathlineto{\pgfqpoint{1.030205in}{0.524170in}}%
\pgfpathlineto{\pgfqpoint{0.988117in}{0.524170in}}%
\pgfpathlineto{\pgfqpoint{0.988117in}{0.524170in}}%
\pgfpathclose%
\pgfusepath{fill}%
\end{pgfscope}%
\begin{pgfscope}%
\pgfpathrectangle{\pgfqpoint{0.651412in}{0.524170in}}{\pgfqpoint{4.629690in}{2.558193in}}%
\pgfusepath{clip}%
\pgfsetbuttcap%
\pgfsetmiterjoin%
\definecolor{currentfill}{rgb}{0.007843,0.619608,0.450980}%
\pgfsetfillcolor{currentfill}%
\pgfsetfillopacity{0.700000}%
\pgfsetlinewidth{0.000000pt}%
\definecolor{currentstroke}{rgb}{0.000000,0.000000,0.000000}%
\pgfsetstrokecolor{currentstroke}%
\pgfsetstrokeopacity{0.700000}%
\pgfsetdash{}{0pt}%
\pgfpathmoveto{\pgfqpoint{1.030205in}{0.524170in}}%
\pgfpathlineto{\pgfqpoint{1.072293in}{0.524170in}}%
\pgfpathlineto{\pgfqpoint{1.072293in}{0.524170in}}%
\pgfpathlineto{\pgfqpoint{1.030205in}{0.524170in}}%
\pgfpathlineto{\pgfqpoint{1.030205in}{0.524170in}}%
\pgfpathclose%
\pgfusepath{fill}%
\end{pgfscope}%
\begin{pgfscope}%
\pgfpathrectangle{\pgfqpoint{0.651412in}{0.524170in}}{\pgfqpoint{4.629690in}{2.558193in}}%
\pgfusepath{clip}%
\pgfsetbuttcap%
\pgfsetmiterjoin%
\definecolor{currentfill}{rgb}{0.007843,0.619608,0.450980}%
\pgfsetfillcolor{currentfill}%
\pgfsetfillopacity{0.700000}%
\pgfsetlinewidth{0.000000pt}%
\definecolor{currentstroke}{rgb}{0.000000,0.000000,0.000000}%
\pgfsetstrokecolor{currentstroke}%
\pgfsetstrokeopacity{0.700000}%
\pgfsetdash{}{0pt}%
\pgfpathmoveto{\pgfqpoint{1.072293in}{0.524170in}}%
\pgfpathlineto{\pgfqpoint{1.114381in}{0.524170in}}%
\pgfpathlineto{\pgfqpoint{1.114381in}{0.524170in}}%
\pgfpathlineto{\pgfqpoint{1.072293in}{0.524170in}}%
\pgfpathlineto{\pgfqpoint{1.072293in}{0.524170in}}%
\pgfpathclose%
\pgfusepath{fill}%
\end{pgfscope}%
\begin{pgfscope}%
\pgfpathrectangle{\pgfqpoint{0.651412in}{0.524170in}}{\pgfqpoint{4.629690in}{2.558193in}}%
\pgfusepath{clip}%
\pgfsetbuttcap%
\pgfsetmiterjoin%
\definecolor{currentfill}{rgb}{0.007843,0.619608,0.450980}%
\pgfsetfillcolor{currentfill}%
\pgfsetfillopacity{0.700000}%
\pgfsetlinewidth{0.000000pt}%
\definecolor{currentstroke}{rgb}{0.000000,0.000000,0.000000}%
\pgfsetstrokecolor{currentstroke}%
\pgfsetstrokeopacity{0.700000}%
\pgfsetdash{}{0pt}%
\pgfpathmoveto{\pgfqpoint{1.114381in}{0.524170in}}%
\pgfpathlineto{\pgfqpoint{1.156469in}{0.524170in}}%
\pgfpathlineto{\pgfqpoint{1.156469in}{0.524170in}}%
\pgfpathlineto{\pgfqpoint{1.114381in}{0.524170in}}%
\pgfpathlineto{\pgfqpoint{1.114381in}{0.524170in}}%
\pgfpathclose%
\pgfusepath{fill}%
\end{pgfscope}%
\begin{pgfscope}%
\pgfpathrectangle{\pgfqpoint{0.651412in}{0.524170in}}{\pgfqpoint{4.629690in}{2.558193in}}%
\pgfusepath{clip}%
\pgfsetbuttcap%
\pgfsetmiterjoin%
\definecolor{currentfill}{rgb}{0.007843,0.619608,0.450980}%
\pgfsetfillcolor{currentfill}%
\pgfsetfillopacity{0.700000}%
\pgfsetlinewidth{0.000000pt}%
\definecolor{currentstroke}{rgb}{0.000000,0.000000,0.000000}%
\pgfsetstrokecolor{currentstroke}%
\pgfsetstrokeopacity{0.700000}%
\pgfsetdash{}{0pt}%
\pgfpathmoveto{\pgfqpoint{1.156469in}{0.524170in}}%
\pgfpathlineto{\pgfqpoint{1.198557in}{0.524170in}}%
\pgfpathlineto{\pgfqpoint{1.198557in}{0.524170in}}%
\pgfpathlineto{\pgfqpoint{1.156469in}{0.524170in}}%
\pgfpathlineto{\pgfqpoint{1.156469in}{0.524170in}}%
\pgfpathclose%
\pgfusepath{fill}%
\end{pgfscope}%
\begin{pgfscope}%
\pgfpathrectangle{\pgfqpoint{0.651412in}{0.524170in}}{\pgfqpoint{4.629690in}{2.558193in}}%
\pgfusepath{clip}%
\pgfsetbuttcap%
\pgfsetmiterjoin%
\definecolor{currentfill}{rgb}{0.007843,0.619608,0.450980}%
\pgfsetfillcolor{currentfill}%
\pgfsetfillopacity{0.700000}%
\pgfsetlinewidth{0.000000pt}%
\definecolor{currentstroke}{rgb}{0.000000,0.000000,0.000000}%
\pgfsetstrokecolor{currentstroke}%
\pgfsetstrokeopacity{0.700000}%
\pgfsetdash{}{0pt}%
\pgfpathmoveto{\pgfqpoint{1.198557in}{0.524170in}}%
\pgfpathlineto{\pgfqpoint{1.240645in}{0.524170in}}%
\pgfpathlineto{\pgfqpoint{1.240645in}{0.524170in}}%
\pgfpathlineto{\pgfqpoint{1.198557in}{0.524170in}}%
\pgfpathlineto{\pgfqpoint{1.198557in}{0.524170in}}%
\pgfpathclose%
\pgfusepath{fill}%
\end{pgfscope}%
\begin{pgfscope}%
\pgfpathrectangle{\pgfqpoint{0.651412in}{0.524170in}}{\pgfqpoint{4.629690in}{2.558193in}}%
\pgfusepath{clip}%
\pgfsetbuttcap%
\pgfsetmiterjoin%
\definecolor{currentfill}{rgb}{0.007843,0.619608,0.450980}%
\pgfsetfillcolor{currentfill}%
\pgfsetfillopacity{0.700000}%
\pgfsetlinewidth{0.000000pt}%
\definecolor{currentstroke}{rgb}{0.000000,0.000000,0.000000}%
\pgfsetstrokecolor{currentstroke}%
\pgfsetstrokeopacity{0.700000}%
\pgfsetdash{}{0pt}%
\pgfpathmoveto{\pgfqpoint{1.240645in}{0.524170in}}%
\pgfpathlineto{\pgfqpoint{1.282734in}{0.524170in}}%
\pgfpathlineto{\pgfqpoint{1.282734in}{0.524170in}}%
\pgfpathlineto{\pgfqpoint{1.240645in}{0.524170in}}%
\pgfpathlineto{\pgfqpoint{1.240645in}{0.524170in}}%
\pgfpathclose%
\pgfusepath{fill}%
\end{pgfscope}%
\begin{pgfscope}%
\pgfpathrectangle{\pgfqpoint{0.651412in}{0.524170in}}{\pgfqpoint{4.629690in}{2.558193in}}%
\pgfusepath{clip}%
\pgfsetbuttcap%
\pgfsetmiterjoin%
\definecolor{currentfill}{rgb}{0.007843,0.619608,0.450980}%
\pgfsetfillcolor{currentfill}%
\pgfsetfillopacity{0.700000}%
\pgfsetlinewidth{0.000000pt}%
\definecolor{currentstroke}{rgb}{0.000000,0.000000,0.000000}%
\pgfsetstrokecolor{currentstroke}%
\pgfsetstrokeopacity{0.700000}%
\pgfsetdash{}{0pt}%
\pgfpathmoveto{\pgfqpoint{1.282734in}{0.524170in}}%
\pgfpathlineto{\pgfqpoint{1.324822in}{0.524170in}}%
\pgfpathlineto{\pgfqpoint{1.324822in}{0.524170in}}%
\pgfpathlineto{\pgfqpoint{1.282734in}{0.524170in}}%
\pgfpathlineto{\pgfqpoint{1.282734in}{0.524170in}}%
\pgfpathclose%
\pgfusepath{fill}%
\end{pgfscope}%
\begin{pgfscope}%
\pgfpathrectangle{\pgfqpoint{0.651412in}{0.524170in}}{\pgfqpoint{4.629690in}{2.558193in}}%
\pgfusepath{clip}%
\pgfsetbuttcap%
\pgfsetmiterjoin%
\definecolor{currentfill}{rgb}{0.007843,0.619608,0.450980}%
\pgfsetfillcolor{currentfill}%
\pgfsetfillopacity{0.700000}%
\pgfsetlinewidth{0.000000pt}%
\definecolor{currentstroke}{rgb}{0.000000,0.000000,0.000000}%
\pgfsetstrokecolor{currentstroke}%
\pgfsetstrokeopacity{0.700000}%
\pgfsetdash{}{0pt}%
\pgfpathmoveto{\pgfqpoint{1.324822in}{0.524170in}}%
\pgfpathlineto{\pgfqpoint{1.366910in}{0.524170in}}%
\pgfpathlineto{\pgfqpoint{1.366910in}{0.524170in}}%
\pgfpathlineto{\pgfqpoint{1.324822in}{0.524170in}}%
\pgfpathlineto{\pgfqpoint{1.324822in}{0.524170in}}%
\pgfpathclose%
\pgfusepath{fill}%
\end{pgfscope}%
\begin{pgfscope}%
\pgfpathrectangle{\pgfqpoint{0.651412in}{0.524170in}}{\pgfqpoint{4.629690in}{2.558193in}}%
\pgfusepath{clip}%
\pgfsetbuttcap%
\pgfsetmiterjoin%
\definecolor{currentfill}{rgb}{0.007843,0.619608,0.450980}%
\pgfsetfillcolor{currentfill}%
\pgfsetfillopacity{0.700000}%
\pgfsetlinewidth{0.000000pt}%
\definecolor{currentstroke}{rgb}{0.000000,0.000000,0.000000}%
\pgfsetstrokecolor{currentstroke}%
\pgfsetstrokeopacity{0.700000}%
\pgfsetdash{}{0pt}%
\pgfpathmoveto{\pgfqpoint{1.366910in}{0.524170in}}%
\pgfpathlineto{\pgfqpoint{1.408998in}{0.524170in}}%
\pgfpathlineto{\pgfqpoint{1.408998in}{0.524170in}}%
\pgfpathlineto{\pgfqpoint{1.366910in}{0.524170in}}%
\pgfpathlineto{\pgfqpoint{1.366910in}{0.524170in}}%
\pgfpathclose%
\pgfusepath{fill}%
\end{pgfscope}%
\begin{pgfscope}%
\pgfpathrectangle{\pgfqpoint{0.651412in}{0.524170in}}{\pgfqpoint{4.629690in}{2.558193in}}%
\pgfusepath{clip}%
\pgfsetbuttcap%
\pgfsetmiterjoin%
\definecolor{currentfill}{rgb}{0.007843,0.619608,0.450980}%
\pgfsetfillcolor{currentfill}%
\pgfsetfillopacity{0.700000}%
\pgfsetlinewidth{0.000000pt}%
\definecolor{currentstroke}{rgb}{0.000000,0.000000,0.000000}%
\pgfsetstrokecolor{currentstroke}%
\pgfsetstrokeopacity{0.700000}%
\pgfsetdash{}{0pt}%
\pgfpathmoveto{\pgfqpoint{1.408998in}{0.524170in}}%
\pgfpathlineto{\pgfqpoint{1.451086in}{0.524170in}}%
\pgfpathlineto{\pgfqpoint{1.451086in}{0.524170in}}%
\pgfpathlineto{\pgfqpoint{1.408998in}{0.524170in}}%
\pgfpathlineto{\pgfqpoint{1.408998in}{0.524170in}}%
\pgfpathclose%
\pgfusepath{fill}%
\end{pgfscope}%
\begin{pgfscope}%
\pgfpathrectangle{\pgfqpoint{0.651412in}{0.524170in}}{\pgfqpoint{4.629690in}{2.558193in}}%
\pgfusepath{clip}%
\pgfsetbuttcap%
\pgfsetmiterjoin%
\definecolor{currentfill}{rgb}{0.007843,0.619608,0.450980}%
\pgfsetfillcolor{currentfill}%
\pgfsetfillopacity{0.700000}%
\pgfsetlinewidth{0.000000pt}%
\definecolor{currentstroke}{rgb}{0.000000,0.000000,0.000000}%
\pgfsetstrokecolor{currentstroke}%
\pgfsetstrokeopacity{0.700000}%
\pgfsetdash{}{0pt}%
\pgfpathmoveto{\pgfqpoint{1.451086in}{0.524170in}}%
\pgfpathlineto{\pgfqpoint{1.493174in}{0.524170in}}%
\pgfpathlineto{\pgfqpoint{1.493174in}{0.524170in}}%
\pgfpathlineto{\pgfqpoint{1.451086in}{0.524170in}}%
\pgfpathlineto{\pgfqpoint{1.451086in}{0.524170in}}%
\pgfpathclose%
\pgfusepath{fill}%
\end{pgfscope}%
\begin{pgfscope}%
\pgfpathrectangle{\pgfqpoint{0.651412in}{0.524170in}}{\pgfqpoint{4.629690in}{2.558193in}}%
\pgfusepath{clip}%
\pgfsetbuttcap%
\pgfsetmiterjoin%
\definecolor{currentfill}{rgb}{0.007843,0.619608,0.450980}%
\pgfsetfillcolor{currentfill}%
\pgfsetfillopacity{0.700000}%
\pgfsetlinewidth{0.000000pt}%
\definecolor{currentstroke}{rgb}{0.000000,0.000000,0.000000}%
\pgfsetstrokecolor{currentstroke}%
\pgfsetstrokeopacity{0.700000}%
\pgfsetdash{}{0pt}%
\pgfpathmoveto{\pgfqpoint{1.493174in}{0.524170in}}%
\pgfpathlineto{\pgfqpoint{1.535262in}{0.524170in}}%
\pgfpathlineto{\pgfqpoint{1.535262in}{0.524170in}}%
\pgfpathlineto{\pgfqpoint{1.493174in}{0.524170in}}%
\pgfpathlineto{\pgfqpoint{1.493174in}{0.524170in}}%
\pgfpathclose%
\pgfusepath{fill}%
\end{pgfscope}%
\begin{pgfscope}%
\pgfpathrectangle{\pgfqpoint{0.651412in}{0.524170in}}{\pgfqpoint{4.629690in}{2.558193in}}%
\pgfusepath{clip}%
\pgfsetbuttcap%
\pgfsetmiterjoin%
\definecolor{currentfill}{rgb}{0.007843,0.619608,0.450980}%
\pgfsetfillcolor{currentfill}%
\pgfsetfillopacity{0.700000}%
\pgfsetlinewidth{0.000000pt}%
\definecolor{currentstroke}{rgb}{0.000000,0.000000,0.000000}%
\pgfsetstrokecolor{currentstroke}%
\pgfsetstrokeopacity{0.700000}%
\pgfsetdash{}{0pt}%
\pgfpathmoveto{\pgfqpoint{1.535262in}{0.524170in}}%
\pgfpathlineto{\pgfqpoint{1.577350in}{0.524170in}}%
\pgfpathlineto{\pgfqpoint{1.577350in}{0.524170in}}%
\pgfpathlineto{\pgfqpoint{1.535262in}{0.524170in}}%
\pgfpathlineto{\pgfqpoint{1.535262in}{0.524170in}}%
\pgfpathclose%
\pgfusepath{fill}%
\end{pgfscope}%
\begin{pgfscope}%
\pgfpathrectangle{\pgfqpoint{0.651412in}{0.524170in}}{\pgfqpoint{4.629690in}{2.558193in}}%
\pgfusepath{clip}%
\pgfsetbuttcap%
\pgfsetmiterjoin%
\definecolor{currentfill}{rgb}{0.007843,0.619608,0.450980}%
\pgfsetfillcolor{currentfill}%
\pgfsetfillopacity{0.700000}%
\pgfsetlinewidth{0.000000pt}%
\definecolor{currentstroke}{rgb}{0.000000,0.000000,0.000000}%
\pgfsetstrokecolor{currentstroke}%
\pgfsetstrokeopacity{0.700000}%
\pgfsetdash{}{0pt}%
\pgfpathmoveto{\pgfqpoint{1.577350in}{0.524170in}}%
\pgfpathlineto{\pgfqpoint{1.619438in}{0.524170in}}%
\pgfpathlineto{\pgfqpoint{1.619438in}{0.562029in}}%
\pgfpathlineto{\pgfqpoint{1.577350in}{0.562029in}}%
\pgfpathlineto{\pgfqpoint{1.577350in}{0.524170in}}%
\pgfpathclose%
\pgfusepath{fill}%
\end{pgfscope}%
\begin{pgfscope}%
\pgfpathrectangle{\pgfqpoint{0.651412in}{0.524170in}}{\pgfqpoint{4.629690in}{2.558193in}}%
\pgfusepath{clip}%
\pgfsetbuttcap%
\pgfsetmiterjoin%
\definecolor{currentfill}{rgb}{0.007843,0.619608,0.450980}%
\pgfsetfillcolor{currentfill}%
\pgfsetfillopacity{0.700000}%
\pgfsetlinewidth{0.000000pt}%
\definecolor{currentstroke}{rgb}{0.000000,0.000000,0.000000}%
\pgfsetstrokecolor{currentstroke}%
\pgfsetstrokeopacity{0.700000}%
\pgfsetdash{}{0pt}%
\pgfpathmoveto{\pgfqpoint{1.619438in}{0.524170in}}%
\pgfpathlineto{\pgfqpoint{1.661526in}{0.524170in}}%
\pgfpathlineto{\pgfqpoint{1.661526in}{0.543099in}}%
\pgfpathlineto{\pgfqpoint{1.619438in}{0.543099in}}%
\pgfpathlineto{\pgfqpoint{1.619438in}{0.524170in}}%
\pgfpathclose%
\pgfusepath{fill}%
\end{pgfscope}%
\begin{pgfscope}%
\pgfpathrectangle{\pgfqpoint{0.651412in}{0.524170in}}{\pgfqpoint{4.629690in}{2.558193in}}%
\pgfusepath{clip}%
\pgfsetbuttcap%
\pgfsetmiterjoin%
\definecolor{currentfill}{rgb}{0.007843,0.619608,0.450980}%
\pgfsetfillcolor{currentfill}%
\pgfsetfillopacity{0.700000}%
\pgfsetlinewidth{0.000000pt}%
\definecolor{currentstroke}{rgb}{0.000000,0.000000,0.000000}%
\pgfsetstrokecolor{currentstroke}%
\pgfsetstrokeopacity{0.700000}%
\pgfsetdash{}{0pt}%
\pgfpathmoveto{\pgfqpoint{1.661526in}{0.524170in}}%
\pgfpathlineto{\pgfqpoint{1.703614in}{0.524170in}}%
\pgfpathlineto{\pgfqpoint{1.703614in}{0.580959in}}%
\pgfpathlineto{\pgfqpoint{1.661526in}{0.580959in}}%
\pgfpathlineto{\pgfqpoint{1.661526in}{0.524170in}}%
\pgfpathclose%
\pgfusepath{fill}%
\end{pgfscope}%
\begin{pgfscope}%
\pgfpathrectangle{\pgfqpoint{0.651412in}{0.524170in}}{\pgfqpoint{4.629690in}{2.558193in}}%
\pgfusepath{clip}%
\pgfsetbuttcap%
\pgfsetmiterjoin%
\definecolor{currentfill}{rgb}{0.007843,0.619608,0.450980}%
\pgfsetfillcolor{currentfill}%
\pgfsetfillopacity{0.700000}%
\pgfsetlinewidth{0.000000pt}%
\definecolor{currentstroke}{rgb}{0.000000,0.000000,0.000000}%
\pgfsetstrokecolor{currentstroke}%
\pgfsetstrokeopacity{0.700000}%
\pgfsetdash{}{0pt}%
\pgfpathmoveto{\pgfqpoint{1.703614in}{0.524170in}}%
\pgfpathlineto{\pgfqpoint{1.745703in}{0.524170in}}%
\pgfpathlineto{\pgfqpoint{1.745703in}{0.562029in}}%
\pgfpathlineto{\pgfqpoint{1.703614in}{0.562029in}}%
\pgfpathlineto{\pgfqpoint{1.703614in}{0.524170in}}%
\pgfpathclose%
\pgfusepath{fill}%
\end{pgfscope}%
\begin{pgfscope}%
\pgfpathrectangle{\pgfqpoint{0.651412in}{0.524170in}}{\pgfqpoint{4.629690in}{2.558193in}}%
\pgfusepath{clip}%
\pgfsetbuttcap%
\pgfsetmiterjoin%
\definecolor{currentfill}{rgb}{0.007843,0.619608,0.450980}%
\pgfsetfillcolor{currentfill}%
\pgfsetfillopacity{0.700000}%
\pgfsetlinewidth{0.000000pt}%
\definecolor{currentstroke}{rgb}{0.000000,0.000000,0.000000}%
\pgfsetstrokecolor{currentstroke}%
\pgfsetstrokeopacity{0.700000}%
\pgfsetdash{}{0pt}%
\pgfpathmoveto{\pgfqpoint{1.745703in}{0.524170in}}%
\pgfpathlineto{\pgfqpoint{1.787791in}{0.524170in}}%
\pgfpathlineto{\pgfqpoint{1.787791in}{0.628283in}}%
\pgfpathlineto{\pgfqpoint{1.745703in}{0.628283in}}%
\pgfpathlineto{\pgfqpoint{1.745703in}{0.524170in}}%
\pgfpathclose%
\pgfusepath{fill}%
\end{pgfscope}%
\begin{pgfscope}%
\pgfpathrectangle{\pgfqpoint{0.651412in}{0.524170in}}{\pgfqpoint{4.629690in}{2.558193in}}%
\pgfusepath{clip}%
\pgfsetbuttcap%
\pgfsetmiterjoin%
\definecolor{currentfill}{rgb}{0.007843,0.619608,0.450980}%
\pgfsetfillcolor{currentfill}%
\pgfsetfillopacity{0.700000}%
\pgfsetlinewidth{0.000000pt}%
\definecolor{currentstroke}{rgb}{0.000000,0.000000,0.000000}%
\pgfsetstrokecolor{currentstroke}%
\pgfsetstrokeopacity{0.700000}%
\pgfsetdash{}{0pt}%
\pgfpathmoveto{\pgfqpoint{1.787791in}{0.524170in}}%
\pgfpathlineto{\pgfqpoint{1.829879in}{0.524170in}}%
\pgfpathlineto{\pgfqpoint{1.829879in}{0.656677in}}%
\pgfpathlineto{\pgfqpoint{1.787791in}{0.656677in}}%
\pgfpathlineto{\pgfqpoint{1.787791in}{0.524170in}}%
\pgfpathclose%
\pgfusepath{fill}%
\end{pgfscope}%
\begin{pgfscope}%
\pgfpathrectangle{\pgfqpoint{0.651412in}{0.524170in}}{\pgfqpoint{4.629690in}{2.558193in}}%
\pgfusepath{clip}%
\pgfsetbuttcap%
\pgfsetmiterjoin%
\definecolor{currentfill}{rgb}{0.007843,0.619608,0.450980}%
\pgfsetfillcolor{currentfill}%
\pgfsetfillopacity{0.700000}%
\pgfsetlinewidth{0.000000pt}%
\definecolor{currentstroke}{rgb}{0.000000,0.000000,0.000000}%
\pgfsetstrokecolor{currentstroke}%
\pgfsetstrokeopacity{0.700000}%
\pgfsetdash{}{0pt}%
\pgfpathmoveto{\pgfqpoint{1.829879in}{0.524170in}}%
\pgfpathlineto{\pgfqpoint{1.871967in}{0.524170in}}%
\pgfpathlineto{\pgfqpoint{1.871967in}{0.685072in}}%
\pgfpathlineto{\pgfqpoint{1.829879in}{0.685072in}}%
\pgfpathlineto{\pgfqpoint{1.829879in}{0.524170in}}%
\pgfpathclose%
\pgfusepath{fill}%
\end{pgfscope}%
\begin{pgfscope}%
\pgfpathrectangle{\pgfqpoint{0.651412in}{0.524170in}}{\pgfqpoint{4.629690in}{2.558193in}}%
\pgfusepath{clip}%
\pgfsetbuttcap%
\pgfsetmiterjoin%
\definecolor{currentfill}{rgb}{0.007843,0.619608,0.450980}%
\pgfsetfillcolor{currentfill}%
\pgfsetfillopacity{0.700000}%
\pgfsetlinewidth{0.000000pt}%
\definecolor{currentstroke}{rgb}{0.000000,0.000000,0.000000}%
\pgfsetstrokecolor{currentstroke}%
\pgfsetstrokeopacity{0.700000}%
\pgfsetdash{}{0pt}%
\pgfpathmoveto{\pgfqpoint{1.871967in}{0.524170in}}%
\pgfpathlineto{\pgfqpoint{1.914055in}{0.524170in}}%
\pgfpathlineto{\pgfqpoint{1.914055in}{0.760790in}}%
\pgfpathlineto{\pgfqpoint{1.871967in}{0.760790in}}%
\pgfpathlineto{\pgfqpoint{1.871967in}{0.524170in}}%
\pgfpathclose%
\pgfusepath{fill}%
\end{pgfscope}%
\begin{pgfscope}%
\pgfpathrectangle{\pgfqpoint{0.651412in}{0.524170in}}{\pgfqpoint{4.629690in}{2.558193in}}%
\pgfusepath{clip}%
\pgfsetbuttcap%
\pgfsetmiterjoin%
\definecolor{currentfill}{rgb}{0.007843,0.619608,0.450980}%
\pgfsetfillcolor{currentfill}%
\pgfsetfillopacity{0.700000}%
\pgfsetlinewidth{0.000000pt}%
\definecolor{currentstroke}{rgb}{0.000000,0.000000,0.000000}%
\pgfsetstrokecolor{currentstroke}%
\pgfsetstrokeopacity{0.700000}%
\pgfsetdash{}{0pt}%
\pgfpathmoveto{\pgfqpoint{1.914055in}{0.524170in}}%
\pgfpathlineto{\pgfqpoint{1.956143in}{0.524170in}}%
\pgfpathlineto{\pgfqpoint{1.956143in}{0.713466in}}%
\pgfpathlineto{\pgfqpoint{1.914055in}{0.713466in}}%
\pgfpathlineto{\pgfqpoint{1.914055in}{0.524170in}}%
\pgfpathclose%
\pgfusepath{fill}%
\end{pgfscope}%
\begin{pgfscope}%
\pgfpathrectangle{\pgfqpoint{0.651412in}{0.524170in}}{\pgfqpoint{4.629690in}{2.558193in}}%
\pgfusepath{clip}%
\pgfsetbuttcap%
\pgfsetmiterjoin%
\definecolor{currentfill}{rgb}{0.007843,0.619608,0.450980}%
\pgfsetfillcolor{currentfill}%
\pgfsetfillopacity{0.700000}%
\pgfsetlinewidth{0.000000pt}%
\definecolor{currentstroke}{rgb}{0.000000,0.000000,0.000000}%
\pgfsetstrokecolor{currentstroke}%
\pgfsetstrokeopacity{0.700000}%
\pgfsetdash{}{0pt}%
\pgfpathmoveto{\pgfqpoint{1.956143in}{0.524170in}}%
\pgfpathlineto{\pgfqpoint{1.998231in}{0.524170in}}%
\pgfpathlineto{\pgfqpoint{1.998231in}{0.827044in}}%
\pgfpathlineto{\pgfqpoint{1.956143in}{0.827044in}}%
\pgfpathlineto{\pgfqpoint{1.956143in}{0.524170in}}%
\pgfpathclose%
\pgfusepath{fill}%
\end{pgfscope}%
\begin{pgfscope}%
\pgfpathrectangle{\pgfqpoint{0.651412in}{0.524170in}}{\pgfqpoint{4.629690in}{2.558193in}}%
\pgfusepath{clip}%
\pgfsetbuttcap%
\pgfsetmiterjoin%
\definecolor{currentfill}{rgb}{0.007843,0.619608,0.450980}%
\pgfsetfillcolor{currentfill}%
\pgfsetfillopacity{0.700000}%
\pgfsetlinewidth{0.000000pt}%
\definecolor{currentstroke}{rgb}{0.000000,0.000000,0.000000}%
\pgfsetstrokecolor{currentstroke}%
\pgfsetstrokeopacity{0.700000}%
\pgfsetdash{}{0pt}%
\pgfpathmoveto{\pgfqpoint{1.998231in}{0.524170in}}%
\pgfpathlineto{\pgfqpoint{2.040319in}{0.524170in}}%
\pgfpathlineto{\pgfqpoint{2.040319in}{0.732396in}}%
\pgfpathlineto{\pgfqpoint{1.998231in}{0.732396in}}%
\pgfpathlineto{\pgfqpoint{1.998231in}{0.524170in}}%
\pgfpathclose%
\pgfusepath{fill}%
\end{pgfscope}%
\begin{pgfscope}%
\pgfpathrectangle{\pgfqpoint{0.651412in}{0.524170in}}{\pgfqpoint{4.629690in}{2.558193in}}%
\pgfusepath{clip}%
\pgfsetbuttcap%
\pgfsetmiterjoin%
\definecolor{currentfill}{rgb}{0.007843,0.619608,0.450980}%
\pgfsetfillcolor{currentfill}%
\pgfsetfillopacity{0.700000}%
\pgfsetlinewidth{0.000000pt}%
\definecolor{currentstroke}{rgb}{0.000000,0.000000,0.000000}%
\pgfsetstrokecolor{currentstroke}%
\pgfsetstrokeopacity{0.700000}%
\pgfsetdash{}{0pt}%
\pgfpathmoveto{\pgfqpoint{2.040319in}{0.524170in}}%
\pgfpathlineto{\pgfqpoint{2.082407in}{0.524170in}}%
\pgfpathlineto{\pgfqpoint{2.082407in}{0.789185in}}%
\pgfpathlineto{\pgfqpoint{2.040319in}{0.789185in}}%
\pgfpathlineto{\pgfqpoint{2.040319in}{0.524170in}}%
\pgfpathclose%
\pgfusepath{fill}%
\end{pgfscope}%
\begin{pgfscope}%
\pgfpathrectangle{\pgfqpoint{0.651412in}{0.524170in}}{\pgfqpoint{4.629690in}{2.558193in}}%
\pgfusepath{clip}%
\pgfsetbuttcap%
\pgfsetmiterjoin%
\definecolor{currentfill}{rgb}{0.007843,0.619608,0.450980}%
\pgfsetfillcolor{currentfill}%
\pgfsetfillopacity{0.700000}%
\pgfsetlinewidth{0.000000pt}%
\definecolor{currentstroke}{rgb}{0.000000,0.000000,0.000000}%
\pgfsetstrokecolor{currentstroke}%
\pgfsetstrokeopacity{0.700000}%
\pgfsetdash{}{0pt}%
\pgfpathmoveto{\pgfqpoint{2.082407in}{0.524170in}}%
\pgfpathlineto{\pgfqpoint{2.124495in}{0.524170in}}%
\pgfpathlineto{\pgfqpoint{2.124495in}{0.827044in}}%
\pgfpathlineto{\pgfqpoint{2.082407in}{0.827044in}}%
\pgfpathlineto{\pgfqpoint{2.082407in}{0.524170in}}%
\pgfpathclose%
\pgfusepath{fill}%
\end{pgfscope}%
\begin{pgfscope}%
\pgfpathrectangle{\pgfqpoint{0.651412in}{0.524170in}}{\pgfqpoint{4.629690in}{2.558193in}}%
\pgfusepath{clip}%
\pgfsetbuttcap%
\pgfsetmiterjoin%
\definecolor{currentfill}{rgb}{0.007843,0.619608,0.450980}%
\pgfsetfillcolor{currentfill}%
\pgfsetfillopacity{0.700000}%
\pgfsetlinewidth{0.000000pt}%
\definecolor{currentstroke}{rgb}{0.000000,0.000000,0.000000}%
\pgfsetstrokecolor{currentstroke}%
\pgfsetstrokeopacity{0.700000}%
\pgfsetdash{}{0pt}%
\pgfpathmoveto{\pgfqpoint{2.124495in}{0.524170in}}%
\pgfpathlineto{\pgfqpoint{2.166583in}{0.524170in}}%
\pgfpathlineto{\pgfqpoint{2.166583in}{0.789185in}}%
\pgfpathlineto{\pgfqpoint{2.124495in}{0.789185in}}%
\pgfpathlineto{\pgfqpoint{2.124495in}{0.524170in}}%
\pgfpathclose%
\pgfusepath{fill}%
\end{pgfscope}%
\begin{pgfscope}%
\pgfpathrectangle{\pgfqpoint{0.651412in}{0.524170in}}{\pgfqpoint{4.629690in}{2.558193in}}%
\pgfusepath{clip}%
\pgfsetbuttcap%
\pgfsetmiterjoin%
\definecolor{currentfill}{rgb}{0.007843,0.619608,0.450980}%
\pgfsetfillcolor{currentfill}%
\pgfsetfillopacity{0.700000}%
\pgfsetlinewidth{0.000000pt}%
\definecolor{currentstroke}{rgb}{0.000000,0.000000,0.000000}%
\pgfsetstrokecolor{currentstroke}%
\pgfsetstrokeopacity{0.700000}%
\pgfsetdash{}{0pt}%
\pgfpathmoveto{\pgfqpoint{2.166583in}{0.524170in}}%
\pgfpathlineto{\pgfqpoint{2.208672in}{0.524170in}}%
\pgfpathlineto{\pgfqpoint{2.208672in}{0.902762in}}%
\pgfpathlineto{\pgfqpoint{2.166583in}{0.902762in}}%
\pgfpathlineto{\pgfqpoint{2.166583in}{0.524170in}}%
\pgfpathclose%
\pgfusepath{fill}%
\end{pgfscope}%
\begin{pgfscope}%
\pgfpathrectangle{\pgfqpoint{0.651412in}{0.524170in}}{\pgfqpoint{4.629690in}{2.558193in}}%
\pgfusepath{clip}%
\pgfsetbuttcap%
\pgfsetmiterjoin%
\definecolor{currentfill}{rgb}{0.007843,0.619608,0.450980}%
\pgfsetfillcolor{currentfill}%
\pgfsetfillopacity{0.700000}%
\pgfsetlinewidth{0.000000pt}%
\definecolor{currentstroke}{rgb}{0.000000,0.000000,0.000000}%
\pgfsetstrokecolor{currentstroke}%
\pgfsetstrokeopacity{0.700000}%
\pgfsetdash{}{0pt}%
\pgfpathmoveto{\pgfqpoint{2.208672in}{0.524170in}}%
\pgfpathlineto{\pgfqpoint{2.250760in}{0.524170in}}%
\pgfpathlineto{\pgfqpoint{2.250760in}{0.921692in}}%
\pgfpathlineto{\pgfqpoint{2.208672in}{0.921692in}}%
\pgfpathlineto{\pgfqpoint{2.208672in}{0.524170in}}%
\pgfpathclose%
\pgfusepath{fill}%
\end{pgfscope}%
\begin{pgfscope}%
\pgfpathrectangle{\pgfqpoint{0.651412in}{0.524170in}}{\pgfqpoint{4.629690in}{2.558193in}}%
\pgfusepath{clip}%
\pgfsetbuttcap%
\pgfsetmiterjoin%
\definecolor{currentfill}{rgb}{0.007843,0.619608,0.450980}%
\pgfsetfillcolor{currentfill}%
\pgfsetfillopacity{0.700000}%
\pgfsetlinewidth{0.000000pt}%
\definecolor{currentstroke}{rgb}{0.000000,0.000000,0.000000}%
\pgfsetstrokecolor{currentstroke}%
\pgfsetstrokeopacity{0.700000}%
\pgfsetdash{}{0pt}%
\pgfpathmoveto{\pgfqpoint{2.250760in}{0.524170in}}%
\pgfpathlineto{\pgfqpoint{2.292848in}{0.524170in}}%
\pgfpathlineto{\pgfqpoint{2.292848in}{0.845974in}}%
\pgfpathlineto{\pgfqpoint{2.250760in}{0.845974in}}%
\pgfpathlineto{\pgfqpoint{2.250760in}{0.524170in}}%
\pgfpathclose%
\pgfusepath{fill}%
\end{pgfscope}%
\begin{pgfscope}%
\pgfpathrectangle{\pgfqpoint{0.651412in}{0.524170in}}{\pgfqpoint{4.629690in}{2.558193in}}%
\pgfusepath{clip}%
\pgfsetbuttcap%
\pgfsetmiterjoin%
\definecolor{currentfill}{rgb}{0.007843,0.619608,0.450980}%
\pgfsetfillcolor{currentfill}%
\pgfsetfillopacity{0.700000}%
\pgfsetlinewidth{0.000000pt}%
\definecolor{currentstroke}{rgb}{0.000000,0.000000,0.000000}%
\pgfsetstrokecolor{currentstroke}%
\pgfsetstrokeopacity{0.700000}%
\pgfsetdash{}{0pt}%
\pgfpathmoveto{\pgfqpoint{2.292848in}{0.524170in}}%
\pgfpathlineto{\pgfqpoint{2.334936in}{0.524170in}}%
\pgfpathlineto{\pgfqpoint{2.334936in}{0.969016in}}%
\pgfpathlineto{\pgfqpoint{2.292848in}{0.969016in}}%
\pgfpathlineto{\pgfqpoint{2.292848in}{0.524170in}}%
\pgfpathclose%
\pgfusepath{fill}%
\end{pgfscope}%
\begin{pgfscope}%
\pgfpathrectangle{\pgfqpoint{0.651412in}{0.524170in}}{\pgfqpoint{4.629690in}{2.558193in}}%
\pgfusepath{clip}%
\pgfsetbuttcap%
\pgfsetmiterjoin%
\definecolor{currentfill}{rgb}{0.007843,0.619608,0.450980}%
\pgfsetfillcolor{currentfill}%
\pgfsetfillopacity{0.700000}%
\pgfsetlinewidth{0.000000pt}%
\definecolor{currentstroke}{rgb}{0.000000,0.000000,0.000000}%
\pgfsetstrokecolor{currentstroke}%
\pgfsetstrokeopacity{0.700000}%
\pgfsetdash{}{0pt}%
\pgfpathmoveto{\pgfqpoint{2.334936in}{0.524170in}}%
\pgfpathlineto{\pgfqpoint{2.377024in}{0.524170in}}%
\pgfpathlineto{\pgfqpoint{2.377024in}{0.893298in}}%
\pgfpathlineto{\pgfqpoint{2.334936in}{0.893298in}}%
\pgfpathlineto{\pgfqpoint{2.334936in}{0.524170in}}%
\pgfpathclose%
\pgfusepath{fill}%
\end{pgfscope}%
\begin{pgfscope}%
\pgfpathrectangle{\pgfqpoint{0.651412in}{0.524170in}}{\pgfqpoint{4.629690in}{2.558193in}}%
\pgfusepath{clip}%
\pgfsetbuttcap%
\pgfsetmiterjoin%
\definecolor{currentfill}{rgb}{0.007843,0.619608,0.450980}%
\pgfsetfillcolor{currentfill}%
\pgfsetfillopacity{0.700000}%
\pgfsetlinewidth{0.000000pt}%
\definecolor{currentstroke}{rgb}{0.000000,0.000000,0.000000}%
\pgfsetstrokecolor{currentstroke}%
\pgfsetstrokeopacity{0.700000}%
\pgfsetdash{}{0pt}%
\pgfpathmoveto{\pgfqpoint{2.377024in}{0.524170in}}%
\pgfpathlineto{\pgfqpoint{2.419112in}{0.524170in}}%
\pgfpathlineto{\pgfqpoint{2.419112in}{0.978481in}}%
\pgfpathlineto{\pgfqpoint{2.377024in}{0.978481in}}%
\pgfpathlineto{\pgfqpoint{2.377024in}{0.524170in}}%
\pgfpathclose%
\pgfusepath{fill}%
\end{pgfscope}%
\begin{pgfscope}%
\pgfpathrectangle{\pgfqpoint{0.651412in}{0.524170in}}{\pgfqpoint{4.629690in}{2.558193in}}%
\pgfusepath{clip}%
\pgfsetbuttcap%
\pgfsetmiterjoin%
\definecolor{currentfill}{rgb}{0.007843,0.619608,0.450980}%
\pgfsetfillcolor{currentfill}%
\pgfsetfillopacity{0.700000}%
\pgfsetlinewidth{0.000000pt}%
\definecolor{currentstroke}{rgb}{0.000000,0.000000,0.000000}%
\pgfsetstrokecolor{currentstroke}%
\pgfsetstrokeopacity{0.700000}%
\pgfsetdash{}{0pt}%
\pgfpathmoveto{\pgfqpoint{2.419112in}{0.524170in}}%
\pgfpathlineto{\pgfqpoint{2.461200in}{0.524170in}}%
\pgfpathlineto{\pgfqpoint{2.461200in}{0.931157in}}%
\pgfpathlineto{\pgfqpoint{2.419112in}{0.931157in}}%
\pgfpathlineto{\pgfqpoint{2.419112in}{0.524170in}}%
\pgfpathclose%
\pgfusepath{fill}%
\end{pgfscope}%
\begin{pgfscope}%
\pgfpathrectangle{\pgfqpoint{0.651412in}{0.524170in}}{\pgfqpoint{4.629690in}{2.558193in}}%
\pgfusepath{clip}%
\pgfsetbuttcap%
\pgfsetmiterjoin%
\definecolor{currentfill}{rgb}{0.007843,0.619608,0.450980}%
\pgfsetfillcolor{currentfill}%
\pgfsetfillopacity{0.700000}%
\pgfsetlinewidth{0.000000pt}%
\definecolor{currentstroke}{rgb}{0.000000,0.000000,0.000000}%
\pgfsetstrokecolor{currentstroke}%
\pgfsetstrokeopacity{0.700000}%
\pgfsetdash{}{0pt}%
\pgfpathmoveto{\pgfqpoint{2.461200in}{0.524170in}}%
\pgfpathlineto{\pgfqpoint{2.503288in}{0.524170in}}%
\pgfpathlineto{\pgfqpoint{2.503288in}{0.921692in}}%
\pgfpathlineto{\pgfqpoint{2.461200in}{0.921692in}}%
\pgfpathlineto{\pgfqpoint{2.461200in}{0.524170in}}%
\pgfpathclose%
\pgfusepath{fill}%
\end{pgfscope}%
\begin{pgfscope}%
\pgfpathrectangle{\pgfqpoint{0.651412in}{0.524170in}}{\pgfqpoint{4.629690in}{2.558193in}}%
\pgfusepath{clip}%
\pgfsetbuttcap%
\pgfsetmiterjoin%
\definecolor{currentfill}{rgb}{0.007843,0.619608,0.450980}%
\pgfsetfillcolor{currentfill}%
\pgfsetfillopacity{0.700000}%
\pgfsetlinewidth{0.000000pt}%
\definecolor{currentstroke}{rgb}{0.000000,0.000000,0.000000}%
\pgfsetstrokecolor{currentstroke}%
\pgfsetstrokeopacity{0.700000}%
\pgfsetdash{}{0pt}%
\pgfpathmoveto{\pgfqpoint{2.503288in}{0.524170in}}%
\pgfpathlineto{\pgfqpoint{2.545376in}{0.524170in}}%
\pgfpathlineto{\pgfqpoint{2.545376in}{0.940622in}}%
\pgfpathlineto{\pgfqpoint{2.503288in}{0.940622in}}%
\pgfpathlineto{\pgfqpoint{2.503288in}{0.524170in}}%
\pgfpathclose%
\pgfusepath{fill}%
\end{pgfscope}%
\begin{pgfscope}%
\pgfpathrectangle{\pgfqpoint{0.651412in}{0.524170in}}{\pgfqpoint{4.629690in}{2.558193in}}%
\pgfusepath{clip}%
\pgfsetbuttcap%
\pgfsetmiterjoin%
\definecolor{currentfill}{rgb}{0.007843,0.619608,0.450980}%
\pgfsetfillcolor{currentfill}%
\pgfsetfillopacity{0.700000}%
\pgfsetlinewidth{0.000000pt}%
\definecolor{currentstroke}{rgb}{0.000000,0.000000,0.000000}%
\pgfsetstrokecolor{currentstroke}%
\pgfsetstrokeopacity{0.700000}%
\pgfsetdash{}{0pt}%
\pgfpathmoveto{\pgfqpoint{2.545376in}{0.524170in}}%
\pgfpathlineto{\pgfqpoint{2.587464in}{0.524170in}}%
\pgfpathlineto{\pgfqpoint{2.587464in}{0.912227in}}%
\pgfpathlineto{\pgfqpoint{2.545376in}{0.912227in}}%
\pgfpathlineto{\pgfqpoint{2.545376in}{0.524170in}}%
\pgfpathclose%
\pgfusepath{fill}%
\end{pgfscope}%
\begin{pgfscope}%
\pgfpathrectangle{\pgfqpoint{0.651412in}{0.524170in}}{\pgfqpoint{4.629690in}{2.558193in}}%
\pgfusepath{clip}%
\pgfsetbuttcap%
\pgfsetmiterjoin%
\definecolor{currentfill}{rgb}{0.007843,0.619608,0.450980}%
\pgfsetfillcolor{currentfill}%
\pgfsetfillopacity{0.700000}%
\pgfsetlinewidth{0.000000pt}%
\definecolor{currentstroke}{rgb}{0.000000,0.000000,0.000000}%
\pgfsetstrokecolor{currentstroke}%
\pgfsetstrokeopacity{0.700000}%
\pgfsetdash{}{0pt}%
\pgfpathmoveto{\pgfqpoint{2.587464in}{0.524170in}}%
\pgfpathlineto{\pgfqpoint{2.629553in}{0.524170in}}%
\pgfpathlineto{\pgfqpoint{2.629553in}{0.997411in}}%
\pgfpathlineto{\pgfqpoint{2.587464in}{0.997411in}}%
\pgfpathlineto{\pgfqpoint{2.587464in}{0.524170in}}%
\pgfpathclose%
\pgfusepath{fill}%
\end{pgfscope}%
\begin{pgfscope}%
\pgfpathrectangle{\pgfqpoint{0.651412in}{0.524170in}}{\pgfqpoint{4.629690in}{2.558193in}}%
\pgfusepath{clip}%
\pgfsetbuttcap%
\pgfsetmiterjoin%
\definecolor{currentfill}{rgb}{0.007843,0.619608,0.450980}%
\pgfsetfillcolor{currentfill}%
\pgfsetfillopacity{0.700000}%
\pgfsetlinewidth{0.000000pt}%
\definecolor{currentstroke}{rgb}{0.000000,0.000000,0.000000}%
\pgfsetstrokecolor{currentstroke}%
\pgfsetstrokeopacity{0.700000}%
\pgfsetdash{}{0pt}%
\pgfpathmoveto{\pgfqpoint{2.629553in}{0.524170in}}%
\pgfpathlineto{\pgfqpoint{2.671641in}{0.524170in}}%
\pgfpathlineto{\pgfqpoint{2.671641in}{0.827044in}}%
\pgfpathlineto{\pgfqpoint{2.629553in}{0.827044in}}%
\pgfpathlineto{\pgfqpoint{2.629553in}{0.524170in}}%
\pgfpathclose%
\pgfusepath{fill}%
\end{pgfscope}%
\begin{pgfscope}%
\pgfpathrectangle{\pgfqpoint{0.651412in}{0.524170in}}{\pgfqpoint{4.629690in}{2.558193in}}%
\pgfusepath{clip}%
\pgfsetbuttcap%
\pgfsetmiterjoin%
\definecolor{currentfill}{rgb}{0.007843,0.619608,0.450980}%
\pgfsetfillcolor{currentfill}%
\pgfsetfillopacity{0.700000}%
\pgfsetlinewidth{0.000000pt}%
\definecolor{currentstroke}{rgb}{0.000000,0.000000,0.000000}%
\pgfsetstrokecolor{currentstroke}%
\pgfsetstrokeopacity{0.700000}%
\pgfsetdash{}{0pt}%
\pgfpathmoveto{\pgfqpoint{2.671641in}{0.524170in}}%
\pgfpathlineto{\pgfqpoint{2.713729in}{0.524170in}}%
\pgfpathlineto{\pgfqpoint{2.713729in}{0.902762in}}%
\pgfpathlineto{\pgfqpoint{2.671641in}{0.902762in}}%
\pgfpathlineto{\pgfqpoint{2.671641in}{0.524170in}}%
\pgfpathclose%
\pgfusepath{fill}%
\end{pgfscope}%
\begin{pgfscope}%
\pgfpathrectangle{\pgfqpoint{0.651412in}{0.524170in}}{\pgfqpoint{4.629690in}{2.558193in}}%
\pgfusepath{clip}%
\pgfsetbuttcap%
\pgfsetmiterjoin%
\definecolor{currentfill}{rgb}{0.007843,0.619608,0.450980}%
\pgfsetfillcolor{currentfill}%
\pgfsetfillopacity{0.700000}%
\pgfsetlinewidth{0.000000pt}%
\definecolor{currentstroke}{rgb}{0.000000,0.000000,0.000000}%
\pgfsetstrokecolor{currentstroke}%
\pgfsetstrokeopacity{0.700000}%
\pgfsetdash{}{0pt}%
\pgfpathmoveto{\pgfqpoint{2.713729in}{0.524170in}}%
\pgfpathlineto{\pgfqpoint{2.755817in}{0.524170in}}%
\pgfpathlineto{\pgfqpoint{2.755817in}{0.864903in}}%
\pgfpathlineto{\pgfqpoint{2.713729in}{0.864903in}}%
\pgfpathlineto{\pgfqpoint{2.713729in}{0.524170in}}%
\pgfpathclose%
\pgfusepath{fill}%
\end{pgfscope}%
\begin{pgfscope}%
\pgfpathrectangle{\pgfqpoint{0.651412in}{0.524170in}}{\pgfqpoint{4.629690in}{2.558193in}}%
\pgfusepath{clip}%
\pgfsetbuttcap%
\pgfsetmiterjoin%
\definecolor{currentfill}{rgb}{0.007843,0.619608,0.450980}%
\pgfsetfillcolor{currentfill}%
\pgfsetfillopacity{0.700000}%
\pgfsetlinewidth{0.000000pt}%
\definecolor{currentstroke}{rgb}{0.000000,0.000000,0.000000}%
\pgfsetstrokecolor{currentstroke}%
\pgfsetstrokeopacity{0.700000}%
\pgfsetdash{}{0pt}%
\pgfpathmoveto{\pgfqpoint{2.755817in}{0.524170in}}%
\pgfpathlineto{\pgfqpoint{2.797905in}{0.524170in}}%
\pgfpathlineto{\pgfqpoint{2.797905in}{0.893298in}}%
\pgfpathlineto{\pgfqpoint{2.755817in}{0.893298in}}%
\pgfpathlineto{\pgfqpoint{2.755817in}{0.524170in}}%
\pgfpathclose%
\pgfusepath{fill}%
\end{pgfscope}%
\begin{pgfscope}%
\pgfpathrectangle{\pgfqpoint{0.651412in}{0.524170in}}{\pgfqpoint{4.629690in}{2.558193in}}%
\pgfusepath{clip}%
\pgfsetbuttcap%
\pgfsetmiterjoin%
\definecolor{currentfill}{rgb}{0.007843,0.619608,0.450980}%
\pgfsetfillcolor{currentfill}%
\pgfsetfillopacity{0.700000}%
\pgfsetlinewidth{0.000000pt}%
\definecolor{currentstroke}{rgb}{0.000000,0.000000,0.000000}%
\pgfsetstrokecolor{currentstroke}%
\pgfsetstrokeopacity{0.700000}%
\pgfsetdash{}{0pt}%
\pgfpathmoveto{\pgfqpoint{2.797905in}{0.524170in}}%
\pgfpathlineto{\pgfqpoint{2.839993in}{0.524170in}}%
\pgfpathlineto{\pgfqpoint{2.839993in}{0.978481in}}%
\pgfpathlineto{\pgfqpoint{2.797905in}{0.978481in}}%
\pgfpathlineto{\pgfqpoint{2.797905in}{0.524170in}}%
\pgfpathclose%
\pgfusepath{fill}%
\end{pgfscope}%
\begin{pgfscope}%
\pgfpathrectangle{\pgfqpoint{0.651412in}{0.524170in}}{\pgfqpoint{4.629690in}{2.558193in}}%
\pgfusepath{clip}%
\pgfsetbuttcap%
\pgfsetmiterjoin%
\definecolor{currentfill}{rgb}{0.007843,0.619608,0.450980}%
\pgfsetfillcolor{currentfill}%
\pgfsetfillopacity{0.700000}%
\pgfsetlinewidth{0.000000pt}%
\definecolor{currentstroke}{rgb}{0.000000,0.000000,0.000000}%
\pgfsetstrokecolor{currentstroke}%
\pgfsetstrokeopacity{0.700000}%
\pgfsetdash{}{0pt}%
\pgfpathmoveto{\pgfqpoint{2.839993in}{0.524170in}}%
\pgfpathlineto{\pgfqpoint{2.882081in}{0.524170in}}%
\pgfpathlineto{\pgfqpoint{2.882081in}{0.845974in}}%
\pgfpathlineto{\pgfqpoint{2.839993in}{0.845974in}}%
\pgfpathlineto{\pgfqpoint{2.839993in}{0.524170in}}%
\pgfpathclose%
\pgfusepath{fill}%
\end{pgfscope}%
\begin{pgfscope}%
\pgfpathrectangle{\pgfqpoint{0.651412in}{0.524170in}}{\pgfqpoint{4.629690in}{2.558193in}}%
\pgfusepath{clip}%
\pgfsetbuttcap%
\pgfsetmiterjoin%
\definecolor{currentfill}{rgb}{0.007843,0.619608,0.450980}%
\pgfsetfillcolor{currentfill}%
\pgfsetfillopacity{0.700000}%
\pgfsetlinewidth{0.000000pt}%
\definecolor{currentstroke}{rgb}{0.000000,0.000000,0.000000}%
\pgfsetstrokecolor{currentstroke}%
\pgfsetstrokeopacity{0.700000}%
\pgfsetdash{}{0pt}%
\pgfpathmoveto{\pgfqpoint{2.882081in}{0.524170in}}%
\pgfpathlineto{\pgfqpoint{2.924169in}{0.524170in}}%
\pgfpathlineto{\pgfqpoint{2.924169in}{1.025805in}}%
\pgfpathlineto{\pgfqpoint{2.882081in}{1.025805in}}%
\pgfpathlineto{\pgfqpoint{2.882081in}{0.524170in}}%
\pgfpathclose%
\pgfusepath{fill}%
\end{pgfscope}%
\begin{pgfscope}%
\pgfpathrectangle{\pgfqpoint{0.651412in}{0.524170in}}{\pgfqpoint{4.629690in}{2.558193in}}%
\pgfusepath{clip}%
\pgfsetbuttcap%
\pgfsetmiterjoin%
\definecolor{currentfill}{rgb}{0.007843,0.619608,0.450980}%
\pgfsetfillcolor{currentfill}%
\pgfsetfillopacity{0.700000}%
\pgfsetlinewidth{0.000000pt}%
\definecolor{currentstroke}{rgb}{0.000000,0.000000,0.000000}%
\pgfsetstrokecolor{currentstroke}%
\pgfsetstrokeopacity{0.700000}%
\pgfsetdash{}{0pt}%
\pgfpathmoveto{\pgfqpoint{2.924169in}{0.524170in}}%
\pgfpathlineto{\pgfqpoint{2.966257in}{0.524170in}}%
\pgfpathlineto{\pgfqpoint{2.966257in}{0.883833in}}%
\pgfpathlineto{\pgfqpoint{2.924169in}{0.883833in}}%
\pgfpathlineto{\pgfqpoint{2.924169in}{0.524170in}}%
\pgfpathclose%
\pgfusepath{fill}%
\end{pgfscope}%
\begin{pgfscope}%
\pgfpathrectangle{\pgfqpoint{0.651412in}{0.524170in}}{\pgfqpoint{4.629690in}{2.558193in}}%
\pgfusepath{clip}%
\pgfsetbuttcap%
\pgfsetmiterjoin%
\definecolor{currentfill}{rgb}{0.007843,0.619608,0.450980}%
\pgfsetfillcolor{currentfill}%
\pgfsetfillopacity{0.700000}%
\pgfsetlinewidth{0.000000pt}%
\definecolor{currentstroke}{rgb}{0.000000,0.000000,0.000000}%
\pgfsetstrokecolor{currentstroke}%
\pgfsetstrokeopacity{0.700000}%
\pgfsetdash{}{0pt}%
\pgfpathmoveto{\pgfqpoint{2.966257in}{0.524170in}}%
\pgfpathlineto{\pgfqpoint{3.008345in}{0.524170in}}%
\pgfpathlineto{\pgfqpoint{3.008345in}{0.950087in}}%
\pgfpathlineto{\pgfqpoint{2.966257in}{0.950087in}}%
\pgfpathlineto{\pgfqpoint{2.966257in}{0.524170in}}%
\pgfpathclose%
\pgfusepath{fill}%
\end{pgfscope}%
\begin{pgfscope}%
\pgfpathrectangle{\pgfqpoint{0.651412in}{0.524170in}}{\pgfqpoint{4.629690in}{2.558193in}}%
\pgfusepath{clip}%
\pgfsetbuttcap%
\pgfsetmiterjoin%
\definecolor{currentfill}{rgb}{0.007843,0.619608,0.450980}%
\pgfsetfillcolor{currentfill}%
\pgfsetfillopacity{0.700000}%
\pgfsetlinewidth{0.000000pt}%
\definecolor{currentstroke}{rgb}{0.000000,0.000000,0.000000}%
\pgfsetstrokecolor{currentstroke}%
\pgfsetstrokeopacity{0.700000}%
\pgfsetdash{}{0pt}%
\pgfpathmoveto{\pgfqpoint{3.008345in}{0.524170in}}%
\pgfpathlineto{\pgfqpoint{3.050433in}{0.524170in}}%
\pgfpathlineto{\pgfqpoint{3.050433in}{1.006875in}}%
\pgfpathlineto{\pgfqpoint{3.008345in}{1.006875in}}%
\pgfpathlineto{\pgfqpoint{3.008345in}{0.524170in}}%
\pgfpathclose%
\pgfusepath{fill}%
\end{pgfscope}%
\begin{pgfscope}%
\pgfpathrectangle{\pgfqpoint{0.651412in}{0.524170in}}{\pgfqpoint{4.629690in}{2.558193in}}%
\pgfusepath{clip}%
\pgfsetbuttcap%
\pgfsetmiterjoin%
\definecolor{currentfill}{rgb}{0.007843,0.619608,0.450980}%
\pgfsetfillcolor{currentfill}%
\pgfsetfillopacity{0.700000}%
\pgfsetlinewidth{0.000000pt}%
\definecolor{currentstroke}{rgb}{0.000000,0.000000,0.000000}%
\pgfsetstrokecolor{currentstroke}%
\pgfsetstrokeopacity{0.700000}%
\pgfsetdash{}{0pt}%
\pgfpathmoveto{\pgfqpoint{3.050433in}{0.524170in}}%
\pgfpathlineto{\pgfqpoint{3.092522in}{0.524170in}}%
\pgfpathlineto{\pgfqpoint{3.092522in}{0.845974in}}%
\pgfpathlineto{\pgfqpoint{3.050433in}{0.845974in}}%
\pgfpathlineto{\pgfqpoint{3.050433in}{0.524170in}}%
\pgfpathclose%
\pgfusepath{fill}%
\end{pgfscope}%
\begin{pgfscope}%
\pgfpathrectangle{\pgfqpoint{0.651412in}{0.524170in}}{\pgfqpoint{4.629690in}{2.558193in}}%
\pgfusepath{clip}%
\pgfsetbuttcap%
\pgfsetmiterjoin%
\definecolor{currentfill}{rgb}{0.007843,0.619608,0.450980}%
\pgfsetfillcolor{currentfill}%
\pgfsetfillopacity{0.700000}%
\pgfsetlinewidth{0.000000pt}%
\definecolor{currentstroke}{rgb}{0.000000,0.000000,0.000000}%
\pgfsetstrokecolor{currentstroke}%
\pgfsetstrokeopacity{0.700000}%
\pgfsetdash{}{0pt}%
\pgfpathmoveto{\pgfqpoint{3.092522in}{0.524170in}}%
\pgfpathlineto{\pgfqpoint{3.134610in}{0.524170in}}%
\pgfpathlineto{\pgfqpoint{3.134610in}{0.864903in}}%
\pgfpathlineto{\pgfqpoint{3.092522in}{0.864903in}}%
\pgfpathlineto{\pgfqpoint{3.092522in}{0.524170in}}%
\pgfpathclose%
\pgfusepath{fill}%
\end{pgfscope}%
\begin{pgfscope}%
\pgfpathrectangle{\pgfqpoint{0.651412in}{0.524170in}}{\pgfqpoint{4.629690in}{2.558193in}}%
\pgfusepath{clip}%
\pgfsetbuttcap%
\pgfsetmiterjoin%
\definecolor{currentfill}{rgb}{0.007843,0.619608,0.450980}%
\pgfsetfillcolor{currentfill}%
\pgfsetfillopacity{0.700000}%
\pgfsetlinewidth{0.000000pt}%
\definecolor{currentstroke}{rgb}{0.000000,0.000000,0.000000}%
\pgfsetstrokecolor{currentstroke}%
\pgfsetstrokeopacity{0.700000}%
\pgfsetdash{}{0pt}%
\pgfpathmoveto{\pgfqpoint{3.134610in}{0.524170in}}%
\pgfpathlineto{\pgfqpoint{3.176698in}{0.524170in}}%
\pgfpathlineto{\pgfqpoint{3.176698in}{0.789185in}}%
\pgfpathlineto{\pgfqpoint{3.134610in}{0.789185in}}%
\pgfpathlineto{\pgfqpoint{3.134610in}{0.524170in}}%
\pgfpathclose%
\pgfusepath{fill}%
\end{pgfscope}%
\begin{pgfscope}%
\pgfpathrectangle{\pgfqpoint{0.651412in}{0.524170in}}{\pgfqpoint{4.629690in}{2.558193in}}%
\pgfusepath{clip}%
\pgfsetbuttcap%
\pgfsetmiterjoin%
\definecolor{currentfill}{rgb}{0.007843,0.619608,0.450980}%
\pgfsetfillcolor{currentfill}%
\pgfsetfillopacity{0.700000}%
\pgfsetlinewidth{0.000000pt}%
\definecolor{currentstroke}{rgb}{0.000000,0.000000,0.000000}%
\pgfsetstrokecolor{currentstroke}%
\pgfsetstrokeopacity{0.700000}%
\pgfsetdash{}{0pt}%
\pgfpathmoveto{\pgfqpoint{3.176698in}{0.524170in}}%
\pgfpathlineto{\pgfqpoint{3.218786in}{0.524170in}}%
\pgfpathlineto{\pgfqpoint{3.218786in}{0.883833in}}%
\pgfpathlineto{\pgfqpoint{3.176698in}{0.883833in}}%
\pgfpathlineto{\pgfqpoint{3.176698in}{0.524170in}}%
\pgfpathclose%
\pgfusepath{fill}%
\end{pgfscope}%
\begin{pgfscope}%
\pgfpathrectangle{\pgfqpoint{0.651412in}{0.524170in}}{\pgfqpoint{4.629690in}{2.558193in}}%
\pgfusepath{clip}%
\pgfsetbuttcap%
\pgfsetmiterjoin%
\definecolor{currentfill}{rgb}{0.007843,0.619608,0.450980}%
\pgfsetfillcolor{currentfill}%
\pgfsetfillopacity{0.700000}%
\pgfsetlinewidth{0.000000pt}%
\definecolor{currentstroke}{rgb}{0.000000,0.000000,0.000000}%
\pgfsetstrokecolor{currentstroke}%
\pgfsetstrokeopacity{0.700000}%
\pgfsetdash{}{0pt}%
\pgfpathmoveto{\pgfqpoint{3.218786in}{0.524170in}}%
\pgfpathlineto{\pgfqpoint{3.260874in}{0.524170in}}%
\pgfpathlineto{\pgfqpoint{3.260874in}{0.931157in}}%
\pgfpathlineto{\pgfqpoint{3.218786in}{0.931157in}}%
\pgfpathlineto{\pgfqpoint{3.218786in}{0.524170in}}%
\pgfpathclose%
\pgfusepath{fill}%
\end{pgfscope}%
\begin{pgfscope}%
\pgfpathrectangle{\pgfqpoint{0.651412in}{0.524170in}}{\pgfqpoint{4.629690in}{2.558193in}}%
\pgfusepath{clip}%
\pgfsetbuttcap%
\pgfsetmiterjoin%
\definecolor{currentfill}{rgb}{0.007843,0.619608,0.450980}%
\pgfsetfillcolor{currentfill}%
\pgfsetfillopacity{0.700000}%
\pgfsetlinewidth{0.000000pt}%
\definecolor{currentstroke}{rgb}{0.000000,0.000000,0.000000}%
\pgfsetstrokecolor{currentstroke}%
\pgfsetstrokeopacity{0.700000}%
\pgfsetdash{}{0pt}%
\pgfpathmoveto{\pgfqpoint{3.260874in}{0.524170in}}%
\pgfpathlineto{\pgfqpoint{3.302962in}{0.524170in}}%
\pgfpathlineto{\pgfqpoint{3.302962in}{0.883833in}}%
\pgfpathlineto{\pgfqpoint{3.260874in}{0.883833in}}%
\pgfpathlineto{\pgfqpoint{3.260874in}{0.524170in}}%
\pgfpathclose%
\pgfusepath{fill}%
\end{pgfscope}%
\begin{pgfscope}%
\pgfpathrectangle{\pgfqpoint{0.651412in}{0.524170in}}{\pgfqpoint{4.629690in}{2.558193in}}%
\pgfusepath{clip}%
\pgfsetbuttcap%
\pgfsetmiterjoin%
\definecolor{currentfill}{rgb}{0.007843,0.619608,0.450980}%
\pgfsetfillcolor{currentfill}%
\pgfsetfillopacity{0.700000}%
\pgfsetlinewidth{0.000000pt}%
\definecolor{currentstroke}{rgb}{0.000000,0.000000,0.000000}%
\pgfsetstrokecolor{currentstroke}%
\pgfsetstrokeopacity{0.700000}%
\pgfsetdash{}{0pt}%
\pgfpathmoveto{\pgfqpoint{3.302962in}{0.524170in}}%
\pgfpathlineto{\pgfqpoint{3.345050in}{0.524170in}}%
\pgfpathlineto{\pgfqpoint{3.345050in}{0.921692in}}%
\pgfpathlineto{\pgfqpoint{3.302962in}{0.921692in}}%
\pgfpathlineto{\pgfqpoint{3.302962in}{0.524170in}}%
\pgfpathclose%
\pgfusepath{fill}%
\end{pgfscope}%
\begin{pgfscope}%
\pgfpathrectangle{\pgfqpoint{0.651412in}{0.524170in}}{\pgfqpoint{4.629690in}{2.558193in}}%
\pgfusepath{clip}%
\pgfsetbuttcap%
\pgfsetmiterjoin%
\definecolor{currentfill}{rgb}{0.007843,0.619608,0.450980}%
\pgfsetfillcolor{currentfill}%
\pgfsetfillopacity{0.700000}%
\pgfsetlinewidth{0.000000pt}%
\definecolor{currentstroke}{rgb}{0.000000,0.000000,0.000000}%
\pgfsetstrokecolor{currentstroke}%
\pgfsetstrokeopacity{0.700000}%
\pgfsetdash{}{0pt}%
\pgfpathmoveto{\pgfqpoint{3.345050in}{0.524170in}}%
\pgfpathlineto{\pgfqpoint{3.387138in}{0.524170in}}%
\pgfpathlineto{\pgfqpoint{3.387138in}{0.931157in}}%
\pgfpathlineto{\pgfqpoint{3.345050in}{0.931157in}}%
\pgfpathlineto{\pgfqpoint{3.345050in}{0.524170in}}%
\pgfpathclose%
\pgfusepath{fill}%
\end{pgfscope}%
\begin{pgfscope}%
\pgfpathrectangle{\pgfqpoint{0.651412in}{0.524170in}}{\pgfqpoint{4.629690in}{2.558193in}}%
\pgfusepath{clip}%
\pgfsetbuttcap%
\pgfsetmiterjoin%
\definecolor{currentfill}{rgb}{0.007843,0.619608,0.450980}%
\pgfsetfillcolor{currentfill}%
\pgfsetfillopacity{0.700000}%
\pgfsetlinewidth{0.000000pt}%
\definecolor{currentstroke}{rgb}{0.000000,0.000000,0.000000}%
\pgfsetstrokecolor{currentstroke}%
\pgfsetstrokeopacity{0.700000}%
\pgfsetdash{}{0pt}%
\pgfpathmoveto{\pgfqpoint{3.387138in}{0.524170in}}%
\pgfpathlineto{\pgfqpoint{3.429226in}{0.524170in}}%
\pgfpathlineto{\pgfqpoint{3.429226in}{0.959551in}}%
\pgfpathlineto{\pgfqpoint{3.387138in}{0.959551in}}%
\pgfpathlineto{\pgfqpoint{3.387138in}{0.524170in}}%
\pgfpathclose%
\pgfusepath{fill}%
\end{pgfscope}%
\begin{pgfscope}%
\pgfpathrectangle{\pgfqpoint{0.651412in}{0.524170in}}{\pgfqpoint{4.629690in}{2.558193in}}%
\pgfusepath{clip}%
\pgfsetbuttcap%
\pgfsetmiterjoin%
\definecolor{currentfill}{rgb}{0.007843,0.619608,0.450980}%
\pgfsetfillcolor{currentfill}%
\pgfsetfillopacity{0.700000}%
\pgfsetlinewidth{0.000000pt}%
\definecolor{currentstroke}{rgb}{0.000000,0.000000,0.000000}%
\pgfsetstrokecolor{currentstroke}%
\pgfsetstrokeopacity{0.700000}%
\pgfsetdash{}{0pt}%
\pgfpathmoveto{\pgfqpoint{3.429226in}{0.524170in}}%
\pgfpathlineto{\pgfqpoint{3.471314in}{0.524170in}}%
\pgfpathlineto{\pgfqpoint{3.471314in}{0.817579in}}%
\pgfpathlineto{\pgfqpoint{3.429226in}{0.817579in}}%
\pgfpathlineto{\pgfqpoint{3.429226in}{0.524170in}}%
\pgfpathclose%
\pgfusepath{fill}%
\end{pgfscope}%
\begin{pgfscope}%
\pgfpathrectangle{\pgfqpoint{0.651412in}{0.524170in}}{\pgfqpoint{4.629690in}{2.558193in}}%
\pgfusepath{clip}%
\pgfsetbuttcap%
\pgfsetmiterjoin%
\definecolor{currentfill}{rgb}{0.007843,0.619608,0.450980}%
\pgfsetfillcolor{currentfill}%
\pgfsetfillopacity{0.700000}%
\pgfsetlinewidth{0.000000pt}%
\definecolor{currentstroke}{rgb}{0.000000,0.000000,0.000000}%
\pgfsetstrokecolor{currentstroke}%
\pgfsetstrokeopacity{0.700000}%
\pgfsetdash{}{0pt}%
\pgfpathmoveto{\pgfqpoint{3.471314in}{0.524170in}}%
\pgfpathlineto{\pgfqpoint{3.513403in}{0.524170in}}%
\pgfpathlineto{\pgfqpoint{3.513403in}{0.845974in}}%
\pgfpathlineto{\pgfqpoint{3.471314in}{0.845974in}}%
\pgfpathlineto{\pgfqpoint{3.471314in}{0.524170in}}%
\pgfpathclose%
\pgfusepath{fill}%
\end{pgfscope}%
\begin{pgfscope}%
\pgfpathrectangle{\pgfqpoint{0.651412in}{0.524170in}}{\pgfqpoint{4.629690in}{2.558193in}}%
\pgfusepath{clip}%
\pgfsetbuttcap%
\pgfsetmiterjoin%
\definecolor{currentfill}{rgb}{0.007843,0.619608,0.450980}%
\pgfsetfillcolor{currentfill}%
\pgfsetfillopacity{0.700000}%
\pgfsetlinewidth{0.000000pt}%
\definecolor{currentstroke}{rgb}{0.000000,0.000000,0.000000}%
\pgfsetstrokecolor{currentstroke}%
\pgfsetstrokeopacity{0.700000}%
\pgfsetdash{}{0pt}%
\pgfpathmoveto{\pgfqpoint{3.513403in}{0.524170in}}%
\pgfpathlineto{\pgfqpoint{3.555491in}{0.524170in}}%
\pgfpathlineto{\pgfqpoint{3.555491in}{0.836509in}}%
\pgfpathlineto{\pgfqpoint{3.513403in}{0.836509in}}%
\pgfpathlineto{\pgfqpoint{3.513403in}{0.524170in}}%
\pgfpathclose%
\pgfusepath{fill}%
\end{pgfscope}%
\begin{pgfscope}%
\pgfpathrectangle{\pgfqpoint{0.651412in}{0.524170in}}{\pgfqpoint{4.629690in}{2.558193in}}%
\pgfusepath{clip}%
\pgfsetbuttcap%
\pgfsetmiterjoin%
\definecolor{currentfill}{rgb}{0.007843,0.619608,0.450980}%
\pgfsetfillcolor{currentfill}%
\pgfsetfillopacity{0.700000}%
\pgfsetlinewidth{0.000000pt}%
\definecolor{currentstroke}{rgb}{0.000000,0.000000,0.000000}%
\pgfsetstrokecolor{currentstroke}%
\pgfsetstrokeopacity{0.700000}%
\pgfsetdash{}{0pt}%
\pgfpathmoveto{\pgfqpoint{3.555491in}{0.524170in}}%
\pgfpathlineto{\pgfqpoint{3.597579in}{0.524170in}}%
\pgfpathlineto{\pgfqpoint{3.597579in}{0.940622in}}%
\pgfpathlineto{\pgfqpoint{3.555491in}{0.940622in}}%
\pgfpathlineto{\pgfqpoint{3.555491in}{0.524170in}}%
\pgfpathclose%
\pgfusepath{fill}%
\end{pgfscope}%
\begin{pgfscope}%
\pgfpathrectangle{\pgfqpoint{0.651412in}{0.524170in}}{\pgfqpoint{4.629690in}{2.558193in}}%
\pgfusepath{clip}%
\pgfsetbuttcap%
\pgfsetmiterjoin%
\definecolor{currentfill}{rgb}{0.007843,0.619608,0.450980}%
\pgfsetfillcolor{currentfill}%
\pgfsetfillopacity{0.700000}%
\pgfsetlinewidth{0.000000pt}%
\definecolor{currentstroke}{rgb}{0.000000,0.000000,0.000000}%
\pgfsetstrokecolor{currentstroke}%
\pgfsetstrokeopacity{0.700000}%
\pgfsetdash{}{0pt}%
\pgfpathmoveto{\pgfqpoint{3.597579in}{0.524170in}}%
\pgfpathlineto{\pgfqpoint{3.639667in}{0.524170in}}%
\pgfpathlineto{\pgfqpoint{3.639667in}{0.741861in}}%
\pgfpathlineto{\pgfqpoint{3.597579in}{0.741861in}}%
\pgfpathlineto{\pgfqpoint{3.597579in}{0.524170in}}%
\pgfpathclose%
\pgfusepath{fill}%
\end{pgfscope}%
\begin{pgfscope}%
\pgfpathrectangle{\pgfqpoint{0.651412in}{0.524170in}}{\pgfqpoint{4.629690in}{2.558193in}}%
\pgfusepath{clip}%
\pgfsetbuttcap%
\pgfsetmiterjoin%
\definecolor{currentfill}{rgb}{0.007843,0.619608,0.450980}%
\pgfsetfillcolor{currentfill}%
\pgfsetfillopacity{0.700000}%
\pgfsetlinewidth{0.000000pt}%
\definecolor{currentstroke}{rgb}{0.000000,0.000000,0.000000}%
\pgfsetstrokecolor{currentstroke}%
\pgfsetstrokeopacity{0.700000}%
\pgfsetdash{}{0pt}%
\pgfpathmoveto{\pgfqpoint{3.639667in}{0.524170in}}%
\pgfpathlineto{\pgfqpoint{3.681755in}{0.524170in}}%
\pgfpathlineto{\pgfqpoint{3.681755in}{0.827044in}}%
\pgfpathlineto{\pgfqpoint{3.639667in}{0.827044in}}%
\pgfpathlineto{\pgfqpoint{3.639667in}{0.524170in}}%
\pgfpathclose%
\pgfusepath{fill}%
\end{pgfscope}%
\begin{pgfscope}%
\pgfpathrectangle{\pgfqpoint{0.651412in}{0.524170in}}{\pgfqpoint{4.629690in}{2.558193in}}%
\pgfusepath{clip}%
\pgfsetbuttcap%
\pgfsetmiterjoin%
\definecolor{currentfill}{rgb}{0.007843,0.619608,0.450980}%
\pgfsetfillcolor{currentfill}%
\pgfsetfillopacity{0.700000}%
\pgfsetlinewidth{0.000000pt}%
\definecolor{currentstroke}{rgb}{0.000000,0.000000,0.000000}%
\pgfsetstrokecolor{currentstroke}%
\pgfsetstrokeopacity{0.700000}%
\pgfsetdash{}{0pt}%
\pgfpathmoveto{\pgfqpoint{3.681755in}{0.524170in}}%
\pgfpathlineto{\pgfqpoint{3.723843in}{0.524170in}}%
\pgfpathlineto{\pgfqpoint{3.723843in}{0.808114in}}%
\pgfpathlineto{\pgfqpoint{3.681755in}{0.808114in}}%
\pgfpathlineto{\pgfqpoint{3.681755in}{0.524170in}}%
\pgfpathclose%
\pgfusepath{fill}%
\end{pgfscope}%
\begin{pgfscope}%
\pgfpathrectangle{\pgfqpoint{0.651412in}{0.524170in}}{\pgfqpoint{4.629690in}{2.558193in}}%
\pgfusepath{clip}%
\pgfsetbuttcap%
\pgfsetmiterjoin%
\definecolor{currentfill}{rgb}{0.007843,0.619608,0.450980}%
\pgfsetfillcolor{currentfill}%
\pgfsetfillopacity{0.700000}%
\pgfsetlinewidth{0.000000pt}%
\definecolor{currentstroke}{rgb}{0.000000,0.000000,0.000000}%
\pgfsetstrokecolor{currentstroke}%
\pgfsetstrokeopacity{0.700000}%
\pgfsetdash{}{0pt}%
\pgfpathmoveto{\pgfqpoint{3.723843in}{0.524170in}}%
\pgfpathlineto{\pgfqpoint{3.765931in}{0.524170in}}%
\pgfpathlineto{\pgfqpoint{3.765931in}{0.789185in}}%
\pgfpathlineto{\pgfqpoint{3.723843in}{0.789185in}}%
\pgfpathlineto{\pgfqpoint{3.723843in}{0.524170in}}%
\pgfpathclose%
\pgfusepath{fill}%
\end{pgfscope}%
\begin{pgfscope}%
\pgfpathrectangle{\pgfqpoint{0.651412in}{0.524170in}}{\pgfqpoint{4.629690in}{2.558193in}}%
\pgfusepath{clip}%
\pgfsetbuttcap%
\pgfsetmiterjoin%
\definecolor{currentfill}{rgb}{0.007843,0.619608,0.450980}%
\pgfsetfillcolor{currentfill}%
\pgfsetfillopacity{0.700000}%
\pgfsetlinewidth{0.000000pt}%
\definecolor{currentstroke}{rgb}{0.000000,0.000000,0.000000}%
\pgfsetstrokecolor{currentstroke}%
\pgfsetstrokeopacity{0.700000}%
\pgfsetdash{}{0pt}%
\pgfpathmoveto{\pgfqpoint{3.765931in}{0.524170in}}%
\pgfpathlineto{\pgfqpoint{3.808019in}{0.524170in}}%
\pgfpathlineto{\pgfqpoint{3.808019in}{0.760790in}}%
\pgfpathlineto{\pgfqpoint{3.765931in}{0.760790in}}%
\pgfpathlineto{\pgfqpoint{3.765931in}{0.524170in}}%
\pgfpathclose%
\pgfusepath{fill}%
\end{pgfscope}%
\begin{pgfscope}%
\pgfpathrectangle{\pgfqpoint{0.651412in}{0.524170in}}{\pgfqpoint{4.629690in}{2.558193in}}%
\pgfusepath{clip}%
\pgfsetbuttcap%
\pgfsetmiterjoin%
\definecolor{currentfill}{rgb}{0.007843,0.619608,0.450980}%
\pgfsetfillcolor{currentfill}%
\pgfsetfillopacity{0.700000}%
\pgfsetlinewidth{0.000000pt}%
\definecolor{currentstroke}{rgb}{0.000000,0.000000,0.000000}%
\pgfsetstrokecolor{currentstroke}%
\pgfsetstrokeopacity{0.700000}%
\pgfsetdash{}{0pt}%
\pgfpathmoveto{\pgfqpoint{3.808019in}{0.524170in}}%
\pgfpathlineto{\pgfqpoint{3.850107in}{0.524170in}}%
\pgfpathlineto{\pgfqpoint{3.850107in}{0.893298in}}%
\pgfpathlineto{\pgfqpoint{3.808019in}{0.893298in}}%
\pgfpathlineto{\pgfqpoint{3.808019in}{0.524170in}}%
\pgfpathclose%
\pgfusepath{fill}%
\end{pgfscope}%
\begin{pgfscope}%
\pgfpathrectangle{\pgfqpoint{0.651412in}{0.524170in}}{\pgfqpoint{4.629690in}{2.558193in}}%
\pgfusepath{clip}%
\pgfsetbuttcap%
\pgfsetmiterjoin%
\definecolor{currentfill}{rgb}{0.007843,0.619608,0.450980}%
\pgfsetfillcolor{currentfill}%
\pgfsetfillopacity{0.700000}%
\pgfsetlinewidth{0.000000pt}%
\definecolor{currentstroke}{rgb}{0.000000,0.000000,0.000000}%
\pgfsetstrokecolor{currentstroke}%
\pgfsetstrokeopacity{0.700000}%
\pgfsetdash{}{0pt}%
\pgfpathmoveto{\pgfqpoint{3.850107in}{0.524170in}}%
\pgfpathlineto{\pgfqpoint{3.892195in}{0.524170in}}%
\pgfpathlineto{\pgfqpoint{3.892195in}{0.855438in}}%
\pgfpathlineto{\pgfqpoint{3.850107in}{0.855438in}}%
\pgfpathlineto{\pgfqpoint{3.850107in}{0.524170in}}%
\pgfpathclose%
\pgfusepath{fill}%
\end{pgfscope}%
\begin{pgfscope}%
\pgfpathrectangle{\pgfqpoint{0.651412in}{0.524170in}}{\pgfqpoint{4.629690in}{2.558193in}}%
\pgfusepath{clip}%
\pgfsetbuttcap%
\pgfsetmiterjoin%
\definecolor{currentfill}{rgb}{0.007843,0.619608,0.450980}%
\pgfsetfillcolor{currentfill}%
\pgfsetfillopacity{0.700000}%
\pgfsetlinewidth{0.000000pt}%
\definecolor{currentstroke}{rgb}{0.000000,0.000000,0.000000}%
\pgfsetstrokecolor{currentstroke}%
\pgfsetstrokeopacity{0.700000}%
\pgfsetdash{}{0pt}%
\pgfpathmoveto{\pgfqpoint{3.892195in}{0.524170in}}%
\pgfpathlineto{\pgfqpoint{3.934283in}{0.524170in}}%
\pgfpathlineto{\pgfqpoint{3.934283in}{0.751325in}}%
\pgfpathlineto{\pgfqpoint{3.892195in}{0.751325in}}%
\pgfpathlineto{\pgfqpoint{3.892195in}{0.524170in}}%
\pgfpathclose%
\pgfusepath{fill}%
\end{pgfscope}%
\begin{pgfscope}%
\pgfpathrectangle{\pgfqpoint{0.651412in}{0.524170in}}{\pgfqpoint{4.629690in}{2.558193in}}%
\pgfusepath{clip}%
\pgfsetbuttcap%
\pgfsetmiterjoin%
\definecolor{currentfill}{rgb}{0.007843,0.619608,0.450980}%
\pgfsetfillcolor{currentfill}%
\pgfsetfillopacity{0.700000}%
\pgfsetlinewidth{0.000000pt}%
\definecolor{currentstroke}{rgb}{0.000000,0.000000,0.000000}%
\pgfsetstrokecolor{currentstroke}%
\pgfsetstrokeopacity{0.700000}%
\pgfsetdash{}{0pt}%
\pgfpathmoveto{\pgfqpoint{3.934283in}{0.524170in}}%
\pgfpathlineto{\pgfqpoint{3.976372in}{0.524170in}}%
\pgfpathlineto{\pgfqpoint{3.976372in}{0.808114in}}%
\pgfpathlineto{\pgfqpoint{3.934283in}{0.808114in}}%
\pgfpathlineto{\pgfqpoint{3.934283in}{0.524170in}}%
\pgfpathclose%
\pgfusepath{fill}%
\end{pgfscope}%
\begin{pgfscope}%
\pgfpathrectangle{\pgfqpoint{0.651412in}{0.524170in}}{\pgfqpoint{4.629690in}{2.558193in}}%
\pgfusepath{clip}%
\pgfsetbuttcap%
\pgfsetmiterjoin%
\definecolor{currentfill}{rgb}{0.007843,0.619608,0.450980}%
\pgfsetfillcolor{currentfill}%
\pgfsetfillopacity{0.700000}%
\pgfsetlinewidth{0.000000pt}%
\definecolor{currentstroke}{rgb}{0.000000,0.000000,0.000000}%
\pgfsetstrokecolor{currentstroke}%
\pgfsetstrokeopacity{0.700000}%
\pgfsetdash{}{0pt}%
\pgfpathmoveto{\pgfqpoint{3.976372in}{0.524170in}}%
\pgfpathlineto{\pgfqpoint{4.018460in}{0.524170in}}%
\pgfpathlineto{\pgfqpoint{4.018460in}{0.789185in}}%
\pgfpathlineto{\pgfqpoint{3.976372in}{0.789185in}}%
\pgfpathlineto{\pgfqpoint{3.976372in}{0.524170in}}%
\pgfpathclose%
\pgfusepath{fill}%
\end{pgfscope}%
\begin{pgfscope}%
\pgfpathrectangle{\pgfqpoint{0.651412in}{0.524170in}}{\pgfqpoint{4.629690in}{2.558193in}}%
\pgfusepath{clip}%
\pgfsetbuttcap%
\pgfsetmiterjoin%
\definecolor{currentfill}{rgb}{0.007843,0.619608,0.450980}%
\pgfsetfillcolor{currentfill}%
\pgfsetfillopacity{0.700000}%
\pgfsetlinewidth{0.000000pt}%
\definecolor{currentstroke}{rgb}{0.000000,0.000000,0.000000}%
\pgfsetstrokecolor{currentstroke}%
\pgfsetstrokeopacity{0.700000}%
\pgfsetdash{}{0pt}%
\pgfpathmoveto{\pgfqpoint{4.018460in}{0.524170in}}%
\pgfpathlineto{\pgfqpoint{4.060548in}{0.524170in}}%
\pgfpathlineto{\pgfqpoint{4.060548in}{0.685072in}}%
\pgfpathlineto{\pgfqpoint{4.018460in}{0.685072in}}%
\pgfpathlineto{\pgfqpoint{4.018460in}{0.524170in}}%
\pgfpathclose%
\pgfusepath{fill}%
\end{pgfscope}%
\begin{pgfscope}%
\pgfpathrectangle{\pgfqpoint{0.651412in}{0.524170in}}{\pgfqpoint{4.629690in}{2.558193in}}%
\pgfusepath{clip}%
\pgfsetbuttcap%
\pgfsetmiterjoin%
\definecolor{currentfill}{rgb}{0.007843,0.619608,0.450980}%
\pgfsetfillcolor{currentfill}%
\pgfsetfillopacity{0.700000}%
\pgfsetlinewidth{0.000000pt}%
\definecolor{currentstroke}{rgb}{0.000000,0.000000,0.000000}%
\pgfsetstrokecolor{currentstroke}%
\pgfsetstrokeopacity{0.700000}%
\pgfsetdash{}{0pt}%
\pgfpathmoveto{\pgfqpoint{4.060548in}{0.524170in}}%
\pgfpathlineto{\pgfqpoint{4.102636in}{0.524170in}}%
\pgfpathlineto{\pgfqpoint{4.102636in}{0.704001in}}%
\pgfpathlineto{\pgfqpoint{4.060548in}{0.704001in}}%
\pgfpathlineto{\pgfqpoint{4.060548in}{0.524170in}}%
\pgfpathclose%
\pgfusepath{fill}%
\end{pgfscope}%
\begin{pgfscope}%
\pgfpathrectangle{\pgfqpoint{0.651412in}{0.524170in}}{\pgfqpoint{4.629690in}{2.558193in}}%
\pgfusepath{clip}%
\pgfsetbuttcap%
\pgfsetmiterjoin%
\definecolor{currentfill}{rgb}{0.007843,0.619608,0.450980}%
\pgfsetfillcolor{currentfill}%
\pgfsetfillopacity{0.700000}%
\pgfsetlinewidth{0.000000pt}%
\definecolor{currentstroke}{rgb}{0.000000,0.000000,0.000000}%
\pgfsetstrokecolor{currentstroke}%
\pgfsetstrokeopacity{0.700000}%
\pgfsetdash{}{0pt}%
\pgfpathmoveto{\pgfqpoint{4.102636in}{0.524170in}}%
\pgfpathlineto{\pgfqpoint{4.144724in}{0.524170in}}%
\pgfpathlineto{\pgfqpoint{4.144724in}{0.798649in}}%
\pgfpathlineto{\pgfqpoint{4.102636in}{0.798649in}}%
\pgfpathlineto{\pgfqpoint{4.102636in}{0.524170in}}%
\pgfpathclose%
\pgfusepath{fill}%
\end{pgfscope}%
\begin{pgfscope}%
\pgfpathrectangle{\pgfqpoint{0.651412in}{0.524170in}}{\pgfqpoint{4.629690in}{2.558193in}}%
\pgfusepath{clip}%
\pgfsetbuttcap%
\pgfsetmiterjoin%
\definecolor{currentfill}{rgb}{0.007843,0.619608,0.450980}%
\pgfsetfillcolor{currentfill}%
\pgfsetfillopacity{0.700000}%
\pgfsetlinewidth{0.000000pt}%
\definecolor{currentstroke}{rgb}{0.000000,0.000000,0.000000}%
\pgfsetstrokecolor{currentstroke}%
\pgfsetstrokeopacity{0.700000}%
\pgfsetdash{}{0pt}%
\pgfpathmoveto{\pgfqpoint{4.144724in}{0.524170in}}%
\pgfpathlineto{\pgfqpoint{4.186812in}{0.524170in}}%
\pgfpathlineto{\pgfqpoint{4.186812in}{0.751325in}}%
\pgfpathlineto{\pgfqpoint{4.144724in}{0.751325in}}%
\pgfpathlineto{\pgfqpoint{4.144724in}{0.524170in}}%
\pgfpathclose%
\pgfusepath{fill}%
\end{pgfscope}%
\begin{pgfscope}%
\pgfpathrectangle{\pgfqpoint{0.651412in}{0.524170in}}{\pgfqpoint{4.629690in}{2.558193in}}%
\pgfusepath{clip}%
\pgfsetbuttcap%
\pgfsetmiterjoin%
\definecolor{currentfill}{rgb}{0.007843,0.619608,0.450980}%
\pgfsetfillcolor{currentfill}%
\pgfsetfillopacity{0.700000}%
\pgfsetlinewidth{0.000000pt}%
\definecolor{currentstroke}{rgb}{0.000000,0.000000,0.000000}%
\pgfsetstrokecolor{currentstroke}%
\pgfsetstrokeopacity{0.700000}%
\pgfsetdash{}{0pt}%
\pgfpathmoveto{\pgfqpoint{4.186812in}{0.524170in}}%
\pgfpathlineto{\pgfqpoint{4.228900in}{0.524170in}}%
\pgfpathlineto{\pgfqpoint{4.228900in}{0.808114in}}%
\pgfpathlineto{\pgfqpoint{4.186812in}{0.808114in}}%
\pgfpathlineto{\pgfqpoint{4.186812in}{0.524170in}}%
\pgfpathclose%
\pgfusepath{fill}%
\end{pgfscope}%
\begin{pgfscope}%
\pgfpathrectangle{\pgfqpoint{0.651412in}{0.524170in}}{\pgfqpoint{4.629690in}{2.558193in}}%
\pgfusepath{clip}%
\pgfsetbuttcap%
\pgfsetmiterjoin%
\definecolor{currentfill}{rgb}{0.007843,0.619608,0.450980}%
\pgfsetfillcolor{currentfill}%
\pgfsetfillopacity{0.700000}%
\pgfsetlinewidth{0.000000pt}%
\definecolor{currentstroke}{rgb}{0.000000,0.000000,0.000000}%
\pgfsetstrokecolor{currentstroke}%
\pgfsetstrokeopacity{0.700000}%
\pgfsetdash{}{0pt}%
\pgfpathmoveto{\pgfqpoint{4.228900in}{0.524170in}}%
\pgfpathlineto{\pgfqpoint{4.270988in}{0.524170in}}%
\pgfpathlineto{\pgfqpoint{4.270988in}{0.845974in}}%
\pgfpathlineto{\pgfqpoint{4.228900in}{0.845974in}}%
\pgfpathlineto{\pgfqpoint{4.228900in}{0.524170in}}%
\pgfpathclose%
\pgfusepath{fill}%
\end{pgfscope}%
\begin{pgfscope}%
\pgfpathrectangle{\pgfqpoint{0.651412in}{0.524170in}}{\pgfqpoint{4.629690in}{2.558193in}}%
\pgfusepath{clip}%
\pgfsetbuttcap%
\pgfsetmiterjoin%
\definecolor{currentfill}{rgb}{0.007843,0.619608,0.450980}%
\pgfsetfillcolor{currentfill}%
\pgfsetfillopacity{0.700000}%
\pgfsetlinewidth{0.000000pt}%
\definecolor{currentstroke}{rgb}{0.000000,0.000000,0.000000}%
\pgfsetstrokecolor{currentstroke}%
\pgfsetstrokeopacity{0.700000}%
\pgfsetdash{}{0pt}%
\pgfpathmoveto{\pgfqpoint{4.270988in}{0.524170in}}%
\pgfpathlineto{\pgfqpoint{4.313076in}{0.524170in}}%
\pgfpathlineto{\pgfqpoint{4.313076in}{0.770255in}}%
\pgfpathlineto{\pgfqpoint{4.270988in}{0.770255in}}%
\pgfpathlineto{\pgfqpoint{4.270988in}{0.524170in}}%
\pgfpathclose%
\pgfusepath{fill}%
\end{pgfscope}%
\begin{pgfscope}%
\pgfpathrectangle{\pgfqpoint{0.651412in}{0.524170in}}{\pgfqpoint{4.629690in}{2.558193in}}%
\pgfusepath{clip}%
\pgfsetbuttcap%
\pgfsetmiterjoin%
\definecolor{currentfill}{rgb}{0.007843,0.619608,0.450980}%
\pgfsetfillcolor{currentfill}%
\pgfsetfillopacity{0.700000}%
\pgfsetlinewidth{0.000000pt}%
\definecolor{currentstroke}{rgb}{0.000000,0.000000,0.000000}%
\pgfsetstrokecolor{currentstroke}%
\pgfsetstrokeopacity{0.700000}%
\pgfsetdash{}{0pt}%
\pgfpathmoveto{\pgfqpoint{4.313076in}{0.524170in}}%
\pgfpathlineto{\pgfqpoint{4.355164in}{0.524170in}}%
\pgfpathlineto{\pgfqpoint{4.355164in}{0.798649in}}%
\pgfpathlineto{\pgfqpoint{4.313076in}{0.798649in}}%
\pgfpathlineto{\pgfqpoint{4.313076in}{0.524170in}}%
\pgfpathclose%
\pgfusepath{fill}%
\end{pgfscope}%
\begin{pgfscope}%
\pgfpathrectangle{\pgfqpoint{0.651412in}{0.524170in}}{\pgfqpoint{4.629690in}{2.558193in}}%
\pgfusepath{clip}%
\pgfsetbuttcap%
\pgfsetmiterjoin%
\definecolor{currentfill}{rgb}{0.007843,0.619608,0.450980}%
\pgfsetfillcolor{currentfill}%
\pgfsetfillopacity{0.700000}%
\pgfsetlinewidth{0.000000pt}%
\definecolor{currentstroke}{rgb}{0.000000,0.000000,0.000000}%
\pgfsetstrokecolor{currentstroke}%
\pgfsetstrokeopacity{0.700000}%
\pgfsetdash{}{0pt}%
\pgfpathmoveto{\pgfqpoint{4.355164in}{0.524170in}}%
\pgfpathlineto{\pgfqpoint{4.397253in}{0.524170in}}%
\pgfpathlineto{\pgfqpoint{4.397253in}{0.760790in}}%
\pgfpathlineto{\pgfqpoint{4.355164in}{0.760790in}}%
\pgfpathlineto{\pgfqpoint{4.355164in}{0.524170in}}%
\pgfpathclose%
\pgfusepath{fill}%
\end{pgfscope}%
\begin{pgfscope}%
\pgfpathrectangle{\pgfqpoint{0.651412in}{0.524170in}}{\pgfqpoint{4.629690in}{2.558193in}}%
\pgfusepath{clip}%
\pgfsetbuttcap%
\pgfsetmiterjoin%
\definecolor{currentfill}{rgb}{0.007843,0.619608,0.450980}%
\pgfsetfillcolor{currentfill}%
\pgfsetfillopacity{0.700000}%
\pgfsetlinewidth{0.000000pt}%
\definecolor{currentstroke}{rgb}{0.000000,0.000000,0.000000}%
\pgfsetstrokecolor{currentstroke}%
\pgfsetstrokeopacity{0.700000}%
\pgfsetdash{}{0pt}%
\pgfpathmoveto{\pgfqpoint{4.397253in}{0.524170in}}%
\pgfpathlineto{\pgfqpoint{4.439341in}{0.524170in}}%
\pgfpathlineto{\pgfqpoint{4.439341in}{0.827044in}}%
\pgfpathlineto{\pgfqpoint{4.397253in}{0.827044in}}%
\pgfpathlineto{\pgfqpoint{4.397253in}{0.524170in}}%
\pgfpathclose%
\pgfusepath{fill}%
\end{pgfscope}%
\begin{pgfscope}%
\pgfpathrectangle{\pgfqpoint{0.651412in}{0.524170in}}{\pgfqpoint{4.629690in}{2.558193in}}%
\pgfusepath{clip}%
\pgfsetbuttcap%
\pgfsetmiterjoin%
\definecolor{currentfill}{rgb}{0.007843,0.619608,0.450980}%
\pgfsetfillcolor{currentfill}%
\pgfsetfillopacity{0.700000}%
\pgfsetlinewidth{0.000000pt}%
\definecolor{currentstroke}{rgb}{0.000000,0.000000,0.000000}%
\pgfsetstrokecolor{currentstroke}%
\pgfsetstrokeopacity{0.700000}%
\pgfsetdash{}{0pt}%
\pgfpathmoveto{\pgfqpoint{4.439341in}{0.524170in}}%
\pgfpathlineto{\pgfqpoint{4.481429in}{0.524170in}}%
\pgfpathlineto{\pgfqpoint{4.481429in}{0.751325in}}%
\pgfpathlineto{\pgfqpoint{4.439341in}{0.751325in}}%
\pgfpathlineto{\pgfqpoint{4.439341in}{0.524170in}}%
\pgfpathclose%
\pgfusepath{fill}%
\end{pgfscope}%
\begin{pgfscope}%
\pgfpathrectangle{\pgfqpoint{0.651412in}{0.524170in}}{\pgfqpoint{4.629690in}{2.558193in}}%
\pgfusepath{clip}%
\pgfsetbuttcap%
\pgfsetmiterjoin%
\definecolor{currentfill}{rgb}{0.007843,0.619608,0.450980}%
\pgfsetfillcolor{currentfill}%
\pgfsetfillopacity{0.700000}%
\pgfsetlinewidth{0.000000pt}%
\definecolor{currentstroke}{rgb}{0.000000,0.000000,0.000000}%
\pgfsetstrokecolor{currentstroke}%
\pgfsetstrokeopacity{0.700000}%
\pgfsetdash{}{0pt}%
\pgfpathmoveto{\pgfqpoint{4.481429in}{0.524170in}}%
\pgfpathlineto{\pgfqpoint{4.523517in}{0.524170in}}%
\pgfpathlineto{\pgfqpoint{4.523517in}{0.760790in}}%
\pgfpathlineto{\pgfqpoint{4.481429in}{0.760790in}}%
\pgfpathlineto{\pgfqpoint{4.481429in}{0.524170in}}%
\pgfpathclose%
\pgfusepath{fill}%
\end{pgfscope}%
\begin{pgfscope}%
\pgfpathrectangle{\pgfqpoint{0.651412in}{0.524170in}}{\pgfqpoint{4.629690in}{2.558193in}}%
\pgfusepath{clip}%
\pgfsetbuttcap%
\pgfsetmiterjoin%
\definecolor{currentfill}{rgb}{0.007843,0.619608,0.450980}%
\pgfsetfillcolor{currentfill}%
\pgfsetfillopacity{0.700000}%
\pgfsetlinewidth{0.000000pt}%
\definecolor{currentstroke}{rgb}{0.000000,0.000000,0.000000}%
\pgfsetstrokecolor{currentstroke}%
\pgfsetstrokeopacity{0.700000}%
\pgfsetdash{}{0pt}%
\pgfpathmoveto{\pgfqpoint{4.523517in}{0.524170in}}%
\pgfpathlineto{\pgfqpoint{4.565605in}{0.524170in}}%
\pgfpathlineto{\pgfqpoint{4.565605in}{0.751325in}}%
\pgfpathlineto{\pgfqpoint{4.523517in}{0.751325in}}%
\pgfpathlineto{\pgfqpoint{4.523517in}{0.524170in}}%
\pgfpathclose%
\pgfusepath{fill}%
\end{pgfscope}%
\begin{pgfscope}%
\pgfpathrectangle{\pgfqpoint{0.651412in}{0.524170in}}{\pgfqpoint{4.629690in}{2.558193in}}%
\pgfusepath{clip}%
\pgfsetbuttcap%
\pgfsetmiterjoin%
\definecolor{currentfill}{rgb}{0.007843,0.619608,0.450980}%
\pgfsetfillcolor{currentfill}%
\pgfsetfillopacity{0.700000}%
\pgfsetlinewidth{0.000000pt}%
\definecolor{currentstroke}{rgb}{0.000000,0.000000,0.000000}%
\pgfsetstrokecolor{currentstroke}%
\pgfsetstrokeopacity{0.700000}%
\pgfsetdash{}{0pt}%
\pgfpathmoveto{\pgfqpoint{4.565605in}{0.524170in}}%
\pgfpathlineto{\pgfqpoint{4.607693in}{0.524170in}}%
\pgfpathlineto{\pgfqpoint{4.607693in}{0.685072in}}%
\pgfpathlineto{\pgfqpoint{4.565605in}{0.685072in}}%
\pgfpathlineto{\pgfqpoint{4.565605in}{0.524170in}}%
\pgfpathclose%
\pgfusepath{fill}%
\end{pgfscope}%
\begin{pgfscope}%
\pgfpathrectangle{\pgfqpoint{0.651412in}{0.524170in}}{\pgfqpoint{4.629690in}{2.558193in}}%
\pgfusepath{clip}%
\pgfsetbuttcap%
\pgfsetmiterjoin%
\definecolor{currentfill}{rgb}{0.007843,0.619608,0.450980}%
\pgfsetfillcolor{currentfill}%
\pgfsetfillopacity{0.700000}%
\pgfsetlinewidth{0.000000pt}%
\definecolor{currentstroke}{rgb}{0.000000,0.000000,0.000000}%
\pgfsetstrokecolor{currentstroke}%
\pgfsetstrokeopacity{0.700000}%
\pgfsetdash{}{0pt}%
\pgfpathmoveto{\pgfqpoint{4.607693in}{0.524170in}}%
\pgfpathlineto{\pgfqpoint{4.649781in}{0.524170in}}%
\pgfpathlineto{\pgfqpoint{4.649781in}{0.704001in}}%
\pgfpathlineto{\pgfqpoint{4.607693in}{0.704001in}}%
\pgfpathlineto{\pgfqpoint{4.607693in}{0.524170in}}%
\pgfpathclose%
\pgfusepath{fill}%
\end{pgfscope}%
\begin{pgfscope}%
\pgfpathrectangle{\pgfqpoint{0.651412in}{0.524170in}}{\pgfqpoint{4.629690in}{2.558193in}}%
\pgfusepath{clip}%
\pgfsetbuttcap%
\pgfsetmiterjoin%
\definecolor{currentfill}{rgb}{0.007843,0.619608,0.450980}%
\pgfsetfillcolor{currentfill}%
\pgfsetfillopacity{0.700000}%
\pgfsetlinewidth{0.000000pt}%
\definecolor{currentstroke}{rgb}{0.000000,0.000000,0.000000}%
\pgfsetstrokecolor{currentstroke}%
\pgfsetstrokeopacity{0.700000}%
\pgfsetdash{}{0pt}%
\pgfpathmoveto{\pgfqpoint{4.649781in}{0.524170in}}%
\pgfpathlineto{\pgfqpoint{4.691869in}{0.524170in}}%
\pgfpathlineto{\pgfqpoint{4.691869in}{0.722931in}}%
\pgfpathlineto{\pgfqpoint{4.649781in}{0.722931in}}%
\pgfpathlineto{\pgfqpoint{4.649781in}{0.524170in}}%
\pgfpathclose%
\pgfusepath{fill}%
\end{pgfscope}%
\begin{pgfscope}%
\pgfpathrectangle{\pgfqpoint{0.651412in}{0.524170in}}{\pgfqpoint{4.629690in}{2.558193in}}%
\pgfusepath{clip}%
\pgfsetbuttcap%
\pgfsetmiterjoin%
\definecolor{currentfill}{rgb}{0.007843,0.619608,0.450980}%
\pgfsetfillcolor{currentfill}%
\pgfsetfillopacity{0.700000}%
\pgfsetlinewidth{0.000000pt}%
\definecolor{currentstroke}{rgb}{0.000000,0.000000,0.000000}%
\pgfsetstrokecolor{currentstroke}%
\pgfsetstrokeopacity{0.700000}%
\pgfsetdash{}{0pt}%
\pgfpathmoveto{\pgfqpoint{4.691869in}{0.524170in}}%
\pgfpathlineto{\pgfqpoint{4.733957in}{0.524170in}}%
\pgfpathlineto{\pgfqpoint{4.733957in}{0.713466in}}%
\pgfpathlineto{\pgfqpoint{4.691869in}{0.713466in}}%
\pgfpathlineto{\pgfqpoint{4.691869in}{0.524170in}}%
\pgfpathclose%
\pgfusepath{fill}%
\end{pgfscope}%
\begin{pgfscope}%
\pgfpathrectangle{\pgfqpoint{0.651412in}{0.524170in}}{\pgfqpoint{4.629690in}{2.558193in}}%
\pgfusepath{clip}%
\pgfsetbuttcap%
\pgfsetmiterjoin%
\definecolor{currentfill}{rgb}{0.007843,0.619608,0.450980}%
\pgfsetfillcolor{currentfill}%
\pgfsetfillopacity{0.700000}%
\pgfsetlinewidth{0.000000pt}%
\definecolor{currentstroke}{rgb}{0.000000,0.000000,0.000000}%
\pgfsetstrokecolor{currentstroke}%
\pgfsetstrokeopacity{0.700000}%
\pgfsetdash{}{0pt}%
\pgfpathmoveto{\pgfqpoint{4.733957in}{0.524170in}}%
\pgfpathlineto{\pgfqpoint{4.776045in}{0.524170in}}%
\pgfpathlineto{\pgfqpoint{4.776045in}{0.722931in}}%
\pgfpathlineto{\pgfqpoint{4.733957in}{0.722931in}}%
\pgfpathlineto{\pgfqpoint{4.733957in}{0.524170in}}%
\pgfpathclose%
\pgfusepath{fill}%
\end{pgfscope}%
\begin{pgfscope}%
\pgfpathrectangle{\pgfqpoint{0.651412in}{0.524170in}}{\pgfqpoint{4.629690in}{2.558193in}}%
\pgfusepath{clip}%
\pgfsetbuttcap%
\pgfsetmiterjoin%
\definecolor{currentfill}{rgb}{0.007843,0.619608,0.450980}%
\pgfsetfillcolor{currentfill}%
\pgfsetfillopacity{0.700000}%
\pgfsetlinewidth{0.000000pt}%
\definecolor{currentstroke}{rgb}{0.000000,0.000000,0.000000}%
\pgfsetstrokecolor{currentstroke}%
\pgfsetstrokeopacity{0.700000}%
\pgfsetdash{}{0pt}%
\pgfpathmoveto{\pgfqpoint{4.776045in}{0.524170in}}%
\pgfpathlineto{\pgfqpoint{4.818133in}{0.524170in}}%
\pgfpathlineto{\pgfqpoint{4.818133in}{0.789185in}}%
\pgfpathlineto{\pgfqpoint{4.776045in}{0.789185in}}%
\pgfpathlineto{\pgfqpoint{4.776045in}{0.524170in}}%
\pgfpathclose%
\pgfusepath{fill}%
\end{pgfscope}%
\begin{pgfscope}%
\pgfpathrectangle{\pgfqpoint{0.651412in}{0.524170in}}{\pgfqpoint{4.629690in}{2.558193in}}%
\pgfusepath{clip}%
\pgfsetbuttcap%
\pgfsetmiterjoin%
\definecolor{currentfill}{rgb}{0.007843,0.619608,0.450980}%
\pgfsetfillcolor{currentfill}%
\pgfsetfillopacity{0.700000}%
\pgfsetlinewidth{0.000000pt}%
\definecolor{currentstroke}{rgb}{0.000000,0.000000,0.000000}%
\pgfsetstrokecolor{currentstroke}%
\pgfsetstrokeopacity{0.700000}%
\pgfsetdash{}{0pt}%
\pgfpathmoveto{\pgfqpoint{4.818133in}{0.524170in}}%
\pgfpathlineto{\pgfqpoint{4.860222in}{0.524170in}}%
\pgfpathlineto{\pgfqpoint{4.860222in}{0.704001in}}%
\pgfpathlineto{\pgfqpoint{4.818133in}{0.704001in}}%
\pgfpathlineto{\pgfqpoint{4.818133in}{0.524170in}}%
\pgfpathclose%
\pgfusepath{fill}%
\end{pgfscope}%
\begin{pgfscope}%
\pgfpathrectangle{\pgfqpoint{0.651412in}{0.524170in}}{\pgfqpoint{4.629690in}{2.558193in}}%
\pgfusepath{clip}%
\pgfsetbuttcap%
\pgfsetmiterjoin%
\definecolor{currentfill}{rgb}{0.007843,0.619608,0.450980}%
\pgfsetfillcolor{currentfill}%
\pgfsetfillopacity{0.700000}%
\pgfsetlinewidth{0.000000pt}%
\definecolor{currentstroke}{rgb}{0.000000,0.000000,0.000000}%
\pgfsetstrokecolor{currentstroke}%
\pgfsetstrokeopacity{0.700000}%
\pgfsetdash{}{0pt}%
\pgfpathmoveto{\pgfqpoint{4.860222in}{0.524170in}}%
\pgfpathlineto{\pgfqpoint{4.902310in}{0.524170in}}%
\pgfpathlineto{\pgfqpoint{4.902310in}{0.732396in}}%
\pgfpathlineto{\pgfqpoint{4.860222in}{0.732396in}}%
\pgfpathlineto{\pgfqpoint{4.860222in}{0.524170in}}%
\pgfpathclose%
\pgfusepath{fill}%
\end{pgfscope}%
\begin{pgfscope}%
\pgfpathrectangle{\pgfqpoint{0.651412in}{0.524170in}}{\pgfqpoint{4.629690in}{2.558193in}}%
\pgfusepath{clip}%
\pgfsetbuttcap%
\pgfsetmiterjoin%
\definecolor{currentfill}{rgb}{0.007843,0.619608,0.450980}%
\pgfsetfillcolor{currentfill}%
\pgfsetfillopacity{0.700000}%
\pgfsetlinewidth{0.000000pt}%
\definecolor{currentstroke}{rgb}{0.000000,0.000000,0.000000}%
\pgfsetstrokecolor{currentstroke}%
\pgfsetstrokeopacity{0.700000}%
\pgfsetdash{}{0pt}%
\pgfpathmoveto{\pgfqpoint{4.902310in}{0.524170in}}%
\pgfpathlineto{\pgfqpoint{4.944398in}{0.524170in}}%
\pgfpathlineto{\pgfqpoint{4.944398in}{0.770255in}}%
\pgfpathlineto{\pgfqpoint{4.902310in}{0.770255in}}%
\pgfpathlineto{\pgfqpoint{4.902310in}{0.524170in}}%
\pgfpathclose%
\pgfusepath{fill}%
\end{pgfscope}%
\begin{pgfscope}%
\pgfpathrectangle{\pgfqpoint{0.651412in}{0.524170in}}{\pgfqpoint{4.629690in}{2.558193in}}%
\pgfusepath{clip}%
\pgfsetbuttcap%
\pgfsetmiterjoin%
\definecolor{currentfill}{rgb}{0.007843,0.619608,0.450980}%
\pgfsetfillcolor{currentfill}%
\pgfsetfillopacity{0.700000}%
\pgfsetlinewidth{0.000000pt}%
\definecolor{currentstroke}{rgb}{0.000000,0.000000,0.000000}%
\pgfsetstrokecolor{currentstroke}%
\pgfsetstrokeopacity{0.700000}%
\pgfsetdash{}{0pt}%
\pgfpathmoveto{\pgfqpoint{4.944398in}{0.524170in}}%
\pgfpathlineto{\pgfqpoint{4.986486in}{0.524170in}}%
\pgfpathlineto{\pgfqpoint{4.986486in}{0.732396in}}%
\pgfpathlineto{\pgfqpoint{4.944398in}{0.732396in}}%
\pgfpathlineto{\pgfqpoint{4.944398in}{0.524170in}}%
\pgfpathclose%
\pgfusepath{fill}%
\end{pgfscope}%
\begin{pgfscope}%
\pgfpathrectangle{\pgfqpoint{0.651412in}{0.524170in}}{\pgfqpoint{4.629690in}{2.558193in}}%
\pgfusepath{clip}%
\pgfsetbuttcap%
\pgfsetmiterjoin%
\definecolor{currentfill}{rgb}{0.007843,0.619608,0.450980}%
\pgfsetfillcolor{currentfill}%
\pgfsetfillopacity{0.700000}%
\pgfsetlinewidth{0.000000pt}%
\definecolor{currentstroke}{rgb}{0.000000,0.000000,0.000000}%
\pgfsetstrokecolor{currentstroke}%
\pgfsetstrokeopacity{0.700000}%
\pgfsetdash{}{0pt}%
\pgfpathmoveto{\pgfqpoint{4.986486in}{0.524170in}}%
\pgfpathlineto{\pgfqpoint{5.028574in}{0.524170in}}%
\pgfpathlineto{\pgfqpoint{5.028574in}{0.685072in}}%
\pgfpathlineto{\pgfqpoint{4.986486in}{0.685072in}}%
\pgfpathlineto{\pgfqpoint{4.986486in}{0.524170in}}%
\pgfpathclose%
\pgfusepath{fill}%
\end{pgfscope}%
\begin{pgfscope}%
\pgfpathrectangle{\pgfqpoint{0.651412in}{0.524170in}}{\pgfqpoint{4.629690in}{2.558193in}}%
\pgfusepath{clip}%
\pgfsetbuttcap%
\pgfsetmiterjoin%
\definecolor{currentfill}{rgb}{0.007843,0.619608,0.450980}%
\pgfsetfillcolor{currentfill}%
\pgfsetfillopacity{0.700000}%
\pgfsetlinewidth{0.000000pt}%
\definecolor{currentstroke}{rgb}{0.000000,0.000000,0.000000}%
\pgfsetstrokecolor{currentstroke}%
\pgfsetstrokeopacity{0.700000}%
\pgfsetdash{}{0pt}%
\pgfpathmoveto{\pgfqpoint{5.028574in}{0.524170in}}%
\pgfpathlineto{\pgfqpoint{5.070662in}{0.524170in}}%
\pgfpathlineto{\pgfqpoint{5.070662in}{0.694536in}}%
\pgfpathlineto{\pgfqpoint{5.028574in}{0.694536in}}%
\pgfpathlineto{\pgfqpoint{5.028574in}{0.524170in}}%
\pgfpathclose%
\pgfusepath{fill}%
\end{pgfscope}%
\begin{pgfscope}%
\pgfpathrectangle{\pgfqpoint{0.651412in}{0.524170in}}{\pgfqpoint{4.629690in}{2.558193in}}%
\pgfusepath{clip}%
\pgfsetrectcap%
\pgfsetroundjoin%
\pgfsetlinewidth{0.803000pt}%
\definecolor{currentstroke}{rgb}{0.450000,0.450000,0.450000}%
\pgfsetstrokecolor{currentstroke}%
\pgfsetdash{}{0pt}%
\pgfpathmoveto{\pgfqpoint{0.861853in}{0.524170in}}%
\pgfpathlineto{\pgfqpoint{0.861853in}{3.082363in}}%
\pgfusepath{stroke}%
\end{pgfscope}%
\begin{pgfscope}%
\pgfsetbuttcap%
\pgfsetroundjoin%
\definecolor{currentfill}{rgb}{0.000000,0.000000,0.000000}%
\pgfsetfillcolor{currentfill}%
\pgfsetlinewidth{0.803000pt}%
\definecolor{currentstroke}{rgb}{0.000000,0.000000,0.000000}%
\pgfsetstrokecolor{currentstroke}%
\pgfsetdash{}{0pt}%
\pgfsys@defobject{currentmarker}{\pgfqpoint{0.000000in}{-0.048611in}}{\pgfqpoint{0.000000in}{0.000000in}}{%
\pgfpathmoveto{\pgfqpoint{0.000000in}{0.000000in}}%
\pgfpathlineto{\pgfqpoint{0.000000in}{-0.048611in}}%
\pgfusepath{stroke,fill}%
}%
\begin{pgfscope}%
\pgfsys@transformshift{0.861853in}{0.524170in}%
\pgfsys@useobject{currentmarker}{}%
\end{pgfscope}%
\end{pgfscope}%
\begin{pgfscope}%
\definecolor{textcolor}{rgb}{0.000000,0.000000,0.000000}%
\pgfsetstrokecolor{textcolor}%
\pgfsetfillcolor{textcolor}%
\pgftext[x=0.861853in,y=0.426948in,,top]{\color{textcolor}\rmfamily\fontsize{8.000000}{9.600000}\selectfont \(\displaystyle {0}\)}%
\end{pgfscope}%
\begin{pgfscope}%
\pgfpathrectangle{\pgfqpoint{0.651412in}{0.524170in}}{\pgfqpoint{4.629690in}{2.558193in}}%
\pgfusepath{clip}%
\pgfsetrectcap%
\pgfsetroundjoin%
\pgfsetlinewidth{0.803000pt}%
\definecolor{currentstroke}{rgb}{0.450000,0.450000,0.450000}%
\pgfsetstrokecolor{currentstroke}%
\pgfsetdash{}{0pt}%
\pgfpathmoveto{\pgfqpoint{1.703614in}{0.524170in}}%
\pgfpathlineto{\pgfqpoint{1.703614in}{3.082363in}}%
\pgfusepath{stroke}%
\end{pgfscope}%
\begin{pgfscope}%
\pgfsetbuttcap%
\pgfsetroundjoin%
\definecolor{currentfill}{rgb}{0.000000,0.000000,0.000000}%
\pgfsetfillcolor{currentfill}%
\pgfsetlinewidth{0.803000pt}%
\definecolor{currentstroke}{rgb}{0.000000,0.000000,0.000000}%
\pgfsetstrokecolor{currentstroke}%
\pgfsetdash{}{0pt}%
\pgfsys@defobject{currentmarker}{\pgfqpoint{0.000000in}{-0.048611in}}{\pgfqpoint{0.000000in}{0.000000in}}{%
\pgfpathmoveto{\pgfqpoint{0.000000in}{0.000000in}}%
\pgfpathlineto{\pgfqpoint{0.000000in}{-0.048611in}}%
\pgfusepath{stroke,fill}%
}%
\begin{pgfscope}%
\pgfsys@transformshift{1.703614in}{0.524170in}%
\pgfsys@useobject{currentmarker}{}%
\end{pgfscope}%
\end{pgfscope}%
\begin{pgfscope}%
\definecolor{textcolor}{rgb}{0.000000,0.000000,0.000000}%
\pgfsetstrokecolor{textcolor}%
\pgfsetfillcolor{textcolor}%
\pgftext[x=1.703614in,y=0.426948in,,top]{\color{textcolor}\rmfamily\fontsize{8.000000}{9.600000}\selectfont \(\displaystyle {10}\)}%
\end{pgfscope}%
\begin{pgfscope}%
\pgfpathrectangle{\pgfqpoint{0.651412in}{0.524170in}}{\pgfqpoint{4.629690in}{2.558193in}}%
\pgfusepath{clip}%
\pgfsetrectcap%
\pgfsetroundjoin%
\pgfsetlinewidth{0.803000pt}%
\definecolor{currentstroke}{rgb}{0.450000,0.450000,0.450000}%
\pgfsetstrokecolor{currentstroke}%
\pgfsetdash{}{0pt}%
\pgfpathmoveto{\pgfqpoint{2.545376in}{0.524170in}}%
\pgfpathlineto{\pgfqpoint{2.545376in}{3.082363in}}%
\pgfusepath{stroke}%
\end{pgfscope}%
\begin{pgfscope}%
\pgfsetbuttcap%
\pgfsetroundjoin%
\definecolor{currentfill}{rgb}{0.000000,0.000000,0.000000}%
\pgfsetfillcolor{currentfill}%
\pgfsetlinewidth{0.803000pt}%
\definecolor{currentstroke}{rgb}{0.000000,0.000000,0.000000}%
\pgfsetstrokecolor{currentstroke}%
\pgfsetdash{}{0pt}%
\pgfsys@defobject{currentmarker}{\pgfqpoint{0.000000in}{-0.048611in}}{\pgfqpoint{0.000000in}{0.000000in}}{%
\pgfpathmoveto{\pgfqpoint{0.000000in}{0.000000in}}%
\pgfpathlineto{\pgfqpoint{0.000000in}{-0.048611in}}%
\pgfusepath{stroke,fill}%
}%
\begin{pgfscope}%
\pgfsys@transformshift{2.545376in}{0.524170in}%
\pgfsys@useobject{currentmarker}{}%
\end{pgfscope}%
\end{pgfscope}%
\begin{pgfscope}%
\definecolor{textcolor}{rgb}{0.000000,0.000000,0.000000}%
\pgfsetstrokecolor{textcolor}%
\pgfsetfillcolor{textcolor}%
\pgftext[x=2.545376in,y=0.426948in,,top]{\color{textcolor}\rmfamily\fontsize{8.000000}{9.600000}\selectfont \(\displaystyle {20}\)}%
\end{pgfscope}%
\begin{pgfscope}%
\pgfpathrectangle{\pgfqpoint{0.651412in}{0.524170in}}{\pgfqpoint{4.629690in}{2.558193in}}%
\pgfusepath{clip}%
\pgfsetrectcap%
\pgfsetroundjoin%
\pgfsetlinewidth{0.803000pt}%
\definecolor{currentstroke}{rgb}{0.450000,0.450000,0.450000}%
\pgfsetstrokecolor{currentstroke}%
\pgfsetdash{}{0pt}%
\pgfpathmoveto{\pgfqpoint{3.387138in}{0.524170in}}%
\pgfpathlineto{\pgfqpoint{3.387138in}{3.082363in}}%
\pgfusepath{stroke}%
\end{pgfscope}%
\begin{pgfscope}%
\pgfsetbuttcap%
\pgfsetroundjoin%
\definecolor{currentfill}{rgb}{0.000000,0.000000,0.000000}%
\pgfsetfillcolor{currentfill}%
\pgfsetlinewidth{0.803000pt}%
\definecolor{currentstroke}{rgb}{0.000000,0.000000,0.000000}%
\pgfsetstrokecolor{currentstroke}%
\pgfsetdash{}{0pt}%
\pgfsys@defobject{currentmarker}{\pgfqpoint{0.000000in}{-0.048611in}}{\pgfqpoint{0.000000in}{0.000000in}}{%
\pgfpathmoveto{\pgfqpoint{0.000000in}{0.000000in}}%
\pgfpathlineto{\pgfqpoint{0.000000in}{-0.048611in}}%
\pgfusepath{stroke,fill}%
}%
\begin{pgfscope}%
\pgfsys@transformshift{3.387138in}{0.524170in}%
\pgfsys@useobject{currentmarker}{}%
\end{pgfscope}%
\end{pgfscope}%
\begin{pgfscope}%
\definecolor{textcolor}{rgb}{0.000000,0.000000,0.000000}%
\pgfsetstrokecolor{textcolor}%
\pgfsetfillcolor{textcolor}%
\pgftext[x=3.387138in,y=0.426948in,,top]{\color{textcolor}\rmfamily\fontsize{8.000000}{9.600000}\selectfont \(\displaystyle {30}\)}%
\end{pgfscope}%
\begin{pgfscope}%
\pgfpathrectangle{\pgfqpoint{0.651412in}{0.524170in}}{\pgfqpoint{4.629690in}{2.558193in}}%
\pgfusepath{clip}%
\pgfsetrectcap%
\pgfsetroundjoin%
\pgfsetlinewidth{0.803000pt}%
\definecolor{currentstroke}{rgb}{0.450000,0.450000,0.450000}%
\pgfsetstrokecolor{currentstroke}%
\pgfsetdash{}{0pt}%
\pgfpathmoveto{\pgfqpoint{4.228900in}{0.524170in}}%
\pgfpathlineto{\pgfqpoint{4.228900in}{3.082363in}}%
\pgfusepath{stroke}%
\end{pgfscope}%
\begin{pgfscope}%
\pgfsetbuttcap%
\pgfsetroundjoin%
\definecolor{currentfill}{rgb}{0.000000,0.000000,0.000000}%
\pgfsetfillcolor{currentfill}%
\pgfsetlinewidth{0.803000pt}%
\definecolor{currentstroke}{rgb}{0.000000,0.000000,0.000000}%
\pgfsetstrokecolor{currentstroke}%
\pgfsetdash{}{0pt}%
\pgfsys@defobject{currentmarker}{\pgfqpoint{0.000000in}{-0.048611in}}{\pgfqpoint{0.000000in}{0.000000in}}{%
\pgfpathmoveto{\pgfqpoint{0.000000in}{0.000000in}}%
\pgfpathlineto{\pgfqpoint{0.000000in}{-0.048611in}}%
\pgfusepath{stroke,fill}%
}%
\begin{pgfscope}%
\pgfsys@transformshift{4.228900in}{0.524170in}%
\pgfsys@useobject{currentmarker}{}%
\end{pgfscope}%
\end{pgfscope}%
\begin{pgfscope}%
\definecolor{textcolor}{rgb}{0.000000,0.000000,0.000000}%
\pgfsetstrokecolor{textcolor}%
\pgfsetfillcolor{textcolor}%
\pgftext[x=4.228900in,y=0.426948in,,top]{\color{textcolor}\rmfamily\fontsize{8.000000}{9.600000}\selectfont \(\displaystyle {40}\)}%
\end{pgfscope}%
\begin{pgfscope}%
\pgfpathrectangle{\pgfqpoint{0.651412in}{0.524170in}}{\pgfqpoint{4.629690in}{2.558193in}}%
\pgfusepath{clip}%
\pgfsetrectcap%
\pgfsetroundjoin%
\pgfsetlinewidth{0.803000pt}%
\definecolor{currentstroke}{rgb}{0.450000,0.450000,0.450000}%
\pgfsetstrokecolor{currentstroke}%
\pgfsetdash{}{0pt}%
\pgfpathmoveto{\pgfqpoint{5.070662in}{0.524170in}}%
\pgfpathlineto{\pgfqpoint{5.070662in}{3.082363in}}%
\pgfusepath{stroke}%
\end{pgfscope}%
\begin{pgfscope}%
\pgfsetbuttcap%
\pgfsetroundjoin%
\definecolor{currentfill}{rgb}{0.000000,0.000000,0.000000}%
\pgfsetfillcolor{currentfill}%
\pgfsetlinewidth{0.803000pt}%
\definecolor{currentstroke}{rgb}{0.000000,0.000000,0.000000}%
\pgfsetstrokecolor{currentstroke}%
\pgfsetdash{}{0pt}%
\pgfsys@defobject{currentmarker}{\pgfqpoint{0.000000in}{-0.048611in}}{\pgfqpoint{0.000000in}{0.000000in}}{%
\pgfpathmoveto{\pgfqpoint{0.000000in}{0.000000in}}%
\pgfpathlineto{\pgfqpoint{0.000000in}{-0.048611in}}%
\pgfusepath{stroke,fill}%
}%
\begin{pgfscope}%
\pgfsys@transformshift{5.070662in}{0.524170in}%
\pgfsys@useobject{currentmarker}{}%
\end{pgfscope}%
\end{pgfscope}%
\begin{pgfscope}%
\definecolor{textcolor}{rgb}{0.000000,0.000000,0.000000}%
\pgfsetstrokecolor{textcolor}%
\pgfsetfillcolor{textcolor}%
\pgftext[x=5.070662in,y=0.426948in,,top]{\color{textcolor}\rmfamily\fontsize{8.000000}{9.600000}\selectfont \(\displaystyle {50}\)}%
\end{pgfscope}%
\begin{pgfscope}%
\definecolor{textcolor}{rgb}{0.000000,0.000000,0.000000}%
\pgfsetstrokecolor{textcolor}%
\pgfsetfillcolor{textcolor}%
\pgftext[x=2.966257in,y=0.272725in,,top]{\color{textcolor}\rmfamily\fontsize{10.000000}{12.000000}\selectfont Resistance in \unit{\ohm}}%
\end{pgfscope}%
\begin{pgfscope}%
\definecolor{textcolor}{rgb}{0.000000,0.000000,0.000000}%
\pgfsetstrokecolor{textcolor}%
\pgfsetfillcolor{textcolor}%
\pgftext[x=5.281103in,y=0.286614in,right,top]{\color{textcolor}\rmfamily\fontsize{8.000000}{9.600000}\selectfont \(\displaystyle \times{10^{6}}{}\)}%
\end{pgfscope}%
\begin{pgfscope}%
\pgfpathrectangle{\pgfqpoint{0.651412in}{0.524170in}}{\pgfqpoint{4.629690in}{2.558193in}}%
\pgfusepath{clip}%
\pgfsetrectcap%
\pgfsetroundjoin%
\pgfsetlinewidth{0.803000pt}%
\definecolor{currentstroke}{rgb}{0.450000,0.450000,0.450000}%
\pgfsetstrokecolor{currentstroke}%
\pgfsetdash{}{0pt}%
\pgfpathmoveto{\pgfqpoint{0.651412in}{0.524170in}}%
\pgfpathlineto{\pgfqpoint{5.281103in}{0.524170in}}%
\pgfusepath{stroke}%
\end{pgfscope}%
\begin{pgfscope}%
\pgfsetbuttcap%
\pgfsetroundjoin%
\definecolor{currentfill}{rgb}{0.000000,0.000000,0.000000}%
\pgfsetfillcolor{currentfill}%
\pgfsetlinewidth{0.803000pt}%
\definecolor{currentstroke}{rgb}{0.000000,0.000000,0.000000}%
\pgfsetstrokecolor{currentstroke}%
\pgfsetdash{}{0pt}%
\pgfsys@defobject{currentmarker}{\pgfqpoint{-0.048611in}{0.000000in}}{\pgfqpoint{-0.000000in}{0.000000in}}{%
\pgfpathmoveto{\pgfqpoint{-0.000000in}{0.000000in}}%
\pgfpathlineto{\pgfqpoint{-0.048611in}{0.000000in}}%
\pgfusepath{stroke,fill}%
}%
\begin{pgfscope}%
\pgfsys@transformshift{0.651412in}{0.524170in}%
\pgfsys@useobject{currentmarker}{}%
\end{pgfscope}%
\end{pgfscope}%
\begin{pgfscope}%
\definecolor{textcolor}{rgb}{0.000000,0.000000,0.000000}%
\pgfsetstrokecolor{textcolor}%
\pgfsetfillcolor{textcolor}%
\pgftext[x=0.344310in, y=0.485614in, left, base]{\color{textcolor}\rmfamily\fontsize{8.000000}{9.600000}\selectfont \(\displaystyle {0.00}\)}%
\end{pgfscope}%
\begin{pgfscope}%
\pgfpathrectangle{\pgfqpoint{0.651412in}{0.524170in}}{\pgfqpoint{4.629690in}{2.558193in}}%
\pgfusepath{clip}%
\pgfsetrectcap%
\pgfsetroundjoin%
\pgfsetlinewidth{0.803000pt}%
\definecolor{currentstroke}{rgb}{0.450000,0.450000,0.450000}%
\pgfsetstrokecolor{currentstroke}%
\pgfsetdash{}{0pt}%
\pgfpathmoveto{\pgfqpoint{0.651412in}{0.820300in}}%
\pgfpathlineto{\pgfqpoint{5.281103in}{0.820300in}}%
\pgfusepath{stroke}%
\end{pgfscope}%
\begin{pgfscope}%
\pgfsetbuttcap%
\pgfsetroundjoin%
\definecolor{currentfill}{rgb}{0.000000,0.000000,0.000000}%
\pgfsetfillcolor{currentfill}%
\pgfsetlinewidth{0.803000pt}%
\definecolor{currentstroke}{rgb}{0.000000,0.000000,0.000000}%
\pgfsetstrokecolor{currentstroke}%
\pgfsetdash{}{0pt}%
\pgfsys@defobject{currentmarker}{\pgfqpoint{-0.048611in}{0.000000in}}{\pgfqpoint{-0.000000in}{0.000000in}}{%
\pgfpathmoveto{\pgfqpoint{-0.000000in}{0.000000in}}%
\pgfpathlineto{\pgfqpoint{-0.048611in}{0.000000in}}%
\pgfusepath{stroke,fill}%
}%
\begin{pgfscope}%
\pgfsys@transformshift{0.651412in}{0.820300in}%
\pgfsys@useobject{currentmarker}{}%
\end{pgfscope}%
\end{pgfscope}%
\begin{pgfscope}%
\definecolor{textcolor}{rgb}{0.000000,0.000000,0.000000}%
\pgfsetstrokecolor{textcolor}%
\pgfsetfillcolor{textcolor}%
\pgftext[x=0.344310in, y=0.781745in, left, base]{\color{textcolor}\rmfamily\fontsize{8.000000}{9.600000}\selectfont \(\displaystyle {0.25}\)}%
\end{pgfscope}%
\begin{pgfscope}%
\pgfpathrectangle{\pgfqpoint{0.651412in}{0.524170in}}{\pgfqpoint{4.629690in}{2.558193in}}%
\pgfusepath{clip}%
\pgfsetrectcap%
\pgfsetroundjoin%
\pgfsetlinewidth{0.803000pt}%
\definecolor{currentstroke}{rgb}{0.450000,0.450000,0.450000}%
\pgfsetstrokecolor{currentstroke}%
\pgfsetdash{}{0pt}%
\pgfpathmoveto{\pgfqpoint{0.651412in}{1.116431in}}%
\pgfpathlineto{\pgfqpoint{5.281103in}{1.116431in}}%
\pgfusepath{stroke}%
\end{pgfscope}%
\begin{pgfscope}%
\pgfsetbuttcap%
\pgfsetroundjoin%
\definecolor{currentfill}{rgb}{0.000000,0.000000,0.000000}%
\pgfsetfillcolor{currentfill}%
\pgfsetlinewidth{0.803000pt}%
\definecolor{currentstroke}{rgb}{0.000000,0.000000,0.000000}%
\pgfsetstrokecolor{currentstroke}%
\pgfsetdash{}{0pt}%
\pgfsys@defobject{currentmarker}{\pgfqpoint{-0.048611in}{0.000000in}}{\pgfqpoint{-0.000000in}{0.000000in}}{%
\pgfpathmoveto{\pgfqpoint{-0.000000in}{0.000000in}}%
\pgfpathlineto{\pgfqpoint{-0.048611in}{0.000000in}}%
\pgfusepath{stroke,fill}%
}%
\begin{pgfscope}%
\pgfsys@transformshift{0.651412in}{1.116431in}%
\pgfsys@useobject{currentmarker}{}%
\end{pgfscope}%
\end{pgfscope}%
\begin{pgfscope}%
\definecolor{textcolor}{rgb}{0.000000,0.000000,0.000000}%
\pgfsetstrokecolor{textcolor}%
\pgfsetfillcolor{textcolor}%
\pgftext[x=0.344310in, y=1.077875in, left, base]{\color{textcolor}\rmfamily\fontsize{8.000000}{9.600000}\selectfont \(\displaystyle {0.50}\)}%
\end{pgfscope}%
\begin{pgfscope}%
\pgfpathrectangle{\pgfqpoint{0.651412in}{0.524170in}}{\pgfqpoint{4.629690in}{2.558193in}}%
\pgfusepath{clip}%
\pgfsetrectcap%
\pgfsetroundjoin%
\pgfsetlinewidth{0.803000pt}%
\definecolor{currentstroke}{rgb}{0.450000,0.450000,0.450000}%
\pgfsetstrokecolor{currentstroke}%
\pgfsetdash{}{0pt}%
\pgfpathmoveto{\pgfqpoint{0.651412in}{1.412561in}}%
\pgfpathlineto{\pgfqpoint{5.281103in}{1.412561in}}%
\pgfusepath{stroke}%
\end{pgfscope}%
\begin{pgfscope}%
\pgfsetbuttcap%
\pgfsetroundjoin%
\definecolor{currentfill}{rgb}{0.000000,0.000000,0.000000}%
\pgfsetfillcolor{currentfill}%
\pgfsetlinewidth{0.803000pt}%
\definecolor{currentstroke}{rgb}{0.000000,0.000000,0.000000}%
\pgfsetstrokecolor{currentstroke}%
\pgfsetdash{}{0pt}%
\pgfsys@defobject{currentmarker}{\pgfqpoint{-0.048611in}{0.000000in}}{\pgfqpoint{-0.000000in}{0.000000in}}{%
\pgfpathmoveto{\pgfqpoint{-0.000000in}{0.000000in}}%
\pgfpathlineto{\pgfqpoint{-0.048611in}{0.000000in}}%
\pgfusepath{stroke,fill}%
}%
\begin{pgfscope}%
\pgfsys@transformshift{0.651412in}{1.412561in}%
\pgfsys@useobject{currentmarker}{}%
\end{pgfscope}%
\end{pgfscope}%
\begin{pgfscope}%
\definecolor{textcolor}{rgb}{0.000000,0.000000,0.000000}%
\pgfsetstrokecolor{textcolor}%
\pgfsetfillcolor{textcolor}%
\pgftext[x=0.344310in, y=1.374006in, left, base]{\color{textcolor}\rmfamily\fontsize{8.000000}{9.600000}\selectfont \(\displaystyle {0.75}\)}%
\end{pgfscope}%
\begin{pgfscope}%
\pgfpathrectangle{\pgfqpoint{0.651412in}{0.524170in}}{\pgfqpoint{4.629690in}{2.558193in}}%
\pgfusepath{clip}%
\pgfsetrectcap%
\pgfsetroundjoin%
\pgfsetlinewidth{0.803000pt}%
\definecolor{currentstroke}{rgb}{0.450000,0.450000,0.450000}%
\pgfsetstrokecolor{currentstroke}%
\pgfsetdash{}{0pt}%
\pgfpathmoveto{\pgfqpoint{0.651412in}{1.708692in}}%
\pgfpathlineto{\pgfqpoint{5.281103in}{1.708692in}}%
\pgfusepath{stroke}%
\end{pgfscope}%
\begin{pgfscope}%
\pgfsetbuttcap%
\pgfsetroundjoin%
\definecolor{currentfill}{rgb}{0.000000,0.000000,0.000000}%
\pgfsetfillcolor{currentfill}%
\pgfsetlinewidth{0.803000pt}%
\definecolor{currentstroke}{rgb}{0.000000,0.000000,0.000000}%
\pgfsetstrokecolor{currentstroke}%
\pgfsetdash{}{0pt}%
\pgfsys@defobject{currentmarker}{\pgfqpoint{-0.048611in}{0.000000in}}{\pgfqpoint{-0.000000in}{0.000000in}}{%
\pgfpathmoveto{\pgfqpoint{-0.000000in}{0.000000in}}%
\pgfpathlineto{\pgfqpoint{-0.048611in}{0.000000in}}%
\pgfusepath{stroke,fill}%
}%
\begin{pgfscope}%
\pgfsys@transformshift{0.651412in}{1.708692in}%
\pgfsys@useobject{currentmarker}{}%
\end{pgfscope}%
\end{pgfscope}%
\begin{pgfscope}%
\definecolor{textcolor}{rgb}{0.000000,0.000000,0.000000}%
\pgfsetstrokecolor{textcolor}%
\pgfsetfillcolor{textcolor}%
\pgftext[x=0.344310in, y=1.670136in, left, base]{\color{textcolor}\rmfamily\fontsize{8.000000}{9.600000}\selectfont \(\displaystyle {1.00}\)}%
\end{pgfscope}%
\begin{pgfscope}%
\pgfpathrectangle{\pgfqpoint{0.651412in}{0.524170in}}{\pgfqpoint{4.629690in}{2.558193in}}%
\pgfusepath{clip}%
\pgfsetrectcap%
\pgfsetroundjoin%
\pgfsetlinewidth{0.803000pt}%
\definecolor{currentstroke}{rgb}{0.450000,0.450000,0.450000}%
\pgfsetstrokecolor{currentstroke}%
\pgfsetdash{}{0pt}%
\pgfpathmoveto{\pgfqpoint{0.651412in}{2.004822in}}%
\pgfpathlineto{\pgfqpoint{5.281103in}{2.004822in}}%
\pgfusepath{stroke}%
\end{pgfscope}%
\begin{pgfscope}%
\pgfsetbuttcap%
\pgfsetroundjoin%
\definecolor{currentfill}{rgb}{0.000000,0.000000,0.000000}%
\pgfsetfillcolor{currentfill}%
\pgfsetlinewidth{0.803000pt}%
\definecolor{currentstroke}{rgb}{0.000000,0.000000,0.000000}%
\pgfsetstrokecolor{currentstroke}%
\pgfsetdash{}{0pt}%
\pgfsys@defobject{currentmarker}{\pgfqpoint{-0.048611in}{0.000000in}}{\pgfqpoint{-0.000000in}{0.000000in}}{%
\pgfpathmoveto{\pgfqpoint{-0.000000in}{0.000000in}}%
\pgfpathlineto{\pgfqpoint{-0.048611in}{0.000000in}}%
\pgfusepath{stroke,fill}%
}%
\begin{pgfscope}%
\pgfsys@transformshift{0.651412in}{2.004822in}%
\pgfsys@useobject{currentmarker}{}%
\end{pgfscope}%
\end{pgfscope}%
\begin{pgfscope}%
\definecolor{textcolor}{rgb}{0.000000,0.000000,0.000000}%
\pgfsetstrokecolor{textcolor}%
\pgfsetfillcolor{textcolor}%
\pgftext[x=0.344310in, y=1.966266in, left, base]{\color{textcolor}\rmfamily\fontsize{8.000000}{9.600000}\selectfont \(\displaystyle {1.25}\)}%
\end{pgfscope}%
\begin{pgfscope}%
\pgfpathrectangle{\pgfqpoint{0.651412in}{0.524170in}}{\pgfqpoint{4.629690in}{2.558193in}}%
\pgfusepath{clip}%
\pgfsetrectcap%
\pgfsetroundjoin%
\pgfsetlinewidth{0.803000pt}%
\definecolor{currentstroke}{rgb}{0.450000,0.450000,0.450000}%
\pgfsetstrokecolor{currentstroke}%
\pgfsetdash{}{0pt}%
\pgfpathmoveto{\pgfqpoint{0.651412in}{2.300952in}}%
\pgfpathlineto{\pgfqpoint{5.281103in}{2.300952in}}%
\pgfusepath{stroke}%
\end{pgfscope}%
\begin{pgfscope}%
\pgfsetbuttcap%
\pgfsetroundjoin%
\definecolor{currentfill}{rgb}{0.000000,0.000000,0.000000}%
\pgfsetfillcolor{currentfill}%
\pgfsetlinewidth{0.803000pt}%
\definecolor{currentstroke}{rgb}{0.000000,0.000000,0.000000}%
\pgfsetstrokecolor{currentstroke}%
\pgfsetdash{}{0pt}%
\pgfsys@defobject{currentmarker}{\pgfqpoint{-0.048611in}{0.000000in}}{\pgfqpoint{-0.000000in}{0.000000in}}{%
\pgfpathmoveto{\pgfqpoint{-0.000000in}{0.000000in}}%
\pgfpathlineto{\pgfqpoint{-0.048611in}{0.000000in}}%
\pgfusepath{stroke,fill}%
}%
\begin{pgfscope}%
\pgfsys@transformshift{0.651412in}{2.300952in}%
\pgfsys@useobject{currentmarker}{}%
\end{pgfscope}%
\end{pgfscope}%
\begin{pgfscope}%
\definecolor{textcolor}{rgb}{0.000000,0.000000,0.000000}%
\pgfsetstrokecolor{textcolor}%
\pgfsetfillcolor{textcolor}%
\pgftext[x=0.344310in, y=2.262397in, left, base]{\color{textcolor}\rmfamily\fontsize{8.000000}{9.600000}\selectfont \(\displaystyle {1.50}\)}%
\end{pgfscope}%
\begin{pgfscope}%
\pgfpathrectangle{\pgfqpoint{0.651412in}{0.524170in}}{\pgfqpoint{4.629690in}{2.558193in}}%
\pgfusepath{clip}%
\pgfsetrectcap%
\pgfsetroundjoin%
\pgfsetlinewidth{0.803000pt}%
\definecolor{currentstroke}{rgb}{0.450000,0.450000,0.450000}%
\pgfsetstrokecolor{currentstroke}%
\pgfsetdash{}{0pt}%
\pgfpathmoveto{\pgfqpoint{0.651412in}{2.597083in}}%
\pgfpathlineto{\pgfqpoint{5.281103in}{2.597083in}}%
\pgfusepath{stroke}%
\end{pgfscope}%
\begin{pgfscope}%
\pgfsetbuttcap%
\pgfsetroundjoin%
\definecolor{currentfill}{rgb}{0.000000,0.000000,0.000000}%
\pgfsetfillcolor{currentfill}%
\pgfsetlinewidth{0.803000pt}%
\definecolor{currentstroke}{rgb}{0.000000,0.000000,0.000000}%
\pgfsetstrokecolor{currentstroke}%
\pgfsetdash{}{0pt}%
\pgfsys@defobject{currentmarker}{\pgfqpoint{-0.048611in}{0.000000in}}{\pgfqpoint{-0.000000in}{0.000000in}}{%
\pgfpathmoveto{\pgfqpoint{-0.000000in}{0.000000in}}%
\pgfpathlineto{\pgfqpoint{-0.048611in}{0.000000in}}%
\pgfusepath{stroke,fill}%
}%
\begin{pgfscope}%
\pgfsys@transformshift{0.651412in}{2.597083in}%
\pgfsys@useobject{currentmarker}{}%
\end{pgfscope}%
\end{pgfscope}%
\begin{pgfscope}%
\definecolor{textcolor}{rgb}{0.000000,0.000000,0.000000}%
\pgfsetstrokecolor{textcolor}%
\pgfsetfillcolor{textcolor}%
\pgftext[x=0.344310in, y=2.558527in, left, base]{\color{textcolor}\rmfamily\fontsize{8.000000}{9.600000}\selectfont \(\displaystyle {1.75}\)}%
\end{pgfscope}%
\begin{pgfscope}%
\pgfpathrectangle{\pgfqpoint{0.651412in}{0.524170in}}{\pgfqpoint{4.629690in}{2.558193in}}%
\pgfusepath{clip}%
\pgfsetrectcap%
\pgfsetroundjoin%
\pgfsetlinewidth{0.803000pt}%
\definecolor{currentstroke}{rgb}{0.450000,0.450000,0.450000}%
\pgfsetstrokecolor{currentstroke}%
\pgfsetdash{}{0pt}%
\pgfpathmoveto{\pgfqpoint{0.651412in}{2.893213in}}%
\pgfpathlineto{\pgfqpoint{5.281103in}{2.893213in}}%
\pgfusepath{stroke}%
\end{pgfscope}%
\begin{pgfscope}%
\pgfsetbuttcap%
\pgfsetroundjoin%
\definecolor{currentfill}{rgb}{0.000000,0.000000,0.000000}%
\pgfsetfillcolor{currentfill}%
\pgfsetlinewidth{0.803000pt}%
\definecolor{currentstroke}{rgb}{0.000000,0.000000,0.000000}%
\pgfsetstrokecolor{currentstroke}%
\pgfsetdash{}{0pt}%
\pgfsys@defobject{currentmarker}{\pgfqpoint{-0.048611in}{0.000000in}}{\pgfqpoint{-0.000000in}{0.000000in}}{%
\pgfpathmoveto{\pgfqpoint{-0.000000in}{0.000000in}}%
\pgfpathlineto{\pgfqpoint{-0.048611in}{0.000000in}}%
\pgfusepath{stroke,fill}%
}%
\begin{pgfscope}%
\pgfsys@transformshift{0.651412in}{2.893213in}%
\pgfsys@useobject{currentmarker}{}%
\end{pgfscope}%
\end{pgfscope}%
\begin{pgfscope}%
\definecolor{textcolor}{rgb}{0.000000,0.000000,0.000000}%
\pgfsetstrokecolor{textcolor}%
\pgfsetfillcolor{textcolor}%
\pgftext[x=0.344310in, y=2.854658in, left, base]{\color{textcolor}\rmfamily\fontsize{8.000000}{9.600000}\selectfont \(\displaystyle {2.00}\)}%
\end{pgfscope}%
\begin{pgfscope}%
\definecolor{textcolor}{rgb}{0.000000,0.000000,0.000000}%
\pgfsetstrokecolor{textcolor}%
\pgfsetfillcolor{textcolor}%
\pgftext[x=0.288755in,y=1.803266in,,bottom,rotate=90.000000]{\color{textcolor}\rmfamily\fontsize{10.000000}{12.000000}\selectfont Normalised resistance density in \unit{\per \ohm}}%
\end{pgfscope}%
\begin{pgfscope}%
\definecolor{textcolor}{rgb}{0.000000,0.000000,0.000000}%
\pgfsetstrokecolor{textcolor}%
\pgfsetfillcolor{textcolor}%
\pgftext[x=0.651412in,y=3.124029in,left,base]{\color{textcolor}\rmfamily\fontsize{8.000000}{9.600000}\selectfont \(\displaystyle \times{10^{\ensuremath{-}7}}{}\)}%
\end{pgfscope}%
\begin{pgfscope}%
\pgfsetrectcap%
\pgfsetmiterjoin%
\pgfsetlinewidth{0.803000pt}%
\definecolor{currentstroke}{rgb}{0.000000,0.000000,0.000000}%
\pgfsetstrokecolor{currentstroke}%
\pgfsetdash{}{0pt}%
\pgfpathmoveto{\pgfqpoint{0.651412in}{0.524170in}}%
\pgfpathlineto{\pgfqpoint{0.651412in}{3.082363in}}%
\pgfusepath{stroke}%
\end{pgfscope}%
\begin{pgfscope}%
\pgfsetrectcap%
\pgfsetmiterjoin%
\pgfsetlinewidth{0.803000pt}%
\definecolor{currentstroke}{rgb}{0.000000,0.000000,0.000000}%
\pgfsetstrokecolor{currentstroke}%
\pgfsetdash{}{0pt}%
\pgfpathmoveto{\pgfqpoint{5.281103in}{0.524170in}}%
\pgfpathlineto{\pgfqpoint{5.281103in}{3.082363in}}%
\pgfusepath{stroke}%
\end{pgfscope}%
\begin{pgfscope}%
\pgfsetrectcap%
\pgfsetmiterjoin%
\pgfsetlinewidth{0.803000pt}%
\definecolor{currentstroke}{rgb}{0.000000,0.000000,0.000000}%
\pgfsetstrokecolor{currentstroke}%
\pgfsetdash{}{0pt}%
\pgfpathmoveto{\pgfqpoint{0.651412in}{0.524170in}}%
\pgfpathlineto{\pgfqpoint{5.281103in}{0.524170in}}%
\pgfusepath{stroke}%
\end{pgfscope}%
\begin{pgfscope}%
\pgfsetrectcap%
\pgfsetmiterjoin%
\pgfsetlinewidth{0.803000pt}%
\definecolor{currentstroke}{rgb}{0.000000,0.000000,0.000000}%
\pgfsetstrokecolor{currentstroke}%
\pgfsetdash{}{0pt}%
\pgfpathmoveto{\pgfqpoint{0.651412in}{3.082363in}}%
\pgfpathlineto{\pgfqpoint{5.281103in}{3.082363in}}%
\pgfusepath{stroke}%
\end{pgfscope}%
\begin{pgfscope}%
\pgfsetbuttcap%
\pgfsetmiterjoin%
\definecolor{currentfill}{rgb}{1.000000,1.000000,1.000000}%
\pgfsetfillcolor{currentfill}%
\pgfsetfillopacity{0.800000}%
\pgfsetlinewidth{1.003750pt}%
\definecolor{currentstroke}{rgb}{0.800000,0.800000,0.800000}%
\pgfsetstrokecolor{currentstroke}%
\pgfsetstrokeopacity{0.800000}%
\pgfsetdash{}{0pt}%
\pgfpathmoveto{\pgfqpoint{3.180370in}{2.351697in}}%
\pgfpathlineto{\pgfqpoint{5.203325in}{2.351697in}}%
\pgfpathquadraticcurveto{\pgfqpoint{5.225547in}{2.351697in}}{\pgfqpoint{5.225547in}{2.373919in}}%
\pgfpathlineto{\pgfqpoint{5.225547in}{3.004585in}}%
\pgfpathquadraticcurveto{\pgfqpoint{5.225547in}{3.026807in}}{\pgfqpoint{5.203325in}{3.026807in}}%
\pgfpathlineto{\pgfqpoint{3.180370in}{3.026807in}}%
\pgfpathquadraticcurveto{\pgfqpoint{3.158148in}{3.026807in}}{\pgfqpoint{3.158148in}{3.004585in}}%
\pgfpathlineto{\pgfqpoint{3.158148in}{2.373919in}}%
\pgfpathquadraticcurveto{\pgfqpoint{3.158148in}{2.351697in}}{\pgfqpoint{3.180370in}{2.351697in}}%
\pgfpathlineto{\pgfqpoint{3.180370in}{2.351697in}}%
\pgfpathclose%
\pgfusepath{stroke,fill}%
\end{pgfscope}%
\begin{pgfscope}%
\pgfsetbuttcap%
\pgfsetmiterjoin%
\definecolor{currentfill}{rgb}{0.003922,0.450980,0.698039}%
\pgfsetfillcolor{currentfill}%
\pgfsetfillopacity{0.700000}%
\pgfsetlinewidth{0.000000pt}%
\definecolor{currentstroke}{rgb}{0.000000,0.000000,0.000000}%
\pgfsetstrokecolor{currentstroke}%
\pgfsetstrokeopacity{0.700000}%
\pgfsetdash{}{0pt}%
\pgfpathmoveto{\pgfqpoint{3.202593in}{2.899029in}}%
\pgfpathlineto{\pgfqpoint{3.424815in}{2.899029in}}%
\pgfpathlineto{\pgfqpoint{3.424815in}{2.976807in}}%
\pgfpathlineto{\pgfqpoint{3.202593in}{2.976807in}}%
\pgfpathlineto{\pgfqpoint{3.202593in}{2.899029in}}%
\pgfpathclose%
\pgfusepath{fill}%
\end{pgfscope}%
\begin{pgfscope}%
\definecolor{textcolor}{rgb}{0.000000,0.000000,0.000000}%
\pgfsetstrokecolor{textcolor}%
\pgfsetfillcolor{textcolor}%
\pgftext[x=3.513704in,y=2.899029in,left,base]{\color{textcolor}\rmfamily\fontsize{8.000000}{9.600000}\selectfont HCS, \qty{0.05}{\percent} tolerance}%
\end{pgfscope}%
\begin{pgfscope}%
\pgfsetbuttcap%
\pgfsetmiterjoin%
\definecolor{currentfill}{rgb}{0.870588,0.560784,0.019608}%
\pgfsetfillcolor{currentfill}%
\pgfsetfillopacity{0.700000}%
\pgfsetlinewidth{0.000000pt}%
\definecolor{currentstroke}{rgb}{0.000000,0.000000,0.000000}%
\pgfsetstrokecolor{currentstroke}%
\pgfsetstrokeopacity{0.700000}%
\pgfsetdash{}{0pt}%
\pgfpathmoveto{\pgfqpoint{3.202593in}{2.738585in}}%
\pgfpathlineto{\pgfqpoint{3.424815in}{2.738585in}}%
\pgfpathlineto{\pgfqpoint{3.424815in}{2.816363in}}%
\pgfpathlineto{\pgfqpoint{3.202593in}{2.816363in}}%
\pgfpathlineto{\pgfqpoint{3.202593in}{2.738585in}}%
\pgfpathclose%
\pgfusepath{fill}%
\end{pgfscope}%
\begin{pgfscope}%
\definecolor{textcolor}{rgb}{0.000000,0.000000,0.000000}%
\pgfsetstrokecolor{textcolor}%
\pgfsetfillcolor{textcolor}%
\pgftext[x=3.513704in,y=2.738585in,left,base]{\color{textcolor}\rmfamily\fontsize{8.000000}{9.600000}\selectfont HCS, \qty{0.01}{\percent} tolerance}%
\end{pgfscope}%
\begin{pgfscope}%
\pgfsetbuttcap%
\pgfsetmiterjoin%
\definecolor{currentfill}{rgb}{0.835294,0.368627,0.000000}%
\pgfsetfillcolor{currentfill}%
\pgfsetfillopacity{0.700000}%
\pgfsetlinewidth{0.000000pt}%
\definecolor{currentstroke}{rgb}{0.000000,0.000000,0.000000}%
\pgfsetstrokecolor{currentstroke}%
\pgfsetstrokeopacity{0.700000}%
\pgfsetdash{}{0pt}%
\pgfpathmoveto{\pgfqpoint{3.202593in}{2.578141in}}%
\pgfpathlineto{\pgfqpoint{3.424815in}{2.578141in}}%
\pgfpathlineto{\pgfqpoint{3.424815in}{2.655919in}}%
\pgfpathlineto{\pgfqpoint{3.202593in}{2.655919in}}%
\pgfpathlineto{\pgfqpoint{3.202593in}{2.578141in}}%
\pgfpathclose%
\pgfusepath{fill}%
\end{pgfscope}%
\begin{pgfscope}%
\definecolor{textcolor}{rgb}{0.000000,0.000000,0.000000}%
\pgfsetstrokecolor{textcolor}%
\pgfsetfillcolor{textcolor}%
\pgftext[x=3.513704in,y=2.578141in,left,base]{\color{textcolor}\rmfamily\fontsize{8.000000}{9.600000}\selectfont Improved HCS, \qty{0.05}{\percent} tolerance}%
\end{pgfscope}%
\begin{pgfscope}%
\pgfsetbuttcap%
\pgfsetmiterjoin%
\definecolor{currentfill}{rgb}{0.007843,0.619608,0.450980}%
\pgfsetfillcolor{currentfill}%
\pgfsetfillopacity{0.700000}%
\pgfsetlinewidth{0.000000pt}%
\definecolor{currentstroke}{rgb}{0.000000,0.000000,0.000000}%
\pgfsetstrokecolor{currentstroke}%
\pgfsetstrokeopacity{0.700000}%
\pgfsetdash{}{0pt}%
\pgfpathmoveto{\pgfqpoint{3.202593in}{2.417697in}}%
\pgfpathlineto{\pgfqpoint{3.424815in}{2.417697in}}%
\pgfpathlineto{\pgfqpoint{3.424815in}{2.495474in}}%
\pgfpathlineto{\pgfqpoint{3.202593in}{2.495474in}}%
\pgfpathlineto{\pgfqpoint{3.202593in}{2.417697in}}%
\pgfpathclose%
\pgfusepath{fill}%
\end{pgfscope}%
\begin{pgfscope}%
\definecolor{textcolor}{rgb}{0.000000,0.000000,0.000000}%
\pgfsetstrokecolor{textcolor}%
\pgfsetfillcolor{textcolor}%
\pgftext[x=3.513704in,y=2.417697in,left,base]{\color{textcolor}\rmfamily\fontsize{8.000000}{9.600000}\selectfont Improved HCS, \qty{0.01}{\percent} tolerance}%
\end{pgfscope}%
\end{pgfpicture}%
\makeatother%
\endgroup%
% data/plot_ltspice_monte-carlo.py
    \caption{Histogram of the Monte Carlo simulations of the output impedance for the classic Howland current source with different resistor tolerances, assuming $A = \infty$, $3 \sigma = T$, $R=\qty{1}{\kilo\ohm}$. The improved Howland current sources uses $R = \qty{10}{\kilo\ohm}$ and $R_{2b} = \qty{1}{\kilo\ohm}$. The number of simulation runs was \num{e5}.}
    \label{fig:ltpsice_howland_mc_output_impedance}
\end{figure}

Figure \ref{fig:ltpsice_howland_mc_output_impedance} gives a more complete picture of the expected output impedance distribution when implementing an (improved) Howland current source without trimming. Do note, that figure \ref{fig:ltpsice_howland_mc_output_impedance} only shows the absolute value of the output impedance. The impedance can be either positive or negative, depending on the resistor ratios. The probability is evenly distributed between negative and positive impedance, producing a left-handed negative copy of the plot in figure \ref{fig:ltpsice_howland_mc_output_impedance}. For the purpose of improving readability of the plot, only the absolute value is plotted. A negative impedance means that with increasing load voltage, the output current increases as well. For the purpose of of driving laser diodes, this distinction is irrelevant as the output impedance mostly determines the noise immunity.

First, the basic Howland current source is discussed. Using \qty{0.05}{\percent} tolerance resistors the lower limit of \qty{500}{\kilo\ohm} can be easily identified as the leftmost bin of the histogram that contains a non-zero number of counts. The maximum probability is reached at around \qty{2}{\mega\ohm}. Integrating over the probability density of the values gives a \qty{31.5}{\percent} probability to end up with an output impedance of at least \qty{7.5}{\mega\ohm}. This is not nearly enough to meet the specificaction. Going to \qty{0.01}{\percent} tolerance resistors, the lower limit of \qty{2.5}{\mega\ohm} can again be identified by the absence of counts in the lower bins. The maximum probability is around \qtyrange{10}{20}{\mega\ohm} and the chance of getting an output impedance of more than the targeted \qty{7.5}{\mega\ohm} is \qty{95.6}{\percent}, a number far closer to the desired $3\sigma$ value (\qty{99.7}{\percent}) and can be considered acceptable given the high cost of acquiring a resistor array with better specifications.

The improved Howland current source was simulated with $R=\qty{10}{\kilo\ohm}$ and $R_{2b} = \qty{1}{\kilo\ohm}$. It is therefore expected that according to equation \ref{eqn:improved_howland_output_impedance_equal_resistors} the minimum output impedance is about a factor of two higher than compared to the basic Howland current source. Looking at figure \ref{fig:ltpsice_howland_mc_output_impedance}, this assumption holds. For a resistor tolerance of $T=\qty{0.05}{\percent}$, the minimum bin is \qty{1.5}{\mega\ohm}, with the maximum probability at \qty{4.5}{\mega\ohm}. The chance of having an output impedance of more than \qty{7.5}{\mega\ohm} is \qty{79.7}{\percent}, which is quite an improvement over the \qty{31.5}{\percent} of the basic current source, but still not usable. Using $T=\qty{0.01}{\percent}$ resistors, the improved Howland current source has a minimum impedance of \qty{8.5}{\mega\ohm}, which is sufficient to meet the requirements and the maximum is between \qty{20}{\giga\ohm} and \qty{30}{\giga\ohm}. The results are summarised again in table \ref{tab:howland_current_source_summary}.
\begin{table}[hb]
    \centering
    \begin{tabular}{lccc}
        \toprule
        Configuration& Tolerance $T$ & $R_{out, min}$ Bin& P($R_{out} \geq \qty{7.5}{\mega\ohm}$) \\
        \midrule
        \multirow{2}{*}{HCS} & \qty{0.05}{\percent}& \qty{500}{\kilo\ohm}& \qty{31.5}{\percent}\\
        & \qty{0.01}{\percent} & \qty{4}{\mega\ohm}& \qty{95.6}{\percent} \\
        \multirow{2}{*}{Improved HCS} & \qty{0.05}{\percent}& \qty{1.5}{\mega\ohm}& \qty{79.7}{\percent} \\
        & \qty{0.01}{\percent}& \qty{8.5}{\mega\ohm}& $\qty{100}{\percent}$\\
        \bottomrule
    \end{tabular}
    \caption{Summary of the Monte Carlo simulations for different current source configurations.}
    \label{tab:howland_current_source_summary}
\end{table}

In comparison to the MOSFET based precision current source, for which an output impedance of \qty{40}{\giga\ohm} was calculated in equation \ref{eqn:current_source_output_impedance}, the Howland current source is the weak link, when both are combined in a laser driver. To improve this situation the resistors can either be trimmed or a JFET or MOSFET cascode can be added to improve the output impedance. Do note a cascode is not bidirectional though. Several trimming options were discussed by \citeauthor{howland_pease} \cite{howland_pease}, but trimming at this level will prove difficult. The desired trim for the classic Howland current source and $R=\qty{1}{\kilo\ohm}$ can be calculated as
\begin{align*}
    R_{out} &\approx \frac{R}{\epsilon} \approx \frac{R}{4T} \geq \qty{7.5}{\mega\ohm}\\
    \Rightarrow T &\approx \qty{33}{\micro\ohm \per\ohm}\,.
\end{align*}

The final aspect that needs to be discussed is the compliance voltage. The compliance voltage of the HCS and improved HCS was calculated in equations \ref{eqn:howland_current_compliance_voltage} and \ref{eqn:improved_howland_current_compliance_voltage} as
\begin{align*}
    V_{c, HCS} &\leq \frac{1}{2} V_{o,max} & V_{c, iHCS} &\leq V_{o,max} - V_{in}
\end{align*}

From those equations it can be seen that there is a significant difference between the the basic Howland current source and the improved HCS. The compliance voltage of the basic Howland current source only depends on the maximum output voltage of the op amp and hence the supply voltage. The improved HCS depends on both the supply voltage and the input voltage. This makes it unsuitable for a laser driver modulation source, because the maximum modulation input typically is roughly the same voltage as the maximum output of a typical op-amp, since the laser driver is modulated or steered by another box using the same power rail. The improved Howland current source can only by employed in situations, where the input is well defined and a lot lower than the power supply rail of the HCS op-amp. For a laser driver modulation current source, the basic HCS is more suitable since it is independent of the input. To achieve a high output impedance, a high quality matched array is necessary, or alternatively, a difference amplifier like the ADI \device{LT1997} \cite{datasheet_LT1997} can be used. Those amplifiers contain both the amplifier and a resistor network that is tightly matched. In case of the \device{LT1997}, the matching is better than \qty{60}{\micro\ohm \per \ohm}. Unfortunately, the choice of resistor values in this case is even more limited than the range of resistor arrays.

The problem of a limited choice of resistor values as arrays or integrated resistors can be addressed by a circuit that combines both the basic HCS and the improved HCS. This circuit is shown in figure \ref{fig:buffered_howland_current_source}.
\begin{figure}[hb]
    \centering
    %\scalebox{0.5} % scalebox
    \caption{A buffered Howland current source combining the improved HCS and the basic HCS.}
    \label{fig:buffered_howland_current_source}
\end{figure}

By adding another op-amp $R_2a$ is now connected to a low impedance node, just like in case of the basic Howland current source. This leads to the same formulae regarding the output impedance and the compliance voltage as the improved HCS, but removes the matching requirement of $R_2a$ and $R_2b$. The full derivation can be found in the Python notebook at \external{data/simulations/howland\_current\_source.ipynb} as part of the online supplemental material \cite{supplemental_material}.

With the buffered HCS it is possible to use either \num{4} matched resistors or a difference amplifier with integrated resistors. The single resistor $R_5$ is used to set the output current and can be chosen independently to result in
\begin{equation}
    I_{out} = \frac{V_{in}}{R_5}\,.
\end{equation}

% TODO: Investigation of Long-Term Drift of NTC Temperature Sensors with less than 1 mK Uncertainty
